%%
%% This is file `sample-sigconf.tex',
%% generated with the docstrip utility.
%%
%% The original source files were:
%%
%% samples.dtx  (with options: `all,proceedings,bibtex,sigconf')
%% 
%% IMPORTANT NOTICE:
%% 
%% For the copyright see the source file.
%% 
%% Any modified versions of this file must be renamed
%% with new filenames distinct from sample-sigconf.tex.
%% 
%% For distribution of the original source see the terms
%% for copying and modification in the file samples.dtx.
%% 
%% This generated file may be distributed as long as the
%% original source files, as listed above, are part of the
%% same distribution. (The sources need not necessarily be
%% in the same archive or directory.)
%%
%%
%% Commands for TeXCount
%TC:macro \cite [option:text,text]
%TC:macro \citep [option:text,text]
%TC:macro \citet [option:text,text]
%TC:envir table 0 1
%TC:envir table* 0 1
%TC:envir tabular [ignore] word
%TC:envir displaymath 0 word
%TC:envir math 0 word
%TC:envir comment 0 0
%%
%%
%% The first command in your LaTeX source must be the \documentclass
%% command.
%%
%% For submission and review of your manuscript please change the
%% command to \documentclass[manuscript, screen, review]{acmart}.
%%
%% When submitting camera ready or to TAPS, please change the command
%% to \documentclass[sigconf]{acmart} or whichever template is required
%% for your publication.
%%
%%
\documentclass[sigconf, screen]{acmart}

%%
%% \BibTeX command to typeset BibTeX logo in the docs
\AtBeginDocument{%
  \providecommand\BibTeX{{%
    Bib\TeX}}}

%% Rights management information.  This information is sent to you
%% when you complete the rights form.  These commands have SAMPLE
%% values in them; it is your responsibility as an author to replace
%% the commands and values with those provided to you when you
%% complete the rights form.
\setcopyright{none} 
%\setcopyright{acmlicensed}
\copyrightyear{2024}
\acmYear{2024}
\acmDOI{XXXXXXX.XXXXXXX}

%% These commands are for a PROCEEDINGS abstract or paper.
\acmConference[`ISCA 2025']{The 52nd IEEE/ACM International Symposium on Computer Architecture}{June 21--25, 2025}{Tokyo, Japan}
%%
%%  Uncomment \acmBooktitle if the title of the proceedings is different
%%  from ``Proceedings of ...''!
%%
%%\acmBooktitle{Woodstock '18: ACM Symposium on Neural Gaze Detection,
%%  June 03--05, 2018, Woodstock, NY}
\acmISBN{978-1-4503-XXXX-X/18/06}


%%
%% Submission ID.
%% Use this when submitting an article to a sponsored event. You'll
%% receive a unique submission ID from the organizers
%% of the event, and this ID should be used as the parameter to this command.
%%\acmSubmissionID{123-A56-BU3}

%%
%% For managing citations, it is recommended to use bibliography
%% files in BibTeX format.
%%
%% You can then either use BibTeX with the ACM-Reference-Format style,
%% or BibLaTeX with the acmnumeric or acmauthoryear sytles, that include
%% support for advanced citation of software artefact from the
%% biblatex-software package, also separately available on CTAN.
%%
%% Look at the sample-*-biblatex.tex files for templates showcasing
%% the biblatex styles.
%%

%%
%% The majority of ACM publications use numbered citations and
%% references.  The command \citestyle{authoryear} switches to the
%% "author year" style.
%%
%% If you are preparing content for an event
%% sponsored by ACM SIGGRAPH, you must use the "author year" style of
%% citations and references.
%% Uncommenting
%% the next command will enable that style.
%%\citestyle{acmauthoryear}
\settopmatter{printfolios=true}
\settopmatter{printacmref=false}
%\pagestyle{plain}

\usepackage{makecell}
\usepackage{multirow} 
\usepackage[normalem]{ulem}
\usepackage[linesnumbered, ruled, vlined]{algorithm2e}
\usepackage{pifont}
\usepackage{amsfonts}
\usepackage{framed}
\usepackage{graphicx}
\newcommand{\bone}{\ding{182}}
\newcommand{\btwo}{\ding{183}}
\newcommand{\bthree}{\ding{184}}
\newcommand{\bfour}{\ding{185}}
\newcommand{\bfive}{\ding{186}}
\newcommand{\bsix}{\ding{187}}
\newcommand{\bseven}{\ding{188}}
\newcommand{\beight}{\ding{189}}
\newcommand{\bnine}{\ding{190}}
\newcommand{\bten}{\ding{191}}
\newcommand{\beleven}{\ding{192}}

\newcommand{\blfootnote}[1]{%
  \begingroup
  \renewcommand\thefootnote{}\footnote{#1}%
  \addtocounter{footnote}{-1}%
  \endgroup
}

\renewcommand\footnotetextcopyrightpermission[1]{}

%%
%% end of the preamble, start of the body of the document source.
\begin{document}

%%
%% The "title" command has an optional parameter,
%% allowing the author to define a "short title" to be used in page headers.
\title{FPGA-based Emulation and Device-Side Management for CXL-based Memory Tiering Systems}

% 添加作者和单位
\author{
{Yiqi Chen\textsuperscript{*1}, Xiping Dong\textsuperscript{*1}, Zhe Zhou\textsuperscript{1,2}, Zhao Wang\textsuperscript{1,2}, Jie Zhang\textsuperscript{2},  Guangyu Sun\textsuperscript{$\dagger$}\textsuperscript{1}} \vspace{0.5em} \\
% Dimin Niu\textsuperscript{2}, Hongzhong Zheng\textsuperscript{2}\\\\
\textsuperscript{1}\emph{School of Integrated Circuits}, \textsuperscript{2}\emph{School of Computer Science}, 
\emph{Peking University}\\
\emph{\{yiqi.chen, zhou.zhe, wangzhao21, jiez, gsun\}@pku.edu.cn} \\  \emph{dxp@stu.pku.edu.cn}\\
}

% \subtitle{\normalsize{ISCA 2025 Submission
%     \textbf{\#1125} -- Confidential Draft -- Do NOT Distribute!!}}
%%
%% The "author" command and its associated commands are used to define
%% the authors and their affiliations.
%% Of note is the shared affiliation of the first two authors, and the
%% "authornote" and "authornotemark" commands
%% used to denote shared contribution to the research.
%\author{\normalsize{ISCA 2025 Submission
 %   \textbf{\#NaN} -- Confidential Draft -- Do NOT Distribute!!}}

%%
%% By default, the full list of authors will be used in the page
%% headers. Often, this list is too long, and will overlap
%% other information printed in the page headers. This command allows
%% the author to define a more concise list
%% of authors' names for this purpose.

%%
%% The abstract is a short summary of the work to be presented in the
%% article.

\fancyhf{}


\begin{abstract}
\label{sec:abstract}
\begin{abstract}
Retrieval-Augmented Generation (RAG) is often used with Large Language Models (LLMs) to infuse domain knowledge or user-specific information. In RAG, given a user query, a retriever extracts chunks of relevant text from a knowledge base. These chunks are sent to an LLM as part of the input prompt. Typically, any given chunk is repeatedly retrieved across user questions. However, currently, for every question, attention-layers in LLMs fully compute the key values (KVs) repeatedly for the input chunks, as state-of-the-art methods cannot reuse KV-caches when chunks appear at arbitrary locations with arbitrary contexts. Naive reuse leads to output quality degradation.  This leads to potentially redundant computations on expensive GPUs and increases latency. In this work, we propose \sys, a system for managing and reusing precomputed KVs corresponding to the text chunks (we call \textit{chunk-caches}) in RAG-based systems. We present how to identify \hl{\textit{chunk-caches} that are reusable}, how to efficiently perform a small fraction of recomputation to \textit{fix} the cache to maintain output quality, and how to efficiently store and evict \textit{chunk-caches} in the hardware for maximizing reuse while masking any overheads. With real production workloads as well as synthetic datasets, we show that \sys reduces redundant computation by \textbf{51\%} over SOTA prefix-caching and \textbf{75\%} over full recomputation.
\hl{Additionally, with continuous batching on a real production workload, we get a \textbf{1.6$\times$} speedup in throughput and a \textbf{2$\times$} reduction in end-to-end response latency over prefix-caching while maintaining quality, for both the \llama-3-8B and \llama-3-70B models. 
}
\end{abstract}





\end{abstract}

%%
%% The code below is generated by the tool at http://dl.acm.org/ccs.cfm.
%% Please copy and paste the code instead of the example below.
%%
%\begin{CCSXML}
%<ccs2012>
% <concept>
%  <concept_id>00000000.0000000.0000000</concept_id>
%  <concept_desc>Do Not Use This Code, Generate the Correct Terms for Your Paper</concept_desc>
%  <concept_significance>500</concept_significance>
% </concept>
% <concept>
%  %<concept_id>00000000.00000000.00000000</concept_id>
%  <concept_desc>Do Not Use This Code, Generate the Correct Terms for Your Paper</concept_desc>
%  <concept_significance>300</concept_significance>
% </concept>
% <concept>
%  %<concept_id>00000000.00000000.00000000</concept_id>
%  <concept_desc>Do Not Use This Code, Generate the Correct Terms for Your Paper</concept_desc>
%  <concept_significance>100</concept_significance>
% </concept>
% <concept>
 % <concept_id>00000000.00000000.00000000</concept_id>
%  <concept_desc>Do Not Use This Code, Generate the Correct Terms for Your Paper</concept_desc>
%  <concept_significance>100</concept_significance>
% </concept>
%</ccs2012>
%\end{CCSXML}

%\ccsdesc[500]{Do Not Use This Code~Generate the Correct Terms for Your Paper}
%\ccsdesc[300]{Do Not Use This Code~Generate the Correct Terms for Your Paper}
%\ccsdesc{Do Not Use This Code~Generate the Correct Terms for Your Paper}
%\ccsdesc[100]{Do Not Use This Code~Generate the Correct Terms for Your Paper}

%%
%% Keywords. The author(s) should pick words that accurately describe
%% the work being presented. Separate the keywords with commas.
% \keywords{CXL, Memory Tiering System, Emulation Platform, FPGA}


%\received{20 February 2007}
%\received[revised]{12 March 2009}
%\received[accepted]{5 June 2009}
%%
%% This command processes the author and affiliation and title
%% information and builds the first part of the formatted document.
\maketitle

\section{Introduction}
\label{sec:introduction}
\documentclass[../main.tex]{subfiles}
\graphicspath{{../images/}}
\makeatletter
\def\input@path{{../images/}}
\makeatother
\begin{document}
\section{Introduction}
\begin{figure}
\centering
\begin{tikzpicture}
\node[inner sep=0pt] (ws) at (0, 0) {
\includegraphics[height=.4\textwidth, trim={10cm 0 10cm 0},clip]{world_space.png}};
\node[inner sep=0pt] (cs) at (6,0) {\includegraphics[height=.4\textwidth, trim={10cm 1cm 10cm 4cm},clip]{conf_space.png}};
\end{tikzpicture}
\vspace{-5pt}
\label{fig:pbrm_intro}
\caption{\textbf{Left}: Shows world space obstacles as grey spheres. Robots start and goal configuration is colored red and green, respectively. Configurations along the computed path are colored transparent blue. \textbf{Right:} Mapped world space scenario to configuration space. Obstacle region is the grey mesh. Red spheres are collision-free regions computed by the neural SCDF. The optimized shortest path in the convex corridor is the blue curve.}
\vspace{-25pt}
\end{figure}
Motion planning is the problem of finding a collision-free trajectory that connects a given start and goal configuration. The planning takes place in the configuration space of the robot. For single body robots, like mobile robots or drones, the configuration space and the world space are usually the same. This simplifies the planning, since explicit obstacle representations are available which enables geometrical tools like separating hyperplanes, smallest distance to obstacles etc., to be used when designing motion planning algorithms. For multi-body robots like manipulators, the situation is completely different. The world space obstacles are usually mapped to non-convex regions, and to make the problem even harder, the mapping is usually not known. Forming explicit representations of the obstacle region in the configuration space is usually too expensive or intractable. Despite all of this, sampling based planners are used with great success, which mainly is due to their use of implicit representations of the obstacle region. The basic idea is to construct a graph in the configuration space that covers and connects the collision-free region. From this graph, a path can be extracted that connects a given start and goal configuration. The approach is computationally expensive, since the graph is constructed with the smallest geometrical building block available, points, which represents a collision-check. Furthermore, the extracted paths from the graph are non-smooth and jagged due to the stochastic nature of the approach. This adds an additional post-processing step to the process, where the paths are shortcutted and smoothened, before the path can be used for tracking. Clearly a lot of time is invested to form this graph and produce smooth paths. Thus, if the obstacles start to move, then all of this work is done in no use, since all points that make up this graph need to be re-verified, which is simply too time consuming to be done in real time.
\\\\
In this work, we want to address the existing drawbacks of the sampling based planners. Our main contribution is an improved motion planner where each vertex in the graph covers a collision-free region in the form of a sphere instead of a point and where the edges are formed with neighboring intersecting spheres. This representation has the advantage of instead of returning piecewise linear paths, returning a sequence of overlapping spheres, i.e. a convex corridor, that connects a given start and goal configuration, illustrated in Figure \ref{fig:pbrm_intro}. This convex corridor allows us to use convex optimization to produce smooth trajectories, instead of computationally expensive post-processing methods. The representation further allows us to estimate the coverage of the collision-free space, which gives us awareness and feedback in the offline roadmap construction phase. Finally, our representation is simple to adapt to moving obstacles, simply requery for the new radii and recheck for intersections. 
\\\\
The spherical collision-free regions are formed using a signed distance function (SDF), which is a function that returns the smallest distance from an arbitrary point to the boundary of an obstacle. As the name implies, the distance is signed, thus if the point is inside the obstacle it is negative otherwise positive. If the distance is positive, a sphere with radius equal to the distance is guaranteed to cover a collision-free region. Using an SDF in motion planning is not new, but what is novel about our approach is that we express the distance in the configuration space instead of the world space and by doing so allows us to form these convex collision-free regions. We refer to the resulting SDF as a signed configuration distance function (SCDF). Computing an SCDF analytically is non-trivial, our approach is therefore to parameterize the SCDF with a deep neural network and learn the mapping by supervised learning. Our resulting neural SCDF can compute distances for different parameter values of obstacle shapes and we also show how multiple distances can be combined, thus making our approach flexible.
\section{Related work}
Motion planning algorithms can roughly be divided into three families, grid-based, sampling based and optimization based methods. Grid-based methods (GBM) discretize the planning space from which a graph is then compiled. A standard search method is A$^\star$ \citep{a_star}, which is classified as an \textit{informed} search method, since it employs a heuristic function to speed up the search. A$^\star$ guarantees to return an optimal path at the level of discretization used. GBMs usually discretize the planning space by a regular lattice and this limits the GBMs to problems with low dimensionality due to the curse of dimensionality. Thus, GBMs are usually limited to single-body robots where the degrees of freedom (DOF) are low. To overcome the inherent scaling problem with the GBMs, stochastic methods are usually used for multi-body robots. These methods are termed as sampling-based methods (SBM) and core members within this family are the rapidly-exploring random trees (RRT) \citep{rrt} and the probabilistic roadmap (PRM) \citep{prm}. RRT grows a tree from the start configuration and explores the collision-free region in a rapid way until it is able to connect to the goal region. RRT is usually improved by bi-directional planning \citep{rrt_connect}, i.e. an additional tree is grown from the goal configuration and the trees are tested for connection after any tree has been expanded. RRT is a single-query method, thus it searches for a path from scratch each time it is queried. Contrary to this, PRM is a multi-query method, which solves for multiple queries without starting from scratch. PRM does this by creating a roadmap (graph) that covers the collision-free space as an offline step. The graph is then used to solve for multiple queries. PRMs are used in cases where the environment does not change since the extra offline step is too computationally costly and needs to be re-done if the environment is changed. In our work, we address this inherent issue by using a different roadmap representation. Our vertices in the graph cover a collision-free region in the form of spheres and we form the edges by checking for intersecting spheres. If something in the environment changes, we recompute the spheres radii and recheck the intersections, without relying on collision detection. We use a trained neural network to compute the sphere radius, therefore querying for the radius can be done fast, hence our representation enables the PRM for dynamic environments.
\\\\
In the recent decades, optimization based methods (OBM) \citep{chomp, schulman, itomp, stomp} have been introduced as an alternative to SBM for multi-body robots. Like the SBM, the OBMs scale well to higher dimensional problems and produce smoother motion. It is common to use a SDF in the optimization since it is a smooth function, thus enabling gradient-based methods. However, the standard way of expressing the SDF is in world space. The distance therefore needs to be mapped to the configuration space by the forward kinematics. This mapping makes the optimization problem a non-linear program (NLP), which is computationally expensive to solve. Recently, a different approach has been proposed. In \cite{mp_gcs} motion planning is formulated as a convex optimization problem by using the graph of convex sets framework \citep{gcs}. The underlying idea is to decompose the collision-free space into intersecting convex sets from which a convex optimization problem is formulated. In cases where an explicit representation of the obstacles in the configuration space exists, like for single-body robots, creating collision-free convex regions can be done fast \citep{iris}. For multi-body robots, this is non-trivial. Existing work does this successfully \citep{iris_nlp, iris_c} by an optimization based approach, but the methods are still too time consuming to be used in the presence of moving obstacles. Our approach is instead to use deep learning to learn an SDF expressed in the configuration space. With this, we can query for shortest distances to the collision boundary, which allows us to expand spherical regions which are collision-free. Our approach is fast and therefore enables our suggested roadmap planner to be used in dynamic environments.
\\\\
Recent research has focused on learning collision detection \citep{fk_kernel_distance, diffco, graphdistnet} by predicting the signed distance between the robot links and the surrounding obstacles in the world space. The learned SDF is used in trajectory optimization but since the distance is expressed in the world space, the problem becomes an NLP and therefore takes a long time to solve. We take a novel approach and suggest to instead express the signed distance in the configuration space. This allows us to improve the PRM at the same time as it enables convex optimization for trajectory optimization, which runs faster and is more reliable than NLP solvers. In \cite{cspf} a learned signed distance function in the configuration space is proposed similar to our approach. However, their approach is restricted to point cloud representations, while we propose to represent the obstacles as parameterized geometric shapes, e.g. spheres. Furthermore, we also show how to use our learned SCDF to improve an existing roadmap planner.
\section{Problem formulation}
A robot is located in the world space, $\W \subset \R^3 $. The unique location of the robot is given by its configuration $\q \in \C$, where $\C$ is the configuration space. The set of points covered by the robots bodies at a certain configuration is expressed as $\B(\q) \subset \W$. The robot is surrounded by $\NrObst$ obstacles $\O = \bigcup_{i=1}^{\NrObst} \O_i$, where  $\O_i \subset \W$. The representation of the obstacle in the configuration space is the set $\C\O_i = \{\q \in \C \: |\: \B(\q) \cap \O_i \neq \emptyset \}$. The obstacle space is formed as $\Co = \bigcup_{i=1}^{\NrObst} \C \O_i$. The complement is referred to as the free space, $\Cf = \C \setminus \Co$. The path planning problem is a tuple, ($\Cf$, $\qStart$, $\qGoal$), where we want to connect a query pair, consisting of a start, $\qStart$, and goal configuration, $\qGoal$, with a geometric path, $\q(s): [0, 1] \mapsto \Cf$, such that $\q(0)=\qStart$ and $\q(1)=\qGoal$, or report correctly when such a path does not exist.
\end{document}


\section{Background \& Motivation}
\label{sec:background}
\section{Background and Motivation} \label{sec:background_motivation}

\subsection{LLM Training \& Token Filtering} \label{sec:background_llm}

\begin{figure}[t]
	\centering
	\includegraphics[scale=0.55]{figures/llm_training.pdf}
	\caption{An overview of LLM training.}
	\label{fig:llm_training}
    % \vspace{+2mm}
\end{figure}

Training LLMs is a computationally intensive process that that demands substantial computational resources. Figure~\ref{fig:llm_training} shows an overview of the LLM training process. The training data is first tokenized and fed into the LLM, which consists of multiple transformer layers. The model processes the input data and generates predictions, which are compared to the ground truth labels (\ie, next tokens) to compute the loss. Finally, gradients are computed based on the loss to update the model parameters. 
Two primary factors significantly influence the computational cost: the size of the model (\eg, the number of layers) and the number of training tokens. For instance, training foundation models like LLaMA3-70B requires approximately 7 million GPU hours and involves processing more than 15 trillion tokens. Additionally, LLM-based applications necessitate extensive domain knowledge to fine-tune the model, which can also be computationally expensive—particularly for applications that require frequent updates (\eg, LLM-based recommender systems~\cite{DBLP:conf/sigir/LinWLYFWC24}).
Accelerating LLM training is crucial to enable faster application development, reducing costs, and minimizing the environmental impact of training LLMs.

Existing studies have explored various techniques to accelerate LLM training. However, many of them either leave limited room for further improvement or adversely affect model utility. Specifically, distributed training systems~\cite{MegatronLM,MegaScale} have been proposed to effectively leverage computational resources in parallel and reduce idle time in the computation pipeline by overlapping communication and data input/output (I/O) with computations. State-of-the-art LLM training systems~\cite{MegaScale} have achieved 55.2\% model FLOPs utilization (MFU) while training on more than 10,000 GPUs. Further enhancing the utilization rate of hardware remains a challenging task with limited room for improvement.
Techniques such as layer freezing~\cite{SmartFRZ,yiding-layer-freezing}, model pruning~\cite{DBLP:conf/nips/MaFW23,DBLP:conf/iclr/Sun0BK24}, and low-rank fine-tuning~\cite{LoRA} have been explored to reduce the number of trainable model parameters and improve training efficiency. However, decreasing the number of trainable parameters may negatively impact the model's utility and generalization ability~\cite{DBLP:journals/natmi/DingQYWYSHCCCYZWLZCLTLS23}.

Token filtering is a recently proposed technology that has been well recognized by the AI community. The core idea is to identify and filter out tokens that are either noisy or unlikely to contribute meaningfully to the training process, which implicitly improves the quality of training data to benefit the model utility. Moreover, by reducing the total number of tokens to be trained, token filtering also brings opportunity for efficiency improvement.

Existing token filtering works can be categorized into two types: \emph{forward token filtering} and \emph{backward token filtering}. As illustrated in \Cref{fig:token_filter_intro}, forward token filtering techniques remove training tokens during the forward pass, whereas backward token filtering methods eliminate tokens exclusively during the backward pass.

\begin{figure}[t]
	\centering
	\includegraphics[scale=0.6]{figures/token_filter_intro.pdf}
	\caption{An overview of existing token filter studies. Forward token filtering methods (a) filter hidden states during forward process, while backward token filtering methods (b) filter training loss during the backward process.}
	\label{fig:token_filter_intro}
    \vspace{+2mm}
\end{figure}

Forward token filtering methods have been extensively studied in previous works~\cite{DBLP:conf/acl/HouPZWSSZ22,DBLP:conf/acl/ZhongDL0ZDT23,DBLP:journals/corr/abs-2211-11586,DBLP:journals/corr/abs-2401-15293}. However, they typically underperform compared to backward filtering methods due to semantic losses~\cite{DBLP:conf/acl/ZhongDL0ZDT23,DBLP:journals/corr/abs-2211-11586,RHO}. As shown in \Cref{fig:token_filter_intro}, forward token filtering methods filter tokens at each layer of the forward computation, such that each layer of the model only processes partial context. However, this approach has been shown to cause semantic loss and potential harm model utility~\cite{DBLP:conf/acl/ZhongDL0ZDT23,DBLP:journals/corr/abs-2211-11586}. Evaluations in existing forward filtering studies~\cite{DBLP:conf/acl/HouPZWSSZ22,DBLP:conf/acl/ZhongDL0ZDT23,DBLP:journals/corr/abs-2211-11586,DBLP:journals/corr/abs-2401-15293} report only similar or lower model utilities and fail to achieve the improvements in utility seen with backward filtering methods~\cite{RHO}.

Backward token filtering is an effective solution for enhancing model utility and is widely accepted within the AI community. Existing work~\cite{RHO} has demonstrated that backward filtering methods can not only reduce the number of training tokens processed during the backward pass but also improve model utility by eliminating inconsequential tokens. As illustrated in \Cref{fig:token_filter_intro}, the backward filtering method maintains standard forward computation while performing selective token training in the output layer.
Existing studies leverage a reference model to assess the importance of each token. For instance, when training a target model to enhance mathematical reasoning, the reference model is trained on a small but high-quality mathematical corpus (\eg, clean datasets with clear instructions and derivations). During the training process, the loss of the target model (\ie, the model being trained) is compared to the loss of the reference model. Tokens with high excessive loss (\ie, the loss of the target model minus the loss of the reference model) are considered important, while those with lower excessive loss are filtered out during the backward pass.
Empirically, tokens with high excessive loss have larger room to be trained, and lower loss in the reference model also indicates that the tokens match the distribution of high-quality data. Mathematically, backward token filtering can be formulated as follows~\cite{RHO}:
\begin{equation}
    \mathcal{L}_{filter} = -\frac{1}{N \times k\%} \sum^N_{i=1} I_{k\%}(\mathbf{x}_i) \log P_{\theta}(\mathbf{x}_i|\mathbf{x}_{<i};\theta)
\end{equation}
\begin{equation}
	I_{k\%}(\mathbf{x}_i) = \left\{
	\begin{aligned}
		1, & \ if \ \mathbf{x}_i \ \in \ top \ k\% \ of \ (\mathcal{L}_{\theta}(\mathbf{x}_i)-\mathcal{L}_{ref}(\mathbf{x}_i)) \\
		0, & \ \text{otherwise}
	\end{aligned}
	\right.
\end{equation}
where $\mathcal{L}_{\theta}$ is the loss of the target model, $\mathcal{L}_{ref}$ is the loss of the reference model, and $\mathcal{L}_{filter}$ is the actual loss to train the target model while keeping $k\%$ of tokens.

In this paper, we mainly focus on backward token filtering due to its aforementioned advantages.

\subsection{Existing Token Filtering Fails to Improve Efficiency} \label{sec:efficiency_motivation}

Although backward token filtering has shown promising results in improving model utility, its potential of improving training efficiency remains unexplored. In principle, reducing the number of training tokens should bring significant efficiency improvement due to the reduced computation workload. However, existing studies fail to improve training efficiency due to the following two reasons: (1) insufficient sparsity after token filtering; and (2) inefficiency of sparse GEMM implementations.

\parab{Insufficient sparsity after token filtering.} 
The potential for efficiency improvements in token filtering methods arises from the sparsity achieved by filtering out unimportant tokens. The gradients of the filtered tokens become zero, allowing for a reduction in computational costs during the backward process. Essentially, backpropagation involves computations between gradients and activations, which are intermediate results specifically stored for the backward pass.
Current methods filter the loss of unimportant tokens at the output layer, resulting in sparse gradients. However, they leave all dense activations unchanged. Consequently, after being multiplied by these dense activations, the gradients are no longer sparse once they pass through the final attention block. Therefore, existing backward filtering methods~\cite{RHO} exhibit insufficient sparsity, even after filtering the loss at the output layer.

\Cref{eq:attention} shows the forward computation of the attention block in the transformer model, where $softmax(\mathbf{Q}\mathbf{K}^T/\sqrt{d})$ is stored as activation\footnote{Eager implementation of attention block in PyTorch stores the softmax as intermediate results. The FlashAttention recomputes the the softmax matrix during backward, which is mathematically equalivent to storing the matrix.}.
\begin{equation} \label{eq:attention}
	\mathbf{X} = softmax(\frac{\mathbf{Q}\mathbf{K}^T}{\sqrt{d}}) \times \mathbf{V}
\end{equation}
\begin{figure}[t]
	\centering
	\includegraphics[scale=0.45]{figures/dv.pdf}
	\vspace{+0.5mm}
	\caption{Leaving the activation (\ie, $softmax$) of filtered tokens unchanged makes the $\mathbf{V}$'s gradients computed by the attention block not sparse anymore after the backpropagation. The dense gradients $\mathbf{G}_V$ will be passed to the front layers, undermining sparsity in all the rest computations .}
	\label{fig:dv}
    \vspace{+1mm}
\end{figure}
\Cref{fig:dv} illustrates the process of computing gradients for $\mathbf{V}$ (\ie, $\mathbf{G_V}$) using sparse gradients while maintaining unchanged activations (\ie, activations of all tokens are retained). After filtering the tokens based on loss, the gradients of the corresponding tokens become zero, as depicted in \Cref{fig:dv}. However, because the activations of the filtered tokens remain unchanged, the gradients of $\mathbf{V}$ are no longer sparse. Consequently, the backward computation following the first attention block lacks sparsity, limiting efficiency improvements solely within the output layer.

Following the setting in existing work~\cite{RHO}, we can estimate the upper bound of efficiency improvement with existing token filtering schemes. Taking TinyLlama, a model with 22 layers and 1.1B parameters, as an example. Filtering 40\% tokens will only linearly improve the efficiency on backward propagation of the last layer, while no front layers can be improved. Thus, the overall backward efficiency can only be improved by 1.8\%. Given that backpropagation consumes 66\% of the whole training~\cite{MegatronLM}, the end-to-end efficiency improvement is only 1.2\%.

To unlock the full efficiency of token filtering, we propose to further filter the activations to retain the sparsity in the whole backpropagation, as we illustrate in \S\ref{sec:system:filter_activation}.

\parab{Inefficient sparse GEMM.} Existing sparse GEMM implementations are not well-suited for token filtering training. Although sparse GEMM is a hot research topic and PyTorch has provided a sparse tensor implementation (\ie, \texttt{torch.sparse}), the efficiency of existing sparse GEMM is only improved when the data has very high sparsity (\eg, 95\%). Furthermore, \texttt{torch.sparse} does not fully support model training. For instance, the commonly used Compressed Sparse Row (CSR) format only accommodates 2D tensors, whereas the data in transformer models is typically represented as 3D or 4D tensors.
% Moreover, \texttt{torch.sparse} fails to provide full support for modrts 2D tensors while the data in transformer models is typically 3D or 4D tensors.el training. For example, the frequently used CSR-sparse format only suppo

\begin{figure}[t]
	\centering
	\includegraphics[scale=0.52]{figures/sparse_gemm_eff.pdf}
	\caption{PyTorch sparse GEMM outperforms regular GEMM only when filtering more than 95\% tokens and cannot improve efficiency of token filtering training which typically drops 30\% $\sim$ 40\% tokens \cite{RHO}.}
	\label{fig:sparse_gemm_eff}
    % \vspace{+2mm}
\end{figure}

To demonstrate the problem, we perform experiments on our testbed (details in \S\ref{sec:eval:setup}).
We compare the efficiency of sparse GEMM in PyTorch and regular GEMM in the scenario of token filtering, \ie, the matrix is sparse by row or columns. \Cref{fig:sparse_gemm_eff} shows the comparison results under different ratios of token filtering and batch sizes. The sparse GEMM is more efficient only when over 95\% of all tokens are filtered, which is unrealistic for token filtering. Under the typical filtering rate of 40\%, sparse GEMM is even 10$\times$ slower than regular GEMM. 



\section{HeteroBox Platform}
\label{sec:heterobox}
\begin{figure*} [t]
    \centering
    \includegraphics[width=1.0\linewidth]{fig/heterobox.pdf} 
    % \vspace{-1.2em}
    \vspace{-0.7cm}
    \caption{Overview of HeteroBox Emulation Platform.}
     \label{fig:heterobox_overview}
     \vspace{-0.4cm}
\end{figure*}

% \section{System Overview}
% \section{System Overview}
\label{sec:system_overview}
The primary goal of this work is to accurately predict the MFSP in building structures subjected to various fire scenarios \revise{by utilizing the MIDR as a metric for the overall lateral stability.} To achieve this, we propose an integrated framework that combines GNNs and FEA. The system architecture is illustrated in \figref{fig:system_overview} and comprises two key components: the MIDR predictor and the MFSP predictor. \revise{There are two stages corresponding to the MIDR predictor's different modes. In stage 1, we use data with ground truth MIDR values obtained from FEA simulations to train the MIDR predictor. Then, in stage 2, with the well-trained MIDR predictor working as a differentiable agent of fire simulation, we train the MFSP predictor to determine the point that maximize the MIDR.}
This framework seamlessly integrates physics-based simulations, GNN-driven predictive modeling, and data-driven techniques, enabling efficient and accurate MFSP prediction to provide a powerful tool for proactive fire safety analysis and risk mitigation in building structures.
\begin{figure*} [h!]
    \centering
    \includegraphics[width=0.8\textwidth]{figures/system_overview.pdf}
    \caption{\revise{Proposed framework for predicting the MFSP in building structures. Trapezoids  and gray rectangles represent the NN predictor and processing, respectively. Dashed and solid lines indicate that MIDR predictor is in the respective training mode and evaluation mode with parameters fixed, acting as a differentiable agent.}}
    \label{fig:system_overview}
\end{figure*}

{\blockRevise
\subsection{Structural Stability Metrics} 
The Interstory Drift Ratio (IDR) of each node serves as a critical parameter for evaluating structural lateral stability and deformation under external forces. IDR quantifies the relative displacement between two consecutive floors (interstory displacement) as a percentage of the floor height. Mathematically, the IDR for a given node $i$ is defined as follows:
\begin{equation}
    d_i = \left.\sqrt{\left(\Delta x_i\right)^2 + \left(\Delta y_i\right)^2}\right/ H \times 100 \%,
\end{equation}
where the numerator represents the relative displacement (in the horizontal plane $xy$) of node $i$ with respect to the corresponding node on the floor below, and $H$ denotes the story height between these two floors. Excessively high drift ratios can indicate significant structural deformation, potentially leading to significant damage or collapse due to lateral instability. The Maximum IDR among all the nodes of a structure, i.e., MIDR, is chosen to be a representative example metric for assessing the overall structural stability performance during fire events. Although MIDR may not capture all possible failure modes, such as local collapses due to midspan softening, local buckling, or loss of vertical elements, we emphasize that the proposed method is not limited to MIDR. The framework can be easily adapted to other performance indicators of interest, by simply replacing the MIDR with the desired metric.

\textit{Remark:} Selecting MIDR as the primary metric for assessing the structural integrity under fire conditions is motivated by its effectiveness in quantifying global deformation patterns. In fire-induced scenarios, thermal expansion, stiffness degradation, and gravity-induced deformations contribute to structural instability, which can be captured through relative floor displacements. MIDR provides a direct and interpretable measure of structural vulnerability by identifying floors experiencing excessive lateral deformations that may lead to global instability or loss of vertical load-carrying capacity. While MIDR is commonly used for seismic and wind-induced responses, its application to fire scenarios is justified as a fire-driven thermal effect also leads to large-scale deformation patterns, particularly in multi-story steel structures. Importantly, MIDR serves as an effective proxy for overall building stability in computational frameworks where parameterizing every potential failure mode (e.g., local buckling, connection failure, progressive collapse) is infeasible. However, fire-induced failure mechanisms extend beyond interstory drift, and localized effects are not explicitly captured by MIDR. These mechanisms typically develop locally and may not always translate into immediate global structural instability. While our current framework focuses on identifying the MFSP based on a worst-case drift metric, future work could integrate alternative failure criteria to further refine the fire vulnerability predictions.

Note that ``M'' in MIDR and MFSP represents different concepts. In the case of MIDR, for a given structure and fire source point, the IDR is computed at each node, and the MIDR is defined as the maximum IDR among all nodes. In contrast, MFSP refers to the fire source location that results in the highest MIDR across all possible fire source points within the structure. In summary, an MIDR is associated with a specific structure and fire source point pair, whereas an MFSP characterizes an entire structure by identifying the most critical fire source location.

}

\subsection{MIDR \& MFSP Predictors}

The MIDR predictor is a GNN-based model designed to estimate the MIDR of a building under a given fire scenario. The inputs to this model include:
\begin{itemize}
    \item {\bf{Structural configuration}}: Building geometry, material property, and gravity loads. 
    \item {\bf{Fire location}}: The specific point where the fire is initiated within the building.
\end{itemize}
A GNN processes this input to represent the structural configuration and fire location as a graph. The MIDR predictor is trained on labeled data generated using OpenSeesRT, a robust open-source FEA framework \cite{perez2024openseesrt}. These labels represent detailed structural responses under various fire conditions. Once trained, the MIDR predictor functions as a {\em{differentiable agent}}, offering computationally efficient MIDR estimates. Its capabilities include:
\begin{itemize}
    \item {\bf{Annotating datasets}}: Assigning  MIDR values to support subsequent analyses.
    \item {\bf{Integrating with NNs}}: Reducing the computational cost typically associated with simulation-based methods.
\end{itemize}

% \subsection{MFSP Predictor}
The MFSP predictor acts as an ``argmaxer module'' for the MIDR predictor, identifing the fire location that results in the highest MIDR. This location corresponds to the point of the greatest structural vulnerability. By leveraging the structural graph as input and utilizing the MIDR predictor's outputs, the MFSP predictor efficiently pinpoints the critical fire location.

\subsection{Data Generation and Training Pipeline}
To ensure robustness and generalizability, we introduce a comprehensive data generator pipeline:
\begin{enumerate}
    \item {\bf{Structure data generator}}: This component creates synthetic datasets for diverse building configurations, including geometry, material, and gravity loads.
    \item {\bf{FEA simulations}}: With the high-fidelity FEA simulation software, OpenSeesRT, the gravity simulation is first conducted to confirm the rationality of the synthetic dataset. Further, a subset of the generated configurations undergoes fire scenario simulations using OpenSeesRT based on a rule-based thermal load generation method. These simulations produce \textbf{labeled data} detailing  the structural responses to various fire locations, forming training and testing sets for the MIDR predictor.
    \item {\bf{Unlabeled data utilization}}: The remaining configurations, without MIDR labels from the FEA simulation, are also used to train and test the MFSP predictor, leveraging the MIDR predictor as a computationally efficient, yet accurate, {\em{surrogate}} model. Although the structural configurations with unlabeled data do not undergo FEA simulations, they can be rapidly and efficiently  \emph{pseudo labeled} using this surrogate model.
\end{enumerate}



\section{HeteroMem Solution}
\label{sec:heteromem}
% In this section, we detail the hardware architecture of each component of the HeteroBox emulation platform and the HeteroMem memory tiering system.


\subsection{Overview of HeteroMem System}
\begin{figure}[t]
  \centering
  \includegraphics[width=\columnwidth]{fig/heterobox_system_overview.pdf}
  \vspace{-0.5cm}
  \caption{System Overview of HeteroMem}
  \vspace{-0.7cm}
  \label{fig:heterobox_system_overview}
\end{figure}

% The HeteroMem memory tiering system act as an intermediate layer between host CPU and device-side memory controller. The system is consisted of three main components: \textbf{Remap Table}, \textbf{Profile Unit} and \textbf{Migrate Unit}. Besides, we implement software interface support for the enumeration platform, including software driver in the kernel and BAR regs in the device for configuration and debugging. Without losing generality, in the following evaluation of HeteroMem memory tiering system, we use two tiers memory system, which consists of a fast memory tier and a slow memory tier.

% The HeteroMem memory tiering system acts as an intermediate layer between the host CPU and the device-side memory controller. 
The HeteroMem memory tiering system acts as an intermediate layer between host CPU and CXL-extended memory. 
As shown in Figure~\ref{fig:heterobox_system_overview}, the HeteroMem system comprises three main components: the \textbf{Remapping Unit}, the \textbf{Profiling Unit}, and the \textbf{Migration Unit}. 
% Additionally, we provide software interface support for the emulation platform, including a software driver in the kernel and BAR registers in the device for configuration and debugging. 
Additionally, we provide software interface support for HeteroMem memory tiering system, including a software driver in the kernel and BAR registers in the device for configuration and profiling. 
For the following evaluation of HeteroMem memory tiering system, we use a two-tier memory system, consisting of a fast memory tier and a slow memory tier, without losing generality.


% \begin{figure*} [t!]
%     \centering
    % \includegraphics[width=1.0\linewidth]{fig/remap_table_working_process.pdf} 
%     \vspace{-1.2em}
%     \caption{Remap Table Working Process.}
%          \label{fig:remap_table_working_process}
%          \vspace{-1em}
% \end{figure*}

\begin{figure*}[t!]
    \centering
    % \vspace*{-1.5cm}
    \includegraphics[width=1.0\linewidth]{fig/remap_table_working_process.pdf}
    \vspace{-0.7cm}
    \caption{Remapping Unit Working Process.}
    \label{fig:remap_table_working_process}
    \vspace*{-0.5cm}
\end{figure*}
\begin{figure}[t]
  \centering
  \includegraphics[width=\columnwidth]{fig/remap_table_v2.pdf}
  \vspace{-0.5cm}
  \caption{Block Diagram of the Remapping Unit}
  \vspace{-0.6cm}
  \label{fig:remap_table}
\end{figure}


\noindent{\textbf{Remapping Unit:}} 
% HeteroMem proposes to manage the heterogeneous memory system totally at device-side. This includes migrating the hot data from slow memory media to fast memory media, and migrating the cold data from fast memory media to slow memory media. As a result, the device needs to record the actual address of data when receiving a memory request issued by host. We build a remap table module to do this, which receives memory requests from host and translate the memory request to its actual address in the device, and then forward the translated request to the succeeding modules. The remap table should be carefully designed, because: 1) It is on the critical path of a memory request, which shouldn't introduce too much latency, otherwise it will affect the overall performance the memory system. 2) The translation granularity of remap table further determines the migration granularity, in which smaller granularity means larger overhead of remap table yet less migration amplification, larger granularity introduce less overhead in remap table yet cause more serious migration amplification. We will discuss these trade-offs of remap table in section~\ref{sec:heterobox_hardware}.
% HeteroMem proposes to manage the heterogeneous memory system entirely on the device side. This includes migrating hot data from slow memory media to fast memory media and migrating cold data from fast memory media to slow memory media. Consequently, the device must record the actual address of data when receiving a memory request from the host. We implement a Remap Unit module to handle this, which receives memory requests from the host, translates the memory request to its actual address in the device, and then forwards the translated request to the subsequent modules. The Remap Unit must be carefully designed since it is on the critical path of a memory request, and should not introduce significant latency, as this would affect the overall performance of the memory system. 
% 2) The translation granularity of the remap table determines the migration granularity. Smaller granularity results in larger overhead for the remap unit but less migration amplification, whereas larger granularity introduces less overhead in the remap table but causes more significant migration amplification. 
% We will discuss the design of Remap Unit in section~\ref{sec:heteromem}.
% \xp{also discuss remapping overhead in section~\ref{sec:evaluation}}
% HeteroMem manages the heterogeneous memory system entirely on the device side, including migrating hot data to fast memory and cold data to slow memory. 
% To keep the device side data migration transparent to CPU, the device needs a translation layer which can map the request address from host to its actual address in the device. 
% We implement a Remap Unit to handle this, translating memory requests to their actual addresses on the device and forwarding them to subsequent modules. Since the Remap Unit is on the critical path of a memory request, it must be carefully designed to avoid introducing significant latency, which would impact overall system performance.
HeteroMem manages heterogeneous memory system entirely on the device side, migrating hot data to fast memory and cold data to slow memory. To keep this migration transparent to CPU, the device needs a translation layer to map hPA to dPA. We implement a Remapping Unit to handle this, translating memory requests and forwarding them to subsequent modules. Since the Remapping Unit is on the critical path of memory requests, it must minimize added latency to ensure optimal system performance.


\noindent{\textbf{Profiling Unit:}} 
% To optimize the performance the heterogeneous memory system, we need to migrate the hot data to fast memory media. We build Profile Unit to measure the hotness of data at different address in the device. The Profile Unit take memory requests which are already translated by remap table as input and record this access. When the Profile Unit recognize data at a address as hot, it will issue a signal to remap table to indicate that a migration event should be executed. The signal contains a pair of address, which locate in fast memory region and slow memory region respectively. The Profile Unit should classify hot page and cold page accurately to do the right migration decision. For hot pages, we implement a sketch to evaluate the access time of each page and choose those with a access time larger than a threshold to be hot.
% Cold pages, however, are harder to detect. Building a LRU-like structure in a GB-level address space is impracticable because of it will consume large amount of hardware resource. Fortunately, cold page tends to remain cold in relatively longer time~\cite{tmts_asplos2023}. Based on this insight, we use a periodically scanning method to balance the hardware overhead and profiling accuracy.
% %For cold pages, we scan the address space periodically and choose those with smallest access time as cold pages.
% Besides, we design a dynamic hotness threshold setting mechanism to ensure that the hottest part of data remain in fast memory. We will discuss the design of Profile Unit in detail in section~\ref{sec:heterobox_hardware}.
% To optimize the performance of the heterogeneous memory system, we need to migrate hot data to fast memory media. We develop the Profile Unit to measure the hotness of data at different addresses in the CXL extended memory space. 
% The Profile Unit takes memory requests that have already been translated by the Remap Unit as input and records this access. When the Profile Unit identifies data at an address as hot, it issues a signal to the Remap Unit to indicate that a migration event should be executed. This signal contains a pair of addresses, located in the fast memory region and slow memory region, respectively. The Profile Unit must accurately classify hot pages and cold pages to make correct migration decisions. For hot pages, we implement a sketch to evaluate the access times of each page and select those with access times larger than a threshold to be considered hot.
% Detecting cold pages, however, is more challenging. Constructing an LRU-like structure in a GB-level address space is impractical due to the large amount of hardware resources it would consume. Fortunately, cold pages tend to remain cold for relatively longer periods~\cite{tmts_asplos2023}. Based on this insight, we use a periodic scanning method to balance hardware overhead and profiling accuracy. Additionally, we design a dynamic hotness threshold setting mechanism to ensure that the hottest portion of data remains in fast memory. 
% We will discuss the design of the Profile Unit in detail in section~\ref{sec:heteromem}.
% The Profile Unit is used for hotness and coldness profiling of the CXL-extended memory. The Profile Unit uses translated memory requests from the Remap Unit as input and records access patterns. The Profile Unit proactively profiles the data hotness in slow memory and data coldness in fast memory. Detected cold data will be temporarily buffered. When it identifies hot data, it signals the Remap Unit to initiate a migration event, providing a pair of hot data address and cold data address for the succeeding migration transaction.
% The Profile Unit profiles the hotness and coldness of data in CXL-extended memory using translated memory requests from the Remap Unit. It records access patterns and proactively profiles data hotness in slow memory and data coldness in fast memory, temporarily buffering detected cold data. When hot data is identified, the Profile Unit signals the Remap Unit to initiate a migration event, providing addresses for the hot data and previously buffered cold data for the migration transaction.
The Profiling Unit assesses the hotness and coldness of data in CXL-extended memory using translated memory requests from the Remapping Unit. It records access patterns, profiling hot data in slow memory and cold data in fast memory, temporarily buffering the dPA of detected cold data. When hot data is identified, the Profiling Unit signals Remapping Unit to initiate a migration event, providing the dPA for hot data and the dPA of previously buffered cold data for subsequent migration transaction.


\noindent{\textbf{Migration Unit:}} 
% The Migrate Unit is responsible for data migration. It takes in two address, corresponding to two pages, and it swap the content of these two pages. The key point in designing a Migrate Unit is that we need to keep all migration event transparent to CPU. To achieve that, we need to update the Remap Table when migration happens. Also, we need to ensure that at the when the Migrate Unit has already finished migrating data and the Remap Table has not updated its translation rule, no memory request can go through the Remap Table, being translated and sent to the DRAM module, otherwise it will cause broken data. We will discuss our design of Migrate Unit and how it co-operate with Remap Table in detail in section~\ref{sec:heterobox_hardware}.
% The Migrate Unit is responsible for data migration. 
% It takes in migration requests issued from Remap Unit, which contain two memory addresses corresponding to hot and cold data detected by Profile Unit, and swaps their memory location. 
% The key point in designing the Migrate Unit is to ensure that all migration events remain transparent to the CPU. To achieve this, we need to update the Remap Unit when migration occurs. Additionally, we must ensure that during the period when the Migrate Unit has finished migrating data but the Remap Unit has not yet updated its translation rules, no memory requests can pass through the Remap Unit, be translated, and be sent to the DRAM module; otherwise, this would result in data corruption. 
% We will discuss our design of the Migrate Unit and how it cooperates with the Remap Unit in detail in section~\ref{sec:heteromem}.
The Migration Unit initiates data migration upon receiving requests from the Remapping Unit, which contain dPAs for hot and cold data identified by the Profiling Unit, and swaps their memory locations. To keep migrations transparent to the CPU, the Remapping Unit must be updated whenever a migration occurs. 
% During this transaction, it is crucial to prevent memory requests from being translated by the Remapping Unit until the translation rules are updated to avoid data corruption.


% \noindent{\textbf{Latency Module:}} We build our enumeration platform on a FPGA board with homogeneous DRAM memory, yet we need to enumerate heterogeneous memory system with different configuration, such as different capacity ratio and different latency. To achieve that, we build a Latency module as an abstract layer between memory controller and other logic. It can be configured to devide the DRAM into two regions, which make an illusion that the two regions has different latency attribute, thus forming a abstraction of heterogeneous memory system with two-tiered memory. We will discuss the design of Latency Module in detail in section ?. 





\subsection{Remapping Unit}


% The Remap Table provides an additional translation layer between CPU and device. It needs to carefully designed to introduce low overhead and keep the translation process transparent to CPU. 
The Remapping Unit provides an additional translation layer between the CPU and the device. It needs to be carefully designed to introduce minimal overhead and keep the translation process transparent to the CPU.

\noindent{\textbf{Memory Layout:}} 
% As shown in Figure \ref{fig:remap_table_working_process}, the whole memory space is consisted of 1GB of fast memory and 15GB of slow memory. We modify the Linux kernel to software reserve the first 128MB of the fast memory. When the machine power on, the remap table will send a series of write request to the reserved fast memory to initialize the remap table meta data. The remap table meta data consists of a remap table array and a reversed remap table array. Access the remap table array with the index of a page index will get the translated page index corresponding to the original page index. The reversed remap table is responsible for the reversed translation.
% As shown in Figure \ref{fig:remap_table_working_process}-(a), the entire memory space consists of a fast memory region and a slow memory region. 
% The Remap Unit stores the metadata used for address translation at the beginning of fast memory.
% The memory used to store metadata is software reserved to avoid being accessed by host CPU.
% When the machine powers on, the Remap Unit sends a series of write requests to the reserved fast memory to initialize the metadata.
% This metadata consists of a remap table array and a reverse remap table array. 
% Both remap table array and reverse remap table array consist of 4 bytes entries, each entry contains a page index. Using 4 bytes page index can describe a memory space up to 16TB, and we can use longer page index (for example, 8 bytes) for larger memory space.
% Accessing the remap table array with a page index yields the translated page index corresponding to the original page index, while the reverse remap table contains the inverse mapping of the remap table. 
As shown in Figure \ref{fig:remap_table_working_process}-(a), the entire memory space consists of a fast memory region and a slow memory region. 
The Remapping Unit stores the metadata for address translation at the beginning of the fast memory, which is software-reserved to avoid access by the host CPU. 
When the machine powers on, the Remapping Unit initializes this metadata by sending a series of write requests to the reserved fast memory.
This metadata includes a remapping table array and a reverse remapping table array, both consisting of 4-byte entries, each containing a page index (for dPA). 
% HeteroMem manages the memory in 4KB granularity, so a 4-byte page index can describe a memory space of up to 16TB, and a longer index (e.g., 8 bytes) can be used for larger memory space. 
HeteroMem manages memory in 4KB page granularity, so a 4-byte page index can describe a memory space of up to 16TB. For larger memory spaces, a longer index (e.g., 8 bytes) can be used.
Accessing the remapping table array with a page index yields the translated page index, while the reverse remapping table contains the inverse mapping of the remapping table.


\noindent{\textbf{Architecture of Remapping Unit:}} 
% The architecture of the Remap Table is shown in Figure \ref{fig:remap_table}. When a memory request is issued to the Remap Table, it first come into a remap cache which buffers the remap table meta data stored in the memory. If the cache contains the meta data which is needed to translate the current request, the remap table translate the current request with the meta data and send it to the memory. If the cache miss, the current request will be buffered in a fifo to pipelining the translation process of the succeeding requests, and the remap cache will issue a read request to the remap table data in the memory. When the response of the read request returned to the remap table, the remap cache will be updated, and the request buffered in the fifo will be translated and sent to the memory. 
% The architecture of the Remap Unit is shown in Figure \ref{fig:remap_table}. When a memory request is issued to the Remap Unit, it first enters a remap cache, which buffers the remap table metadata stored in the memory. If the cache contains the metadata needed to translate the current request, the Remap Unit translates the request using the metadata and sends it to the memory. If there is a cache miss, the current request is buffered in a FIFO to pipeline the translation process for subsequent requests, and the remap cache issues a read request to fetch the remap table data from the memory. Once the read request response returns to the Remap Unit, the remap cache is updated, and the request buffered in the FIFO is translated and sent to the memory.
% The architecture of the Remapping Unit is shown in Figure \ref{fig:remap_table}. When a memory request reaches the Remapping Unit, it first enters a remapping cache that buffers the remapping table metadata stored in memory. If the cache contains the necessary metadata to translate the request, the Remapping Unit performs the translation and forwards the request to memory. In the case of a remapping cache miss, the current request is buffered in a FIFO to pipeline the translation process for subsequent requests, and the remapping cache issues a read request to fetch the remapping table data from memory. Once the read response returns, the remapping cache is updated, and the buffered request in the FIFO is translated and sent to memory.
% Figure~\ref{fig:remap_table_working_process}-(b) shows the translation process of Remap Unit when a remap cache miss happens.
The Remapping Unit's architecture, shown in Figure \ref{fig:remap_table}, works as follows: when a memory request arrives, it first checks the remapping cache, which stores metadata for translating requests. If the required metadata is found, the request is translated and sent to memory. On a cache miss, the request is buffered in a FIFO, and the remapping cache issues a memory read to retrieve the needed remapping table data. Once the data returns, the cache is updated, and the buffered request is translated and forwarded to memory.
Figure~\ref{fig:remap_table_working_process}-(b) illustrates the translation process of the Remapping Unit when a remapping cache miss occurs.



\begin{figure*} [t]
    \centering
    \includegraphics[width=0.9\linewidth]{fig/profile_unit.pdf} 
    % \vspace{-1.2em}
    \vspace{-0.4cm}
    \caption{Profiling Unit Overview.}
     \label{fig:profiling_unit_overview}
     \vspace{-0.4cm}
\end{figure*}

\noindent{\textbf{Migration Transaction:}} 
% As shown in Figure \ref{fig:remap_table_working_process}.b, when the remap table receive a migration request from the profile unit(\bone), it will start a migration transaction. The migration request contains a pair of page index which locate in fast memory and slow memory respectively. The page index in the migration request is page index which is already translated. Upon receive the migration request, the remap table will block the succeeding request from the host to ensure the atomic attribute of the migration. Then, remap table will issue a memory read request to the reverse remap table in the memory to get the original request address(\btwo). When the read response containing the original request address returns(\bthree), the remap table will update the remap table data and reverse remap table data according to the two original address and two translated address(\bfour). The Remap Table will concurrently send an enable signal to Migrate Unit after it has sent the memory read request to overlap the memory read latency. Then Migrate Unit will read out the data of two pages in the two page index given by the migration request(\bsix \bseven), swap their index and then write them back(\beight \bnine). After the migrate module has finished the migration process and the remap table has updated its remap table data and reverse remap table data, the remap table will allow succeeding memory request to go into the remap table.
As shown in Figure~\ref{fig:remap_table_working_process}-(c), when the Remapping Unit receives a migration request from the Profiling Unit (\bone), it initiates a migration transaction. The migration request contains a pair of page indexes located in fast memory and slow memory, respectively. The page indexes in the migration request have already been translated. 
Upon receiving the migration request, the Remapping Unit blocks subsequent requests from the host to ensure the atomicity of the migration. 
Then, the Remapping Unit issues two memory read requests to the reverse remapping table in the memory to retrieve the hPAs of the hot and cold page (\btwo). 
When the read responses containing the hPAs return (\bthree), the Remapping Unit updates the remapping table data and the reverse remapping table data according to the two hPAs and two dPAs (\bfour). 
The Remapping Unit concurrently sends an enable signal to the Migration Unit after issuing the memory read request to overlap the memory read latency. 
The Migration Unit then reads the data of the two pages at the two page indexes specified by the migration request (\bsix \bseven), swaps their indexes, and writes them back (\beight \bnine). 
After the Migration Unit has completed the migration process and the Remapping Unit has updated its remapping and reverse remapping table data, subsequent memory requests are allowed to proceed.


\subsection{Profiling Unit}



% The profile unit is responsible for profiling hotness of data in memory. In our HeteroMem design, we profile hotness in 4kB page granularity. The Profile Unit is linked after the remap table, off the critical path, to get the remapped request address trace. We only profile read requests since write requests are not on the critical path of the execution of programs. As shown in Figure~\ref{fig:profile_unit_overview}-(a), in the case of a CXL extended memory system with two tiers, the Profile Unit splits the read request into two streams, one target fast memory and the other target slow memory. For the read request stream target fast memory, the Profile Unit measure the hotness of each page through its access frequency and report it as hot page if its hotness is greater than a given threshold. For the read request stream target slow memory, the Profile Unit detect cold pages through its access history. The profile Unit will send pairs of hot pages and cold pages to Remap Table, whose number is limited within a given constraint, and the logic of swapping pages between fast memory and slow memory in Remap Table will do the work of data movement. If there are more than two tiers, we just need to simultaneously profile coldness and hotness of pages in each tier.

The Profiling Unit is responsible for profiling the hotness and coldness of data in memory. In our HeteroMem design, we profile data hotness and coldness at a 4kB page granularity. 
The Profiling Unit is linked after the Remapping Unit, off the critical path, to capture the dPA trace. 
% We only profile read requests, as write requests are not on the critical path of program execution. 
We profile only read requests, as write requests are generally not on the critical path of program execution.
As shown in Figure~\ref{fig:profiling_unit_overview}-(a), in the case of a CXL-extended memory system with two tiers, the Profiling Unit splits the read requests into two streams: one targeting fast memory and the other targeting slow memory. 
% For the read request stream targeting fast memory, the Profile Unit detects cold pages based on their access history.
% The detected cold pages are temporarily buffered in a cold pages buffer. Because cold pages tend to remain cold for a relatively long time, the pages in the cold pages buffer remain cold in the subsequent process.
% For the read request stream targeting slow memory, the Profile Unit measures the hotness of each page based on its access frequency and reports it as a hot page if its hotness exceeds a given threshold.
% When the Profile Unit detects a hot page, it fetches a cold page in the cold page buffer and sends the pair detected hot page and fetched cold page to the Remap Unit. 
% Then, the logic in the Remap Unit for swapping pages between fast memory and slow memory handles the data movement.
% The number of pairs of hot pages and cold pages sent to the Remap Unit is limited to a given constraint (for example, 32 pairs of pages per 100000 cycles).
For the read request stream targeting fast memory, the Profiling Unit detects cold pages based on their access history. These detected cold pages are temporarily buffered in a cold pages buffer. Since cold pages tend to remain cold for a relatively long time~\cite{tmts_asplos2023}, they typically remain cold when fetched from the buffer in subsequent processes.
For the read request stream targeting slow memory, the Profiling Unit measures the hotness of each page based on its access frequency, reporting a page as hot if its hotness exceeds a given threshold. Upon detecting a hot page, the Profiling Unit fetches a cold page from the cold pages buffer and sends the detected hot page and the fetched cold page to the Remapping Unit.
The Remapping Unit then manages the data movement, swapping pages between fast memory and slow memory. The number of these hot and cold page pairs sent to the Remapping Unit is limited by a given constraint (e.g., 32 pairs of pages per 100,000 cycles).
% Then, the logic for swapping pages between fast memory and slow memory in the Remap Unit handles the data movement. 

% If there are more than two tiers, we simultaneously profile the coldness and hotness of pages in each tier.

\noindent{\textbf{Hotness Profiling:}} 
% For memory read request target fast memory, we use a Count-Min Sketch structure to profile the hotness of each page. Count-Min Sketch is an hash-based algorithm which can be used to estimate heavy hitter within a stream. As shown in Figure~\ref{fig:profile_unit_overview}-(b), the main body of the Count-Min Sketch is a $\textbf{D}*\textbf{W}$ counter array, in which $\textbf{D}$ is the depth of the array and $\textbf{W}$ is the width of the array. The $\textbf{W}$ entries in the same line is called a lane. When a request comes in, its address will be mapped to $\textbf{D}$ different lanes with $\textbf{D}$ different hash functions. The counters being mapped to will be added by one, and when the counters overflow, we make them stay at the max value. Then the minimal value of the D mapped counters will be sent to sketch controllers, and if the minimal value is greater than a given threshold, the accessed page will be detected as hot page. We add a hot bit for every counter and set it when the corresponding page is detected as hot page. We will not report a page as hot page if all of its D hot bits is set, thus avoiding repeated reporting the same page as hot page. All the counters will be reset in a given period. Considering that counters in Count-Min Sketch give approximate value of access time, and the error bound of the Count-Min Sketch is decided by the length of the stream~\cite{count_min_sketch}, we control the reset period to limit the error bound of the sketch.
For memory read requests targeting fast memory, we use a Count-Min Sketch~\cite{count_min_sketch} structure to profile the hotness of each page. Count-Min Sketch is a hash-based algorithm that can estimate heavy hitters within a stream. As shown in Figure~\ref{fig:profiling_unit_overview}-(b), the main body of the Count-Min Sketch is a $\textbf{D}*\textbf{W}$ counter array, where $\textbf{D}$ is the depth of the array and $\textbf{W}$ is the width. The $\textbf{W}$ entries in the same row are referred to as a lane. When a request comes in, its dPA is mapped to $\textbf{D}$ different lanes using $\textbf{D}$ different hash functions. The mapped counters are incremented by one, and when the counters overflow, they remain at their maximum value. 
% The minimal value of the $\textbf{D}$ mapped counters is then sent to the succeeding hot pages detect logic. If this minimal value exceeds a given threshold, the accessed page is marked as a hot page.
The minimum value of the $\textbf{D}$ mapped counters is then sent to the subsequent hot page detection logic. If this minimum value exceeds a given threshold, the accessed page is marked as hot.
Each counter has an associated hot bit that is set when the corresponding page is detected as hot. A page will not be reported as hot if all of its $\textbf{D}$ hot bits are already set, thus avoiding repeated reporting of the same hot page. All counters are reset periodically. Since the counters in the Count-Min Sketch provide an approximate value of access frequency and the error bound of the Count-Min Sketch is determined by the length of the stream, we control the reset period to limit the sketch's error bound.


\noindent{\textbf{Coldness Profiling:}} 
% The Migrate Module use a swap mechanism to move hot data to fast memory, so when we detect a hot page in slow memory, we need to find a cold page in the fast memory to swap its location with the detected hot page. We use a ping-pong bitmap to record the access history of each page in the fast memory. The ping-pong bitmap structure consists of two bitmaps array, in which each entry is a bit corresponding to a page in the fast memory. In a given period, one bitmap, say bitmap $A$, is used to record memory access in the current period. When Profile Unit receive a memory request, it will set the bit corresponding to the accessed page in the bitmap $A$. Meanwhile, the Profile Unit will scan the other bitmap, say bitmap $B$, to get these pages with their corresponding bit in bitmap $B$ unset as cold pages. In the next period, we reset all the bits in bitmap $B$ and change the function of bitmap $A$ and bitmap $B$, which means bitmap $B$ is responsible for recording incoming memory read request, while bitmap $A$ provides cold pages to swap with hot pages. In this way, we can make sure the detected cold pages are not accessed during the last period, thus detecting cold pages accurately.
As detailed in Section~\ref{sec:heteromem}.D, the Migration Unit uses a swap mechanism to move hot data to fast memory. 
Therefore, when a hot page is detected in slow memory, we need to find a cold page in fast memory to swap its location with the detected hot page. We use a ping-pong bitmap to record the access history of each page in fast memory. The ping-pong bitmap structure consists of two bitmap arrays, each entry being a bit corresponding to a page in fast memory.
In a given period, one bitmap, say bitmap $A$, is used to record memory access in the current period. When Profiling Unit receives a memory request, it sets the bit corresponding to the accessed page in bitmap $A$. Meanwhile, Profiling Unit scans the other bitmap, say bitmap $B$, to identify pages with their corresponding bit in bitmap $B$ unset as cold pages. In the next period, we reset all bits in bitmap $B$ and switch the functions of bitmap $A$ and bitmap $B$. 
% This means bitmap $B$ now records incoming memory read requests, while bitmap $A$ provides cold pages to swap with hot pages.
This approach ensures the detected cold pages were not accessed during the last period, thus accurately identifying cold pages.

%\noindent{\textbf{Buffer and Pipeline Design:}} We implement the Profile Unit in a 200M Hz clock domain. To satisfy the timing requirement and deal with burst situation, we design a series of pipeline and buffer structure in the Profile Unit.

\subsection{Migration Unit}
% The migrate unit moves data between fast memory and slow memory. In the device side, we have no knowledge about whether a address has valid data or not, so every migration operation is achieved as a swap operation, which swap the data in an address of fast memory and an address of slow memory. The migrate module takes in a pair of address and an enable signal. When the enable signal is set high, the migrate unit stage the two input address and read data in the two address to buffer, and then write them to swapped address to complete the process. When the migrate unit is working, it blocks all the coming requests until it finishes migrating process. Blocking request while migrating will not significantly affect performance, because: 1) When migrating, the bandwidth of the memory controller is fully occupied by the migrate unit. 2) We control the frequency of migration at a reasonable level, as described in Section~\ref{sec:evaluation}.
The Migration Unit moves data between fast memory and slow memory. 
On the device side, we lack knowledge about whether an address contains valid data, so every migration operation is executed as a swap operation, swapping the data between fast memory and slow memory. 
The Migration Unit takes in a pair of dPAs and an enable signal. 
When the enable signal is set high, the Migration Unit stages the two input dPAs, reads the data from these dPAs into a buffer, and then writes the data to the swapped dPAs to complete the process. 
% While the Migration Unit is operating, it blocks all incoming requests until the migration process is finished.
% The Migration Unit issues read requests to memory controller in a non-blocking manner and issues write requests as soon as a read response returns, meanwhile blocking incoming memory requests from host during a migration, minimizing the bubbles and maximizing the migration bandwidth. 
The Migration Unit issues read requests to the memory controller in a non-blocking manner and initiates write requests as soon as a read response returns, while simultaneously blocking incoming memory requests from the host during migration, thereby minimizing idle cycles and maximizing migration bandwidth.
% Blocking requests during migration can achieve similar performance compared to non-blocking design, because: 
% 1) During migration, the bandwidth of the memory controller is fully occupied by the Migrate Unit. 2) We control the frequency of migrations to a reasonable level, as described in Section~\ref{sec:evaluation}.

% Blocking requests during migration can achieve similar performance to a non-blocking design because:
% 1) During migration, the memory controller's bandwidth is fully occupied by the Migration Unit.
% 2) We regulate the frequency of migrations to a reasonable level, as described in Section~\ref{sec:evaluation}.


\subsection{Software Interface}

% Our HeteroBox emulation platform can be runtime configured to emulate different configurations of a CXL extended memory tiering system. We have implemented corresponding configuration registers in our design. 
HeteroMem can manage the memory tiering system transparently without CPU intervention. To better understand the behavior of HeteroMem, we have implemented a series of profiling registers, each indicating the number of specific events. Additionally, HeteroMem's behavior can be tuned through several parameters, which we have also implemented as configuration registers.
We expose these configuration and profiling registers in the PCIe BAR space of the CXL memory device. The host CPU can read or write these registers through MMIO. In the OS kernel space, we have implemented a device driver that exposes the MMIO address of these registers to user space as files under the $/sys/kernel/mm/heteromem$ directory. Users can read or write these files to configure HeteroMem and obtain profiling information. For example, users can read $/sys/kernel/mm/heteromem/migrated\_pages\_cnt$ to get the number of pages migrated by HeteroMem. 
% We list all the files in Table~\ref{tab:interface_table}.

% \begin{table}[t]\centering
    \caption{User Space Software Interface.}
    \vspace{-0.2cm}
    \label{tab:interface_table}
    \resizebox{0.48\textwidth}{!}{
    \begin{tabular}{|l|l|}
        \hline
       \textbf{File Name} & \text{Description} \\
        % \hline
        % heterobox\_on & Activate the HeteroBox emulation platform \\
        % slow\_memory\_region & Configure the region of slow memory \\
        % slow\_memory\_latency & Configure the latency of slow memory \\
        \hline
        remapping\_cache\_info & Hit and miss number of remapping cache.  \\
        migrate\_pages\_cnt & Number of pages migrated by HeteroMem. \\
        fast\_memeory\_access\_cnt & Number of accesses to fast memory.\\
        slow\_memeory\_access\_cnt & Number of accesses to slow memory.\\
        memeory\_access\_per\_page & Number of accesses to each 2MB page.\\
        \hline
        cold\_page\_scan\_period & Period to reset the access bitmap. \\
        sketch\_reset\_period & Period to reset counters in sketch. \\
        sketch\_threshold & Threshold of sketch counters to detect hot pages. \\
        migrate\_pages\_limit & Upper bound of migrated pages number in a period. \\
        migrate\_pages\_limit\_period & Period of limiting the page migration. \\
        \hline
    \end{tabular}
    }
    \vspace{-0.5cm}
\end{table}

\label{sec:heterobox_software_interface}

% Our HeteroBox emulation platform can be runtime configured to emulate different configuration of CXL extended memory tiering system. We implement corresponding configuration registers in our design.
% HeteroMem can manage the memory tiering system transparently, without the intervention of CPU. In order to better understand the behaviour of HeteroMem, we implement a series of profiling registers, each indicating the number of an event. Meanwhile, the behaviour of HeteroMem can be tuned through several parameters, so we implement these parameters as configuration registers as well. 
% We list all the registers in Table ?.

% We implement a series of software interface for HeteroBox configuration and HeteroMem parameters tuning. We expose the configuration registers of our design in the PCIE BAR space. The host can read and write these registers through MMIO interface



% We expose these configuration and profiling registers at the PCIE BAR space of the CXL memory device. Host CPU can read or write these registers through MMIO. In the OS kernel space, we implement a device driver. The device driver expose the MMIO address of these registers to user space as files under $/sys/kernel/mm/heterobox$ directory. User can read or write these files to configure HeteroBox and HeteroMem, and get profiling information. For example, user can read $/sys/kernel/mm/heterobox/migrated\_pages\_cnt$ to get the number of pages which have been migrated by HeteroMem.
% We list all the files in Table~\ref{tab:interface_table}.


\section{Implementation}
\label{sec:implementation}
\section{Implementation Environment}
\label{sec:implementation_environment}

Here we introduce the detailed implementation details and environment for reproducibility purpose. For our model, we choose hyperparameters based on the performance on validation set (Document classification task in the main paper explains how we split validation set). The results in the main paper are obtain by 5 independent runs. The standard deviations reported in the main paper are 1-sigma error bars and are obtained by calling its corresponding function in Excel library. All the experiments were done on Linux server with an NVIDIA A40 GPU with 46,068 MiB. Its operating system is CentOS Linux 7 (Core). We implemented our proposed model GTFormer using Python 3.10 as programming language and PyTorch 2.0.0 as deep learning library. Other frameworks include NumPy 1.23.1, sklearn 0.23.2, and scipy 1.5.2. We emphasize that the main focus of our model is effectiveness, instead of running efficiency. But for completeness, we still make a short comment on execution time. Our model is efficient, on the largest dataset Web, the training takes less than 40 hours to converge. We will release code and datasets upon publication.

\section{Evaluation}
\label{sec:evaluation}
\section{Evaluation}
We provide three sets of insights into this section, organised as \textit{findings (F*)}. We quantitatively study the effect of the adversarial and counterfactual perturbations on the performance of informal reasoners and autoformalisation methods. Then, we dive deeper into method variants. Finally, 
we analyse the nature of formalisation errors made by the models.

\subsection{Robustness Analysis}
\paragraph{\textbf{\emph{F1: Noise perturbations have a stronger effect on formalisation methods than informal \ac{LLM} reasoners.}}}
Table~\ref{tab:distraction_k4_formalisation} shows that, on average, the accuracy of both direct and \ac{CoT} informal reasoning remains between $73\%$ and $74\%$ in the face of added noise. While the autoformalisation method performs similarly to informal reasoners on the original dataset, its performance decreases between $4\%$ and $11\%$. The accuracy drops especially with logical (L) and tautological (T) distractions, whose logical language formats trick the \ac{LLM} into formalizing the noisy clauses. On the other hand, the linguistically complex and more natural sentences of encyclopedic distractions show a minor effect, suggesting that \acp{LLM} successfully avoids formalizing the more complicated sentences.

\paragraph{\textbf{\emph{F2: All \ac{LLM}-based reasoning methods suffer a drop for counterfactual perturbations.}}} % influence .}}}
Table~\ref{tab:distraction_k4_formalisation} shows that counterfactual statements cause a significant decrease in performance for both the informal reasoners and autoformalisation methods of between $12\%$ and $13\%$ on average. 
Moreover, this observation also holds for all tested models, i.e., none are robust towards counterfactual perturbations across every evaluated dimension. Even the strongest model, GPT 4o-mini, yields a performance of 63-68\%, which is relatively close to the random performance of 50\%. The high impact of counterfactual statements (the single ``not'' inserted) could be due to the inability of \acp{LLM} to overwrite prior knowledge with explicitly stated information or memorization of the answers. We study the error sources further in §\ref{subsec:errors}.  

\noindent \paragraph{\textbf{\emph{F3: Introducing multiple noise sentences has an effect only for logical distractions.}}}
We show the impact of introducing between one and four sentences for the two top-performing autoformalisation models in Figure~\ref{fig:length_distraction}. The figure shows similar trends with and without counterfactual perturbations.
As additional logical distractions are introduced, the model performance consistently decreases. Tautological (T) distractions lead to a decline in accuracy with a single disruptive sentence, yet adding more noise does not worsen the outcome. 
The tautological corpus introduces truth constants for all sentences as a persistent unseen logical construct. Given that this leads only to a decrease for a single occurrence, we can assume that a model can consistently handle the same unseen logical construct. In contrast, the logical corpus increases the chance of adding text, requiring new, previously unseen reasoning constructs for each added sentence. The impact of encyclopedic noise remains negligible, generalising F1 to $k$ sentences. Similarly, counterfactual perturbations remain much more effective for all settings, generalising F2.

\begin{table}[!t]
\small
\setlength{\modelspacing}{2pt}
\setlength{\tabcolsep}{1.7pt} % Default value: 6pt
\setlength{\belowrulesep}{4pt}
\begin{threeparttable}
    \centering
    \begin{tabular}{cc l r rrr @{\quad} rrrr}
\toprule
\multirow{2}{*}{} & \multirow{2}{*}{} & Reasoning & \multirow{2}{*}{O} & \multicolumn{3}{c}{Distraction} & \multicolumn{4}{c}{Counterfactual} \\
 & & Format & & E& L & T & $\text{O}_C$ & $\text{E}_C$& $\text{L}_C$ & $\text{T}_C$\\
\midrule
\multirow{6}{*}{\rotatebox{90}{Gemma-2}} & \multirow{3}{*}{\rotatebox{90}{9b}}
   & Informal (direct) & \textbf{0.78} & \textbf{0.80} & \textbf{0.79} & \textbf{0.77} & 0.58 & 0.52 & 0.50 & 0.59 \\
 & & Informal (CoT) & 0.72 & 0.78 & 0.73 & 0.76 & 0.61 & \textbf{0.57} & \textbf{0.60} & \textbf{0.66} \\
 & & Formal (FOL) & 0.62 & 0.58 & 0.52 & 0.53 & \textbf{0.63} & 0.52 & 0.46 & 0.46 \\[\modelspacing]
\cmidrule{2-11}
 & \multirow{3}{*}{\rotatebox{90}{27b}} 
   & Informal (direct) & 0.71 & 0.69 & \textbf{0.66} & \textbf{0.68} & 0.59 & 0.51 & 0.54 & 0.59 \\
 & & Informal (CoT) & 0.66 & 0.65 & 0.64 & 0.63 & 0.62 & 0.58 & \textbf{0.62} & \textbf{0.64} \\
 & & Formal (FOL) & \textbf{0.74} & \textbf{0.74} & 0.61 & 0.61 & \underline{\textbf{0.72}} & \underline{\textbf{0.67}} & 0.58 & 0.51 \\[\modelspacing]
\midrule
\multirow{6}{*}{\rotatebox{90}{Mistral}} & \multirow{3}{*}{\rotatebox{90}{7B}} 
   & Informal (direct) & 0.77 & \textbf{0.77} & 0.75 & \textbf{0.79} & \textbf{0.63} & \textbf{0.54} & \textbf{0.54} & \textbf{0.66} \\
 & & Informal (CoT) & \textbf{0.79} & 0.75 & \textbf{0.77} & 0.78 & 0.55 & 0.52 & \textbf{0.54} & 0.58 \\
 & & Formal (FOL) & 0.62 & 0.58 & 0.54 & 0.57 & 0.50 & \textbf{0.54} & 0.51 & 0.52 \\[\modelspacing]
\cmidrule{2-11}
 & \multirow{3}{*}{\rotatebox{90}{Small}} 
   & Informal (direct) & \textbf{0.77} & \textbf{0.76} & \textbf{0.76} & \textbf{0.75} & 0.61 & 0.51 & 0.56 & 0.59 \\
 & & Informal (CoT) & 0.72 & 0.72 & 0.72 & 0.71 & \textbf{0.62} & \textbf{0.59} & \textbf{0.62} & \textbf{0.68} \\
 & & Formal (FOL) & 0.68 & 0.59 & 0.53 & 0.64 & 0.54 & 0.55 & 0.49 & 0.51 \\[\modelspacing]
\midrule
\multirow{6}{*}{\rotatebox{90}{Llama-3.1}} & \multirow{3}{*}{\rotatebox{90}{8B}} 
   & Informal (direct) & 0.63 & 0.61 & 0.64 & 0.66 & 0.61 & \textbf{0.62} & 0.59 & 0.61 \\
 & & Informal (CoT) & 0.73 & \textbf{0.73} & \textbf{0.71} & \textbf{0.72} & \textbf{0.62} & 0.59 & \textbf{0.61} & \textbf{0.65} \\
 & & Formal (FOL) & \textbf{0.77} & 0.71 & 0.63 & 0.52 & 0.60 & 0.58 & 0.55 & 0.52 \\[\modelspacing]
\cmidrule{2-11}
 & \multirow{3}{*}{\rotatebox{90}{70B}} 
   & Informal (direct) & 0.77 & 0.74 & 0.74 & 0.73 & 0.62 & 0.53 & 0.56 & 0.64 \\
 & & Informal (CoT) & \textbf{0.78} & \textbf{0.75} & \textbf{0.76} & \textbf{0.76} & 0.64 & 0.61 & \textbf{0.66} & \underline{\textbf{0.73}} \\
 & & Formal (FOL) & 0.74 & 0.73 & 0.71 & 0.71 & \textbf{0.66} & \textbf{0.62} & 0.59 & 0.57 \\[\modelspacing]
 \midrule
\multirow{3}{*}{\rotatebox{90}{GPT}} & \multirow{3}{*}{\rotatebox{90}{4o-mini}} 
   & Informal (direct) & 0.78 & 0.77 & 0.79 & 0.79 & 0.64 & 0.61 & 0.61 & 0.63 \\
 & & Informal (CoT) & 0.80 & 0.80 & \underline{\textbf{0.81}} & \underline{\textbf{0.82}} & \textbf{0.68} & \textbf{0.63} & \underline{\textbf{0.68}} & \textbf{0.64} \\
 & & Formal (FOL) & \underline{\textbf{0.84}} & \underline{\textbf{0.82}} & 0.73 & 0.79 & 0.63 & 0.62 & 0.57 & 0.54 \\[\modelspacing]
 \midrule
\multicolumn{2}{c}{\multirow{3}{*}{\textbf{Avg}}} 
 & Informal (direct) & 0.74 & 0.73 & 0.73 & 0.73 & 0.61 & 0.55 & 0.56 & 0.62 \\
 & & Informal (CoT) & 0.74 & 0.74 & 0.73 & 0.74 & 0.62 & 0.58 & 0.62 & 0.65 \\
  & & Formal (FOL) & 0.72 & 0.68 &	0.61 & 0.62 & 0.61 & 0.59 & 0.54 & 0.52 \\
\bottomrule
\end{tabular}
\caption{Accuracies of informal and autoformalisation-based deductive reasoners. The best overall model per dataset is underlined; the best model version is marked in bold.}
\label{tab:distraction_k4_formalisation}
\end{threeparttable}
\end{table} 

\begin{figure}[!t]
    \centering
    \scriptsize
    \begin{tikzpicture}
        \begin{axis}[name=gpt,
            title={GPT-4o-mini},
            width=0.6\linewidth,
            height=0.6\linewidth,
            xlabel={\# Noise sentences},
            ylabel={Accuracy},
            xmin=-0.1, xmax=4.1,
            ymin=0.5, ymax=0.9,
            xtick={1,2,4},
            ytick={0.55, 0.6, 0.65, 0.75, 0.8, 0.85},
            title style={yshift=-0.6em},
            legend style={at={(1,-0.15)},
	           anchor=north,legend columns=-1},
            x label style={at={(axis description cs:1,-0.05)},anchor=north},
            y label style={at={(axis description cs:-0.15,0.5)},anchor=south},
            ymajorgrids=true,
            grid style=dashed,
        ]
            \addplot[color=blue, mark=square,]
                coordinates {
                (0,0.848076939582825)(1,0.823076903820038)(2,0.826923072338104)(4,0.821153819561005)
                };
            \addplot[color=red, mark=triangle,]
                coordinates {
                (0,0.848076939582825)(1,0.817307710647583)(2,0.801923096179962)(4,0.759615361690521)
                };
            \addplot[color=green, mark=diamond,] 
                coordinates {
                (0,0.848076939582825)(1,0.767307698726654)(2,0.769230782985687)(4,0.803846180438995)
                };
            \addplot[color=blue, mark=square*] 
                coordinates {
                (0,0.627777755260468)(1,0.622222244739533)(2,0.600000023841858)(4,0.633333325386047)
                };
            \addplot[color=red, mark=triangle*,] 
                coordinates {
                (0,0.627777755260468)(1,0.611111104488373)(2,0.611111104488373)(4,0.594444453716278)
                };
            \addplot[color=green, mark=diamond*,] 
                coordinates {
                (0,0.627777755260468)(1,0.572222232818604)(2,0.538888871669769)(4,0.555555582046509)
                };
                \legend{E,L,T,$\text{E}_C$, $\text{L}_C$ , $\text{T}_C$}
        \end{axis}

        \begin{axis}[name=llama, at={($(gpt.east)+(0.1cm,0)$)},anchor=west,
            title={Llama 3.1 70b},
            width=0.6\linewidth,
            height=0.6\linewidth,
            xmin=-0.1,, xmax=4.1,
            ymin=0.5, ymax=0.9,
            xtick={1,2,4},
            ytick={0.55, 0.6, 0.65, 0.75, 0.8, 0.85},
            title style={yshift=-0.6em},
            yticklabel=\empty,
            ymajorgrids=true,
            grid style=dashed,
        ]
            \addplot[color=blue, mark=square,]
                coordinates {
                (0,0.838461518287659)(1,0.817307710647583)(2,0.805769205093384)(4,0.817307710647583)
                };
            \addplot[color=red, mark=triangle,]
                coordinates {
                (0,0.838461518287659)(1,0.819230794906616)(2,0.803846180438995)(4,0.771153867244721)
                };
            \addplot[color=green, mark=diamond,]
                coordinates {
                (0,0.838461518287659)(1,0.803846180438995)(2,0.807692289352417)(4,0.805769205093384)
                };
            \addplot[color=blue, mark=square*]
                coordinates {
                (0,0.627777755260468)(1,0.622222244739533)(2,0.577777802944183)(4,0.594444453716278)
                };
            \addplot[color=red, mark=triangle*,]
                coordinates {
                (0,0.627777755260468)(1,0.583333313465118)(2,0.561111092567444)(4,0.577777802944183)
                };
            \addplot[color=green, mark=diamond*,]
                coordinates {
                (0,0.627777755260468)(1,0.627777755260468)(2,0.566666662693024)(4,0.577777802944183)
                };
        \end{axis}
    \end{tikzpicture}
    \caption{Influence of the number of noisy sentences for FOL.}
    \label{fig:length_distraction}
\end{figure}



\subsection{Impact of Method Design}
\paragraph{\textbf{\emph{F4: \ac{CoT} prompting is most impactful when both noise and counterfactual perturbations are applied.}}}
The accuracies for the individual \acp{LLM} in Table~\ref{tab:distraction_k4_formalisation} show that the impact of \ac{CoT} is negligible for noise-only datasets (first four columns). Meanwhile, the benefit from \ac{CoT} is most pronounced in the datasets that combine noise and counterfactual perturbations.
The better-performing informal prompting strategy for a model remains stable for all types of distractions. Still, the decline in performance due to counterfactuals leads to a less consistent preference for a specific prompting style.

\paragraph{\textbf{\emph{F5: The best-performing grammar differs per model and is unstable across data versions.}}}

The evaluation of different logical forms for formal \ac{LLM}-based reasoning in Table~\ref{tab:distraction_k4_logical_form} shows the preference of some models for specific syntactic formats.
Llama 3.1 70B has a considerable improvement of $12\%$ with TPTP syntax on the original set, while Llama 3.1 8B benefits from the R-FOL syntax. However, all grammars show a declining accuracy trend and increased syntax errors for noise perturbations, where the best grammar loses its advantage over the rest. 
When comparing the grammars on the counterfactual partitions, we observe that TPTP is consistently more robust than the standard first-order logic grammar. Here, GPT 4o-mini shows a reduction from $O$ to $O_C$ of $20\%$ for FOL and only $12\%$ for the TPTP grammar. Since this does not correlate with fewer syntax errors, the formalisation in TPTP prevents semantical errors for counterfactual premises. 
A positive reading of these results, especially the minor differences between FOL and R-FOL, is that autoformalisation \acp{LLM} can adapt to the grammar syntax prescribed in the prompt without further loss in performance.

\begin{table}[!t]
\small
\setlength{\modelspacing}{2pt}
\setlength{\tabcolsep}{1.7pt} % Default value: 6pt
\setlength{\belowrulesep}{4pt}
\begin{threeparttable}
    \centering
    \begin{tabular}{cc l r rrr @{\quad} rrrr}
\toprule
\multirow{2}{*}{} & \multirow{2}{*}{} & Grammar & \multirow{2}{*}{O} & \multicolumn{3}{c}{Distraction} & \multicolumn{4}{c}{Counterfactual} \\
 & & Syntax & & E& L & T & $\text{O}_C$ & $\text{E}_C$& $\text{L}_C$ & $\text{T}_C$\\
\midrule
\multirow{6}{*}{\rotatebox{90}{Llama-3.1}} & \multirow{3}{*}{\rotatebox{90}{8B}} 
   & FOL & 0.77 & \textbf{0.71} & 0.61 & \textbf{0.53} & 0.58 & \textbf{0.55} & 0.52 & \textbf{0.56} \\
 & & R-FOL & \textbf{0.78} & 0.69 & \textbf{0.62} & \textbf{0.53} & 0.58 & \textbf{0.55} & \textbf{0.54} & 0.52 \\
 & & TPTP & 0.73 & 0.67 & 0.55 & 0.51 & \textbf{0.68} & 0.54 & 0.46 & 0.51 \\[\modelspacing]
\cmidrule{2-11}
 & \multirow{3}{*}{\rotatebox{90}{70B}} 
   & FOL & 0.76 & 0.73 & 0.71 & \textbf{0.72} & 0.67 & 0.57 & 0.63 & 0.56 \\
 & & R-FOL & 0.76 & 0.73 & 0.67 & 0.71 & 0.64 & 0.57 & 0.53 & 0.64 \\
 & & TPTP & \underline{\textbf{0.88}} & \underline{\textbf{0.84}} & \underline{\textbf{0.81}} & \textbf{0.72} & \underline{\textbf{0.81}} & \underline{\textbf{0.68}} & \underline{\textbf{0.67}} & \underline{\textbf{0.68}} \\[\modelspacing]
\midrule
\multirow{3}{*}{\rotatebox{90}{GPT}} & \multirow{3}{*}{\rotatebox{90}{4o-mini}} 
   & FOL & \textbf{0.84} & \textbf{0.82} & \textbf{0.72} & \underline{\textbf{0.78}} & 0.64 & \textbf{0.63} & \textbf{0.61} & 0.51 \\
 & & R-FOL & \textbf{0.84} & 0.77 & 0.70 & \underline{\textbf{0.78}} & \textbf{0.72} & 0.56 & 0.54 & \textbf{0.63} \\
 & & TPTP & 0.83 & \textbf{0.82} & 0.71 & 0.71 & 0.69 & \textbf{0.63} & 0.57 & 0.57 \\
\bottomrule
\end{tabular}
\caption{Accuracies of different formalisation grammars for autoformalisation.}
\label{tab:distraction_k4_logical_form}
\end{threeparttable}
\end{table} 

\paragraph{\textbf{\emph{F6: Feedback does not help \acp{LLM} self-correct to mitigate robustness issues.}}}
\autoref{tab:distraction_k4_feedback} shows the results with different error recovery mechanisms. The results indicate that no feedback strategy emerges as a winner in the different datasets. 
All feedback variants reduce syntax errors for noise perturbations, but given the lack of a consistent increase in accuracy, the corrected formalisations are most likely to contain semantic errors still. 
The type of feedback message only has a minor influence on correcting syntax errors, whereas Llama 3.1 70b and GPT 4o-mini correct slightly more syntax errors with specific error messages. This finding aligns with \cite{huang2023large}, who also found that \acp{LLM} cannot consistently self-correct their reasoning after receiving relevant feedback.

\begin{table}[!ht]
\small
\setlength{\modelspacing}{2pt}
\setlength{\tabcolsep}{1.7pt} % Default value: 6pt
\setlength{\belowrulesep}{4pt}
\begin{threeparttable}
    \centering
    \begin{tabular}{cc l r rrr @{\quad} rrrr}
\toprule
\multirow{2}{*}{} & \multirow{2}{*}{} & \multirow{2}{*}{Feedback} & \multirow{2}{*}{O} & \multicolumn{3}{c}{Distraction} & \multicolumn{4}{c}{Counterfactual} \\
 & & & & E& L & T & $\text{O}_C$ & $\text{E}_C$& $\text{L}_C$ & $\text{T}_C$\\
\midrule
\multirow{8}{*}{\rotatebox{90}{Llama-3.1}} & \multirow{4}{*}{\rotatebox{90}{8B}} 
   & No recovery & 0.77 & \textbf{0.72} & 0.62 & 0.53 & 0.59 & 0.58 & 0.56 & \textbf{0.56} \\
 & & Error type & \textbf{0.79} & 0.71 & 0.63 & \textbf{0.56} & \textbf{0.66} & 0.54 & 0.52 & 0.51 \\
 & & Error message & 0.78 & 0.71 & \textbf{0.67} & 0.55 & 0.59 & 0.53 & \underline{\textbf{0.64}} & 0.49 \\
 & & Warning & 0.74 & 0.66 & 0.58 & 0.55 & 0.55 & \textbf{0.60} & 0.49 & 0.49 \\[\modelspacing]
\cmidrule{2-11}
 & \multirow{4}{*}{\rotatebox{90}{70B}} 
   & No recovery & \textbf{0.77} & \textbf{0.72} & \textbf{0.73} & 0.71 & \textbf{0.64} & 0.59 & \textbf{0.61} & 0.56 \\
 & & Error type & 0.72 & 0.70 & 0.72 & \textbf{0.73} & 0.62 & 0.56 & 0.60 & 0.58 \\
 & & Error message & 0.71 & 0.70 & \textbf{0.73} & 0.71 & \textbf{0.64} & 0.59 & 0.54 & \underline{\textbf{0.64}} \\
 & & Warning & 0.69 & \textbf{0.72} & 0.72 & 0.72 & 0.62 & \underline{\textbf{0.65}} & \textbf{0.61} & 0.63 \\[\modelspacing]
\midrule
\multirow{4}{*}{\rotatebox{90}{GPT}} & \multirow{4}{*}{\rotatebox{90}{4o-mini}} 
   & No recovery & \underline{\textbf{0.84}} & \underline{\textbf{0.82}} & 0.73 & 0.79 & 0.64 & \textbf{0.62} & 0.56 & \textbf{0.56} \\
 & & Error type & 0.83 & 0.79 & 0.74 & 0.76 & 0.67 & 0.57 & 0.56 & \textbf{0.56} \\
 & & Error message & \underline{\textbf{0.84}} & 0.78 & \underline{\textbf{0.77}} & \underline{\textbf{0.80}} & 0.62 & 0.59 & 0.56 & \textbf{0.56} \\
 & & Warning & \underline{\textbf{0.84}} & 0.75 & 0.73 & 0.76 & \underline{\textbf{0.70}} & 0.61 & \textbf{0.61} & 0.55 \\
 \bottomrule
\end{tabular}
\caption{Accuracies of error recovery strategies.}
\label{tab:distraction_k4_feedback}
\end{threeparttable}
\end{table} 

\subsection{Error Analysis}
\label{subsec:errors}
\paragraph{\textbf{\emph{F7: Autoformalisation increases syntax errors for noise perturbations.}}}
The low performance for noise perturbations correlates with more syntax errors for all models and distraction categories (cf. execution rates in Table~\ref{tab:appendix_k4_formalisation_exec}). The three worst-performing models (both Mistral models, Gemma-2 9b) generate, at best, for $37\%$  and, at worst, for only $4\%$ of the samples, a valid logical form.
Gemma-2 9b and Llama3.1 8b produce more syntax errors than the larger counterparts, suggesting that larger models are more robust towards noise perturbations. 
The accuracy of syntactically valid samples is higher than the informal reasoning methods for most distractions (Table~\ref{tab:appendix_k4_formalisation_vacc}), motivating informal reasoning as a backup strategy for formal reasoning. The error message feedback reveals two common syntax errors: 1) errors by models with an initial low execution rate exhibit issues with the template structure, including using incorrect keywords or adding conversational phrases;
2) perturbation-related errors, the most common of which is using undefined truth constants as part of tautological distractions. 

\paragraph{\textbf{\emph{F8: Autoformalisation increases semantic errors for counterfactuals.}}}
Unlike the introduced noise, counterfactual perturbations do not lead to more syntax errors. The execution rate in Table~\ref{tab:appendix_k4_formalisation_exec} is stable or improves for counterfactuals. However, we see a drop in accuracy for the counterfactual column $\text{O}_C$ in Table~\ref{tab:distraction_k4_formalisation} and can conclude that the number of logical forms with semantic errors has to increase. This suggests that the introduced negation is not correctly formalised. Looking at the warnings generated by the feedback mechanism, for GPT 4o-mini, $161$ warning messages are generated on the unperturbed data. $54$ of these were fixed with a single iteration. Not considering predicates and individuals as part of the context is the most frequent warning across all models. 

\section{Related Work}
\label{sec:related_work}
\section{Related Work}
% \subsection{Vision Language Model}
% 시각장애인에서 상황을 설명할 DB가 없으니 만들었다. 그리고 이를 VLM에 튜닝했다.
\subsection{Technical approaches for assisting the visually-impaired}


\subsection{Datasets for visual instruction tuning}


\section{Discussion \& Future Work}
\label{sec:discussion}
This work identifies signal collapse as a critical bottleneck in one-shot neural network pruning. Performance loss in pruned networks is due to \textbf{signal collapse} in addition to the removal of critical parameters. We propose \textbf{REFLOW} (\textbf{Re}storing \textbf{F}low of \textbf{Low}-variance signals), a simple yet effective method that mitigates signal collapse without computationally expensive weight updates. By focusing on signal preservation, REFLOW highlights the importance of mitigating signal collapse in sparse networks and enables magnitude pruning to match or surpass state-of-the-art one-shot pruning methods such as CHITA, CBS, and WF.

REFLOW consistently achieves state-of-the-art accuracy across diverse architectures, restoring ResNeXt-101 from under 4.1\% to 78.9\% top-1 accuracy at 80\% sparsity on ImageNet. Its lightweight design makes it a practical solution for both research and deployment, delivering high-quality sparse models without the overhead of traditional approaches. These findings challenge the traditional emphasis on weight selection strategies and underscore the critical role of signal propagation for achieving high-quality sparse networks in the context of one-shot pruning.




\section{Conclusion}
\label{sec:conclusion}
\section*{Conclusion}
This paper aims to enhance our understanding of the computational complexity of computing various Shapley value variants. We found that for various ML models --- including decision trees, regression tree ensembles, weighted automata, and linear regression --- both local and global interventional and baseline SHAP can be computed in polynomial time under HMM modeled distributions. This extends popular algorithms, such as TreeSHAP, beyond their empirical distributional scope. We also establish strict complexity gaps between the various SHAP variants (baseline, interventional, and conditional) and prove the intractability of computing SHAP for tree ensembles and neural networks in simplified scenarios. Overall, we present SHAP as a versatile framework whose complexity depends on four key factors: \begin{inparaenum}[(i)] \item model type, \item SHAP variant, \item distribution modeling approach, \item and local vs. global explanations\end{inparaenum}. We believe this perspective provides deeper insight into the computational complexity of SHAP, paving the way for future work.




%We believe that our framework provides a more intricate understanding of SHAP computation complexity across different models, distributions, and variants, paving the way for further research.

Our work opens promising directions for future research. First, expanding our computational analysis to other SHAP-related metrics, such as asymmetric SHAP~\citep{frye20} and SAGE~\citep{covert2020understanding}, would be valuable. Additionally, we aim to explore more expressive distribution classes and relaxed assumptions beyond those in Section \ref{sec:tractable} while maintaining tractable SHAP computation. Finally, when exact computation is intractable (Section \ref{sec:intractable}), investigating the approximability of SHAP metrics through approximation and parameterized complexity theory~\citep{downey2012parameterized} is an important direction.

%Our work opens several promising avenues for future research on the computational properties of explainable AI methods, with a particular focus on SHAP. First, it would be interesting to broaden the computational analysis conducted in this work to include other popular SHAP-related metrics in the literature, such as asymmetric SHAP \cite{frye20} and SAGE \cite{covert2020understanding}. Also, in the future, we aim to explore more expressive distribution classes and relaxed distributional assumptions—extending beyond those examined in Section \ref{sec:tractable} —that still yield tractable SHAP computation. Finally, when exact computation proves intractable (Section \ref{sec:intractable}), it is worthwhile to theoretically investigate the question of the approximability of computing the SHAP metrics across various configurations, through the lens of approximation and parametrized complexity theory \cite{arora2009computational}.

%This paper aims to deepen our understanding of the computational complexity involved in obtaining different Shapley value variants. We found that for a variety of ML models, including decision trees, tree ensembles for regression, weighted automata, and linear regression models — computing both local and global interventional and baseline SHAP can be done in polynomial time when distributions are modeled by HMMs. This extends the distributional scope of popular algorithms like TreeSHAP, which is limited to empirical distributions. Additionally, we demonstrate a strict complexity gap between SHAP variants, showing that interventional and baseline SHAP can be strictly easier to compute than conditional SHAP. Despite these positive results, we uncovered intractability for various SHAP variants in neural networks and tree ensembles. Finally, we provided generalized complexity relations across SHAP variants. We believe that our framework offers a deeper understanding of the complexity involved in computing SHAP across various variants, models, distributions, as well as in both local and global computations, laying the groundwork for future research.

\bibliographystyle{ACM-Reference-Format}
\bibliography{refs}



\end{document}
\endinput
%%
%% End of file `sample-sigconf.tex'.
