\section{Verde: Dispute resolution for neural networks}
\label{sec:game}

Consider a client delegating the same ML program (e.g. training, inference) to $k>1$ compute providers, whom we'll generically call `\servers'.
Each \server responds with a claimed output (or a binding commitment to an output).
If all \servers return the same output, then there is no need for dispute resolution.
If the \servers disagree on the output, they'll engage in a dispute resolution protocol with a neutral referee (which could be the client themselves).
For simplicity, we assume all $k$ \servers claims a different output. Any \servers claiming the same output can be merged into one.
At the end of the protocol, the referee selects one of the $k$ outputs to accept. 

The security guarantee of dispute resolution is as follows:
If there is an honest \server, their output is guaranteed to be the one accepted.
The referee will identify $k-1$ dishonest \servers, proving that they reported incorrect output.
An important limitation is that if \emph{all} \servers are dishonest, the protocol will still identify $k-1$ dishonest providers, but the referee will also an incorrect output and not find proof of incorrectness for one dishonest provider.



{\bf Program setup:} 
The client specifies a neural network with initial weights, training data, and training metadata 
  like the optimizer and global batch size. 
The model is a topologically-sorted computational graph, 
  consisting of operators (using standardized formats like \onnx\cite{onnx2017}). 
A ``training step'' comprises a forward pass, backward pass, parameter updates 
  and an optimizer state update. 

For simplicity of presentation, we consider only the case of $k=2$ \servers.\footnote{Extending to $k>2$ \servers can be achieved trivially by repeating the 2-\server case iteratively or in a hierarchical tournament. More efficient combination is possible by combining some steps.}
We assume bitwise reproducibility between the referee and all \servers, ensuring that honest execution 
always produces the same output (Section~\ref{sec:repops} explains how \repops enables this 
on different hardware setups).
The referee has access to the computational graph defining the model, the initial parameters, and the training data. 
However, they are assumed to be computationally limited, and unable to train the model themselves. 

During dispute resolution, the referee interacts with the \servers, 
analyzing commitments to their intermediate computation states to narrow down the 
dispute to a smaller segment of the program.
In adapting this to neural network training, we design 
a protocol that proceeds in two phases.
\begin{enumerate}
  \item {\bf Phase 1} identifies the exact training step 
    that the \trainers first diverged at.
  \item {\bf Phase 2} identifies the first operator in the computational graph
    defining the model where they diverge. 
\end{enumerate}

Sections \ref{sec:bisection-phase-1} and \ref{sec:bisection-phase-2} 
describe the two phases, respectively.
While we focus on training, our techniques immediately extend 
to inference. 

\subsection{Phase 1: Identifying the diverging training step.}
\label{sec:bisection-phase-1}

A training program specifies the neural network as a computational graph 
with a training dataset and fixed training metadata. 
We start by abstracting the training procedure as a state machine, where the 
``state'' is the values of the learnable parameters and optimizer state 
and the ``transition function'' is each training step 
performing a forward pass, backward pass, and optimizer state update. 
At each training step, we derive a ``checkpoint'' that includes the state and 
can be \emph{committed} using a cryptographically binding commitment scheme. 
We define a precise checkpoint and commitment format later, 
but it suffices for now to assume that the checkpoint consists of just the aforementioned state 
and is committed to using a standard collision-resistant hash function like SHA-256. 

The training program transcript can thus be viewed as a sequence of checkpoints 
$C_0 \rightarrow C_1 \rightarrow \ldots \rightarrow C_n$. 
If two trainers start with $C_0$ 
but produce different outputs after $n$ steps, $C_n$ and $C_n'$, 
then there must be a training step $i$ where they first ``diverged'':
that is, they claim different transitions 
$C_i \rightarrow C_{i+1}$ and $C_i \rightarrow C_{i+1}'$, 
respectively, at step $i$. 

{\bf A first attempt.} 
The naive way for the referee to identify this diverging step is to 
receive from each \trainer the 
hashes of all their checkpoints, and then find the diverging checkpoint with a linear search.
The prohibitive issue here is that hashing 
the model state after every single step can be very expensive for large models.
For example, hashing the weights and Adam optimizer state~\cite{kingma14adam}
(the optimizer state is double the size of the weights alone)
in FP32 precision for 
\distilbert (66 million parameters) takes under a second, 
for \llama-1B takes around 2.5 seconds, and for \llama-8B around 15 seconds, 
on a CPU with 128 GB RAM and the Apple M3 chip.

\newcommand{\starthash}{\mathtt{h_{start}}\xspace}
\newcommand{\hend}{\mathtt{{h}_{end}}\xspace}
\newcommand{\disphash}{\mathtt{{h}_{end}}\xspace}
\newcommand{\disphashes}{\mathtt{\overline{h}_{end}}\xspace}

\begin{algorithm}
  \begin{algorithmic}[1]
\STATE {\bf Participants}: Trainers $T_0, T_1$, Referee $R$
\STATE {\bf Inputs}: Starting checkpoint $C_0$, with hash $\starthash$. 
\STATE {\bf Parameter}: Checkpoint counts $k_0, k_1, k_2, \ldots$
\STATE Trainers send hash of output checkpoint to referee.
\STATE \textbullet\ $T_0\rightarrow R$: $\hend_1$
\STATE \textbullet\ $T_1\rightarrow R$: $\hend_2$
\STATE $R$: If $\hend_1 == \hend_2$, then output no dispute.
\STATE Let $\disphashes = [\hend_1, \hend_2]$
\FOR{$\ell$ = 0, 1, \ldots}
  \STATE Trainers send $k_\ell$ checkpoint hashes to $R$: 
  \STATE \textbullet\ $T_0\rightarrow R$: $\starthash, \tth_1, \ldots, \disphashes[0]$
  \STATE \textbullet\ $T_1\rightarrow R$: $\starthash, \tth_1', \ldots, \disphashes[1]$
  \STATE Referee identifies diverging checkpoint: 
  \STATE $R \rightarrow T_1, T_2$: $d$ s.t. $\tth_j = \tth_j'$ $\forall\ $ $j\leq d \land \tth_{d+1} \neq \tth_{d+1}'$
  \STATE If checkpoint interval length $> 1$: 
  \STATE \textbullet\ Repeat with $\starthash \leftarrow \tth_d$ and $\disphashes \leftarrow [\tth_{d+1}, \tth_{d+1}']$
  \STATE Else: exit loop.
\ENDFOR
\STATE $R$: output $\starthash, \disphashes$ 
as the starting hash and disputed ending hashes of the first diverging training step. 
\end{algorithmic}
\caption{\underline{Dispute Resolution Phase 1:} The \trainers and the referee 
identify the first step the two \trainers diverged at in their executions.}
\label{alg:protocol_phase_1}
\end{algorithm}



{\bf Optimization: multi-level checkpointing.}
We identify the diverging step by narrowing down the interval it occurs in 
through multiple rounds of interaction (see Algorithm~\ref{sec:bisection-phase-1})
During training, the \trainers log checkpoints only at specified steps, instead of every step.
Upon receiving these initial checkpoints sequences, 
the referee identifies the first diverging checkpoint as before.\footnote{Note that we eschew the binary search of Canetti et al., as the number of checkpoint commitments is so low ($N<100$) it is more efficient in practice to send all of them in a single round.}
Then, the parties recurse: they re-run the diverging segment of training and 
log more granular checkpoints within.
These are again sent to the referee 
who identifies and shares the new diverging segment. 
This repeats until the granularity of the checkpoints is a single training step.

\textbf{Communication and storage costs.}
As only short hashes are communicated at this stage (and not the checkpoints themselves), 
the communication overhead is minimal for all parties. 
There is a trade-off between the checkpointing frequency (and thus, the hashing and 
storage requirements) during training and the amount of re-execution needed by the \trainers 
after training during dispute resolution. 
For example, if the trainers log $N$ initial checkpoints 
at each level, 
then the \trainers re-execute a $(\frac{1}{N} + \frac{1}{N^2} + \frac{1}{N^3} + \ldots)$ 
fraction of the original training program. When $N=20$, this 
comes to under $6\%$. 
When considering just the learnable parameters of \llama-8B with precision FP32, 
this requires a few hundred gigabytes of storage.
With $N=100$, the amount of re-execution reduces to under $1.1\%$ but 
the storage requirements reaches a few terabytes. 
As mentioned earlier, the time taken to hash all of the checkpoints is 
negligible compared to the training program itself (and can be done in parallel). 




\subsection{Phase 2: Narrowing down the diverging operation.}
\label{sec:bisection-phase-2}

At the end of Phase 1, 
the referee identifies the diverging training step, along with the 
starting commitment of the step (common to \trainers) and the 
two disputed ending commitments. 
To decide which of the two \trainers' claims is dishonest, 
the referee could re-run the entire training step themselves. 
This requires them to receive the starting checkpoint state from either \trainer,
perform the step, 
and compare the hash of the output with that of the two \trainers.
However, this incurs high communication and computation costs for the referee. 
A checkpoint itself can be many gigabytes and 
a training step requires a peak memory usage larger 
than the checkpoint itself. 

To improve on this, we observe that the work done by a training step function is defined as a computational graph.
We can narrow down the dispute further to the 
the first \emph{operator} in the graph 
the two \trainers diverged at. 
Resolving the output of just a single operator can reduce 
the communication and compute requirements of the referee
by two orders of magnitude. 
For example, 
the largest operator involved in most transformer-based models, 
the key-query matrix multiplication in the self-attention layer, 
can be represented as a series of 
matrix-vector multiplication operators, each of which require 
memory in the order of dozens of megabytes even for large sequence lengths. 

\textbf{Computational graph and node representations.}
We start with the representation of neural networks as a directed acyclic computational 
graph and extend it to capture all the work done in a training step: 
the forward pass, backward pass, and optimizer state updates.
This extended graph can be implicitly derived 
from the computational graph representing the 
forward pass of the model, such as in a format like \onnx, 
and automatic differentiation library like \autograd.
An example of this expanded graph is shown in Figure \ref{fig:extended-graph}
for a neural network with one operator. 
Note that we topologically sort the graph to ensure a common order for all parties. 

\begin{figure}[t]
  \centering
  \includegraphics[width=0.35\textwidth]{figures/extended-graph.pdf}
  \caption{Extended computational graph for a neural network with a single operator. 
Yellow represents nodes that initialize tensor values from either the training data or the 
a training checkpoint. 
    Blue nodes are forward pass operators, and red ones are backward pass operators. 
For clarity, we label the edge transferring context (also called ``saved tensors''' in autograd) 
from the forward pass to the corresponding backward pass operator.
  In the dispute resolution algorithm, these nodes are specified as \cgraphnode objects. 
} 
  \label{fig:extended-graph}
\end{figure}

We augment these nodes with information 
allowing the referee to later resolve disputes without needing to 
run other parts of the graph.
On top of information about the node's connections (inputs nodes, output nodes), 
the operation (operator and attribute details), 
we add the hashes of all tensors sent into and emitted out of the node.

\begin{tcolorbox}[colback=azure!5!white, colframe=azure!75!black, title={\cgraphnode}]
\textbullet\ Input node pointers\\
\textbullet\ Output node pointers\\
\textbullet\ Operator and attributes\\
\textbullet\ {\bf Input tensor hashes}: list of hashes of the tensors input to this node \\
\textbullet\ {\bf Output tensor hashes}: list of hashes of the tensors output from this node 
\end{tcolorbox}

If the \trainers start with the same checkpoint 
(which we know to be the case after Phase 1) at the start of a training step 
but finish with different weights or optimizer state, then there must be a node 
in the above computational graph that the \trainers start with the same inputs 
but compute different outputs for. 
Similar to Algorithm~\ref{alg:protocol_phase_1}, we can identify this 
node by comparing the sequence of hashes of the nodes produced by each \trainer. 
When performing a training step during dispute resolution,  
the \trainers populate and hash these nodes, and send the sequence to the 
referee. 
Algorithm~\ref{alg:protocol_phase_2} shows the full protocol.

\begin{algorithm}
  \begin{algorithmic}[1]
\STATE {\bf Participants}: Trainers $T_0, T_1$, Referee $R$
\STATE {\bf Inputs}: checkpoint commitments $\starthash$, $\disphashes$ 
  \STATE $T_1, T_2$ send node hashes to $R$. 
  \STATE \quad\textbullet\ $T_1$ sends ${\tt seq}_0 = \tth_1, \tth_2, \ldots, \tth_k$
  \STATE \quad\textbullet\ $T_2$ sends ${\tt seq}_1 = \tth_1', \tth_2', \ldots, \tth_k$
  \STATE $R$ verifies consistency of the claimed checkpoint states:
  \STATE \quad\textbullet\ $\disphashes[i] \stackrel{?}{=} \text{MerkleHash}({\tt seq}_i)$ for $i=0, 1$
  \STATE $R$ finds and sends the first node index $d$ such that: 
  \STATE \quad\textbullet\ $\tth_j = \tth_j'$ for $j<d$ but $\tth_{d} \neq \tth_{d}'$
  \STATE $T_1, T_2$ open their $d^{th}$ nodes by sending \cgraphnode $\node_0, \node_1$, respectively.  
  \STATE Referee runs decision algorithm using $\node_0, \node_1$. 
  \end{algorithmic}
\caption{\uline{Dispute Resolution Phase 2.} The \trainers and the referee 
identify the first node in the computational graph of the disputed step 
the two \trainers diverged at in their executions.}
\label{alg:protocol_phase_2}
\end{algorithm}



\paragraph{Checkpoint hash format.}

We can now specify the precise checkpoint format used in Phase 1. 
To ensure continuity between the \trainers claims in Phase 1 and 2, 
the checkpoint after a given training step is the Merkle tree 
with the leaves being the 
\emph{hash of all the {\cgraphnode} nodes} 
of the training step leading to the checkpoint, as shown in Figure~\ref{fig:checkpoint-merkle-tree}.  
As these contain hashes of all updates performed in the step, they serve as a 
commitment to the new training state. 
Importantly, 
they disallow a trainer from using inconsistent commitments between Phase 1 and Phase 2. 
The referee ensures that the sequence of node hashes sent by the 
\trainers in Phase 2 conform exactly to the 
values of $\disphash$ obtained in Phase 1 
(line \textred{7} in Algorithm~\ref{alg:protocol_phase_2}). 

\begin{figure}[h]
  \centering
  \includegraphics[width=0.4\textwidth]{figures/checkpoint_commitment.pdf}
  \caption{The nodes of the computational graph of the latest training step 
    serve as the checkpoint used in Phase 1. 
    It's committed to using a Merkle (binary hash) tree 
    and verified by the referee in Phase 2 (line {7}).   
    Merkle trees provide efficient proofs of membership for its leaves,
    facilitating efficient dispute resolution when \trainers disagree on the 
    values from the weights, optimizer state, or training data. 
  }
  \label{fig:checkpoint-merkle-tree}
\end{figure}



\subsection{Referee's decision algorithm.}
\label{sec:decision}

Finally, we outline the algorithm used by the referee to determine the dishonest party 
in the dispute.
At the end of Phase 2, the referee contains objects $\node_0, \node_1$ representing 
\cgraphnode nodes in the extended computational graph.
As the node hashes are different, there must differ in at least one of the fields 
specified in the \cgraphnode definition. 
As we enforce deterministic execution and control for 
randomness, at most one node can represent a valid execution.
The referee compares the fields within the nodes and determines which of them 
represents a valid execution. 

Depending on how $\node_0$ and $\node_1$ differ, the referee resolves the 
dispute. There are 3 main cases, depending on what differs between the nodes.
\begin{enumerate}
  \item {\bf Case 1:} Difference in graph structure: \texttt{inputs, outputs, operator}  
  \item {\bf Case 2:} Difference in an input tensor hash.
  \item {\bf Case 3:} Difference in an output tensor hash.
\end{enumerate}

We provide brief descriptions of how each case is handled. 
{\bf Case 1} is the easiest 
to resolve as it means one \trainer constructed the graph incorrectly.
The referee, knowing the model specified by the client, can check which structure is correct. 

{\bf Case 2} involves multiple sub-cases 
depending on whether the input tensor corresponding to the differing hash 
is from from (a) the starting checkpoint (such as a weight or 
optimizer state value) or from (b) another node in the graph.
If (a), then the referee requests a \emph{Merkle membership proof} 
of the input hash in the checkpoint from the two \trainers. 
This short and easy-to-check proof can only be produced by the honest \trainer.
If (b), the referee asks for the 
source node's opening from the trainers to check which of the disputed tensor hash 
is present as an output tensor hash of that node.
As nodes $\node_0, \node_1$ are the first diverging node, 
    they agree on source node and in particular, the \texttt{emitted tensor hashes} in it. 
    The disputed input tensor hash must be an emitted tensor hash in the other node. 

{\bf Case 3} implies that the two trainers claim different emitted tensor values. 
This is the only scenario where the referee needs to run the operator 
and decide which \trainer is correct. 
This can be done in multiple ways, depending on the operator and 
compute capabilities of the referee. 
The referee can recompute the operator themselves by receiving from the trainers 
  the input tensors of the node. 
We can also further decompose the operator into smaller operators and perform more phases 
  to narrow down the dispute arbitrarily. 
  







