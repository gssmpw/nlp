\appendix

\section{Datasets}
Natural Language Understanding (NLU) Dataset information is detailed in Table~\ref{tab:dataset}. All NLU datasets are loaded from HuggingFace Hub. For most NLU datasets, we report the results on the test set; while for the datasets MNLI and QNLI, we report the results on the validation set due to restricted access to their test sets. 

\begin{table*}[t]
    \centering
    \resizebox{0.8\linewidth}{!}{
    \begin{tabular}{lcccc}
    \toprule
    \textbf{Dataset} & \textbf{Task} & \textbf{\# of Classes} & \textbf{Data Split} \\ \midrule
     \textbf{SST-2}  &  Sentiment Classification & 2 & 6920/872/1821 \\
     \textbf{SST-5}  &  Sentiment Classification & 5 & 8544/1101/2210 \\
     \textbf{CR}  &  Sentiment Classification & 2 & 3394/0/376 \\
     \textbf{Subj}  &  Subjectivity Analysis & 2 & 8000/0/2000 \\
     \textbf{AgNews} &  Topic Classification & 4 & 120000/0/7600 \\
     \textbf{MNLI}   &  Natural Language Inference & 3 & 392702/19647/19643 \\
      \textbf{QNLI}  &  Natural Language Inference & 2 & 104743/5463/5463 \\
      \bottomrule
      \end{tabular}}
    \caption{\textbf{Details of NLU datasets.}}
    \label{tab:dataset}
\end{table*}

\section{Templates}
\begin{table*}[t ]
\centering
\resizebox{0.8\linewidth}{!}{
\begin{tabular}{lll}
\toprule
\textbf{Task} & \textbf{Prompt} & \textbf{Class} \\
\hline
\multirow{2}{*}{SST-2} & Review: "<X>" Sentiment: positive & positive \\ 
& Review: "<X>" Sentiment: negative & negative \\
\midrule
\multirow{5}{*}{SST-5}
& Review: "<X>" Sentiment: terrible  & terrible \\ 
& Review: "<X>" Sentiment: bad  & bad \\ 
& Review: "<X>" Sentiment: okay  & okay \\ 
& Review: "<X>" Sentiment: good  & good \\ 
& Review: "<X>" Sentiment: great  & great \\ 
\midrule
\multirow{2}{*}{Subj}
& Input: "<X>" Type: objective  & objective \\ 
& Input: "<X>" Type: subjective  & subjective \\ 
\midrule
\multirow{2}{*}{CR} & Review: "<X>" Sentiment: positive & positive \\ 
& Review: "<X>" Sentiment: negative & negative \\
\midrule
\multirow{4}{*}{AgNews}
& Input: "<X>" Type: world  & World \\ 
& Input: "<X>" Type: sports  & Sports \\ 
& Input: "<X>" Type: business  & Business \\ 
& Input: "<X>" Type: technology  & Sci/Tech \\ 
\midrule
\multirow{3}{*}{MNLI}
& Premise: <C> Hypothesis: <X> Prediction: entailment  & Entailment \\ 
& Premise: <C> Hypothesis: <X> Prediction: neutral  & Neutral \\ 
& Premise: <C> Hypothesis: <X>? Prediction: contradiction  & Contradiction \\ 
\midrule
\multirow{2}{*}{QNLI}
& <C> Can we know <X>? Yes.  & Entailment \\ 
& <C> Can we know <X>? No.  & Contradiction \\ 
\midrule
\end{tabular}
}
\caption{\textbf{Templates of NLU tasks.} Placeholders (e.g., <X> and <C>) will be replaced by real inputs.}
\label{tab:templates}
\end{table*}

The templates of NLU tasks used in this paper are detailed in Table~\ref{tab:templates}.
