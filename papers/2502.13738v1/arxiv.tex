% This must be in the first 5 lines to tell arXiv to use pdfLaTeX, which is strongly recommended.
\pdfoutput=1
% In particular, the hyperref package requires pdfLaTeX in order to break URLs across lines.

\documentclass[11pt]{article}

% Change "review" to "final" to generate the final (sometimes called camera-ready) version.
% Change to "preprint" to generate a non-anonymous version with page numbers.
% \usepackage[review]{acl}
\usepackage[]{acl}

% Standard package includes
\usepackage{times}
\usepackage{latexsym}

% For proper rendering and hyphenation of words containing Latin characters (including in bib files)
\usepackage[T1]{fontenc}
% For Vietnamese characters
% \usepackage[T5]{fontenc}
% See https://www.latex-project.org/help/documentation/encguide.pdf for other character sets

% This assumes your files are encoded as UTF8
\usepackage[utf8]{inputenc}
% This is not strictly necessary, and may be commented out,
% but it will improve the layout of the manuscript,
% and will typically save some space.
\usepackage{microtype}
\usepackage{arydshln}

% This is also not strictly necessary, and may be commented out.
% However, it will improve the aesthetics of text in
% the typewriter font.
\usepackage{inconsolata}

\usepackage{color}
\usepackage{multirow}
\usepackage{booktabs}
\usepackage{amsmath}

%Including images in your LaTeX document requires adding
%additional package(s)
\usepackage{graphicx}

% If the title and author information does not fit in the area allocated, uncomment the following
%
%\setlength\titlebox{<dim>}
%
% and set <dim> to something 5cm or larger.

\title{Enhancing Input-Label Mapping in In-Context Learning with\\ Contrastive Decoding}


\author{%
  Keqin Peng$^{1}$,
  Liang Ding$^{2}$,
  Yuanxin Ouyang$^{1}$,
  Meng Fang$^{3}$,
  Yancheng Yuan$^{4}$,
  Dacheng Tao$^{5}$\\
  $^{1}$Beihang University $^{2}$The University of Sydney $^{3}$University of Liverpool\\
  $^{4}$The Hong Kong Polytechnic University $^{5}$Nanyang Technological University\\
  \texttt{keqin.peng@buaa.edu.cn},
  \texttt{liangding.liam@gmail.com}}
  

\begin{document}
\maketitle
\begin{abstract}
Large language models (LLMs) excel at a range of tasks through in-context learning (ICL), where only a few task examples guide their predictions. However, prior research highlights that LLMs often overlook input-label mapping information in ICL, relying more on their pre-trained knowledge. To address this issue, we introduce In-Context Contrastive Decoding (ICCD), a novel method that emphasizes input-label mapping by contrasting the output distributions between positive and negative in-context examples. Experiments on 7 natural language understanding (NLU) tasks show that our ICCD method brings consistent and significant improvement (up to +2.1 improvement on average) upon 6 different scales of LLMs without requiring additional training. Our approach is versatile, enhancing performance with various demonstration selection methods, demonstrating its broad applicability and effectiveness. The code and scripts will be publicly released.

\end{abstract}

\section{Introduction}
In-context learning (ICL,~\citealp[]{DBLP:conf/nips/BrownMRSKDNSSAA20}) is one of the most remarkable emergent capabilities of large language models (LLMs,~\citealp[]{achiam2023gpt,dubey2024llama}). By leveraging just a few carefully selected input-output examples, ICL enables models to adapt to new tasks without parameter updating~\cite{dong2022survey,peng-etal-2024-revisiting}. This approach has proven highly effective in unlocking the advanced capabilities of LLMs and has become a standard technique for tackling a spectrum of tasks, like translation, coding, and reasoning~\cite{peng2023towards,wang2025leveraging,wibisono2024context}.

Previous studies~\cite{pan-etal-2023-context,wei2023larger} have identified two critical factors for successful ICL: \textit{task recognition (TR)}, which involves identifying the task from the demonstrations and utilizing prior knowledge to make predictions, and \textit{task learning (TL)}, which focuses on directly learning the input-label mappings from the demonstrations. However, ICL faces challenges in overcoming the biases introduced by pretraining~\cite{kossen2024context}, and LLMs tend to underutilize input-label mapping information~\cite{min-etal-2022-rethinking}. For example, in tasks like SST-2~\cite{socher2013recursive}, the model may default to using its internal knowledge rather than learning the specific input-label mappings provided in the context.

To address this issue, we propose a simple yet effective method called \textit{in-context contrastive decoding} (ICCD). Our method is inspired by the contrastive decoding technique~\cite{li-etal-2023-contrastive,sennrich2024mitigating,kim2024instructive,zhong-etal-2024-rose,wang2024mathbb}, which increases the probability of the desired output by suppressing undesired outputs, and our ICCD enhances the model’s attention to input-label mapping during generation. Specifically, we construct negative in-context examples by altering the inputs of the demonstrations, creating incorrect input-label mappings while keeping the labels unchanged. By comparing the output distributions between positive and negative examples, ICCD effectively emphasizes the correct input-label mappings, integrating this information into the original ICL process. Notably, our method works with any pretrained LLMs without requiring additional training.
\textbf{Experimental results} across seven NLU tasks demonstrate that our ICCD consistently and significantly improves model performance across various datasets and model sizes. Moreover, ICCD can be seamlessly integrated with different demonstration selection methods, showcasing its robustness and universal applicability.

\section{Methodology}
\subsection{Background} 
Given an input query $x$, the probability of generating the target $y$ using a casual LLM $M$ parameterized by $\theta$ can be formulated as follows:
\begin{equation}
    y \sim p_\theta(y \mid \boldsymbol{c}, \mathcal{T}(x)),
    %p(y|\mathbf{x}) = \mathcal{P}\left(\mathcal{V}(y) | c, \mathcal{T}(\mathbf{x})\right),    
\end{equation}
where $\mathcal{T(\cdot)}$ is the template used to wrap up inputs and $c = \mathcal{T}(x_1),\cdots, \mathcal{T}(x_k)$ is the context string concatenating $k$ in-context examples, $p_\theta(y \mid \boldsymbol{c}, \mathcal{T}(x))= \operatorname{softmax}[\ \operatorname{logit}_\theta(y \mid \boldsymbol{c}, \mathcal{T}(x))]$ is the probability for the predicted token. For obtaining the desired $y$, the regular decoding method is to choose the token with the highest probability (\textit{i.e.}, greedy decoding) or sampling from its distribution (\textit{e.g.}, top-k decoding).

Here, we can observe that there are two kinds of knowledge contributing to model prediction, models' prior knowledge and input-label mapping information in in-context learning. However, LLMs usually prioritize prior knowledge over input-label mapping information~\cite{kossen2024context}, leading to ICL's struggle to fully overcome prediction preferences acquired from pre-training.

\subsection{In-Context Contrastive Decoding} 
\label{subsec:cd}
To mitigate the issue above, we construct negative in-context examples to factor out the input-label mapping from the models' original output distribution contrastively. Specifically, in addition to the origin in-context examples $\boldsymbol{c}$, we construct negative in-context examples $\boldsymbol{c}^-$ with incorrect input-label mapping. We then subtract the negative output $\boldsymbol{\mathrm{z}}_t$ from the positive output $\boldsymbol{\mathrm{z}}_{t}^{-}$ to isolate the knowledge of input-label mapping. Finally, we integrate this knowledge with the original in-context learning to reinforce the importance of input-label mapping:
\begin{equation}
\label{eq:log}
y_t \sim \mathrm{softmax}(\boldsymbol{\mathrm{z}}_t + \alpha (\boldsymbol{\mathrm{z}}_{t} - \boldsymbol{\mathrm{z}}_{t}^{-}))\text{,}
\end{equation}
where $\alpha$ is a hyperparameter that governs the importance of input-label mapping information. Equivalently,
\begin{align}
    y_t &\sim \tilde p_{\theta}(y | \boldsymbol{c},\boldsymbol{c}^-,\mathcal{T}(x)) \\
    &\propto p_{\theta}(y | \boldsymbol{c},\mathcal{T}(x))\left( \frac{ p_{\theta}(y | \boldsymbol{c},\mathcal{T}(x)}{ p_{\theta}(y | \boldsymbol{c}^{-},\mathcal{T}(x))}\right)^{\alpha}\text{.}
\end{align}

\paragraph{Construction of $\boldsymbol{c}^-$.} The negative in-context examples $\boldsymbol{c}^-$ is the key to the success of the in-context contrastive decoding method (ICCD). Considering the label bias~\cite{zhao2021calibrate} of in-context learning, directly altering the labels of demonstrations may introduce a completely different label bias, potentially distorting the input-label mapping information. Hence, we adjust the inputs instead of the labels to change input-label mapping information. Specifically, for each demonstration ${(x_i, y_i)}$, we first randomly select a different label $y_j(y_j \neq y_i)$ from the label space. Then we randomly choose an input $x_j$ whose label is $y_j$ from the demonstrations pool to construct the negative demonstration ${(x_j, y_i)}$. We compare the effect of different $\boldsymbol{c}^-$ in Section~\ref{subsect:analysis}.

\section{Experimental Setup}
\paragraph{Models and Baselines.} We perform experiments across different sizes of models, including Llama-series: Llama3.2-1B (1B), Llama3.2-3B (3B) and Llama3.1-8B (8B) ~\cite{dubey2024llama} and Qwen2 series: Qwen2-0.5B (0.5B), Qwen2-1.5B (1.5B) and Qwen2-7B (7B)~\cite{yang2024qwen2technicalreport}, which are all widely-used decoder-only dense LMs. We also conduct experiments on extensive alignment models, e.g., Llama3.2-1B-Instruct, Llama3.2-3B-Instruct, and Llama3.1-8B-Instruct~\cite{dubey2024llama} to verify the generalizability of our approach. For the baseline, we use the regular decoding methods following prior work~\cite{shi-etal-2024-trusting,zhao-etal-2024-enhancing}. 

\paragraph{Demonstration Selection methods.} To verify that our method is complementary to different demonstration selection methods, we mainly consider three different demonstration selection methods that do not require additional training.
\begin{itemize}
 \item \textbf{Random} baseline randomly select in context examples for each testing sample.
  \item \textbf{BM25}~\cite{robertson2009probabilistic} baseline uses BM25 to calculate the word-overlap similarity between samples and test input and select the high-similarity samples as demonstrations.
  \item \textbf{TopK}~\cite{liu2022makes} baseline uses the nearest neighbors of a given test sample as the corresponding in-context examples.
\end{itemize}

\paragraph{Datasets and Metrics.} We conduct a systematic study across 7 NLU tasks, including binary, multi-class classification tasks (SST-2, SST-5~\cite{socher-etal-2013-recursive}, CR, Subj~\cite{wang-etal-2018-glue}) and natural language inference tasks: MNLI~\cite{williams-etal-2018-broad} and QNLI~\cite{wang-etal-2018-glue}. We will report the accuracy to show the performance.

\paragraph{Experimental Details.} Our method introduces a hyperparameter $\alpha$ to control the input-label mapping information. For simplicity, we set $\alpha=1$ for all models and settings. We ran all experiments 3 times with different random seeds and reported the average accuracies. We use 16-shot ICL for all models. Without a special statement, we report the results of the random selection method.
% 需要补充一些细节

\section{Main Results}
 We demonstrate the effectiveness of our method in 7 NLU tasks described in the Datasets and Metrics section. We summarize the results in Table~\ref{tab:ret}, Table~\ref{tab:task_model}, and Figure~\ref{fig:ins}. Based on the results, we can find that: 
\paragraph{Our method brings gain across different tasks and model scales.} Results on Table~\ref{tab:task_model} show that our method can achieve consistently better performance across the majority of tasks under different model scales than the regular decoding method. Specifically, our method brings over 1.5 improvements (in accuracy) in all Llama-series models and Qwen2-series models. It's worth highlighting that ICCD brings +3.1 gains on average in the Qwen2-1.5B model. 
Furthermore, it is noteworthy that our approach can achieve more significant improvements in challenging tasks with the increase of model scale, such as QNLI and MNLI tasks, respectively bringing 5.6\% (2.0\%) and 1.7\% (1.8\%) gains compared to regular decoding in Llama3.1-8B (Qwen2-7B), demonstrating the effectiveness and universality of our method. 

\paragraph{Our method consistently improves the performance with different in-context examples selection methods.} Table~\ref{tab:ret} lists the performance of different models with different demonstration selection methods. Clearly, our method can achieve better performance with different demonstration selection methods. When the model scale increases, our method can achieve more improvement gains compared to the regular decoding method, +0.5 and +1.1 with BM25 method under Llama3.2-3B and Llama3.1-8B, respectively. These results prove that ICCD can be complementary with different demonstration selection methods.

\paragraph{Our method works for aligned chat models.} To verify the effectiveness of our method for the chat
LLMs, we conducted experiments on different instruction-tuned and RLHF-tuned LLMs. Figure~\ref{fig:ins} show that our method can achieve consistent improvement in different chat models, demonstrating that our method also works
for instruction-tuned and safety-enhanced models.

\begin{table}
\centering
\scalebox{0.75}{
\begin{tabular}{ccccc} 
\toprule
\multirow{2}{*}{\textbf{Model}}       & \multirow{2}{*}{\textbf{Decoding}} & \multicolumn{3}{c}{\textbf{Method}}                                                               \\ 
\cmidrule(lr){3-5}
                                      &                                    & \textbf{Random}                & \textbf{BM25}                  & \textbf{TopK}                   \\ 
\midrule
\multirow{2}{*}{\textbf{Llama3.2-1B}} & Regular                            & 66.1                           & 72.5                               & 73.6                                \\
                                      & Ours                               & \textcolor{red}{\textbf{68.5}} & \textcolor{red}{\textbf{73.0}}                               &  73.6                               \\ 
\midrule
\multirow{2}{*}{\textbf{Llama3.2-3B}} & Regular                            & 72.9                           & 76.6                           & 76.7                            \\
                                      & Ours                               & \textbf{\textcolor{red}{74.6}} & \textbf{\textcolor{red}{77.1}} & \textbf{\textcolor{red}{77.1}}  \\ 
\midrule
\multirow{2}{*}{\textbf{Llama3.1-8B}} & Regular                            & 77.6                           & 79.7                           & 80.2                            \\
                                      & Ours                               & \textbf{\textcolor{red}{79.4}} & \textbf{\textcolor{red}{80.8}} & \textbf{\textcolor{red}{80.9}}  \\
\bottomrule
\end{tabular}
}
\caption{\textbf{Average performance of 7 Natural Language Understanding (NLU) tasks with different in-context example selection methods.} \textbf{\textcolor{red}{Red}} results indicate that our method brings improvement over the regular decoding, while \textbf{\textcolor{green}{Green}} results denote no improvement.}
\label{tab:ret}
\end{table}

\begin{table*}
\centering
\scalebox{0.90}{
\begin{tabular}{cccccccccc} 
\toprule
\textbf{Model}                        & \textbf{Decoding} & \textbf{SST2}                    & \textbf{CR}                    & \textbf{SST5}                    & \textbf{Subj}                  & \textbf{QNLI}                  & \textbf{MNLI}                    & \textbf{AG\_NEWS}                & \textbf{\textit{Avg.}}                 \\ 
\midrule
\multirow{2}{*}{\textbf{Llama3.2-1B}} & Regular           & 89.8                             & 83.0                           & 43.7                             & 72.8                           & 53.5                           & 36.6                             & 83.3                             & 66.1                                   \\
                                      & Ours              & \textbf{\textcolor{red}{90.6}}   & \textbf{\textcolor{red}{83.5}} & \textbf{\textcolor{green}{43.6}} & \textbf{\textcolor{red}{84.9}} & \textbf{\textcolor{red}{54.1}} & \textbf{\textcolor{red}{39.0}}   & \textbf{\textcolor{red}{83.9}}   & \textbf{\textcolor{red}{68.5 (+2.4)}}  \\ 
\hdashline
\multirow{2}{*}{\textbf{Llama3.2-3B}} & Regular           & 93.7                             & 87.2                           & 46.2                             & 86.0                           & 54.2                           & 56.9                             & 86.4                             & 72.9                                   \\
                                      & Ours              & \textbf{\textcolor{red}{93.8}}   & \textbf{\textcolor{red}{89.4}} & \textbf{\textcolor{red}{47.2}}   & \textbf{\textcolor{red}{91.9}} & \textbf{\textcolor{red}{55.9}} & \textbf{\textcolor{red}{57.4}}   & \textbf{\textcolor{red}{86.7}}   & \textbf{\textcolor{red}{74.6 (+1.7)}}  \\ 
\hdashline
\multirow{2}{*}{\textbf{Llama3.1-8B}} & Regular           & 96.7                             & 92.3                           & 48.0                             & 94.0                           & 60.3                           & 65.3                             & 86.7                             & 77.6                                   \\
                                      & Ours              & \textbf{\textcolor{green}{96.6}} & \textcolor{red}{\textbf{93.1}} & \textcolor{red}{\textbf{49.5}}   & \textcolor{red}{\textbf{95.9}} & \textcolor{red}{\textbf{65.9}} & \textcolor{red}{\textbf{67.0}}   & \textcolor{red}{\textbf{87.6}}   & \textcolor{red}{\textbf{79.4 (+1.8)}}  \\ 
\midrule
\multirow{2}{*}{\textbf{Qwen2-0.5B}}  & Regular           & 87.9                             & 89.4                           & 34.5                             & 62.2                           & 52.5                           & 47.6                             & 78.1                             & 64.6                                   \\
                                      & Ours              & \textcolor{red}{\textbf{89.6}}   & \textcolor{red}{\textbf{89.6}} & \textcolor{red}{\textbf{35.0}}   & \textcolor{red}{\textbf{70.6}} & \textcolor{red}{\textbf{53.3}} & \textbf{\textcolor{green}{47.4}} & \textbf{\textcolor{green}{78.0}} & \textcolor{red}{\textbf{66.2 (+1.8)}}  \\ 
\hdashline
\multirow{2}{*}{\textbf{Qwen2-1.5B}}  & Regular           & 95.2                             & 91.0                           & 49.0                             & 72.3                           & 60.2                           & 61.8                             & 76.7                             & 72.3                                   \\
                                      & Ours              & \textbf{\textcolor{green}{95.1}} & \textbf{\textcolor{red}{91.5}} & \textbf{\textcolor{red}{49.3}}   & \textbf{\textcolor{red}{86.0}} & \textbf{\textcolor{red}{61.8}} & \textbf{\textcolor{red}{65.5}}   & \textbf{\textcolor{red}{78.5}}   & \textbf{\textcolor{red}{75.4 (+3.1)}}  \\ 
\hdashline
\multirow{2}{*}{\textbf{Qwen2-7B}}    & Regular           & 96.0                             & 91.5                           & 51.9                             & 82.3                           & 71.4                           & 78.7                             & 83.8                             & 79.4                                   \\
                                      & Ours              & \textcolor{red}{\textbf{96.3}}   & \textcolor{red}{\textbf{92.3}} & \textcolor{red}{\textbf{52.5}}   & \textcolor{red}{\textbf{89.5}} & \textcolor{red}{\textbf{73.4}} & \textcolor{red}{\textbf{80.5}}   & \textcolor{red}{\textbf{85.3}}   & \textcolor{red}{\textbf{81.4 (+2.0)}}  \\
\bottomrule
\end{tabular}
}
\caption{\textbf{Performance of different models across 7 Natural Language Understanding (NLU) tasks.} \textbf{\textcolor{red}{Red}} results indicate our method brings improvement over the regular decoding, while \textbf{\textcolor{green}{Green}} denote no improvement.}
\label{tab:task_model}
\end{table*}


\section{Analysis}
\label{subsect:analysis}
To further explore the impact of different factors on the effectiveness of our method, we conduct further analysis with the Llama3.2-8B models.

\paragraph{Effects of Different Negative In-context Examples.} As mentioned in Section \ref{subsec:cd}, the choice of negative in-context examples is important to the performance of our methods. Here, we conduct contrastive experiments to analyze the impact of different negative examples. Specifically, we refer to the selected negative examples as \textbf{Input}, if the input-label mapping is altered by modifying the inputs of the demonstrations. Additionally, we construct another variant, \textbf{Label}, in which the labels of the demonstrations are changed. For comparison, we also include \textbf{NULL}, which does not use any negative demonstrations, similar to ~\citet{shi-etal-2024-trusting}. The results in Table~\ref{tab:neg_exp} show that \textbf{Input} outperforms the other counterparts, thus leaving as our default setting in this work.

\begin{table}
\centering
\scalebox{0.95}{
\begin{tabular}{cccc} 
\toprule
\multirow{2}{*}{\textbf{Method}} & \multicolumn{3}{c}{\textbf{Selection Method}}                                                                             \\ 
\cmidrule(lr){2-4}
                                 & \textbf{Random}                        & \textbf{BM25}                          & \textbf{TopK}                           \\ 
\midrule
\textbf{Regular Decoding}        & 77.6                                   & 79.7                                   & 80.2                                    \\ 
\midrule
\multicolumn{4}{l}{\textbf{\textit{Equipped with our method}}}                                                                                               \\ 
\hdashline
\textbf{+NULL}                   & \textbf{\textcolor{green}{73.0}}       & \textcolor{green}{\textbf{75.8}}       & \textcolor{green}{76.5}                 \\
\textbf{+Label}                  & \textbf{\textcolor{green}{77.3}}                                       & \textbf{\textcolor{green}{79.5}}                                       &  \textbf{\textcolor{green}{80.0}}                                       \\
\textbf{+Input}                  & \textcolor{red}{\textbf{79.4}} & \textcolor{red}{\textbf{80.8}} & \textcolor{red}{\textbf{80.9}}  \\
\bottomrule
\end{tabular}}
\caption{\textbf{Average performance with different negative in-context examples.} \textbf{\textcolor{red}{Red}} results indicate that our method brings improvement over the regular decoding, while \textbf{\textcolor{green}{Green}} results denote no improvement.}
\label{tab:neg_exp}
\end{table}

\begin{figure}[t!]
    \centering
    \includegraphics[width=0.95\columnwidth]{figures/instruct.pdf}
    \caption{\textbf{Performance with different chat models.}}
    \label{fig:ins}
\end{figure}

\paragraph{Effects of Different number of shots.}
We gradually increase the number of in-context examples (denoted as N) from 1 to 16 to verify the influence of the number of shots in our method. Figure~\ref{fig:shots} reports the average performance of 7 NLU tasks and the different task QNLI. We see that our method can consistently outperform the regular decoding method with a different number of shots on average. For the task QNLI, as the number of shots increases, the performance gains of our method also improve. We attribute this to the model acquiring more input-label mapping information from the demonstrations, which aligns with previous findings~\cite{pan-etal-2023-context}.

\paragraph{Effects of $\alpha$.} The factor $\alpha$ in Eq.~\ref{eq:log}, which controls the importance of input-label mapping information, is an important hyper-parameter. In this part, we analyze its influence by evaluating the performance on SST5 and MNLI varying $\alpha$ from 0 to 2. The results on Table~\ref{tab:alpha} show that: 1) the performance improves with the increase of $\alpha$, and it becomes stable when $\alpha \geq 1.0$, we set $\alpha=1$ as default; 2) For advanced demonstration selection methods(e.g.TopK), too large positive $\alpha$ values lead to performance degradation. 

\begin{table}
\centering
\scalebox{0.86}{
\begin{tabular}{ccccccc} 
\toprule
\multirow{2}{*}{\textbf{Dataset}} & \multirow{2}{*}{\textbf{Method}} & \multicolumn{5}{c}{\textbf{$\alpha$}}                                                                              \\ 
\cmidrule(lr){3-7}
                                  &                                  & 0.0  & 0.5  & 1.0                            & 1.5                            & 2.0                             \\ 
\midrule
\multirow{3}{*}{\textbf{SST5}}    & \textbf{Random}                  & 48.0 & 49.0 & 49.5                           & 49.5                           & \textcolor{red}{\textbf{49.6}}  \\
                                  & \textbf{BM25}                    & 53.0 & 53.8 & 53.8                           & \textcolor{red}{\textbf{53.9}} & 53.8                            \\
                                  & \textbf{TopK}                    & 53.0 & 52.9 & \textbf{\textcolor{red}{53.0}} & 52.9                           & 52.8                            \\ 
\midrule
\multirow{3}{*}{\textbf{MNLI}}    & \textbf{Random}                  & 65.3 & 66.3 & 67.0                           & 67.2                           & \textbf{\textcolor{red}{67.3}}  \\
                                  & \textbf{BM25}                    & 65.8 & 66.1 & 66.6                           & 66.7                           & \textbf{\textcolor{red}{66.9}}  \\
                                  & \textbf{TopK}                    & 65.9 & 66.6 & \textbf{\textcolor{red}{67.0}} & 67.0                           & 66.9                            \\
\bottomrule
\end{tabular}
}
\caption{\textbf{The SST5 and MNLI performance with different $\alpha$.}}
\label{tab:alpha}
\end{table}

\begin{figure}[t!]
    \centering
    \includegraphics[width=1\columnwidth]{figures/shot.pdf}
    \caption{\textbf{The performance with different shots.}}
    \label{fig:shots}
\end{figure}
\section{Conclusion}
Large language models suffer from insufficient attention to the input-label mapping compared to their prior knowledge in in-context learning, leading to an unfaithful generation of the input query. In this work, we present a simple yet effective in-context contrastive decoding method that highlights input-label mapping by contrasting positive and negative in-context examples. Our experiments across various datasets and model architectures demonstrate the effectiveness and broad applicability of our approach, confirming its potential to enhance in-context learning.


\section*{Limitations}
While the results presented in this paper demonstrate the effectiveness of our In-Context Contrastive Decoding (ICCD) method, there are a few limitations that warrant future exploration. First, our experiments were conducted on models up to 8B parameters, primarily due to computational limitations. Extending our method to even larger models (e.g., 70B parameters) could provide further insights into its scalability and effectiveness. Second, while our method shows promise across various Natural Language Understanding (NLU) tasks, its performance in specialized domains, such as legal or medical texts, has yet to be thoroughly examined. Future work will explore the generalizability of ICCD to these domains, as well as investigate its interaction with domain-specific datasets. Additionally, while we focused on classification tasks, other NLP tasks like text generation and summarization remain unexplored.

\bibliography{arxiv}

\subsection{Lloyd-Max Algorithm}
\label{subsec:Lloyd-Max}
For a given quantization bitwidth $B$ and an operand $\bm{X}$, the Lloyd-Max algorithm finds $2^B$ quantization levels $\{\hat{x}_i\}_{i=1}^{2^B}$ such that quantizing $\bm{X}$ by rounding each scalar in $\bm{X}$ to the nearest quantization level minimizes the quantization MSE. 

The algorithm starts with an initial guess of quantization levels and then iteratively computes quantization thresholds $\{\tau_i\}_{i=1}^{2^B-1}$ and updates quantization levels $\{\hat{x}_i\}_{i=1}^{2^B}$. Specifically, at iteration $n$, thresholds are set to the midpoints of the previous iteration's levels:
\begin{align*}
    \tau_i^{(n)}=\frac{\hat{x}_i^{(n-1)}+\hat{x}_{i+1}^{(n-1)}}2 \text{ for } i=1\ldots 2^B-1
\end{align*}
Subsequently, the quantization levels are re-computed as conditional means of the data regions defined by the new thresholds:
\begin{align*}
    \hat{x}_i^{(n)}=\mathbb{E}\left[ \bm{X} \big| \bm{X}\in [\tau_{i-1}^{(n)},\tau_i^{(n)}] \right] \text{ for } i=1\ldots 2^B
\end{align*}
where to satisfy boundary conditions we have $\tau_0=-\infty$ and $\tau_{2^B}=\infty$. The algorithm iterates the above steps until convergence.

Figure \ref{fig:lm_quant} compares the quantization levels of a $7$-bit floating point (E3M3) quantizer (left) to a $7$-bit Lloyd-Max quantizer (right) when quantizing a layer of weights from the GPT3-126M model at a per-tensor granularity. As shown, the Lloyd-Max quantizer achieves substantially lower quantization MSE. Further, Table \ref{tab:FP7_vs_LM7} shows the superior perplexity achieved by Lloyd-Max quantizers for bitwidths of $7$, $6$ and $5$. The difference between the quantizers is clear at 5 bits, where per-tensor FP quantization incurs a drastic and unacceptable increase in perplexity, while Lloyd-Max quantization incurs a much smaller increase. Nevertheless, we note that even the optimal Lloyd-Max quantizer incurs a notable ($\sim 1.5$) increase in perplexity due to the coarse granularity of quantization. 

\begin{figure}[h]
  \centering
  \includegraphics[width=0.7\linewidth]{sections/figures/LM7_FP7.pdf}
  \caption{\small Quantization levels and the corresponding quantization MSE of Floating Point (left) vs Lloyd-Max (right) Quantizers for a layer of weights in the GPT3-126M model.}
  \label{fig:lm_quant}
\end{figure}

\begin{table}[h]\scriptsize
\begin{center}
\caption{\label{tab:FP7_vs_LM7} \small Comparing perplexity (lower is better) achieved by floating point quantizers and Lloyd-Max quantizers on a GPT3-126M model for the Wikitext-103 dataset.}
\begin{tabular}{c|cc|c}
\hline
 \multirow{2}{*}{\textbf{Bitwidth}} & \multicolumn{2}{|c|}{\textbf{Floating-Point Quantizer}} & \textbf{Lloyd-Max Quantizer} \\
 & Best Format & Wikitext-103 Perplexity & Wikitext-103 Perplexity \\
\hline
7 & E3M3 & 18.32 & 18.27 \\
6 & E3M2 & 19.07 & 18.51 \\
5 & E4M0 & 43.89 & 19.71 \\
\hline
\end{tabular}
\end{center}
\end{table}

\subsection{Proof of Local Optimality of LO-BCQ}
\label{subsec:lobcq_opt_proof}
For a given block $\bm{b}_j$, the quantization MSE during LO-BCQ can be empirically evaluated as $\frac{1}{L_b}\lVert \bm{b}_j- \bm{\hat{b}}_j\rVert^2_2$ where $\bm{\hat{b}}_j$ is computed from equation (\ref{eq:clustered_quantization_definition}) as $C_{f(\bm{b}_j)}(\bm{b}_j)$. Further, for a given block cluster $\mathcal{B}_i$, we compute the quantization MSE as $\frac{1}{|\mathcal{B}_{i}|}\sum_{\bm{b} \in \mathcal{B}_{i}} \frac{1}{L_b}\lVert \bm{b}- C_i^{(n)}(\bm{b})\rVert^2_2$. Therefore, at the end of iteration $n$, we evaluate the overall quantization MSE $J^{(n)}$ for a given operand $\bm{X}$ composed of $N_c$ block clusters as:
\begin{align*}
    \label{eq:mse_iter_n}
    J^{(n)} = \frac{1}{N_c} \sum_{i=1}^{N_c} \frac{1}{|\mathcal{B}_{i}^{(n)}|}\sum_{\bm{v} \in \mathcal{B}_{i}^{(n)}} \frac{1}{L_b}\lVert \bm{b}- B_i^{(n)}(\bm{b})\rVert^2_2
\end{align*}

At the end of iteration $n$, the codebooks are updated from $\mathcal{C}^{(n-1)}$ to $\mathcal{C}^{(n)}$. However, the mapping of a given vector $\bm{b}_j$ to quantizers $\mathcal{C}^{(n)}$ remains as  $f^{(n)}(\bm{b}_j)$. At the next iteration, during the vector clustering step, $f^{(n+1)}(\bm{b}_j)$ finds new mapping of $\bm{b}_j$ to updated codebooks $\mathcal{C}^{(n)}$ such that the quantization MSE over the candidate codebooks is minimized. Therefore, we obtain the following result for $\bm{b}_j$:
\begin{align*}
\frac{1}{L_b}\lVert \bm{b}_j - C_{f^{(n+1)}(\bm{b}_j)}^{(n)}(\bm{b}_j)\rVert^2_2 \le \frac{1}{L_b}\lVert \bm{b}_j - C_{f^{(n)}(\bm{b}_j)}^{(n)}(\bm{b}_j)\rVert^2_2
\end{align*}

That is, quantizing $\bm{b}_j$ at the end of the block clustering step of iteration $n+1$ results in lower quantization MSE compared to quantizing at the end of iteration $n$. Since this is true for all $\bm{b} \in \bm{X}$, we assert the following:
\begin{equation}
\begin{split}
\label{eq:mse_ineq_1}
    \tilde{J}^{(n+1)} &= \frac{1}{N_c} \sum_{i=1}^{N_c} \frac{1}{|\mathcal{B}_{i}^{(n+1)}|}\sum_{\bm{b} \in \mathcal{B}_{i}^{(n+1)}} \frac{1}{L_b}\lVert \bm{b} - C_i^{(n)}(b)\rVert^2_2 \le J^{(n)}
\end{split}
\end{equation}
where $\tilde{J}^{(n+1)}$ is the the quantization MSE after the vector clustering step at iteration $n+1$.

Next, during the codebook update step (\ref{eq:quantizers_update}) at iteration $n+1$, the per-cluster codebooks $\mathcal{C}^{(n)}$ are updated to $\mathcal{C}^{(n+1)}$ by invoking the Lloyd-Max algorithm \citep{Lloyd}. We know that for any given value distribution, the Lloyd-Max algorithm minimizes the quantization MSE. Therefore, for a given vector cluster $\mathcal{B}_i$ we obtain the following result:

\begin{equation}
    \frac{1}{|\mathcal{B}_{i}^{(n+1)}|}\sum_{\bm{b} \in \mathcal{B}_{i}^{(n+1)}} \frac{1}{L_b}\lVert \bm{b}- C_i^{(n+1)}(\bm{b})\rVert^2_2 \le \frac{1}{|\mathcal{B}_{i}^{(n+1)}|}\sum_{\bm{b} \in \mathcal{B}_{i}^{(n+1)}} \frac{1}{L_b}\lVert \bm{b}- C_i^{(n)}(\bm{b})\rVert^2_2
\end{equation}

The above equation states that quantizing the given block cluster $\mathcal{B}_i$ after updating the associated codebook from $C_i^{(n)}$ to $C_i^{(n+1)}$ results in lower quantization MSE. Since this is true for all the block clusters, we derive the following result: 
\begin{equation}
\begin{split}
\label{eq:mse_ineq_2}
     J^{(n+1)} &= \frac{1}{N_c} \sum_{i=1}^{N_c} \frac{1}{|\mathcal{B}_{i}^{(n+1)}|}\sum_{\bm{b} \in \mathcal{B}_{i}^{(n+1)}} \frac{1}{L_b}\lVert \bm{b}- C_i^{(n+1)}(\bm{b})\rVert^2_2  \le \tilde{J}^{(n+1)}   
\end{split}
\end{equation}

Following (\ref{eq:mse_ineq_1}) and (\ref{eq:mse_ineq_2}), we find that the quantization MSE is non-increasing for each iteration, that is, $J^{(1)} \ge J^{(2)} \ge J^{(3)} \ge \ldots \ge J^{(M)}$ where $M$ is the maximum number of iterations. 
%Therefore, we can say that if the algorithm converges, then it must be that it has converged to a local minimum. 
\hfill $\blacksquare$


\begin{figure}
    \begin{center}
    \includegraphics[width=0.5\textwidth]{sections//figures/mse_vs_iter.pdf}
    \end{center}
    \caption{\small NMSE vs iterations during LO-BCQ compared to other block quantization proposals}
    \label{fig:nmse_vs_iter}
\end{figure}

Figure \ref{fig:nmse_vs_iter} shows the empirical convergence of LO-BCQ across several block lengths and number of codebooks. Also, the MSE achieved by LO-BCQ is compared to baselines such as MXFP and VSQ. As shown, LO-BCQ converges to a lower MSE than the baselines. Further, we achieve better convergence for larger number of codebooks ($N_c$) and for a smaller block length ($L_b$), both of which increase the bitwidth of BCQ (see Eq \ref{eq:bitwidth_bcq}).


\subsection{Additional Accuracy Results}
%Table \ref{tab:lobcq_config} lists the various LOBCQ configurations and their corresponding bitwidths.
\begin{table}
\setlength{\tabcolsep}{4.75pt}
\begin{center}
\caption{\label{tab:lobcq_config} Various LO-BCQ configurations and their bitwidths.}
\begin{tabular}{|c||c|c|c|c||c|c||c|} 
\hline
 & \multicolumn{4}{|c||}{$L_b=8$} & \multicolumn{2}{|c||}{$L_b=4$} & $L_b=2$ \\
 \hline
 \backslashbox{$L_A$\kern-1em}{\kern-1em$N_c$} & 2 & 4 & 8 & 16 & 2 & 4 & 2 \\
 \hline
 64 & 4.25 & 4.375 & 4.5 & 4.625 & 4.375 & 4.625 & 4.625\\
 \hline
 32 & 4.375 & 4.5 & 4.625& 4.75 & 4.5 & 4.75 & 4.75 \\
 \hline
 16 & 4.625 & 4.75& 4.875 & 5 & 4.75 & 5 & 5 \\
 \hline
\end{tabular}
\end{center}
\end{table}

%\subsection{Perplexity achieved by various LO-BCQ configurations on Wikitext-103 dataset}

\begin{table} \centering
\begin{tabular}{|c||c|c|c|c||c|c||c|} 
\hline
 $L_b \rightarrow$& \multicolumn{4}{c||}{8} & \multicolumn{2}{c||}{4} & 2\\
 \hline
 \backslashbox{$L_A$\kern-1em}{\kern-1em$N_c$} & 2 & 4 & 8 & 16 & 2 & 4 & 2  \\
 %$N_c \rightarrow$ & 2 & 4 & 8 & 16 & 2 & 4 & 2 \\
 \hline
 \hline
 \multicolumn{8}{c}{GPT3-1.3B (FP32 PPL = 9.98)} \\ 
 \hline
 \hline
 64 & 10.40 & 10.23 & 10.17 & 10.15 &  10.28 & 10.18 & 10.19 \\
 \hline
 32 & 10.25 & 10.20 & 10.15 & 10.12 &  10.23 & 10.17 & 10.17 \\
 \hline
 16 & 10.22 & 10.16 & 10.10 & 10.09 &  10.21 & 10.14 & 10.16 \\
 \hline
  \hline
 \multicolumn{8}{c}{GPT3-8B (FP32 PPL = 7.38)} \\ 
 \hline
 \hline
 64 & 7.61 & 7.52 & 7.48 &  7.47 &  7.55 &  7.49 & 7.50 \\
 \hline
 32 & 7.52 & 7.50 & 7.46 &  7.45 &  7.52 &  7.48 & 7.48  \\
 \hline
 16 & 7.51 & 7.48 & 7.44 &  7.44 &  7.51 &  7.49 & 7.47  \\
 \hline
\end{tabular}
\caption{\label{tab:ppl_gpt3_abalation} Wikitext-103 perplexity across GPT3-1.3B and 8B models.}
\end{table}

\begin{table} \centering
\begin{tabular}{|c||c|c|c|c||} 
\hline
 $L_b \rightarrow$& \multicolumn{4}{c||}{8}\\
 \hline
 \backslashbox{$L_A$\kern-1em}{\kern-1em$N_c$} & 2 & 4 & 8 & 16 \\
 %$N_c \rightarrow$ & 2 & 4 & 8 & 16 & 2 & 4 & 2 \\
 \hline
 \hline
 \multicolumn{5}{|c|}{Llama2-7B (FP32 PPL = 5.06)} \\ 
 \hline
 \hline
 64 & 5.31 & 5.26 & 5.19 & 5.18  \\
 \hline
 32 & 5.23 & 5.25 & 5.18 & 5.15  \\
 \hline
 16 & 5.23 & 5.19 & 5.16 & 5.14  \\
 \hline
 \multicolumn{5}{|c|}{Nemotron4-15B (FP32 PPL = 5.87)} \\ 
 \hline
 \hline
 64  & 6.3 & 6.20 & 6.13 & 6.08  \\
 \hline
 32  & 6.24 & 6.12 & 6.07 & 6.03  \\
 \hline
 16  & 6.12 & 6.14 & 6.04 & 6.02  \\
 \hline
 \multicolumn{5}{|c|}{Nemotron4-340B (FP32 PPL = 3.48)} \\ 
 \hline
 \hline
 64 & 3.67 & 3.62 & 3.60 & 3.59 \\
 \hline
 32 & 3.63 & 3.61 & 3.59 & 3.56 \\
 \hline
 16 & 3.61 & 3.58 & 3.57 & 3.55 \\
 \hline
\end{tabular}
\caption{\label{tab:ppl_llama7B_nemo15B} Wikitext-103 perplexity compared to FP32 baseline in Llama2-7B and Nemotron4-15B, 340B models}
\end{table}

%\subsection{Perplexity achieved by various LO-BCQ configurations on MMLU dataset}


\begin{table} \centering
\begin{tabular}{|c||c|c|c|c||c|c|c|c|} 
\hline
 $L_b \rightarrow$& \multicolumn{4}{c||}{8} & \multicolumn{4}{c||}{8}\\
 \hline
 \backslashbox{$L_A$\kern-1em}{\kern-1em$N_c$} & 2 & 4 & 8 & 16 & 2 & 4 & 8 & 16  \\
 %$N_c \rightarrow$ & 2 & 4 & 8 & 16 & 2 & 4 & 2 \\
 \hline
 \hline
 \multicolumn{5}{|c|}{Llama2-7B (FP32 Accuracy = 45.8\%)} & \multicolumn{4}{|c|}{Llama2-70B (FP32 Accuracy = 69.12\%)} \\ 
 \hline
 \hline
 64 & 43.9 & 43.4 & 43.9 & 44.9 & 68.07 & 68.27 & 68.17 & 68.75 \\
 \hline
 32 & 44.5 & 43.8 & 44.9 & 44.5 & 68.37 & 68.51 & 68.35 & 68.27  \\
 \hline
 16 & 43.9 & 42.7 & 44.9 & 45 & 68.12 & 68.77 & 68.31 & 68.59  \\
 \hline
 \hline
 \multicolumn{5}{|c|}{GPT3-22B (FP32 Accuracy = 38.75\%)} & \multicolumn{4}{|c|}{Nemotron4-15B (FP32 Accuracy = 64.3\%)} \\ 
 \hline
 \hline
 64 & 36.71 & 38.85 & 38.13 & 38.92 & 63.17 & 62.36 & 63.72 & 64.09 \\
 \hline
 32 & 37.95 & 38.69 & 39.45 & 38.34 & 64.05 & 62.30 & 63.8 & 64.33  \\
 \hline
 16 & 38.88 & 38.80 & 38.31 & 38.92 & 63.22 & 63.51 & 63.93 & 64.43  \\
 \hline
\end{tabular}
\caption{\label{tab:mmlu_abalation} Accuracy on MMLU dataset across GPT3-22B, Llama2-7B, 70B and Nemotron4-15B models.}
\end{table}


%\subsection{Perplexity achieved by various LO-BCQ configurations on LM evaluation harness}

\begin{table} \centering
\begin{tabular}{|c||c|c|c|c||c|c|c|c|} 
\hline
 $L_b \rightarrow$& \multicolumn{4}{c||}{8} & \multicolumn{4}{c||}{8}\\
 \hline
 \backslashbox{$L_A$\kern-1em}{\kern-1em$N_c$} & 2 & 4 & 8 & 16 & 2 & 4 & 8 & 16  \\
 %$N_c \rightarrow$ & 2 & 4 & 8 & 16 & 2 & 4 & 2 \\
 \hline
 \hline
 \multicolumn{5}{|c|}{Race (FP32 Accuracy = 37.51\%)} & \multicolumn{4}{|c|}{Boolq (FP32 Accuracy = 64.62\%)} \\ 
 \hline
 \hline
 64 & 36.94 & 37.13 & 36.27 & 37.13 & 63.73 & 62.26 & 63.49 & 63.36 \\
 \hline
 32 & 37.03 & 36.36 & 36.08 & 37.03 & 62.54 & 63.51 & 63.49 & 63.55  \\
 \hline
 16 & 37.03 & 37.03 & 36.46 & 37.03 & 61.1 & 63.79 & 63.58 & 63.33  \\
 \hline
 \hline
 \multicolumn{5}{|c|}{Winogrande (FP32 Accuracy = 58.01\%)} & \multicolumn{4}{|c|}{Piqa (FP32 Accuracy = 74.21\%)} \\ 
 \hline
 \hline
 64 & 58.17 & 57.22 & 57.85 & 58.33 & 73.01 & 73.07 & 73.07 & 72.80 \\
 \hline
 32 & 59.12 & 58.09 & 57.85 & 58.41 & 73.01 & 73.94 & 72.74 & 73.18  \\
 \hline
 16 & 57.93 & 58.88 & 57.93 & 58.56 & 73.94 & 72.80 & 73.01 & 73.94  \\
 \hline
\end{tabular}
\caption{\label{tab:mmlu_abalation} Accuracy on LM evaluation harness tasks on GPT3-1.3B model.}
\end{table}

\begin{table} \centering
\begin{tabular}{|c||c|c|c|c||c|c|c|c|} 
\hline
 $L_b \rightarrow$& \multicolumn{4}{c||}{8} & \multicolumn{4}{c||}{8}\\
 \hline
 \backslashbox{$L_A$\kern-1em}{\kern-1em$N_c$} & 2 & 4 & 8 & 16 & 2 & 4 & 8 & 16  \\
 %$N_c \rightarrow$ & 2 & 4 & 8 & 16 & 2 & 4 & 2 \\
 \hline
 \hline
 \multicolumn{5}{|c|}{Race (FP32 Accuracy = 41.34\%)} & \multicolumn{4}{|c|}{Boolq (FP32 Accuracy = 68.32\%)} \\ 
 \hline
 \hline
 64 & 40.48 & 40.10 & 39.43 & 39.90 & 69.20 & 68.41 & 69.45 & 68.56 \\
 \hline
 32 & 39.52 & 39.52 & 40.77 & 39.62 & 68.32 & 67.43 & 68.17 & 69.30  \\
 \hline
 16 & 39.81 & 39.71 & 39.90 & 40.38 & 68.10 & 66.33 & 69.51 & 69.42  \\
 \hline
 \hline
 \multicolumn{5}{|c|}{Winogrande (FP32 Accuracy = 67.88\%)} & \multicolumn{4}{|c|}{Piqa (FP32 Accuracy = 78.78\%)} \\ 
 \hline
 \hline
 64 & 66.85 & 66.61 & 67.72 & 67.88 & 77.31 & 77.42 & 77.75 & 77.64 \\
 \hline
 32 & 67.25 & 67.72 & 67.72 & 67.00 & 77.31 & 77.04 & 77.80 & 77.37  \\
 \hline
 16 & 68.11 & 68.90 & 67.88 & 67.48 & 77.37 & 78.13 & 78.13 & 77.69  \\
 \hline
\end{tabular}
\caption{\label{tab:mmlu_abalation} Accuracy on LM evaluation harness tasks on GPT3-8B model.}
\end{table}

\begin{table} \centering
\begin{tabular}{|c||c|c|c|c||c|c|c|c|} 
\hline
 $L_b \rightarrow$& \multicolumn{4}{c||}{8} & \multicolumn{4}{c||}{8}\\
 \hline
 \backslashbox{$L_A$\kern-1em}{\kern-1em$N_c$} & 2 & 4 & 8 & 16 & 2 & 4 & 8 & 16  \\
 %$N_c \rightarrow$ & 2 & 4 & 8 & 16 & 2 & 4 & 2 \\
 \hline
 \hline
 \multicolumn{5}{|c|}{Race (FP32 Accuracy = 40.67\%)} & \multicolumn{4}{|c|}{Boolq (FP32 Accuracy = 76.54\%)} \\ 
 \hline
 \hline
 64 & 40.48 & 40.10 & 39.43 & 39.90 & 75.41 & 75.11 & 77.09 & 75.66 \\
 \hline
 32 & 39.52 & 39.52 & 40.77 & 39.62 & 76.02 & 76.02 & 75.96 & 75.35  \\
 \hline
 16 & 39.81 & 39.71 & 39.90 & 40.38 & 75.05 & 73.82 & 75.72 & 76.09  \\
 \hline
 \hline
 \multicolumn{5}{|c|}{Winogrande (FP32 Accuracy = 70.64\%)} & \multicolumn{4}{|c|}{Piqa (FP32 Accuracy = 79.16\%)} \\ 
 \hline
 \hline
 64 & 69.14 & 70.17 & 70.17 & 70.56 & 78.24 & 79.00 & 78.62 & 78.73 \\
 \hline
 32 & 70.96 & 69.69 & 71.27 & 69.30 & 78.56 & 79.49 & 79.16 & 78.89  \\
 \hline
 16 & 71.03 & 69.53 & 69.69 & 70.40 & 78.13 & 79.16 & 79.00 & 79.00  \\
 \hline
\end{tabular}
\caption{\label{tab:mmlu_abalation} Accuracy on LM evaluation harness tasks on GPT3-22B model.}
\end{table}

\begin{table} \centering
\begin{tabular}{|c||c|c|c|c||c|c|c|c|} 
\hline
 $L_b \rightarrow$& \multicolumn{4}{c||}{8} & \multicolumn{4}{c||}{8}\\
 \hline
 \backslashbox{$L_A$\kern-1em}{\kern-1em$N_c$} & 2 & 4 & 8 & 16 & 2 & 4 & 8 & 16  \\
 %$N_c \rightarrow$ & 2 & 4 & 8 & 16 & 2 & 4 & 2 \\
 \hline
 \hline
 \multicolumn{5}{|c|}{Race (FP32 Accuracy = 44.4\%)} & \multicolumn{4}{|c|}{Boolq (FP32 Accuracy = 79.29\%)} \\ 
 \hline
 \hline
 64 & 42.49 & 42.51 & 42.58 & 43.45 & 77.58 & 77.37 & 77.43 & 78.1 \\
 \hline
 32 & 43.35 & 42.49 & 43.64 & 43.73 & 77.86 & 75.32 & 77.28 & 77.86  \\
 \hline
 16 & 44.21 & 44.21 & 43.64 & 42.97 & 78.65 & 77 & 76.94 & 77.98  \\
 \hline
 \hline
 \multicolumn{5}{|c|}{Winogrande (FP32 Accuracy = 69.38\%)} & \multicolumn{4}{|c|}{Piqa (FP32 Accuracy = 78.07\%)} \\ 
 \hline
 \hline
 64 & 68.9 & 68.43 & 69.77 & 68.19 & 77.09 & 76.82 & 77.09 & 77.86 \\
 \hline
 32 & 69.38 & 68.51 & 68.82 & 68.90 & 78.07 & 76.71 & 78.07 & 77.86  \\
 \hline
 16 & 69.53 & 67.09 & 69.38 & 68.90 & 77.37 & 77.8 & 77.91 & 77.69  \\
 \hline
\end{tabular}
\caption{\label{tab:mmlu_abalation} Accuracy on LM evaluation harness tasks on Llama2-7B model.}
\end{table}

\begin{table} \centering
\begin{tabular}{|c||c|c|c|c||c|c|c|c|} 
\hline
 $L_b \rightarrow$& \multicolumn{4}{c||}{8} & \multicolumn{4}{c||}{8}\\
 \hline
 \backslashbox{$L_A$\kern-1em}{\kern-1em$N_c$} & 2 & 4 & 8 & 16 & 2 & 4 & 8 & 16  \\
 %$N_c \rightarrow$ & 2 & 4 & 8 & 16 & 2 & 4 & 2 \\
 \hline
 \hline
 \multicolumn{5}{|c|}{Race (FP32 Accuracy = 48.8\%)} & \multicolumn{4}{|c|}{Boolq (FP32 Accuracy = 85.23\%)} \\ 
 \hline
 \hline
 64 & 49.00 & 49.00 & 49.28 & 48.71 & 82.82 & 84.28 & 84.03 & 84.25 \\
 \hline
 32 & 49.57 & 48.52 & 48.33 & 49.28 & 83.85 & 84.46 & 84.31 & 84.93  \\
 \hline
 16 & 49.85 & 49.09 & 49.28 & 48.99 & 85.11 & 84.46 & 84.61 & 83.94  \\
 \hline
 \hline
 \multicolumn{5}{|c|}{Winogrande (FP32 Accuracy = 79.95\%)} & \multicolumn{4}{|c|}{Piqa (FP32 Accuracy = 81.56\%)} \\ 
 \hline
 \hline
 64 & 78.77 & 78.45 & 78.37 & 79.16 & 81.45 & 80.69 & 81.45 & 81.5 \\
 \hline
 32 & 78.45 & 79.01 & 78.69 & 80.66 & 81.56 & 80.58 & 81.18 & 81.34  \\
 \hline
 16 & 79.95 & 79.56 & 79.79 & 79.72 & 81.28 & 81.66 & 81.28 & 80.96  \\
 \hline
\end{tabular}
\caption{\label{tab:mmlu_abalation} Accuracy on LM evaluation harness tasks on Llama2-70B model.}
\end{table}

%\section{MSE Studies}
%\textcolor{red}{TODO}


\subsection{Number Formats and Quantization Method}
\label{subsec:numFormats_quantMethod}
\subsubsection{Integer Format}
An $n$-bit signed integer (INT) is typically represented with a 2s-complement format \citep{yao2022zeroquant,xiao2023smoothquant,dai2021vsq}, where the most significant bit denotes the sign.

\subsubsection{Floating Point Format}
An $n$-bit signed floating point (FP) number $x$ comprises of a 1-bit sign ($x_{\mathrm{sign}}$), $B_m$-bit mantissa ($x_{\mathrm{mant}}$) and $B_e$-bit exponent ($x_{\mathrm{exp}}$) such that $B_m+B_e=n-1$. The associated constant exponent bias ($E_{\mathrm{bias}}$) is computed as $(2^{{B_e}-1}-1)$. We denote this format as $E_{B_e}M_{B_m}$.  

\subsubsection{Quantization Scheme}
\label{subsec:quant_method}
A quantization scheme dictates how a given unquantized tensor is converted to its quantized representation. We consider FP formats for the purpose of illustration. Given an unquantized tensor $\bm{X}$ and an FP format $E_{B_e}M_{B_m}$, we first, we compute the quantization scale factor $s_X$ that maps the maximum absolute value of $\bm{X}$ to the maximum quantization level of the $E_{B_e}M_{B_m}$ format as follows:
\begin{align}
\label{eq:sf}
    s_X = \frac{\mathrm{max}(|\bm{X}|)}{\mathrm{max}(E_{B_e}M_{B_m})}
\end{align}
In the above equation, $|\cdot|$ denotes the absolute value function.

Next, we scale $\bm{X}$ by $s_X$ and quantize it to $\hat{\bm{X}}$ by rounding it to the nearest quantization level of $E_{B_e}M_{B_m}$ as:

\begin{align}
\label{eq:tensor_quant}
    \hat{\bm{X}} = \text{round-to-nearest}\left(\frac{\bm{X}}{s_X}, E_{B_e}M_{B_m}\right)
\end{align}

We perform dynamic max-scaled quantization \citep{wu2020integer}, where the scale factor $s$ for activations is dynamically computed during runtime.

\subsection{Vector Scaled Quantization}
\begin{wrapfigure}{r}{0.35\linewidth}
  \centering
  \includegraphics[width=\linewidth]{sections/figures/vsquant.jpg}
  \caption{\small Vectorwise decomposition for per-vector scaled quantization (VSQ \citep{dai2021vsq}).}
  \label{fig:vsquant}
\end{wrapfigure}
During VSQ \citep{dai2021vsq}, the operand tensors are decomposed into 1D vectors in a hardware friendly manner as shown in Figure \ref{fig:vsquant}. Since the decomposed tensors are used as operands in matrix multiplications during inference, it is beneficial to perform this decomposition along the reduction dimension of the multiplication. The vectorwise quantization is performed similar to tensorwise quantization described in Equations \ref{eq:sf} and \ref{eq:tensor_quant}, where a scale factor $s_v$ is required for each vector $\bm{v}$ that maps the maximum absolute value of that vector to the maximum quantization level. While smaller vector lengths can lead to larger accuracy gains, the associated memory and computational overheads due to the per-vector scale factors increases. To alleviate these overheads, VSQ \citep{dai2021vsq} proposed a second level quantization of the per-vector scale factors to unsigned integers, while MX \citep{rouhani2023shared} quantizes them to integer powers of 2 (denoted as $2^{INT}$).

\subsubsection{MX Format}
The MX format proposed in \citep{rouhani2023microscaling} introduces the concept of sub-block shifting. For every two scalar elements of $b$-bits each, there is a shared exponent bit. The value of this exponent bit is determined through an empirical analysis that targets minimizing quantization MSE. We note that the FP format $E_{1}M_{b}$ is strictly better than MX from an accuracy perspective since it allocates a dedicated exponent bit to each scalar as opposed to sharing it across two scalars. Therefore, we conservatively bound the accuracy of a $b+2$-bit signed MX format with that of a $E_{1}M_{b}$ format in our comparisons. For instance, we use E1M2 format as a proxy for MX4.

\begin{figure}
    \centering
    \includegraphics[width=1\linewidth]{sections//figures/BlockFormats.pdf}
    \caption{\small Comparing LO-BCQ to MX format.}
    \label{fig:block_formats}
\end{figure}

Figure \ref{fig:block_formats} compares our $4$-bit LO-BCQ block format to MX \citep{rouhani2023microscaling}. As shown, both LO-BCQ and MX decompose a given operand tensor into block arrays and each block array into blocks. Similar to MX, we find that per-block quantization ($L_b < L_A$) leads to better accuracy due to increased flexibility. While MX achieves this through per-block $1$-bit micro-scales, we associate a dedicated codebook to each block through a per-block codebook selector. Further, MX quantizes the per-block array scale-factor to E8M0 format without per-tensor scaling. In contrast during LO-BCQ, we find that per-tensor scaling combined with quantization of per-block array scale-factor to E4M3 format results in superior inference accuracy across models. 


\end{document}
