\section{Background and Related Work}
The evolution of financial sentiment analysis has followed several key trajectories in the literature. Early work by Loughran and McDonald \cite{Loughran11} established the importance of domain-specific dictionaries for financial text analysis, highlighting how general-purpose sentiment tools often fail in financial contexts. In parallel, the development of comprehensive financial reinforcement learning frameworks like FinRL-Meta \cite{Liu20} has provided standardized environments for developing and evaluating trading strategies.

Recent developments in prompt engineering have shown promising results in various domains. Wei et al. \cite{Wei22} demonstrated how carefully constructed prompts can elicit domain-specific knowledge from LLMs without fine-tuning, while Vatsal and Dubey \cite{Vatsal23} provided various methods / frameworks for evaluating prompt effectiveness in various NLP tasks. However, applications in financial sentiment analysis have been limited, with most approaches focusing on model architecture modifications rather than prompt optimization.