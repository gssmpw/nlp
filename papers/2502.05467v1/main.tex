%%%%%%%% ICML 2025 EXAMPLE LATEX SUBMISSION FILE %%%%%%%%%%%%%%%%%

\documentclass{article}

% Recommended, but optional, packages for figures and better typesetting:
\usepackage{microtype}
\usepackage{graphicx}
\usepackage{subfigure}
\usepackage{booktabs} % for professional tables

% hyperref makes hyperlinks in the resulting PDF.
% If your build breaks (sometimes temporarily if a hyperlink spans a page)
% please comment out the following usepackage line and replace
% \usepackage{icml2025} with \usepackage[nohyperref]{icml2025} above.
\usepackage{hyperref}


% Attempt to make hyperref and algorithmic work together better:
\newcommand{\theHalgorithm}{\arabic{algorithm}}

% Use the following line for the initial blind version submitted for review:
% \usepackage{icml2025}

% If accepted, instead use the following line for the camera-ready submission:
\usepackage[accepted]{icml2025}

% For theorems and such
\usepackage{amsmath}
\usepackage{amssymb}
\usepackage{mathtools}
\usepackage{amsthm}

\usepackage{tcolorbox}
\usepackage{tikz}
\usepackage[edges]{forest}

% if you use cleveref..
\usepackage[capitalize,noabbrev]{cleveref}

%%%%%%%%%%%%%%%%%%%%%%%%%%%%%%%%
% THEOREMS
%%%%%%%%%%%%%%%%%%%%%%%%%%%%%%%%
\theoremstyle{plain}
\newtheorem{theorem}{Theorem}[section]
\newtheorem{proposition}[theorem]{Proposition}
\newtheorem{lemma}[theorem]{Lemma}
\newtheorem{corollary}[theorem]{Corollary}
\theoremstyle{definition}
\newtheorem{definition}[theorem]{Definition}
\newtheorem{assumption}[theorem]{Assumption}
\theoremstyle{remark}
\newtheorem{remark}[theorem]{Remark}

% Todonotes is useful during development; simply uncomment the next line
%    and comment out the line below the next line to turn off comments
%\usepackage[disable,textsize=tiny]{todonotes}
\usepackage[textsize=tiny]{todonotes}


% The \icmltitle you define below is probably too long as a header.
% Therefore, a short form for the running title is supplied here:
% \icmltitlerunning{Submission and Formatting Instructions for ICML 2025}
\icmltitlerunning{Position: LLMs Can be Good Tutors in Foreign Language Education}

\begin{document}

\twocolumn[
\icmltitle{Position: LLMs Can be Good Tutors in Foreign Language Education}

% It is OKAY to include author information, even for blind
% submissions: the style file will automatically remove it for you
% unless you've provided the [accepted] option to the icml2025
% package.

% List of affiliations: The first argument should be a (short)
% identifier you will use later to specify author affiliations
% Academic affiliations should list Department, University, City, Region, Country
% Industry affiliations should list Company, City, Region, Country

% You can specify symbols, otherwise they are numbered in order.
% Ideally, you should not use this facility. Affiliations will be numbered
% in order of appearance and this is the preferred way.
\icmlsetsymbol{equal}{*}
\icmlsetsymbol{correspondence}{\dag}

\begin{icmlauthorlist}
\icmlauthor{Jingheng Ye}{Squirrel,THU}
\icmlauthor{Shen Wang}{Squirrel}
\icmlauthor{Deqing Zhou}{THU}
\icmlauthor{Yibo Yan}{HKUST}
\icmlauthor{Kun Wang}{Squirrel}\\
\icmlauthor{Hai-Tao Zheng}{correspondence,THU,PCL}
\icmlauthor{Zenglin Xu}{Fudan}
\icmlauthor{Irwin King}{CUHK}
\icmlauthor{Philip S. Yu}{UIC}
\icmlauthor{Qingsong Wen}{correspondence,Squirrel}

%\icmlauthor{}{sch}
%\icmlauthor{}{sch}
\end{icmlauthorlist}

\icmlaffiliation{THU}{Tsinghua University}
\icmlaffiliation{Squirrel}{Squirrel Ai Learning}
\icmlaffiliation{HKUST}{The Hong Kong University of Science and Technology (Guangzhou)}
\icmlaffiliation{Fudan}{Fudan University}
\icmlaffiliation{CUHK}{The Chinese University of Hong Kong}
\icmlaffiliation{UIC}{University of Illinois at Chicago}
\icmlaffiliation{PCL}{Peng Cheng Laboratory}

\icmlcorrespondingauthor{Hai-Tao Zheng}{zheng.haitao@sz.tsinghua.edu.cn}
\icmlcorrespondingauthor{Qingsong Wen}{qingsongedu@gmail.com}

% You may provide any keywords that you
% find helpful for describing your paper; these are used to populate
% the "keywords" metadata in the PDF but will not be shown in the document
\icmlkeywords{Machine Learning, ICML}

\vskip 0.3in
]
% Jingheng ye 

% this must go after the closing bracket ] following \twocolumn[ ...

% This command actually creates the footnote in the first column
% listing the affiliations and the copyright notice.
% The command takes one argument, which is text to display at the start of the footnote.
% The \icmlEqualContribution command is standard text for equal contribution.
% Remove it (just {}) if you do not need this facility.

\printAffiliationsAndNotice{}  % leave blank if no need to mention equal contribution
% \printAffiliationsAndNotice{\icmlEqualContribution} % otherwise use the standard text.

\begin{abstract}  
Test time scaling is currently one of the most active research areas that shows promise after training time scaling has reached its limits.
Deep-thinking (DT) models are a class of recurrent models that can perform easy-to-hard generalization by assigning more compute to harder test samples.
However, due to their inability to determine the complexity of a test sample, DT models have to use a large amount of computation for both easy and hard test samples.
Excessive test time computation is wasteful and can cause the ``overthinking'' problem where more test time computation leads to worse results.
In this paper, we introduce a test time training method for determining the optimal amount of computation needed for each sample during test time.
We also propose Conv-LiGRU, a novel recurrent architecture for efficient and robust visual reasoning. 
Extensive experiments demonstrate that Conv-LiGRU is more stable than DT, effectively mitigates the ``overthinking'' phenomenon, and achieves superior accuracy.
\end{abstract}  
\section{Introduction}
\label{sec:introduction}
The business processes of organizations are experiencing ever-increasing complexity due to the large amount of data, high number of users, and high-tech devices involved \cite{martin2021pmopportunitieschallenges, beerepoot2023biggestbpmproblems}. This complexity may cause business processes to deviate from normal control flow due to unforeseen and disruptive anomalies \cite{adams2023proceddsriftdetection}. These control-flow anomalies manifest as unknown, skipped, and wrongly-ordered activities in the traces of event logs monitored from the execution of business processes \cite{ko2023adsystematicreview}. For the sake of clarity, let us consider an illustrative example of such anomalies. Figure \ref{FP_ANOMALIES} shows a so-called event log footprint, which captures the control flow relations of four activities of a hypothetical event log. In particular, this footprint captures the control-flow relations between activities \texttt{a}, \texttt{b}, \texttt{c} and \texttt{d}. These are the causal ($\rightarrow$) relation, concurrent ($\parallel$) relation, and other ($\#$) relations such as exclusivity or non-local dependency \cite{aalst2022pmhandbook}. In addition, on the right are six traces, of which five exhibit skipped, wrongly-ordered and unknown control-flow anomalies. For example, $\langle$\texttt{a b d}$\rangle$ has a skipped activity, which is \texttt{c}. Because of this skipped activity, the control-flow relation \texttt{b}$\,\#\,$\texttt{d} is violated, since \texttt{d} directly follows \texttt{b} in the anomalous trace.
\begin{figure}[!t]
\centering
\includegraphics[width=0.9\columnwidth]{images/FP_ANOMALIES.png}
\caption{An example event log footprint with six traces, of which five exhibit control-flow anomalies.}
\label{FP_ANOMALIES}
\end{figure}

\subsection{Control-flow anomaly detection}
Control-flow anomaly detection techniques aim to characterize the normal control flow from event logs and verify whether these deviations occur in new event logs \cite{ko2023adsystematicreview}. To develop control-flow anomaly detection techniques, \revision{process mining} has seen widespread adoption owing to process discovery and \revision{conformance checking}. On the one hand, process discovery is a set of algorithms that encode control-flow relations as a set of model elements and constraints according to a given modeling formalism \cite{aalst2022pmhandbook}; hereafter, we refer to the Petri net, a widespread modeling formalism. On the other hand, \revision{conformance checking} is an explainable set of algorithms that allows linking any deviations with the reference Petri net and providing the fitness measure, namely a measure of how much the Petri net fits the new event log \cite{aalst2022pmhandbook}. Many control-flow anomaly detection techniques based on \revision{conformance checking} (hereafter, \revision{conformance checking}-based techniques) use the fitness measure to determine whether an event log is anomalous \cite{bezerra2009pmad, bezerra2013adlogspais, myers2018icsadpm, pecchia2020applicationfailuresanalysispm}. 

The scientific literature also includes many \revision{conformance checking}-independent techniques for control-flow anomaly detection that combine specific types of trace encodings with machine/deep learning \cite{ko2023adsystematicreview, tavares2023pmtraceencoding}. Whereas these techniques are very effective, their explainability is challenging due to both the type of trace encoding employed and the machine/deep learning model used \cite{rawal2022trustworthyaiadvances,li2023explainablead}. Hence, in the following, we focus on the shortcomings of \revision{conformance checking}-based techniques to investigate whether it is possible to support the development of competitive control-flow anomaly detection techniques while maintaining the explainable nature of \revision{conformance checking}.
\begin{figure}[!t]
\centering
\includegraphics[width=\columnwidth]{images/HIGH_LEVEL_VIEW.png}
\caption{A high-level view of the proposed framework for combining \revision{process mining}-based feature extraction with dimensionality reduction for control-flow anomaly detection.}
\label{HIGH_LEVEL_VIEW}
\end{figure}

\subsection{Shortcomings of \revision{conformance checking}-based techniques}
Unfortunately, the detection effectiveness of \revision{conformance checking}-based techniques is affected by noisy data and low-quality Petri nets, which may be due to human errors in the modeling process or representational bias of process discovery algorithms \cite{bezerra2013adlogspais, pecchia2020applicationfailuresanalysispm, aalst2016pm}. Specifically, on the one hand, noisy data may introduce infrequent and deceptive control-flow relations that may result in inconsistent fitness measures, whereas, on the other hand, checking event logs against a low-quality Petri net could lead to an unreliable distribution of fitness measures. Nonetheless, such Petri nets can still be used as references to obtain insightful information for \revision{process mining}-based feature extraction, supporting the development of competitive and explainable \revision{conformance checking}-based techniques for control-flow anomaly detection despite the problems above. For example, a few works outline that token-based \revision{conformance checking} can be used for \revision{process mining}-based feature extraction to build tabular data and develop effective \revision{conformance checking}-based techniques for control-flow anomaly detection \cite{singh2022lapmsh, debenedictis2023dtadiiot}. However, to the best of our knowledge, the scientific literature lacks a structured proposal for \revision{process mining}-based feature extraction using the state-of-the-art \revision{conformance checking} variant, namely alignment-based \revision{conformance checking}.

\subsection{Contributions}
We propose a novel \revision{process mining}-based feature extraction approach with alignment-based \revision{conformance checking}. This variant aligns the deviating control flow with a reference Petri net; the resulting alignment can be inspected to extract additional statistics such as the number of times a given activity caused mismatches \cite{aalst2022pmhandbook}. We integrate this approach into a flexible and explainable framework for developing techniques for control-flow anomaly detection. The framework combines \revision{process mining}-based feature extraction and dimensionality reduction to handle high-dimensional feature sets, achieve detection effectiveness, and support explainability. Notably, in addition to our proposed \revision{process mining}-based feature extraction approach, the framework allows employing other approaches, enabling a fair comparison of multiple \revision{conformance checking}-based and \revision{conformance checking}-independent techniques for control-flow anomaly detection. Figure \ref{HIGH_LEVEL_VIEW} shows a high-level view of the framework. Business processes are monitored, and event logs obtained from the database of information systems. Subsequently, \revision{process mining}-based feature extraction is applied to these event logs and tabular data input to dimensionality reduction to identify control-flow anomalies. We apply several \revision{conformance checking}-based and \revision{conformance checking}-independent framework techniques to publicly available datasets, simulated data of a case study from railways, and real-world data of a case study from healthcare. We show that the framework techniques implementing our approach outperform the baseline \revision{conformance checking}-based techniques while maintaining the explainable nature of \revision{conformance checking}.

In summary, the contributions of this paper are as follows.
\begin{itemize}
    \item{
        A novel \revision{process mining}-based feature extraction approach to support the development of competitive and explainable \revision{conformance checking}-based techniques for control-flow anomaly detection.
    }
    \item{
        A flexible and explainable framework for developing techniques for control-flow anomaly detection using \revision{process mining}-based feature extraction and dimensionality reduction.
    }
    \item{
        Application to synthetic and real-world datasets of several \revision{conformance checking}-based and \revision{conformance checking}-independent framework techniques, evaluating their detection effectiveness and explainability.
    }
\end{itemize}

The rest of the paper is organized as follows.
\begin{itemize}
    \item Section \ref{sec:related_work} reviews the existing techniques for control-flow anomaly detection, categorizing them into \revision{conformance checking}-based and \revision{conformance checking}-independent techniques.
    \item Section \ref{sec:abccfe} provides the preliminaries of \revision{process mining} to establish the notation used throughout the paper, and delves into the details of the proposed \revision{process mining}-based feature extraction approach with alignment-based \revision{conformance checking}.
    \item Section \ref{sec:framework} describes the framework for developing \revision{conformance checking}-based and \revision{conformance checking}-independent techniques for control-flow anomaly detection that combine \revision{process mining}-based feature extraction and dimensionality reduction.
    \item Section \ref{sec:evaluation} presents the experiments conducted with multiple framework and baseline techniques using data from publicly available datasets and case studies.
    \item Section \ref{sec:conclusions} draws the conclusions and presents future work.
\end{itemize}
\section{Background}\label{sec:backgrnd}

\subsection{Cold Start Latency and Mitigation Techniques}

Traditional FaaS platforms mitigate cold starts through snapshotting, lightweight virtualization, and warm-state management. Snapshot-based methods like \textbf{REAP} and \textbf{Catalyzer} reduce initialization time by preloading or restoring container states but require significant memory and I/O resources, limiting scalability~\cite{dong_catalyzer_2020, ustiugov_benchmarking_2021}. Lightweight virtualization solutions, such as \textbf{Firecracker} microVMs, achieve fast startup times with strong isolation but depend on robust infrastructure, making them less adaptable to fluctuating workloads~\cite{agache_firecracker_2020}. Warm-state management techniques like \textbf{Faa\$T}~\cite{romero_faa_2021} and \textbf{Kraken}~\cite{vivek_kraken_2021} keep frequently invoked containers ready, balancing readiness and cost efficiency under predictable workloads but incurring overhead when demand is erratic~\cite{romero_faa_2021, vivek_kraken_2021}. While these methods perform well in resource-rich cloud environments, their resource intensity challenges applicability in edge settings.

\subsubsection{Edge FaaS Perspective}

In edge environments, cold start mitigation emphasizes lightweight designs, resource sharing, and hybrid task distribution. Lightweight execution environments like unikernels~\cite{edward_sock_2018} and \textbf{Firecracker}~\cite{agache_firecracker_2020}, as used by \textbf{TinyFaaS}~\cite{pfandzelter_tinyfaas_2020}, minimize resource usage and initialization delays but require careful orchestration to avoid resource contention. Function co-location, demonstrated by \textbf{Photons}~\cite{v_dukic_photons_2020}, reduces redundant initializations by sharing runtime resources among related functions, though this complicates isolation in multi-tenant setups~\cite{v_dukic_photons_2020}. Hybrid offloading frameworks like \textbf{GeoFaaS}~\cite{malekabbasi_geofaas_2024} balance edge-cloud workloads by offloading latency-tolerant tasks to the cloud and reserving edge resources for real-time operations, requiring reliable connectivity and efficient task management. These edge-specific strategies address cold starts effectively but introduce challenges in scalability and orchestration.

\subsection{Predictive Scaling and Caching Techniques}

Efficient resource allocation is vital for maintaining low latency and high availability in serverless platforms. Predictive scaling and caching techniques dynamically provision resources and reduce cold start latency by leveraging workload prediction and state retention.
Traditional FaaS platforms use predictive scaling and caching to optimize resources, employing techniques (OFC, FaasCache) to reduce cold starts. However, these methods rely on centralized orchestration and workload predictability, limiting their effectiveness in dynamic, resource-constrained edge environments.



\subsubsection{Edge FaaS Perspective}

Edge FaaS platforms adapt predictive scaling and caching techniques to constrain resources and heterogeneous environments. \textbf{EDGE-Cache}~\cite{kim_delay-aware_2022} uses traffic profiling to selectively retain high-priority functions, reducing memory overhead while maintaining readiness for frequent requests. Hybrid frameworks like \textbf{GeoFaaS}~\cite{malekabbasi_geofaas_2024} implement distributed caching to balance resources between edge and cloud nodes, enabling low-latency processing for critical tasks while offloading less critical workloads. Machine learning methods, such as clustering-based workload predictors~\cite{gao_machine_2020} and GRU-based models~\cite{guo_applying_2018}, enhance resource provisioning in edge systems by efficiently forecasting workload spikes. These innovations effectively address cold start challenges in edge environments, though their dependency on accurate predictions and robust orchestration poses scalability challenges.

\subsection{Decentralized Orchestration, Function Placement, and Scheduling}

Efficient orchestration in serverless platforms involves workload distribution, resource optimization, and performance assurance. While traditional FaaS platforms rely on centralized control, edge environments require decentralized and adaptive strategies to address unique challenges such as resource constraints and heterogeneous hardware.



\subsubsection{Edge FaaS Perspective}

Edge FaaS platforms adopt decentralized and adaptive orchestration frameworks to meet the demands of resource-constrained environments. Systems like \textbf{Wukong} distribute scheduling across edge nodes, enhancing data locality and scalability while reducing network latency. Lightweight frameworks such as \textbf{OpenWhisk Lite}~\cite{kravchenko_kpavelopenwhisk-light_2024} optimize resource allocation by decentralizing scheduling policies, minimizing cold starts and latency in edge setups~\cite{benjamin_wukong_2020}. Hybrid solutions like \textbf{OpenFaaS}~\cite{noauthor_openfaasfaas_2024} and \textbf{EdgeMatrix}~\cite{shen_edgematrix_2023} combine edge-cloud orchestration to balance resource utilization, retaining latency-sensitive functions at the edge while offloading non-critical workloads to the cloud. While these approaches improve flexibility, they face challenges in maintaining coordination and ensuring consistent performance across distributed nodes.


\section{LLMs as Data Enhancers}\label{sec:enhancer}
Education is a high-stake area where any form of LLMs' hallucination could cause devastating harm to humans' cognition activities~\cite{ho2024mitigating}. One of the hallucination causes is from data~\cite{huang2023survey}. Therefore, high-quality and diverse data resources~\cite{long-etal-2024-llms} are critical to ensuring the reliability of incorporating LLMs into FLE. The 1) \textit{creation}, 2) \textit{reformation}, and 3) \textit{annotation} of educational materials are crucial to delivering effective and engaging teaching. Traditional resource development methods often lack the scalability, adaptability, and personalization necessary to meet the diverse needs of learners~\cite{feng-etal-2021-survey,shorten2021text}. In contrast, LLMs emerge as transformative tools capable of enhancing these processes~\cite{wang2024survey,liu2024best}. This section explores how LLMs serve as data enhancers in FLE.


\subsection{Data Creation}
Creating pedagogically sound and learner-specific data is a cornerstone of personalized learning. However, manually creating such resources is time-consuming and often fails to address the wide range of learner needs~\cite{cochran2022improving}. LLMs can revolutionize this process by generating tailored and diverse educational content or responses on demand~\cite{zha2023data,cochran2023improving}.

% \paragraph{Educational Materials Generation.} One of the most direct applications of LLMs is the \textit{generation of educational questions} tailored to specific learning objectives. Many question generation methods have been proposed, from traditional rule-based to advanced neural network-based methods~\cite{kurdi2020systematic,rathod-etal-2022-educational,mulla2023automatic}. LLMs have been proven adept at generating contextually relevant questions and distractors for various learning objectives~\cite{doughty2024comparative,biancini2024multiple,lee2024few} than previous methods. They can produce answer-aware (whose target answer is known) or answer-agnostic (whose answer is open)~\cite{zhang2021review} for reading comprehension questions~\cite{xiao-etal-2023-evaluating} based on a specific topic. The generated questions can be combined into a systematic simulated exam for students' learning diagnosis, a prerequisite for the subsequent design of personalized learning paths. In addition to text-based materials~\cite{rashid2024humanizing}, LLMs show promise in contributing to the creating multimodal educational resources that integrate slides, images, audio, and video.

\paragraph{Educational Materials Generation.} A primary use of LLMs in data creation is the \textit{generation of educational questions} aligned with specific learning objectives. Due to their superior contextual understanding, classic rule-based approaches have largely been eclipsed by neural network-based techniques~\cite{kurdi2020systematic,rathod-etal-2022-educational,mulla2023automatic}. LLMs can produce answer-aware (whose target answer is known) or answer-agnostic (whose answer is open)~\cite{zhang2021review}, resulting in more nuanced exercises and assessments~\cite{xiao-etal-2023-evaluating}.

\paragraph{Student Simulation.} Simulating the learner’s perspective is crucial for designing adaptive instructional materials. Traditional surveys and standardized tests often fail to capture the complexity of dynamic learner behaviors~\cite{kaser2024simulated}. In contrast, LLM-based approaches enable high-fidelity, context-aware \textit{student simulations}~\cite{liu2024personality,yue2024mathvc}, generating synthetic learners who exhibit realistic mastery levels and evolving behaviors. For instance, \textit{Generative Students}~\cite{lu2024generative} create simulated learners with various competency levels, while \textit{EduAgent}~\cite{xu2024eduagent} integrates cognitive priors to model complex learning trajectories and behaviors better.

\paragraph{Discussion.} While LLMs excel at generating educational content, current approaches mainly focus on question creation, leaving many areas of FLE underexplored. Essential tasks like generating culturally rich reading materials, context-dependent writing prompts, or dynamic comprehension exercises are still lacking in diversity and depth. Additionally, the student simulations created by LLMs often fail to reflect long-term learning trajectories or the intricacies of individual learning progress, leading to challenges in creating truly adaptive systems.


\subsection{Data Reformation}
In addition to creating new content, LLMs can adapt \textit{existing} materials to better align with current needs. This process, commonly referred to as data reformation, involves (1) changing data types or modalities, (2) paraphrasing materials to match learner proficiency, and (3) enriching raw data with auxiliary signals or contextual content.

\paragraph{Teaching Material Transformation.} Transforming existing materials into different forms can yield more comprehensive and immersive learning experiences. For example, \textit{Book2Dial}\cite{wang-etal-2024-book2dial} generates teacher-student dialogues grounded in textbooks, keeping the content both relevant and informative. Their approach includes multi-turn question generation and answering\cite{kim-etal-2022-generating}, dialogue inpainting~\cite{dai2022dialog}, and role-playing. Likewise, \textit{Slide2Lecture}~\cite{zhang2024awaking} automatically converts lecture slides into structured teaching agendas, enabling interactive follow-up and deeper learner engagement.

\paragraph{Simplification and Paraphrasing.} Another vital application is simplifying or paraphrasing complex texts to specified readability levels~\cite{huang2024generating} without losing key concepts~\cite{al2021automated}. This is particularly beneficial in FLE settings, where language beginners often face advanced vocabulary and complex structures~\cite{day2025evaluating}. Recent advancements in controllable generation~\cite{zhang2023survey} leverage model fine-tuning on curated datasets~\cite{zeng-etal-2023-seen} or decoding-time interventions~\cite{liang2024controllable}, thereby allowing educators to specify text complexity, style, or tone.

\paragraph{Cultural Context Adaptation.} Beyond linguistic correctness, cultural nuance is pivotal in FLE~\cite{byram1989cultural,byram2008foreign}. LLMs can facilitate this process by recontextualizing existing materials to reflect the cultural and social norms of the target language~\cite{liu2024culturally}. For instance, a short story originally set in an English-speaking environment may be adapted for French or Japanese learners by adjusting the characters’ names, idiomatic expressions, or social customs, while still preserving core instructional goals. This cultural adaptation not only enhances learner engagement but also strengthens cross-cultural competencies.

\paragraph{Discussion.} While LLM-based data reformation can significantly enhance FLE, several gaps warrant attention. Most current studies prioritize textual forms or single-modal approaches, which may overlook valuable \textit{multimodal} resources such as interactive video and audio-based content~\cite{ghosal2023text}. Furthermore, cultural adaptation, although promising, remains underexplored in practical classroom scenarios, particularly for underrepresented personas and culturally sensitive topics. \citet{alkhamissi-etal-2024-investigating} demonstrate how cultural misalignment can increase bias, suggesting \textit{Anthropological Prompting} to strengthen cultural alignment. However, robust empirical \textit{evaluations} are still limited across diverse learners and linguistic backgrounds. These issues highlight the importance of interdisciplinary research, where educators, linguists, and technologists collaborate to refine data reformation techniques that broaden inclusivity and effectiveness in FLE.

\subsection{Data Annotation}
While \textit{Data Creation} focuses on generating learner-specific data, it often prioritizes diversity and adaptability over precision. The approach is particularly useful for tasks with large label spaces~\cite{ding-etal-2024-data}. In contrast, \textit{Data Annotation} emphasizes producing high-quality, meticulously labeled data that is essential for tasks requiring accuracy and consistency. Unlike data creation, annotated data often undergoes rigorous validation to ensure its accuracy and relevancy~\cite{artemova2024hands}.

\paragraph{Annotation Generation.} LLMs can be central to generating a variety of annotations, including categorical labels, rationales, pedagogical feedback, and linguistic features such as discourse relations. Recent prompt engineering and fine-tuning techniques have further expanded LLMs’ annotation capabilities. For instance, \citet{ye2024excgec} leverage GPT-4 to annotate structured explanations for Chinese grammatical error correction, while \citet{samuel-etal-2024-llms} examine GPT-4 as a surrogate for human annotators in low-resource reading comprehension tasks. Likewise, \citet{li-etal-2024-eden} deploy GPT-4-Turbo for audio transcript annotations. However, inconsistencies across LLMs~\cite{tornberg2024best} remain a serious challenge, posing risks to educational reliability.

\paragraph{Annotation Assessment.} Although LLM-based annotation generation is efficient, it also raises critical issues of bias, calibration, and validity—particularly in low-resource language contexts~\cite{bhat-varma-2023-large,jadhav2024limitations}. Automated or semi-automated evaluation strategies have emerged to address these quality concerns. For example, LLMs-as-Judges~\cite{li2024generation,li2024llms,gu2024survey} reduce human overhead by automating evaluation, an approach increasingly explored in education-focused applications~\cite{chiang-etal-2024-large,zhou-etal-2024-llm}. However, purely automated frameworks can still propagate errors or bias.

% if LLM outputs are taken at face value.

\paragraph{Discussion.} Although LLMs provide efficient data annotation, the inconsistency across different models remains a critical concern, affecting the quality and reliability of annotated educational materials. These discrepancies hinder the creation of universally reliable FLE content, especially in diverse linguistic and cultural contexts. Additionally, automated annotations often lack the nuance needed for pedagogical applications, making it essential to involve human oversight in critical cases to mitigate errors or biases.

\begin{tcolorbox}[top=1pt, bottom=1pt, left=1pt, right=1pt]
\textbf{Our position.} We acknowledge the current limitations in LLM-based data creation, reformation, and annotation for FLE. However, we believe that with continued interdisciplinary collaboration, including input from educators, linguists, and technologists, these challenges can be addressed. \textit{Future advancements} should focus on enhancing the accuracy and diversity of generated content, improving multimodal and culturally sensitive learning materials, and integrating more robust systems for human-LLM collaboration~\cite{li-etal-2023-coannotating,wang2024human} in data annotation. This approach will ensure that LLMs can fully realize their potential as effective tutors in FLE.
\end{tcolorbox}

% adaptive and effective tutors in FLE.

\section{LLMs as Task Predictors}\label{sec:predictor}
\textit{Task-Based Language Learning (TBLL)}~\cite{nunan1989designing,willis2021framework} as a methodological approach is one of the critical factors of modern foreign language teaching. LLMs have demonstrated remarkable capabilities in understanding and generating human language, making them well-suited for addressing numerous tasks in FLE. These tasks can be broadly categorized into three types based on their nature and the role of LLMs: 1) \textit{Discriminative}, 2) \textit{Generative}, and 3) \textit{Mixed} of the above two roles.

\subsection{Discriminative Task Predictors}
Discriminative tasks in FLE primarily involve classifying learner inputs or grading their future performance. Below are some applications that are still calling for improvements:

% \paragraph{Automated Assessment.} The task aims to automatically grade students' assignments, including automatic essay scoring~\cite{sessler2024can,li2024applying,syamkumar2024improving}, short answer grading~\cite{schneider2023towards,henkel2024can}, and spoken language assessment~\cite{gao2023investigation,fu2024pronunciation}. For instance, LLMs can assess the grammatical accuracy, lexical variety, and coherence of essays, offering both holistic scores and detailed feedback on specific areas for improvement. Similarly, for spoken assessments, LLMs can evaluate pronunciation, fluency, and prosody~\cite{kopparapu2024spoken}, which are often challenging to assess consistently in traditional settings. However, this area still faces challenges, such as misalignment of assessment with expert instructors~\cite{kundu2024large} and a lack of empathy~\cite{sharma2024comuniqa}.

\paragraph{Automated Assessment.}  The task aims to automatically grade students’ assignments, including essay scoring~\cite{sessler2024can,li2024applying,syamkumar2024improving}, short answer grading~\cite{schneider2023towards,henkel2024can}, and spoken language evaluation~\cite{gao2023investigation,fu2024pronunciation}. LLMs can process learners’ submissions to judge grammar, lexical diversity, coherence, and even spoken fluency, providing instant feedback. This scalability is particularly appealing for large classes, where human evaluators are often overwhelmed and unable to provide timely, personalized critique~\cite{mizumoto2023exploring}.

% \paragraph{Knowledge Tracing.} Given sequences of learning interactions in online learning systems, Knowledge Tracing aims to monitor students’ evolving knowledge states during the learning process and predict their performance on future exercises. The measured knowledge states can be further applied to individualize students’ learning schemes to maximize their learning efficiency~\cite{shen2024survey,xu2023learning}. Recent studies have explored the incorporation of LLMs into knowledge tracing and have demonstrated the effectiveness and generalization in solving cold-start problems~\cite{zhan2024knowledge,jung2024clst}, but challenges still exist in this area~\cite{cho2024systematic}.


\paragraph{Knowledge Tracing.} Given sequences of learning interactions in online learning systems, Knowledge Tracing identifies and tracks students’ evolving mastery of target skills~\cite{shen2024survey,xu2023learning}. LLM-based methods have been explored in cold-start scenarios~\cite{zhan2024knowledge,jung2024clst}, offering strong generalization by inferring latent learner states from limited data. These approaches can support adaptive learning pathways, giving personalized recommendations based on predicted performance and knowledge gaps.


\paragraph{Discussion.} Despite their promise in automating and personalizing these discriminative tasks, LLMs still grapple with notable limitations that hinder their utility as robust tutoring tools. First, \emph{misalignment of assessment with expert instructors} poses risks: machine-generated scores may deviate from established rubrics or neglect qualitative nuances, leading to potential discrepancies in grading quality~\cite{kundu2024large}. Second, the \textit{lack of empathy} compounds this issue, as assessments devoid of human judgment risk discouraging learners or overlooking subtle motivational factors~\cite{sharma2024comuniqa}. Knowledge tracing approaches, while promising in cold-start scenarios, struggle with capturing the complexity of long-term learning trajectories and deeper cognitive processes~\cite{cho2024systematic}. These concerns point to the need for more transparent and human-centered methods in utilizing LLMs for assessment.

% Similarly, while LLM-based \textit{knowledge tracing} shows success in cold-start contexts, it remains unclear how well these models capture complex, long-term learner trajectories in real-world classrooms~\cite{cho2024systematic}. Critics argue that relying solely on algorithmic estimations of knowledge states may obscure students’ deeper cognitive processes, highlighting the need for more transparent and context-sensitive methods. Together, these concerns underscore that while LLMs offer scalable and immediate solutions, they must be carefully aligned with expert standards and learner-centered practices to realize their full potential in FLE.


\subsection{Generative Task Predictors}
Generative tasks involve producing new content or responses. LLMs are known to be adept in these tasks due to their natural language generation capabilities.

\paragraph{Grammatical Error Correction and Explanation.} In foreign language writing, errors often reveal learners’ gaps in grammar and vocabulary~\cite{hyland2006feedback}. LLMs can detect and correct these errors~\cite{bryant2023grammatical,ye-etal-2023-mixedit}, offering concise explanations~\cite{ye2024excgec} that reinforce language rules. By streamlining error detection and pedagogically framing corrections, learners deepen their linguistic understanding.

\paragraph{Feedback Generation.} Quizzes and exercises remain vital in FLE for practice and targeted remediation~\cite{rashov2024modern}. LLMs enhance this process by delivering prompt, personalized feedback that pinpoints strengths and addresses weaknesses~\cite{borges-etal-2024-teach}. This scalability enables learners to self-regulate and refine their skills without relying solely on human graders~\cite{stamper2024enhancing}.

\paragraph{Socratic Dialogue.} Moving beyond straightforward Q\&A, Socratic questioning promotes critical thinking and self-reflection~\cite{paul2007critical}. \textit{SocraticLM}~\cite{liusocraticlm}, for example, aligns an LLM with open-ended, inquiry-based teaching principles, guiding learners through iterative exploration rather than prescriptive correction. In theory, this fosters deeper conceptual understanding and active learner engagement.

\paragraph{Discussion.} Despite the promise of LLM-based generation in FLE, multiple uncertainties persist. \textit{Determining how to provide automatic feedback that genuinely maximizes learning outcomes} is an ongoing challenge~\cite{stamper2024enhancing}, particularly given education’s risk-averse culture and high accountability standards~\cite{xiao2024humanaicollaborativeessayscoring}. Moreover, while LLMs like SocraticLM have demonstrated success in domains like mathematics, their applicability to FLE contexts has not been thoroughly validated~\cite{liusocraticlm}. As such, the design of strategies and follow-up queries remains an open question in ensuring that these systems track and respond to learners' cognitive states effectively.


\subsection{Mixed Task Predictors}
Mixed tasks integrate discriminative and generative elements, requiring LLMs to evaluate learner inputs and generate meaningful feedback or suggestions. These tasks are particularly valuable in fostering an interactive and adaptive learning experience, as they bridge the gap between evaluation and instruction.

\paragraph{Automated Assessment with Feedback.} While discriminative systems for essay scoring and speech evaluation primarily focus on assigning grades, LLMs extend these capabilities by simultaneously generating formative feedback~\cite{katuka2024investigating,stahl2024exploring}. For example, an LLM can evaluate the coherence and lexical diversity of a written assignment, then offer specific revision strategies. In speaking practice, it can measure fluency and pronunciation accuracy while suggesting drills to refine intonation or stress patterns. Through this combination of scoring and tailored advice, learners gain a deeper understanding of their strengths and areas for improvement.

\paragraph{Error Analysis.} Error Analysis systematically uncovers and categorizes learners’ missteps, from syntactic lapses in writing to flawed pronunciations in speaking~\cite{james2013errors,erdougan2005contribution}. LLMs functioning in a mixed capacity can classify these errors and generate corrective guidance, providing revised sentences, clarifications of grammatical rules, or remediation exercises for identified weaknesses~\cite{myles2002second,mashoor2020error}. Such insight facilitates targeted interventions that enhance language proficiency across modalities, including reading and listening.

% \paragraph{Error Analysis.} Error Analysis is the systematic identification, classification, and explanation of errors made by learners in the process of acquiring a foreign language~\cite{james2013errors}. It is a critical component of language education, offering insights into learners' developmental stages and areas of difficulty~\cite{erdougan2005contribution}. This task is inherently mixed, as it requires discriminative capabilities to classify error types (e.g., grammatical, lexical, phonological) and generative capabilities to provide corrective feedback, explanations, and actionable improvement suggestions. For instance, in writing, LLMs can identify and categorize errors in syntax or word choice while generating revised sentences and explanations of the underlying grammar rules~\cite{myles2002second}. In reading, they can highlight misinterpretations of text and suggest strategies for improving comprehension~\cite{leu1982oral}. In listening and speaking~\cite{mashoor2020error}, LLMs can pinpoint pronunciation errors, mishearings, or unnatural phrasing, offering tailored exercises to enhance fluency and accuracy~\cite{kumar2025sealspeakererrorcorrection}. By addressing errors across these modalities, LLMs can deliver a holistic and adaptive learning experience that fosters deeper language proficiency.


\paragraph{Discussion.} Mixed-task systems hold promise by combining assessment and feedback generation, but they face notable challenges. One major issue is the \textit{weak alignment} between scoring mechanisms and the quality of feedback provided~\cite{stahl2024exploring}. For example, while essay scoring systems may deliver comprehensive evaluations, the feedback often lacks specificity, limiting its instructional value. Additionally, although error analysis has potential, \textit{the absence of standardized pedagogical benchmarks}, especially in oral tasks, hampers the reliability and comparability of LLM-based tools~\cite{leu1982oral}. Furthermore, by addressing errors across these modalities, LLMs can deliver a holistic and adaptive learning experience that fosters deeper language proficiency~\cite{zhao2024embodied}.


\begin{tcolorbox}[top=1pt, bottom=1pt, left=1pt, right=1pt]
\textbf{Our position.} While LLMs offer scalable solutions for task prediction in FLE, their current limitations—such as misalignment with expert assessments, lack of empathy, and weak alignment between assessment and feedback—require ongoing refinement. \textit{Future research} should focus on improving model transparency, enhancing the cultural and emotional sensitivity of LLMs, and refining task predictors to better reflect long-term learning trajectories and learner motivation. Additionally, developing standardized pedagogical benchmarks for error analysis, especially in oral tasks, will help ensure the consistency and reliability of LLM-generated feedback.
\end{tcolorbox}

\section{AI agents: Today and future}
\label{sec:agent}

We start by reviewing the state-of-the-art in AI agents, with a focus on LLM-based agents, and shares our perspectives on potential development that could lead to advanced AI agents.  While it is challenging to predict the trajectory of future AI development, this discussion lays the groundwork for the analyses and proposals in the following.  We will also briefly review prior work on safety of AI agents and FMs. 

\subsection{Advanced AI agents}

An AI agent is defined as ``anything that can be viewed as perceiving its environment through sensors and acting upon that environment through actuators'' \cite{russell2016artificial}.  Based on the observations received from its environment, the controller of an AI agent selects an action, which could range from uttering a word to executing a physical movement, such as the motion of a robotic arm.

LLMs are significantly accelerating the development of advanced AI agents \cite{xi2023rise,wang2024survey,sumers2024cognitive}. LLMs can function as controllers for these agents, utilizing their internal reasoning capabilities, such as those demonstrated by Chain-of-Thought \cite{wei2022chain}. Alternatively, LLMs can be utilized to solve individual subtasks, with an external controller orchestrating the overall plan by breaking the original task into multiple subtasks and coordinating their solutions. The external controller may use simple search methods \cite{yao2023tree,besta2024graph,wang2024math,wang2024multistep} or advanced methods, such as Monte Carlo Tree Search (MCTS) \cite{luo2024improve,zhang2024restmcts} and domain-independent planners \cite{guan2023leveraging,liu2023llmp,dagan2023dynamic}.

%selecting an output from multiple samples \cite{wang2023selfconsistency}, iteratively refining outputs \cite{madaan2023selfrefine}, and conducting debate \cite{du2023improving} or

Reasoning capabilities \cite{huang2023towards,qiao2023reasoning,plaat2024reasoning,xu2025large} are central to such controllers, as they involve searching for and planning sequences of actions to achieve a specified goal. An emerging direction is developing large reasoning models---FMs specifically optimized for reasoning tasks \cite{xu2025large}. Early FMs in this direction include o1 \cite{zhong2024evaluation}, OpenR \cite{wang2024openr}, and LLaMA-Berry \cite{zhang2024llamaberry}. Importantly, strong reasoning capabilities typically stem from inference-time computation rather than pre-training or fine-tuning \cite{ji2025testtime}.

This trend could eventually lead to the development of long-term planning agents (LTPAs), capable of planning over extended time horizons far more effectively than humans.  \citet{cohen2024regulating} warn that LTPAs could ``take humans out of the loop, if it has the opportunity, ... deceive humans and thwart human control'' to achieve their goals. Ensuring the safety of such agents is particularly challenging: their risks cannot be fully tested in real environments due to the inherent danger, nor in simulated environments, since they may behave harmlessly in testing to achieve their goals once deployed. As a result, \citet{cohen2024regulating} compellingly argue that LTPAs should never be developed.

Advanced AI agents may also evolve continually over time. Similar to humans, their cognitive processes may consist of dual systems: System~1, which makes instantaneous and intuitive decisions, and System~2, which performs slower but more deliberate reasoning \cite{kahneman2003perspective,ji2025testtime}. These two systems can interleave in their operations. For instance, System~2 may devise a plan, after which System~1 is updated or retrained to execute similar tasks in the future without requiring further planning \cite{yu2024distilling}. Over time, this enables System~2 to conduct more complex reasoning processes by bypassing previously learned steps.  For such continually learning AI agents, the distinction between training and inference becomes blurred.

In addition to search and planning, advanced AI agents will possess strategic reasoning capabilities \cite{zhang2024llm,feng2024survey,goktas2025strategic}, enabling them to interact with other AI agents and humans in cooperative or competitive ways \cite{guo2024large,jiang2024multimodal}. The effectiveness of multi-agent coordination has already been demonstrated with current LLMs through methods such as debate \cite{du2023improving} and dialog \cite{qian2024chatdev}, which help thems achieve better solutions than a single LLM could alone.  These showcase the potential for sophisticated strategic behavior in complex multi-agent environments.


\subsection{Safety of AI agents and FMs}

While our primary focus is on safety against existential risks, there is a substantial body of literature addressing other types of risks associated with AI agents. Here, we briefly review the prior work on the risks posed by AI agents and FMs along with the approaches to mitigate these risks.

Prior work has identified various risks and safety issues associated with FMs and other generative models \cite{chua2024ai,wang2024security,shayegani2023survey,longpre2024position}. These risks include the generation of toxic, harmful, biased, false, or misleading content, including hallucinations; violations of privacy, copyright, or other legal protections; misalignment with human instructions and values, including ethical and moral considerations; and vulnerabilities to adversarial attacks. 

There have been extensive efforts to develop approaches for mitigating these risks. These approaches include pre-training or fine-tuning with data selection \cite{albalak2024survey} and human feedback \cite{kaufmann2024survey}, establishing guardrails \cite{dong2024safeguarding}, and conducting empirical evaluations through testing \cite{chang2024survey} and red teaming \cite{lin2024achilles}. In particular, \citet{longpre2024position} advocate the importance of evaluation and red teaming by independent third parties.

A particularly relevant risk for advanced AI agents is misalignment, which can result in reward hacking and negative side effects \cite{skalse2022defining,ngo2024alignment,gabriel2020artificial,shen2023large,amodei2016concrete,taylor2020alignment}.  Namely, AI agents can maximize rewards by exploiting misspecifications or misalignment in the reward function, resulting in unintended and potentially high risk behaviors.  One approach to avoiding negative side effects is to avoid any side effect by ensuring that the actions have low impacts on the environment \cite{armstrong2012mathematics,armstrong2017low,amodei2016concrete}.  Representative measures of impact include attainable utility \cite{turner2020conservative,turner2020avoiding}, relative reachability \cite{krakovna2019penalizing}, and other reachability-based measures.  Reachability-base measures are grounded on the idea that reachability to certain states should be maintained, while attainable utility is to maintain the achievability of certain goals, which are different from the goal given to the agent.

%, should be maintained as a result of taking an action (e.g. as compared to not taking any actions).

While these impact measures provide clear guidance on how the safety of AI agents may be ensured, their applicability is limited to relatively simple environments such as grid worlds.  In particular, the requirement on the observability of states makes it difficult to apply existing impact measures to regulate advanced AI agents\footnote{There has been little research on impact measures under partial observability \cite{naiff2023low}.}, since they can operate in complex and open-ended environments that cannot be fully observed.  While the study on impact measures and other techniques towards AI safety remain crucial and can even contribute to mitigating existential risks, the magnitude of existential risks demands additional measures that are broadly applicable in that they require minimal knowledge and assumptions about the environments and the AI agents.

%catastrophic convergence conjecture: unaligned goals tend to have catastrophe-inducing policies because of power-seeking incentives \cite{turner2020catastrophic}
%optimal agents would try to gain control over their environment, because default is not preferable \cite{turner2021optimal}

%Instrumental convergence:  there are some instrumental goals likely to be pursued by almost any intelligent agent (self-preservation, resource acquisition) \cite{bostrom2012superintelligent}
%self-correction \cite{shinn2023reflexion}

% inference time computation (generate multiple responses, and select one): \cite{yu2024selfgenerated,wang2023selfconsistency,zhang2024generative,ankner2024critiqueoutloud}

%\subsection{Impact Minimization}
%\subsubsection{Reachability}
%Consider actions to have high impact if they make some states unreachable. 
%reachability of state states \cite{moldovan2012safe,eysenbach2018leave}
%reachability of safe regions \cite{mitchell2005time,gillula2012guaranteed,fisac2019general}
%- it is insensitive to the magnitude of the irreversible disruption
%- irreversible transitions can happen spontaneously (due to the forces of nature, the actions of other agents, etc).  the agent has an incentive to interfere to prevent them

%\cite{hadfield2021principal} Assume some of the features $K$ are not mentioned.  Then optimize under the constraint that the unmentioned features are not changed.
%\begin{align}
%    \min_{\phi\in\Phi} & U(\phi) \\
%    s.t. & \phi_K = \phi_K^{(0)} \\
%    & C(\phi) \le 0
%\end{align}

%\cite{krakovna2019penalizing} empirically compare various deviation measures: unreachability, relative reachability, attainable utility, value difference measures
%\cite{krakovna2020avoiding} Avoiding Side Effects By Considering Future Tasks

\section{Challenges when Guiding with History}
\label{sec:history_guidance_challenges}

% \begin{figure*}[!t]
%     \centering    
%     \includegraphics[width=.84\linewidth]{assets/detailed_architecture.pdf}
%     \caption{\textbf{Detailed architecture illustration} of \sname if it is incoporated with W.A.L.T~\citep{gupta2023photorealistic}. 
%     } 
%     \label{fig:detailed_architecture}
% \end{figure*}
\section{Sampling Procedure}
\label{appen:sampling}
We provide detailed sampling procedure of \sname in Algorithm~\ref{algo:sampling}.

\begin{algorithm}[h!]
\begin{spacing}{1.05}
\caption{\sname Diffusion}\label{algo:sampling}
\begin{algorithmic}[1]
\For{$n=1$ to $N$} \Comment{{\it Autoregressively generate $n$-th latent vector $\bz^n$.}}
\State Sample the random noise $\bz_T^{n} \sim p(\bz_T)$.
\For{$i$ in $\{0, \ldots, M-1\}$}
\State Compute the score $\bm{\epsilon}_i \leftarrow D_{\bm{\theta}}(\bz_{i}^{n}, t;\, \bh^{n-1}, \bc)$.
\State Compute $\bz_{i+1}^n \leftarrow \bz_i^n + (t_{i+1} - t_{i})\bm{\epsilon}_i$ \Comment{{\it Euler solver; can be different with other solvers.}}
\EndFor
\State $\bh^{n} = \mathrm{HiddenState}\big(D_{\bm{\theta}} (\bz_M^{n}, 0;\, \bh^{i-1}, \bc)\big)$ \Comment{{\it Compute memory latent vector.}}
\EndFor
\State Decode $[\bx^1,\ldots,\bx^{N}]$ from generated latent vectors $[\bz^1,\ldots,\bz^{N}]$.
\State Output the generated video $[\bx^1,\ldots,\bx^{N}]$.
\end{algorithmic}
\end{spacing}
\end{algorithm}
 


Video diffusion models are conditional diffusion models  $p(\bx|\bc)$, where $\bx$ denotes frames to be generated, and $\bc$ represents the conditioning (e.g. text prompt, or a few observed prior frames). For simplicity, we refer to the latter as \emph{history}, even when the observed images could be e.g. a subset of keyframes that are spaced across time. Our discussion of $\bc$ will focus exclusively on history conditioning and exclude text or other forms of conditioning in notation. Formally, let $\bx_{\cT}$ denote a $T$-frame video clips with indices $\cT = \{1, 2, \ldots, T\}$. Define $\cH \subset \cT$ as the indices of history frames used for conditioning, and $\cG = \cT \setminus \cH$ as the indices of the frames to be generated. Our objective is to model the conditional distribution $p(\xG | \xH)$ with a diffusion model. 

We aim to extend classifier-free guidance (CFG) to this setting. Since the history $\xH$ serves as conditioning, sampling can be performed by estimating the following score: 
\begin{equation} 
\label{eq:history_guidance}
\score p_k(\xGk)
+ \omega \big[\score p_k(\xGk|\xH)  - \score p_k(\xGk)\big].
\end{equation}
This approach differs from conventional CFG in two ways: 1) The generation $\xG$ and conditioning history $\xH$ belong to the same signal $\bx_{\cT}$, differing only in their indices $\cG, \cH\subset \cT$; thus, the generated $\xG$ can be reused as conditioning $\xH$ for generating subsequent frames. 2) The history $\xH$ can be any subset of $\cT$, allowing its length to vary. Guiding with history, therefore, requires a model that can estimate both conditional and unconditional scores given arbitrary subsets of video frames. Below, we analyze how these differences present challenges for implementation within the current paradigm of video diffusion models (VDMs).

\textbf{Architectures with fixed-length conditioning.}
As shown in \cref{fig:architecture-conventional}, DiT~\cite{peebles2023scalable} or U-Net-based diffusion models~\cite{bao2023all,rombach2022high} typically inject conditioning using AdaLN~\cite{peebles2023scalable, perez2018film} layers or by concatenating the conditioning with noisy input frames along the channel dimension.
This design constrains conditioning to a fixed-size vector. While some models adopt sequence encoders for variable-length conditioning (e.g., for text inputs), these encoders are often pre-trained~\cite{yang2024cogvideox} and cannot share parameters with the diffusion model to encode history frames. %
Consequently, guidance has been limited to fixed-length and generally short history~\cite{blattmann2023stable, xing2023dynamicrafter, yang2024cogvideox, watson2024controlling}.

\textbf{Framewise Binary Dropout performs poorly.} 
Classifier-free guidance is typically implemented using a single network that jointly represents the conditional and unconditional models. These are trained via \emph{binary dropout}, where the conditioning variable $\bc$ is randomly masked during training  with a certain probability. 
History guidance can, in principle, be achieved by randomly dropping out subsets of history frames during training.
However, our ablations (Sec.~\ref{sec:exp_ablation}) reveal that this approach performs poorly. We hypothesize that this is due to inefficient token utilization: although the model processes all $|\cT|$ frames via attention, only a random subset of $|\cG|$ frames contribute to the loss. This becomes more pronounced as videos grow longer, making framewise binary dropout a suboptimal choice.




\section{Conclusion}
In this work, we propose a simple yet effective approach, called SMILE, for graph few-shot learning with fewer tasks. Specifically, we introduce a novel dual-level mixup strategy, including within-task and across-task mixup, for enriching the diversity of nodes within each task and the diversity of tasks. Also, we incorporate the degree-based prior information to learn expressive node embeddings. Theoretically, we prove that SMILE effectively enhances the model's generalization performance. Empirically, we conduct extensive experiments on multiple benchmarks and the results suggest that SMILE significantly outperforms other baselines, including both in-domain and cross-domain few-shot settings.


% In the unusual situation where you want a paper to appear in the
% references without citing it in the main text, use \nocite
% \nocite{langley00}

\bibliography{example_paper}
\bibliographystyle{icml2025}


%%%%%%%%%%%%%%%%%%%%%%%%%%%%%%%%%%%%%%%%%%%%%%%%%%%%%%%%%%%%%%%%%%%%%%%%%%%%%%%
%%%%%%%%%%%%%%%%%%%%%%%%%%%%%%%%%%%%%%%%%%%%%%%%%%%%%%%%%%%%%%%%%%%%%%%%%%%%%%%
% APPENDIX
%%%%%%%%%%%%%%%%%%%%%%%%%%%%%%%%%%%%%%%%%%%%%%%%%%%%%%%%%%%%%%%%%%%%%%%%%%%%%%%
%%%%%%%%%%%%%%%%%%%%%%%%%%%%%%%%%%%%%%%%%%%%%%%%%%%%%%%%%%%%%%%%%%%%%%%%%%%%%%%
\subsection{Lloyd-Max Algorithm}
\label{subsec:Lloyd-Max}
For a given quantization bitwidth $B$ and an operand $\bm{X}$, the Lloyd-Max algorithm finds $2^B$ quantization levels $\{\hat{x}_i\}_{i=1}^{2^B}$ such that quantizing $\bm{X}$ by rounding each scalar in $\bm{X}$ to the nearest quantization level minimizes the quantization MSE. 

The algorithm starts with an initial guess of quantization levels and then iteratively computes quantization thresholds $\{\tau_i\}_{i=1}^{2^B-1}$ and updates quantization levels $\{\hat{x}_i\}_{i=1}^{2^B}$. Specifically, at iteration $n$, thresholds are set to the midpoints of the previous iteration's levels:
\begin{align*}
    \tau_i^{(n)}=\frac{\hat{x}_i^{(n-1)}+\hat{x}_{i+1}^{(n-1)}}2 \text{ for } i=1\ldots 2^B-1
\end{align*}
Subsequently, the quantization levels are re-computed as conditional means of the data regions defined by the new thresholds:
\begin{align*}
    \hat{x}_i^{(n)}=\mathbb{E}\left[ \bm{X} \big| \bm{X}\in [\tau_{i-1}^{(n)},\tau_i^{(n)}] \right] \text{ for } i=1\ldots 2^B
\end{align*}
where to satisfy boundary conditions we have $\tau_0=-\infty$ and $\tau_{2^B}=\infty$. The algorithm iterates the above steps until convergence.

Figure \ref{fig:lm_quant} compares the quantization levels of a $7$-bit floating point (E3M3) quantizer (left) to a $7$-bit Lloyd-Max quantizer (right) when quantizing a layer of weights from the GPT3-126M model at a per-tensor granularity. As shown, the Lloyd-Max quantizer achieves substantially lower quantization MSE. Further, Table \ref{tab:FP7_vs_LM7} shows the superior perplexity achieved by Lloyd-Max quantizers for bitwidths of $7$, $6$ and $5$. The difference between the quantizers is clear at 5 bits, where per-tensor FP quantization incurs a drastic and unacceptable increase in perplexity, while Lloyd-Max quantization incurs a much smaller increase. Nevertheless, we note that even the optimal Lloyd-Max quantizer incurs a notable ($\sim 1.5$) increase in perplexity due to the coarse granularity of quantization. 

\begin{figure}[h]
  \centering
  \includegraphics[width=0.7\linewidth]{sections/figures/LM7_FP7.pdf}
  \caption{\small Quantization levels and the corresponding quantization MSE of Floating Point (left) vs Lloyd-Max (right) Quantizers for a layer of weights in the GPT3-126M model.}
  \label{fig:lm_quant}
\end{figure}

\begin{table}[h]\scriptsize
\begin{center}
\caption{\label{tab:FP7_vs_LM7} \small Comparing perplexity (lower is better) achieved by floating point quantizers and Lloyd-Max quantizers on a GPT3-126M model for the Wikitext-103 dataset.}
\begin{tabular}{c|cc|c}
\hline
 \multirow{2}{*}{\textbf{Bitwidth}} & \multicolumn{2}{|c|}{\textbf{Floating-Point Quantizer}} & \textbf{Lloyd-Max Quantizer} \\
 & Best Format & Wikitext-103 Perplexity & Wikitext-103 Perplexity \\
\hline
7 & E3M3 & 18.32 & 18.27 \\
6 & E3M2 & 19.07 & 18.51 \\
5 & E4M0 & 43.89 & 19.71 \\
\hline
\end{tabular}
\end{center}
\end{table}

\subsection{Proof of Local Optimality of LO-BCQ}
\label{subsec:lobcq_opt_proof}
For a given block $\bm{b}_j$, the quantization MSE during LO-BCQ can be empirically evaluated as $\frac{1}{L_b}\lVert \bm{b}_j- \bm{\hat{b}}_j\rVert^2_2$ where $\bm{\hat{b}}_j$ is computed from equation (\ref{eq:clustered_quantization_definition}) as $C_{f(\bm{b}_j)}(\bm{b}_j)$. Further, for a given block cluster $\mathcal{B}_i$, we compute the quantization MSE as $\frac{1}{|\mathcal{B}_{i}|}\sum_{\bm{b} \in \mathcal{B}_{i}} \frac{1}{L_b}\lVert \bm{b}- C_i^{(n)}(\bm{b})\rVert^2_2$. Therefore, at the end of iteration $n$, we evaluate the overall quantization MSE $J^{(n)}$ for a given operand $\bm{X}$ composed of $N_c$ block clusters as:
\begin{align*}
    \label{eq:mse_iter_n}
    J^{(n)} = \frac{1}{N_c} \sum_{i=1}^{N_c} \frac{1}{|\mathcal{B}_{i}^{(n)}|}\sum_{\bm{v} \in \mathcal{B}_{i}^{(n)}} \frac{1}{L_b}\lVert \bm{b}- B_i^{(n)}(\bm{b})\rVert^2_2
\end{align*}

At the end of iteration $n$, the codebooks are updated from $\mathcal{C}^{(n-1)}$ to $\mathcal{C}^{(n)}$. However, the mapping of a given vector $\bm{b}_j$ to quantizers $\mathcal{C}^{(n)}$ remains as  $f^{(n)}(\bm{b}_j)$. At the next iteration, during the vector clustering step, $f^{(n+1)}(\bm{b}_j)$ finds new mapping of $\bm{b}_j$ to updated codebooks $\mathcal{C}^{(n)}$ such that the quantization MSE over the candidate codebooks is minimized. Therefore, we obtain the following result for $\bm{b}_j$:
\begin{align*}
\frac{1}{L_b}\lVert \bm{b}_j - C_{f^{(n+1)}(\bm{b}_j)}^{(n)}(\bm{b}_j)\rVert^2_2 \le \frac{1}{L_b}\lVert \bm{b}_j - C_{f^{(n)}(\bm{b}_j)}^{(n)}(\bm{b}_j)\rVert^2_2
\end{align*}

That is, quantizing $\bm{b}_j$ at the end of the block clustering step of iteration $n+1$ results in lower quantization MSE compared to quantizing at the end of iteration $n$. Since this is true for all $\bm{b} \in \bm{X}$, we assert the following:
\begin{equation}
\begin{split}
\label{eq:mse_ineq_1}
    \tilde{J}^{(n+1)} &= \frac{1}{N_c} \sum_{i=1}^{N_c} \frac{1}{|\mathcal{B}_{i}^{(n+1)}|}\sum_{\bm{b} \in \mathcal{B}_{i}^{(n+1)}} \frac{1}{L_b}\lVert \bm{b} - C_i^{(n)}(b)\rVert^2_2 \le J^{(n)}
\end{split}
\end{equation}
where $\tilde{J}^{(n+1)}$ is the the quantization MSE after the vector clustering step at iteration $n+1$.

Next, during the codebook update step (\ref{eq:quantizers_update}) at iteration $n+1$, the per-cluster codebooks $\mathcal{C}^{(n)}$ are updated to $\mathcal{C}^{(n+1)}$ by invoking the Lloyd-Max algorithm \citep{Lloyd}. We know that for any given value distribution, the Lloyd-Max algorithm minimizes the quantization MSE. Therefore, for a given vector cluster $\mathcal{B}_i$ we obtain the following result:

\begin{equation}
    \frac{1}{|\mathcal{B}_{i}^{(n+1)}|}\sum_{\bm{b} \in \mathcal{B}_{i}^{(n+1)}} \frac{1}{L_b}\lVert \bm{b}- C_i^{(n+1)}(\bm{b})\rVert^2_2 \le \frac{1}{|\mathcal{B}_{i}^{(n+1)}|}\sum_{\bm{b} \in \mathcal{B}_{i}^{(n+1)}} \frac{1}{L_b}\lVert \bm{b}- C_i^{(n)}(\bm{b})\rVert^2_2
\end{equation}

The above equation states that quantizing the given block cluster $\mathcal{B}_i$ after updating the associated codebook from $C_i^{(n)}$ to $C_i^{(n+1)}$ results in lower quantization MSE. Since this is true for all the block clusters, we derive the following result: 
\begin{equation}
\begin{split}
\label{eq:mse_ineq_2}
     J^{(n+1)} &= \frac{1}{N_c} \sum_{i=1}^{N_c} \frac{1}{|\mathcal{B}_{i}^{(n+1)}|}\sum_{\bm{b} \in \mathcal{B}_{i}^{(n+1)}} \frac{1}{L_b}\lVert \bm{b}- C_i^{(n+1)}(\bm{b})\rVert^2_2  \le \tilde{J}^{(n+1)}   
\end{split}
\end{equation}

Following (\ref{eq:mse_ineq_1}) and (\ref{eq:mse_ineq_2}), we find that the quantization MSE is non-increasing for each iteration, that is, $J^{(1)} \ge J^{(2)} \ge J^{(3)} \ge \ldots \ge J^{(M)}$ where $M$ is the maximum number of iterations. 
%Therefore, we can say that if the algorithm converges, then it must be that it has converged to a local minimum. 
\hfill $\blacksquare$


\begin{figure}
    \begin{center}
    \includegraphics[width=0.5\textwidth]{sections//figures/mse_vs_iter.pdf}
    \end{center}
    \caption{\small NMSE vs iterations during LO-BCQ compared to other block quantization proposals}
    \label{fig:nmse_vs_iter}
\end{figure}

Figure \ref{fig:nmse_vs_iter} shows the empirical convergence of LO-BCQ across several block lengths and number of codebooks. Also, the MSE achieved by LO-BCQ is compared to baselines such as MXFP and VSQ. As shown, LO-BCQ converges to a lower MSE than the baselines. Further, we achieve better convergence for larger number of codebooks ($N_c$) and for a smaller block length ($L_b$), both of which increase the bitwidth of BCQ (see Eq \ref{eq:bitwidth_bcq}).


\subsection{Additional Accuracy Results}
%Table \ref{tab:lobcq_config} lists the various LOBCQ configurations and their corresponding bitwidths.
\begin{table}
\setlength{\tabcolsep}{4.75pt}
\begin{center}
\caption{\label{tab:lobcq_config} Various LO-BCQ configurations and their bitwidths.}
\begin{tabular}{|c||c|c|c|c||c|c||c|} 
\hline
 & \multicolumn{4}{|c||}{$L_b=8$} & \multicolumn{2}{|c||}{$L_b=4$} & $L_b=2$ \\
 \hline
 \backslashbox{$L_A$\kern-1em}{\kern-1em$N_c$} & 2 & 4 & 8 & 16 & 2 & 4 & 2 \\
 \hline
 64 & 4.25 & 4.375 & 4.5 & 4.625 & 4.375 & 4.625 & 4.625\\
 \hline
 32 & 4.375 & 4.5 & 4.625& 4.75 & 4.5 & 4.75 & 4.75 \\
 \hline
 16 & 4.625 & 4.75& 4.875 & 5 & 4.75 & 5 & 5 \\
 \hline
\end{tabular}
\end{center}
\end{table}

%\subsection{Perplexity achieved by various LO-BCQ configurations on Wikitext-103 dataset}

\begin{table} \centering
\begin{tabular}{|c||c|c|c|c||c|c||c|} 
\hline
 $L_b \rightarrow$& \multicolumn{4}{c||}{8} & \multicolumn{2}{c||}{4} & 2\\
 \hline
 \backslashbox{$L_A$\kern-1em}{\kern-1em$N_c$} & 2 & 4 & 8 & 16 & 2 & 4 & 2  \\
 %$N_c \rightarrow$ & 2 & 4 & 8 & 16 & 2 & 4 & 2 \\
 \hline
 \hline
 \multicolumn{8}{c}{GPT3-1.3B (FP32 PPL = 9.98)} \\ 
 \hline
 \hline
 64 & 10.40 & 10.23 & 10.17 & 10.15 &  10.28 & 10.18 & 10.19 \\
 \hline
 32 & 10.25 & 10.20 & 10.15 & 10.12 &  10.23 & 10.17 & 10.17 \\
 \hline
 16 & 10.22 & 10.16 & 10.10 & 10.09 &  10.21 & 10.14 & 10.16 \\
 \hline
  \hline
 \multicolumn{8}{c}{GPT3-8B (FP32 PPL = 7.38)} \\ 
 \hline
 \hline
 64 & 7.61 & 7.52 & 7.48 &  7.47 &  7.55 &  7.49 & 7.50 \\
 \hline
 32 & 7.52 & 7.50 & 7.46 &  7.45 &  7.52 &  7.48 & 7.48  \\
 \hline
 16 & 7.51 & 7.48 & 7.44 &  7.44 &  7.51 &  7.49 & 7.47  \\
 \hline
\end{tabular}
\caption{\label{tab:ppl_gpt3_abalation} Wikitext-103 perplexity across GPT3-1.3B and 8B models.}
\end{table}

\begin{table} \centering
\begin{tabular}{|c||c|c|c|c||} 
\hline
 $L_b \rightarrow$& \multicolumn{4}{c||}{8}\\
 \hline
 \backslashbox{$L_A$\kern-1em}{\kern-1em$N_c$} & 2 & 4 & 8 & 16 \\
 %$N_c \rightarrow$ & 2 & 4 & 8 & 16 & 2 & 4 & 2 \\
 \hline
 \hline
 \multicolumn{5}{|c|}{Llama2-7B (FP32 PPL = 5.06)} \\ 
 \hline
 \hline
 64 & 5.31 & 5.26 & 5.19 & 5.18  \\
 \hline
 32 & 5.23 & 5.25 & 5.18 & 5.15  \\
 \hline
 16 & 5.23 & 5.19 & 5.16 & 5.14  \\
 \hline
 \multicolumn{5}{|c|}{Nemotron4-15B (FP32 PPL = 5.87)} \\ 
 \hline
 \hline
 64  & 6.3 & 6.20 & 6.13 & 6.08  \\
 \hline
 32  & 6.24 & 6.12 & 6.07 & 6.03  \\
 \hline
 16  & 6.12 & 6.14 & 6.04 & 6.02  \\
 \hline
 \multicolumn{5}{|c|}{Nemotron4-340B (FP32 PPL = 3.48)} \\ 
 \hline
 \hline
 64 & 3.67 & 3.62 & 3.60 & 3.59 \\
 \hline
 32 & 3.63 & 3.61 & 3.59 & 3.56 \\
 \hline
 16 & 3.61 & 3.58 & 3.57 & 3.55 \\
 \hline
\end{tabular}
\caption{\label{tab:ppl_llama7B_nemo15B} Wikitext-103 perplexity compared to FP32 baseline in Llama2-7B and Nemotron4-15B, 340B models}
\end{table}

%\subsection{Perplexity achieved by various LO-BCQ configurations on MMLU dataset}


\begin{table} \centering
\begin{tabular}{|c||c|c|c|c||c|c|c|c|} 
\hline
 $L_b \rightarrow$& \multicolumn{4}{c||}{8} & \multicolumn{4}{c||}{8}\\
 \hline
 \backslashbox{$L_A$\kern-1em}{\kern-1em$N_c$} & 2 & 4 & 8 & 16 & 2 & 4 & 8 & 16  \\
 %$N_c \rightarrow$ & 2 & 4 & 8 & 16 & 2 & 4 & 2 \\
 \hline
 \hline
 \multicolumn{5}{|c|}{Llama2-7B (FP32 Accuracy = 45.8\%)} & \multicolumn{4}{|c|}{Llama2-70B (FP32 Accuracy = 69.12\%)} \\ 
 \hline
 \hline
 64 & 43.9 & 43.4 & 43.9 & 44.9 & 68.07 & 68.27 & 68.17 & 68.75 \\
 \hline
 32 & 44.5 & 43.8 & 44.9 & 44.5 & 68.37 & 68.51 & 68.35 & 68.27  \\
 \hline
 16 & 43.9 & 42.7 & 44.9 & 45 & 68.12 & 68.77 & 68.31 & 68.59  \\
 \hline
 \hline
 \multicolumn{5}{|c|}{GPT3-22B (FP32 Accuracy = 38.75\%)} & \multicolumn{4}{|c|}{Nemotron4-15B (FP32 Accuracy = 64.3\%)} \\ 
 \hline
 \hline
 64 & 36.71 & 38.85 & 38.13 & 38.92 & 63.17 & 62.36 & 63.72 & 64.09 \\
 \hline
 32 & 37.95 & 38.69 & 39.45 & 38.34 & 64.05 & 62.30 & 63.8 & 64.33  \\
 \hline
 16 & 38.88 & 38.80 & 38.31 & 38.92 & 63.22 & 63.51 & 63.93 & 64.43  \\
 \hline
\end{tabular}
\caption{\label{tab:mmlu_abalation} Accuracy on MMLU dataset across GPT3-22B, Llama2-7B, 70B and Nemotron4-15B models.}
\end{table}


%\subsection{Perplexity achieved by various LO-BCQ configurations on LM evaluation harness}

\begin{table} \centering
\begin{tabular}{|c||c|c|c|c||c|c|c|c|} 
\hline
 $L_b \rightarrow$& \multicolumn{4}{c||}{8} & \multicolumn{4}{c||}{8}\\
 \hline
 \backslashbox{$L_A$\kern-1em}{\kern-1em$N_c$} & 2 & 4 & 8 & 16 & 2 & 4 & 8 & 16  \\
 %$N_c \rightarrow$ & 2 & 4 & 8 & 16 & 2 & 4 & 2 \\
 \hline
 \hline
 \multicolumn{5}{|c|}{Race (FP32 Accuracy = 37.51\%)} & \multicolumn{4}{|c|}{Boolq (FP32 Accuracy = 64.62\%)} \\ 
 \hline
 \hline
 64 & 36.94 & 37.13 & 36.27 & 37.13 & 63.73 & 62.26 & 63.49 & 63.36 \\
 \hline
 32 & 37.03 & 36.36 & 36.08 & 37.03 & 62.54 & 63.51 & 63.49 & 63.55  \\
 \hline
 16 & 37.03 & 37.03 & 36.46 & 37.03 & 61.1 & 63.79 & 63.58 & 63.33  \\
 \hline
 \hline
 \multicolumn{5}{|c|}{Winogrande (FP32 Accuracy = 58.01\%)} & \multicolumn{4}{|c|}{Piqa (FP32 Accuracy = 74.21\%)} \\ 
 \hline
 \hline
 64 & 58.17 & 57.22 & 57.85 & 58.33 & 73.01 & 73.07 & 73.07 & 72.80 \\
 \hline
 32 & 59.12 & 58.09 & 57.85 & 58.41 & 73.01 & 73.94 & 72.74 & 73.18  \\
 \hline
 16 & 57.93 & 58.88 & 57.93 & 58.56 & 73.94 & 72.80 & 73.01 & 73.94  \\
 \hline
\end{tabular}
\caption{\label{tab:mmlu_abalation} Accuracy on LM evaluation harness tasks on GPT3-1.3B model.}
\end{table}

\begin{table} \centering
\begin{tabular}{|c||c|c|c|c||c|c|c|c|} 
\hline
 $L_b \rightarrow$& \multicolumn{4}{c||}{8} & \multicolumn{4}{c||}{8}\\
 \hline
 \backslashbox{$L_A$\kern-1em}{\kern-1em$N_c$} & 2 & 4 & 8 & 16 & 2 & 4 & 8 & 16  \\
 %$N_c \rightarrow$ & 2 & 4 & 8 & 16 & 2 & 4 & 2 \\
 \hline
 \hline
 \multicolumn{5}{|c|}{Race (FP32 Accuracy = 41.34\%)} & \multicolumn{4}{|c|}{Boolq (FP32 Accuracy = 68.32\%)} \\ 
 \hline
 \hline
 64 & 40.48 & 40.10 & 39.43 & 39.90 & 69.20 & 68.41 & 69.45 & 68.56 \\
 \hline
 32 & 39.52 & 39.52 & 40.77 & 39.62 & 68.32 & 67.43 & 68.17 & 69.30  \\
 \hline
 16 & 39.81 & 39.71 & 39.90 & 40.38 & 68.10 & 66.33 & 69.51 & 69.42  \\
 \hline
 \hline
 \multicolumn{5}{|c|}{Winogrande (FP32 Accuracy = 67.88\%)} & \multicolumn{4}{|c|}{Piqa (FP32 Accuracy = 78.78\%)} \\ 
 \hline
 \hline
 64 & 66.85 & 66.61 & 67.72 & 67.88 & 77.31 & 77.42 & 77.75 & 77.64 \\
 \hline
 32 & 67.25 & 67.72 & 67.72 & 67.00 & 77.31 & 77.04 & 77.80 & 77.37  \\
 \hline
 16 & 68.11 & 68.90 & 67.88 & 67.48 & 77.37 & 78.13 & 78.13 & 77.69  \\
 \hline
\end{tabular}
\caption{\label{tab:mmlu_abalation} Accuracy on LM evaluation harness tasks on GPT3-8B model.}
\end{table}

\begin{table} \centering
\begin{tabular}{|c||c|c|c|c||c|c|c|c|} 
\hline
 $L_b \rightarrow$& \multicolumn{4}{c||}{8} & \multicolumn{4}{c||}{8}\\
 \hline
 \backslashbox{$L_A$\kern-1em}{\kern-1em$N_c$} & 2 & 4 & 8 & 16 & 2 & 4 & 8 & 16  \\
 %$N_c \rightarrow$ & 2 & 4 & 8 & 16 & 2 & 4 & 2 \\
 \hline
 \hline
 \multicolumn{5}{|c|}{Race (FP32 Accuracy = 40.67\%)} & \multicolumn{4}{|c|}{Boolq (FP32 Accuracy = 76.54\%)} \\ 
 \hline
 \hline
 64 & 40.48 & 40.10 & 39.43 & 39.90 & 75.41 & 75.11 & 77.09 & 75.66 \\
 \hline
 32 & 39.52 & 39.52 & 40.77 & 39.62 & 76.02 & 76.02 & 75.96 & 75.35  \\
 \hline
 16 & 39.81 & 39.71 & 39.90 & 40.38 & 75.05 & 73.82 & 75.72 & 76.09  \\
 \hline
 \hline
 \multicolumn{5}{|c|}{Winogrande (FP32 Accuracy = 70.64\%)} & \multicolumn{4}{|c|}{Piqa (FP32 Accuracy = 79.16\%)} \\ 
 \hline
 \hline
 64 & 69.14 & 70.17 & 70.17 & 70.56 & 78.24 & 79.00 & 78.62 & 78.73 \\
 \hline
 32 & 70.96 & 69.69 & 71.27 & 69.30 & 78.56 & 79.49 & 79.16 & 78.89  \\
 \hline
 16 & 71.03 & 69.53 & 69.69 & 70.40 & 78.13 & 79.16 & 79.00 & 79.00  \\
 \hline
\end{tabular}
\caption{\label{tab:mmlu_abalation} Accuracy on LM evaluation harness tasks on GPT3-22B model.}
\end{table}

\begin{table} \centering
\begin{tabular}{|c||c|c|c|c||c|c|c|c|} 
\hline
 $L_b \rightarrow$& \multicolumn{4}{c||}{8} & \multicolumn{4}{c||}{8}\\
 \hline
 \backslashbox{$L_A$\kern-1em}{\kern-1em$N_c$} & 2 & 4 & 8 & 16 & 2 & 4 & 8 & 16  \\
 %$N_c \rightarrow$ & 2 & 4 & 8 & 16 & 2 & 4 & 2 \\
 \hline
 \hline
 \multicolumn{5}{|c|}{Race (FP32 Accuracy = 44.4\%)} & \multicolumn{4}{|c|}{Boolq (FP32 Accuracy = 79.29\%)} \\ 
 \hline
 \hline
 64 & 42.49 & 42.51 & 42.58 & 43.45 & 77.58 & 77.37 & 77.43 & 78.1 \\
 \hline
 32 & 43.35 & 42.49 & 43.64 & 43.73 & 77.86 & 75.32 & 77.28 & 77.86  \\
 \hline
 16 & 44.21 & 44.21 & 43.64 & 42.97 & 78.65 & 77 & 76.94 & 77.98  \\
 \hline
 \hline
 \multicolumn{5}{|c|}{Winogrande (FP32 Accuracy = 69.38\%)} & \multicolumn{4}{|c|}{Piqa (FP32 Accuracy = 78.07\%)} \\ 
 \hline
 \hline
 64 & 68.9 & 68.43 & 69.77 & 68.19 & 77.09 & 76.82 & 77.09 & 77.86 \\
 \hline
 32 & 69.38 & 68.51 & 68.82 & 68.90 & 78.07 & 76.71 & 78.07 & 77.86  \\
 \hline
 16 & 69.53 & 67.09 & 69.38 & 68.90 & 77.37 & 77.8 & 77.91 & 77.69  \\
 \hline
\end{tabular}
\caption{\label{tab:mmlu_abalation} Accuracy on LM evaluation harness tasks on Llama2-7B model.}
\end{table}

\begin{table} \centering
\begin{tabular}{|c||c|c|c|c||c|c|c|c|} 
\hline
 $L_b \rightarrow$& \multicolumn{4}{c||}{8} & \multicolumn{4}{c||}{8}\\
 \hline
 \backslashbox{$L_A$\kern-1em}{\kern-1em$N_c$} & 2 & 4 & 8 & 16 & 2 & 4 & 8 & 16  \\
 %$N_c \rightarrow$ & 2 & 4 & 8 & 16 & 2 & 4 & 2 \\
 \hline
 \hline
 \multicolumn{5}{|c|}{Race (FP32 Accuracy = 48.8\%)} & \multicolumn{4}{|c|}{Boolq (FP32 Accuracy = 85.23\%)} \\ 
 \hline
 \hline
 64 & 49.00 & 49.00 & 49.28 & 48.71 & 82.82 & 84.28 & 84.03 & 84.25 \\
 \hline
 32 & 49.57 & 48.52 & 48.33 & 49.28 & 83.85 & 84.46 & 84.31 & 84.93  \\
 \hline
 16 & 49.85 & 49.09 & 49.28 & 48.99 & 85.11 & 84.46 & 84.61 & 83.94  \\
 \hline
 \hline
 \multicolumn{5}{|c|}{Winogrande (FP32 Accuracy = 79.95\%)} & \multicolumn{4}{|c|}{Piqa (FP32 Accuracy = 81.56\%)} \\ 
 \hline
 \hline
 64 & 78.77 & 78.45 & 78.37 & 79.16 & 81.45 & 80.69 & 81.45 & 81.5 \\
 \hline
 32 & 78.45 & 79.01 & 78.69 & 80.66 & 81.56 & 80.58 & 81.18 & 81.34  \\
 \hline
 16 & 79.95 & 79.56 & 79.79 & 79.72 & 81.28 & 81.66 & 81.28 & 80.96  \\
 \hline
\end{tabular}
\caption{\label{tab:mmlu_abalation} Accuracy on LM evaluation harness tasks on Llama2-70B model.}
\end{table}

%\section{MSE Studies}
%\textcolor{red}{TODO}


\subsection{Number Formats and Quantization Method}
\label{subsec:numFormats_quantMethod}
\subsubsection{Integer Format}
An $n$-bit signed integer (INT) is typically represented with a 2s-complement format \citep{yao2022zeroquant,xiao2023smoothquant,dai2021vsq}, where the most significant bit denotes the sign.

\subsubsection{Floating Point Format}
An $n$-bit signed floating point (FP) number $x$ comprises of a 1-bit sign ($x_{\mathrm{sign}}$), $B_m$-bit mantissa ($x_{\mathrm{mant}}$) and $B_e$-bit exponent ($x_{\mathrm{exp}}$) such that $B_m+B_e=n-1$. The associated constant exponent bias ($E_{\mathrm{bias}}$) is computed as $(2^{{B_e}-1}-1)$. We denote this format as $E_{B_e}M_{B_m}$.  

\subsubsection{Quantization Scheme}
\label{subsec:quant_method}
A quantization scheme dictates how a given unquantized tensor is converted to its quantized representation. We consider FP formats for the purpose of illustration. Given an unquantized tensor $\bm{X}$ and an FP format $E_{B_e}M_{B_m}$, we first, we compute the quantization scale factor $s_X$ that maps the maximum absolute value of $\bm{X}$ to the maximum quantization level of the $E_{B_e}M_{B_m}$ format as follows:
\begin{align}
\label{eq:sf}
    s_X = \frac{\mathrm{max}(|\bm{X}|)}{\mathrm{max}(E_{B_e}M_{B_m})}
\end{align}
In the above equation, $|\cdot|$ denotes the absolute value function.

Next, we scale $\bm{X}$ by $s_X$ and quantize it to $\hat{\bm{X}}$ by rounding it to the nearest quantization level of $E_{B_e}M_{B_m}$ as:

\begin{align}
\label{eq:tensor_quant}
    \hat{\bm{X}} = \text{round-to-nearest}\left(\frac{\bm{X}}{s_X}, E_{B_e}M_{B_m}\right)
\end{align}

We perform dynamic max-scaled quantization \citep{wu2020integer}, where the scale factor $s$ for activations is dynamically computed during runtime.

\subsection{Vector Scaled Quantization}
\begin{wrapfigure}{r}{0.35\linewidth}
  \centering
  \includegraphics[width=\linewidth]{sections/figures/vsquant.jpg}
  \caption{\small Vectorwise decomposition for per-vector scaled quantization (VSQ \citep{dai2021vsq}).}
  \label{fig:vsquant}
\end{wrapfigure}
During VSQ \citep{dai2021vsq}, the operand tensors are decomposed into 1D vectors in a hardware friendly manner as shown in Figure \ref{fig:vsquant}. Since the decomposed tensors are used as operands in matrix multiplications during inference, it is beneficial to perform this decomposition along the reduction dimension of the multiplication. The vectorwise quantization is performed similar to tensorwise quantization described in Equations \ref{eq:sf} and \ref{eq:tensor_quant}, where a scale factor $s_v$ is required for each vector $\bm{v}$ that maps the maximum absolute value of that vector to the maximum quantization level. While smaller vector lengths can lead to larger accuracy gains, the associated memory and computational overheads due to the per-vector scale factors increases. To alleviate these overheads, VSQ \citep{dai2021vsq} proposed a second level quantization of the per-vector scale factors to unsigned integers, while MX \citep{rouhani2023shared} quantizes them to integer powers of 2 (denoted as $2^{INT}$).

\subsubsection{MX Format}
The MX format proposed in \citep{rouhani2023microscaling} introduces the concept of sub-block shifting. For every two scalar elements of $b$-bits each, there is a shared exponent bit. The value of this exponent bit is determined through an empirical analysis that targets minimizing quantization MSE. We note that the FP format $E_{1}M_{b}$ is strictly better than MX from an accuracy perspective since it allocates a dedicated exponent bit to each scalar as opposed to sharing it across two scalars. Therefore, we conservatively bound the accuracy of a $b+2$-bit signed MX format with that of a $E_{1}M_{b}$ format in our comparisons. For instance, we use E1M2 format as a proxy for MX4.

\begin{figure}
    \centering
    \includegraphics[width=1\linewidth]{sections//figures/BlockFormats.pdf}
    \caption{\small Comparing LO-BCQ to MX format.}
    \label{fig:block_formats}
\end{figure}

Figure \ref{fig:block_formats} compares our $4$-bit LO-BCQ block format to MX \citep{rouhani2023microscaling}. As shown, both LO-BCQ and MX decompose a given operand tensor into block arrays and each block array into blocks. Similar to MX, we find that per-block quantization ($L_b < L_A$) leads to better accuracy due to increased flexibility. While MX achieves this through per-block $1$-bit micro-scales, we associate a dedicated codebook to each block through a per-block codebook selector. Further, MX quantizes the per-block array scale-factor to E8M0 format without per-tensor scaling. In contrast during LO-BCQ, we find that per-tensor scaling combined with quantization of per-block array scale-factor to E4M3 format results in superior inference accuracy across models. 


\end{document}


% This document was modified from the file originally made available by
% Pat Langley and Andrea Danyluk for ICML-2K. This version was created
% by Iain Murray in 2018, and modified by Alexandre Bouchard in
% 2019 and 2021 and by Csaba Szepesvari, Gang Niu and Sivan Sabato in 2022.
% Modified again in 2023 and 2024 by Sivan Sabato and Jonathan Scarlett.
% Previous contributors include Dan Roy, Lise Getoor and Tobias
% Scheffer, which was slightly modified from the 2010 version by
% Thorsten Joachims & Johannes Fuernkranz, slightly modified from the
% 2009 version by Kiri Wagstaff and Sam Roweis's 2008 version, which is
% slightly modified from Prasad Tadepalli's 2007 version which is a
% lightly changed version of the previous year's version by Andrew
% Moore, which was in turn edited from those of Kristian Kersting and
% Codrina Lauth. Alex Smola contributed to the algorithmic style files.
