\section{LLM-empowered Agent}\label{sec:agent}
In this section, we delve into the potential of LLMs as intelligent agents in FLE. LLMs can act as catalysts for personalized learning, addressing the long-standing scalability, adaptability, and inclusivity challenges in traditional teaching paradigms.

\subsection{Fundamental Abilities}
This section highlights five key abilities of LLM-empowered agents that enable them to function as adaptive tutors.

\paragraph{Knowledge Integration.} LLMs excel at merging structured educational knowledge graphs~\cite{abu2024knowledge,hu2024foke} with unstructured textual data~\cite{li2024supporting,modran2024llm}, providing rich, contextualized information on linguistic constructs and cultural nuances. Their ability to perform real-time knowledge editing~\cite{wang2024knowledge,zhang2024comprehensive} ensures learners receive content aligned with evolving language usage, addressing the inherent limitations of static materials.

\paragraph{Pedagogical Alignment.} LLMs require embedding with pedagogical principles to facilitate genuine learning experiences~\cite{carroll1965contributions,taneja1995educational}. Recent work incorporates theoretical frameworks, such as Bloom’s taxonomy~\cite{bloom1956taxonomy}, to guide LLMs in systematically addressing different cognitive levels~\cite{jiang2024llms}. Approaches like \textit{Pedagogical Chain of Thought}~\cite{jiang2024llms} and \textit{preference learning}~\cite{sonkar-etal-2024-pedagogical,rafailov2024direct} focus on aligning model responses with educational objectives.

% \paragraph{Pedagogical Alignment.} To truly function as effective educational agents, LLMs must be imbued with pedagogical modes of thought~\cite{carroll1965contributions,taneja1995educational}, which can be achieved through pedagogical alignment~\cite{razafinirina2024pedagogical}. Inspired by Bloom Cognitive Model~\cite{bloom1956taxonomy}, which categorizes student abilities into six cognitive levels—Remember, Understand, Apply, Analyze, Evaluate, and Create—\citet{jiang2024llms} proposes \textit{Pedagogical Chain of Thought} to enhance mistake correction of LLMs without requiring fine-turning. \citet{sonkar-etal-2024-pedagogical} develop a synthetic preference dataset embedded with pedagogical principles and explore whether preference learning techniques~\cite{rafailov2024direct,azar2024general,ethayarajh2024kto} can align LLMs with educational objectives. The conversational CLASS~\cite{sonkar-etal-2023-class} framework equips intelligent tutoring systems with tutor-like step-by-step guiding strategies rather than merely providing direct answers to students.


\paragraph{Planning.} By assisting in crafting teaching objectives and lesson designs, LLMs can handle complex tasks such as differentiated instruction~\cite{hu2024teaching}. LessonPlanner~\cite{fan2024lessonplanner} has been proposed to assist novice teachers in preparing lesson plans, with expert interviews confirming its effectiveness. \citet{zheng2024automatic} propose a three-stage process to produce customized lesson plans, using Retrieval-Augmented Generation (RAG), self-critique, and subsequent refinement.

\paragraph{Memory.} Effective tutoring systems track learner histories and tailor subsequent interactions accordingly~\cite{jiang2024ai,chen2024empowering}. When serving as memory-augmented agents, LLMs can retain individualized data—such as repeated grammar mistakes or overlooked vocabulary—thereby improving continuity and enabling consistent scaffolding of future learning tasks.

\paragraph{Tool Using.} Beyond textual interactions, LLM-based agents can integrate specialized tools to streamline the educational ecosystem, from cognitive diagnosis modules~\cite{ma2019cognitive} to report generators~\cite{zhou2025study}. By orchestrating these resources, LLMs seamlessly unify diverse utilities under a single interface, enhancing learner experience and instructional efficiency.

\paragraph{Discussion.} Existing research often overlooks the interplay among listening, speaking, reading, and writing in real-world language learning~\cite{caines2023application,shetye2024evaluation}. Most systems concentrate on text-oriented features, lacking robust benchmarks and methodologies to evaluate integrated, multimodal interactions. Recent work on pedagogical alignment~\cite{razafinirina2024pedagogical} largely addresses textual data, leaving out real-time speaking and listening dynamics that demand complex, rapid feedback. Similarly, while memory modules can track repeated written errors, their effectiveness in monitoring and improving learning efficacy remains underexplored.


% \paragraph{Planning.} Planning is a cornerstone of successful education, involving setting instructional objectives, identifying teaching priorities, organizing teaching activities, articulating subject content, and selecting methods and strategies~\cite{hu2024teaching}. The design of teaching plans entails an abundant knowledge base and rich teaching experiences to tailor to learners' diverse needs,  with the constraint of available resources. Recent advances in LLMs demonstrate significant potential in planning complex tasks~\cite{huang2024understanding}. LessonPlanner~\cite{fan2024lessonplanner} has been proposed to assist novice teachers in preparing lesson plans, with expert interviews confirming its effectiveness. \citet{zheng2024automatic} propose a three-stage process to produce customized lesson plans, using Retrieval-Augmented Generation (RAG), self-critique, and subsequent refinement.

% \paragraph{Memory.} The ability to retain and recall information about a learner's progress is essential for fostering consistent and personalized education. LLMs can serve as memory-augmented agents~\cite{jiang2024ai}, maintaining detailed records of a learner's strengths, weaknesses, and historical performance. By leveraging this memory, LLMs can provide targeted feedback, revisit challenging topics, and build on prior knowledge in a coherent manner~\cite{chen2024empowering}. This continuity not only enhances the learner's experience but also ensures that educational interventions are data-driven and evidence-based.

% \paragraph{Tool Using.} LLMs can further extend their utility by interfacing with a variety of educational tools, such as cognitive diagnosis~\cite{ma2019cognitive} and report generation~\cite{zhou2025study}. LLMs can integrate these tools into a cohesive learning ecosystem by serving as a central orchestrator, streamlining the learner's experience. This ability to seamlessly incorporate external resources amplifies the effectiveness of LLM-driven educational systems.


\subsection{Applications}
Although still in its early stages, LLM-empowered agents have already started to show promising applications in FLE.
% These applications leverage the capabilities of LLMs to create more interactive and personalized learning environments, offering new ways to address traditional challenges in language instruction. Despite being in development, these applications have the potential to transform how language learners engage with content and interact with tutoring systems.


\paragraph{Classroom Simulation.} Classroom simulation leverages LLM-empowered agents to recreate complex, interactive learning settings without the logistical hurdles of organizing physical classrooms~\cite{zhang2024simulating}. By simulating virtual students and tutors, researchers can study pedagogical strategies at scale, generate diverse learner interactions, and refine teaching techniques. Moreover, this virtual data can be used to fine-tune LLMs for specific educational contexts and learner profiles~\cite{liusocraticlm}, offering a cost-effective and adaptable approach to language instruction.

\paragraph{Intelligent Tutoring System (ITS).} LLM-based agents have demonstrated the capacity to provide dynamic, personalized tutoring experiences~\cite{kwon-etal-2024-biped}, effectively identifying learner weaknesses through large-scale linguistic analysis~\cite{caines2023application}. This makes them promising for delivering individualized instruction at scale. Although current ITS applications in mathematics~\cite{pal2024autotutor} and science~\cite{stamper2024enhancing} have shown success, the extension to FLE requires nuanced handling of cultural and contextual elements, as well as the unpredictability of human language usage.

\paragraph{Discussion.} Despite the promise of these applications, critical challenges remain. Existing classroom simulation frameworks often \textit{lack standardized benchmarks for FLE}, making it difficult to assess the efficacy and generalizability of developed systems~\cite{zhang2024simulating}. In addition, evaluating language-specific tutoring strategies, including real-time conversational practice and holistic skill integration, remains an underexplored frontier. Addressing these gaps requires \textit{new datasets and metrics} centered on holistic skill development, as well as interdisciplinary collaboration.

\begin{tcolorbox}[top=1pt, bottom=1pt, left=1pt, right=1pt]
\textbf{Our position.} We argue that to overcome current limitations, \textit{future research} should focus on the integration of multimodal learning tasks~\cite{sonlu2024effects} and the development of standardized frameworks for evaluating FLE simulations. Moreover, LLMs should evolve beyond text-based capabilities to provide real-time, context-sensitive feedback, particularly in speaking and listening. Interdisciplinary collaboration and the creation of new datasets tailored to FLE are crucial for refining these systems and ensuring their scalability and inclusivity in language instruction. Additionally, addressing the complexities of cultural context and learner variability will be key to the success of LLMs as effective agents in FLE.
\end{tcolorbox}


% \paragraph{Classroom Simulation.} Classroom simulation is a powerful application of LLM-empowered agents that addresses the limitations of traditional language learning environments. By simulating interactions among virtual students and tutors, multi-agent setups create realistic, dynamic classroom scenarios that reflect real-world complexities. This approach not only facilitates the study of pedagogical strategies but also accelerates educational research by eliminating the logistical challenges of organizing physical classrooms~\cite{zhang2024simulating}. Additionally, simulated classroom data can be used to fine-tune LLMs, enhancing their ability to adapt to diverse teaching contexts and learner profiles~\cite{liusocraticlm}. These simulations offer scalability, cost-efficiency, and the opportunity to test various instructional methods in controlled yet realistic environments.

% \paragraph{Intelligent Tutoring System (ITS).} LLM-powered agents have the potential to create an immersive learning environment and provide a personalized alternative to traditional one-on-one tutoring~\cite{kwon-etal-2024-biped}. Furthermore, their ability to process vast amounts of linguistic data enables them to identify and address specific learner weaknesses~\cite{caines2023application}, making them particularly effective in delivering tailored instruction at scale. However, recent efforts are mainly on non-language subjects like mathematics~\cite{pal2024autotutor} and science~\cite{stamper2024enhancing} which hold more objective learning goals.
