\begin{abstract}
While recent efforts have begun integrating large language models (LLMs) into foreign language education (FLE), they often rely on traditional approaches to learning tasks without fully embracing educational methodologies, thus lacking adaptability to language learning. To address this gap, we argue that \textbf{LLMs have the potential to serve as effective tutors in FLE}. Specifically, LLMs can play three critical roles: (1) as \textit{data enhancers}, improving the creation of learning materials or serving as student simulations; (2) as \textit{task predictors}, serving as learner assessment or optimizing learning pathway; and (3) as \textit{agents}, enabling personalized and inclusive education. We encourage interdisciplinary research to explore these roles, fostering innovation while addressing challenges and risks, ultimately advancing FLE through the thoughtful integration of LLMs.

\end{abstract}
