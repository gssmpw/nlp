\section{Related Work}
%\subsection{Related Work on Inverse and Forward Modeling}
Inverse modeling is currently the predominant paradigm in ML and Data Mining (DM). 
Work on the integration of forward and inverse modeling, however, %, although called for, 
is currently found mainly outside ML and DM. A paper from applied physics  \cite{RobertsHedayati2021}  presents a deep learning approach to the forward and inverse designs of plasmonic metasurface structural color. Another applied physics paper  describes a deep learning framework for forward and inverse problems in time-domain electromagnetics \cite{Huetal2021}. 
%Another physics application of combined forward and inverse modeling comes from the domain of predicting radiative properties for dielectric and metallic particles \cite{Elzoukaetal2020}. 
Closer in spirit are robotics approaches for tracking and prediction based on deep forward and inverse perceptual models \cite{Lambertetal2018}. Deriving data from simulations of theoretical models and then training classification, regression, and ranking models is quite common in ML applied to particle physics \cite{Koeppeletal2021}. One has to note that the integration of forward and inverse modeling is similar to abduction \cite{Kakas2017}, a currently underrepresented topic in DM \cite{Wickeretal2015}.

In case of traffic surveillance on airport grounds, there is no actual standard for sensor systems for the tasks of object detection, speed estimation, tracking, and classification. Consequently, different technological systems are explored in the literature, where most of them focus on trajectory optimization during flight \cite{ref:tian2020}, mid-air classification of aircraft using radar \cite{ref:xia2022} or aircraft classification based on top-view images \cite{ref:gao2022,ref:azam2021}. %, e.g. magnetometers \cite{ref:tang2020}, \emph{Light Detection and Ranging} (LiDAR) \cite{ref:koppanyi2018}, cameras or images \cite{ref:gao2022}, radar \cite{ref:xia2022}, satellite \cite{ref:azam2021}, and others \cite{ref:tian2020}. %The most prominent solutions in vehicle traffic monitoring are either based on LiDAR or \emph{magnetometers} \cite{ref:won2020}, as these sensors typically deliver more robust measurements and observations than cameras or radars. 
%They mostly focus on trajectory optimization during flight \cite{ref:tian2020}, mid-air classification of aircraft \cite{ref:xia2022} or aircraft classification based on top-view images \cite{ref:gao2022,ref:azam2021}. 
However, in case of surface traffic surveillance on airports, aircraft need to be classified on the ground. Hence, visual-based technologies may struggle with obstacles and visually bad weather conditions like heavy rain, snow or fog \cite{ref:zhang2023}. On the other hand, magnetometers are not affected as they measure the Earth's magnetic field. They may also be installed directly below the surface of the holding point positions that need to be monitored, functioning as on-site sensors. Consequently, magnetometers have a high potential to be a part of traffic surveillance systems on airport grounds.   

%-------------------------------------------------
%------------------- Framework
%-------------------------------------------------