\section{Related Work}
\textbf{Methods for LRP with Predefined Depot Candidates:}
In addressing the LRP with Predefined Depot Candidates, traditional methods have predominantly employed exact and heuristic approaches. 
Exact methods, such as Mixed Integer Programming (MIP) models enhanced by branch-and-cut ____ or column generation techniques ____, 
offer precision but often struggle with scalability in larger and complex scenarios due to an exponential increase in binary variables. 
This limitation has pivoted attention towards heuristic methods, which are categorized into: matheuristic approaches ____ that blend heuristic rules with exact methods, learning-aided heuristics ____ that leverage learning-based algorithms to refine heuristic operations, and pure meta-heuristic algorithms. 
Among pure meta-heuristics, cluster-based heuristics ____ and iterative methods ____ have been notable.
However, the cluster-based heuristics, which focus on geographically clustering customers, exhibit limitations in handling additional constraints like customer-specific time windows, 
while the iterative methods present insufficient conjugation between the two stages of depot-selecting and route-planning.
Besides, these methods typically require initiating a new search process for each case, leading to inefficiencies when even minor alterations occur to current problem instance.


% (Pending) [Related DRL method but not exactly LRP, more of MDVRP]
The advancements in DRL have shown promise in addressing routing problems, both in ``learn-to-construct/generalize'' ____ and ``learn-to-improve/decompose'' ____. % 3-23 chen2019learning, % 9-30 guo2023deep, lu2020learning, 
However, its application in LRP, which integrates the challenges of facility locating with routing problem, 
still remains notably underexplored due to their inherent limitations in problem formulation for scenario involving multiple depots and the inability to organically integrate depot-selecting with route-planning. 
The works ____ focus on resolving routing problems involving multiple depots, but without considering the depot-relate cost,
which technically confine them as multi-depot VRP, instead of LRP which considers both route-related cost and depot-related cost.
The work ____ considers depot-related cost but adopts a two-stage process, 
clustering customers with an assigned depot location first and planning routes second,
thereby separating depot selection from route planning, 
which fails to capture the interdependencies between these two critical aspects, 
while also lacking verification on standard LRP setup align with real-world datasets.
Most importantly, all these methods are constrained to the predefined depot candidates, 
falling short in dealing with LRP without predefined depot choices.

\textbf{Exploration of LRP without Predefined Depot Candidates:}
% Most of the methods focus on LRP with a finite set of potential depot candidates, and 
Only a scant number of studies explore the LRP without predefined depot candidates, 
predominantly employing heuristic strategies for attempting new depots across a planar area devoid of predefined depot choices. 
%which are all solved by specifically designed heuristic methods.
The work ____ concentrate on single-depot scenario, where a single and uncapacitated depot is to be selected from a planar area. 
Specifically, ____ extends the cluster-based method in ____, proposing a learning-aided heuristic method to recurrently initiate new depot. 
Based on the same single-depot scenario, the work ____ proposes a hierarchical heuristic method to iteratively update candidate circle to select new depot and then plan routes based on this depot.
Furthermore, ____ extends the iterative heuristic method to explore multi-depot scenario, but only manages to deal with cases with up to two depots. 

It is notable that, compared with our method's endeavors on actively and directly generating the recommended depots, 
these works employ heuristic algorithms to iteratively attempt new depot and then decide if it is a better one by re-planning routes based on it, 
limited to single or double-depot scenarios.
Moreover, all these works lack ability on considering specific location constraints for depots, 
highlighting the necessity for a more adaptable and flexible solution.


%%%%%%%%%%%%%%%%%%%%%%%%%%%%%%%%%%%%%%%%%%%%%%%%%%%%%%%%%%%%%%%%%%%%%%%%%%%%%%%%%%%%%%%%%%%%%%%%%%%%%%%%%%%%%%%%%%%%%%%%%%%%%%%%%%%%%%%%%%%%%%%%%%%%%%%%%%%%%%%%%%%%%%%%%%%%%%%%%%%%%%%%%%%%%%
%%%%%%%%%%%%%%%%%%%%%%%%%%%%%%%%%%%%%%%%%%%%%%%%%%%%%%%%%%%%%%%%%%%%%%%%%%%%%%%%%%%%%%%%%%%%%%%%%%%%%%%%%%%%%%%%%%%%%%%%%%%%%%%%%%%%%%%%%%%%%%%%%%%%%%%%%%%%%%%%%%%%%%%%%%%%%%%%%%%%%%%%%%%%%%