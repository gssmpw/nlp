\subsection{Motivation}  
% or Motivation


Previous studies have shown the limitations of guide dogs in providing perceptual intelligence for the visually impaired, as well as the key considerations for describing specific object in the environment to assist the visually impaired for navigation \cite{Hochul2024}\cite{Hoogsteen2022}.
The findings of these studies suggest that the preferred methods of description and the frequency of requests among visually impaired individuals vary based on factors like physical balance, residual vision, age, whether they are early or late blind, and the type of mobility aid used, such as a white cane or guide dog.
For example, early blind participants, who asked fewer questions, tended to focus on identifying a building through its address, whereas late blind participants prioritized locating it directly from their own perspective. 
Individuals using a white cane preferred guidance-related descriptions over collision-related ones \cite{Hoogsteen2022}.
Some guide dog handlers want to be informed about every small obstacle, while others prefer to bypass minor ones like low-rise toe-trip hazards \cite{Hochul2024}. 
Despite the variety of opinions, \textit{a common preference was to receive \textbf{essential} walking information in a \textbf{concise} format.}

In this context, we conducted interviews to determine what descriptions are essential for the visually impaired if walking situations can be captured through the camera of a guide dog robot. 
A total of three face-to-face, in-depth interviews were conducted with two visually impaired individuals and four instructors from guide dog training schools.
Based on these findings, we can summarize the essential information in walking situations as follows:
\begin{itemize}
    \item \textit{A brief description should cover the left and right sides of the path, the path itself including any obstacles, and whether it is possible to walk along it.}
    \item \textit{If it is possible to walk to the destination, no explanation is needed.}
    \item \textit{If walking to the destination is not possible, provide a brief reason of the situation.}
\end{itemize}

\subsection{Problem Definition}
From the interview results, we formulate the problem as determining how to provide essential and concise information to the visually impaired.
To clarify, we assume that an image $I$ is captured by an RGB camera mounted on the guide dog robot, with a goal position (GP) representing the user's intended destination in the image.
The GP is a point $\mathbf{p_{goal}} = (x, y)$ in the image $I$. 
The first stage is determining the path to the GP.
At this stage, we set the path width to 2 meters, incorporating a safety buffer with the larger value within the range of 1.2 to 1.5 meters, which is known to accommodate the safe movement of a visually impaired person and a guide dog \cite{MacLennan2015}.
Based on the detected path, descriptions are given for the path, its left and right sides, and the destination, represented as $t_{path}$, $t_{left}$, $t_{right}$, and $t_{dest}$, respectively.
Since the descriptions must be concise and essential, we focus on nearby objects or obstacles, as a survey \cite{Hoogsteen2022} found they are more useful than distant ones.
After providing these descriptions, a decision $t_{reco}$ is made on whether it is possible to proceed along the path, with a brief explanation of the reasons for the decision.
Thus, by receiving an image $I$ and a query prompt $q_{goal}$ containing a goal position $\mathbf{p_{goal}}$ as input, a VLM, $f$, provides four types of spatial descriptions and makes decisions based on that information as follows:
\begin{equation}
t_{dest}, t_{left}, t_{right}, t_{path}, t_{reco} = f(I, q_{goal})
\end{equation}
