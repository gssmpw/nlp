\section{Limitations}
Below we discuss the limitations of the proposed work.
\begin{enumerate}
    \item In this work we focused on evaluating/mitigating name bias in text-embedding models using texts from English language. The work presented here does not cover other languages. Further, the work also does not cover name bias issues arising in multi language texts.
    

    \item While our proposed anonymization solution enhances thematic similarity, it is not ideal for situations requiring the preservation of identity that we are removing through anonymization.   A partial and straightforward solution might involve anonymizing only non-critical identifying information depending upon the use-case.  Many real world use cases may require dynamically balancing identity and thematic preservation to suit the specific needs of each use case.

    \item In our work, we adopted similarity between text-embeddings as a proxy for their semantic similarity. While commonly used, it is still an estimate of semantic similarity and may overlook deeper semantic relationships that require reasoning. A limitation of this work is that we capture thematic similarity only to the extent that it is captured by the cosine similarity (and the Euclidean distance similarity is studied in the Appendix).
   

\end{enumerate}