\label{sec:method}




\section{Methodology: Overcoming Bias through Anonymization}
Previously, we showed that how just changing person names/country names can impact the embeddings significantly. In this section, we introduce a simple \textit{inference-time anonymization} technique to mitigate the bias caused by names. The core idea is to mitigate the influence of names on embeddings, and making the resulting \textit{debiased} anonymized embeddings to be more generalizable and less prone to biases related to particular individuals or entities.  

The anonymization of a text $T$ during inference is achieved through the following process. We first identify in $T$, occurrences of desired entities such as person names, locations and organizations relevant to the use case. We \textit{anonymize} the text by removing those occurrences from $T$.  The anonymized text referred to as $T_{anon}$ retains the overall structure and meaning of the original text $T$ while removing any specific references to person names etc.  This anonymization can be achieved via tools such as Large Language Models(LLMs)~\citep{zhao2023survey} or Named Entity Recognition tools~\citep{jehangir2023survey}. In our work, we used \textit{gemini} and \textit{anthropic.claude-3-5-sonnet} text-generative models for anonymization using prompts. Depending upon the use-case, different names in text such as person names, cities, countries, organizations can be removed. We would like to clarify that the same process of anonymization can also be done through Named-Entity Recognition(NER) tools~\citep{jehangir2023survey}, however in our initial experiments we found LLMs to be more accurate. Sample prompts for anonymization are presented in Table ~\ref{tab:prompts}. Post anonymization, the embeddings become independent of identity specific details such as person names/ country names etc.~\footnote{ The type of anonymization i.e removing person names and/or country names and/or city names etc. used determines the exact level of independence.} Overall, the \textit{debiased} embeddings generated on anonymized text promise reduced sensitivity to biases associated with particular individuals or entities. Note that the embeddings generated for sentences that differ solely in their named entities (e.g., character names) will now have a cosine similarity of 1.
\noindent
An alternate to removing \textit{named} content for anonymization is to replace names with specific non-identifying placeholder words. This approach with its associated challenges is further examined in App. ~\ref{sec:anon_vs_replacement}.




% \textcolor{red}{Add pseudo code? expand method.}
% Specifically, 

\begin{table}[h!]
\centering
% \small
\scalebox{0.8}{
\begin{tabular}{p{2cm}p{6.5cm}}
\hline
 Purpose & Prompt \\
\hline
 Remove person names and location names & Given below text, please COMPLETELY DELETE all Person/Character names which are PROPER NOUNS and City/ Country/ Village/ Town/ Continent/ River/  Organization names which are PROPER NOUNS etc. Wherever they occur replace with empty string. Completely remove them and not anything else. Do not delete monument/landmark names like Eiffel tower etc. Do not remove He/She/him/her etc.. Output contains the modified text only....  The text is provided below :::: \\ \hline
 Remove person names only & Given below text, please COMPLETELY DELETE all Person/Character names which are PROPER NOUNS. Wherever they occur replace with empty string. Completely remove them and not anything else. Do not remove He/She/him/her etc.. Output contains the modified text only....  The text is provided below :::: \\
\hline
\end{tabular}}
\caption{Prompts for Anonymization. In our experiments, we select the first prompt. Based upon the use case, the suitable prompt can be selected or modified accordingly. }
\label{tab:prompts}
\end{table}