\begin{figure*}[p!]
    \captionsetup[subfigure]{skip=0pt}
    \centering%
    \begin{promptbox}{\texttt{CSC In-Context Prompt for Baseline}}{black}{white}
        \footnotesize{
            \textbf{\texttt{System Prompt:}}\\
            你是一个优秀的中文纠错模型,中文纠错模型即更正用户输入句子中的错误。\return

            \textbf{\texttt{User Prompt:}}\\
            你需要识别并纠正用户输入的句子中可能的错别字并输出正确的句子,在纠正错别字的同时尽可能减少对原句子的改动(不新增、删除和修改标点符号)。只输出没有错误的句子,不要添加任何其他解释或说明。如果句子没有错误,就直接输出和输入相同的句子。\return\return

            \jsonkey{<Example>}\return\\
            输入:\jsonkey{\{INPUT\_EXAMPLE\_1:\  $\substring{x}_1$\}}\return\\
            输出:\jsonkey{\{OUTPUT\_EXAMPLE\_1: $\substring{y}_1$\}}\return\return\\
            输入:\jsonkey{\{INPUT\_EXAMPLE\_2:\  $\substring{x}_2$\}}\return\\
            输出:\jsonkey{\{OUTPUT\_EXAMPLE\_2: $\substring{y}_2$\}}\return\return\\
            \jsonkey{</Example>}\return\return\\

            输入:\jsonkey{\{INPUT: $\substring{x}$\}}\return
            输出:
        }
        \tcblower
        \scriptsize
        \textbf{\texttt{System Prompt:}}\\
        You are an excellent Chinese error correction model, which means you correct errors in the sentences entered by the user.\return

        \textbf{\texttt{User Prompt:}}\\
        You need to identify and correct possible misspelled characters in the user's input sentence, and output the correct sentence. While making corrections, try to minimize changes to the original sentence (without adding, deleting, or modifying punctuation). Only output the corrected sentence; do not add any further explanations or notes. If the sentence is error-free, output the exact same sentence as the input.\return

        \jsonkey{<Example>}\return\\
        \rlap{Input:}\phantom{Output:} \jsonkey{\{INPUT\_EXAMPLE\_1:  $\substring{x}_1$\}}\return\\
        Output: \jsonkey{\{OUTPUT\_EXAMPLE\_1: $\substring{y}_1$\}}\return\return\\
        \rlap{Input:}\phantom{Output:} \jsonkey{\{INPUT\_EXAMPLE\_2:  $\substring{x}_2$\}}\return\\
        Output: \jsonkey{\{OUTPUT\_EXAMPLE\_2: $\substring{y}_2$\}}\return\return\\
        \jsonkey{</Example>}\return\return\\\\
        \rlap{Input:}\phantom{Output:} \jsonkey{\{INPUT: $\substring{x}$\}}\return\\
        Output:

    \end{promptbox}
    \begin{promptbox}{\texttt{C2EC In-Context Prompt for Baseline}}{black}{white}
        \footnotesize{
            \textbf{\texttt{System Prompt:}}\\
            你是一个优秀的中文纠错模型,中文纠错模型即更正用户输入句子中的错误。\return

            \textbf{\texttt{User Prompt:}}\\
            你需要识别并纠正用户输入的句子中可能的\correct{错别字、多字、漏字}并输出正确的句子,在修改的同时尽可能减少对原句子的改动(不新增、删除和修改标点符号)。只输出没有错误的句子,不要添加任何其他解释或说明。如果句子没有错误,就直接输出和输入相同的句子。\return\return

            \jsonkey{<Example>}\return\\
            输入:\jsonkey{\{INPUT\_EXAMPLE\_1:\  $\substring{x}_1$\}}\return\\
            输出:\jsonkey{\{OUTPUT\_EXAMPLE\_1: $\substring{y}_1$\}}\return\return\\
            输入:\jsonkey{\{INPUT\_EXAMPLE\_2:\  $\substring{x}_2$\}}\return\\
            输出:\jsonkey{\{OUTPUT\_EXAMPLE\_2: $\substring{y}_2$\}}\return\return\\
            \jsonkey{</Example>}\return\return\\

            输入:\jsonkey{\{INPUT: $\substring{x}$\}}\return
            输出:
        }
        \tcblower
        \scriptsize
        \textbf{\texttt{System Prompt:}}\\
        You are an excellent Chinese error correction model, which means you correct errors in the sentences entered by the user.\return

        \textbf{\texttt{User Prompt:}}\\
        You need to identify and correct possible misspellings, redundant and missing characters in the user's input sentence, and output the correct sentence. While making corrections, try to minimize changes to the original sentence (without adding, deleting, or modifying punctuation). Only output the corrected sentence; do not add any further explanations or notes. If the sentence is error-free, output the exact same sentence as the input.\return

        \jsonkey{<Example>}\return\\
        \rlap{Input:}\phantom{Output:} \jsonkey{\{INPUT\_EXAMPLE\_1:  $\substring{x}_1$\}}\return\\
        Output: \jsonkey{\{OUTPUT\_EXAMPLE\_1: $\substring{y}_1$\}}\return\return\\
        \rlap{Input:}\phantom{Output:} \jsonkey{\{INPUT\_EXAMPLE\_2:  $\substring{x}_2$\}}\return\\
        Output: \jsonkey{\{OUTPUT\_EXAMPLE\_2: $\substring{y}_2$\}}\return\return\\
        \jsonkey{</Example>}\return\return\\\\
        \rlap{Input:}\phantom{Output:} \jsonkey{\{INPUT: $\substring{x}$\}}\return\\
        Output:

    \end{promptbox}
    \caption{
        In-context learning prompts for baseline models.
    }
    \label{fig:icl_detail_prompt}
\end{figure*}
