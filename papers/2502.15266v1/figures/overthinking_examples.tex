\begin{figure*}[ph!]
    \newtcolorbox{scopebox}[3]{%
        left=0pt,
        right=0pt,
        top=0pt,
        bottom=0pt,
        boxsep=3pt,
        middle=0pt,
        colback={#3},
        colframe={#2},
        title={#1},
    }
    \centering
    \footnotesize
    \begin{scopebox}{Example 1: \textit{Correct Thinking (with a little flaw)}}{black}{white}
        \begin{scopebox}{\texttt{Input}}{black!45}{white}
            松原市开展清\wrong{埋}整顿和\wrong{视}范房地产市场大检查
        \end{scopebox}
        \begin{scopebox}{\texttt{Reference}}{json_blue!60}{white}
            松原市开展清\correct{理}整顿和\correct{规}范房地产市场大检查
            \tcblower
            \textit{The city of Songyuan is conducting a major inspection to clean up, regulate, and standardize the real estate market.}
        \end{scopebox}
        \begin{scopebox}{\texttt{DeepSeek\,V3 Output}}{json_blue!80}{white}
            松原市开展清\correct{理}整顿和\correct{规}范房地产市场大检查
        \end{scopebox}
        \begin{scopebox}{\texttt{DeepSeek\,R1 Think}}{thinking_yellow}{white}
            \qquad{}好的,我现在需要处理用户输入的句子,找出可能的错别字并进行纠正,同时尽量减少对原句的改动,不新增、删除或修改标点符号。首先,我要仔细阅读用户提供的例子,了解他们期望的处理方式。\return

            \qquad{}用户给的例子中,比如第一个输入中的“班马王子”被纠正为“斑马王子”,这里“班”和“斑”是同音字,但“斑马”是正确的写法。另一个例子是“自然延生”改为“自然诞生”,这里“延”和“诞”字形相近,但“诞生”是正确的。还有“登录”改为“登陆”,虽然发音相同,但“登陆”在这里是正确的动词。这些例子显示,需要关注同音字、形近字以及特定语境下的正确用词。\return

            \qquad{}现在来看当前的输入句子:“松原市开展清埋整顿和视范房地产市场大检查”。我需要逐词分析可能的错别字。\return

            \qquad{}首先,“清埋”这个词是否正确?正确的应该是“清理”吗?\wrong{因为“埋”和“理”发音相近},但“清理整顿”是常见的搭配,比如清理市场、整顿秩序。所以这里“清埋”可能是“清理”的错别字。\return

            \qquad{}接下来是“视范”,正确的应该是“规范”吗?“视”和“规”在字形上有点相似,但发音不同。“规范房地产市场”是合理的表达,而“视范”这个词组不太常见,可能是“规范”的笔误或形近错误。\return

            \qquad{}因此,纠正后的句子应该是:“松原市开展清理整顿和规范房地产市场大检查”。这样改动后的句子只替换了两个字,符合用户要求的最小改动原则,并且没有影响标点符号。\return

            \qquad{}再检查一遍是否有其他可能的错误。比如“整顿”是否正确?是的,没有问题。“房地产市场”也是正确的。标点符号也没有需要改动的地方。所以最终的输出应该是纠正后的这个句子。\return

            \tcblower
            \qquad{}\textit{Okay, I now need to process the user's input sentence, identify possible typos, and correct them while minimizing changes to the original sentence, without adding, deleting, or modifying punctuation. First, I need to carefully read the examples provided by the user to understand their expected handling method.}\return

            \qquad{}\textit{In the examples given by the user, for instance, the first input "班马王子" was corrected to "斑马王子", where "班" and "斑" are homophones, but "斑马" is the correct spelling. Another example is "自然延生" changed to "自然诞生", where "延" and "诞" have similar shapes, but "诞生" is correct. Also, "登录" changed to "登陆", although they sound the same, "登陆" is the correct verb here. These examples show the need to pay attention to homophones, similar-looking characters, and the correct word usage in specific contexts.}\return

            \qquad{}\textit{Now, let's look at the current input sentence: "松原市开展清埋整顿和视范房地产市场大检查". I need to analyze each word for possible typos.}\return

            \qquad{}\textit{First, is the word "清埋" correct? Should it be "清理"? \wrong{Because "埋" and "理" sound similar}, but "清理整顿" is a common phrase, like cleaning up the market and rectifying order. So "清埋" might be a typo for "清理".}\return

            \qquad{}\textit{Next is "视范", should it be "规范"? "视" and "规" look somewhat similar, but sound different. "规范房地产市场" is a reasonable expression, while "视范" is not common and might be a typo or similar-looking error for "规范".}\return

            \qquad{}\textit{Therefore, the corrected sentence should be: "松原市开展清理整顿和规范房地产市场大检查". This revised sentence only replaces two characters, adhering to the user's principle of minimal changes, and does not affect punctuation.}\return

            \qquad{}\textit{Check again for any other possible errors. For example, is "整顿" correct? Yes, no problem. "房地产市场" is also correct. There is no need to change punctuation. So the final output should be this corrected sentence.}\return

        \end{scopebox}
        \begin{scopebox}{\texttt{DeepSeek\,R1 Output}}{json_blue!80}{white}
            松原市开展清\correct{理}整顿和\correct{规}范房地产市场大检查
        \end{scopebox}
    \end{scopebox}
    \caption{
        An example of {Correct Thinking}
    }
    \label{fig:thinking_example_1}
\end{figure*}

\begin{figure*}[ph!]
    \newtcolorbox{scopebox}[3]{%
        left=0pt,
        right=0pt,
        top=0pt,
        bottom=0pt,
        boxsep=3pt,
        middle=0pt,
        colback={#3},
        colframe={#2},
        title={#1},
    }
    \centering
    \footnotesize
    \begin{scopebox}{Example 2: \textit{Correct Thinking}}{black}{white}
        \begin{scopebox}{\texttt{Input}}{black!45}{white}
            目前\wrong{上}暂无任何一种治疗方式可保证高热惊厥不复发。
        \end{scopebox}
        \begin{scopebox}{\texttt{Reference}}{json_blue!60}{white}
            目前暂无任何一种治疗方式可保证高热惊厥不复发。
            \tcblower
            \textit{Currently, there is no treatment available that can guarantee the prevention of recurrent febrile seizures.}
        \end{scopebox}
        \begin{scopebox}{\texttt{DeepSeek\,V3 Output}}{figure_red!80}{white}
            目前\wrong{尚}暂无任何一种治疗方式可保证高热惊厥不复发。
        \end{scopebox}
        \begin{scopebox}{\texttt{DeepSeek\,R1 Think}}{thinking_yellow}{white}
            \qquad{}好的,我现在需要处理用户的输入,找出可能的错别字、多字或漏字,并尽可能少地修改原句。首先,用户提供的例子显示,他们希望保持标点符号不变,只调整文字部分。\return

            \qquad{}现在来看当前的输入句子:“目前上暂无任何一种治疗方式可保证高热惊厥不复发。” 我需要仔细分析每个部分是否有错误。\return

            \qquad{}首先,“目前上暂无”这里可能有问题。“目前上”听起来不太自然,正确的表达应该是“目前”或者“目前临床上”。但原句中的“上”是否多余呢?比如,正确的说法可能是“目前暂无”,而“上”可能是多出来的字。例如,用户可能想表达“目前暂无”,但误加了“上”字。因此,“目前上”应该改为“目前”,去掉“上”字。\return

            \qquad{}接下来检查其他部分:“暂无任何一种治疗方式可保证高热惊厥不复发。” 这里的“暂无”是否正确?“暂无”表示暂时没有,符合语境。其他词语如“治疗方式”、“保证”、“复发”都没有明显错误。标点符号也没有问题,句尾是句号,符合要求。\return

            \qquad{}所以,修改后的句子应该是:“目前暂无任何一种治疗方式可保证高热惊厥不复发。” 去除了多余的“上”字,使句子更通顺,同时没有改动其他部分,符合用户的要求。\return

            \tcblower
            \qquad{}\textit{Alright, I now need to process the user's input, identify possible typos, extra words, or missing words, and make minimal changes to the original sentence. First, the examples provided by the user show that they want to keep the punctuation unchanged and only adjust the text part.}\return

            \qquad{}\textit{Now, let's look at the current input sentence: "目前上暂无任何一种治疗方式可保证高热惊厥不复发." I need to carefully analyze each part for errors.}\return

            \qquad{}\textit{First, "目前上暂无" might have an issue. "目前上" doesn't sound natural; the correct expression should be "目前" or "目前临床上". But is the "上" in the original sentence redundant? For example, the correct phrase might be "目前暂无", and "上" might be an extra word. For example, the user might have intended to say "目前暂无" but mistakenly added "上". Therefore, "目前上" should be changed to "目前", removing the "上".}\return

            \qquad{}\textit{Next, check the other parts: "暂无任何一种治疗方式可保证高热惊厥不复发." Is "暂无" correct here? "暂无" means temporarily unavailable, which fits the context. Other words like "治疗方式", "保证", "复发" have no obvious errors. The punctuation is also correct, with a period at the end, meeting the requirements.}\return

            \qquad{}\textit{So, the revised sentence should be: "目前暂无任何一种治疗方式可保证高热惊厥不复发." The extra "上" is removed, making the sentence smoother, while no other parts are changed, meeting the user's requirements.}\return
        \end{scopebox}
        \begin{scopebox}{\texttt{DeepSeek\,R1 Output}}{json_blue!80}{white}
            目前暂无任何一种治疗方式可保证高热惊厥不复发。
        \end{scopebox}
    \end{scopebox}
    \caption{
        Another example of {Correct Thinking}
    }
    \label{fig:thinking_example_2}
\end{figure*}

\begin{figure*}[ph!]
    \newtcolorbox{scopebox}[3]{%
        left=0pt,
        right=0pt,
        top=0pt,
        bottom=0pt,
        boxsep=3pt,
        middle=0pt,
        colback={#3},
        colframe={#2},
        title={#1},
    }
    \centering
    \footnotesize
    \begin{scopebox}{Example 2: \textit{Overthinking}}{black}{white}
        \begin{scopebox}{\texttt{Input}}{black!45}{white}
            新抚区已清理违法刀匾近70处将扩大整治范围发现一处拆除一处
        \end{scopebox}
        \begin{scopebox}{\texttt{Reference}}{json_blue!60}{white}
            新抚区已清理违法刀匾近70处将扩大整治范围发现一处拆除一处
            \tcblower
            \textit{Xinfu District has removed nearly 70 illegal protruding vertical signboards and will expand the rectification area - removing them as they are discovered.}
        \end{scopebox}
        \begin{scopebox}{\texttt{DeepSeek\,V3 Output}}{json_blue!80}{white}
            新抚区已清理违法刀匾近70处将扩大整治范围发现一处拆除一处
        \end{scopebox}
        \begin{scopebox}{\texttt{DeepSeek\,R1 Think}}{thinking_yellow}{white}
            \qquad{}好的,我现在需要处理用户的输入,找出可能的错别字并进行纠正,同时尽量少改动原句,不增删或修改标点。首先,我需要仔细阅读用户提供的例子,理解他们的处理方式和规则。\return

            \qquad{}用户的输入是:“新抚区已清理违法刀匾近70处将扩大整治范围发现一处拆除一处”。我要逐词分析,看看有没有可能的错别字。首先,“刀匾”这个词听起来不太对,可能是一个错误。通常我们会说“刀牌”或者“招牌”,但“刀匾”似乎不太常见。可能用户是想表达“刀牌”,而“匾”可能是一个错别字,正确的应该是“刀牌”或者“招牌”。比如,违法设置的广告牌、招牌等,所以这里“刀匾”可能是“刀牌”的误写,或者“招牌”的错误。但“刀牌”也不太常见,可能是“招牌”中的“招”被误写为“刀”。因为“招”和“刀”在字形上有点相似,尤其是在手写或快速输入时容易出错。所以可能的正确词是“招牌”,即“违法招牌”。\return

            \qquad{}接下来,验证这个推测是否符合语境。原文在说清理违法设置的物品,通常这类整治行动针对的是违规广告牌、招牌等,所以“招牌”更符合语境。因此,“刀匾”应纠正为“招牌”。这样修改后,句子变为:“新抚区已清理违法招牌近70处将扩大整治范围发现一处拆除一处”。检查其他部分是否有错误,比如“近70处将扩大”这里的“将”是否正确,是的,没有问题。其他部分也没有明显的错别字,所以最终的纠正应该是将“刀匾”改为“招牌”。\return

            \qquad{}另外,需要注意标点符号是否被改动,原句中没有标点,所以输出也应该保持无标点,但根据例子中的情况,用户可能允许保持原标点结构。不过在这个输入中,原句没有使用标点,所以输出也不需要添加。但可能用户希望保持原样,所以不需要添加标点。因此,最终输出应为:“新抚区已清理违法招牌近70处将扩大整治范围发现一处拆除一处”。\return
            \tcblower
            \qquad{}\textit{Alright, I now need to process the user's input, identify possible typos, and make corrections while minimizing changes to the original sentence, without adding, deleting, or modifying punctuation. First, I need to carefully read the examples provided by the user to understand their processing methods and rules.}\return

            \qquad{}\textit{The user's input is: "新抚区已清理违法刀匾近70处将扩大整治范围发现一处拆除一处". I need to analyze word by word to see if there are any possible typos. First, the word "刀匾" doesn't sound quite right and might be an error. We usually say "刀牌" or "招牌", but "刀匾" seems uncommon. The user might mean "刀牌", and "匾" might be a typo, with the correct word being "刀牌" or "招牌". For example, illegal advertising signs, signboards, etc., so here "刀匾" might be a miswriting of "刀牌" or an error for "招牌". However, "刀牌" is also uncommon, and it might be that "招" in "招牌" was mistakenly written as "刀". Since "招" and "刀" are somewhat similar in shape, especially in handwriting or fast input, they are easy to confuse. Therefore, the correct word might be "招牌", meaning "illegal signboards".}\return

            \qquad{}\textit{Next, verify if this hypothesis fits the context. The original text talks about clearing illegally set items, and such rectification actions usually target illegal advertising signs, signboards, etc., so "招牌" fits the context better. Therefore, "刀匾" should be corrected to "招牌". After this modification, the sentence becomes: "新抚区已清理违法招牌近70处将扩大整治范围发现一处拆除一处". Check if there are errors in other parts, such as whether "将" in "近70处将扩大" is correct. Yes, there is no problem. There are no obvious typos in other parts, so the final correction should be changing "刀匾" to "招牌".}\return

            \qquad{}\textit{Additionally, pay attention to whether punctuation has been altered. The original sentence has no punctuation, so the output should also remain without punctuation. However, based on the examples, the user might allow maintaining the original punctuation structure. But in this input, the original sentence does not use punctuation, so the output does not need to add any. But, the user might want to keep it as is, so no punctuation needs to be added. Therefore, the final output should be: "新抚区已清理违法招牌近70处将扩大整治范围发现一处拆除一处".}\return
        \end{scopebox}
        \begin{scopebox}{\texttt{DeepSeek\,R1 Output}}{figure_red!80}{white}
            新抚区已清理违法\wrong{招牌}近70处将扩大整治范围发现一处拆除一处
        \end{scopebox}
    \end{scopebox}
    \caption{
        An example of {Incorrect Thinking}
    }
    \label{fig:thinking_example_3}
\end{figure*}

\begin{figure*}[ph!]
    \newtcolorbox{scopebox}[3]{%
        left=0pt,
        right=0pt,
        top=0pt,
        bottom=0pt,
        boxsep=3pt,
        middle=0pt,
        colback={#3},
        colframe={#2},
        title={#1},
    }
    \centering
    \footnotesize
    \begin{scopebox}{Example 3: \textit{Oversight}}{black}{white}
        \begin{scopebox}{\texttt{Input}}{black!45}{white}
            行窃\wrong{是}触电身亡,小偷家属索赔,赔还是不赔
        \end{scopebox}
        \begin{scopebox}{\texttt{Reference}}{json_blue!60}{white}
            行窃\correct{时}触电身亡,小偷家属索赔,赔还是不赔
            \tcblower
            Died of electrocution while stealing, the thief's family claims compensation—compensate or not.
        \end{scopebox}
        \begin{scopebox}{\texttt{DeepSeek\,V3 Output}}{json_blue!80}{white}
            行窃\correct{时}触电身亡,小偷家属索赔,赔还是不赔
        \end{scopebox}
        \begin{scopebox}{\texttt{DeepSeek\,R1 Think}}{thinking_yellow}{white}
            \qquad{}好,我现在需要处理用户输入的句子,找出可能的错别字并进行纠正,同时尽量少改动原句。首先,我需要仔细阅读输入的句子:\return

            \qquad{}输入: 行窃是触电身亡,小偷家属索赔,赔还是不赔\return

            \qquad{}首先,逐词检查有没有明显的错别字。第一个部分“行窃是触电身亡”,看起来没问题,“行窃”指的是偷窃行为,“触电身亡”正确。接着是“小偷家属索赔”,这里“家属”正确,没有错别字。最后是“赔还是不赔”,这里“赔”字重复使用,但结构正确,没有问题。\return

            \qquad{}再检查标点符号是否正确。用户使用了逗号分隔各部分,没有使用错误,比如句号或分号,所以标点符号不需要改动。\return

            \qquad{}接下来,考虑是否有同音字或形近字错误。例如,“触电”是否应为“触电”?是的,正确。“身亡”正确。其他词汇如“行窃”、“家属”、“索赔”都正确。\return

            \qquad{}有没有可能被忽略的错误?比如“行窃”是否应为“行窃”?不,是正确的。“触电”是否有其他写法?比如“电击”,但原句用“触电”更准确,指接触电流导致死亡,正确。\return

            \qquad{}所以整个句子看起来没有错别字,结构和用词都正确。因此,输出应该与输入相同,不需要修改。\return

            \tcblower
            \qquad{}\textit{Ok, I now need to process the user's input sentence, identify possible typos, and make corrections while minimizing changes to the original sentence. First, I need to carefully read the input sentence:}\return

            \qquad{}\textit{Input: 行窃是触电身亡,小偷家属索赔,赔还是不赔}\return

            \qquad{}\textit{First, check word by word for any obvious typos. The first part "行窃是触电身亡" seems fine, "行窃" refers to the act of stealing, "触电身亡" is correct. Next is "小偷家属索赔", here "家属" is correct, no typos. Finally, "赔还是不赔", the word "赔" is repeated, but the structure is correct, no issues.}\return

            \qquad{}\textit{Then check if the punctuation is correct. The user used commas to separate each part, no errors like periods or semicolons, so the punctuation does not need to be changed.}\return

            \qquad{}\textit{Next, consider if there are homophones or similar-looking character errors. For example, should "触电" be "触电"? Yes, it's correct. "身亡" is correct. Other words like "行窃", "家属", "索赔" are all correct.}\return

            \qquad{}\textit{Are there any errors that might have been overlooked? For example, should "行窃" be "行窃"? No, it's correct. Is there another way to write "触电"? Like "电击", but the original sentence uses "触电" more accurately, referring to death by electric shock, correct.}\return

            \qquad{}\textit{So the whole sentence seems to have no typos, the structure and wording are correct. Therefore, the output should be the same as the input, no modifications needed.}\return
        \end{scopebox}
        \begin{scopebox}{\texttt{DeepSeek\,R1 Output}}{figure_red!80}{white}
            行窃\wrong{是}触电身亡,小偷家属索赔,赔还是不赔
        \end{scopebox}
    \end{scopebox}
    \caption{
        Another example of {Incorrect Thinking}
    }
    \label{fig:thinking_example_4}
\end{figure*}
