\section{Experimental Setup}
\subsection{Datasets}
\label{sec:datasets}
We evaluate our approach on both conventional CSC datasets and our newly constructed dataset, \textbf{C2EC}.
For conventional CSC, following \citet{li-etal-2024-cllm}, we use two representative datasets: \textbf{CSCD-NS} \cite{hu-etal-2024-cscd} and \textbf{Lemon} \cite{wu-etal-2023-rethinking}.
Both datasets contain text written by native speakers.
CSCD-NS focuses on general domain performance, while Lemon evaluates zero-shot cross-domain capabilities.
The statistics of the datasets are shown in Table~\ref{tab:dataset_statistics}.
As we are unable to access the ECMR-2023 dataset \cite{he-etal-2023-umrspell}, we do not include it in our experiments.

\subsection{Evaluation Metrics}
Following previous works \cite{li-etal-2024-cllm,zhou-etal-2024-simple}, we use character-level correction $F_1$ as our main evaluation metric.
Since sentence-level metrics are widely used in previous works, we also report them in Appendix~\ref{subsec:sentence_level_results}.
Details of the metrics can be found in Appendix~\ref{subsec:evaluation_implementation_and_settings}.

\begin{table*}[tb!]
    \centering
    \renewcommand{\arraystretch}{0.95}
    \setlength{\tabcolsep}{4.0pt}
    \scalebox{0.9}{
        \begin{NiceTabular}{lccccccccc;c;c|>{\columncolor{figure_light_red!6}}c}
            \toprule
            \rowcolor[gray]{.9}
                                                                                            &                               &                                 & \Block[c]{1-9}{\textbf{\textsc{Conventional CSC}}} &                &                &                &                &                &                &                                  &                               & \textbf{\textsc{C2EC}} \\
            \Block[l]{2-1}{\textbf{Model}}                                                  & \Block[c]{2-1}{\textbf{Size}} & \Block[c]{2-1}{\textbf{Method}} & \Block[c]{1-7}{\textbf{Lemon}}                     &                &                &                &                &                &                & \Block[c]{1-1}{\textbf{CSCD-NS}} & \Block[c]{2-1}{\textbf{Avg.}} & \textbf{C2EC}          \\
                                                                                            &                               &                                 & \textit{Car}                                       & \textit{Cot}   & \textit{Enc}   & \textit{Gam}   & \textit{Mec}   & \textit{New}   & \textit{Nov}   & \textit{test}                    &                               & \textit{test}          \\
            \midrule
            \rowcolor[gray]{1.0}
            \Block[c]{1-13}{\textit{Supervised Fine-tuning SoTAs} \cite{li-etal-2024-cllm}} &                               &                                 &                                                                                                                                                                                                                                                      \\
            \midrule
            \texttt{SCOPE}                                                                  & \texttt{0.1B}                 & \texttt{Full}                   & 50.71                                              & 54.89          & 45.23          & 24.74          & 44.44          & 48.72          & 33.17          & 71.70                            & 46.70                         & --                     \\
            \Block[l]{2-1}{\texttt{Qwen1.5}}                                                & \texttt{7B}                   & \texttt{LoRA}                   & 53.87                                              & 58.04          & 54.57          & 37.43          & 61.16          & 60.07          & 41.42          & 71.64                            & 54.77                         & --                     \\
                                                                                            & \texttt{14B}                  & \texttt{LoRA}                   & 57.54                                              & 60.40          & 56.48          & 38.02          & 65.31          & 64.49          & 43.92          & \textbf{73.80}                   & 57.49                         & --                     \\
            \midrule
            \rowcolor[gray]{1.0}
            \Block[c]{1-13}{\textit{Training-free Methods}}                                 &                               &                                                                                                                                                                                                                                                                                        \\
            \midrule
            \texttt{GPT4o-mini}                                                             & \texttt{N/A}                  & \texttt{ICL}                    & 32.13                                              & 29.57          & 41.43          & 12.46          & 34.12          & 33.53          & 26.67          & 40.32                            & 31.28                         & 30.73                  \\
            \texttt{GPT4o}                                                                  & \texttt{N/A}                  & \texttt{ICL}                    & 54.68                                              & 57.25          & 63.02          & 14.60          & 62.92          & 61.37          & 54.10          & 65.43                            & 54.17                         & 45.71                  \\
            \texttt{DeepSeek\,V3}                                                           & \texttt{671B}                 & \texttt{ICL}                    & 60.12                                              & 69.91          & \textbf{68.56} & 38.10          & \textbf{70.34} & 69.67          & \textbf{61.41} & 69.43                            & 63.44                         & 54.62                  \\
            \texttt{DeepSeek\,R1}                                                           & \texttt{671B}                 & \texttt{ICL}                    & 57.80                                              & 63.47          & 62.38          & 45.37          & 69.38          & 68.39          & 58.21          & 62.68                            & 61.61                         & 48.85                  \\
            \hdashedline
            \Block[l]{4-1}{\texttt{Qwen2.5}}                                                & \Block[c]{4-1}{\texttt{7B}}   & \texttt{ICL}                    & 28.98                                              & 39.78          & 36.89          & \wz9.89        & 38.22          & 27.98          & 22.09          & 32.61                            & 29.56                         & 29.03                  \\
                                                                                            &                               & \texttt{ICL-RR}                 & 41.39                                              & 55.61          & 49.23          & 24.13          & 51.19          & 44.48          & 34.14          & 54.20                            & 44.30                         & 41.01                  \\
                                                                                            &                               & \texttt{TfPf}                   & 55.25                                              & 64.31          & 53.89          & 41.08          & 57.47          & 63.39          & 45.53          & 62.45                            & 55.42                         & 41.50                  \\
                                                                                            &                               & \texttt{OUR}                    & 61.80                                              & \textbf{71.05} & 65.86          & 51.67          & 68.65          & 69.34          & 51.89          & 71.03                            & 63.91                         & 56.02                  \\
            \hdashedline
            \Block[l]{4-1}{\texttt{Qwen2.5}}                                                & \Block[c]{4-1}{\texttt{14B}}  & \texttt{ICL}                    & 35.93                                              & 49.65          & 42.04          & 26.45          & 45.37          & 38.02          & 31.45          & 38.39                            & 38.41                         & 32.52                  \\
                                                                                            &                               & \texttt{ICL-RR}                 & 48.80                                              & 55.39          & 52.63          & 40.02          & 53.67          & 55.46          & 44.43          & 55.68                            & 50.76                         & 42.86                  \\
                                                                                            &                               & \texttt{TfPf}                   & 55.51                                              & 62.50          & 54.43          & 37.90          & 56.58          & 64.25          & 46.74          & 62.53                            & 55.06                         & 40.96                  \\
                                                                                            &                               & \texttt{OUR}                    & \textbf{64.62}                                     & 70.81          & 68.50          & \textbf{51.92} & 68.24          & \textbf{71.85} & 53.68          & 72.75                            & \textbf{65.30}                & \textbf{57.41}         \\
            \bottomrule
        \end{NiceTabular}
    }
    \caption{
        Comparison between our method and the baseline methods on conventional CSC datasets.
    }
    \label{tab:main_results}
\end{table*}

\subsection{Baselines}
We compare our approach against three training-free baselines:
\begin{inparaenum}[\itshape a)]
    \item \textbf{In-context Learning} (\texttt{ICL}): This method prompts LLMs with 10 exemplars (5 erroneous, 5 correct sentences) randomly selected and shuffled for each input. During inference, we use beam search with the same beam size as our approach;\footnote{For conventional CSC datasets, exemplars are randomly sampled from the CSCD-NS training set, while for C2EC, they come from its development set. Due to the random nature of \texttt{ICL}, we report the results averaged across 3 runs.}
    \item \textbf{Training-free Prompt-free Method} (\texttt{TfPf}, \citet{zhou-etal-2024-simple}); and
    \item \textbf{ICL with Reranking} (\texttt{ICL-RR}): This hybrid method first generates $K$ candidates using \texttt{ICL}, then reranks them using an extended \texttt{TfPf} model that supports insertion and deletion operations.
\end{inparaenum}

For reference, we also include \texttt{ICL} results from leading LLMs through API calls: \texttt{GPT4o-mini}\rlap{,}\footnote{Version: \texttt{gpt-4o-mini-2024-07-18}} \texttt{GPT4o} \cite{hurst-etal-2024-gpt4o}\rlap{,}\footnote{Version: \texttt{gpt-4o-2024-11-20}} and the 671B parameter \texttt{DeepSeek\,V3} and \texttt{R1} \cite{deepseekai-2024-deepseek-v3,deepseek-r1-2025}\rlap{.}\footnote{Due to API constraints and high cost, we use greedy decoding (\texttt{temperature=0.0}) instead of beam search, and report single-run results.}

Additionally, the supervised fine-tuning (\texttt{SFT}) results on conventional CSC datasets from \citet{li-etal-2024-cllm} are also included for better understanding.

\paragraph{Training Details of SFT Baselines}
The \texttt{SFT} models were trained on a combined dataset consisting of 271k pseudo sentence pairs \cite{wang-etal-2018-hybrid} and the \textbf{CSCD-NS} training data.
This means that while CSCD-NS is an in-domain evaluation, the Lemon dataset serves as a cross-domain dataset for evaluating the generalization capabilities.

\subsection{Model Selection}
We use the \texttt{Qwen2.5} model series \cite{yang-etal-2024-qwen25} for our experiments, as it is one of the most recent open-source LLMs with strong Chinese language capabilities.

For \texttt{ICL} experiments, we use the ``\texttt{Instruct}'' version, which is optimized for instruction following.
As for our approach, we use the ``\texttt{Base}'' version.
To reduce the computational cost, we use \texttt{Qwen2.5\,7B} for detailed analysis.

\subsection{Hyperparameters}
We largely follow the hyperparameter settings in \citet{zhou-etal-2024-simple}, using a beam size of 8 and $\alpha$ for length reward of 2.5.
Through development set tuning, we set the weights for insertion and deletion as 8.5 and 9.0, respectively, and the temperature of the prompt-based scoring as 1.5.
