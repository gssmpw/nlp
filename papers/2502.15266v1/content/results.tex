\section{Main Results}
As shown in Table \ref{tab:main_results}, our method performs better than \texttt{ICL}, \texttt{ICL-RR}, and \texttt{TfPf} on both conventional CSC datasets and the C2EC dataset.
Compared to \texttt{TfPf}, from which our method is extended, we achieve improvements of 8.49 and 10.24 on average with 7B and 14B models, respectively.
Compared to \texttt{ICL-RR}, our method is shown to be a better way to combine the advantages of prompt-based LLMs and \texttt{TfPf}.
This is because \texttt{ICL-RR} can only choose from the top $K$ candidates from \texttt{ICL}.
If none of these candidates are good, reranking cannot improve the final result.

Compared to \texttt{SFT} models from \citet{li-etal-2024-cllm}, which are trained on the training set of the CSCD-NS dataset, our method shows better performance on the \textbf{out-of-domain} Lemon dataset\rlap{.}\footnote{Since the \texttt{SFT} models are trained with the \texttt{Qwen1.5} series, which may not be a fair comparison with our method, we also provide our results under the \texttt{Qwen1.5} series in Appendix~\ref{app:supervised_fine_tuning}.}
This indicates that \texttt{SFT} methods may not generalize well to new data they have not seen during training.

Compared to recent leading LLMs (e.g., the 671B parameter \texttt{DeepSeek\,V3}), our method enables much smaller 7B and 14B models to be on par with them without any training.

For a better understanding of the performance of our method, we also provide several qualitative results in Appendix~\ref{subsec:qualitative_analysis}.
\begin{table*}[tb!]
    \centering
    \scalebox{0.9}{
        \begin{NiceTabular}{ccccc;ccc;ccc}
            \toprule
            \rowcolor[gray]{1.0}
                       & \Block[c]{2-1}{\texttt{Prompt}                                                                                                                                                                                                                                                         \\[-2pt]\texttt{Template}}         & \Block[c]{2-1}{\texttt{Prompt}\\[-2pt]\texttt{LLM}}          & \Block[c]{2-1}{\texttt{Pure}\\[-2pt]\texttt{LLM}}         & \Block[c]{2-1}{\texttt{Distance}\\[-2pt]\texttt{Metric}}      & \Block[c]{1-3}{\textbf{CSCD-NS} \textit{dev}} &                &                & \Block[c]{1-3}{\textbf{C2EC} \textit{dev}} &                &                \\
                       &                                            &                                            &                                            &                                           & \textbf{P}     & \textbf{R}     & \textbf{F$_1$} & \textbf{P}     & \textbf{R}     & \textbf{F$_1$} \\
            \midrule
            \rowcolor[gray]{0.98}
            \texttt{1} & \texttt{Minimal}                           & \texttt{Base}                              & \texttt{Base}                              & \texttt{Levenshtein}                      & 67.70          & 73.69          & 70.57          & \textbf{65.14} & 49.18          & 56.05          \\
            \midrule
            \texttt{2} & \texttt{Minimal}                           & \texttt{Base}                              & \texttt{Base}                              & \textcolor{figure_blue}{\texttt{Hamming}} & \textbf{72.49} & \textbf{74.90} & \textbf{73.68} & 65.06          & 41.71          & 50.83          \\
            \texttt{3} & \textcolor{figure_blue}{--}                & \textcolor{figure_blue}{--}                & \texttt{Base}                              & \texttt{Levenshtein}                      & 60.04          & 65.23          & 62.53          & 54.80          & 37.43          & 44.48          \\
            \texttt{4} & \texttt{Minimal}                           & \texttt{Base}                              & \textcolor{figure_blue}{--}                & \texttt{Levenshtein}                      & 61.12          & 57.91          & 59.47          & 55.96          & 46.18          & 50.60          \\
            \hdashedline
            \texttt{5} & \textcolor{figure_blue}{\texttt{Detailed}} & \texttt{Base}                              & \texttt{Base}                              & \texttt{Levenshtein}                      & 67.60          & 71.65          & 69.57          & 65.12          & 45.90          & 53.85          \\
            \texttt{6} & \texttt{Minimal}                           & \textcolor{figure_blue}{\texttt{Instruct}} & \textcolor{figure_blue}{\texttt{Instruct}} & \texttt{Levenshtein}                      & 58.35          & 69.50          & 63.44          & 55.04          & 50.27          & 52.55          \\
            \texttt{7} & \textcolor{figure_blue}{\texttt{Detailed}} & \textcolor{figure_blue}{\texttt{Instruct}} & \textcolor{figure_blue}{\texttt{Instruct}} & \texttt{Levenshtein}                      & 67.33          & 68.44          & 67.88          & 62.21          & \textbf{51.73} & \textbf{56.49} \\
            \bottomrule
        \end{NiceTabular}
    }
    \caption{
        Ablation study on different parts of our approach.
    }
    \label{tab:ablation:parts}
\end{table*}


An interesting phenomenon worth noting is that the reasoning model \texttt{DeepSeek\,R1} shows lower performance than \texttt{DeepSeek\,V3}, its non-reasoning variant, on both CSC and C2EC tasks.
We find this may be caused by incorrect reasoning.
More discussion of this is provided in Appendix~\ref{subsec:incorrect_thinking_may_lead_to_wrong_corrections}.
