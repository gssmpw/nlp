% This must be in the first 5 lines to tell arXiv to use pdfLaTeX, which is strongly recommended.
\pdfoutput=1
% In particular, the hyperref package requires pdfLaTeX in order to break URLs across lines.

\documentclass[11pt]{article}

% Change "review" to "final" to generate the final (sometimes called camera-ready) version.
% Change to "preprint" to generate a non-anonymous version with page numbers.
\usepackage[preprint]{acl}

% Standard package includes
\usepackage{times}
\usepackage{latexsym}

% For proper rendering and hyphenation of words containing Latin characters (including in bib files)
\usepackage[T1]{fontenc}
% For Vietnamese characters
% \usepackage[T5]{fontenc}
% See https://www.latex-project.org/help/documentation/encguide.pdf for other character sets

% This assumes your files are encoded as UTF8
\usepackage[utf8]{inputenc}

% This is not strictly necessary, and may be commented out,
% but it will improve the layout of the manuscript,
% and will typically save some space.
\usepackage{microtype}

% This is also not strictly necessary, and may be commented out.
% However, it will improve the aesthetics of text in
% the typewriter font.
\usepackage{inconsolata}

%Including images in your LaTeX document requires adding
%additional package(s)
\usepackage{graphicx}
\usepackage{array}
\usepackage{tabularx}
\usepackage{float}
\usepackage{multicol, multirow}
\usepackage{booktabs}
\usepackage{amsmath}
\usepackage{tcolorbox}
\usepackage{enumitem}
\usepackage{bookmark}
\usepackage{adjustbox}
\usepackage{makecell}
\usepackage{xurl}
\usepackage{colortbl}  % For coloring
\usepackage{tikz}      % For drawing the oval
\usepackage{enumitem}
\usepackage{soul}

\usetikzlibrary{shapes.multipart, positioning}
\raggedbottom

\newcommand{\thought}[1]{{\color[rgb]{0.2,0.39,0.66}(#1)}}
\newcommand{\todo}[1]{{\color[rgb]{1.0,0.0,0.0}(#1)}}
\newcommand{\hsh}[1]{{\color{green!50!black} Henrik: #1}}
\newcommand{\st}[1]{{\color{red!50!black} Sebastian: #1}}

\newcommand{\ulm}[1]{_{\scaleto{\mathrm{#1}}{3pt}}}
\newcommand\at[2]{\left.#1\right|_{#2}}











\newtheorem{assumption}{Assumption}

\DeclareMathOperator*{\argmax}{arg\,max}
\DeclareMathOperator*{\argmin}{arg\,min}

\newcommand{\swname}[1]{\texttt{#1}}
\newcommand{\ie}{i\/.\/e\/.,\/~}
\newcommand{\eg}{e\/.\/g\/.,\/~}
\newcommand{\cf}{cf\/.\/~}

\newcommand{\fig}{Fig\/.\/~}
\newcommand{\defn}{Def\/.\/~}
\newcommand{\sect}{Sec\/.\/~}
\newcommand{\tabl}{Tab\/.\/~}
\newcommand{\algo}{Algorithm~}
\newcommand{\theo}{Theorem~}

\newcommand{\bnnl}{3 hidden layers}
\newcommand{\bnnn}{50 neurons}
\newcommand{\bnna}{tanh activations}

\newcommand{\capt}[1]{\mdseries{\emph{#1}}}

\newcommand{\videolink}{at \url{https://youtu.be/_d7AqTRjz6g}}
\newcommand{\codelink}{\url{https://github.com/wheelbot/mini-wheelbot}}

\newcommand{\fakepar}[1]{\vspace{0mm}\noindent\textbf{#1.}}

\newcommand{\needref}{\textcolor{red}{[REF]}}

\newcommand{\plotfontsize}{9pt}


\definecolor{myblue}{RGB}{173, 216, 230} % Light blue background
\definecolor{myborder}{RGB}{0, 102, 204} % Dark blue border

\setlength{\aboverulesep}{1.5pt}  % Reduce space above \midrule
\setlength{\belowrulesep}{1.5pt}  % Reduce space below \midrule

\newcommand{\rulesep}{\unskip\ \vrule\ }
\newcolumntype{Y}{>{\centering\arraybackslash}X}
\newtcbox{\myovalbox}{colback=cyan,boxrule=0pt,arc=2pt,
  boxsep=0pt,left=1pt,right=1pt,top=0pt,bottom=0pt}

\newcommand{\psulogo}{\raisebox{3.4pt}{\includegraphics[scale=0.025]{./images/logos/penn-state-shield.jpeg}}}
\newcommand{\nittlogo}{\raisebox{3.4pt}{\includegraphics[scale=0.025]{./images/logos/nitt_logo.png}}}
% If the title and author information does not fit in the area allocated, uncomment the following
%
%\setlength\titlebox{<dim>}
%
% and set <dim> to something 5cm or larger.

\title{{\name}: Can Multimodal Large Language Models Effectively Understand Coherence Relationships?}

\author{
  \textbf{Aashish Anantha Ramakrishnan\psulogo},
  \textbf{Aadarsh Anantha Ramakrishnan\nittlogo},
  \textbf{Dongwon Lee\psulogo}
\\
  The Pennsylvania State University\textsuperscript{\psulogo},
  National Institute of Technology, Tiruchirappalli\textsuperscript{\nittlogo}
\\
  \texttt{
    \{aza6352, dul13\}@psu.edu\psulogo, 106121001@nitt.edu\nittlogo
  }
}


\begin{document}
\maketitle


\begin{abstract}
Multimodal Large Language Models (MLLMs) are renowned for their superior instruction-following and reasoning capabilities across diverse problem domains. However, existing benchmarks primarily focus on assessing factual and logical correctness in downstream tasks, with limited emphasis on evaluating MLLMs' ability to interpret pragmatic cues and intermodal relationships. To address this gap, we assess the competency of MLLMs in performing {\em Multimodal Discourse Analysis} (MDA) using Coherence Relations. Our benchmark, {\name}, encompasses a broad spectrum of Coherence Relations across 3 different discourse domains at varying levels of granularity. Through our experiments on 10+ MLLMs employing different prompting strategies, we show that even top models like Gemini 1.5 Pro and GPT-4o fail to match the performance of simple classifier-based baselines. This study emphasizes the need to move beyond similarity-based metrics and adopt a discourse-driven framework for evaluating MLLMs, providing a more nuanced assessment of their capabilities. The benchmark and code are available at: \url{https://github.com/aashish2000/CORDIAL}.


\end{abstract}

\section{Introduction}
\label{sec:intro}
% Image editing methods in diffusion models depend on user-defined control directions - users can unlock their creativity using these methods by specifying the desired manipulation through prompts~\cite{gandikota2023concept}, reference images~\cite{ruiz2022dreambooth, kumari2022customdiffusion, gal2022image, chen2024trainingfreeregionalpromptingdiffusion}, or attribute vectors~\cite{parmar2023zero,hertz2022prompt}. In this work, we ask a fundamentally different question: \emph{Can we automatically discover the underlying visual structure of a concept within diffusion model's knowledge?} %Rather than requiring user-specified controls, we aim to decompose the model's internal knowledge into meaningful directions.

% This question touches on a fundamental limitation in how we interact with diffusion models. Current control methods ~\cite{zhang2023addingconditionalcontroltexttoimage, gandikota2023concept, ye2023ipadaptertextcompatibleimage,ye2023ipadaptertextcompatibleimage, hertz2024stylealignedimagegeneration, li2023photomaker, shi2024instantbooth, chen2024trainingfreeregionalpromptingdiffusion} require users to specify their desired manipulations in advance, limiting interactive creativity. This contrasts with natural human artistic workflows, where creators dynamically explore creative ideas while jointly refining them toward meaningful artistic outcomes~\cite{hoffmann2016modeling}. This synergy between specification and exploration is not new to generative models. Early GAN architectures naturally developed disentangled latent spaces that enabled continuous\cite{harkonen2020ganspace,radford2015unsupervised, wu2021stylespace, shen2020interfacegan}, compositional control over generated images. Users could explore these spaces to discover interesting variations that would be difficult to describe in words~\cite{wu2021stylespace}, then combine them to achieve their creative goals~\cite{grabe2022towards}. 


% While diffusion models have largely superseded GANs in conditional image synthesis~\cite{dhariwal2021diffusion},  their underlying structure remains less understood. Diffusion models achieve remarkable diversity through high-dimensional latents, unlike GANs' compact latent spaces.  With a single prompt, diffusion models can generate radically different variations through different random initializations of input noise. We ask - Is it possible to discover interpretable structure within this vast space of variations?

Text-to-image diffusion models are capable of generating remarkable visual variations from a single prompt through different random initializations. However, this vast creative potential remains largely opaque to users---while we can generate diverse images, we lack understanding of the underlying structure of these variations. This presents a fundamental challenge: how can we discover and expose the latent visual capabilities encoded within these models?

\let\thefootnote\relax \footnote{$^{*}$Correspondence to \texttt{gandikota.ro@northeastern.edu}}

The challenge touches on a key limitation in how we interact with diffusion models today. Current control methods require users to explicitly specify their desired edits in advance through prompts~\cite{gandikota2023concept}, reference images~\cite{zhang2023addingconditionalcontroltexttoimage, chen2024trainingfreeregionalpromptingdiffusion, ruiz2022dreambooth,kumari2022customdiffusion, Ryu_lora, hu2021lora}, or attribute vectors~\cite{ye2023ipadaptertextcompatibleimage, hertz2024stylealignedimagegeneration, li2023photomaker, shi2024instantbooth,parmar2023zero,hertz2022prompt}. That contrasts sharply with natural human creative workflows, where artists dynamically explore creative ideas and jointly refine them toward meaningful artistic outcomes~\cite{hoffmann2016modeling}. The need for pre-specified controls creates a barrier between users and the full creative potential of these models.

Interestingly, earlier generative models like GANs~\cite{gans,karras2019style,brock2018large} naturally developed more interpretable internal structures. Their compact latent spaces often exhibited emergent disentanglement~\cite{harkonen2020ganspace,radford2015unsupervised, wu2021stylespace, shen2020interfacegan}, enabling continuous and compositional control over generated images. Users could explore these spaces to discover interesting variations that would be difficult to describe in words~\cite{wu2021stylespace}, then combine them to achieve their creative goals~\cite{grabe2022towards}.

Diffusion models have largely superseded GANs in conditional image synthesis~\cite{dhariwal2021diffusion}, achieving greater diversity through much higher-dimensional latents. And yet an understanding of the underlying structure of these larger latent spaces has remained elusive. In this work, we ask a fundamental question: \emph{Can we automatically discover the visual structure within a diffusion model's knowledge of a concept?} Rather than requiring user-specified controls, we aim to decompose the model's internal representations into expressive directions that users can explore and combine.

To address these needs, we present \textbf{SliderSpace}, a framework that brings systematic explorability to diffusion models. Given just a text prompt, SliderSpace discovers a canonical set of meaningful, diverse, and controllable directions within the model's knowledge of that concept. Each direction is implemented as a low-rank adapter~\cite{hu2021lora} that can be scaled and composed with others, allowing users to explore and smoothly combine different aspects of variation, as shown in Figure~\ref{fig:intro}.

We ground SliderSpace discovery in three key requirements for meaningful decomposition of a diffusion model's visual manifold: 
\begin{enumerate}
    \item \textbf{Unsupervised Discovery:} The decomposition process should emerge from the intrinsic structure of the model's learned representation, rather than being guided by predefined attributes. This ensures we capture the true topology of the model's knowledge space rather than projecting our assumptions onto it.
    
    \item \textbf{Semantic Orthogonality:} Each discovered control must represent a distinct semantic direction. This is enforced in a semantic feature space, like CLIP, where every slider has an orthogonal effect in embeddings. This prevents discovering multiple controls that create similar semantic effects, making the system more efficient and easier.
    
    \item \textbf{Distribution Consistency:} Directions must induce consistent transformations across both random seeds and prompt variations. 
\end{enumerate}

These requirements naturally lead to our proposed framework, which we formalize in Section~\ref{sec:method}. As we show in our experiments, SliderSpace is architecture-agnostic, working with both conventional U-Net based models like Stable Diffusion~\cite{rombach2022high, rombach2022sd20, podell2023sdxl, turbo, dmd} and recent transformer-based architectures like Flux~\cite{flux}.

We demonstrate the expressiveness of SliderSpace through three applications: First, we show how SliderSpace can decompose high-level concepts into diverse and expressive components, revealing the natural axes of variation in the model's understanding. Second, we explore artistic style variation, where SliderSpace discovers directions that match or exceed the diversity of manually curated artist lists while being judged more useful by human evaluators. Finally, we show how SliderSpace can help reverse the mode collapse commonly observed in distilled diffusion models, restoring diversity while maintaining generation speed.

Beyond providing practical creative control, SliderSpace opens new avenues for understanding and utilizing the latent capabilities of diffusion models. By mapping these models' visual potential into intuitive, composable directions, we take a step toward making their creative possibilities more accessible and interpretable to users.

% Image editing methods in diffusion models unlock the creativity of users. In this work we ask an alternate question: \emph{Can we organize and expose what of the diffusion model is already capable of?}.
% Existing methods for controlling image generation typically require users to manually specify edit directions for desired changes. This process is time-consuming, requires technical expertise, and limits the spontaneity of the creative process. For instance, if a user wants to adjust the smile of a generated person, they must explicitly request this edit, often through imprecise prompt engineering or model fine-tuning. This approach of predefined controls or manual specifications restricts users from fully exploring the latent capabilities of the model. There may be interesting stylistic variations or attributes that the model can generate, but users have no easy way to discover or utilize these.

% Natural visual disentanglement was an emergent property in the latent space of Generative Adversarial Models (GANs) \cite{harkonen2020ganspace,radford2015unsupervised, wu2021stylespace, shen2020interfacegan}. In particular, it has been observed that StyleGAN~\cite{karras2019style} stylespace neurons offer detailed control over many meaningful aspects of images that would be difficult to describe in words~\cite{wu2021stylespace}. However, diffusion models do not share such a compact latent space~\cite{park2023unsupervised}; and efforts to uncover such a space in the semantic embeddings of the text conditioning have met with limited success \nik{Nick - is there a specific citation you were thinking about?}.

% In this work we introduce \textbf{SliderSpace}, which takes a step towards uncovering an analogous low dimensional representation of diffusion models' visual breadth; in essence treating the diffusion model as many generators sharing parameters, where a particular generator is defined by a specific prompt. For a given prompt we sample many random seeds (and optionally prompt expansions using an LLM), generate the corresponding images, and apply an off the shelf feature extractor (in this work CLIP, but our method can be applied to any differentiable feature extractor). We use PCA to analyze these features, and for each of the leading $k$ principal components we train a LoRA \cite{} which causes the diffusion model to produces images which increase the feature magnitude along that component when passed back through the same feature extractor. This leads to a 'Slider' for each principal component, because each LoRA can be scaled and applied to the original diffusion model, continuously varying those visual features in the generated results (as measured, in our case, by CLIP).

% There are many other works that enhance the controllability of diffusion models. One common approach is enabling users to add spatial constraints to a generation either manually, or via a reference image \cite{zhang2023addingconditionalcontroltexttoimage, chen2024trainingfreeregionalpromptingdiffusion}, a second is leveraging more abstract embeddings (e.g. identity, style) extracted from a reference image \cite{ye2023ipadaptertextcompatibleimage, hertz2024stylealignedimagegeneration, li2023photomaker, shi2024instantbooth}, a third is finetuning a foundation model to better generate a concept important to the user \cite{ruiz2022dreambooth, kumari2022customdiffusion, Ryu_lora, hu2021lora}, and a fourth (most relevant to this work) is finding low-rank adaptors of the model based on a prompt or small training set which can be scaled to provide continous control over one aspect of generated image (e.g. night vs day, basic vs luxury, etc.) \cite{gandikota2023concept}. SliderSpace is complementary to all of these methods and offers something distinct. All of the other methods we are aware require the user (and / or model designer) to know in advance what type of control they want. In contrast SliderSpace assists users in discovering and controlling hidden capabilities present in the diffusion model's distribution of possible generations.

%We propose that truly intuitive creative control in a text-to-image model should meet three key criteria: \emph{discoverability}, \emph{intuitiveness}, and \emph{specificity}. The model should reveal controllable attributes that may not be immediately obvious, offer controls that are easy to understand and manipulate, and ensure each control affects a distinct attribute of the generated image.

% We demonstrate the utility and power of SliderSpace using three applications built on top of SDXL-DMD \cite{dmd}, because its fast generation speed lends itself well to the continuous control offered by SliderSpace.

% First, we study concept decomposition (Section \ref{sec:concept_exp}), where we learn sliders for a specific concept (e.g. 'monster', 'waterfall', 'car'). Through quantitative metrics of diversity and text alignment we demonstrate that the learned sliders dramatically boost the diversity of generations when randomly applied without harming text alignment; we also ask humans to qualitatively judge these results in a user study where they find the SliderSpace results to be more 'Diverse', 'Useful', and 'Creative' than our baselines.

% Second, we attempt to compare the automatic discoveries of SliderSpace to a large scale manual study of artistic styles (Section \ref{sec:art_exp}), open-sourced by ParrotZone \cite{parrotzone}. In this study SDXL was prompted with over 4300 artist names,  and based on visual inspection the cases of successful stylistic mimicry recorded. Quantitatively SliderSpace more closely matches the distribution of artistic variation discovered by ParrotZone than other baselines, and in our user studies was judged to be significantly more 'Diverse' and 'Useful' than the baselines. To our surprise humans even judged SliderSpace results to be slightly more 'Diverse' than the results generated by the manually discovered artist names of \cite{parrotzone}.

% Third, we attempt to use SliderSpace to reverse the mode collapse commonly observed in distilled few-step diffusion models relative to the original teacher model (Section \ref{sec:diverse_exp}). We quantitatively demonstrate that applying SliderSpace to SDXL-DMD leads to more closely matching the distribution of images by the original teacher, SDXL.

%Through extensive experiments on various state-of-the-art text-to-image models, we demonstrate that SliderSpace significantly enhances user control and creative expression in AI-assisted image generation tasks. Our method enables a range of applications, including concept decomposition and control, diversity improvement in generated images, customization dissection and edits, and the exploration of artistic styles inherent in the model.

% SliderSpace goes beyond providing a practical tool for enhanced creative control. By mapping the visual potential of diffusion models it can open new avenues for generative creativity and deepens our understanding of each model's hidden potential.
\section{Related Work}
\label{sec:related_work}

The original investigation \cite{gibson1979ecological} on the relationship between visual perception and human action defines \emph{affordance} as the opportunities for interaction with the surrounding environment. Behavioral studies on regular and cognitively impaired persons have shown evidence that perception results in both visual and motor signals in the human brain. An extended study \cite{anderson2002attentional} shows that visual attention to the spatial characteristics of the perceived objects initiates automatic motor signals for different actions. In computer vision, human affordance learning involves novel pose prediction such that the estimated pose represents a valid human action within the scene context. The task is fundamental to many problems requiring robust semantic reasoning about the environment, such as human motion synthesis \cite{wang2021scene} and scene-aware human pose generation \cite{wang2017binge, roy2016multi, zhang2022inpaint, yao2023scene}.

Earlier methods of affordance learning have explored knowledge mining \cite{zhu2014reasoning} and multimodal feature cues \cite{roy2016multi} to address the problem. In \cite{zhu2014reasoning}, the authors use a Markov Logic Network for constructing a knowledge base by extracting several object attributes from different image and metadata sources, which can perform various downstream visual inference tasks without any additional classifier, including zero-shot affordance prediction. In \cite{roy2016multi}, the authors use depth map, surface normals, and segmentation map as multimodal cues to train a multi-scale convolutional neural network (CNN) for scene-level semantic label assignment associated with specific human actions. In \cite{do2018affordancenet}, the authors design a multi-branch end-to-end CNN with two separate pathways for object detection and affordance label assignment to achieve high real-time inference throughput. Researchers \cite{chuang2018learning} have also explored socially imposed constraints for affordance learning. In \cite{chuang2018learning}, the authors propose a graph neural network (GNN) to propagate contextual scene information from egocentric views for action-object affordance reasoning.

Probabilistic modeling of scene-aware human motion generation also involves semantic reasoning of human interaction with the environment. Initial works on human motion synthesis have taken different architectural approaches, such as sequence-to-sequence models \cite{barsoum2018hp}, generative adversarial networks (GAN) \cite{barsoum2018hp, cai2018deep, yang2018pose}, graph convolutional networks (GCN) \cite{yan2019convolutional}, and variational autoencoders (VAE) \cite{guo2020action2motion}. However, these methods have mostly ignored the role of environmental semantics. Due to potential uncertainty in human motion, in a recent approach \cite{wang2021scene}, the authors address such motion synthesis with a GAN conditioned on scene attributes and motion trajectory to predict probable body pose dynamics.

One key challenge of human affordance generation in 2D scenes is the lack of large-scale datasets with rich pose annotations. In \cite{wang2017binge}, the authors compile the only public dataset of annotated human body poses in complex 2D indoor scenes by extracting frames from sitcom videos. Aiming to generate a contextually valid human affordance at a user-defined location, the authors propose sampling the scale and deformation parameters for an existing human pose template using a VAE conditioned on the localized image patches as scene context. In \cite{zhang2022inpaint}, the authors introduce a two-stage GAN architecture for achieving a similar goal by estimating the affine bounding box parameters to localize a probable human in the scene and then generating a potential body pose at that location. The method uses the input scene, corresponding depth, and segmentation maps as semantic guidance. In \cite{yao2023scene}, the authors propose a transformer-based approach with knowledge distillation for generating human affordances in 2D indoor scenes.


% introduce PDDL domains
% why Gripper env as testing context
% motivation: comparing classical vs LLM planners
% - classical: PDDL solver fast-downward
% - LLM: gpt-4o
% explanation and refinement are two distinguishing features of LLM planners
% - how we demonstrate explanation and refinement in the study
We evaluate user trust in two planners over a set of planning problems and study the potential factors influencing user trust in the planners. In particular, we compare a language-model-based planner, denoted as an \emph{LLM Planner}, with a traditional graph-search-based planner, denoted as a \emph{PDDL Solver}. The PDDL Solver uses Fast Downwards \cite{fastdownward} as its underlying model, processing planning problems described in PDDL to generate an optimal solution. In comparison, the LLM Planner employs GPT-4o to interpret the planning problem and extract a solution generated by the language model. Unlike the PDDL Solver, the LLM Planner can reason through the planning problem, explain its proposed solution, and iteratively refine the solution based on external feedback. This study investigates how the correctness of solutions, the quality of explanations, and the refinement process influence user trust.

\subsection{Planning Problem}
% \begin{wrapfigure}{r}{0.4\textwidth}
% % \begin{figure}[t]
%     \centering
%     \includegraphics[width=\linewidth]{figures/problem-example.pdf}
%     \caption{A running example of a planning problem in our study.}
%     \Description{Planning Problem Example}
%     \label{fig: problem-example}
% % \end{figure}
% \end{wrapfigure}

We describe each planning problem in the \emph{Planning Domain Definition Language (PDDL)} and propose two planners to generate plans that solve the problem. We select the \emph{gripper} planning problems from the International Planning Competition \cite{IPC} for plan generation and evaluation. In a gripper planning problem, a robot moves balls between a set of rooms using two grippers. The objective is to create a plan for the robot to move the balls to the target rooms we defined. We present a few running examples of the gripper problem in Figure \ref{fig: correctness}.

A planning problem consists of a \emph{planning domain} and a \emph{problem description}, expressed in PDDL. 

\paragraph{Planning Domain}
A planning domain refers to the universal aspects of a problem that remains consistent across different instances of the problem. In particular, it defines the types of objects, predicates, and actions that exist in the planning problem. We present an example of the gripper problem in Appendix \ref{app: grippers}.

\paragraph{Problem Description} A problem description specifies the particular instance of a planning task within a given domain. It includes the planning domain to which it pertains, a set of objects, the initial state of these objects, and the goal state to be achieved.

\paragraph{Plan}
A plan is a sequence of actions with specific input parameters. Recall that an action corresponds to a state transition. If a plan (a sequence of actions) transits from the initial state to the goal state defined by a problem, then we consider the plan to be \emph{correct}. If a plan does not transit to the goal state or there exists an action violating its precondition, then the plan is \emph{wrong}.

\begin{figure}[t]
    \centering
    \includegraphics[width=0.8\linewidth]{figures/correct.jpeg}
    \caption{Examples where LLM Planner correctly generates a plan for the gripper planning problem.}
    \Description{Planning Problem Correctness}
    \label{fig: correct}
\end{figure}

\subsection{PDDL Solver}
The PDDL Solver takes the planning domain and the problem description as inputs and then generates a plan described in PDDL. 
% It generates a plan in the following format:
% \vspace{4pt}
% \begin{lstlisting}[language=completion]
% (move robot1 room1 room3)
% (pick robot1 ball2 room3 rgripper1)
% (move robot1 room3 room2) ......
% \end{lstlisting}
Next, we convert the generated plan into natural language for user studies following the procedure in \cite{seipp-et-al-zenodo2022} and display it to users. We present an example in Figure \ref{fig: correct}.

The PDDL Solver applies a graph search algorithm to find a path (i.e., a list of transitions) from the initial state to the goal state. It either generates a \emph{correct} plan---defined as the shortest path between the initial and goal states---or returns a signal indicating that no solution exists for the given problem.

\subsection{LLM Planner}

The LLM Planner addresses planning problems by querying a large language model. In particular, it transmits the planning domain and problem description to the language model using a structured prompt format. The planner then retrieves a natural language plan from the language model. We use GPT-4o as the language model for the planner. To ensure the output adheres to the desired format, we include a few in-context examples within the prompts.

A language model solves a planning problem by interpreting the domain and problem descriptions, simulating state transitions, and generating a sequence of actions to achieve the goal. While effective for reasoning and plan generation, language models may struggle with large state spaces. Unlike the PDDL Solver, the LLM Planner may generate \emph{incorrect} plans that violate the problem specifications (e.g., preconditions of actions) or fail to achieve the goal.

\subsection{Explanation and Refinement}
Alongside the generated plans, we offer detailed explanations of all the plans and revisions of any incorrect plans. This study examines how these explanations and refinements influence human trust in the two planners.

\paragraph{LLM Planner with Explanation (LLM+Expl)}
For each generated plan, we manually provide a natural language explanation. This explanation includes an assessment of the plan’s correctness, identification of any violations of action preconditions, and an analysis of inconsistencies between the final state achieved and the intended goal state. We present examples of explanations in Figure \ref{fig: explain} in Appendix.

In particular, if a plan is correct, the explanation is simply ``the plan successfully satisfies the goal conditions.'' 
If a plan is incorrect, we identify the underlying cause as either a violation of action preconditions or a failure to achieve the goal state. In cases involving precondition violations, we specify the action responsible for the issue. For example, consider the action ``robot moves from room 1 to room 2,'' but the robot is initially located in room 3. This scenario constitutes a violation of the precondition for the ``move'' action. In the latter case, we describe the differences between the final state achieved and the intended goal state, e.g., ``fail to move ball 2 to room 2.''

% \begin{wrapfigure}{r}{0.5\textwidth}
%     \centering
%     \includegraphics[width=0.98\linewidth]{figures/refine.jpeg}
%     \includegraphics[width=0.98\linewidth]{figures/refine-correct.jpeg}
%     \includegraphics[width=0.98\linewidth]{figures/refine-wrong.jpeg}
%     \caption{Plan refinement by the LLM Planner. The top row presents two choices of plan refinement (where the refinement starts). The second and third row shows the refinement outcomes of the two choices, where the second row shows a correctly refined plan and the third row shows an incorrect plan.}
%     \Description{Refinement}
%     \label{fig: refine}
% \end{wrapfigure}

\paragraph{LLM Planner with Refinement (LLM+Refine)}
Note that a plan generated by the LLM Planner could be incorrect. Therefore, we offer a prompting mechanism for the LLM Planner to refine the generated plan according to the user feedback. The mechanism works as follows:

1. Request the user to indicate the step number of the first action in the plan that is incorrect, such as the step where an action’s precondition is violated. We present a sample user interface on the left of Figure \ref{fig: refine} in Appendix.

2. Send the planning domain, problem description, and the original plan to the language model. Then, query the model to rewrite the subsequent steps starting from the user-specified step number. We present a sample input prompt in Figure \ref{fig: refine-prompt} in the Appendix.

3. Replace the original plan with the newly refined plan and display it to the user.

This mechanism allows users to interact with the language model to refine the plan. It enables the language model to focus on a subset of steps, facilitating a deeper interpretation of the incorrect component. However, the correctness of the refined plan is not guaranteed. Figure \ref{fig: refine} in the Appendix shows an example of a correctly refined plan and an incorrectly refined plan.

\section{Experiments}
\label{sec:experiments}

\begin{figure*}[t]
\vspace{-6mm}
    \centering
    \includegraphics[width=0.8\linewidth]{figs/compare.pdf}
    \vspace{-4mm}
    \caption{\textbf{Qualitative comparison} with the baseline for generating a sequence of novel view images.  
    The results demonstrate that our method synthesizes more consistent multi-view images compared to our baseline model (Zero123). In addition, compared to SyncDreamer, our method visually maintains better similarity to the conditioned image and appears more natural.}
    \label{fig:sota_compare}
\vspace{-5mm}
\end{figure*}

\subsection{Experimental Setups}
\textbf{Dataset.}
Following previous work~\cite{zero123, SyncDreamer}, we evaluate our work on the Google Scanned Object (GSO)~\cite{GSO} dataset to verify the zero-shot novel view image synthesis capability. 
We also provide results for additional datasets in the Supplementary Material.
Specifically, we randomly select 30 objects from the GSO dataset with various object categories. 
Unlike recent approaches~\cite{mvdream, SyncDreamer} that aim to enhance the consistency of novel view synthesis models by generating multiple fixed-view images, our method can generate images from any camera pose and any number of views. Therefore, we conduct experiments under different camera pose settings to validate our approach:
specifically, 
1) \textit{16-views with free camera pose}: for each object, we circularly render 16 views with the elevation angles ranging in $[-10\degree, 40\degree]$ and the azimuth angles are evenly distributed in $[0\degree, 360\degree]$. 
2) \textit{16-views with fixed camera pose}: We maintain a constant elevation angle of $30\degree$ and uniformly sample azimuth angles (same as SyncDreamer~\cite{SyncDreamer}).
3) \textit{32-views with free camera pose}: Similar to the first setting, but we sample 32 views.
It's important to note that our method does not require additional training or fine-tuning on any datasets.

\noindent\textbf{Metrics.}
To validate the effectiveness of our method, we mainly evaluate it based on three criteria:
1) \textit{Quality Score}. We evaluate the image quality of synthesized multi-view images by measuring their similarity with ground truth images. Following prior research~\cite{zero123, sparsefusion}, we report the similarity between the synthesized images and the ground truth images with standard metrics: PSNR, SSIM~\cite{ssim}, and LPIPS~\cite{lpips}.
2) \textit{Multi-view Consistency Score}. As the primary goal of our work is to improve the consistency of generated images, we also employ the 3D consistency score~\cite{3dim} to verify the consistency among the synthesized images. Specifically, we train an Instant-NGP~\cite{instant_ngp} with the input image and part of the synthesized novel view images of our model and evaluate the similarity between the remaining synthesized images and the rendered images of Instant-NGP. For the synthesized multi-view images of each object, we allocate $3/4$ for training and reserve the remaining $1/4$ for validation.
Intuitively, if the consistency of synthesized images is improved, the NeRF-like model will train a better object representation, and the re-rendered images will agree more with the validation images.
3) \textit{Input Consistency Score}. To assess the faithfulness of synthesized images in preserving the identity of the input condition image, we introduce the input consistency score. This score calculates the similarity of each synthesized image with the input condition image, utilizing the LPIPS metric.

In addition, we use synthesized multi-view images to train a neural 3D reconstruction model (NeuS~\cite{neus}) and report commonly used Chamfer Distances (CD) and Volume IoUs between the trained 3D model and the ground truth.

\noindent\textbf{Baselines.}
Given that our main goal is to improve the consistency of the trained baseline model without further fine-tuning, we mainly compare our approach with the used baseline model Zero123~\cite{zero123}. Additionally, we compare our method to the SOTA approaches such as PGD~\cite{tseng2023consistent} and SyncDreamer~\cite{SyncDreamer} using the same Zero123 base model.

\noindent\textbf{Implementation Details.}
We use the official checkpoint provided by Zero123~\cite{zero123}, which is trained on objaverse~\cite{objaverse} for 165,000 steps. We inject our epipolar attention layer after step $T=4$ and layer $L=10$ by default. We find that feature fusion weight $\alpha=0.5$, and the number of context views $M=2$ work better.

\begin{table}[t]
\centering
\caption{Comparison of multi-view consistency, image quality, and input consistency of synthesized multi-view images at the 16-view setting with free camera pose.}
\label{tab:view16_free_compare}
\vspace{-2mm}
\scalebox{0.6}{
\begin{tabular}{c ccc ccc c}
\toprule
              & \multicolumn{3}{c}{Multi-view Consistency} & \multicolumn{3}{c}{Quality Score} & \multicolumn{1}{c}{Input Consis.} \\
              \cmidrule(lr){2-4} \cmidrule(lr){5-7} \cmidrule(lr){8-8}
              & PSNR$\uparrow$  & SSIM$\uparrow$ & LPIPS$\downarrow$ 
              & PSNR$\uparrow$  & SSIM$\uparrow$ & LPIPS$\downarrow$ 
              & LPIPS$\downarrow$ 
              \\ \midrule

Zero123
& 15.225        & 0.645       & 0.408
& 14.255        & 0.747       &	0.208
& 0.303         
\\
SyncDreamer
& 14.830        & 0.626       & 0.434
& 12.650        & 0.713       &	0.254
& 0.317         
\\
Ours 
& \best{18.300}	& \best{0.734}	& \best{0.355}
& \best{14.947}	& \best{0.763}	& \best{0.191}
& \best{0.282}
\\

\bottomrule
\end{tabular}
}
\end{table}

\begin{table}[t]
\vspace{-1mm}
\centering
\caption{Comparison of multi-view consistency, image quality, and input consistency at the 16-view setting with fixed camera pose as SyncDreamer~\cite{SyncDreamer}.}
\label{tab:view16_fxied_compare}
\vspace{-3mm}
\scalebox{0.6}{
\begin{tabular}{c ccc ccc c}
\toprule
              & \multicolumn{3}{c}{Multi-view Consistency} & \multicolumn{3}{c}{Quality Score} & \multicolumn{1}{c}{Input Consis.} \\
              \cmidrule(lr){2-4} \cmidrule(lr){5-7} \cmidrule(lr){8-8}
              & PSNR$\uparrow$  & SSIM$\uparrow$ & LPIPS$\downarrow$ 
              & PSNR$\uparrow$  & SSIM$\uparrow$ & LPIPS$\downarrow$ 
              & LPIPS$\downarrow$ 
              \\ \midrule

Zero123
& 16.556        & 0.682       & 0.378
& 14.592        & 0.750       &	0.207
& 0.305         
\\
SyncDreamer
& \best{22.424}        & \best{0.812}       & \best{0.268}
& 15.269        & 0.749       &	0.196
& 0.300         
\\
Ours 
& 21.151	& 0.780	& 0.302
& \best{15.293}	& \best{0.764}	& \best{0.184}
& \best{0.287}
\\

\bottomrule
\end{tabular}
}
\vspace{-4mm}
\end{table}


\subsection{Comparison With Baseline Models}
The quantitative comparison on three settings are shown in Tab.~\ref{tab:view16_free_compare}, Tab.~\ref{tab:view16_fxied_compare}, and Tab.~\ref{tab:view32_free_compare}. The qualitative comparison is shown in Fig.~\ref{fig:sota_compare}.

\begin{table}[t]
\centering
\caption{Comparison of multi-view consistency and image quality scores of synthesized multi-view images at the 32-view setting with free camera pose.}
\vspace{-3mm}
\label{tab:view32_free_compare}
\scalebox{0.7}{
\begin{tabular}{c ccc ccc}
\toprule
              & \multicolumn{3}{c}{Multi-view Consistency} & \multicolumn{3}{c}{Quality Score} \\
              \cmidrule(lr){2-4} \cmidrule(lr){5-7}
              & PSNR$\uparrow$  & SSIM$\uparrow$ & LPIPS$\downarrow$ 
              & PSNR$\uparrow$  & SSIM$\uparrow$ & LPIPS$\downarrow$ 
              \\ \midrule

Zero123
& 16.515        & 0.694       & 0.378
& 15.142        & 0.733       &	0.211
\\
PGD~\cite{tseng2023consistent}
& 18.481        & 0.720       & 0.343
& 15.281        & 0.739       &	0.205
\\
Ours 
& \best{20.655}	& \best{0.792}	& \best{0.305}
& \best{15.268}	& \best{0.742}	& \best{0.203}
\\

\bottomrule
\end{tabular}
}
\vspace{-3mm}
\end{table}

\begin{table*}
  [t]
  \centering
  \resizebox{\textwidth}{!}{%
  \begin{tabular}{cccccccccccc}
    \toprule \multicolumn{2}{c}{Components}                                                             & \multicolumn{5}{c}{Re-executability Rate (\%)} & \multicolumn{5}{c}{Readability (\#)} \\
    \cmidrule(lr){1-2} \cmidrule(lr){3-7} \cmidrule(lr){8-12}        \hspace{8pt}\labelemoji\hspace{8pt}                                                                & \hspace{8pt}\toolemoji\hspace{8pt}                                      & O0                                 & O1             & O2             & O3             & AVG            & O0             & O1             & O2             & O3             & AVG            \\
    \hline
    \rowcolor[rgb]{0.93,0.93,0.93}\multicolumn{12}{c}{\textbf{Initialize with LLM4Decompile-End-6.7B~\citep{llm4decompile}}}   \\
    \xmark                                                                                              & \xmark                                    & 69.51                              & 46.95          & 50.61          & 46.34          & 53.35          & 3.98 & 3.41 & 3.44 & 3.38 & 3.55 \\
    \cmark                                                                                              & \xmark                                    & 75.61                              & 50.61          & 50.00          & 50.00          & 56.55          & 4.01 & 3.44 & 3.39 & \textbf{3.49} & 3.58 \\
    \xmark                                                                                              & \cmark                                    & 83.54                     & \textbf{56.10}          & 51.22          & 50.61 & 60.37 & 4.05 & 3.51 & 3.51 & 3.42 & 3.62 \\
    \cmark                                                                                              & \cmark                                    & \textbf{85.37}                            & \textbf{56.10}                     & \textbf{51.83} & \textbf{52.43}          & \textbf{61.43} & \textbf{4.13} & \textbf{3.60} & \textbf{3.54} & \textbf{3.49} & \textbf{3.69} \\

    \rowcolor[rgb]{0.93,0.93,0.93}\multicolumn{12}{c}{\textbf{Initialize with Deepseek-Coder-6.7B-base~\citep{deepseekcoder}}} \\
    \xmark                                                                                              & \xmark                                    & 59.15                              & 35.98          & 39.02          & 37.80          & 42.99          & 3.71 & 3.05 & 3.16 & 3.05 & 3.24 \\
    \cmark                                                                                              & \xmark                                    & 66.46                              & 41.46          & 38.41          & 36.59          & 45.73          & 3.76 & 3.17 & \textbf{3.21} & 3.08 & 3.31 \\
    \xmark                                                                                              & \cmark                                    & 70.73                              & 39.63          & 39.02          & 40.24          & 47.41          & 3.90 & 3.17 & 3.08 & 3.11 & 3.31 \\
    \cmark                                                                                              & \cmark                                    & \textbf{79.88}                     & \textbf{45.73} & \textbf{43.90} & \textbf{42.68} & \textbf{53.05} & \textbf{3.96} & \textbf{3.21} & 3.18 & \textbf{3.19} & \textbf{3.38} \\
    \bottomrule
  \end{tabular}%
  }
  \caption{The ablation study of different methods across four optimization levels
  (O0, O1, O2, O3), as well as their average scores (AVG). The results in bold represent the optimal performance. The ~\labelemoji~ and ~\toolemoji~ means Relabedling and Function Call. \textbf{Bold} denotes the best performance.}
  \label{tab:ablation}
\end{table*}



\begin{figure*}[ht]
    \centering
    \begin{minipage}{0.65\textwidth}
        \centering
        \includegraphics[width=0.95\linewidth]{figs/ablation.pdf}
        \vspace{-2mm}
        \captionof{figure}{Qualitative Comparison for different design choices. Our method, employing multi-view epipolar attention, demonstrates the best consistency.}
        \label{fig:ablation}
    \end{minipage}\hfill
    \begin{minipage}{0.33\textwidth}
        \centering
        \includegraphics[width=0.8\linewidth]{figs/neus_ver.pdf}
        \vspace{-3mm}
        \caption{Our method shows better direct 3D reconstruction~\cite{neus}.}
        \label{fig:neus}
    \end{minipage}
    \vspace{-5mm}
\end{figure*}

\noindent\textbf{Multi-view Consistency.}
Tab.~\ref{tab:view16_fxied_compare} presents the 3D consistency scores compared to our baseline model (Zero123) and SyncDreamer. The results indicate a significant improvement across all three metrics achieved by our method when compared with Zero123.
While our method exhibits a marginally lower numerical consistency score compared to SyncDreamer, it enables the synthesis of images with arbitrary camera poses.	
This capability is illustrated in Tab.~\ref{tab:view16_free_compare}, where our method consistently enhances consistency with changes in camera pose settings, whereas SyncDreamer fails to do so and exhibits inferior results compared to Zero123.
Furthermore, our method facilitates the synthesis of multi-view images with any number of camera views. This versatility is demonstrated in Tab.~\ref{tab:view32_free_compare}, where our method continues to achieve significant improvements in consistency scores, while SyncDreamer is unable to operate under such conditions.	

Meanwhile, Fig.~\ref{fig:sota_compare} provides a qualitative comparison with the baseline. While both our method and SyncDreamer enhance consistency, our method visually preserves better similarity to the input image, including color and texture details. The input consistency score further corroborates this.

\noindent\textbf{Image Quality.}
While our primary goal centers around enhancing the consistency of synthesized multi-view images, we also evaluate the image quality by comparing the similarity with the ground truth images. The results shown in Tab.~\ref{tab:view16_free_compare}, Tab.~\ref{tab:view16_fxied_compare}, and Tab.~\ref{tab:view32_free_compare} indicate that our method also enhances the image quality under different settings besides improving the consistency.
Moreover, our method shows better image quality compared with SyncDreamer even in the 16-view setting with fixed camera pose.

\noindent\textbf{Input Consistency.}
Input consistency terms whether the results align with the input image.
Fig.~\ref{fig:sota_compare} illustrates that both our method and SyncDreamer enhance multi-view consistency. However, the color and texture details of SyncDreamer's results diverge from the input image and appear visually unnatural.
This discrepancy is evident in the input consistency score presented in Tab.~\ref{tab:view16_fxied_compare}, indicating lower similarity with the condition image in the SyncDreamer results.	

\subsection{Ablation Study}
The overall quantitative results are shown in Tab.~\ref{tab:ablation}, and the qualitative comparisons are shown in Fig.~\ref{fig:ablation}.

\noindent \textbf{Full Attention \vs Epipolar Attention.}
The results presented in Tab.\ref{tab:ablation} and Fig.\ref{fig:ablation} demonstrate that our epipolar attention mechanism can synthesize more consistent multi-view images compared with full attention. Furthermore, our epipolar attention achieves a greater performance improvement compared to full attention when using multiple reference images. This could be attributed to the fact that our epipolar attention more effectively localizes target information, as depicted in Fig.~\ref{fig:full_attn_compare}, thereby reducing noise from the reference images. In the multi-view setting, where multiple reference images are utilized, this noise reduction becomes particularly crucial.
Moreover, it is noteworthy that the epipolar attention mechanism consumes less GPU memory compared to our baseline, as discussed in Sec.~\ref{sec:attn_analysis}.

\noindent \textbf{Attending Single-View \vs Multi-View.}
Applying the epipolar attention significantly improves the consistency between the input and target views. However, the consistency between different views in the unobserved regions of the input view is not well preserved.
After implementing our epipolar attention in the multi-view setting, the consistency across the generated multi-view images is further improved. The last row in Tab.~\ref{tab:ablation} shows that after applying our multi-view epipolar attention, the consistency score is further improved compared with the single-view setting. Besides, the qualitative result in Fig.~\ref{fig:ablation} also shows better consistency among different target views.



\begin{table}[t]
\centering
\vspace{-1mm}
\caption{Comparison of 3D reconstruction results. Our method significantly improves the reconstruction quality.}
\vspace{-3mm}
\label{tab:neus}
\scalebox{0.7}{
\begin{tabular}{c cc}
\toprule
              &  Chamfer Dist.$\downarrow$  & Volume IoU$\uparrow$
\\ \midrule

            Zero123         & 0.017         & 0.819    \\
            SyncDreamer     & \best{0.013}         & \best{0.847}    \\
            Ours            & 0.014	& 0.842 \\

\bottomrule
\end{tabular}
}
\vspace{-5mm}
\end{table}


\vspace{-2mm}
\subsection{Downstream Application}
\vspace{-2mm}
To demonstrate the effectiveness of our method, we also applied it to the downstream 3D reconstruction task. Specifically, we trained the NeuS model~\cite{neus} directly using images synthesized by our method, Zero123, and SyncDreamer, respectively.
The quantitative results in Tab.~\ref{tab:neus} show that the consistent multi-view images synthesized by our method can significantly improve the 3D reconstruction quality.
Additionally, our method exhibits similar performance to SyncDreamer which requires time-consuming re-training.
The qualitative results in Fig.~\ref{fig:neus} show that it is challenging to train the NeuS model directly due to the lack of consistency in the images generated by Zero123. In contrast, our method generates more consistent multi-view images and, therefore, better reconstructs the geometry and texture details.
We show improvements on other downstream applications such as image-to-3D in the Supplementary Material.


\section{Limitations} 

In this work, we compared the effectiveness and interplay of SFT and RL-based methods, under fixed data constraints. In particular, we chose offline methods like DPO and KTO as the baseline implementation of the RL method because it eliminates the need for reward modeling or iterative finetuning. This means that the process of development is limited to collecting an offline dataset and fientuning it - making it the most fair comparable to SFT in terms of implementation effort, compute costs and annotation efforts. Since this baseline RL method shows optimal performance over SFT, we hope that this motivates future work to study more complex RL-based methods and their interplay with SFT. In addition, we used GPT4o annotation for synthetic data generation, and also for evaluating Summarization and Helpfulness, which could include potential biases inherited from the model. 

In addition, we limited the size of the model to under 10 Billion parameters, to keep the finetuning cost low enough to ignore as compared to the data annotation costs. In addition, it would be extremely compute resource intensive to run thousands of finetuning runs with larger model sizes like 70B parameters. We hope that future work would study the scaling trends of RL-based methods against different model sizes, and also study the compute-data trade-off in-depth.


\section{Limitations}

While our approach is typically more general than current training-based approaches, it still has limitations. One limitation arises from surprising entanglements in the CLIP and diffusion feature spaces. For example, when attempting to combine a zebra's body with a leopard fur pattern (\cref{fig:limitations} (top)), the diffusion model tends to produce animals with the head of a giraffe, even though no giraffe appears in either input image. We hypothesize that this may be related to the tendency of diffusion models to represent some concepts as a composition of more basic visual components~\citep{chefer2023hidden}, but leave further investigation to future work.

On the other hand, some concepts may be \textit{more} disentangled in CLIP-space than intuitively expected. For example, outfit types and colors are disentangled in CLIP-space, hence, an ``outfit'' subspace spanned with descriptions of different types of outfits (``dress'', ``tuxedo''...) will not preserve outfit colors (\cref{fig:limitations} (bottom)). However, this can be easily amended by also specifying colors in the spanning texts (``\textit{red} dress'', ``\textit{blue} tuxedo''...).



Finally, we note that IP-Adapter itself is limited in the level of detail captured from the input image. Hence, our approach will not be sufficient for capturing delicate details such as exact identities. Stronger encoders may achieve higher fidelity, but it is not clear that our embedding-space projections would generalize to more complex feature spaces.

\section{Conclusions}

We presented IP-Composer, a training-free method that allows a user to compose novel images from visual concepts derived through a set of input images. To do so, our approach uses a CLIP-based IP-Adapter, leveraging their joint disentangled subspace structure. Through this approach, we achieve comparable or better performance compared with existing training-based methods, and can more easily generalize to novel concepts derived solely from textual descriptions. 

We hope that our work can serve as an additional component of the creative toolbox, and open the way to additional composable-concept discovery methods. 

\section{Acknowledgment}
We would like to thank Ron Mokady and Yoad Tewel for providing feedback and helpful suggestions.



\newpage
\section*{Limitations}
While our proposed benchmark provides a comprehensive assessment of intermodal reasoning in current MLLMs, several limitations must be acknowledged. Firstly, the benchmark is currently limited to analyzing coherence relations in single-turn discourses. This is due to a lack of publically available datasets that provide multi-turn image-text pairs with annotated coherence relations. We plan to extend our benchmark to include multi-turn discourse relations as future work. Secondly, although we analyze different discourse domains in our benchmark, we lack a unified set of coherence relations that can be applied across all domains. The difficulty in defining a universal set of coherence relations is due to the varying nature of discourse in different domains. This limits our ability to analyze the inter-domain performance of MLLMs on the same set of relations. Finally, our benchmark is currently limited to the English language and must be extended to multi-lingual discourses as well.

\section*{Acknowledgments}
This research was in part supported by the U.S. National Science Foundation (NSF) award \#1820609. Part of the research results were obtained using the computational resources provided by CloudBank (\url{https://www.cloudbank.org/}), which was supported by the NSF award \#1925001.

\bibliography{acl_latex}

\clearpage
\appendix
\appendix
\newpage



\section{Baseline methods.}

We compare our approach to 18 baseline methods from IGMBT\footnote{https://github.com/kinit-sk/IMGTB} with its default setting \citep{Spiegel.2023}, which are categorized into metric-based and pretrained model-based methods. The metric-based methods include Binoculars ~\cite{hans2401spotting}, DetectLLM-LLR ~\cite{su2023detectllm}, DNAGPT ~\cite{yang2023dna}, Entropy ~\cite{gehrmann2019gltr}, FastDetectGPT ~\cite{bao2023fast}, GLTR ~\cite{gehrmann2019gltr}, LLMDeviation ~\cite{wu2023mfd}, Loglikelihood ~\cite{solaiman2019release}, LogRank ~\cite{Mitchell.2023}, MFD ~\cite{wu2023mfd}, Rank ~\cite{gehrmann2019gltr}, and S5 ~\cite{Spiegel.2023}. The model-based methods include NTNU-D ~\cite{sivesind2023turning}, ChatGPT-D ~\cite{guo2023close}, OpenAI-D ~\cite{solaiman2019release}, OpenAI-D-lrg ~\cite{solaiman2019release}, RADAR-D ~\cite{solaiman2019release}, and MAGE-D ~\cite{li2024mage}. 

\label{app:baselines}
\subsection{Metric based methods}
\subsubsection{Binoculars} 
Binoculars ~\cite{hans2401spotting}, analyzes text through two perspectives. First, it calculates the log perplexity of the text using an observer LLM. Then, a performer LLM generates next-token predictions, whose perplexity is evaluated by the observer—this metric is termed cross-perplexity. The ratio of perplexity to cross-perplexity serves as a strong indicator for detecting LLM-generated text.

\subsubsection{DNAGPT}
DNAGPT ~\cite{yang2023dna} is a training-free detection method designed to identify machine-generated text. Unlike conventional approaches that rely on training models, DNAGPT uses Divergent N-Gram Analysis (DNA) to detect discrepancies in text origin. The method works by truncating a given text at the midpoint and using the preceding portion as input to an LLM to regenerate the missing section. By comparing the regenerated text with the original through N-gram analysis (black-box) or probability divergence (white-box), DNAGPT reveals distributional differences between human and machine-written text, offering a flexible and explainable detection strategy.

\subsubsection{Entropy} Similar to the Rank score, the Entropy score for a text is determined by averaging the entropy values of each word, conditioned on its preceding context ~\cite{gehrmann2019gltr}.

\subsubsection{GLTR} The Entropy score, like the Rank score, is computed by averaging the entropy values of each word within a text, considering the preceding context ~\cite{gehrmann2019gltr}.

\subsubsection{MFD}
The Multi-level Fine-grained Detection (MFD) ~\cite{wu2023mfd} framework enhances text detection by combining statistical, semantic, and linguistic features at the sentence level. It first extracts low-level statistical features like readability and author style to quantify sentence structure. Simultaneously, high-level semantic differences are captured using an encoder with contrastive learning to distinguish LLM-generated text from human-written content. Additionally, advanced LLMs analyze the full text, extracting deep linguistic features related to lexicon, grammar, and syntax for more precise detection.

\subsubsection{Loglikelihood} This method utilizes a language model to compute the token-wise log probability. Specifically, given a text, the log probability of each token is averaged to produce a final score. A higher score indicates a greater likelihood that the text is machine-generated ~\cite{solaiman2019release}.

\subsubsection{LogRank} Unlike the Rank metric, which relies on absolute rank values, the Log-Rank score is derived by applying a logarithmic function to the rank value of each word ~\cite{Mitchell.2023}.

\subsubsection{Rank} The Rank score is calculated by determining the absolute rank of each word in a text based on its preceding context. The final score is obtained by averaging the rank values across the text. A lower score suggests a higher probability that the text was machine-generated ~\cite{gehrmann2019gltr}.

\subsubsection{DetectLLM-LLR} This approach integrates Log-Likelihood and Log-Rank scores, leveraging their complementary properties to analyze a given text ~\cite{su2023detectllm}.

\subsubsection{FastDetectGPT} This method assesses changes in a model’s log probability function when small perturbations are introduced to a text. The underlying idea is that LLM-generated text often resides in a local optimum of the model’s probability function. Consequently, minor perturbations to machine-generated text typically result in lower log probabilities, whereas perturbations to human-written text may lead to either an increase or decrease in log probability ~\cite{Mitchell.2023}.

\subsection{Model-based methods}
\subsubsection{NTNU-D}
It is a fine-tuned classification model based on the RoBERTa-base model, and three sizes of the bloomz-models ~\cite{sivesind2023turning}
\subsubsection{ChatGPT-D}
The ChatGPT Detector ~\cite{guo2023close} is designed to differentiate between human-written text and content generated by ChatGPT. It is based on a RoBERTa model that has been fine-tuned for this specific task. The authors propose two training approaches: one that trains the model solely on generated responses and another that incorporates both question-answer pairs for joint training. In our evaluation, we adopt the first approach to maintain consistency with other detection methods.

\subsubsection{OpenAI-D and RADAR-D}
The OpenAI Detector ~\cite{solaiman2019release} are models fine-tuned on RoBERTa to identify outputs generated by GPT-2. Specifically, it was trained using text generated by the largest GPT-2 model (1.5B parameters) and is capable of determining whether a given text is machine-generated.


\subsubsection{MAGE-D}
MAGE (MAchine-GEnerated text detection) ~\cite{li2024mage} is a large-scale benchmark designed for detecting AI-generated text. It compiles human-written content from seven diverse writing tasks, including story generation, news writing, and scientific writing. Corresponding machine-generated texts are produced using 27 different LLMs, such as ChatGPT, LLaMA, and Bloom, across three representative prompt types.

\section{Dataset Details}
\subsection{Dataset File Structure}
The calibration, test, and extended sets are in separate directories. Each directory contains subdirectories for different models that were used to generate AI peer review samples. In each model's subdirectory, you will find multiple CSV files, with each file representing peer review samples of a specific conference. The directory and file structure are outlined below.
\\

{\fontsize{8.2pt}{10pt}
\begin{verbatim}
|-- calibration
    |-- gpt4o
        |-- (format: <conference>.<subset>.<LLM>.csv)
        |-- ICLR2017.calibration.gpt-4o.csv
        |-- ...
        |-- ICLR2024.calibration.gpt-4o.csv
        |-- NeurIPS2016.calibration.gpt-4o.csv
        |-- ...
        |-- NeurIPS2024.calibration.gpt-4o.csv
    |-- claude
        |-- ...
    |-- gemini
        |-- ...
    |-- llama
        |-- ...
    |-- qwen
        |-- ...
|-- extended
    |-- gpt4o
        |-- ICLR2018.extended.gpt-4o.csv
        |-- ...
        |-- ICLR2024.extended.gpt-4o.csv
        |-- NeurIPS2016.extended.gpt-4o.csv
        |-- ...
        |-- NeurIPS2024.extended.gpt-4o.csv
    |-- llama
        |-- ...
|-- test
    |-- gpt4o
        |-- ICLR2017.test.gpt-4o.csv
        |-- ...
        |-- ICLR2024.test.gpt-4o.csv
        |-- NeurIPS2016.test.gpt-4o.csv
        |-- ...
        |-- NeurIPS2024.test.gpt-4o.csv
    |-- claude
        |-- ...
    |-- gemini
        |-- ...
    |-- llama
        |-- ...
    |-- qwen
        |-- ...
\end{verbatim}
}

\subsection{CSV File Content}
CSV files may differ in their column structures across conferences and years. These differences are due to updates in the required review fields over time as well as variations between conferences. See Table \ref{tab:review_template_fields} for review fields of individual conferences.


\begin{table}%
\centering
\resizebox{1\columnwidth}{!}{
\begin{tabularx}{.51\textwidth}{l X}
\toprule
Conference & Required Fields \\
\midrule
ICLR2017 & review, rating, confidence \\
ICLR2018 & review, rating, confidence \\
ICLR2019 & review, rating, confidence \\
ICLR2020 & review, rating, confidence, experience assessment, checking correctness of derivations and theory, checking correctness of experiments, thoroughness in paper reading \\
ICLR2021 & review, rating, confidence \\
ICLR2022 & summary of the paper, main review, summary of the review, correctness, technical novelty and significance, empirical novelty and significance, flag for ethics review, recommendation, confidence \\
ICLR2023 & summary of the paper, strength and weaknesses, clarity quality novelty and reproducibility, summary of the review, rating, confidence \\
ICLR2024 & summary, strengths, weaknesses, questions, soundness, presentation, contribution, flag for ethics review, rating, confidence \\
NeurIPS2016 & review, rating, confidence \\
NeurIPS2017 & review, rating, confidence \\
NeurIPS2018 & review, overall score, confidence score \\
NeurIPS2019 & review, overall score, confidence score, contribution \\
NeurIPS2021 & summary, main review, limitations and societal impact, rating, confidence, needs ethics review, ethics review area \\
NeurIPS2022 & summary, strengths and weaknesses, questions, limitations, ethics flag, ethics review area, rating, confidence, soundness, presentation, contribution \\
NeurIPS2023 & summary, strengths, weaknesses, questions, limitations, ethics flag, ethics review area, rating, confidence, soundness, presentation, contribution \\
NeurIPS2024 & summary, strengths, weaknesses, questions, limitations, ethics flag, ethics review area, rating, confidence, soundness, presentation, contribution \\
\bottomrule
\end{tabularx}
}
\caption{Required fields in the review templates for each conference.}
\label{tab:review_template_fields}
\end{table}

\FloatBarrier
\subsection{Dataset Sample Numbers per Conference Year}
\label{sec:sample_breakdown}
In this section, we present further breakdowns of sample numbers by conference, year, and LLM, as shown in Table~\ref{tab:dataset-statistics}.
 
\begin{table}[h!]
\centering
\resizebox{.59\columnwidth}{!}{
\begin{tabular}{lrr}
\toprule
Conference & gpt4o & llama \\
\midrule
ICLR2017 & 2926 & 2918 \\
ICLR2018 & 5460 & 5434 \\
ICLR2019 & 9414 & 9378 \\
ICLR2020 & 15426 & 15366 \\
ICLR2021 & 18786 & 18768 \\
ICLR2022 & 20042 & 20026 \\
ICLR2023 & 28562 & 28560 \\
ICLR2024 & 55714 & 55672 \\
NeurIPS2016 & 6296 & 6284 \\
NeurIPS2017 & 3848 & 3774 \\
NeurIPS2018 & 5990 & 5938 \\
NeurIPS2019 & 8444 & 8398 \\
NeurIPS2021 & 21170 & 21164 \\
NeurIPS2022 & 20472 & 20408 \\
NeurIPS2023 & 30264 & 30194 \\
NeurIPS2024 & 33206 & 33104 \\
\bottomrule
\end{tabular}

}
\caption{Entire set sample size, including both human and AI reviews. They are exactly balanced.}
\label{tab:sample_numbers_entire}
\end{table}

\begin{table}[h!]
\centering
\resizebox{1\columnwidth}{!}
{
\begin{tabular}{lrrrrr}
\toprule
Conference & gemini & claude & qwen & gpt4o & llama \\
\midrule
ICLR2017 & 2924 & 2926 & 2918 & 2926 & 2918 \\
ICLR2018 & 3000 & 3004 & 2988 & 3004 & 2992 \\
ICLR2019 & 3002 & 3010 & 3000 & 3010 & 2998 \\
ICLR2020 & 3016 & 3022 & 3000 & 3022 & 3010 \\
ICLR2021 & 3840 & 3842 & 3830 & 3842 & 3838 \\
ICLR2022 & 3896 & 3900 & 3838 & 3900 & 3898 \\
ICLR2023 & 3816 & 3816 & 3816 & 3816 & 3814 \\
ICLR2024 & 3784 & 3820 & 3800 & 3822 & 3816 \\
NeurIPS2016 & 5522 & 5534 & 5534 & 5536 & 5526 \\
NeurIPS2017 & 2854 & 2858 & 2850 & 2858 & 2812 \\
NeurIPS2018 & 3000 & 3006 & 2916 & 3006 & 2982 \\
NeurIPS2019 & 2930 & 2938 & 2922 & 2940 & 2928 \\
NeurIPS2021 & 3884 & 3884 & 3884 & 3884 & 3884 \\
NeurIPS2022 & 3606 & 3622 & 3598 & 3622 & 3610 \\
NeurIPS2023 & 4436 & 4440 & 4382 & 4440 & 4432 \\
NeurIPS2024 & 3914 & 3926 & 3880 & 3926 & 3912 \\
\bottomrule
\end{tabular}

}
\caption{Test set sample size, including both human and AI reviews. They are exactly balanced.}
\label{tab:sample_numbers_test}
\end{table}

\begin{table}[h!]
\centering
\resizebox{1\columnwidth}{!}{
\begin{tabular}{lrrrrr}
\toprule
Conference & gemini & claude & qwen & gpt4o & llama \\
\midrule
ICLR2021 & 3826 & 3828 & 3802 & 3828 & 3828 \\
ICLR2022 & 3878 & 3844 & 3860 & 3882 & 3880 \\
NeurIPS2021 & 3828 & 3830 & 3818 & 3830 & 3828 \\
NeurIPS2022 & 3648 & 3654 & 3634 & 3654 & 3644 \\
\bottomrule
\end{tabular}

}
\caption{Calibration set sample size, including both human and AI reviews. They are exactly balanced.}
\label{tab:sample_numbers_calibration}
\end{table}

\FloatBarrier

\section{Additional results}

\subsection{Calibration using ICLR + NeurIPS reviews}
\label{app:main-result-iclr-plus-neurips-calibration}

Our main results in Table~\ref{tab:main-results} of Section~\ref{sec:experiments-main-result} utilized ICLR review from our calibration set to calibrate each detection method. This simulates the scenario in which some of the reviews in the test set are "out-of-domain" in the sense that they belong to a different conference than the reviews used for calibration. In Table~\ref{tab:main-results-iclr-plus-neurips-calibration}, we provide additional results for the same evaluation setting as before, but using both ICLR and NeurIPS reviews for calibration (i.e., fully ``in domain''). We generally see similar trends regarding relative performance between methods as before, with the exception that the Binoculars method achieves slightly higher detection rates than our Anchor method for Gemini reviews. This suggests that existing methods such as Binoculars may be more sensitive to the use of in-domain data during calibration. 

\begin{table}[h!]
\begin{center}
\resizebox{1\columnwidth}{!}
{
\begin{tabular}{p{0.1cm}lcccccc}
\toprule
& Target FPR: & \multicolumn{2}{c}{0.1\%} & \multicolumn{2}{c}{0.5\%} & \multicolumn{2}{c}{1\%}\\
\cmidrule(lr){3-4}
\cmidrule(lr){5-6}
\cmidrule(lr){7-8}
& & FPR & TPR & FPR & TPR & FPR & TPR \\
\midrule
\multirow{8}{*}{\rotatebox[origin=c]{90}{GPT-4o Reviews}} 
& Anchor & 0.1 & \textbf{61.4} & 0.3 & \textbf{80.1} & 0.8 & \textbf{87.4} \\
& Binoculars & 0.3 & 18.8 & 0.7 & 37.5 & 1.2 & 49.3 \\
& MAGE-D & 0.1 & 2.3 & 0.7 & 9.6 & 1.3 & 14.5  \\
& s5 & 0.3 & 0.7 & 1.0 & 8.0 & 1.6 & 16.3 \\
& MFD & 0.3 & 0.9 & 0.9 & 7.8 & 1.6 & 14.9 \\
& GLTR & 0.1 & 0.1 & 0.5 & 2.4 & 1.2 & 5.9 \\
& DetectGPT & 0.1 & 0.2 & 0.6 & 1.1 & 1.0 & 2.1 \\
& Loglikelihood & 0.1 & 0.0 & 0.3 & 0.2 & 0.6 & 1.0 \\
\midrule
\multirow{10}{*}{\rotatebox[origin=c]{90}{Gemini Reviews}} 
& Binoculars & 0.3 & \textbf{63.8} & 0.7 & \textbf{80.9} & 1.1 & \textbf{87.6} \\
& Anchor & 0.2 & 57.2 & 0.5 & 75.5 & 1.1 & 84.2 \\
& MFD & 0.1 & 1.9 & 0.6 & 11.0 & 1.0 & 18.1 \\
& s5 & 0.1 & 1.4 & 0.5 & 10.5 & 1.0 & 18.3 \\
& GLTR & 0.2 & 0.6 & 1.0 & 6.3 & 1.8 & 12.9 \\
& FastDetectGPT & 0.1 & 1.1 & 0.4 & 4.9 & 0.9 & 8.9 \\
& MAGE-D & 0.1 & 0.4 & 0.7 & 3.8 & 1.3 & 6.9 \\
& DetectGPT & 0.1 & 0.5 & 0.5 & 3.2 & 1.1 & 6.3  \\
& Loglikelihood & 0.1 & 0.0 & 0.3 & 0.2 & 0.6 & 1.6 \\
& NTNU-D & 11.5 & 0.0 & 21.8 & 0.0 & 26.3 & 0.1 \\
\midrule
\multirow{6}{*}{\rotatebox[origin=c]{90}{Claude Reviews}}
& Anchor & 0.1 & \textbf{53.8} & 0.3 & \textbf{72.6} & 0.8 & \textbf{80.0} \\
& Binoculars & 0.3 & 46.4 & 0.7 & 70.2 & 1.1 & \textbf{80.0} \\
& MFD & 0.0 & 1.0 & 0.2 & 8.7 & 0.4 & 15.9  \\
& s5 & 0.0 & 0.7 & 0.2 & 8.4 & 0.4 & 16.4 \\
& DetectGPT & 0.1 & 0.6 & 0.5 & 4.9 & 1.0 & 10.1 \\
& GLTR & 0.0 & 0.0 & 0.3 & 0.7 & 0.6 & 1.9 \\
\bottomrule
\end{tabular}
}
\caption{Actual FPR and TPR calculated from the withheld test dataset at varying detection thresholds, which are calibrated using ICLR and NeurIPS reviews from our calibration set at different target FPRs. Best TPRs are in \textbf{bold}.}
\label{tab:main-results-iclr-plus-neurips-calibration}
\end{center}
\end{table}










\subsection{Additional results for test set: Llama and Qwen Detection Results}
\label{app:llama-qwen-results}
Table~10 is organized similarly to Table \ref{tab:main-results}, but it presents results for Llama and Qwen reviews. Both tables use the same set of thresholds for each method, which were calibrated using ICLR reviews generated by GPT-4o, Gemini, and Claude along with their matching human-written reviews.

\begin{table}[h]
\centering
\resizebox{1\columnwidth}{!}
{
\begin{tabular}{p{0.1cm}lcccccc}
\toprule
& Target FPR: & \multicolumn{2}{c}{0.1\%} & \multicolumn{2}{c}{0.5\%} & \multicolumn{2}{c}{1\%}\\
\cmidrule(lr){3-4}
\cmidrule(lr){5-6}
\cmidrule(lr){7-8}
& & FPR & TPR & FPR & TPR & FPR & TPR \\
\midrule
\multirow{15}{*}{\rotatebox[origin=c]{90}{Llama Reviews}} &Binoculars & 0.2 & \textbf{98.4} & 0.6 & \textbf{99.0} & 1.0 & \textbf{99.2} \\
&MAGE-D & 0.1 & 63.8 & 0.7 & 91.8 & 1.3 & 95.7 \\
&MFD & 0.0 & 57.6 & 0.1 & 81.9 & 0.1 & 87.4 \\
&GLTR & 0.1 & 55.4 & 0.2 & 75.2 & 0.3 & 81.7 \\
&FastDetectGPT & 0.1 & 54.7 & 0.5 & 73.8 & 1.2 & 80.6 \\
&s5 & 0.0 & 53.5 & 0.1 & 83.2 & 0.1 & 88.0 \\
&OpenAI-D & 0.2 & 38.8 & 0.6 & 48.3 & 1.6 & 57.1 \\
&LLMDeviation & 0.0 & 37.0 & 0.0 & 37.0 & 0.0 & 37.0 \\
&ChatGPT-D & 0.0 & 26.4 & 0.0 & 26.4 & 0.0 & 26.4 \\
&DetectLLM-{LLR} & 0.0 & 19.3 & 0.0 & 19.3 & 0.0 & 19.3 \\
&LogRank & 0.0 & 18.2 & 0.0 & 18.2 & 0.0 & 18.2 \\
&Loglikelihood & 0.0 & 13.4 & 0.3 & 76.0 & 0.5 & 85.5 \\
&Anchor & 0.1 & 12.2 & 0.5 & 28.5 & 1.0 & 36.8 \\
&DetectGPT & 0.1 & 1.7 & 0.7 & 10.8 & 1.3 & 19.7 \\
&RADAR-D & 0.9 & 0.0 & 2.6 & 1.6 & 4.2 & 12.5 \\
\midrule
\multirow{8}{*}{\rotatebox[origin=c]{90}{Qwen Reviews}} & Binoculars & 0.2 & \textbf{99.4} & 0.6 & \textbf{99.8} & 1.0 & \textbf{99.9} \\
& Anchor & 0.1 & 67.1 & 0.5 & 83.8 & 1.0 & 88.1 \\
& FastDetectGPT & 0.1 & 54.3 & 0.5 & 77.6 & 1.1 & 85.4  \\
& MFD & 0.1 & 37.6 & 0.4 & 73.0 & 0.6 & 82.3 \\
& s5 & 0.1 & 34.7 & 0.4 & 76.0 & 0.6 & 83.9 \\
& MAGE-D  & 0.1 & 33.2 & 0.7 & 73.6 & 1.3 & 86.2  \\
& GLTR & 0.1 & 31.1 & 0.3 & 64.7 & 0.6 & 77.4 \\
& Loglikelihood  & 0.0 & 0.5 & 0.3 & 30.8 & 0.5 & 56.6 \\
\bottomrule
\end{tabular}
}
\caption{Actual FPR and TPR calculated from the withheld test dataset at varying detection thresholds, which are calibrated using ICLR reviews from our calibration set at different target FPRs. Best TPRs are in \textbf{bold}.}

\label{tab:main-results_open-source-llm}

\end{table}

\FloatBarrier
\subsection{Experimental Results on Full Dataset}
\label{app:extended-set}

We test existing AI text generation text detection models on our entire dataset (i.e., the test set + the extended set). The results are shown in Tables \ref{tab:iclr_entire_dataset} for ICLR reviews and \ref{tab:neurips_entire_set} for NeurIPS reviews.


\begin{table}[h!]
\begin{center}
\resizebox{1\columnwidth}{!}
{
\begin{tabular}{p{0.1cm}lcccccccc}
\toprule
& Target FPR: & \multicolumn{2}{c}{0.1\%} & \multicolumn{2}{c}{0.5\%} & \multicolumn{2}{c}{1\%} \\
\cmidrule(lr){3-4}
\cmidrule(lr){5-6}
\cmidrule(lr){7-8}
& & FPR & TPR & FPR & TPR & FPR & TPR \\
\midrule
\multirow{2}{*}{\rotatebox[origin=c]{90}{GPT}} 
& Binoculars& 0.8\% & 23.4\%& 1.5\% & 41.7\%& 2.2\% & 54.2\% \\ 
 & GLTR& 0.7\% & 3.3\%& 2.1\% & 12.0\%& 3.8\% & 21.0\% \\ 
\midrule
\multirow{2}{*}{\rotatebox[origin=c]{90}{Llama}} 
& Binoculars& 0.8\% & 98.9\%& 1.5\% & 99.4\%& 2.2\% & 99.6\% \\ 
 & GLTR& 0.4\% & 80.6\%& 1.0\% & 92.5\%& 1.6\% & 95.3\% \\ 
\bottomrule
\end{tabular}
}
\caption{Actual FPR and TPR calculated from the ICLR reviews at varying detection thresholds, which are calibrated using the ICLR calibration dataset at different target FPRs.}
\label{tab:iclr_entire_dataset}
\end{center}
\end{table}





\begin{table}[h!]
\begin{center}
\resizebox{1\columnwidth}{!}
{
\begin{tabular}{p{0.1cm}lcccccccc}
\toprule
& Target FPR: & \multicolumn{2}{c}{0.1\%} & \multicolumn{2}{c}{0.5\%} & \multicolumn{2}{c}{1\%} \\
\cmidrule(lr){3-4}
\cmidrule(lr){5-6}
\cmidrule(lr){7-8}
& & FPR & TPR & FPR & TPR & FPR & TPR \\
\midrule
\multirow{2}{*}{\rotatebox[origin=c]{90}{GPT}} 
&BinocularsMetric& 0.3\% & 26.1\% & 0.6\% & 45.3\% & 0.9\% & 54.1\% \\
& GLTRMetric& 0.3\% & 1.3\% & 5.1\% & 31.9\% & 10.6\% & 52.6\% \\
\midrule
\multirow{2}{*}{\rotatebox[origin=c]{90}{Llama}} 
& BinocularsMetric& 0.3\% & 98.9\% & 0.6\% & 99.3\% & 0.9\% & 99.4\% \\
& GLTRMetric& 0.3\% & 76.3\% & 0.9\% & 91.0\% & 1.4\% & 93.9\% \\
\bottomrule
\end{tabular}
}
\caption{Actual FPR and TPR calculated from the NeurIPS reviews at varying detection thresholds, which are calibrated using the ICLR calibration dataset at different target FPRs.}
\label{tab:neurips_entire_set}
\end{center}
\end{table}



\FloatBarrier
\begin{table*}[]
\centering
\resizebox{1\textwidth}{!}
{
\begin{tabular}{p{3cm} p{10cm} p{10cm}}
    \toprule
    Category & Human review example & GPT-4o review examples \\
    \midrule
    References to specific details in the paper & ``Table 2 confirms that MDR outperforms Graph Rec Retriever (Asai et al.). This result shows the feasibility of a more accurate multi-hop QA model without external knowledge such as Wikipedia hyperlinks.'' & ``The paper extensively evaluates on multiple datasets and situates the contributions clearly within existing literature, substantiating claims with thorough quantitative analysis.'' \\
    \midrule
    Specific references to prior work & ``My only serious concern is the degree of novelty with respect to (Yuan et al., 2020), which was published at ECCV 2020. The main difference seems to be that in the proposed method the graph is dynamic (i.e., it depends on the input sentences), instead in (Yuan et al., 2018) the graph is learned but fixed for all the input samples.'' & ``The novelty of the TDM is not strong enough relative to prior work.'' \\
    \midrule
    Generic criticisms & N/A & ``Lack of clarity'' (without pointing to specific statements in the paper which need clarification); ``lack of discussion of limitations or computational considerations''; ``need more discussion of hyperparameter sensitivity''; ``need comparisons to more datasets'' (without suggesting any in particular); ``technical language used in the paper may be difficult to follow for unfamiliar readers''\\
    \bottomrule
\end{tabular}
}
\caption{Examples of differences identified in human analysis of human and AI-written peer reviews}
\label{tab:human_analysis_examples}
\end{table*}

\section{Additional analyses}

\subsection{Examples from human analysis of differences between human and AI-written peer reviews}
\label{app:human_analysis}

Table~\ref{tab:human_analysis_examples} provides examples of the issues identified in our qualitative analysis of human and AI-written peer reviews. In general, we observe that GPT-4o reviews lack references to specific details in the paper, lack references to specific prior work, and contain overly generic criticisms. See Section~\ref{sec:human-analysis} for additional discussion.



\subsection{Comparison of numeric scores assigned by human and AI reviewers}
\label{app:numeric_scores}
While Section~\ref{sec:analysis_numeric_score} focused on the misalignment between human and AI peer reviews from three commercial LLMs (GPT-4o, Gemini, and Claude) from the NeurIPS2022 review samples, this section presents the corresponding results for two open-source LLMs (Llama and Qwen), as shown in Figure~\ref{fig:numeric_scores}. The main findings from GPT-4o, Gemini, and Claude also hold for these two open-source models, with one notable difference: Llama and Qwen exhibit an even larger divergence in Presentation scores than Claude, the most overly-positive one amongst the commercial LLMs across all categories. In terms of Contribution scores, the evaluations from Llama and Qwen were similar to those of Claude.

In addition, we examine data from three other conferences (NeurIPS2023, NeurIPS2024, and ICLR2024). Although the results from these conferences are slightly less reliable---given that human reviews may have been influenced by AI use following the release of ChatGPT---the overall trend persists: LLMs tend to inflate the quality of papers compared to human reviewers.


\begin{figure*}
    \centering
    \underline{NeurIPS2022}
    \includegraphics[width=1\linewidth]{figures/figs_nuemric_scores/NeurIPS2022.ai_vs_human.png}
    \vspace{.01em}\\
    \underline{NeurIPS2023}
    \includegraphics[width=1\linewidth]{figures/figs_nuemric_scores/NeurIPS2023.ai_vs_human.png}
    \vspace{.01em}\\
    \underline{NeurIPS2024}
    \includegraphics[width=1\linewidth]{figures/figs_nuemric_scores/NeurIPS2024.ai_vs_human.png}
    \vspace{.01em}\\
    \underline{ICLR2024}
    \includegraphics[width=1\linewidth]{figures/figs_nuemric_scores/ICLR2024.ai_vs_human.png}
    \caption{Difference between AI and human scores. For each matched review (aligned by paper ID and recommendation), score differences were computed and displayed as histograms. Scores range from 1 to 4 for all metrics except Confidence, which ranges from 1 to 5. Statistical significance was assessed using a two-sided Wilcoxon signed‐rank test, with p-values shown in the legend. This figure includes only NeurIPS2022--2024 and ICLR2024, because they are the onyl conferences that required reviewers to submit these scores in their review templates.}
    \label{fig:numeric_scores}
\end{figure*}


\section{Prompts}
\label{sec:prompts}
This section includes the prompts we used to generate AI peer review texts. Due to space limitations, we provide only the ICLR2022 review guideline and review template here. Those for other years and other conferences (e.g., NeurIPS) are available on the respective conference official websites\footnote{https://icml.cc/Conferences/\{2016..2024\} \\ and https://neurips.cc/Conferences/\{2016..2024\} }.
\subsection{Prompts for Generating Reviews}
\noindent\underline{System prompt}:

\begin{Verbatim}[breaklines, breaksymbolleft={}, fontsize=\small]
You are an AI researcher reviewing a paper submitted to a prestigious AI research conference. 
You will be provided with the manuscript text, the conference's reviewer guidelines, and the decision for the paper.
Your objective is to thoroughly evaluate the paper, adhering to the provided guidelines, and return a detailed assessment that supports the given decision using the specified response template.
Ensure your evaluation is objective, comprehensive, and aligned with the conference standards.

{reviewer_guideline}

{review_template}
\end{Verbatim}

\noindent\underline{User prompt}:
\begin{Verbatim}[breaklines, breaksymbolleft={}, fontsize=\small]
Here is the paper you are asked to review. Write a well-justified review of this paper that aligns with a '{human_reviewer_decision}' decision.

```
{text}
```
\end{Verbatim}

\noindent\underline{ICLR2022 Reviewer Guideline}\\
\underline{(\texttt{\{reviewer\_guideline\}} in the system prompt)}:
\begin{Verbatim}[breaklines, breaksymbolleft={}, fontsize=\small]
## Reviewer Guidelines

1. Read the paper: It’s important to carefully read through the entire paper, and to look up any related work and citations that will help you comprehensively evaluate it. Be sure to give yourself sufficient time for this step.

2. While reading, consider the following:
    - Objective of the work: What is the goal of the paper? Is it to better address a known application or problem, draw attention to a new application or problem, or to introduce and/or explain a new theoretical finding? A combination of these? Different objectives will require different considerations as to potential value and impact.
    - Strong points: is the submission clear, technically correct, experimentally rigorous, reproducible, does it present novel findings (e.g. theoretically, algorithmically, etc.)?
    - Weak points: is it weak in any of the aspects listed in b.?
    - Be mindful of potential biases and try to be open-minded about the value and interest a paper can hold for the entire ICLR community, even if it may not be very interesting for you.

3. Answer three key questions for yourself, to make a recommendation to Accept or Reject:
    - What is the specific question and/or problem tackled by the paper?
    - Is the approach well motivated, including being well-placed in the literature?
    - Does the paper support the claims? This includes determining if results, whether theoretical or empirical, are correct and if they are scientifically rigorous.

4. Write your initial review, organizing it as follows: 
    - Summarize what the paper claims to contribute. Be positive and generous.
    - List strong and weak points of the paper. Be as comprehensive as possible.
    - Clearly state your recommendation (accept or reject) with one or two key reasons for this choice.
    - Provide supporting arguments for your recommendation.
    - Ask questions you would like answered by the authors to help you clarify your understanding of the paper and provide the additional evidence you need to be confident in your assessment. 
    - Provide additional feedback with the aim to improve the paper. Make it clear that these points are here to help, and not necessarily part of your decision assessment.

5. General points to consider:
    - Be polite in your review. Ask yourself whether you’d be happy to receive a review like the one you wrote.
    - Be precise and concrete. For example, include references to back up any claims, especially claims about novelty and prior work
    - Provide constructive feedback.
    - It’s also fine to explicitly state where you are uncertain and what you don’t quite understand. The authors may be able to resolve this in their response.
    - Don’t reject a paper just because you don’t find it “interesting”. This should not be a criterion at all for accepting/rejecting a paper. The research community is so big that somebody will find some value in the paper (maybe even a few years down the road), even if you don’t see it right now.
\end{Verbatim}

\noindent\underline{ICLR2022 Review Template}\\ 
\underline{(\texttt{\{reviewer\_template\}} in the system prompt)}:
\begin{Verbatim}[breaklines, breaksymbolleft={}, fontsize=\small]
## Response template (JSON format)

Provide the review in valid JSON format with the following fields. Ensure all fields are completed as described below. The response must be a valid JSON object.

- "summary_of_the_paper": Briefly summarize the paper and its contributions. This is not the place to critique the paper; the authors should generally agree with a well-written summary. You may use paragraphs and bulleted lists for formatting, but ensure that the content remains a single, continuous text block. Do not use nested JSON or include additional fields.

- "main_review": "Provide review comments as a single text field (a string). Consider including assessment on the following dimensions: a comprehensive list of strong and weak points of the paper, your recommendation, supporting arguments for your recommendation, questions to clarify your understanding of the paper or request additional evidence, and additional feedback with the aim to improve the paper. You may use paragraphs and bulleted lists for formatting, but ensure that the content remains a single, continuous text block. Do not use nested JSON or include additional fields."

- "summary_of_the_review": Concise summary of 'main_review'. You may use paragraphs and bulleted lists for formatting, but ensure that the content remains a single, continuous text block. Do not use nested JSON or include additional fields.

- "correctness": A numerical rating on the following scale to indicate that the claims and methods are correct. The value should be between 1 and 4, where:
    - 1 = The main claims of the paper are incorrect or not at all supported by theory or empirical results.
    - 2 = Several of the paper’s claims are incorrect or not well-supported.
    - 3 = Some of the paper’s claims have minor issues. A few statements are not well-supported, or require small changes to be made correct.
    - 4 = All of the claims and statements are well-supported and correct.

- "technical_novelty_and_significance": A numerical rating on the following scale to indicate technical novelty and significance. The value should be between 1 and 4, where:
    - 1 = The contributions are neither significant nor novel.
    - 2 = The contributions are only marginally significant or novel.
    - 3 = The contributions are significant and somewhat new. Aspects of the contributions exist in prior work.
    - 4 = The contributions are significant and do not exist in prior works.
       
- "empirical_novelty_and_significance": A numerical rating on the following scale to indicate empirical novelty and significance. The value should be between 1 and 4, or -999 if not applicable, where:
    - 1 = The contributions are neither significant nor novel.
    - 2 = The contributions are only marginally significant or novel.
    - 3 = The contributions are significant and somewhat new. Aspects of the contributions exist in prior work.
    - 4 = The contributions are significant and do not exist in prior works.
    - -999 = Not applicable.

- "flag_for_ethics_review": A boolean value (`true` or `false`) indicating whether there are ethical concerns in the work.

- "recommendation": A string indicating the final decision, which must strictly be one of the following options: 'strong reject', 'reject, not good enough', 'marginally below the acceptance threshold', 'marginally above the acceptance threshold', 'accept, good paper', or 'strong accept, should be highlighted at the conference'.

- "confidence": A nuemrical values to indicate how confident you are in your evaluation. The value should be between 1 and 5, where:
    - 1 = You are unable to assess this paper and have alerted the ACs to seek an opinion from different reviewers.
    - 2 = You are willing to defend your assessment, but it is quite likely that you did not understand the central parts of the submission or that you are unfamiliar with some pieces of related work. Math/other details were not carefully checked.
    - 3 = You are fairly confident in your assessment. It is possible that you did not understand some parts of the submission or that you are unfamiliar with some pieces of related work. Math/other details were not carefully checked.
    - 4 = You are confident in your assessment, but not absolutely certain. It is unlikely, but not impossible, that you did not understand some parts of the submission or that you are unfamiliar with some pieces of related work.
    - 5 = You are absolutely certain about your assessment. You are very familiar with the related work and checked the math/other details carefully.
\end{Verbatim}


\subsection{Anchor Review Generation Prompt}
\label{sec:prompt_anchor}
\noindent\underline{System prompt}:
\begin{Verbatim}[breaklines, breaksymbolleft={}, fontsize=\small]
You are an AI research scientist tasked with reviewing paper submissions for a top AI research conference. Carefully read the provided paper, then write a detailed review following a common AI conference review format (e.g., including summary, strengths and weakness, limitations, questions, suggestions for improvement). Make sure to include recommendation for the paper, either 'Accept' or 'Reject'. Your review should be fair and objective.
\end{Verbatim}

\noindent\underline{User prompt}:
\begin{Verbatim}[breaklines, breaksymbolleft={}, fontsize=\small]
Here is the paper you are asked to review:
```
{text}
```
\end{Verbatim}

\subsection{Editing Prompts}
\label{sec:prompt_editing}
\label{sec:editing_prompt}
\noindent\underline{Minimal Editing}:
\begin{Verbatim}[breaklines, breaksymbolleft={}, fontsize=\small]
Please proofread my review for typos and grammatical errors without altering the content. Keep the original wordings as much as you can, except for typo or grammatical  error.
\end{Verbatim}
\noindent\underline{Moderate Editing}:
\begin{Verbatim}[breaklines, breaksymbolleft={}, fontsize=\small]
Please polish my review to improve sentence structure and readability while keeping the original intent clear.
\end{Verbatim}
\noindent\underline{Extensive Editing}:
\begin{Verbatim}[breaklines, breaksymbolleft={}, fontsize=\small]
Please rewrite my review into a polished, professional piece that effectively communicates its main points.
\end{Verbatim}
\noindent\underline{Maximum Editing}:
\begin{Verbatim}[breaklines, breaksymbolleft={}, fontsize=\small]
Please transform my review into a high quality piece, using professional language and a polished tone. Please also extend my review with additional details from the oringial paper.
\end{Verbatim}

\section{Artifact Use Consistent With Intended Use}
In our work, we ensured that the external resources we utilized were applied in a manner that aligns with their intended purposes. We used several LLMs (including GPT-4o, Gemini, Claude, Qwen, and Llama) as well as an open-source package, IMGTB, with a focus on advancing research in a non-commercial, open-source context. The artifacts from our work will be non-commercial, for-research, and open-sourced.

\section{Use of AI Tool}
The authors of this paper used Github Co-pilot for coding assistance for analysis. 


\end{document}
