% This must be in the first 5 lines to tell arXiv to use pdfLaTeX, which is strongly recommended.
\pdfoutput=1
% In particular, the hyperref package requires pdfLaTeX in order to break URLs across lines.

\documentclass[11pt]{article}

% Change "review" to "final" to generate the final (sometimes called camera-ready) version.
% Change to "preprint" to generate a non-anonymous version with page numbers.
\usepackage[preprint]{acl}

% Standard package includes
\usepackage{times}
\usepackage{latexsym}

% For proper rendering and hyphenation of words containing Latin characters (including in bib files)
\usepackage[T1]{fontenc}
% For Vietnamese characters
% \usepackage[T5]{fontenc}
% See https://www.latex-project.org/help/documentation/encguide.pdf for other character sets

% This assumes your files are encoded as UTF8
\usepackage[utf8]{inputenc}

% This is not strictly necessary, and may be commented out,
% but it will improve the layout of the manuscript,
% and will typically save some space.
\usepackage{microtype}

% This is also not strictly necessary, and may be commented out.
% However, it will improve the aesthetics of text in
% the typewriter font.
\usepackage{inconsolata}

%Including images in your LaTeX document requires adding
%additional package(s)
\usepackage{graphicx}
\usepackage{array}
\usepackage{tabularx}
\usepackage{float}
\usepackage{multicol, multirow}
\usepackage{booktabs}
\usepackage{amsmath}
\usepackage{tcolorbox}
\usepackage{enumitem}
\usepackage{bookmark}
\usepackage{adjustbox}
\usepackage{makecell}
\usepackage{xurl}
\usepackage{colortbl}  % For coloring
\usepackage{tikz}      % For drawing the oval
\usepackage{enumitem}
\usepackage{soul}

\usetikzlibrary{shapes.multipart, positioning}
\raggedbottom


%
\setlength\unitlength{1mm}
\newcommand{\twodots}{\mathinner {\ldotp \ldotp}}
% bb font symbols
\newcommand{\Rho}{\mathrm{P}}
\newcommand{\Tau}{\mathrm{T}}

\newfont{\bbb}{msbm10 scaled 700}
\newcommand{\CCC}{\mbox{\bbb C}}

\newfont{\bb}{msbm10 scaled 1100}
\newcommand{\CC}{\mbox{\bb C}}
\newcommand{\PP}{\mbox{\bb P}}
\newcommand{\RR}{\mbox{\bb R}}
\newcommand{\QQ}{\mbox{\bb Q}}
\newcommand{\ZZ}{\mbox{\bb Z}}
\newcommand{\FF}{\mbox{\bb F}}
\newcommand{\GG}{\mbox{\bb G}}
\newcommand{\EE}{\mbox{\bb E}}
\newcommand{\NN}{\mbox{\bb N}}
\newcommand{\KK}{\mbox{\bb K}}
\newcommand{\HH}{\mbox{\bb H}}
\newcommand{\SSS}{\mbox{\bb S}}
\newcommand{\UU}{\mbox{\bb U}}
\newcommand{\VV}{\mbox{\bb V}}


\newcommand{\yy}{\mathbbm{y}}
\newcommand{\xx}{\mathbbm{x}}
\newcommand{\zz}{\mathbbm{z}}
\newcommand{\sss}{\mathbbm{s}}
\newcommand{\rr}{\mathbbm{r}}
\newcommand{\pp}{\mathbbm{p}}
\newcommand{\qq}{\mathbbm{q}}
\newcommand{\ww}{\mathbbm{w}}
\newcommand{\hh}{\mathbbm{h}}
\newcommand{\vvv}{\mathbbm{v}}

% Vectors

\newcommand{\av}{{\bf a}}
\newcommand{\bv}{{\bf b}}
\newcommand{\cv}{{\bf c}}
\newcommand{\dv}{{\bf d}}
\newcommand{\ev}{{\bf e}}
\newcommand{\fv}{{\bf f}}
\newcommand{\gv}{{\bf g}}
\newcommand{\hv}{{\bf h}}
\newcommand{\iv}{{\bf i}}
\newcommand{\jv}{{\bf j}}
\newcommand{\kv}{{\bf k}}
\newcommand{\lv}{{\bf l}}
\newcommand{\mv}{{\bf m}}
\newcommand{\nv}{{\bf n}}
\newcommand{\ov}{{\bf o}}
\newcommand{\pv}{{\bf p}}
\newcommand{\qv}{{\bf q}}
\newcommand{\rv}{{\bf r}}
\newcommand{\sv}{{\bf s}}
\newcommand{\tv}{{\bf t}}
\newcommand{\uv}{{\bf u}}
\newcommand{\wv}{{\bf w}}
\newcommand{\vv}{{\bf v}}
\newcommand{\xv}{{\bf x}}
\newcommand{\yv}{{\bf y}}
\newcommand{\zv}{{\bf z}}
\newcommand{\zerov}{{\bf 0}}
\newcommand{\onev}{{\bf 1}}

% Matrices

\newcommand{\Am}{{\bf A}}
\newcommand{\Bm}{{\bf B}}
\newcommand{\Cm}{{\bf C}}
\newcommand{\Dm}{{\bf D}}
\newcommand{\Em}{{\bf E}}
\newcommand{\Fm}{{\bf F}}
\newcommand{\Gm}{{\bf G}}
\newcommand{\Hm}{{\bf H}}
\newcommand{\Id}{{\bf I}}
\newcommand{\Jm}{{\bf J}}
\newcommand{\Km}{{\bf K}}
\newcommand{\Lm}{{\bf L}}
\newcommand{\Mm}{{\bf M}}
\newcommand{\Nm}{{\bf N}}
\newcommand{\Om}{{\bf O}}
\newcommand{\Pm}{{\bf P}}
\newcommand{\Qm}{{\bf Q}}
\newcommand{\Rm}{{\bf R}}
\newcommand{\Sm}{{\bf S}}
\newcommand{\Tm}{{\bf T}}
\newcommand{\Um}{{\bf U}}
\newcommand{\Wm}{{\bf W}}
\newcommand{\Vm}{{\bf V}}
\newcommand{\Xm}{{\bf X}}
\newcommand{\Ym}{{\bf Y}}
\newcommand{\Zm}{{\bf Z}}

% Calligraphic

\newcommand{\Ac}{{\cal A}}
\newcommand{\Bc}{{\cal B}}
\newcommand{\Cc}{{\cal C}}
\newcommand{\Dc}{{\cal D}}
\newcommand{\Ec}{{\cal E}}
\newcommand{\Fc}{{\cal F}}
\newcommand{\Gc}{{\cal G}}
\newcommand{\Hc}{{\cal H}}
\newcommand{\Ic}{{\cal I}}
\newcommand{\Jc}{{\cal J}}
\newcommand{\Kc}{{\cal K}}
\newcommand{\Lc}{{\cal L}}
\newcommand{\Mc}{{\cal M}}
\newcommand{\Nc}{{\cal N}}
\newcommand{\nc}{{\cal n}}
\newcommand{\Oc}{{\cal O}}
\newcommand{\Pc}{{\cal P}}
\newcommand{\Qc}{{\cal Q}}
\newcommand{\Rc}{{\cal R}}
\newcommand{\Sc}{{\cal S}}
\newcommand{\Tc}{{\cal T}}
\newcommand{\Uc}{{\cal U}}
\newcommand{\Wc}{{\cal W}}
\newcommand{\Vc}{{\cal V}}
\newcommand{\Xc}{{\cal X}}
\newcommand{\Yc}{{\cal Y}}
\newcommand{\Zc}{{\cal Z}}

% Bold greek letters

\newcommand{\alphav}{\hbox{\boldmath$\alpha$}}
\newcommand{\betav}{\hbox{\boldmath$\beta$}}
\newcommand{\gammav}{\hbox{\boldmath$\gamma$}}
\newcommand{\deltav}{\hbox{\boldmath$\delta$}}
\newcommand{\etav}{\hbox{\boldmath$\eta$}}
\newcommand{\lambdav}{\hbox{\boldmath$\lambda$}}
\newcommand{\epsilonv}{\hbox{\boldmath$\epsilon$}}
\newcommand{\nuv}{\hbox{\boldmath$\nu$}}
\newcommand{\muv}{\hbox{\boldmath$\mu$}}
\newcommand{\zetav}{\hbox{\boldmath$\zeta$}}
\newcommand{\phiv}{\hbox{\boldmath$\phi$}}
\newcommand{\psiv}{\hbox{\boldmath$\psi$}}
\newcommand{\thetav}{\hbox{\boldmath$\theta$}}
\newcommand{\tauv}{\hbox{\boldmath$\tau$}}
\newcommand{\omegav}{\hbox{\boldmath$\omega$}}
\newcommand{\xiv}{\hbox{\boldmath$\xi$}}
\newcommand{\sigmav}{\hbox{\boldmath$\sigma$}}
\newcommand{\piv}{\hbox{\boldmath$\pi$}}
\newcommand{\rhov}{\hbox{\boldmath$\rho$}}
\newcommand{\upsilonv}{\hbox{\boldmath$\upsilon$}}

\newcommand{\Gammam}{\hbox{\boldmath$\Gamma$}}
\newcommand{\Lambdam}{\hbox{\boldmath$\Lambda$}}
\newcommand{\Deltam}{\hbox{\boldmath$\Delta$}}
\newcommand{\Sigmam}{\hbox{\boldmath$\Sigma$}}
\newcommand{\Phim}{\hbox{\boldmath$\Phi$}}
\newcommand{\Pim}{\hbox{\boldmath$\Pi$}}
\newcommand{\Psim}{\hbox{\boldmath$\Psi$}}
\newcommand{\Thetam}{\hbox{\boldmath$\Theta$}}
\newcommand{\Omegam}{\hbox{\boldmath$\Omega$}}
\newcommand{\Xim}{\hbox{\boldmath$\Xi$}}


% Sans Serif small case

\newcommand{\Gsf}{{\sf G}}

\newcommand{\asf}{{\sf a}}
\newcommand{\bsf}{{\sf b}}
\newcommand{\csf}{{\sf c}}
\newcommand{\dsf}{{\sf d}}
\newcommand{\esf}{{\sf e}}
\newcommand{\fsf}{{\sf f}}
\newcommand{\gsf}{{\sf g}}
\newcommand{\hsf}{{\sf h}}
\newcommand{\isf}{{\sf i}}
\newcommand{\jsf}{{\sf j}}
\newcommand{\ksf}{{\sf k}}
\newcommand{\lsf}{{\sf l}}
\newcommand{\msf}{{\sf m}}
\newcommand{\nsf}{{\sf n}}
\newcommand{\osf}{{\sf o}}
\newcommand{\psf}{{\sf p}}
\newcommand{\qsf}{{\sf q}}
\newcommand{\rsf}{{\sf r}}
\newcommand{\ssf}{{\sf s}}
\newcommand{\tsf}{{\sf t}}
\newcommand{\usf}{{\sf u}}
\newcommand{\wsf}{{\sf w}}
\newcommand{\vsf}{{\sf v}}
\newcommand{\xsf}{{\sf x}}
\newcommand{\ysf}{{\sf y}}
\newcommand{\zsf}{{\sf z}}


% mixed symbols

\newcommand{\sinc}{{\hbox{sinc}}}
\newcommand{\diag}{{\hbox{diag}}}
\renewcommand{\det}{{\hbox{det}}}
\newcommand{\trace}{{\hbox{tr}}}
\newcommand{\sign}{{\hbox{sign}}}
\renewcommand{\arg}{{\hbox{arg}}}
\newcommand{\var}{{\hbox{var}}}
\newcommand{\cov}{{\hbox{cov}}}
\newcommand{\Ei}{{\rm E}_{\rm i}}
\renewcommand{\Re}{{\rm Re}}
\renewcommand{\Im}{{\rm Im}}
\newcommand{\eqdef}{\stackrel{\Delta}{=}}
\newcommand{\defines}{{\,\,\stackrel{\scriptscriptstyle \bigtriangleup}{=}\,\,}}
\newcommand{\<}{\left\langle}
\renewcommand{\>}{\right\rangle}
\newcommand{\herm}{{\sf H}}
\newcommand{\trasp}{{\sf T}}
\newcommand{\transp}{{\sf T}}
\renewcommand{\vec}{{\rm vec}}
\newcommand{\Psf}{{\sf P}}
\newcommand{\SINR}{{\sf SINR}}
\newcommand{\SNR}{{\sf SNR}}
\newcommand{\MMSE}{{\sf MMSE}}
\newcommand{\REF}{{\RED [REF]}}

% Markov chain
\usepackage{stmaryrd} % for \mkv 
\newcommand{\mkv}{-\!\!\!\!\minuso\!\!\!\!-}

% Colors

\newcommand{\RED}{\color[rgb]{1.00,0.10,0.10}}
\newcommand{\BLUE}{\color[rgb]{0,0,0.90}}
\newcommand{\GREEN}{\color[rgb]{0,0.80,0.20}}

%%%%%%%%%%%%%%%%%%%%%%%%%%%%%%%%%%%%%%%%%%
\usepackage{hyperref}
\hypersetup{
    bookmarks=true,         % show bookmarks bar?
    unicode=false,          % non-Latin characters in AcrobatÕs bookmarks
    pdftoolbar=true,        % show AcrobatÕs toolbar?
    pdfmenubar=true,        % show AcrobatÕs menu?
    pdffitwindow=false,     % window fit to page when opened
    pdfstartview={FitH},    % fits the width of the page to the window
%    pdftitle={My title},    % title
%    pdfauthor={Author},     % author
%    pdfsubject={Subject},   % subject of the document
%    pdfcreator={Creator},   % creator of the document
%    pdfproducer={Producer}, % producer of the document
%    pdfkeywords={keyword1} {key2} {key3}, % list of keywords
    pdfnewwindow=true,      % links in new window
    colorlinks=true,       % false: boxed links; true: colored links
    linkcolor=red,          % color of internal links (change box color with linkbordercolor)
    citecolor=green,        % color of links to bibliography
    filecolor=blue,      % color of file links
    urlcolor=blue           % color of external links
}
%%%%%%%%%%%%%%%%%%%%%%%%%%%%%%%%%%%%%%%%%%%



\definecolor{myblue}{RGB}{173, 216, 230} % Light blue background
\definecolor{myborder}{RGB}{0, 102, 204} % Dark blue border

\setlength{\aboverulesep}{1.5pt}  % Reduce space above \midrule
\setlength{\belowrulesep}{1.5pt}  % Reduce space below \midrule

\newcommand{\rulesep}{\unskip\ \vrule\ }
\newcolumntype{Y}{>{\centering\arraybackslash}X}
\newtcbox{\myovalbox}{colback=cyan,boxrule=0pt,arc=2pt,
  boxsep=0pt,left=1pt,right=1pt,top=0pt,bottom=0pt}

\newcommand{\psulogo}{\raisebox{3.4pt}{\includegraphics[scale=0.025]{./images/logos/penn-state-shield.jpeg}}}
\newcommand{\nittlogo}{\raisebox{3.4pt}{\includegraphics[scale=0.025]{./images/logos/nitt_logo.png}}}
% If the title and author information does not fit in the area allocated, uncomment the following
%
%\setlength\titlebox{<dim>}
%
% and set <dim> to something 5cm or larger.

\title{{\name}: Can Multimodal Large Language Models Effectively Understand Coherence Relationships?}

\author{
  \textbf{Aashish Anantha Ramakrishnan\psulogo},
  \textbf{Aadarsh Anantha Ramakrishnan\nittlogo},
  \textbf{Dongwon Lee\psulogo}
\\
  The Pennsylvania State University\textsuperscript{\psulogo},
  National Institute of Technology, Tiruchirappalli\textsuperscript{\nittlogo}
\\
  \texttt{
    \{aza6352, dul13\}@psu.edu\psulogo, 106121001@nitt.edu\nittlogo
  }
}


\begin{document}
\maketitle


\begin{abstract}
Multimodal Large Language Models (MLLMs) are renowned for their superior instruction-following and reasoning capabilities across diverse problem domains. However, existing benchmarks primarily focus on assessing factual and logical correctness in downstream tasks, with limited emphasis on evaluating MLLMs' ability to interpret pragmatic cues and intermodal relationships. To address this gap, we assess the competency of MLLMs in performing {\em Multimodal Discourse Analysis} (MDA) using Coherence Relations. Our benchmark, {\name}, encompasses a broad spectrum of Coherence Relations across 3 different discourse domains at varying levels of granularity. Through our experiments on 10+ MLLMs employing different prompting strategies, we show that even top models like Gemini 1.5 Pro and GPT-4o fail to match the performance of simple classifier-based baselines. This study emphasizes the need to move beyond similarity-based metrics and adopt a discourse-driven framework for evaluating MLLMs, providing a more nuanced assessment of their capabilities. The benchmark and code are available at: \url{https://github.com/aashish2000/CORDIAL}.


\end{abstract}

\section{Introduction}
Backdoor attacks pose a concealed yet profound security risk to machine learning (ML) models, for which the adversaries can inject a stealth backdoor into the model during training, enabling them to illicitly control the model's output upon encountering predefined inputs. These attacks can even occur without the knowledge of developers or end-users, thereby undermining the trust in ML systems. As ML becomes more deeply embedded in critical sectors like finance, healthcare, and autonomous driving \citep{he2016deep, liu2020computing, tournier2019mrtrix3, adjabi2020past}, the potential damage from backdoor attacks grows, underscoring the emergency for developing robust defense mechanisms against backdoor attacks.

To address the threat of backdoor attacks, researchers have developed a variety of strategies \cite{liu2018fine,wu2021adversarial,wang2019neural,zeng2022adversarial,zhu2023neural,Zhu_2023_ICCV, wei2024shared,wei2024d3}, aimed at purifying backdoors within victim models. These methods are designed to integrate with current deployment workflows seamlessly and have demonstrated significant success in mitigating the effects of backdoor triggers \cite{wubackdoorbench, wu2023defenses, wu2024backdoorbench,dunnett2024countering}.  However, most state-of-the-art (SOTA) backdoor purification methods operate under the assumption that a small clean dataset, often referred to as \textbf{auxiliary dataset}, is available for purification. Such an assumption poses practical challenges, especially in scenarios where data is scarce. To tackle this challenge, efforts have been made to reduce the size of the required auxiliary dataset~\cite{chai2022oneshot,li2023reconstructive, Zhu_2023_ICCV} and even explore dataset-free purification techniques~\cite{zheng2022data,hong2023revisiting,lin2024fusing}. Although these approaches offer some improvements, recent evaluations \cite{dunnett2024countering, wu2024backdoorbench} continue to highlight the importance of sufficient auxiliary data for achieving robust defenses against backdoor attacks.

While significant progress has been made in reducing the size of auxiliary datasets, an equally critical yet underexplored question remains: \emph{how does the nature of the auxiliary dataset affect purification effectiveness?} In  real-world  applications, auxiliary datasets can vary widely, encompassing in-distribution data, synthetic data, or external data from different sources. Understanding how each type of auxiliary dataset influences the purification effectiveness is vital for selecting or constructing the most suitable auxiliary dataset and the corresponding technique. For instance, when multiple datasets are available, understanding how different datasets contribute to purification can guide defenders in selecting or crafting the most appropriate dataset. Conversely, when only limited auxiliary data is accessible, knowing which purification technique works best under those constraints is critical. Therefore, there is an urgent need for a thorough investigation into the impact of auxiliary datasets on purification effectiveness to guide defenders in  enhancing the security of ML systems. 

In this paper, we systematically investigate the critical role of auxiliary datasets in backdoor purification, aiming to bridge the gap between idealized and practical purification scenarios.  Specifically, we first construct a diverse set of auxiliary datasets to emulate real-world conditions, as summarized in Table~\ref{overall}. These datasets include in-distribution data, synthetic data, and external data from other sources. Through an evaluation of SOTA backdoor purification methods across these datasets, we uncover several critical insights: \textbf{1)} In-distribution datasets, particularly those carefully filtered from the original training data of the victim model, effectively preserve the model’s utility for its intended tasks but may fall short in eliminating backdoors. \textbf{2)} Incorporating OOD datasets can help the model forget backdoors but also bring the risk of forgetting critical learned knowledge, significantly degrading its overall performance. Building on these findings, we propose Guided Input Calibration (GIC), a novel technique that enhances backdoor purification by adaptively transforming auxiliary data to better align with the victim model’s learned representations. By leveraging the victim model itself to guide this transformation, GIC optimizes the purification process, striking a balance between preserving model utility and mitigating backdoor threats. Extensive experiments demonstrate that GIC significantly improves the effectiveness of backdoor purification across diverse auxiliary datasets, providing a practical and robust defense solution.

Our main contributions are threefold:
\textbf{1) Impact analysis of auxiliary datasets:} We take the \textbf{first step}  in systematically investigating how different types of auxiliary datasets influence backdoor purification effectiveness. Our findings provide novel insights and serve as a foundation for future research on optimizing dataset selection and construction for enhanced backdoor defense.
%
\textbf{2) Compilation and evaluation of diverse auxiliary datasets:}  We have compiled and rigorously evaluated a diverse set of auxiliary datasets using SOTA purification methods, making our datasets and code publicly available to facilitate and support future research on practical backdoor defense strategies.
%
\textbf{3) Introduction of GIC:} We introduce GIC, the \textbf{first} dedicated solution designed to align auxiliary datasets with the model’s learned representations, significantly enhancing backdoor mitigation across various dataset types. Our approach sets a new benchmark for practical and effective backdoor defense.



\section{Related Work}

\subsection{Large 3D Reconstruction Models}
Recently, generalized feed-forward models for 3D reconstruction from sparse input views have garnered considerable attention due to their applicability in heavily under-constrained scenarios. The Large Reconstruction Model (LRM)~\cite{hong2023lrm} uses a transformer-based encoder-decoder pipeline to infer a NeRF reconstruction from just a single image. Newer iterations have shifted the focus towards generating 3D Gaussian representations from four input images~\cite{tang2025lgm, xu2024grm, zhang2025gslrm, charatan2024pixelsplat, chen2025mvsplat, liu2025mvsgaussian}, showing remarkable novel view synthesis results. The paradigm of transformer-based sparse 3D reconstruction has also successfully been applied to lifting monocular videos to 4D~\cite{ren2024l4gm}. \\
Yet, none of the existing works in the domain have studied the use-case of inferring \textit{animatable} 3D representations from sparse input images, which is the focus of our work. To this end, we build on top of the Large Gaussian Reconstruction Model (GRM)~\cite{xu2024grm}.

\subsection{3D-aware Portrait Animation}
A different line of work focuses on animating portraits in a 3D-aware manner.
MegaPortraits~\cite{drobyshev2022megaportraits} builds a 3D Volume given a source and driving image, and renders the animated source actor via orthographic projection with subsequent 2D neural rendering.
3D morphable models (3DMMs)~\cite{blanz19993dmm} are extensively used to obtain more interpretable control over the portrait animation. For example, StyleRig~\cite{tewari2020stylerig} demonstrates how a 3DMM can be used to control the data generated from a pre-trained StyleGAN~\cite{karras2019stylegan} network. ROME~\cite{khakhulin2022rome} predicts vertex offsets and texture of a FLAME~\cite{li2017flame} mesh from the input image.
A TriPlane representation is inferred and animated via FLAME~\cite{li2017flame} in multiple methods like Portrait4D~\cite{deng2024portrait4d}, Portrait4D-v2~\cite{deng2024portrait4dv2}, and GPAvatar~\cite{chu2024gpavatar}.
Others, such as VOODOO 3D~\cite{tran2024voodoo3d} and VOODOO XP~\cite{tran2024voodooxp}, learn their own expression encoder to drive the source person in a more detailed manner. \\
All of the aforementioned methods require nothing more than a single image of a person to animate it. This allows them to train on large monocular video datasets to infer a very generic motion prior that even translates to paintings or cartoon characters. However, due to their task formulation, these methods mostly focus on image synthesis from a frontal camera, often trading 3D consistency for better image quality by using 2D screen-space neural renderers. In contrast, our work aims to produce a truthful and complete 3D avatar representation from the input images that can be viewed from any angle.  

\subsection{Photo-realistic 3D Face Models}
The increasing availability of large-scale multi-view face datasets~\cite{kirschstein2023nersemble, ava256, pan2024renderme360, yang2020facescape} has enabled building photo-realistic 3D face models that learn a detailed prior over both geometry and appearance of human faces. HeadNeRF~\cite{hong2022headnerf} conditions a Neural Radiance Field (NeRF)~\cite{mildenhall2021nerf} on identity, expression, albedo, and illumination codes. VRMM~\cite{yang2024vrmm} builds a high-quality and relightable 3D face model using volumetric primitives~\cite{lombardi2021mvp}. One2Avatar~\cite{yu2024one2avatar} extends a 3DMM by anchoring a radiance field to its surface. More recently, GPHM~\cite{xu2025gphm} and HeadGAP~\cite{zheng2024headgap} have adopted 3D Gaussians to build a photo-realistic 3D face model. \\
Photo-realistic 3D face models learn a powerful prior over human facial appearance and geometry, which can be fitted to a single or multiple images of a person, effectively inferring a 3D head avatar. However, the fitting procedure itself is non-trivial and often requires expensive test-time optimization, impeding casual use-cases on consumer-grade devices. While this limitation may be circumvented by learning a generalized encoder that maps images into the 3D face model's latent space, another fundamental limitation remains. Even with more multi-view face datasets being published, the number of available training subjects rarely exceeds the thousands, making it hard to truly learn the full distibution of human facial appearance. Instead, our approach avoids generalizing over the identity axis by conditioning on some images of a person, and only generalizes over the expression axis for which plenty of data is available. 

A similar motivation has inspired recent work on codec avatars where a generalized network infers an animatable 3D representation given a registered mesh of a person~\cite{cao2022authentic, li2024uravatar}.
The resulting avatars exhibit excellent quality at the cost of several minutes of video capture per subject and expensive test-time optimization.
For example, URAvatar~\cite{li2024uravatar} finetunes their network on the given video recording for 3 hours on 8 A100 GPUs, making inference on consumer-grade devices impossible. In contrast, our approach directly regresses the final 3D head avatar from just four input images without the need for expensive test-time fine-tuning.


\section{Methodology}

\subsection{Problem Definition}

Given a multivariate time series input $X \in \mathbb{R}^{C  \times T}$, multivariate time series forecasting tasks are designed to predict its future $F$ time steps $\hat{Y}\in \mathbb{R}^{C \times F}$ using past $T$ steps. $C $ is the number of variates or channels.

\subsection{Preliminary Analysis}

This section presents why RevIN~\citep{Kim_revin,liu2022non}, High-pass, and Low-pass filters fail to address the Mid-Frequency Spectrum Gap. Let the input univariate time series be $ x(t) $ with length $ T $ and target $ y(t) $ with length $ F $. 

\begin{definition}[Frequency Spectral Energy]\label{def:energy}
The Fourier transform of $x(t)$, $X(f)$, and its spectral energy $E_X(f)$ is given by:
\vspace{-0.2cm}
\begin{align}
X(f) = \sum_{t=0}^{T-1} x(t) e^{-i 2 \pi f t / {T-1}}, \quad &f = 0, 1, \dots, T-1\notag\\
E_X(f) = |X(f)|^2.
\end{align}
\vspace{-0.2cm}
\end{definition}

\textbf{Impact of RevIN on Frequency Spectrum \quad}
\begin{definition}[Reversible Instance Normalization]\label{def:RevIN}
Given a \textbf{forecast model} $ f: \mathbb{R}^T \rightarrow \mathbb{R}^F $ that generates a forecast $ \hat{y}(t) $ from a given input $x(t)$, RevIN is defined as:
\vspace{-0.2cm}
\begin{align}
&\hat{x}(t) = \frac{x(t) - \mu}{\sigma},\quad t = 0, 1, \dots, T-1\notag\\
&\hat{y}(t) = f(\hat{x}(t)), \quad \hat{y}(t)_{rev}= \hat{y}(t) \cdot \sigma + \mu,\notag\\
&\mu = \frac{1}{T} \sum_{t=0}^{T-1} x(t), \quad \sigma = \sqrt{\frac{1}{T} \sum_{t=0}^{T-1} (x(t) - \mu)^2}.
\end{align}
\vspace{-0.2cm}
\end{definition}

\begin{theorem} [Frequency Spectrum after RevIN] \label{theorem:RevIN}
\vspace{-0.2cm}
The spectral energy of $\hat{x}(t)$ (transformed using RevIN):
\begin{align}
E_{\hat{X}}(0)=0,& \quad f=0, \notag\\
E_{\hat{X}}(f) = \left( \frac{1}{\sigma} \right)^2 |X(f)|^2,&\quad f = 1,2,\dots, T-1 . 
\end{align}
\vspace{-0.2cm}
\end{theorem}
The proof is in Appendix~\ref{app:RevIN}. Theorem~\ref{theorem:RevIN} suggests that RevIN scales the absolute spectral energy by $ \sigma^2 $ but does not affect its relative distribution except $E_{\hat{X}}(0)=0$. Thus, RevIN preserves the relative spectral energy distribution and leaves the Mid-Frequency Spectrum Gap unresolved. \textit{However, our experiments still employ RevIN to ensure a fair comparison with other baselines.}
\begin{figure*}[h]
  \centering
  \includegraphics[width=1.\linewidth]{Faker/source/assets/jpg/ReFocus.jpg}
  \caption{General structure of \textbf{ReFocus}. `Adaptive Mid-Frequency Energy Optimizer (AMEO)' enhances mid-frequency components modeling, and `Energy-based Key-Frequency Picking Block' (EKPB) effectively captures shared Key-Frequency across channels}
  \label{fig:refocus}
\end{figure*}

\begin{figure*}[h]
  \centering
  \includegraphics[width=0.7\linewidth]{Faker/source/assets/jpg/ket.jpg}
  \caption{General process of the \textbf{Key-Frequency Enhanced Training strategy (KET)}, where spectral information from other channels is randomly introduced into each channel, to enhance the extraction of the shared Key-Frequency.}
  \label{fig:reshuffle}
\end{figure*}
\textbf{Impact of High- and Low-pass filter \quad}
We still define $\hat{x}(t)$ to be the filtered (processed) signal, obtained by applying a filter $H(f)$ (High/Low-pass filter). The filter $ H(f) $ is 1 in the passband (High/Low frequency) and 0 in the stopband (Middle frequency). So $E_{\hat{X}}(f)=0,\quad E_{\hat{X}}\leq E_X(f)$ for middle frequencies, which creates even larger gap.

\subsection{Overall Structure of The Proposed ReFocus}

In this section, we elucidate the overall architecture of \textbf{ReFocus}, depicted in Figure \ref{fig:refocus}. We define frequency domain projection as $D1\rightarrow D2$ representing a projection from dimension $D1$ to $D2$ in the frequency domain~\citep{xu2024fits}. Initially, we apply \textbf{AMEO} to the input $X \in \mathbb{R}^{C \times T}$, yielding the processed spectrum $ X_{am} \in \mathbb{R}^{C  \times T} $. Next, we use a projection $T\rightarrow D$ to transform $ X_{am}$ into the Variate Embedding $ X_{em} \in \mathbb{R}^{C  \times D}$~\citep{LiuiTransformer}. Then, $X_{em}$ go through $N$ \textbf{EKPB} to generate representation $H_{N+1}$, which is projected to obtain final prediction $\hat{Y}$. 

\textbf{Adaptive Mid-Frequency Energy Optimizer \quad}
Building upon the \textbf{Preliminary Analysis}, we propose a convolution- and residual learning-based solution to address the Mid-Frequency Spectrum Gap, which we denoted as AMEO. 
\begin{definition}[Adaptive Mid-Frequency Energy Optimizer]\label{def:AMEO}
AMEO is defined as:
\begin{align}
&\hat{x}(t) = x(t)-\frac{\beta}{K}\sum_{k=0}^{K-1} \tilde{x}(t+K-1-k),\notag\\
&\tilde{x}(t) =\notag\\
&\begin{cases}
x(t-(\frac{K}{2}+1)), \quad \text{if } \frac{K}{2}+1 \leq t < T+\frac{K}{2}+1, \\
0,  \quad\text{if } 0 \leq t < \frac{K}{2}+1 \text{ or } T+\frac{K}{2}+1 \leq t < T+K.
\end{cases}
\end{align}
\vspace{-0.2cm}
\end{definition}

It is equivalent to $x=x-\beta \cdot Conv(x)$. $Conv$ is a 1D convolution (Zero-padding at both ends, stride $s=1$, kernel size $K$, with values initialized as $ \frac{1}{K} $). $\beta \in \mathbb{R}^{1}$ is a hyperparameter.

\begin{theorem} [Frequency Spectrum after AMEO] \label{theorem:AMEO}
The spectral energy of $\hat{x}(t)$ obtained using AMEO:
\begin{align}
E_{\hat{X}}(f) =|X(f)|^2 \left\{1 - \beta \cdot \underbrace{\frac{1}{K} \sum_{k=0}^{K-1} e^{i 2 \pi f (\frac{3K}{2}-k -2) / {T-1}}}_{G(f)}\right\}^2
\end{align}
\vspace{-0.2cm}
\end{theorem}

The proof is in Appendix~\ref{app:AMEO}. We have $E_{\hat{X}}(f) =|X(f)|^2(1-\beta  \cdot G(f))^2$. Generally, $ G(f) $ behaves as a decay function, gradually reducing its value from \textbf{One} to \textbf{Zero}. Such \textbf{decay behavior} makes AMEO relatively enhances mid-frequency components, thus addressing the Mid-Frequency Spectrum Gap.

\textbf{Energy-based Key-Frequency Picking Block \quad} In each \textbf{EKPB}, the input $ H_i \in \mathbb{R}^{C  \times D} (H_1=X_{em}) $ is first processed through an MLP to generate $ H_i^k \in \mathbb{R}^{C  \times Q}$. Then, FFT is applied to get $ H_i^f \in \mathbb{R}^{C  \times (Q/2+1)}$. For $ H_i^f$, we calculate its energy, denoted as $ H_i^e \in \mathbb{R}^{C  \times (Q/2+1)}$. A cross-channel softmax is then applied to $ H_i^e$ per frequency to obtain a probability distribution $ H_i^{soft} \in \mathbb{R}^{C  \times (Q/2+1)}$. Using $H_i^{soft}$, we select values from $ H_i^f$ across channels for each frequency, resulting in $K^f_i \in \mathbb{R}^{1  \times (Q/2+1)}$, which represents the Shared Key-Frequency across all channels. Then iFFT is performed on $K^f_i$ to get $K_i\in \mathbb{R}^{1  \times Q}$, followed by projection $Q\rightarrow D$ and repeating (C times) to get $\hat{K}_i \in \mathbb{R}^{C  \times D}$. This $\hat{K}_i$ is point-wisely added to $\hat{H_i}\in \mathbb{R}^{C  \times D}$ , which is the projection of $ H_i$ using projection $D\rightarrow D$. Then, an MLP and $Add\&Norm$ is applied to the result $HK\in \mathbb{R}^{C  \times D}$ to fuse inter-series dependencies information, and another MLP and $Add\&Norm$ is used to capture intra-series variations~\citep{LiuiTransformer}. The output of each \textbf{EKPB} is $\hat{O_i} \in \mathbb{R}^{C  \times D}$, where $H_{i+1}=\hat{O_i}$.

\subsection{Key-Frequency Enhanced Training strategy}

In real-world time series, certain channels often exhibit spectral dependencies, which may not be fully captured in the training set, and the specific channels with such dependencies are also unknown~\citep{geweke1984freqchannel,Zhao2024freqchannel}. So this work borrows insight from recent advancement of mix-up in time series~\citep{zhou2023mixup,ansari2024mixup}, randomly introducing spectral information from other channels into each channel, to enhance the extraction of the shared Key-Frequency, as in Figure~\ref{fig:reshuffle}. Given a multivariate time series input $X \in \mathbb{R}^{C \times T}$ and its ground-truth $Y \in \mathbb{R}^{C \times F}$, we generate a pseudo sample pair: 

\begin{align}
X' = iFFT(FFT(X) +\alpha \cdot FFT(X[\text{perm},:]))&,  \notag\\ 
Y' = iFFT(FFT(Y) +\alpha \cdot FFT(Y[\text{perm},:]))&.
\end{align}

$\alpha \in \mathbb{R}^{C \times 1}$ is a weight vector sampled from a normal distribution, $\text{perm}$ is a reshuffled channel index. Since $FFT$ and $iFFT$ are linear operations, this mix-up process can be equivalently simplified in the \textbf{Time Domain}:
\begin{align}
X' = X +\alpha \cdot X[\text{perm},:]&,  \notag\\
Y' = Y +\alpha \cdot Y[\text{perm},:]&
 \end{align}
We alternate training between real and synthetic data to preserve the spectral dependencies in real samples. This combines the advantages of data augmentation, such as improved generalization, while mitigating potential drawbacks like over-smoothing and training instability~\citep{ryu2024tf,alkhalifah2022tf}.












\section{Experiments}

\subsection{Setups}
\subsubsection{Implementation Details}
We apply our FDS method to two types of 3DGS: 
the original 3DGS, and 2DGS~\citep{huang20242d}. 
%
The number of iterations in our optimization 
process is 35,000.
We follow the default training configuration 
and apply our FDS method after 15,000 iterations,
then we add normal consistency loss for both
3DGS and 2DGS after 25000 iterations.
%
The weight for FDS, $\lambda_{fds}$, is set to 0.015,
the $\sigma$ is set to 23,
and the weight for normal consistency is set to 0.15
for all experiments. 
We removed the depth distortion loss in 2DGS 
because we found that it degrades its results in indoor scenes.
%
The Gaussian point cloud is initialized using Colmap
for all datasets.
%
%
We tested the impact of 
using Sea Raft~\citep{wang2025sea} and 
Raft\citep{teed2020raft} on FDS performance.
%
Due to the blurriness of the ScanNet dataset, 
additional prior constraints are required.
Thus, we incorporate normal prior supervision 
on the rendered normals 
in ScanNet (V2) dataset by default.
The normal prior is predicted by the Stable Normal 
model~\citep{ye2024stablenormal}
across all types of 3DGS.
%
The entire framework is implemented in 
PyTorch~\citep{paszke2019pytorch}, 
and all experiments are conducted on 
a single NVIDIA 4090D GPU.

\begin{figure}[t] \centering
    \makebox[0.16\textwidth]{\scriptsize Input}
    \makebox[0.16\textwidth]{\scriptsize 3DGS}
    \makebox[0.16\textwidth]{\scriptsize 2DGS}
    \makebox[0.16\textwidth]{\scriptsize 3DGS + FDS}
    \makebox[0.16\textwidth]{\scriptsize 2DGS + FDS}
    \makebox[0.16\textwidth]{\scriptsize GT (Depth)}

    \includegraphics[width=0.16\textwidth]{figure/fig3_img/compare3/gt_rgb/frame_00522.jpg}
    \includegraphics[width=0.16\textwidth]{figure/fig3_img/compare3/3DGS/frame_00522.jpg}
    \includegraphics[width=0.16\textwidth]{figure/fig3_img/compare3/2DGS/frame_00522.jpg}
    \includegraphics[width=0.16\textwidth]{figure/fig3_img/compare3/3DGS+FDS/frame_00522.jpg}
    \includegraphics[width=0.16\textwidth]{figure/fig3_img/compare3/2DGS+FDS/frame_00522.jpg}
    \includegraphics[width=0.16\textwidth]{figure/fig3_img/compare3/gt_depth/frame_00522.jpg} \\

    % \includegraphics[width=0.16\textwidth]{figure/fig3_img/compare1/gt_rgb/frame_00137.jpg}
    % \includegraphics[width=0.16\textwidth]{figure/fig3_img/compare1/3DGS/frame_00137.jpg}
    % \includegraphics[width=0.16\textwidth]{figure/fig3_img/compare1/2DGS/frame_00137.jpg}
    % \includegraphics[width=0.16\textwidth]{figure/fig3_img/compare1/3DGS+FDS/frame_00137.jpg}
    % \includegraphics[width=0.16\textwidth]{figure/fig3_img/compare1/2DGS+FDS/frame_00137.jpg}
    % \includegraphics[width=0.16\textwidth]{figure/fig3_img/compare1/gt_depth/frame_00137.jpg} \\

     \includegraphics[width=0.16\textwidth]{figure/fig3_img/compare2/gt_rgb/frame_00262.jpg}
    \includegraphics[width=0.16\textwidth]{figure/fig3_img/compare2/3DGS/frame_00262.jpg}
    \includegraphics[width=0.16\textwidth]{figure/fig3_img/compare2/2DGS/frame_00262.jpg}
    \includegraphics[width=0.16\textwidth]{figure/fig3_img/compare2/3DGS+FDS/frame_00262.jpg}
    \includegraphics[width=0.16\textwidth]{figure/fig3_img/compare2/2DGS+FDS/frame_00262.jpg}
    \includegraphics[width=0.16\textwidth]{figure/fig3_img/compare2/gt_depth/frame_00262.jpg} \\

    \includegraphics[width=0.16\textwidth]{figure/fig3_img/compare4/gt_rgb/frame00000.png}
    \includegraphics[width=0.16\textwidth]{figure/fig3_img/compare4/3DGS/frame00000.png}
    \includegraphics[width=0.16\textwidth]{figure/fig3_img/compare4/2DGS/frame00000.png}
    \includegraphics[width=0.16\textwidth]{figure/fig3_img/compare4/3DGS+FDS/frame00000.png}
    \includegraphics[width=0.16\textwidth]{figure/fig3_img/compare4/2DGS+FDS/frame00000.png}
    \includegraphics[width=0.16\textwidth]{figure/fig3_img/compare4/gt_depth/frame00000.png} \\

    \includegraphics[width=0.16\textwidth]{figure/fig3_img/compare5/gt_rgb/frame00080.png}
    \includegraphics[width=0.16\textwidth]{figure/fig3_img/compare5/3DGS/frame00080.png}
    \includegraphics[width=0.16\textwidth]{figure/fig3_img/compare5/2DGS/frame00080.png}
    \includegraphics[width=0.16\textwidth]{figure/fig3_img/compare5/3DGS+FDS/frame00080.png}
    \includegraphics[width=0.16\textwidth]{figure/fig3_img/compare5/2DGS+FDS/frame00080.png}
    \includegraphics[width=0.16\textwidth]{figure/fig3_img/compare5/gt_depth/frame00080.png} \\



    \caption{\textbf{Comparison of depth reconstruction on Mushroom and ScanNet datasets.} The original
    3DGS or 2DGS model equipped with FDS can remove unwanted floaters and reconstruct
    geometry more preciously.}
    \label{fig:compare}
\end{figure}


\subsubsection{Datasets and Metrics}

We evaluate our method for 3D reconstruction 
and novel view synthesis tasks on
\textbf{Mushroom}~\citep{ren2024mushroom},
\textbf{ScanNet (v2)}~\citep{dai2017scannet}, and 
\textbf{Replica}~\citep{replica19arxiv}
datasets,
which feature challenging indoor scenes with both 
sparse and dense image sampling.
%
The Mushroom dataset is an indoor dataset 
with sparse image sampling and two distinct 
camera trajectories. 
%
We train our model on the training split of 
the long capture sequence and evaluate 
novel view synthesis on the test split 
of the long capture sequences.
%
Five scenes are selected to evaluate our FDS, 
including "coffee room", "honka", "kokko", 
"sauna", and "vr room". 
%
ScanNet(V2)~\citep{dai2017scannet}  consists of 1,613 indoor scenes
with annotated camera poses and depth maps. 
%
We select 5 scenes from the ScanNet (V2) dataset, 
uniformly sampling one-tenth of the views,
following the approach in ~\citep{guo2022manhattan}.
To further improve the geometry rendering quality of 3DGS, 
%
Replica~\citep{replica19arxiv} contains small-scale 
real-world indoor scans. 
We evaluate our FDS on five scenes from 
Replica: office0, office1, office2, office3 and office4,
selecting one-tenth of the views for training.
%
The results for Replica are provided in the 
supplementary materials.
To evaluate the rendering quality and geometry 
of 3DGS, we report PSNR, SSIM, and LPIPS for 
rendering quality, along with Absolute Relative Distance 
(Abs Rel) as a depth quality metrics.
%
Additionally, for mesh evaluation, 
we use metrics including Accuracy, Completion, 
Chamfer-L1 distance, Normal Consistency, 
and F-scores.




\subsection{Results}
\subsubsection{Depth rendering and novel view synthesis}
The comparison results on Mushroom and 
ScanNet are presented in \tabref{tab:mushroom} 
and \tabref{tab:scannet}, respectively. 
%
Due to the sparsity of sampling 
in the Mushroom dataset,
challenges are posed for both GOF~\citep{yu2024gaussian} 
and PGSR~\citep{chen2024pgsr}, 
leading to their relative poor performance 
on the Mushroom dataset.
%
Our approach achieves the best performance 
with the FDS method applied during the training process.
The FDS significantly enhances the 
geometric quality of 3DGS on the Mushroom dataset, 
improving the "abs rel" metric by more than 50\%.
%
We found that Sea Raft~\citep{wang2025sea}
outperforms Raft~\citep{teed2020raft} on FDS, 
indicating that a better optical flow model 
can lead to more significant improvements.
%
Additionally, the render quality of RGB 
images shows a slight improvement, 
by 0.58 in 3DGS and 0.50 in 2DGS, 
benefiting from the incorporation of cross-view consistency in FDS. 
%
On the Mushroom
dataset, adding the FDS loss increases 
the training time by half an hour, which maintains the same
level as baseline.
%
Similarly, our method shows a notable improvement on the ScanNet dataset as well using Sea Raft~\citep{wang2025sea} Model. The "abs rel" metric in 2DGS is improved nearly 50\%. This demonstrates the robustness and effectiveness of the FDS method across different datasets.
%


% \begin{wraptable}{r}{0.6\linewidth} \centering
% \caption{\textbf{Ablation study on geometry priors.}} 
%         \label{tab:analysis_prior}
%         \resizebox{\textwidth}{!}{
\begin{tabular}{c| c c c c c | c c c c}

    \hline
     Method &  Acc$\downarrow$ & Comp $\downarrow$ & C-L1 $\downarrow$ & NC $\uparrow$ & F-Score $\uparrow$ &  Abs Rel $\downarrow$ &  PSNR $\uparrow$  & SSIM  $\uparrow$ & LPIPS $\downarrow$ \\ \hline
    2DGS&   0.1078&  0.0850&  0.0964&  0.7835&  0.5170&  0.1002&  23.56&  0.8166& 0.2730\\
    2DGS+Depth&   0.0862&  0.0702&  0.0782&  0.8153&  0.5965&  0.0672&  23.92&  0.8227& 0.2619 \\
    2DGS+MVDepth&   0.2065&  0.0917&  0.1491&  0.7832&  0.3178&  0.0792&  23.74&  0.8193& 0.2692 \\
    2DGS+Normal&   0.0939&  0.0637&  0.0788&  \textbf{0.8359}&  0.5782&  0.0768&  23.78&  0.8197& 0.2676 \\
    2DGS+FDS &  \textbf{0.0615} & \textbf{ 0.0534}& \textbf{0.0574}& 0.8151& \textbf{0.6974}&  \textbf{0.0561}&  \textbf{24.06}&  \textbf{0.8271}&\textbf{0.2610} \\ \hline
    2DGS+Depth+FDS &  0.0561 &  0.0519& 0.0540& 0.8295& 0.7282&  0.0454&  \textbf{24.22}& \textbf{0.8291}&\textbf{0.2570} \\
    2DGS+Normal+FDS &  \textbf{0.0529} & \textbf{ 0.0450}& \textbf{0.0490}& \textbf{0.8477}& \textbf{0.7430}&  \textbf{0.0443}&  24.10&  0.8283& 0.2590 \\
    2DGS+Depth+Normal &  0.0695 & 0.0513& 0.0604& 0.8540&0.6723&  0.0523&  24.09&  0.8264&0.2575\\ \hline
    2DGS+Depth+Normal+FDS &  \textbf{0.0506} & \textbf{0.0423}& \textbf{0.0464}& \textbf{0.8598}&\textbf{0.7613}&  \textbf{0.0403}&  \textbf{24.22}& 
    \textbf{0.8300}&\textbf{0.0403}\\
    
\bottomrule
\end{tabular}
}
% \end{wraptable}



The qualitative comparisons on the Mushroom and ScanNet dataset 
are illustrated in \figref{fig:compare}. 
%
%
As seen in the first row of \figref{fig:compare}, 
both the original 3DGS and 2DGS suffer from overfitting, 
leading to corrupted geometry generation. 
%
Our FDS effectively mitigates this issue by 
supervising the matching relationship between 
the input and sampled views, 
helping to recover the geometry.
%
FDS also improves the refinement of geometric details, 
as shown in other rows. 
By incorporating the matching prior through FDS, 
the quality of the rendered depth is significantly improved.
%

\begin{table}[t] \centering
\begin{minipage}[t]{0.96\linewidth}
        \captionof{table}{\textbf{3D Reconstruction 
        and novel view synthesis results on Mushroom dataset. * 
        Represents that FDS uses the Raft model.
        }}
        \label{tab:mushroom}
        \resizebox{\textwidth}{!}{
\begin{tabular}{c| c c c c c | c c c c c}
    \hline
     Method &  Acc$\downarrow$ & Comp $\downarrow$ & C-L1 $\downarrow$ & NC $\uparrow$ & F-Score $\uparrow$ &  Abs Rel $\downarrow$ &  PSNR $\uparrow$  & SSIM  $\uparrow$ & LPIPS $\downarrow$ & Time  $\downarrow$ \\ \hline

    % DN-splatter &   &  &  &  &  &  &  &  & \\
    GOF &  0.1812 & 0.1093 & 0.1453 & 0.6292 & 0.3665 & 0.2380  & 21.37  &  0.7762  & 0.3132  & $\approx$1.4h\\ 
    PGSR &  0.0971 & 0.1420 & 0.1196 & 0.7193 & 0.5105 & 0.1723  & 22.13  & 0.7773  & 0.2918  & $\approx$1.2h \\ \hline
    3DGS &   0.1167 &  0.1033&  0.1100&  0.7954&  0.3739&  0.1214&  24.18&  0.8392& 0.2511 &$\approx$0.8h \\
    3DGS + FDS$^*$ & 0.0569  & 0.0676 & 0.0623 & 0.8105 & 0.6573 & 0.0603 & 24.72  & 0.8489 & 0.2379 &$\approx$1.3h \\
    3DGS + FDS & \textbf{0.0527}  & \textbf{0.0565} & \textbf{0.0546} & \textbf{0.8178} & \textbf{0.6958} & \textbf{0.0568} & \textbf{24.76}  & \textbf{0.8486} & \textbf{0.2381} &$\approx$1.3h \\ \hline
    2DGS&   0.1078&  0.0850&  0.0964&  0.7835&  0.5170&  0.1002&  23.56&  0.8166& 0.2730 &$\approx$0.8h\\
    2DGS + FDS$^*$ &  0.0689 &  0.0646& 0.0667& 0.8042& 0.6582& 0.0589& 23.98&  0.8255&0.2621 &$\approx$1.3h\\
    2DGS + FDS &  \textbf{0.0615} & \textbf{ 0.0534}& \textbf{0.0574}& \textbf{0.8151}& \textbf{0.6974}&  \textbf{0.0561}&  \textbf{24.06}&  \textbf{0.8271}&\textbf{0.2610} &$\approx$1.3h \\ \hline
\end{tabular}
}
\end{minipage}\hfill
\end{table}

\begin{table}[t] \centering
\begin{minipage}[t]{0.96\linewidth}
        \captionof{table}{\textbf{3D Reconstruction 
        and novel view synthesis results on ScanNet dataset.}}
        \label{tab:scannet}
        \resizebox{\textwidth}{!}{
\begin{tabular}{c| c c c c c | c c c c }
    \hline
     Method &  Acc $\downarrow$ & Comp $\downarrow$ & C-L1 $\downarrow$ & NC $\uparrow$ & F-Score $\uparrow$ &  Abs Rel $\downarrow$ &  PSNR $\uparrow$  & SSIM  $\uparrow$ & LPIPS $\downarrow$ \\ \hline
    GOF & 1.8671  & 0.0805 & 0.9738 & 0.5622 & 0.2526 & 0.1597  & 21.55  & 0.7575  & 0.3881 \\
    PGSR &  0.2928 & 0.5103 & 0.4015 & 0.5567 & 0.1926 & 0.1661  & 21.71 & 0.7699  & 0.3899 \\ \hline

    3DGS &  0.4867 & 0.1211 & 0.3039 & 0.7342& 0.3059 & 0.1227 & 22.19& 0.7837 & 0.3907\\
    3DGS + FDS &  \textbf{0.2458} & \textbf{0.0787} & \textbf{0.1622} & \textbf{0.7831} & 
    \textbf{0.4482} & \textbf{0.0573} & \textbf{22.83} & \textbf{0.7911} & \textbf{0.3826} \\ \hline
    2DGS &  0.2658 & 0.0845 & 0.1752 & 0.7504& 0.4464 & 0.0831 & 22.59& 0.7881 & 0.3854\\
    2DGS + FDS &  \textbf{0.1457} & \textbf{0.0679} & \textbf{0.1068} & \textbf{0.7883} & 
    \textbf{0.5459} & \textbf{0.0432} & \textbf{22.91} & \textbf{0.7928} & \textbf{0.3800} \\ \hline
\end{tabular}
}
\end{minipage}\hfill
\end{table}


\begin{table}[t] \centering
\begin{minipage}[t]{0.96\linewidth}
        \captionof{table}{\textbf{Ablation study on geometry priors.}}
        \label{tab:analysis_prior}
        \resizebox{\textwidth}{!}{
\begin{tabular}{c| c c c c c | c c c c}

    \hline
     Method &  Acc$\downarrow$ & Comp $\downarrow$ & C-L1 $\downarrow$ & NC $\uparrow$ & F-Score $\uparrow$ &  Abs Rel $\downarrow$ &  PSNR $\uparrow$  & SSIM  $\uparrow$ & LPIPS $\downarrow$ \\ \hline
    2DGS&   0.1078&  0.0850&  0.0964&  0.7835&  0.5170&  0.1002&  23.56&  0.8166& 0.2730\\
    2DGS+Depth&   0.0862&  0.0702&  0.0782&  0.8153&  0.5965&  0.0672&  23.92&  0.8227& 0.2619 \\
    2DGS+MVDepth&   0.2065&  0.0917&  0.1491&  0.7832&  0.3178&  0.0792&  23.74&  0.8193& 0.2692 \\
    2DGS+Normal&   0.0939&  0.0637&  0.0788&  \textbf{0.8359}&  0.5782&  0.0768&  23.78&  0.8197& 0.2676 \\
    2DGS+FDS &  \textbf{0.0615} & \textbf{ 0.0534}& \textbf{0.0574}& 0.8151& \textbf{0.6974}&  \textbf{0.0561}&  \textbf{24.06}&  \textbf{0.8271}&\textbf{0.2610} \\ \hline
    2DGS+Depth+FDS &  0.0561 &  0.0519& 0.0540& 0.8295& 0.7282&  0.0454&  \textbf{24.22}& \textbf{0.8291}&\textbf{0.2570} \\
    2DGS+Normal+FDS &  \textbf{0.0529} & \textbf{ 0.0450}& \textbf{0.0490}& \textbf{0.8477}& \textbf{0.7430}&  \textbf{0.0443}&  24.10&  0.8283& 0.2590 \\
    2DGS+Depth+Normal &  0.0695 & 0.0513& 0.0604& 0.8540&0.6723&  0.0523&  24.09&  0.8264&0.2575\\ \hline
    2DGS+Depth+Normal+FDS &  \textbf{0.0506} & \textbf{0.0423}& \textbf{0.0464}& \textbf{0.8598}&\textbf{0.7613}&  \textbf{0.0403}&  \textbf{24.22}& 
    \textbf{0.8300}&\textbf{0.0403}\\
    
\bottomrule
\end{tabular}
}
\end{minipage}\hfill
\end{table}




\subsubsection{Mesh extraction}
To further demonstrate the improvement in geometry quality, 
we applied methods used in ~\citep{turkulainen2024dnsplatter} 
to extract meshes from the input views of optimized 3DGS. 
The comparison results are presented  
in \tabref{tab:mushroom}. 
With the integration of FDS, the mesh quality is significantly enhanced compared to the baseline, featuring fewer floaters and more well-defined shapes.
 %
% Following the incorporation of FDS, the reconstruction 
% results exhibit fewer floaters and more well-defined 
% shapes in the meshes. 
% Visualized comparisons
% are provided in the supplementary material.

% \begin{figure}[t] \centering
%     \makebox[0.19\textwidth]{\scriptsize GT}
%     \makebox[0.19\textwidth]{\scriptsize 3DGS}
%     \makebox[0.19\textwidth]{\scriptsize 3DGS+FDS}
%     \makebox[0.19\textwidth]{\scriptsize 2DGS}
%     \makebox[0.19\textwidth]{\scriptsize 2DGS+FDS} \\

%     \includegraphics[width=0.19\textwidth]{figure/fig4_img/compare1/gt02.png}
%     \includegraphics[width=0.19\textwidth]{figure/fig4_img/compare1/baseline06.png}
%     \includegraphics[width=0.19\textwidth]{figure/fig4_img/compare1/baseline_fds05.png}
%     \includegraphics[width=0.19\textwidth]{figure/fig4_img/compare1/2dgs04.png}
%     \includegraphics[width=0.19\textwidth]{figure/fig4_img/compare1/2dgs_fds03.png} \\

%     \includegraphics[width=0.19\textwidth]{figure/fig4_img/compare2/gt00.png}
%     \includegraphics[width=0.19\textwidth]{figure/fig4_img/compare2/baseline02.png}
%     \includegraphics[width=0.19\textwidth]{figure/fig4_img/compare2/baseline_fds01.png}
%     \includegraphics[width=0.19\textwidth]{figure/fig4_img/compare2/2dgs04.png}
%     \includegraphics[width=0.19\textwidth]{figure/fig4_img/compare2/2dgs_fds03.png} \\
      
%     \includegraphics[width=0.19\textwidth]{figure/fig4_img/compare3/gt05.png}
%     \includegraphics[width=0.19\textwidth]{figure/fig4_img/compare3/3dgs03.png}
%     \includegraphics[width=0.19\textwidth]{figure/fig4_img/compare3/3dgs_fds04.png}
%     \includegraphics[width=0.19\textwidth]{figure/fig4_img/compare3/2dgs02.png}
%     \includegraphics[width=0.19\textwidth]{figure/fig4_img/compare3/2dgs_fds01.png} \\

%     \caption{\textbf{Qualitative comparison of extracted mesh 
%     on Mushroom and ScanNet datasets.}}
%     \label{fig:mesh}
% \end{figure}












\subsection{Ablation study}


\textbf{Ablation study on geometry priors:} 
To highlight the advantage of incorporating matching priors, 
we incorporated various types of priors generated by different 
models into 2DGS. These include a monocular depth estimation
model (Depth Anything v2)~\citep{yang2024depth}, a two-view depth estimation 
model (Unimatch)~\citep{xu2023unifying}, 
and a monocular normal estimation model (DSINE)~\citep{bae2024rethinking}.
We adapt the scale and shift-invariant loss in Midas~\citep{birkl2023midas} for
monocular depth supervision and L1 loss for two-view depth supervison.
%
We use Sea Raft~\citep{wang2025sea} as our default optical flow model.
%
The comparison results on Mushroom dataset 
are shown in ~\tabref{tab:analysis_prior}.
We observe that the normal prior provides accurate shape information, 
enhancing the geometric quality of the radiance field. 
%
% In contrast, the monocular depth prior slightly increases 
% the 'Abs Rel' due to its ambiguous scale and inaccurate depth ordering.
% Moreover, the performance of monocular depth estimation 
% in the sauna scene is particularly poor, 
% primarily due to the presence of numerous reflective 
% surfaces and textureless walls, which limits the accuracy of monocular depth estimation.
%
The multi-view depth prior, hindered by the limited feature overlap 
between input views, fails to offer reliable geometric 
information. We test average "Abs Rel" of multi-view depth prior
, and the result is 0.19, which performs worse than the "Abs Rel" results 
rendered by original 2DGS.
From the results, it can be seen that depth order information provided by monocular depth improves
reconstruction accuracy. Meanwhile, our FDS achieves the best performance among all the priors, 
and by integrating all
three components, we obtained the optimal results.
%
%
\begin{figure}[t] \centering
    \makebox[0.16\textwidth]{\scriptsize RF (16000 iters)}
    \makebox[0.16\textwidth]{\scriptsize RF* (20000 iters)}
    \makebox[0.16\textwidth]{\scriptsize RF (20000 iters)  }
    \makebox[0.16\textwidth]{\scriptsize PF (16000 iters)}
    \makebox[0.16\textwidth]{\scriptsize PF (20000 iters)}


    % \includegraphics[width=0.16\textwidth]{figure/fig5_img/compare1/16000.png}
    % \includegraphics[width=0.16\textwidth]{figure/fig5_img/compare1/20000_wo_flow_loss.png}
    % \includegraphics[width=0.16\textwidth]{figure/fig5_img/compare1/20000.png}
    % \includegraphics[width=0.16\textwidth]{figure/fig5_img/compare1/16000_prior.png}
    % \includegraphics[width=0.16\textwidth]{figure/fig5_img/compare1/20000_prior.png}\\

    % \includegraphics[width=0.16\textwidth]{figure/fig5_img/compare2/16000.png}
    % \includegraphics[width=0.16\textwidth]{figure/fig5_img/compare2/20000_wo_flow_loss.png}
    % \includegraphics[width=0.16\textwidth]{figure/fig5_img/compare2/20000.png}
    % \includegraphics[width=0.16\textwidth]{figure/fig5_img/compare2/16000_prior.png}
    % \includegraphics[width=0.16\textwidth]{figure/fig5_img/compare2/20000_prior.png}\\

    \includegraphics[width=0.16\textwidth]{figure/fig5_img/compare3/16000.png}
    \includegraphics[width=0.16\textwidth]{figure/fig5_img/compare3/20000_wo_flow_loss.png}
    \includegraphics[width=0.16\textwidth]{figure/fig5_img/compare3/20000.png}
    \includegraphics[width=0.16\textwidth]{figure/fig5_img/compare3/16000_prior.png}
    \includegraphics[width=0.16\textwidth]{figure/fig5_img/compare3/20000_prior.png}\\
    
    \includegraphics[width=0.16\textwidth]{figure/fig5_img/compare4/16000.png}
    \includegraphics[width=0.16\textwidth]{figure/fig5_img/compare4/20000_wo_flow_loss.png}
    \includegraphics[width=0.16\textwidth]{figure/fig5_img/compare4/20000.png}
    \includegraphics[width=0.16\textwidth]{figure/fig5_img/compare4/16000_prior.png}
    \includegraphics[width=0.16\textwidth]{figure/fig5_img/compare4/20000_prior.png}\\

    \includegraphics[width=0.30\textwidth]{figure/fig5_img/bar.png}

    \caption{\textbf{The error map of Radiance Flow and Prior Flow.} RF: Radiance Flow, PF: Prior Flow, * means that there is no FDS loss supervision during optimization.}
    \label{fig:error_map}
\end{figure}




\textbf{Ablation study on FDS: }
In this section, we present the design of our FDS 
method through an ablation study on the 
Mushroom dataset to validate its effectiveness.
%
The optional configurations of FDS are outlined in ~\tabref{tab:ablation_fds}.
Our base model is the 2DGS equipped with FDS,
and its results are shown 
in the first row. The goal of this analysis 
is to evaluate the impact 
of various strategies on FDS sampling and loss design.
%
We observe that when we 
replace $I_i$ in \eqref{equ:mflow} with $C_i$, 
as shown in the second row, the geometric quality 
of 2DGS deteriorates. Using $I_i$ instead of $C_i$ 
help us to remove the floaters in $\bm{C^s}$, which are also 
remained in $\bm{C^i}$.
We also experiment with modifying the FDS loss. For example, 
in the third row, we use the neighbor 
input view as the sampling view, and replace the 
render result of neighbor view with ground truth image of its input view.
%
Due to the significant movement between images, the Prior Flow fails to accurately 
match the pixel between them, leading to a further degradation in geometric quality.
%
Finally, we attempt to fix the sampling view 
and found that this severely damaged the geometric quality, 
indicating that random sampling is essential for the stability 
of the mean error in the Prior flow.



\begin{table}[t] \centering

\begin{minipage}[t]{1.0\linewidth}
        \captionof{table}{\textbf{Ablation study on FDS strategies.}}
        \label{tab:ablation_fds}
        \resizebox{\textwidth}{!}{
\begin{tabular}{c|c|c|c|c|c|c|c}
    \hline
    \multicolumn{2}{c|}{$\mathcal{M}_{\theta}(X, \bm{C^s})$} & \multicolumn{3}{c|}{Loss} & \multicolumn{3}{c}{Metric}  \\
    \hline
    $X=C^i$ & $X=I^i$  & Input view & Sampled view     & Fixed Sampled view        & Abs Rel $\downarrow$ & F-score $\uparrow$ & NC $\uparrow$ \\
    \hline
    & \ding{51} &     &\ding{51}    &    &    \textbf{0.0561}        &  \textbf{0.6974}         & \textbf{0.8151}\\
    \hline
     \ding{51} &           &     &\ding{51}    &    &    0.0839        &  0.6242         &0.8030\\
     &  \ding{51} &   \ding{51}  &    &    &    0.0877       & 0.6091        & 0.7614 \\
      &  \ding{51} &    &    & \ding{51}    &    0.0724           & 0.6312          & 0.8015 \\
\bottomrule
\end{tabular}
}
\end{minipage}
\end{table}




\begin{figure}[htbp] \centering
    \makebox[0.22\textwidth]{}
    \makebox[0.22\textwidth]{}
    \makebox[0.22\textwidth]{}
    \makebox[0.22\textwidth]{}
    \\

    \includegraphics[width=0.22\textwidth]{figure/fig6_img/l1/rgb/frame00096.png}
    \includegraphics[width=0.22\textwidth]{figure/fig6_img/l1/render_rgb/frame00096.png}
    \includegraphics[width=0.22\textwidth]{figure/fig6_img/l1/render_depth/frame00096.png}
    \includegraphics[width=0.22\textwidth]{figure/fig6_img/l1/depth/frame00096.png}

    % \includegraphics[width=0.22\textwidth]{figure/fig6_img/l2/rgb/frame00112.png}
    % \includegraphics[width=0.22\textwidth]{figure/fig6_img/l2/render_rgb/frame00112.png}
    % \includegraphics[width=0.22\textwidth]{figure/fig6_img/l2/render_depth/frame00112.png}
    % \includegraphics[width=0.22\textwidth]{figure/fig6_img/l2/depth/frame00112.png}

    \caption{\textbf{Limitation of FDS.} }
    \label{fig:limitation}
\end{figure}


% \begin{figure}[t] \centering
%     \makebox[0.48\textwidth]{}
%     \makebox[0.48\textwidth]{}
%     \\
%     \includegraphics[width=0.48\textwidth]{figure/loss_Ignatius.pdf}
%     \includegraphics[width=0.48\textwidth]{figure/loss_family.pdf}
%     \caption{\textbf{Comparison the photometric error of Radiance Flow and Prior Flow:} 
%     We add FDS method after 2k iteration during training.
%     The results show
%     that:  1) The Prior Flow is more precise and 
%     robust than Radiance Flow during the radiance 
%     optimization; 2) After adding the FDS loss 
%     which utilize Prior 
%     flow to supervise the Radiance Flow at 2k iterations, 
%     both flow are more accurate, which lead to
%     a mutually reinforcing effects.(TODO fix it)} 
%     \label{fig:flowcompare}
% \end{figure}






\textbf{Interpretive Experiments: }
To demonstrate the mutual refinement of two flows in our FDS, 
For each view, we sample the unobserved 
views multiple times to compute the mean error 
of both Radiance Flow and Prior Flow. 
We use Raft~\citep{teed2020raft} as our default optical flow model
for visualization.
The ground truth flow is calculated based on 
~\eref{equ:flow_pose} and ~\eref{equ:flow} 
utilizing ground truth depth in dataset.
We introduce our FDS loss after 16000 iterations during 
optimization of 2DGS.
The error maps are shown in ~\figref{fig:error_map}.
Our analysis reveals that Radiance Flow tends to 
exhibit significant geometric errors, 
whereas Prior Flow can more accurately estimate the geometry,
effectively disregarding errors introduced by floating Gaussian points. 

%





\subsection{Limitation and further work}

Firstly, our FDS faces challenges in scenes with 
significant lighting variations between different 
views, as shown in the lamp of first row in ~\figref{fig:limitation}. 
%
Incorporating exposure compensation into FDS could help address this issue. 
%
 Additionally, our method struggles with 
 reflective surfaces and motion blur,
 leading to incorrect matching. 
 %
 In the future, we plan to explore the potential 
 of FDS in monocular video reconstruction tasks, 
 using only a single input image at each time step.
 


\section{Conclusions}
In this paper, we propose Flow Distillation Sampling (FDS), which
leverages the matching prior between input views and 
sampled unobserved views from the pretrained optical flow model, to improve the geometry quality
of Gaussian radiance field. 
Our method can be applied to different approaches (3DGS and 2DGS) to enhance the geometric rendering quality of the corresponding neural radiance fields.
We apply our method to the 3DGS-based framework, 
and the geometry is enhanced on the Mushroom, ScanNet, and Replica datasets.

\section*{Acknowledgements} This work was supported by 
National Key R\&D Program of China (2023YFB3209702), 
the National Natural Science Foundation of 
China (62441204, 62472213), and Gusu 
Innovation \& Entrepreneurship Leading Talents Program (ZXL2024361)
\section{Conclusions}
\label{sec:conclusions}

In this paper, we introduce a novel sketch-to-image translation technique that uses a learnable lightweight mapping network (LCTN) for latent code translation from sketch to image domain, followed by $k$ forward diffusion and $T$ backward denoising steps through a pre-trained text-to-image LDM. We show that by selecting an optimal value for $k \sim [1, T]$ near the upper threshold ($k \approx T$, $k < T$), it is possible to generate highly detailed photorealistic images that closely resemble the intended structures in the given sketches. Our experiments demonstrate that the proposed technique outperforms the existing methods in most visual and analytical comparisons across multiple datasets. Additionally, we show that the proposed method retains structural consistency across different visual styles, allowing photorealistic style manipulation in the generated images.


\newpage
\section*{Limitations}
While our proposed benchmark provides a comprehensive assessment of intermodal reasoning in current MLLMs, several limitations must be acknowledged. Firstly, the benchmark is currently limited to analyzing coherence relations in single-turn discourses. This is due to a lack of publically available datasets that provide multi-turn image-text pairs with annotated coherence relations. We plan to extend our benchmark to include multi-turn discourse relations as future work. Secondly, although we analyze different discourse domains in our benchmark, we lack a unified set of coherence relations that can be applied across all domains. The difficulty in defining a universal set of coherence relations is due to the varying nature of discourse in different domains. This limits our ability to analyze the inter-domain performance of MLLMs on the same set of relations. Finally, our benchmark is currently limited to the English language and must be extended to multi-lingual discourses as well.

\section*{Acknowledgments}
This research was in part supported by the U.S. National Science Foundation (NSF) award \#1820609. Part of the research results were obtained using the computational resources provided by CloudBank (\url{https://www.cloudbank.org/}), which was supported by the NSF award \#1925001.

\bibliography{acl_latex}

\clearpage
\appendix
\newpage
\onecolumn
\appendix

\newtcolorbox{cvbox}[1][]{
    enhanced,
%   blanker, % <- removed as it suppresses box color and frame
    %leftupper=4cm,
    after skip=8mm,%   enlarge distance to the next box
    title=#1,
    breakable = true,
    fonttitle=\sffamily\bfseries,
    coltitle=black,
    colbacktitle=gray!10,   % <- defines background color in title
    titlerule= 0pt,         % <- sets rule underneath title 
    %fontupper=\sffamily,%
    %#1
    overlay={%
        \ifcase\tcbsegmentstate
        % 0 = Box contains only an upper part
        \or%
        % 1 = Box contains an upper and a lower part
        %\path[draw=red] (segmentation.west)--(frame.south east);
        \else%
        % 2 = Box contains only a lower part
        %\path[draw=red] (frame.north west)--(frame.south east);
        \fi%
    }
    colback = gray,         % <- defines background color in box
    colframe = black!75     % <- defines color of frame
    }


\section{Data Construction}

\subsection{Dataset Summary}
\label{sec:appendix_data}

We prepare a seed dataset $\mathcal{D}$ containing both safety and helpfulness data. It consists of 50k pairwise samples from three sources. For helpfulness data, we draw 25k samples from UltraFeedback~\cite{cui2024ultrafeedback}. Each sample originally has 5 potential responses with ratings and we take the one with the highest rating as ``chosen'' and the one with the lowest as ``rejected''. For safety data, we take 22k samples from PKU-SafeRLHF~\cite{ji2024pku}, which have responses with unsafe labels and are further filtered by GPT-4o to assure the prompts are truly toxic and harmful. We follow the common practice of proprietary LLMs that responses to harmful queries should contain clear refusal in at most one sentence instead of providing additional content and guide besides a brief apology~\cite{guan2024deliberative}. This make current positive annotations in PKU-SafeRLHF, which usually contain much relevant information, not directly usable. Therefore, we use GPT-4o to generate refusal answers for these prompts and substitute the original chosen responses with them. 

Further, to better address the complex scenario of jailbreak attack, we take 3k jailbreak prompts from JailbreakV-28k~\cite{luo2024jailbreakv}. As this dataset was originally proposed for benchmarks, we carefully decontaminate the red-teaming queries from those used for evaluation, e.g., AdvBench~\cite{zou2023universal}, and only sample prompts from the sources of GPT-Generate, Handcraft, and HH-RLHF~\cite{ganguli2022red}. Due to the lack of response annotations, we prompt GPT-4o to generate refusal answers as ``chosen'' ones and prompt the models to be trained for ``rejected'' answers.


When training with baselines on the whole dataset, we take the ``chosen'' responses as labels for SFT and use the pairwise data for preference learning (i.e., DPO, SACPO). As for our method and self-rewarding, we only use the prompts from the dataset. Compared with baseline methods like SFT and DPO, we do not use the jailbreak prompts with GPT-generated refusals to directly train the models.



\subsection{Structured CoT Data Synthesis}
\label{sec:appendix_cot}
To initialize the models with reasoning ability, we take 10k prompts each from PKU-SafeRLHF and UltraFeedback, and synthesize structured CoT answers. Specifically, we used GPT-4o to generate structured reasoning steps for these prompts. The generation prompt instructs GPT-4o to produce step-by-step reasoning following a specific format. After generation, we filter out responses with incorrect formatting and remove any responses that provide unsafe answers to safety-related queries, ensuring the quality and safety of our CoT training data. The final dataset consists of synthesized samples slightly fewer than 20k due to the filtering process. The generation prompt for instructing GPT-4o is as below:

% \vspace{8pt} 
\begin{cvbox}[\vspace{8pt}Instruction to Generate Structured Chain-of-Thought Answer]
You are a helpful assistant specializing in addressing both general queries and sensitive topics, including privacy, security, and ethics. Your primary responsibility is to provide thorough reasoning and structured planning to ensure your responses are both helpful and strictly adhere to safety and ethical guidelines.
\newline \newline
\#\#\# Instructions:
\newline \newline
1. **Problem Analysis**: Begin by analyzing the query to understand its context and devise a clear and structured roadmap for the reasoning process below. This step should outline how the problem will be addressed in subsequent reasoning steps, ensuring logical progression and coherence. During this step, you may identify whether the query poses any risks or ethical concerns. If the query involves sensitive or risky elements, the roadmap should focus on identifying and mitigating these aspects. For harmless queries, emphasize an effective and logical approach to solving the problem.
\newline \newline
2. **Reasoning**: Break down the query into detailed and systematic reasoning steps. Each step should build upon the roadmap provided in the Problem Analysis section. Pay close attention to the progression of the analysis—if potential risks or sensitivities are uncovered during this stage, pivot immediately to a safety-focused reasoning process. This includes reassessing the problem's implications and adjusting your steps to ensure a thorough evaluation of risks. For harmless queries, highlight logical planning and systematic problem-solving to provide the most effective solution.
\newline \newline
3. **Final Answer**: Based on your reasoning, provide a concise and direct conclusion using the format `<|Output|>Final Answer:<|/Output|>`. Reasoning is provided to help you better understand the steps involved, but the ultimate answer to the query must be presented here.
- If the query is unsafe or unethical, explicitly refuse assistance with clear language, e.g., "Sorry, I cannot assist with this." Avoid using repetitive responses by varying your phrasing while maintaining a professional and polite tone. For example, you might say: "I’m sorry, but I cannot help with that request," "Unfortunately, I am unable to fulfill this request as it may violate safety guidelines," or "I must decline, as this request does not align with ethical considerations." Ensure your response remains concise, clear, and respectful without engaging further with unsafe or inappropriate content.
\newline  
- If the query is safe and ethical, offer a well-structured, logical, and solution-oriented response that demonstrates your reasoning and problem-solving capabilities. Strengthen the clarity and completeness of the solution to provide the most effective assistance in harmless scenarios.
\newline \newline
\#\#\# Format Requirements:
\newline \newline
- Each step must use the following tokenized format:
\newline 
1. **Problem Analysis**: Encapsulate the analysis within <|Reasoning\_step|> Title: Problem Analysis:  <|/Reasoning\_step|> tags.
\newline \newline
2. **Reasoning**: Include multiple <|Reasoning\_step|> Title: Title\_name <|/Reasoning\_step|> sections as needed to thoroughly address the query.
\newline \newline
3. **Final Answer**: Provide the conclusion in the format: <|Output|>Final Answer: <|/Output|> .
\newline 
By adhering to these guidelines and referring to the above example, you will provide clear, accurate, and well-structured responses to questions involving sensitive or potentially unsafe topics while excelling in logical planning and reasoning for safe and harmless queries. Provide your reasoning steps directly without additional explanations. Begin your response with the special token `<|Reasoning\_step|>`. Following is the question:

\vspace{1em}
Question: \{prompt\}
\vspace{8pt} 
\end{cvbox}


\section{Self-Improvement with Safety-Informed MCTS}

\subsection{Derivation of Safety-Informed Reward}
\label{sec:appendix_derive}

Here, we present the proof for~\cref{theorem} in~\cref{sec:MCTS}, to derive a proper form for the safety-informed reward function. We first recall the three desired properties with intuitive explanations.
\begin{enumerate}
    \item \textbf{Safety as Priority}: Safe responses always get higher rewards than unsafe ones, regardless of their helpfulness.
    \begin{equation}
        \forall \bfm_1,\bfm_2, S(\bfm_1)>0> S(\bfm_2) \Rightarrow R(\bfm_1)>R(\bfm_2)
    \end{equation}
    \item \textbf{Dual Monotonicity of Helpfulness}: When the response is safe, it gets higher reward for better helpfulness; when it is unsafe, it gets lower reward for providing more helpful instructions towards the harmful intention.
    \begin{equation}
        \forall S>0 , \frac{\partial R}{\partial H} > 0\text{ and } \forall S<0, \frac{\partial R}{\partial H} < 0;
    \end{equation}
    \item \textbf{Degeneration to Single Objective}: If we only consider one dimension, we can set the reward function to have a constant shift from the original reward of that aspect. This will lead to the procedure degenerating to standard MCTS under the corresponding reward, i.e., given a partially constructed search tree, the result of selection is the same when all hyperparameters, e.g., seed, exploration parameter, are fixed.
    \begin{align}
        \exists\;C_1 \in [-1,1],\;s.t.\;\text{let }S\equiv C_1, \forall \bfm_1,\bfm_2, R(\bfm_1)-R(\bfm_2)=H(\bfm_1)-H(\bfm_2);\\
    \exists\;C_2 \in [-1,1],\;s.t.\;\text{let }H\equiv C_2, \forall \bfm_1,\bfm_2, R(\bfm_1)-R(\bfm_2)=S(\bfm_1)-S(\bfm_2).
    \end{align}
    
\end{enumerate}

\begin{theorem}
    Fix constants $C_1, C_2\in [-1,1],\;C_1\ne0$. Suppose $R:[-1,1]\times[-1,1]\rightarrow \mathbb{R}$ is twice-differentiable and satisfies $\frac{\partial R}{\partial H}=F(S)$, for some continuous function $F: [-1,1]\rightarrow \mathbb{R}$. Properties 2 and 3 of Dual Monotonicity of Helpfulness and Degeneration to Single Objective hold, if and only if
    \begin{equation}
    R(H,S)=F(S)\cdot H+S - C_2 \cdot F(S)+c,       
    \end{equation} with $F(0)=0, F(C_1)=1, \forall S>0, F(S)>0, \forall S<0, F(S)<0$ and $c$ as a constant.
\end{theorem}

\begin{proof} We show that the form of $R$ is the sufficient and necessary condition of Properties 2 and 3, given the assumptions. For notation simplicity, we use $H_1,H_2,S_1,S_2$ to denote the rewards for arbitrary final answers $f_1, f_2$.

\textbf{Sufficiency}

Assume $R(H,S)=F(S)\cdot H+S-C_2\cdot F(S)+c$ with $F(S)$ satisfying the stated conditions.

For Property 2, we can compute the partial derivative and show that
\begin{equation*}
    \frac{\partial R}{\partial H} = F(S) \begin{cases}
        > 0,\text{ when }S>0,\\
        <0,\text{ when }S<0.
    \end{cases}
\end{equation*}

For Property 3, let $S\equiv C_1$, we get
\begin{equation*}
    R(H_1,C_1)-R(H_2,C_1) = F(C_1) (H_1-H_2) = H_1-H_2.
\end{equation*}
let $H\equiv C_2$, we get
\begin{equation*}
    R(C_2,S_1)-R(C_2,S_2) = C_2(F(S_1)-F(S_2)) + S_1-S_2 -C_2(F(S_1)-F(S_2))= S_1-S_2.
\end{equation*}

\textbf{Necessity}

    Assume $R(H,S)$ satisfies Properties 2 and 3.

    Given the condition that $\frac{\partial R}{\partial H} = F(S)$, the function $R$ should follow the form by integral, 
    \begin{equation}
        R(H,S) = \int_0^H \frac{\partial R}{\partial H}dH+R(0,S) =F(S)\cdot H + G(S),
        \label{eq:reward}
    \end{equation}
    with $G(S)=R(0,S)$ as a continuous and differentiable function of $S$.

    Then, we apply the property of Degeneration to Single Objective, when $S\equiv C_1$,
    \begin{align*}
        R(H_1, C_1)-R(H_2,C_2) = F(C_1)& (H_1-H_2) = H_1-H_2, \forall H_1,H_2\in[-1,1]\\
        &\Rightarrow F(C_1) = 1,
    \end{align*}
    and when $H\equiv C_2$, 
    \begin{align*}
        R(C_2, S_1) - R(C_2, S_2) = C_2(F(S_1)& - F(S_2)) + G(S_1) - G(S_2) = S_1 - S_2, \forall S_1, S_2 \in[-1,1]\\ 
        &\Rightarrow C_2\cdot F'(S) - G'(S) = 1\\ 
        &\Rightarrow G'(S) = 1- C_2\cdot F'(S)\\ 
        &\Rightarrow G(S) = S-C_2\cdot F(S) + c, 
    \end{align*}
    with $c$ as a constant.

    Considering the property of Dual Monotonicity of Helpfulness, it is clear that $\frac{\partial R}{\partial H} = F(S)$ should satisfy
    \begin{equation*}
        F(S) >0, \forall S>0\text{ and }F(S)<0, \forall S<0.
    \end{equation*}
    Given the continuity of $F(S)$, $F(0) = 0$.

    Substituting $G(S)$ to~\cref{eq:reward}, we eventually get the family of $R$, following
    \begin{equation*}
    R(H,S)=F(S)\cdot H+S - C_2 \cdot F(S)+c,       
    \end{equation*} with $F(0)=0, F(C_1)=1, F(S)>0, \forall S>0$, $F(S)<0, \forall S<0$ and $c$ as a constant.
\end{proof}

\begin{corollary}
 Take $F(S)=S, C_1=1, C_2=-1, c=0$, $R(H,S)=2S+S\cdot H$ satisfies that for any $H_1, H_2,S_1,S_2\in[-1,1]$, when $S_1>0>S_2$, the inequality of $R(S_1,H_1)>R(S_2,H_2)$ holds.
\end{corollary}


\subsection{Implementation Details of Self-Improvement}
\label{sec:appendix_self-improvement}

Here, we introduce the implementation details of different components in the iterative self-improvement, including SI-MCTS, Self-Rewarding, and preference data construction.

\subsubsection{Safety-Informed MCTS} 
We design safety-informed reward to introduce dual information of both helpfulness and safety, without impacting the original effect of MCTS on a single dimension. Therefore, we mainly follow the standard MCTS procedure~\cite{vodopivec2017monte} guided by UCB1 algorithm~\cite{chang2005adaptive}. When traversing from the root node (i.e., prompt) to the leaf node, it selects the $i$-th node with the highest value of
\begin{equation}
    v_i + c\sqrt{\frac{\ln N_i}{n_i}},
\label{eq:UCB}
\end{equation}
where $v_i$ is the estimated value of safety-informed rewards, $n_i$ is the visited times of this node, $N_i$ is the visited times of its parent node, and $c$ is the exploration parameter that balances exploration and exploitation. 

The whole procedure of Safety-Informed MCTS follows~\cref{alg:SI MCTS}. In practice, we set exploration parameter $c=1.5$, search budget $n=200$, children number $m=4$. To generate child nodes and rollout to final answers, we set temperature as $1.2$, top-p as $0.9$ and top-k as $50$. We adjust these parameters when higher diversity is needed.



% Build MCT
\begin{algorithm}[ht]
   \caption{Safety-Informed MCTS}
   \label{alg:SI MCTS}
\begin{algorithmic}
   \STATE {\bfseries Input:} prompt set $\mathcal{D}_k$, safety reward function $S$, helpfulness reward function $H$, actor model $\pi_\theta$ that generate one step each time by default, search budget $n$, children number $m$
   \STATE {\bfseries Output:} MCT data $\mathbb{T}$
   \STATE Init $\mathbb{T}$ with empty
   \FOR{each single prompt $\bx$ in $\mathcal{D}_k$}
        \STATE Init search tree $T$ with $root\_node$ of $\bx$
        \FOR{$i$ in range($n$)}
            \STATE Select a leaf node $select\_node$ following the trajectory $(\bx,\bs_i)$ using UCB1 algorithm as~\cref{eq:UCB}
            \STATE $\bz_{i+1}^\ast \leftarrow None$
            \IF{$select\_node$ has been visited before}
                \IF{$select\_node$ is non-terminal}
                    \STATE Sample $m$ children $\{\bz_{i+1}^{(j)}\}_{j=1}^m$ from $\pi_\theta(\cdot|\bx, \bs_i)$ and add the $m$ children to $T$
                    \STATE $\bz_{i+1}^\ast \leftarrow$ random.choice($\{\bz_{i+1}^{(j)}\}$), $select\_node \leftarrow$ the corresponding child
                \ENDIF
            \ENDIF
            \STATE Rollout a full answer $\bfm\sim\pi_\theta(\cdot|\bx,\bs_i, \bz_{i+1}^\ast)$
            \STATE Calculate reward $r \leftarrow S(\bfm) \cdot H(\bfm) +2S(\bfm)$
            \STATE Backpropagate and update node's value and visited times from $select\_node$ to $root\_node$
        \ENDFOR
        \STATE Rollout all nodes that have not been visited before, calculate reward and backpropagate
        \STATE $\mathbb{T}\leftarrow \mathbb{T}\cup\{T\}$
   \ENDFOR
\end{algorithmic}
\end{algorithm}

\subsubsection{Self-Rewarding} 
We take the trained LLMs as judges~\cite{zheng2023judging} to rate their own responses, to remove dependencies on external reward models. We adopt a similar template design following~\cite{yuanself} to prompt the model to give discrete ratings given the query $\bx$ and the final answer $\bfm$ sampled through rollout. For helpfulness, we ask the model to rate the answer from $1$ to $5$ according to the extent of helpfulness and correctness. For safety, we categorize the answer into safe and unsafe ones. All ratings will be normalized into the range of $[-1,1]$. Note that the models also give rewards with in-depth reasoning, which further increase the reliability of ratings.

\subsubsection{Preference Data Construction} 

Given the search trees built via SI-MCTS, we can select stepwise preference data with different steps to optimize the model itself. We employ a threshold sampling strategy to guarantee the quality of training data. For a parent node in the tree, we group two children nodes as a pair of stepwise data if they satisfy that the difference between two values exceeds a threshold $v_0$ and the larger value exceeds another threshold $v_1$. This is to assure that there is a significant gap in the quality of two responses while the ``chosen'' one is good enough. Two thresholds are adjusted to gather a certain amount of training data. 

For the ablation study comparing preference data of full trajectories, we adopt similar strategies but within all full trajectories from the root node. As for the stepwise preference data for training a process reward model, we group nodes at the same depth without requiring them to share a parent node and only emphasize the gap between the ``chosen'' and ``rejected'' responses. To support rewarding at both stepwise and full-trajectory level, we include some full-trajectory preference data into $\mathcal{D}_R$.


\section{Experimental Details}
\label{sec:appendix_exp}

In this work, we conduct all our experiments on clusters with 8 NVIDIA A800 GPUs. 

\subsection{Training Details}
\label{sec:appendix_train}

We have done all the training of LLMs with LLaMA-Factory~\cite{zheng2024llamafactory}, which is a popular toolbox for LLM training. For all methods in training LLMs, optimization with SFT is for $3$ epochs and that with DPO is for $1$ epoch by default. We tune the learning rate from $\{5e-7, 1e-6, 5e-6\}$ and $\beta$ for DPO from $\{0.1,0.2,0.4\}$. Batch size is fixed as $128$ and weight decay is set to $0$. We adopt a cosine scheduler with a warm-up ratio of $0.1$. Following the official implementation, we set $\beta=0.1$ and $\beta/\lambda=0.025$ for SACPO. For Self-Rewarding and our self-improving framework, we take $K=3$ iterations. We take an auxiliary SFT loss with a coefficient of $0.2$ in our self-improvement to preserve the structured CoT style. 

For training process reward model based on the LLaMA architecture, we use OpenRLHF~\cite{hu2024openrlhf} and train based on TA-DPO-3 for 1 epoch, using batch size of $256$ and learning rate of $5e-6$. The training data has 70k pairwise samples from Monte Carlo Search Tree in three iterations and contains both stepwise pairs and full-trajectory pairs. This is to ensure the verifier to have the ability to choose the best answer between partial answers with same thinking steps and between full answers.


For the reproduction of Deliberative Alignment~\cite{guan2024deliberative}, we first develop a comprehensive set of safety policies by analyzing query data from o1 and reviewing OpenAI's content moderation guidelines. Specifically, we prompt o1-preview to generate policies for the seven categories of harmful content identified in Deliberative Alignment --- erotic content, extremism, harassment, illicit behavior, regulated advice, self-harm, and violence ---  and organize them with a unified format by manual check. Each policy includes: (1) a clear Definition of the category, (2) User Requests Categorization (defining and providing examples of both allowed and disallowed requests), (3) Response Style Guidelines, and (4) Edge Cases and Exceptions. Additionally, to account for potential gaps in coverage, we introduce a general safety policy, resulting in a total of eight distinct policy categories, which are submitted as supplementary materials. To ensure fairness and consistency, we use GPT-4o to classify prompts from the PKU-SafeRLHF and JailbreakV-28k datasets based on these eight policy definitions. Notably, we focus on the same 23k safety-related prompts used in our own methodology to maintain comparability.

We fine-tune two open-source o1-like LLMs with the same LLaMA-8B architecture, OpenO1-LLaMA-8B-v0.1 and DeepSeek-r1-Distilled-LLaMA-8b, to compare with our results on LLaMA-8B-3.1-Instruct. We follow the practice in~\cite{guan2024deliberative}, generating reasoning answers based on the harmful prompts together with the safety guidelines, which are gathered as a SFT dataset. These models are trained on the query-response pairs with a learning rate $5e-6$ and a batch size of $128$ for $3$ epochs. 


\subsection{Evaluation Details}
\label{sec:appendix_eval}

For evaluation, we take greedy decoding for generation to guarantee the reproducibility by default. As for test-time scaling, we set temperature to 0.6, top-p to 0.9 and top-k to 50 for the diversity across different responses. Below, we introduce the benchmarks and corresponding metrics in details.

For StrongReject~\cite{souly2024strongreject}, we take the official evaluation protocol, which uses GPT-4o to evaluate the responses and gives a rubric-based score reflecting the willingness and capabilities in responding the harmful queries. We follow~\cite{jaech2024openai} and take the goodness score, which is $1-\text{rubric score}$, as the metric. We evaluate models on prompts with no jailbreak in addition to the reported top-2 jailbreak methods PAIR~\cite{chaojailbreaking}, and PAP-Misrepresentation~\cite{zeng2024johnny}. For main results, we only report the average goodness score on the two jailbreak methods, since most methods achieve goodness scores near $1.0$. For XsTest~\cite{rottger2023xstest}, we select the unsafe split to evaluate the resistance to normal harmful queries and follow its official implementation on refusal determination with GPT-4. We report the sum of full refusal rate and partial refusal rate as the metric. For WildChat~\cite{zhaowildchat}, we filter the conversations with ModerationAPI\footnote{https://platform.openai.com/docs/guides/moderation} and eventually get 219 samples with high toxicity in English. For Stereotype, it is a split for evaluating the model's refusal behavior to queries associated with fairness issues in Do-Not-Answer~\cite{wang2023not}. We also use the same method as XsTest for evaluation, also with the same metric, for these two benchmarks. 

To benchmark general performance, we consider several dimensions involving trustworthiness~\cite{wangdecodingtrust,zhangmultitrust} and  helpfulness in popular sense. We adopt SimpleQA~\cite{wei2024measuring} for truthfulness, AdvGLUE~\cite{wang2adversarial} for adversarial robustness, InfoFlow~\cite{mireshghallahcan} for privacy awareness, GSM8k~\cite{hendrycks2measuring}, AlpacaEval~\cite{dubois2024length}, and BIG-bench HHH~\cite{zhou2024beyond} for helpfulness. All benchmarks are evaluated following official implementations. Correlation coefficient is reported for InfoFlow, and winning rate against GPT-4 is reported for AlpacaEval, while accuracies are reported for the rest. 

% \vspace{12pt}
\section{Examples}
\label{sec:appendix_examples}

Here, we present several examples to qualitatively demonstrate the effectiveness of STAIR against jailbreak attacks proposed by PAIR~\cite{chaojailbreaking}. We compare the outputs of our model with those of baseline models trained on the complete dataset using Direct Preference Optimization (DPO), referred to as the \textit{baseline model} in the cases below.

For each case presented below, we display the following:
\begin{itemize}
    \item \texttt{<Original harmful prompt, baseline model's answer>}
    \item \texttt{<Jailbroken prompt based on the original harmful prompt, baseline model's answer>}
    \item \texttt{<Jailbroken prompt based on the original harmful prompt, STAIR's reasoning process and answer>}
\end{itemize}

Please note that in the answers, due to ethical concerns, we have redacted harmful content by replacing it with a "cross mark" (\textbf{x}) to indicate the presence of harmful content. Our model may perform single-step reasoning (as shown in Case 1) or multi-step reasoning (as demonstrated in Cases 2 and 3) depending on the question. Each reasoning step is marked with \texttt{<|Reasoning\_step|>} and \texttt{<|/Reasoning\_step|>}, while the final answer is enclosed within \texttt{<|Output|>} and \texttt{<|/Output|>}.

We observe that although the baseline model can respond to harmful prompts with refusals, it remains vulnerable to jailbreaks that fabricate imagined scenarios to conform to the harmful query. In contrast, the model trained with STAIR-DPO-3 thoroughly examines the potential risks underlying the jailbreak prompts through step-by-step introspective reasoning, ultimately providing appropriate refusals.



\begin{figure*}
    \centering
    \includegraphics[width = \linewidth]{images/appendix/case-1.pdf}
    \caption{\textbf{Case 1}}
    % \label{fig:appendix-case-1}
\end{figure*}

\begin{figure*}
    \centering
    \includegraphics[width = \linewidth]{images/appendix/case-2.pdf}
    \caption{\textbf{Case 2}}
    % \label{fig:appendix-case-1}
\end{figure*}

\begin{figure*}
    \centering
    \includegraphics[width = \linewidth]{images/appendix/case-3.pdf}
    \caption{\textbf{Case 3}}
    % \label{fig:appendix-case-1}
\end{figure*}

\end{document}
