\section{Implementation and Experiments}\label{sec:results}
\begin{figure*}[t]
 \centering
  \includegraphics[width=\linewidth]{figures/bunny-fig} 
 \caption{Starting with a reference bunny (blue) which has a trajectory going into a bin, we construct an adversarial bunny (pink). A greater portion of the kinetic energy in the adversarial case is rotational, causing it to travel less distance and instead bounce off the rim of the bin.} 
 \label{fig:bunny-example}
\end{figure*}
%
\begin{table*}[t]
\caption{Comparison of reference and adversarial moments of mass for all of our examples. In all cases, the moments closely match.}
\label{table:moments-comparison}
\rowcolors{2}{white}{cvprblue!25}
\resizebox{\textwidth}{!}{%
\begin{tabular}{|lrrrrrrrrrr|}
    \hline
    \textit{Object} & \textbf{$m_0$} & \textbf{$m_{1,x}$} & \textbf{$m_{1,y}$} & \textbf{$m_{1,z}$} & \textbf{$m_{2,xx}$} & \textbf{$m_{2,yy}$} & \textbf{$m_{2,zz}$} & \textbf{$m_{2,xy}$}  & \textbf{$m_{2,xz}$} & \textbf{$m_{2,yz}$} \\
    Ball (ref) & 1.825e+01 & -3.215e-05 & -4.682e-05 & 3.930e-05 & 1.058e-01 & 1.059e-01 &  1.055e-01 &  7.475e-08 & -1.050e-06 &  2.079e-06 \\
    Ball (adv) & 1.825e+01 & -3.215e-05 & -4.682e-05 &  3.930e-05 &  1.058e-01 &1.059e-01 &  1.055e-01 &  7.462e-08 & -1.050e-06 &  2.080e-06 \\
    Star (ref) & 8.811e+02 &  3.314e-03 & -3.201e-03 &  1.016e-02 &  2.178e+02 &1.155e+02 & 1.155e+02 &-5.076e-04 & 1.560e-03 &-4.151e-03 \\
    Star (adv) & 8.811e+02 &  3.311e-03 & -3.200e-03 & 1.016e-02 & 2.178e+02 &1.155e+02 & 1.155e+02 & -5.259e-04 &  1.565e-03 & -4.142e-03 \\
    Bunny (ref) & 3.901e+03& -2.403e-13& 3.411e-13&-4.263e-13&1.382e+03&8.974e+02& 1.028e+03&9.705e-01&  4.553e+01& 2.682e+02 \\
    Bunny (adv) & 3.901e+03&-1.920e-07& 2.939e-08& 2.131e-06&  1.382e+03&8.974e+02&  1.028e+03& 9.705e-01& 4.553e+01&  2.682e+02 \\
    Cube (ref) & 3.124e-01 & 9.922e-07 &  3.564e-07 & 2.571e-07 & 1.301e-04 &1.301e-04 & 1.301e-04 & 1.164e-08 & 3.443e-08 &-1.833e-08 \\
    Cube (adv) & 3.124e-01&  9.922e-07&  3.565e-07&  2.571e-07& 1.301e-04&1.301e-04& 1.301e-04&  1.164e-08& 3.443e-08& -1.833e-08 \\
    Bat (ref) & 4.132e+00 & 3.082e-05 & -2.229e-05 & 2.287e+00 & 1.453e+00&1.453e+00&  1.713e-03& 2.258e-07& -1.887e-05&  2.489e-05 \\
    Bat(adv) & 4.132e+00 &-8.139e-05 & -4.273e-05 &  2.287e+00& 1.453e+00&1.453e+00&  1.930e-03& 3.406e-06& -2.030e-05&  2.410e-05 \\
    \hline
\end{tabular}}
\end{table*}

We implement our simulation in \textsc{Python}, using \textsc{PyTorch}  \cite{pytorch-citation} as our autodifferentiation framework. We use the sparse direct solver \textsc{Cholespy} \cite{inverse-rendering-geometry} for all of our linear system solves. We use \textsc{libigl} \cite{libigl} for all necessary geometry processing, and \textsc{Blender} and \textsc{Polyscope} \cite{polyscope} for visualization.
Our simulations are run with a timestep of 0.01 seconds, and an IPC barrier distance of $10^{-3}$ unless stated otherwise. We run all of our experiments on CPU on a 2021 M1 Macbook Pro. We demonstrate the power of our method by conducting black box attacks, where our rigid reference objects are simulated in \textsc{Bullet} \cite{pybullet}, and adversarial objects are constructed using our custom simulator and demonstrated on \textsc{Polyfem} \cite{polyfem}. 

Unless noted otherwise, we use following bounds on material parameters for all examples. The Young's modulus is allowed to range from 2.5 GPa to 650 GPa which corresponds respectively to ABS plastic and tungsten carbide. The Poisson's ratio is allowed to range from 0.2 to 0.4 which corresponds to most metals and plastics. The mass density is allowed to range from 0.8 g/cc to 11.3 g/cc, which corresponds respectively to PMP plastic and lead. 

We find that due to the chaotic nature of contact, it is possible to successfully create adversarial results even with very simple object and contact geometries. 
In Fig. \ref{fig:teaser}, we construct an adversarial ball from just the first contact off the backboard, and the resulting trajectory difference is greatly amplified by the subsequent contacts off the rim. Similarly, in Fig. \ref{fig:bunny-example}, we construct an adversarial bunny from the first collision off the ground, such that it has a different center of mass trajectory than its reference, with more energy being stored in rotational kinetic energy. This trajectory difference is sufficient for the bunny to bounce off the rim of a bin rather than fall into it.
In Fig. \ref{fig:base-example}, we have a more complex geometry, with a star bouncing off of two perpendicular planes. In this case, our optimized optimization includes contact off of both planes. The star example is run with a timestep of 1/30 seconds, and the minimum Young's modulus is raised to 5GPa. Due to the thin shape of the star, we only optimize the Young's modulus, Poisson's ratio, and density; we leave the material occupancy fixed.

In Fig. \ref{fig:stacking}, we show a more complex scenario with a cube striking a stack of three other cubes with a timestep of 1e-3 seconds. Using reference blocks, the stack remains upright post collision, while with adversarial blocks, it falls over. The adversarial cube is created from planar impacts from above and below. Even though our deformable simulation is frictionless, these adversarial cubes are successful at attacking a simulator with a friction coefficient of 0.4.

Thus far, all examples have been undirected attacks. In Fig. \ref{fig:baseball-example}, we change our cost function to demonstrate a directed attack, where an adversarial bat is constructed specifically so that the trajectory of the struck ball is as close to the center as possible causing a significant angular deviation from the reference.

For all of these examples, the comparison of their moments of mass and that of the corresponding reference object is given in Table \ref{table:moments-comparison}. For all reference objects, we determine moments by using a constant mass density of 2.5g/cc. We provide further experimental details in the Supplemental, as well as baseline comparisons with simulations of stiff uniform material objects.