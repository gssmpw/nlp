\section{Conclusion and Future Work} \label{sec:conclusion}
In this paper we introduce the adversarial setting of rigid body simulators, and present a method of constructing perceptually rigid adversarial objects that will behave identically to a reference object in a rigid body simulation but maximally different in deformable simulations. We do so by implementing a differentiable deformable simulator, and defining a cost function that encodes both the difference in the trajectories of the adversarial and reference object as well as the first three moments of mass. We then minimize this cost using the ADAM algorithm. We have shown over several examples that there can be significant deviations in the trajectories even through simple contact. 
\revision{
While it is unknown how common these adversarial objects can accidentally occur or how hard physical adversarial objects would be to manufacture, knowing that they can happen at all is interesting and important in consideration of rigid body simulation tools being used in safety critical applications (e.g. robotics control in industrial settings).
}

The primary limitation of our method is that the materials parameters are allowed to continuously span some range. This means that even though the material parameters individually are in reasonable ranges, there is no guarantee that a constructed material will actually exist. In the future, it would be interesting to incorporate a materials library that can be populated with real world materials. By our choice of hyperelastic constitutive model, we are limited to isotropic materials. One direction of future work could be to investigate if we can cause even greater trajectory differences by allowing anisotropic materials or multiscale modelling, where even microstructures are considered in the optimization. Another limitation is that our method can occasionally result in disconnected pieces after rounding occupancies, which we currently fix as a postprocess (via \texttt{steiner\_tree} \cite{networkx}). We leave more sophisticated topological postprocesses (e.g., \cite{joining-cv}) or reparameterizations (e.g., \cite{fast-quasi-harmonic}) as future work. Our method currently considers a fully deterministic setting and a single trajectory. It would be interesting to study the construction and performance of adversarial objects in a stochastic setting.
%
\begin{figure}[t]
	\includegraphics[width=\linewidth]{figures/stacked-cubes}
	\vspace{-0.7cm}
 \caption{Here, we simulate a collision between a block and a stack of blocks. In the reference blocks (left), the stack is left intact after the collision. Using adversarial blocks (right), the stack falls down.} 
	 \label{fig:stacking}
\end{figure}
%

\revision{
Having identified a deficiency in current simulation techniques, it is interesting to consider how they can be improved. An immediate first approach could be using our adversarial objects to improve robustness of learned physics models through adversarial training.  
More broadly, stiff dynamics simulation is typically seen as a dichotomy between fast rigid body simulation, and very slow deformable simulation. In computer graphics and animation, there has been research in between those regimes, for example, elastodynamics modelling via adding visual effects to augment rigid body simulation (e.g. \cite{comp-dynamics}).
Our work suggests there could be value in studying more accurate methods for stiff materials. Recent work in this area includes \citet{bounce-map}, who incorporate spatially varying coefficient of restitution to rigid body simulation. Interesting future work in simulation could be to incorporate our adversarial objects to evaluate methods in this area of research with the ultimate goal of sufficiently performant and accurate simulation techniques for stiff objects that can be used in the current robotics and ML pipelines.
}

We hope that our work encourages members of the machine learning and robotics community to investigate whether their rigid body simulations are sufficiently accurate for their needs and further consider what steps can be taken to improve simulator robustness.%, and the broader research community to take a closer look at what model assumptions they may be making.
\begin{figure}[t]
	\includegraphics[width=\linewidth]{figures/baseball-fig}
	\vspace{-0.7cm}
 \caption{We construct an adversarial bat that aims to get the trajectory of the struck ball as close as possible towards the center (i.e. $x=0$). } 
	 \label{fig:baseball-example}
\end{figure}
