\section{Related Works}\label{sec:Related Works}
To construct adversarial objects, our work draws on ideas from inverse problems, machine learning, and simulation.

\subsection{Inverse Problems}
Consider a system whose behavior is governed by some set of parameters. In a \textit{forward problem}, the goal is to compute the evolution of that system given its parameters. In the corresponding \textit{inverse problem}, the goal is to determine the parameters of the system from observations of the evolution of the system - either from physical measurements or via simulation (see e.g. \cite{inverse-problems-book1, inverse-problems-book2}). 
\revision{In machine learning and vision, common inverse problems include inverse kinematics (e.g. \cite{invkin2, invkin3}) and imaging (e.g. \cite{invimage2, invimage3}), though there have been works that tackle more broad inverse problems in physics using invertible neural networks \cite{inverse-phys-inn} and approximate inverse simulation with a learned correction function \cite{inverse-phys-score-matching}}.

In the context of this work, the system we consider is the elastodynamics simulation of an object, and our inverse problem to determine its material parameters and internal geometry.
%In computer graphics, inverse problems appear both in animation (e.g., \cite{sag-free-deformable}) and in fabrication (e.g., \cite{perez2015}). 
In mechanics, \citet{XU2023115852} use physics informed neural networks to solve inverse problems with linear elastic and hyperelastic materials where they determine loads and internal pressures.
In computer animation, there has been some work in solving inverse problems to optimize the trajectories of objects. Approaches include determining control forces via sequential quadratic programming \cite{spacetime-constraints}, determining physical parameters (i.e. initial positions and velocities, surface normal variations, etc.) that minimize an energy subject to position constraints of an object at chosen times \cite{popovic2000interactive}, and simulating backwards using time-reversed simulators to determine initial conditions \cite{backward-rigid-body}. Unlike these prior works, we are optimizing for material properties rather than loads, control forces, or initial conditions.

Optimizing the internal geometry of our adversarial object leads us to the topology optimization inverse problem, where one finds the optimal allocation of a fixed amount of material over a design domain such that the compliance of the design is minimized when subjected to a load. There are a few broad approaches to topology optimization including nodal/element based design variables \cite{XIE1993885,simp-bendsoe}, level sets \cite{ALLAIRE20021125, WANG2003227}, and implicit neural representations \cite{ntopo}.
In our work, we follow the solid isotropic material with penalization parametrization (SIMP) approach of defining a per element occupancy and using a penalizing power-law to scale stiffness (see \cite{bendsoe-book}).
However, unlike the standard topology optimization problem, we are not optimizing for minimum compliance, do not have a fixed amount of material, have a continuous spread of allowed materials rather than one or a few, and furthermore, we are interested in optimizing over dynamic loads. 
Instead of a fixed amount of material, our constraint is matching moments of mass. This is a very similar problem to \citet{same-stats}, where the authors construct datasets of 2D points such that they have the same mean (first moment), standard deviation (second moment about the mean), and Pearson's R value to those of some reference dataset. By using a simulated annealing optimization scheme, they can direct the new dataset to be along a target shape. We instead leave our material density distribution as an untargeted optimization, where it may take any values such that the moments match (see Fig. \ref{fig:density-shapes}).
\begin{figure}[t!]
	\centering
	\includegraphics[width=\linewidth]{figures/density-shapes}
	\vspace{-0.7cm}
	\caption{For a given geometry, there can be many mass density distributions that have identical moments of mass. Above are three visually different density distributions on a mesh that have the same total mass, center of mass, and moment of inertia.}
	\label{fig:density-shapes}
\end{figure}

\subsection{Adversarial Attacks}
Our method is inspired by adversarial attacks in machine learning, where an adversarial input is constructed by causing small perturbations to a reference.
There has been considerable interest for adversarial attacks in various contexts and applications, including images \cite{Szegedy2014, goodfellow2014explaining, moosavi2016deepfool,kurakin2018adversarial, parametric-adversary-derek}, videos \cite{Li2018AdversarialPA,sparse-adv-video, linxi-video,Li2021AdversarialAO,10378215-kim}, audio \cite{smack-audio,Chen2019WhoIR,aaecaptcha,Liu2019WeightedSamplingAA}, \revision{geometric representations of 3D shapes \cite{lang2021geometric-adv, hu2023geometric-adv, Stolik2022SAGASA}}, 3D printed objects that fool classifiers \cite{pmlr-v80-athalye18b, Huang2023TowardsTT}, machine learning components of industrial control systems \cite{aa-erba, ANTHI2021102717, figueroa-ic}, and even control signals in robots \cite{otomo2022adversarial}.

Most relevant to our work is the intersection of physical inverse design problems and adversarial attacks. To our knowledge, the only study combining the two fields is \citet{azakami2022adversarial}, who perform an attack over the shape of legged robots that are simulated in MuJoCo. These robots have body components parameterized by their lengths and thicknesses. The authors use a differential evolution optimization scheme to find the parameters that are close to those of the original robot but induce a failure in the control, causing them to fall over. However, the parameter changes may cause the new design to have different centers of mass and broken symmetries.

In machine learning, adversarial attacks have led to adversarial training, where model robustness is improved by training on adversarial examples \cite{ijcai2021p591, QIAN2022108889}. We think our adversarial rigid objects can similarly be used for improved validation and training of learned physics simulation and robotics planning and control methods.

\subsection{Simulation}
Both rigid body (RBD) and deformable simulation techniques are popular in robotics \cite{liu2021role}.

\subsubsection{Rigid Body Simulation}
\revision{
Dynamic simulation of highly stiff objects is a common task across several fields. A rigid body is an idealized model, where the simulated object is treated as infinitely stiff - external stimulus like forces or impulses are propagated almost instantaneously across the object eliminating relative deformations \cite{affine-body-dynamics}. As a result, rigid body simulators require only per object rotational and translational degrees of freedom, rather than requiring degrees of freedom for each material point in the object to capture deformations. This multiple orders of magnitude reduction in degrees of freedom per object makes the rigid body model extremely performant and useful in a variety of interactive applications. Its speed also means that rigid body simulators are often used for simulating highly stiff objects in robotics and are used in robotics control \cite{posa-control, rigid-control-2, rigid-soft-robot}. 
}

\revision{Rigid body simulation has been used for modelling multi object scenes with collision since the late 1980's \cite{baraff1989rbd, hahn1988realistic, moore1988collision}, and a comprehensive survey on the topic can be found in \citet{rbd-survey}.}
\revision{The most involved process in rigid body simulation is in the handling of collision and contact.} Two main classes of rigid body simulators have emerged, some handling contact using penalties or barriers (e.g. \cite{RigidIPC, drumwright-rbd}), and others using constraints \revision{encoding non-penetration} formulated as linear complementarity or velocity-level approximate linear complementarity problems (e.g. \cite{erleben2007velocity, trinkle-rbd}).

\subsubsection{Deformable Simulation}
The simulation of deformable bodies has been extensively studied both in robotics and in the broader academic community \revision{through various methods including mass-spring systems, the finite element method (FEM), the boundary element method, and the material point method.}
We choose to use FEM for our deformable simulation, which is used to solve boundary value problems over potentially intricate domains by using a discretization into elements (in our case tetrahedra). For a detailed introduction, see \cite{fem-book, dynamic-deformables, sifakis2012fem}. In FEM, one can choose between different constitutive models for materials depending on the application - we choose the stable Neohookean energy \cite{dynamic-deformables, smith-kim-nh}. While FEM can be very accurate and is constantly being improved, it may still be too slow to be used in certain applications.

\subsubsection{Differentiable Simulation}
In order to solve the inverse problem using a descent method, we need to be able to calculate gradients through the simulation. Differentiable simulation is a powerful tool for physics based learning and control problems, and has picked up in prevalence both in the context of rigid bodies \cite{neuralsim, JADE, BelbutePeres2018EndtoEndDP, diffsdfsim} and deformable objects \cite{gradsim, du2021diffpd, qiao2021differentiable, gjoka2022differentiable, add-diff-phys, diffcloth}. We follow the approach taken by \citet{gradsim}, where we implement our physics simulator with an autodifferentiation framework to which we provide manual derivatives at each timestep.
