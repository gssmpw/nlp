\section{INTRODUCTION}

% Keep trajectory representation polynomial and avoid using non-linear optimization 
% How to handle uncertainty? \cite{wang2023polynomial}

Rapid online trajectory adaptation is essential for deploying agile autonomous vehicles in dynamic environments. 
This capability is crucial in real-world scenarios, where unseen obstacles may appear suddenly and target positions can change unexpectedly. 
In these situations, the planning algorithm must swiftly adapt the trajectory to accommodate unforeseen obstacles or varying target points. 
This is challenging since the algorithm should efficiently handle the two critical components—obstacle avoidance and dynamic feasibility—with minimal computation to ensure quick adaptation.

\begin{figure}[tb]
    \centering
    \includegraphics[width=0.48\textwidth,trim=0.1cm 0.1cm 0.1cm 0.1cm,clip]{figures/algorithm_overview.png}
    \vspace{-1\baselineskip}
    \caption{Overview of the proposed algorithm: 
    (top left) simulation environment for drone interception experiments;
    (top right) generation of the point-mass path towards target drone position candidates; 
    (bottom left) extraction of waypoints along these candidate paths; 
    (bottom right) parallel optimization of the trajectories along the waypoints.}
    \label{fig:alg_overview}
    \vspace{-0.5\baselineskip}
\end{figure}

% Existing path-finding methods, such as graph traversal algorithms or sampling-based planning, can be used to find the obstacle-avoiding path.
% These methods are capable of efficiently navigating complex environments and generating collision-free routes.
% However, incorporating a fully accurate dynamics model into these methods is challenging, as it increases the dimension of the search space.
% As a result, these methods often resort to simplified kinodynamic models, like Dubins paths or velocity constraints, to approximate vehicle dynamics without overburdening the search process~\cite{pivtoraiko2005efficient, liu2017search}. 
% However, relying on simplified models can potentially result in infeasible trajectories, especially during high-speed maneuvers.
% \cite{penicka2022minimum} considers a full dynamics model while reducing planning complexity. This is achieved by using informed sampling around a point-mass initial path that does not consider a dynamics model.

% On the other hand, non-linear optimization methods, such as model predictive control or iLQR, can be used. 
% These approaches offer the advantage of incorporating complex vehicle dynamics directly into the trajectory planning process. 
% However, the collision avoidance constraints need to be approximated to keep the optimization in solvable form.
% To address this challenge, recent studies~\cite{liu2017planning, marcucci2024shortest, shao2023design, wang2023speed} propose a two-step approach. .
% The first step involves generating an initial path without considering dynamics constraints. 
% This initial collision-free path can be efficiently found using simple graph traversal algorithms or sampling-based methods. 
% The second step consists of performing a convex decomposition of the surrounding free space. 
% This decomposition method generates linear collision constraints around the initial path, which can be incorporated into the non-linear optimization.

% For both types of methods, a common strategy to manage the complexity of considering both collision-avoidance and full dynamics model is to constrain the trajectory optimization around the initial path.
% Specifically, this method only considers trajectories within the same homotopy class as the initial path, i.e., trajectories that can be continuously deformed from the initial path.
% However, this strategy becomes problematic in dynamic environments where uncertainties in target and obstacle positions require evaluating multiple homotopy classes. 
% For instance, \cite{beyer2024riskpredictive, schmittle2024multi, zhou2021raptor} demonstrate that a path optimized primarily for minimal distance might prove less optimal than longer alternatives with more predictable surrounding conditions.
% Moreover, even when constraining to a single homotopy class, the final step of trajectory generation involving the full dynamics model remains computationally intensive and often unsuitable for high-update-rate online planning.

Existing pathfinding methods efficiently navigate complex environments but struggle to incorporate full vehicle dynamics due to increased search space complexity. 
They often use simplified kinodynamic models, potentially resulting in infeasible trajectories during high-speed maneuvers~\cite{pivtoraiko2005efficient, liu2017search}. 
An alternative approach reduces planning complexity while considering full dynamics by using informed sampling around a point-mass initial path~\cite{penicka2022minimum}.
Nonlinear optimization methods can incorporate complex vehicle dynamics but require approximating collision avoidance constraints~\cite{liu2017planning, marcucci2024shortest, shao2023design, wang2023speed}. 
These approaches also constrain trajectory optimization to the initial path's homotopy class, which is problematic in dynamic environments requiring evaluation of multiple homotopy classes~\cite{beyer2024riskpredictive, schmittle2024multi, zhou2021raptor}. 
Furthermore, even when constrained to a single homotopy class, the final trajectory generation step remains computationally intensive and often unsuitable for high-update-rate online planning.

To address these challenges, we present a sampling-based trajectory generation method for uncertain target positions as shown in \cref{fig:alg_overview}.
Our approach builds upon recent works that replace time-consuming nonlinear optimization with neural network inference~\cite{ryou2024multi, wu2023learning, penicka2022learning, zhao2024learning}.
Utilization of these methods not only dramatically reduces computation times but also allows the exploration of different homotopy classes by optimizing multiple trajectories simultaneously.
Based on this idea, the proposed algorithm first generates multiple initial paths towards sampled candidate target positions, followed by parallel trajectory optimization using a neural network policy. 
The algorithm is applied to the drone interception problem and generates optimal trajectories towards the predicted target drone's positions.

The main contributions of this paper are as follows:
% \begin{itemize}

% \item Implementation of a neural network policy to determine trajectories for multiple initial paths towards candidate targets. The policy selects the path that minimizes traversal time and meets dynamic feasibility, significantly enhancing adaptability to unpredictable movements in unstructured environments.

% \item Application to the drone interception problem, where a defense drone intercepts a target while avoiding collisions and handling imperfect target predictions. The algorithm assesses trajectory reachability by comparing traversal times with the target drone's arrival time, selecting the minimum-time reachable trajectory.

% \item Validation in both simulated and real-world environments, demonstrating high-rate online replanning in the drone interception scenario.
% \end{itemize}
\begin{itemize}
    \item Utilization of a neural network policy to iteratively determine time-optimal and dynamically feasible trajectories for multiple initial paths, enhancing adaptability in unstructured environments.
    \item Application to drone interception, addressing collision avoidance and imperfect target predictions by assessing trajectory reachability and selecting the minimum-time feasible path.
    \item Validation through simulations and real-world experiments, demonstrating high-rate online replanning capabilities.
\end{itemize}
