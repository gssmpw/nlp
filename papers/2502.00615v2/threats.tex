%\textit{\textbf{Definition of Abandonment:}} 
%\textit{\textbf{Generalizability to Other Ecosystems:}}

%Our study faces several threats to validity. 
We define abandoned libraries as those without releases during the last two years of our observation window. While this definition may not fully capture true abandonment, i.e., some libraries may become active again in the future, or developers may maintain repositories without releasing new versions, we argue that using releases as the basis for abandonment is reasonable. This is because developers typically rely on package management systems for downloading libraries, and the lack of new releases may pose risks, even if the source code is maintained elsewhere. Additionally, we find that only 6.6\% of the libraries in our dataset experience a release gap of more than two years during their lifespan, suggesting that reactivation after a two-year period of inactivity is highly unlikely. Thus, we believe that the two-year cutoff does not significantly affect our conclusions. However, this definition and cutoff may not generalize to all ecosystems or account for libraries reaching stability or being replaced by alternatives. Furthermore, our study focuses exclusively on the Maven ecosystem, which may limit the applicability of our findings to ecosystems such as npm or PyPI. External factors, such as industry trends or shifts in technology, could also influence release patterns, which our dataset may not fully capture. 





