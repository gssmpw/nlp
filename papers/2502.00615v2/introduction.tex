Open-source software (OSS) libraries play a crucial role in modern software development, as they are frequently adopted as dependencies in millions of open-source and commercial software systems. These dependencies enable developers to build applications efficiently by leveraging pre-existing solutions. However, the sustainability of these libraries is often uncertain, as they typically depend on volunteer contributors who may discontinue due to various reasons~\cite{lin2017developer,forsgren20212020,kaur2022exploring}. This reliance on volunteer contributors makes libraries vulnerable to turnover and, ultimately, \textit{abandonment}, which presents risks such as unresolved bugs and unpatched security vulnerabilities, which can severely impact dependent systems~\cite{zimmermann2019small,miller2023we}. 

Most existing studies on OSS project abandonment focus on analyzing the impact and risk associated with project abandonment~\cite{valiev2018ecosystem,avelino2019abandonment,coelho2020github,cogo2021deprecation,ait2022empirical,miller2023we,miller2025understanding}. To the best of our knowledge, the recent study by Miller et al.~\cite{miller2025understanding} is the only one that proposes strategies to mitigate the risk brought by abandonment. Their study highlights the importance of proactive notifications from maintainers before abandonment occurs. However, this approach relies heavily on maintainers actively signaling their intent to abandon a project, which is not always feasible or reliable. This highlights a research gap in identifying more indicators of package abandonment that can warn dependent projects about potential risks. Our study seeks to address this gap through a data-driven investigation of libraries' release activities. The focus on release activities is driven by the intuition that abandoned projects may exhibit a slowdown in their release speed prior to abandonment.
%Throughout the paper, the \textit{abandoned projects} refers to the packages that no longer receive any update, i.e., release, for a long period (more than 2 years). 
To achieve our goal, we conduct a case study on libraries in the Maven ecosystem. Despite its popularity, many Maven projects have been abandoned over time, raising concerns about the ecosystem’s long-term sustainability~\cite{shen2025understanding}. Our study addresses the following two research questions: 
\begin{description}
 \item[\textbf{RQ1:}] \rqone
 \item[\textbf{RQ2:}] \rqtwo
\end{description}

The first question examines the severity and evolution of abandonment in the Maven ecosystem, and the second investigates release patterns to differentiate abandoned libraries from active ones. By answering these questions, our study provides insights into the extent of abandonment and early warning signs, offering guidance for developers in dependency management and promoting ecosystem resilience.



\noindent \textbf{Replication package:} The dataset and code for reproducing our empirical study results are available at: \url{https://github.com/RISElabQueens/analyzing-abandonment-and-slowdowns}.