%Open-source projects are critical to millions of software systems, both open-source and proprietary. However, the sustainability of these projects is often uncertain, as they largely rely on volunteer contributors who may stop maintaining them for various reasons~\cite{lin2017developer,forsgren20212020}. This uncertainty leaves developers with little support or guidance when facing dependency abandonment. Miller et al.~\cite{miller2023we} explored how developers manage such situations. Through interviews with 33 developers, they reported how developers prepare for and deal with open-source dependency abandonment. The key stages include risk assessment before adoption, monitoring, maintenance during use, and strategies such as switching, forking, or creating workarounds post-abandonment. Our study is driven by the critical need to analyze software abandonment. Unlike Miller et al., who employed a qualitative approach, we analyzed ecosystem-level library release patterns quantitatively. Our study leverages 10 years of release data from libraries in the Maven ecosystem to uncover trends and release patterns of abandoned projects.

Existing studies have analyzed software development dynamics within ecosystems~\cite{german2013evolution,plakidas2017evolution,decan2019empirical,polese2022adoption}. Among these, the most relevant to our work are those investigating release patterns and project abandonment (survival) at the ecosystem level. Jaime et al.~\cite{jaime2022preliminary} introduced two metrics, i.e., rhythm (time intervals between releases) and speed (average releases per day), to analyze release dynamics in the Maven ecosystem. We also analyze the release speed, but at a more fine-grained level, quartile, and link it with the abandonment of libraries. Ait et al.~\cite{ait2022empirical} analyzed the survival of 1,127 repositories across four ecosystems, i.e., NPM, R, WordPress, and Laravel, over a six-year period. They found that over half of the projects became inactive (abandoned) within four years, with survival rates dropping below 50\% after five years. Miller et al.~\cite{miller2025understanding} analyzed dependency abandonment in the NPM ecosystem. They found that 15\% of widely-used NPM packages were abandoned within six years. Their analysis also highlighted the significant impact of abandonment on dependent projects, many of which fail to respond effectively. Similar to these studies, we find that abandonment is common within the Maven ecosystem. Unlike them, we also investigate library abandonment by examining their release behaviors.

