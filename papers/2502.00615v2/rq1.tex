%We formulated two sub-RQs to explore the prevalence of abandonment in the Maven ecosystem. 

\subsection{RQ1.1: What is the trend of abandonment over the years? }

% \begin{table*}[h]
% \centering
% \footnotesize
% \setlength{\tabcolsep}{3pt} % Reduce column spacing
% \renewcommand{\arraystretch}{0.99} % Reduce line spacing
% \caption{Summary of abandoned and active libraries over the years}
% \label{tab:year_wise_summary}
% \begin{tabular}{|c|c|c|c|c|c|c|}
% \hline
% \textbf{Years} &
%   \textbf{\begin{tabular}[c]{@{}c@{}}Continuing libraries \\ from previous year\end{tabular}} &
%   \textbf{\begin{tabular}[c]{@{}c@{}}Newly created libraries\\  in current year\end{tabular}} &
%   \textbf{\begin{tabular}[c]{@{}c@{}}Total active libraries \\ in currernt year\end{tabular}} &
%   \textbf{\begin{tabular}[c]{@{}c@{}}Total abandoned\\ libraries in current year\end{tabular}} &
%   \textbf{\begin{tabular}[c]{@{}c@{}}Cumulative abandoned \\ libraries\end{tabular}} &
%   \textbf{\begin{tabular}[c]{@{}c@{}}Abandonment \\ rate per year\end{tabular}} \\ \hline
% 2014 & 0      & 26158 & 26158  & 1783  & 1783   & 0.068163 \\ \hline
% 2015 & 24375  & 34620 & 58995  & 12354 & 14137  & 0.209408 \\ \hline
% 2016 & 46641  & 34680 & 81321  & 19604 & 33741  & 0.241069 \\ \hline
% 2017 & 61717  & 34405 & 96122  & 21523 & 55264  & 0.223913 \\ \hline
% 2018 & 74599  & 36440 & 111039 & 27110 & 82374  & 0.244148 \\ \hline
% 2019 & 83929  & 40550 & 124479 & 28075 & 110449 & 0.225540 \\ \hline
% 2020 & 96404  & 43149 & 139553 & 30925 & 141374 & 0.221600 \\ \hline
% 2021 & 108628 & 63612 & 172240 & 37655 & 179029 & 0.218619 \\ \hline
% 2022 & 134585 & 52235 & 186820 & 27739 & 206768 & 0.148480 \\ \hline
% 2023 & 159081 & 37199 & 196280 & 0     & 206768 & 0.000000 \\ \hline
% %2024 & 196280 & 0     & 196280 & 0     & 206768 & 0.000000 \\ \hline
% \end{tabular}
% \end{table*}

\begin{table*}[h]
\centering
\footnotesize
\setlength{\tabcolsep}{3pt} % Reduce column spacing
\renewcommand{\arraystretch}{0.75} % Reduce line spacing
\caption{Statistics of library abandonment in the maven ecosystem over 10 years (2014–2023).}
\vspace{-0.2cm}
\label{tab:year_wise_summary}
\begin{tabular}{ccccccc}
\hline
\textbf{Year} &
  \textbf{\begin{tabular}[c]{@{}c@{}}Continuing libraries \\ from previous year\end{tabular}} &
  \textbf{\begin{tabular}[c]{@{}c@{}}Newly created libraries \\ in current year\end{tabular}} &
  \textbf{\begin{tabular}[c]{@{}c@{}}Total active libraries \\ in current year\end{tabular}} &
  \textbf{\begin{tabular}[c]{@{}c@{}}Total abandoned \\ libraries in current year\end{tabular}} &
  \textbf{\begin{tabular}[c]{@{}c@{}}Abandonment rate \\ per year (\%)\end{tabular}} &
  \textbf{\begin{tabular}[c]{@{}c@{}}Cumulative abandoned \\ libraries\end{tabular}}  \\ \hline
2014 & 0      & 26,158 & 26,158  & 1,783   & 6.8\% & 1,783  \\ 
2015 & 24,375  & 34,620 & 58,995  & 12,354  & 20.9\% & 14,137 \\ 
2016 & 46,641  & 34,680 & 81,321  & 19,604  & 24.1\% & 33,741  \\ 
2017 & 61,717  & 34,405 & 96,122  & 21,523  & 22.3\% & 55,264 \\ 
2018 & 74,599  & 36,440 & 111,039 & 27,110  & 24.4\% & 82,374 \\ 
2019 & 83,929  & 40,550 & 124,479 & 28,075  & 22.5\% & 110,449 \\ 
2020 & 96,404  & 43,149 & 139,553 & 30,925  & 22.1\% & 141,374 \\ 
2021 & 108,628 & 63,612 & 172,240 & 37,655  & 21.8\% & 179,029  \\ 
2022 & 134,585 & 52,235 & 186,820 & 27,739  & 14.8\% & 206,768\\ 
2023 & 159,081 & 37,199 & 196,280 & 0  & 0.0\% & 206,768 \\ \hline
\end{tabular}
\vspace{-0.2cm}
\end{table*}

\noindent \textbf{Approach:} Following the definitions outlined in Section~\ref{dataset}, we classify libraries as either abandoned or active. For each library, we determine its entry into the ecosystem based on its first release date, and for abandoned libraries, we identify the date of abandonment. Subsequently, we collect key statistics related to abandonment for each observation year.

%We determined the very first release year for each project (both active and abandoned) to understand when it entered the ecosystem. Then we characterized continuing libraries(libraries active in the previous year that were not yet abandoned in the current year), newly created (libraries whose first release occurred in the current year) and total active libraries (combined set of continuing and newly created libraries) based on their release timestamps. The yearly abandonment rate was calculated by dividing the number of libraries abandoned in a given year by the total number of active libraries in that year. 

%As our observation window started from 2014, we do not have the data to measure the continuing libraries from previous year. But there's a substantial growth of newly created libraries from 2014 to 2015. 
\noindent \textbf{Results:} Table~\ref{tab:year_wise_summary} summarizes, for each year, the number of active libraries, the libraries abandoned that year, and the percentage of total libraries abandoned. \textbf{We observed that as the Maven ecosystem rapidly expanded, the number of abandoned libraries also increased, with high annual abandonment rates consistently ranging from 20.9\% to 24.4\% during 2015 to 2021.} Note that the abandonment rate for 2014 is significantly lower at 6.8\%, likely because the observation window begins in 2014, providing only one year for libraries created in that year to be abandoned. Similarly, the lower abandonment rate in 2022 may be influenced by its proximity to the end of the observation window, which limits the time available for libraries to be classified as abandoned.

%This indicates a steady-state trend, while the ecosystem continuously added new libraries. As the abandoned rate is consistently present throughout our observation period, it is worth analyzing that why abandonment happens in the Maven ecosystem. 

\subsection{RQ1.2: What is the abandonment rate of libraries over their life cycle?}

% \begin{table*}[]
% \centering
% \footnotesize
% \setlength{\tabcolsep}{3pt} % Reduce column spacing
% \renewcommand{\arraystretch}{0.99}
% \caption{Summary of abandoned libraries based on their cohort year. The abandoned rates indicates the \% of libraries that got abandoned by x year.  }
% \label{tab:survival_stats}
% \begin{tabular}{cccccccc}
% \textbf{Year} &
%   \textbf{Total libraries} &
%   \textbf{\begin{tabular}[c]{@{}c@{}}Abandoned \\ libraries\end{tabular}} &
%   \textbf{\begin{tabular}[c]{@{}c@{}}Active \\ libraries\end{tabular}} &
%   \multicolumn{4}{c}{\textbf{\begin{tabular}[c]{@{}c@{}}Abandonment \\ rates (\%)\end{tabular}}} \\ \hline
%      &       &       &       & \textbf{0yr} & \textbf{1yr} & \textbf{2yr} & \textbf{3yr} \\ \hline
% 2014 & 26158 & 19122 & 7036  & 6.82         & 25.67        & 41.13        & 49.71        \\
% 2015 & 34620 & 27256 & 7364  & 21.44        & 41.98        & 53.54        & 62.18        \\
% 2016 & 34680 & 27706 & 6974  & 24.37        & 44.34        & 58.27        & 65.70        \\
% 2017 & 34405 & 26038 & 8367  & 24.26        & 47.16        & 57.83        & 66.73        \\
% 2018 & 36440 & 26016 & 10424 & 26.30        & 48.05        & 59.41        & 67.14        \\
% 2019 & 40550 & 25197 & 15353 & 26.48        & 46.52        & 56.10        & 62.14        \\
% 2020 & 43149 & 22212 & 20937 & 24.89        & 42.75        & 51.48        & 0.00         \\
% 2021 & 63612 & 25263 & 38349 & 26.77        & 39.71        & 0.00         & 0.00         \\
% 2022 & 52235 & 7958  & 44277 & 15.23        & 0.00         & 0.00         & 0.00         \\
% 2023 & 37199 & 0     & 37199 & 0.00         & 0.00         & 0.00         & 0.00          
% \end{tabular}
% \end{table*}

%The abandonment rates indicate the \% of libraries abandoned within x year.
% \begin{table*}[h]
% \centering
% \footnotesize
% \setlength{\tabcolsep}{2pt} % Reduce column spacing
% \renewcommand{\arraystretch}{0.75}
% \caption{Summary of abandoned libraries based on their creation year. }
% \vspace{-0.2cm}
% \label{tab:survival_stats}
% \begin{tabular}{ccccccccc}
% \hline
% \textbf{Year} &
%   \textbf{Total libraries} &
%   \textbf{\begin{tabular}[c]{@{}c@{}}Abandoned \\ libraries\end{tabular}} &
%   \textbf{\begin{tabular}[c]{@{}c@{}}Active \\ libraries\end{tabular}} &
%   \multicolumn{5}{c}{\textbf{\begin{tabular}[c]{@{}c@{}}Abandonment rates (\%)\end{tabular}}} \\ \cline{5-9}
%      &       &       &       & \textbf{within 1 year}            & \textbf{within 2 year}            & \textbf{within 3 year}            & \textbf{within 4 year}            & \textbf{within 5 and remaining years}           \\ \hline
% 2014 & 26,158 & 19,122 & 7,036  & 1,783 (6.8\%)           & 6,714 (25.6\%)          & 10,758 (41.1\%)         & 13,003 (49.7\%)         & 19,122 (73.1\%)         \\
% 2015 & 34,620 & 27,256 & 7,364  & 7,423 (21.4\%)          & 14,533 (41.9\%)         & 18,537 (53.5\%)         & 21,528 (62.1\%)         & 27,256 (78.7\%)         \\
% 2016 & 34,680 & 27,706 & 6,974  & 8,450 (24.3\%)          & 15,377 (44.3\%)         & 20,208 (58.2\%)         & 22,784 (65.7\%)         & 27,706 (79.8\%)         \\
% 2017 & 34,405 & 26,038 & 8,367  & 8,347 (24.2\%)          & 16,225 (47.1\%)         & 19,896 (57.8\%)         & 22,959 (66.7\%)         & 26,038 (75.6\%)         \\
% 2018 & 36,440 & 26,016 & 10,424 & 9,585 (26.3\%)          & 17,510 (48.1\%)         & 21,649 (59.4\%)         & 24,464 (67.1\%)         & 26,016 (71.3\%)         \\
% 2019 & 40,550 & 25,197 & 15,353 & 10,736 (26.5\%)         & 18,865 (46.5\%)         & 22,749 (56.1\%)         & 25,197 (62.1\%)         & 0 (0.0\%)              \\
% 2020 & 43,149 & 22,212 & 20,937 & 10,741 (24.9\%)         & 18,445 (42.7\%)         & 22,212 (51.4\%)         & 0 (0.0\%)              & 0 (0.0\%)              \\
% 2021 & 63,612 & 25,263 & 38,349 & 17,026 (26.7\%)         & 25,263 (39.7\%)         & 0 (0.00\%)              & 0 (0.0\%)              & 0 (0.0\%)              \\
% 2022 & 52,235 & 7,958  & 44,277 & 7,958 (15.2\%)          & 0 (0.0\%)              & 0 (0.0\%)              & 0 (0.0\%)              & 0 (0.0\%)              \\
% 2023 & 37,199 & 0     & 37,199 & 0 (0.0\%)              & 0 (0.0\%)              & 0 (0.0\%)              & 0 (0.0\%)              & 0 (0.0\%)              \\
% \hline
% \end{tabular}
% \end{table*}


\begin{table*}[h]
\centering
\footnotesize
\setlength{\tabcolsep}{2pt} % Reduce column spacing
\renewcommand{\arraystretch}{0.75}
\caption{Summary of abandoned libraries based on their creation year.}
\vspace{-0.2cm}
\label{tab:survival_stats}
\begin{tabular}{ccccccccc}
\hline
\textbf{Year} &
  \textbf{Total libraries} &
  \textbf{\begin{tabular}[c]{@{}c@{}}Abandoned \\ libraries\end{tabular}} &
  \textbf{\begin{tabular}[c]{@{}c@{}}Active \\ libraries\end{tabular}} &
  \multicolumn{5}{c}{\textbf{\begin{tabular}[c]{@{}c@{}}Abandonment rates (\%)\end{tabular}}} \\ \cline{5-9}
     &       &       &       & \textbf{within 1 year} & \textbf{within 2 year} & \textbf{within 3 year} & \textbf{within 4 year} & \textbf{\begin{tabular}[c]{@{}c@{}}within 5 and  remaining years\end{tabular}} \\ \hline
2014 & 26,158 & 19,122 & 7,036  & 6.8\%   & 25.6\%  & 41.1\%  & 49.7\%  & 73.1\%  \\
2015 & 34,620 & 27,256 & 7,364  & 21.4\%  & 41.9\%  & 53.5\%  & 62.1\%  & 78.7\%  \\
2016 & 34,680 & 27,706 & 6,974  & 24.3\%  & 44.3\%  & 58.2\%  & 65.7\%  & 79.8\%  \\
2017 & 34,405 & 26,038 & 8,367  & 24.2\%  & 47.1\%  & 57.8\%  & 66.7\%  & 75.6\%  \\
2018 & 36,440 & 26,016 & 10,424 & 26.3\%  & 48.1\%  & 59.4\%  & 67.1\%  & 71.3\%  \\
2019 & 40,550 & 25,197 & 15,353 & 26.5\%  & 46.5\%  & 56.1\%  & 62.1\%  & --      \\
2020 & 43,149 & 22,212 & 20,937 & 24.9\%  & 42.7\%  & 51.4\%  & --      & --      \\
2021 & 63,612 & 25,263 & 38,349 & 26.7\%  & 39.7\%  & --      & --      & --      \\
2022 & 52,235 & 7,958  & 44,277 & 15.2\%  & --      & --      & --      & --      \\
2023 & 37,199 & 0      & 37,199 & --      & --      & --      & --      & --      \\ \hline
\end{tabular}
\end{table*}





%\AM{ For Yuan, You may have questions on why after 0 years, I explained in approach below about this. The main logic was i want to make it consistent with first table. it remains consistent in both of the tables. After completing the table 2, this table is showing the same results which i presented in survival plot. The survival plot table is still in shared document, you may check. It varies by 2-3\% which is making sense as that was an estimator.}
 

\noindent \textbf{Approach:} RQ1.1 demonstrates that abandonment is prevalent in the ecosystem. However, the abandonment ratio presented in Table~\ref{tab:year_wise_summary} does not tell when a library will likely be abandoned, which motivates RQ1.2. To answer this question, we performed a cohort-based, time-to-event analysis. We define a \textit{cohort} as a group of libraries created in the same observation year. This design enables us to examine how abandonment patterns evolve. For each cohort, we calculated the proportion of libraries created in that year that were abandoned within 1 to 4 years, or later, from their creation date.

%'Abandoned\_after\_x\_years'  represents how many libraries from that starting group (cohort) were no longer active after x years. For example, abandoned\_after\_0\_years means these libraries stopped receiving updates in the same year they were created, abandoned\_after\_1\_years means they were abandoned by the end of their first full year, and so on. 

%From a survival point of view, 50.29\% libraries still active at after three years. 
\noindent \textbf{Results:} The cohort-wise abandonment analysis (Table~\ref{tab:survival_stats}) shows varying abandonment patterns by library creation year. Libraries created in 2014 had the lowest first-year abandonment rate (6.8\%), but this rose sharply to 49.7\% within three years. \textbf{In contrast, libraries created between 2015 and 2021 showed higher first-year abandonment rates (21.4\%–26.7\%), with roughly one in four failing within their first year.} Second-year abandonment was particularly pronounced, driving the two-year cumulative rate to nearly 50\%. Notably, libraries created in 2018 and 2019 experienced the highest overall abandonment rates.



%Additionally, between 62.14\% and 67.14\% of libraries were abandoned by the fourth year after their creation. 

%This consistent pattern indicates that a large fraction of libraries get abandoned within their creation year and it grows positively over time. 

%Libraries that were stopped receiving updates in the same year they were created indicates interesting patterns. Across all the years, around 75\% of the libraries survived after first creation year. It refers abandonment exists but less common after first year. From 2016-2021, the immediate abandonment rate stays within 21.44\% to 26.48\%. This steady pattern suggests around one in every four libraries do not cross it's creation year. These libraries could be experimental or test libraries. 

% \AM{i may change it}. Within 2015-16, it could be the libraries are experimental or developers quickly disregarded the ideas. Within 2017-2021, many new libraries may have been prototypes, test libraries, or one-off experiments that never moved beyond the initial release. We observed this trend in all the cohort years. 

%For cohorts with enough observation time (2014–2020), survival rates after three years ranges around 32.86\% to 50.29\%. After completing 3 years, it is observed in 2014 that around 50\% libraries survived which indicates half of the libraries created got abandoned. The number increases in later years (2015-2019) which makes it worth analyzing the abandonment in the Maven ecosystem. 

\vspace{-0.1cm}
\begin{tcolorbox}[colframe=white, colback=blue!5, left=1mm, right=1mm, sharp corners]
\textbf{RQ1 Summary:} Library abandonment is common in the Maven ecosystem, with a consistently high annual rate exceeding 20\%. Between 2015 and 2021, approximately 25\% of newly introduced libraries were abandoned within their first year, increasing to 39.7\%–48.1\% by the end of their second year.
\end{tcolorbox}

% colframe=blue!50!black
