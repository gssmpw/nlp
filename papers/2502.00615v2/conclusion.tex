%Our study on the Maven ecosystem shows that 21.4\% to 26.7\% of libraries are abandoned within their first year, with over 62\% abandoned by the third year. This aligns with prior research, highlighting the need to understand and mitigate library abandonment risks. 
Library abandonment poses significant challenges for software ecosystems, affecting their reliability and maintainability. Our study shows that a slowdown in release speed, particularly in the later stages of a library's lifecycle, often signals impending abandonment. While not all abandoned libraries exhibit this trend, developers are encouraged to monitor dependency release activity, watching for sudden slowdowns or prolonged inactivity, to address potential risks proactively. The high abandonment rate after two years further highlights the need for vigilance and adaptability in managing dependencies. Our findings underscore the importance of release patterns and library lifecycles in understanding abandonment. Future work should aim to identify predictive indicators and develop tools to track library age and release patterns, offering early warnings for at-risk dependencies. The steady high rate of library abandonment in Maven highlights the need for ecosystem-level interventions. Package platforms could introduce automated alerts for libraries showing signs of abandonment, such as inactivity or reduced release frequency, to support maintainers and minimize abandonment risks.

%This study, centered on the widely-used Maven ecosystem, highlights the severity of library abandonment and identifies critical patterns, such as the influence of release speed, lifespan, and inactivity on abandonment likelihood. However, shared characteristics between active and abandoned libraries complicate precise risk prediction. While indicators like slow release speed and extended inactivity are common precursors to abandonment, further research is needed to refine prediction models and develop robust strategies for dependency management in software ecosystems.

