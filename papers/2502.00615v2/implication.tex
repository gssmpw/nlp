
\noindent\textbf{Developer Perspective:}
Our study shows that 21.44\% to 26.77\% of newly introduced libraries are abandoned within their first year, with a decreasing survival rate in the following years. We observed slowdown patterns in later stages which could potentially serve as an abandonment flag. Developers should monitor release activity patterns of their dependencies, such as sudden slowdowns and prolonged inactivity in the within life cycle as warning signs of potential abandonment. Developers should consider the survival rate of libraries when selecting dependencies. Additionally, the high abandonment rate after two years emphasizes the need for proper monitoring and preparedness to handle dependency shifts.






\noindent\textbf{Research Perspective:} 
Our findings suggest that release patterns and library lifecycles are important factors to explore. Future research should focus on identifying the indicators that lead to abandonment and developing tools to predict which libraries are at risk. These tools could help developers track the age and release patterns of libraries, allowing them to identify at-risk projects early.


\noindent\textbf{Ecosystem Perspective:}
The steady rate of project abandonment in the Maven ecosystem underscores the need for ecosystem-level interventions to support maintainers and reduce the risk of abandonment. Platforms should consider offering automatic alerts for libraries showing signs of abandonment.
