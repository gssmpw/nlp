
%RQ1 findings reveal the stable trend of libraries being abandoned in Maven ecosystem with many being short lived. This trend correlates directly with the release activity patterns explored in this research question.  


%We analyze the release activities of abandoned libraries from two perspectives: their overall lifespan and release speed, as well as the release speed patterns throughout their evolution. These aspects are compared with active libraries, forming RQ2.1 and RQ2.2.



\subsection{RQ2.1: How do the lifespan and release speed of abandoned libraries compare to those of active libraries?}





% new table after yuan's suggestion starts below:
%Each cell shows the number of libraries followed by their percentages.
% \begin{table*}[]
% \centering
% \footnotesize
% \setlength{\tabcolsep}{2pt} % Adjust column spacing
% \renewcommand{\arraystretch}{1.1} % Adjust row spacing
% \caption{Summary of the lifespan of libraries and release speed categories for active and abandoned libraries. }
% \vspace{-0.2cm}
% \label{tab:comparison_lifespan_speed}
% \resizebox{\textwidth}{!}{%
% \begin{tabular}{ccccccccc}
% \hline
% \multirow{2}{*}{\textbf{Speed category}} &
%   \multicolumn{4}{c}{\textbf{Abandoned libraries}} &
%   \multicolumn{4}{c}{\textbf{Active libraries}} \\ \cline{2-9} 
%  &
%   \textbf{\textless{}1 year} &
%   \textbf{1-2 years} &
%   \textbf{\textgreater{}2 years} &
%   \textbf{Total} &
%   \textbf{\textless{}1 year} &
%   \textbf{1-2 years} &
%   \textbf{\textgreater{}2 years} &
%   \textbf{Total} \\ \hline
% \textbf{\textless{}1} &
%   42,039 (20.3\%) &
%   28,718 (13.9\%) &
%   39,468 (19.1\%) &
%   110,225 (53.3\%) &
%   16,438 (8.4\%) &
%   26,920 (13.7\%) &
%   75,999 (38.7\%) &
%   119,357 (60.8\%) \\ 
% \textbf{1-2} &
%   42,061 (20.3\%) &
%   5,401 (2.6\%) &
%   5,153 (2.5\%) &
%   52,615 (25.5\%) &
%   11,930 (6.1\%) &
%   8,301 (4.2\%) &
%   22,179 (11.3\%) &
%   42,410 (21.6\%) \\ 
% \textbf{\textgreater{}2} &
%   37,907 (18.3\%) &
%   3,312 (1.6\%) &
%   2,709 (1.3\%) &
%   43,928 (21.3\%) &
%   10,771 (5.5\%) &
%   7,394 (3.7\%) &
%   16,348 (8.3\%) &
%   34,513 (17.5\%) \\ \hline
% \textbf{Total} &
%   122,007 (59.1\%) &
%   37,431 (18.1\%) &
%   47,330 (22.8\%) &
%   206,768 (100.0\%) &
%   39,139 (19.9\%) &
%   42,615 (21.7\%) &
%   114,526 (58.3\%) &
%   196,280 (100.0\%) \\ \hline
% \end{tabular}%
% }
% \end{table*}

\begin{table*}[]
\centering
\footnotesize
\setlength{\tabcolsep}{2pt} % Adjust column spacing
\renewcommand{\arraystretch}{1.1} % Adjust row spacing
\caption{Summary of the lifespan of libraries and release speed categories for active and abandoned libraries.}
\vspace{-0.2cm}
\label{tab:comparison_lifespan_speed}
\resizebox{\textwidth}{!}{%
\begin{tabular}{ccccc|cccc} % Vertical line added after the 5th column
\hline
\multirow{2}{*}{\textbf{Speed category}} &
  \multicolumn{4}{c}{\textbf{Abandoned libraries}} &
  \multicolumn{4}{c}{\textbf{Active libraries}} \\ \cline{2-9} 
 &
  \textbf{\textless{}1 year} &
  \textbf{1-2 years} &
  \textbf{\textgreater{}2 years} &
  \textbf{Total} &
  \textbf{\textless{}1 year} &
  \textbf{1-2 years} &
  \textbf{\textgreater{}2 years} &
  \textbf{Total} \\ \hline
\textbf{\textless{}1} &
  42,039 (20.3\%) &
  28,718 (13.9\%) &
  39,468 (19.1\%) &
  110,225 (53.3\%) &
  16,438 (8.4\%) &
  26,920 (13.7\%) &
  75,999 (38.7\%) &
  119,357 (60.8\%) \\ 
\textbf{1-2} &
  42,061 (20.3\%) &
  5,401 (2.6\%) &
  5,153 (2.5\%) &
  52,615 (25.5\%) &
  11,930 (6.1\%) &
  8,301 (4.2\%) &
  22,179 (11.3\%) &
  42,410 (21.6\%) \\ 
\textbf{\textgreater{}2} &
  37,907 (18.3\%) &
  3,312 (1.6\%) &
  2,709 (1.3\%) &
  43,928 (21.3\%) &
  10,771 (5.5\%) &
  7,394 (3.7\%) &
  16,348 (8.3\%) &
  34,513 (17.5\%) \\ \hline
\textbf{Total} &
  122,007 (59.1\%) &
  37,431 (18.1\%) &
  47,330 (22.8\%) &
  206,768 (100.0\%) &
  39,139 (19.9\%) &
  42,615 (21.7\%) &
  114,526 (58.3\%) &
  196,280 (100.0\%) \\ \hline
\end{tabular}%
}
\end{table*}




\noindent \textbf{Approach:} Release speed is a critical characteristic of libraries in a software ecosystem~\cite{jaime2022preliminary}. Thus, we first analyze the release activities of abandoned libraries by calculating this metric within the whole library lifespan. Recognizing that release speed may vary depending on a library's lifespan, we categorize libraries based on their lifespan and classify their release speeds into distinct groups. Specifically, we use 1 and 2 years as thresholds to define short-lived, moderate-lived, and long-lived libraries, and 1 and 2 releases per month to distinguish between fast, moderate, and slow release speeds. These thresholds are determined based on observations of the overall distribution of lifespan and release speed and their interpretability.

%To investigate the distinct release activity patterns of active and abandoned libraries, we performed a comprehensive quantitative analysis of their release histories. We adopted \textit{speed} metric defined by Damien et al \cite{jaime2022preliminary}. In our study, we calculated the lifespan \HH{do we need to explain the definition of lifespan?} and speed (release per month) of each libraries. To facilitate meaningful comparison, we categorized speed and their lifespan in different categories, summarized in Table \ref{tab:comparison_lifespan_speed}. 

\noindent \textbf{Results:} Table~\ref{tab:comparison_lifespan_speed} summarizes the number of libraries in each group, categorized by lifespan and release speed. \textbf{Our analysis reveals that 59.1\% of abandoned libraries are short-lived, with the majority (68.9\%) exhibiting slow or moderate release speeds.} 20.3\% of abandoned libraries exhibit slow release speed and short-lived, significantly higher than the 8.4\% observed in active libraries. This indicates that short-lived libraries, particularly those with slow release speed, are more prone to abandonment. This trend remains relatively consistent across moderate (1–2 releases per month) and high-speed (\textgreater 2 releases per month) categories, where short-lived abandoned libraries are more prevalent than their active counterparts. 

%However, high-speed libraries are less commonly observed among abandoned libraries compared to active ones. Notably, rapid release activity does not guarantee long-term maintenance, as 18.33\% of high-speed libraries were ultimately abandoned.

22.8\% of abandoned libraries are long-lived, suggesting that these libraries may have reached maturity and no longer require frequent updates or new releases. \textbf{Additionally, 43,928 (21.3\%) abandoned libraries exhibit high release speeds. This indicates high release speed alone may not reliably signal the risk of abandonment.}

%In active libraries, the matured (\textgreater 2 years) libraries maintain slow and moderate release speed (38.72\%) over months. In contrast, only 19.09\% abandoned libraries contain this same pattern. This indicates that some libraries may suddenly get abandoned 
%This refers abandoned libraries too maintain slow and steady release activity but suddenly get abandoned. 


%Less yet alarming consistent present of libraries getting abandoned even being a long lived (1-2 years and \textgreater2 years) library, indicates the fact that they might slow down in their later stages which resulted in abandonment. This is making it worth analyzing how their release patterns changes over time before getting abandoned and how it is different from active groups.




\subsection{RQ2.2: How does release speed evolve over the lifespan of abandoned libraries compared to active libraries?}


%RQ2.1 findings highlight that abandonment can occur in libraries with different lifespan. The key distinction appears to be the gradual slowing of release activity prior to abandonment. This pattern making it worth exploring the release sequences, aiming to identify early warning indicators of impending abandonment within  libraries and to understand how their release patterns are different from those of active libraries.  

\noindent \textbf{Approach:} The release speed of a library can vary significantly over its lifespan. Thus, in RQ2.2, we perform an in-depth analysis on how a library's release speed evolves across the four quartiles of its lifespan. To ensure meaningful pattern analysis, we include only libraries with at least four releases, providing a sufficiently rich release history for analysis. For each library, we begin by identifying the releases within each of the four quartiles (Q1, Q2, Q3, Q4) based on their release dates. For instance, if the lifespan of a library is 1 year, then releases made in the first three months belong to Q1. Next, we calculate the average lifespan of releases within each quartile and categorize it as ``Fast'', ``Normal'', or ``Slow'' by comparing the quartile-specific lifespan against the overall average lifespan of releases for that library. Specifically, a quartile is labeled as \textit{``Fast''} if its average release lifespan is 20\% less than the library’s overall average.
A quartile is labeled as \textit{``Slow''} if its average release lifespan is 20\% longer than the library’s overall average. Quartiles with intervals between these thresholds are labeled as \textit{``Normal''}. In two cases, quartile-specific average release lifespans cannot be calculated and are labeled as \textit{``nan''}: in either Q2 or Q3, if no releases occur, and in Q4, if only the final release exists, as it coincides with the end of the library's lifespan.

%This process assigns each library a sequence of four statuses corresponding to the release speed in its quartiles. Finally, we analyze and summarize the patterns observed across these sequences to identify trends in release speed dynamics over a library's lifespan.


%\AM{for yuan, i am unsure how could i explain the threshold here.} \HH{I think Nan is not a good word to use in writing. Suggestion: use None, not available, or no release instead?}

\noindent \textbf{Results:} Tables~\ref{tab:quartile_abandoned_libraries} and~\ref{tab:quartile_active_libraries} present the top-10 sequence patterns observed in abandoned and active libraries, covering 18.2\% of total abandoned libraries and 26.2\% of total active libraries, respectively.

% \begin{table}[h]
% \centering
% \footnotesize
% \caption{Top 10 identified patterns in abandoned libraries.}
% \label{tab:quartile_abandoned_libraries}
% \resizebox{\columnwidth}{!}{%
% \begin{tabular}{cccccccc}
% \hline
% \textbf{\# of libraries} &
% \textbf{(\%)} &
% \textbf{Patterns} &
% \multicolumn{4}{c}{\textbf{Release statistics}} \\ \cline{4-7} 
%  &  &  & \textbf{Min} & \textbf{Mean} & \textbf{Median} & \textbf{Max} \\ \hline
% 6,540 & 5.65 & Fast \textgreater Slow \textgreater nan \textgreater nan       & 5  & 8.84  & 7.0  & 150  \\
% 5,428 & 4.69 & Slow \textgreater nan \textgreater nan \textgreater Fast       & 5  & 7.66  & 6.0  & 208  \\
% 4,233 & 3.66 & Normal \textgreater nan \textgreater nan \textgreater nan      & 5  & 7.016  & 6.0  & 66   \\
% 4,196 & 3.62 & Fast \textgreater Slow \textgreater nan \textgreater Fast      & 5  & 12.52 & 9.0  & 219  \\
% 3,651 & 3.15 & Fast \textgreater Slow \textgreater Slow \textgreater nan      & 5  & 13.52 & 11.0 & 230  \\
% 2,991 & 2.58 & Fast \textgreater Normal \textgreater Slow \textgreater nan    & 5  & 12.86 & 10.0 & 317  \\
% 2,831 & 2.44 & Fast \textgreater Normal \textgreater Slow \textgreater Slow   & 7  & 52.45 & 39.0 & 1,043 \\
% 2,625 & 2.27 & Fast \textgreater Slow \textgreater Slow \textgreater Fast     & 6  & 21.40 & 15.0 & 263  \\
% 2,566 & 2.21 & Normal \textgreater Normal \textgreater Normal \textgreater Normal & 5  & 40.79 & 22.0 & 1,630 \\
% 2,487 & 2.15 & Slow \textgreater nan \textgreater Fast \textgreater Fast      & 5  & 12.25 & 8.0  & 206  \\ \hline
% \end{tabular}%
% }
% \end{table}


%after removing release stats:

\begin{table}[h]
\centering
\tiny
\caption{Top 10 identified patterns in abandoned libraries.}
\vspace{-0.2cm}
\label{tab:quartile_abandoned_libraries}
\resizebox{\columnwidth}{!}{%
\begin{tabular}{ccc}
\hline
\textbf{\# of libraries} & \textbf{(\%)} & \textbf{Patterns} \\ \hline
6,540 & 5.7 & Fast \textgreater Slow \textgreater nan \textgreater nan       \\
5,428 & 4.7 & Slow \textgreater nan \textgreater nan \textgreater Fast       \\
4,233 & 3.7 & Normal \textgreater nan \textgreater nan \textgreater nan      \\
4,196 & 3.6 & Fast \textgreater Slow \textgreater nan \textgreater Fast      \\
3,651 & 3.2 & Fast \textgreater Slow \textgreater Slow \textgreater nan      \\
2,991 & 2.6 & Fast \textgreater Normal \textgreater Slow \textgreater nan    \\
2,831 & 2.4 & Fast \textgreater Normal \textgreater Slow \textgreater Slow   \\
2,625 & 2.3 & Fast \textgreater Slow \textgreater Slow \textgreater Fast     \\
2,566 & 2.2 & Normal \textgreater Normal \textgreater Normal \textgreater Normal \\
2,487 & 2.2 & Slow \textgreater nan \textgreater Fast \textgreater Fast      \\ \hline
\end{tabular}%
}
\end{table}

% \begin{table}[h]
% \centering
% \footnotesize
% \caption{Top 10 identified patterns in active libraries. \HH{why do we want to have the Releases statistics included in the table? } \AM{see the results section to understand}}
% \label{tab:quartile_active_libraries}
% \resizebox{\columnwidth}{!}{%
% \begin{tabular}{ccccccc}
% \hline
% \textbf{\# of libraries} &
% \textbf{(\%)} &
% \textbf{Patterns} &
% \multicolumn{4}{c}{\textbf{Releases statistics}} \\ \cline{4-7} 
%  &  &  & \textbf{Min} & \textbf{Mean} & \textbf{Median} & \textbf{Max} \\ \hline
% 11,047 & 6.70 & Normal \textgreater Normal \textgreater Normal \textgreater Normal & 5  & 128.91 & 55.0  & 1,835 \\
% 8,032  & 4.87 & Fast \textgreater Normal \textgreater Slow \textgreater Slow       & 6  & 104.35 & 51.0  & 3,819 \\
% 4,914  & 2.98 & Fast \textgreater Slow \textgreater Slow \textgreater Slow         & 7  & 63.73  & 36.0  & 1,516 \\
% 4,622  & 2.80 & Fast \textgreater Slow \textgreater nan \textgreater nan           & 5  & 9.82   & 7.0   & 150  \\
% 4,351  & 2.64 & Slow \textgreater nan \textgreater nan \textgreater Fast           & 5  & 8.97   & 7.0   & 269  \\
% 3,921  & 2.38 & Fast \textgreater Slow \textgreater Slow \textgreater Fast         & 6  & 27.50  & 20.0  & 398  \\
% 3,759  & 2.28 & Slow \textgreater Normal \textgreater Normal \textgreater Fast     & 5  & 103.29 & 59.0  & 1,286 \\
% 3,685  & 2.23 & Fast \textgreater Fast \textgreater Slow \textgreater Slow         & 8  & 90.18  & 51.0  & 3,023 \\
% 3,587  & 2.17 & Fast \textgreater Slow \textgreater nan \textgreater Fast          & 5  & 12.68  & 10.0  & 295  \\
% 3,509  & 2.13 & Slow \textgreater Slow \textgreater Fast \textgreater Fast         & 5  & 77.02  & 39.0  & 1,288 \\ \hline
% \end{tabular}%
% }
% \end{table}


%after removing release stats

\begin{table}[h]
\centering
\tiny
\caption{Top 10 identified patterns in active libraries.}
\label{tab:quartile_active_libraries}

\resizebox{\columnwidth}{!}{%
\begin{tabular}{ccc}
\hline
\textbf{\# of libraries} & \textbf{(\%)} & \textbf{Patterns} \\ \hline
11,047 & 6.7 & Normal \textgreater Normal \textgreater Normal \textgreater Normal \\
8,032  & 4.9 & Fast \textgreater Normal \textgreater Slow \textgreater Slow       \\
4,914  & 2.9 & Fast \textgreater Slow \textgreater Slow \textgreater Slow         \\
4,622  & 2.8 & Fast \textgreater Slow \textgreater nan \textgreater nan           \\
4,351  & 2.6 & Slow \textgreater nan \textgreater nan \textgreater Fast           \\
3,921  & 2.4 & Fast \textgreater Slow \textgreater Slow \textgreater Fast         \\
3,759  & 2.3 & Slow \textgreater Normal \textgreater Normal \textgreater Fast     \\
3,685  & 2.2 & Fast \textgreater Fast \textgreater Slow \textgreater Slow         \\
3,587  & 2.2 & Fast \textgreater Slow \textgreater nan \textgreater Fast          \\
3,509  & 2.1 & Slow \textgreater Slow \textgreater Fast \textgreater Fast         \\ \hline
\end{tabular}%
}
\end{table}
In abandoned libraries, the most common pattern is \textit{``Fast → Slow → nan → nan''}, which exists in 6,540 libraries (5.7\%). This means there are only releases in the first half of the lifespan before the final release, and the release speed is quickly slowed down from the first to the second quartile. \textbf{This aligns with our intuitions, i.e., abandoned libraries already slowed down or have long relative inactive period before abandonment.} Note that this pattern also exists in active libraries, but accounts for only 2.8\% of the cases. These active libraries may still be new, i.e., having their latest release in the final two years of our observation window but may be prone to abandonment soon. The general slowdown patterns can also be observed in other patterns, such as \textit{``Fast → Slow → Slow → nan''} and \textit{``Fast → Normal → Slow → nan''}. Four out of the top-10 patterns represent slowdown, accounting for 7.8\% of the abandoned libraries. The inactive general pattern (containing ``nan'') can also be observed in patterns such as \textit{``Slow → nan → nan → Fast''}, accounting for 7 out of 10 patterns and 14.3\% of the abandoned libraries. This number is much lower in active libraries, with only three patterns containing ``nan''. Interestingly, there are some cases where the libraries recovered from slow release but still get abandoned, i.e., 3.9\% libraries with patterns such as \textit{``Slow → nan → nan → Fast''} and \textit{``Fast → Slow → Slow → Fast''}. This suggests that a temporary increase in release activity is insufficient to prevent abandonment. 

For active libraries, the pattern \textit{``Normal → Normal → Normal → Normal''} is the most common, representing 6.7\% of all active libraries. This suggests that many active libraries maintain a consistent release speed over their lifespan. Although this pattern is also observed in abandoned libraries, it occurs at a significantly lower rate (2.2\%).

%This refers that in first 25\% lifecycle of library, the release speed was relatively fast, in next 25\% life cycle, it slowed down and has no activity in next 50\% of lifestyle and got abandoned with the final release made in the final quartile. Looking into their release statistics, this abandoned pattern indicates these libraries never matured into stable libraries as they have 8.8 mean releases. Similarly, the “Normal → nan → nan → nan” (4,233 libraries, 3.66\%), refers libraries got abandoned after an initial start. With a median release count of 6, these libraries never advanced beyond their early stage, aligning with RQ2.1 findings that libraries often abandon early.

%“Fast → Normal → Slow → Slow” (2,991 libraries, 2.587\%) has a mean release count of 12.866 and a maximum of 317 releases. This indicates that some libraries once maintained a robust release schedule yet gradually declined and eventually got abandoned. It also indicates that observing slowdowns increases the risk of abandonment. Similar pattern also seen in active libraries, but they managed to comeback and continue contributing. 


%We also encountered “Fast → Slow → Slow → Fast” patterns when there's fast release activities in later stages (2.21\%), but still got abandoned. 


%In active libraries, most frequent pattern is “Normal → Normal → Normal → Normal,” found in 11,047 libraries (6.71\%), substantially higher than its occurrence among abandoned libraries (only 2.22\%). These libraries show a median of 55 releases and a maximum of 1,835, suggesting sustained, steady maintenance over time. 


%In summary, patterns that started face slowdown in later quartiles along with their associated release statistics provides an automated mechanism for flagging libraries that may be at risk. 


\vspace{-0.1cm}
\begin{tcolorbox}[colframe=white, colback=blue!5, left=1mm, right=1mm, sharp corners]
\textbf{RQ2 Summary:} In general, a slowdown in release speed and periods of relative inactivity can signal the potential abandonment of a library. However, some abandoned libraries exhibit fast release speeds toward the end of their lifespan, suggesting that rapid release alone is not always indicative of sustained maintenance.
\end{tcolorbox}


