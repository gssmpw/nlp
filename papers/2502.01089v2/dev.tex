\documentclass[10pt, conference]{IEEEtran}
\IEEEoverridecommandlockouts
% The preceding line is only needed to identify funding in the first footnote. If that is unneeded, please comment it out.
\usepackage{cite}
\usepackage{amsmath,amssymb,amsfonts}
\usepackage{algorithmic}
\usepackage{graphicx}
\usepackage{textcomp}
\usepackage{xcolor}
\usepackage{comment}
\usepackage{soul, xcolor}
\usepackage{balance}

\usepackage[acronym,toc,shortcuts]{glossaries}

\graphicspath{{figures/}}
\usepackage{url}

 

%%%%%%%%%%%%%%%%%%%%%%%%%%%%%%%%%%%%%%%%%%%%%%%%%%%%%%%%%%%%%%%%%%%
%%%%%%%%%%%%%%%%%%%%%%%%%%%%%%%%%% ACRONYMS
%\makeglossaries
\newacronym{AI}{AI}{Artificial Intelligence}
\newacronym{DDoS}{DDoS}{Distributed Denial-of-Service}
\newacronym{IACS}{IACS}{Industrial Automated Control System}
\newacronym{ICPS}{ICPS}{Industrial Cyber-Physical Systems}
\newacronym{FaaS}{FaaS}{Function-as-a-Service}
\newacronym{HMI}{HMI}{Human-Machine Interface}
\newacronym{IIoT}{IIoT}{Industrial IoT}
\newacronym{IT}{IT}{Information Technology}
\newacronym{LLM}{LLM}{Large Language Model}
\newacronym{NPN}{NPN}{Non-Public Network}
\newacronym{OT}{OT}{Operational Technology}
\newacronym{PLC}{PLC}{Programmable Logic Controller}
\newacronym{PQC}{PQC}{Post-Quantum Cryptography}
\newacronym{RAN}{RAN}{Radio Access Network}
\newacronym{SBA}{SBA}{Service-Based Architectures}
\newacronym{SOC}{SOC}{Security Operations Center}
\newacronym{SSI}{SSI}{Self-Sovereign Identity}
\newacronym{UE}{UE}{User Equipment}
\newacronym{V2X}{V2X}{Vehicle to Everything}
\newacronym{ZTA}{ZTA}{Zero Trust Architecture}

\def\BibTeX{{\rm B\kern-.05em{\sc i\kern-.025em b}\kern-.08em
    T\kern-.2em\lower.7ex\hbox{E}\kern-.125emX}}


    
\begin{document}

\title{Advanced Architectures Integrated with Agentic AI for Next-Generation Wireless Networks\\
%{\footnotesize \textsuperscript{*}Note: Sub-titles are not captured in Xplore and
%should not be used}
\thanks{This work was partially funded by “ERDF A way of making Europe” project under grant PID2021-126431OB-I00 and Spanish  Ministry of Economy and Competitiveness (MINECO)—Program UNICO I+D funded by MCIN/AEI/ 10.13039/501100011033 (grants TSI-063000-2021-54 and -55), Generalitat de Catalunya grant 2021 SGR 00770 and by UNITY-6G project, funded from European Union’s Horizon Europe Smart Networks and Services Joint Undertaking (SNS JU) research and innovation programme under the Grant Agreement No 101192650.}
}

\author{Kapal Dev$^{\ast}$, Sunder Ali Khowaja$^{\diamond}$, Engin Zeydan$^{\bigwedge}$, Keshav Singh$^{\square}$, Merouane Debbah$^{\maltese}$\\
$^{\ast} $Munster Technological University,Ireland
$^{\diamond} $Dublin City University, Dublin, Ireland\\
$^{\bigwedge} $Centre Tecnològic de Telecomunicacions de Catalunya (CTTC),  Spain\\
$^{\square}$National Sun Yat-sen University, Taiwan
$^{\maltese} $Khalifa University, Abu Dhabi, UAE\\ 
\protect Email:  kapal.dev@ieee.org, Sunderali.khowaja@dcu.ie, engin.zeydan@cttc.cat,\\ keshav.singh@mail.nysu.edu.tw, merouane.debbah@ku.ac.ae
}

%\author{\IEEEauthorblockN{Kapal Dev}
%\IEEEauthorblockA{\textit{Department of Computer Science, } \\
%\textit{Munster Technological University,Ireland}\\
%kapal.dev@ieee.org}
%\and

%\IEEEauthorblockN{Sunder Ali Khowaja}
%\IEEEauthorblockA{\textit{School of Computing} \\
%\textit{Dublin City University, Dublin, Ireland.}\\
%Sunderali.khowaja@dcu.ie}
%\and
%\\ % Start a new row

%\hfill
%\IEEEauthorblockN{Engin Zeydan}
%\IEEEauthorblockA{\textit{Centre Tecnològic de Telecomunicacions} \\
%\textit{de Catalunya, Barcelona, Spain}\\
%engin.zeydan@cttc.cat}
%\\
%\hfill
%\and 

%\IEEEauthorblockN{Merouane Debbah}
%\IEEEauthorblockA{\textit{Computer and Information Engineering } \\
%\textit{Khalifa University, Abu Dhabi, UAE}\\
% merouane.debbah@ku.ac.ae}

%}

\maketitle

\begin{abstract}

This paper investigates a range of cutting-edge technologies and architectural innovations aimed at simplifying network operations, reducing operational expenditure (OpEx), and enabling the deployment of new service models. The focus is on (i) Proposing novel, more efficient 6G architectures, with both Control and User planes enabling the seamless expansion of services, while addressing long-term 6G network evolution. (ii) Exploring advanced techniques for constrained artificial intelligence (AI) operations, particularly the design of AI agents for real-time learning, optimizing energy consumption, and the allocation of computational resources. (iii) Identifying technologies and architectures that support the orchestration of back-end services using serverless computing models across multiple domains, particularly for vertical industries. (iv) Introducing optically-based, ultra-high-speed, low-latency network architectures, with fast optical switching and real-time control, replacing conventional electronic switching to reduce power consumption by an order of magnitude.

%The expected outcomes are addressing a number of technologies and architectures that have the potential to simplify the operations of a network, decrease OpEx, or allow for new service models to be deployed. They target mainly: The definition and design of innovative more efficient and simplified 6G architectures (Control and/or User plane) enabling a seamless grow of service deployments without constraints, targeting long-term evolution of 6G architectures. The definition and performance characterisation of novel techniques for constrained AI operations, notably from an energy perspective, from real time learning perspective and for optimal deployment of computation resources. Identification and performance characterisation of technologies and architectures for provision/orchestration of a multiplicity of telco or verticals backend services through serverless computing across multiple domains and for vertical use cases. Entirely new optically based, ultra-high speed, low-latency network architectures with fast optical switching and real-time control to replace electronic switching, routing and memory, within IP routers, and reduce overall power consumption by an order of magnitude.
\end{abstract}

\begin{IEEEkeywords}
6G architectures, AI Agents, constrained AI, serverless computing, optical networks, deep learning, goal-oriented communication
\end{IEEEkeywords}



%draw.io link: https://drive.google.com/file/d/1rksl0Ai2-sjDlLZthhuMStJ_eEVICNVH/view?usp=sharing


%\hl{All final submissions should be written in English with a maximum paper length of six (6) printed pages (10-point font) including figures without incurring additional page charges (maximum 3 additional pages with over length page charge of US\$100 if accepted). Papers exceeding 9 pages will not be accepted by EDAS.}

%\tableofcontents

\section{Introduction}

%The continuous evolution of telecommunications systems necessitates the development of novel network architectures and technologies to address emerging challenges and requirements. As we move toward 6G, there is an increasing demand for scalable, efficient, and sustainable network solutions.  This paper provides an in-depth analysis of key architectural innovations that address these challenges by simplifying network operations, enhancing resource optimization, and supporting intelligent, decentralized network functions. These innovations are crucial to overcoming current limitations and advancing the state of network systems, particularly in the context of 6G.


The rapid development of telecommunications systems continues to drive the need for novel network architectures that meet the requirements of future technologies, especially in the transition to 6G \cite{raddo2021transition}. Unlike previous generations, 6G is set to revolutionize not only the performance and capacity of networks, but also enable new, intelligent services that support a variety of use cases, from immersive media to autonomous systems \cite{raddo2021transition}. 6G is also a communication and computing integrated platform as AI is considered to be its fundamental key enabler. The AI-integrated 6G system will be able to achieve higher level of automation, such as intent-based networking and automatic orchestration and maintenance (OAM). With the current evolution of AI towards large language models (LLMs) and Agentic approaches, the OAM can just simply issue intent for the network configuration to the AI agent, which then run the execution-feed-back-adjust-monitoring process along with the network tasks in an automated manner. Although, the Agentic AI is at its peak, the increasing complexity of the methods along with its support towards networks that are more scalable, efficient, and sustainable has not been discussed before. In response to these new challenges, the development of advanced network architectures integrated with Agentic AI should focus on several key aspects, such as, Reducing operational expenditure (OpEx), simplifying network operations, enabling dynamic service orchestration and integrating intelligent and autonomous functions. In this paper, we take a closer look at these architectural innovations and present solutions that go beyond the current limitations of 5G and early 6G frameworks.


One of the key innovations of this paper lies in the design of simplified network architectures that separate the control and user planes, allowing operators to deploy and manage services more efficiently through AI Agents while seamlessly integrating new network domains. This approach also addresses the limitations of existing \ac{SBA} and provides a clear path for the long-term evolution of 6G networks. The increased flexibility of such architectures ensures that devices can operate as dynamic network nodes, enabling uninterrupted roaming across multiple operators and network technologies. In addition to architectural simplification, the integration of AI Agents \ac{AI} into network management \cite{blanco2023ai} opens up new opportunities for optimization, especially in environments with energy constraints. Agentic AI-based networks can make intelligent decisions based on real-time data, but these models must work within strict energy and security constraints to ensure sustainable and secure operations. The concept of constrained AI operations, which is explored in this paper, is crucial for achieving these goals. It focuses on optimising energy consumption and learning in real-time while maintaining security protocols. In addition, the emergence of serverless computing and \ac{FaaS} offers new solutions for the orchestration of services in distributed network environments \cite{raza2021sok}. Serverless computing decouples application execution from the underlying infrastructure and enables real-time execution of lightweight functions at the network edge. This paradigm shift supports the flexible placement of AI agents and orchestration of tasks, a key requirement for the future 6G ecosystem where services must dynamically adapt to changing network conditions.

Another fundamental component of future 6G architectures is the use of autonomous cognitive agents that introduce decentralized decision-making processes \cite{wang2024survey}. These agents work without a central controller and enable dynamic and spontaneous interactions that optimize network resources and operations. This is a significant departure from traditional static architectures and enables more adaptable and resilient networks that can meet future requirements. The concept of goal-oriented communication, a revolutionary approach that enables AI-equipped devices to transmit only the information necessary to achieve specific goals \cite{strinati20216g}. This reduces unnecessary communication, computing load and energy consumption, while allowing devices to communicate more efficiently. Such innovations are crucial for next-generation machine-to-machine (M2M) communication, where optimising resource usage is key. In addition, the potential for cloudification and management of RAN functions in space underscores the increasing role of distributed data centers, including those in space, in the dynamic orchestration of RAN functions \cite{liu2022cloud}. This approach contrasts with existing static methods of resource allocation and provides a more flexible and adaptable solution that can meet the diverse requirements of 6G communication scenarios. Cloudification of RAN functions is expected to play a critical role in achieving ultra- reliable, low-latency communications for a wide range of applications. The concept of Neural Radio Protocol Stacks (NRPS) can be used  to automate the generation of radio protocol stacks tailored to specific network requirements using  Agentic AI techniques. NRPSs can be a significant advance in the development of customizable and efficient communication protocols. By integrating Agentic AI into the development of radio stacks \cite{radiostack}, these solutions can offer a high degree of flexibility and adaptability, addressing issues such as interpretability, cross-vendor compatibility and computational efficiency.  The main contributions of the paper are as follows:
\begin{itemize}
    \item We introduce novel 6G architectures and network simplification through user control plane separation and seamlessly integrating network domains through Agentic AI, enabling flexible, scalable service deployments while overcoming the limitations of current \ac{SBA}.
    \item We discuss advances in constrained Agentic AI techniques applied to network operations, focusing on energy-efficient AI models, real-time learning optimization, and secure, adaptive AI-driven processes for improved network management.
    \item We explore innovative technologies such as real-time serverless computing for dynamic orchestration of functions, autonomous cognitive agents for decentralized network operations, goal-oriented communication protocols, cloudification of the RAN in space, and neural radio protocol stacks to support advanced, efficient and adaptive next-generation networks.
\end{itemize}
    
The rest of the paper is organized as follows: Section \ref{agentic} discusses new 6G architecture models that leverage Agentic AI for flexible service provisioning, efficient resource management and real-time learning under energy and security constraints. It highlights serverless computing frameworks, decentralized cognitive agents for adaptive network management, goal-oriented communication protocols for bandwidth and energy efficiency, cloudification of \ac{RAN} functions using space-based data centers, and customizable radio protocol stacks for real-time adaptation in 6G networks.  Section \ref{experimental} provides an experimental setup and an evaluation of agent-based AI within the  \ac{V2X} scenario. Finally, Section \ref{conclusion} summarizes the key innovations discussed and outlines the key steps to realizing the potential of 6G networks, focusing on scalability, intelligence and efficiency.

%Section \ref{agentic} discusses new 6G architectural models that simplify network operations by decoupling user and control planes for flexible, seamless service deployments using Agentic AI and explores the use of Agentic AI with energy and security constraints to optimize network operations, focusing on efficient resource management and real-time learning.  It  presents serverless computing frameworks, such as \ac{FaaS}, to enable dynamic function deployment and low-latency execution in 6G networks, highlights the role of decentralized, autonomous cognitive agents that enable self-organizing, adaptive network management and decision-making. It also introduces goal-oriented communication protocols that optimize data transmission by exchanging only task-relevant information, reducing bandwidth and energy consumption. It  covers the cloudification of \ac{RAN} functions using space-based data centers for dynamic and real-time orchestration of network services and  examines Agentic AI-driven, customizable radio protocol stacks that enable efficient and real-time adaptation of communication protocols in 6G environments. 


\section{Agentic AIs}
\label{agentic}


\subsection{Novel Architectural Solutions for Simplified Networks with Agentic AI}
%\label{arch}

Fig. \ref{fig:general_arch} provides a layered diagram illustrating the evolution from 5G to 6G.  A primary objective in the design of future 6G networks is the simplification of network architectures to facilitate more efficient service deployment and operation using AI agents. In this context, the separation of the User and Control planes allows for more flexible network management and resource allocation. This separation supports seamless integration of new network domains and ensures that devices can function as autonomous network nodes with the help of AI-support services. Such architectures allow for uninterrupted user roaming across various operators and network technologies in an adaptive manner. The limitations of the existing \ac{SBA} in 5G networks are evident when considering the evolution toward 6G. Early models, such as the Hexa-X II framework \cite{hexa_x_ii_2024}, offer a foundation for addressing these shortcomings. However, further architectural refinements are needed to fully realize the potential of 6G, particularly with regard to supporting dynamic and adaptive services. Fig. \ref{fig:new_arch} shows the network diagram  that uses simplified user-control plane architecture with seamless domain integration using Agentic AI, where AI agents will be designed to optimize the services, communication, flow of data using context, semantics, and sustainability characteristics. Hence, realizing the true potential of 6G.

\begin{figure}[htp!]
\includegraphics[width=\linewidth]{General_arch_rev.png}
\centering
\caption{A layered diagram illustrating the evolution from 5G to 6G.}
\label{fig:general_arch}
\end{figure}





\begin{figure}[htp!]
\includegraphics[width=\linewidth]{Figure1.jpg}
\centering
\caption{A network diagram showing a simplified user-control plane architecture with seamless domain integration with Agentic AI}
\label{fig:new_arch}
\end{figure}



%\section{New architectural solutions targeting the simplification of the architecture }

%(user and control plane) enabling operators to introduce and operate services more efficiently, effortlessly integrate and connect new network domains (considering that any device can operate as a network node), as well as enable users to seamlessly roam across operators, and network technologies and domains. It covers current limitation of the SBA, legacy constraints and evolution from early 6G architectures (e.g. Hexa-X II model


\subsection{Agentic AI for Constrained Network Operations}



Fig. \ref{fig:Flowchart} shows the flowchart for the Agentic AI implementation, showing a network optimized for both energy and security. The integration of Agentic \ac{AI} into network operations offers significant optimization opportunities, especially in resource-constrained environments while adapting to new threats or energy constraints. Constrained AI refers to the application of predefined constraints, such as energy efficiency and security requirements, in the design and deployment of AI agents in networks. For example, AI agents can be assigned the task of collecting and analyzing data, optimizing energy efficiency and monitoring it sequentially. Same can be applied for the security constraints. However, these agents can also work together to make the decisions with semantics to integrate security and energy efficiency together. By incorporating these constraints into the learning process, it is possible to improve the performance of AI-driven network operations while ensuring compliance with domain-specific requirements.

\begin{figure}[htp!]
\includegraphics[width=\linewidth]{Figure2.jpg}
\centering
\caption{A flowchart illustrating the process of deploying Agentic AI under energy and security constraints in network operations.}
\label{fig:Flowchart}
\end{figure}

In particular, the optimization of energy usage is a critical point when using AI agents in networks. Such a limitation also has an impact on the scalability of network operations, such as the support of massive machine type communication (mMTC). Energy-efficient AI solutions are essential for sustainable network operations, especially in the context of 6G, where the density of network infrastructure is expected to increase dramatically. In addition, the 6G networks must enable communication between space-terrestrial-ground- air, so the AI agents must be both energy efficient and computationally friendly. Furthermore, ensuring the security and robustness of AI agents in real-world conditions is of paramount importance as AI is increasingly used for autonomous network management and service provisioning. With the emergence of security vulnerabilities in AI models, concerns about the deployment of AI agents in 6G networks have grown \cite{SPIN}. Therefore, AI models must be resilient to specific AI model attacks such as model inversion, model poisoning, and membership inference attacks. As for communication between AI agents, security can be performed together with energy efficiency. For example, the autonomous agents can collect and analyze data, but instead of sending it for optimization, the security agent can perform privacy checks and encrypt the data. This will not only protect the data from intruders, but also learn to optimize the encrypted data. Also, the interactions between agents will help to understand the energy constraints so that privacy checks and encryption are performed using lightweight methods.


Although there is no specific information about reducing energy consumption in constrained AI operations, we can gain some relevant insights about AI-driven energy reduction in other contexts. Regarding AI-driven energy reduction in data centers, Google and DeepMind reported significant energy savings in data center cooling. DeepMind's machine learning system reduced the amount of energy used for cooling by up to 40\%. This resulted in a 15\% reduction in overall Power Usage Effectiveness (PUE) overhead after accounting for electrical losses and other non- cooling inefficiencies\footnote{https://deepmind.google/discover/blog/
deepmind-ai-reduces-google-data-centre-cooling-bill-by-40/}. Researchers at the MIT Lincoln Laboratory Supercomputing Center (LLSC) have developed techniques to reduce energy consumption when training AI models.
By implementing power capping on GPUs, they achieved a reduction in energy consumption of 12\% to 15\%. By using an early stopping technique, they were able to reduce the energy consumption for model training by a dramatic 80\% \cite{mit_ai_energy_2023}. With regard to AI-supported energy reduction in Radio Access Networks (RAN), Ericsson reported on energy savings in the operation of mobile networks. An ML/AI-based approach to dynamically configure cell sleep modes resulted in a 10-12\% energy reduction at pilot sites \footnote{https://www.ericsson.com/en/blog/2023/1/ai-powered-ran-energy-efficiency}.



%\section{The development and evaluation of deep learning models }
%that include predefined constraints either for from a network operation viewpoint or from a user service provision viewpoint. The constraints may be in the form of physical law, logical rule, or any other domain specific knowledge, with an end-to-end connectivity perspective. Of particular interest are problems related to AI optimization under energy constraints and under security constraints. The scope covers constrained deep learning by incorporating constraints into the learning process for network processes optimisation.

\subsection{Real time serverless computing with Agentic AI for 6G Networks}
\label{serverless}

Serverless computing, particularly in the form of \ac{FaaS}  \cite{liu2023faaslight}, offers a promising solution for the dynamic orchestration of network services while employing Agentic AI as shown in Fig. \ref{fig:FaaS}. Achieving seamless scaling is difficult in existing network architectures due to the accessibility issues. By decoupling the execution of services from the underlying infrastructure, serverless computing enables the deployment of lightweight, highly-scalable functions across the network capable of automating and optimizing the task, including at the edge. In the context of 6G integrated with Agentic AI, real-time serverless computing can address several key challenges. These include mitigating the cold start problem, orchestrating high-dimensional tasks across multiple network domains, and ensuring security in distributed computing environments. The serverless computing with AI agents can also address the seamless scaling issues by employing proactive resource allocation and function pooling. The ability to dynamically allocate resources based on real-time network conditions is critical to achieving the low-latency, high-reliability performance required in next-generation networks. The latency of serverless computing (\ac{FaaS}) with Agentic AI in real time can vary greatly depending on the platform, complexity of the function and network conditions. As detailed in \cite{moreno2023latency}, cold start latency can range from 100 milliseconds to several seconds. In a typical machine learning application for image classification, cold start latencies of about 2-3 seconds. Warm start latency, which occurs when an agent is called for a particular function shortly after a previous call, is typically in the range of 10-100 milliseconds  and in a cloud scenario where a device sends image files to an AWS Lambda function deployed in a remote Amazon region, network latencies were between 70 and 120 milliseconds. In an edge scenario, where the device sends files from the same region as the Lambda function, network latencies were reduced to 0.5 to 2 milliseconds.


\begin{figure}[htp!]
\includegraphics[width=\linewidth]{Figure3.jpg}
\centering
\caption{A distributed network diagram showing the real-time execution of serverless functions with Agentic AI across edge and cloud environments.}
\label{fig:FaaS}
\end{figure}




%The scope covers provision and orchestration of database and storage services towards the implementation of “Function-as-a-Service” (FaaS). It allows execution of code on various parts of the network (e.g. edge) and support the versatile/optimised function placement expected in 6G. Also in scope is instant start to combat the cold start characteristics of the scheme, high-dimensional task orchestration across multiple stakeholders, and inherent security of such infrastructures.

\subsection{Autonomous Cognitive Agents in Network Architectures}
\label{autonomous}

The use of autonomous cognitive agents in network architectures represents a significant shift from the traditional, centralized network management models as shown in Fig. \ref{fig:agents}. Multi-Agent Systems (MAS), potentially using \glspl{LLM}, enable decentralized decision making where individual agents interact and cooperate to achieve network goals \cite{zeeshan2024large}. While it can be argued that the LLMs require computationally intensive operations, which could hinder the realization of the said network architecture, various techniques such as low-rank adapters (LoRA), quantization and others can be used to mitigate the computational complexity issues of the LLMs while reducing the computational complexity of the system. These agents can dynamically invoke each other, enabling a flexible and adaptable network infrastructure that operates in a more organic and decentralized manner. The design of architectures that support autonomous agents requires careful consideration of scalability, security and the seamless integration of service composition and knowledge handling. Decentralizing network operations not only improves efficiency, but also increases the network's ability to adapt to changing conditions, providing robustness and resilience.


\begin{figure}[htp!]
\includegraphics[width=\linewidth]{agents.png}
\centering
\caption{A decentralized agent-based network structure where autonomous agents communicate and coordinate without a central controller.}
\label{fig:agents}
\end{figure}




%The definition of architectures with a focus on simplicity, scalability, and security with a focus on the holistic combination of service composition and knowledge handling in a network-compute continuum where the network and applications merge in an organic way and operations are carried out in a decentralized manner. The Multi-agent systems, possibly but not necessarily based on LLMs, can excel the performance of individual agents. The agents can invoke each other spontaneously and can be operated in a decentralized manner, thus departing from the relatively static architectures we know today.

% \section{Goal-oriented Communication}

% Goal-oriented communication protocols \cite{getu2023making} mark a significant departure from traditional communication models, particularly in networks that rely heavily on AI-driven devices as shown in Fig. \ref{fig:goal}. These protocols are designed to optimize communication by transmitting only information that is relevant to achieving specific goals, thus reducing the overall communication overhead. By focusing on goal-relevant information, devices can minimize their computing, storage, and energy consumption.

% \begin{figure}[htp!]
% \includegraphics[width=\linewidth]{goal.png}
% \centering
% \caption{A communication model between AI-equipped devices, showing how goal-oriented protocols minimize data exchange.}
% \label{fig:goal}
% \end{figure}



% The design of goal-oriented communication protocols is particularly important in scenarios involving machine-to-machine communication, where vast amounts of data are exchanged between AI-equipped devices. By utilizing generative AI technologies, and combining with light-weight containers, devices can efficiently extract and transmit the most pertinent data in various waveform formats, thereby enhancing the overall efficiency of the network. It should also be noted that the aforementioned example only provides unimodal characteristics of generative AI, however, the same can be used to perform multimodal communication as well. 

%The definition and design of revolutionized effective goal-oriented communication protocols, languages, and media among devices and machines equipped with AI, especially those with generative AI technologies, allowing them to extract and communicate only goal-relevant information, possibly directly in various waveform formats, to reduce communication, computing, storage, and energy consumptions.


\begin{table*}[htp!]
\centering
\scriptsize
\caption{Comparison of 6G Enablers: Characteristics, Pros, and Cons}
\begin{tabular}{|p{2.5cm}|p{4cm}|p{4cm}|p{4cm}|}
\hline
\textbf{6G Enabler} & \textbf{Characteristics} & \textbf{Pros} & \textbf{Cons} \\ \hline
\textbf{Novel 6G Architectures} & 
Separation of User and Control planes using AI agents, Flexible network management and resource allocation, and autonomous network nodes & 
- Simplifies network operations \newline
- Seamless integration with new network domains \newline
- Uninterrupted user roaming & 
- Additional interfaces \newline
- Management Overhead \newline
- More read and store stateful data \\ \hline

\textbf{Agentic AI for Constrained Network Operations} & 
Energy-aware AI, Security-driven learning, Real-time optimization algorithms for network processes & 
- Reduce energy consumption 
\newline
- Sustainable network operations \newline
- Secure and Resilient system & 
- High computational requirements \newline
- Latency trade-offs in complex models \newline
- Sensitive to model privacy attacks \\ \hline

\textbf{Real-time Serverless Computing with Agentic AI} & 
Dynamic function orchestration, Edge execution, Function placement at different network layers (cloud to edge) & 
- Reduces latency to as low as 10 ms with warm start \cite{moreno2023latency} \newline
- Seamless scaling \newline
- Support for high-dimensional tasks across multiple network domains & 
- Cost of long-running processes \newline
- Lack of custom control \newline
- Difficulty in Function Management \\ \hline

\textbf{Autonomous Cognitive Agents} & 
Decentralized decision-making, Multi-agent systems (MAS), Knowledge sharing, LLM-driven coordination & 
- Hypercustomization \newline
- Improved fault tolerance (20\% increase) \newline
- Reduces dependency on centralized control & 
- Stable Training and Tokenization \newline
- Security vulnerabilities (Membership Inference Attacks) \newline
- Seamless Integration \\ \hline

% \textbf{Goal-oriented Communication Protocols} & 
% AI-based goal extraction, Reduced data transmission, Task-specific optimization protocols, Support for Generative AI & 
% - Reduces communication overhead by up to 50\% \newline
% - Optimizes bandwidth and energy usage \newline
% - Supports Generative AI capability & 
% - Complex protocol design and implementation \newline
% - Computationally Intensive \newline
% - AI model reliability and heterogeneity \\ \hline

% \textbf{In-space RAN Cloudification and Management} & 
% Dynamic orchestration of RAN functionalities via space-based data centers, high-throughput communication, Real-time RAN virtualization & 
% - Low-latency communication (below 1ms) \newline
% - Improved spectrum utilization \newline
% - Extended Footprint \newline
% - Supports multi-domain orchestration & 
% - Disaggregation with high interdependency in RAN \newline
% - High-performance computing under constrained resources \newline
% - Managing deployments geographically distributed from core and Centralized Unit (CU) \\ \hline

\textbf{Neural Radio Protocol Stacks with Agentic AI} & 
ML-based automated protocol generation, Customizable and efficient protocol stacks, Cross-vendor compatibility & 
- Optimized protocol stacks for specific use cases \newline
- Performance Enhancement \newline
- Enables real-time protocol adaptation & 
- Challenges in interpretability and explainability \newline
- Hardware constraints at the radio edge \newline
- Limited real-world PoC demonstrations \\ \hline
\end{tabular}
\label{tab:comparison}
\end{table*}



% \section{In-space RAN functions cloudification and management}

% The cloudification of Radio Access Network (RAN) functions represents a critical development in the evolution of network architectures \cite{habibi2021towards}. By enabling the dynamic orchestration of RAN functionalities, in contrast to static assignments, 6G networks can better meet the demands of diverse communication scenarios as shown in Fig. \ref{fig:satellite}. This vision is further supported by the advent of in-space data centers, which are expected to play a pivotal role in enabling low-latency, high-throughput communication services in the medium- to long-term future. In-space cloudification allows for the flexible allocation of RAN resources, optimizing performance based on real-time network conditions. In-space cloudification also extends the footprint and reachability of the network to areas, where the provision of communication services has not been made available. Domains such as mission-critical applications, rescue operations, telemedicine and e-health, can largely benefit from the In-space cloudification. This approach is particularly relevant in scenarios requiring ultra-reliable, low-latency communication (URLLC), where static resource assignments may lead to suboptimal performance.


% \begin{figure}[htp!]
% \includegraphics[width=.9\linewidth]{satellite.png}
% \centering
% \caption{A diagram showing the cloudification of RAN functions in space, with satellite-based data centers dynamically orchestrating services.}
% \label{fig:satellite}
% \end{figure}




%, allowing to dynamically orchestrate RAN functionalities (vs. static assignments in the foreseeable future) to effectively meet the requirements of different communication scenarios. This long term vision leverages on the advent of in-space data centers in the mid-term.

\subsection{Neural Radio Protocols Stacks (NRPS) with Agentic AI}

The concept of Neural Radio Protocol Stacks (NRPS) with Agentic AI uses autonomous AI agents to adapt, manage and optimize radio communication protocols in real time. The NRPS help to adapt to dynamic network conditions while collaboratively optimizing the communication networks. The use of Agentic AI for NRPS will also enable autonomous decision-making for spectrum sensing and allocation, dynamic protocol adaptation, cooperative communication and multi-agent collaboration, security and anomaly detection, energy-efficient communication, and learning and evolving protocols \cite{racz2022full}.
This introduces machine learning-based techniques to the design and customization of radio protocol stacks as shown in Fig. \ref{fig:neural}. By using neural networks, it is possible to automatically create lean, adaptive protocol stacks that are optimized for specific network conditions. However, there are still some challenges, including ensuring the interpretability of neural stacks, achieving cross-vendor compatibility, and optimizing computational efficiency.

\begin{figure}[htp!]
\includegraphics[width=\linewidth]{Figure4.jpg}
\centering
\caption{A diagram showing the automatic generation of neural radio protocol stacks using neural Agentic AI, focusing on interpretability and efficiency.}
\label{fig:neural}
\end{figure}


The development of NRPS with Agentic AI requires not only advances in algorithmic techniques, but also innovations in the hardware architectures that can support the deployment of such stacks at the radio edge. Proof-of-concept (PoC) demonstrations are essential to validate the feasibility and performance of Agentic AI-enabled NRPS in real-world scenarios. Recently, however, AI chips have been announced that promise to provide Agentic AI-based services on the edge devices. Similarly, model compression techniques and edge-optimized software frameworks could help in implementing the concept of Agentic AI-based NRPS to achieve the desired adaptation of radio protocol stacks. 

%This topic seeks innovative solutions for the automated generation of lean, customizable radio protocol stacks using ML techniques. Proposals should address the practical challenges of neural stack interpretability, cross-vendor compatibility, and computational efficiency, culminating in PoC demonstrators. Research may not only focus on algorithmic advancements but also explore novel hardware architectures that enable the deployment of NRPS at the radio edge.

Finally, Table \ref{tab:comparison} provides a comparison of the 6G enablers using Agentic AI as discussed in this paper along with their features, advantages and disadvantages.

\section{Experimental Setup and Evaluation}
\label{experimental}

To validate Agentic \ac{AI} for next-generation network architectures, we conduct a simple experiment to optimize packet delivery ratio and connection robustness in a \ac{V2X} environment. For the development of AI agents, we use LLaMA2-13B and LLaMA2-70B as our foundational pre-trained models. We design three agents from the perspective of performance optimization, sustainability (reducing carbon footprint) and spectrum efficiency using crewAI\footnote{https://github.com/crewAIInc/crewAI} and develop a Retrieval Augmented Generation (RAG) using the Telecom Q\&A \cite{TelecomQA}, ArXiv papers on telecommunication/5G/6G and 3GPP documentation. Working together is at the heart of CrewAI. When multiple AI agents collaborate, they can tackle complex problems that would overwhelm a single agent working alone. By sharing what they know and dividing up tasks, the agents create a multiplier effect - their combined intelligence lets them solve challenges more effectively than any one agent could on its own. Think of it like a well-coordinated team where each member brings their unique strengths to the table. We used about 60 ArXiv papers for the development of RAG. To make it fair and simple, we did not apply any pre-processing or specialized curation of information to train AI agents, other than feeding the RAG. We modified the method proposed in \cite{V2X} and performed an evaluation without the use of AI agents. We simulate the scenario described above for 100 - 500 devices. The mobility parameter was set to 10 - 30 m/s for vehicles in the \ac{V2X} environment. Interference was set to "Moderate to High" due to multiple active links, and the topology was set to random and clustered node placement. We evaluate performance based on the packet delivery ratio (PDR), i.e. the percentage of packets successfully delivered, and the robustness of the connection, i.e. the mean signal-to-noise ratio (SNR) and retransmission rate. Note that we repeat the simulation for a different number of parameters and give average results accordingly. Table \ref{table2} shows the comparative analysis for the parameter optimization of PDR and Link Robustness in \ac{V2X} with and without Agentic AI. These results show that Agentic AI outperforms conventional methods when optimizing parameters, even with a simple RAG and few documents passed as training data. The Agentic AI approach achieves lower latency (20ms vs. 50ms), fewer retransmissions (17\% vs. 5\%), improved link robustness (20dB vs. 30dB), and higher packet delivery ratios (95\% vs. 79\%), respectively. We believe the improvement is due to the agent's ability to dynamically adapt to changes in the network while optimizing resources and predicting failures in real time. 

\begin{table}[htp!]
\centering
\caption{Comparative analysis for parameter optimization of PDR and Link Robustness in V2X with and without Agentic AI. }
\label{tab:my-table}
\begin{tabular}{|c|c|c|}
\hline
\textbf{Metric}              & \textbf{Without AI Agents} & \textbf{With AI Agents} \\ \hline
\textbf{PDR}                 & 79\% ($\pm$ 5\%)            & 95\% ($\pm$ 2\%)         \\ \hline
\textbf{Mean SNR}            & 20 dB ($\pm$ 2dB)          & 30 dB ($\pm$ 1dB)       \\ \hline
\textbf{Retransmission Rate} & 17\%                       & 5\%                     \\ \hline
\textbf{Adaptability}        & Low                       & High                  \\ \hline
\textbf{Latency}             & 50 ms ($\pm$ 10 ms)        & 20 ms ($\pm$ 5 ms)      \\ \hline
\end{tabular}
\label{table2}
\end{table}


\section{Conclusion}
\label{conclusion}

In this paper, we provide a comprehensive examination of the key architectural advances that will shape the future of 6G networks when integrated with Agentic AI. From novel 6G architectures to Agentic AI for constrained network operations, serverless computing with Agentic AI and neural protocol stacks with Agentic AI, the innovations discussed in this paper are expected to transform the telecommunications landscape, enabling networks that are smarter, more efficient and capable of supporting a wide range of new applications. Key contributions such as novel architectural solutions that simplify network operations through the decoupling of the user and control planes through autonomous AI agents, and introducing Agenting AI-based constrained optimization techniques to reudce energy consumption and improve security in real-time network operations. Real-time serverless computing with Agentic AI offers a flexible approach to dynamic function orchestration, while autonomous cognitive agents enable decentralized and adaptive network management. 

Additionally, we explored the Neural Radio Protocol Stacks using Agentic AI that highlights the potential for customizable and efficient radio protocols in future networks. We conducted an experiment for parameter optimizing in the context of V2X with and without AI agents. We designed 3 AI agents to provide optimized parameters recursively based on the current state and evaluated the performance using PDR, link robustness, retransmission rate, adaptability, and latency. We show that the parameter optimization with Agentic AI is far better than the conventional techniques. 
% we introduced goal-oriented communication protocols, which enhance machine-to-machine interactions by reducing unnecessary data exchanges, and in-space RAN cloudification, which enables dynamic orchestration of RAN functionalities for ultra-reliable communication scenarios. 
Together, these advancements represent critical steps toward realizing the 6G vision, where networks are more adaptable, intelligent, and capable of supporting diverse applications with greater efficiency and flexibility.



  
 
%\vspace{12pt}
\balance


%\bibliographystyle{abbrv}

\bibliographystyle{plain} % or any other style you prefer
\bibliography{biblio} % without the .bib extension
%\bibliographystyle{ieeetr}
%\bibliography{biblio}  % vldb_sample.bib is the name of 


\end{document}
