%%%%%%%% ICML 2025 EXAMPLE LATEX SUBMISSION FILE %%%%%%%%%%%%%%%%%

\documentclass{article}

% Recommended, but optional, packages for figures and better typesetting:
\usepackage{microtype}
\usepackage{graphicx}
\usepackage{subfigure}
\usepackage{booktabs} % for professional tables
\usepackage{tabularx,tabulary,multirow}

% hyperref makes hyperlinks in the resulting PDF.
% If your build breaks (sometimes temporarily if a hyperlink spans a page)
% please comment out the following usepackage line and replace
% \usepackage{icml2025} with \usepackage[nohyperref]{icml2025} above.
\usepackage{hyperref}
\usepackage{caption}
\usepackage{url}            % simple URL typesetting
\usepackage{natbib}

% Attempt to make hyperref and algorithmic work together better:
\newcommand{\theHalgorithm}{\arabic{algorithm}}

% Use the following line for the initial blind version submitted for review:
\usepackage[accepted]{icml2025}

% If accepted, instead use the following line for the camera-ready submission:
% \usepackage[accepted]{icml2025}

% For theorems and such
\usepackage{amsmath}
\usepackage{amssymb}
\usepackage{mathtools}
\usepackage{amsthm}
% if you use cleveref..
\usepackage[capitalize,noabbrev]{cleveref}

%%%%%%%%%%%%%%%%%%%%%%%%%%%%%%%%
% THEOREMS
%%%%%%%%%%%%%%%%%%%%%%%%%%%%%%%%
\theoremstyle{plain}
\newtheorem{theorem}{Theorem}[section]
\newtheorem{proposition}[theorem]{Proposition}
\newtheorem{lemma}[theorem]{Lemma}
\newtheorem{corollary}[theorem]{Corollary}
\theoremstyle{definition}
\newtheorem{definition}[theorem]{Definition}
\newtheorem{assumption}[theorem]{Assumption}
\theoremstyle{remark}
\newtheorem{remark}[theorem]{Remark}

% Todonotes is useful during development; simply uncomment the next line
%    and comment out the line below the next line to turn off comments
%\usepackage[disable,textsize=tiny]{todonotes}
\usepackage{listings}
\lstdefinestyle{mystyle}{
  backgroundcolor=\color{gray!10},  % Light gray background
  basicstyle=\ttfamily\scriptsize, % Small font size
  breakatwhitespace=false,         % Lines break at non-whitespace
  breaklines=true,                 % Lines are broken
  captionpos=b,                    % Caption placed below
  commentstyle=\color{green!50!black}, % Green comments
  keepspaces=true,                 % Keep spaces in text
  keywordstyle=\color{blue},       % Blue keywords
  language=Python,                 % Set language to Python
  numbers=none,                    % Line numbers on left
  numbersep=5pt,                   % Distance between line numbers and code
  numberstyle=\tiny\color{gray},   % Tiny gray line numbers
  rulecolor=\color{black},         % Frame color
  showspaces=false,                % Don't show spaces as special characters
  showstringspaces=false,          % Don't show spaces in strings
  showtabs=false,                  % Don't show tabs as special characters
  stringstyle=\color{red!70!black}, % Red strings
  tabsize=2,                       % 2 spaces per tab
}
\newcolumntype{Y}{>{\centering\arraybackslash}X}
\newcommand{\A}[0]{{\ttfamily{A}}}
\newcommand{\B}[0]{{\ttfamily{B}}}
\newcommand{\C}[0]{{\ttfamily{C}}}
\newcommand{\D}[0]{{\ttfamily{D}}}


% The \icmltitle you define below is probably too long as a header.
% Therefore, a short form for the running title is supplied here:
\icmltitlerunning{Recursive Inference Scaling}

\begin{document}

\twocolumn[
\icmltitle{Recursive Inference Scaling:\\A Winning Path to Scalable Inference in Language and Multimodal Systems}


% It is OKAY to include author information, even for blind
% submissions: the style file will automatically remove it for you
% unless you've provided the [accepted] option to the icml2025
% package.

% List of affiliations: The first argument should be a (short)
% identifier you will use later to specify author affiliations
% Academic affiliations should list Department, University, City, Region, Country
% Industry affiliations should list Company, City, Region, Country

% You can specify symbols, otherwise they are numbered in order.
% Ideally, you should not use this facility. Affiliations will be numbered
% in order of appearance and this is the preferred way.
\icmlsetsymbol{equal}{*}

\begin{icmlauthorlist}
\icmlauthor{Ibrahim Alabdulmohsin}{gdm}
\icmlauthor{Xiaohua Zhai}{gdm}
% \icmlauthor{Firstname3 Lastname3}{comp}
% \icmlauthor{Firstname4 Lastname4}{sch}
% \icmlauthor{Firstname5 Lastname5}{yyy}
% \icmlauthor{Firstname6 Lastname6}{sch,yyy,comp}
% \icmlauthor{Firstname7 Lastname7}{comp}
% %\icmlauthor{}{sch}
% \icmlauthor{Firstname8 Lastname8}{sch}
% \icmlauthor{Firstname8 Lastname8}{yyy,comp}
%\icmlauthor{}{sch}
%\icmlauthor{}{sch}
\end{icmlauthorlist}

\icmlaffiliation{gdm}{Google Deepmind, Z\"urich, Switzerland}
% \icmlaffiliation{comp}{Company Name, Location, Country}
% \icmlaffiliation{sch}{School of ZZZ, Institute of WWW, Location, Country}

\icmlcorrespondingauthor{Ibrahim Alabdulmohsin}{ibomohsin@google.com}
% You may provide any keywords that you
% find helpful for describing your paper; these are used to populate
% the "keywords" metadata in the PDF but will not be shown in the document
\icmlkeywords{Machine Learning, ICML}

\vskip 0.3in
]

% this must go after the closing bracket ] following \twocolumn[ ...

% This command actually creates the footnote in the first column
% listing the affiliations and the copyright notice.
% The command takes one argument, which is text to display at the start of the footnote.
% The \icmlEqualContribution command is standard text for equal contribution.
% Remove it (just {}) if you do not need this facility.

%\printAffiliationsAndNotice{}  % leave blank if no need to mention equal contribution
\printAffiliationsAndNotice{
%\icmlEqualContribution
} % otherwise use the standard text.

\begin{abstract}
Recent research in language modeling reveals two scaling effects: the well-known improvement from increased training compute, and a lesser-known boost from applying more sophisticated or computationally intensive inference methods. Inspired by recent findings on the fractal geometry of language, we introduce \emph{Recursive INference Scaling} (RINS) as a complementary, plug-in recipe for scaling inference time. For a given fixed model architecture and training compute budget, RINS substantially improves language modeling performance. It also generalizes beyond pure language tasks, delivering gains in multimodal systems, including a +2\% improvement in 0-shot ImageNet accuracy for SigLIP-B/16. Additionally, by deriving data scaling laws, we show that RINS improves both the asymptotic performance limits and the scaling exponents. These advantages are maintained even when compared to state-of-the-art recursive techniques like the ``repeat-all-over'' (RAO) strategy in Mobile LLM. Finally, \emph{stochastic} RINS not only can enhance performance further but also provides the flexibility to optionally forgo increased inference computation at test time with minimal performance degradation.
\end{abstract}

\section{Introduction}
\label{sect:intro}
\section{Introduction}


\begin{figure}[t]
\centering
\includegraphics[width=0.6\columnwidth]{figures/evaluation_desiderata_V5.pdf}
\vspace{-0.5cm}
\caption{\systemName is a platform for conducting realistic evaluations of code LLMs, collecting human preferences of coding models with real users, real tasks, and in realistic environments, aimed at addressing the limitations of existing evaluations.
}
\label{fig:motivation}
\end{figure}

\begin{figure*}[t]
\centering
\includegraphics[width=\textwidth]{figures/system_design_v2.png}
\caption{We introduce \systemName, a VSCode extension to collect human preferences of code directly in a developer's IDE. \systemName enables developers to use code completions from various models. The system comprises a) the interface in the user's IDE which presents paired completions to users (left), b) a sampling strategy that picks model pairs to reduce latency (right, top), and c) a prompting scheme that allows diverse LLMs to perform code completions with high fidelity.
Users can select between the top completion (green box) using \texttt{tab} or the bottom completion (blue box) using \texttt{shift+tab}.}
\label{fig:overview}
\end{figure*}

As model capabilities improve, large language models (LLMs) are increasingly integrated into user environments and workflows.
For example, software developers code with AI in integrated developer environments (IDEs)~\citep{peng2023impact}, doctors rely on notes generated through ambient listening~\citep{oberst2024science}, and lawyers consider case evidence identified by electronic discovery systems~\citep{yang2024beyond}.
Increasing deployment of models in productivity tools demands evaluation that more closely reflects real-world circumstances~\citep{hutchinson2022evaluation, saxon2024benchmarks, kapoor2024ai}.
While newer benchmarks and live platforms incorporate human feedback to capture real-world usage, they almost exclusively focus on evaluating LLMs in chat conversations~\citep{zheng2023judging,dubois2023alpacafarm,chiang2024chatbot, kirk2024the}.
Model evaluation must move beyond chat-based interactions and into specialized user environments.



 

In this work, we focus on evaluating LLM-based coding assistants. 
Despite the popularity of these tools---millions of developers use Github Copilot~\citep{Copilot}---existing
evaluations of the coding capabilities of new models exhibit multiple limitations (Figure~\ref{fig:motivation}, bottom).
Traditional ML benchmarks evaluate LLM capabilities by measuring how well a model can complete static, interview-style coding tasks~\citep{chen2021evaluating,austin2021program,jain2024livecodebench, white2024livebench} and lack \emph{real users}. 
User studies recruit real users to evaluate the effectiveness of LLMs as coding assistants, but are often limited to simple programming tasks as opposed to \emph{real tasks}~\citep{vaithilingam2022expectation,ross2023programmer, mozannar2024realhumaneval}.
Recent efforts to collect human feedback such as Chatbot Arena~\citep{chiang2024chatbot} are still removed from a \emph{realistic environment}, resulting in users and data that deviate from typical software development processes.
We introduce \systemName to address these limitations (Figure~\ref{fig:motivation}, top), and we describe our three main contributions below.


\textbf{We deploy \systemName in-the-wild to collect human preferences on code.} 
\systemName is a Visual Studio Code extension, collecting preferences directly in a developer's IDE within their actual workflow (Figure~\ref{fig:overview}).
\systemName provides developers with code completions, akin to the type of support provided by Github Copilot~\citep{Copilot}. 
Over the past 3 months, \systemName has served over~\completions suggestions from 10 state-of-the-art LLMs, 
gathering \sampleCount~votes from \userCount~users.
To collect user preferences,
\systemName presents a novel interface that shows users paired code completions from two different LLMs, which are determined based on a sampling strategy that aims to 
mitigate latency while preserving coverage across model comparisons.
Additionally, we devise a prompting scheme that allows a diverse set of models to perform code completions with high fidelity.
See Section~\ref{sec:system} and Section~\ref{sec:deployment} for details about system design and deployment respectively.



\textbf{We construct a leaderboard of user preferences and find notable differences from existing static benchmarks and human preference leaderboards.}
In general, we observe that smaller models seem to overperform in static benchmarks compared to our leaderboard, while performance among larger models is mixed (Section~\ref{sec:leaderboard_calculation}).
We attribute these differences to the fact that \systemName is exposed to users and tasks that differ drastically from code evaluations in the past. 
Our data spans 103 programming languages and 24 natural languages as well as a variety of real-world applications and code structures, while static benchmarks tend to focus on a specific programming and natural language and task (e.g. coding competition problems).
Additionally, while all of \systemName interactions contain code contexts and the majority involve infilling tasks, a much smaller fraction of Chatbot Arena's coding tasks contain code context, with infilling tasks appearing even more rarely. 
We analyze our data in depth in Section~\ref{subsec:comparison}.



\textbf{We derive new insights into user preferences of code by analyzing \systemName's diverse and distinct data distribution.}
We compare user preferences across different stratifications of input data (e.g., common versus rare languages) and observe which affect observed preferences most (Section~\ref{sec:analysis}).
For example, while user preferences stay relatively consistent across various programming languages, they differ drastically between different task categories (e.g. frontend/backend versus algorithm design).
We also observe variations in user preference due to different features related to code structure 
(e.g., context length and completion patterns).
We open-source \systemName and release a curated subset of code contexts.
Altogether, our results highlight the necessity of model evaluation in realistic and domain-specific settings.






\section{Recursive Inference Scaling}
\label{sect:pre}
\section{Preliminary}
\label{sec:3}
% 在本节,我们将分别定义实现高效CBE的三个要素:accuracy,convergence, scalability,并分析影响它们的要素。
In this section, we start by symbolically introducing the working process of CBE. Afterwards, we introduce the key objectives for achieving efficient CBE: accuracy, convergence, and scalability, and analyze the factors that influence them.
We mainly discuss the pair-wise evaluation scenario (where the judge provides preference between two models per time) for its wide applications \citep{paireval,allpair}. Actually, list-wise preferences can be easily converted into pair-wise ones, as demonstrated in \S\ref{sec:5.4-2}, so the discussions below are general for CBE.
% 考虑到pair-wise comparison(Judge每次给出对两个models的preference)是CBE的最典型场景,我们在本节围绕pair-wsie evaluation展开讨论。事实上,list-wise preference results可以轻松地转化成pair-wise,正如第三届所示的那样。
\subsection{Process of CBE}
%一个CBE method 可以被划分为三部分:preference allocation strategy, sampling strategy以及model score estimation strategy。
Generally, a CBE method $f$ can be divided into three parts: budget allocation strategy $f^{ba}$, tuple sampling strategy $f^{ts}$, and preference aggregation strategy $f^{pa}$.
Given benchmark $\mathcal{D}:s_{1:N}$ and models under evaluation $\mathcal{M}:m_{1:M}$, we iterate the following steps: 
\textit{step 1.} applying $f^{ba}$ to attain sampling matrix $P^l$ at iteration $l$, where $P_{i,j,k}^l$ denotes the probability to select tuple $(m_i,m_j,s_k)$ for judging; 
\textit{step 2.} applying $f^{ts}$ to sample certain tuple $(m^{l1},m^{l2},s^l)$ based on $P^l$; 
\textit{step 3.} attaining preference result $r^l$ from the judge, where $r^l \in [0,1]$ denotes the degree $m^{l1}$ wins over $m^{l2}$ (0.5 means tie). 
We stop this iterative process when the preset preference budget $T$ is achieved and then apply $f^{pa}$ on preference results $\{(m^{l1},m^{l2},s^l,r^l)\}_{l=1}^T$ to attain estimated model scores $u_{1:M}$.

\subsection{Accuracy}
\label{sec:3.2}
Theoretically, if we have a budget of $\hat{T}=\frac{NM(M-1)}{2}$, we can explore all tuples to obtain the ground truth estimation for the model scores $\hat{u}_{1:M}$. 
However, typically $T$ is much smaller than $\hat{T}$ in reality considering the preciousness of preference signals. 
% 然而,在现实中考虑到preference的珍贵性T通常远小于Y. 
Previous studies \citep{samplebias1,samplebias2} have discussed the risks of introducing sampling bias in incomplete sampling scenarios, which we believe could similarly lead to potential risks in CBE. % due to inappropriate $f^{ba}$. 
% 考虑到每次采样的内容是(ml-1, ml-2, sl),我们认为sample bias存在两方面.
Considering that the content of each sample is $(m^{l1},m^{l2},s^l)$, we think the sample bias exists across both samples and models. 
% 首先,由于不同的模型可能擅长回答不同类型的问题,因此模型的能力值会随采样的sample而发生变化:

\begin{figure}[ht]
    \centering
    \subfigure[Sampling bias with different preference aggregation strategies across samples and models.]{
    \label{fig:bias1}
        \includegraphics[width=0.47\textwidth]{figs/score_bias2.pdf}
    }
    \hfill
    \subfigure[Interval distribution of bias across samples with $f^{pa}_{BT}$ as preference aggregation strategy.]{
    \label{fig:bias2}
        \includegraphics[width=0.48\textwidth]{figs/score_bias1.pdf}
    } \\
    \vspace{-0.2cm}
    \subfigure[Bias across models with $f^{pa}_{BT}$ as preference aggregation strategy.]{
    \label{fig:bias3}
        \includegraphics[width=0.95\textwidth]{figs/score_bias3.pdf}
    }
    \vspace{-0.2cm}
    \caption{Analyses of potential sampling bias risks in CBE.}
    \vspace{-0.4cm}
    \label{fig:samplebias}
\end{figure}


\paragraph{Bias across Samples.} Since different models may excel at answering different types of queries, the model scores can vary depending on the sampled data:
\begin{equation}
    \ \ \ \ \ \ \ \ \ \ \ \ \ \ \ \ \ \ \ u_t=f^{pa}(\{(m_i,m_j,s_k,r_{i,j,k})\}_{i\in 1:M,j\in i+1:M})_t = \hat{u}_t + \eta_{m_t,\text{-},s_k} \ \ \ \ \ \ \ \ \ \ \ \ \text{for} \ \ \ \forall \  t,k
    \label{eq:1}
\end{equation}
% 为了验证this,我们在alpacaeval benchmark上以GPT-4o为judge,随机选择了20个模型进行了如下实验:
% 我们首先在每个sample上遍历所有的model pair得到对应的N组preference results。然后应用f计算相应的u,并计算u与s之间差值的绝对值,也即k,并取均值。
% s中提到的三种Preference Aggregation方法计算对应的
where $\eta_{m_t,\text{-},s_k}$ represents the bias between the observed model score $u_t$ of $m_t$ and the ground truth $\hat{u}_t$ when sorely assessing on sample $s_k$.
To verify this, we conduct experiments on the AlpacaEval benchmark \citep{alpacaeval} using GPT-4o \citep{4o} as the judge across randomly selected 20 LLMs (listed in Figure~\ref{fig:bias3}). We first traversed all model pairs for samples $s_{1:N}$ to obtain corresponding $N$ sets of preference results and then calculate the respective $|\eta_{m_i,\text{-},s_k}|$ for $\ i\in 1:M$ and $\ k\in 1:N$ according to~\eqref{eq:1} (model scores are normalized to an average of 1). We calculate the average value of $|\eta_{m_i,\text{-},s_k}|$ across models and samples using different preference aggregation strategies $f^{pa}$ discussed in \S\ref{sec:2-pa}. As shown in Figure~\ref{fig:bias1}, with all kinds of $f^{pa}$, the average difference between the model scores estimated on single sample and the ground truth values exceeds 0.25, indicating a significant bias across samples. We further analyze the proportion of samples with different biases using $f^{pa}_{BT}$ in Figure~\ref{fig:bias2} and find that they overall follow a Gaussian distribution, showing the wide existence of sample bias in CBE.
\paragraph{Bias across Models.} Just as humans may perform differently when facing different opponents, models may also have varying scores when competing against different models:
\begin{equation}
    \ \ \ \ \ \ \ \ \ \ \ \ \ \ \ \ \ \ \ u_i=f^{pa}(\{(m_i,m_j,s_k,r_{i,j,k})\}_{k\in 1:N})_i = \hat{u}_i + \eta_{m_i,m_j,\text{-}} \ \ \ \ \ \ \ \ \ \ \ \ \text{for} \ \ \ \forall \  i,j
    \label{eq:2}
\end{equation}
We validate this from two perspectives:  
(1) We calculate the average $|\eta_{m_i,m_j,\text{-}}|$ according to~\eqref{eq:2} like the process above and show the results in Figure~\ref{fig:bias1}. Overall, although the bias across models is significantly lower than the bias across samples, it still exists at a scale around 0.05. We further visualize the pair-wise model score bias in Figure~\ref{fig:bias3} to validate its wide existence. 
(2) We obtain over 1.7 million pairwise preference results across 129 LLMs collected by Chatbot Arena \footnote{\url{https://storage.googleapis.com/arena_external_data/public/clean_battle_20240814_public.json}}. After excluding pairs with fewer than 50 comparisons, we calculate the pairwise win rates and find non-transitivity in 81 model triplets (win rate: $A > B$, $B > C$, $C > A$), which also verifies the existence of bias across models.

% 在



\paragraph{Uniform Allocation Brings the Least Bias.} 
Based on the discussions above, we analyze the budget allocation strategy that can introduce the least bias. 
Considering the presence of sampling bias, the estimation error of $u_i$ with $T$ evaluation budget can be expressed as follows:
%考虑到采样偏差的存在,在T次采样时,我们有以下equation:
\begin{equation}
    u_i-\hat{u}_i = \sum_{l=1}^T \1_{m^{l1}=m_i} \times \eta_{m^{l1},m^{l2},s^l}
    \label{eq:3}
\end{equation}

Considering that $u=\hat{u}$ when all the tuples are traversed, we have the following equation:
\begin{equation}
    \ \ \ \ \ \ \ \ \ \ 0 = u_i-\hat{u}_i = \sum_{j=1}^M\sum_{k=1}^N \eta_{i,j,k} \ \ \ \ \ \ \ \ \ \ \ \ \text{for} \ \ \ \forall \  i
    \label{eq:4}
\end{equation}
%此时获得最小的estimation error of $u_i$ 的目标转化为了从U个和为0的数中采样V个数,使这V个数的和的绝对值最小。
The goal of obtaining the minimum estimation error for $u_i$ is transformed into sampling \( T \) numbers (~\eqref{eq:3}) from \( MN \) numbers that sum to zero (~\eqref{eq:4}), such that the absolute value of the sum of these \( T \) numbers is minimized. We have provided a detailed proof in Appendix~\ref{app:proof} that the best strategy is completely uniform sampling. \textit{\textbf{This denotes that the score estimation error can be minimized when the preference budgets are uniformly distributed across models and samples to bring the least sampling bias.}}

% Given that $\sum_{i=1}^U X_i = 0$, we want to attain sampling set $\mathcal{S}$ that satisfies $|\mathcal{S}|=V$ and:
% \begin{equation}
%     \mathcal{S} = argmin_{\mathcal{S}} | \sum\limits_{i \in \mathcal{S}} X_i |
%     \label{eq-ap:1}
% \end{equation}
% It is easy to know that for any sampling set $\mathcal{S}$:
% \begin{equation}
%     \mathbb{E}[ \sum\limits_{i \in \mathcal{S}} X_i ]=0
%     \label{eq-ap:2}
% \end{equation}
% Thus, 
% \begin{equation}
%     \mathbb{E}[|\sum\limits_{i \in \mathcal{S}} X_i|] = \mathbb{E}[\mathrm{StdVar}[ \sum\limits_{i \in \mathcal{S}} X_i ]] =\mathbb{E}[  \sqrt{\mathrm{Var}[ \sum\limits_{i \in \mathcal{S}} X_i ]} ]
%     \label{eq-ap:3}
% \end{equation}
% Considering that:
% \begin{equation}
%     \mathrm{Var}[ \sum\limits_{i \in \mathcal{S}} X_i ] =  \sum\limits_{i \in \mathrm{set}(\mathcal{S})} c_i^2\mathrm{Var}(X_i)
%     \label{eq-ap:4}
% \end{equation}
% where $c_i$ denotes the number of $X_i$ in $\mathcal{S}$. On this basis, we derive that:
% \begin{equation}
% \begin{aligned}
%     \mathbb{E}[|\sum\limits_{i \in \mathcal{S}} X_i|] &= \sqrt{\mathbb{E}[\mathrm{Var}(X)]\sum\limits_{i \in \mathrm{set}(\mathcal{S})} c_i^2}\\
% &=\sqrt{\mathbb{E}[\mathrm{Var}(X)]}\sqrt{\sum\limits_{i \in \mathrm{set}(\mathcal{S})} c_i^2}\\
% &\geq \sqrt{\mathbb{E}[\mathrm{Var}(X)]}\sqrt{(\sum\limits_{i \in \mathrm{set}(\mathcal{S})} c_i)^2/|\mathrm{set}(\mathcal{S})|}\\
% &=V\sqrt{\mathbb{E}[\mathrm{Var}(X)]}\sqrt{ |\mathrm{set}(\mathcal{S})|^{-1} } \\
% &\geq V\sqrt{\mathbb{E}[\mathrm{Var}(X)]}\sqrt{\mathrm{min}(U,V)^{-1}}
% \end{aligned}
% \label{eq-ap:5}
% \end{equation}
% The equality condition of the first inequality is: the number of samples taken from each category is equal. The equality condition of the second inequality is: the number of sampled categories equals to $\mathrm{min}(U,V)$. These two conditions imply that a completely uniform sampling strategy is optimal.

% using $f^{pa}$ and compute the absolute difference between them and $\hat$, denoted as k, and took the average.
% 当T<T时,
% 
% 
\subsection{Convergence}
\label{sec:3.3}
% 在评测进行的过程中,随着新的preference results被观测到,模型间的胜负关系和对模型score的预估值都在发生变化。
During the evaluation process, as new preference results are continuously observed, the estimated values of the models win rate matrix and model scores also change constantly. 
To accelerate the convergence process, we analyze the uncertainty of the win rate matrix as follows.
Defining that:
\begin{equation}
    X^l_{i,j} = \frac{1}{P^l_{i,j}}r^l\1_{m^{l1}=m_i \And m^{l2}=m_j} + \frac{1}{P^l_{j,i}}(1-r^l)\1_{m^{l1}=m_j \And m^{l2}=m_i}
    \label{eq:6}
\end{equation}
The unbiased estimated win rate matrix $\Phi$ at iteration $L$ can be calculated as follows:
\begin{equation}
    \Phi^L = \frac{1}{L}\sum_{l=1}^L X^l
    \label{eq:7}
\end{equation}
We further estimate the variance matrix $\Theta$ as:
\begin{equation}
    \Theta^L = \frac{1}{L}\sum_{l=1}^L(X^l-\Phi^L)\circ (X^l-\Phi^L)
    \label{eq:8}
\end{equation}
Denoting if the model pair $(m_i,m_j)$ has been compared on sample $s_k$ after $l$ iterations as $C^l_{i,j,k}$, the uncertainty (standard deviation) of each element in the win rate matrix is as follows:
\begin{equation}
    \epsilon^l_{i,j} = \sqrt{\frac{\Theta^l_{i,j}}{\sum_{k=1}^{N} C^l_{i,j,k}}}
    \label{eq:9}
\end{equation}
Allocating the next preference budget on $(m_i,m_j)$ can reduce the uncertainty of their win rate by:
\begin{equation}
    \sqrt{\frac{\Theta^l_{i,j}}{\sum_{k=1}^{N} C^l_{i,j,k}}} - \sqrt{\frac{\Theta^l_{i,j}}{\sum_{k=1}^{N} C^l_{i,j,k}+1}}
    \label{eq:10}
\end{equation}
%在此基础上,考虑到我们的核心目标是对于所有的模型都进行准确的能力评估,因此我们需要globally平衡胜率矩阵不确定度的descent process以避免。
Considering that our core objective is to conduct accurate capability assessments for all models and estimate their ranking relationship, \textit{\textbf{we should globally ensure the uniformity of the win rate uncertainty matrix during its descending process to achieve smooth evaluation convergence}}.


\subsection{Scalability}
\label{sec:3.4}
Due to the continuous emergence of new LLMs, the demand for scalability in evaluation method is becoming increasingly prominent \citep{scalable}. 
Considering that we have evaluated $m_{1:M}$ with $T$ budgets, when model $m_{M+1}$ is introduced for assessment, a well-scalable CBE method should be able to quickly calibrate the capability estimates of $m_{1:M+1}$ with minimal additional preference budget. 
In this scenario, at the beginning stage when \( m_{M+1} \) is introduced, $\mathrm{avg}(C_{M+1,\text{-},\text{-}})$ is much smaller than $\mathrm{avg}(C_{\neq M+1,\text{-},\text{-}})$. 
According to~\eqref{eq:9}, the uncertainty at this point mainly arises from $\epsilon_{M+1}$, which is also intuitively easy to understand.
\textit{\textbf{Therefore, the key to improving scalability lies in allocating sufficient evaluation budgets to the newly added models to ensure
the uniform allocation among models, reducing the updating uncertainty.}}
%依据公式7可知此时评测的不确定性主要来自于对$m_{1:M+1}$,这也是直观上容易理解的。
%因此,提升可扩展性的关键在于sufficient evaluation budgets being allocated to new added models 从而快速减少更新带来的不确定性
% 


% 定义accuracy,convergence, scalability 
% 明确他们分别与sampling bias, observation variance, updating bias的关系
% 推导降低这三者的方法分别在于XXX

\section{Experimental Results}
\label{sect:app}

%%%%%%%%%%%%%%%%%%%%%%%%%%%%%%%%%%%%%%%%%%%%%%%%%%%%%%%%%%%%%%%%%%%%%%%%%%%%%%%%%%%%%%%%%%%%%%%%%%%%%%

%%%%%%%%%%%%%%%%%%%%%%%%%%%%%%%%%%%%%%%%%%%%%%
\begin{table*}[t]
\setlength{\tabcolsep}{3pt}
\centering
\renewcommand{\arraystretch}{1.1}
\tabcolsep=0.2cm
\begin{adjustbox}{max width=\textwidth}  % Set the maximum width to text width
\begin{tabular}{c| cccc ||  c| cc cc}
\toprule
General & \multicolumn{3}{c}{Preference} & Accuracy & Supervised & \multicolumn{3}{c}{Preference} & Accuracy \\ 
LLMs & PrefHit & PrefRecall & Reward & BLEU & Alignment & PrefHit & PrefRecall & Reward & BLEU \\ 
\midrule
GPT-J & 0.2572 & 0.6268 & 0.2410 & 0.0923 & Llama2-7B & 0.2029 & 0.803 & 0.0933 & 0.0947 \\
Pythia-2.8B & 0.3370 & 0.6449 & 0.1716 & 0.1355 & SFT & 0.2428 & 0.8125 & 0.1738 & 0.1364 \\
Qwen2-7B & 0.2790 & 0.8179 & 0.1593 & 0.2530 & Slic & 0.2464 & 0.6171 & 0.1700 & 0.1400 \\
Qwen2-57B & 0.3086 & 0.6481 & 0.6854 & 0.2568 & RRHF & 0.3297 & 0.8234 & 0.2263 & 0.1504 \\
Qwen2-72B & 0.3212 & 0.5555 & 0.6901 & 0.2286 & DPO-BT & 0.2500 & 0.8125 & 0.1728 & 0.1363 \\ 
StarCoder2-15B & 0.2464 & 0.6292 & 0.2962 & 0.1159 & DPO-PT & 0.2572 & 0.8067 & 0.1700 & 0.1348 \\
ChatGLM4-9B & 0.2246 & 0.6099 & 0.1686 & 0.1529 & PRO & 0.3025 & 0.6605 & 0.1802 & 0.1197 \\ 
Llama3-8B & 0.2826 & 0.6425 & 0.2458 & 0.1723 & \textbf{\shortname}* & \textbf{0.3659} & \textbf{0.8279} & \textbf{0.2301} & \textbf{0.1412} \\ 
\bottomrule
\end{tabular}
\end{adjustbox}
\caption{Main results on the StaCoCoQA. The left shows the performance of general LLMs, while the right presents the performance of the fine-tuned LLaMA2-7B across various strong benchmarks for preference alignment. Our method SeAdpra is highlighted in \textbf{bold}.}
\label{main}
\vspace{-0.2cm}
\end{table*}
%%%%%%%%%%%%%%%%%%%%%%%%%%%%%%%%%%%%%%%%%%%%%%%%%%%%%%%%%%%%%%%%%%%%%%%%%%%%%%%%%%%%%%%%%%%%%%%%%%%%
\begin{table}[h]
\centering
\renewcommand{\arraystretch}{1.02}
% \tabcolsep=0.1cm
\begin{adjustbox}{width=0.48\textwidth} % Adjust table width
\begin{tabularx}{0.495\textwidth}{p{1.2cm} p{0.7cm} p{0.95cm}p{0.95cm}p{0.7cm}p{0.7cm}}
     \toprule
    \multirow{2}{*}{\small \textbf{Dataset}} & \multirow{2}{*}{\small Model} & \multicolumn{2}{c}{\small Preference} & \multicolumn{2}{c}{\small Acc } \\ 
    & & \small \textit{PrefHit} & \small \textit{PrefRec} & \small \textit{Reward} & \small \textit{Rouge} \\ 
    \midrule
    \multirow{2}{*}{\small \textbf{Academia}}   & \small PRO & 33.78 & 59.56 & 69.94 & 9.84 \\ 
                                & \small \textbf{Ours} & 36.44 & 60.89 & 70.17 & 10.69 \\ 
    \midrule
    \multirow{2}{*}{\small \textbf{Chemistry}}  & \small PRO & 36.31 & 63.39 & 69.15 & 11.16 \\ 
                                & \small \textbf{Ours} & 38.69 & 64.68 & 69.31 & 12.27 \\ 
    \midrule
    \multirow{2}{*}{\small \textbf{Cooking}}    & \small PRO & 35.29 & 58.32 & 69.87 & 12.13 \\ 
                                & \small \textbf{Ours} & 38.50 & 60.01 & 69.93 & 13.73 \\ 
    \midrule
    \multirow{2}{*}{\small \textbf{Math}}       & \small PRO & 30.00 & 56.50 & 69.06 & 13.50 \\ 
                                & \small \textbf{Ours} & 32.00 & 58.54 & 69.21 & 14.45 \\ 
    \midrule
    \multirow{2}{*}{\small \textbf{Music}}      & \small PRO & 34.33 & 60.22 & 70.29 & 13.05 \\ 
                                & \small \textbf{Ours} & 37.00 & 60.61 & 70.84 & 13.82 \\ 
    \midrule
    \multirow{2}{*}{\small \textbf{Politics}}   & \small PRO & 41.77 & 66.10 & 69.52 & 9.31 \\ 
                                & \small \textbf{Ours} & 42.19 & 66.03 & 69.74 & 9.38 \\ 
    \midrule
    \multirow{2}{*}{\small \textbf{Code}} & \small PRO & 26.00 & 51.13 & 69.17 & 12.44 \\ 
                                & \small \textbf{Ours} & 27.00 & 51.77 & 69.46 & 13.33 \\ 
    \midrule
    \multirow{2}{*}{\small \textbf{Security}}   & \small PRO & 23.62 & 49.23 & 70.13 & 10.63 \\ 
                                & \small \textbf{Ours} & 25.20 & 49.24 & 70.92 & 10.98 \\ 
    \midrule
    \multirow{2}{*}{\small \textbf{Mean}}       & \small PRO & 32.64 & 58.05 & 69.64 & 11.51 \\ 
                                & \small \textbf{Ours} & \textbf{34.25} & \textbf{58.98} & \textbf{69.88} & \textbf{12.33} \\ 
    \bottomrule
\end{tabularx}
\end{adjustbox}
\caption{Main results (\%) on eight publicly available and popular CoQA datasets, comparing the strong list-wise benchmark PRO and \textbf{ours with bold}.}
\label{public}
\end{table}



%%%%%%%%%%%%%%%%%%%%%%%%%%%%%%%%%%%%%%%%%%%%%%%%%%%%%
\begin{table}[h]
\centering
\renewcommand{\arraystretch}{1.02}
\begin{tabularx}{0.48\textwidth}{p{1.45cm} p{0.56cm} p{0.6cm} p{0.6cm} p{0.50cm} p{0.45cm} X}
\toprule
\multirow{2}{*}{Method} & \multicolumn{3}{c}{Preference \((\uparrow)\)} & \multicolumn{3}{c}{Accuracy \((\uparrow)\)} \\ \cmidrule{2-4} \cmidrule{5-7}
& \small PrefHit & \small PrefRec & \small Reward & \small CoSim & \small BLEU & \small Rouge \\ \midrule
\small{SeAdpra} & \textbf{34.8} & \textbf{82.5} & \textbf{22.3} & \textbf{69.1} & \textbf{17.4} & \textbf{21.8} \\ 
\small{-w/o PerAl} & \underline{30.4} & 83.0 & 18.7 & 68.8 & \underline{12.6} & 21.0 \\
\small{-w/o PerCo} & 32.6 & 82.3 & \underline{24.2} & 69.3 & 16.4 & 21.0 \\
\small{-w/o \(\Delta_{Se}\)} & 31.2 & 82.8 & 18.6 & 68.3 & \underline{12.4} & 20.9 \\
\small{-w/o \(\Delta_{Po}\)} & \underline{29.4} & 82.2 & 22.1 & 69.0 & 16.6 & 21.4 \\
\small{\(PerCo_{Se}\)} & 30.9 & 83.5 & 15.6 & 67.6 & \underline{9.9} & 19.6 \\
\small{\(PerCo_{Po}\)} & \underline{30.3} & 82.7 & 20.5 & 68.9 & 14.4 & 20.1 \\ 
\bottomrule
\end{tabularx}
\caption{Ablation Results (\%). \(PerCo_{Se}\) or \(PerCo_{Po}\) only employs Single-APDF in Perceptual Comparison, replacing \(\Delta_{M}\) with \(\Delta_{Se}\) or \(\Delta_{Po}\). The bold represents the overall effect. The underlining highlights the most significant metric for each component's impact.}
\label{ablation}
% \vspace{-0.2cm}
\end{table}

\subsection{Dataset}

% These CoQA datasets contain questions and answers from the Stack Overflow data dump\footnote{https://archive.org/details/stackexchange}, intended for training preference models. 

Due to the additional challenges that programming QA presents for LLMs and the lack of high-quality, authentic multi-answer code preference datasets, we turned to StackExchange \footnote{https://archive.org/details/stackexchange}, a platform with forums that are accompanied by rich question-answering metadata. Based on this, we constructed a large-scale programming QA dataset in real-time (as of May 2024), called StaCoCoQA. It contains over 60,738 programming directories, as shown in Table~\ref{tab:stacocoqa_tags}, and 9,978,474 entries, with partial data statistics displayed in Figure~\ref{fig:dataset}. The data format of StaCoCoQA is presented in Table~\ref{fig::stacocoqa}.

The initial dataset \(D_I\) contains 24,101,803 entries, and is processed by the following steps:
(1) Select entries with "Questioner-picked answer" pairs to represent the preferences of the questioners, resulting in 12,260,106 entries in the \(D_Q\).
(2) Select data where the question includes at least one code block to focus on specific-domain programming QA, resulting in 9,978,474 entries in the dataset \(D_C\).
(3) All HTML tags were cleaned using BeautifulSoup \footnote{https://beautiful-soup-4.readthedocs.io/en/latest/} to ensure that the model is not affected by overly complex and meaningless content.
(4) Control the quality of the dataset by considering factors such as the time the question was posted, the size of the response pool, the difference between the highest and lowest votes within a pool, the votes for each response, the token-level length of the question and the answers, which yields varying sizes: 3K, 8K, 18K, 29K, and 64K. 
The controlled creation time variable and the data details after each processing step are shown in Table~\ref{tab:statistics}.

To further validate the effectiveness of SeAdpra, we also select eight popular topic CoQA datasets\footnote{https://huggingface.co/datasets/HuggingFaceH4/stack-exchange-preferences}, which have been filtered to meet specific criteria for preference models \cite{askell2021general}. Their detailed data information is provided in Table~\ref{domain}.
% Examples of some control variables are shown in Table~\ref{tab:statistics}.
% \noindent\textbf{Baselines}. 
% Following the DPO \cite{rafailov2024direct}, we evaluated several existing approaches aligned with human preference, including GPT-J \cite{gpt-j} and Pythia-2.8B \cite{biderman2023pythia}.  
% Next, we assessed StarCoder2 \cite{lozhkov2024starcoder}, which has demonstrated strong performance in code generation, alongside several general-purpose LLMs: Qwen2 \cite{qwen2}, ChatGLM4 \cite{wang2023cogvlm, glm2024chatglm} and LLaMA serials \cite{touvron2023llama,llama3modelcard}.
% Finally, we fine-tuned LLaMA2-7B on the StaCoCoQA and compared its performance with other strong baselines for supervised learning in preference alignment, including SFT, RRHF \cite{yuan2024rrhf}, Silc \cite{zhao2023slic}, DPO, and PRO \cite{song2024preference}.
%%%%%%%%%%%%%%%%%%%%%%%%%%%%%%%%%%%%%%%%%%%%%%%%%%%%%%%%%%%%%%%%%%%%%%%%%%%%%%%%%%%%%%%%%%%%%%%%%%%%%%%%%%%%%%%%%%%%%%%%%%%%%%%%%%

% For preference evaluation, traditional win-rate assessments are costly and not scalable. For instance, when an existing model \(M_A\) is evaluated against comparison methods \((M_B, M_C, M_D)\) in terms of win rates, upgrading model \(M_A\) would necessitate a reevaluation of its win rates against other models. Furthermore, if a new comparison method \(M_E\) is introduced, the win rates of model \(M_A\) against \(M_E\) would also need to be reassessed. Whether AI or humans are employed as evaluation mediators, binary preference between preferred and non-preferred choices or to score the inference results of the modified model, the costs of this process are substantial. 
% Therefore, from an economic perspective, we propose a novel list preference evaluation method. We utilize manually ranking results as the gold standard for assessing human preferences, to calculate the Hit and Recall, referred to as PrefHit and PrefRecall, respectively. Regardless of whether improving model \(M_A\) or expanding comparison method \(M_E\), only the calculation of PrefHit and PrefRecall for the modified model is required, eliminating the need for human evaluation. 
% We also employ a professional reward model\footnote{https://huggingface.co/OpenAssistant/reward-model-deberta-v3-large}
% for evaluation, denoted as the Reward metric.

% \subsection{Baseline} 
% Following the DPO \cite{rafailov2024direct}, we evaluated several existing approaches aligned with human preference, including GPT-J \cite{gpt-j} and Pythia-2.8B \cite{biderman2023pythia}.  
% Next, we assessed StarCoder2 \cite{lozhkov2024starcoder}, which has demonstrated strong performance in code generation, alongside several general-purpose LLMs: Qwen2 \cite{qwen2}, ChatGLM4 \cite{wang2023cogvlm, glm2024chatglm} and LLaMA serials \cite{touvron2023llama,llama3modelcard}.
% Finally, we fine-tuned LLaMA2-7B on the StaCoCoQA and compared its performance with other strong baselines for supervised learning in preference alignment, including SFT, RRHF \cite{yuan2024rrhf}, Silc \cite{zhao2023slic}, DPO, and PRO \cite{song2024preference}.
\subsection{Evaluation Metrics}
\label{sec: metric}
For preference evaluation, we design PrefHit and PrefRecall, adhering to the "CSTC" criterion outlined in Appendix \ref{sec::cstc}, which overcome the limitations of existing evaluation methods, as detailed in Appendix \ref{metric::mot}.
In addition, we demonstrate the effectiveness of thees new evaluation from two main aspects: 1) consistency with traditional metrics, and 2) applicability in different application scenarios in Appendix \ref{metric::ana}.
Following the previous \cite{song2024preference}, we also employ a professional reward.
% Following the previous \cite{song2024preference}, we also employ a professional reward model\footnote{https://huggingface.co/OpenAssistant/reward-model-deberta-v3-large} \cite{song2024preference}, denoted as the Reward.

For accuracy evaluation, we alternately employ BLEU \cite{papineni2002bleu}, RougeL \cite{lin2004rouge}, and CoSim. Similar to codebertscore \cite{zhou2023codebertscore}, CoSim not only focuses on the semantics of the code but also considers structural matching.
Additionally, the implementation details of SeAdpra are described in detail in the Appendix \ref{sec::imp}.
\subsection{Main Results}
We compared the performance of \shortname with general LLMs and strong preference alignment benchmarks on the StaCoCoQA dataset, as shown in Table~\ref{main}. Additionally, we compared SeAdpra with the strongly supervised alignment model PRO \cite{song2024preference} on eight publicly available CoQA datasets, as presented in Table~\ref{public} and Figure~\ref{fig::public}.

\textbf{Larger Model Parameters, Higher Preference.}
Firstly, the Qwen2 series has adopted DPO \cite{rafailov2024direct} in post-training, resulting in a significant enhancement in Reward.
In a horizontal comparison, the performance of Qwen2-7B and LLaMA2-7B in terms of PrefHit is comparable.
Gradually increasing the parameter size of Qwen2 \cite{qwen2} and LLaMA leads to higher PrefHit and Reward.
Additionally, general LLMs continue to demonstrate strong capabilities of programming understanding and generation preference datasets, contributing to high BLEU scores.
These findings indicate that increasing parameter size can significantly improve alignment.

\textbf{List-wise Ranking Outperforms Pair-wise Comparison.}
Intuitively, list-wise DPO-PT surpasses pair-wise DPO-{BT} on PrefHit. Other list-wise methods, such as RRHF, PRO, and our \shortname, also undoubtedly surpass the pair-wise Slic.

\textbf{Both Parameter Size and Alignment Strategies are Effective.}
Compared to other models, Pythia-2.8B achieved impressive results with significantly fewer parameters .
Effective alignment strategies can balance the performance differences brought by parameter size. For example, LLaMA2-7B with PRO achieves results close to Qwen2-57B in PrefHit. Moreover, LLaMA2-7B combined with our method SeAdpra has already far exceeded the PrefHit of Qwen2-57B.

\textbf{Rather not Higher Reward, Higher PrefHit.}
It is evident that Reward and PrefHit are not always positively correlated, indicating that models do not always accurately learn human preferences and cannot fully replace real human evaluation. Therefore, relying solely on a single public reward model is not sufficiently comprehensive when assessing preference alignment.

% In conclusion, during ensuring precise alignment, SeAdpra will focuse on PrefHit@1, even though the trade-off between PrefHit and PrefRecall is a common issue and increasing recall may sometimes lead to a decrease in hit rate. The positive correlation between Reward and BLEU, indicates that improving the quality of the generated text typically enhances the Reward. 
% Most importantly, evaluating preferences solely based on reward is clearly insufficient, as a high reward does not necessarily correspond to a high PrefHit or PrefRecall.
%%%%%%%%%%%%%%%%%%%%%%%%%%%%%%%%%%%%%%%%%%%
%%%%%%%%%%%%
\begin{figure}
  \centering
  \begin{subfigure}{0.49\linewidth}
    \includegraphics[width=\linewidth]{latex/pic/hit.png}
    \caption{The PrefHit}
    \label{scale:hit}
  \end{subfigure}
  \begin{subfigure}{0.49\linewidth}
    \includegraphics[width=\linewidth]{latex/pic/Recall.png}
    \caption{The PrefRecall}
    \label{scale:recall}
  \end{subfigure}
  \medskip
  \begin{subfigure}{0.48\linewidth}
    \includegraphics[width=\linewidth]{latex/pic/reward.png}
    \caption{The Reward}
    \label{scale:reward}
  \end{subfigure}
  \begin{subfigure}{0.48\linewidth}
    \includegraphics[width=\linewidth]{latex/pic/bleu.png}
    \caption{The BLEU}
    \label{scale:bleu}
  \end{subfigure}
  \caption{The performance with Confidence Interval (CI) of our SeAdpra and PRO at different data scales.}
  \label{fig:scale}
  % \vspace{-0.2cm}
\end{figure}
%%%%%%%%%%%%%%%%%%%%%%%%%%%%%%%%%%%%%%%%%%%%%%%%%%%%%%%%%%%%%%%%%%%%%%%%%%%%%%%%%%%%%%%%%%%%%%%%%%%%%%%%%%%%%%%%

\subsection{Ablation Study}

In this section, we discuss the effectiveness of each component of SeAdpra and its impact on various metrics. The results are presented in Table \ref{ablation}.

\textbf{Perceptual Comparison} aims to prevent the model from relying solely on linguistic probability ordering while neglecting the significance of APDF. Removing this Reward will significantly increase the margin, but PrefHit will decrease, which may hinder the model's ability to compare and learn the preference differences between responses.

\textbf{Perceptual Alignment} seeks to align with the optimal responses; removing it will lead to a significant decrease in PrefHit, while the Reward and accuracy metrics like CoSim will significantly increase, as it tends to favor preference over accuracy.

\textbf{Semantic Perceptual Distance} plays a crucial role in maintaining semantic accuracy in alignment learning. Removing it leads to a significant decrease in BLEU and Rouge. Since sacrificing accuracy recalls more possibilities, PrefHit decreases while PrefRecall increases. Moreover, eliminating both Semantic Perceptual Distance and Perceptual Alignment in \(PerCo_{Po}\) further increases PrefRecall, while the other metrics decline again, consistent with previous observations.


\textbf{Popularity Perceptual Distance} is most closely associated with PrefHit. Eliminating it causes PrefHit to drop to its lowest value, indicating that the popularity attribute is an extremely important factor in code communities.

% In summary, each module has a varying impact on preference and accuracy, but all outperform their respective foundation models and other baselines, as shown in Table \ref{main}, proving their effectiveness.


\subsection{Analysis and Discussion}

\textbf{SeAdpra adept at high-quality data rather than large-scale data.}
In StaCoCoQA, we tested PRO and SeAdpra across different data scales, and the results are shown in Figure~\ref{fig:scale}.
Since we rely on the popularity and clarity of questions and answers to filter data, a larger data scale often results in more pronounced deterioration in data quality. In Figure~\ref{scale:hit}, SeAdpra is highly sensitive to data quality in PrefHit, whereas PRO demonstrates improved performance with larger-scale data. Their performance on Prefrecall is consistent. In the native reward model of PRO, as depicted in Figure~\ref{scale:reward}, the reward fluctuations are minimal, while SeAdpra shows remarkable improvement.

\textbf{SeAdpra is relatively insensitive to ranking length.} 
We assessed SeAdpra's performance on different ranking lengths, as shown in Figure 6a. Unlike PRO, which varied with increasing ranking length, SeAdpra shows no significant differences across different lengths. There is a slight increase in performance on PrefHit and PrefRecall. Additionally, SeAdpra performs better at odd lengths compared to even lengths, which is an interesting phenomenon warranting further investigation.


\textbf{Balance Preference and Accuracy.} 
We analyzed the effect of control weights for Perceptual Comparisons in the optimization objective on preference and accuracy, with the findings presented in Figure~\ref{para:weight}.
When \( \alpha \) is greater than 0.05, the trends in PrefHit and BLEU are consistent, indicating that preference and accuracy can be optimized in tandem. However, when \( \alpha \) is 0.01, PrefHit is highest, but BLEU drops sharply.
Additionally, as \( \alpha \) changes, the variations in PrefHit and Reward, which are related to preference, are consistent with each other, reflecting their unified relationship in the optimization. Similarly, the variations in Recall and BLEU, which are related to accuracy, are also consistent, indicating a strong correlation between generation quality and comprehensiveness. 

%%%%%%%%%%%%%%%%%%%%%%%%%%%%%%%%%%%%%%%%%%%%%%%%%%%%%%%%%%%%%%%%%%%%%%%%%%%%%%%%%
\begin{figure}
  \centering
  \begin{subfigure}{0.475\linewidth}
    \includegraphics[width=\linewidth]{latex/pic/Rank1.png}
    \caption{Ranking length}
    \label{para:rank}
  \end{subfigure}
  \begin{subfigure}{0.475\linewidth}
    \includegraphics[width=\linewidth]{latex/pic/weights1.png}
    \caption{The \(\alpha\) in \(Loss\)}
    \label{para:weight}
  \end{subfigure}
  \caption{Parameters Analysis. Results of experiments on different ranking lengths and the weight \(\alpha\) in \(Loss\).}
  \label{fig:para}
  % \vspace{-0.2cm}
\end{figure}
%%%%%%%%%%%%%%%%%%%%%%%%%%%%%%%%%%%%%%%%%%%%
\begin{figure*}
  \centering
  \begin{subfigure}{0.305\linewidth}
    \includegraphics[width=\linewidth]{latex/pic/se2.pdf}
    \caption{The \(\Delta_{Se}\)}
    \label{visual:se}
  \end{subfigure}
  \begin{subfigure}{0.305\linewidth}
    \includegraphics[width=\linewidth]{latex/pic/po2.pdf}
    \caption{The \(\Delta_{Po}\)}
    \label{visual:po}
  \end{subfigure}
  \begin{subfigure}{0.305\linewidth}
    \includegraphics[width=\linewidth]{latex/pic/sv2.pdf}
    \caption{The \(\Delta_{M}\)}
    \label{visual:sv}
  \end{subfigure}
  \caption{The Visualization of Attribute-Perceptual Distance Factors (APDF) matrix of five responses. The blue represents the response with the highest APDF, and SeAdpra aligns with the fifth response corresponding to the maximum Multi-APDF in (c). The green represents the second response that is next best to the red one.}
  \label{visual}
  % \vspace{-0.2cm}
\end{figure*}
%%%%%%%%%%%%%%%%%%%%%%%%%%%%%%%%%%%%%%%%%
\textbf{Single-APDF Matrix Cannot Predict the Optimal Response.} We randomly selected a pair with a golden label and visualized its specific iteration in Figure~\ref{visual}.
It can be observed that the optimal response in a Single-APDF matrix is not necessarily the same as that in the Multi-APDF matrix.
Specifically, the optimal response in the Semantic Perceptual Factor matrix \(\Delta_{Se}\) is the fifth response in Figure~\ref{visual:se}, while in the Popularity Perceptual Factor matrix \(\Delta_{Po}\) (Figure~\ref{visual:po}), it is the third response. Ultimately, in the Multiple Perceptual Distance Factor matrix \(\Delta_{M}\), the third response is slightly inferior to the fifth response (0.037 vs. 0.038) in Figure~\ref{visual:sv}, and this result aligns with the golden label.
More key findings regarding the ADPF are described in Figure \ref{fig::hot1} and Figure \ref{fig::hot2}.

\section{Stochastic Recursive Inference Scaling}
\label{sect:stoch}
\begin{figure}
    \centering

\begin{lstlisting}[language=Python, style=mystyle]
class RecursiveBlock():
  config: Dict  # single block config
  signature: str = "a"
  degree: int = 1
  p_skip: Tuple[float, ...]]  # skip prob for blocks

  def __call__(self, x):
    """Call model on input x."""
    if degree == 1:
      blocks = {c: SingleBlock(config=self.config
      ) for c in set(self.signature)}
    else:
      blocks = {c: RecursiveBlock(
        config=self.config, signature=self.signature,
        degree=self.degree - 1, p_skip=self.p_skip
      ) for c in set(self.signature)}
    inputs = x
    # forward pass
    for i in range(len(self.signature)):
      c = self.signature[i]
      choice = random.uniform()
      if self.degree == 1:  # stochastic RINS
        if choice > self.p_skip[i]:
            x = blocks[c](x)  # else skip
      else:
        x = blocks[c](x)
    return x
\end{lstlisting}
    \caption{Numpy-like syntax for models with a fixed signature and degree. When no stochastic depth is applied, we have $p_{s} = 0$. In RINS, we expand $p_{s}$ into a tuple of the form $(0, p_s, p_s, \ldots, p_s, 0)$, where the first and last entries are zero to guarantee they are executed, which is equivalent to sampling the number of recursion rounds from a binomial distribution as described in Section~\ref{sect:stoch}.}
    \label{fig:ris_algorithm}
\end{figure}

\begin{figure}[t]
    \centering
    \includegraphics[width=0.49\columnwidth]{figures/ris_lm/llm_long_dur_c4.pdf}
    \includegraphics[width=0.49\columnwidth]{figures/ris_lm/llm_long_dur_sp.pdf}
    \caption{
Performance of RINS (\A$^2$\B, \A$^3$\B, \A$^4$\B) vs. RAO (\A$^2$, \A$^3$, \A$^4$) with increasing compute on 600M-parameter models. All models have the same size. The performance advantage of RINS grows with the computational budget. The long-sequence baseline is not shown since it underperforms other methods in Figure~\ref{fig:llm_sweep}.}
    \label{fig:llm_long_dur}
\end{figure}

Next, we investigate the effect of stochastically dropping blocks during training, inspired by the regularization technique of stochastic depth~\cite{huang2016deepnetworksstochasticdepth}. Our primary goal is to determine whether this approach can further enhance the performance of RINS while simultaneously offering the flexibility of \emph{reverting} to non-recursive inference without significant degradation in model quality.

To recall, RINS has the signature \A$^r$\B\ for some $r>1$. To implement stochastic RINS, we introduce a skip probability $p_s\in[0, 1)$ and sample during training the number of recursion rounds in each step to be $1+\eta$, where $\eta$ is a binomial random variable with probability of success $1-p_s$ and number of trials $r-1$. Thus, block \A\ is always executed at least once. During inference, we are free to choose how to scale inference compute by setting the value of $r\ge 1$. See the detailed pseudocode in Figure~\ref{fig:ris_algorithm}.


Here, we train bigger models with 1 billion parameters. We use an embedding dimension 2,048 and MLP dimension 8,192. All models have 18 decoder blocks. We train for 500K steps and compare RINS with signature \A$^3$\B\ against the non-recursive baseline.

Figure~\ref{fig:stoch_lm} summarizes the advantages of stochastic RINS. Notably, we observe that as $p_s>0$ increases, stochastic RINS mitigates the performance degradation incurred when scaling is not applied at inference time, while still offering big gains when inference time is scaled. Not surprisingly, though, scaling inference time is less effective when $p_s$ increases, suggesting a tradeoff between flexibility at inference time and the magnitude of potential gains from scaling. As shown in Figure~\ref{fig:infinite_lim}, similar conclusions hold even in the asymptotic (infinite-compute) regime, under the assumption that loss follows a power law relation~\cite{kaplan2020scaling}.


\begin{figure}[t]
    \centering
    \includegraphics[width=0.49\columnwidth]{figures/stoch/llm_stochdepth_c4_0.pdf}
    \includegraphics[width=0.49\columnwidth]{figures/stoch/llm_stochdepth_sp_0.pdf}
    \includegraphics[width=0.49\columnwidth]{figures/stoch/llm_stochdepth_c4_1.pdf}
    \includegraphics[width=0.49\columnwidth]{figures/stoch/llm_stochdepth_sp_1.pdf}
    \includegraphics[width=0.49\columnwidth]{figures/stoch/llm_stochdepth_c4_2.pdf}
    \includegraphics[width=0.49\columnwidth]{figures/stoch/llm_stochdepth_sp_2.pdf}
    \caption{Performance of stochastic RINS (\A$^3$\B) with varying inference costs for 1B parameter LMs. The $x$-axis represents the training compute cost. The legend indicates the inference cost of each stochastic RINS configuration relative to the baseline; e.g. $1.5x$ denotes 50\% increase in inference cost. For $p_s=0$, RINS@1x is significantly worse, with perplexity scores $>3$. As expected,  RINS converges in performance to the baseline as $p_s\to 1$.}
    \label{fig:stoch_lm}
\end{figure}

\begin{figure}[t]
    \centering
    \includegraphics[width=0.49\columnwidth]{figures/stoch/infinite_lim_c4_stoch.pdf}
    \includegraphics[width=0.49\columnwidth]{figures/stoch/infinite_lim_sp_stoch.pdf}
    \caption{Asymptotic performance of 1B-parameter LMs, evaluated in C4 (left) and SlimPajama (right). Dotted line is for baseline. We observe a tradeoff between inference flexibility (gap between 1x \& 2x) and magnitude of gain from inference scaling (2x result).}
    \label{fig:infinite_lim}
\end{figure}

\section{Other Modalities}
\label{sect:others}
\subsection{Vision}\label{sect:vision}
As previously discussed, the performance gains in RINS are consistent with the self-similar nature of language. By performing a recursive, scale-invariant decoding, RINS introduces an inductive bias that encourages the model to recognize and exploit recurring patterns at different scales (see Appendix~\ref{app:selfsim} for further discussion). To test if this is likely the source of its advantage, we conduct a similar empirical evaluation in vision, a domain lacking self-similarity.

\textbf{Setup.} We train an encoder-only vision transformer ViT-B/16~\cite{dosovitskiy2021imageworth16x16words} on ImageNet-ILSRCV2012~\cite{deng2009imagenet}. The non-recursive baseline model is trained for either 300 or 1,000 epochs using a batch size 1,024, while recursive models are trained on fewer epochs to match the same total training compute FLOPs. We apply MixUp (probability 0.2) during training~\cite{zhang2018mixupempiricalriskminimization} and use learned position embedding. The optimizer is Adam where we tune the learning rate for each architecture in the set $\mathrm{lr}\in\{10^{-3}, 7\times10^{-4}, 3\times10^{-4}, 10^{-4}\}$ with weight decay $\mathrm{wd}=\mathrm{lr}/10$, on a small validation split. We use a cosine learning rate schedule with 10K warm-up steps. Images are resized to $224\times224$ and $16\times16$ patch sizes are used. The full training configuration is in Appendix~\ref{sect:app_config}.

\textbf{Results.} As presented in Table~\ref{tab:vision}, parameter-sharing techniques, including RINS, do not confer any advantage in supervised image classification. The non-recursive baseline, when trained on longer sequence lengths (i.e., higher image resolution) to match the inference cost of recursive architectures, surpasses all other methods. This starkly differs from the results observed in language modeling, where RINS provides significant gains even when compared against models that scale inference by increasing the sequence length.  

\begin{table}[t]
    \centering\scriptsize
    \begin{tabularx}{\columnwidth}{@{}l|YYY|Y@{}}
    \toprule
    \bf Architecture &\bf Val &\bf ReaL &\bf v2 &\bf Avg \\
    \midrule
    \multicolumn{5}{c}{300 epochs}\\ \midrule
    (\A\B\B)$_2$ &\bf75.2 &\bf 81.4 &\bf 62.7 &\bf 73.1\\
    \A@336 &\bf75.7 &\bf81.0 &\bf62.3 &\bf73.0 \\
    \A\A\B & 75.2 & 80.5 & 61.4 & 72.4\\
    \A\B\B\C & 75.1 & 80.2 & 61.3 & 72.2 \\
    \A@224 & 74.9 & 80.1 & 61.3 & 72.1 \\
    \midrule
    \multicolumn{5}{c}{1,000 epochs}\\ \midrule
    \A@336	&\bf77.6 &\bf82.7 &\bf64.6 & \bf75.0 \\
    \A\B\B\C	&76.3	&81.6	&63.2	&73.7 \\
    \A\A\B\C	&76.0	&81.4	&62.4	&73.2 \\
    \A\A	&75.8	&81.4	&62.4	&73.3 \\
    \A\A\A	&75.7	&80.8	&62.0	&72.8 \\
 \bottomrule
    \end{tabularx}
    \caption{Performance of the top 5 architectures on ILSRCV2012 classification. The table compares the performance of recursive architectures with a baseline trained 224- and 336-resolution images. The baseline \A@336, trained on higher-resolution to match the inference cost of the recursive models, outperforms all parameter-sharing architectures. We use the notation $(\mathrm{signature})_\mathrm{degree}$.}
    \label{tab:vision}
\end{table}

\subsection{Contrastive Mutlimodal Systems}\label{sect:siglip}
\paragraph{Setup.}
Finally, we also study the impact of RINS in vision-language pretraining, motivated by the fact that such models also process natural language. We pretrain SigLIP-B/16 models~\cite{zhai2023sigmoidlosslanguageimage}, which are contrastive models trained using the sigmoid loss on English-language image–text pairs. We  follow~\cite{zhai2023sigmoidlosslanguageimage} in most aspects. Images are resized to $256\times256$ and we use $16\times16$ patch sizes.  Texts, on the other hand, are tokenized using C4 tokenizer~\cite{t5} with a vocabulary size of 32K, and we keep a maximum
of 64 tokens. The optimizer is Adam with learning rate $10^{-3}$ and weight decay $10^{-4}$, using an inverse square root decay schedule with 5K warm-up and cool-down steps. For the baseline, we use SigLIP-B/16 pretrained on 10B training examples. Again, recursive models are trained on fewer examples to match the total training compute cost. Due to the amount of compute involved in these experiments, we only compare the non-recursive baseline (with signature \A\B) against RAO (with signature \A\B\A\B) and RINS (signature \A$^2$\B) with degree 1.

\begin{table}[t]
    \centering\scriptsize
    \begin{tabularx}{\columnwidth}{@{}l|YYYY@{}}
    \toprule
    \bf Metric &\multicolumn{4}{c}{\bf Architecture}\\
    &\A\B &\ttfamily Long-Seq & \A\B\A\B & \A\A\B\\ \midrule 
    \multicolumn{5}{c}{\em Zero-shot classification}\\[2pt]
    ImageNet & 73.4 &
    73.0 
    & 72.7 & \bf 74.1 \\
CIAFR100 & 68.9 &
63.6 
& 65.3 & \bf 72.2 \\
Pet & 90.4 &
90.0
& \bf 90.7 & 90.0 \\

    \multicolumn{5}{c}{\em Retrieval}\\[2pt]
COCO img2txt@1 & \bf 62.7 & 
62.0
& 61.1 & 62.3 \\
COCO img2txt@5 & \bf 84.8 & 
84.1
& 82.9 & 84.1 \\
COCO img2txt@10 & \bf 90.7 &
90.4
& 89.5 & 90.2 \\[2pt]
COCO txt2img@1 & 44.6 &
44.2
& 43.2 & \bf 45.1 \\
COCO txt2img@5 & 69.6 &
69.4 
& 68.4 & \bf 70.0 \\
COCO txt2img@10 & 78.8 &
78.7
& 77.5 & \bf 79.0 \\[2pt]
Flickr img2txt@1 & \bf 89.6 &
87.2
& 87.9 & 88.9 \\
Flickr img2txt@5 & 98.0 &
97.8
& 97.5 & \bf 98.5 \\
Flickr img2txt@10 & 99.1 &
98.9
& 98.9 & \bf 99.3 \\[2pt]
Flickr txt2img@1 & \bf 75.1 &
73.7
& 73.2 & 74.3 \\
Flickr txt2img@5 & 92.3 &
92.1
& 91.5 & \bf92.4 \\
Flickr txt2img@10 & 95.6 &
95.6
& 95.6 & \bf95.8 \\
 \bottomrule
    \end{tabularx}
    \caption{Performance of SigLIP-B/16 on various datasets.  Results are shown for ImageNet~\cite{deng2009imagenet}, CIFAR100~\cite{Krizhevsky09learningmultiple}, Pet~\cite{parkhi12a}, COCO~\cite{chen2015microsoft}, and Flickr~\cite{young-etal-2014-image}. All models are identical in size to SigLIP-B/16 and have the same training compute FLOPs. Long-Seq is SigLIP-B/16 trained on higher resolution of 280 and text length 80 (25\% increase in sequence length $\rightarrow$ 50\% increase in inference cost compared to baseline, similar to \A\A\B). Using Wilcoxon signed rank test~\cite{wilcoxon1992individual}, we obtain $p=0.003$ so the evidence in favor of RINS is statistically significant at the 99\% confidence level.}
    \label{tab:siglip}
\end{table}


\textbf{Results.} Table~\ref{tab:siglip} shows that RINS (with signature \A$^2$\B) outperforms the non-recursive baseline, including with long sequence length, and RAO in zero-shot and retrieval evaluations. Of importance is the impact in ImageNet 0-shot classification, where we see an improvement of about $+0.7\%$. 

\textbf{Overtraining.} Next, we demonstrate that RINS can substantially advance state-of-the-art results in multimodal systems for a given  model size. We train a recursive variant of SigLIP-B/16, denoted  SigLIP-RINS-B/16, using the signature \A$^3$\B. In light of the findings in Section~\ref{sect:analysis}, we increase the number of recursions here given the  increase in compute.

We adhere to the training protocol outlined above, with the exception that SigLIP-RINS-B/16 is now trained on 40B examples, matching the training data scale of the widely-used, publicly available SigLIP-B/16 checkpoint. Note that both models have an identical size. Moreover, following~\citet{pouget2024no}, we utilize a training mixture comprising both English and multilingual data to enhance cultural diversity, and report the cultural metrics recommended in~\citet{pouget2024no} as well as multilinguality evaluations using XM3600 dataset~\cite{thapliyal2022crossmodal}. So, to ensure an appropriate comparison of results, we re-train SigLIP-B/16 on 40B examples from the same data mixture. The primary datasets for cultural diversity evaluation are Dollar Street~\cite{rojas2022dollar}, GeoDE~\cite{ramaswamy2024geode}, and Google Landmarks Dataset v2 (GLDv2)~\cite{weyand2020google}. We use the Gemma tokenizer~\cite{gemmateam2024gemmaopenmodelsbased}.

As shown in Table~\ref{tab:long_siglip_results}, SigLIP-RINS-B/16 significantly outperforms the standard SigLIP-B/16 across all benchmarks. In fact, \emph{stochastic} RINS with skip probability $p_s=\frac{1}{4}$ improves results even further. Crucially, these results demonstrate that RINS offers a fundamental advantage in multimodal learning that are not replicated by simply overtraining a non-recursive counterpart.

\begin{table}[t]
    \centering\scriptsize
    \begin{tabularx}{\columnwidth}{@{}l|c|YYY@{}}
    \toprule
    \bf Metric &\textbf{SigLIP-B/16} &\multicolumn{3}{c}{\textbf{SigLIP-RINS-B/16}}\\
    &\textbf{(370M params)} &\multicolumn{3}{c}{\textbf{(370M params)}}\\
    & & $p_s=0$ & $\frac{1}{4}$ & $\frac{1}{2}$\\ \midrule
    \multicolumn{5}{c}{\em Zero-shot classification}\\[2pt]
ImageNet & 77.3 & 79.0 & \bf79.6 & \underline{79.2} \\
CIAFR100 & 70.3 & 78.5 & \bf80.7 & \underline{81.7} \\
Pet & 92.6 & \bf94.8 & \underline{94.4} & 92.7 \\[2pt]
    \multicolumn{5}{c}{\em Cultural Diversity}\\[2pt]
GeoLoc:Dollar Street & 17.6 & \bf19.7 & 19.2 & \underline{19.3} \\
GeoLoc:GeoDE-Country & 22.5 & 23.3 & \bf24.7 & \underline{23.9} \\
GeoLoc:GeoDE-Region & 36.1 & 37.7 & \bf39.7 & \underline{38.7} \\
Dollar Street & 51.5 & 52.9 & \bf53.1 & \bf53.1 \\
GeoDE & 93.1 & 93.7 & \bf94.3 & \underline{94.2} \\
GLDv2 & 51.6 & \bf53.7 & 52.4 & \underline{52.5} \\[2pt]
    \multicolumn{5}{c}{\em Multilinguality}\\[2pt]
XM3600 img2txt@1 & 48.4 & \bf53.5 & \underline{52.6} & 51.8 \\
XM3600 img2txt@5 & 68.4 & \bf72.6 & \underline{72.0} & 70.9 \\
XM3600 img2txt@10 & 74.4 & \bf78.1 & \underline{77.5} & 76.5 \\[2pt]

XM3600 txt2img@1 & 39.5 & \underline{43.0} & \bf43.1 & 41.9 \\
XM3600 txt2img@5 & 59.5 & \underline{63.0} & \bf63.1 & 61.6 \\
XM3600 txt2img@10 & 66.1 & \bf69.3 & \bf69.3 & 68.0 \\[2pt]
    \multicolumn{5}{c}{\em Retrieval}\\[2pt]
COCO img2txt@1 & 67.6 & 69.4 & \bf70.0 & \underline{69.5} \\
COCO img2txt@5 & 87.2 & \underline{88.6} & \bf88.9 & 88.3 \\
COCO img2txt@10 & 92.6 & \underline{93.2} & \bf93.5 & \underline{93.2} \\[2pt]
COCO txt2img@1 & 50.3 & 51.5 & \bf52.4 & \underline{52.0} \\
COCO txt2img@5 & 74.7 & 75.2 & \bf76.0 & \underline{75.7} \\
COCO txt2img@10 & 82.6 & 83.0 & \bf83.5 & \underline{83.2} \\[2pt]
Flickr img2txt@1 & 91.9 & \underline{92.9} & \bf93.5 & 92.4 \\
Flickr img2txt@5 & \underline{99.3} & 98.7 & \bf99.5 & 98.7 \\
Flickr img2txt@10 & \underline{99.6} & 99.5 & \bf99.7 & 99.3 \\[2pt]
Flickr txt2img@1 & 80.1 & \underline{80.7} & \bf81.4 & 80.5 \\
Flickr txt2img@5 & 94.6 & 94.3 & \bf95.4 & \underline{95.0} \\
Flickr txt2img@10 & \underline{97.1} & 97.0 & \bf97.4 & 97.0 \\
 \bottomrule
    \end{tabularx}
    \caption{
    Performance of multilingual SigLIP models on various datasets under an overtraining regime, where training compute is not a constraint. All models are identical in size to SigLIP-B/16. As shown in the rightmost columns, stochastic RINS ($p_s>0$) outperforms the other architectures. Detailed results on multilinguality are provided in Appendix~\ref{sect:app_multiling}.
    }
    \label{tab:long_siglip_results}
\end{table}



\section{Further Analysis}
\label{sect:analysis}
\section{Analysis}
\label{sec:analysis}
In the following sections, we will analyze European type approval regulation\footnote{Strictly speaking, the German enabling act (AFGBV) does not regulate type-approval, but how test \& operating permits are issued for SAE-Level-4 systems. Type-approval regulation for SAE-Level-3 systems follows UN Regulation No. 157 (UN-ECE-ALKS) \parencite{un157}.} regarding the underlying notions of ``safety'' and ``risk''.
We will classify these notions according to their absolute or relative character, underlying risk sources, or underlying concepts of harm.

\subsection{Classification of Safety Notions}
\label{sec:safety-notions}
We will refer to \emph{absolute} notions of safety as conceptualizations that assume the complete absence of any kind of risk.
Opposed to this, \emph{relative} notions of safety are based on a conceptualization that specifically includes risk acceptance criteria, e.g., in terms of ``tolerable'' risk or ``sufficient'' safety.

For classifying notions of safety by their underlying risk (or rather ``hazard'') sources, and different concepts of harm, \Cref{fig:hazard-sources} provides an overview of our reasoning, which is closely in line with the argumentation provided by Waymo in \parencite{favaro2023}.
We prefer ``hazard sources'' over ``risk sources'', as a risk must always be related to a \emph{cause} or \emph{source of harm} (i.e., a hazard \parencite[p.~1, def. 3.2]{iso51}).
Without a concrete (scenario) context that the system is operating in, a hazard is \emph{latent}: E.g., when operating in public traffic, there is a fundamental possibility that a \emph{collision with a pedestrian} leads to (physical) harm for that pedestrian. 
However, only if an automated vehicle shows (potentially) hazardous behavior (e.g., not decelerating properly) \emph{and} is located near a pedestrian (context), the hazard is instantiated and leads to a hazardous event.
\begin{figure*}
    \includeimg[width=.9\textwidth]{hazard-sources0.pdf}
    \caption{Graphical summary of a taxonomy of risk related to automated vehicles, extended based on ISO 21448 (\parencite{iso21448}) and \parencite{favaro2023}. Top: Causal chain from hazard sources to actual harm; bottom: summary of the individual elements' contributions to a resulting risk. Graphic translated from \parencite{nolte2024} \label{fig:hazard-sources}}
\end{figure*}
If the hazardous event cannot be mitigated or controlled, we see a loss event in which the pedestrian's health is harmed.
Note that this hypothetical chain of events is summarized in the definition of risk:
The probability of occurrence of harm is determined by a) the frequency with which hazard sources manifest, b) the time for which the system operates in a context that exposes the possibility of harm, and c) by the probability with which a hazardous event can be controlled.
A risk can then be determined as a function of the probability of harm and the severity of the harm potentially inflicted on the pedestrian.

In the following, we will apply this general model to introduce different types of hazard sources and also different types of harm.
\cref{fig:hazard-sources} shows two distinct hazard sources, i.e., functional insufficiencies and E/E-failures that can lead to hazardous behavior.
ISO~21488 \parencite{iso21448} defines functional insufficiencies as insufficiencies that stem from an incomplete or faulty system specification (specification insufficiencies).
In addition, the standard considers insufficiencies that stem from insufficient technical capability to operate inside the targeted Operational Design Domain (performance insufficiencies).
Functional insufficiencies are related to the ``Safety of the Intended Functionality (SOTIF)'' (according to ISO~21448), ``Behavioral Safety'' (according to Waymo \parencite{waymo2018}), or ``Operational Safety'' (according to UN Regulation No. 157 \parencite{un157}).
E/E-Failures are related to classic functional safety and are covered exhaustively by ISO~26262 \parencite{iso2018}.
Additional hazard sources can, e.g., be related to malicious security attacks (ISO~21434), or even to mechanical failures that should be covered (in the US) in the Federal Motor Vehicle Safety Standards (FMVSS).

For the classification of notions of safety by the related harm, in \parencite{salem2024, nolte2024}, we take a different approach compared to \parencite{koopman2024}:
We extend the concept of harm to the violation of stakeholder \emph{values}, where values are considered to be a ``standard of varying importance among other such standards that, when combined, form a value pattern that reduces complexity for stakeholders [\ldots] [and] determines situational actions [\ldots].'' \parencite{albert2008}
In this sense, values are profound, personal determinants for individual or collective behavior.
The notion of values being organized in a weighted value pattern shows that values can be ranked according to importance.
For automated vehicles, \emph{physical wellbeing} and \emph{mobility} can, e.g., be considered values which need to be balanced to achieve societal acceptance, in line with the discussion of required tradeoffs in \cref{sec:terminology}.
For the analysis of the following regulatory frameworks, we will evaluate if the given safety or risk notions allow tradeoffs regarding underlying stakeholder values. 

\subsection{UN Regulation No. 157 \& European Implementing Regulation (EU) 2022/1426}
\label{sec:enabling-act}
UN Regulation No. 157 \parencite{un157} and the European Implementing Regulation 2022/1426 \parencite{eu1426} provide type approval regulation for automated vehicles equipped with SAE-Level-3 (UN Reg. 157) and Level 4 (EU 2022/1426) systems on an international (UN Reg. 157) and European (EU 2022/1426) level.

Generally, EU type approval considers UN ECE regulations mandatory for its member states ((EU) 2018/858, \parencite{eu858}), while the EU largely forgoes implementing EU-specific type approval rules, it maintains the right to alter or to amend UN ECE regulation \parencite{eu858}.

In this respect, the terminology and conceptualizations in the EU Implementing Act closely follow those in UN Reg. No. 157.
The EU Implementing Act gives a clear reference to UN Reg. No. 157 \parencite[][Preamble,  Paragraph 1]{eu1426}.
Hence, the documents can be assessed in parallel.
Differences will be pointed out as necessary.

Both acts are written in rather technical language, including the formulation of technical requirements (e.g., regarding deceleration values or speeds in certain scenarios).
While providing exhaustive definitions and terminology, neither of both documents provide an actual definition of risk or safety.
The definition of ``unreasonable'' risk in both documents does not define risk, but only what is considered \emph{unreasonable}. It states that the ``overall level of risk for [the driver, (only in UN Reg. 157)] vehicle occupants and other road users which is increased compared to a competently and carefully driven manual vehicle.''
The pertaining notions of safety and risk can hence only be derived from the context in which they are used.

\subsubsection{Absolute vs. Relative Notions of Safety}
In line with the technical detail provided in the acts, both clearly imply a \emph{relative} notion of safety and refer to the absence of \emph{unreasonable} risk throughout, which is typical for technical safety definitions.

Both acts require sufficient proof and documentation that the to-be-approved automated driving systems are ``free of unreasonable safety risks to vehicle occupants and other road users'' for type approval.\footnote{As it targets SAE-Level-3 systems, UN Reg. 157 also refers to the driver, where applicable.}
In this respect, both acts demand that the manufacturers perform verification and validation activities for performance requirements that include ``[\ldots] the conclusion that the system is designed in such a way that it is free from unreasonable risks [\ldots]''.
Additionally, \emph{risk minimization} is a recurring theme when it comes to the definition of Minimum Risk Maneuvers (MRM).

Finally, supporting the relative notions of safety and risk, UN Reg. 157 introduces the concept of ``reasonable foreseeable and preventable'' \parencite[Article 1, Clause 5.1.1.]{un157} collisions, which implies that a residual risk will remain with the introduction of automated vehicles.
\parencite[][Appendix 3, Clause 3.1.]{un157} explicitly states that only \emph{some} scenarios that are unpreventable for a competent human driver can actually be prevented by an automated driving system.
While this concept is not applied throughout the EU Implementing Act, both documents explicitly refer to \emph{residual} risks that are related to the operation of automated driving systems (\parencite[][Annex I, Clause 1]{un157}, \parencite[][Annex II, Clause 7.1.1.]{eu1426}).

\subsubsection{Hazard Sources}
Hazard sources that are explicitly differentiated in UN Reg. 157 and (EU) 2022/1426 are E/E-failures that are in scope of functional safety (ISO~26262) and functional insufficiencies that are in scope of behavioral (or ``operational'') safety (ISO~21448).
Both documents consistently differentiate both sources when formulating requirements.

While the acts share a common definition of ``operational'' safety (\parencite[][Article 2, def. 30.]{eu1426}, \parencite[][Annex 4, def. 2.15.]{un157}), the definitions for functional safety differ.
\parencite{un157} defines functional safety as the ``absence of unreasonable risk under the occurrence of hazards caused by a malfunctioning behaviour of electric/electronic systems [\ldots]'', \parencite{eu1426} drops the specification of ``electric/electronic systems'' from the definition.
When taken at face value, this definition would mean that functional safety included all possible hazard sources, regardless of their origin, which is a deviation from the otherwise precise usage of safety-related terminology.

\subsubsection{Harm Types}
As the acts lack explicit definitions of safety and risk, there is no consistent and explicit notion of different harm types that could be differentiated.

\parencite{un157} gives little hints regarding different considered harm types.
``The absence of unreasonable risk'' in terms of human driving performance could hence be related to any chosen performance metric that allows a comparison with a competent careful human driver including, e.g., accident statistics, statistics about rule violations, or changes in traffic flow.

In \parencite{eu1426}, ``safety'' is, implicitly, attributed to the absence of unreasonable risk to life and limb of humans.
This is supported by the performance requirements that are formulated:
\parencite[][Annex II, Clause 1.1.2. (d)]{eu1426} demands that an automated driving system can adapt the vehicle behavior in a way that it minimizes risk and prioritizes the protection of human life.

Both acts demand the adherence to traffic rules (\parencite[][Annex 2, Clause 1.3.]{eu1426}, \parencite[][Clause 5.1.2.]{un157}).
\parencite[][Annex II, Clause 1.1.2. (c)]{eu1426} also demands that an automated driving system shall adapt its behavior to surrounding traffic conditions, such as the current traffic flow.
With the relative notion of risk in both acts, the unspecific clear statement that there may be unpreventable accidents \parencite{un157}, and a demand of prioritization of human life in \parencite{eu1426}, both acts could be interpreted to allow developers to make tradeoffs as discussed in \cref{sec:terminology}.


\subsubsection{Conclusion}
To summarize, the UN Reg. 157 and the (EU) 2022/1426 both clearly support the technical notion of safety as the absence of unreasonable risk.
The notion is used consistently throughout both documents, providing a sufficiently clear terminology for the developers of automated vehicles.
Uncertainty remains when it comes to considered harm types: Both acts do not explicitly allow for broader notions of safety, in the sense of \parencite{koopman2024} or \parencite{salem2024}.
Finally, a minor weak spot can be seen in the definition of risk acceptance criteria: Both acts take the human driving performance as a baseline.
While (EU) 2022/1426 specifies that these criteria are specific to the systems' Operational Design Domain \parencite[][Annex II, Clause 7.1.1.]{eu1426}, the reference to the concrete Operational Design Domain is missing in UN Reg. 157.
Without a clearly defined notion of safety, however, it remains unclear, how aspects beyond net accident statistics (which are given as an example in \parencite[][Annex II, Clause 7.1.1.]{eu1426}), can be addressed practically, as demanded by \parencite{koopman2024}.

\subsection{German Regulation (StVG \& AFGBV)}
\label{sec:afgbv}
The German L3 (Automated Driving Act) and L4 (Act on Autonomous Driving) Acts from 2017 and 2021,\footnote{Formally, these are amendments to the German Road Traffic Act (StVG): 06/21/2017, BGBl. I p. 1648, 07/12/2021 BGBl. I p. 3108.} respectively, provide enabling regulation for the operation of SAE-Level-3 and 4 vehicles on German roads.
The German Implementing Regulation (\parencite{afgbv}, AFGBV) defines how this enabling regulation is to be implemented for granting testing permits for SAE-Level-3 and -4 and driving permits for SAE-Level-3 and -4 automated driving systems.\footnote{Note that these permits do not grant EU-wide type approval, but serve as a special solution for German roads only. At the same time, the AFGBV has the same scope as (EU) 2022/1426.}
With all three acts, Germany was the first country to regulate the approval of automated vehicles for a domestic market.
All acts are subject to (repeated) evaluation until the year 2030 regarding their impact on the development of automated driving technology.
An assessment of the German AFGBV and comparisons to (EU) 2022/1426 have been given in \cite{steininger2022} in German.

Just as for UN Reg. 157 and (EU) 2022/1426, neither the StVG nor the AFGBV provide a clear definition of ``safety'' or ``risk'' -- even though the "safety" of the road traffic is one major goal of the StVG and StVO.
Again, different implicit notions of both concepts can only be interpreted from the context of existing wording.
An additional complication that is related to the German language is that ``safety'' and ``security'' can both be addressed as ``Sicherheit'', adding another potential source of unclarity.
Literal Quotations in this section are our translations from the German act.

\subsubsection{Absolute vs. Relative Notions of Safety}
For assessing absolute vs. relative notions of safety in German regulation, it should be mentioned that the main goal of the German StVO is to ensure the ``safety and ease of traffic flow'' -- an already diametral goal that requires human drivers to make tradeoffs.\footnote{For human drivers, this also creates legal uncertainty which can sometimes only be settled in a-posteriori court cases.}
While UN and EU regulation clearly shows a relative notion of safety\footnote{And even the StVG contains sections that use wording such as ``best possible safety for vehicle occupants'' (§1d (4) StVG) and acknowledges that there are unavoidable hazards to human life (§1e (2) No. 2c)).}, the German AFGBV contains ambiguous statements in this respect:
Several paragraphs contain a demand for a hazard free operation of automated vehicles.
§4 (1) No. 4 AFGBV, e.g., states that ``the operation of vehicles with autonomous driving functions must neither negatively impact road traffic safety or traffic flow, nor endanger the life and limb of persons.''
Additionally, §6 (1) AFGBV states that the permits for testing and operation have to be revoked, if it becomes apparent that a ``negative impact on road traffic safety or traffic flow, or hazards to the life and limb of persons cannot be ruled out''.
The same wording is used for the approval of operational design domains regulated in §10 (1) No. 1.
A particularly misleading statement is made regarding the requirements for technical supervision instances which are regulated in §14 (3) AFGBV which states that an automated vehicle has to be  ``immediately removed from the public traffic space if a risk minimal state leads to hazards to road traffic safety or traffic flow''.
Considering the argumentation in \cref{sec:terminology}, that residual risks related to the operation of automated driving systems are inevitable, these are strong statements which, if taken at face value, technically prohibit the operation of automated vehicles.
It suggests an \emph{absolute} notion of safety that requires the complete absence of risk.  
The last statement above is particularly contradictory in itself, considering that a risk \emph{minimal} state always implies a residual risk.

In addition to these absolute safety notions, there are passages which suggest a relative notion of safety:
The approval for Operational Design Domains is coupled to the proof that the operation of an automated vehicle ``neither negatively impacts road traffic safety or traffic flow, nor significantly endangers the life and limb of persons beyond the general risk of an impact that is typical of local road traffic'' (§9 (2) No. 3 AFGBV).
The addition of a relative risk measure ``beyond the general risk of an impact'' provides a relaxation (cf. also \cite{steininger2022}, who criticizes the aforementioned absolute safety notion) that also yields an implicit acceptance criterion (\emph{statistically as good as} human drivers) similar to the requirements stated in UN Reg. 157 and (EU) 2022/1426.

Additional hints for a relative notion of safety can be found in Annex 1, Part 1, No. 1.1 and Annex 1, Part 2, No. 10.
Part 1, No 1.1 specifies collision-avoidance requirements and acknowledges that not all collisions can be avoided.\footnote{The same is true for Part 2, No. 10, Clause 10.2.5.}
Part 2, No. 10 specifies requirements for test cases.
It demands that test cases are suitable to provide evidence that the ``safety of a vehicle with an autonomous driving function is increased compared to the safety of human-driven vehicles''.
This does not only acknowledge residual risks, but also yields an acceptance criterion (\emph{better} than human drivers) that is different from the implied acceptance criterion given in §9 (2) No. 3 AFGBV.

\subsubsection{Hazard Sources}
Regarding hazard sources, Annex 1 and 3 AFGBV explicitly refer to ISO~26262 and ISO~21448 (or rather its predecessor ISO/PAS~21448:2019).
However, regarding the discussion of actual hazard sources, the context in which both standards are mentioned is partially unclear:
Annex 1, Clause 1.3 discusses requirements for path and speed planning.
Clause 1.3 d) demands that in intersections, a Time to Collision (TTC) greater than 3 seconds must be guaranteed.
If manufacturers deviate from this, it is demanded that ``state-of-the-art, systematic safety evaluations'' are performed.
Fulfillment of the state of the art is assumed if ``the guidelines of ISO~26262:2018-12 Road Vehicles -- Functional Safety are fulfilled''.
Technically, ISO~26262 is not suitable to define the state of the art in this context, as the requirements discussed fall in the scope of operational (or behavioral) safety (ISO~21448).
A hazard source ``violated minimal time to collision'' is clearly a functional insufficiency, not an E/E-failure.

Similar unclarity presents itself in Annex 3, Clause 1 AFGBV: 
Clause 1 specifies the contents of the ``functional specification''.
The ``specification of the functionality'' is an artifact which is demanded in ISO~21448:2022 (Clause 5.3) \parencite{iso21448}.
However, Annex 3, Clause 1 AFGBV states that the ``functional specification'' is considered to comply to the state of the art, if the ``functional specification'' adheres to ISO~26262-3:2018 (Concept Phase).
Again, this assumes SOTIF-related contents as part of ISO~26262, which introduces the ``Item Definition'' as an artifact, which is significantly different from the ``specification of the functionality'' which is demanded by ISO~21448.
Finally, Annex 3, Clause 3 AFGBV demands a ``documentation of the safety concept'' which ``allows a functional safety assessment''.
A safety concept that is related to operational / behavioral safety is not demanded.
Technically, the unclarity with respect to the addressed harm types lead to the fact that the requirements provided by the AFGBV do not comply with the state of the art in the field, providing questionable regulation.

\subsubsection{Harm Types}
Just like UN Reg. 157 and (EU) 2022/1426, the German StVG and AFGBV do not explicitly differentiate concrete harm types for their notions of safety.
However, the AFGBV mentions three main concerns for the operation of automated vehicles which are \emph{traffic flow} (e.g., §4 (1) No. 4 AFGBV), compliance to \emph{traffic law} (e.g., §1e (2) No. 2 StVG), and the \emph{life and limb of humans} (e.g., §4 (1) No. 4 AFGBV).

Again, there is some ambiguity in the chosen wording:
The conflict between traffic flow and safety has already been argued in \cref{sec:terminology}.
The wording given in §4 (1) No. 4 and §6 (1) AFGBV  demand to ensure (absolute) safety \emph{and} traffic flow at the same time, which is impossible (cf. \cref{sec:terminology}) from an engineering perspective.
§1e (2) No. 2 StVG defines that ``vehicles with an autonomous driving function must [\ldots] be capable to comply to [\ldots] traffic rules in a self-contained manner''.
Taken at face value, this wording implies that an automated driving system could lose its testing or operating permit as soon as it violates a traffic rule.
A way out could be provided by §1 of the German Traffic Act (StVO) which demands careful and considerate behavior of all traffic participants and by that allows judgement calls for human drivers.
However, if §1 is applicable in certain situations is often settled in court cases. 
For developers, the application of §1 StVO during system design hence remains a legal risk.

While there are rather absolute statements as mentioned above, sections of the AFGBV and StVG can be interpreted to allow tradeoffs:
§1e (2) No. 2 b) demands that a system,  ``in case of an inevitable, alternative harm to legal objectives, considers the significance of the legal objectives, where the protection of human life has highest priority''.
This exact wording \emph{could} provide some slack for the absolute demands in other parts of the acts, enabling tradeoffs between (tolerable) risk and mobility as discussed in \cref{sec:terminology}.
However, it remains unclear if this interpretation is legally possible.

\subsubsection{Conclusion}
Compared to UN Reg. 157 and (EU) 2022/1426, the German StVG and AFGBV introduce openly inconsistent notions of safety and risk which are partially directly contradictory:
The wording partially implies absolute and relative notions of safety and risk at the same time.
The implied validation targets (``better'' or ``as good as'' human drivers) are equally contradictory. 
The partially implied absolute notions of safety, when taken at face value, prohibit engineers from making the tradeoffs required to develop a system that is safe and provides customer benefit at the same time. 
In consequence, the wording in the acts is prone to introducing legal uncertainty.
This uncertainty creates additional clarification need and effort for manufacturers and engineers who design and develop SAE-Level-3 and -4 automated driving systems. The use of undefined legal terms not only makes it more difficult for engineers to comply with the law, but also complicates the interpretation of the law and leads to legal uncertainty.

\subsection{UK Automated Vehicles Act 2024 (2024 c. 10)}
The UK has issued a national enabling act for regulating the approval of automated vehicles on the roads in the UK.
To the best of our knowledge, concrete implementing regulation has not been issued yet.
Regarding terminology, the act begins with a dedicated terminology section to clarify the terms used in the act \parencite[Part 1, Chapter 1, Section 1]{ukav2024}.
In that regard, the act defines a vehicle to drive ```autonomously' if --- (a)
it is being controlled not by an individual but by equipment of the vehicle, and (b) neither the vehicle nor its surroundings are being monitored by an individual with a view to immediate intervention in the driving of the vehicle.''
The act hence covers SAE-Level-3 to SAE-Level-5 automated driving systems.

\subsubsection{Absolute vs. Relative Notions of Safety}
While not providing an explicit definition of safety and risk, the UK Automated Vehicles Act (``UK AV Act'') \parencite{ukav2024} explicitly refers to a relative notion of safety.
Part~1, Chapter~1, Section~1, Clause (7)~(a) defines that an automated vehicle travels ```safely' if it travels to an acceptably safe standard''.
This clarifies that absolute safety is not achievable and that acceptance criteria to prove the acceptability of residual risk are required, even though a concrete safety definition is not given.
The act explicitly tasks the UK Secretary of State\footnote{Which means, that concrete implementation regulation needs to be enacted.} to install safety principles to determine the ``acceptably safe standard'' in Part~1, Chapter~1, Section~1, Clause (7)~(a).
In this respect, the act also provides one general validation target as it demands that the safety principles must ensure that ``authorized automated vehicles will achieve a level of safety equivalent to, or higher than, that of careful and competent human drivers''.
Hence, the top-level validation risk acceptance criterion assumed for UK regulation is ``\emph{at least as good} as human drivers''.

\subsubsection{Hazard Sources}
The UK AV Act contains no statements that could be directly related to different hazard sources.
Note that, in contrast to the rest of the analyzed documents, the UK AV Act is enabling rather than implementing regulation.
It is hence comparable to the German StVG, which does not refer to concrete hazard sources as well.

\subsubsection{Types of Harm}
Even though providing a clear relative safety notion, the missing definition of risk also implies a lack of explicitly differentiable types of harm.
Implicitly, three different types of harm can be derived from the wording in the act.
This includes the harm to life and limb of humans\footnote{Part~1, Chapter~3, Section~25 defines ``aggravated offence where death or serious injury occurs'' \parencite{ukav2024}.}, the violation of traffic rules\footnote{Part~1, Chapter~1, Clause~(7)~(b) defines that an automated vehicle travels ```legally' if it travels with an acceptably low risk of committing a traffic infraction''}, and the cause of inconvenience to the public \parencite[Part~1, Chapter~1, Section~58, Clause (2)~(d)]{ukav2024}.

The act connects all the aforementioned types of harm to ``risk'' or ``acceptable safety''.
While the act generally defines criminal offenses for providing ``false or misleading information about safety'', it also acknowledges possible defenses if it can be proven that ``reasonable precautions'' were taken and that ``due diligence'' was exercised to ``avoid the commission of the offence''.
This statement could enable tradeoffs within the scope of ``reasonable risk'' to the life and limb of humans, the violation of traffic rules, or to the cause of inconvenience to the public, as we argued in \cref{sec:terminology}.

\subsubsection{Conclusion}
From the set of reviewed documents, the current UK AV Act is the one with the most obvious relative notions of safety and risk and the one that seems to provide a legal framework for permitting tradeoffs.
In our review, we did not spot major inconsistency beyond a missing definitions of safety and risk\footnote{Note that with the Office for Product Safety and Standards (OPSS), there is a British government agency that maintains an exhaustive and widely focussed ``Risk Lexicon'' that provides suitable risk definitions. For us, it remains unclear, to what extent this terminology is assumed general knowledge in British legislation.}.
The general, relative notion of safety and the related alleged ability for designers to argue well-founded development tradeoffs within the legal framework could prove beneficial for the actual implementation of automated driving systems.
While the act thus appears as a solid foundation for the market introduction of automated vehicles, without accompanying implementing regulation, it is too early to draw definite conclusions.

\section{Discussion and Related Works}
\label{sect:related}
\putsec{related}{Related Work}

\noindent \textbf{Efficient Radiance Field Rendering.}
%
The introduction of Neural Radiance Fields (NeRF)~\cite{mil:sri20} has
generated significant interest in efficient 3D scene representation and
rendering for radiance fields.
%
Over the past years, there has been a large amount of research aimed at
accelerating NeRFs through algorithmic or software
optimizations~\cite{mul:eva22,fri:yu22,che:fun23,sun:sun22}, and the
development of hardware
accelerators~\cite{lee:cho23,li:li23,son:wen23,mub:kan23,fen:liu24}.
%
The state-of-the-art method, 3D Gaussian splatting~\cite{ker:kop23}, has
further fueled interest in accelerating radiance field
rendering~\cite{rad:ste24,lee:lee24,nie:stu24,lee:rho24,ham:mel24} as it
employs rasterization primitives that can be rendered much faster than NeRFs.
%
However, previous research focused on software graphics rendering on
programmable cores or building dedicated hardware accelerators. In contrast,
\name{} investigates the potential of efficient radiance field rendering while
utilizing fixed-function units in graphics hardware.
%
To our knowledge, this is the first work that assesses the performance
implications of rendering Gaussian-based radiance fields on the hardware
graphics pipeline with software and hardware optimizations.

%%%%%%%%%%%%%%%%%%%%%%%%%%%%%%%%%%%%%%%%%%%%%%%%%%%%%%%%%%%%%%%%%%%%%%%%%%
\myparagraph{Enhancing Graphics Rendering Hardware.}
%
The performance advantage of executing graphics rendering on either
programmable shader cores or fixed-function units varies depending on the
rendering methods and hardware designs.
%
Previous studies have explored the performance implication of graphics hardware
design by developing simulation infrastructures for graphics
workloads~\cite{bar:gon06,gub:aam19,tin:sax23,arn:par13}.
%
Additionally, several studies have aimed to improve the performance of
special-purpose hardware such as ray tracing units in graphics
hardware~\cite{cho:now23,liu:cha21} and proposed hardware accelerators for
graphics applications~\cite{lu:hua17,ram:gri09}.
%
In contrast to these works, which primarily evaluate traditional graphics
workloads, our work focuses on improving the performance of volume rendering
workloads, such as Gaussian splatting, which require blending a huge number of
fragments per pixel.

%%%%%%%%%%%%%%%%%%%%%%%%%%%%%%%%%%%%%%%%%%%%%%%%%%%%%%%%%%%%%%%%%%%%%%%%%%
%
In the context of multi-sample anti-aliasing, prior work proposed reducing the
amount of redundant shading by merging fragments from adjacent triangles in a
mesh at the quad granularity~\cite{fat:bou10}.
%
While both our work and quad-fragment merging (QFM)~\cite{fat:bou10} aim to
reduce operations by merging quads, our proposed technique differs from QFM in
many aspects.
%
Our method aims to blend \emph{overlapping primitives} along the depth
direction and applies to quads from any primitive. In contrast, QFM merges quad
fragments from small (e.g., pixel-sized) triangles that \emph{share} an edge
(i.e., \emph{connected}, \emph{non-overlapping} triangles).
%
As such, QFM is not applicable to the scenes consisting of a number of
unconnected transparent triangles, such as those in 3D Gaussian splatting.
%
In addition, our method computes the \emph{exact} color for each pixel by
offloading blending operations from ROPs to shader units, whereas QFM
\emph{approximates} pixel colors by using the color from one triangle when
multiple triangles are merged into a single quad.



\section*{Impact Statement}
\label{sect:impact}
\section{Discussion of Assumptions}\label{sec:discussion}
In this paper, we have made several assumptions for the sake of clarity and simplicity. In this section, we discuss the rationale behind these assumptions, the extent to which these assumptions hold in practice, and the consequences for our protocol when these assumptions hold.

\subsection{Assumptions on the Demand}

There are two simplifying assumptions we make about the demand. First, we assume the demand at any time is relatively small compared to the channel capacities. Second, we take the demand to be constant over time. We elaborate upon both these points below.

\paragraph{Small demands} The assumption that demands are small relative to channel capacities is made precise in \eqref{eq:large_capacity_assumption}. This assumption simplifies two major aspects of our protocol. First, it largely removes congestion from consideration. In \eqref{eq:primal_problem}, there is no constraint ensuring that total flow in both directions stays below capacity--this is always met. Consequently, there is no Lagrange multiplier for congestion and no congestion pricing; only imbalance penalties apply. In contrast, protocols in \cite{sivaraman2020high, varma2021throughput, wang2024fence} include congestion fees due to explicit congestion constraints. Second, the bound \eqref{eq:large_capacity_assumption} ensures that as long as channels remain balanced, the network can always meet demand, no matter how the demand is routed. Since channels can rebalance when necessary, they never drop transactions. This allows prices and flows to adjust as per the equations in \eqref{eq:algorithm}, which makes it easier to prove the protocol's convergence guarantees. This also preserves the key property that a channel's price remains proportional to net money flow through it.

In practice, payment channel networks are used most often for micro-payments, for which on-chain transactions are prohibitively expensive; large transactions typically take place directly on the blockchain. For example, according to \cite{river2023lightning}, the average channel capacity is roughly $0.1$ BTC ($5,000$ BTC distributed over $50,000$ channels), while the average transaction amount is less than $0.0004$ BTC ($44.7k$ satoshis). Thus, the small demand assumption is not too unrealistic. Additionally, the occasional large transaction can be treated as a sequence of smaller transactions by breaking it into packets and executing each packet serially (as done by \cite{sivaraman2020high}).
Lastly, a good path discovery process that favors large capacity channels over small capacity ones can help ensure that the bound in \eqref{eq:large_capacity_assumption} holds.

\paragraph{Constant demands} 
In this work, we assume that any transacting pair of nodes have a steady transaction demand between them (see Section \ref{sec:transaction_requests}). Making this assumption is necessary to obtain the kind of guarantees that we have presented in this paper. Unless the demand is steady, it is unreasonable to expect that the flows converge to a steady value. Weaker assumptions on the demand lead to weaker guarantees. For example, with the more general setting of stochastic, but i.i.d. demand between any two nodes, \cite{varma2021throughput} shows that the channel queue lengths are bounded in expectation. If the demand can be arbitrary, then it is very hard to get any meaningful performance guarantees; \cite{wang2024fence} shows that even for a single bidirectional channel, the competitive ratio is infinite. Indeed, because a PCN is a decentralized system and decisions must be made based on local information alone, it is difficult for the network to find the optimal detailed balance flow at every time step with a time-varying demand.  With a steady demand, the network can discover the optimal flows in a reasonably short time, as our work shows.

We view the constant demand assumption as an approximation for a more general demand process that could be piece-wise constant, stochastic, or both (see simulations in Figure \ref{fig:five_nodes_variable_demand}).
We believe it should be possible to merge ideas from our work and \cite{varma2021throughput} to provide guarantees in a setting with random demands with arbitrary means. We leave this for future work. In addition, our work suggests that a reasonable method of handling stochastic demands is to queue the transaction requests \textit{at the source node} itself. This queuing action should be viewed in conjunction with flow-control. Indeed, a temporarily high unidirectional demand would raise prices for the sender, incentivizing the sender to stop sending the transactions. If the sender queues the transactions, they can send them later when prices drop. This form of queuing does not require any overhaul of the basic PCN infrastructure and is therefore simpler to implement than per-channel queues as suggested by \cite{sivaraman2020high} and \cite{varma2021throughput}.

\subsection{The Incentive of Channels}
The actions of the channels as prescribed by the DEBT control protocol can be summarized as follows. Channels adjust their prices in proportion to the net flow through them. They rebalance themselves whenever necessary and execute any transaction request that has been made of them. We discuss both these aspects below.

\paragraph{On Prices}
In this work, the exclusive role of channel prices is to ensure that the flows through each channel remains balanced. In practice, it would be important to include other components in a channel's price/fee as well: a congestion price  and an incentive price. The congestion price, as suggested by \cite{varma2021throughput}, would depend on the total flow of transactions through the channel, and would incentivize nodes to balance the load over different paths. The incentive price, which is commonly used in practice \cite{river2023lightning}, is necessary to provide channels with an incentive to serve as an intermediary for different channels. In practice, we expect both these components to be smaller than the imbalance price. Consequently, we expect the behavior of our protocol to be similar to our theoretical results even with these additional prices.

A key aspect of our protocol is that channel fees are allowed to be negative. Although the original Lightning network whitepaper \cite{poon2016bitcoin} suggests that negative channel prices may be a good solution to promote rebalancing, the idea of negative prices in not very popular in the literature. To our knowledge, the only prior work with this feature is \cite{varma2021throughput}. Indeed, in papers such as \cite{van2021merchant} and \cite{wang2024fence}, the price function is explicitly modified such that the channel price is never negative. The results of our paper show the benefits of negative prices. For one, in steady state, equal flows in both directions ensure that a channel doesn't loose any money (the other price components mentioned above ensure that the channel will only gain money). More importantly, negative prices are important to ensure that the protocol selectively stifles acyclic flows while allowing circulations to flow. Indeed, in the example of Section \ref{sec:flow_control_example}, the flows between nodes $A$ and $C$ are left on only because the large positive price over one channel is canceled by the corresponding negative price over the other channel, leading to a net zero price.

Lastly, observe that in the DEBT control protocol, the price charged by a channel does not depend on its capacity. This is a natural consequence of the price being the Lagrange multiplier for the net-zero flow constraint, which also does not depend on the channel capacity. In contrast, in many other works, the imbalance price is normalized by the channel capacity \cite{ren2018optimal, lin2020funds, wang2024fence}; this is shown to work well in practice. The rationale for such a price structure is explained well in \cite{wang2024fence}, where this fee is derived with the aim of always maintaining some balance (liquidity) at each end of every channel. This is a reasonable aim if a channel is to never rebalance itself; the experiments of the aforementioned papers are conducted in such a regime. In this work, however, we allow the channels to rebalance themselves a few times in order to settle on a detailed balance flow. This is because our focus is on the long-term steady state performance of the protocol. This difference in perspective also shows up in how the price depends on the channel imbalance. \cite{lin2020funds} and \cite{wang2024fence} advocate for strictly convex prices whereas this work and \cite{varma2021throughput} propose linear prices.

\paragraph{On Rebalancing} 
Recall that the DEBT control protocol ensures that the flows in the network converge to a detailed balance flow, which can be sustained perpetually without any rebalancing. However, during the transient phase (before convergence), channels may have to perform on-chain rebalancing a few times. Since rebalancing is an expensive operation, it is worthwhile discussing methods by which channels can reduce the extent of rebalancing. One option for the channels to reduce the extent of rebalancing is to increase their capacity; however, this comes at the cost of locking in more capital. Each channel can decide for itself the optimum amount of capital to lock in. Another option, which we discuss in Section \ref{sec:five_node}, is for channels to increase the rate $\gamma$ at which they adjust prices. 

Ultimately, whether or not it is beneficial for a channel to rebalance depends on the time-horizon under consideration. Our protocol is based on the assumption that the demand remains steady for a long period of time. If this is indeed the case, it would be worthwhile for a channel to rebalance itself as it can make up this cost through the incentive fees gained from the flow of transactions through it in steady state. If a channel chooses not to rebalance itself, however, there is a risk of being trapped in a deadlock, which is suboptimal for not only the nodes but also the channel.

\section{Conclusion}
This work presents DEBT control: a protocol for payment channel networks that uses source routing and flow control based on channel prices. The protocol is derived by posing a network utility maximization problem and analyzing its dual minimization. It is shown that under steady demands, the protocol guides the network to an optimal, sustainable point. Simulations show its robustness to demand variations. The work demonstrates that simple protocols with strong theoretical guarantees are possible for PCNs and we hope it inspires further theoretical research in this direction.

\bibliography{main}
\bibliographystyle{icml2025}


%%%%%%%%%%%%%%%%%%%%%%%%%%%%%%%%%%%%%%%%%%%%%%%%%%%%%%%%%%%%%%%%%%%%%%%%%%%%%%%
%%%%%%%%%%%%%%%%%%%%%%%%%%%%%%%%%%%%%%%%%%%%%%%%%%%%%%%%%%%%%%%%%%%%%%%%%%%%%%%
% APPENDIX
%%%%%%%%%%%%%%%%%%%%%%%%%%%%%%%%%%%%%%%%%%%%%%%%%%%%%%%%%%%%%%%%%%%%%%%%%%%%%%%
%%%%%%%%%%%%%%%%%%%%%%%%%%%%%%%%%%%%%%%%%%%%%%%%%%%%%%%%%%%%%%%%%%%%%%%%%%%%%%%
\newpage
\appendix
\onecolumn
\section{Training Configurations}\label{sect:app_config}


\section{Unexpected Questions}
\label{app:question}
Real-world questions do not always have the correct premises. For example, in the question "\begin{CJK}{UTF8}{gbsn}水俣病的传染途径是什么?\end{CJK}(What is the route of infection for Minamata disease?)", Minamata disease is not an infectious disease. Taking this situation into account, we add a small number of human-written questions with incorrect premises and LLM-generated questions with hard-to-verify premises in the question collection phase. The number of these questions in the total number of questions is about 3\%.

\section{Prompt for LLM Augmentation}
\label{app:aug}

\definecolor{titlecolor}{rgb}{0.9, 0.5, 0.1}
\definecolor{anscolor}{rgb}{0.2, 0.5, 0.8}
\definecolor{labelcolor}{HTML}{48a07e}
\begin{table*}[h]
	\centering
	
 % \vspace{-0.2cm}
	
	\begin{center}
		\begin{tikzpicture}[
				chatbox_inner/.style={rectangle, rounded corners, opacity=0, text opacity=1, font=\sffamily\scriptsize, text width=5in, text height=9pt, inner xsep=6pt, inner ysep=6pt},
				chatbox_prompt_inner/.style={chatbox_inner, align=flush left, xshift=0pt, text height=11pt},
				chatbox_user_inner/.style={chatbox_inner, align=flush left, xshift=0pt},
				chatbox_gpt_inner/.style={chatbox_inner, align=flush left, xshift=0pt},
				chatbox/.style={chatbox_inner, draw=black!25, fill=gray!7, opacity=1, text opacity=0},
				chatbox_prompt/.style={chatbox, align=flush left, fill=gray!1.5, draw=black!30, text height=10pt},
				chatbox_user/.style={chatbox, align=flush left},
				chatbox_gpt/.style={chatbox, align=flush left},
				chatbox2/.style={chatbox_gpt, fill=green!25},
				chatbox3/.style={chatbox_gpt, fill=red!20, draw=black!20},
				chatbox4/.style={chatbox_gpt, fill=yellow!30},
				labelbox/.style={rectangle, rounded corners, draw=black!50, font=\sffamily\scriptsize\bfseries, fill=gray!5, inner sep=3pt},
			]
											
			\node[chatbox_user] (q1) {
				\textbf{System prompt}
				\newline
				\newline
				You are a helpful and precise assistant for segmenting and labeling sentences. We would like to request your help on curating a dataset for entity-level hallucination detection.
				\newline \newline
                We will give you a machine generated biography and a list of checked facts about the biography. Each fact consists of a sentence and a label (True/False). Please do the following process. First, breaking down the biography into words. Second, by referring to the provided list of facts, merging some broken down words in the previous step to form meaningful entities. For example, ``strategic thinking'' should be one entity instead of two. Third, according to the labels in the list of facts, labeling each entity as True or False. Specifically, for facts that share a similar sentence structure (\eg, \textit{``He was born on Mach 9, 1941.''} (\texttt{True}) and \textit{``He was born in Ramos Mejia.''} (\texttt{False})), please first assign labels to entities that differ across atomic facts. For example, first labeling ``Mach 9, 1941'' (\texttt{True}) and ``Ramos Mejia'' (\texttt{False}) in the above case. For those entities that are the same across atomic facts (\eg, ``was born'') or are neutral (\eg, ``he,'' ``in,'' and ``on''), please label them as \texttt{True}. For the cases that there is no atomic fact that shares the same sentence structure, please identify the most informative entities in the sentence and label them with the same label as the atomic fact while treating the rest of the entities as \texttt{True}. In the end, output the entities and labels in the following format:
                \begin{itemize}[nosep]
                    \item Entity 1 (Label 1)
                    \item Entity 2 (Label 2)
                    \item ...
                    \item Entity N (Label N)
                \end{itemize}
                % \newline \newline
                Here are two examples:
                \newline\newline
                \textbf{[Example 1]}
                \newline
                [The start of the biography]
                \newline
                \textcolor{titlecolor}{Marianne McAndrew is an American actress and singer, born on November 21, 1942, in Cleveland, Ohio. She began her acting career in the late 1960s, appearing in various television shows and films.}
                \newline
                [The end of the biography]
                \newline \newline
                [The start of the list of checked facts]
                \newline
                \textcolor{anscolor}{[Marianne McAndrew is an American. (False); Marianne McAndrew is an actress. (True); Marianne McAndrew is a singer. (False); Marianne McAndrew was born on November 21, 1942. (False); Marianne McAndrew was born in Cleveland, Ohio. (False); She began her acting career in the late 1960s. (True); She has appeared in various television shows. (True); She has appeared in various films. (True)]}
                \newline
                [The end of the list of checked facts]
                \newline \newline
                [The start of the ideal output]
                \newline
                \textcolor{labelcolor}{[Marianne McAndrew (True); is (True); an (True); American (False); actress (True); and (True); singer (False); , (True); born (True); on (True); November 21, 1942 (False); , (True); in (True); Cleveland, Ohio (False); . (True); She (True); began (True); her (True); acting career (True); in (True); the late 1960s (True); , (True); appearing (True); in (True); various (True); television shows (True); and (True); films (True); . (True)]}
                \newline
                [The end of the ideal output]
				\newline \newline
                \textbf{[Example 2]}
                \newline
                [The start of the biography]
                \newline
                \textcolor{titlecolor}{Doug Sheehan is an American actor who was born on April 27, 1949, in Santa Monica, California. He is best known for his roles in soap operas, including his portrayal of Joe Kelly on ``General Hospital'' and Ben Gibson on ``Knots Landing.''}
                \newline
                [The end of the biography]
                \newline \newline
                [The start of the list of checked facts]
                \newline
                \textcolor{anscolor}{[Doug Sheehan is an American. (True); Doug Sheehan is an actor. (True); Doug Sheehan was born on April 27, 1949. (True); Doug Sheehan was born in Santa Monica, California. (False); He is best known for his roles in soap operas. (True); He portrayed Joe Kelly. (True); Joe Kelly was in General Hospital. (True); General Hospital is a soap opera. (True); He portrayed Ben Gibson. (True); Ben Gibson was in Knots Landing. (True); Knots Landing is a soap opera. (True)]}
                \newline
                [The end of the list of checked facts]
                \newline \newline
                [The start of the ideal output]
                \newline
                \textcolor{labelcolor}{[Doug Sheehan (True); is (True); an (True); American (True); actor (True); who (True); was born (True); on (True); April 27, 1949 (True); in (True); Santa Monica, California (False); . (True); He (True); is (True); best known (True); for (True); his roles in soap operas (True); , (True); including (True); in (True); his portrayal (True); of (True); Joe Kelly (True); on (True); ``General Hospital'' (True); and (True); Ben Gibson (True); on (True); ``Knots Landing.'' (True)]}
                \newline
                [The end of the ideal output]
				\newline \newline
				\textbf{User prompt}
				\newline
				\newline
				[The start of the biography]
				\newline
				\textcolor{magenta}{\texttt{\{BIOGRAPHY\}}}
				\newline
				[The ebd of the biography]
				\newline \newline
				[The start of the list of checked facts]
				\newline
				\textcolor{magenta}{\texttt{\{LIST OF CHECKED FACTS\}}}
				\newline
				[The end of the list of checked facts]
			};
			\node[chatbox_user_inner] (q1_text) at (q1) {
				\textbf{System prompt}
				\newline
				\newline
				You are a helpful and precise assistant for segmenting and labeling sentences. We would like to request your help on curating a dataset for entity-level hallucination detection.
				\newline \newline
                We will give you a machine generated biography and a list of checked facts about the biography. Each fact consists of a sentence and a label (True/False). Please do the following process. First, breaking down the biography into words. Second, by referring to the provided list of facts, merging some broken down words in the previous step to form meaningful entities. For example, ``strategic thinking'' should be one entity instead of two. Third, according to the labels in the list of facts, labeling each entity as True or False. Specifically, for facts that share a similar sentence structure (\eg, \textit{``He was born on Mach 9, 1941.''} (\texttt{True}) and \textit{``He was born in Ramos Mejia.''} (\texttt{False})), please first assign labels to entities that differ across atomic facts. For example, first labeling ``Mach 9, 1941'' (\texttt{True}) and ``Ramos Mejia'' (\texttt{False}) in the above case. For those entities that are the same across atomic facts (\eg, ``was born'') or are neutral (\eg, ``he,'' ``in,'' and ``on''), please label them as \texttt{True}. For the cases that there is no atomic fact that shares the same sentence structure, please identify the most informative entities in the sentence and label them with the same label as the atomic fact while treating the rest of the entities as \texttt{True}. In the end, output the entities and labels in the following format:
                \begin{itemize}[nosep]
                    \item Entity 1 (Label 1)
                    \item Entity 2 (Label 2)
                    \item ...
                    \item Entity N (Label N)
                \end{itemize}
                % \newline \newline
                Here are two examples:
                \newline\newline
                \textbf{[Example 1]}
                \newline
                [The start of the biography]
                \newline
                \textcolor{titlecolor}{Marianne McAndrew is an American actress and singer, born on November 21, 1942, in Cleveland, Ohio. She began her acting career in the late 1960s, appearing in various television shows and films.}
                \newline
                [The end of the biography]
                \newline \newline
                [The start of the list of checked facts]
                \newline
                \textcolor{anscolor}{[Marianne McAndrew is an American. (False); Marianne McAndrew is an actress. (True); Marianne McAndrew is a singer. (False); Marianne McAndrew was born on November 21, 1942. (False); Marianne McAndrew was born in Cleveland, Ohio. (False); She began her acting career in the late 1960s. (True); She has appeared in various television shows. (True); She has appeared in various films. (True)]}
                \newline
                [The end of the list of checked facts]
                \newline \newline
                [The start of the ideal output]
                \newline
                \textcolor{labelcolor}{[Marianne McAndrew (True); is (True); an (True); American (False); actress (True); and (True); singer (False); , (True); born (True); on (True); November 21, 1942 (False); , (True); in (True); Cleveland, Ohio (False); . (True); She (True); began (True); her (True); acting career (True); in (True); the late 1960s (True); , (True); appearing (True); in (True); various (True); television shows (True); and (True); films (True); . (True)]}
                \newline
                [The end of the ideal output]
				\newline \newline
                \textbf{[Example 2]}
                \newline
                [The start of the biography]
                \newline
                \textcolor{titlecolor}{Doug Sheehan is an American actor who was born on April 27, 1949, in Santa Monica, California. He is best known for his roles in soap operas, including his portrayal of Joe Kelly on ``General Hospital'' and Ben Gibson on ``Knots Landing.''}
                \newline
                [The end of the biography]
                \newline \newline
                [The start of the list of checked facts]
                \newline
                \textcolor{anscolor}{[Doug Sheehan is an American. (True); Doug Sheehan is an actor. (True); Doug Sheehan was born on April 27, 1949. (True); Doug Sheehan was born in Santa Monica, California. (False); He is best known for his roles in soap operas. (True); He portrayed Joe Kelly. (True); Joe Kelly was in General Hospital. (True); General Hospital is a soap opera. (True); He portrayed Ben Gibson. (True); Ben Gibson was in Knots Landing. (True); Knots Landing is a soap opera. (True)]}
                \newline
                [The end of the list of checked facts]
                \newline \newline
                [The start of the ideal output]
                \newline
                \textcolor{labelcolor}{[Doug Sheehan (True); is (True); an (True); American (True); actor (True); who (True); was born (True); on (True); April 27, 1949 (True); in (True); Santa Monica, California (False); . (True); He (True); is (True); best known (True); for (True); his roles in soap operas (True); , (True); including (True); in (True); his portrayal (True); of (True); Joe Kelly (True); on (True); ``General Hospital'' (True); and (True); Ben Gibson (True); on (True); ``Knots Landing.'' (True)]}
                \newline
                [The end of the ideal output]
				\newline \newline
				\textbf{User prompt}
				\newline
				\newline
				[The start of the biography]
				\newline
				\textcolor{magenta}{\texttt{\{BIOGRAPHY\}}}
				\newline
				[The ebd of the biography]
				\newline \newline
				[The start of the list of checked facts]
				\newline
				\textcolor{magenta}{\texttt{\{LIST OF CHECKED FACTS\}}}
				\newline
				[The end of the list of checked facts]
			};
		\end{tikzpicture}
        \caption{GPT-4o prompt for labeling hallucinated entities.}\label{tb:gpt-4-prompt}
	\end{center}
\vspace{-0cm}
\end{table*}
\begin{table*}

\centering

\begin{tabular}{|p{\textwidth}|}
\hline
\\ [2pt]
\par Here is a statement and a corresponding piece of reference text. Please complete the task as follows, strictly following the format I have given for the output:
\par (1) Find all the original key passages in the reference text that directly support the information in the statement (there may be more than one, find each one). Output one original key passage per line and the information in the statement it directly supports in the format “Key passage {number}: {key passage} (information in the supporting statement: {supporting information})”.
\par (2) Please group key passages, each group contains key passages supporting the same or related information in the statement, output one line of the grouping results in the format of “Key passage grouping: Group 1: (first group of key passage numbers), Group 2: (second group of key passage numbers) ...”. For example, if there are 2 pieces of information in the statement, key paragraph 1 supports information 1, key paragraph 2 supports information 2, and key paragraph 3 supports information 1, then the output is “Key Paragraph Grouping: Group 1: (1, 3), Group 2: (2)”.
\par (3) Select a group of key text segments and modify the parts of them that support the information in the statement to meet the following requirements:
\par - The modification should make it impossible for the key passage to fully support the corresponding information in the statement.
\par - The modifications should maintain the logical flow of the key passages and no contradictions between the information in the key passages.
\par - The modification should keep the key paragraph logically coherent in the context of the reference text and not contradict the rest of the reference text.
\par - Modify only the parts that support the information in a statement, leaving the rest unchanged.
\par - If there is more than one key passage in a set, the information in them should remain consistent after revision.
\par You need to try two methods of modification:
\par - Changing the message: modifying the message in one part of the key paragraph to another. Do not make changes that directly conflict with the original information. For example, if the original message is “The Audi A7 Signature Edition has a faster top speed than its predecessor”, an appropriate change would be “The Audi A7 Luxury Edition has a faster top speed than its predecessor”, and an inappropriate change1 would be “The Audi A7 Signature Edition has a slower top speed than its predecessor” (using an antonym, which is in direct conflict with the original message), and inappropriate modification 2 is ‘The top speed of the Audi A7 Signature Edition is not faster than the previous generation’ (adding a negative word, which is in direct conflict with the original message).
\par - Delete Information: Remove information from a place in a key paragraph. If the key paragraph is a complete sentence, it should still be a complete sentence after deleting the information. For example, if the original paragraph reads “Due to weather conditions, the project was delayed until March 15” (complete sentence), an appropriate change would be “Due to weather conditions, the project was delayed until March” (still a complete sentence), an inappropriate change would be “Due to the weather” (no longer a complete sentence).
\par For each method, output the key passage that was modified and check its logical fluency, giving an integer within 1 to 10 as a rating (higher means more fluent). Output one modified key passage per line in the format “{method}-modified key passage {number}: {modified key passage} (logical fluency: {score})”.
\\ [5pt]
\\ [5pt]
\hline

\end{tabular}

\caption{\label{tab:prompt_en} The complete prompt for the LLM augmentation (translated into English).}
\end{table*}

See Table~\ref{tab:prompt} for the prompt for LLM augmentation. Table~\ref{tab:prompt_en} provides an English version.

\section{Instructions for Annotators}
\label{app:ann}
\subsection{First Stage}
In the first stage, we provide the annotators with the question, answer, statement, and cited documents. What LLM considers to be key segments are highlighted in red in the cited documents (see Figure~\ref{fig:stage} for an example). We instruct the annotators to follow the process below:

\par (1) First look at the highlighted text. If the highlighted text fully supports the statement, then the annotation is positive; if the highlighted text contradicts the statement, then the annotation is negative.

\par (2) If the annotation cannot be derived from the highlighted text, then look at the rest of the documents to make the annotation. When the documents fully support the statement, the label is positive, and when there is any information in the statement that contradicts the documents or information that is not mentioned in the documents, the label is negative.

\subsection{Second Stage}

In the second stage, we provide the annotator with the statement and the modified documents. In the documents, the modified parts are highlighted in green, where the dashed and crossed-out text is deleted and the rest is added (see Figure~\ref{fig:stage} for examples). 

For the annotation of whether the quality of the modification is acceptable, the annotators are instructed to note that qualified modifications need to satisfy the following two requirements: (1) There are no contradictions within each modified document. (2) The modified key segments are fluent in their own right and in the context of the document. The annotation for support is the same as the first stage, but based on the modified documents.

 
\section{Input and Training Details}
\label{app:detail}
We input the statement and the cited documents into the model and ask the model to determine whether the statement is fully supported by the documents, outputting yes or no. For input, we label and concatenate the cited documents in order (as shown in Table~\ref{tab:dataset}). For training, we use the following settings: For training, we use the following settings: learning rate is 5e-4, number of epochs is 10, scheduler is cosine scheduler, warmup ratio is 0.03, batch size is 256, and LoRA setting is $r=8$, $a=32$ and 0.1 dropout. We report the model performance for the epoch that achieves the best performance on the dev set.
\label{app:detail}



\section{Related Works}
Language models are known to produce hallucinations - statements that are inaccurate or unfounded~\citep{MaynezNBM20,HuCLGWYG24}. To address this limitation, recent research has focused on augmenting LLMs with external tools such as retrievers~\citep{GuuLTPC20,BorgeaudMHCRM0L22,LiuCtrla2024} and search engines~\citep{WebGPT2021, Komeili0W22, TanGSXLFLWSLS24}. While this approach suggests that generated content is supported by external references, the reliability of such attribution requires careful examination. Recent studies have investigated the validity of these attributions. \citet{DBLP:conf/emnlp/LiuZL23} conducted human evaluations to assess the verifiability of responses from generative search engines. \citet{hu2024evaluate} further investigate the reliability of such attributions when giving adversarial questions to RAG systems. Their findings revealed frequent occurrences of unsupported statements and inaccurate citations, highlighting the need for rigorous attribution verification~\citep{RashkinNLA00PTT23}. However, human evaluation processes are resource-intensive and time-consuming. To overcome these limitations, existing efforts~\citep{GaoDPCCFZLLJG23,DBLP:conf/emnlp/GaoYYC23} proposed an automated approach using Natural Language Inference models to evaluate attribution accuracy. While several English-language benchmarks have been developed for this purpose~\citep{DBLP:conf/emnlp/YueWCZS023}, comparable resources in Chinese are notably lacking. Creating such datasets presents unique challenges, particularly in generating realistic negative samples (unsupported citations).  To address this gap, we introduce the first large-scale Chinese dataset for citation faithfulness detection, developed through a cost-effective two-stage manual annotation process.

\begin{figure*}
    \centering
    \includegraphics[width=0.3\textwidth]{appendix/s1.png}
    \includegraphics[width=0.3\textwidth]{appendix/s2-m.png}
    \includegraphics[width=0.3\textwidth]{appendix/s2-d.png}
    \caption{Examples of interfaces that provide samples to the annotators. The first figure shows an example of the first stage. The last two images show the second stage with the same sample modified (information changed/deleted).}
    \label{fig:stage}
\end{figure*}



\end{document}


% This document was modified from the file originally made available by
% Pat Langley and Andrea Danyluk for ICML-2K. This version was created
% by Iain Murray in 2018, and modified by Alexandre Bouchard in
% 2019 and 2021 and by Csaba Szepesvari, Gang Niu and Sivan Sabato in 2022.
% Modified again in 2023 and 2024 by Sivan Sabato and Jonathan Scarlett.
% Previous contributors include Dan Roy, Lise Getoor and Tobias
% Scheffer, which was slightly modified from the 2010 version by
% Thorsten Joachims & Johannes Fuernkranz, slightly modified from the
% 2009 version by Kiri Wagstaff and Sam Roweis's 2008 version, which is
% slightly modified from Prasad Tadepalli's 2007 version which is a
% lightly changed version of the previous year's version by Andrew
% Moore, which was in turn edited from those of Kristian Kersting and
% Codrina Lauth. Alex Smola contributed to the algorithmic style files.
