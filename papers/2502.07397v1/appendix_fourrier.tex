\section{Elements of Fourier Analysis}\label{app: fourier}

\subsection{Formal definitions}\label{subsec: fourier defs}

To define the Fourier transform on $L^2(\Rb^d;\varrho)$, we will extend it from a dense subspace (see \cref{def: schwartz space}) of $L^2(\Rb^d;\varrho)$ to the whole space. This technical construction arises as a consequence of the fact that $L^2(\Rb^d;\varrho)\not\subset L^1(\Rb^d;\varrho)$, meaning the right-hand side of~\eqref{eq: fourier transform} may not be defined and $\fourier$ is ill-posed on $L^2(\Rb^d;\varrho)$, despite the fact that~\eqref{eq: fourier transform} is well-posed for $f\in L^1(\Rb^d;\varrho)$. The following is summarised from~\cite[Ch.5--6]{constantin_fourier_2016}, refer therein for a more detailed treatment or, e.g.,\ to \citep{folland_fourier_1992}. 

\begin{definition}\label{def: schwartz space}
    The Schwartz space $\Sc(\Rb^d)$ is defined as 
    \[ 
        \left\{\phi\in\Cc^\infty(\Rb^d;\Cb) : \sup_{x\in\Rb^d}\abs{x^\alpha\partial_\beta\phi(x)}<+\infty \mbox{ for any } \alpha,\beta\in\Nb^d\right\}
    \]
    in which $\alpha,\beta\in\Nb^d$ are multi-indices so that $x^{\alpha}:={(x_i^{\alpha_i})}_{i=1}^d$, and $\partial_\beta:=\partial_{x_1}^{\beta_1}\cdots\partial_{x_d}^{\beta_d}$.
\end{definition}

Note that $\Sc(\Rb^d)$ is a dense subspace of $L^2(\Rb^d;\varrho)$ and $L^1(\Rb^d;\varrho)$ as it contains $\Cc^\infty_c(\Rb^d;\Cb)$ the space of infinitely-differentiable compactly-supported (a.k.a.\ test) functions, which is dense in both $L^2(\Rb^d;\varrho)$ and  $L^1(\Rb^d;\varrho)$.

\begin{theorem}[{\cite[Thm.~6.1]{constantin_fourier_2016}}]\label{thm: constantin def fourier on Schwartz}
    Consider the Fourier transform operator  $\fourier$ on the Schwartz space, with 
    \begin{align}
        \fourier: \phi\in\Sc(\Rb^d)\mapsto\int \phi(x)e^{-2\pi i\langle x\vert \cdot\rangle_2}\de\varrho(x)\,.\label{eq: fourier transform}
    \end{align}
    This operator maps $\Sc(\Rb^d)$ maps onto itself and is an isometric bijection. Moreover, 
    \begin{align}
        \fourier^{-1}=\fourier\reflection\,,\label{eq: inverse fourier (formal operator notation)}
    \end{align}    
    in which $\reflection:\phi\in\Sc(\Rb^d)\mapsto \phi(-\cdot)\in\Sc(\Rb^d)$ is the \emph{reflection} operator.
\end{theorem}

\begin{theorem}[{\cite[Thm.~6.4]{constantin_fourier_2016}}]\label{thm: constantin fourier extension}
    The fourier transform $\fourier$ can be extended to a unitary operator on $L^2(\Rb^d;\varrho)$ and~\eqref{eq: inverse fourier (formal operator notation)} holds on $L^2(\Rb^d;\varrho)$ for this extension.
\end{theorem}

The formal inversion property~\eqref{eq: inverse fourier (formal operator notation)} is easily shown to recover the classical inversion formula 
\begin{align}
    f(x)=\int \fourier f(\xi)e^{2\pi i\langle x\vert \xi\rangle}\de\varrho(\xi) \mbox{ for $\varrho$-a.e. }x\in\state\, \label{eq: inverse fourier transform}
\end{align}
as soon as $f\in L^1(\Rb^d;\varrho)\cap L^2(\Rb^d;\varrho)$. In our case $\varrho$ is a finite measure so $L^2(\Rb^d;\varrho)\subset L^1(\Rb^d;\varrho)$ and the inversion formula always holds. If $\varrho$ is only $\sigma$-finite (e.g.\ the Lebesgue measure), one must take slightly higher care. Namely 
the difference between~\eqref{eq: inverse fourier (formal operator notation)} and~\eqref{eq: inverse fourier transform} is whether the integral in~\eqref{eq: inverse fourier transform} is well defined for $f\in L^2(\Rb^d;\varrho)$, which is not guaranteed. 

This technicality reflects the limits used in the definition of the extension which are hidden by the abstract statement of \cref{thm: constantin fourier extension}. Nevertheless, since the Schwartz space $\Sc(\Rb^d)$ is dense in both $L^1(\Rb^d;\varrho)$ and $L^2(\Rb^d;\varrho)$, we can always take an arbitrarily close function in $\Sc(\Rb^d)$ and invert that, the result will remain arbitrarily close in $L^2(\Rb^d;\varrho)$.


The Schwartzian framework turns out to be a robust one for Fourier analysis more generally, and we can also use to extend $\fourier$ beyond $L^2(\Rb^d;\varrho)$. In particular, it can be used to unify the definitions we gave for the Fourier transform of a function and a measure, refer to~\cite[\S~6.1.2]{constantin_fourier_2016} for more details. Precisely, one extends to the topological dual of $\Sc(\Rb^d)$ (the space of tempered distributions $\Sc'(\Rb^d)$), which includes $\measures(\Rb^d)$ and $L^2(\Rb^d;\varrho)$ as sub-spaces.


A fundamental consequence of the various formulations of the  Fourier transform is that measures whose transforms are in $L^2(\Rb^d;\varrho)$ are exactly those which have an $L^2(\Rb^d;\varrho)$ density with respect to $\varrho$. We will denote the density of a measure $\mu$ with respect to $\varrho$ using the Radon-Nikodym notation $\de\mu/\de\varrho$, even when this tempered distribution can be identified with a function.

\begin{lemma}\label{lemma: fundamental facts about fourier transform of measure}
    Let $\gamma\in\measures(\state)$ be a finite Radon measure, if it has density with respect to $\varrho$ and $\de\gamma/\de\varrho\in L^2(\Rb^d;\varrho)$, then 
    \[
        \fourier\gamma = \fourier \frac{\de\gamma}{\de\rho}\in L^2(\Rb^d;\varrho)\,.
    \]
    Conversely, if $\fourier\gamma\in L^2(\Rb^d;\varrho)$, then $\gamma$ has a density with respect to $\varrho$ and $\de\gamma
/\de\varrho\in L^2(\Rb^d;\varrho)$.
\end{lemma}
\begin{proof}
    The first part is a direct consequence of the definitions of the Fourier transforms of a measure and an $L^2(\Rb^d;\varrho)$ function. For the converse, the fact that $\fourier\gamma
\in L^2(\Rb^d;\varrho)$ implies $\gamma
 \ll \varrho$ involves some technical minutiae due to the different topologies $\measures(\state)$ can be equipped with, which we won't reproduce for conciseness, refer to e.g.\ \cite[Lemma~1.1]{fournier_absolute_2010}. That the density is then in $L^2(\varrho)$ is a simple consequence of Plancherel's theorem:
    \[
        \norm{\frac{\de\gamma}{\de\rho}}_{L^2(\Rb^d;\rho)} =\int_{\Rb^d}\abs{F\gamma(\xi)}^2\de\rho(\xi) = \norm{F\gamma}_{L^2(\Rb^d;\rho)}\,.%\qedhere
    \]
\end{proof}



\subsection{Technical details of Section {\ref{subsec: measure valued actions}}}



Let $\Cc_0(\Rb^d,\Kb)$ denote the space of continuous functions from $\Rb^d$ to $\Kb\in\{\Rb;\Cb\}$, $\measures(\Rb^d)$ denote the space of finite Borel measures over $\Rb^d$, and let us define the Fourier operator on this space by using the same notation, i.e.\ $\fourier: \gamma\in\measures(\Rb^d)\mapsto\fourier\gamma\in\Cc_0(\Rb^d;\Cb)$ with
\begin{align}
    \fourier\gamma: \xi\in\Rb^d \mapsto \int e^{-2\pi i\langle x\vert \xi\rangle_2}\de\gamma(x)\,
    .\label{eq: fourier transform of measure}
\end{align}
Note that we will eschew the standard notations $\hat f$ and $\hat\gamma$ in favour of $\fourier f$ and $\fourier\gamma$ to avoid confusion with the least-squares estimator, which we will denote using its standard hat.


% 

The Riesz-Markov theorem shows that $(\measures^*(\Rb^d),\norm{\cdot}_\infty)$, the space of finite signed Borel measures on $\Rb^d$ (endowed with the total variation norm $\norm{\cdot}_\infty$), is the topological dual of $(\Cc_0(\Rb^d),\norm{\cdot}_\infty)$, the space of continuous functions which vanish at infinity (endowed with the supremum norm $\norm{\cdot}_\infty$), refer e.g.\ to~\cite[p.~242]{constantin_fourier_2016}. This duality is characterised by the pairing
\[
    \langle \cdot\vert\cdot\rangle: (f,\gamma
)\in\Cc_0(\Rb^d)\x\measures^*(\Rb^d)\mapsto \int f \de \gamma
 \in \Rb\,.
\]
This pairing applies in particular to all functions $f\in\Cc(\state;\Rb)$ if $\state$ is compact and to all positive finite Borel measures $\gamma\in\measures^+(\state)$, and we will use the pairing notation in this case too. In general we will use the notation for arbitrary functions, understood that it will be well defined, see also \cref{remark: assumption L2 case}. In particular:
\[
\kant(\mu,\nu,c) = \inf_{\pi\in\Pi(\mu,\nu)}\langle c\vert \pi\rangle\,.
\] 

\begin{lemma}\label{lemma: finiteness of IP}
    For any finite Borel measure $\rho\in\measures(\Rb^d)$, any $\gamma\in\measures(\Rb^d)$ finite and with $\de\mu/\de\rho\in L^2(\Rb^d;\rho)$, and any $f\in L^2(\Rb^d;\rho)\cap L^1(\Rb^d;\rho)$, we have
    \[
        \langle f\vert \gamma\rangle = \langle \fourier\reflection f\vert \fourier\gamma\rangle_{L^2(\Rb^d;\rho)}\,
    \]
    and 
    \[
        \abs{\langle f\vert \gamma\rangle}\le \norm{f}_{L^2(\Rb^d;\rho)}\abs{\rho(\Rb^d)}\abs{\gamma(\Rb^d)}\,.
    \]
\end{lemma}

\begin{proof}
    By~\eqref{eq: inverse fourier transform}, 
    \begin{align}
        \langle f\vert\gamma\rangle:=\int f\de \gamma &=\int\int \fourier f(\xi)e^{2\pi i\langle x\vert \xi\rangle}\de\rho(\xi)\de\gamma(x)\,.\label{eq: fourier double integral}
    \end{align}
Let $\varphi:(x,\xi)\mapsto e^{2\pi i \langle x\vert\xi\rangle}$. Using~\eqref{eq: fourier double integral}, since by the Cauchy-Schwartz inequality
\begin{align}
    \abs{\langle f\vert\gamma\rangle} &\le \norm{Ff}_{L^2(\Rb^d\x\Rb^d;\gamma\tensor\rho)}\norm{1}_{L^2(\Rb^d\x\Rb^d;\gamma\tensor\rho)}\notag\\
    &= \norm{Ff}_{L^2(\Rb^d;\rho)}\gamma{(\Rb^d)}^2\rho(\Rb^d)<+\infty\label{eq: bound of IP in fourier space}\,,
\end{align}
the integrand in~\eqref{eq: fourier double integral} is $\gamma\tensor\rho$-integrable, and thus we can apply the Fubini-Lebesgue theorem to obtain
\begin{align*}
    \langle f\vert\gamma\rangle&= \int \fourier f(\xi) e^{2\pi i\langle x\vert \xi\rangle}\de[\gamma\tensor\rho](\xi,x)=\langle Ff\vert \varphi\rangle_{L^2(\Rb^d\times\Rb^d;\gamma\tensor\rho)}\,.
\end{align*}
Furthermore,
    \begin{align*}
        \langle f\vert\gamma\rangle&= \int \fourier f(\xi)\int e^{2\pi i\langle x\vert \xi\rangle}\de\gamma(x)\de\rho(\xi)\\
        &= \langle \fourier\reflection f\vert \fourier \gamma\rangle_{L^2(\Rb^d;\rho)}.
    \end{align*}
By~\eqref{eq: bound of IP in fourier space}, we have once more:
    \[
        \abs{\langle f\vert\gamma\rangle}=\abs{\langle \fourier\reflection f\vert \fourier \gamma\rangle_{L^2(\Rb^d;\rho)}} \le \norm{Ff}_{L^2(\Rb^d;\rho)}\gamma{(\Rb^d)}^2\rho(\Rb^d)\,. %\qedhere
    \]
\end{proof}

The benefit of \cref{lemma: finiteness of IP} may not be immediately apparent, but it is revealed when one notices that the $L^2(\Rb^d;\rho)$ inner products and norms considered on the right hand side depend only on the measure $\rho$ and not on $\gamma$. Thus, we are able to assume only integrability of $c^*$ only with respect to our reference measure $\varrho$ (recall~\eqref{eq: def entropy}) and still manipulate the duality product $\langle c^*\vert \gamma\rangle$ for any $\gamma$. In particular, by taking $\varrho=\mu\tensor\nu$ given marginals $\mu$ and $\nu$ and playing $\pi_t$ such that $\entf(c^*,\pi_t)<+\infty$ (recall~\eqref{eq: entropic OT def}) we can reduce $\langle c^*\vert \pi_t\rangle$ to a $L^2(\Rb^d;\varrho)$ inner product, moving our problem to a Hilbert space.

\begin{remark}\label{remark: assumption L2 case}
    \Cref{lemma: finiteness of IP} opens the subject of discussing \cref{asmp: L2 case}. Let us remark that if $S:=\supp(\mu\tensor\nu)$ is compact, continuity of $c^*$ on the closure of $S$ is sufficient to obtain these results. Similarly, if $c^*$ is bounded. However, \cref{asmp: L2 case} allows for many more functions, for instance it allows $c^*:(x,y)=\norm{x-y}^2_2$ if $(\mu,\nu)\in\Ps_2(\Rb^d)$, where $\Ps_2(\Rb^d)$ denotes measures with a finite second moment. This is of value as it covers the Wasserstein distances which are of broad interest. In general, one can develop finer assumptions based on $(\mu,\nu)$ even if $\varrho$ is not finite, but we do not detail this for brevity.
\end{remark}


