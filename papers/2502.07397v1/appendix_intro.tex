\section{Preliminaries}\label{app: intro}

\subsection{Organisation of Appendices}\label{subsec: organisation of appendices}

The following appendices are organised thematically and are mostly independent completions of various parts of the text. \Cref{subsec: notational precisions} contains notations and clarifications that are shared across them. 

\Cref{app: fourier} provides a rigourous treatment of necessary  Fourier analysis notions, which allow for a rigourous outlining of the schema detailed in \cref{subsec: measure valued actions}. 

\Cref{app: technical details,app: regret bounds,app: computation} contains the majority of the technical contributions of this work, including the major lemmata used in the proofs of the main text. \Cref{app: technical details} is dedicated to the details of the constructions in \cref{sec: Preliminaries}, while \cref{app: regret bounds} focuses on the general regret proofs of \cref{sec: regret of learning}, specifically the proofs of \cref{thm: regret Kantorovich,thm: entropic regret}. Finally \cref{app: computation} is dedicated to the details of \cref{subsec: computability} on specific regularity dependent regret.
Some miscellaneous minor results, or reproductions of results from prior works are collected in \cref{app: lemmas}.

The remaining appendices contain complements to the text and discussion of topics not mentioned therein for the sake of brevity. \Cref{app: open problems} contains more detailed discussions of the open problems mentioned in \cref{sec: directions}. \Cref{app: biblio} contains bibliographical notes on statistical optimal transport which readers unfamiliar with the field might find of interest to understand the context of the paper. It is a complement to \cref{sec:RWCC}.


\subsection{Notational precisions}\label{subsec: notational precisions}

Throughout the text, for a reference measure $\varrho$, let $L^p(\state,\Kb ;\varrho)$, $p\in[1,\infty]$ and $\Kb\in\{\Rb,\Cb\}$, denote the space of functions $f:\state\to\Kb$ that are $p$-integrable. When $\state$, $\Kb$, or $\varrho$ are clear from context we will drop them for brevity; by default $\Kb=\Cb$. We allow complex functions ($\Kb=\Cb$) to deal with the Fourier transforms, but this has no noticeable effect as it does not impact the Hilbertian structure of the space $L^2(\Rb^d,\Kb;\varrho)$. 

In the following, let $\langle \cdot\vert\cdot\rangle_{L^2(\Rb^d,\varrho)}$ denote the inner product on $L^2(\Rb^d,\Kb;\varrho)$, $\langle\cdot\vert\cdot\rangle_{\ell_2(\Rb^d)}$ the one on $\ell^2(\Rb,\Kb)$ (the space of square integrable real sequences) with $\norm{\cdot}_{\ell_2(\Rb^d)}$ denoting its associated norm. On $\Rb^d$, $\langle\cdot\vert\cdot\rangle_{2}$ denotes the inner product, $\norm{\cdot}_2$ the Euclidean norm. As before, let  $\langle\cdot\vert\cdot\rangle$ denote the duality pairing between $\measures(\Rb^d)$ (the space of finite Radon measures) and $\Cc_0(\Rb^d)$ (the space functions vanishing at infinity). The operator norm of a linear operator (in finite or infinite dimension) $A$ is denoted by $\norm{A}_{\op}$.

Throughout, all probabilistic statements are understood as holding in the filtered probability space $(\Omega,\Fc_\infty,\Fb,\Pb)$, in which $\Fb:={(\Fc_t)}_{t\in\Nb}$ is the natural filtration of ${(\xi_t)}_{t\in\Nb}$, and $\Fc_\infty=\lim_{t\to\infty}\Fc_t$.

For two measures $(\gamma,\rho)\in\measures(\Rb^d)$, $\gamma\ll\rho$ denotes that $\gamma$ is absolutely continuous with respect to $\rho$, in which case we use ${\de \gamma}/{\de \rho}$ to denote the Radon-Nikodym derivative (a.k.a.\ the density) of $\gamma$ with respect to $\rho$.