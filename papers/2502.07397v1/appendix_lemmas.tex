\section{Miscellaneous lemmas and proofs}\label{app: lemmas}

\subsection{Sub-Gaussian Analysis}\label{app: sub-gaussian}

\begin{definition}\label{def: sub-G}
    A random variable $\xi:\Omega\to\Rb$ is $\sigma^2$-sub-Gaussian if
    \[
        \Eb\left[\exp\left(t\xi\right)\right]\leq\exp\left(\frac{\sigma^2t^2}{2}\right) \quad\mbox{ for any } t\in\Rb\,.
    \]
    A stochastic process ${(\xi_i)}_{i\in\Nb}:\Omega\to \Rb^{\Nb}$ is $\sigma^2$-conditionally sub-Gaussian if
    \[
        \Eb\left[\exp\left(t\xi_i\right)\middle|\sigma({(\xi_j)}_{j<i})\right]\leq\exp\left(\frac{\sigma^2t^2}{2}\right)\quad \mbox{ for all } i\in\Nb \mbox{ and any } t\in\Rb\,.
    \]
\end{definition}

\begin{lemma}\label{lemma: sub-gaussian sum}
    Let ${(\xi_i)}_{i\in\Nb}$ be a $\sigma^2$-conditionally sub-Gaussian process, 
    \[ 
        \Pb\left(\sum_{i=1}^n \xi_i \ge \sigma\sqrt{2n\log\left(\frac 1\delta\right)}\right)\le \delta \quad\mbox{ for any }(n,\delta)\in\Nb\x(0,1).
    \]
\end{lemma}
\begin{proof}
    The proof follows Chernoff's method, by exponentiating $\sum_{i=1}^n\xi_i$ using $x\mapsto e^{tx}$, applying Markov's inequality, the tower rule accompanied by conditional sub-Gaussianity, and finally optimising the bound over the parameter $t>0$. 
\end{proof}




\subsection{A common summation identity}

\begin{lemma}\label{lemma: sum of terms from pegon bound}
    For $\alpha\in(0,1)$, let $\phi:u\in(0,+\infty)\to \alpha u^{-\alpha}\log(u)\in\Rb_+^*$, then for any $N\in\Nb$,
    \[
        \sum_{u=1}^N \phi(u)\le \frac{\alpha}{1-\alpha}N^{1-\alpha}\log(N) +\frac{\alpha}{2^{\alpha}}\log(6)\,.
    \]
    In particular, if $\alpha=1/2$, then 
    \[ 
        \sum_{u=1}^N \phi(u)\le\sqrt{N}\log(N)+\frac{1}{2\sqrt{2}}\log(6)\,.
    \]
\end{lemma}

\begin{proof}
    Notice that $\phi$ is differentiable, with $\phi'(u)=\alpha u^{-(1+\alpha)}(1-\alpha\log(u))$, so that it is decreasing on $(e^{1/\alpha},+\infty)$. Since $\sup_{\alpha>1}e^{1/\alpha}=e<3$, comparison between the sum and the integral of $\phi$ yields
    \[
        \sum_{u=1}^N\phi(u) \le \phi(1)+\phi(2)+\phi(3)+\int_3^N \phi(u)\de u\,.
    \]
    The remaining integral can be computed by parts, for $(a,b)\in\Rb_+^2$, $a<b$, 
    \begin{align*}
        \int_a^b\phi(u)\de u &= \frac{\alpha}{1-\alpha}\left({\left[u^{1-\alpha}\log(u)\right]}_a^b - \int_a^b u^{-{\alpha}}\de u\right)\\
        &=\frac{\alpha}{1-\alpha}\left({\left[u^{1-\alpha}\left(\log(u) -\frac{1}{\alpha-1}\right)\right]}_a^b\right)\\
        &\le \frac{\alpha}{1-\alpha}b^{1-\alpha}\log(b)
    \end{align*}
    for every $\alpha\in(0,1)$. Computing yields $\phi(1)=0$, $\phi(2)=\alpha 2^{-\alpha}\log(2)$, and $\phi(3)= \alpha 3^{-\alpha}\log(3)$, so that $\phi(1)+\phi(2)+\phi(3)\le \alpha 2^{-\alpha}\log(6)$.  Combining the results yields the desired inequality.
\end{proof}

