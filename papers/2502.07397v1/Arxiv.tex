
\documentclass[12pt]{article}

\usepackage{amsmath,hyperref,amsthm,microtype}
\usepackage{thm-restate}

\usepackage{times}

% \coltauthor{%
%  \Name{Lorenzo Croissant} \Email{lorenzo.croissant@ensae.fr}\\
%  \addr{CREST, ENSAE, CNRS, Palaiseau, France \\ INRIA FairPlay Team, INRIA, Palaiseau, France\\ }%
% }

\author{Lorenzo Croissant\thanks{The author would like to thank Nadav Merlis and Hugo Richard for their thoughtful comments on the manuscript, as well as Austin J. Stromme for sharing his insights in statistical optimal transport.} \\CREST, ENSAE, \& INRIA FairPlay Team, Palaiseau, France.\\ \texttt{lorenzo.croissant@ensae.fr}
}

%%%%%%%%%%%%%%%%%%%%%%%%%%%%%%%
% Editing tools
%%%%%%%%%%%%%%%%%%%%%%%%%%%%%%%

%\usepackage[textsize=tiny]{todonotes}

% highlight
\newcommand{\lc}[1]{{\color{purple} #1}}

%strikethrough
\usepackage[normalem]{ulem}
\newcommand{\lzost}[1]{\lc{\sout{#1}}}

%Margin comment 
\newcommand{\ltodo}[1]{\marginpar{\lc{#1}}}


\usepackage[a4paper, total={6in, 10in}]{geometry}
\usepackage{algorithm2e}
\usepackage{natbib}
\bibliographystyle{plainnat}
\setcitestyle{authoryear,open={(},close={)}}
\usepackage{cleveref}
\newcommand{\CG}{\mathcal{G}\xspace}
\newcommand{\CV}{\mathcal{V}\xspace}
\newcommand{\CE}{\mathcal{E}\xspace}
\newcommand{\CA}{\mathcal{A}\xspace}
\newcommand{\CF}{\mathcal{F}\xspace}
\newcommand{\CR}{\mathcal{R}\xspace}
\newcommand{\CB}{\mathcal{B}\xspace}
\newcommand{\CX}{\mathcal{X}\xspace}
\newcommand{\CK}{\mathcal{K}\xspace}
\newcommand{\CM}{\mathcal{M}\xspace}
\newcommand{\CC}{\mathcal{C}\xspace}
\newcommand{\CL}{\mathcal{L}\xspace}
\newcommand{\CI}{\mathcal{I}\xspace}
\newcommand{\CQ}{\mathcal{Q}\xspace}
\newcommand{\CO}{\mathcal{O}\xspace}
\newcommand{\CP}{\mathcal{P}\xspace}
\newcommand{\CS}{\mathcal{S}\xspace}
\newcommand{\CT}{\mathcal{T}\xspace}
\newcommand{\CJ}{\mathcal{J}\xspace}
\usepackage[para]{footmisc}
\usepackage{subfig}
% \usepackage{subcaption}
% \usepackage{array}
% \usepackage{colortbl}


%%% REVIEW
\newcommand{\tocite}{{\color{red}CITE} }
\newcommand{\toref}{{\color{red}REF} }

%%% LOGO
\newcommand{\usc}{\raisebox{-1pt}{\includegraphics[height=0.8em]{figures/usc_logo.png}}}
\newcommand{\vuam}{\raisebox{-1pt}{\includegraphics[height=0.8em]{figures/vu_logo.png}}}

%%% SIGNS and SYMBOLS
\newcommand{\grad}{\texttt{grad-CROP}}
\newcommand{\att}{\texttt{att-CROP}}
\newcommand{\seg}{\texttt{seg}}
\newcommand{\clip}{\texttt{clip-CROP}}
\newcommand{\sam}{\texttt{sam-CROP}}
\newcommand{\yolo}{\texttt{yolo-CROP}}
\newcommand{\hc}{\texttt{human-CROP}}
\newcommand{\zsvqa}{\texttt{ZSVQA}}
\newcommand{\vic}{\textbf{ViCrop}}
\newcommand{\xmark}{\text{\ding{55}}}
\newcommand{\cmark}{\text{\ding{51}}}
\newcommand{\success}{\texttt{\color{green} \cmark}}
\newcommand{\failure}{\texttt{\color{red} \xmark}}
\newcommand{\rel}{\texttt{rel-att}}
\newcommand{\gra}{\texttt{grad-att}}
\newcommand{\pgra}{\texttt{pure-grad}}
\newcommand{\relh}{\texttt{rel-att$^h$}}
\newcommand{\grah}{\texttt{grad-att$^h$}}
\newcommand{\pgrah}{\texttt{pure-grad$^h$}}


%%% Text Abb.
\makeatletter
\DeclareRobustCommand\onedot{\futurelet\@let@token\@onedot}
\def\@onedot{\ifx\@let@token.\else.\null\fi\xspace}

\def\aka{\emph{a.k.a}\onedot} \def\Eg{\emph{E.g}\onedot}
\def\eg{\emph{e.g}\onedot} \def\Eg{\emph{E.g}\onedot}
\def\ie{\emph{i.e}\onedot} \def\Ie{\emph{I.e}\onedot}
\def\cf{\emph{c.f}\onedot} \def\Cf{\emph{C.f}\onedot}
\def\etc{\emph{etc}\onedot} \def\vs{\emph{vs}\onedot}
\def\wrt{w.r.t\onedot} \def\dof{d.o.f\onedot}
\def\etal{\emph{et al}\onedot}
\makeatletter



\definecolor{myred}{HTML}{FF8577}
\definecolor{mygreen}{HTML}{0FA958}
\definecolor{myblue}{HTML}{1982C4}
\definecolor{codegreen}{rgb}{0,0.5,0}
\definecolor{codegray}{rgb}{0.5,0.5,0.5}
\definecolor{codepurple}{rgb}{0.07,0,0.53}
\definecolor{codered}{RGB}{189,41,0}
\definecolor{codecomment}{RGB}{153,153,153}
\definecolor{backcolour}{rgb}{0.96,0.96,0.96}
\definecolor{royalblue}{rgb}{0.0, 0.14, 0.4}
\definecolor{egyptianblue}{rgb}{0.06, 0.2, 0.65}
\definecolor{royalazure}{rgb}{0.0, 0.22, 0.66}
\definecolor{portlandorange}{rgb}{1.0, 0.35, 0.21}
\definecolor{sienna}{RGB}{183,105,68}
\definecolor{saddlebrown}{RGB}{139,69,19}
\definecolor{mediumbrown}{RGB}{83,41,11}
\definecolor{darkbrown}{RGB}{58,28,7}
\hypersetup{
    colorlinks=true,
    linkcolor=sienna,
    urlcolor=royalblue,
    citecolor=royalblue,
}
\newcommand{\thought}[1]{{\color[rgb]{0.2,0.39,0.66}(#1)}}
\newcommand{\todo}[1]{{\color[rgb]{1.0,0.0,0.0}(#1)}}
\newcommand{\hsh}[1]{{\color{green!50!black} Henrik: #1}}
\newcommand{\st}[1]{{\color{red!50!black} Sebastian: #1}}

\newcommand{\ulm}[1]{_{\scaleto{\mathrm{#1}}{3pt}}}
\newcommand\at[2]{\left.#1\right|_{#2}}











\newtheorem{assumption}{Assumption}

\DeclareMathOperator*{\argmax}{arg\,max}
\DeclareMathOperator*{\argmin}{arg\,min}

\newcommand{\swname}[1]{\texttt{#1}}
\newcommand{\ie}{i\/.\/e\/.,\/~}
\newcommand{\eg}{e\/.\/g\/.,\/~}
\newcommand{\cf}{cf\/.\/~}

\newcommand{\fig}{Fig\/.\/~}
\newcommand{\defn}{Def\/.\/~}
\newcommand{\sect}{Sec\/.\/~}
\newcommand{\tabl}{Tab\/.\/~}
\newcommand{\algo}{Algorithm~}
\newcommand{\theo}{Theorem~}

\newcommand{\bnnl}{3 hidden layers}
\newcommand{\bnnn}{50 neurons}
\newcommand{\bnna}{tanh activations}

\newcommand{\capt}[1]{\mdseries{\emph{#1}}}

\newcommand{\videolink}{at \url{https://youtu.be/_d7AqTRjz6g}}
\newcommand{\codelink}{\url{https://github.com/wheelbot/mini-wheelbot}}

\newcommand{\fakepar}[1]{\vspace{0mm}\noindent\textbf{#1.}}

\newcommand{\needref}{\textcolor{red}{[REF]}}

\newcommand{\plotfontsize}{9pt}



%%%%%%%%%%%%%%%%%%%%%%%%%%%%%%%%
% THEOREMS
%%%%%%%%%%%%%%%%%%%%%%%%%%%%%%%%
\theoremstyle{plain}
\newtheorem{theorem}{Theorem}[section]
\newtheorem{proposition}[theorem]{Proposition}
\newtheorem{lemma}[theorem]{Lemma}
\newtheorem{corollary}[theorem]{Corollary}

\theoremstyle{definition}
\newtheorem{definition}{Definition}

\theoremstyle{remark}
\newtheorem{remark}{Remark}[section]
\newtheorem{example}{Example}[section]

%%%%%%%%%%%%%%%%%%%%%%%%%%%%%%%
% layout
%%%%%%%%%%%%%%%%%%%%%%%%%%%%%%%

\newcommand{\mypar}[1]{{\textbf{#1}}}


\begin{document}
\title{Bandit Optimal Transport}

\maketitle
\begin{abstract}%
    Despite the impressive progress in statistical Optimal Transport (OT) in recent years, there has been little interest in the study of the \emph{sequential learning} of OT. Surprisingly so, as this problem is both practically motivated and a challenging extension of existing settings such as linear bandits. This article considers (for the first time) the stochastic bandit problem of learning to solve generic Kantorovich and entropic OT problems from repeated interactions when the marginals are known but the cost is unknown. We provide $\tilde{\mathcal O}(\sqrt{T})$ regret algorithms for both problems by extending linear bandits on Hilbert spaces. These results provide a reduction to infinite-dimensional linear bandits. To deal with the dimension, we provide a method to exploit the intrinsic regularity of the cost to learn, yielding corresponding regret bounds which interpolate between $\tilde{\mathcal O}(\sqrt{T})$ and $\tilde{\mathcal O}(T)$. 
\end{abstract}

\section{Introduction}
\label{sec:intro}

\begin{figure*}[tb]
    \centering
    \includegraphics[width=0.848\linewidth]{figs/circuitnn.pdf} 
    \caption{Illustration of differentiable CircuitNN. CircuitNN is designed based on differentiable NAND gates. After DAS is guided by PI and PO pairs of the truth table, CircuitNN can get the precise circuit architecture logic equivalent to the truth table.}
    \label{fig:circuitnn}
\end{figure*}

% 1. Describe the importance of logic synthesis
% 2. Existing Problems
% (a) Neural Architecture Search: Unstable, Predefined Setting, etc.
% (b) Circuit Generation: Probabilistic Model, Logic Equivalence

With the rapid advancement of technology, the scale of integrated circuits (ICs) has expanded exponentially. 
This expansion has introduced significant challenges in chip manufacturing, particularly concerning power and area metrics.
A primary objective in IC design is achieving the same circuit function with fewer transistors, thereby reducing power usage and area occupancy.

Logic synthesis~\cite{hachtel2005logicsynth}, a critical step in electronic design automation (EDA), transforms behavioral-level circuit designs into optimized gate-level circuits, ultimately yielding the final IC layout. 
The primary goal of logic synthesis is to identify the physical implementation with the fewest gates for a given circuit function. 
This task constitutes a challenging NP-hard combinatorial optimization problem. 
Current logic synthesis tools~\cite{brayton2010abc, wolf2013yosys} rely on human-designed heuristics, often leading to sub-optimal outcomes.

Differentiable architecture search (DAS) techniques~\cite{liu2018darts, chu2020darts} offer novel perspectives on addressing challenges in this problem.
Circuit functions can be represented through truth tables, which map binary inputs to their corresponding outputs. 
Truth tables provide a precise representation of input-output relationships, ensuring the design of functionally equivalent circuits.
Inspired by this, researchers~\cite{deepmind2024ai4sys, wang2024tnet} have begun exploring the application of DAS to synthesize circuits directly from truth tables.
Specifically, \citet{deepmind2024ai4sys} proposed CircuitNN, a framework that learns differentiable connection structures with logic gates, enabling the automatic generation of logic circuits from truth tables.
This approach significantly reduces the complexity of traditional circuit generation. 
Building on this, \citet{wang2024tnet} introduced T-Net, a triangle-shaped variant of CircuitNN, incorporating regularization techniques to enhance the efficiency of DAS.

Despite these advancements, several challenges remain. 
The computational complexity of DAS grows quadratically with the number of gates, posing scalability issues.
Although triangle-shaped architecture~\cite{wang2024tnet} partially mitigates this problem, redundancy persists. 
%Additionally, DAS is susceptible to converging to local optima, limiting the ability to search architectures that satisfy the given truth tables~\cite{liu2018darts}. 
%Furthermore, hyperparameters (network depth and layer width) require extensive searches, introducing complexity and prolonging the synthesis process. 
Additionally, DAS is susceptible to converging to local optima~\cite{liu2018darts} and hyperparameters (network depth and layer width) require extensive searches. 
The challenges arise from the vast search space in DAS. 
% Even with predefined settings for CircuitNN, finding a configuration that meets the truth table requires extensive trial and error during the DAS process. 
Intuitively, limiting the search space through predefined parameters (network depth, gates per layer, and connection probabilities) can significantly reduce the complexity.

Recent advances~\cite{openai2023gpt4, abramson2024alphafold3, esser2024sd3, li2024mar} in conditional generative models have demonstrated remarkable performance across language, vision, and graph generation tasks. 
Motivated by these developments, we propose a novel approach to circuit generation that generates preliminary circuit structures to guide DAS in generating refined circuits matching specified truth tables. 
Firstly, we introduce CircuitVQ, a tokenizer with a discrete codebook for circuit tokenization. 
Built upon our Circuit AutoEncoder framework~\cite{hou2022graphmae,li2023maskgae,wu2025mgvga}, CircuitVQ is trained through a circuit reconstruction task. 
Specifically, the CircuitVQ encoder encodes input circuits into discrete tokens using a learnable codebook, while the decoder reconstructs the circuit adjacency matrix based on these tokens.
Subsequently, the CircuitVQ encoder serves as a circuit tokenizer for CircuitAR pretraining, which employs a masked autoregressive modeling paradigm~\cite{chang2022maskgit, li2023mage}. 
In this process, the discrete codes function as supervision signals. 
After training, CircuitAR can generate discrete tokens progressively, which can be decoded into initial circuit structures by the decoder of the CircuitVQ. 
These prior insights can guide DAS in producing refined circuits that match the target truth tables precisely.

Our key contributions can be summarized as follows:
\begin{itemize}
\item We introduce CircuitVQ, a circuit tokenizer that facilitates graph autoregressive modeling for circuit generation, based on our Circuit AutoEncoder framework;
\item Develop CircuitAR, a model trained using masked autoregressive modeling, which generates initial circuit structures conditioned on given truth tables;
\item Propose a refinement framework that integrates differentiable architecture search to produce functionally equivalent circuits guided by target truth tables;
\item Comprehensive experiments demonstrating the scalability and capability emergence of our CircuitAR and the superior performance of the proposed circuit generation approach.
\end{itemize}

% Motivation
% (a) Diffusion (Vision, Graph), Autoregressive (Language, Vision)
% (b) Circuit Generation for Predefined Setting
% (c) Neural Architecture Search for Strict Logic Equivalence

% Contribution
% (a) Circuit Tokenizer (new transformer arch, training strategy)
% (b) CircuitAR (train and gen strategies, post-ar strategy)
% (c) Extensive Evaluation including BitD (Bit Distance) for Scalability

We study (stochastic) gradient descent on the empirical risk
\begin{equation*}
\cL(w) = \frac{1}{n}\sum_{i=1}^n l(p_i(w))\, ,
\end{equation*}
where the loss function $l$ and the functions  $(p_i)_{i=1}^n$  are specified in the following assumptions. Note that the empirical risk for binary classification from Equation~\eqref{def:emp_risk_intro} is a special case of the above objective.

\begin{assumption}\label{hyp:loss_exp_log}\phantom{=}
  \begin{enumerate}[label=\roman*)]
    \item The loss is either the exponential loss, $l(q) = e^{-q}$, or the logistic loss, $l(q) = \log(1{+}e^{-q})$.
    \item There exists an integer $L \in \mathbb{N}^*$  such that, for all $1 \leq i \leq n$, the function $p_i$ is $L$-homogeneous\footnote{We recall that a mapping $f : \mathbb{R}^d \rightarrow \mathbb{R}$ is positively $L$-homogeneous if $f(\lambda w) = \lambda^L f(w)$ for all $w \in \mathbb{R}^d$ and $\lambda >0$.}, locally Lipschitz continuous and semialgebraic.
  \end{enumerate}
\end{assumption}
If the $p_i$'s were differentiable with respect to $w$, the chain rule would guarantee that
\begin{align*}
\nabla \mathcal{L}(w) = \frac{1}{n}\sum_{i=1}^n  l'(p_i(w)) \nabla p_i(w)\enspace.
\end{align*}
However, we only assume that the $p_i$'s are semialgebraic. While we could consider Clarke subgradients, the Clarke subgradient of operations on functions (e.g., addition, composition, and minimum) is only contained within the composition of the respective Clarke subgradients. This, as noted in Section~\ref{sec:cons_field}, implies that the output of backpropagation is usually not an element of a Clarke subgradient but a selection of some conservative set-valued field.
Consequently, for $1\leq i \leq n$, we consider $D_i : \bbR^d \rightrightarrows\bbR^d$, a conservative set-valued field of $p_i$, and a function $\sa_i : \bbR^d \rightarrow \bbR^d$ such that for all $w \in \bbR^d$, $\sa_i(w) \in D_i(w)$. Given a step-size $\gamma >0$, gradient descent (GD)\footnote{More precisely, this refers to conservative gradient descent. We use the term GD for simplicity, as conservative gradients behave similarly to standard gradients.} is then expressed as
\begin{equation*}\label{eq:gd_new}\tag{GD}
  w_{k+1} = w_k - \frac{\gamma}{n} \sum_{i=1}^n l'(p_i(w_k))\sa_i(w_k)\,.
\end{equation*}
For its stochastic counterpart, stochastic gradient descent (SGD), we fix a batch-size $1\leq n_b \leq n$. At each iteration $k \in \bbN$, we randomly and uniformly draw a batch $B_k \subset \{1, \ldots, n \}$ of size $n_b$. The update rule is then given by 
\begin{equation*}\label{eq:sgd_new}\tag{SGD}
  w_{k+1} = w_k -  \frac{\gamma}{n_b}\sum_{i\in B_k} l'(p_i(w_k)) \sa_i(w_k)\, .
\end{equation*}
The considered conservative set-valued fields will satisfy an Euler lemma-type assumption.
%\nic{Smoother transition}
\begin{assumption}\phantom{=}\label{hyp:conserv}
  For every $i \leq n$, $\sa_i$ is measurable and $D_i$ is semialgebraic. Moreover, for every $w \in \bbR^d$ and $\lambda \geq 0$, $\sa_i(w)  \in D_i(w)$,
  \begin{equation*}
    D_i(\lambda w) = \lambda^{L-1} D_i(w)\, , \textrm{ and } \quad   L p_i(w) = \scalarp{\sa_i(w)}{w}\, .
  \end{equation*}
\end{assumption}
%\nic{Smoother transition}
Having in mind the binary classification setting, in which $p_i(w) = y_i \Phi(x_i, w)$, we define the margin
\begin{equation}\label{def:marg}
  \sm: \bbR^d \rightarrow \bbR, \quad \sm(w) = \min_{1\leq i \leq n} p_i(w)\, .
\end{equation}
It quantifies the quality of a prediction rule $\Phi(\cdot, w)$. In particular,  the training data is perfectly separated when $\sm(w) >0$. A binary prediction for $x$ is given by the sign of $\Phi(x, w)$, and under the homogeneity assumption, it depends only on the normalized direction $w / \norm{w}$. Consequently, we will focus on the sequence of directions $u_k := w_k / \norm{w_k}$. Our final assumption ensures that the normalized directions $(u_k)$ have stabilized in a region where the training data is correctly classified.

\begin{assumption}\label{hyp:marg_lowb}
  Almost surely, $\liminf \sm(u_k) >0$.
\end{assumption}
Before presenting our main result, we comment on our assumptions.

\paragraph{On Assumption~\ref{hyp:loss_exp_log}.} As discussed in the introduction, the primary example we consider is when $p_i(w) = y_i \Phi(x_i;w)$ is the signed prediction of a feedforward neural network without biases and with piecewise linear activation functions on a labeled dataset $((x_i,y_i))_{i \leq n}$. In this case,
\begin{equation}\label{eq:NN}
 p_i(w) = y_i \Phi(w;x_i) = y_i V_L(W_L) \sigma(V_{L-1}(W_{L-1}) \sigma(V_{L-1}(W_{L-2}) \ldots \sigma(V_{1}(W_1 x_i))))\, ,
\end{equation}
where $w = [W_1, \ldots, W_L]$, $W_i$ represents the weights of the $i$-th layer, $V_i$ is a linear function in the space of matrices (with $V_i$ being the identity for fully-connected layers) and $\sigma$ is a coordinate-wise activation function such as $z \mapsto \max(0,z)$ ($\ReLU$), $z \mapsto \max(az, z)$ for a small parameter $a>0$ (LeakyReLu) or $z \mapsto z$. Note that the mapping $w \mapsto p_i(w)$ is semialgebraic and $L$-homogeneous for any of these activation functions. Regarding the loss functions, the logistic and exponential losses are among the most commonly studied and widely used. In Appendix~\ref{app:gen_sett}, we extend our results to a broader class of losses, including $l(q) = e^{-q^a}$ and $l(q) = \ln (1 + e^{-q^a})$ for any $a \geq 1$.

\paragraph{On Assumption~\ref{hyp:conserv}.} Assumption~\ref{hyp:conserv} holds automatically  if $D_i$ is the Clarke subgradient of $p_i$. Indeed, at any vector $w \in \bbR^d$, where $p_i$ is differentiable it holds that $p_i(\lambda w) = \lambda^{L} p_i(w)$. Differentiating relatively to $w$ and $\lambda$ (noting that $p_i$ remains differentiable at $\lambda w$ due to homogeneity), we obtain $\lambda \nabla p_i(\lambda w) = \lambda^{L} \nabla p_i(w)$ and $\scalarp{\nabla p_i(\lambda w)}{w} = L \lambda^{L-1} p_i(w)$. The expression for any element of the Clarke subgradient then follows from~\eqref{eq:def_clarke}. 

However, for an arbitrary conservative set-valued field, Assumption~\ref{hyp:conserv} does not necessarily hold. For instance, $D(x) = \mathds{1}(x \in \mathbb{N})$ is a conservative set-valued field for $p \equiv 0$, which does not satisfy Assumption~\ref{hyp:conserv}. Nevertheless, in practice, conservative set-valued fields naturally arise from a formal application of the chain rule. For a non-smooth but homogeneous activation function $\sigma$, one selects an element $e \in \partial \sigma (0)$, and computes $\sa_i(w)$ via backpropagation. Whenever a gradient candidate of $\sigma$ at zero is required (i.e., in~\eqref{eq:NN}, for some $j$, $V_j(W_j)$ contains a zero entry), it is replaced by $e$. 
Since $V_j(W_j)$ and $V_j(\lambda W_j)$ have the same zero elements, it follows that for every such $w$, $
\sa_i(\lambda w) = \lambda^L \sa_i(w)$. The conservative set-valued field $D_i$ is then obtained by associating to each $w$ the set of all possible outcomes of the chain rule, with $e$ ranging over all elements of $\partial \sigma(0)$. Thus, for such fields, Assumption~\ref{hyp:conserv} holds.


\paragraph{On Assumption~\ref{hyp:marg_lowb}.} Training typically continues even after the training error reaches zero.
Assumption~\ref{hyp:marg_lowb} characterizes this late-training phase, where our result applies. 
As noted earlier, since $\sm$ is $L$-homogeneous, the classification rule is determined by the direction of the  iterates $u_k=w_k/\norm{w_k}$. Assumption~\ref{hyp:marg_lowb} then states that, beyond some iteration, the normalized margin remains positive. 
This assumption is natural in the context of studying the implicit bias of SGD: we \emph{assume} that we reached the phase in which the dataset is correctly classified and \emph{then} characterize the limit points. A similar perspective was taken in  \cite{nacson2019lexicographic}, where the implicit bias of GF was analyzed under the assumption that the sequence of directions and the loss converge. However, unlike their approach, ours does not require assuming such convergence a priori.

Earlier works such as \cite{ji2020directional,Lyu_Li_maxmargin}, which analyze subgradient flow or smooth GD, establish convergence by assuming the existence of a single iterate $w_{k_0}$ satisfying $\sm(w_{k_0}) > \varepsilon$ and then proving that $\lim \sm(u_{k}) > 0$. Their approach relies on constructing a smooth approximation of the margin, which increases during training, ensuring that $\sm(u_k) > 0$ for all iterates with $k \geq k_0$. This is feasible in their setting, as they study either subgradient flow or GD with smooth $p_i$’s, allowing them to leverage the descent lemma.

In contrast, our analysis considers a nonsmooth and stochastic setting, in which, even if an iterate $w_{k_0}$ satisfying $\sm(w_{k_0}) > \varepsilon$ exists, there is no a priori assurance that subsequent iterates remain in the region where Assumption~\ref{hyp:marg_lowb} holds. From this perspective, Assumption~\ref{hyp:marg_lowb} can be viewed as a stability assumption, ensuring that iterates continue to classify the dataset correctly. Establishing stability for stochastic and nonsmooth algorithms is notoriously hard, and only partial results in restrictive settings exist \cite{borkar2000ode,ramaswamy2017generalization,josz2024global}.

%Finally, note that Assumption~\ref{hyp:marg_lowb} only needs to hold almost surely. Specifically, with probability 1, there exist $k_0$ and $\varepsilon$ such that for all $k \geq k_0$, $\sm(u_k) \geq \varepsilon > 0$. In the case of~\eqref{eq:sgd_new}, $k_0$ and $\delta$ are random variables and may take different values across different realizations. 

%\paragraph{On constant stepsizes.}
%We allow the step size to be a constant of arbitrary magnitude, subject to the stability Assumption~\ref{hyp:marg_lowb}. This may seem surprising in a nonsmooth and stochastic setting, where a vanishing step size is typically required to ensure convergence (see, e.g., \cite{majewski2018analysis, dav-dru-kak-lee-19, bolte2023subgradient, le2024nonsmooth}).

\section{Challenges, related work, and contributions}\label{sec:RWCC}
Giving an exhaustive account of the vast literature of Optimal Transport would be outside the scope of this article. As it focuses on aspects of online learning, we will limit our attention to this narrow view. Nevertheless, we provide the curious reader a modest bibliography in \cref{app: biblio}.

\subsection{On optimal transport and learning}\label{subsec: OT and learning RW}
        \paragraph{Estimation of OT functionals}
        Much of the early work in statistical OT focused on estimating the value of the functional $\kant(\mu,\nu,c)$ when $(\mu,\nu)$ are unknown, but $c$ is known and highly regular, e.g.\ \citep{horowitz_mean_1994,weed_sharp_2019}. These regularity assumptions are motivated by the study of Wasserstein distances between probability measures (i.e.\ $c=\norm{\cdot-\cdot}^p$, $p\ge1$) via sampling. With the increased interest in the entropic OT problem, many works have asked the same questions about $\ent(\mu,\nu,c,\ve)$, e.g.\ \citep{rigollet_sample_2022,stromme_minimum_2024}.   

        This line of work is orthogonal to our investigation, as we know $(\mu,\nu)$ but not $c^*$. The critical object in this line of work is the regularity structure of $\kant(\mu,\cdot,c)$, when $c$ is strongly regular. For our problem, the relevant geometry is that of the transport functional $\pi\in\Pi(\mu,\nu)\mapsto \langle c^*\vert \pi\rangle$.

        \paragraph{Online matchings}
        Concurrently, Matching (discrete marginal OT), has been actively studied by computer scientists and economists. These works, such as \citep{perrot_mapping_2016}, are often directly inspired by applications, and have yielded many creative extensions to the OT problem:~\cite{alon_learning_2004} aims to learn an optimal matching using queries to an oracle;~\cite{johari_matching_2021} to identify \emph{types} of nodes;~\cite{min_learn_2022} to design a welfare-maximising social planner; etc.

        The common thread amongst these works is the nature of the \emph{market} on which they work: at each time $t$, a new supply becomes available to \emph{match} (i.e.\ transport from), and the agent must decide to which of its available demands to transport it. This decision problem is fundamentally different from our repeated OT problem as mistakes in the matching are permanent, while we replay a whole matching at each step.
        Furthermore, the information structure is different. \cite{jagadeesan_learning_2021,sentenac_pure_2021,sentenac_learning_2023} (amongst others) have highlighted that this problem is a combinatorial semi-bandit problem, in which there is feedback about each connection made. In our problem the agent receives feedback only about the matching as a whole (full bandit). These two differences make the problems seem superficially similar, but they are fundamentally different.
                          
        \paragraph{Online Learning to Transport}
        The first paper to take interest in online learning of optimal transport itself appears to be \citep{guo_online_2022-1}. In this article, the authors take an Online Convex Optimisation (OCO) approach to the problem, meaning that an adversary chooses a cost function $c_t$ at each round $t$ from a class of suitably regular (convex) functions. The learner aims to choose a sequence of transport plans $\pi_t$ which has a small regret with respect to the best fixed transport plan in hindsight. While this work pioneered the study of online (repeated) optimal transport, there are no direct reductions between this paper and their work.

        Most of the work of~\cite{guo_online_2022-1} is done under a full-information adversarial setting (as is typical in OCO): the transport problem changes at each round and is completely revealed after a coupling $\pi_t$ is played. However, in section 3, the authors provide a $0$-order semi-bandit scheme based on a discretisation of $\state$. In contrast, our work is directed at a stochastic setting under complete bandit feedback (only $\int c\de \pi_t$ is observed, with some noise).
        
        Due to the use of OCO techniques, as well as PDE-based optimal transport tools based on the work of~\cite{brenier_least_1989}, the results of~\cite{guo_online_2022-1} are only valid under strong assumptions on the regularity of the cost functional (and thus the cost function) and the marginals. In contrast, we work without specific assumptions on the cost function and marginals, beyond the minimal ones for~\eqref{eq: kantorovich def} to be well-defined. This difference arises because they consider general functionals on the Wasserstein space, while our work focuses on the specific regularity of OT functionals.

        This work was followed by~\cite{zhu_semidiscrete_2023} which considers the first online learning problem in semi-discrete optimal transport (i.e.\ $\mu$ discrete, $\nu$ continuous). They construct a semi-myopic algorithm with forced exploration which can learn to behave as the optimal plan from samples of the continuous marginal. Unfortunately, they do not study a general problem but rather only the case in which the cost $c^*$ is a linear parametric model. This choice obfuscates a large part of the complexity of the general problem and dilutes any insights about the geometry of the problem.
        Moreover, \cite{zhu_semidiscrete_2023} do not provide direct regret bounds, but rather performance metrics which may be converted into regret bounds. Sadly, these metrics fail to generalise to the continuous marginal case, and their analysis breaks down in the general setting.

\subsection{Bandit Algorithms}

As \cref{subsec: OT and learning RW} shows, bandits and optimal transport have been in peripheral contact repeatedly. Nevertheless, despite its interest in many optimisation problems, the bandit literature has remained  uninterested in the general optimal transport problem. Still, let us highlight the key elements of this theory on which we can build to solve the bandit optimal transport problem. 

\mypar{Multi-armed bandits}. The classical bandit problem \citep{thompson1933likelihood,lai_asymptotically_1985,auer_finite-time_2002} considered the issue of choosing the best amongst a finite set of arms based on bandit feedback about arm rewards. Since then, bandit theorists have taken some interest in higher-dimensional optimisation problems either linear or non-linear. 
For instance, \cite{tran-thanh_functional_2014} show regret bounds for learning a general functional using bandit feedback but sadly still considers only finitely many arms. While a general theory of bandits for functionals remains elusive \citep{wang_beyond_2022}, bandits under weak assumptions on the set of arms have been studied.

\mypar{Lipschitz bandits}. Several papers \citep{bubeck_lipschitz_2011,magureanu_lipschitz_2014,kleinberg_bandits_2019} have leveraged Lipschitz reward functions to provide regret bounds and algorithms, even on arbitrary metric spaces. Unfortunately, the bounds for general Lipschitz functions using these methodology are of the order of $\Theta(T^{{(d+1)}/{(d+2)}})$, in dimension $d\in\Nb$ \citep{kleinberg_bandits_2019}. In the case of the continuous optimal transport problem, this dimension is infinite, and the regret bounds become vacuous. The infinite dimensional nature of our problem also prevents the practical usability of most discretisations, even sophisticated ones like the tree-based scheme of~\cite{bubeck2011X-armed}. 

\mypar{Linear bandits}. In the hope of circumventing this problem, we can take inspiration from Kantorovich and recall that~\eqref{eq: kantorovich def} is linear program. Indeed, linear functions have much stronger global regularity than Lipschitz ones, meaning that linear bandits may escape vacuity even when $d=+\infty$.   

The setting of linear bandits was introduced by~\cite{auer_using_2003}, and refined by many subsequent works \citep{abeille_linear_2017,vernade_linear_2020,hao_high-dimensional_2020}, most notably for us~\cite{abbasi-yadkori_improved_2011}. In his doctoral thesis, Y.~\cite{abbasi-yadkori_online_2012} includes a version of this article in which the technical results are given not just for $\Rb^d$, but for an arbitrary Hilbert space. These works all use the celebrated Optimism in the Face of Uncertainty (OFU) principle to tackle the previously mentioned exploration-exploitation dilemma.

Nevertheless, in spite of its generality,~\cite{abbasi-yadkori_online_2012} is not sufficient to solve the bandit optimal transport problem, because the action space of our bandit is not a Hilbert space, and in fact the actions do not live in the same space as $c^*$. This fundamentally breaks the assumptions of this work, in spite of the fact that the duality product $\langle c\vert \pi \rangle$ defining $\kant$ is a linear form. 

\mypar{Kernel bandits.} Kernel methods intrinsically consider infinite-dimensional linear rewards, and may appear, at first, an ideal solution for solving bandit optimal transport. Kernel bandits have seen extensive work \citep{chowdhury_kernelized_2017,janz_bandit_2020,takemori_approximation_2021},  including~\cite{valko_finite-time_2013} which comes closest to our approach by introducing a kernelised OFU algorithm. These methods posit a particular structure for the reward function $c^*$, and then use the representer theorem to reduce the problem to a linear problem. Our problem, in contrast, is already linear so it should not require any such assumptions. 

One place where kernel methods shine is in making infinite-dimensional problems computationally tractable. While they can be used for this purpose in our setting, we will show that we can obtain similar bounds directly from the regularity of $c^*$ without assuming an RKHS structure. 

  
\subsection{Challenges and contributions}

\paragraph{Challenges} The specificities and challenges of the general BOT problem, can be summarised in three main points. 

\textbf{A)} The actions of this bandit problem are probability measures. In the discrete optimal transport (matching) problems previously studied in the literature, probability measures remain finite dimensional and can be represented using an inner product. This hides the true complexity of the general case in which one must confront a continuum of infinite-dimensional actions which require sophisticated tools to analyse. Moreover, this is compounded by the fact that the space of probability measures has a difficult geometry. 

\textbf{B)} The cost function $c^*$, which plays the role of a ``parameter'' to estimate, is a continuous function. Since the optimal transport problem only requires minimal integrability assumptions on $c^*$, the natural hypothesis classes for $c^*$ will be large function spaces\footnote{Circumventing this difficulty by parametrising $c^*$ as in~\cite{zhu_semidiscrete_2023} would dilute any insight about the geometry of the problem.} such as $L^2$.  This creates a significant difficulty for estimation and thus for bandit algorithms based on least-squares. The construction of estimators and confidence sets that permit the use of OFU algorithms is  challenged by the infinite dimensionality of $c^*$. 

\textbf{C)} Even if estimators for $c^*$ can be constructed, they must face the infinite-dimensionality of $c^*$. This raises the challenge of efficient approximation of infinite-dimensional estimators under weak assumptions, and of their associated regrets.

\paragraph{Contributions} This paper is the first study of the general stochastic bandit optimal transport problem. It provides a general framework for further work in this area, by showing that the problem is learnable under weak assumptions. Beyond this, the technical contributions can be summarised as follows.

\textbf{1)} To overcome challenge \textbf{A}, we construct a phase-space representation of the optimal transport problem which allows us to transform the problem into a linear bandit on a Hilbert space. This is enabled by the regularity of the entropic problem and tools from the Fourier analysis of measures.  

\textbf{2)} Combining \textbf{1} with the framework of~\cite{abbasi-yadkori_online_2012} we are able to construct the necessary confidence sets and estimators to estimate $c^*$ and address challenge \textbf{B}. By regularising optimism by entropy, and using the dual problem of~\eqref{eq: entropic OT def}, we are able to ensure our algorithm maintains the validity of the phase-space representation as it learns, unlocking regret analysis. In the regret analysis, we leverage the regularity of the entropic problem to prove bounds on the Kantorovich regret through the entropic one.

\textbf{3)} To face the infinite-dimensional quantities which arise in the general regret bounds, we construct a general estimation method based on the regularity of the cost function. This method addresses challenge \textbf{C} by allowing us to obtain regret bounds of order $\tilde\Oc(\sqrt{T})$ in simple cases, and an interpolation up to $\tilde\Oc(T)$ dependent on the regularity of $c^*$. 


% A few local macros that are used by the example content.
\newcommand{\expect}[2]{\mathds{E}_{{#1}} \left[ {#2} \right]}
\newcommand{\myvec}[1]{\boldsymbol{#1}}
\newcommand{\myvecsym}[1]{\boldsymbol{#1}}
\newcommand{\vx}{\myvec{x}}
\newcommand{\vy}{\myvec{y}}
\newcommand{\vz}{\myvec{z}}
\newcommand{\vtheta}{\myvecsym{\theta}}

\section{Introduction}

Recent advances in large language models (LLMs) are based on the principle that model performance improves predictably with increased model size and training data, following well-characterized scaling laws~\citep{kaplan2020scalinglawsneurallanguage}. At the same time, the computational and memory costs of  large models have driven significant interest in model compression techniques like quantization and sparsification. While model compression has been studied extensively in isolation, and in particular for post-training compression, we do not yet have a good understanding about how they interact with scaling of LLMs when performing compressed training from scratch. 

\noindent\textbf{The Compression Scaling Law.} Understanding these relationships  not only provides theoretical insights into compression, but also enable more principled approaches to building efficient language models. This report analyzes  how different forms of compression affect model scaling through an ``effective parameter count'' parameter that is associated to each form of compression. 
We show that, across both sparsity and quantization, the following general \textbf{compressed training law} holds:

\begin{equation} 
\label{eqn:main_law}
L(N, D, C) = \frac{a}{(N \cdot \text{eff}(C))^b} + \frac{c}{D^d} + e,
\end{equation}

\noindent where $N$ is the number of parameters, $D$ is the amount of data, $C$ is the compression type, and $N \cdot \text{eff}(C)$ denotes the \textbf{effective parameter count} for a given form of compression $C$, which would simply be $N$ for full precision. 
The form of this law is derived following prior work on sparsity scaling laws~\citep{frantar2023scalinglawssparselyconnectedfoundation}. 


Intuitively, $\text{eff}(C)$ indicates, on average, how much information one parameter compressed using the compressed representation $C$ (e.g., sparsity or quantization) is worth relative to a standard uncompressed BF16 parameter. For low compression, we expect this value to be close to 1; for high compression, it should be close to 0, based on the intuition that a 1-bit parameter cannot carry the same amount of information as a 16-bit one. 
Assuming this model is correct, up to constants, there are several notable consequences:

\begin{enumerate}
    \item To determine $\text{eff}(C)$, we only need to sweep over model sizes with fixed data, e.g., Chinchilla-optimal training~\citep{hoffmann2022trainingcomputeoptimallargelanguage}. In particular, we do not have to vary data amounts, leading to faster exploration.
    \item Second, if we determine this multiplier over short training runs (within reason), we  should still be able to transfer results to longer training runs. Notably, in the case of sparsity scaling laws, longer training durations were actually \emph{necesary} to reach  optimality for training over compressed models~\citep{frantar2023scalinglawssparselyconnectedfoundation}.
    \item Finally, a simple implication of this is formula the Chinchilla optimal amount of training data to use for a model compressed using technique $C$ is the Chinchilla-optimal amount of its uncompressed version, of size $N$, multiplied  by $\text{eff}(C)$. In other words, the parameter/size ratio remains the same, but model size is now expressed in terms of effective parameter count.
\end{enumerate}

The rest of this technical report is organized as follows: we first discuss our experimental setup in Section~\ref{sec:experimental_setup}. We then validate and analyze compression scaling laws for weight-only quantization, weight-and-activation quantization, and sparsity, in Section~\ref{sec:scaling_law_weight_quant}. We discuss related and concurrent work in Section~\ref{sec:related-work}.  We conclude in Section~\ref{sec:discussion}. 


\section{The Compression Scaling Law}
\label{sec:scaling_law_weight_quant}

\subsection{Proposed Law}

\paragraph{Definition.} We posit that, up to constants, the following general \textbf{compressed scaling law} for training holds:

\begin{equation} 
\label{eqn:main_law}
L(N, D, C) = \frac{a}{(N \cdot \text{eff}(C))^b} + \frac{c}{D^d} + e,
\end{equation}

\noindent where $N$ is the number of parameters, $D$ is the amount of data, $C$ is the compression type, $e$ is the irreducible error, and $N \cdot \text{eff}(C)$ denotes the \textbf{effective parameter count} for a given form of compression $C$, which would simply be $N$ for full precision training without compression. Thus, we call $\text{eff}(C)$ the \textbf{effective parameter multiplier (EPM)} that depends on the model architecture and compression type, but is independent of the parameter count or dataset size. 

The rest of this paper is dedicated to experimental validation of this law, and an exploration of its implications for sparse and quantized representations. 

\subsection{Experimental Setup}
\label{sec:experimental_setup}

\paragraph{Models and data.} 
We begin by detailing the experimental setup for validating the scaling law. 
Our experiments focus on Llama-type models~\citep{touvron2023llama2openfoundation} trained on a variant of the C4 dataset~\citep{raffel2020exploring}, following the Chinchilla ratios for tokens-per-parameter scaling, and well- optimized learning rate and batch size combinations for the setting.


\paragraph{Hyper-parameter transfer.}
One key aspect of scaling studies involving compression is that different hyper-parameter tuning between compressed and uncompressed baseline can skew results in a major way. For instance, training the compressed model with much higher learning rate can be a major advantage~\citep{wang2023bitnet}. At the same time, a poor choice of hyper-parameters can significantly disadvantage compressed training. 

Fortunately, it appears that good hyper-parameter settings for standard dense training seem to transfer quite well also to compressed training. In Figure~\ref{fig:lr_INT3} we illustrate this by showing the results of a learning rate sweep for models trained using BF16 and INT3 weights, finding that the relative ordering of settings remains identical, in both cases.\footnote{We were not able to reproduce some prior claims in literature that compressed models tolerate higher learning rates during training; both models seem to diverge at similar values in our experiments. } 
Thus, we always use the same hyper-parameters for dense and compressed training, guaranteeing fairness relative to the baseline while seemingly not hindering compression accuracy by too much.
\begin{figure}[h]
\centering
\includegraphics[width=0.45\textwidth]{figures/lr_sweep.pdf}
\caption{Learning rate sweep comparison for BF16 and INT3 models.}
\label{fig:lr_INT3}
\end{figure}


\paragraph{Data independence.}
One of our key simplifying assumptions is that compression multipliers are approximately \emph{constant}, in particular, they are independent of the amount of training data used. This has been suggested by~\citet{frantar2023scalinglawssparselyconnectedfoundation}; however, we perform some further validation in our experimental setup. For this, we repeat sweeps with two particularly interesting compression settings at 2x Chinchilla-training and compare the resulting lines to those predicted from the fitted multipliers of our 1x experiments. As can be seen in Figure~\ref{fig:data_independence} below, the lines are virtually identical, thus providing further evidence of the data independence.

\begin{figure}[h]
\centering
\includegraphics[width=0.45\textwidth]{figures/data-independence.pdf}
\caption{Validation of the independence between scaling laws and the amount of data used. The legend represents the number of bits per weight.}
\label{fig:data_independence}
\end{figure}

\begin{table*}[h]
\centering
\begin{tabular}{|l|c|}
\hline
\textbf{Llama 100M for 10B tokens @ 2-bit} & \textbf{Validation loss} \\
\hline
Centered = 2 zero point & 3.268 \\
Symmetric = 1.5 zero point & 3.250 \\
\hline
\end{tabular}
\caption{Tuning results for 2-bit quantization.}
\label{tab:weight-only-2bit}
\end{table*}

\paragraph{Quantized estimation.}
For weight quantization, we roughly follow BitNet~\citep{wang2023bitnet, ma2024era1bitllmslarge} and train quantized models via vanilla Straight-Through Estimation (STE). That is, we run forward passes with quantized weights, with scales computed dynamically from weight statistics. (As opposed to classic methods such as PACT~\citep{choi2018pact} or LSQ~\citep{esser2019learned}, scale parameters are not learned.) 

\subsection{Validation: Weight-Only Quantization}

\begin{figure*}[h]
\centering
\includegraphics[width=0.45\textwidth]{figures/weight-only-loss.pdf}
\includegraphics[width=0.45\textwidth]{figures/weight-only-fit.pdf}
\caption{Scaling results (loss and fit) for weight-only quantization.}
\label{fig:scaling-weight-quantization}
\end{figure*}

\begin{figure*}[ht!]
\centering
\includegraphics[width=0.45\textwidth]{figures/scaling-full-quant-linear.pdf}
\includegraphics[width=0.45\textwidth]{figures/scaling-full-quant-quadratic.pdf}
\caption{Scaling results for full quantization with linear and quadratic speedup counting.}
\label{fig:full-quantization}
\end{figure*}

Scaling experiments with weight-only quantization, shown in Figure~\ref{fig:scaling-weight-quantization} show clear trends, both in terms of raw data, as well as in terms of the scaling law fit. The corresponding bitwidth-effectiveness multipliers, shown in Table~\ref{tab:multipliers-eff-parameters}, suggest that 4-bit is indeed relatively close to lossless (matching state-of-the-art post-training quantization results). Significant size reductions at the same loss level can be made by further dropping the bit-width, coupled with corresponding quantization-aware pretraining. Although returns are clearly diminishing, further improvements still appear quite useful, e.g., 4-bit $\rightarrow$ 1-bit seems to give $\sim$2x improvement.

\begin{table}[h]
\centering
\begin{tabular}{|l|c|c|c|}
\hline
& 4-bit & 2-bit & 1-bit \\
\hline
EPM & 0.923 & 0.702 & 0.466 \\
Size gain & 3.69x & 5.62x & 7.46x \\
\hline
\end{tabular}
\caption{Effective parameter multipliers (EPM) and size gains for different bit-widths.}
\label{tab:multipliers-eff-parameters}
\end{table}



\subsection{Full Weight-and-activation Quantization}

We now extend the above approach to quantize not just the weights but also the corresponding activations. We note that, in these experiments, we only quantize inputs to all linear, non-embedding layers; this means attention itself is currently not quantized.
The results are shown in Figure~\ref{fig:full-quantization}, with two variants: with linear and quadratic speedup counting.  


\paragraph{Scaling.} We observe that, using the quantization strategy described in Section~\ref{sec:experimental_setup}, we can train low-bitwidth models in a surprisingly stable manner. One interesting observation is that when counting speedups linearly (i.e., 4-bit being 4x faster, as is the case on current hardware), then bit-widths lower than 4 are clearly not Pareto optimal. However, if we count speedups quadratically (the theoretical scaling of multiplication cost), 2-bit still brings a noticeable improvements. Moreover, the 1-bit option is far from optimal, but it is surprising that it trains stably.



The fit between the empirical and the predicted law is provided in Figure~\ref{fig:full-quantization-fit}. While this fit is not quite as clean as for weight-only quantization, it still looks very reasonable, especially for more than $ 1$-bit per parameter. Overall, we obtain reasonable  effectiveness multipliers up to 4-bit, after which they drop off quite sharply.

\begin{figure}[t]
\centering
\includegraphics[width=0.45\textwidth]{figures/full-quantization-scaling-law-fit.pdf}
\caption{Scaling law fit for full quantization.}
\label{fig:full-quantization-fit}
\end{figure}



\begin{table}[h]
\centering
% \scriptsize
\begin{tabular}{|l|c|c|c|c|}
\hline
& 8-bit & 4-bit & 2-bit & 1-bit \\
\hline
EPM & 0.857 & 0.747 & 0.289 & 0.067 \\
\hline
\end{tabular}
\caption{Effective parameter multipliers (EPM) for full quantization.}
\end{table}



\subsection{Mixed Activation-Weight Quantization}


\begin{figure}[h]
\centering
\includegraphics[width=0.45\textwidth]{figures/mixed-quantization-loss.pdf}
\caption{Scaling behavior of mixed quantization.}
\label{fig:mixed}
\end{figure}

The final experiment in this series is exploring mixed weight and activation bitwidth. 
The results for various combinations of weight and activation bitwidths are shown in Figure~\ref{fig:mixed}. 

While the overall scaling behavior seems to be as expected, we make the  observation that the Pareto lines of 2W4A and 2W2A as well as of 4W2A and 4W4A essentially overlap. A possible explanation for this effect could be that, unlike weights, activation bitwidth (below some level) scales linearly with the loss. That is, \emph{reducing activation bitwidth by a factor of 2 produces a loss roughly equivalent to a model with half the size}. This could be verified further, for instance, by running experiments with only quantized activations and full precision weights.



\section{Sparsity vs. Quantization}

While scaling behaviour of weight-sparsity is studied in depth in previous work~\citep{frantar2023scalinglawssparselyconnectedfoundation}, we revisit it here under a different setup:  for decoder-only architectures, larger models, and better tuned training parameters. Our goal is to compare the ``effective parameters'' of sparse vs. quantized representations, under similar computational cost. 

First, we find that standard STE-based optimization (top-k sparsification before every forward) works well. This is useful since sparsifying this way involves less hyper-parameter tuning (e.g., gradual pruning schedules) and is hence what adopt this setup across all subsequent experiments.

\paragraph{Method tuning.} However, compared to quantization, applying top-k on entire tensors before every forward pass can be relatively slow; hence, we investigated the importance of global per-tensor top-k in some isolated experiments. 
 Large N:M sparsity with larger values of N and M seems to work almost  identically to full per-tensor top-k, but is much faster ($>$ 2x in our particular setup).

\begin{table*}[h]
\centering
\begin{tabular}{|l|c|}
\hline
\textbf{Llama 100M for 10B tokens @ 50\% sparsity} & \textbf{Validation loss} \\
\hline
2:4 & 3.274 \\
64:128 & 3.223 \\
Per-output & 3.223 \\
Per-tensor & 3.224 \\
\hline
\end{tabular}
\caption{Tuning results for weight sparsity}
\end{table*}


Eventually, we use a \emph{per-row} configuration in our scaling sweep in order to make sure we are not restricting higher sparsity levels too much while still getting substantial speedups from smaller top-k windows.


\begin{figure}[h]
\centering
\includegraphics[width=0.45\textwidth]{figures/weight-sparsity-loss.pdf}
\includegraphics[width=0.45\textwidth]{figures/weight-sparsity-fit.pdf}
\caption{Scaling results for weight sparsity.}
\label{fig:sparsity-scaling}
\end{figure}

\paragraph{Scaling.} 
The scaling results for weight sparsity are shown in Figure~\ref{fig:sparsity-scaling}, across three different sparsities, showing a very good law fit. 
As expected, weight sparsity also leads to clear constant multiplier scaling. The size-gain multipliers are approximately 10\% better than identified in  previous work~\citep{frantar2023scalinglawssparselyconnectedfoundation}; this is a reasonable improvement (presumably due to a significantly better optimized setup overall), but still suggests that those multipliers remain relatively stable across many setup changes.

\paragraph{Comparing sparse vs. quantized representations.} 
The multipliers for sparsity are overall significantly lower than for weight-only quantization. However, sparsity also reduces FLOPs proportionally:  remarkably, 50\% sparsity yields a very similar multiplier as 8W8A (0.871 vs 0.857, respectively), which has exactly the same  FLOP reduction under linear scaling. 
Beyond that point, e.g. when comparing full 4-bit quantization (0.747) vs 75\% sparsity (0.622), \emph{quantization  seems to have noticeably better effective parameter count, even under a stringent linear FLOP scaling}. But we note that this balance is largely dependent on the baseline precision and we expect sparsity to become more competitive as we reduce the default precision for our models.  

\begin{table}[h]
\centering
\begin{tabular}{|l|c|c|}
\hline
& 50\% & 75\% \\
\hline
EPM & 0.871 & 0.622 \\
Size-gain & 1.74x & 2.49x \\
\hline
\end{tabular}
\caption{Effectiveness multipliers and size gains for weight sparsity.}
\label{tab:tuning-weight-sparsity}
\end{table}

\section{Related Work}
\label{sec:related-work}

Our work builds on and connects several lines of research around scaling laws, model compression techniques, and the intersection between them.

\paragraph{Scaling Laws for Language Models.} The foundation of this work builds on established scaling laws for language models that characterize how performance improves with model size and training data. \citet{kaplan2020scalinglawsneurallanguage} established the first comprehensive scaling laws showing that loss follows power law relationships with both parameters and data. \citet{hoffmann2022trainingcomputeoptimallargelanguage} refined these results with the Chinchilla scaling laws, suggesting that previous models were over-parameterized and that parameters and data should be scaled roughly equally. Recent work has revealed additional nuances in scaling behavior - for example, when considering data redundancy~\citep{muennighoff2023scaling}, or different model architectures~\citep{clark2022unified}.

\paragraph{Model Compression and Sparsity.} Parallel work has focused on making models more efficient through compression techniques. For sparsity, \citet{frantar2023scalinglawssparselyconnectedfoundation} established the first scaling laws characterizing how sparsity interacts with model and data scaling, showing that sparsity acts as a consistent multiplier on effective parameter count. Their work demonstrated that optimal sparsity levels increase with longer training, as dense models hit diminishing returns. This report directly builds on this earlier work, studying how different representations affect scaling. 

\paragraph{Quantization for Language Models.} 
Recent advances in quantization have enabled dramatically reduced precision while maintaining performance. Post-training quantization methods like GPTQ~\citep{frantar2022gptq} and AWQ~\citep{lin2023awq} have shown strong results for inference. For quantization-aware training, BitNet~\citep{wang2023bitnet} and  follow-up work~\citep{ma2024era1bitllmslarge, kaushal2024spectra} demonstrated stable training with binary and ternary weights, although a precise comparison against dense model scaling is not possible in their setting given the different hyper-parameters used. 

This work complements these efforts by characterizing how quantization during training affects fundamental scaling behavior - showing for instance that weight-only quantization maintains strong parameter efficiency even at very low bitwidths, for both weights and activations. 
In this respect, the thrust of our work is similar to that of concurrent work by~\citet{kumar2024scaling}: relative to their results, we propose a relatively simpler scaling law formulation, albeit in a narrower setting. Moreover, we reveal much more stable precision scaling, since we obtain results suggesting that 4-bit weights and activations may be Pareto-optimal, relative to their findings claiming that there is a precision barrier at around 8-bit precision. In addition, the main goal of our work is different, as we wish to consider a scaling comparison between sparsity and quantization. 

Our work advances this literature by providing the first unified scaling framework encompassing both sparsity and quantization, enabling principled comparison of these compression approaches in the context of large-scale training. The effective parameter framework we propose helps clarify when and how different compression techniques are most beneficial.


\section{Discussion}
\label{sec:discussion}

We have proposed and given evidence of a unified ``effective parameter count'' metric in compresed training of LLMs, revealing a few insights about compression scaling in language models. For instance, through this lens, we find that weight-only quantization maintains strong parameter efficiency even at very low bitwidths, with 4-bit to 1-bit showing approximately 2x improvement. However, when quantizing both weights and activations, we observe clear diminishing returns below 4 bits. This suggests that 4-bit quantization may be optimal for full model quantization. This challenges previous work suggesting ever-lower bitwidths are always better~\citep{ma2024era1bitllmslarge} or that 8-bit presents a clear complexity barrier~\citep{kumar2024scaling}.

One implication concerns the comparison between sparsity and quantization under similar compute constraints. Interestingly, at 50\% sparsity, the effective parameter multiplier (EPM) of 0.871 nearly matches 8-bit quantization (0.857). However, at higher compression rates (e.g., 75\% sparsity vs 4-bit quantization), quantization demonstrates better parameter efficiency. These results suggest that different compression approaches may be optimal depending on the training regime – weight-only quantization provides strong benefits for inference scenarios, while full quantization may be preferable for training efficiency but should likely not go below 4 bits, at least when using the methods from our study.
Importantly, the effectiveness of compression appears to increase with longer training, suggesting particular benefits for heavily-overtrained models. This insight, combined with our unified scaling framework, enables principled selection of compression methods based on compute budgets and deployment scenarios.

Several important questions remain open for future work. First, while our results focus on autoregressive language models, understanding how these scaling relationships transfer to other architectures would be valuable. Second, the potential benefits of combining different compression approaches – for instance, using both sparsity and quantization – remains unexplored. Finally, investigating the role of different quantization schemes (e.g., floating point vs integer) could reveal additional insights into compression scaling.
\section{Computing Optimal Robust Contracts}\label{sec:computational_problem}



Now, we present our main result: a polynomial-time algorithm to compute an optimal $\delta$-robust contract.
%
%\bac{and we show that, differently from Stackeberg games, the problem admits a polynomial time solution. }
%



\subsection{Characterizing an Optimal Contract}

We begin by presenting an optimization problem that characterizes an optimal $\delta$-robust contract. 
Consider an arbitrary optimal $\delta$-robust contract $p^\star$, as well as two arbitrary actions $a^\star \in A(p^\star)$ and $a^\delta \in \arg\min_{a \in A^\delta(p^\star)} u^\sfP(p^\star, a)$.
By fixing $a^\star$ and $a^\delta$, we show that the following optimization problem, over the variable $p \in \mathbb{R}_+^m$, characterizes an optimal $\delta$-robust contract $p^\star$.
% optimal contracts can be characterized as optimal solutions to the following optimization problem, over variable $p$.
\begin{align}
	\label{eq:opt}
	\max_{p\in  \mathbb{R}_+^m}
	\quad  
	u^\sfP(p, a^\delta)
	%F_{a^\delta} \cdot (\rr - p) 
\end{align}
subject to the following \emph{disjunctive} constraints, which must hold for every agent's action $a \in \mathcal{A}$:
\begin{align}
	\hspace{-2mm}\Big( u^\sfA(p, a) \le u^\sfA(p, a^\star) -   \delta  \Big) \vee \Big(u^\sfP(p, a) \ge u^\sfP(p, a^\delta)\Big).\tag{\ref*{eq:opt}a}\label{eq:opt-1}
	% &\Big( u^\sfA(p, a) \le u^\sfA(p, a^\star) - \delta \Big) 
	%\;\bigvee \nonumber\\
	%&\qquad\qquad\qquad
	%\Big( u^\sfP(p, a \ge u^\sfP(p, a^\delta) \Big).
	%\tag{\ref*{eq:opt}a}\label{eq:opt-1}
\end{align}

% \albo{An intuition on the constraint would be helpful.}

Intuitively, \Cref{eq:opt-1} requires that each action $a$ is either \emph{not} a $\delta$-best response (first inequality) or no worse than $a^\delta$ for the principal (second inequality). This ensures that the objective function $u^\sfP(p,a^\delta)$ captures the principal's utility in a $\delta$-robust contract $p$.
More formally, we establish the following lemma (recall that $\Psi(p) = \min_{a \in A^\delta(p)} u^\sfP(p, a)$ denotes the principal's utility in a $\delta$-robust contract $p$).


% \begin{lemma}
	% If $p$ satisfies \Cref{eq:opt-1}, then $u^\sfP(p, a^\delta) \le \Psi(p)$.
	% Moreover, if $p$ is an optimal solution to \Cref{eq:opt}, then $u^\sfP(p, a^\delta) = \Psi(p)$. 
	% \end{lemma}

% \begin{proof}
	
	% \end{proof}

% \begin{lemma}
	% If $p$ is an optimal solution to \Cref{eq:opt}, then $u^\sfP(p, a^\delta) = \Psi(p)$. Additionally, $a^\star \in \BR(p)$ and $a^\delta \in \BR_\delta(p)$. 
	% \end{lemma}

% \begin{proof}
	
	% \end{proof}

\begin{lemma}
	\label{lmm:Psi-opt}
	% Suppose that $p^\star$ is an optimal $\delta$-robust contract; moreover, $a^\star \in \BR(p^\star)$, and $a^\delta \in \arg\min_{a \in \BR_\delta(p^\star)} u^\sfP(p^\star, a)$.
	Every optimal solution $p \in \mathbb{R}_+^m$ to Problem~\eqref{eq:opt} is an optimal $\delta$-robust contract, i.e., $\Psi(p) = \Psi(p^\star)$.
\end{lemma}

\begin{proof}
	First, observe that $p^\star$ is a feasible solution to Problem~\eqref{eq:opt}.
	Indeed, if an action $a$ is {\em not} a $\delta$-best response to $p^\star$, then by definition this means $u^\sfA(p^\star, a) \le u^\sfA(p^\star, a^\star) - \delta$; otherwise, it must be that $u^\sfP(p^\star, a) \ge u^\sfP(p^\star, a^\delta)$ as $a^\delta$ is by definition the worst $\delta$-best response in $p^\star$.
	As a result, \Cref{eq:opt-1} holds for $p^\star$ for every $a \in \mathcal{A}$.
	
	Furthermore, notice that, according to the definition of $a^\delta$:
	\[
	\Psi(p^\star) = \min_{a \in A_\delta(p^\star)} u^\sfP(p^\star, a) = u^\sfP(p^\star, a^\delta).
	\]
	
	In order to complete the proof, it is sufficient to show that $u^\sfP(p, a^\delta) \le \Psi(p)$ for every feasible solution $p\in \mathbb{R}_+^m$ to Problem~\eqref{eq:opt}.
	%to complete the proof.
	Indeed, if the condition above holds, for any arbitrary optimal solution $p'\in \mathbb{R}_+^m$ to Problem~\eqref{eq:opt}, we have:
	\begin{align*}
		\Psi(p^\star)
		= u^\sfP(p^\star, a^\delta) 
		\le u^\sfP(p', a^\delta)
		\le \Psi(p') 
		\le \Psi(p^\star),
	\end{align*}
	where $u^\sfP(p^\star, a^\delta) \le u^\sfP(p', a^\delta)$ since $p^\star$ is a feasible solution to Problem~\eqref{eq:opt} and $p'$ is optimal,
	while $\Psi(p') \le \Psi(p^\star)$ since $p^\star$ is an optimal $\delta$-robust contract by definition.
	%
	Then, it must be the case that $\Psi(p') = \Psi(p^\star)$.
	
	Now, to complete the proof, consider any feasible solution $p\in \mathbb{R}_+^m$ to Problem~\eqref{eq:opt}. We show that $u^\sfP(p, a^\delta) \le \Psi(p)$.
	%to complete the proof.
	%
	Pick an arbitrary best-response action $a' \in A(p)$ and consider any $a \in A^\delta(p)$.
	By definition,
	\begin{align*}
		u^\sfA(p, a) > u^\sfA(p, a') - \delta = \max_{a \in \Actions} u^\sfA(p, a) - \delta 
		\ge u^\sfA(p, a^\star) - \delta.
	\end{align*}
	Hence, for \Cref{eq:opt-1} to hold for action $a$, it must be that $u^\sfP(p, a) \ge u^\sfP(p, a^\delta)$.
	Since the choice of $a$ is arbitrary, this holds for every $a \in A^\delta(p)$.
	Consequently,
	\[
	\Psi(p) = \min_{a \in A^\delta(p)} u^\sfP(p, a) \ge u^\sfP(p, a^\delta).
	\qedhere
	\]
\end{proof}

The above lemma implies that we can effectively ``guess'' $a^\star$ and $a^\delta$, fixing these actions in Problem~\eqref{eq:opt} and solving the optimization problem to obtain $p^\star$ (or possibly a different, but still optimal, $\delta$-robust contract).
Since there are only $O(n^2)$ possible combinations of the values of $a^\star$ and $a^\delta$, the approach is efficient as long as Problem~\eqref{eq:opt} can be solved efficiently.
A correct guess yields a contract $p$ such that $\Psi(p) = \Psi(p^\star)$, whereas $\Psi(p) \le \Psi(p^\star)$ for incorrect guesses.
%
Thus, by comparing the $\Psi$ values, we can identify a correct guess and a corresponding optimal $\delta$-robust contract.

It remains to show how to efficiently solve Problem~\eqref{eq:opt}.

\subsection{Solving Problem~\eqref{eq:opt}}

Problem~\eqref{eq:opt} does \emph{not} directly admit any efficient solution algorithm due to the non-convex constraint in~\Cref{eq:opt-1}.
To deal with this issue, we rewrite \Cref{eq:opt-1} as follows:
\begin{align}
	\Big( F_a \cdot p \le c_a + u^\sfA(p, a^\star) - \delta \Big) \;\vee  \;  \Big( F_a \cdot p \le F_a \cdot r - u^\sfP(p, a^\delta) \Big),
	\label{eq:opt-1-re}
\end{align}
by expanding the utilities as $u^\sfA(p, a) = F_a \cdot p - c_a$ and $u^\sfP(p, a) = F_a \cdot (r - p)$ and rearranging the terms.

Hence, \Cref{eq:opt-1} is satisfied for action $a \in \mathcal{A}$ if and only if $F_a \cdot p$ is smaller than the maximum of the right-hand sides of the two inequalities in~\Cref{eq:opt-1-re}.
The constraint effectively reduces to the second inequality for all $p \in \Rset$ such that 
\begin{align}
	\label{eq:right-sides}
	c_a + u^\sfA(p, a^\star) - \delta \le F_a \cdot r - u^\sfP(p, a^\delta),
\end{align} 
%while it reduces to the second inequality for all other $p\in \Rset$.
while for the other $p\in \Rset$, it reduces to the first inequality.

Consequently, we can partition the contract space based on the satisfiability of \Cref{eq:right-sides}, considering all $a \in \mathcal{A}$.
Within each subspace in the partition, only one inequality in \Cref{eq:opt-1-re} is active for every $a \in \mathcal{A}$. So, effectively, \Cref{eq:opt-1-re} reduces to a linear constraint for each action $a$, and the optimization problem to an LP.
%
It then suffices to solve an LP for every subspace, each generating an optimal contract within its corresponding subspace.
%
Among these contracts, the one providing the highest utility is an optimal solution to Problem~\eqref{eq:opt}.


Now, if the linear inequalities in \Cref{eq:right-sides} (one for each action $a \in \mathcal{A}$) were $n$ arbitrary inequalities, the above partition may consist of exponentially many subspaces, making the approach inefficient. 
Fortunately, the partition is much more well-structured, since the hyperplanes corresponding to the inequalities are {\em parallel}, as the coefficients of $p$ in \Cref{eq:right-sides} are invariant with respect to $a$.
% Surprisingly, in the context of contract design, the partition is well-structured and involves only $O(n)$ subspaces as we demonstrate next.
As a result, they partition the space into only $O(n)$ subspaces.


Next, we show how to exploit the above observation to solve Problem~\eqref{eq:opt}, by solving $O(n)$ suitable subproblems instead.

\subsection{Formulating the Subproblems}

Let us first rearrange \Cref{eq:right-sides} as follows:
\begin{align}
	\label{eq:right-sides-re}
	u^\sfA(p, a^\star) + u^\sfP(p, a^\delta) - \delta \le \nu_a \coloneqq F_a \cdot r - c_a, 
\end{align} 
where $\nu_a$ is exactly the social welfare generated by action $a $ (which is independent of the specific contract adopted).

Then, we can re-order agent's actions $a_1, \dots, a_n$ in such a way that $\nu_{a_1} \le \nu_{a_2} \le \dots \le \nu_{a_n}$. For simplicity, we write $\nu_j = \nu_{a_j}$, and we let $\nu_0 = -\infty$ and $\nu_{n+1} = +\infty$.
% It is then straightforward that for each $j=1,\dots,n+1$, the following is a subspace in the partition:
% \[
% \left\{ 
% p \in \mathbb{R}^{m}_{+} :  
% \nu_{j-1} \le u^\sfA(p, a^\star) + u^\sfP(p, a^\delta) - \delta \le \nu_{j}
% \right\}.
% \]
Then, the following lemma is straightforward.

\begin{lemma}
	\label{lmm:right-sides}
	For every contract $p \in \mathbb{R}^{m}_{+}$, if it holds that $\nu_{j-1} \leq u^\sfA(p, a^\star) + u^\sfP(p, a^\delta) - \delta \leq \nu_{j}$, then:
	%
	% If $u^\sfA(p, a^\star) + u^\sfP(p, a^\delta) - \delta \in [\nu_{j-1}, \nu_{j}]$, then:
	\begin{itemize}
		\item $F_a \cdot p \le F_a \cdot r - u^\sfP(p, a^\delta) \Longleftrightarrow$ \Cref{eq:opt-1-re} holds for all actions $a  \in \{a_\ell \mid \ell \le j - 1\}$; and 
		%
		\item $F_a \cdot p \le c_a + u^\sfA(p, a^\star) - \delta \Longleftrightarrow$ \Cref{eq:opt-1-re} holds for all actions $a  \in \{a_\ell \mid j \le \ell \}$.
	\end{itemize}
\end{lemma}

In other words, the condition in the lemma defines a suitable subspace of contracts $\mathcal{P}_j$ for each $j \in \{1,\dots,n+1\}$. 
The following LP solves for a $p \in \Rset$ that is optimal within $\mathcal{P}_j$. 
\begin{align}
	\label{eq:lp-nu-j}
	\max_{p \in \mathbb{R}_+^m}
	\quad
	u^\sfP(p, a^\delta)
	% F_{a^\delta} \cdot (\rr - p) 
	\tag{\ref*{eq:lp-nu-j-const}}
\end{align}
subject to the following constraints:
\begin{subequations}
	\label{eq:lp-nu-j-const}
	\begin{align}
		%&\hspace{-2mm}
		%p_\omega \ge 0 \quad \text{ for } \omega \in \Omega \\
		&%\hspace{-3mm}
		\nu_{j-1} \le u^\sfA(p, a^\star) + u^\sfP(p, a^\delta) - \delta \le  \nu_j \label{eq:lp-nu-j-1} \\
		&%\hspace{-3mm}
		F_a {\cdot} \, p \le F_a \cdot r - u^\sfP(p, a^\delta)
		\quad \forall a \in \{a_\ell \mid  \ell \le j{-}1\}  \label{eq:lp-nu-j-2} \\
		&%\hspace{-3mm}
		F_a {\cdot} \, p \le c_a + u^\sfA(p, a^\star) - \delta
		\quad\quad \forall a \in \{a_\ell \mid j \le \ell \}. \label{eq:lp-nu-j-3}
	\end{align}
\end{subequations}
Since $\bigcup_{j=1}^{n+1} \mathcal{P}_j = \mathbb{R}^m_+$, solving the LP in Problem~\eqref{eq:lp-nu-j-const} for all $j \in \{1,\dots,n+1\}$ and picking the best among the obtained solutions gives an optimal solution to Problem~\eqref{eq:opt}.


\begin{remark}
	The left-hand side of \Cref{eq:right-sides-re} is roughly (up to a $\delta$ difference) the social welfare of an optimal contract.
	Thus, \Cref{lmm:right-sides} can be interpreted as follows.
	For low-social-welfare actions $a_1,\dots, a_{j-1}$, yielding a sufficiently high utility for the principal automatically provides a low utility for the agent.
	%
	This fulfills the first inequality in \Cref{eq:opt-1-re} (and \Cref{eq:opt-1}).
	Conversely, for high-social-welfare actions $a_j,\dots, a_n$, a sufficiently low utility for the agent automatically gives a high utility for the principal.
	%
	This fulfills the second inequality in \Cref{eq:opt-1-re} (and \Cref{eq:opt-1}).
\end{remark}

\begin{algorithm}[!htp]
	\caption{Compute an optimal $\delta$-robust contract}
	\label{alg:robust_poly}
	\begin{algorithmic}[1]
		% \REQUIRE \bac{Princiapl-agent isnstance and $\delta>0$} \jiarui{this require is somewhat obvious. we can omit?} \bac{Yess, you are right!}
		\State $p^\star \leftarrow \texttt{null}$,\; 
		$\psi^\star \leftarrow -\infty$
		\ForAll{$(a^\star,a^\delta) \in \mathcal{A} \times \mathcal{A}$} \label{ln:inner-for}
		\ForAll{$j=1,\dots,n+1$}
		\State 
		% \begin{align}
			% \label{eq:lp-nu-j}
			% \max
			% \quad
			% u^\sfP(p, a^\delta)
			% % F_{a^\delta} \cdot (\rr - p) 
			% \tag{\ref*{eq:lp-nu-j-const}}
			% \end{align}
		Solve Problem~\eqref{eq:lp-nu-j-const} instantiated with $(a^\star,a^\delta)$ s.t. \Cref{eq:lp-nu-j-1,eq:lp-nu-j-2,eq:lp-nu-j-3},
		% solve $\max_{p \in \mathbb{R}_+^m} u^\sfP(p, a^\delta)$,
		% s.t. 
		% Eqs.~\eqref{eq:lp-nu-j-1}--\eqref{eq:lp-nu-j-3}, 
		and let an optimal solution be $p'$
		
		\If{$\Psi(p') > \psi^\star$}
		\State $p^\star \leftarrow p'$ 
		\State $\psi^\star \leftarrow \Psi(p')$
		\EndIf
		
		% the following constraints:
		% \begin{subequations}
			% \label{eq:lp-nu-j-const}
			% \begin{align}
				% %&\hspace{-2mm}
				% %p_\omega \ge 0 \quad \text{ for } \omega \in \Omega \\
				% &\hspace{-2mm}
				% \nu_{j-1} \le u^\sfA(p, a^\star) + u^\sfP(p, a^\delta) - \delta \le  \nu_j \label{eq:lp-nu-j-1} \\
				% &\hspace{-2mm}
				% F_a \cdot p \le F_a \cdot r - u^\sfP(p, a^\delta)
				% \,\, \forall a \in \{a_{i} \}_{i<j} \label{eq:lp-nu-j-2} \\
				% &\hspace{-2mm}
				% F_a \cdot p \le c_a + u^\sfA(p, a^\star) - \delta
				% \,\, \forall a \in \{a_i\}_{i\ge j} \label{eq:lp-nu-j-3}
				% \end{align}
			% \end{subequations}
		\label{ln:inner-for-end}
		\EndFor
		\EndFor
		\State {\bf return} $p^\star$
	\end{algorithmic}
\end{algorithm}
%
%\bac{ABUSE OF NOTATION IN THE ABOVE LP $a_0$}
% \albo{Maybe we should write that $\{ a_1, a_0 \}$ and $\{ a_n, a_{n+1} \}$ are to be considered empty set?}
% \jiarui{What about writing $a \in \{a_\ell : 1\le \ell \le j-1\}$ in \Cref{eq:lp-nu-j-2,eq:lp-nu-j-3}}
% \albo{Better ;)}

\subsection{Putting All Together}

We summarize our results in this section into \Cref{alg:robust_poly} and the following main theorem.

\begin{theorem}
	\Cref{alg:robust_poly} computes an optimal $\delta$-robust contract in polynomial time.
\end{theorem}

\begin{proof}[Proof Sketch]
	The polynomial runtime of \Cref{alg:robust_poly} is obvious as it enumerates $O(n^3)$ value combinations and solves an LP for each of them.
	To see the correctness of the algorithm, note that the inner for-loop of \Cref{alg:robust_poly} effectively solves Problem~\eqref{eq:opt}, for the pair $(a^\star, a^\delta)$ enumerated in the outer for-loop.
	Now, if $a^\star$ and $a^\delta$ happen to be the agent's exact and $\delta$-best responses, respectively, under some optimal contract, then according to \Cref{lmm:Psi-opt}, the outer loop produces an optimal contract $p'$, with $\Psi(p') \ge \Psi(p)$ for all $p \in \mathbb{R}_+^m$. By comparing the $\Psi$ values, the algorithm identifies and outputs such an optimal contract.
\end{proof}

%\clearpage
\section*{Conclusion}
This paper aims to enhance our understanding of the computational complexity of computing various Shapley value variants. We found that for various ML models --- including decision trees, regression tree ensembles, weighted automata, and linear regression --- both local and global interventional and baseline SHAP can be computed in polynomial time under HMM modeled distributions. This extends popular algorithms, such as TreeSHAP, beyond their empirical distributional scope. We also establish strict complexity gaps between the various SHAP variants (baseline, interventional, and conditional) and prove the intractability of computing SHAP for tree ensembles and neural networks in simplified scenarios. Overall, we present SHAP as a versatile framework whose complexity depends on four key factors: \begin{inparaenum}[(i)] \item model type, \item SHAP variant, \item distribution modeling approach, \item and local vs. global explanations\end{inparaenum}. We believe this perspective provides deeper insight into the computational complexity of SHAP, paving the way for future work.




%We believe that our framework provides a more intricate understanding of SHAP computation complexity across different models, distributions, and variants, paving the way for further research.

Our work opens promising directions for future research. First, expanding our computational analysis to other SHAP-related metrics, such as asymmetric SHAP~\citep{frye20} and SAGE~\citep{covert2020understanding}, would be valuable. Additionally, we aim to explore more expressive distribution classes and relaxed assumptions beyond those in Section \ref{sec:tractable} while maintaining tractable SHAP computation. Finally, when exact computation is intractable (Section \ref{sec:intractable}), investigating the approximability of SHAP metrics through approximation and parameterized complexity theory~\citep{downey2012parameterized} is an important direction.

%Our work opens several promising avenues for future research on the computational properties of explainable AI methods, with a particular focus on SHAP. First, it would be interesting to broaden the computational analysis conducted in this work to include other popular SHAP-related metrics in the literature, such as asymmetric SHAP \cite{frye20} and SAGE \cite{covert2020understanding}. Also, in the future, we aim to explore more expressive distribution classes and relaxed distributional assumptions—extending beyond those examined in Section \ref{sec:tractable} —that still yield tractable SHAP computation. Finally, when exact computation proves intractable (Section \ref{sec:intractable}), it is worthwhile to theoretically investigate the question of the approximability of computing the SHAP metrics across various configurations, through the lens of approximation and parametrized complexity theory \cite{arora2009computational}.

%This paper aims to deepen our understanding of the computational complexity involved in obtaining different Shapley value variants. We found that for a variety of ML models, including decision trees, tree ensembles for regression, weighted automata, and linear regression models — computing both local and global interventional and baseline SHAP can be done in polynomial time when distributions are modeled by HMMs. This extends the distributional scope of popular algorithms like TreeSHAP, which is limited to empirical distributions. Additionally, we demonstrate a strict complexity gap between SHAP variants, showing that interventional and baseline SHAP can be strictly easier to compute than conditional SHAP. Despite these positive results, we uncovered intractability for various SHAP variants in neural networks and tree ensembles. Finally, we provided generalized complexity relations across SHAP variants. We believe that our framework offers a deeper understanding of the complexity involved in computing SHAP across various variants, models, distributions, as well as in both local and global computations, laying the groundwork for future research.


\newpage
\bibliography{biblio.bib}

%\acks{The author would like to thank Nadav Merlis and Hugo Richard for their thoughtful comments on the manuscript, as well as Austin J. Stromme for sharing his insights in statistical optimal transport.}

%%%%%%%%%%%%%%%%%%%%%%%%%%%%%%%%%%%%%%%%%%%%%%%%%%%%%%%%%%%%%%%%%%%%%%%%%%%%%%%
%%%%%%%%%%%%%%%%%%%%%%%%%%%%%%%%%%%%%%%%%%%%%%%%%%%%%%%%%%%%%%%%%%%%%%%%%%%%%%%
% APPENDIX
%%%%%%%%%%%%%%%%%%%%%%%%%%%%%%%%%%%%%%%%%%%%%%%%%%%%%%%%%%%%%%%%%%%%%%%%%%%%%%%
%%%%%%%%%%%%%%%%%%%%%%%%%%%%%%%%%%%%%%%%%%%%%%%%%%%%%%%%%%%%%%%%%%%%%%%%%%%%%%%

\newpage

\appendix
\crefalias{section}{appendix} % uncomment if you are using cleveref
\crefalias{subsection}{subappendix} 
\part*{Appendices}
\section{Preliminaries}\label{app: intro}

\subsection{Organisation of Appendices}\label{subsec: organisation of appendices}

The following appendices are organised thematically and are mostly independent completions of various parts of the text. \Cref{subsec: notational precisions} contains notations and clarifications that are shared across them. 

\Cref{app: fourier} provides a rigourous treatment of necessary  Fourier analysis notions, which allow for a rigourous outlining of the schema detailed in \cref{subsec: measure valued actions}. 

\Cref{app: technical details,app: regret bounds,app: computation} contains the majority of the technical contributions of this work, including the major lemmata used in the proofs of the main text. \Cref{app: technical details} is dedicated to the details of the constructions in \cref{sec: Preliminaries}, while \cref{app: regret bounds} focuses on the general regret proofs of \cref{sec: regret of learning}, specifically the proofs of \cref{thm: regret Kantorovich,thm: entropic regret}. Finally \cref{app: computation} is dedicated to the details of \cref{subsec: computability} on specific regularity dependent regret.
Some miscellaneous minor results, or reproductions of results from prior works are collected in \cref{app: lemmas}.

The remaining appendices contain complements to the text and discussion of topics not mentioned therein for the sake of brevity. \Cref{app: open problems} contains more detailed discussions of the open problems mentioned in \cref{sec: directions}. \Cref{app: biblio} contains bibliographical notes on statistical optimal transport which readers unfamiliar with the field might find of interest to understand the context of the paper. It is a complement to \cref{sec:RWCC}.


\subsection{Notational precisions}\label{subsec: notational precisions}

Throughout the text, for a reference measure $\varrho$, let $L^p(\state,\Kb ;\varrho)$, $p\in[1,\infty]$ and $\Kb\in\{\Rb,\Cb\}$, denote the space of functions $f:\state\to\Kb$ that are $p$-integrable. When $\state$, $\Kb$, or $\varrho$ are clear from context we will drop them for brevity; by default $\Kb=\Cb$. We allow complex functions ($\Kb=\Cb$) to deal with the Fourier transforms, but this has no noticeable effect as it does not impact the Hilbertian structure of the space $L^2(\Rb^d,\Kb;\varrho)$. 

In the following, let $\langle \cdot\vert\cdot\rangle_{L^2(\Rb^d,\varrho)}$ denote the inner product on $L^2(\Rb^d,\Kb;\varrho)$, $\langle\cdot\vert\cdot\rangle_{\ell_2(\Rb^d)}$ the one on $\ell^2(\Rb,\Kb)$ (the space of square integrable real sequences) with $\norm{\cdot}_{\ell_2(\Rb^d)}$ denoting its associated norm. On $\Rb^d$, $\langle\cdot\vert\cdot\rangle_{2}$ denotes the inner product, $\norm{\cdot}_2$ the Euclidean norm. As before, let  $\langle\cdot\vert\cdot\rangle$ denote the duality pairing between $\measures(\Rb^d)$ (the space of finite Radon measures) and $\Cc_0(\Rb^d)$ (the space functions vanishing at infinity). The operator norm of a linear operator (in finite or infinite dimension) $A$ is denoted by $\norm{A}_{\op}$.

Throughout, all probabilistic statements are understood as holding in the filtered probability space $(\Omega,\Fc_\infty,\Fb,\Pb)$, in which $\Fb:={(\Fc_t)}_{t\in\Nb}$ is the natural filtration of ${(\xi_t)}_{t\in\Nb}$, and $\Fc_\infty=\lim_{t\to\infty}\Fc_t$.

For two measures $(\gamma,\rho)\in\measures(\Rb^d)$, $\gamma\ll\rho$ denotes that $\gamma$ is absolutely continuous with respect to $\rho$, in which case we use ${\de \gamma}/{\de \rho}$ to denote the Radon-Nikodym derivative (a.k.a.\ the density) of $\gamma$ with respect to $\rho$.
\section{Elements of Fourier Analysis}\label{app: fourier}

\subsection{Formal definitions}\label{subsec: fourier defs}

To define the Fourier transform on $L^2(\Rb^d;\varrho)$, we will extend it from a dense subspace (see \cref{def: schwartz space}) of $L^2(\Rb^d;\varrho)$ to the whole space. This technical construction arises as a consequence of the fact that $L^2(\Rb^d;\varrho)\not\subset L^1(\Rb^d;\varrho)$, meaning the right-hand side of~\eqref{eq: fourier transform} may not be defined and $\fourier$ is ill-posed on $L^2(\Rb^d;\varrho)$, despite the fact that~\eqref{eq: fourier transform} is well-posed for $f\in L^1(\Rb^d;\varrho)$. The following is summarised from~\cite[Ch.5--6]{constantin_fourier_2016}, refer therein for a more detailed treatment or, e.g.,\ to \citep{folland_fourier_1992}. 

\begin{definition}\label{def: schwartz space}
    The Schwartz space $\Sc(\Rb^d)$ is defined as 
    \[ 
        \left\{\phi\in\Cc^\infty(\Rb^d;\Cb) : \sup_{x\in\Rb^d}\abs{x^\alpha\partial_\beta\phi(x)}<+\infty \mbox{ for any } \alpha,\beta\in\Nb^d\right\}
    \]
    in which $\alpha,\beta\in\Nb^d$ are multi-indices so that $x^{\alpha}:={(x_i^{\alpha_i})}_{i=1}^d$, and $\partial_\beta:=\partial_{x_1}^{\beta_1}\cdots\partial_{x_d}^{\beta_d}$.
\end{definition}

Note that $\Sc(\Rb^d)$ is a dense subspace of $L^2(\Rb^d;\varrho)$ and $L^1(\Rb^d;\varrho)$ as it contains $\Cc^\infty_c(\Rb^d;\Cb)$ the space of infinitely-differentiable compactly-supported (a.k.a.\ test) functions, which is dense in both $L^2(\Rb^d;\varrho)$ and  $L^1(\Rb^d;\varrho)$.

\begin{theorem}[{\cite[Thm.~6.1]{constantin_fourier_2016}}]\label{thm: constantin def fourier on Schwartz}
    Consider the Fourier transform operator  $\fourier$ on the Schwartz space, with 
    \begin{align}
        \fourier: \phi\in\Sc(\Rb^d)\mapsto\int \phi(x)e^{-2\pi i\langle x\vert \cdot\rangle_2}\de\varrho(x)\,.\label{eq: fourier transform}
    \end{align}
    This operator maps $\Sc(\Rb^d)$ maps onto itself and is an isometric bijection. Moreover, 
    \begin{align}
        \fourier^{-1}=\fourier\reflection\,,\label{eq: inverse fourier (formal operator notation)}
    \end{align}    
    in which $\reflection:\phi\in\Sc(\Rb^d)\mapsto \phi(-\cdot)\in\Sc(\Rb^d)$ is the \emph{reflection} operator.
\end{theorem}

\begin{theorem}[{\cite[Thm.~6.4]{constantin_fourier_2016}}]\label{thm: constantin fourier extension}
    The fourier transform $\fourier$ can be extended to a unitary operator on $L^2(\Rb^d;\varrho)$ and~\eqref{eq: inverse fourier (formal operator notation)} holds on $L^2(\Rb^d;\varrho)$ for this extension.
\end{theorem}

The formal inversion property~\eqref{eq: inverse fourier (formal operator notation)} is easily shown to recover the classical inversion formula 
\begin{align}
    f(x)=\int \fourier f(\xi)e^{2\pi i\langle x\vert \xi\rangle}\de\varrho(\xi) \mbox{ for $\varrho$-a.e. }x\in\state\, \label{eq: inverse fourier transform}
\end{align}
as soon as $f\in L^1(\Rb^d;\varrho)\cap L^2(\Rb^d;\varrho)$. In our case $\varrho$ is a finite measure so $L^2(\Rb^d;\varrho)\subset L^1(\Rb^d;\varrho)$ and the inversion formula always holds. If $\varrho$ is only $\sigma$-finite (e.g.\ the Lebesgue measure), one must take slightly higher care. Namely 
the difference between~\eqref{eq: inverse fourier (formal operator notation)} and~\eqref{eq: inverse fourier transform} is whether the integral in~\eqref{eq: inverse fourier transform} is well defined for $f\in L^2(\Rb^d;\varrho)$, which is not guaranteed. 

This technicality reflects the limits used in the definition of the extension which are hidden by the abstract statement of \cref{thm: constantin fourier extension}. Nevertheless, since the Schwartz space $\Sc(\Rb^d)$ is dense in both $L^1(\Rb^d;\varrho)$ and $L^2(\Rb^d;\varrho)$, we can always take an arbitrarily close function in $\Sc(\Rb^d)$ and invert that, the result will remain arbitrarily close in $L^2(\Rb^d;\varrho)$.


The Schwartzian framework turns out to be a robust one for Fourier analysis more generally, and we can also use to extend $\fourier$ beyond $L^2(\Rb^d;\varrho)$. In particular, it can be used to unify the definitions we gave for the Fourier transform of a function and a measure, refer to~\cite[\S~6.1.2]{constantin_fourier_2016} for more details. Precisely, one extends to the topological dual of $\Sc(\Rb^d)$ (the space of tempered distributions $\Sc'(\Rb^d)$), which includes $\measures(\Rb^d)$ and $L^2(\Rb^d;\varrho)$ as sub-spaces.


A fundamental consequence of the various formulations of the  Fourier transform is that measures whose transforms are in $L^2(\Rb^d;\varrho)$ are exactly those which have an $L^2(\Rb^d;\varrho)$ density with respect to $\varrho$. We will denote the density of a measure $\mu$ with respect to $\varrho$ using the Radon-Nikodym notation $\de\mu/\de\varrho$, even when this tempered distribution can be identified with a function.

\begin{lemma}\label{lemma: fundamental facts about fourier transform of measure}
    Let $\gamma\in\measures(\state)$ be a finite Radon measure, if it has density with respect to $\varrho$ and $\de\gamma/\de\varrho\in L^2(\Rb^d;\varrho)$, then 
    \[
        \fourier\gamma = \fourier \frac{\de\gamma}{\de\rho}\in L^2(\Rb^d;\varrho)\,.
    \]
    Conversely, if $\fourier\gamma\in L^2(\Rb^d;\varrho)$, then $\gamma$ has a density with respect to $\varrho$ and $\de\gamma
/\de\varrho\in L^2(\Rb^d;\varrho)$.
\end{lemma}
\begin{proof}
    The first part is a direct consequence of the definitions of the Fourier transforms of a measure and an $L^2(\Rb^d;\varrho)$ function. For the converse, the fact that $\fourier\gamma
\in L^2(\Rb^d;\varrho)$ implies $\gamma
 \ll \varrho$ involves some technical minutiae due to the different topologies $\measures(\state)$ can be equipped with, which we won't reproduce for conciseness, refer to e.g.\ \cite[Lemma~1.1]{fournier_absolute_2010}. That the density is then in $L^2(\varrho)$ is a simple consequence of Plancherel's theorem:
    \[
        \norm{\frac{\de\gamma}{\de\rho}}_{L^2(\Rb^d;\rho)} =\int_{\Rb^d}\abs{F\gamma(\xi)}^2\de\rho(\xi) = \norm{F\gamma}_{L^2(\Rb^d;\rho)}\,.%\qedhere
    \]
\end{proof}



\subsection{Technical details of Section {\ref{subsec: measure valued actions}}}



Let $\Cc_0(\Rb^d,\Kb)$ denote the space of continuous functions from $\Rb^d$ to $\Kb\in\{\Rb;\Cb\}$, $\measures(\Rb^d)$ denote the space of finite Borel measures over $\Rb^d$, and let us define the Fourier operator on this space by using the same notation, i.e.\ $\fourier: \gamma\in\measures(\Rb^d)\mapsto\fourier\gamma\in\Cc_0(\Rb^d;\Cb)$ with
\begin{align}
    \fourier\gamma: \xi\in\Rb^d \mapsto \int e^{-2\pi i\langle x\vert \xi\rangle_2}\de\gamma(x)\,
    .\label{eq: fourier transform of measure}
\end{align}
Note that we will eschew the standard notations $\hat f$ and $\hat\gamma$ in favour of $\fourier f$ and $\fourier\gamma$ to avoid confusion with the least-squares estimator, which we will denote using its standard hat.


% 

The Riesz-Markov theorem shows that $(\measures^*(\Rb^d),\norm{\cdot}_\infty)$, the space of finite signed Borel measures on $\Rb^d$ (endowed with the total variation norm $\norm{\cdot}_\infty$), is the topological dual of $(\Cc_0(\Rb^d),\norm{\cdot}_\infty)$, the space of continuous functions which vanish at infinity (endowed with the supremum norm $\norm{\cdot}_\infty$), refer e.g.\ to~\cite[p.~242]{constantin_fourier_2016}. This duality is characterised by the pairing
\[
    \langle \cdot\vert\cdot\rangle: (f,\gamma
)\in\Cc_0(\Rb^d)\x\measures^*(\Rb^d)\mapsto \int f \de \gamma
 \in \Rb\,.
\]
This pairing applies in particular to all functions $f\in\Cc(\state;\Rb)$ if $\state$ is compact and to all positive finite Borel measures $\gamma\in\measures^+(\state)$, and we will use the pairing notation in this case too. In general we will use the notation for arbitrary functions, understood that it will be well defined, see also \cref{remark: assumption L2 case}. In particular:
\[
\kant(\mu,\nu,c) = \inf_{\pi\in\Pi(\mu,\nu)}\langle c\vert \pi\rangle\,.
\] 

\begin{lemma}\label{lemma: finiteness of IP}
    For any finite Borel measure $\rho\in\measures(\Rb^d)$, any $\gamma\in\measures(\Rb^d)$ finite and with $\de\mu/\de\rho\in L^2(\Rb^d;\rho)$, and any $f\in L^2(\Rb^d;\rho)\cap L^1(\Rb^d;\rho)$, we have
    \[
        \langle f\vert \gamma\rangle = \langle \fourier\reflection f\vert \fourier\gamma\rangle_{L^2(\Rb^d;\rho)}\,
    \]
    and 
    \[
        \abs{\langle f\vert \gamma\rangle}\le \norm{f}_{L^2(\Rb^d;\rho)}\abs{\rho(\Rb^d)}\abs{\gamma(\Rb^d)}\,.
    \]
\end{lemma}

\begin{proof}
    By~\eqref{eq: inverse fourier transform}, 
    \begin{align}
        \langle f\vert\gamma\rangle:=\int f\de \gamma &=\int\int \fourier f(\xi)e^{2\pi i\langle x\vert \xi\rangle}\de\rho(\xi)\de\gamma(x)\,.\label{eq: fourier double integral}
    \end{align}
Let $\varphi:(x,\xi)\mapsto e^{2\pi i \langle x\vert\xi\rangle}$. Using~\eqref{eq: fourier double integral}, since by the Cauchy-Schwartz inequality
\begin{align}
    \abs{\langle f\vert\gamma\rangle} &\le \norm{Ff}_{L^2(\Rb^d\x\Rb^d;\gamma\tensor\rho)}\norm{1}_{L^2(\Rb^d\x\Rb^d;\gamma\tensor\rho)}\notag\\
    &= \norm{Ff}_{L^2(\Rb^d;\rho)}\gamma{(\Rb^d)}^2\rho(\Rb^d)<+\infty\label{eq: bound of IP in fourier space}\,,
\end{align}
the integrand in~\eqref{eq: fourier double integral} is $\gamma\tensor\rho$-integrable, and thus we can apply the Fubini-Lebesgue theorem to obtain
\begin{align*}
    \langle f\vert\gamma\rangle&= \int \fourier f(\xi) e^{2\pi i\langle x\vert \xi\rangle}\de[\gamma\tensor\rho](\xi,x)=\langle Ff\vert \varphi\rangle_{L^2(\Rb^d\times\Rb^d;\gamma\tensor\rho)}\,.
\end{align*}
Furthermore,
    \begin{align*}
        \langle f\vert\gamma\rangle&= \int \fourier f(\xi)\int e^{2\pi i\langle x\vert \xi\rangle}\de\gamma(x)\de\rho(\xi)\\
        &= \langle \fourier\reflection f\vert \fourier \gamma\rangle_{L^2(\Rb^d;\rho)}.
    \end{align*}
By~\eqref{eq: bound of IP in fourier space}, we have once more:
    \[
        \abs{\langle f\vert\gamma\rangle}=\abs{\langle \fourier\reflection f\vert \fourier \gamma\rangle_{L^2(\Rb^d;\rho)}} \le \norm{Ff}_{L^2(\Rb^d;\rho)}\gamma{(\Rb^d)}^2\rho(\Rb^d)\,. %\qedhere
    \]
\end{proof}

The benefit of \cref{lemma: finiteness of IP} may not be immediately apparent, but it is revealed when one notices that the $L^2(\Rb^d;\rho)$ inner products and norms considered on the right hand side depend only on the measure $\rho$ and not on $\gamma$. Thus, we are able to assume only integrability of $c^*$ only with respect to our reference measure $\varrho$ (recall~\eqref{eq: def entropy}) and still manipulate the duality product $\langle c^*\vert \gamma\rangle$ for any $\gamma$. In particular, by taking $\varrho=\mu\tensor\nu$ given marginals $\mu$ and $\nu$ and playing $\pi_t$ such that $\entf(c^*,\pi_t)<+\infty$ (recall~\eqref{eq: entropic OT def}) we can reduce $\langle c^*\vert \pi_t\rangle$ to a $L^2(\Rb^d;\varrho)$ inner product, moving our problem to a Hilbert space.

\begin{remark}\label{remark: assumption L2 case}
    \Cref{lemma: finiteness of IP} opens the subject of discussing \cref{asmp: L2 case}. Let us remark that if $S:=\supp(\mu\tensor\nu)$ is compact, continuity of $c^*$ on the closure of $S$ is sufficient to obtain these results. Similarly, if $c^*$ is bounded. However, \cref{asmp: L2 case} allows for many more functions, for instance it allows $c^*:(x,y)=\norm{x-y}^2_2$ if $(\mu,\nu)\in\Ps_2(\Rb^d)$, where $\Ps_2(\Rb^d)$ denotes measures with a finite second moment. This is of value as it covers the Wasserstein distances which are of broad interest. In general, one can develop finer assumptions based on $(\mu,\nu)$ even if $\varrho$ is not finite, but we do not detail this for brevity.
\end{remark}




\section{Technical contributions in Bandit Theory}\label{app: technical details} %might actually become redundant and end up in appendices

\subsection{Confidence sets and Regularised least-squares}\label{subsec: conf sets and RLS}

%Notice that, since $\fourier$ is a linear isometry on $L^2(\Rb^d,\Cb;\varrho)$ (\cref{thm: constantin fourier extension}), $\ffset:=\fourier\fset$ is also convex.

Recall that $R_t:=\langle c^*\vert \pi_t\rangle +\xi_i=\langle \fourier\reflection c^*\vert \fourier \pi_t\rangle_{L^2(\Rb^d;\varrho)} + \xi_t$ (by \cref{lemma: finiteness of IP}), in which by \cref{asmp: estimate + subG} we have ${(\xi_i)}_{i\in\Nb}$ a conditionally $\sigma$-sub-Gaussian sequence. Let $a_t:=\fourier\pi_t\in L^2(\Rb^d;\varrho)$ for $t\in\Nb$, and $\vec{a}_t:={(a_i)}_{i=1}^t$ and $\vec{R}_t:={(R_i)}_{i=1}^t$.

Let us begin by defining the regularised least-squares estimator of $c^*$. 
Let $J_\cdot[\cdot]:\Nb\x\ffset\to\Rb$ be the (random) functional defined by 
\[
    (t,f)\mapsto J_t[f]:= \sum_{s=1}^t\norm{R_s - \langle f\vert a_s\rangle_{L^2(\Rb^d;\varrho)}}^2_2 \,.
\]
Consider $\Lambda:L^2(\Rb^d;\varrho)\to\Rb$, a strongly convex and continuously Fréchet-differentiable functional whose Fréchet derivative, denoted $\De \Lambda$, satisfies
\begin{align}
    \frac1{M_\Lambda}\norm{f}_{L^2(\Rb^d;\varrho)}\le \De^2\Lambda[f] \le M_\Lambda \norm{f}_{L^2(\Rb^d;\varrho)} \quad\mbox{for any}\quad f\in L^2(\Rb^d;\varrho)\label{eq: Fréchet derivative of Lambda}
\end{align} 
for some $M_\Lambda>0$, e.g.\ $\Lambda=\frac12\norm{\cdot}_{L^2(\Rb^d;\varrho)}^2$ with $M_\Lambda=1$. Let us recall that the Fréchet derivative of a strongly convex Fréchet-differentiable functional is a (strongly) positive-definite operator denoted $\De \Lambda$.
It is clear that $J_t+\lambda\Lambda$ is a strongly convex functional for any $\lambda>0$ and $t\in\Nb^*$ as $\ffset$ is convex. Therefore, we can define the $\Lambda$-regularised least-squares estimator of $c^*$ to be
\[
    \hat f_\lambda:=\argmin_{f\in L^2(\Rb^d;\varrho)} J_t[f]+ \lambda\Lambda[f]\,.
\]
\begin{proposition}\label{prop: least squares estimator}
    Assume \cref{asmp: L2 case}, then for any $\lambda>0$, and $t\in\Nb^*$, we have 
    \begin{align}
        \hat f_t^\lambda = {(M_t^*M_t+\lambda\De \Lambda)}^{-1} M^*_t \vec{R}_t\,,\label{eq: reg least squares estimator}
    \end{align}
    in which, for every $t\in\Nb^*$, $M_t:L^2(\Rb^d;\varrho)\to \Rb^{t}$ is the linear a.s.\ bounded operator defined by 
    \begin{align}
        M_t: f\in L^2(\Rb^d;\varrho) \mapsto {(\langle f\vert a_t \rangle_{L^2(\Rb^d;\varrho)})}_{i=1}^t\in\Rb^t  \,,
    \end{align}
    and $M^*_t:\Rb^t\to L^2(\Rb^d;\varrho)$ is its adjoint, defined by
    \begin{align}
        M^*_t: v\in\Rb^t\mapsto \sum_{s=1}^t v_s a_s \in L^2(\Rb^d;\varrho) \quad \mbox{ for any } v\in\Rb_t\,.
    \end{align}
\end{proposition}

\begin{proof}
    This proof extends the standard arguments for finite-dimensional least-squares, we include it for completeness focusing on the differences owing to infinite dimensions, cf.\ e.g.\ \citep[\S~3.2]{abbasi-yadkori_online_2012}. One first computes the Fréchet derivative of $J_t$, by studying a variation $\delta f\in L^2(\Rb^d;\varrho)$ and
    \[
    J_t[f+\delta f] - J_t[f]\,.
    \]
    One sees that the Fréchet derivative of $J_t$ exists for all $t$ and is given by 
    \[
        f\mapsto \sum_{s=1}^t\langle f\vert a_s\rangle_{L^2(\Rb^d;\varrho)}a_s- R_s a_s +\lambda\De\Lambda f= (M^*_t M_t+\lambda\De\Lambda)f -M^*_t\vec{R}_t\,.
    \]
    Note that the right-hand side is easily checked by expanding the definition of $M_t$ and $M^*_t$, and in doing so one easily checks that $M^*_t$ is indeed the adjoint of $M_t$. Carrying on, by first order optimality, the normal equations are 
    \[
        (M^*_t M_t+ \lambda\De\Lambda)\hat f_\lambda = M^*_t \vec{R}_t\,.
    \]
    Since $M^*_t M_t$ is positive semi-definite and $\De\Lambda$ is positive definite,~\eqref{eq: reg least squares estimator} follows. 
\end{proof}

Let $\design_t:=M^*_t M_t$ and $\designl_t:=\design_t+\lambda\De\Lambda$ denote the design and regularised design operators at time $t\in\Nb$. Let
    \begin{align}
        \event_t(\delta):=\left\{\hspace{-1pt}\norm{\hat f_t^\lambda - \fourier c^*}_{\designl_t}\hspace{-1pt}\le \sigma\sqrt{\log\hspace{-3pt}\left(\frac{4\det(\De\Lambda+\lambda^{-1}M_t M^*_t)}{\delta^2}\right)} +{\left(\frac{\lambda}{\norm{\De\Lambda}_{\op}}\right)}^{\hspace{-2pt}\frac12}\hspace{-5pt}\norm{\fourier c^*}_{L^2(\Rb^d;\varrho)}\right\}.\label{eq: event def}
    \end{align}
for $t\in\Nb$.

\begin{lemma}[{\cite[Cor.~3.6]{abbasi-yadkori_online_2012}}]\label{lemma: probability of confsets unif in N}
    For every $\delta\in(0,1)$, $\lambda>0$, under \cref{asmp: L2 case,asmp: estimate + subG} we have 
    \[  \Pb\left(c^*\in \bigcap_{t\in\Nb}\fourier^{-1}\confset_t(\delta)\right)\ge\Pb\left(\bigcap_{t\in\Nb} \event_t(\delta/2)\right)\ge 1-\frac\delta2 \,.\]
\end{lemma}

\begin{proof}
    Recall that $\fourier$ is an isometry on $L^2(\Rb^d;\varrho)$, and so is $\fourier^{-1}$, so $\fourier^{-1}\confset_t$ is a confidence set for $c^*$ in $L^2(\Rb^d;\varrho)$, and it is a ball of identical radius $\width_t(\delta)$ centred at $\fourier^{-1} \hat f_t^\lambda$. 
    A direct combination of \cref{asmp: estimate + subG},~\eqref{eq: event def}, and~\cite[Cor.~3.6]{abbasi-yadkori_online_2012} yields 
    \[
        \Pb\left(\bigcap_{t\in\Nb} \event_t(\delta/2)\right)\ge 1-\frac\delta2\,.
    \]
    The second results follow by comparison of~\eqref{eq: event def} and~\eqref{eq: confidence set width def}.
\end{proof}



\begin{lemma}\label{lemma: bound on width term}
    Under \cref{asmp: L2 case,asmp: estimate + subG}, on the event $\{c^*\in \cap_{t\in\Nb}\fourier^{-1}\confset_t(\delta)\}$, for any $T\in\Nb$ and ${(c_t)}_{t=1}^T$ with $c_t\in\fourier^{-1}\confset_t(\delta)$ for $t\in[T]$, we have
    \[
        \sum_{t=1}^T\langle c^* - c_t\vert \tilde\pi_t\rangle \le 2\ubnorm\width_T(\delta)\sqrt{T\log\det\left(\Id + \frac{1}{2\lambda\ubnorm}M_t{(\De\Lambda)}^{-1}M_t^*\right)}
    \]
\end{lemma}
\begin{proof}
    Consider $t\ge 0$, $c_t\in\confset_t(\delta)$, and let $\varphi_t:=\langle c^* - c_t\vert \tilde\pi_i\rangle$. Recall that $a_t:=\fourier\pi_t$. By \cref{lemma: probability of confsets unif in N} and the Cauchy-Schwartz inequality, on the event $\{c^*\in \cap_{t\in\Nb}\fourier^{-1}\confset_t(\delta)\}$, we have 
    \begin{align*}
        \abs{\varphi_t} \le \width_t(\delta)\norm{a_t}_{{(\designl_t)}^{-1}}\,,
    \end{align*}
    while, by the Cauchy-Schwartz inequality, \cref{asmp: L2 case}, and using the fact that $\fourier$ is an isometry on $L^2(\Rb^d;\varrho)$, we have 
    \begin{align*}
        \abs{\varphi_t} &\le \norm{\fourier\reflection c^* - \fourier\reflection c}_{L^2(\Rb^d;\varrho)}\norm{a_t}_{L^2(\Rb^d;\varrho)}= \norm{c^*-c}_{L^2(\Rb^d;\varrho)}\pi_t(\Rb^d)\le 2\ubnorm\,.
    \end{align*}
    Combining yields
    \begin{align*}
        \abs{\varphi_t}\le \width_t(\delta)\min\{\norm{a_t}_{{(\designl_t)}^{-1}},2\ubnorm\}=2\ubnorm\width_t(\delta)\left(\frac{1}{2\ubnorm}\norm{a_t}_{{(\designl_t)}^{-1}}\wedge 1 \right) \,.
    \end{align*}
    Squaring and applying the inequality $u\le 2\log(1+u)$, which holds on $[0,1]$, to the final term, yields
    \begin{align*}
        \abs{\varphi_t}^2\le 8\ubnorm^2{\width_t(\delta)}^2\log\left(1+ \frac{1}{2\ubnorm}\norm{a_t}_{{(\designl_t)}^{-1}} \right)
    \end{align*}
    and, summing up,
    \begin{align}
        \sum_{t=1}^T \varphi_t\le \sqrt{T\sum_{t=1}^T \abs{\varphi_t}^2}\le 2\ubnorm\width_t(\delta)\sqrt{T\sum_{t=1}^T \log\left(1+ \frac{1}{2\ubnorm}\norm{a_t}_{{(\designl_t)}^{-1}}\right)} \label{eq: summation of widths in proof of the bound on width term lemma}\,.
    \end{align}
    % On the one hand, by \cref{lemma: bound on size of actions}, we have
    % \[ 
    %     \varphi^2\le 4\ubnorm^2\width_t(\delta)^2\log\left(1+\frac{1}{2\ubnorm\lambda M_\Lambda}\right)\,
    % \]
    % so that 
    % \begin{align}
    %     \sum_{t=1}^n \varphi_t\le \sqrt{n\sum_{t=1}^n \varphi_t^2}\le 2\ubnorm\width_t(\delta)\sqrt{n\sum_{t=1}^n\log\left(1+\frac{1}{2\ubnorm\lambda M_\Lambda}\right)} 
    % \end{align}
    By definition of $M_T$ and $\designl_T$, we have
    \begin{align}
        \sum_{t=1}^T \log\left(1+ \frac{1}{2\ubnorm}\norm{a_t}_{{(\design_t^\lambda)}^{-1}}\right)=\log\left(\prod_{t=1}^T \left(1+ \frac{1}{2\ubnorm}\norm{a_t}_{{(\design_t^\lambda)}^{-1}}\right)\right)\notag\\
        =\log\det\left(\Id + \frac{1}{2\lambda\ubnorm}M_T{(\De\Lambda)}^{-1}M_T^*\right)\,\label{eq: sum of log is logdet}
    \end{align}
    as wanted.
\end{proof}

% \begin{lemma}\label{lemma: bound on action sizes}
%     Under \cref{asmp: L2 case}, for any $n\in\Nb$, we have
%     \begin{align*}
%         \sum_{t=1}^n \norm{\fourier \pi_t}_{(\designl_t)^{-1}}\wedge 1 \le \min\left\{2\log\det \left(\Id + \lambda^{-1} M_t\De\Lambda^{-1} M_t^*\right), 2\log(1+(\lambda M_\Lambda)^{-\frac12})\right\}\,.
%     \end{align*}
% \end{lemma}
% \begin{proof}
%     The first term of the minimum is the well known \cite[Lemma~E.2]{abbasi-abbasi-yadkori_online_2012}, which relies on the fact that $u\le 2\log(1+u)$ on $[0,1]$ implying
%     \[ 
%         \sum_{i=1}^n \norm{\fourier \pi_i}_{(\designl_t)^{-1}}\wedge 1 \le2\log\left(1+\norm{\fourier \pi_i}_{(\designl_t)^{-1}}\right)\,.
%     \]
%     Going from this same equation in another direction, namely taking \cref{lemma: bound on size of actions} into account, we have
%     \[
%         \norm{\fourier \pi_i}_{(\designl_t)^{-1}}\le \frac1{\lambda M_\Lambda}\norm{\fourier\pi_i}_\infty=\frac1{\sqrt{\lambda M_\Lambda}}\,
%     \]
%     which readily yields the second term.
% \end{proof}



% \begin{lemma}\label{lemma: bound on size of actions}
%     Under \cref{asmp: L2 case},
%     \[
%         \norm{a_t}_{(\designl_t)^{-1}} \le \frac1{\sqrt{\lambda M_\Lambda}} \quad\mbox{ for }t\in\Nb\,.
%     \]
% \end{lemma}
% \begin{proof}
%     Let $\varrho\in\measures(\Rb^d)$ be an arbitrary finite Borel measure with $\varrho(\state)=1$. Let us first observe that $\designl_t:=M^*_tM_t+\lambda\De\Lambda$ is positive definite, with 
%     \[ 
%         \langle \designl_tf\vert f\rangle_{\varrho}= \sum_{i=1}^t\langle f\vert a_i\rangle^2_\varrho + \lambda\norm{\De\Lambda f}_{\varrho}^2\ge \lambda M_\Lambda \norm{f}_{\varrho}^2\,.
%     \]
%     Thus, 
%     \[ 
%         \inf\left\{\norm{\designl_tf}_\varrho : f\in L^2(\Rb^d,\Rb;\varrho),\, \norm{f}_\varrho=1\right\} \ge \sqrt{\lambda M_\Lambda} >0
%     \]
%     and $\norm{(\designl_t)^{-1}}_{\op,\varrho}\le (\lambda M_\Lambda)^{-1/2}$, in which $\norm{\cdot}_{\op,\varrho}$ denotes the operator norm induced by $\norm{\cdot}_\varrho$. 
%     One obtains the bound by recalling that $\norm{a_t}_\infty=\pi_t(\state)=1$ and using $\norm{a_t}_{(\designl_t)^{-1}}\le \norm{(\designl_t)^{-1}}_{\op,\varrho}\norm{a_t}_\infty\varrho(\state)$. 
% \end{proof}

\section{Regret bounds}\label{app: regret bounds}

\subsection{Entropic regret bounds}\label{subsec: entropic regret}


To facilitate the analysis and the presentation of results, recall the entropic transport functional
\[
    \entf:(c,\pi)\in L^2(\Rb^d;\varrho)\times\Pi(\mu,\nu)\mapsto \langle c\vert\pi\rangle + \ve\entropy(\pi)\,.
\]
Thus, $\ent(\mu,\nu,c,\ve)=\inf_{\pi\in\Pi(\mu,\nu)}\entf(c,\pi)$ and~\eqref{eq: entropic regret} becomes
\[
    \regret_T^{\entropy,\ve}(\pi):= \sum_{t=1}^T \entf(c^*,\pi_t)+\xi_t - \ent(\mu,\nu,c^*,\ve)\,.
\]

\ThmEntropicRegret*


\begin{proof}
    Recall the we identify $\Ac$ with the $\Fb$-adapted process $\bm{\pi}:={(\pi_t)}_{t\in\Nb}\subset\Pi(\mu,\nu)$ of transport plans played. The instantaneous regret of the algorithm at time $t\in\Nb$ is defined as
    \[
        r_t:= \entf(c^*,\pi_t)+\xi_t - \ent(\mu,\nu,c^*,\ve)\,.
    \]
    It is clear that $\regret_T^\entropy(\Ac)=\sum_{t=1}^T r_t$.
    Before pursuing further, let us apply \cref{lemma: sub-gaussian sum} to the sequence ${(\xi_t)}_{t\in\Nb}$, in view of \cref{asmp: estimate + subG}, to obtain that for any $\delta>0$ we have
    \begin{align}
        \Pb\left(\sum_{i=1}^T \xi_i \le \sigma\sqrt{2T\log\left(\frac2\delta\right)}\right)\ge 1-\frac\delta2\,.\label{eq: sub-gaussian sum bound regret}
    \end{align}
    Now, let $\bar r_t :=  \entf(c^*,\pi_t)- \ent(\mu,\nu,c^*,\ve)$ as we continue the decomposition. By definition of the algorithm, let 
    \begin{align} 
        (\tilde\pi_t,\tilde c_t)\in\argmin_{\substack{\pi\in\Pi(\mu,\nu)\\c\in\confset_t(\delta)}} \entf(c,\pi)\notag\,,
    \end{align}
    where $\confset_t(\delta)$ is the confidence set defined in~\eqref{eq: confidence set}.
    
    Let us place ourselves on the event $\cap_{t=1}^\infty\left\{c^*\in\fourier^{-1}\confset_t(\delta)\right\}$, an event which happens with probability at least $1-\delta/2$ by \cref{lemma: probability of confsets unif in N}. By optimism, we have 
    \[
        \bar r_t \le  \entf(c^*,\pi_t)- \ent(\mu,\nu,\tilde c_t,\ve)
    \]
    The instant regret can be decomposed as
    \begin{align*}
        \bar r_t &= \entf(c^*,\pi_t) -\entf(\tilde c_t,\pi_t) + \entf(\tilde c_t,\pi_t) -\ent(\mu,\nu,\tilde c_t,\ve) 
    \end{align*}
    The first term $\entf(c^*,\pi_t) -\entf(\tilde c_i,\pi_t)=\langle c^*-\tilde c_t\vert \pi_t\rangle$ can be bounded by \cref{lemma: bound on width term}, while the second term is $0$ by definition of \cref{alg: alg shared}. The proof is completed by taking a union bound over $\cap_{t=1}^\infty\left\{c^*\in\fourier^{-1}\confset_t(\delta)\right\}$ and the event of~\eqref{eq: sub-gaussian sum bound regret}.
\end{proof}


\subsection{Kantorovich regret bounds}\label{subsec: kantorovich regret}

Let us begin by giving the requisite results on approximation of the Kantorovich problem by the entropic one. 
Let $d_{\entropy}(\gamma)$ (for $\gamma\in\{\mu,\nu\}$) denote the \textit{upper Renyi dimension} of $\gamma$, defined by 
\[ d_\entropy(\gamma):=\limsup_{\epsilon\downarrow0}\frac{H_\varepsilon(\gamma)}{\log(\varepsilon^{-1})}\,\]
in which $H_\varepsilon(\gamma)$ is the infimum (over all countable partitions of $\supp(\gamma)$ into Borel subsets of diameter at most $\varepsilon$) of the discrete entropy of $\gamma$ with respect to the partition, see~\cite{carlier_convergence_2023}.

\begin{lemma}[{\cite[Prop.~3.1]{carlier_convergence_2023}}]\label{lemma carlier pegon lipschitz UB}
    If $c^*$ is $L$-Lipschitz on $\supp(\mu)\times\supp(\nu)$, then
    \[\ent(\mu,\nu,c^*,\ve) - \kant(\mu,\nu,c^*)\le \varepsilon\left((d_{\entropy}(\mu)\wedge d_{\entropy}(\nu))\log(\varepsilon^{-1}) + L \right) \,\]
    as $\varepsilon\downarrow0$.
\end{lemma}

Extensions of this result exist for more general absolute continuity conditions, see~\cite[Rem.~3.4]{carlier_convergence_2023}. This constant is sharp, but tighter bounds may be obtained under stronger regularity assumptions, see e.g.~\cite[Prop.~3.7]{carlier_convergence_2023}. In view of \cref{lemma carlier pegon lipschitz UB}, we can define $\kappa:= (d_{\entropy}(\mu)\wedge d_{\entropy}(\nu)) + L$. In spite of its apparent complexity, upper Renyi dimension is a relatively well behaved object, and can be bounded in many common situations, see the following remarks.

\begin{remark}[{\cite[Prop.~3.2]{carlier_convergence_2023}}]
    If $\gamma$ is a measure on $\Rb^d$ satisfying
    \[ \int 0\vee\log(\norm{x}_2)\de\gamma(x)<+\infty\]
    then $d_{\entropy}(\gamma)\le d$.
\end{remark}

\begin{remark}[{\cite[Rem.~3.5]{carlier_convergence_2023}}]
    If $\gamma$ is finitely supported, then $d_\entropy(\gamma)=0$. 
\end{remark}



\ThmKantorovichRegret*



\begin{proof}
    The proof follows the same lines as the proof of \cref{thm: entropic regret}. Again, we identify $\Bc$ with the transport plans $\bm{\pi}:={(\pi_t)}_{t\in\Nb}\subset\Pi(\mu,\nu)$ it plays. The instantaneous regret is different due to the change of objective, it is given by 
    \[
        r_t:= R_t - \kant(\mu,\nu,c^*) = \langle c^*\vert \pi_t-\pi^*\rangle +\xi_t\,.
    \]
    As before, apply \cref{lemma: sub-gaussian sum} to the sequence ${(\xi_i)}_{i\in\Nb}$, and pass to $\bar r_t := \langle c^*\vert \pi_t-\pi^*\rangle$, which can be decomposed as
    \begin{align}
        \bar r_t &= \langle c^*\vert \pi_t\rangle - \langle c^*\vert \pi^*\rangle\notag\\
        &= \langle c^*\vert \pi_t\rangle  - \ent(\mu,\nu,c^*,\ve)+\ent(\mu,\nu,c^*,\ve) - \kant(\mu,\nu,c^*)\notag\\
        &\le \langle c^*\vert \pi_t\rangle -\ent(\mu,\nu,c^*,\ve) + \ve (d_\entropy(\mu)\wedge d_\entropy(\nu))\log(\ve^{-1}) + L\ve\notag
    \end{align}
    for any $\ve>0$, by \cref{lemma carlier pegon lipschitz UB}. In particular, for $\ve=\ve_t$ as used by \cref{alg: alg shared}, we have
    \begin{align}
        \sum_{t=1}^T \ve_t(d_\entropy(\mu)\wedge d_\entropy(\nu))\log(\ve_t^{-1}) + L\ve_t \le \frac{\kappa\alpha}{1-\alpha}\left(T^{1-\alpha}\log(T) + \frac{\alpha}{2^\alpha}\log(6)\right)\,.\label{eq: bound on sum epsilons}
    \end{align}
    by \cref{lemma: sum of terms from pegon bound}.
    Let us recall that optimism implies that 
    \begin{align} 
        (\tilde\pi_t,\tilde c_t)\in\argmin_{\substack{\pi\in\Pi(\mu,\nu)\\c\in\confset_t(\delta)}} \Psi^{\ve_t}_{\mu,\nu}(c,\pi)\notag\,,
    \end{align}
    for $\ve_t>0$ as used by \cref{alg: alg shared}, so that 
    \[
    \ent(\mu,\nu,\tilde c_t,\ve_t)\le \ent(\mu,\nu,c^*,\ve_t) \mbox{ on } \event_t(\delta)\,.
    \]

    Let us place ourselves on the event $\cap_{t=1}^\infty\left\{c^*\in\fourier^{-1}\confset_t(\delta)\right\}\supset \cap_{t=1}^\infty \Ec_t(\delta)$, which happens with probability at least $1-\delta/2$ by \cref{lemma: probability of confsets unif in N}.
   On this event, we thus have 
    \begin{align*}
        \langle c^*\vert \pi_t\rangle -\ent(\mu,\nu,c^*,\ve_t) &\le \langle c^*\vert \pi_t-\tilde\pi_t\rangle + \langle c^*-\tilde c_t\vert \tilde\pi_t\rangle\,,
    \end{align*}
    since $\entropy\ge 0$.
    Applying $\pi_t=\tilde\pi_t$ we obtain the desired result, up to combining the $\bar r_t$ over $t\in\Nb$, recalling~\eqref{eq: bound on sum epsilons} and taking a union bound over the two events.
\end{proof}





\section{Controlling the infinite dimensional terms}\label{app: computation}



The parametric and RKHS estimation methodologies are highly standard in bandit theory, because they seamlessly fit into the general Hilbert Space analysis of~\cite{abbasi-yadkori_online_2012} while giving a control on the resulting regret bounds in terms of finite dimensional quantities. In Fourier analysis and fields which rely on it, such as functional regression \citep{morris_functional_2015}, it is more natural to look for approximations by decomposing $\fourier c^*$ and $a_t$ into an orthonormal basis and truncating it at some finite order. We detail this learning methodology below.

We being in \cref{subsec: Fourier basis representation} by presenting the general concept of basis decomposition as an approximation method. Then, in \cref{subsubsec: finite order} we truncate at a fixed order and derive the regret bounds for this case. Before moving on to changing the truncation order with $t\in\Nb$ in \cref{subsubsec: increasing order}, we give a brief discussion in \cref{subsubsec: parametric rate /matching} of some examples in which a finite basis is sufficient. Finally, we give a brief treatment of kernel methods in \cref{subsec: tikhonov and RKHS} for completeness.



\subsection{Intrinsic regularity and fourier basis decay}\label{subsec: Fourier basis representation}

To simplify notation, let $f^*:=\fourier c^*$. Recall the chosen orthonormal basis ${(\phi_i)}_{i\in\Nb}$ of the space $L^2(\Rb^d;\varrho)$, in which $(f^*,a_t)\in {L^2(\Rb^d;\varrho)}^2$, $t\in\Nb$, admit representations 
\[
    f^*= \sum_{i=0}^\infty \gamma_i^*\phi_i\quad\mbox{ and }\quad a_t = \sum_{i=0}^\infty \vartheta^{(t)}_{i}\phi_i\,, \quad \mbox{ for some }\quad (\gamma^*,\vartheta^{(t)})\in{\ell_2(\Rb)}^2\,.
\] 
Classical choices for ${(\phi_i)}_{i=\in\Nb}$ are wavelet systems such as the Haar or Hermitian systems, and the Fourier basis if $\supp(\mu)\x\supp(\nu)$ is bounded. The choice of a specific basis is made \textit{ad hoc} from knowledge of the structure of the problem; we present the general argument. 

By definition of ${(\phi_i)}_{i\in\Nb}$ as an orthonormal basis, we have 
\[
    \langle f\vert a_t\rangle_{L^2(\Rb^d;\varrho)}=\langle \gamma^*\vert \vartheta^{(t)}\rangle_{\ell_2(\Rb)}=\sum_{n=1}^{+\infty} \gamma_n\vartheta^{(t)}_n\,.
\]

Let $f^*\vert_n:=\sum_{i=0}^n\gamma_i^*\phi_i$ be the truncation of the basis expansion of $f$ at order $n\in\Nb$. By abuse of notation, and only when it is clear from context, we will override notation and denote $c^*\vert_n$ the result of applying the inverse fourier transform to $(f^*)\vert_n$, the basis truncation of $f^*:=\fourier c^*$.
A straightforward derivation yields the approximation bound of \cref{lemma: fourier decay bound}.

\begin{lemma}\label{lemma: fourier decay bound}
    Let ${(\phi_i)}_{i\in\Nb}$ be an orthonormal basis of $L^2(\Rb^d;\varrho)$, and let $f\in L^2(\Rb^d;\varrho)$ with $f:=\sum_{i=0}^\infty \gamma_i\phi_i$. Then, we have
    \[
    \abs{\left\langle f - f\vert_n\vert g\right\rangle }_{L^2(\Rb^d;\varrho)}\le \norm{g}_{L^2(\Rb^d;\varrho)}\sum_{k=n+1}^{+\infty}\abs{\gamma_n^*}\quad \mbox{ for every } g\in L^2(\Rb^d;\varrho).
    \]
\end{lemma}

For our purpose, $g=a_t$ is bounded by $1$ since $\norm{\fourier\pi_t}_\infty\le\pi_t(\Rb^d)=1$ and $\varrho(\Rb^d)=1$, so that the resulting approximation error is controlled entirely by the decay of the coefficients ${(\gamma_i^*)}_{i\in\Nb}$. Consequently, regret analysis can leverage \cref{lemma: fourier decay bound} to move the problem into a finite dimensional regression problem on the coefficients ${(\gamma_i^*)}_{i\in\Nb}$. We begin by setting the stage with a fixed order (i.e.\ $n$ independent of $t$) methodology. Later, we will derive regret guarantees when $n$ is allowed to grow with $t$ in order to control the approximation error. 


\subsection{Fixed order basis truncation}\label{subsubsec: finite order}

In this section, let $n\in\Nb$ be fixed. One can approximately regress $\bm{R}_t$ against $a_t$ up to order $n$ by solving the $n$-dimensional Regularised Least-Squares (RLS) problem
\begin{align}
    \hat \gamma^{n,\lambda}_t:=\argmin_{\gamma\in\Rb^n} \sum_{s=1}^t \norm{R_s - \sum_{i=1}^n\gamma_i\vartheta^{(s)}_i}_2^2 + \lambda \Lambda_n(\gamma)\,,\label{eq: RLS  basis truncation}
\end{align}
in which $\Lambda_n:\Rb^n\to[0,+\infty)$ is a strictly convex continuously Fréchet-differentiable regulariser such that its Fréchet derivative $\De\Lambda_n$ satisfies
\[
    \frac1{M_{\Lambda_n}}\Id\preceq \De\Lambda_n\preceq M_{\Lambda_n}\Id\,.
\] 

For clarity, let $\vartheta^{(s,n)}$ denote the truncation of $\vartheta^{(s)}\in\Rb^\Nb$ at order $n$, so that $\vartheta^{(s,n)}\in\Rb^{n}$ and $\vartheta^{(s,n)}_i=\vartheta^{(s)}_i$ for all $i\in[n]$.
Following the standard arguments for online linear regression (omitted for brevity, see e.g.\ \cite{abbasi-yadkori_improved_2011,abbasi-yadkori_online_2012}), one can construct the (valid, by \cref{cor: probability of confsets order n}) confidence sets
\begin{align}
    \tilde\confset_{t}^{n}(\delta):=\left\{\gamma\in\Rb^{n}: \norm{\gamma -\hat \gamma^{n,\lambda}_t}_{\tilde\design_t^{\lambda,n}}\le \tilde\width_{t,n}(\delta)\right\}\label{eq: confidence set fixed order}\,,
\end{align}
in which $\tilde\design_t^{\lambda,n}:= \lambda\De\Lambda_n + \sum_{s=1}^t \vartheta^{(s,n)}{\vartheta^{(s,n)}}^\top$ and 
\begin{align}
    \tilde\width_{t}^{n}(\delta):=\sigma\sqrt{\log\left(\frac{4\det\left(\De\Lambda_n+\lambda^{-1}\sum_{s=1}^t \vartheta^{(s,n)}{\vartheta^{(s,n)}}^\top\right)}{\delta^2}\right)} +{\left(\frac\lambda{\norm{\De\Lambda_n}_\op}\right)}^{\frac12}\ubnorm\,.
\label{eq: width fixed order}
\end{align}
Notice that $C>\norm{c^*}_{L^2(\Rb^d;\varrho)}$ implies that $C\ge \norm{\gamma^*}_{\ell_2(\Rb)}$ by definition of ${(\phi_i)}_{i\in\Nb}$, so that $\ubnorm$ is a valid upper bound on $\norm{\gamma^*}_{\ell_2(\Rb)}$. To verify the validity of the confidence sets (see \cref{cor: probability of confsets order n}), let 
    \begin{align}
        \tilde{\event_t^{n}}(\delta):=\left\{ \norm{\gamma -\hat \gamma^{n,\lambda}_t}_{\tilde\design_t^\lambda}\le \tilde\width_{t}^{n}(\delta) \right\}\quad \mbox{ for }\quad (t,n)\in\Nb^2\,.\label{eq: event def for fixed order approx}
    \end{align} 

\begin{corollary}\label{cor: probability of confsets order n}
    Under \cref{asmp: estimate + subG,asmp: L2 case}, for every $\delta>0$, $\lambda>0$, $n\in\Nb$, 
    \[
        \Pb\left(\bigcap_{t=1}^\infty \tilde{\event_t^{n}}(\delta)\right)\ge 1-\frac\delta2\,.
    \]
\end{corollary}

\begin{proof}
    Follow the proof method of \cref{lemma: probability of confsets unif in N} (or apply \cref{lemma: confidence sets with varying basis order} below).
\end{proof}


Applying this learning methodology to \cref{alg: alg shared} in place of the infinite-dimensional RLS, and with the optimistic choice of belief-action pairs
\begin{align}
    (\tilde\pi_t,\tilde \gamma_t^n)\in \argmin_{\substack{\pi\in\Pi(\mu,\nu)\\ \gamma\in \tilde\confset_{t}^{n}(\delta)}}\entf\left(\mu,\nu,\sum_{i=1}^n\gamma_{i}\phi_i,\ve\right)\label{eq: optimism for finite order}
\end{align}
yields \cref{alg: alg shared + approx} with ${(n_t)}_{t\in\Nb}={(n)}_{t\in\Nb}$ and the regret bound of \cref{cor: regret for fixed approximation order}.

\begin{algorithm}
    \caption{\namealgtwo{}\label{alg: alg shared + approx}}
    \SetKwFunction{approxpi}{ApproximateAction}
    \KwData{Confidence $\delta$, regularization level $\lambda$, entropy penalisation ${(\ve_t)}_{t\in\Nb}$, orders ${(n_t)}_{t\in\Nb}$.}
    \For{$t\in\Nb$}{
            Compute the RLS estimator $\hat \gamma^{n_t,\lambda}_t$ using~\eqref{eq: RLS  basis truncation}\;
            Construct the confidence set $\tilde\confset_{t}^{n_t}(\delta)$ using~\eqref{eq: confidence set fixed order} and~\eqref{eq: width fixed order}\;
            Optimism:  pick $ (\tilde\pi_t,\tilde\gamma^{n_t}_t)$ according to~\eqref{eq: optimism for finite order}\;
        Play $\pi_t=\tilde\pi_t$ if $t>0$, else $\pi_0=\mu\tensor\nu$; receive feedback $R_t$\;
    }
\end{algorithm}

\begin{restatable}{corollary}{RegretCorrForFixedApproximationOrder}\label{cor: regret for fixed approximation order}
    Under \cref{asmp: estimate + subG,asmp: L2 case}, for any $\delta>0$, $\lambda>0$, $T\in\Nb$, using \cref{alg: alg shared + approx} with ${(\ve_t)}_{t\in\Nb}={(\ve)}_{t\in\Nb}$ and ${(n_t)}_{t\in\Nb}={(n)}_{t\in\Nb}$ (denoted $\Ac_n$) yields
    \begin{align}
        \regret_T^{\entropy,\ve} (\Ac_n)&\le  \sigma\sqrt{2T\log\left(\frac2\delta\right)} + 2\ubnorm\sqrt{nT}\left(\log\left(\frac{M_{\Lambda_n}}\lambda + \frac{t \ubnorm^2}n\right)+\frac{n}{2(1\wedge\lambda\ubnorm)}\log M_{\Lambda_n} \right)\notag\\
        &\quad +2T \sum_{k=n+1}^{+\infty}\abs{\gamma_k^*}\,, \label{eq: entropic regret for fixed approximation order} \\
        \intertext{ while using ${(\ve_t)}_{t\in\Nb}={(\alpha t^{-\alpha})}_{t\in\Nb}$ and ${(n_t)}_{t\in\Nb}={(n)}_{t\in\Nb}$ (denoted $\Bc_n$) yields }
        \regret_T(\Bc) &\le \sigma\sqrt{2T\log\left(\frac2\delta\right)} + 2\ubnorm\sqrt{nT}\left(\log\left(\frac{M_{\Lambda_n}}\lambda + \frac{t \ubnorm^2}n\right)+\frac{n}{2(1\wedge\lambda\ubnorm)}\log M_{\Lambda_n} \right)\notag\\
        &\quad +\frac{\kappa\alpha}{1-\alpha}\left(T^{1-\alpha}\log(T) + \frac{\alpha}{2^\alpha}\log(6)\right) + 2T \sum_{k=n+1}^{+\infty}\abs{\gamma_k^*}\,.  \label{eq: Kantorovich regret for fixed approximation order}
    \end{align}
\end{restatable}

\begin{proof}
    The proof follows the usual decomposition up the following modifications which are the same for both~\eqref{eq: entropic regret for fixed approximation order} and~\eqref{eq: Kantorovich regret for fixed approximation order}. We give the modification for \cref{thm: entropic regret}, the same modifications need only be applied to \cref{thm: regret Kantorovich} to complete the proof of the second bound.  

    At the second step of the proof, let $\bar\pi^{\epsilon}$ be an $\epsilon$-minimiser of $\ent(\mu,\nu,c^*,\ve)$, for $\epsilon>0$, and decompose $\bar r_t := \entf (c^*,\pi_t)-\ent(\mu,\nu,c^*,\ve)$ as
    \begin{align}
        \bar r_t &\le  \epsilon + \entf (c^*,\pi_t) - \entf(c^*, \bar\pi^{\epsilon})\notag \\
        & \le \epsilon +\entf (c^*\vert_n,\pi_t) - \ent(\mu,\nu,c^*\vert_n,\ve) \notag \\
        &\quad +\entf (c^*,\pi_t) -\entf (c^*\vert_n,\pi_t)  + \entf(c^*\vert_n, \bar\pi^{\epsilon})- \entf(c^*, \bar\pi^{\epsilon})\notag \\% \entf(c^*\vert_n,\pi_t) - \ent(\mu,\nu,c^*,\ve)\,.
        &\le \epsilon +\entf (c^*\vert_n,\pi_t) - \ent(\mu,\nu,c^*\vert_n,\ve) + 2\sum_{k=n+1}^{+\infty}\abs{\gamma_k^*}\,,\notag
    \end{align}
    by a double application of \cref{lemma: fourier decay bound} combined with the bound $\norm{a_t}_{L^2(\Rb^2;\varrho)}\le 1$. 
    Sending $\epsilon\to0$ allows one to then continue the proof, up to replacing the events $\event_t(\delta)$ by $\tilde{\event_t^{n}}(\delta)$, and \cref{lemma: probability of confsets unif in N} by \cref{cor: probability of confsets order n}. 

    Finally, let us introduce $\tilde c_t^{\,n}:= \sum_{i=1}^n \tilde\gamma_{t,i}^{n}\phi_i$ for $t\in\Nb$, so that by \cref{lemma: bound on width term}, we can directly derive
    \begin{align}
        \sum_{t=1}^T\langle c^*\vert_n -\tilde c_t^{\,n}\vert \tilde\pi_t\rangle\le 2C\tilde\width_{T,n}(\delta)\sqrt{nT\log\det\left(\idmat+\frac{1}{2\lambda C}\sum_{t=1}^T \vartheta^{(t,n)}_t\De\Lambda_n^{-1}{\vartheta^{(t,n)}}^\top\right)}\,.\notag
    \end{align}

    To obtain the stated bounds, it remains to bound 
    \[
        \det\left(\De\Lambda_n+\lambda^{-1}\sum_{t=1}^T \vartheta^{(t,n)}{\vartheta^{(t,n)}}^\top\right) \quad\mbox{and}\quad \det\left(\idmat+\frac{1}{2\lambda C}\sum_{t=1}^T \vartheta^{(t,n)}\De\Lambda_n^{-1}{\vartheta^{(t,n)}}^\top\right)
    \]
    using \cref{lemma: bounds for the determinants in finite dimension}.
\end{proof}


\begin{lemma}\label{lemma: bounds for the determinants in finite dimension}
    Under \cref{asmp: estimate + subG,asmp: L2 case}, for $(n,t)\in\Nb^2$, we have
    \begin{align}
        \log\det\left(\De\Lambda_n+\lambda^{-1}\sum_{t=1}^T \vartheta^{(t,n)}{\vartheta^{(t,n)}}^\top\right)\le \log\left(\frac{M_{\Lambda_n}}\lambda + \frac{t \ubnorm^2}n\right)+n\log M_{\Lambda_n}\,.\label{eq: bound for the determinindant of beta in finite dimension}\\
        \log\det\left(\idmat+\frac{1}{2\lambda C}\sum_{t=1}^T \vartheta^{(t,n)}_t\De\Lambda_n^{-1}{\vartheta^{(t,n)}}^\top\right)\le \log\left(\frac{M_{\Lambda_n}}\lambda + \frac{t \ubnorm^2}n\right)+\frac{n}{2\lambda\ubnorm}\log M_{\Lambda_n}\,.
        \label{eq: bound for the determinindant of width in finite dimension}
    \end{align}
\end{lemma}

\begin{proof}
    We take the two bounds in turn. First, apply the matrix determinant lemma to obtain
    \[
        \det\left(\De\Lambda_n+\lambda^{-1}\sum_{t=1}^T \vartheta^{(t,n)}{\vartheta^{(t,n)}}^\top\right)\le \det(\De\Lambda_n)\det\left(\idmat+\lambda^{-1}\sum_{t=1}^T \vartheta^{(t,n)}\De\Lambda_n^{-1}{\vartheta^{(t,n)}}^\top\right)\,,
    \]
    which can be readily bounded as in \citep[Lemma E.3]{abbasi-yadkori_online_2012} by noticing that $\snorm{\vartheta^{(t,n)}}_2\le\norm{c^*}_{L^2(\Rb^d;\varrho)}$ (with $\det(\De\Lambda_n)\le M_{\lambda_n}^n\vee 1$) to obtain~\eqref{eq: bound for the determinindant of beta in finite dimension}.

    For the second bound, apply \citep[Lemma E.3]{abbasi-yadkori_online_2012} directly to obtain
    \begin{align}
        \det\left(\idmat+\frac{1}{2\lambda C}\sum_{t=1}^T \vartheta^{(t,n)}\De\Lambda_n^{-1}{\vartheta^{(t,n)}}^\top\right) \le {\left(\frac{2\lambda\ubnorm \Tr(\De\Lambda_n)+t\ubnorm^2 }n\right)}^n\left(2\lambda \ubnorm\det(\De\Lambda_n)\right)\notag
    \end{align}
    wherefrom~\eqref{eq: bound for the determinindant of width in finite dimension} follows.
\end{proof}


\subsection{Finite order bases: matching and parametric models}\label{subsubsec: parametric rate /matching}


At this point, let us recall \cref{asmp: basis decay} which provides the quantification of the regularity of $c^*$ which we will use to set $n$.  We will now discuss some examples in which a finite basis is sufficient to control the approximation error.

\asmpthree*


\begin{proposition}\label{cor: regret for fixed approximation order with bounded basis}
    Under \cref{asmp: basis decay,asmp: estimate + subG,asmp: L2 case}, with $\zeta(n)\1_{\cdot\ge N}$ for some $N\in\Nb$ (i.e.\ if $\gamma_i^*=0$ for every $i>N$), then under the conditions of \cref{cor: regret for fixed approximation order} with $n=N$, $\Lambda_n=\norm{\cdot}_{L^2(\Rb^d;\varrho)}/2$, and $\alpha=1/2$ the bounds of \cref{cor: regret for fixed approximation order} become
    \begin{align}
        \regret_T^{\entropy,\ve} (\Ac_n)&\le \sigma\sqrt{2T\log\left(\frac2\delta\right)} + 2\ubnorm\sqrt{NT}\log\left(\frac{1}\lambda + \frac{T \ubnorm^2}N\right) \label{eq: entropic regret for fixed approximation order with bounded basis}\\
        \intertext{ and }
        \regret_T(\Bc_n)&\le  \sigma\sqrt{2T\log\left(\frac2\delta\right)} + 2\ubnorm\sqrt{NT}\log\left(\frac{1}\lambda + \frac{T \ubnorm^2}N\right) +\kappa(1+\sqrt{T}\log(T))\label{eq: kantorovich regret for fixed approximation order with bounded basis}
    \end{align}
\end{proposition}


Naturally, the assumption that $\gamma_i^*=0$ for any $i>N$ is not satisfactory, but it is verified for several existing models and serves to demonstrate that some learning problems in BOT are learnable at the rate $\tilde\Oc(\sqrt{T})$ given only knowledge of an upper bound on $N$ and $\norm{\gamma^*}_{\ell_2(\Rb)}$ and an appropriate basis ${(\phi_i)}_{i\in\Nb}$.  

Consider a matching problem in which the measures $\mu$ and $\nu$ are supported on $K$ and $K'$ loci respectively. Let $\{x_1,\ldots, x_K\}=\supp(\mu)$ and $\{x'_1,\ldots, x'_{K'}\}=\supp(\nu)$ denote these loci. We can let $c^*$ assume arbitrarily values outside of $\state=\{(x_i,x_j'):(i,)\in[K]\x [K']\}$ without loss of generality. Let $\epsilon<\inf\{\norm{u-v}:(u,v)\in\state^2\,,\; u\neq v\}$ and define the functions 
\[
    \phi_{i,j}:= \frac{6}{\pi\epsilon^3}\1_{\{B_2({(x_i,x_j')}^\top,\epsilon/2)\}} \quad \mbox{ for } (i,j)\in[K]\times[K']\,.
\]
Re-indexing the functions by $k\in[K\times K']$, and adding suitable functions for $k>KK'$, we obtain an orthonormal basis ${(\phi_k)}_{k\in\Nb}$ of $L^2(\Rb^d;\varrho)$, in which $c^*:=\sum_{k=1}^{KK'}\gamma_k^*\phi_k$. Consequently, we can apply \cref{cor: regret for fixed approximation order with bounded basis} with $N=KK'$ to obtain a regret bound of $\tilde\Oc(\sqrt{KK'T})$ for the learning problem.

Alternatively, consider that there is a parametric model for $c^*$, i.e.\ there is $\theta^*\in\Rb^p$ such that 
\[
    c^*(x,y) = \sum_{i=1}^p {\theta^*}_i\Phi_i(x,y)\,,
\]
for some embedding function $\Phi:\Rb^d\x\Rb^d\to\Rb^p$. When the embedding function is known, one can construct a basis through the Gram-Schmidt process. Let $\phi_1:=\Phi_1/\norm{\Phi_1}_{L^2(\Rb^d;\varrho)}$, and for $i\le p$, define $S_i:={\{\phi_k:k<i\}}^\perp$ the orthogonal complement of the sequence this far. Now, repeatedly project the feature dimensions onto $S_i$ to construct $\phi_i:=P_{S_i}\Phi_i/\norm{P_{S_i}\Phi_i}_{L^2(\Rb^d;\varrho)}$. For $i>p$, take any orthonormal basis of $S_p$ to complete the basis, it will not be used anyway. Consequently, we can also apply \cref{cor: regret for fixed approximation order with bounded basis} with $N=p$ to obtain a regret bound of $\tilde\Oc(\sqrt{pT})$ for the learning problem.


These results are summarised in \cref{cor: on finite basis regret}, but notice that higher order polynomial models can be readily considered as well, such as quadratic costs 
\[
    c^*(x,y) = {\Phi(x,y)}^\top\Theta^*\Phi(x,y)\,,
\]
for $\Theta^*\in\Rb^{p\times p}$, by simply reparametrising it as a linear model in dimension $p^2$ and applying the same construction. Many other models can be considered in this manner, and would benefit from further specialised investigation.

\CorOnFiniteBasis*

\subsection{Increasing order basis truncation}\label{subsubsec: increasing order}

In this section, we will extend the results of \cref{subsubsec: finite order} to let $n$ change with $t\in\Nb$ along the learning process. We will denote the corresponding sequence by ${(n_t)}_{t\in\Nb}\subset\Nb$. It is relatively simple to see that the proofs of the key properties of online least-squares estimation will extend, but we include the key proof sketches for completeness. We begin by diagonalising the validity of the confidence sets in \cref{lemma: confidence sets with varying basis order}.

\begin{lemma}\label{lemma: confidence sets with varying basis order}
    Under \cref{asmp: estimate + subG,asmp: L2 case},
\[
    \Pb\left(\bigcap_{t=1}^\infty \tilde{\event}_t^{n_t}(\delta)\right)\ge 1-\frac\delta2\,.
\]
\end{lemma}

\begin{proof}
    The proof only requires diagonalisation of the standard stopping time construction. For $(\delta,t)\in(0,1)\x\Nb$, on the filtered probability space $(\Omega,\Fc_\infty,\Fb,\Pb)$ define 
    \[ 
        B_t(\delta):=\left\{\omega\in\Omega: \norm{\gamma^* -\hat \gamma^{n_t,\lambda}_t}_{\tilde\design_t^{\lambda,n_t}}\le \tilde\width_{t,n_t}(\delta) \right\}\overset{\mbox{a.s.}}{=}\{\omega\in\Omega: c^*\vert_{n_t} \not\in \tilde\confset_{t}^{n_t}(\delta)\}\,,
    \]
    be the $t\textsuperscript{th}$ ``bad event'', and let $\tau_\delta: \omega\in\Omega\to\inf\{t\in\Nb: \omega\in B_t(\delta)\}$, which is a stopping time. We have 
    \[
        \{\tau <+\infty\}= \bigcup_{t\in\Nb} B_t(\delta)\,.
    \]
    By construction, in the classical manner:
    \begin{align*}
        \Pb\left(\bigcup_{t\in\Nb} B_t(\delta)\right)&= \Pb(\tau<+\infty, B_t(\delta))\le \Pb\left(\tilde\event_t^{n_t}(\delta)\right)\le \frac{\delta}2\,.
    \end{align*}
\end{proof}

The confidence sets using for non-constant ${(n_t)}_{t\in\Nb}$ are simply instantiations of~\eqref{eq: confidence set fixed order} and~\eqref{eq: width fixed order} with $n_t$ in place of $n$. This change of basis with time however requires a modification of the proof of \cref{lemma: bound on width term} as the steps summed up in~\eqref{eq: summation of widths in proof of the bound on width term lemma} are no longer homogenous. In particular,~\eqref{eq: sum of log is logdet} is no longer valid. 

\begin{lemma}\label{lemma: bound on width term with varying basis order}
    Under \cref{asmp: estimate + subG,asmp: L2 case}, if $\Lambda_n:=\frac{1}{2}\norm{\cdot}_{2}^2$ with the norm being on $\Rb^n$, then
    \begin{align}
        \sum_{t=1}^T\langle c^*\vert_{n_t} -\tilde c_t^{\,n_t}\vert \tilde\pi_t\rangle\le 2C\sigma\left( \sqrt{2\log\left(\frac{\lambda^{-1}+\frac{T\ubnorm^2}{n_T}}{\delta}\right)}+\sqrt{\lambda}\ubnorm\right) \sqrt{n_T T\log\left(1+ \frac{T}{n_T\ubnorm^2}\right)}\,.\notag
    \end{align}
\end{lemma}

\begin{proof}
    Recall the notation of \cref{lemma: bound on width term}, which adapts to $\varphi_t:=\langle c^*\vert_{n_t}-c_t^{n_t}\vert \tilde\pi_t\rangle$ for $t\in\Nb$ and $\tilde c_t^{\, n_t}:= \sum_{i=1}^{n_t}\tilde\gamma^{n_t}_{t,i}\phi_i$. The proof of \cref{lemma: bound on width term} yields 
    \begin{align}
        \sum_{t=1}^T\varphi_t \le 2C\beta_{T,n_T}(\delta)\sqrt{T\sum_{t=1}^T\log\left(1+\frac1{2\ubnorm}\norm{\vartheta^{(t,n_t)}}_{{({\tilde{D}_t^{\lambda,n_t}})}^{-1}}\right)}\,.
    \end{align}
    
    First, one can bound $\tilde\width_{T,n_T}(\delta)$ by~\eqref{eq: bound for the determinindant of width in finite dimension} in \cref{lemma: bounds for the determinants in finite dimension}. 

    It remains to adapt the logarithmic term into a log-determinant of the desired form by conforming the vectors $\vartheta^{(t,n_t)}$. To do so, let us define the block matrices
    \[
    Z_t:= \begin{pmatrix}
        {(\tilde\design_t^{\lambda,n_t})}^{-1} & \bm{0} \\
        \bm{0} & \bm{0}\\
        \end{pmatrix} \quad \mbox{ for } t\in\Nb\,,
    \]
    so that we may use the rank one update formula to write
    \[
    \prod_{t=1}^T \left(1+\frac1{2\ubnorm}\norm{\vartheta^{(t,n_t)}}_{{({\tilde{D}_t^{\lambda,n_t}})}^{-1}}\right) = \frac{\det\left(\De\Lambda_n+ \sum_{t=1}^T \vartheta^{(t,n_t)}Z_t{\vartheta^{(t,n_t)}}^\top\right)}{\det(\De\Lambda_n)}\,.
    \]
    Taking $\Lambda_n=\frac{1}{2}\norm{\cdot}_{2}^2$ as given, we can bound the determinant of the numerator by
    \[
        {\det\left(\De\Lambda_n+ \sum_{t=1}^T \vartheta^{(t,n_t)}Z_t{\vartheta^{(t,n_t)}}^\top\right)} \le {\left(1+ \frac{T}{n_T\ubnorm^2}\right)}^{n_T}
    \]
    as in~\cite[Lemma E.3]{abbasi-yadkori_online_2012}. Combining with the bound on $\tilde\width_{T,n_T}(\delta)$ completes the proof.
\end{proof}


Having established the technical lemmata, we now turn to the regret guarantees of the varying order basis truncation version of \cref{alg: alg shared + approx}. In particular, recall \cref{asmp: basis decay} to give a quantification of the regularity of $c^*$, which in turn will allow us to tune ${(n_t)}_{t\in\Nb}$ to obtain the best possible regret bounds in \cref{thm: regret for varying approximation}.

\begin{restatable}{theorem}{RegretForVaryingApprox}\label{thm: regret for varying approximation}
    Assume \cref{asmp: estimate + subG,asmp: L2 case,asmp: basis decay} and  $\zeta(n)=1-n^{-q}$ for some $q>0$. For any $\delta\in(0,1)$, $\lambda>0$, $\ve>0$, let $\tilde\Ac$ (resp. $\tilde\Bc$) denote \cref{alg: alg shared + approx} with ${(n_t)}_{t\in\Nb}={(\ceil{t^{\frac1{2q+1}}})}_{t\in\Nb}$, $\Lambda_n=\frac12\norm{\cdot}_{2}^2$, for all $n\in\Nb$, and ${(\ve_t)}_{t\in\Nb}={(\ve)}_{t\in\Nb}$ (resp. ${(\ve_t)}_{t\in\Nb}= {(\alpha t^{-\alpha})}_{t\in\Nb}$). For any $T\in\Nb$, the following regret bounds hold:
    \begin{align}
        \regret_T^{\entropy,\ve} (\tilde\Ac)&\le \sigma\sqrt{2T\log\left(\frac2\delta\right)} + \ubnorm\left(1+\frac{qT^{\frac{q+1}{2q+1}}}{2q+1}\right) \notag\\
        &\quad + 2\ubnorm\sigma T^{\frac{q+1}{2q+1}}\left(\sqrt{2\log\left(\frac{\lambda^-1+2T^{\frac{2q}{2q+1}}\ubnorm^2}{\delta}\right)}+\sqrt\lambda \ubnorm\right)\sqrt{\log\left(1+\frac{2T^{\frac{2q}{2q+1}}}{\ubnorm^2}\right)}\,,\notag
        %\label{eq: entropic regret for varying approximation order with bounded basis}\\
        \intertext{ and }
        \regret_T(\tilde\Bc)&\le  \sigma\sqrt{2T\log\left(\frac2\delta\right)} + \ubnorm\left(1+\frac{qT^{\frac{q+1}{2q+1}}}{2q+1}\right)+\kappa(1+\sqrt{T}\log(T))\notag\\
        &\quad + 2\ubnorm\sigma T^{\frac{q+1}{2q+1}}\left(\sqrt{2\log\left(\frac{\lambda^-1+2T^{\frac{2q}{2q+1}}\ubnorm^2}{\delta}\right)}+\sqrt\lambda \ubnorm\right)\sqrt{\log\left(1+\frac{2T^{\frac{2q}{2q+1}}}{\ubnorm^2}\right)}\,,\notag
        %\label{eq: kantorovich regret for varying approximation order with bounded basis}
    \end{align}
\end{restatable}

\begin{proof}
    The proof requires only two steps from the one of \cref{cor: regret for fixed approximation order}. First, we bound the approximation error term.
    \Cref{lemma: fourier decay bound} readily implies that
    \[
        \abs{\langle c^* - c^*\vert_{n_t}\vert \pi_t\rangle} \le \sum_{k=n_t+1}^{+\infty}\abs{\gamma_i^*}\,.
    \]
    Summing over $t\in\Nb$, one obtains
    \begin{align}
        \sum_{t=1}^T \abs{\langle c^* - c^*\vert_{n_t}\vert \pi_t\rangle}\le \sum_{t=1}^T \sum_{i=n_t+1}^{+\infty}\abs{\gamma_i^*}\,.\label{eq: regret term for truncation, summed}
    \end{align}
    By \cref{asmp: basis decay}, for any $n\in\Nb$, we have
    \[ 
        \sum_{i=n_t+1}^\infty\abs{\gamma_i^*} = \norm{c^*}_{L^2(\Rb^d;\varrho)}- \sum_{i=1}^{n_t}\abs{\gamma_i^*} \le \norm{c^*}_{L^2(\Rb^d;\varrho)}(1-\zeta(n_t))
    \]
    so that for any $u>0$, the choice $n_t:=\ceil{\zeta^{-1}((1-t^{-u}))}=\ceil{t^{\frac uq}}$ ($q>0$) yields
    \[
        \sum_{i=n_t+1}^\infty \abs{\gamma_i^*}\le  \norm{c^*}_{L^2(\Rb^d;\varrho)} t^{-u}\,.
    \]
    This follows from the fact that $\zeta$ can be made a bijection of $\Rb_+\to(0,1]$, and that $\zeta$ is increasing. Injecting $\sum_{s=n_t+1}^{+\infty}\abs{\gamma_s^*}\le \norm{c^*} t^{-u}$ into~\eqref{eq: regret term for truncation, summed} yields
    \[
        \sum_{t=1}^T \abs{\langle c^* - c^*\vert_{n(t)}\vert \pi_t\rangle}\le\norm{c^*}_{L^2(\Rb^d;\varrho)}\left(1+\frac{T^{1-u}}u\right)\,.
    \]
    The second step simply involves applying \cref{lemma: bound on width term with varying basis order} for $n_T:=\ceil{T^{\frac{u}q}}\le 2T^{u/q}$ to obtain a bound of order $\Oc(T^{\frac12+\frac{u}{2q}})$. Setting $u=\frac{q}{2q+1}$ yields the stated bounds.
\end{proof}

%In terms of computational complexity, 

%\lc{In terms of computational complexity, suppose for example that $\zeta(t)=1-t^{-\iota}$ for some $\iota>0$. Note that one may always assume\footnote{Summability of $\{\abs{\gamma_i}\}_{i\in\Nb}$ implies that $\{1-\zeta(n)\}_{n\in\Nb}$ should be summable.} (up to multiplying $\zeta$ by a constant) that $\iota>1$. At step $t\in\Nb$, solving the linear system of \eqref{eq: RLS basis truncation} (of dimension $t^{1/2\iota}\times t$) thus has a complexity of $\Oc(t^{1+1/\iota})$.} 



\subsection{Tikhonov regularisation and RKHS theory}\label{subsec: tikhonov and RKHS}

In this section, we will assume that $\Lambda=\frac12\norm{\cdot}^2_2$ for simplicity. In general any increasing positive function of $\norm{\cdot}_2$ will suffice to use the representer theorem as per our argument. Suppose we are given $(\Hf,K)$ a Reproducing Kernel Hilbert Space\footnote{Understood, of course, up to the identifications necessary for the RKHS to be a space of functions. Recall that $L^2(\Rb^d;\varrho)$ is \emph{not} an RKHS due to a subtlety of this nature.} (RKHS) such that $\Hf\subset L^2(\Rb^d;\varrho)$. We may specialise the RLS estimator (see \cref{prop: least squares estimator}) to this case by noting that $M_t:={(K(a_0,\cdot),\dots,K(a_{t-1},\cdot))}^\top$. 

By the representer theorem, at any step $t\in\Nb$,the solution to the regularised least squares problem in $\Hf$ is given by
\[
    \hat f_t^\lambda = \sum_{i=0}^{t-1} \upsilon_i K(a_i,\cdot)\,,
\]
for some ${(\upsilon_i)}_{i=0}^{t-1}\in\Rb^t$. The problem can therefore be reduced to the finite dimensional optimisation problem 
\[
    \min_{\upsilon\in\Rb^t} \norm{\vec{R}_t - K_t\upsilon}_2^2 + \lambda \upsilon^\top K_t\upsilon\,,
\]
in which $K_t:={[K(a_i,a_j)]}_{i,j}\in\Rb^{t\x t}$ is the kernel (Grammian) matrix. The rest of the standard developments follow, and one arrives at the approximation bound 
\[
    \sum_{t=1}^T \langle c^* - \tilde c_t\vert \tilde\pi_t\rangle \le 2\ubnorm\beta_T(\delta)\sqrt{2T\log\det\left(I+{(\lambda \ubnorm)}^{-1}K_T\right)}
\] 
by \cref{lemma: bound on width term}, and corresponding regret bounds easily follows via \cref{thm: entropic regret,thm: regret Kantorovich}. From here, one can easily recover bounds ad-hoc or by following the general methodology of \cref{subsec: Fourier basis representation}.

One of the main benefits of kernel methods is that they can be used to learn in infinite-dimensional spaces efficiently. While they are inherently efficient thanks to the kernel trick, works in this field have suggested further efficiency refinements such as~\cite{takemori_approximation_2021} which uses approximation theory to reduce learning in an RKHS to a finite-dimensional approximation on a well chosen basis. This resembles the methodology used above, further developments in this direction appear an interesting avenue for research. 
\begin{proof}[Proof of Lemma~\ref{lem:VstarVplan}]

The lemma is proven by induction over $h$, starting from $V^{\star}_{H+1} = V_{H+1}=\bm{0}$. We have
\begin{align*}
    &Q^{\star}_h(s,a) - Q_{h}(s,a) - \sum_{h'=h}^H\left(\beta(\delta)(\sigma_{h,N}(s_h,a_h)+\frac{2}{\sqrt{N}})+\frac{2}{N}\right)\\
    &= r(s,a)+ [P_hV^{\star}_{h+1}](s,a)- r(s,a) - \hat{g}_{h}(s,a)- \sum_{h'=h}^H\left(\beta(\delta)(\sigma_{h,N}(s,a)+\frac{2}{\sqrt{N}})+\frac{2}{N}\right)\\
    &\le [P_hV^{\star}_{h+1}](s,a) -[P_hV_{h+1}](s,a)- \sum_{h'=h+1}^H\left(\beta(\delta)(\sigma_{h,N}(s,a)+\frac{2}{\sqrt{N}})+\frac{2}{N}\right)\\
    &=[P_h(V^{\star}_{h+1}-V_{h+1})](s,a)+\beta(\delta)(\sigma_{h,N}(s,a)(s,a)+\frac{2}{\sqrt{n}})+\frac{2}{n}.\\
\end{align*}
The inequality holds by $\Ec$.
Then, we have
\begin{align*}
    V^{\star}_h(s_h) - V_h(s_h) &=
    \max_{a\in\Ac} Q^{\star}_h(s,a) - \max_{a\in\Ac}Q_{h}(s,a)\\
    &\le \max_{a\in\Ac} \{Q^{\star}_h(s,a) - Q_{h}(s,a)\}\\
    &\le 0
\end{align*}
That proves the lemma. 

\end{proof}




\begin{proof}[Proof of Lemma~\ref{VHVpisum}]
Note that $V_{H+1}=V^{\pi}_{H+1}=\bm{0}$. We next obtain a recursive relationship for the difference $V_h(s)-V^{\pi}_h(s)$. 
\begin{align*}
    V_h(s_h)-V^{\pi}_h(s_h) &= Q_h\left(s_h, \pi(s_h)\right) -  Q^{\pi}_h\left(s_h, \pi(s_h)\right) \\
    &=r\left(s_h, \pi(s_h)\right) + \hat{g}_h\left(s_h, \pi(s_h)\right) + \beta\sigma^N_h\left(s_h, \pi(s_h)\right) - r\left(s_h, \pi(s_h)\right) - [P_hV^{\pi}_{h+1}]\left(s_h, \pi(s_h)\right)\\
    &\le [P_hV_{h+1}]\left(s_h, \pi(s_h)\right) +2\beta\sigma^N_h\left(s_h, \pi(s_h)\right) - [P_hV^{\pi}_{h+1}]\left(s_h, \pi(s_h)\right),\\
\end{align*}
where the inequality is due to $\Ec_1$. 

Recursive application of the above inequality over $h=H, H-1, \cdots, 1$, we obtain
\begin{align*}
    V_1(s_1) - V_1^{\pi}(s_1)& \le \E_{s_{h+1}\sim P(\cdot|s_h,\pi(s_h)), h \le H}\left[\sum_{h=1}^H2\beta\sigma^N_h\left(s_h, \pi(s_h)\right)\right]\\
    &=2HV_1^{\pi}(s_1; \alpha/H )
\end{align*}


\end{proof}


\begin{proof}[Proof of Lemma~\ref{lem:vnstar_vn}]

    
    The lemma is proven by induction, starting from $V^{\star}_{h_0+1}(\cdot; \tilde{\sigma}^{h_0}_{ n})=V_{h_0+1,n}=\bm{0}$. We have, for $h\le h_0$
    \begin{align}\nn
        V^{\star}_{h}(s; \tilde{\sigma}^{h_0}_{ n})-V_{h,n}(s) &= \max_{a\in\Ac} Q^{\star}_{h}(s,a; \tilde{\sigma}^{h_0}_{ n})-\max_{a\in\Ac}Q_{h,n}(s,a)\\\nn
        &\le \max_{a\in\Ac} \left\{Q^{\star}_{h}(s,a; \tilde{\sigma}^{h_0}_{ n})-Q_{h,n}(s,a)\right\}\\\nn
        & =\max_{a\in\Ac} \left\{[P_hV^{\star}_{h+1}](s,a; \tilde{\sigma}^{h_0}_{ n})-[P_hV_{h+1,n}](s,a)\right\}\\\nn
        &\le 0. 
    \end{align}
    The fist inequality is due to rearrangement of $\max$ and the second inequality is by the base of induction. We thus prove the lemma. 
\end{proof}


\begin{proof}[Proof of Lemma~\ref{lem:v_minus_vpi}]
We have
\begin{align}
    V_{h,n}(s_h)- V^{\pi_n}_h(s_h; \tilde{\sigma}^{h_0}_n)
    &=
    Q_{h,n}(s_h, a_h)- Q^{\pi_n}_h(s_h, a_h; \tilde{\sigma}^{h_0}_n)\\\nn &\le r(s_h, a_h) + [P_hV_{h+1,n}](s_h, a_h) + 2\beta(\delta)\sigma_{n,h}(s_h, a_h)- r(s_h, a_h) - [P_hV^{\pi_n}_{h+1}](s_h, a_h; \tilde{\sigma}^{h_0}_n)\\\nn
    &= V_{h+1, n}(s_{h+1}) - V^{\pi_n}_{h+1}(s_{h+1}; \tilde{\sigma}^{h_0}_n)+ 2\beta(\delta)\sigma_{n,h}(s_h, a_h) \\\nn
    &+ ([P_hV_{h+1,n}](s_h, a_h) - V_{h+1, n}(s_{h+1})) \\\nn
    & + (V^{\pi_n}_{h+1}(s_{h+1}; \tilde{\sigma}^{h_0}_n) - [P_hV^{\pi_n}_{h+1}](s_h, a_h; \tilde{\sigma}^{h_0}_n))
\end{align}

Applying this recursively, we obtain

\begin{align}
    V_{1,n}(s_1)- V^{\pi_n}_1(s_1; \tilde{\sigma}^{h_0}_n) \le \sum_{h=1}^H 2\beta(\delta)\sigma_{n,h}(s_h, a_h) + ...
\end{align}
    
\end{proof}

\section{Discussion of some open problems}\label{app: open problems}

\subsection{Practical computation of actions and action-set violations}\label{subsec: action feasibility}


In \cref{alg: alg shared,alg: alg shared + approx} we used a black-box solver for an entropic optimal transport problem. This is a computational abstraction and not implementable in practice. Implementing a computationally feasible resolution raises several questions. 

\subsubsection{Numerical resolution of the Kantorovich problem}

Sinkhorn's algorithm is the standard method for solving entropic optimal transport problems. It relies on the dual formulation of the entropic problem, that is
\begin{align}
    \ent(\mu,\nu,c,\ve) = \sup_{\substack{\varphi\in L^1(\mu)\\\psi\in L^1(\nu)\\\varphi\oplus\psi\le c}} \left\{\int \varphi\de\mu + \int \psi\de\nu - \ve\int e^{\ve^{-1}(\varphi+\psi-c)}\de(\mu\tensor\nu) + \ve\right\}\,\notag%\label{eq: entropic dual}
\end{align}
in the case $c\in L^1(\mu\tensor\nu)$, see e.g.\ \cite[Thm.~4.7]{nutz_introduction_2022}. The solution of the dual problem is given by the pair $(\varphi^*,\psi^*)$ which satisfies the Schrödinger system
\begin{align*}
    \varphi^*&=-\ve\log\left(\int e^{\frac{\psi^*(y)-c(\cdot,y)}\ve}\de\nu(y)\right) \quad \mu\mbox{-a.s.}\\
    \psi^*&=-\ve\log\left(\int e^{\frac{\varphi^*(x)-c(x,\cdot)}\ve}\de\mu(x)\right) \quad \nu\mbox{-a.s.}\,.
\end{align*}

Sinkhorn's algorithm \citep{sinkhorn_concerning_1967}, in its application to this problem \citep{cuturi_sinkhorn_2013}, is a fixed-point iteration which improves one potential at a time. In other words, for $n\in\Nb$, it computes
\begin{align*}
    \varphi_{2n+1}&=-\ve\log\left(\int e^{\frac{\psi_{2n}(y)-c(\cdot,y)}\ve}\de\nu(y)\right) \\
    \intertext{and}
    \psi_{2n}&=-\ve\log\left(\int e^{\frac{\varphi_{2n-1}(x)-c(x,\cdot)}\ve}\de\mu(x)\right)\,.
\end{align*}

A primal solution to $\ent(\mu,\nu,c,\ve)$ can be recovered from the optimal dual potentials $(\varphi^*,\psi^*)$ via
\[ 
    \de\pi^*=e^{\frac{\varphi^*\oplus\psi^*-c}\ve}\de[\mu\tensor\nu]\,,
\]  
in which $\varphi^*\oplus\psi^*: (x,y)\mapsto \varphi^*(x)+\psi^*(y)$. Through an analogue for ${(\varphi_{2n+1},\psi_{2n})}_{n\in\Nb}$, we can obtain iterates ${(\varpi_n)}_{n\in\Nb}$.

\begin{lemma}[{\cite[Thm.~3.15]{eckstein_quantitative_2022}}]\label{lemma: bound on inf memory sinkhorn}
    If $c$ is Lipschitz on $\supp(\mu)\times\supp(\nu)$, and $\mu,\nu$ are sub-Gaussian measures, then the iterates ${\{\varpi_n(c)\}}_{n\in\Nb}$ of Sinkhorn's algorithm satisfy
    \[
        \entf(c^*,\varpi_n(c)) - \ent(\mu,\nu,c,\ve)\le C_0\ve n^{-\frac14}\,,
    \]
    for every $\ve>0$, in which $C_0$ is a numerical constant independent of $n$.
\end{lemma}
We omit the explicit dependencies in the constant $C_0$ as they are quite technical and require parsing a large part of~\cite{eckstein_quantitative_2022}, which proceeds from within a highly general framework. We should note, however, that consequently their bound is valid under much weaker assumptions than the ones stated here, and that the rate can, in fact, be improved if $c$ has sub-linear growth.


Unfortunately for regret minimisation, $\varpi_n$ need not be a transport plan in $\Pi(\mu,\nu)$, meaning it is not a valid action. Removing the requirement that $\pi_t\in\Pi(\mu,\nu)$ entirely would render the problem meaningless, as the regret can be made negative by finding a single point such that $c(x,y)<\kant(\mu,\nu,c)$, and playing $\delta_{(x,y)}$. 

As an auxiliary remark, this problem is one of the main hurdle to adapting \cref{alg: alg shared} to unknown marginals, as there would be no conceivable way to pick valid transport plans, which renders the analysis a non-starter.


\subsubsection{On action violations}


Two possible directions appear to resolve this issue: one at the level of bandit design, and one at the level of numerical optimal transport. The former revolves around the idea of incorporating action-set violations to regret analysis, the latter around the idea of modifying Sinkhorn's algorithm to produce valid primal iterates at each step, e.g.\ by projecting onto $\Pi(\mu,\nu)$.

The question of violating action sets has been posed before in Bandit Theory and has also arisen in practical use-cases in Reinforcement Learning, see \citep{seurin_im_2020}. It is a staple topic in the context of fairness, see e.g.\ \citep{joseph_fairness_2016} and of contextual bandits (including linear stochastic bandits) in which various other types constraint have also been considered, see e.g. \citep{liu_efficient_2024}. These types of constraints typically, in effect, disable certain arms at certain times, a generic setting which has been considered as well, e.g.\ by~\cite{kleinberg_regret_2010,abensur_productization_2019}. 

These works adopt a range of strategies to formulate the problem in a meaningful way, but their perspectives don't really fit with the real challenge we have with the OT problem. The problem isn't so much that the constraints placed on the action set are complicated: $\Pi(\mu,\nu)$ is a convex, compact set defined by linear inequalities. The problem arises entirely from the facts that $\Pi(\mu,\nu)$ is infinite-dimensional, and that it is a subspace of $\Ps(\state)$, whose geometry is far from straightforward.

A preliminary exploration of this topic would likely require a taxonomy of the different possible violations of $\Pi(\mu,\nu)$. Indeed, $\pi_t$ could violate one or both marginal constraints, or it could even fail to be a probability measure through the total mass or positivity conditions. It appears likely that these will have quite different impacts both on the problem's geometry and on practical usefulness. Thereafter, one might consider whether guaranteeing finitely many violations, as~\cite{liu_efficient_2024} do, or developing a penalised regret is more appropriate.

The alternative would be to design an algorithm which optimises the entropic or Kantorovich problems through while staying within the constraint set $\Pi(\mu,\nu)$ (either for all time, or once it reaches a desired precision). On the one hand, there are finite-dimensional intuitions for this to work as Sinkhorn's algorithm can be viewed as a form of gradient descent \citep{leger_gradient_2021}, which could be projected onto $\Pi(\mu,\nu)$ (which is convex and compact). On the other hand, the geometry of $\Pi(\mu,\nu)$ as an infinite-dimensional probability space is likely to make rigourously doing so (and deriving convergence rates) quite arduous work. 


\subsection{Extensions to the Monge problem}\label{subsec: Monge pb}

The \emph{Monge} optimal transport problem associated to $(\mu,\nu,c)$ is
\begin{align}
    \monge(\mu,\nu,c):= \inf_{T\in\Ts} \int c(x,T(x))\de\mu(x)\,,
    \label{eq: monge def}
\end{align}
in which $\Ts$ is the set of all $\mu$-measurable maps  $T:\Mc_\mu\to\Mc_\nu$ such that $\mu(T^{-1}(\cdot))=\nu$. Chronologically, this is in fact the original formulation of the OT problem \citep{monge1781memoire}. 

The Monge problem is best approached through finite-dimensional practical applications such as \emph{matchings} of students to universities, employees to employers, etc. The requirement that the map $T$ be a function imposes an \emph{indivisibility} of the mass $T$ moves from $\mu$ to $\nu$ (i.e.\ one university per student). This makes the resolution of the problem much more difficult. For example, if $\mu$ and $\nu$ each have two atoms with weights $(1/2,1/2)$ and $(1/3,2/3)$ respectively, then $\Ts=\emptyset$, meaning $\monge(\mu,\nu,\cdot)\equiv +\infty$, and the problem is never solvable. 

If $\mu,\nu$ are non-atomic, $\monge(\mu,\nu,c)$ can be interpreted as the cheapest way (w.r.t. $c$) to transport a $\mu$-shaped pile of infinitesimally small things into a $\nu$-shaped one, but its geometry remains complicated. 
The Kantorovich relaxation drastically simplified the geometry of the problem and remains one of the most effective tools to approach the Monge problem, which is why it is accepted as the standard in modern OT theory.

Note that the relaxation from $\monge(\mu,\nu,c)$ to $\kant(\mu,\nu,c)$ is known to be exact in some cases, such as $c=\norm{\cdot-\cdot}^2/2$ with $\Mc_\mu=\Mc_\nu=\Rb^d$, $(\mu,\nu)$ having second-order moments and $\mu$ being absolutely continuous w.r.t.\ the Lebesgue measure~\cite[Thm.~5.2]{ambrosio_lectures_2021}. See also \citep[Thm.~5.30]{villani_optimal_2009} for weaker conditions. 
But it is also known (e.g.\ via the above example) that this  relaxation is not without loss.

If we want to learn a Monge problem, we must, of course, make sufficient assumptions for it to be solvable, but more importantly we must face the issue that~\eqref{eq: monge def} is now a non-linear functional and that $\Ts$ is not as docile a set as $\Pi(\mu,\nu)$. Here, the recent work in statistical optimal transport on learning Monge maps (i.e.\ the solutions to~\eqref{eq: monge def}) is highly relevant, see e.g.~\cite[Ch.~3]{chewi_statistical_2024} or the paragraph in \cref{app: biblio} below. Though once again most work focuses on the batch sampling of marginals, not on online learning. This line of work would appear to also require more general results about the learning of minima of non-linear functionals, which are not yet available in the literature. Overall, it remains unclear if the Monge problem is on a similar or different level of difficulty to the Kantorovich problem as it is not clear that the techniques to reduce to online least-squares we used will transfer. 

Beyond these statistical issues, one should also expect the problems of effective optimisation from \cref{subsec: action feasibility} to return with a vengeance as the Monge problem is a fully non-linear problem unlike the Kantorovich problem which is an (infinite-dimensional) linear program.


\section{Bibliographical complements on statistical optimal transport}\label{app: biblio}

    An excellent detailed history of the development of OT as a mathematical theory, replete with bibliographical notes, can be found in~\cite[Ch.~3]{villani_topics_2003}. Summarising this field's venerable history further would be of little value. Instead, we will expand on relevant research specifically about \textit{learning} optimal transport problems. We touch on key aspects of the literature below, and refer to the forthcoming book~\cite{chewi_statistical_2024}, for a deeper longitudinal overview.
  
        \paragraph{Estimation of Wasserstein distances}
            One of the most important contributions of optimal transport is a family of useful distances between probability measures: the Wasserstein metrics. The study of these distances has allowed major progress on the geometry of spaces of probability measures, and has been used in many applications. It is therefore natural that the estimation of these distances has been a major topic of interest in the learning of optimal transport. 

            The key question here is the convergence in Wasserstein distance of an empirical distribution to the true distribution. Pioneering work on this topic began in the 80s and 90s, see \citep{ajtai_optimal_1984,talagrand_transportation_1994}, with the study of \emph{Matching} (i.e.\ discrete optimal transport). Key statistical analysis of this problem includes finite sample bounds, see \citep{horowitz_mean_1994} and more recently \citep{fournier_rate_2015,weed_sharp_2019} among others, as well as distributional limits, see e.g.\
            \citep{tameling_empirical_2019} and references therein. 

            Sadly, most work has remained limited to Wasserstein distances rather than generic cost functions, owing to a reliance on the pleasant geometric properties that they enjoy.

        \paragraph{Estimation of Entropic OT}
            Motivated by the success of Entropic OT in designing numerical solution to OT problems, see \citep{cuturi_sinkhorn_2013}, work on the Entropic problem has focused on estimating $\ent(\mu,\nu,c,\ve)$ using $\ent(\hat\mu_n,\hat\nu_n,c,\ve)$, for empirical measures $(\hat\mu_n,\hat\nu_n)$. This has often gone together with estimation for the Schrödinger potentials $(\varphi,\psi)$ of~\eqref{eq: entropic dual}.

            While this is very much the same type of study as for the Kantorovich problem in Wasserstein metrics, it should be noted that the entropic problem exhibits qualitatively different behaviour. While learning the Kantorovich problem exhibits a curse of dimensionality, the entropic problem exhibits parametric-rate (dimension-free) convergence, as shown by~\cite{genevay_sample_2019,rigollet_sample_2022}. This was tempered by large dependencies in other problem quantities, which were reduced over time \citep{stromme_minimum_2024} and were complemented by distributional limits, see e.g.\ \citep{gonzalez-sanz_weak_2024}.

        \paragraph{Estimation of Monge maps}
            While the estimation of Wasserstein distances is mostly motivated by statistical applications, the estimation of Monge maps is motived by effectively solving transport problems in an applied context. Here, one sees samples from two marginals $\mu$ and $\nu$, and attempts to estimate $T^*$ the minimiser of~\eqref{eq: monge def}. 
            
            There has been a significant amount of machine learning and statistics literature on this topic, following on from \citep{hutter_minimax_2021,gunsilius_convergence_2022}. Various types of estimators have been constructed, either derived from optimal transport theory \citep{hutter_minimax_2021}, or from plug-in estimates using classical machine learning methods such as $k$-NN \citep{manole_plugin_2024,deb_rates_2021}. 

        % \paragraph{Estimation of Dual potentials}
        %     (related via resolution of the Kantorovich problem, but offline and full information)
        
        \paragraph{Optimal transport applied to learning}

        While these bibliographical notes concern learning in optimal transport let us conclude by underline that the machine learning community has used optimal transport to impressive success in applications. One could highlight in particular Wassertein GANs \citep{arjovsky_wasserstein_2017} and subsequent works, e.g.\ \citep{salimans_improving_2018} as well as the field of domain adaptation \citep{courty_joint_2017,torres_survey_2021}



%%%%%%%%%%%%%%%%%%%%%%%%%%%%%%%%%%%%%%%%%%%%%%%%%%%%%%%%%%%%%%%%%%%%%%%%%%%%%%%
%%%%%%%%%%%%%%%%%%%%%%%%%%%%%%%%%%%%%%%%%%%%%%%%%%%%%%%%%%%%%%%%%%%%%%%%%%%%%%%
\end{document}