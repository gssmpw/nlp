
\documentclass[12pt]{article}

\usepackage{amsmath,hyperref,amsthm,microtype}
\usepackage{thm-restate}

\usepackage{times}

% \coltauthor{%
%  \Name{Lorenzo Croissant} \Email{lorenzo.croissant@ensae.fr}\\
%  \addr{CREST, ENSAE, CNRS, Palaiseau, France \\ INRIA FairPlay Team, INRIA, Palaiseau, France\\ }%
% }

\author{Lorenzo Croissant\thanks{The author would like to thank Nadav Merlis and Hugo Richard for their thoughtful comments on the manuscript, as well as Austin J. Stromme for sharing his insights in statistical optimal transport.} \\CREST, ENSAE, \& INRIA FairPlay Team, Palaiseau, France.\\ \texttt{lorenzo.croissant@ensae.fr}
}

%%%%%%%%%%%%%%%%%%%%%%%%%%%%%%%
% Editing tools
%%%%%%%%%%%%%%%%%%%%%%%%%%%%%%%

%\usepackage[textsize=tiny]{todonotes}

% highlight
\newcommand{\lc}[1]{{\color{purple} #1}}

%strikethrough
\usepackage[normalem]{ulem}
\newcommand{\lzost}[1]{\lc{\sout{#1}}}

%Margin comment 
\newcommand{\ltodo}[1]{\marginpar{\lc{#1}}}


\usepackage[a4paper, total={6in, 10in}]{geometry}
\usepackage{algorithm2e}
\usepackage{natbib}
\bibliographystyle{plainnat}
\setcitestyle{authoryear,open={(},close={)}}
\usepackage{cleveref}
%
% --- inline annotations
%
\newcommand{\red}[1]{{\color{red}#1}}
\newcommand{\todo}[1]{{\color{red}#1}}
\newcommand{\TODO}[1]{\textbf{\color{red}[TODO: #1]}}
% --- disable by uncommenting  
% \renewcommand{\TODO}[1]{}
% \renewcommand{\todo}[1]{#1}



\newcommand{\VLM}{LVLM\xspace} 
\newcommand{\ours}{PeKit\xspace}
\newcommand{\yollava}{Yo’LLaVA\xspace}

\newcommand{\thisismy}{This-Is-My-Img\xspace}
\newcommand{\myparagraph}[1]{\noindent\textbf{#1}}
\newcommand{\vdoro}[1]{{\color[rgb]{0.4, 0.18, 0.78} {[V] #1}}}
% --- disable by uncommenting  
% \renewcommand{\TODO}[1]{}
% \renewcommand{\todo}[1]{#1}
\usepackage{slashbox}
% Vectors
\newcommand{\bB}{\mathcal{B}}
\newcommand{\bw}{\mathbf{w}}
\newcommand{\bs}{\mathbf{s}}
\newcommand{\bo}{\mathbf{o}}
\newcommand{\bn}{\mathbf{n}}
\newcommand{\bc}{\mathbf{c}}
\newcommand{\bp}{\mathbf{p}}
\newcommand{\bS}{\mathbf{S}}
\newcommand{\bk}{\mathbf{k}}
\newcommand{\bmu}{\boldsymbol{\mu}}
\newcommand{\bx}{\mathbf{x}}
\newcommand{\bg}{\mathbf{g}}
\newcommand{\be}{\mathbf{e}}
\newcommand{\bX}{\mathbf{X}}
\newcommand{\by}{\mathbf{y}}
\newcommand{\bv}{\mathbf{v}}
\newcommand{\bz}{\mathbf{z}}
\newcommand{\bq}{\mathbf{q}}
\newcommand{\bff}{\mathbf{f}}
\newcommand{\bu}{\mathbf{u}}
\newcommand{\bh}{\mathbf{h}}
\newcommand{\bb}{\mathbf{b}}

\newcommand{\rone}{\textcolor{green}{R1}}
\newcommand{\rtwo}{\textcolor{orange}{R2}}
\newcommand{\rthree}{\textcolor{red}{R3}}
\usepackage{amsmath}
%\usepackage{arydshln}
\DeclareMathOperator{\similarity}{sim}
\DeclareMathOperator{\AvgPool}{AvgPool}

\newcommand{\argmax}{\mathop{\mathrm{argmax}}}     


%%%%%%%%%%%---SETME-----%%%%%%%%%%%%%
%replace @@ with the submission number submission site.
\newcommand{\thiswork}{INF$^2$\xspace}
%%%%%%%%%%%%%%%%%%%%%%%%%%%%%%%%%%%%


%\newcommand{\rev}[1]{{\color{olivegreen}#1}}
\newcommand{\rev}[1]{{#1}}


\newcommand{\JL}[1]{{\color{cyan}[\textbf{\sc JLee}: \textit{#1}]}}
\newcommand{\JW}[1]{{\color{orange}[\textbf{\sc JJung}: \textit{#1}]}}
\newcommand{\JY}[1]{{\color{blue(ncs)}[\textbf{\sc JSong}: \textit{#1}]}}
\newcommand{\HS}[1]{{\color{magenta}[\textbf{\sc HJang}: \textit{#1}]}}
\newcommand{\CS}[1]{{\color{navy}[\textbf{\sc CShin}: \textit{#1}]}}
\newcommand{\SN}[1]{{\color{olive}[\textbf{\sc SNoh}: \textit{#1}]}}

%\def\final{}   % uncomment this for the submission version
\ifdefined\final
\renewcommand{\JL}[1]{}
\renewcommand{\JW}[1]{}
\renewcommand{\JY}[1]{}
\renewcommand{\HS}[1]{}
\renewcommand{\CS}[1]{}
\renewcommand{\SN}[1]{}
\fi

%%% Notion for baseline approaches %%% 
\newcommand{\baseline}{offloading-based batched inference\xspace}
\newcommand{\Baseline}{Offloading-based batched inference\xspace}


\newcommand{\ans}{attention-near storage\xspace}
\newcommand{\Ans}{Attention-near storage\xspace}
\newcommand{\ANS}{Attention-Near Storage\xspace}

\newcommand{\wb}{delayed KV cache writeback\xspace}
\newcommand{\Wb}{Delayed KV cache writeback\xspace}
\newcommand{\WB}{Delayed KV Cache Writeback\xspace}

\newcommand{\xcache}{X-cache\xspace}
\newcommand{\XCACHE}{X-Cache\xspace}


%%% Notions for our methods %%%
\newcommand{\schemea}{\textbf{Expanding supported maximum sequence length with optimized performance}\xspace}
\newcommand{\Schemea}{\textbf{Expanding supported maximum sequence length with optimized performance}\xspace}

\newcommand{\schemeb}{\textbf{Optimizing the storage device performance}\xspace}
\newcommand{\Schemeb}{\textbf{Optimizing the storage device performance}\xspace}

\newcommand{\schemec}{\textbf{Orthogonally supporting Compression Techniques}\xspace}
\newcommand{\Schemec}{\textbf{Orthogonally supporting Compression Techniques}\xspace}



% Circular numbers
\usepackage{tikz}
\newcommand*\circled[1]{\tikz[baseline=(char.base)]{
            \node[shape=circle,draw,inner sep=0.4pt] (char) {#1};}}

\newcommand*\bcircled[1]{\tikz[baseline=(char.base)]{
            \node[shape=circle,draw,inner sep=0.4pt, fill=black, text=white] (char) {#1};}}

%
\setlength\unitlength{1mm}
\newcommand{\twodots}{\mathinner {\ldotp \ldotp}}
% bb font symbols
\newcommand{\Rho}{\mathrm{P}}
\newcommand{\Tau}{\mathrm{T}}

\newfont{\bbb}{msbm10 scaled 700}
\newcommand{\CCC}{\mbox{\bbb C}}

\newfont{\bb}{msbm10 scaled 1100}
\newcommand{\CC}{\mbox{\bb C}}
\newcommand{\PP}{\mbox{\bb P}}
\newcommand{\RR}{\mbox{\bb R}}
\newcommand{\QQ}{\mbox{\bb Q}}
\newcommand{\ZZ}{\mbox{\bb Z}}
\newcommand{\FF}{\mbox{\bb F}}
\newcommand{\GG}{\mbox{\bb G}}
\newcommand{\EE}{\mbox{\bb E}}
\newcommand{\NN}{\mbox{\bb N}}
\newcommand{\KK}{\mbox{\bb K}}
\newcommand{\HH}{\mbox{\bb H}}
\newcommand{\SSS}{\mbox{\bb S}}
\newcommand{\UU}{\mbox{\bb U}}
\newcommand{\VV}{\mbox{\bb V}}


\newcommand{\yy}{\mathbbm{y}}
\newcommand{\xx}{\mathbbm{x}}
\newcommand{\zz}{\mathbbm{z}}
\newcommand{\sss}{\mathbbm{s}}
\newcommand{\rr}{\mathbbm{r}}
\newcommand{\pp}{\mathbbm{p}}
\newcommand{\qq}{\mathbbm{q}}
\newcommand{\ww}{\mathbbm{w}}
\newcommand{\hh}{\mathbbm{h}}
\newcommand{\vvv}{\mathbbm{v}}

% Vectors

\newcommand{\av}{{\bf a}}
\newcommand{\bv}{{\bf b}}
\newcommand{\cv}{{\bf c}}
\newcommand{\dv}{{\bf d}}
\newcommand{\ev}{{\bf e}}
\newcommand{\fv}{{\bf f}}
\newcommand{\gv}{{\bf g}}
\newcommand{\hv}{{\bf h}}
\newcommand{\iv}{{\bf i}}
\newcommand{\jv}{{\bf j}}
\newcommand{\kv}{{\bf k}}
\newcommand{\lv}{{\bf l}}
\newcommand{\mv}{{\bf m}}
\newcommand{\nv}{{\bf n}}
\newcommand{\ov}{{\bf o}}
\newcommand{\pv}{{\bf p}}
\newcommand{\qv}{{\bf q}}
\newcommand{\rv}{{\bf r}}
\newcommand{\sv}{{\bf s}}
\newcommand{\tv}{{\bf t}}
\newcommand{\uv}{{\bf u}}
\newcommand{\wv}{{\bf w}}
\newcommand{\vv}{{\bf v}}
\newcommand{\xv}{{\bf x}}
\newcommand{\yv}{{\bf y}}
\newcommand{\zv}{{\bf z}}
\newcommand{\zerov}{{\bf 0}}
\newcommand{\onev}{{\bf 1}}

% Matrices

\newcommand{\Am}{{\bf A}}
\newcommand{\Bm}{{\bf B}}
\newcommand{\Cm}{{\bf C}}
\newcommand{\Dm}{{\bf D}}
\newcommand{\Em}{{\bf E}}
\newcommand{\Fm}{{\bf F}}
\newcommand{\Gm}{{\bf G}}
\newcommand{\Hm}{{\bf H}}
\newcommand{\Id}{{\bf I}}
\newcommand{\Jm}{{\bf J}}
\newcommand{\Km}{{\bf K}}
\newcommand{\Lm}{{\bf L}}
\newcommand{\Mm}{{\bf M}}
\newcommand{\Nm}{{\bf N}}
\newcommand{\Om}{{\bf O}}
\newcommand{\Pm}{{\bf P}}
\newcommand{\Qm}{{\bf Q}}
\newcommand{\Rm}{{\bf R}}
\newcommand{\Sm}{{\bf S}}
\newcommand{\Tm}{{\bf T}}
\newcommand{\Um}{{\bf U}}
\newcommand{\Wm}{{\bf W}}
\newcommand{\Vm}{{\bf V}}
\newcommand{\Xm}{{\bf X}}
\newcommand{\Ym}{{\bf Y}}
\newcommand{\Zm}{{\bf Z}}

% Calligraphic

\newcommand{\Ac}{{\cal A}}
\newcommand{\Bc}{{\cal B}}
\newcommand{\Cc}{{\cal C}}
\newcommand{\Dc}{{\cal D}}
\newcommand{\Ec}{{\cal E}}
\newcommand{\Fc}{{\cal F}}
\newcommand{\Gc}{{\cal G}}
\newcommand{\Hc}{{\cal H}}
\newcommand{\Ic}{{\cal I}}
\newcommand{\Jc}{{\cal J}}
\newcommand{\Kc}{{\cal K}}
\newcommand{\Lc}{{\cal L}}
\newcommand{\Mc}{{\cal M}}
\newcommand{\Nc}{{\cal N}}
\newcommand{\nc}{{\cal n}}
\newcommand{\Oc}{{\cal O}}
\newcommand{\Pc}{{\cal P}}
\newcommand{\Qc}{{\cal Q}}
\newcommand{\Rc}{{\cal R}}
\newcommand{\Sc}{{\cal S}}
\newcommand{\Tc}{{\cal T}}
\newcommand{\Uc}{{\cal U}}
\newcommand{\Wc}{{\cal W}}
\newcommand{\Vc}{{\cal V}}
\newcommand{\Xc}{{\cal X}}
\newcommand{\Yc}{{\cal Y}}
\newcommand{\Zc}{{\cal Z}}

% Bold greek letters

\newcommand{\alphav}{\hbox{\boldmath$\alpha$}}
\newcommand{\betav}{\hbox{\boldmath$\beta$}}
\newcommand{\gammav}{\hbox{\boldmath$\gamma$}}
\newcommand{\deltav}{\hbox{\boldmath$\delta$}}
\newcommand{\etav}{\hbox{\boldmath$\eta$}}
\newcommand{\lambdav}{\hbox{\boldmath$\lambda$}}
\newcommand{\epsilonv}{\hbox{\boldmath$\epsilon$}}
\newcommand{\nuv}{\hbox{\boldmath$\nu$}}
\newcommand{\muv}{\hbox{\boldmath$\mu$}}
\newcommand{\zetav}{\hbox{\boldmath$\zeta$}}
\newcommand{\phiv}{\hbox{\boldmath$\phi$}}
\newcommand{\psiv}{\hbox{\boldmath$\psi$}}
\newcommand{\thetav}{\hbox{\boldmath$\theta$}}
\newcommand{\tauv}{\hbox{\boldmath$\tau$}}
\newcommand{\omegav}{\hbox{\boldmath$\omega$}}
\newcommand{\xiv}{\hbox{\boldmath$\xi$}}
\newcommand{\sigmav}{\hbox{\boldmath$\sigma$}}
\newcommand{\piv}{\hbox{\boldmath$\pi$}}
\newcommand{\rhov}{\hbox{\boldmath$\rho$}}
\newcommand{\upsilonv}{\hbox{\boldmath$\upsilon$}}

\newcommand{\Gammam}{\hbox{\boldmath$\Gamma$}}
\newcommand{\Lambdam}{\hbox{\boldmath$\Lambda$}}
\newcommand{\Deltam}{\hbox{\boldmath$\Delta$}}
\newcommand{\Sigmam}{\hbox{\boldmath$\Sigma$}}
\newcommand{\Phim}{\hbox{\boldmath$\Phi$}}
\newcommand{\Pim}{\hbox{\boldmath$\Pi$}}
\newcommand{\Psim}{\hbox{\boldmath$\Psi$}}
\newcommand{\Thetam}{\hbox{\boldmath$\Theta$}}
\newcommand{\Omegam}{\hbox{\boldmath$\Omega$}}
\newcommand{\Xim}{\hbox{\boldmath$\Xi$}}


% Sans Serif small case

\newcommand{\Gsf}{{\sf G}}

\newcommand{\asf}{{\sf a}}
\newcommand{\bsf}{{\sf b}}
\newcommand{\csf}{{\sf c}}
\newcommand{\dsf}{{\sf d}}
\newcommand{\esf}{{\sf e}}
\newcommand{\fsf}{{\sf f}}
\newcommand{\gsf}{{\sf g}}
\newcommand{\hsf}{{\sf h}}
\newcommand{\isf}{{\sf i}}
\newcommand{\jsf}{{\sf j}}
\newcommand{\ksf}{{\sf k}}
\newcommand{\lsf}{{\sf l}}
\newcommand{\msf}{{\sf m}}
\newcommand{\nsf}{{\sf n}}
\newcommand{\osf}{{\sf o}}
\newcommand{\psf}{{\sf p}}
\newcommand{\qsf}{{\sf q}}
\newcommand{\rsf}{{\sf r}}
\newcommand{\ssf}{{\sf s}}
\newcommand{\tsf}{{\sf t}}
\newcommand{\usf}{{\sf u}}
\newcommand{\wsf}{{\sf w}}
\newcommand{\vsf}{{\sf v}}
\newcommand{\xsf}{{\sf x}}
\newcommand{\ysf}{{\sf y}}
\newcommand{\zsf}{{\sf z}}


% mixed symbols

\newcommand{\sinc}{{\hbox{sinc}}}
\newcommand{\diag}{{\hbox{diag}}}
\renewcommand{\det}{{\hbox{det}}}
\newcommand{\trace}{{\hbox{tr}}}
\newcommand{\sign}{{\hbox{sign}}}
\renewcommand{\arg}{{\hbox{arg}}}
\newcommand{\var}{{\hbox{var}}}
\newcommand{\cov}{{\hbox{cov}}}
\newcommand{\Ei}{{\rm E}_{\rm i}}
\renewcommand{\Re}{{\rm Re}}
\renewcommand{\Im}{{\rm Im}}
\newcommand{\eqdef}{\stackrel{\Delta}{=}}
\newcommand{\defines}{{\,\,\stackrel{\scriptscriptstyle \bigtriangleup}{=}\,\,}}
\newcommand{\<}{\left\langle}
\renewcommand{\>}{\right\rangle}
\newcommand{\herm}{{\sf H}}
\newcommand{\trasp}{{\sf T}}
\newcommand{\transp}{{\sf T}}
\renewcommand{\vec}{{\rm vec}}
\newcommand{\Psf}{{\sf P}}
\newcommand{\SINR}{{\sf SINR}}
\newcommand{\SNR}{{\sf SNR}}
\newcommand{\MMSE}{{\sf MMSE}}
\newcommand{\REF}{{\RED [REF]}}

% Markov chain
\usepackage{stmaryrd} % for \mkv 
\newcommand{\mkv}{-\!\!\!\!\minuso\!\!\!\!-}

% Colors

\newcommand{\RED}{\color[rgb]{1.00,0.10,0.10}}
\newcommand{\BLUE}{\color[rgb]{0,0,0.90}}
\newcommand{\GREEN}{\color[rgb]{0,0.80,0.20}}

%%%%%%%%%%%%%%%%%%%%%%%%%%%%%%%%%%%%%%%%%%
\usepackage{hyperref}
\hypersetup{
    bookmarks=true,         % show bookmarks bar?
    unicode=false,          % non-Latin characters in AcrobatÕs bookmarks
    pdftoolbar=true,        % show AcrobatÕs toolbar?
    pdfmenubar=true,        % show AcrobatÕs menu?
    pdffitwindow=false,     % window fit to page when opened
    pdfstartview={FitH},    % fits the width of the page to the window
%    pdftitle={My title},    % title
%    pdfauthor={Author},     % author
%    pdfsubject={Subject},   % subject of the document
%    pdfcreator={Creator},   % creator of the document
%    pdfproducer={Producer}, % producer of the document
%    pdfkeywords={keyword1} {key2} {key3}, % list of keywords
    pdfnewwindow=true,      % links in new window
    colorlinks=true,       % false: boxed links; true: colored links
    linkcolor=red,          % color of internal links (change box color with linkbordercolor)
    citecolor=green,        % color of links to bibliography
    filecolor=blue,      % color of file links
    urlcolor=blue           % color of external links
}
%%%%%%%%%%%%%%%%%%%%%%%%%%%%%%%%%%%%%%%%%%%




%%%%%%%%%%%%%%%%%%%%%%%%%%%%%%%%
% THEOREMS
%%%%%%%%%%%%%%%%%%%%%%%%%%%%%%%%
\theoremstyle{plain}
\newtheorem{theorem}{Theorem}[section]
\newtheorem{proposition}[theorem]{Proposition}
\newtheorem{lemma}[theorem]{Lemma}
\newtheorem{corollary}[theorem]{Corollary}

\theoremstyle{definition}
\newtheorem{definition}{Definition}

\theoremstyle{remark}
\newtheorem{remark}{Remark}[section]
\newtheorem{example}{Example}[section]

%%%%%%%%%%%%%%%%%%%%%%%%%%%%%%%
% layout
%%%%%%%%%%%%%%%%%%%%%%%%%%%%%%%

\newcommand{\mypar}[1]{{\textbf{#1}}}


\begin{document}
\title{Bandit Optimal Transport}

\maketitle
\begin{abstract}%
    Despite the impressive progress in statistical Optimal Transport (OT) in recent years, there has been little interest in the study of the \emph{sequential learning} of OT. Surprisingly so, as this problem is both practically motivated and a challenging extension of existing settings such as linear bandits. This article considers (for the first time) the stochastic bandit problem of learning to solve generic Kantorovich and entropic OT problems from repeated interactions when the marginals are known but the cost is unknown. We provide $\tilde{\mathcal O}(\sqrt{T})$ regret algorithms for both problems by extending linear bandits on Hilbert spaces. These results provide a reduction to infinite-dimensional linear bandits. To deal with the dimension, we provide a method to exploit the intrinsic regularity of the cost to learn, yielding corresponding regret bounds which interpolate between $\tilde{\mathcal O}(\sqrt{T})$ and $\tilde{\mathcal O}(T)$. 
\end{abstract}

\section{Introduction}


\begin{figure}[t]
\centering
\includegraphics[width=0.6\columnwidth]{figures/evaluation_desiderata_V5.pdf}
\vspace{-0.5cm}
\caption{\systemName is a platform for conducting realistic evaluations of code LLMs, collecting human preferences of coding models with real users, real tasks, and in realistic environments, aimed at addressing the limitations of existing evaluations.
}
\label{fig:motivation}
\end{figure}

\begin{figure*}[t]
\centering
\includegraphics[width=\textwidth]{figures/system_design_v2.png}
\caption{We introduce \systemName, a VSCode extension to collect human preferences of code directly in a developer's IDE. \systemName enables developers to use code completions from various models. The system comprises a) the interface in the user's IDE which presents paired completions to users (left), b) a sampling strategy that picks model pairs to reduce latency (right, top), and c) a prompting scheme that allows diverse LLMs to perform code completions with high fidelity.
Users can select between the top completion (green box) using \texttt{tab} or the bottom completion (blue box) using \texttt{shift+tab}.}
\label{fig:overview}
\end{figure*}

As model capabilities improve, large language models (LLMs) are increasingly integrated into user environments and workflows.
For example, software developers code with AI in integrated developer environments (IDEs)~\citep{peng2023impact}, doctors rely on notes generated through ambient listening~\citep{oberst2024science}, and lawyers consider case evidence identified by electronic discovery systems~\citep{yang2024beyond}.
Increasing deployment of models in productivity tools demands evaluation that more closely reflects real-world circumstances~\citep{hutchinson2022evaluation, saxon2024benchmarks, kapoor2024ai}.
While newer benchmarks and live platforms incorporate human feedback to capture real-world usage, they almost exclusively focus on evaluating LLMs in chat conversations~\citep{zheng2023judging,dubois2023alpacafarm,chiang2024chatbot, kirk2024the}.
Model evaluation must move beyond chat-based interactions and into specialized user environments.



 

In this work, we focus on evaluating LLM-based coding assistants. 
Despite the popularity of these tools---millions of developers use Github Copilot~\citep{Copilot}---existing
evaluations of the coding capabilities of new models exhibit multiple limitations (Figure~\ref{fig:motivation}, bottom).
Traditional ML benchmarks evaluate LLM capabilities by measuring how well a model can complete static, interview-style coding tasks~\citep{chen2021evaluating,austin2021program,jain2024livecodebench, white2024livebench} and lack \emph{real users}. 
User studies recruit real users to evaluate the effectiveness of LLMs as coding assistants, but are often limited to simple programming tasks as opposed to \emph{real tasks}~\citep{vaithilingam2022expectation,ross2023programmer, mozannar2024realhumaneval}.
Recent efforts to collect human feedback such as Chatbot Arena~\citep{chiang2024chatbot} are still removed from a \emph{realistic environment}, resulting in users and data that deviate from typical software development processes.
We introduce \systemName to address these limitations (Figure~\ref{fig:motivation}, top), and we describe our three main contributions below.


\textbf{We deploy \systemName in-the-wild to collect human preferences on code.} 
\systemName is a Visual Studio Code extension, collecting preferences directly in a developer's IDE within their actual workflow (Figure~\ref{fig:overview}).
\systemName provides developers with code completions, akin to the type of support provided by Github Copilot~\citep{Copilot}. 
Over the past 3 months, \systemName has served over~\completions suggestions from 10 state-of-the-art LLMs, 
gathering \sampleCount~votes from \userCount~users.
To collect user preferences,
\systemName presents a novel interface that shows users paired code completions from two different LLMs, which are determined based on a sampling strategy that aims to 
mitigate latency while preserving coverage across model comparisons.
Additionally, we devise a prompting scheme that allows a diverse set of models to perform code completions with high fidelity.
See Section~\ref{sec:system} and Section~\ref{sec:deployment} for details about system design and deployment respectively.



\textbf{We construct a leaderboard of user preferences and find notable differences from existing static benchmarks and human preference leaderboards.}
In general, we observe that smaller models seem to overperform in static benchmarks compared to our leaderboard, while performance among larger models is mixed (Section~\ref{sec:leaderboard_calculation}).
We attribute these differences to the fact that \systemName is exposed to users and tasks that differ drastically from code evaluations in the past. 
Our data spans 103 programming languages and 24 natural languages as well as a variety of real-world applications and code structures, while static benchmarks tend to focus on a specific programming and natural language and task (e.g. coding competition problems).
Additionally, while all of \systemName interactions contain code contexts and the majority involve infilling tasks, a much smaller fraction of Chatbot Arena's coding tasks contain code context, with infilling tasks appearing even more rarely. 
We analyze our data in depth in Section~\ref{subsec:comparison}.



\textbf{We derive new insights into user preferences of code by analyzing \systemName's diverse and distinct data distribution.}
We compare user preferences across different stratifications of input data (e.g., common versus rare languages) and observe which affect observed preferences most (Section~\ref{sec:analysis}).
For example, while user preferences stay relatively consistent across various programming languages, they differ drastically between different task categories (e.g. frontend/backend versus algorithm design).
We also observe variations in user preference due to different features related to code structure 
(e.g., context length and completion patterns).
We open-source \systemName and release a curated subset of code contexts.
Altogether, our results highlight the necessity of model evaluation in realistic and domain-specific settings.





We study (stochastic) gradient descent on the empirical risk
\begin{equation*}
\cL(w) = \frac{1}{n}\sum_{i=1}^n l(p_i(w))\, ,
\end{equation*}
where the loss function $l$ and the functions  $(p_i)_{i=1}^n$  are specified in the following assumptions. Note that the empirical risk for binary classification from Equation~\eqref{def:emp_risk_intro} is a special case of the above objective.

\begin{assumption}\label{hyp:loss_exp_log}\phantom{=}
  \begin{enumerate}[label=\roman*)]
    \item The loss is either the exponential loss, $l(q) = e^{-q}$, or the logistic loss, $l(q) = \log(1{+}e^{-q})$.
    \item There exists an integer $L \in \mathbb{N}^*$  such that, for all $1 \leq i \leq n$, the function $p_i$ is $L$-homogeneous\footnote{We recall that a mapping $f : \mathbb{R}^d \rightarrow \mathbb{R}$ is positively $L$-homogeneous if $f(\lambda w) = \lambda^L f(w)$ for all $w \in \mathbb{R}^d$ and $\lambda >0$.}, locally Lipschitz continuous and semialgebraic.
  \end{enumerate}
\end{assumption}
If the $p_i$'s were differentiable with respect to $w$, the chain rule would guarantee that
\begin{align*}
\nabla \mathcal{L}(w) = \frac{1}{n}\sum_{i=1}^n  l'(p_i(w)) \nabla p_i(w)\enspace.
\end{align*}
However, we only assume that the $p_i$'s are semialgebraic. While we could consider Clarke subgradients, the Clarke subgradient of operations on functions (e.g., addition, composition, and minimum) is only contained within the composition of the respective Clarke subgradients. This, as noted in Section~\ref{sec:cons_field}, implies that the output of backpropagation is usually not an element of a Clarke subgradient but a selection of some conservative set-valued field.
Consequently, for $1\leq i \leq n$, we consider $D_i : \bbR^d \rightrightarrows\bbR^d$, a conservative set-valued field of $p_i$, and a function $\sa_i : \bbR^d \rightarrow \bbR^d$ such that for all $w \in \bbR^d$, $\sa_i(w) \in D_i(w)$. Given a step-size $\gamma >0$, gradient descent (GD)\footnote{More precisely, this refers to conservative gradient descent. We use the term GD for simplicity, as conservative gradients behave similarly to standard gradients.} is then expressed as
\begin{equation*}\label{eq:gd_new}\tag{GD}
  w_{k+1} = w_k - \frac{\gamma}{n} \sum_{i=1}^n l'(p_i(w_k))\sa_i(w_k)\,.
\end{equation*}
For its stochastic counterpart, stochastic gradient descent (SGD), we fix a batch-size $1\leq n_b \leq n$. At each iteration $k \in \bbN$, we randomly and uniformly draw a batch $B_k \subset \{1, \ldots, n \}$ of size $n_b$. The update rule is then given by 
\begin{equation*}\label{eq:sgd_new}\tag{SGD}
  w_{k+1} = w_k -  \frac{\gamma}{n_b}\sum_{i\in B_k} l'(p_i(w_k)) \sa_i(w_k)\, .
\end{equation*}
The considered conservative set-valued fields will satisfy an Euler lemma-type assumption.
%\nic{Smoother transition}
\begin{assumption}\phantom{=}\label{hyp:conserv}
  For every $i \leq n$, $\sa_i$ is measurable and $D_i$ is semialgebraic. Moreover, for every $w \in \bbR^d$ and $\lambda \geq 0$, $\sa_i(w)  \in D_i(w)$,
  \begin{equation*}
    D_i(\lambda w) = \lambda^{L-1} D_i(w)\, , \textrm{ and } \quad   L p_i(w) = \scalarp{\sa_i(w)}{w}\, .
  \end{equation*}
\end{assumption}
%\nic{Smoother transition}
Having in mind the binary classification setting, in which $p_i(w) = y_i \Phi(x_i, w)$, we define the margin
\begin{equation}\label{def:marg}
  \sm: \bbR^d \rightarrow \bbR, \quad \sm(w) = \min_{1\leq i \leq n} p_i(w)\, .
\end{equation}
It quantifies the quality of a prediction rule $\Phi(\cdot, w)$. In particular,  the training data is perfectly separated when $\sm(w) >0$. A binary prediction for $x$ is given by the sign of $\Phi(x, w)$, and under the homogeneity assumption, it depends only on the normalized direction $w / \norm{w}$. Consequently, we will focus on the sequence of directions $u_k := w_k / \norm{w_k}$. Our final assumption ensures that the normalized directions $(u_k)$ have stabilized in a region where the training data is correctly classified.

\begin{assumption}\label{hyp:marg_lowb}
  Almost surely, $\liminf \sm(u_k) >0$.
\end{assumption}
Before presenting our main result, we comment on our assumptions.

\paragraph{On Assumption~\ref{hyp:loss_exp_log}.} As discussed in the introduction, the primary example we consider is when $p_i(w) = y_i \Phi(x_i;w)$ is the signed prediction of a feedforward neural network without biases and with piecewise linear activation functions on a labeled dataset $((x_i,y_i))_{i \leq n}$. In this case,
\begin{equation}\label{eq:NN}
 p_i(w) = y_i \Phi(w;x_i) = y_i V_L(W_L) \sigma(V_{L-1}(W_{L-1}) \sigma(V_{L-1}(W_{L-2}) \ldots \sigma(V_{1}(W_1 x_i))))\, ,
\end{equation}
where $w = [W_1, \ldots, W_L]$, $W_i$ represents the weights of the $i$-th layer, $V_i$ is a linear function in the space of matrices (with $V_i$ being the identity for fully-connected layers) and $\sigma$ is a coordinate-wise activation function such as $z \mapsto \max(0,z)$ ($\ReLU$), $z \mapsto \max(az, z)$ for a small parameter $a>0$ (LeakyReLu) or $z \mapsto z$. Note that the mapping $w \mapsto p_i(w)$ is semialgebraic and $L$-homogeneous for any of these activation functions. Regarding the loss functions, the logistic and exponential losses are among the most commonly studied and widely used. In Appendix~\ref{app:gen_sett}, we extend our results to a broader class of losses, including $l(q) = e^{-q^a}$ and $l(q) = \ln (1 + e^{-q^a})$ for any $a \geq 1$.

\paragraph{On Assumption~\ref{hyp:conserv}.} Assumption~\ref{hyp:conserv} holds automatically  if $D_i$ is the Clarke subgradient of $p_i$. Indeed, at any vector $w \in \bbR^d$, where $p_i$ is differentiable it holds that $p_i(\lambda w) = \lambda^{L} p_i(w)$. Differentiating relatively to $w$ and $\lambda$ (noting that $p_i$ remains differentiable at $\lambda w$ due to homogeneity), we obtain $\lambda \nabla p_i(\lambda w) = \lambda^{L} \nabla p_i(w)$ and $\scalarp{\nabla p_i(\lambda w)}{w} = L \lambda^{L-1} p_i(w)$. The expression for any element of the Clarke subgradient then follows from~\eqref{eq:def_clarke}. 

However, for an arbitrary conservative set-valued field, Assumption~\ref{hyp:conserv} does not necessarily hold. For instance, $D(x) = \mathds{1}(x \in \mathbb{N})$ is a conservative set-valued field for $p \equiv 0$, which does not satisfy Assumption~\ref{hyp:conserv}. Nevertheless, in practice, conservative set-valued fields naturally arise from a formal application of the chain rule. For a non-smooth but homogeneous activation function $\sigma$, one selects an element $e \in \partial \sigma (0)$, and computes $\sa_i(w)$ via backpropagation. Whenever a gradient candidate of $\sigma$ at zero is required (i.e., in~\eqref{eq:NN}, for some $j$, $V_j(W_j)$ contains a zero entry), it is replaced by $e$. 
Since $V_j(W_j)$ and $V_j(\lambda W_j)$ have the same zero elements, it follows that for every such $w$, $
\sa_i(\lambda w) = \lambda^L \sa_i(w)$. The conservative set-valued field $D_i$ is then obtained by associating to each $w$ the set of all possible outcomes of the chain rule, with $e$ ranging over all elements of $\partial \sigma(0)$. Thus, for such fields, Assumption~\ref{hyp:conserv} holds.


\paragraph{On Assumption~\ref{hyp:marg_lowb}.} Training typically continues even after the training error reaches zero.
Assumption~\ref{hyp:marg_lowb} characterizes this late-training phase, where our result applies. 
As noted earlier, since $\sm$ is $L$-homogeneous, the classification rule is determined by the direction of the  iterates $u_k=w_k/\norm{w_k}$. Assumption~\ref{hyp:marg_lowb} then states that, beyond some iteration, the normalized margin remains positive. 
This assumption is natural in the context of studying the implicit bias of SGD: we \emph{assume} that we reached the phase in which the dataset is correctly classified and \emph{then} characterize the limit points. A similar perspective was taken in  \cite{nacson2019lexicographic}, where the implicit bias of GF was analyzed under the assumption that the sequence of directions and the loss converge. However, unlike their approach, ours does not require assuming such convergence a priori.

Earlier works such as \cite{ji2020directional,Lyu_Li_maxmargin}, which analyze subgradient flow or smooth GD, establish convergence by assuming the existence of a single iterate $w_{k_0}$ satisfying $\sm(w_{k_0}) > \varepsilon$ and then proving that $\lim \sm(u_{k}) > 0$. Their approach relies on constructing a smooth approximation of the margin, which increases during training, ensuring that $\sm(u_k) > 0$ for all iterates with $k \geq k_0$. This is feasible in their setting, as they study either subgradient flow or GD with smooth $p_i$’s, allowing them to leverage the descent lemma.

In contrast, our analysis considers a nonsmooth and stochastic setting, in which, even if an iterate $w_{k_0}$ satisfying $\sm(w_{k_0}) > \varepsilon$ exists, there is no a priori assurance that subsequent iterates remain in the region where Assumption~\ref{hyp:marg_lowb} holds. From this perspective, Assumption~\ref{hyp:marg_lowb} can be viewed as a stability assumption, ensuring that iterates continue to classify the dataset correctly. Establishing stability for stochastic and nonsmooth algorithms is notoriously hard, and only partial results in restrictive settings exist \cite{borkar2000ode,ramaswamy2017generalization,josz2024global}.

%Finally, note that Assumption~\ref{hyp:marg_lowb} only needs to hold almost surely. Specifically, with probability 1, there exist $k_0$ and $\varepsilon$ such that for all $k \geq k_0$, $\sm(u_k) \geq \varepsilon > 0$. In the case of~\eqref{eq:sgd_new}, $k_0$ and $\delta$ are random variables and may take different values across different realizations. 

%\paragraph{On constant stepsizes.}
%We allow the step size to be a constant of arbitrary magnitude, subject to the stability Assumption~\ref{hyp:marg_lowb}. This may seem surprising in a nonsmooth and stochastic setting, where a vanishing step size is typically required to ensure convergence (see, e.g., \cite{majewski2018analysis, dav-dru-kak-lee-19, bolte2023subgradient, le2024nonsmooth}).

\section{Challenges, related work, and contributions}\label{sec:RWCC}
Giving an exhaustive account of the vast literature of Optimal Transport would be outside the scope of this article. As it focuses on aspects of online learning, we will limit our attention to this narrow view. Nevertheless, we provide the curious reader a modest bibliography in \cref{app: biblio}.

\subsection{On optimal transport and learning}\label{subsec: OT and learning RW}
        \paragraph{Estimation of OT functionals}
        Much of the early work in statistical OT focused on estimating the value of the functional $\kant(\mu,\nu,c)$ when $(\mu,\nu)$ are unknown, but $c$ is known and highly regular, e.g.\ \citep{horowitz_mean_1994,weed_sharp_2019}. These regularity assumptions are motivated by the study of Wasserstein distances between probability measures (i.e.\ $c=\norm{\cdot-\cdot}^p$, $p\ge1$) via sampling. With the increased interest in the entropic OT problem, many works have asked the same questions about $\ent(\mu,\nu,c,\ve)$, e.g.\ \citep{rigollet_sample_2022,stromme_minimum_2024}.   

        This line of work is orthogonal to our investigation, as we know $(\mu,\nu)$ but not $c^*$. The critical object in this line of work is the regularity structure of $\kant(\mu,\cdot,c)$, when $c$ is strongly regular. For our problem, the relevant geometry is that of the transport functional $\pi\in\Pi(\mu,\nu)\mapsto \langle c^*\vert \pi\rangle$.

        \paragraph{Online matchings}
        Concurrently, Matching (discrete marginal OT), has been actively studied by computer scientists and economists. These works, such as \citep{perrot_mapping_2016}, are often directly inspired by applications, and have yielded many creative extensions to the OT problem:~\cite{alon_learning_2004} aims to learn an optimal matching using queries to an oracle;~\cite{johari_matching_2021} to identify \emph{types} of nodes;~\cite{min_learn_2022} to design a welfare-maximising social planner; etc.

        The common thread amongst these works is the nature of the \emph{market} on which they work: at each time $t$, a new supply becomes available to \emph{match} (i.e.\ transport from), and the agent must decide to which of its available demands to transport it. This decision problem is fundamentally different from our repeated OT problem as mistakes in the matching are permanent, while we replay a whole matching at each step.
        Furthermore, the information structure is different. \cite{jagadeesan_learning_2021,sentenac_pure_2021,sentenac_learning_2023} (amongst others) have highlighted that this problem is a combinatorial semi-bandit problem, in which there is feedback about each connection made. In our problem the agent receives feedback only about the matching as a whole (full bandit). These two differences make the problems seem superficially similar, but they are fundamentally different.
                          
        \paragraph{Online Learning to Transport}
        The first paper to take interest in online learning of optimal transport itself appears to be \citep{guo_online_2022-1}. In this article, the authors take an Online Convex Optimisation (OCO) approach to the problem, meaning that an adversary chooses a cost function $c_t$ at each round $t$ from a class of suitably regular (convex) functions. The learner aims to choose a sequence of transport plans $\pi_t$ which has a small regret with respect to the best fixed transport plan in hindsight. While this work pioneered the study of online (repeated) optimal transport, there are no direct reductions between this paper and their work.

        Most of the work of~\cite{guo_online_2022-1} is done under a full-information adversarial setting (as is typical in OCO): the transport problem changes at each round and is completely revealed after a coupling $\pi_t$ is played. However, in section 3, the authors provide a $0$-order semi-bandit scheme based on a discretisation of $\state$. In contrast, our work is directed at a stochastic setting under complete bandit feedback (only $\int c\de \pi_t$ is observed, with some noise).
        
        Due to the use of OCO techniques, as well as PDE-based optimal transport tools based on the work of~\cite{brenier_least_1989}, the results of~\cite{guo_online_2022-1} are only valid under strong assumptions on the regularity of the cost functional (and thus the cost function) and the marginals. In contrast, we work without specific assumptions on the cost function and marginals, beyond the minimal ones for~\eqref{eq: kantorovich def} to be well-defined. This difference arises because they consider general functionals on the Wasserstein space, while our work focuses on the specific regularity of OT functionals.

        This work was followed by~\cite{zhu_semidiscrete_2023} which considers the first online learning problem in semi-discrete optimal transport (i.e.\ $\mu$ discrete, $\nu$ continuous). They construct a semi-myopic algorithm with forced exploration which can learn to behave as the optimal plan from samples of the continuous marginal. Unfortunately, they do not study a general problem but rather only the case in which the cost $c^*$ is a linear parametric model. This choice obfuscates a large part of the complexity of the general problem and dilutes any insights about the geometry of the problem.
        Moreover, \cite{zhu_semidiscrete_2023} do not provide direct regret bounds, but rather performance metrics which may be converted into regret bounds. Sadly, these metrics fail to generalise to the continuous marginal case, and their analysis breaks down in the general setting.

\subsection{Bandit Algorithms}

As \cref{subsec: OT and learning RW} shows, bandits and optimal transport have been in peripheral contact repeatedly. Nevertheless, despite its interest in many optimisation problems, the bandit literature has remained  uninterested in the general optimal transport problem. Still, let us highlight the key elements of this theory on which we can build to solve the bandit optimal transport problem. 

\mypar{Multi-armed bandits}. The classical bandit problem \citep{thompson1933likelihood,lai_asymptotically_1985,auer_finite-time_2002} considered the issue of choosing the best amongst a finite set of arms based on bandit feedback about arm rewards. Since then, bandit theorists have taken some interest in higher-dimensional optimisation problems either linear or non-linear. 
For instance, \cite{tran-thanh_functional_2014} show regret bounds for learning a general functional using bandit feedback but sadly still considers only finitely many arms. While a general theory of bandits for functionals remains elusive \citep{wang_beyond_2022}, bandits under weak assumptions on the set of arms have been studied.

\mypar{Lipschitz bandits}. Several papers \citep{bubeck_lipschitz_2011,magureanu_lipschitz_2014,kleinberg_bandits_2019} have leveraged Lipschitz reward functions to provide regret bounds and algorithms, even on arbitrary metric spaces. Unfortunately, the bounds for general Lipschitz functions using these methodology are of the order of $\Theta(T^{{(d+1)}/{(d+2)}})$, in dimension $d\in\Nb$ \citep{kleinberg_bandits_2019}. In the case of the continuous optimal transport problem, this dimension is infinite, and the regret bounds become vacuous. The infinite dimensional nature of our problem also prevents the practical usability of most discretisations, even sophisticated ones like the tree-based scheme of~\cite{bubeck2011X-armed}. 

\mypar{Linear bandits}. In the hope of circumventing this problem, we can take inspiration from Kantorovich and recall that~\eqref{eq: kantorovich def} is linear program. Indeed, linear functions have much stronger global regularity than Lipschitz ones, meaning that linear bandits may escape vacuity even when $d=+\infty$.   

The setting of linear bandits was introduced by~\cite{auer_using_2003}, and refined by many subsequent works \citep{abeille_linear_2017,vernade_linear_2020,hao_high-dimensional_2020}, most notably for us~\cite{abbasi-yadkori_improved_2011}. In his doctoral thesis, Y.~\cite{abbasi-yadkori_online_2012} includes a version of this article in which the technical results are given not just for $\Rb^d$, but for an arbitrary Hilbert space. These works all use the celebrated Optimism in the Face of Uncertainty (OFU) principle to tackle the previously mentioned exploration-exploitation dilemma.

Nevertheless, in spite of its generality,~\cite{abbasi-yadkori_online_2012} is not sufficient to solve the bandit optimal transport problem, because the action space of our bandit is not a Hilbert space, and in fact the actions do not live in the same space as $c^*$. This fundamentally breaks the assumptions of this work, in spite of the fact that the duality product $\langle c\vert \pi \rangle$ defining $\kant$ is a linear form. 

\mypar{Kernel bandits.} Kernel methods intrinsically consider infinite-dimensional linear rewards, and may appear, at first, an ideal solution for solving bandit optimal transport. Kernel bandits have seen extensive work \citep{chowdhury_kernelized_2017,janz_bandit_2020,takemori_approximation_2021},  including~\cite{valko_finite-time_2013} which comes closest to our approach by introducing a kernelised OFU algorithm. These methods posit a particular structure for the reward function $c^*$, and then use the representer theorem to reduce the problem to a linear problem. Our problem, in contrast, is already linear so it should not require any such assumptions. 

One place where kernel methods shine is in making infinite-dimensional problems computationally tractable. While they can be used for this purpose in our setting, we will show that we can obtain similar bounds directly from the regularity of $c^*$ without assuming an RKHS structure. 

  
\subsection{Challenges and contributions}

\paragraph{Challenges} The specificities and challenges of the general BOT problem, can be summarised in three main points. 

\textbf{A)} The actions of this bandit problem are probability measures. In the discrete optimal transport (matching) problems previously studied in the literature, probability measures remain finite dimensional and can be represented using an inner product. This hides the true complexity of the general case in which one must confront a continuum of infinite-dimensional actions which require sophisticated tools to analyse. Moreover, this is compounded by the fact that the space of probability measures has a difficult geometry. 

\textbf{B)} The cost function $c^*$, which plays the role of a ``parameter'' to estimate, is a continuous function. Since the optimal transport problem only requires minimal integrability assumptions on $c^*$, the natural hypothesis classes for $c^*$ will be large function spaces\footnote{Circumventing this difficulty by parametrising $c^*$ as in~\cite{zhu_semidiscrete_2023} would dilute any insight about the geometry of the problem.} such as $L^2$.  This creates a significant difficulty for estimation and thus for bandit algorithms based on least-squares. The construction of estimators and confidence sets that permit the use of OFU algorithms is  challenged by the infinite dimensionality of $c^*$. 

\textbf{C)} Even if estimators for $c^*$ can be constructed, they must face the infinite-dimensionality of $c^*$. This raises the challenge of efficient approximation of infinite-dimensional estimators under weak assumptions, and of their associated regrets.

\paragraph{Contributions} This paper is the first study of the general stochastic bandit optimal transport problem. It provides a general framework for further work in this area, by showing that the problem is learnable under weak assumptions. Beyond this, the technical contributions can be summarised as follows.

\textbf{1)} To overcome challenge \textbf{A}, we construct a phase-space representation of the optimal transport problem which allows us to transform the problem into a linear bandit on a Hilbert space. This is enabled by the regularity of the entropic problem and tools from the Fourier analysis of measures.  

\textbf{2)} Combining \textbf{1} with the framework of~\cite{abbasi-yadkori_online_2012} we are able to construct the necessary confidence sets and estimators to estimate $c^*$ and address challenge \textbf{B}. By regularising optimism by entropy, and using the dual problem of~\eqref{eq: entropic OT def}, we are able to ensure our algorithm maintains the validity of the phase-space representation as it learns, unlocking regret analysis. In the regret analysis, we leverage the regularity of the entropic problem to prove bounds on the Kantorovich regret through the entropic one.

\textbf{3)} To face the infinite-dimensional quantities which arise in the general regret bounds, we construct a general estimation method based on the regularity of the cost function. This method addresses challenge \textbf{C} by allowing us to obtain regret bounds of order $\tilde\Oc(\sqrt{T})$ in simple cases, and an interpolation up to $\tilde\Oc(T)$ dependent on the regularity of $c^*$. 


\section{Introduction}
\label{sec:introduction}
Discrete-time (DT) linear parameter-varying (LPV) input-output (IO) models \cite{Shamma2012, Toth2012} are a powerful model class in identification for capturing time and parameter-related system variations, e.g., due to varying operating conditions. In LPV models, the signal relations are linear, but the coefficients describing these relations are functions of a time-varying scheduling signal $p$ that is assumed to be measurable online. Specifically, an DT-LPV-IO model is represented by a higher-order linear difference equation in terms of previous inputs and outputs, and the coefficients in this difference equation are a function of $p$. The resulting parameter-varying behavior can also embed certain nonlinear characteristics under the correct choice of $p$ \cite{Leith2000a}. 
Data-driven system identification methods using LPV-IO models has been thoroughly developed in the last decades \cite{Bachnas2014,Bamieh2000,Bamieh2002, Previdi2003, Butcher2008, Laurain2010, Mercere2011a, Zhao2012a}.

A state-space (SS) realization of these LPV-IO models enables analysis and controller design using Lyapunov/dissipativity theory\cite{Shamma2012, LMIsLectureNotes,Caverly2019}. For example, such a SS realization can be used to compute the maximum energy amplification ($\ell_2$-gain) of the model using the LPV bounded-real lemma \cite{Apkarian1995}. Such a state-space realization can be either minimal or nonminimal, in which a realization is minimal if there exists no realization with a smaller state dimension \cite{Silverman1969, Gohberg1992}. In a minimal SS realization, all states are relevant for the IO behavior of the SS realization, i.e., all directions in the state space are observable and reachable for some scheduling signal $p$ \cite{Gohberg1992}.
In a nonminimal realization, there are directions in the state space that are irrelevant for the IO behavior of the SS realization, which can complicate analysis and control if these directions are marginally stable/unstable. Consequently, minimal realizations are often preferred over nonminimal realizations. In the linear time-invariant (LTI) setting, minimal realizations can straightforwardly be obtained from an identified transfer function \cite{DeSchutter2000}.

In the LPV case, obtaining a minimal state-space realization of an LPV-IO models alters the way in which the coefficient functions of the realization depends on the scheduling, complicating analysis and design based on this minimal realization \cite{Wollnack2015}. Furthermore, obtaining a minimal realization is complex from an algorithmic and computational point of view. Specifically, minimal realizations can be computed by using canonical state constructions \cite{Guidorzi2003, Weiss2005, Toth2007} or by using the behavioral approach \cite{Toth2010}. For both approaches, the resulting state-space realization generally has coefficient functions that are nonlinear functions of the coefficient functions of the original IO model. Furthermore, the SS coefficient functions generally depend not only on the current value of the scheduling signal $p_k$, but also on its shifted values, e.g., $p_{k+1}$, referred to as dynamic dependence. This nonlinear and dynamic dependence severely complicates the analysis and controller design based on a minimal realization \cite{Wollnack2015}. Only in special cases can these effects be avoided \cite{Toth2012b}. 

In contrast, a SS realization that is potentially nonminimal can directly be constructed from an LPV-IO model without altering the scheduling dependency \cite{Toth2013, Goodwin1984,Pearson2004, DePersis2020}. 
This construction uses the delayed outputs and inputs as a state basis and the resulting realization can directly be specified in terms of the coefficient functions of the LPV-IO model, avoiding the nonlinear and dynamic dependency introduced in minimal realizations.
However, this realization is not guaranteed to be minimal. 

Even though there exist methods to convert LPV-IO models to an LPV-SS realization, at present IO models obtained through identification cannot straightforwardly be used for stability analysis and controller design. The goal of this paper is to illustrate that the direct nonminimal realization can be used for this purpose, and that the nonminimality of this realization is usually not a problem for analysis and control.

The main contribution of this paper is an analysis of the direct nonminimal state-space realization in terms of its reachability and observability properties. This is achieved through the following subcontributions.
\begin{itemize}[topsep=0pt,noitemsep]
    \item[C1] A sufficient condition is derived under which the direct realization is reachable. Namely, that it is reachable for constant scheduling trajectories under a coprimeness condition on the IO coefficient functions and a well-posedness condition on the representation (Section \ref{sec:reachability}).
    \item[C2] It is illustrated that the direct realization is always unobservable, i.e., that there exist states that cannot be observed in the output. Next, it is shown that these unobservable states always decay to zero in finite amount of steps, i.e., the realization is reconstructible, as no unobservable unstable states are introduced (Section \ref{sec:observability}).
\end{itemize}
The developed theory is exemplified through examples. 
%However, in the LTI case, minimal realizations can straightforwardly be computed without any of the drawbacks that occur in the LPV case \cite{DeSchutter2000}.

It is highlighted that these results are also important for data-driven control in both the LPV as well as the LTI setting, especially for behavioral type of approaches relying state realizations consisting of shifted inputs and outputs \cite{Coulson2019, Dai2023, DePersis2020, Park2009, VanWaarde2022, Verlioek2021}. Specifically, the studied direct nonminimal state-space realization is often used in these methods to characterize the behavior of the system through data, and subsequently analyze stability/dissipativity properties and design controllers directly from data. The reachability and observability properties of this state-space realization play an important role in this field as these are intimately coupled to rank conditions necessary for controller synthesis. As such, a clear understanding of the source of nonminimality is needed to avoid possible misinterpretations, as is shown in the examples.

\subsubsection*{Notation} $z_k \in \Ree^{n_z}$ denotes the value of the signal $z: \Zee \mapsto \Ree^{n_z}$ at time index $k\in\Zee$. $\signalspaceDT$ denotes the set of all signals that are finite at all times, i.e., $\signalspaceDT = \{z \ | \ \norm{z_k} < \infty \ \forall  k \in\Zee \}$. Given a matrix $A\in\Ree^{m\times n}$, $A^\dag$ denotes its pseudoinverse, $\ker\ A$ denotes its kernel (or nullspace), and $(\ker\ A)^\perp$ the orthogonal complement of $\ker\ A$. Given those matrices $A,B$, the block diagonal combination $\begin{bmatrix} A & 0 \\ 0 & B
\end{bmatrix}$ is denoted by $\blkdiag(A,B)$.

\section{Problem Formulation}
\label{sec:problem_formulation}
Consider the multi-input multi-output DT-LPV-IO model $G$ mapping input signal $u: \Zee \mapsto \mathbb{R}^{n_u}$ to output signal $y: \Zee \mapsto \Ree^{n_y}$ according to the difference equation
\begin{equation}
    G : \ y_k = -\sum_{i=1}^{n_a} A_i(p_k) y_{k-i} + \sum_{i=0}^{n_b-1} B_i(p_k) u_{k-i},\label{eq:LPV_IO_DT}
\end{equation}
with coefficient functions $A_i: \Pee \mapsto \Ree^{n_y \times n_y}$, $B_i: \Pee \mapsto \Ree^{n_y\times n_u}$ describing the dependency of \eqref{eq:LPV_IO_DT} on the scheduling signal $p:\Zee \mapsto \Pee \subseteq \Ree^{n_p}$, and $n_a \geq 0, n_b \geq 1$ the order of the IO representation, with $n_a=0$ representing a model in which $y_k$ does not depend on $y_{k-i}$.
%Such an LPV-IO representation is often used in data-based identification of systems \cite{lots of citations, previdi, Zhao2012, Toth book, chemical thing} 

For stability/dissipativity analysis and controller/observer synthesis using a model in the form of \eqref{eq:LPV_IO_DT}, it is desired to obtain a state-space realization of \eqref{eq:LPV_IO_DT}. A nonminimal SS realization of \eqref{eq:LPV_IO_DT} can directly be constructed by setting the state equal to the delayed inputs and outputs. Specifically in case $n_a > 0, n_b > 1$, the state is defined as
\begin{subequations}
 \label{eq:maximum_ss_state_def}
 \begin{align}
    x_k &= \begin{bmatrix} \bar{y}_k^\T & \bar{u}_k^\T \end{bmatrix}^\T \in \mathbb{R}^{n_x}, \\
    \bar{y}_k &= \begin{bmatrix} y_{k-1}^\T & \hdots & y_{k-n_a}^\T \end{bmatrix}^\T \in \mathbb{R}^{n_y n_a}, \\
    \bar{u}_k &= \begin{bmatrix} u_{k-1}^\T & \hdots & u_{k-n_b+1}^\T \end{bmatrix}^\T \in \mathbb{R}^{n_u (n_b-1)}.
\end{align}
\end{subequations}
with $n_x = n_y n_a + n_u (n_b -1)$. Given this state vector, the evolution of \eqref{eq:LPV_IO_DT} can be written as
\begin{subequations}
\label{eq:maximum_state_space}
\begin{equation}
\label{eq:maximum_state_space_state}
\resizebox{\textwidth}{!}{$
    x_{k+1}
    \!=\!
    \left[
    \begin{array}{ccccc|ccccc}
        \!-A_1(p_k) \! & \! -A_2(p_k) \! & \! \hdots \! & \! -A_{n_a-1}(p_k)  \! &  \! -A_{n_a}(p_k) & B_1(p_k) \! & \! B_2(p_k) \! & \! \hdots \! & \! B_{n_b-2}(p_k) \! & \! B_{n_b-1}(p_k) \! \\
        I &  & & 0 & 0 & 0 & 0 & \hdots & 0 & 0\\
         & I &  &  & 0 & 0 & 0 & \hdots & 0 & 0\\
        & & \ddots  &  & \vdots & \vdots & \vdots & & \vdots & \vdots \\
        0 &  & & I & 0 & 0 & 0 & \hdots & 0 & 0\\
        \hline
        0 & 0 & \hdots & 0 & 0 & 0 & 0 & \hdots & 0 & 0\\
        0 & 0 & \hdots & 0 & 0 & I &  &  & 0 & 0\\
        0 & 0 & \hdots & 0 & 0 &  & I &  &  & 0\\
        \vdots & \vdots & & \vdots & \vdots & & & \ddots & & \vdots \\
        0 & 0 & \hdots & 0 & 0 & 0 & & & I & 0
    \end{array}
    \right] \!
    x_k
    \! + \!
    \left[
    \begin{array}{c}
        \! B_0(p_k) \! \\
        0 \\
        0 \\
        \vdots \\
        0 \\
        \hline
        I \\
        0 \\
        0 \\
        \vdots \\
        0
    \end{array}
    \right] \!
    u_k,
    $}
\end{equation}
\begin{equation} 
\label{eq:maximum_state_space_output}
\resizebox{\textwidth}{!}{$
    y_k\! =\! \left[ \begin{array}{ccccc|ccccc} 
        \! -A_1(p_k) \! & \! -A_2(p_k) \! & \! \hdots \! & \! -A_{n_a-1}(p_k) \! & \! -A_{n_a}(p_k) \! & \! B_1(p_k) \! & \! B_2(p_k) \! & \! \hdots \! & \! B_{n_b-2}(p_k) \! & \! B_{n_b-1}(p_k) \!
    \end{array}\right] \! x_k
    \! + \! B_0(p_k) u_k,
    $}
\end{equation}
\end{subequations}
thus directly obtaining an SS realization of \eqref{eq:LPV_IO_DT}. 
State-space realization \eqref{eq:maximum_state_space} can be compactly written as
\begin{subequations}
    \label{eq:maximum_ss_ABCD}
    \begin{align}\label{eq:maximum_ss_FcalGcal}
        \left[\begin{array}{c}
            \bar{y}_{k+1} \\
            \bar{u}_{k+1}
        \end{array}\right]
        &= 
        \left[\begin{array}{c|c}
            F_a - G_a A(p_k) & G_a B(p_k) \\
            \hline \hspace{-5pt} 
            0 & F_b
        \end{array}\right] 
        \left[\begin{array}{c}
            \bar{y}_{k} \\
            \bar{u}_{k}
        \end{array}\right]
        + 
        \left[\begin{array}{c}
        G_a B_0 (p_k) \\
        \hline
        G_b
        \end{array}\right]
        u_k
        = \Fcal(p_k) x_k + \Gcal(p_k) u_k, \\
        % +
        % \left[\begin{array}{c}
        %     G \\ \hline 0
        % \end{array}\right]
        % \left[\begin{array}{c|c}
        %      -K(p) & L(p)
        % \end{array}\right] 
        % &
        % \Bcal(p) &= \left[\begin{array}{c}
        % G_a B_0 (p) \\
        % \hline
        % G_b
        % \end{array}\right] \nonumber \\
        y_k &= \left[\begin{array}{c|c}
             -A(p_k) & B(p_k)
        \end{array}\right]
        x_k + B_0(p_k) u_k = \Hcal(p_k) x_k + \Jcal(p_k) u_k,
        \end{align}
\end{subequations}
with
\begin{subequations}
\begin{align}
    A &= \begin{bmatrix} A_1 & \hdots & A_{n_a-1} & A_{n_a} \end{bmatrix}: \Pee \mapsto \mathbb{R}^{n_y \times n_y n_a}, \label{eq:maximum_ss_A} \\
    B &= \begin{bmatrix} B_1 &  \hdots & B_{n_b-2} & B_{n_b-1} \end{bmatrix}: \Pee \in \mathbb{R}^{n_y \times n_u (n_b-1)}, \label{eq:maximum_ss_B}
\end{align}
\end{subequations}
and
\begin{subequations}
\label{eq:maximum_ss_FGFaGaFbGb}
\begin{align}
    F_a &= \begin{bmatrix}
        0 & 0_{n_y \times n_y} \\ %_{n_y \times ny(n_a-1)}
        I_{n_y(n_a-1)} & 0
    \end{bmatrix} \in \mathbb{R}^{n_y n_a \times n_y n_a}
    &  &
    G_a = \begin{bmatrix}
        I_{n_y} \\ 0_{ n_y (n_a -1) \times n_y}
    \end{bmatrix}
    \in \mathbb{R}^{n_y n_a \times n_y}, 
    \label{eq:maximum_ss_Fa_Ga}
    \\
    F_b &= \begin{bmatrix}
        0 & 0_{n_u \times n_u} \\ %_{n_u \times n_u(n_b-2)}
        I_{n_u(n_b-2)} & 0
    \end{bmatrix} \in \mathbb{R}^{n_u(n_b-1) \times n_u(n_b -1)}
    & &
    G_b = \begin{bmatrix}
        I_{n_u} \\ 0_{n_u(n_b-2) \times n_u}
    \end{bmatrix}
    \in \mathbb{R}^{n_u(n_b-1) \times n_u},
    \label{eq:maximum_ss_Fb_Gb}
\end{align}
\end{subequations}
Even though realization \eqref{eq:maximum_state_space} can directly be constructed, it might contain more states than necessary, i.e., it might be a nonminimal realization. This can happen through two ways\footnote{Minimality in the Kalman sense is considered. If a realization is desired that also captures unreachable but observable autonomous behavior, then only the second mechanism is relevant.}: 1) it can include state directions that can never be reached, i.e., the state-space contains an unreachable subspace, and 2) it can include state directions that can never be observed, i.e., the state-space contains an unobservable subspace. 

The goal of this paper is to analyze observability and reachability properties \eqref{eq:maximum_state_space}, which are formally defined in the next Sections. Specifically, the goal is to provide numerical conditions in terms of the coefficient functions $A_i,B_i$ that imply observability and reachability.

Next to the general case in which $n_a>0,n_b>1$, also observability/reachability in the special cases that $n_a=0,n_b>1$ or $n_a>0,n_b=1$ are analyzed in this paper. In these cases, a state-space realization of \eqref{eq:LPV_IO_DT} is obtained by deleting the irrelevant blocks in \eqref{eq:maximum_state_space}. Specifically, in case $n_a=0$, \eqref{eq:LPV_IO_DT} reduces to an LPV finite-impulse response (FIR), for which a state-space realization of \eqref{eq:LPV_IO_DT} is given by
\begin{align} \label{eq:maximum_state_space_FIR}
    \bar{u}_{k+1} = F_b \bar{u}_k + G_b u_k  & & y_k = B(p_k) \bar{u}_k + B_0(p_k) u_k.
\end{align}
In case $n_b = 1$, IO representation \eqref{eq:LPV_IO_DT} reduces to an inverse LPV-FIR filter. A state-space realization of \eqref{eq:LPV_IO_DT} in this setting is given by
\begin{align} \label{eq:maximum_state_space_invFIR}
    \bar{y}_{k+1} = (F_a - G_a A(p_k)) \bar{y}_k + G_a B_0(p_k) u_k & & y_k = -A(p_k) \bar{y}_k + B_0(p_k) u_k,
\end{align}
with $A$ and $F_a,G_a$ as in \eqref{eq:maximum_ss_A} and \eqref{eq:maximum_ss_Fa_Ga}. 

\section{Reachability}
\label{sec:reachability}
In this section, a condition for reachability of the realization \eqref{eq:maximum_state_space} is derived in terms of the coefficient functions $A_i,B_i$, constituting contribution C1. 

Intuitively, reachability states that any point in the state space can be reached by an appropriate input. It is formally defined in the LPV case as follows \cite{Silverman1967, Gohberg1992, Toth2010}.

\begin{definition}
\label{def:reachability_DT}
    State-space realization \eqref{eq:maximum_state_space} is said to be structurally $k$-reachable if there exists a scheduling signal $p\in\signalspaceDT$ such that for any initial state $x_{k_1} \in \Xee$ at an arbitrary time $k_1$ and any final state $x_{k_2}\in\Xee$, $x_{k_1}$ can be steered to $x_{k_2}$ in $k$ steps. It is said to be completely $k$-reachable if this steering of the state is possible for all $p\in\signalspaceDT$.
\end{definition}
This definition of reachability mostly aligns with the LTI definition \cite[Section 6.1]{Chen1999}, but differs in two ways. First, Definition \ref{def:reachability_DT} discerns between structural and complete reachability. This stems from the fact that reachability might be lost for some scheduling signals $p$, e.g., through a pole-zero cancellation at a specific point $\bar{p}\in\Pee$. Structural reachability allows for such situations, whereas complete reachability does not. 

Second, Definition \ref{def:reachability_DT} defines reachability inherently with respect to a time interval. This again stems from the difference in behavior for different $p$: whereas all states might be reachable in $k=n_x$ steps for some $p$, they might only be reachable after, e.g., $k=10n_x$ steps for a different $p$. Structural reachability is a prerequisite for minimality of a realization \cite{Gohberg1992}: if all states are reachable for some $p$, these state directions contribute to the dynamics and cannot be projected out to obtain a smaller state dimension.

First, reachability of the state-space realization is investigated in the general case that $n_a > 0, n_b > 1$, i.e., \eqref{eq:maximum_state_space} is considered. The special cases $n_a = 0, n_b>1$ (FIR) or $n_a > 0, n_b =1$ (inverse FIR) are considered after. The next lemma defines a condition to verify structural $k$-reachability in terms of the $k$-step reachability matrix $\Rcal_k$.

\begin{lemma}
\label{lem:reachability_matrix}
    State-space realization \eqref{eq:maximum_state_space} is structurally k-reachable if and only if $\exists p\in \signalspaceDT$ such that $\rank\ \Rcal_k = n_x$ with
    \begin{equation}
        \Rcal_{k}(p_0,\ldots,p_{k-1}) = 
        \begin{bmatrix}
            \prod_{j=1}^{k-1} \Fcal(p_j) \Gcal(p_0) & \prod_{j=2}^{k-1} \Fcal(p_j) \Gcal(p_1) & \hdots & \Fcal(p_{k-1}) \Gcal(p_{k-2}) & \Gcal(p_{k-1})
        \end{bmatrix}
        \in \mathbb{R}^{n_x \times n_u k}, \label{eq:reach_matrix_generic}%
    \end{equation}%
    with $\Fcal,\Gcal$ as in \eqref{eq:maximum_ss_FcalGcal}.
\end{lemma}%
\begin{proof}
The proof follows from standard reachability arguments and is given here for completeness. First, note that $p$ is completely free such that $k_1,k_2$ in Definition \ref{def:reachability_DT} can be chosen as $k_1=0$ and $k_2=k$ without loss of generality. Then the evolution of $x_k$ in \eqref{eq:maximum_state_space} starting from $x_{k_1}  = x_0$ can be written as
    % \begin{equation}
    %     x_{k} = \prod_{j=0}^{k-1} A(p_j) x_0 + \sum_{i=0}^{k-1} \prod_{j=i+1}^{k-1} \Fcal(p_j) \Gcal(p_i) u_i
    % \end{equation}
    \begin{equation}
        x_k = \prod_{j=0}^{k-1} \Fcal(p_j) x_0 + \Rcal_k(p_0,\ldots,p_{k-1}) U_k,
    \end{equation}
    with $U_k = \begin{bmatrix} u_0^\T & u_1^\T & \ldots & u_{k-1}^\T \end{bmatrix}^\T$. Then, $x_0$ can be steered to any $x_k\in\Ree^{n_x}$ in $k$ steps if and only if $\im\ \Rcal_k = \Ree^{n_x}$, i.e., if and only if $\rank\ \Rcal_k = n_x$. A possible input that achieves this is $U_k = \Rcal_k^{\dag}(p_0,\ldots,p_{k-1}) \big(x_k - \prod_{j=0}^{k-1} A(p_j) x_0\big)$.
\end{proof}
Next, a sufficient condition for full rank of $\Rcal_{n_x}$, i.e., reachability in $n_x$ steps, is derived in terms of conditions on the coefficient functions $A_1,\ldots,A_{n_a}$ and $B_1,\ldots,B_{n_b}$ through considering \eqref{eq:maximum_state_space} for constant scheduling trajectories, i.e., $p_k = \bar{p} \ \forall k \in \Zee$ with $\bar{p}\in\Pee$. 

\begin{theorem}
    \label{th:DT_controllability}
    State-space realization \eqref{eq:maximum_state_space} with $n_a > 0, n_b > 1$ is structurally $n_x$-reachable if there exists a constant scheduling signal $p_k = \bar{p} \in \mathbb{P} \ \forall k \in \Zee$ such that
    \begin{subequations}
    \begin{align}
        &\textnormal{\rank} \begin{bmatrix} I + \sum_{i=1}^{n_a} \sigma^{-i}A_i(\bar{p}) & \sum_{i=0}^{n_b-1} \sigma^{-i} B_i(\bar{p}) \end{bmatrix} = n_y \ \forall \sigma \in \Cee \backslash \{ 0 \}, \label{eq:coprime_theorem} \\
        &\textnormal{\rank} \begin{bmatrix} -A_{n_a}(\bar{p}) & B_{n_b-1}(\bar{p}) \end{bmatrix} = n_y. \label{eq:last_coefficient_nullspace}
    \end{align}%
    \end{subequations}%
\end{theorem}
\begin{proof}
    The proof is based on the equivalence between the Popov-Belevitch-Hautus test for reachability and full rank of the associated reachability matrix for a constant scheduling signal \cite{hespanha2018linear}, i.e., in the LTI case, and proceeds similarly to the key reachability lemma \cite[Lemma 3.4.7]{Goodwin1984}. Specifically, define 
    \begin{equation}
        H(\sigma; \bar{p}) = \left[\begin{array}{c|c} \Fcal(\bar{p}) - \sigma I & \Gcal(\bar{p}) \end{array}\right] \in \Cee^{n_x \times (n_x + n_u)},
    \end{equation}
    with $\sigma\in\Cee$. Then $\rank\ H(\sigma; \bar{p}) = n_x \ \forall \sigma \in \mathbb{C}$ if and only if $\rank\ \Rcal_{n_x}(\bar{p},\ldots,\bar{p}) = n_x$ \cite{hespanha2018linear}, such that by Lemma \ref{lem:reachability_matrix}, realization \eqref{eq:maximum_state_space} is structurally $n_x$-reachable if $\rank\ H(\sigma; \bar{p}) = n_x \ \forall \sigma \in \mathbb{C}$. The remainder of this proof shows that $\rank\ H(\sigma; \bar{p}) = n_x \ \forall \sigma \in \mathbb{C}$ if and only if the conditions of Theorem \ref{th:DT_controllability} hold. 
    
    To show this, first substitute $\Fcal(\bar{p}),\Gcal(\bar{p})$ as in \eqref{eq:maximum_ss_ABCD} into $H(\sigma; \bar{p})$ to obtain
    \begin{equation}\label{eq:H_sigma_pbar_full}
    \resizebox{\textwidth}{!}{$
        H(\sigma;\bar{p}) \! = \! \left[
        \begin{array}{ccccc|ccccc|c}
        \!-A_1(\bar{p}) \!-\!\sigma I \! & \! -A_2(\bar{p}) \! & \! \hdots \! & \! -A_{n_a-1}(\bar{p}) \! & \! -A_{n_a}(\bar{p}) \! &  \!B_1(\bar{p}) \! & \! B_2(\bar{p}) \! & \! \hdots \! & \! B_{n_b-2}(\bar{p}) \! & \! B_{n_b-1}(\bar{p}) \! & \! B_0(\bar{p}) \! \\
        I & -\sigma I  & & 0 & 0 & 0 & 0 & \hdots & 0 & 0 & 0 \\
         & I & \hspace{0ex}\raisebox{2pt}{\rotatebox{15}{$\ddots$}} & & 0 & 0 & 0 & \hdots & 0 & 0 & 0 \\
        & & \hspace{0ex}\raisebox{2pt}{\rotatebox{15}{$\ddots$}}  & -\sigma I & \vdots & \vdots & \vdots & & \vdots & \vdots & \vdots \\
        0 &  & & I & -\sigma I & 0 & 0 & \hdots & 0 & 0 & 0\\
        \hline 
        0 & 0 & \hdots & 0 & 0 & -\sigma I &  & & 0  & 0 & I\\
        0 & 0 & \hdots & 0 & 0 & I & -\sigma I &  & & 0 & 0 \\
        0 & 0 & \hdots & 0 & 0 &  & I & \hspace{0ex}\raisebox{2pt}{\rotatebox{15}{$\ddots$}} &  & \vdots & \vdots \\
        \vdots & \vdots & & \vdots & \vdots & & & \hspace{0ex}\raisebox{2pt}{\rotatebox{15}{$\ddots$}} & -\sigma I & 0 & 0 \\
        0 & 0 & \hdots & 0 & 0 & 0 & & & I & -\sigma I & 0
        \end{array}
        \right]\!\!.
        $}
    \end{equation}
    Now consider that $\sigma \neq 0$ such that $\sigma^{-1}$ exists. Then define the matrices $T_{y},T_{u}$ representing elementary column operations as 
    \begin{subequations}
    \begin{align}
        T_{y} &=   \begin{bmatrix} I & & & & \\
                                  & \raisebox{0pt}{\rotatebox{10}{$\ddots$}} & & & \\
                                  & & I & & \\
                                  & & & I &  \\
                                  & & & \sigma^{-1} I & I \end{bmatrix}
                    \begin{bmatrix} I & & & & \\
                                  & \raisebox{0pt}{\rotatebox{10}{$\ddots$}} & & & \\
                                  & & I &   & \\
                                  & & \sigma^{-1} I& I & \\
                                  & & & & I \end{bmatrix}
                    \hdots
                    \begin{bmatrix} I & & & & \\
                                  \sigma^{-1} I& I & & & \\
                                  & & \raisebox{0pt}{\rotatebox{10}{$\ddots$}} &   & \\
                                  & & & I & \\
                                  & & & & I \end{bmatrix}
                    \in\Ree^{n_y n_a \times n_y n_a} \\
    %\end{equation}
    %the column of $B_{n_b-1}(\bar{p})$ with $\sigma^{-1} I$ and add it to the column of $B_{n_b-2}(\bar{p})$, eliminating the lowest off-diagonal $I$. Next, subtract the result in the column of $B_{n_b-2}(\bar{p})$ from the column of $B_{n_b-3}(\bar{p})$. Repeat this procedure until we arrive at $B_0(\bar{p})$, and do the same for the columns containing $A_i(\bar{p})$. to realize that $\rank\ H(\sigma) = \rank\ \bar{H}(\sigma)$ with
    %i.e., a matrix that through right-multiplication recursively adds $\sigma^{-1}$ times the $i^{th}$ column from the $(i-1)^{th}$ column, eliminating the off-diagonal $I$ elements in the (1,1) submatrix of $H(\sigma)$. 
    %\begin{equation}
        T_{u} &=   \begin{bmatrix} I \vspace{-3pt} & & & & \\
                                  & \raisebox{0pt}{\rotatebox{10}{$\ddots$}} \hspace{-5pt} & & & \\
                                  & & \! I & & \\
                                  & & \sigma^{-1} I & I &  \\
                                  & & &  & I \end{bmatrix}
                    \hdots
                    \begin{bmatrix} I & & & & \\
                                  \sigma^{-1} I& I & & & \\
                                  & & \raisebox{0pt}{\rotatebox{10}{$\ddots$}} &   & \\
                                  & & & I & \\
                                  & & & & I \end{bmatrix} 
                    \begin{bmatrix} I & & & & \\
                                  & I & & & \\
                                  & & \raisebox{0pt}{\rotatebox{10}{$\ddots$}} &   & \\
                                  & & & I & \\
                                  & & & \sigma^{-1} I & I \end{bmatrix} \in \Ree^{(n_b+1)n_u \times (n_b+1)n_u}, \hspace{-10pt}
    \end{align}
    \end{subequations}
    such that
    \begin{align}
        &H(\sigma;\bar{p}) \begin{bmatrix}
            T_{y} & 0 \\
            0 & T_{u}
        \end{bmatrix}
        = \hat{H}(\sigma;\bar{p}) =   \label{eq:Hhat} \\
    &\resizebox{\textwidth}{!}{$
        \left[
        \begin{array}{ccccc|ccccc|c}
        \!-A_1(\bar{p})\! -\!\sigma I \!-\! \sum\limits_{i=2}^{n_a} \sigma^{-i+1} A_{i}(\bar{p}) \! & \! \ast \! & \! \hdots \! & \! \ast \! & \! \ast \! & \! \ast \! & \! \ast \! &\! \hdots \!& \! \ast \! & \! \ast \! & \! B_0(\bar{p}) \! +\! \sum\limits_{i=1}^{n_b-1} \sigma^{-i} B_i(\bar{p}) \\
        0 & \! -\sigma I \!  & & & 0 & 0 & 0 & \hdots & 0 & 0 & 0 \\
        0  &  & \raisebox{0pt}{\rotatebox{10}{$\ddots$}} & & & 0 & 0 & \hdots & 0 & 0 & 0 \\
        \vdots & & & \! -\sigma I \! & & \vdots & \vdots & & \vdots & \vdots & \vdots \\
        0 & 0 & &  & -\sigma I & 0 & 0 & \hdots & 0 & 0 & 0\\
        \hline 
        0 & 0 & \hdots & 0 & 0 & \!\!-\sigma I\!\! & & & & 0 & 0\\
        0 & 0 & \hdots & 0 & 0 & & \!\!-\sigma I\!\! &  & & & 0\\
        0 & 0 & \hdots & 0 & 0 &  & & \raisebox{0pt}{\rotatebox{10}{$\ddots$}} &  & & \vdots \\
        \vdots & \vdots & & \vdots & \vdots & & & & \!\!-\sigma I\!\! & & 0 \\
        0 & 0 & \hdots & 0 & 0 & 0 & & & & \!\!-\sigma I \!\!& 0
        \end{array}
        \right]\!,$} \nonumber
    \end{align}
    where $\ast$ indicates elements not of interest as they can be zeroed out using elementary row operations ($\sigma\neq0$). Note that $T_{y},T_{u}$ are full rank, such that $\rank\ H(\sigma;\bar{p}) = \rank\ \hat{H}(\sigma;\bar{p})$. Now $\hat{H}(\sigma;\bar{p})$ is upper block-triangular, and thus has full rank $n_x$ if and only if its block of rows has full rank, i.e., if and only if
    \begin{equation}
        \rank \begin{bmatrix} -\sigma I - A_1(\bar{p}) - \sum_{i=2}^{n_a} \sigma^{-i+1}A_i(\bar{p}) & \ \ \sum_{i=0}^{n_b-1} \sigma^{-i} B_i(\bar{p}) \end{bmatrix} = n_y, \label{eq:coprime_proof}
    \end{equation}
    which is equivalent to \eqref{eq:coprime_theorem} after a rank-preserving post-multiplication with $\textrm{blkdiag}(-\sigma^{-1}I, I)$.
    
    %Last, multiply the first column with $\sigma^{-1}I$. Then this upper block-triangular matrix has full rank if and only if its first rows has full rank as stated in \eqref{eq:coprime_proof}.

    Turning attention to \eqref{eq:last_coefficient_nullspace}, consider that $\sigma = 0$ and define 
    \begin{equation}
        % T = \left[\begin{array}{ccccc|cccccc}
        %     I_{n_y} & A_1(\bar{p}) & A_2(\bar{p}) & \hdots & A_{n_a-1}(\bar{p}) & -B_0(\bar{p}) & -B_1(\bar{p}) & -B_2(\bar{p}) & \hdots & -B_{n_b-2}(\bar{p}) & 0 \\ \hline
        %      & I & & &  \\
        %     & & I & &\\
        %     & & & \raisebox{0pt}{\rotatebox{10}{$\ddots$}} & & \\
        %     & & & & I \\
        %     & & & & & I \\
        %     & & & & & & I \\
        %     & & & & & & & I \\
        %     & & & & & & & & \raisebox{0pt}{\rotatebox{10}{$\ddots$}} \\
        %     & & & & & & & & & I \\
        %     & & & & & & & & & & I
        % \end{array}\right] \in \Ree^{n_x\times n_x}.
        % T = \left[\begin{array}{ccc}
        % I_{n_y} & A(\bar{p}) & -B_0(\bar{p}) & -B_1(\bar{p}) & -B_2(\bar{p}) & \hdots & -B_{n_b-2}(\bar{p}) \\
        % 0 & I_{n_y n_a} & 0 \\
        % 0 & 0 & I_{(n_b-1)n_u}
        % \end{array}\right] \in \Ree^{n_x \times n_x}
        T = \left[\begin{array}{ccccccccccc} I_{n_y} & A_1(\bar{p}) & A_2(\bar{p}) & \hdots & A_{n_a-1}(\bar{p}) & -B_0(\bar{p}) & -B_1(\bar{p}) & -B_2(\bar{p}) & \hdots & -B_{n_b-2}(\bar{p}) & 0 \\ 0 & \multicolumn{10}{c}{I_{n_x-n_y}} \end{array}\right] \in \Ree^{n_x\times n_x}.
    \end{equation} 
    Then pre-multiplying $H(0;\bar{p})$ with $T$ yields
    \begin{equation}
        \bar{H}(0;\bar{p}) = T H(0;\bar{p}) = 
        \left[
        \begin{array}{ccccc|ccccc|c}
        0 & 0 & \hdots & 0 & -A_{n_a}(\bar{p}) & 0 & 0 & \hdots & 0 & B_{n_b-1}(\bar{p}) & 0 \\
        I & 0 & & 0 & 0 & 0 & 0 & \hdots & 0 & 0 & 0 \\
         & I & & & 0 & 0 & 0 & \hdots & 0 & 0 & 0 \\
        & & \ddots  & & \vdots & \vdots & \vdots & & \vdots & \vdots & \vdots \\
        0 &  & & I & 0 & 0 & 0 & \hdots & 0 & 0 & 0\\
        \hline 
        0 & 0 & \hdots & 0 & 0 & 0 & 0 & \hdots & 0 & 0 & I\\
        0 & 0 & \hdots & 0 & 0 & I & &  & 0 & 0 & 0 \\
        0 & 0 & \hdots & 0 & 0 &  & I & &  & 0 & 0 \\
        \vdots & \vdots & & \vdots & \vdots & & & \ddots & & \vdots & \vdots \\
        0 & 0 & \hdots & 0 & 0 & 0 & & & I & 0 & 0
        \end{array}
        \right],
        \label{eq:Hbar}
    \end{equation}
    in which $\rank\ \bar{H}(0;\bar{p}) = \rank\ H(0;\bar{p})$ as $T$ is square and full rank.
    By its structure, $\rank\ \bar{H}(0;\bar{p}) = n_x$ and only if its first row is full rank, i.e., $\rank \ \bar{H}(0;\bar{p}) = n_x$ if and only if \eqref{eq:last_coefficient_nullspace}, completing the proof. 
\end{proof}
Theorem \ref{th:DT_controllability} states that if coefficient functions $A_i,B_i$ satisfy \eqref{eq:coprime_theorem}-\eqref{eq:last_coefficient_nullspace} for some $\bar{p}\in\Pee$, then realization \eqref{eq:maximum_state_space} is structurally $n_x$-reachable. It can be understood by noting that conditions \eqref{eq:coprime_theorem}-\eqref{eq:last_coefficient_nullspace} are necessary and sufficient for reachability of the frozen LTI behavior of \eqref{eq:maximum_state_space} at a point $\bar{p}$, such that \eqref{eq:maximum_state_space} is structurally $n_x$-reachable if the frozen LTI behavior is reachable for some point $\bar{p}$. Condition \eqref{eq:coprime_theorem} states that there should be at least one constant value for the scheduling signal $\bar{p}$ for which no pole-zero cancellations occur, i.e., it represents a coprime condition \cite[Chapter 7]{Chen1999} over $\bar{p}\in\Pee$. In the case of single-input single-output (SISO) systems, this condition merely states that the polynomials $P(\sigma; \bar{p}) = I+\sum_{i=1}^{n_a} A_i(\bar{p}) \sigma^{-i}$ and $Q(\sigma; \bar{p}) = \sum_{i=0}^{n_b-1} B_i(\bar{p}) \sigma^{-i}$ should not have a common root for all $\bar{p}\in\Pee$. In the multivariable case, directionality of these roots is taken into account. Condition \eqref{eq:last_coefficient_nullspace} states that the order of the IO representation \eqref{eq:LPV_IO_DT} should not be unnecessarily high, i.e., it can be interpreted as a well-posedness condition. Specifically, in the SISO case, it states that either $A_{n_a}(\bar{p}) \neq 0$ or $B_{n_b-1}(\bar{p}) \neq 0$. By choosing the model order appropriately, i.e., by not incorporating a coefficient that is zero for all $\bar{p}\in\Pee$, this condition is automatically satisfied. In the multivariable case, the directionality of $A_{n_a}(\bar{p})$ and $B_{n_b-1}(\bar{p})$ is taken into account.

Last note that Theorem \ref{th:DT_controllability} only considers constant scheduling trajectories, and is thus only a sufficient condition for structural reachability of \eqref{eq:maximum_state_space}. In other words, $\Rcal_k$ can be full rank for a varying scheduling signal, even if it is not for any constant scheduling $\bar{p}\in\Pee$.

Condition \eqref{eq:coprime_theorem} imposes a condition for all $\sigma \in \Cee \backslash \{0\}$. However, since $\rank\ P(\sigma;\bar{p}) = n_y$ for $\sigma \in \Cee$ that are not roots of $P(\sigma;\bar{p})$, this condition only has to be verified at these roots. Thus, a practical way to evaluate this condition for a given LPV model is to evaluate the coefficient functions for a grid of $\bar{p}\in\Pee$, calculating the roots of $P(\sigma;\bar{p})$ over this grid, and evaluating \eqref{eq:coprime_theorem} at the roots of $P(\sigma)$, i.e., checking for pole-zero cancellations over a grid of $\bar{p}\in\Pee$.

Lemma \ref{lem:reachability_matrix} and Theorem \ref{th:DT_controllability} can straightforwardly be specialized to the special cases $n_b=1$ (inverse LPV-FIR) with realization \eqref{eq:maximum_state_space_invFIR} or $n_a=0$ (LPV-FIR) with realization \eqref{eq:maximum_state_space_FIR} as follows. 
\begin{corollary}\label{cor:reachability_invFIR}
    State-space realization \eqref{eq:maximum_state_space_invFIR} is structurally $n_x$-reachable if there exists a constant scheduling signal $p_k = \bar{p} \in\Pee\ \forall k\in\Zee$ such that $\rank\ B_0(\bar{p}) = n_y$.
        % \begin{align}\label{eq:reachability_invFIR}
        %     \textnormal{\rank} \begin{bmatrix} I + \sum_{i=1}^{n_a} \sigma^{-1} A_i(\bar{p}) & B_0(\bar{p}) \end{bmatrix} = n_y \ \forall \sigma \in \Cee \backslash \{0\} & & \textnormal{\rank}\ B_0(\bar{p}) = n_y.
        % \end{align}
\end{corollary}
\begin{proof}
    The proof is analogous to the proof of Theorem \ref{th:DT_controllability} and links the rank of $\Rcal_k$ in \eqref{eq:reach_matrix_generic} for a constant scheduling $\bar{p}$ to the rank of the PBH matrix pencil $H(\sigma; \bar{p}) \in \Ree^{n_x \times n_x + n_u}$. Specifically, for this setting, it holds that $H(\sigma;\bar{p}) = \left[\begin{array}{c|c} F_a - G_a A(p_k) & G_a B_0 (p_k) \end{array}\right]$, or 
    \begin{equation}
        H(\sigma;\bar{p}) \! = \! \left[
        \begin{array}{ccccc|c}
        \!-A_1(\bar{p}) \!-\!\sigma I \! & \! -A_2(\bar{p}) \! & \! \hdots \! & \! -A_{n_a-1}(\bar{p}) \! & \! -A_{n_a}(\bar{p}) \! &  \! B_0(\bar{p}) \! \\
        I & -\sigma I  & & 0 & 0 & 0 \\
         & I & \raisebox{0pt}{\rotatebox{10}{$\ddots$}} & & \vdots & \vdots \\
        & & \raisebox{0pt}{\rotatebox{10}{$\ddots$}}  & -\sigma I & 0 & 0 \\
        0 &  & & I & -\sigma I & 0 
        \end{array}
        \right]
    \end{equation}
    Carrying out the same steps as in the proof of Theorem \ref{th:DT_controllability} yields that $\rank\ H(\sigma; \bar{p}) = n_x \ \forall \sigma\in\Cee$ if and only if
    \begin{align}
        \rank \begin{bmatrix} I + \sum_{i=1}^{n_a} \sigma^{-1} A_i(\bar{p}) & B_0(\bar{p}) \end{bmatrix} = n_y \ \forall \sigma \in \Cee \backslash \{0\} & & \textnormal{\rank}\ B_0(\bar{p}) = n_y.
    \end{align}
    Now, the first condition is implied by the second, such that $\rank\ H(\sigma; \bar{p}) = n_x \ \forall \sigma\in\Cee$ if and only if $\rank\ B_0(\bar{p}) = n_y$. Then, by standard LTI arguments, $\rank\ \Rcal_k = n_x$ for some constant scheduling $\bar{p}$ if and only if $\rank\ H(\sigma; \bar{p}) = n_x \ \forall \sigma\in\Cee$, i.e., \eqref{eq:maximum_state_space_invFIR} is structurally reachable if $\rank\ B_0(\bar{p}) = n_y$ for some $\bar{p}\in\Pee$.
\end{proof}
\begin{corollary}\label{cor:reachability_FIR}
    State-space realization \eqref{eq:maximum_state_space_FIR} is completely $n_x$-reachable.
\end{corollary}
\begin{proof}
    For realization \eqref{eq:maximum_state_space_FIR}, the state-transition equation $\bar{u}_{k+1} = F_b \bar{u}_k + G_b u_k$ is time-invariant, such that reachability is no longer dependent on the scheduling signal $p$. Its reachability matrix $R_{n_x}$ satisfies $R_{n_x} = I$ for any $p$. Thus, \eqref{eq:maximum_state_space_FIR} is completely $n_x$-reachable.
\end{proof}
Corollary \ref{cor:reachability_FIR} can be intuitively understood by noting that the states of \eqref{eq:maximum_state_space_FIR} are simply shifted versions of $u_k$, which can be directly set by the input $u_k$.  

Taken together, Theorem \ref{th:DT_controllability} and Corollaries \ref{cor:reachability_invFIR} and \ref{cor:reachability_FIR} show that the state-space resulting from the direct state construction \eqref{eq:maximum_ss_state_def} is reachable under coprimeness and well-posedness conditions.

\section{Observability and Reconstructability}\label{sec:observability}
In this section, observability and reconstructability properties of \eqref{eq:maximum_state_space} are  derived, constituting contribution C2. Specifically, it is shown that for state-space realization \eqref{eq:maximum_state_space}, there exist initial states of past inputs and outputs of which the effect cannot be observed in the output, i.e., \eqref{eq:maximum_state_space} has an unobservable subspace and is thus not minimal. Even though this unobservable subspace is present, it is shown that for any coefficient functions $A_i$ and $B_i$, the states in this unobservable subspace decay to 0 in at most $\max(n_a,n_b-1)$ time steps, such that realization \eqref{eq:maximum_state_space} can never have unstable unobservable modes. To formalize these claims, consider the definition of observability and reconstructability in the LPV case \cite{Silverman1967, Gohberg1992, Toth2010}.

\begin{definition}
\label{def:obsv_DT}
   State-space realization \eqref{eq:maximum_state_space} is said to be structurally observable in $k\in\Nee$ steps if there exists a scheduling signal $p\in\signalspaceDT$ such that for any initial state $x_{k_1}\in\Xee$ at an arbitrary time $k_1\in\Zee$ and any input signal $u\in\signalspaceDT$, $x_{k_1}$ can be uniquely reconstructed from $\{ u_{j}, y_{j} \}_{j=k_1}^{k_1+k-1}$. It is said to be completely $k$-observable if the reconstruction of $x_{k_1}$ is possible for all $p\in\signalspaceDT$.
\end{definition}
Structural observability means that for at least one scheduling signal $p$, any state at time $k_1$ can be inferred from input-output data \textit{after} $k_1$. In contrast, complete observability requires that this is possible for all $p$. Structural observability strongly relates to minimality of the state space realization. Specifically, if a subspace of $\Xee$ is observable for some $p$, i.e., if this subspace is structurally observable, this subspace contributes to the input-output behavior of \eqref{eq:maximum_state_space}. Next to observability, the weaker notion of structural $k$-reconstructability is defined \cite{Silverman1967, Gohberg1992, Toth2010}.

\begin{definition}
\label{def:recon_DT}
    State-space realization \eqref{eq:maximum_state_space} is said to be structurally reconstructible in $k\in\Nee$ steps if there exists a scheduling signal $p\in\signalspaceDT$ such that for any initial state $x_{k_1}\in\Xee$ at an arbitrary time $k_1\in\Zee$ and any input signal $u\in\signalspaceDT$, $x_{k_1+k}$ can be uniquely reconstructed from $\{ u_{j}, y_{j} \}_{j=k_1}^{k_1+k-1}$. It is said to be completely $k$-reconstructible if the reconstruction of $x_{k_1+k}$ is possible for all $p\in\signalspaceDT$.
\end{definition}
Structural reconstructability means that for at least one scheduling signal $p$, the state at time $k_1+k$ can be inferred from input-output data \textit{before} $k_1+k$. In contrast to observability, only the part of $x_{k_1}$ that does not decay to 0 in $k$ steps is required to be reconstructible. Consequently, structural $k$-reconstructability allows for an unobservable subspace as long as states in this subspace decay to $0$ in $k$ steps, and is thus a weaker notion than observability. Last, note that detectability follows as reconstructability for $k\rightarrow\infty$. 
%this requires reconstructing only the part of $x_{k_1}$ that does not decay to 0 in $k$ steps, and simulating $x_{k_1+k}$ from that $x_{k_1}$
% Intuition: allows for estimating $x_0$ up to part in kernel of A^k, as this decays anyway
% Similarly to observability, to reconstruct  this requires reconstructing $x_{k_1}$ using $\{ u_{j}, y_{j} \}_{j=k_1}^{k_1+k-1}$, but in contrast to observability, only the part of $x_{k_1}$ that does not decay to 0 in $k$ steps is required to be reconstructible, thus allowing for certain unobservable states as long as they decay to 0 in $k$ steps. Last, note that detectability follows as reconstructability for 

As in the reachability case, first observability and reconstructability are investigated in the general case that $n_ a> 0,n_b>1$, i.e., \eqref{eq:maximum_state_space} is considered. Afterwards, the cases in which $n_b=1$ or $n_a =0$ are considered, described by \eqref{eq:maximum_state_space_invFIR} and \eqref{eq:maximum_state_space_FIR} respectively. The next lemma provides a condition to verify observability and reconstructability in terms of the rank of the $k$-step observability matrix $\Ocal_k$. 

%Note that structural observability does not guarantee that for every signal $p$, $x_{k_1}$ can be uniquely reconstructed from only the $n_x$ outputs $y_{k_1}, y_{k_2}, \ldots, y_{k_1 + n_x -1}$. Depending on the specific realization of $p$, it could be that an initial condition $x_{k_1}$ can only be uniquely determined after $n \gg n_x$ steps, i.e., from $y_{k_1}, \ldots, y_{k_1 + n}$. Structural observability only implies that there exists a $p$ for which it is possible using $n_x$ output samples. The following Lemma provides a numerical test for determining structural observability.

\begin{lemma}
\label{lem:observability_matrix}
    State-space realization \eqref{eq:maximum_state_space} is structurally k-observable if and only if $\exists p\in \signalspaceDT$ such that $\ker\ \Ocal_k = \emptyset$ with
    \begin{equation}
        \Ocal_{k} = 
        \begin{bmatrix}
            \Hcal(p_{0}) \\
            \Hcal(p_{1}) \Fcal(p_{0}) \\
            \Hcal(p_{2}) \Fcal(p_{1}) \Fcal(p_{0}) \\
            \vdots \\
            \Hcal(p_{k-1}) \prod_{j=0}^{k-2} \Fcal(p_{j})
        \end{bmatrix}
        \in \mathbb{R}^{n_y k \times n_x}
        \label{eq:obsv_matrix_generic}
    \end{equation}
    It is completely $k$-reconstructible if and only if $\ker\ \Ocal_k \subseteq \ker\prod_{j=0}^{k-1} \Fcal(p_{j})$ for all scheduling signals $p\in \signalspaceDT$.
    %for any $x\in \Xee \cap \textrm{ker } \Ocal_{k}$ it holds that $\prod_{k=0}^{k-2} A(p_{k}) x = 0$.
\end{lemma}
\begin{proof}
    The proof follows from standard observability arguments \cite{Gohberg1992} and is given here for completeness. Throughout, $k_1$ in Definitions \ref{def:obsv_DT} and \ref{def:recon_DT} is chosen as $k_1 = 0$ without loss of generality, as the moment of zero time can be chosen arbitrary. Then, the response $y_j$ with $j$ ranging from $k_1=0$ to $k-1$ from initial state $x_0$ is given by
    \begin{equation}\label{eq:DT_response}
    \begin{aligned}
        \underbrace{\begin{bmatrix}
            y_0 \\ y_1 \\ y_{2} \\ \vdots \\ y_{k-1}
        \end{bmatrix}}_{y_{[0:k-1]}}
        &=
        \Ocal_k x_0
        -
        \underbrace{\begin{bmatrix}
        \Jcal(p_0) & & & 0\\
        \Hcal(p_1) \Gcal(p_0) & \Jcal(p_1) & & \\
        \Hcal(p_2) \Fcal(p_1) \Gcal(p_0) & \Hcal(p_2) \Gcal(p_1) & &  \\
        \vdots & \vdots & \raisebox{0pt}{\rotatebox{22}{$\ddots$}}\\
        \Hcal(p_{k-1}) \prod_{j=1}^{k-2} \Fcal(p_j) \Gcal(p_0) & \Hcal(p_{k-1}) \prod_{j=2}^{k-2} \Fcal(p_j) \Gcal(p_1) & \hdots & \Jcal(p_{k-1})
        \end{bmatrix}}_{\Gamma}
        \underbrace{\begin{bmatrix}
            u_0 \\ u_1 \\ u_2 \\ \vdots \\ u_{k-1}
        \end{bmatrix}}_{u_{[0:k-1]}}.
    \end{aligned}
    \end{equation}

    Then, for the observability case, $x_0$ can be uniquely determined from $\{u_j, y_j\}_{j=0}^{k-1}$ by $\Ocal_k x_0 = (y_{[0:k-1]} - \Gamma u_{[0:k-1]})$ if and only if $\ker\ \Ocal_k = \emptyset$, i.e., realization \eqref{eq:maximum_state_space} is structurally $k$-observable if and only if $\exists p\in \signalspaceDT$ such that $\ker\ \Ocal_k = \emptyset$. %Thus, $x_0$ can uniquely be determined from $\{ u_j, y_j \}_{j=0}^{k-1}$ if and only if $\rank\ \Ocal_k = n_x$. 
    % \begin{align}
    %     y_k = \Ccal(p_k) \prod_{i=0}^{k-1} A(p_i) x_0 + \sum_{i=0}^{k-1} \prod_{j=i}^{k-1} A(p_j) B(p_i) u_i.
    % \end{align}
    
    Next, for $k$-reconstructability case, note that $x_{k+k_1} = x_{k}$ is given by
    \begin{equation}
        x_k = \prod_{j=0}^{k-1} \Fcal(p_j) x_0 + \sum_{i=0}^{k-1} \prod_{j=i+1}^{k-1} \Fcal(p_j) \Gcal(p_i) u_i.
        \label{eq:x_k_pred}
    \end{equation}
    Then any reconstruction of $x_0$ from \eqref{eq:DT_response} in case $\ker\ \Ocal_k \neq \emptyset$ yields that the reconstruction $\hat{x}_0$ satisfies $\hat{x}_0 = x_0 + \delta x $ for some $\delta x \in \ker\ \Ocal_k$, as these $\delta x$ cannot be inferred from the output $y_0,\ldots,y_{k-1}$.
    %An example of such an estimator is $\hat{x}_0 = \Ocal_k^\dagger \begin{bmatrix} y_0^\T & \ldots & y_{k-1}^\T \end{bmatrix}^\T$. 
    Substituting this estimate $\hat{x}_0$ for $x_0$ in \eqref{eq:x_k_pred} yields an estimate $\hat{x}_k$, which correctly reconstructs $x_k$ if and only if $\delta x \in \ker\prod_{j=0}^{k-1} \Fcal(p_j)$, i.e., state-space realization \eqref{eq:maximum_state_space} is completely $k$-reconstructible if and only if $\ker\ \Ocal_k \subseteq \ker \prod_{j=0}^{k-1} \Fcal(p_j) \ \forall p \in \signalspaceDT$.
\end{proof}

Lemma \ref{lem:observability_matrix} can be understood by noting that the kernel of the observability matrix $\Ocal_k$ directly gives the directions that cannot be observed from the output. Given this characterization of observability and reconstructability in terms of the kernel of $\Ocal_k$, first it is shown that \eqref{eq:maximum_state_space} is completely reconstructible in $\max(n_a,n_b-1)$ steps.
\begin{theorem}\label{th:reconstructability}
    State-space realization \eqref{eq:maximum_state_space} is completely $max(n_a,n_b-1)$-reconstructible.
\end{theorem}
\begin{proof}
The proof is based on noting that any $x_k$ that does not produce an output at time $k$, i.e., $H(p_k) x_k =0$, also satisfies $x_{k+1} = \Fcal(p_k)x_k = (F+G\Hcal(p_k))x_k = F x_k$, i.e., such $x_k$ is only propogated in time by $F$. By the structure of $F$, any such $x_k$ must fade out in $\max(n_a,n_b-1)$ steps. Formally, set $k = \max(n_a,n_b-1)$ and consider $\Ocal_k$ for any scheduling signal $p$%. If $\Ocal_k$ is full rank, \eqref{eq:maximum_state_space} is trivially completely reconstructible. Alternatively, 
, and consider any initial state $x_{0} \in\ker\ \Ocal_k$. For such an $x_0$, it is claimed that it holds that $\prod_{j=0}^{k-1}\Fcal(p_j) x_0 = F^{k} x_0$. To see this, note that for $x_0 \in \ker\ \Ocal_k$ it must hold that $\Hcal(p_0) x_0 = 0$ by the first rows of $\Ocal_k$, which implies that $\Fcal(p_0) x_0 = (F + G \Hcal(p_0)) x_0 = F x_0$. Similarly, by the second row of $\Ocal_k$, it holds that $\Hcal(p_1) \Fcal(p_0) x_0 = 0$, which implies that $\Fcal(p_1) \Fcal(p_0) x_0 = (F+G\Hcal(p_1)) F x_0 = F^2 x_0$. Continuing, it holds that $\prod_{j=0}^{k-1} \Fcal(p_j) x_0 = F^{k} x_0$. Now, for $F$ it holds that $F^{j} = 0$ for $j\geq k$, i.e., $F$ is a nilpotent matrix. Consequently, any $x_0 \in\ker\ \Ocal_k$ satisfies $\prod_{j=0}^{k-1} \Fcal(p_j) x_0 = 0$, or $\ker\ \Ocal_k \subseteq \ker\prod_{j=0}^{k-1} \Fcal(p_j)$ for $k = \max(n_a,n_b-1)$.
\end{proof}

Intuitively, it should come at no surprise that \eqref{eq:maximum_state_space} is completely $\max(n_a,n_b-1)$-reconstructible: the time segment $\{k,k_1,\ldots,k\!+\max(n_a,n_b\!-\!1)\!-\!1\}$ contains all inputs and outputs that constitute the state of \eqref{eq:maximum_state_space} at $k+\max(n_a,n_b-1)$, see the state definition \eqref{eq:maximum_ss_state_def}. Thus, the state at $k+\max(n_a,n_b-1)$ is trivially reconstructible. Theorem \ref{th:reconstructability} directly implies that state-space realization \eqref{eq:maximum_state_space} cannot have unstable unobservable states.

Even though \eqref{eq:maximum_state_space} is completely reconstructible, it is never observable. Specifically, to analyze observability of \eqref{eq:maximum_state_space} for a given set of coefficient functions, $\Ocal_k$ is characterized in terms of $A_i,B_i$. Using this characterization, it is shown that the kernel of $\Ocal_k$ is not empty for any $k\in\Nee$.
% To analyze the rank and kernel of $\Ocal_k$ for a given set of coefficient functions $A_i,B_i$, i.e., to analyze observability and reconstructability of the realization \eqref{eq:maximum_state_space}, the next Lemma expresses the rank/kernel of $\Ocal_k$ in terms of a simpler matrix that is an explicit function of $A_i,B_i$.

\begin{lemma}\label{lem:Ocalbar}
    Given any $k\in\Nee$, there exists a full rank matrix $T_k \in \Ree^{n_x \times n_x}$ such that the observability matrix $\Ocal_k$ can be written as $T_k \Ocal_k = \bar{\Ocal}_k$ with 
        \begin{equation} \label{eq:Ocalbar}
        \bar{\Ocal}_{k} = \begin{bmatrix}
        -A(p_0) & B(p_0) \\
        -A(p_1) F_a & B(p_1) F_b \\
        -A(p_2) F_a^2 & B(p_2) F_b^2 \\
        \vdots & \vdots \\
        -A(p_{k-1}) F_a^{k-1} & B(p_{k-1}) F_b^{k-1} \\
    \end{bmatrix} \in \Ree^{n_y k \times n_x}.
    \end{equation}
    in which $F_a,F_b$ are given by \eqref{eq:maximum_ss_FGFaGaFbGb}.
\end{lemma}
\begin{proof}
    Define
    \begin{align}
        F &= \left[\begin{array}{c|c}
        F_a & 0 \\ \hline 0 & F_b
        \end{array}\right]
        \in \mathbb{R}^{n_x \times n_x} 
        & & 
        G = \left[\begin{array}{c}
            G_a \\ \hline 0
        \end{array}\right]
        \in \mathbb{R}^{n_x \times n_y}
        \label{eq:maximum_ss_F_G}
    \end{align}
    with $F_a,F_b,G_a$ as in \eqref{eq:maximum_ss_FGFaGaFbGb}. Then $\Fcal$ in \eqref{eq:maximum_ss_ABCD} can be written as $\Fcal(p_j) = F + G \Hcal(p_j)$. Now it is noted that the state-transition matrix $\Fcal$ is a function of the output matrix $\Hcal$, which allows for simplifying the observability matrix in \eqref{eq:obsv_matrix_generic}. Specifically, define $\Ncal_j = \Hcal(p_{j-1}) \prod_{i=0}^{j-2} \Fcal(p_i)$ for $j=1,\ldots,k$, i.e., $\Ncal_j$ is the $j^{\textnormal{th}}$ block of rows of \eqref{eq:obsv_matrix_generic}. With these definitions, it is claimed that it holds that
    \begin{equation}\label{eq:aux}
        \mathcal{N}_j =  \Hcal(p_{j-1}) F^{j-1} +  \Hcal(p_{j-1}) \sum_{i=0}^{j-2} F^{j-i}G \mathcal{N}_i.
    \end{equation}
    To see this intuitively, for $j=1,2,3$ it holds that
    \begin{equation}
    \begin{aligned}
        \mathcal{N}_1 &= \Hcal(p_0) \\
        \mathcal{N}_2 &= \Hcal(p_1) \Fcal(p_0) = \Hcal(p_1) (F + G \Hcal(p_0)) = \Hcal(p_1) F + \Hcal(p_1) G \underbrace{\Hcal(p_0)}_{\mathcal{N}_1} \\
        \mathcal{N}_3 &= \Hcal(p_2) \Fcal(p_1) \Fcal(p_0) = \Hcal(p_2) (F + G \Hcal(p_1)) (F + G \Hcal(p_0)) \\
        &= \Hcal(p_2) (F^2 + G \underbrace{\Hcal(p_1) (F + G \Hcal(p_0))}_{\mathcal{N}_2} + F G \underbrace{\Hcal(p_0)}_{\mathcal{N}_1}). 
    \end{aligned}
    \end{equation}
    Formally, for arbitrary $j$ it holds that
    \begin{equation}
        \begin{aligned}
            \mathcal{N}_j &= \Hcal(p_{j-1}) \prod_{i=0}^{j-2} \Fcal(p_i) = \Hcal(p_{j-1}) (F + G \Hcal(p_{j-2})) \prod_{i=0}^{j-3} (F + G\Hcal(p_i)) \\
            %&= \Hcal(p_{k-1}) G \underbrace{\Hcal(p_{k-2}) \prod_{j=0}^{k-3} (F + G\Hcal(p_j))}_{\mathcal{N}_{k-1}} + \Hcal(p_{k-1}) F \prod_{i=0}^{k-2} (F + GJ(p_i))
            &= \Hcal(p_{j-1}) G \Hcal(p_{j-2}) \prod_{i=0}^{j-3} (F + G\Hcal(p_i)) + \Hcal(p_{j-1}) F \prod_{i=0}^{j-3} (F + G\Hcal(p_i))
            \label{eq:N_K_proof}
        \end{aligned}
    \end{equation}
    where, for the first term, it holds by definition that $\mathcal{N}_{j-1}=\Hcal(p_{j-2}) \prod_{i=0}^{j-3} (F + G\Hcal(p_i))$, see above \eqref{eq:aux}. Applying a similar expansion to the second term results in 
    \begin{equation}
        \begin{aligned}
            \mathcal{N}_j &= \Hcal(p_{j-1}) G \mathcal{N}_{j-1} + \Hcal(p_{j-1}) F (F+GJ(p_{j-3})) \prod_{i=0}^{j-4} (F+G\Hcal(p_i)) \\
            &= \Hcal(p_{j-1}) G \mathcal{N}_{j-1} + \Hcal(p_{j-1}) F G \Hcal(p_{j-3}) \prod_{i=0}^{j-4} (F+G\Hcal(p_i)) + \Hcal(p_{j-1}) F^2 \prod_{i=0}^{j-4} (F+G\Hcal(p_i)),
        \end{aligned}
    \end{equation}
    where for the second term it holds that $\Hcal(p_{j-3}) \prod_{i=0}^{j-4} (F+G\Hcal(p_i)) = \mathcal{N}_{j-2}$. Continuing all the way gives
    %to $\Ncal_1$, it holds that
    \begin{align}
        \mathcal{N}_j &= \Hcal(p_{j-1}) G \mathcal{N}_{j-1} + \Hcal(p_{j-1}) F G \mathcal{N}_{j-2} + \Hcal(p_{j-1}) F^2 G \mathcal{N}_{j-3} + \ldots + \Hcal(p_{j-1}) F^{j-2} G \mathcal{N}_{1} + \Hcal(p_{j-1}) F^{j-1} \nonumber \\
        &= \Hcal(p_{j-1}) F^{j-1} + \Hcal(p_{j-1}) \sum_{i=1}^{j-1} F^{j-1-i}G \mathcal{N}_i.
        \label{eq:N_k_proof2}
    \end{align}
    Using this expansion for $\mathcal{N}_j$, the terms $\Ncal_1,\ldots,\Ncal_k$ satisfy
    %Then by definition of $\mathcal{O}_n$ in \eqref{eq:obsv_matrix_generic}, $\mathcal{N}_k =  C(p_k) \prod_{i=0}^{k-1} A(p_i)$ and $T_n$ as in \eqref{eq:T_n} 
    \begin{equation}
        \underbrace{
        \begin{bmatrix}
            I & & & & &  0 \\
            -\Hcal(p_1) G & I & & & &  \\
            -\Hcal(p_2) F G & -\Hcal(p_2) G & I & & & \\
            \vdots & \vdots & & \raisebox{3pt}{\rotatebox{25}{$\ddots$}} & & \\
            -\Hcal(p_{k-2})F^{k-3}G & \!\! -\Hcal(p_{k-2})F^{k-4}G \!& \hdots & \!\!\!\!\!\!\!\!-\Hcal(p_{k-2})G & I &\\
            -\Hcal(p_{k-1})F^{k-2} G &\!\! -\Hcal(p_{k-1})F^{k-3} G \! & \hdots & \!\!\!-\Hcal(p_{k-1})FG & \! -\Hcal(p_{k-1})G \! & \! I  
        \end{bmatrix}
        }_{T_k}
        \underbrace{
        \begin{bmatrix}
            \mathcal{N}_1 \\
            \mathcal{N}_2 \\
            \mathcal{N}_3 \\
            \vdots \\
            \mathcal{N}_{k-1} \\
            \mathcal{N}_{k}
        \end{bmatrix}
        }_{\Ocal_k}
        \!=\! 
        \underbrace{
        \begin{bmatrix}
            \Hcal(p_0) \\
            \Hcal(p_1) F \\
            \Hcal(p_2) F^2 \\
            \vdots \\
            \Hcal(p_{k-2}) F^{k-2} \\
            \Hcal(p_{k-1}) F^{k-1}
        \end{bmatrix}
        }_{\bar{\Ocal}_k}
        \!.
    \end{equation}
    Now note that $T_k$ is lower block diagonal with $I$ on its main diagonal, i.e., it is unimodular and full rank. Consequently, it holds that $\rank\ \Ocal_k = \rank\ \bar{\Ocal}_k$ and $\ker\ \Ocal_k = \ker\ \bar{\Ocal}_k$, such that $\bar{\Ocal}_k$ can be equivalently considered. Using the partitioned forms $\Hcal(p_j) = \left[\begin{array}{c|c} -A(p_j) & B(p_j) \end{array}\right]$ and $F = \textrm{blkdiag}(F_a,F_b)$, see \eqref{eq:maximum_ss_ABCD} and \eqref{eq:maximum_ss_FGFaGaFbGb}, $\bar{\Ocal}_k$ can be written as \eqref{eq:Ocalbar}.   
\end{proof}
Even though above Lemma is only an algebraic manipulation of $\Ocal_k$, it enables analyzing $\ker\ \Ocal_k$ through $\ker\ \bar{\Ocal}_k$ as $T_k$ is full rank. This enables the following observability result.
\begin{theorem}\label{th:observability}
    For the state-space realization \eqref{eq:maximum_state_space} it holds that $\rank\ \Ocal_{k} < n_x$ for all $k\in\Nee$ and all $p\in\signalspaceDT$, i.e., \eqref{eq:maximum_state_space} is not observable. Furthermore, $\ker\ \Ocal_k$ is given by $\ker\ \bar{\Ocal}_k$ with $\bar{\Ocal}_k$ in \eqref{eq:Ocalbar}.
\end{theorem}
\begin{proof}
    The proof is based on analysis of $\bar{\Ocal}_k$ and nilpotency of shift matrices $F_a$,$F_b$. Specifically, first note that by Lemma \ref{lem:Ocalbar} and full rank of $T_k$, $\ker\ \Ocal_k = \ker\ \bar{\Ocal}_k$ and $\rank\ \Ocal_k = \rank\ \bar{\Ocal}_k$, i.e., $\bar{\Ocal}_k$ can be considered to derive observability properties of  \eqref{eq:maximum_state_space}. Then the rank of $\bar{\Ocal}_k$ is analyzed in 4 separate cases, which taken together imply $\rank\ \Ocal_{k} < n_x$ for all $k$ and all $p\in\signalspaceDT$.
    \begin{enumerate}[leftmargin=*]
        \item $k \leq n_a$: in this case $k n_y < n_x$, as $n_x = n_y n_a + n_u (n_b-1)$ and $n_b > 1$, i.e., $\bar{\Ocal}_k \in \Ree^{k n_y \times n_x}$ is wide and its rank trivially satisfies $\rank\ \bar{\Ocal}_k \leq k n_y < n_x$.
        \item $n_a \geq n_b -1$ and $k > n_a$: for $F_a$, it holds that $F_a^{k} \neq 0$ for $0 \leq k < n_a$ and $F_a^{k} = 0$ for $k \geq n_a$. Similarly, for $n_b$ it holds that $F_b^k \neq 0$ for $0\leq k < n_b-1$ and $F_b^{k} = 0$ for $k \geq n_b-1$. Thus $\bar{\Ocal}_k$ is given by
        \begin{equation}
            % \left[\begin{array}{cccccc|ccc}
            %     -A_0(p_0) & -A_1(p_0) & \hdots & -A_{k_s}(p_0) & \hdots & -A_{n_a}(p_0) & B_1(p_0) & \hdots & B_{n_b-1}(p_0) \\ 
            %     - A_1(p_0) & \ldots & -A_{k_s}(p_0) & \hdots & -A_{n_a}(p_0) & & B_2(p_0) \\
            %     \vdots & \textcolor{red}{\udots} & & \textcolor{red}{\udots} & & & \vdots & \\
            %     -A_{k_s}(p_0) & \hdots & -A_{n_a}(p_0) & & \emptyset & & B_{n_b-1}(p_0) & \emptyset \\
            %     \hdashline
            % \end{array}\right]
            \bar{\Ocal}_k = 
            \left[\begin{array}{cc}
            -A(p_0) & B(p_0) \\
            \vdots & \vdots \\
            -A(p_{n_b-2}) F^{n_b-2} & B(p_{n_b-2}) F^{n_b-2}\\
            \hline
            -A(p_{n_b-1})F^{n_b-1} & 0 \\
            \vdots & \vdots \\
            -A(p_{n_a-1}) F^{n_a-1} & 0 \\
            \hline
            0_{n_y(k - n_a) \times n_y n_a} & 0_{n_y(k-n_a) \times n_u (n_b-1)}
            \end{array}\right],
        \end{equation}
        for $n_a > n_b-1$. In case $n_a = n_b-1$, the middle blocks are not present. In both cases, $\bar{\Ocal}_k$ only has $n_a n_y$ rows that are potentially nonzero depending on $p$, i.e., $\rank\ \bar{\Ocal}_k \leq n_a n_y < n_x$.
        %i.e., its row space satisfies $\dim (\ker\ \bar{\Ocal_k})^\perp \leq n_a n_y < n_x$ as $n_a n_y < n_x$ for $n_b > 1$. 
        \item $n_a < n_b -1$ and $k>n_b-1$ (and thus $k > n_a$): by the same nilpotency properties, $\bar{\Ocal}_k$ is given by
        \begin{equation}\label{eq:aux2}
            % \left[\begin{array}{cccccc|ccc}
            %     -A_0(p_0) & -A_1(p_0) & \hdots & -A_{k_s}(p_0) & \hdots & -A_{n_a}(p_0) & B_1(p_0) & \hdots & B_{n_b-1}(p_0) \\ 
            %     - A_1(p_0) & \ldots & -A_{k_s}(p_0) & \hdots & -A_{n_a}(p_0) & & B_2(p_0) \\
            %     \vdots & \textcolor{red}{\udots} & & \textcolor{red}{\udots} & & & \vdots & \\
            %     -A_{k_s}(p_0) & \hdots & -A_{n_a}(p_0) & & \emptyset & & B_{n_b-1}(p_0) & \emptyset \\
            %     \hdashline
            % \end{array}\right]
            \bar{\Ocal}_k = 
            \left[\begin{array}{cc}
            -A(p_0) & B(p_0) \\
            \vdots & \vdots \\
            -A(p_{n_a-1}) F^{n_a-1} & B(p_{n_a-1}) F^{n_a-1}\\
            \hline
            0 & B(p_{n_a}) F^{n_a} \\
            \vdots & \vdots \\
            0 &  B(p_{n_b-2}) F^{n_b-2} \\
            \hline
            0_{n_y(k - n_b-1) \times n_y n_a} & 0_{n_y(k - n_b-1) \times n_u (n_b-1)}
            \end{array}\right] 
            =
            \left[\begin{array}{cc}
            \bar{\Ocal}_{[0,n_a-1],y} & \bar{\Ocal}_{[0,n_a-1],u} \\ \hline 0 & \bar{\Ocal}_{[n_a,n_b-2],u} \\ \hline 0 & 0
            \end{array}\right],
        \end{equation}
        with $\bar{\Ocal}_{[0,n_a-1],y} \in \Ree^{n_y n_a \times n_y n_a}$, $\bar{\Ocal}_{[0,n_a-1],u} \in\Ree^{n_y n_a \times n_y (n_b-1)}$ and $\bar{\Ocal}_{[n_a,n_b-2],u} \in \Ree^{n_y (n_b - 1 - n_a) \times n_u (n_b-1-n_a)}$. 
        Now, if $n_u \geq n_y$, then $\left[\begin{array}{cc} \bar{\Ocal}_{[0,n_a-1],y} & \bar{\Ocal}_{[0,n_a-1],u} \\ \hline 0 & \bar{\Ocal}_{[n_a,n_b-2],u}\end{array}\right]\in\Ree^{n_y (n_b-1) \times n_x}$ is wide since $n_x = n_a n_y + n_u (n_b-1) > n_y (n_b-1)$ for $n_u > n_y$ and $n_a > 0$. Consequently, $\rank\ \bar{\Ocal}_k \leq  n_y (n_b-1) < n_x$ as the remaining rows are 0. Alternatively, in case $n_u < n_y$, $\bar{\Ocal}_k$ is tall and it holds that
        \begin{equation}
            \dim (\ker \left[\begin{array}{cc} 0 & \bar{\Ocal}_{[n_a,n_b-2],u}\end{array}\right]^\perp) = n_u(n_b-1-n_a),
        \end{equation}
        such that the row space of $\bar{\Ocal}_k$ satisfies 
        \begin{equation}
        \begin{aligned}
            \dim ( (\ker\ \bar{\Ocal_k})^\perp) 
            \leq &
            \dim (\ker \left[\begin{array}{cc}
                \bar{\Ocal}_{[0,n_a-1],y} & \bar{\Ocal}_{[0,n_a-1],u}
            \end{array}\right]^\perp)
            + 
            \dim (\ker \left[\begin{array}{cc}
            0 & \bar{\Ocal}_{[n_a,n_b-2],u}
            \end{array}\right]^\perp) 
            \\
            & \leq n_y n_a + n_u(n_b-1-n_a) < n_x
        \end{aligned}
        \end{equation}
        where the last inequality holds as $n_a > 0$. Consequently, also for $n_u < n_y$, it holds that $\rank\ \bar{\Ocal}_k < n_x$.
        %Then for both $n_u \geq n_y$ and $n_u < n_y$,it holds that $\dim(\ker(\bar{\Ocal}_k)^\perp < n_x$. 
            
        \item $n_a < n_b -1$ with $k>n_a$ but $k \ngtr n_b-1$: in this case, $\bar{\Ocal}_k$ is a submatrix of \eqref{eq:aux2}, such that directly  $\rank\ \bar{\Ocal}_k < n_x$.
    \end{enumerate}
    Summarizing, by cases 2-4, it holds that $\rank\ \bar{\Ocal}_k < n_x$ for $k > n_a$, and together with 1 thus $\rank\ \bar{\Ocal}_k < n_x$ for all $k>0$ and all $p\in\signalspaceDT$.
\end{proof}

Theorem \ref{th:observability} states that realization \eqref{eq:maximum_state_space} has an unobservable subspace of initial conditions, i.e., there exist initial conditions whose effect cannot be observed in the output $y_k$ for any $k\in\Nee$. This unobservability results from the state definition \eqref{eq:maximum_ss_state_def}, in which inputs and outputs are separately included in the state. In contrast, in the SISO LTI case, minimal realizations such as the observable canonical form are constructed by defining state variables as a function of both inputs and outputs according to $x_{k-\ell} = y_{k-\ell} + u_{k-\ell}$, effectively combining $y_{k-\ell}$, $u_{k-\ell}$ with the same lag $\ell$ into the same state variable \cite[section 2.3.3]{Goodwin1984}, requiring only one state dimension. Thus, separately including inputs and outputs in the state is a source of nonminimality of realization \eqref{eq:maximum_state_space}. Note that the minimal state construction of the LTI case is not possible in the LPV case due to the time dependency of the scheduling signal \cite{Toth2007}. 

Next, Theorem \ref{th:observability} is specialized to the cases in which $n_b=1$ (inverse LPV-FIR) with realization \eqref{eq:maximum_state_space_invFIR} or $n_a=0$ (LPV-FIR) with realization \eqref{eq:maximum_state_space_FIR}.

\begin{corollary}
    State-space realization \eqref{eq:maximum_state_space_invFIR} is structurally $n_a$-observable if and only if there exists a $\bar{p}\in\Pee$ such that $\rank\ A_{n_a}(\bar{p}) = n_y$, and completely $n_a$-observable if and only if $\rank\ A_{n_a}(\bar{p}) = n_y$ for all $\bar{p}\in\Pee$. Last, \eqref{eq:maximum_state_space_invFIR} is always completely $n_a$-reconstructible.
    \end{corollary}
\begin{proof}
    These claims follow directly from the proof of Lemma \ref{lem:Ocalbar} and Theorem \ref{th:reconstructability} and \ref{th:observability} by substituting $\Fcal(p_j)$ and $\Hcal(p_j)$ by $(F_a - G_a A(p_j)$ and $-A(p_j)$. Specifically, $\bar{\Ocal}_k$ for \eqref{eq:maximum_state_space_invFIR} is given by
    \begin{equation}
        \bar{\Ocal}_k = \begin{bmatrix}
            -A(p_0) \\
            -A(p_1) F_a \\
            \vdots \\
            -A(p_{k-1}) F_a^{k-1}            
        \end{bmatrix} \in \Ree^{n_y k \times n_x},
    \end{equation}
    with $n_x = n_a n_y$. Then, for $k < n_a$, it holds that $\rank\ \bar{\Ocal}_k \geq n_y k < n_x $. If $k \geq n_a$, $\bar{\Ocal}_k$ can be written as
    \begin{equation}
        \bar{\Ocal}_k = \left[\begin{array}{ccccc}
            -A_1(p_0) & -A_2(p_0) & \hdots & -A_{n_a-1}(p_0) & -A_{n_a}(p_0) \\
            -A_2(p_1) & -A_3(p_1) & \hdots & -A_{n_a}(p_1) &  \\
            \vdots & & \raisebox{-3pt}{\rotatebox{70}{$\ddots$}} & & \\
            -A_{n_a-1}(p_{n_a-2} & -A_{n_a}(p_{n_a-2}) & & & \\
            -A_{n_a}(p_{n_a-1}) & & & & 0 \\
            \hline
            \multicolumn{5}{c}{0_{k - n_a \times n_x}}
        \end{array}\right].
    \end{equation}
    Consequently, by the block-triangular structure, $\rank\ \bar{\Ocal}_k \leq n_a n_y = n_x$ with equality holding if and only if $\rank\ A_{n_a}(p_j) = n_y$ for $j=0,\ldots,k-1$. Consequently,  \eqref{eq:maximum_state_space_invFIR} is completely observable if and only if $\rank\ A_{n_a}(\bar{p})) = n_y \ \forall \bar{p}\in\Pee$, and it is structurally observable if and only if there exists a $\bar{p} \in \Pee$ for which $\rank\ A_{n_a}(\bar{p})) = n_y$. Complete reconstructability follows by the same argument as Theorem \ref{th:reconstructability}.
\end{proof}
Above corollary links observability of \eqref{eq:maximum_state_space_invFIR} to the rank of the coefficient function $A_{n_a}$ related to the highest delayed output $y_{k-n_a}$. This can be understood by noting that if $A_{n_a}$ does not have full rank, some initial conditions for $y_{k-n_a}$ do not contribute to $y_k$ and are immediately shifted out of the state $\bar{y}_k$. Interestingly, this observability condition is the same as the reachability condition in Corollary \eqref{cor:reachability_invFIR}. Next, \eqref{eq:maximum_state_space_FIR} is considered.

\begin{corollary}\label{cor:FIR_observability}
    State-space realization \eqref{eq:maximum_state_space_FIR} with $n_y \geq n_u$ is structurally $(n_b-1)$-observable if there exists a $\bar{p}\in\Pee$ such that $\rank\ B_{n_b-1}(\bar{p}) = n_u$. It is completely $(n_b-1)$-observable if $\rank\ B_{n_b-1}(\bar{p}) = n_u\ \forall \bar{p}\in\Pee$. Last, \eqref{eq:maximum_state_space_FIR} is always completely $(n_b-1)$-reconstructible. State-space realization \eqref{eq:maximum_state_space_FIR} with $n_y < n_u$ is not observable.
\end{corollary}
\begin{proof}
    For realization \eqref{eq:maximum_state_space_FIR}, the observability matrix is directly given by
    \begin{equation}
        \Ocal_k = 
        \begin{bmatrix}
            B(p_0) \\
            B(p_1) F_b \\
            \vdots \\
            B(p_{k-1}) F_b^{k-1}
        \end{bmatrix} \in \Ree^{k n_y \times n_x},
    \end{equation}
    with $n_x =n_u (n_b-1)$. For $k = n_b-1$, $\Ocal_k$ is given by
    \begin{equation}\label{eq:cor_FIR_proof_Ok}
        \Ocal_{n_b-1} = 
        \left[\begin{array}{ccccc}
            B_1(p_0) & B_2(p_0) & \hdots & B_{n_b-2}(p_0) & B_{n_b-1}(p_0) \\
            B_2(p_1) & B_3(p_1) & \hdots & B_{n_b-1}(p_1) &  \\
            \vdots & & \raisebox{-3pt}{\rotatebox{70}{$\ddots$}} & & \\
            B_{n_b-2}(p_{n_b-3} & B{n_b-1}(p_{n_b-3}) & & & \\
            B_{n_b-1}(p_{n_b-2}) & & & & 0 \end{array}\right] \in \Ree^{n_y (n_b-1) \times n_u (n_b-1)},
    \end{equation}
    %The block-triangular structure of $\Ocal_k$ implies $\rank\ \Ocal_k \leq n_y (n_b-1)$. 
    with $k > n_b-1$ only adding zero rows by the nilpotency of $F_b$. The anti block upper triangular structure of $\Ocal_{n_b-1}$ in \eqref{eq:cor_FIR_proof_Ok} immediately allows for imposing conditions on the rank of $B_{n_b-1}$ to guarantee full rank of $\Ocal_{n_b-1}$ as follows.
    \begin{enumerate}[leftmargin=*]
        \item If $n_y \geq n_u$, i.e., the coefficient functions $B_i$ are square or tall, also $\Ocal_{n_b-1}$ is tall. Then, if there exists a $\bar{p}\in\Pee$ such that $\rank\ B_{n_b-1}(\bar{p}) = n_u$, this implies that $\rank\ \Ocal_k = n_x$ for that $\bar{p}$, i.e., realization  \eqref{eq:maximum_state_space_FIR} is structurally $(n_b-1)$-observable. Moreover, if $\rank\ B_{n_b-1}(\bar{p}) = n_u \ \forall \bar{p}\in\Pee$, this implies that  $\rank\ \Ocal_k = n_x$ for all $p$, i.e., realization \eqref{eq:maximum_state_space_FIR} is completely $(n_b-1)$-observable.
        \item  If $n_y < n_u$, i.e., the coefficient functions $B_i$ are wide, also $\Ocal_{n_b-1}$ is wide, i.e., $n_x = n_u (n_b-1) > n_y (n_b-1)$. Then $\rank\ \Ocal_k < n_x$ for all $p$, and \eqref{eq:maximum_state_space_FIR} is not observable. \vspace{-20pt}%
    \end{enumerate} %
\end{proof}
When $n_y \geq n_u$, Corollary \ref{cor:FIR_observability} links observability of \eqref{eq:maximum_state_space_FIR} to the rank of the coefficient function $B_{n_b-1}$ related to the highest delayed input $u_{k-n_b-1}$. Intuitively, if $B_{n_b-1}$ does not have full rank, initial states associated to $u_{k-n_b+1}$, i.e., the last part of the state, cannot be observed. Note that these conditions are only sufficient, as for $n_y \geq n_u$, all initial conditions might be observed after $k< n_b-1$ time steps. Consider for example that $n_y = n_u (n_b-1)$ and $B_i(p_k)$ is a constant full rank matrix. Then all initial conditions are observed in the first output. Last, in case $n_y < n_u$, a wide $B_{n_b-1}$ does not allow for fully observing the initial states associated to $u_{k-n_b+1}$, such that in this setting \eqref{eq:maximum_state_space_FIR} is always unobservable.

To conclude, it is shown that the state-space realization \eqref{eq:maximum_state_space} resulting from the direct state construction \eqref{eq:maximum_ss_state_def} is not observable and thus not minimal for any coefficient functions $A_i,B_i$. However, it is completely $\max(n_a,n_b-1)$-reconstructible. In the (inverse) FIR setting, \eqref{eq:maximum_state_space_invFIR} and \eqref{eq:maximum_state_space_FIR} can be observable depending on the rank of the highest-order coefficient function $A_{n_a}$ or $B_{n_b-1}$. All findings are summarized in Tab. \ref{tab:summary}.

\begin{table}[t]
\caption{Summary of reachability, observability and reconstructability conditions.}
\label{tab:summary}
\resizebox{\textwidth}{!}{
\begin{tabular}{@{}lll@{}}
\toprule
$n_a>0,n_b>1$ & $n_a>0,n_b=1$ & $n_a=0,n_b>1$ \\ \midrule
\textbf{Structurally $n_x$-reachable} &  &  \\
\begin{tabular}[c]{@{}l@{}} $\textnormal{\rank} \begin{bmatrix} I + \sum_{i=1}^{n_a} \sigma^{-i}A_i(\bar{p}) \\ \sum_{i=0}^{n_b-1} \sigma^{-i} B_i(\bar{p}) \end{bmatrix}^\T = n_y \ \forall \sigma \in \Cee \backslash \{ 0 \}$ \\ $\textnormal{\rank} \begin{bmatrix} -A_{n_a}(\bar{p}) & B_{n_b-1}(\bar{p}) \end{bmatrix} = n_y$\end{tabular} & $\textnormal{\rank}\ B_0(\bar{p}) = n_y$ & Yes \\ \midrule
\textbf{Completely $k$-observable} &  &  \\
No & $k = n_a$ iff $\textnormal{\rank}\ A_{n_a}(\bar{p}) = n_y \ \forall \bar{p} \in \Pee$ & \begin{tabular}[c]{@{}l@{}}$n_y \geq n_u$: if $\textnormal{\rank}\ B_{n_b-1}(\bar{p}) = n_u \ \forall \bar{p} \in \Pee$, $k=n_b-1$ \\ $n_y < n_u$: no\end{tabular} \\ \midrule
\textbf{Structurally $k$-observable} &  &  \\
No & $k=n_a$ iff $\exists \bar{p} \in \Pee$ s.t. $\textnormal{\rank}\ A_{n_a}(\bar{p}) = n_y$ & \begin{tabular}[c]{@{}l@{}}$n_y \geq n_u$: if $\exists \bar{p} \in\Pee$ s.t. $\textnormal{\rank}\ B_{n_b-1}(\bar{p}) = n_u$, $k=n_b-1$ \\ $n_y < n_u$: no\end{tabular} \\ \midrule
\textbf{Completely $k$-reconstructible} &  &  \\
Yes, $k=\max(n_a,n_b-1)$ & Yes, $k=n_a$ & Yes, $k=n_b-1$ \\ \bottomrule
\end{tabular}
}
\end{table}

\begin{remark}
    By complete detectability of \eqref{eq:maximum_state_space}, the state $x_k$ of \eqref{eq:maximum_state_space} satisfies $\lim_{k\mapsto \infty} x_k = 0$ if and only if the output $y_k$ of \eqref{eq:LPV_IO_DT} satisfies $\lim_{k\mapsto \infty} y_k = 0$ since the only directions in which $x_k$ can be nonzero, decay to $0$ in $k=\max(n_a,n_b-1)$ steps by Theorem \ref{th:reconstructability}. Particularly, if considering the class of inputs $\{u \ | \ u_k = 0 \ \forall k > k_1 \} \cap \signalspaceDT$ for some $k_1 \in \Zee$, i.e., inputs that are cut off at $k_1$, complete detectability implies that $y_k$ satisfies $\lim_{k\mapsto \infty} y_k = 0$ if there exists a $\Pcal \in \See_{\succ 0}^{n_x}$ such that $\Fcal^\T(p_k) \Pcal \Fcal(p_k) - \Pcal \prec 0 \ \forall p_k\in\Pee$. If considering only a frozen point $\bar{p}\in\Pee$, i.e., in an LTI setting, this is also necessary, i.e., $\lim_{k\mapsto \infty} y_k = 0$ if and only if there exists a $\Pcal \in \See_{\succ 0}^{n_x}$ such that $\Fcal^\T(\bar{p}) \Pcal \Fcal(\bar{p}) - \Pcal \prec 0$.
    %In other words, stability of the state-space realization \eqref{eq:maximum_state_space} implies stability of the IO representation \eqref{eq:LPV_IO_DT}, and vice versa. Specifically,
    %Thus, for a fixed scheduling signal, IO representation is stable if ad only if its nonminimal state-space realization \eqref{eq:maximum_state_space} satisfies $\Fcal(\bar{p})^\T \Pcal \Fcal(\bar{p}) - \Pcal \prec_0$ for some $\Pcal$.
\end{remark}

\section{Examples}
\label{sec:examples}
In this section, the main observability and reachability results in Theorems \ref{th:DT_controllability} and \ref{th:observability} are illustrated through academic examples\footnote{The code for these examples is available at \protect\url{https://gitlab.tue.nl/kon/nonminimum-lpv-ss-realizations/-/tree/main/DT}}. Specifically, the examples illustrate a non-exhaustive collection of mechanisms through which the realization \eqref{eq:maximum_state_space} can lose observability/reachability, and link these mechanisms to the conditions in Theorems \ref{th:DT_controllability} and \ref{th:observability}. The considered mechanisms are as follows.
\begin{enumerate}[noitemsep]
    \item Including previous inputs $u_{k-i}$ as separate states, as opposed to incorporating it in the state variables together with previous outputs, resulting in a loss of observability, illustrating Theorem \ref{th:observability}.
    \item Not taking into account shared directions in $A_i$, resulting in a loss of reachability, illustrating condition \eqref{eq:last_coefficient_nullspace} of Theorem \ref{th:DT_controllability}. In the LTI case, this corresponds to not taking into account the multiplicity of the pole.
    \item Incorporating states for coefficient functions $A,B$ that are not coprime in a functional sense, resulting in a loss of reachability, illustrating condition \eqref{eq:coprime_theorem}. In the LTI case, this corresponds to a pole-zero cancellation.
    \item A multi-input multi-output (MIMO) LTI example in which the true number of poles is not a multiple of the output dimension, resulting in pole-zero cancellations and a loss of reachability, illustrating condition \eqref{eq:coprime_theorem} of Theorem \ref{th:DT_controllability}.
\end{enumerate}

\subsection{Mechanism 1}\label{sec:mechanism1}
This example illustrates mechanism 1 and 3. Specifically, consider the IO representation
\begin{equation}\label{eq:example1_IO}
    y_k = -a_1(p_k) y_{k-1} - a_2(p_k) y_{k-2} + b_0(p_k) u_k + b_1(p_k) u_{k-1},
\end{equation}
with scheduling function $p: \Zee \mapsto \Pee = [1,\infty)$ and coefficient functions $a_1(p_k) = 2p_k$, $a_2(p_k) = p_k^2$, $b_0 = p_k$, and $b_1(p_k) = p_k^{-1}$.
%. A minimal state-space realization of \eqref{eq:example1_IO} is given by \cite{Toth2007}
% \begin{align}\label{eq:example1_minSS}
%     \hat{x}_{k+1} = \begin{bmatrix} 0 & -p_{k+2} \\ 1 & -p_{k+2} \end{bmatrix} x_k + \begin{bmatrix} p_{k+1} \\ -p_{k+2} p_{k+1} \end{bmatrix} u_k & & y_k = \begin{bmatrix} 1 & 0 \end{bmatrix} \hat{x}_k,
% \end{align}
%in which $\hat{x}_k \in \Ree^2$. 
The nonminimal state-space realization of \eqref{eq:example1_IO} is given by
\begin{subequations}\label{eq:example1_nonminSS}
\begin{align}
    \bar{x}_{k+1} &= \left[\begin{array}{c} y_k \\ y_{k-1} \\ \hline u_{k} \end{array}\right] = \left[\begin{array}{cc|c} -a_1(p_k) & -a_2(p_k) & b_1(p_k) \\ 1 & 0 & 0 \\ \hline 0 & 0 & 0 \end{array}\right] \left[\begin{array}{c}y_{k-1} \\ y_{k-2} \\ \hline u_{k-1}\end{array}\right] + \left[\begin{array}{c} b_0(p_k) \\ 0 \\ \hline 1 \end{array}\right] u_k \\
    y_k &= \left[\begin{array}{cc|c} -a_1(p_k) & -a_2(p_k) & b_1(p_k) \end{array}\right] \bar{x}_k + b_0(p_k) u_k,
\end{align}
\end{subequations}
with $\bar{x}_k \in \Ree^3$. Reachability and observability of this realization are discussed next.

\textbf{Reachability}: To conclude on reachability, Theorem \ref{th:DT_controllability} is applied. Specifically, first consider that $p_k = \bar{p} = 1 \ \forall k\in\Zee$, and consider the condition in \eqref{eq:coprime_theorem}. For $\bar{p}=1$, it holds that $1 + \sum_{i=1}^{n_a} a_i(\bar{p}) \sigma^{-i} = 1 + 2 \sigma^{-1} + 1 \sigma^{-2} = 0$ for $\sigma = -1$. Simultaneously, $\sum_{i=0}^{n_b-1} b_i(\bar{p}) \sigma^{-i} = 1 + 1 \sigma^{-1} = 0$ also holds for $\sigma=-1$, i.e., the polynomials associated with $a_i$ and $b_i$ have a common root $\sigma = -1$ for $\bar{p} = 1$. Consequently, $\begin{bmatrix} 1 + \sum_{i=1}^{n_a} \sigma^{-i}a_i(\bar{p}) & \sum_{i=0}^{n_b-1} \sigma^{-i} b_i(\bar{p}) \end{bmatrix} = 0$ for $\bar{p} = 1$ and $\sigma = -1$. Thus, \eqref{eq:example1_nonminSS} is not reachable at $\bar{p}=1$ by Theorem \ref{th:DT_controllability}. Second, consider that $p_k = \bar{p} = 2$. In this setting, $1 + \sum_{i=1}^{n_a} a_i(\bar{p}) \sigma^{-i} = 0$ holds for $\sigma = -2$. However, for $\sigma = -2$, $\sum_{i=0}^{n_b-1} b_i(\bar{p}) \sigma^{-i} = 2 + 0.5 (-2)^{-1} \neq 0$, such that \eqref{eq:example1_nonminSS} is reachable at $\bar{p}=2$ by Theorem \ref{th:DT_controllability}. Consequently \eqref{eq:example1_nonminSS} is structurally reachable.

\textbf{Observability}: To investigate observability, Theorem \ref{th:observability} is applied. Specifically, by Lemma \ref{lem:Ocalbar}, the $4$-step observability matrix $\Ocal_4 \in \Ree^{4 \times 3}$ of \eqref{eq:example1_nonminSS} is defined through
\begin{align}
    \left[\begin{array}{cccc} 1 & 0 & 0 & 0\\ a_{1}(p_{k+1}) & 1 & 0 & 0\\ a_{2}(p_{k+2}) & a_{1}(p_{k+2}) & 1 & 0\\ 0 & a_{2}(p_{k+3}) & a_{1}(p_{k+3}) & 1 \end{array}\right] \Ocal_4 
    =      
    \left[\begin{array}{ccc} -a_{1}(p_k) & -a_{2}(p_k) & b_{1}(p_k) \\ -a_{2}(p_{k+1}) & 0 & 0 \\ 0 & 0 & 0 \\ 0 & 0 & 0 \end{array}\right] = \bar{\Ocal}_4
 \nonumber
\end{align}
Inspecting $\bar{\Ocal}_4$ yields $\rank\ {\Ocal}_4= \rank\ \bar{\Ocal}_4 = 2 < n_x$ for any $p \in \signalspaceDT$ as $a_2(\bar{p}) \geq 1$ for any $\bar{p} \in \Pee = [1,\infty)$. By Lemma \ref{lem:Ocalbar}, $\rank\ \bar{\Ocal}_k \leq 2$ also for $k > 4$ as only zero rows are added to  $\bar{\Ocal}_k$. Thus, \eqref{eq:example1_nonminSS} is unobservable. However, it is reconstructible in 2 time steps by Theorem \ref{th:reconstructability}. 

To obtain a minimal realization for \eqref{eq:example1_IO}, a state-transformation is defined to project out the unobservable subspace. Specifically, define $v_3(p_k) = \begin{bmatrix} 0 & b_1(p_k) & a_2(p_k) \end{bmatrix}^\T$ as a basis for the unobservable subspace, i.e., $\ker\ \bar{\Ocal}_4 = \im\ v_3(p_k)$. Note that this basis depends on $p_k$. Furthermore, define $v_1 = \begin{bmatrix} 1 & 0 & 0 \end{bmatrix}^\T$ and $v_2(p_k) = \begin{bmatrix} 0 & -a_2(p_k) & b_1(p_k) \end{bmatrix}^\T$ as a basis for the observable subspace, i.e., $(\ker\ \bar{\Ocal}_4)^\perp = \im\begin{bmatrix} v_1(p_k) & v_2(p_k) \end{bmatrix}$. Then define $T(p_k) = \begin{bmatrix} v_1 & v_2(p_k) & v_3(p_k) \end{bmatrix}$, for which it is claimed that $\rank\ T(p_k) = 3$ for any $p_k\in\Pee$. To see this, note that $v_1,v_2,v_3$ are mutually orthogonal, i.e., $v_i^\T (p_k) v_j(p_k) = 0 \ \forall p_k \in\Pee$ for $i=1,2,3$, $j=1,2,3$. A state transformation $x_k = T(p_k) z_k$ and thus $x_{k+1} = T(p_{k+1}) z_{k+1}$ yields the equivalent state-space representation
\begin{subequations} \label{eq:example1_minSS}
\begin{align}
    z_{k+1} &= \left[\begin{array}{cc|c}
        -a_1(p_k) & \gamma(p_k) & 0 \\
        -a_2(p_{k+1}) \gamma^{-1}(p_{k+1}) & 0 & 0 \\
        \hline b_1(p_{k+1}) \gamma^{-1}(p_{k+1}) & 0 & 0
    \end{array} \right] z_k + 
    \begin{bmatrix}
        b_0(p_k) \\ b_1(p_{k+1}) \gamma^{-1}(p_{k+1}) \\ \hline a_2(p_{k+1}) \gamma^{-1}(p_{k+1})
    \end{bmatrix} u_k \\
    y_k &=
    \left[\begin{array}{cc|c} -a_1(p_k) & \gamma(p_k) & 0 \end{array} \right] z_k + b_0(p_k) u_k
    % z_{k+1} &= \left[\begin{array}{cc|c} 
    % -a_1(p_k) & a_2(p_k)^2 + b_1(p_k)^2 & 0 \\ 
    % \frac{-a_2(p_{k+1})}{a_2^2(p_{k+1}) + b_1^2(p_{k+1})} & 0 & 0 \\ 
    % \hline \frac{b_1(p_{k+1})}{a_2^2(p_{k+1}) + b_1^2(p_{k+1})} & 0 & 0 \end{array}\right] z_k + \begin{bmatrix} 0 \\ \frac{b_1(p_{k+1})}{a_2^2(p_{k+1}) + b_1^2(p_{k+1})} \\ \frac{a_2(p_{k+1})}{a_2^2(p_{k+1}) + b_1^2(p_{k+1})}\end{bmatrix} u_k, \\
    % y_k &= \left[\begin{array}{cc|c}  -a_1(p_k) & a_2^2(p_{k}) + b_1^2(p_{k}) & 0 \end{array}\right] z_k,
\end{align}
\end{subequations}
in which $\gamma(q) = a_2(q)^2 + b_1(q)^2$. In this transformed representation, the last state $z_{k,3}$ is not observable. Consequently, a minimal realization with state dimension $2$ for \eqref{eq:example1_IO} is obtained by deleting the last rows/columns of \eqref{eq:example1_minSS}.

To conclude, this example illustrates that including the delayed input $u_{k-1}$ in a separate state results in an unobservable subspace, and that this can be avoided by, for example, combining $u_{k-1}$ and $y_{k-2}$ into a single state as prescribed by $v_2$. 

\begin{remark}
    Note that even though \eqref{eq:example1_nonminSS} is not minimal, its coefficient functions only depend on $p_k$, whereas the coefficient functions of \eqref{eq:example1_minSS} also depend on shifts of the scheduling signal $p_{k+1}$, referred to as a dynamic dependency. Furthermore, the coefficient functions of \eqref{eq:example1_nonminSS} depend affinely on $a_i,b_i$, whereas the coefficient functions of \eqref{eq:example1_minSS} are a nonlinear functions of the original $a_i,b_i$. This nonlinear, dynamic dependency on $a_i,b_i$ complicates stability/dissipativity analysis and controller synthesis, such that in these settings, it can be beneficial to employ the nonminimal realization \eqref{eq:example1_nonminSS}.
\end{remark}
%Message: in SISO case, unobservable subspace is exactly equal to states associated with $u$ that could've been incorporated in states associated with $y$.

\subsection{Mechanism 2}
This example illustrates mechanism 2 in an LTI setting. Specifically, consider the LTI IO representation
\begin{equation}\label{eq:example2_IO}
    y_k = -\begin{bmatrix} .3 & 0 \\ 0 & .3 \end{bmatrix} y_{k-1}  + \begin{bmatrix} 3 \\ 1 \end{bmatrix} u_k = -A_1 y_{k-1} + B_0 u_k,
\end{equation}
for which the corresponding transfer function is given by
\begin{equation}\label{eq:example2_TF}
    y_k = \begin{bmatrix}
        \frac{3}{1 + .3 z^{-1}} \\
        \frac{1}{1 + .3 z^{-1}} 
    \end{bmatrix}.
\end{equation}
A minimal realization for \eqref{eq:example2_TF} is obtained as
\begin{align}\label{eq:example2_minSS}
    \hat{x}_{k+1} = -0.3 \hat{x}_{k} + u_k & & y_k = \begin{bmatrix} -0.9 & -0.3 \end{bmatrix}^\T x_k + \begin{bmatrix} 3 & 1 \end{bmatrix}^\T.
\end{align}
Since both entries of \eqref{eq:example2_TF} contain the same pole at $z = -0.3$, this minimal realization has $\hat{x}_k \in \Ree$, i.e., only one state is required for this pole. In contrast, the nonminimal realization for \eqref{eq:example2_IO} is given by
\begin{align}\label{eq:example2_nonminSS}
    \bar{x}_{k+1} = -A_1 \bar{x}_k + B_0 u_k & & y_k = -A_1 x_k + B_0 u_k
\end{align}
with $\bar{x}_k \in \Ree^2$, which has two uncoupled states that could have been combined into a single state.

The nonminimality of \eqref{eq:example2_nonminSS} manifests through an unreachable subspace of $\bar{x}_k \in \Ree^2$. Specifically, by Corollary \ref{cor:reachability_invFIR}, realization \eqref{eq:example2_minSS} is reachable if $\rank\ B_0 = n_y = 2$. However, $\rank\ B_0 = 1$, such that \eqref{eq:example2_nonminSS} is not reachable. Indeed, the reachability matrix $\Rcal_k = \begin{bmatrix} B_0 & 0.3 B_0 & 0.3^2 B_0 & \hdots & 0.3^{k-1} B_0 \end{bmatrix}$ satisfies $\rank\ \Rcal_k = 1$ for any $k$. The unreachable subspace is thus given by $(\im\ B_0)^\perp = \textrm{span} \begin{bmatrix} 1 & -3 \end{bmatrix}^\T$. Last, note that in the LTI case, Corollary \ref{cor:reachability_invFIR} is not only sufficient for reachability, but also necessary \cite{hespanha2018linear}.

\subsection{Mechanism 3}
This example illustrates mechanism 3. Specifically, consider the IO representation
\begin{equation}\label{eq:example3_IO}
    y_k + p_{k} y_{k-1} = u_k + p_k u_{k-1},
\end{equation}
with $p_k\in\Pee = \Ree$. A nonminimal state-space realization is given by
\begin{align}\label{eq:example3_nonminSS}
    \left[\begin{array}{c}
        y_k \\ \hline u_k
    \end{array}\right]
    =
    \left[\begin{array}{c|c}
        -p_{k} & p_k \\ \hline 0 & 0
    \end{array}\right]
    \left[\begin{array}{c}
        y_{k-1} \\ \hline u_{k-1}
    \end{array}\right]
    + \left[\begin{array}{c}
        1 \\ \hline 1
    \end{array}\right] u_k 
    & & y_k = \left[\begin{array}{c|c} -p_{k} & p_k \end{array}\right] x_k + u_k
\end{align}
with $x_k\in\Ree^2$.

\textbf{Reachability:} By Theorem \ref{th:DT_controllability}, \eqref{eq:example3_nonminSS} is reachable if there exists a $\bar{p}\in\Pee$ such that $\rank \begin{bmatrix} 1 + \bar{p}\sigma^{-1} & 1 + \bar{p}\sigma^{-1}\end{bmatrix} = 1 \ \forall \sigma \in \Cee$ and $\rank \begin{bmatrix} -\bar{p} & \bar{p} \end{bmatrix} = n_y$. However, such a $\bar{p}$ does not exist as the first condition fails if $\bar{p} \neq 0$ for $\sigma = -\bar{p}$, and the second one if $\bar{p} = 0$. Consequently, reachability of \eqref{eq:example3_nonminSS} cannot be concluded from Theorem \ref{th:DT_controllability}. This can be understood by noting that for a constant scheduling $p_k = \bar{p}\in\Pee \ \forall k\in\Zee$, both sides of \eqref{eq:example3_IO} associate with the same polynomial $1+\bar{p} \sigma^{-1}$ and are thus not coprime. In other words, \eqref{eq:example3_IO} corresponds to transfer function $G(z) = (1 + \bar{p} z^{-1})(1 + \bar{p} z^{-1})^{-1}$ for which a pole-zero cancellation occurs for all frozen LTI behaviors. 

Since Theorem \ref{th:DT_controllability} is only sufficient for concluding reachability, also the  reachability matrix is analyzed, which for $k=3$ is given by
\begin{equation}
    \Rcal_3 = \begin{bmatrix} \Fcal(p_{k+2}) \Fcal(p_{k+1}) \Gcal & \Fcal(p_{k+2}) \Gcal & 
    \Gcal \end{bmatrix} = \begin{bmatrix} -p_{k+2} (-p_{k+1} + p_{k+1}) & -p_{k+2} + p_{k+2} & 1 \\ 0 & 0 & 1 \end{bmatrix} = \begin{bmatrix} 0 & 0 & 1 \\ 0 & 0 & 1 \end{bmatrix},
\end{equation}
such that $\rank\ \Rcal_3 = 1$. More generally, the reachable subspace is given by $\im\ \Rcal_k = \im\ \begin{bmatrix} 1 & 1\end{bmatrix}^\T$ for any $p\in\signalspaceDT$ since $\Fcal(p_{k+i}) \Gcal = 0$ for any $p_{k+i}$. Thus \eqref{eq:example3_nonminSS} is unreachable and its unreachable subspace is $p$-independent and given by $\im\ \begin{bmatrix} 1& -1\end{bmatrix}^\T$. This can be intuitively understood by noting that from rest, i.e., for $y_{k-1}=u_{k-1} =0$, it holds by \eqref{eq:example3_IO} that $y_k = u_k$. Consequently, in the next time steps, $p_k y_{k-1}$ and $p_k u_{k-1}$ cancel and again $y_k = u_k$, thus resulting in the given unreachable subspace. 
% \begin{equation}
%     \Rcal_2 = \begin{bmatrix} \Fcal(p_{k+1}) \Gcal & 
%     \Gcal \end{bmatrix} = \begin{bmatrix}-p_{k} + p_{k+1} & 1 \\ 0 & 1 \end{bmatrix}.
% \end{equation}
%Then $\rank\ \Rcal_2 = 2$ if $p_{k} \neq p_{k+1}$, i.e., there exists a $p\in\signalspaceDT$ such that $\Rcal_2$ is full rank. Consequently, \eqref{eq:example3_nonminSS} is not completely $k$-reachable for any $k$, but is structurally $2$-reachable.

\textbf{Observability:} The observability matrix of \eqref{eq:example3_nonminSS} is defined by 
\begin{equation}
    \begin{bmatrix} 1 & 0 & 0 \\ p_{k+1} & 1 & 0 \\ 0 & p_{k+2} & 1 \end{bmatrix}\Ocal_3 = \begin{bmatrix} -p_k & p_k \\ 0 & 0 \\ 0 & 0 \end{bmatrix},
\end{equation}
for $k=3$ such that $\rank \ \Ocal_3 = 1 < n_x$. Generally, by Lemma \ref{lem:Ocalbar}, for any $k$ it holds that $\rank \ \Ocal_k = 1$ with the unobservable subspace given by $\ker\ \Ocal_k = \im\ \begin{bmatrix} 1 & 1 \end{bmatrix}^\T$.

%A minimal realization is obtained by projecting out the unobservable and unreachable subspace. Specifically, the intersection between the observable and reachable subspace is given by this case, $(\ker\ \Ocal_k)^\perp \cap\ \im \ \Rcal_k = \emptyset$, i.e., a state direction is either unobservable $(\begin{bmatrix} 1 & 1 \end{bmatrix}^\T)$ or unreachable $(\begin{bmatrix} 1 & -1 \end{bmatrix}^\T)$, such that a minimal realization


\subsection{Mechanism 4}
This example illustrates Theorem \ref{th:DT_controllability} in the MIMO LTI setting and is adapted from \cite{Alsalti2023}. It is shown that if the true amount of poles is not a multiple of $n_y$, then an IO model can contain extra poles which have to be canceled by zeros, resulting in non-coprimeness of $I+\sum_{i=0}^{n_a} A_i \sigma^{-i}$ and $\sum_{i=0}^{n_b-1} B_i \sigma^{-i}$, violating condition \eqref{eq:coprime_theorem} of Theorem \ref{th:DT_controllability}, such that the nonminimal realization \eqref{eq:maximum_state_space} of the IO model is not reachable.

Specifically, consider a MIMO LTI IO representation with
\begin{align}\label{eq:example4_IO_1}
    y_k \!=\! -\underbrace{\left[\begin{array}{cc} 0.435 & -1.52\\ 0.802 & 0.074 \end{array}\right]}_{A_1} y_{k-1} - \underbrace{\left[\begin{array}{cc} -0.584 & -0.272\\ 1.938 & 1.524 \end{array}\right]}_{A_2} y_{k-2} + \underbrace{\left[\begin{array}{cc} 0.1 & -0.3\\ -0.1 & -0.7 \end{array}\right]}_{B_1}u_{k-1} \underbrace{\left[\begin{array}{cc} 0.286 & -0.294\\ -1.097 & 1.267 \end{array}\right]}_{B_2} u_{k-2}.
\end{align}
A direct nonminimal realization of \eqref{eq:example4_IO_1} can directly be constructed using $A_1,A_2,B_1,B_2$ as in \eqref{eq:maximum_state_space} with $n_a=2,n_b=3$, resulting in an eight-dimensional state.

To test reachability of the direct nonminimal realization, Theorem \ref{th:DT_controllability} is applied, stating that this realization is reachable if
\begin{align}\label{eq:example4_reachability}
    \rank \begin{bmatrix} I + \sum_{i=1}^{n_a} A_i \sigma^{-i} & \sum_{i=0}^{n_b} B_i \sigma^{-i} \end{bmatrix} = n_y \ \forall \sigma \in \Cee \backslash \{ 0 \} & & \textrm{and} & &
    \rank \begin{bmatrix}
        -A_2 & B_2
    \end{bmatrix} = n_y.
\end{align}
The second condition is verified as $\rank\ A_2 = n_y$. To address the first condition, note that $\rank( P(\sigma)) = \rank (I + \sum_{i=1}^{n_a} A_i \sigma^{-i}) < n_y$ only for $\sigma \in \mathfrak{S} = \{ 0.3406 \pm 1.7314i, -1.2806, 0.09066\}$, while $\rank\ P(\sigma) = n_y \ \forall \sigma \backslash \mathfrak{S} $, hence the first condition only has to be tested at $\sigma \in \mathfrak{S}$. This yields that $\rank \begin{bmatrix} I + \sum_{i=1}^{n_a} A_i \sigma^{-i} & \sum_{i=0}^{n_b} B_i \sigma^{-i} \end{bmatrix} = 1 < n_y$ for $\sigma = 0.09066$. Since Theorem 3 is also necessary in the LTI case, the direct nonminimal realization is not reachable and has a one-dimensional unreachable subspace. Intuitively, this can be understood by noting that $Q(\sigma) = \sum_{i=0}^{n_b} B_i \sigma^{-i}$ also has a zero at $\sigma = 0.09066$.

To test observability of the realization, Lemma \ref{lem:Ocalbar} is applied. By this lemma, it holds that the observability matrix $\Ocal_8$ of this direct realization satisfies $\rank\ \Ocal_8 = 4$, resulting in a four-dimensional unobservable subspace.

A minimal realization for the IO representation \eqref{eq:example4_IO_1} is given by \cite{Alsalti2023}
\begin{align}\label{eq:SS_example4}
    x_{k+1} = \left[\begin{array}{ccc} -0.5 & 1.4 & 0.4\\ -0.9 & 0.3 & -1.5\\ 1.1 & 1 & -0.4 \end{array}\right] x_k + \left[\begin{array}{cc} 0.1 & -0.3\\ -0.1 & -0.7\\ 0.7 & -1 \end{array}\right] u_k & & 
    y_k =  \left[\begin{array}{ccc} 1 & 0 & 0\\ 0 & 1 & 0 \end{array}\right] x_k,
\end{align}
i.e., with a state dimension of $n_x=3$, matching with the four-dimensional unobservable and one-dimensional unreachable subspaces of the direct nonminimal realization with a state dimension of eight. 

To conclude, in this example it is shown that if the true amount of poles ($3$) is not a multiple of the output dimension ($2$),  $P(\sigma)$ almost always includes an extra pole that has to be canceled by $Q(\sigma)$ \cite{Gevers1986}, resulting in non-coprime $P(\sigma)$ and $Q(\sigma)$. Consequently, the direct nonminimal realization is not reachable. Thus, in the MIMO setting, it is important to have dedicated parametrizations for IO models that match the true order of the system.

\section{Conclusion}
In this paper, a direct state-space realization for LPV input-output representations is provided and numerical conditions are developed to test its observability and reachability properties. It is shown that this direct realization is nonminimal. However, it is always reconstructible and under coprimeness and well-posedness conditions, this realization is also reachable. 

As a consequence, this state-space realization can immediately be used to verify stability/dissipativity properties of an LPV-IO model obtained using system identification. Additionally, it can be used in several LMI-based controller and observer synthesis methods and recently developed LPV data-driven controller design methods, as the unobservable modes are stable by themselves.
We now turn our attention to the computation of solutions to variational inequalities. In what follows, for simplicity, we will restrict ourselves to VIs $(\set, \vioperset)$ in which $\vioperset$ is singleton-valued, which we will for simplicity denote as $(\set, \vioper)$. In future work, the algorithms and results provided in this chapter could be extended to the more general non-singleton-valued VI setting.
% 
We will in this paper consider first-order methods for computing strong solutions of VIs. We will hereafter restrict ourselves to singleton-valued optimality operators $\vioperset(\vartuple) \doteq \{\vioper(\vartuple))$.
Given a VI $(\set, \vioper)$, and an initial iterate $\vartuple[][][0] \in \set$, a \mydef{first-order method} $\kordermethod$ consists of an update function which generates the sequence of iterates $\{\vartuple[][][\numhorizon]\}_{\numhorizon}$ given for all $\numhorizon = 0, 1, \hdots$ by:
$
    \vartuple[][][\numhorizon + 1] \doteq  \kordermethod \left(\bigcup_{i = 0}^{\numhorizon} (\vartuple[][][i],  \vioper(\vartuple[][][i])) \right).
$

When $\kordermethod$ depends solely on the last item in the sequence, we simply write $ \vartuple[][][\numhorizon + 1] = \kordermethod(\vartuple[][][\numhorizon],  \vioper(\vartuple[][][\numhorizon]))$. As is standard in the literature (see, for instance, \citet{cai2022tight}), given a VI $(\set, \vioper)$, the computational complexity measures in this paper will take the number of evaluations of the optimality operator $\vioper$ as the unit of account.
% 
A common assumption for obtain polynomial-time computation for strong solutions of VIs, is Lipschitz-continuity. Given a modulus of continuity $\lsmooth \geq 0$, a $\lsmooth$-\mydef{Lipschitz-continuous} VI is a VI $(\set, \vioper)$ s.t. $\set$ is non-empty, compact, and convex, and $\vioper$ is $\lsmooth$-Lipschitz continuous.


\section{Conclusion}
In this work, we propose a simple yet effective approach, called SMILE, for graph few-shot learning with fewer tasks. Specifically, we introduce a novel dual-level mixup strategy, including within-task and across-task mixup, for enriching the diversity of nodes within each task and the diversity of tasks. Also, we incorporate the degree-based prior information to learn expressive node embeddings. Theoretically, we prove that SMILE effectively enhances the model's generalization performance. Empirically, we conduct extensive experiments on multiple benchmarks and the results suggest that SMILE significantly outperforms other baselines, including both in-domain and cross-domain few-shot settings.


\newpage
\bibliography{biblio.bib}

%\acks{The author would like to thank Nadav Merlis and Hugo Richard for their thoughtful comments on the manuscript, as well as Austin J. Stromme for sharing his insights in statistical optimal transport.}

%%%%%%%%%%%%%%%%%%%%%%%%%%%%%%%%%%%%%%%%%%%%%%%%%%%%%%%%%%%%%%%%%%%%%%%%%%%%%%%
%%%%%%%%%%%%%%%%%%%%%%%%%%%%%%%%%%%%%%%%%%%%%%%%%%%%%%%%%%%%%%%%%%%%%%%%%%%%%%%
% APPENDIX
%%%%%%%%%%%%%%%%%%%%%%%%%%%%%%%%%%%%%%%%%%%%%%%%%%%%%%%%%%%%%%%%%%%%%%%%%%%%%%%
%%%%%%%%%%%%%%%%%%%%%%%%%%%%%%%%%%%%%%%%%%%%%%%%%%%%%%%%%%%%%%%%%%%%%%%%%%%%%%%

\newpage

\appendix
\crefalias{section}{appendix} % uncomment if you are using cleveref
\crefalias{subsection}{subappendix} 
\part*{Appendices}
\section{Preliminaries}\label{app: intro}

\subsection{Organisation of Appendices}\label{subsec: organisation of appendices}

The following appendices are organised thematically and are mostly independent completions of various parts of the text. \Cref{subsec: notational precisions} contains notations and clarifications that are shared across them. 

\Cref{app: fourier} provides a rigourous treatment of necessary  Fourier analysis notions, which allow for a rigourous outlining of the schema detailed in \cref{subsec: measure valued actions}. 

\Cref{app: technical details,app: regret bounds,app: computation} contains the majority of the technical contributions of this work, including the major lemmata used in the proofs of the main text. \Cref{app: technical details} is dedicated to the details of the constructions in \cref{sec: Preliminaries}, while \cref{app: regret bounds} focuses on the general regret proofs of \cref{sec: regret of learning}, specifically the proofs of \cref{thm: regret Kantorovich,thm: entropic regret}. Finally \cref{app: computation} is dedicated to the details of \cref{subsec: computability} on specific regularity dependent regret.
Some miscellaneous minor results, or reproductions of results from prior works are collected in \cref{app: lemmas}.

The remaining appendices contain complements to the text and discussion of topics not mentioned therein for the sake of brevity. \Cref{app: open problems} contains more detailed discussions of the open problems mentioned in \cref{sec: directions}. \Cref{app: biblio} contains bibliographical notes on statistical optimal transport which readers unfamiliar with the field might find of interest to understand the context of the paper. It is a complement to \cref{sec:RWCC}.


\subsection{Notational precisions}\label{subsec: notational precisions}

Throughout the text, for a reference measure $\varrho$, let $L^p(\state,\Kb ;\varrho)$, $p\in[1,\infty]$ and $\Kb\in\{\Rb,\Cb\}$, denote the space of functions $f:\state\to\Kb$ that are $p$-integrable. When $\state$, $\Kb$, or $\varrho$ are clear from context we will drop them for brevity; by default $\Kb=\Cb$. We allow complex functions ($\Kb=\Cb$) to deal with the Fourier transforms, but this has no noticeable effect as it does not impact the Hilbertian structure of the space $L^2(\Rb^d,\Kb;\varrho)$. 

In the following, let $\langle \cdot\vert\cdot\rangle_{L^2(\Rb^d,\varrho)}$ denote the inner product on $L^2(\Rb^d,\Kb;\varrho)$, $\langle\cdot\vert\cdot\rangle_{\ell_2(\Rb^d)}$ the one on $\ell^2(\Rb,\Kb)$ (the space of square integrable real sequences) with $\norm{\cdot}_{\ell_2(\Rb^d)}$ denoting its associated norm. On $\Rb^d$, $\langle\cdot\vert\cdot\rangle_{2}$ denotes the inner product, $\norm{\cdot}_2$ the Euclidean norm. As before, let  $\langle\cdot\vert\cdot\rangle$ denote the duality pairing between $\measures(\Rb^d)$ (the space of finite Radon measures) and $\Cc_0(\Rb^d)$ (the space functions vanishing at infinity). The operator norm of a linear operator (in finite or infinite dimension) $A$ is denoted by $\norm{A}_{\op}$.

Throughout, all probabilistic statements are understood as holding in the filtered probability space $(\Omega,\Fc_\infty,\Fb,\Pb)$, in which $\Fb:={(\Fc_t)}_{t\in\Nb}$ is the natural filtration of ${(\xi_t)}_{t\in\Nb}$, and $\Fc_\infty=\lim_{t\to\infty}\Fc_t$.

For two measures $(\gamma,\rho)\in\measures(\Rb^d)$, $\gamma\ll\rho$ denotes that $\gamma$ is absolutely continuous with respect to $\rho$, in which case we use ${\de \gamma}/{\de \rho}$ to denote the Radon-Nikodym derivative (a.k.a.\ the density) of $\gamma$ with respect to $\rho$.
\section{Elements of Fourier Analysis}\label{app: fourier}

\subsection{Formal definitions}\label{subsec: fourier defs}

To define the Fourier transform on $L^2(\Rb^d;\varrho)$, we will extend it from a dense subspace (see \cref{def: schwartz space}) of $L^2(\Rb^d;\varrho)$ to the whole space. This technical construction arises as a consequence of the fact that $L^2(\Rb^d;\varrho)\not\subset L^1(\Rb^d;\varrho)$, meaning the right-hand side of~\eqref{eq: fourier transform} may not be defined and $\fourier$ is ill-posed on $L^2(\Rb^d;\varrho)$, despite the fact that~\eqref{eq: fourier transform} is well-posed for $f\in L^1(\Rb^d;\varrho)$. The following is summarised from~\cite[Ch.5--6]{constantin_fourier_2016}, refer therein for a more detailed treatment or, e.g.,\ to \citep{folland_fourier_1992}. 

\begin{definition}\label{def: schwartz space}
    The Schwartz space $\Sc(\Rb^d)$ is defined as 
    \[ 
        \left\{\phi\in\Cc^\infty(\Rb^d;\Cb) : \sup_{x\in\Rb^d}\abs{x^\alpha\partial_\beta\phi(x)}<+\infty \mbox{ for any } \alpha,\beta\in\Nb^d\right\}
    \]
    in which $\alpha,\beta\in\Nb^d$ are multi-indices so that $x^{\alpha}:={(x_i^{\alpha_i})}_{i=1}^d$, and $\partial_\beta:=\partial_{x_1}^{\beta_1}\cdots\partial_{x_d}^{\beta_d}$.
\end{definition}

Note that $\Sc(\Rb^d)$ is a dense subspace of $L^2(\Rb^d;\varrho)$ and $L^1(\Rb^d;\varrho)$ as it contains $\Cc^\infty_c(\Rb^d;\Cb)$ the space of infinitely-differentiable compactly-supported (a.k.a.\ test) functions, which is dense in both $L^2(\Rb^d;\varrho)$ and  $L^1(\Rb^d;\varrho)$.

\begin{theorem}[{\cite[Thm.~6.1]{constantin_fourier_2016}}]\label{thm: constantin def fourier on Schwartz}
    Consider the Fourier transform operator  $\fourier$ on the Schwartz space, with 
    \begin{align}
        \fourier: \phi\in\Sc(\Rb^d)\mapsto\int \phi(x)e^{-2\pi i\langle x\vert \cdot\rangle_2}\de\varrho(x)\,.\label{eq: fourier transform}
    \end{align}
    This operator maps $\Sc(\Rb^d)$ maps onto itself and is an isometric bijection. Moreover, 
    \begin{align}
        \fourier^{-1}=\fourier\reflection\,,\label{eq: inverse fourier (formal operator notation)}
    \end{align}    
    in which $\reflection:\phi\in\Sc(\Rb^d)\mapsto \phi(-\cdot)\in\Sc(\Rb^d)$ is the \emph{reflection} operator.
\end{theorem}

\begin{theorem}[{\cite[Thm.~6.4]{constantin_fourier_2016}}]\label{thm: constantin fourier extension}
    The fourier transform $\fourier$ can be extended to a unitary operator on $L^2(\Rb^d;\varrho)$ and~\eqref{eq: inverse fourier (formal operator notation)} holds on $L^2(\Rb^d;\varrho)$ for this extension.
\end{theorem}

The formal inversion property~\eqref{eq: inverse fourier (formal operator notation)} is easily shown to recover the classical inversion formula 
\begin{align}
    f(x)=\int \fourier f(\xi)e^{2\pi i\langle x\vert \xi\rangle}\de\varrho(\xi) \mbox{ for $\varrho$-a.e. }x\in\state\, \label{eq: inverse fourier transform}
\end{align}
as soon as $f\in L^1(\Rb^d;\varrho)\cap L^2(\Rb^d;\varrho)$. In our case $\varrho$ is a finite measure so $L^2(\Rb^d;\varrho)\subset L^1(\Rb^d;\varrho)$ and the inversion formula always holds. If $\varrho$ is only $\sigma$-finite (e.g.\ the Lebesgue measure), one must take slightly higher care. Namely 
the difference between~\eqref{eq: inverse fourier (formal operator notation)} and~\eqref{eq: inverse fourier transform} is whether the integral in~\eqref{eq: inverse fourier transform} is well defined for $f\in L^2(\Rb^d;\varrho)$, which is not guaranteed. 

This technicality reflects the limits used in the definition of the extension which are hidden by the abstract statement of \cref{thm: constantin fourier extension}. Nevertheless, since the Schwartz space $\Sc(\Rb^d)$ is dense in both $L^1(\Rb^d;\varrho)$ and $L^2(\Rb^d;\varrho)$, we can always take an arbitrarily close function in $\Sc(\Rb^d)$ and invert that, the result will remain arbitrarily close in $L^2(\Rb^d;\varrho)$.


The Schwartzian framework turns out to be a robust one for Fourier analysis more generally, and we can also use to extend $\fourier$ beyond $L^2(\Rb^d;\varrho)$. In particular, it can be used to unify the definitions we gave for the Fourier transform of a function and a measure, refer to~\cite[\S~6.1.2]{constantin_fourier_2016} for more details. Precisely, one extends to the topological dual of $\Sc(\Rb^d)$ (the space of tempered distributions $\Sc'(\Rb^d)$), which includes $\measures(\Rb^d)$ and $L^2(\Rb^d;\varrho)$ as sub-spaces.


A fundamental consequence of the various formulations of the  Fourier transform is that measures whose transforms are in $L^2(\Rb^d;\varrho)$ are exactly those which have an $L^2(\Rb^d;\varrho)$ density with respect to $\varrho$. We will denote the density of a measure $\mu$ with respect to $\varrho$ using the Radon-Nikodym notation $\de\mu/\de\varrho$, even when this tempered distribution can be identified with a function.

\begin{lemma}\label{lemma: fundamental facts about fourier transform of measure}
    Let $\gamma\in\measures(\state)$ be a finite Radon measure, if it has density with respect to $\varrho$ and $\de\gamma/\de\varrho\in L^2(\Rb^d;\varrho)$, then 
    \[
        \fourier\gamma = \fourier \frac{\de\gamma}{\de\rho}\in L^2(\Rb^d;\varrho)\,.
    \]
    Conversely, if $\fourier\gamma\in L^2(\Rb^d;\varrho)$, then $\gamma$ has a density with respect to $\varrho$ and $\de\gamma
/\de\varrho\in L^2(\Rb^d;\varrho)$.
\end{lemma}
\begin{proof}
    The first part is a direct consequence of the definitions of the Fourier transforms of a measure and an $L^2(\Rb^d;\varrho)$ function. For the converse, the fact that $\fourier\gamma
\in L^2(\Rb^d;\varrho)$ implies $\gamma
 \ll \varrho$ involves some technical minutiae due to the different topologies $\measures(\state)$ can be equipped with, which we won't reproduce for conciseness, refer to e.g.\ \cite[Lemma~1.1]{fournier_absolute_2010}. That the density is then in $L^2(\varrho)$ is a simple consequence of Plancherel's theorem:
    \[
        \norm{\frac{\de\gamma}{\de\rho}}_{L^2(\Rb^d;\rho)} =\int_{\Rb^d}\abs{F\gamma(\xi)}^2\de\rho(\xi) = \norm{F\gamma}_{L^2(\Rb^d;\rho)}\,.%\qedhere
    \]
\end{proof}



\subsection{Technical details of Section {\ref{subsec: measure valued actions}}}



Let $\Cc_0(\Rb^d,\Kb)$ denote the space of continuous functions from $\Rb^d$ to $\Kb\in\{\Rb;\Cb\}$, $\measures(\Rb^d)$ denote the space of finite Borel measures over $\Rb^d$, and let us define the Fourier operator on this space by using the same notation, i.e.\ $\fourier: \gamma\in\measures(\Rb^d)\mapsto\fourier\gamma\in\Cc_0(\Rb^d;\Cb)$ with
\begin{align}
    \fourier\gamma: \xi\in\Rb^d \mapsto \int e^{-2\pi i\langle x\vert \xi\rangle_2}\de\gamma(x)\,
    .\label{eq: fourier transform of measure}
\end{align}
Note that we will eschew the standard notations $\hat f$ and $\hat\gamma$ in favour of $\fourier f$ and $\fourier\gamma$ to avoid confusion with the least-squares estimator, which we will denote using its standard hat.


% 

The Riesz-Markov theorem shows that $(\measures^*(\Rb^d),\norm{\cdot}_\infty)$, the space of finite signed Borel measures on $\Rb^d$ (endowed with the total variation norm $\norm{\cdot}_\infty$), is the topological dual of $(\Cc_0(\Rb^d),\norm{\cdot}_\infty)$, the space of continuous functions which vanish at infinity (endowed with the supremum norm $\norm{\cdot}_\infty$), refer e.g.\ to~\cite[p.~242]{constantin_fourier_2016}. This duality is characterised by the pairing
\[
    \langle \cdot\vert\cdot\rangle: (f,\gamma
)\in\Cc_0(\Rb^d)\x\measures^*(\Rb^d)\mapsto \int f \de \gamma
 \in \Rb\,.
\]
This pairing applies in particular to all functions $f\in\Cc(\state;\Rb)$ if $\state$ is compact and to all positive finite Borel measures $\gamma\in\measures^+(\state)$, and we will use the pairing notation in this case too. In general we will use the notation for arbitrary functions, understood that it will be well defined, see also \cref{remark: assumption L2 case}. In particular:
\[
\kant(\mu,\nu,c) = \inf_{\pi\in\Pi(\mu,\nu)}\langle c\vert \pi\rangle\,.
\] 

\begin{lemma}\label{lemma: finiteness of IP}
    For any finite Borel measure $\rho\in\measures(\Rb^d)$, any $\gamma\in\measures(\Rb^d)$ finite and with $\de\mu/\de\rho\in L^2(\Rb^d;\rho)$, and any $f\in L^2(\Rb^d;\rho)\cap L^1(\Rb^d;\rho)$, we have
    \[
        \langle f\vert \gamma\rangle = \langle \fourier\reflection f\vert \fourier\gamma\rangle_{L^2(\Rb^d;\rho)}\,
    \]
    and 
    \[
        \abs{\langle f\vert \gamma\rangle}\le \norm{f}_{L^2(\Rb^d;\rho)}\abs{\rho(\Rb^d)}\abs{\gamma(\Rb^d)}\,.
    \]
\end{lemma}

\begin{proof}
    By~\eqref{eq: inverse fourier transform}, 
    \begin{align}
        \langle f\vert\gamma\rangle:=\int f\de \gamma &=\int\int \fourier f(\xi)e^{2\pi i\langle x\vert \xi\rangle}\de\rho(\xi)\de\gamma(x)\,.\label{eq: fourier double integral}
    \end{align}
Let $\varphi:(x,\xi)\mapsto e^{2\pi i \langle x\vert\xi\rangle}$. Using~\eqref{eq: fourier double integral}, since by the Cauchy-Schwartz inequality
\begin{align}
    \abs{\langle f\vert\gamma\rangle} &\le \norm{Ff}_{L^2(\Rb^d\x\Rb^d;\gamma\tensor\rho)}\norm{1}_{L^2(\Rb^d\x\Rb^d;\gamma\tensor\rho)}\notag\\
    &= \norm{Ff}_{L^2(\Rb^d;\rho)}\gamma{(\Rb^d)}^2\rho(\Rb^d)<+\infty\label{eq: bound of IP in fourier space}\,,
\end{align}
the integrand in~\eqref{eq: fourier double integral} is $\gamma\tensor\rho$-integrable, and thus we can apply the Fubini-Lebesgue theorem to obtain
\begin{align*}
    \langle f\vert\gamma\rangle&= \int \fourier f(\xi) e^{2\pi i\langle x\vert \xi\rangle}\de[\gamma\tensor\rho](\xi,x)=\langle Ff\vert \varphi\rangle_{L^2(\Rb^d\times\Rb^d;\gamma\tensor\rho)}\,.
\end{align*}
Furthermore,
    \begin{align*}
        \langle f\vert\gamma\rangle&= \int \fourier f(\xi)\int e^{2\pi i\langle x\vert \xi\rangle}\de\gamma(x)\de\rho(\xi)\\
        &= \langle \fourier\reflection f\vert \fourier \gamma\rangle_{L^2(\Rb^d;\rho)}.
    \end{align*}
By~\eqref{eq: bound of IP in fourier space}, we have once more:
    \[
        \abs{\langle f\vert\gamma\rangle}=\abs{\langle \fourier\reflection f\vert \fourier \gamma\rangle_{L^2(\Rb^d;\rho)}} \le \norm{Ff}_{L^2(\Rb^d;\rho)}\gamma{(\Rb^d)}^2\rho(\Rb^d)\,. %\qedhere
    \]
\end{proof}

The benefit of \cref{lemma: finiteness of IP} may not be immediately apparent, but it is revealed when one notices that the $L^2(\Rb^d;\rho)$ inner products and norms considered on the right hand side depend only on the measure $\rho$ and not on $\gamma$. Thus, we are able to assume only integrability of $c^*$ only with respect to our reference measure $\varrho$ (recall~\eqref{eq: def entropy}) and still manipulate the duality product $\langle c^*\vert \gamma\rangle$ for any $\gamma$. In particular, by taking $\varrho=\mu\tensor\nu$ given marginals $\mu$ and $\nu$ and playing $\pi_t$ such that $\entf(c^*,\pi_t)<+\infty$ (recall~\eqref{eq: entropic OT def}) we can reduce $\langle c^*\vert \pi_t\rangle$ to a $L^2(\Rb^d;\varrho)$ inner product, moving our problem to a Hilbert space.

\begin{remark}\label{remark: assumption L2 case}
    \Cref{lemma: finiteness of IP} opens the subject of discussing \cref{asmp: L2 case}. Let us remark that if $S:=\supp(\mu\tensor\nu)$ is compact, continuity of $c^*$ on the closure of $S$ is sufficient to obtain these results. Similarly, if $c^*$ is bounded. However, \cref{asmp: L2 case} allows for many more functions, for instance it allows $c^*:(x,y)=\norm{x-y}^2_2$ if $(\mu,\nu)\in\Ps_2(\Rb^d)$, where $\Ps_2(\Rb^d)$ denotes measures with a finite second moment. This is of value as it covers the Wasserstein distances which are of broad interest. In general, one can develop finer assumptions based on $(\mu,\nu)$ even if $\varrho$ is not finite, but we do not detail this for brevity.
\end{remark}




\section{Technical contributions in Bandit Theory}\label{app: technical details} %might actually become redundant and end up in appendices

\subsection{Confidence sets and Regularised least-squares}\label{subsec: conf sets and RLS}

%Notice that, since $\fourier$ is a linear isometry on $L^2(\Rb^d,\Cb;\varrho)$ (\cref{thm: constantin fourier extension}), $\ffset:=\fourier\fset$ is also convex.

Recall that $R_t:=\langle c^*\vert \pi_t\rangle +\xi_i=\langle \fourier\reflection c^*\vert \fourier \pi_t\rangle_{L^2(\Rb^d;\varrho)} + \xi_t$ (by \cref{lemma: finiteness of IP}), in which by \cref{asmp: estimate + subG} we have ${(\xi_i)}_{i\in\Nb}$ a conditionally $\sigma$-sub-Gaussian sequence. Let $a_t:=\fourier\pi_t\in L^2(\Rb^d;\varrho)$ for $t\in\Nb$, and $\vec{a}_t:={(a_i)}_{i=1}^t$ and $\vec{R}_t:={(R_i)}_{i=1}^t$.

Let us begin by defining the regularised least-squares estimator of $c^*$. 
Let $J_\cdot[\cdot]:\Nb\x\ffset\to\Rb$ be the (random) functional defined by 
\[
    (t,f)\mapsto J_t[f]:= \sum_{s=1}^t\norm{R_s - \langle f\vert a_s\rangle_{L^2(\Rb^d;\varrho)}}^2_2 \,.
\]
Consider $\Lambda:L^2(\Rb^d;\varrho)\to\Rb$, a strongly convex and continuously Fréchet-differentiable functional whose Fréchet derivative, denoted $\De \Lambda$, satisfies
\begin{align}
    \frac1{M_\Lambda}\norm{f}_{L^2(\Rb^d;\varrho)}\le \De^2\Lambda[f] \le M_\Lambda \norm{f}_{L^2(\Rb^d;\varrho)} \quad\mbox{for any}\quad f\in L^2(\Rb^d;\varrho)\label{eq: Fréchet derivative of Lambda}
\end{align} 
for some $M_\Lambda>0$, e.g.\ $\Lambda=\frac12\norm{\cdot}_{L^2(\Rb^d;\varrho)}^2$ with $M_\Lambda=1$. Let us recall that the Fréchet derivative of a strongly convex Fréchet-differentiable functional is a (strongly) positive-definite operator denoted $\De \Lambda$.
It is clear that $J_t+\lambda\Lambda$ is a strongly convex functional for any $\lambda>0$ and $t\in\Nb^*$ as $\ffset$ is convex. Therefore, we can define the $\Lambda$-regularised least-squares estimator of $c^*$ to be
\[
    \hat f_\lambda:=\argmin_{f\in L^2(\Rb^d;\varrho)} J_t[f]+ \lambda\Lambda[f]\,.
\]
\begin{proposition}\label{prop: least squares estimator}
    Assume \cref{asmp: L2 case}, then for any $\lambda>0$, and $t\in\Nb^*$, we have 
    \begin{align}
        \hat f_t^\lambda = {(M_t^*M_t+\lambda\De \Lambda)}^{-1} M^*_t \vec{R}_t\,,\label{eq: reg least squares estimator}
    \end{align}
    in which, for every $t\in\Nb^*$, $M_t:L^2(\Rb^d;\varrho)\to \Rb^{t}$ is the linear a.s.\ bounded operator defined by 
    \begin{align}
        M_t: f\in L^2(\Rb^d;\varrho) \mapsto {(\langle f\vert a_t \rangle_{L^2(\Rb^d;\varrho)})}_{i=1}^t\in\Rb^t  \,,
    \end{align}
    and $M^*_t:\Rb^t\to L^2(\Rb^d;\varrho)$ is its adjoint, defined by
    \begin{align}
        M^*_t: v\in\Rb^t\mapsto \sum_{s=1}^t v_s a_s \in L^2(\Rb^d;\varrho) \quad \mbox{ for any } v\in\Rb_t\,.
    \end{align}
\end{proposition}

\begin{proof}
    This proof extends the standard arguments for finite-dimensional least-squares, we include it for completeness focusing on the differences owing to infinite dimensions, cf.\ e.g.\ \citep[\S~3.2]{abbasi-yadkori_online_2012}. One first computes the Fréchet derivative of $J_t$, by studying a variation $\delta f\in L^2(\Rb^d;\varrho)$ and
    \[
    J_t[f+\delta f] - J_t[f]\,.
    \]
    One sees that the Fréchet derivative of $J_t$ exists for all $t$ and is given by 
    \[
        f\mapsto \sum_{s=1}^t\langle f\vert a_s\rangle_{L^2(\Rb^d;\varrho)}a_s- R_s a_s +\lambda\De\Lambda f= (M^*_t M_t+\lambda\De\Lambda)f -M^*_t\vec{R}_t\,.
    \]
    Note that the right-hand side is easily checked by expanding the definition of $M_t$ and $M^*_t$, and in doing so one easily checks that $M^*_t$ is indeed the adjoint of $M_t$. Carrying on, by first order optimality, the normal equations are 
    \[
        (M^*_t M_t+ \lambda\De\Lambda)\hat f_\lambda = M^*_t \vec{R}_t\,.
    \]
    Since $M^*_t M_t$ is positive semi-definite and $\De\Lambda$ is positive definite,~\eqref{eq: reg least squares estimator} follows. 
\end{proof}

Let $\design_t:=M^*_t M_t$ and $\designl_t:=\design_t+\lambda\De\Lambda$ denote the design and regularised design operators at time $t\in\Nb$. Let
    \begin{align}
        \event_t(\delta):=\left\{\hspace{-1pt}\norm{\hat f_t^\lambda - \fourier c^*}_{\designl_t}\hspace{-1pt}\le \sigma\sqrt{\log\hspace{-3pt}\left(\frac{4\det(\De\Lambda+\lambda^{-1}M_t M^*_t)}{\delta^2}\right)} +{\left(\frac{\lambda}{\norm{\De\Lambda}_{\op}}\right)}^{\hspace{-2pt}\frac12}\hspace{-5pt}\norm{\fourier c^*}_{L^2(\Rb^d;\varrho)}\right\}.\label{eq: event def}
    \end{align}
for $t\in\Nb$.

\begin{lemma}[{\cite[Cor.~3.6]{abbasi-yadkori_online_2012}}]\label{lemma: probability of confsets unif in N}
    For every $\delta\in(0,1)$, $\lambda>0$, under \cref{asmp: L2 case,asmp: estimate + subG} we have 
    \[  \Pb\left(c^*\in \bigcap_{t\in\Nb}\fourier^{-1}\confset_t(\delta)\right)\ge\Pb\left(\bigcap_{t\in\Nb} \event_t(\delta/2)\right)\ge 1-\frac\delta2 \,.\]
\end{lemma}

\begin{proof}
    Recall that $\fourier$ is an isometry on $L^2(\Rb^d;\varrho)$, and so is $\fourier^{-1}$, so $\fourier^{-1}\confset_t$ is a confidence set for $c^*$ in $L^2(\Rb^d;\varrho)$, and it is a ball of identical radius $\width_t(\delta)$ centred at $\fourier^{-1} \hat f_t^\lambda$. 
    A direct combination of \cref{asmp: estimate + subG},~\eqref{eq: event def}, and~\cite[Cor.~3.6]{abbasi-yadkori_online_2012} yields 
    \[
        \Pb\left(\bigcap_{t\in\Nb} \event_t(\delta/2)\right)\ge 1-\frac\delta2\,.
    \]
    The second results follow by comparison of~\eqref{eq: event def} and~\eqref{eq: confidence set width def}.
\end{proof}



\begin{lemma}\label{lemma: bound on width term}
    Under \cref{asmp: L2 case,asmp: estimate + subG}, on the event $\{c^*\in \cap_{t\in\Nb}\fourier^{-1}\confset_t(\delta)\}$, for any $T\in\Nb$ and ${(c_t)}_{t=1}^T$ with $c_t\in\fourier^{-1}\confset_t(\delta)$ for $t\in[T]$, we have
    \[
        \sum_{t=1}^T\langle c^* - c_t\vert \tilde\pi_t\rangle \le 2\ubnorm\width_T(\delta)\sqrt{T\log\det\left(\Id + \frac{1}{2\lambda\ubnorm}M_t{(\De\Lambda)}^{-1}M_t^*\right)}
    \]
\end{lemma}
\begin{proof}
    Consider $t\ge 0$, $c_t\in\confset_t(\delta)$, and let $\varphi_t:=\langle c^* - c_t\vert \tilde\pi_i\rangle$. Recall that $a_t:=\fourier\pi_t$. By \cref{lemma: probability of confsets unif in N} and the Cauchy-Schwartz inequality, on the event $\{c^*\in \cap_{t\in\Nb}\fourier^{-1}\confset_t(\delta)\}$, we have 
    \begin{align*}
        \abs{\varphi_t} \le \width_t(\delta)\norm{a_t}_{{(\designl_t)}^{-1}}\,,
    \end{align*}
    while, by the Cauchy-Schwartz inequality, \cref{asmp: L2 case}, and using the fact that $\fourier$ is an isometry on $L^2(\Rb^d;\varrho)$, we have 
    \begin{align*}
        \abs{\varphi_t} &\le \norm{\fourier\reflection c^* - \fourier\reflection c}_{L^2(\Rb^d;\varrho)}\norm{a_t}_{L^2(\Rb^d;\varrho)}= \norm{c^*-c}_{L^2(\Rb^d;\varrho)}\pi_t(\Rb^d)\le 2\ubnorm\,.
    \end{align*}
    Combining yields
    \begin{align*}
        \abs{\varphi_t}\le \width_t(\delta)\min\{\norm{a_t}_{{(\designl_t)}^{-1}},2\ubnorm\}=2\ubnorm\width_t(\delta)\left(\frac{1}{2\ubnorm}\norm{a_t}_{{(\designl_t)}^{-1}}\wedge 1 \right) \,.
    \end{align*}
    Squaring and applying the inequality $u\le 2\log(1+u)$, which holds on $[0,1]$, to the final term, yields
    \begin{align*}
        \abs{\varphi_t}^2\le 8\ubnorm^2{\width_t(\delta)}^2\log\left(1+ \frac{1}{2\ubnorm}\norm{a_t}_{{(\designl_t)}^{-1}} \right)
    \end{align*}
    and, summing up,
    \begin{align}
        \sum_{t=1}^T \varphi_t\le \sqrt{T\sum_{t=1}^T \abs{\varphi_t}^2}\le 2\ubnorm\width_t(\delta)\sqrt{T\sum_{t=1}^T \log\left(1+ \frac{1}{2\ubnorm}\norm{a_t}_{{(\designl_t)}^{-1}}\right)} \label{eq: summation of widths in proof of the bound on width term lemma}\,.
    \end{align}
    % On the one hand, by \cref{lemma: bound on size of actions}, we have
    % \[ 
    %     \varphi^2\le 4\ubnorm^2\width_t(\delta)^2\log\left(1+\frac{1}{2\ubnorm\lambda M_\Lambda}\right)\,
    % \]
    % so that 
    % \begin{align}
    %     \sum_{t=1}^n \varphi_t\le \sqrt{n\sum_{t=1}^n \varphi_t^2}\le 2\ubnorm\width_t(\delta)\sqrt{n\sum_{t=1}^n\log\left(1+\frac{1}{2\ubnorm\lambda M_\Lambda}\right)} 
    % \end{align}
    By definition of $M_T$ and $\designl_T$, we have
    \begin{align}
        \sum_{t=1}^T \log\left(1+ \frac{1}{2\ubnorm}\norm{a_t}_{{(\design_t^\lambda)}^{-1}}\right)=\log\left(\prod_{t=1}^T \left(1+ \frac{1}{2\ubnorm}\norm{a_t}_{{(\design_t^\lambda)}^{-1}}\right)\right)\notag\\
        =\log\det\left(\Id + \frac{1}{2\lambda\ubnorm}M_T{(\De\Lambda)}^{-1}M_T^*\right)\,\label{eq: sum of log is logdet}
    \end{align}
    as wanted.
\end{proof}

% \begin{lemma}\label{lemma: bound on action sizes}
%     Under \cref{asmp: L2 case}, for any $n\in\Nb$, we have
%     \begin{align*}
%         \sum_{t=1}^n \norm{\fourier \pi_t}_{(\designl_t)^{-1}}\wedge 1 \le \min\left\{2\log\det \left(\Id + \lambda^{-1} M_t\De\Lambda^{-1} M_t^*\right), 2\log(1+(\lambda M_\Lambda)^{-\frac12})\right\}\,.
%     \end{align*}
% \end{lemma}
% \begin{proof}
%     The first term of the minimum is the well known \cite[Lemma~E.2]{abbasi-abbasi-yadkori_online_2012}, which relies on the fact that $u\le 2\log(1+u)$ on $[0,1]$ implying
%     \[ 
%         \sum_{i=1}^n \norm{\fourier \pi_i}_{(\designl_t)^{-1}}\wedge 1 \le2\log\left(1+\norm{\fourier \pi_i}_{(\designl_t)^{-1}}\right)\,.
%     \]
%     Going from this same equation in another direction, namely taking \cref{lemma: bound on size of actions} into account, we have
%     \[
%         \norm{\fourier \pi_i}_{(\designl_t)^{-1}}\le \frac1{\lambda M_\Lambda}\norm{\fourier\pi_i}_\infty=\frac1{\sqrt{\lambda M_\Lambda}}\,
%     \]
%     which readily yields the second term.
% \end{proof}



% \begin{lemma}\label{lemma: bound on size of actions}
%     Under \cref{asmp: L2 case},
%     \[
%         \norm{a_t}_{(\designl_t)^{-1}} \le \frac1{\sqrt{\lambda M_\Lambda}} \quad\mbox{ for }t\in\Nb\,.
%     \]
% \end{lemma}
% \begin{proof}
%     Let $\varrho\in\measures(\Rb^d)$ be an arbitrary finite Borel measure with $\varrho(\state)=1$. Let us first observe that $\designl_t:=M^*_tM_t+\lambda\De\Lambda$ is positive definite, with 
%     \[ 
%         \langle \designl_tf\vert f\rangle_{\varrho}= \sum_{i=1}^t\langle f\vert a_i\rangle^2_\varrho + \lambda\norm{\De\Lambda f}_{\varrho}^2\ge \lambda M_\Lambda \norm{f}_{\varrho}^2\,.
%     \]
%     Thus, 
%     \[ 
%         \inf\left\{\norm{\designl_tf}_\varrho : f\in L^2(\Rb^d,\Rb;\varrho),\, \norm{f}_\varrho=1\right\} \ge \sqrt{\lambda M_\Lambda} >0
%     \]
%     and $\norm{(\designl_t)^{-1}}_{\op,\varrho}\le (\lambda M_\Lambda)^{-1/2}$, in which $\norm{\cdot}_{\op,\varrho}$ denotes the operator norm induced by $\norm{\cdot}_\varrho$. 
%     One obtains the bound by recalling that $\norm{a_t}_\infty=\pi_t(\state)=1$ and using $\norm{a_t}_{(\designl_t)^{-1}}\le \norm{(\designl_t)^{-1}}_{\op,\varrho}\norm{a_t}_\infty\varrho(\state)$. 
% \end{proof}

\section{Regret bounds}\label{app: regret bounds}

\subsection{Entropic regret bounds}\label{subsec: entropic regret}


To facilitate the analysis and the presentation of results, recall the entropic transport functional
\[
    \entf:(c,\pi)\in L^2(\Rb^d;\varrho)\times\Pi(\mu,\nu)\mapsto \langle c\vert\pi\rangle + \ve\entropy(\pi)\,.
\]
Thus, $\ent(\mu,\nu,c,\ve)=\inf_{\pi\in\Pi(\mu,\nu)}\entf(c,\pi)$ and~\eqref{eq: entropic regret} becomes
\[
    \regret_T^{\entropy,\ve}(\pi):= \sum_{t=1}^T \entf(c^*,\pi_t)+\xi_t - \ent(\mu,\nu,c^*,\ve)\,.
\]

\ThmEntropicRegret*


\begin{proof}
    Recall the we identify $\Ac$ with the $\Fb$-adapted process $\bm{\pi}:={(\pi_t)}_{t\in\Nb}\subset\Pi(\mu,\nu)$ of transport plans played. The instantaneous regret of the algorithm at time $t\in\Nb$ is defined as
    \[
        r_t:= \entf(c^*,\pi_t)+\xi_t - \ent(\mu,\nu,c^*,\ve)\,.
    \]
    It is clear that $\regret_T^\entropy(\Ac)=\sum_{t=1}^T r_t$.
    Before pursuing further, let us apply \cref{lemma: sub-gaussian sum} to the sequence ${(\xi_t)}_{t\in\Nb}$, in view of \cref{asmp: estimate + subG}, to obtain that for any $\delta>0$ we have
    \begin{align}
        \Pb\left(\sum_{i=1}^T \xi_i \le \sigma\sqrt{2T\log\left(\frac2\delta\right)}\right)\ge 1-\frac\delta2\,.\label{eq: sub-gaussian sum bound regret}
    \end{align}
    Now, let $\bar r_t :=  \entf(c^*,\pi_t)- \ent(\mu,\nu,c^*,\ve)$ as we continue the decomposition. By definition of the algorithm, let 
    \begin{align} 
        (\tilde\pi_t,\tilde c_t)\in\argmin_{\substack{\pi\in\Pi(\mu,\nu)\\c\in\confset_t(\delta)}} \entf(c,\pi)\notag\,,
    \end{align}
    where $\confset_t(\delta)$ is the confidence set defined in~\eqref{eq: confidence set}.
    
    Let us place ourselves on the event $\cap_{t=1}^\infty\left\{c^*\in\fourier^{-1}\confset_t(\delta)\right\}$, an event which happens with probability at least $1-\delta/2$ by \cref{lemma: probability of confsets unif in N}. By optimism, we have 
    \[
        \bar r_t \le  \entf(c^*,\pi_t)- \ent(\mu,\nu,\tilde c_t,\ve)
    \]
    The instant regret can be decomposed as
    \begin{align*}
        \bar r_t &= \entf(c^*,\pi_t) -\entf(\tilde c_t,\pi_t) + \entf(\tilde c_t,\pi_t) -\ent(\mu,\nu,\tilde c_t,\ve) 
    \end{align*}
    The first term $\entf(c^*,\pi_t) -\entf(\tilde c_i,\pi_t)=\langle c^*-\tilde c_t\vert \pi_t\rangle$ can be bounded by \cref{lemma: bound on width term}, while the second term is $0$ by definition of \cref{alg: alg shared}. The proof is completed by taking a union bound over $\cap_{t=1}^\infty\left\{c^*\in\fourier^{-1}\confset_t(\delta)\right\}$ and the event of~\eqref{eq: sub-gaussian sum bound regret}.
\end{proof}


\subsection{Kantorovich regret bounds}\label{subsec: kantorovich regret}

Let us begin by giving the requisite results on approximation of the Kantorovich problem by the entropic one. 
Let $d_{\entropy}(\gamma)$ (for $\gamma\in\{\mu,\nu\}$) denote the \textit{upper Renyi dimension} of $\gamma$, defined by 
\[ d_\entropy(\gamma):=\limsup_{\epsilon\downarrow0}\frac{H_\varepsilon(\gamma)}{\log(\varepsilon^{-1})}\,\]
in which $H_\varepsilon(\gamma)$ is the infimum (over all countable partitions of $\supp(\gamma)$ into Borel subsets of diameter at most $\varepsilon$) of the discrete entropy of $\gamma$ with respect to the partition, see~\cite{carlier_convergence_2023}.

\begin{lemma}[{\cite[Prop.~3.1]{carlier_convergence_2023}}]\label{lemma carlier pegon lipschitz UB}
    If $c^*$ is $L$-Lipschitz on $\supp(\mu)\times\supp(\nu)$, then
    \[\ent(\mu,\nu,c^*,\ve) - \kant(\mu,\nu,c^*)\le \varepsilon\left((d_{\entropy}(\mu)\wedge d_{\entropy}(\nu))\log(\varepsilon^{-1}) + L \right) \,\]
    as $\varepsilon\downarrow0$.
\end{lemma}

Extensions of this result exist for more general absolute continuity conditions, see~\cite[Rem.~3.4]{carlier_convergence_2023}. This constant is sharp, but tighter bounds may be obtained under stronger regularity assumptions, see e.g.~\cite[Prop.~3.7]{carlier_convergence_2023}. In view of \cref{lemma carlier pegon lipschitz UB}, we can define $\kappa:= (d_{\entropy}(\mu)\wedge d_{\entropy}(\nu)) + L$. In spite of its apparent complexity, upper Renyi dimension is a relatively well behaved object, and can be bounded in many common situations, see the following remarks.

\begin{remark}[{\cite[Prop.~3.2]{carlier_convergence_2023}}]
    If $\gamma$ is a measure on $\Rb^d$ satisfying
    \[ \int 0\vee\log(\norm{x}_2)\de\gamma(x)<+\infty\]
    then $d_{\entropy}(\gamma)\le d$.
\end{remark}

\begin{remark}[{\cite[Rem.~3.5]{carlier_convergence_2023}}]
    If $\gamma$ is finitely supported, then $d_\entropy(\gamma)=0$. 
\end{remark}



\ThmKantorovichRegret*



\begin{proof}
    The proof follows the same lines as the proof of \cref{thm: entropic regret}. Again, we identify $\Bc$ with the transport plans $\bm{\pi}:={(\pi_t)}_{t\in\Nb}\subset\Pi(\mu,\nu)$ it plays. The instantaneous regret is different due to the change of objective, it is given by 
    \[
        r_t:= R_t - \kant(\mu,\nu,c^*) = \langle c^*\vert \pi_t-\pi^*\rangle +\xi_t\,.
    \]
    As before, apply \cref{lemma: sub-gaussian sum} to the sequence ${(\xi_i)}_{i\in\Nb}$, and pass to $\bar r_t := \langle c^*\vert \pi_t-\pi^*\rangle$, which can be decomposed as
    \begin{align}
        \bar r_t &= \langle c^*\vert \pi_t\rangle - \langle c^*\vert \pi^*\rangle\notag\\
        &= \langle c^*\vert \pi_t\rangle  - \ent(\mu,\nu,c^*,\ve)+\ent(\mu,\nu,c^*,\ve) - \kant(\mu,\nu,c^*)\notag\\
        &\le \langle c^*\vert \pi_t\rangle -\ent(\mu,\nu,c^*,\ve) + \ve (d_\entropy(\mu)\wedge d_\entropy(\nu))\log(\ve^{-1}) + L\ve\notag
    \end{align}
    for any $\ve>0$, by \cref{lemma carlier pegon lipschitz UB}. In particular, for $\ve=\ve_t$ as used by \cref{alg: alg shared}, we have
    \begin{align}
        \sum_{t=1}^T \ve_t(d_\entropy(\mu)\wedge d_\entropy(\nu))\log(\ve_t^{-1}) + L\ve_t \le \frac{\kappa\alpha}{1-\alpha}\left(T^{1-\alpha}\log(T) + \frac{\alpha}{2^\alpha}\log(6)\right)\,.\label{eq: bound on sum epsilons}
    \end{align}
    by \cref{lemma: sum of terms from pegon bound}.
    Let us recall that optimism implies that 
    \begin{align} 
        (\tilde\pi_t,\tilde c_t)\in\argmin_{\substack{\pi\in\Pi(\mu,\nu)\\c\in\confset_t(\delta)}} \Psi^{\ve_t}_{\mu,\nu}(c,\pi)\notag\,,
    \end{align}
    for $\ve_t>0$ as used by \cref{alg: alg shared}, so that 
    \[
    \ent(\mu,\nu,\tilde c_t,\ve_t)\le \ent(\mu,\nu,c^*,\ve_t) \mbox{ on } \event_t(\delta)\,.
    \]

    Let us place ourselves on the event $\cap_{t=1}^\infty\left\{c^*\in\fourier^{-1}\confset_t(\delta)\right\}\supset \cap_{t=1}^\infty \Ec_t(\delta)$, which happens with probability at least $1-\delta/2$ by \cref{lemma: probability of confsets unif in N}.
   On this event, we thus have 
    \begin{align*}
        \langle c^*\vert \pi_t\rangle -\ent(\mu,\nu,c^*,\ve_t) &\le \langle c^*\vert \pi_t-\tilde\pi_t\rangle + \langle c^*-\tilde c_t\vert \tilde\pi_t\rangle\,,
    \end{align*}
    since $\entropy\ge 0$.
    Applying $\pi_t=\tilde\pi_t$ we obtain the desired result, up to combining the $\bar r_t$ over $t\in\Nb$, recalling~\eqref{eq: bound on sum epsilons} and taking a union bound over the two events.
\end{proof}





\section{Controlling the infinite dimensional terms}\label{app: computation}



The parametric and RKHS estimation methodologies are highly standard in bandit theory, because they seamlessly fit into the general Hilbert Space analysis of~\cite{abbasi-yadkori_online_2012} while giving a control on the resulting regret bounds in terms of finite dimensional quantities. In Fourier analysis and fields which rely on it, such as functional regression \citep{morris_functional_2015}, it is more natural to look for approximations by decomposing $\fourier c^*$ and $a_t$ into an orthonormal basis and truncating it at some finite order. We detail this learning methodology below.

We being in \cref{subsec: Fourier basis representation} by presenting the general concept of basis decomposition as an approximation method. Then, in \cref{subsubsec: finite order} we truncate at a fixed order and derive the regret bounds for this case. Before moving on to changing the truncation order with $t\in\Nb$ in \cref{subsubsec: increasing order}, we give a brief discussion in \cref{subsubsec: parametric rate /matching} of some examples in which a finite basis is sufficient. Finally, we give a brief treatment of kernel methods in \cref{subsec: tikhonov and RKHS} for completeness.



\subsection{Intrinsic regularity and fourier basis decay}\label{subsec: Fourier basis representation}

To simplify notation, let $f^*:=\fourier c^*$. Recall the chosen orthonormal basis ${(\phi_i)}_{i\in\Nb}$ of the space $L^2(\Rb^d;\varrho)$, in which $(f^*,a_t)\in {L^2(\Rb^d;\varrho)}^2$, $t\in\Nb$, admit representations 
\[
    f^*= \sum_{i=0}^\infty \gamma_i^*\phi_i\quad\mbox{ and }\quad a_t = \sum_{i=0}^\infty \vartheta^{(t)}_{i}\phi_i\,, \quad \mbox{ for some }\quad (\gamma^*,\vartheta^{(t)})\in{\ell_2(\Rb)}^2\,.
\] 
Classical choices for ${(\phi_i)}_{i=\in\Nb}$ are wavelet systems such as the Haar or Hermitian systems, and the Fourier basis if $\supp(\mu)\x\supp(\nu)$ is bounded. The choice of a specific basis is made \textit{ad hoc} from knowledge of the structure of the problem; we present the general argument. 

By definition of ${(\phi_i)}_{i\in\Nb}$ as an orthonormal basis, we have 
\[
    \langle f\vert a_t\rangle_{L^2(\Rb^d;\varrho)}=\langle \gamma^*\vert \vartheta^{(t)}\rangle_{\ell_2(\Rb)}=\sum_{n=1}^{+\infty} \gamma_n\vartheta^{(t)}_n\,.
\]

Let $f^*\vert_n:=\sum_{i=0}^n\gamma_i^*\phi_i$ be the truncation of the basis expansion of $f$ at order $n\in\Nb$. By abuse of notation, and only when it is clear from context, we will override notation and denote $c^*\vert_n$ the result of applying the inverse fourier transform to $(f^*)\vert_n$, the basis truncation of $f^*:=\fourier c^*$.
A straightforward derivation yields the approximation bound of \cref{lemma: fourier decay bound}.

\begin{lemma}\label{lemma: fourier decay bound}
    Let ${(\phi_i)}_{i\in\Nb}$ be an orthonormal basis of $L^2(\Rb^d;\varrho)$, and let $f\in L^2(\Rb^d;\varrho)$ with $f:=\sum_{i=0}^\infty \gamma_i\phi_i$. Then, we have
    \[
    \abs{\left\langle f - f\vert_n\vert g\right\rangle }_{L^2(\Rb^d;\varrho)}\le \norm{g}_{L^2(\Rb^d;\varrho)}\sum_{k=n+1}^{+\infty}\abs{\gamma_n^*}\quad \mbox{ for every } g\in L^2(\Rb^d;\varrho).
    \]
\end{lemma}

For our purpose, $g=a_t$ is bounded by $1$ since $\norm{\fourier\pi_t}_\infty\le\pi_t(\Rb^d)=1$ and $\varrho(\Rb^d)=1$, so that the resulting approximation error is controlled entirely by the decay of the coefficients ${(\gamma_i^*)}_{i\in\Nb}$. Consequently, regret analysis can leverage \cref{lemma: fourier decay bound} to move the problem into a finite dimensional regression problem on the coefficients ${(\gamma_i^*)}_{i\in\Nb}$. We begin by setting the stage with a fixed order (i.e.\ $n$ independent of $t$) methodology. Later, we will derive regret guarantees when $n$ is allowed to grow with $t$ in order to control the approximation error. 


\subsection{Fixed order basis truncation}\label{subsubsec: finite order}

In this section, let $n\in\Nb$ be fixed. One can approximately regress $\bm{R}_t$ against $a_t$ up to order $n$ by solving the $n$-dimensional Regularised Least-Squares (RLS) problem
\begin{align}
    \hat \gamma^{n,\lambda}_t:=\argmin_{\gamma\in\Rb^n} \sum_{s=1}^t \norm{R_s - \sum_{i=1}^n\gamma_i\vartheta^{(s)}_i}_2^2 + \lambda \Lambda_n(\gamma)\,,\label{eq: RLS  basis truncation}
\end{align}
in which $\Lambda_n:\Rb^n\to[0,+\infty)$ is a strictly convex continuously Fréchet-differentiable regulariser such that its Fréchet derivative $\De\Lambda_n$ satisfies
\[
    \frac1{M_{\Lambda_n}}\Id\preceq \De\Lambda_n\preceq M_{\Lambda_n}\Id\,.
\] 

For clarity, let $\vartheta^{(s,n)}$ denote the truncation of $\vartheta^{(s)}\in\Rb^\Nb$ at order $n$, so that $\vartheta^{(s,n)}\in\Rb^{n}$ and $\vartheta^{(s,n)}_i=\vartheta^{(s)}_i$ for all $i\in[n]$.
Following the standard arguments for online linear regression (omitted for brevity, see e.g.\ \cite{abbasi-yadkori_improved_2011,abbasi-yadkori_online_2012}), one can construct the (valid, by \cref{cor: probability of confsets order n}) confidence sets
\begin{align}
    \tilde\confset_{t}^{n}(\delta):=\left\{\gamma\in\Rb^{n}: \norm{\gamma -\hat \gamma^{n,\lambda}_t}_{\tilde\design_t^{\lambda,n}}\le \tilde\width_{t,n}(\delta)\right\}\label{eq: confidence set fixed order}\,,
\end{align}
in which $\tilde\design_t^{\lambda,n}:= \lambda\De\Lambda_n + \sum_{s=1}^t \vartheta^{(s,n)}{\vartheta^{(s,n)}}^\top$ and 
\begin{align}
    \tilde\width_{t}^{n}(\delta):=\sigma\sqrt{\log\left(\frac{4\det\left(\De\Lambda_n+\lambda^{-1}\sum_{s=1}^t \vartheta^{(s,n)}{\vartheta^{(s,n)}}^\top\right)}{\delta^2}\right)} +{\left(\frac\lambda{\norm{\De\Lambda_n}_\op}\right)}^{\frac12}\ubnorm\,.
\label{eq: width fixed order}
\end{align}
Notice that $C>\norm{c^*}_{L^2(\Rb^d;\varrho)}$ implies that $C\ge \norm{\gamma^*}_{\ell_2(\Rb)}$ by definition of ${(\phi_i)}_{i\in\Nb}$, so that $\ubnorm$ is a valid upper bound on $\norm{\gamma^*}_{\ell_2(\Rb)}$. To verify the validity of the confidence sets (see \cref{cor: probability of confsets order n}), let 
    \begin{align}
        \tilde{\event_t^{n}}(\delta):=\left\{ \norm{\gamma -\hat \gamma^{n,\lambda}_t}_{\tilde\design_t^\lambda}\le \tilde\width_{t}^{n}(\delta) \right\}\quad \mbox{ for }\quad (t,n)\in\Nb^2\,.\label{eq: event def for fixed order approx}
    \end{align} 

\begin{corollary}\label{cor: probability of confsets order n}
    Under \cref{asmp: estimate + subG,asmp: L2 case}, for every $\delta>0$, $\lambda>0$, $n\in\Nb$, 
    \[
        \Pb\left(\bigcap_{t=1}^\infty \tilde{\event_t^{n}}(\delta)\right)\ge 1-\frac\delta2\,.
    \]
\end{corollary}

\begin{proof}
    Follow the proof method of \cref{lemma: probability of confsets unif in N} (or apply \cref{lemma: confidence sets with varying basis order} below).
\end{proof}


Applying this learning methodology to \cref{alg: alg shared} in place of the infinite-dimensional RLS, and with the optimistic choice of belief-action pairs
\begin{align}
    (\tilde\pi_t,\tilde \gamma_t^n)\in \argmin_{\substack{\pi\in\Pi(\mu,\nu)\\ \gamma\in \tilde\confset_{t}^{n}(\delta)}}\entf\left(\mu,\nu,\sum_{i=1}^n\gamma_{i}\phi_i,\ve\right)\label{eq: optimism for finite order}
\end{align}
yields \cref{alg: alg shared + approx} with ${(n_t)}_{t\in\Nb}={(n)}_{t\in\Nb}$ and the regret bound of \cref{cor: regret for fixed approximation order}.

\begin{algorithm}
    \caption{\namealgtwo{}\label{alg: alg shared + approx}}
    \SetKwFunction{approxpi}{ApproximateAction}
    \KwData{Confidence $\delta$, regularization level $\lambda$, entropy penalisation ${(\ve_t)}_{t\in\Nb}$, orders ${(n_t)}_{t\in\Nb}$.}
    \For{$t\in\Nb$}{
            Compute the RLS estimator $\hat \gamma^{n_t,\lambda}_t$ using~\eqref{eq: RLS  basis truncation}\;
            Construct the confidence set $\tilde\confset_{t}^{n_t}(\delta)$ using~\eqref{eq: confidence set fixed order} and~\eqref{eq: width fixed order}\;
            Optimism:  pick $ (\tilde\pi_t,\tilde\gamma^{n_t}_t)$ according to~\eqref{eq: optimism for finite order}\;
        Play $\pi_t=\tilde\pi_t$ if $t>0$, else $\pi_0=\mu\tensor\nu$; receive feedback $R_t$\;
    }
\end{algorithm}

\begin{restatable}{corollary}{RegretCorrForFixedApproximationOrder}\label{cor: regret for fixed approximation order}
    Under \cref{asmp: estimate + subG,asmp: L2 case}, for any $\delta>0$, $\lambda>0$, $T\in\Nb$, using \cref{alg: alg shared + approx} with ${(\ve_t)}_{t\in\Nb}={(\ve)}_{t\in\Nb}$ and ${(n_t)}_{t\in\Nb}={(n)}_{t\in\Nb}$ (denoted $\Ac_n$) yields
    \begin{align}
        \regret_T^{\entropy,\ve} (\Ac_n)&\le  \sigma\sqrt{2T\log\left(\frac2\delta\right)} + 2\ubnorm\sqrt{nT}\left(\log\left(\frac{M_{\Lambda_n}}\lambda + \frac{t \ubnorm^2}n\right)+\frac{n}{2(1\wedge\lambda\ubnorm)}\log M_{\Lambda_n} \right)\notag\\
        &\quad +2T \sum_{k=n+1}^{+\infty}\abs{\gamma_k^*}\,, \label{eq: entropic regret for fixed approximation order} \\
        \intertext{ while using ${(\ve_t)}_{t\in\Nb}={(\alpha t^{-\alpha})}_{t\in\Nb}$ and ${(n_t)}_{t\in\Nb}={(n)}_{t\in\Nb}$ (denoted $\Bc_n$) yields }
        \regret_T(\Bc) &\le \sigma\sqrt{2T\log\left(\frac2\delta\right)} + 2\ubnorm\sqrt{nT}\left(\log\left(\frac{M_{\Lambda_n}}\lambda + \frac{t \ubnorm^2}n\right)+\frac{n}{2(1\wedge\lambda\ubnorm)}\log M_{\Lambda_n} \right)\notag\\
        &\quad +\frac{\kappa\alpha}{1-\alpha}\left(T^{1-\alpha}\log(T) + \frac{\alpha}{2^\alpha}\log(6)\right) + 2T \sum_{k=n+1}^{+\infty}\abs{\gamma_k^*}\,.  \label{eq: Kantorovich regret for fixed approximation order}
    \end{align}
\end{restatable}

\begin{proof}
    The proof follows the usual decomposition up the following modifications which are the same for both~\eqref{eq: entropic regret for fixed approximation order} and~\eqref{eq: Kantorovich regret for fixed approximation order}. We give the modification for \cref{thm: entropic regret}, the same modifications need only be applied to \cref{thm: regret Kantorovich} to complete the proof of the second bound.  

    At the second step of the proof, let $\bar\pi^{\epsilon}$ be an $\epsilon$-minimiser of $\ent(\mu,\nu,c^*,\ve)$, for $\epsilon>0$, and decompose $\bar r_t := \entf (c^*,\pi_t)-\ent(\mu,\nu,c^*,\ve)$ as
    \begin{align}
        \bar r_t &\le  \epsilon + \entf (c^*,\pi_t) - \entf(c^*, \bar\pi^{\epsilon})\notag \\
        & \le \epsilon +\entf (c^*\vert_n,\pi_t) - \ent(\mu,\nu,c^*\vert_n,\ve) \notag \\
        &\quad +\entf (c^*,\pi_t) -\entf (c^*\vert_n,\pi_t)  + \entf(c^*\vert_n, \bar\pi^{\epsilon})- \entf(c^*, \bar\pi^{\epsilon})\notag \\% \entf(c^*\vert_n,\pi_t) - \ent(\mu,\nu,c^*,\ve)\,.
        &\le \epsilon +\entf (c^*\vert_n,\pi_t) - \ent(\mu,\nu,c^*\vert_n,\ve) + 2\sum_{k=n+1}^{+\infty}\abs{\gamma_k^*}\,,\notag
    \end{align}
    by a double application of \cref{lemma: fourier decay bound} combined with the bound $\norm{a_t}_{L^2(\Rb^2;\varrho)}\le 1$. 
    Sending $\epsilon\to0$ allows one to then continue the proof, up to replacing the events $\event_t(\delta)$ by $\tilde{\event_t^{n}}(\delta)$, and \cref{lemma: probability of confsets unif in N} by \cref{cor: probability of confsets order n}. 

    Finally, let us introduce $\tilde c_t^{\,n}:= \sum_{i=1}^n \tilde\gamma_{t,i}^{n}\phi_i$ for $t\in\Nb$, so that by \cref{lemma: bound on width term}, we can directly derive
    \begin{align}
        \sum_{t=1}^T\langle c^*\vert_n -\tilde c_t^{\,n}\vert \tilde\pi_t\rangle\le 2C\tilde\width_{T,n}(\delta)\sqrt{nT\log\det\left(\idmat+\frac{1}{2\lambda C}\sum_{t=1}^T \vartheta^{(t,n)}_t\De\Lambda_n^{-1}{\vartheta^{(t,n)}}^\top\right)}\,.\notag
    \end{align}

    To obtain the stated bounds, it remains to bound 
    \[
        \det\left(\De\Lambda_n+\lambda^{-1}\sum_{t=1}^T \vartheta^{(t,n)}{\vartheta^{(t,n)}}^\top\right) \quad\mbox{and}\quad \det\left(\idmat+\frac{1}{2\lambda C}\sum_{t=1}^T \vartheta^{(t,n)}\De\Lambda_n^{-1}{\vartheta^{(t,n)}}^\top\right)
    \]
    using \cref{lemma: bounds for the determinants in finite dimension}.
\end{proof}


\begin{lemma}\label{lemma: bounds for the determinants in finite dimension}
    Under \cref{asmp: estimate + subG,asmp: L2 case}, for $(n,t)\in\Nb^2$, we have
    \begin{align}
        \log\det\left(\De\Lambda_n+\lambda^{-1}\sum_{t=1}^T \vartheta^{(t,n)}{\vartheta^{(t,n)}}^\top\right)\le \log\left(\frac{M_{\Lambda_n}}\lambda + \frac{t \ubnorm^2}n\right)+n\log M_{\Lambda_n}\,.\label{eq: bound for the determinindant of beta in finite dimension}\\
        \log\det\left(\idmat+\frac{1}{2\lambda C}\sum_{t=1}^T \vartheta^{(t,n)}_t\De\Lambda_n^{-1}{\vartheta^{(t,n)}}^\top\right)\le \log\left(\frac{M_{\Lambda_n}}\lambda + \frac{t \ubnorm^2}n\right)+\frac{n}{2\lambda\ubnorm}\log M_{\Lambda_n}\,.
        \label{eq: bound for the determinindant of width in finite dimension}
    \end{align}
\end{lemma}

\begin{proof}
    We take the two bounds in turn. First, apply the matrix determinant lemma to obtain
    \[
        \det\left(\De\Lambda_n+\lambda^{-1}\sum_{t=1}^T \vartheta^{(t,n)}{\vartheta^{(t,n)}}^\top\right)\le \det(\De\Lambda_n)\det\left(\idmat+\lambda^{-1}\sum_{t=1}^T \vartheta^{(t,n)}\De\Lambda_n^{-1}{\vartheta^{(t,n)}}^\top\right)\,,
    \]
    which can be readily bounded as in \citep[Lemma E.3]{abbasi-yadkori_online_2012} by noticing that $\snorm{\vartheta^{(t,n)}}_2\le\norm{c^*}_{L^2(\Rb^d;\varrho)}$ (with $\det(\De\Lambda_n)\le M_{\lambda_n}^n\vee 1$) to obtain~\eqref{eq: bound for the determinindant of beta in finite dimension}.

    For the second bound, apply \citep[Lemma E.3]{abbasi-yadkori_online_2012} directly to obtain
    \begin{align}
        \det\left(\idmat+\frac{1}{2\lambda C}\sum_{t=1}^T \vartheta^{(t,n)}\De\Lambda_n^{-1}{\vartheta^{(t,n)}}^\top\right) \le {\left(\frac{2\lambda\ubnorm \Tr(\De\Lambda_n)+t\ubnorm^2 }n\right)}^n\left(2\lambda \ubnorm\det(\De\Lambda_n)\right)\notag
    \end{align}
    wherefrom~\eqref{eq: bound for the determinindant of width in finite dimension} follows.
\end{proof}


\subsection{Finite order bases: matching and parametric models}\label{subsubsec: parametric rate /matching}


At this point, let us recall \cref{asmp: basis decay} which provides the quantification of the regularity of $c^*$ which we will use to set $n$.  We will now discuss some examples in which a finite basis is sufficient to control the approximation error.

\asmpthree*


\begin{proposition}\label{cor: regret for fixed approximation order with bounded basis}
    Under \cref{asmp: basis decay,asmp: estimate + subG,asmp: L2 case}, with $\zeta(n)\1_{\cdot\ge N}$ for some $N\in\Nb$ (i.e.\ if $\gamma_i^*=0$ for every $i>N$), then under the conditions of \cref{cor: regret for fixed approximation order} with $n=N$, $\Lambda_n=\norm{\cdot}_{L^2(\Rb^d;\varrho)}/2$, and $\alpha=1/2$ the bounds of \cref{cor: regret for fixed approximation order} become
    \begin{align}
        \regret_T^{\entropy,\ve} (\Ac_n)&\le \sigma\sqrt{2T\log\left(\frac2\delta\right)} + 2\ubnorm\sqrt{NT}\log\left(\frac{1}\lambda + \frac{T \ubnorm^2}N\right) \label{eq: entropic regret for fixed approximation order with bounded basis}\\
        \intertext{ and }
        \regret_T(\Bc_n)&\le  \sigma\sqrt{2T\log\left(\frac2\delta\right)} + 2\ubnorm\sqrt{NT}\log\left(\frac{1}\lambda + \frac{T \ubnorm^2}N\right) +\kappa(1+\sqrt{T}\log(T))\label{eq: kantorovich regret for fixed approximation order with bounded basis}
    \end{align}
\end{proposition}


Naturally, the assumption that $\gamma_i^*=0$ for any $i>N$ is not satisfactory, but it is verified for several existing models and serves to demonstrate that some learning problems in BOT are learnable at the rate $\tilde\Oc(\sqrt{T})$ given only knowledge of an upper bound on $N$ and $\norm{\gamma^*}_{\ell_2(\Rb)}$ and an appropriate basis ${(\phi_i)}_{i\in\Nb}$.  

Consider a matching problem in which the measures $\mu$ and $\nu$ are supported on $K$ and $K'$ loci respectively. Let $\{x_1,\ldots, x_K\}=\supp(\mu)$ and $\{x'_1,\ldots, x'_{K'}\}=\supp(\nu)$ denote these loci. We can let $c^*$ assume arbitrarily values outside of $\state=\{(x_i,x_j'):(i,)\in[K]\x [K']\}$ without loss of generality. Let $\epsilon<\inf\{\norm{u-v}:(u,v)\in\state^2\,,\; u\neq v\}$ and define the functions 
\[
    \phi_{i,j}:= \frac{6}{\pi\epsilon^3}\1_{\{B_2({(x_i,x_j')}^\top,\epsilon/2)\}} \quad \mbox{ for } (i,j)\in[K]\times[K']\,.
\]
Re-indexing the functions by $k\in[K\times K']$, and adding suitable functions for $k>KK'$, we obtain an orthonormal basis ${(\phi_k)}_{k\in\Nb}$ of $L^2(\Rb^d;\varrho)$, in which $c^*:=\sum_{k=1}^{KK'}\gamma_k^*\phi_k$. Consequently, we can apply \cref{cor: regret for fixed approximation order with bounded basis} with $N=KK'$ to obtain a regret bound of $\tilde\Oc(\sqrt{KK'T})$ for the learning problem.

Alternatively, consider that there is a parametric model for $c^*$, i.e.\ there is $\theta^*\in\Rb^p$ such that 
\[
    c^*(x,y) = \sum_{i=1}^p {\theta^*}_i\Phi_i(x,y)\,,
\]
for some embedding function $\Phi:\Rb^d\x\Rb^d\to\Rb^p$. When the embedding function is known, one can construct a basis through the Gram-Schmidt process. Let $\phi_1:=\Phi_1/\norm{\Phi_1}_{L^2(\Rb^d;\varrho)}$, and for $i\le p$, define $S_i:={\{\phi_k:k<i\}}^\perp$ the orthogonal complement of the sequence this far. Now, repeatedly project the feature dimensions onto $S_i$ to construct $\phi_i:=P_{S_i}\Phi_i/\norm{P_{S_i}\Phi_i}_{L^2(\Rb^d;\varrho)}$. For $i>p$, take any orthonormal basis of $S_p$ to complete the basis, it will not be used anyway. Consequently, we can also apply \cref{cor: regret for fixed approximation order with bounded basis} with $N=p$ to obtain a regret bound of $\tilde\Oc(\sqrt{pT})$ for the learning problem.


These results are summarised in \cref{cor: on finite basis regret}, but notice that higher order polynomial models can be readily considered as well, such as quadratic costs 
\[
    c^*(x,y) = {\Phi(x,y)}^\top\Theta^*\Phi(x,y)\,,
\]
for $\Theta^*\in\Rb^{p\times p}$, by simply reparametrising it as a linear model in dimension $p^2$ and applying the same construction. Many other models can be considered in this manner, and would benefit from further specialised investigation.

\CorOnFiniteBasis*

\subsection{Increasing order basis truncation}\label{subsubsec: increasing order}

In this section, we will extend the results of \cref{subsubsec: finite order} to let $n$ change with $t\in\Nb$ along the learning process. We will denote the corresponding sequence by ${(n_t)}_{t\in\Nb}\subset\Nb$. It is relatively simple to see that the proofs of the key properties of online least-squares estimation will extend, but we include the key proof sketches for completeness. We begin by diagonalising the validity of the confidence sets in \cref{lemma: confidence sets with varying basis order}.

\begin{lemma}\label{lemma: confidence sets with varying basis order}
    Under \cref{asmp: estimate + subG,asmp: L2 case},
\[
    \Pb\left(\bigcap_{t=1}^\infty \tilde{\event}_t^{n_t}(\delta)\right)\ge 1-\frac\delta2\,.
\]
\end{lemma}

\begin{proof}
    The proof only requires diagonalisation of the standard stopping time construction. For $(\delta,t)\in(0,1)\x\Nb$, on the filtered probability space $(\Omega,\Fc_\infty,\Fb,\Pb)$ define 
    \[ 
        B_t(\delta):=\left\{\omega\in\Omega: \norm{\gamma^* -\hat \gamma^{n_t,\lambda}_t}_{\tilde\design_t^{\lambda,n_t}}\le \tilde\width_{t,n_t}(\delta) \right\}\overset{\mbox{a.s.}}{=}\{\omega\in\Omega: c^*\vert_{n_t} \not\in \tilde\confset_{t}^{n_t}(\delta)\}\,,
    \]
    be the $t\textsuperscript{th}$ ``bad event'', and let $\tau_\delta: \omega\in\Omega\to\inf\{t\in\Nb: \omega\in B_t(\delta)\}$, which is a stopping time. We have 
    \[
        \{\tau <+\infty\}= \bigcup_{t\in\Nb} B_t(\delta)\,.
    \]
    By construction, in the classical manner:
    \begin{align*}
        \Pb\left(\bigcup_{t\in\Nb} B_t(\delta)\right)&= \Pb(\tau<+\infty, B_t(\delta))\le \Pb\left(\tilde\event_t^{n_t}(\delta)\right)\le \frac{\delta}2\,.
    \end{align*}
\end{proof}

The confidence sets using for non-constant ${(n_t)}_{t\in\Nb}$ are simply instantiations of~\eqref{eq: confidence set fixed order} and~\eqref{eq: width fixed order} with $n_t$ in place of $n$. This change of basis with time however requires a modification of the proof of \cref{lemma: bound on width term} as the steps summed up in~\eqref{eq: summation of widths in proof of the bound on width term lemma} are no longer homogenous. In particular,~\eqref{eq: sum of log is logdet} is no longer valid. 

\begin{lemma}\label{lemma: bound on width term with varying basis order}
    Under \cref{asmp: estimate + subG,asmp: L2 case}, if $\Lambda_n:=\frac{1}{2}\norm{\cdot}_{2}^2$ with the norm being on $\Rb^n$, then
    \begin{align}
        \sum_{t=1}^T\langle c^*\vert_{n_t} -\tilde c_t^{\,n_t}\vert \tilde\pi_t\rangle\le 2C\sigma\left( \sqrt{2\log\left(\frac{\lambda^{-1}+\frac{T\ubnorm^2}{n_T}}{\delta}\right)}+\sqrt{\lambda}\ubnorm\right) \sqrt{n_T T\log\left(1+ \frac{T}{n_T\ubnorm^2}\right)}\,.\notag
    \end{align}
\end{lemma}

\begin{proof}
    Recall the notation of \cref{lemma: bound on width term}, which adapts to $\varphi_t:=\langle c^*\vert_{n_t}-c_t^{n_t}\vert \tilde\pi_t\rangle$ for $t\in\Nb$ and $\tilde c_t^{\, n_t}:= \sum_{i=1}^{n_t}\tilde\gamma^{n_t}_{t,i}\phi_i$. The proof of \cref{lemma: bound on width term} yields 
    \begin{align}
        \sum_{t=1}^T\varphi_t \le 2C\beta_{T,n_T}(\delta)\sqrt{T\sum_{t=1}^T\log\left(1+\frac1{2\ubnorm}\norm{\vartheta^{(t,n_t)}}_{{({\tilde{D}_t^{\lambda,n_t}})}^{-1}}\right)}\,.
    \end{align}
    
    First, one can bound $\tilde\width_{T,n_T}(\delta)$ by~\eqref{eq: bound for the determinindant of width in finite dimension} in \cref{lemma: bounds for the determinants in finite dimension}. 

    It remains to adapt the logarithmic term into a log-determinant of the desired form by conforming the vectors $\vartheta^{(t,n_t)}$. To do so, let us define the block matrices
    \[
    Z_t:= \begin{pmatrix}
        {(\tilde\design_t^{\lambda,n_t})}^{-1} & \bm{0} \\
        \bm{0} & \bm{0}\\
        \end{pmatrix} \quad \mbox{ for } t\in\Nb\,,
    \]
    so that we may use the rank one update formula to write
    \[
    \prod_{t=1}^T \left(1+\frac1{2\ubnorm}\norm{\vartheta^{(t,n_t)}}_{{({\tilde{D}_t^{\lambda,n_t}})}^{-1}}\right) = \frac{\det\left(\De\Lambda_n+ \sum_{t=1}^T \vartheta^{(t,n_t)}Z_t{\vartheta^{(t,n_t)}}^\top\right)}{\det(\De\Lambda_n)}\,.
    \]
    Taking $\Lambda_n=\frac{1}{2}\norm{\cdot}_{2}^2$ as given, we can bound the determinant of the numerator by
    \[
        {\det\left(\De\Lambda_n+ \sum_{t=1}^T \vartheta^{(t,n_t)}Z_t{\vartheta^{(t,n_t)}}^\top\right)} \le {\left(1+ \frac{T}{n_T\ubnorm^2}\right)}^{n_T}
    \]
    as in~\cite[Lemma E.3]{abbasi-yadkori_online_2012}. Combining with the bound on $\tilde\width_{T,n_T}(\delta)$ completes the proof.
\end{proof}


Having established the technical lemmata, we now turn to the regret guarantees of the varying order basis truncation version of \cref{alg: alg shared + approx}. In particular, recall \cref{asmp: basis decay} to give a quantification of the regularity of $c^*$, which in turn will allow us to tune ${(n_t)}_{t\in\Nb}$ to obtain the best possible regret bounds in \cref{thm: regret for varying approximation}.

\begin{restatable}{theorem}{RegretForVaryingApprox}\label{thm: regret for varying approximation}
    Assume \cref{asmp: estimate + subG,asmp: L2 case,asmp: basis decay} and  $\zeta(n)=1-n^{-q}$ for some $q>0$. For any $\delta\in(0,1)$, $\lambda>0$, $\ve>0$, let $\tilde\Ac$ (resp. $\tilde\Bc$) denote \cref{alg: alg shared + approx} with ${(n_t)}_{t\in\Nb}={(\ceil{t^{\frac1{2q+1}}})}_{t\in\Nb}$, $\Lambda_n=\frac12\norm{\cdot}_{2}^2$, for all $n\in\Nb$, and ${(\ve_t)}_{t\in\Nb}={(\ve)}_{t\in\Nb}$ (resp. ${(\ve_t)}_{t\in\Nb}= {(\alpha t^{-\alpha})}_{t\in\Nb}$). For any $T\in\Nb$, the following regret bounds hold:
    \begin{align}
        \regret_T^{\entropy,\ve} (\tilde\Ac)&\le \sigma\sqrt{2T\log\left(\frac2\delta\right)} + \ubnorm\left(1+\frac{qT^{\frac{q+1}{2q+1}}}{2q+1}\right) \notag\\
        &\quad + 2\ubnorm\sigma T^{\frac{q+1}{2q+1}}\left(\sqrt{2\log\left(\frac{\lambda^-1+2T^{\frac{2q}{2q+1}}\ubnorm^2}{\delta}\right)}+\sqrt\lambda \ubnorm\right)\sqrt{\log\left(1+\frac{2T^{\frac{2q}{2q+1}}}{\ubnorm^2}\right)}\,,\notag
        %\label{eq: entropic regret for varying approximation order with bounded basis}\\
        \intertext{ and }
        \regret_T(\tilde\Bc)&\le  \sigma\sqrt{2T\log\left(\frac2\delta\right)} + \ubnorm\left(1+\frac{qT^{\frac{q+1}{2q+1}}}{2q+1}\right)+\kappa(1+\sqrt{T}\log(T))\notag\\
        &\quad + 2\ubnorm\sigma T^{\frac{q+1}{2q+1}}\left(\sqrt{2\log\left(\frac{\lambda^-1+2T^{\frac{2q}{2q+1}}\ubnorm^2}{\delta}\right)}+\sqrt\lambda \ubnorm\right)\sqrt{\log\left(1+\frac{2T^{\frac{2q}{2q+1}}}{\ubnorm^2}\right)}\,,\notag
        %\label{eq: kantorovich regret for varying approximation order with bounded basis}
    \end{align}
\end{restatable}

\begin{proof}
    The proof requires only two steps from the one of \cref{cor: regret for fixed approximation order}. First, we bound the approximation error term.
    \Cref{lemma: fourier decay bound} readily implies that
    \[
        \abs{\langle c^* - c^*\vert_{n_t}\vert \pi_t\rangle} \le \sum_{k=n_t+1}^{+\infty}\abs{\gamma_i^*}\,.
    \]
    Summing over $t\in\Nb$, one obtains
    \begin{align}
        \sum_{t=1}^T \abs{\langle c^* - c^*\vert_{n_t}\vert \pi_t\rangle}\le \sum_{t=1}^T \sum_{i=n_t+1}^{+\infty}\abs{\gamma_i^*}\,.\label{eq: regret term for truncation, summed}
    \end{align}
    By \cref{asmp: basis decay}, for any $n\in\Nb$, we have
    \[ 
        \sum_{i=n_t+1}^\infty\abs{\gamma_i^*} = \norm{c^*}_{L^2(\Rb^d;\varrho)}- \sum_{i=1}^{n_t}\abs{\gamma_i^*} \le \norm{c^*}_{L^2(\Rb^d;\varrho)}(1-\zeta(n_t))
    \]
    so that for any $u>0$, the choice $n_t:=\ceil{\zeta^{-1}((1-t^{-u}))}=\ceil{t^{\frac uq}}$ ($q>0$) yields
    \[
        \sum_{i=n_t+1}^\infty \abs{\gamma_i^*}\le  \norm{c^*}_{L^2(\Rb^d;\varrho)} t^{-u}\,.
    \]
    This follows from the fact that $\zeta$ can be made a bijection of $\Rb_+\to(0,1]$, and that $\zeta$ is increasing. Injecting $\sum_{s=n_t+1}^{+\infty}\abs{\gamma_s^*}\le \norm{c^*} t^{-u}$ into~\eqref{eq: regret term for truncation, summed} yields
    \[
        \sum_{t=1}^T \abs{\langle c^* - c^*\vert_{n(t)}\vert \pi_t\rangle}\le\norm{c^*}_{L^2(\Rb^d;\varrho)}\left(1+\frac{T^{1-u}}u\right)\,.
    \]
    The second step simply involves applying \cref{lemma: bound on width term with varying basis order} for $n_T:=\ceil{T^{\frac{u}q}}\le 2T^{u/q}$ to obtain a bound of order $\Oc(T^{\frac12+\frac{u}{2q}})$. Setting $u=\frac{q}{2q+1}$ yields the stated bounds.
\end{proof}

%In terms of computational complexity, 

%\lc{In terms of computational complexity, suppose for example that $\zeta(t)=1-t^{-\iota}$ for some $\iota>0$. Note that one may always assume\footnote{Summability of $\{\abs{\gamma_i}\}_{i\in\Nb}$ implies that $\{1-\zeta(n)\}_{n\in\Nb}$ should be summable.} (up to multiplying $\zeta$ by a constant) that $\iota>1$. At step $t\in\Nb$, solving the linear system of \eqref{eq: RLS basis truncation} (of dimension $t^{1/2\iota}\times t$) thus has a complexity of $\Oc(t^{1+1/\iota})$.} 



\subsection{Tikhonov regularisation and RKHS theory}\label{subsec: tikhonov and RKHS}

In this section, we will assume that $\Lambda=\frac12\norm{\cdot}^2_2$ for simplicity. In general any increasing positive function of $\norm{\cdot}_2$ will suffice to use the representer theorem as per our argument. Suppose we are given $(\Hf,K)$ a Reproducing Kernel Hilbert Space\footnote{Understood, of course, up to the identifications necessary for the RKHS to be a space of functions. Recall that $L^2(\Rb^d;\varrho)$ is \emph{not} an RKHS due to a subtlety of this nature.} (RKHS) such that $\Hf\subset L^2(\Rb^d;\varrho)$. We may specialise the RLS estimator (see \cref{prop: least squares estimator}) to this case by noting that $M_t:={(K(a_0,\cdot),\dots,K(a_{t-1},\cdot))}^\top$. 

By the representer theorem, at any step $t\in\Nb$,the solution to the regularised least squares problem in $\Hf$ is given by
\[
    \hat f_t^\lambda = \sum_{i=0}^{t-1} \upsilon_i K(a_i,\cdot)\,,
\]
for some ${(\upsilon_i)}_{i=0}^{t-1}\in\Rb^t$. The problem can therefore be reduced to the finite dimensional optimisation problem 
\[
    \min_{\upsilon\in\Rb^t} \norm{\vec{R}_t - K_t\upsilon}_2^2 + \lambda \upsilon^\top K_t\upsilon\,,
\]
in which $K_t:={[K(a_i,a_j)]}_{i,j}\in\Rb^{t\x t}$ is the kernel (Grammian) matrix. The rest of the standard developments follow, and one arrives at the approximation bound 
\[
    \sum_{t=1}^T \langle c^* - \tilde c_t\vert \tilde\pi_t\rangle \le 2\ubnorm\beta_T(\delta)\sqrt{2T\log\det\left(I+{(\lambda \ubnorm)}^{-1}K_T\right)}
\] 
by \cref{lemma: bound on width term}, and corresponding regret bounds easily follows via \cref{thm: entropic regret,thm: regret Kantorovich}. From here, one can easily recover bounds ad-hoc or by following the general methodology of \cref{subsec: Fourier basis representation}.

One of the main benefits of kernel methods is that they can be used to learn in infinite-dimensional spaces efficiently. While they are inherently efficient thanks to the kernel trick, works in this field have suggested further efficiency refinements such as~\cite{takemori_approximation_2021} which uses approximation theory to reduce learning in an RKHS to a finite-dimensional approximation on a well chosen basis. This resembles the methodology used above, further developments in this direction appear an interesting avenue for research. 
\begin{proof}[Proof of Lemma~\ref{lem:VstarVplan}]

The lemma is proven by induction over $h$, starting from $V^{\star}_{H+1} = V_{H+1}=\bm{0}$. We have
\begin{align*}
    &Q^{\star}_h(s,a) - Q_{h}(s,a) - \sum_{h'=h}^H\left(\beta(\delta)(\sigma_{h,N}(s_h,a_h)+\frac{2}{\sqrt{N}})+\frac{2}{N}\right)\\
    &= r(s,a)+ [P_hV^{\star}_{h+1}](s,a)- r(s,a) - \hat{g}_{h}(s,a)- \sum_{h'=h}^H\left(\beta(\delta)(\sigma_{h,N}(s,a)+\frac{2}{\sqrt{N}})+\frac{2}{N}\right)\\
    &\le [P_hV^{\star}_{h+1}](s,a) -[P_hV_{h+1}](s,a)- \sum_{h'=h+1}^H\left(\beta(\delta)(\sigma_{h,N}(s,a)+\frac{2}{\sqrt{N}})+\frac{2}{N}\right)\\
    &=[P_h(V^{\star}_{h+1}-V_{h+1})](s,a)+\beta(\delta)(\sigma_{h,N}(s,a)(s,a)+\frac{2}{\sqrt{n}})+\frac{2}{n}.\\
\end{align*}
The inequality holds by $\Ec$.
Then, we have
\begin{align*}
    V^{\star}_h(s_h) - V_h(s_h) &=
    \max_{a\in\Ac} Q^{\star}_h(s,a) - \max_{a\in\Ac}Q_{h}(s,a)\\
    &\le \max_{a\in\Ac} \{Q^{\star}_h(s,a) - Q_{h}(s,a)\}\\
    &\le 0
\end{align*}
That proves the lemma. 

\end{proof}




\begin{proof}[Proof of Lemma~\ref{VHVpisum}]
Note that $V_{H+1}=V^{\pi}_{H+1}=\bm{0}$. We next obtain a recursive relationship for the difference $V_h(s)-V^{\pi}_h(s)$. 
\begin{align*}
    V_h(s_h)-V^{\pi}_h(s_h) &= Q_h\left(s_h, \pi(s_h)\right) -  Q^{\pi}_h\left(s_h, \pi(s_h)\right) \\
    &=r\left(s_h, \pi(s_h)\right) + \hat{g}_h\left(s_h, \pi(s_h)\right) + \beta\sigma^N_h\left(s_h, \pi(s_h)\right) - r\left(s_h, \pi(s_h)\right) - [P_hV^{\pi}_{h+1}]\left(s_h, \pi(s_h)\right)\\
    &\le [P_hV_{h+1}]\left(s_h, \pi(s_h)\right) +2\beta\sigma^N_h\left(s_h, \pi(s_h)\right) - [P_hV^{\pi}_{h+1}]\left(s_h, \pi(s_h)\right),\\
\end{align*}
where the inequality is due to $\Ec_1$. 

Recursive application of the above inequality over $h=H, H-1, \cdots, 1$, we obtain
\begin{align*}
    V_1(s_1) - V_1^{\pi}(s_1)& \le \E_{s_{h+1}\sim P(\cdot|s_h,\pi(s_h)), h \le H}\left[\sum_{h=1}^H2\beta\sigma^N_h\left(s_h, \pi(s_h)\right)\right]\\
    &=2HV_1^{\pi}(s_1; \alpha/H )
\end{align*}


\end{proof}


\begin{proof}[Proof of Lemma~\ref{lem:vnstar_vn}]

    
    The lemma is proven by induction, starting from $V^{\star}_{h_0+1}(\cdot; \tilde{\sigma}^{h_0}_{ n})=V_{h_0+1,n}=\bm{0}$. We have, for $h\le h_0$
    \begin{align}\nn
        V^{\star}_{h}(s; \tilde{\sigma}^{h_0}_{ n})-V_{h,n}(s) &= \max_{a\in\Ac} Q^{\star}_{h}(s,a; \tilde{\sigma}^{h_0}_{ n})-\max_{a\in\Ac}Q_{h,n}(s,a)\\\nn
        &\le \max_{a\in\Ac} \left\{Q^{\star}_{h}(s,a; \tilde{\sigma}^{h_0}_{ n})-Q_{h,n}(s,a)\right\}\\\nn
        & =\max_{a\in\Ac} \left\{[P_hV^{\star}_{h+1}](s,a; \tilde{\sigma}^{h_0}_{ n})-[P_hV_{h+1,n}](s,a)\right\}\\\nn
        &\le 0. 
    \end{align}
    The fist inequality is due to rearrangement of $\max$ and the second inequality is by the base of induction. We thus prove the lemma. 
\end{proof}


\begin{proof}[Proof of Lemma~\ref{lem:v_minus_vpi}]
We have
\begin{align}
    V_{h,n}(s_h)- V^{\pi_n}_h(s_h; \tilde{\sigma}^{h_0}_n)
    &=
    Q_{h,n}(s_h, a_h)- Q^{\pi_n}_h(s_h, a_h; \tilde{\sigma}^{h_0}_n)\\\nn &\le r(s_h, a_h) + [P_hV_{h+1,n}](s_h, a_h) + 2\beta(\delta)\sigma_{n,h}(s_h, a_h)- r(s_h, a_h) - [P_hV^{\pi_n}_{h+1}](s_h, a_h; \tilde{\sigma}^{h_0}_n)\\\nn
    &= V_{h+1, n}(s_{h+1}) - V^{\pi_n}_{h+1}(s_{h+1}; \tilde{\sigma}^{h_0}_n)+ 2\beta(\delta)\sigma_{n,h}(s_h, a_h) \\\nn
    &+ ([P_hV_{h+1,n}](s_h, a_h) - V_{h+1, n}(s_{h+1})) \\\nn
    & + (V^{\pi_n}_{h+1}(s_{h+1}; \tilde{\sigma}^{h_0}_n) - [P_hV^{\pi_n}_{h+1}](s_h, a_h; \tilde{\sigma}^{h_0}_n))
\end{align}

Applying this recursively, we obtain

\begin{align}
    V_{1,n}(s_1)- V^{\pi_n}_1(s_1; \tilde{\sigma}^{h_0}_n) \le \sum_{h=1}^H 2\beta(\delta)\sigma_{n,h}(s_h, a_h) + ...
\end{align}
    
\end{proof}

\section{Discussion of some open problems}\label{app: open problems}

\subsection{Practical computation of actions and action-set violations}\label{subsec: action feasibility}


In \cref{alg: alg shared,alg: alg shared + approx} we used a black-box solver for an entropic optimal transport problem. This is a computational abstraction and not implementable in practice. Implementing a computationally feasible resolution raises several questions. 

\subsubsection{Numerical resolution of the Kantorovich problem}

Sinkhorn's algorithm is the standard method for solving entropic optimal transport problems. It relies on the dual formulation of the entropic problem, that is
\begin{align}
    \ent(\mu,\nu,c,\ve) = \sup_{\substack{\varphi\in L^1(\mu)\\\psi\in L^1(\nu)\\\varphi\oplus\psi\le c}} \left\{\int \varphi\de\mu + \int \psi\de\nu - \ve\int e^{\ve^{-1}(\varphi+\psi-c)}\de(\mu\tensor\nu) + \ve\right\}\,\notag%\label{eq: entropic dual}
\end{align}
in the case $c\in L^1(\mu\tensor\nu)$, see e.g.\ \cite[Thm.~4.7]{nutz_introduction_2022}. The solution of the dual problem is given by the pair $(\varphi^*,\psi^*)$ which satisfies the Schrödinger system
\begin{align*}
    \varphi^*&=-\ve\log\left(\int e^{\frac{\psi^*(y)-c(\cdot,y)}\ve}\de\nu(y)\right) \quad \mu\mbox{-a.s.}\\
    \psi^*&=-\ve\log\left(\int e^{\frac{\varphi^*(x)-c(x,\cdot)}\ve}\de\mu(x)\right) \quad \nu\mbox{-a.s.}\,.
\end{align*}

Sinkhorn's algorithm \citep{sinkhorn_concerning_1967}, in its application to this problem \citep{cuturi_sinkhorn_2013}, is a fixed-point iteration which improves one potential at a time. In other words, for $n\in\Nb$, it computes
\begin{align*}
    \varphi_{2n+1}&=-\ve\log\left(\int e^{\frac{\psi_{2n}(y)-c(\cdot,y)}\ve}\de\nu(y)\right) \\
    \intertext{and}
    \psi_{2n}&=-\ve\log\left(\int e^{\frac{\varphi_{2n-1}(x)-c(x,\cdot)}\ve}\de\mu(x)\right)\,.
\end{align*}

A primal solution to $\ent(\mu,\nu,c,\ve)$ can be recovered from the optimal dual potentials $(\varphi^*,\psi^*)$ via
\[ 
    \de\pi^*=e^{\frac{\varphi^*\oplus\psi^*-c}\ve}\de[\mu\tensor\nu]\,,
\]  
in which $\varphi^*\oplus\psi^*: (x,y)\mapsto \varphi^*(x)+\psi^*(y)$. Through an analogue for ${(\varphi_{2n+1},\psi_{2n})}_{n\in\Nb}$, we can obtain iterates ${(\varpi_n)}_{n\in\Nb}$.

\begin{lemma}[{\cite[Thm.~3.15]{eckstein_quantitative_2022}}]\label{lemma: bound on inf memory sinkhorn}
    If $c$ is Lipschitz on $\supp(\mu)\times\supp(\nu)$, and $\mu,\nu$ are sub-Gaussian measures, then the iterates ${\{\varpi_n(c)\}}_{n\in\Nb}$ of Sinkhorn's algorithm satisfy
    \[
        \entf(c^*,\varpi_n(c)) - \ent(\mu,\nu,c,\ve)\le C_0\ve n^{-\frac14}\,,
    \]
    for every $\ve>0$, in which $C_0$ is a numerical constant independent of $n$.
\end{lemma}
We omit the explicit dependencies in the constant $C_0$ as they are quite technical and require parsing a large part of~\cite{eckstein_quantitative_2022}, which proceeds from within a highly general framework. We should note, however, that consequently their bound is valid under much weaker assumptions than the ones stated here, and that the rate can, in fact, be improved if $c$ has sub-linear growth.


Unfortunately for regret minimisation, $\varpi_n$ need not be a transport plan in $\Pi(\mu,\nu)$, meaning it is not a valid action. Removing the requirement that $\pi_t\in\Pi(\mu,\nu)$ entirely would render the problem meaningless, as the regret can be made negative by finding a single point such that $c(x,y)<\kant(\mu,\nu,c)$, and playing $\delta_{(x,y)}$. 

As an auxiliary remark, this problem is one of the main hurdle to adapting \cref{alg: alg shared} to unknown marginals, as there would be no conceivable way to pick valid transport plans, which renders the analysis a non-starter.


\subsubsection{On action violations}


Two possible directions appear to resolve this issue: one at the level of bandit design, and one at the level of numerical optimal transport. The former revolves around the idea of incorporating action-set violations to regret analysis, the latter around the idea of modifying Sinkhorn's algorithm to produce valid primal iterates at each step, e.g.\ by projecting onto $\Pi(\mu,\nu)$.

The question of violating action sets has been posed before in Bandit Theory and has also arisen in practical use-cases in Reinforcement Learning, see \citep{seurin_im_2020}. It is a staple topic in the context of fairness, see e.g.\ \citep{joseph_fairness_2016} and of contextual bandits (including linear stochastic bandits) in which various other types constraint have also been considered, see e.g. \citep{liu_efficient_2024}. These types of constraints typically, in effect, disable certain arms at certain times, a generic setting which has been considered as well, e.g.\ by~\cite{kleinberg_regret_2010,abensur_productization_2019}. 

These works adopt a range of strategies to formulate the problem in a meaningful way, but their perspectives don't really fit with the real challenge we have with the OT problem. The problem isn't so much that the constraints placed on the action set are complicated: $\Pi(\mu,\nu)$ is a convex, compact set defined by linear inequalities. The problem arises entirely from the facts that $\Pi(\mu,\nu)$ is infinite-dimensional, and that it is a subspace of $\Ps(\state)$, whose geometry is far from straightforward.

A preliminary exploration of this topic would likely require a taxonomy of the different possible violations of $\Pi(\mu,\nu)$. Indeed, $\pi_t$ could violate one or both marginal constraints, or it could even fail to be a probability measure through the total mass or positivity conditions. It appears likely that these will have quite different impacts both on the problem's geometry and on practical usefulness. Thereafter, one might consider whether guaranteeing finitely many violations, as~\cite{liu_efficient_2024} do, or developing a penalised regret is more appropriate.

The alternative would be to design an algorithm which optimises the entropic or Kantorovich problems through while staying within the constraint set $\Pi(\mu,\nu)$ (either for all time, or once it reaches a desired precision). On the one hand, there are finite-dimensional intuitions for this to work as Sinkhorn's algorithm can be viewed as a form of gradient descent \citep{leger_gradient_2021}, which could be projected onto $\Pi(\mu,\nu)$ (which is convex and compact). On the other hand, the geometry of $\Pi(\mu,\nu)$ as an infinite-dimensional probability space is likely to make rigourously doing so (and deriving convergence rates) quite arduous work. 


\subsection{Extensions to the Monge problem}\label{subsec: Monge pb}

The \emph{Monge} optimal transport problem associated to $(\mu,\nu,c)$ is
\begin{align}
    \monge(\mu,\nu,c):= \inf_{T\in\Ts} \int c(x,T(x))\de\mu(x)\,,
    \label{eq: monge def}
\end{align}
in which $\Ts$ is the set of all $\mu$-measurable maps  $T:\Mc_\mu\to\Mc_\nu$ such that $\mu(T^{-1}(\cdot))=\nu$. Chronologically, this is in fact the original formulation of the OT problem \citep{monge1781memoire}. 

The Monge problem is best approached through finite-dimensional practical applications such as \emph{matchings} of students to universities, employees to employers, etc. The requirement that the map $T$ be a function imposes an \emph{indivisibility} of the mass $T$ moves from $\mu$ to $\nu$ (i.e.\ one university per student). This makes the resolution of the problem much more difficult. For example, if $\mu$ and $\nu$ each have two atoms with weights $(1/2,1/2)$ and $(1/3,2/3)$ respectively, then $\Ts=\emptyset$, meaning $\monge(\mu,\nu,\cdot)\equiv +\infty$, and the problem is never solvable. 

If $\mu,\nu$ are non-atomic, $\monge(\mu,\nu,c)$ can be interpreted as the cheapest way (w.r.t. $c$) to transport a $\mu$-shaped pile of infinitesimally small things into a $\nu$-shaped one, but its geometry remains complicated. 
The Kantorovich relaxation drastically simplified the geometry of the problem and remains one of the most effective tools to approach the Monge problem, which is why it is accepted as the standard in modern OT theory.

Note that the relaxation from $\monge(\mu,\nu,c)$ to $\kant(\mu,\nu,c)$ is known to be exact in some cases, such as $c=\norm{\cdot-\cdot}^2/2$ with $\Mc_\mu=\Mc_\nu=\Rb^d$, $(\mu,\nu)$ having second-order moments and $\mu$ being absolutely continuous w.r.t.\ the Lebesgue measure~\cite[Thm.~5.2]{ambrosio_lectures_2021}. See also \citep[Thm.~5.30]{villani_optimal_2009} for weaker conditions. 
But it is also known (e.g.\ via the above example) that this  relaxation is not without loss.

If we want to learn a Monge problem, we must, of course, make sufficient assumptions for it to be solvable, but more importantly we must face the issue that~\eqref{eq: monge def} is now a non-linear functional and that $\Ts$ is not as docile a set as $\Pi(\mu,\nu)$. Here, the recent work in statistical optimal transport on learning Monge maps (i.e.\ the solutions to~\eqref{eq: monge def}) is highly relevant, see e.g.~\cite[Ch.~3]{chewi_statistical_2024} or the paragraph in \cref{app: biblio} below. Though once again most work focuses on the batch sampling of marginals, not on online learning. This line of work would appear to also require more general results about the learning of minima of non-linear functionals, which are not yet available in the literature. Overall, it remains unclear if the Monge problem is on a similar or different level of difficulty to the Kantorovich problem as it is not clear that the techniques to reduce to online least-squares we used will transfer. 

Beyond these statistical issues, one should also expect the problems of effective optimisation from \cref{subsec: action feasibility} to return with a vengeance as the Monge problem is a fully non-linear problem unlike the Kantorovich problem which is an (infinite-dimensional) linear program.


\section{Bibliographical complements on statistical optimal transport}\label{app: biblio}

    An excellent detailed history of the development of OT as a mathematical theory, replete with bibliographical notes, can be found in~\cite[Ch.~3]{villani_topics_2003}. Summarising this field's venerable history further would be of little value. Instead, we will expand on relevant research specifically about \textit{learning} optimal transport problems. We touch on key aspects of the literature below, and refer to the forthcoming book~\cite{chewi_statistical_2024}, for a deeper longitudinal overview.
  
        \paragraph{Estimation of Wasserstein distances}
            One of the most important contributions of optimal transport is a family of useful distances between probability measures: the Wasserstein metrics. The study of these distances has allowed major progress on the geometry of spaces of probability measures, and has been used in many applications. It is therefore natural that the estimation of these distances has been a major topic of interest in the learning of optimal transport. 

            The key question here is the convergence in Wasserstein distance of an empirical distribution to the true distribution. Pioneering work on this topic began in the 80s and 90s, see \citep{ajtai_optimal_1984,talagrand_transportation_1994}, with the study of \emph{Matching} (i.e.\ discrete optimal transport). Key statistical analysis of this problem includes finite sample bounds, see \citep{horowitz_mean_1994} and more recently \citep{fournier_rate_2015,weed_sharp_2019} among others, as well as distributional limits, see e.g.\
            \citep{tameling_empirical_2019} and references therein. 

            Sadly, most work has remained limited to Wasserstein distances rather than generic cost functions, owing to a reliance on the pleasant geometric properties that they enjoy.

        \paragraph{Estimation of Entropic OT}
            Motivated by the success of Entropic OT in designing numerical solution to OT problems, see \citep{cuturi_sinkhorn_2013}, work on the Entropic problem has focused on estimating $\ent(\mu,\nu,c,\ve)$ using $\ent(\hat\mu_n,\hat\nu_n,c,\ve)$, for empirical measures $(\hat\mu_n,\hat\nu_n)$. This has often gone together with estimation for the Schrödinger potentials $(\varphi,\psi)$ of~\eqref{eq: entropic dual}.

            While this is very much the same type of study as for the Kantorovich problem in Wasserstein metrics, it should be noted that the entropic problem exhibits qualitatively different behaviour. While learning the Kantorovich problem exhibits a curse of dimensionality, the entropic problem exhibits parametric-rate (dimension-free) convergence, as shown by~\cite{genevay_sample_2019,rigollet_sample_2022}. This was tempered by large dependencies in other problem quantities, which were reduced over time \citep{stromme_minimum_2024} and were complemented by distributional limits, see e.g.\ \citep{gonzalez-sanz_weak_2024}.

        \paragraph{Estimation of Monge maps}
            While the estimation of Wasserstein distances is mostly motivated by statistical applications, the estimation of Monge maps is motived by effectively solving transport problems in an applied context. Here, one sees samples from two marginals $\mu$ and $\nu$, and attempts to estimate $T^*$ the minimiser of~\eqref{eq: monge def}. 
            
            There has been a significant amount of machine learning and statistics literature on this topic, following on from \citep{hutter_minimax_2021,gunsilius_convergence_2022}. Various types of estimators have been constructed, either derived from optimal transport theory \citep{hutter_minimax_2021}, or from plug-in estimates using classical machine learning methods such as $k$-NN \citep{manole_plugin_2024,deb_rates_2021}. 

        % \paragraph{Estimation of Dual potentials}
        %     (related via resolution of the Kantorovich problem, but offline and full information)
        
        \paragraph{Optimal transport applied to learning}

        While these bibliographical notes concern learning in optimal transport let us conclude by underline that the machine learning community has used optimal transport to impressive success in applications. One could highlight in particular Wassertein GANs \citep{arjovsky_wasserstein_2017} and subsequent works, e.g.\ \citep{salimans_improving_2018} as well as the field of domain adaptation \citep{courty_joint_2017,torres_survey_2021}



%%%%%%%%%%%%%%%%%%%%%%%%%%%%%%%%%%%%%%%%%%%%%%%%%%%%%%%%%%%%%%%%%%%%%%%%%%%%%%%
%%%%%%%%%%%%%%%%%%%%%%%%%%%%%%%%%%%%%%%%%%%%%%%%%%%%%%%%%%%%%%%%%%%%%%%%%%%%%%%
\end{document}