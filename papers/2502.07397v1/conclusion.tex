
\section{Conclusion and open directions}\label{sec: directions}

This article provided the first regret analysis of the bandit optimal transport problem. We have shown (\cref{thm: entropic regret,thm: regret Kantorovich}) that the entropic and Kantorovich formulations of the problem can be solved with simple algorithms based on OFU. In other words, this problem can be reduced to online least-squares, in spite of the measure-valued actions. Consequently, the regret bounds we obtain are of the same order as the best known bounds for the linear bandit problem. 

These bounds however contain an infinite-dimensional term, whose control is uncertain in general. We showed (\cref{subsec: computability}) that several cases of interest can be readily solved from our general framework (matching, parametric models, RKHS), and that more sophisticated adaptive basis truncation methods can be used, with quantified regret bounds.

Both types of results were arrived at as a consequence of the general analysis framework we developed, which combined Fourier analysis techniques with the intrinsic regularity of the entropic OT problem. This framework opens the door to studying a variety of related problems in BOT, such as the Monge problem (see \cref{subsec: Monge pb}).

Our analysis also raises several questions in OT theory. In order to implement the optimism step~\eqref{eq: optimistic planning}, one would need a numerical algorithm which outputs an $\epsilon$-optimal transport plan after finitely many steps. This appears to be absent from the literature, as Sinkhorn's algorithm does \emph{not} output a valid plan in finite time, only in the limit. This also raises the more general question of the regularity properties of this entropy-regularised bilinear problem.
