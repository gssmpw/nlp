 \begin{figure*}[ht]
    \centering
    \noindent\fbox{
        \parbox{\textwidth}{\vspace{0.1cm}
  You are a coding assistant who is writing code to fill in some incomplete classes. 
  \vspace{0.2cm}
  
    You will be provided with the skeleton of the class with function names and docstrings. You may also be provided with some functions already completed.
  
\vspace{0.2cm}

    Your goal is to complete the code according to the docstrings. Do not add anything else to the code, including natural language that is not part of the code or comments. Your generation should be ready to run without needing any modifications. 
  
\vspace{0.2cm}

  Here is the skeleton:
  \vspace{0.4cm}

  \bluetext{\textit{\{underspec question\}}}
  \vspace{0.4cm}

  Here is your last version of the skeleton:

 \vspace{0.4cm}
 
  \textasciigrave\textasciigrave\textasciigrave python
  
   \vspace{0.2cm}
   
   \bluetext{\textit{\{prev solution\}}}
   
   \vspace{0.2cm}
   
  \textasciigrave\textasciigrave\textasciigrave
  \vspace{0.4cm}
  The user provided the following feedback on the code:
    \vspace{0.4cm}

  \bluetext{\textit{\{user response\}}}
\vspace{0.4cm}
  You can choose to use this response, or you can choose to ignore it. Only incorporate the information that you think is relevant and helpful into the code. 
  
  
  Enclose your solution in a markdown block beginning with \textasciigrave\textasciigrave\textasciigrave python. When you are ready to submit your response, please end the markdown block with \textasciigrave\textasciigrave\textasciigrave on a new line.
  
  \vspace{0.4cm}

    \tantext{
 \textasciigrave\textasciigrave\textasciigrave python

  \vspace{0.2cm}
  
}

        }
        
    }
    \caption{Prompt given to \cm{} in the \para{}, \sent{}, \cf{} setting for Classeval. \bluetext{Blue text} indicates that the relevant text would be inserted at that location in the prompt. \tantext{Red text} is prefill for the \cm{}. }
    \label{prompts:asst-nl_para_sent_classeval}
\end{figure*}