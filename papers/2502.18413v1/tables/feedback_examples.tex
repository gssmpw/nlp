% Please add the following required packages to your document preamble:
% \usepackage[normalem]{ulem}
% \useunder{\uline}{\ul}{}
\begin{table*}
\centering
\resizebox{\textwidth}{!}{
\begin{tabular}{ll}
\toprule
\textbf{Feedback Style}       & \textbf{Generated Feedback on APPS (Interview) }                                                                                                                                                                                                                                                                                                                                                                                                                                                                                                                                                                                                                                                                                     \\ \hline
\sent & \begin{tabular}[c]{@{}l@{}}The current solution doesn't account for the cyclic nature of the strings and may count duplicates,\\so we need to implement a proper cyclic string comparison and ensure uniqueness in the count.\end{tabular}                                                                                                                                                                                                                                                                                                                                                                                                                                                                                \\\hline
\para & \begin{tabular}[c]{@{}l@{}}The current solution misunderstands the problem. It's not counting distinct cyclic strings correctly, \\ as it doesn't consider all possible binary strings of length n that contain s as a substring. \\ The solution should use a more sophisticated approach, possibly involving dynamic programming \\ or KMP algorithm, to generate and count all valid cyclic strings. It needs to handle cases where s \\ can appear at any position in the cyclic string, not just at the beginning. The current implementation\\  also doesn't account for the cyclical nature of the strings properly. A complete overhaul of the \\ algorithm is needed to correctly solve this problem.\end{tabular} \\\hline
\cf   & \begin{tabular}[c]{@{}l@{}}The main issue is that this solution doesn't correctly handle cyclic strings. The line:\\ \\ ```python\\ if s in t or s in t+t:\\ ```\\ \\ should be replaced with:\\ \\ ```python\\ if s in (t * 2):\\ ```\\ \\ This change ensures that all possible cyclic shifts are checked. Additionally, the solution\\  doesn't account for the fact that cyclic shifts of the same string should be counted as one distinct string.\\ A more fundamental restructuring of the algorithm is needed to correctly solve the problem as \\ described in the full problem statement.\end{tabular}                                                                                                            \\\hline
% \ir   & \begin{tabular}[c]{@{}l@{}}Question: Given a binary string s and an integer n (1 $\leq$ n $\leq$ 40, |s| $\leq$ n), count the number of \\ distinct cyclical binary strings of length n containing s as a substring. A cyclical string contains s \\ if any rotation includes s. Distinct cyclical strings differ as strings, even if they're rotations. \\ Use a KMP-like algorithm to build a state machine for matching s, then use dynamic programming \\ to count valid strings. Consider how to handle partial matches at the end of the string that could complete\\  a match when wrapped around to the beginning.\end{tabular} \\ \hline    
\ir{} & \begin{tabular}[c]{@{}l@{}} Question: Calculate the number of unique cyclical binary strings of length n that include \\ a given binary string s as a substring. A cyclical string contains s if any  rotation of the string \\ includes s. Two cyclical strings are different if they're not rotations of each other. Return the count modulo $10^9 + 7$ \end{tabular} \\ \hline
\end{tabular}
}
\caption{Example feedback from \sonnet{}, given in response to a proposed solution for APPS Interview (Question \#42)}
\label{tables:sample-feedback}
\end{table*}