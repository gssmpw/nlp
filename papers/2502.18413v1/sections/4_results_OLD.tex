

\section{Results}
We discuss the overall trends in model behavior when comparing static to interactive settings. \val{mention focus on feedback types}


\input{figtext/edit_distance_vs_steerability_ranking}

\subsection{General trends in model behavior}
Figure \ref{fig:ranking_changes} shows our main results using all \nmodels{} models on the three datasets: APPS, LiveCodeBench, and ClassEval.
The plot shows the relative rankings of each model (measured by their test case accuracy on each dataset) on each of the settings: \vanilla, \para, \sent, \cf, and \ir.

\paragraph{Relative performance between models change between different static and interactive settings.} 
Figure \ref{fig:ranking_changes} shows that the rankings of models often vary depending on which setting is being used to evaluate performance. 
% While \qwenSmall{} performs third in \vanilla{} for LiveCodeBench, it falls to fifth in other feedback settings, and even seventh in \cf. 
% \gemmaLarge{} performs fifth in \vanilla{} and even falls to seventh in \ir, yet jumps to third in \para. \sonnet{} beats \gpt{} in all settings of APPS Interview, except for \cf, where their order inverts.
Tables \jane{ref} shows the normalized Spearman Footrule distances ($\tilde{F})$ between the \vanilla{} and interactive settings. 
It is clear that the choice of evaluation setting has a profound impact on the relative performance of models. 
For ClassEval, $\tilde{F}$ ranges from $0.98$ to $0.67$, indicating anti-correlation between the \vanilla{} and interactive settings.
LiveCodeBench and APPS also demonstrate relatively weak positive correlation on many interactive settings; for instance, for \cf, $\tilde{F}$ is $0.36$ and $0.26$, respectively.
For these two datasets, top models tend to be more consistent, preserving their positions in high positions, whereas weaker models tend to demonstrate more variance in rank.
%\val{i think we should talk about top models being more consistent even with feedback}


% \val{add another paragraph to talk about which feedback types are most helpful}

\paragraph{Dataset difficulty correlates with ranking.}
\jane{Help}

\paragraph{Underspecification hurts model performance, and performance is recoverable only with the incorporation of feedback.}

Figure \ref{fig:vanilla_baseline} plots the best feedback setting for each model, as well as its performance on the \baseline{} and \vanilla{} settings. 
Comparing the blue bars (\vanilla) to the orange bars (\baseline), it is clear that our input perturbation successfully underspecifies the problem, as \baseline{}  consistently underperforms the \vanilla{} setting. 
In order to recover performance comparable with the \vanilla{} setting, the models must interact with the \user{} (as in the \para, \sent, \cf, and \ir{} settings).

\paragraph{\cf{} and \para{} boost performance the most.}


\subsection{Effect of Feedback Quality}
We examine the effects of the feedback type and quality level on performance. \val{provide some intuition for why we look at this}


\paragraph{Feedback quality is consistent across models.}
Table \ref{tab:feedback_quality_max_step_number} compares average feedback quality and the average maximum number of steps.
The feedback quality does not vary greatly across models and often falls above $0.9$.
This suggests that the feedback is high-quality enough to be compared across models on different trajectories of iterative refinement. 
\jane{See HH's comment. Not sure if I captured this sensibly.}

We note that our metric of feedback quality does not correlate with a reduction of the average maximum number of steps ($r=0.48$).
Rather, the maximum number of steps seems to be more related to feedback type, as \cf{} consistently has the the smallest maximum step number, followed by \para{} and \sent.

Surprisingly, when \sonnet{} is the \cm{}, it receives the worst quality feedback compared to the other models, despite being both the \cm{} and the \user{}. 
However, it still completes the solutions the fastest, and consistently has the lowest average maximum step number compared to \gpt{} and \sonnet. 


\paragraph{Code models are robust to \lqf{} feedback.}
Figure~\ref{fig:effect_of_fq} shows the distribution of solutions whose performances improve, do not change, or decrease when comparing \hqf{} feedback to \lqf{} feedback. 
High quality feedback reduces the proportion of feedback that does not have an effect on the solution.
This usually results in a higher rate of solutions whose performances are improved.
\textsc{high-quality} \sent{} feedback usually decreases the rate of worse solutions.
Interestingly, while \cf{} often increases the proportion of improved solutions more than \para{} or \sent, it almost always increases the proportion of worse solutions.




%\paragraph{Average feedback quality varies per feedback type.} 
%\val{this takeaway is weak}


\begin{figure*}[ht]
    \centering
    \includegraphics[width=0.98\textwidth]{figs/edit_distance_vs_steerability_heatmap_1.pdf}

    \caption{
        Distribution of surface-level steerability ($x$-axis) vs. behavioral steerability ($y$-axis) for all models during the first step of iterative refinement. While some models make only surface-level changes that do not induce much behavioral change in code (e.g. \gemmaSmall), others are also able to make highly effective edits that induce large changes in the behavior of the solution (e.g., \sonnet, \qwenLarge, \gpt).
    }
    \vspace{-5pt}
    \label{fig:edit_distance_vs_steerability_heatmap}
\end{figure*}



\subsection{Effect of Feedback Type and Code Model Steerability}

\val{reframe as how models respond to feedback, rename subsection}
The effects of feedback formats and quality level on surface-level and behavioral-level steerability.
In this section, we discuss how different aspects of feedback affect the \cm 's ability to change its previous version of the solution.

\paragraph{\para{} feedback is associated with higher steerability across all models.} 
Figure \ref{fig:edit_distance_vs_steerability_ranking} ranks each feedback type according to how behavioral (left) or surface-level (right) steerability it induces.
\para{} and \cf{} consistently rank the highest in behavioral steerability, ranking first and second, respectively. 
Likewise, \para{} and \sent{} rank the highest in surface-level steerability. 
This suggests that some feedback types tend to induce similar steerability relative to other feedback types on the same model.


\paragraph{Models react differently to the same forms of feedback, both in terms of behavioral- and surface-level steerability.} 
Figure \ref{fig:edit_distance_vs_steerability_heatmap} shows the effect of feedback type on how \cm s change their previous solutions for \gpt{} on APPS Interview.

Some models (e.g. \gemmaSmall{} or \reka) tend to make large number of surface-level changes that do not greatly affect the behavior of the language model.
Other models (e.g. \gpt{} or \sonnet, and \qwenLarge) demonstrate a bimodal distribution, where some edits highly affect code behavior and others do not.
% While \para{} has demonstrates bimodal 
Appendix \ref{app:extra_figs} shows additional tables and figures.

% Figure \ref{fig:steerability_vs_performance_delta} plots the changes in test cases against the change in performance on a step-by-step basis.
% Green points are code solutions where the last step had an improvement in performance; red points are code solutions where the last step did not have an improvement in performance.

% We note that while the first step has relatively few points with a decrease in performance, the following steps have increasingly high "tails" where changes in test case behavior are attributed to high attrition of performance.
% Moreover, these "tails" are predominantly made up of points where the last step had an improvement in performance.
% Because the interaction is generated in a history-less setting, this cannot be attributed to weaker performance induced by long conversation contexts.
% The tails are also present when history is included in the conversation, meaning that the tails also cannot be attributed to loss of information from lack of history. \redtext{Need to double check this!}
% More details about history and its effect on results can be found in \ref{app:ablations}.


