\newpage
\section{Appendix}
\label{sec:appendix}



\subsection{Additional Details on Datasets}
\label{app:datasets}
Our pipeline is designed to accommodate generic static benchmarks with some modifications. 
For instance, because ClassEval requires the model to fill in the skeleton code of a class (rather than providing explicit programming questions), we underspecify its problems by summarizing the docstrings for each method.
Likewise, LiveCodeBench does not provide ground-truth solutions, so we generate solutions for LiveCodeBench by sampling twice from \sonnet{}; if a correct solution is generated, we use it as the ground-truth solution for the question. 
(If not, we do not use the question as our pipeline requires the presence of a ground-truth solution in the \user{} prompt.)  

For evaluation of APPS and LiveCodeBench, Table \ref{tables:perf_per_dataset_setting}, Table \ref{tab:feedback_quality_max_step_number}, and Figure \ref{fig:ranking_changes} average across difficulty levels for brevity.

\subsection{Example Feedback}
\label{app:feedback}
% Please add the following required packages to your document preamble:
% \usepackage[normalem]{ulem}
% \useunder{\uline}{\ul}{}
\begin{table*}
\centering
\resizebox{\textwidth}{!}{
\begin{tabular}{ll}
\toprule
\textbf{Feedback Style}       & \textbf{Generated Feedback on APPS (Interview) }                                                                                                                                                                                                                                                                                                                                                                                                                                                                                                                                                                                                                                                                                     \\ \hline
\sent & \begin{tabular}[c]{@{}l@{}}The current solution doesn't account for the cyclic nature of the strings and may count duplicates,\\so we need to implement a proper cyclic string comparison and ensure uniqueness in the count.\end{tabular}                                                                                                                                                                                                                                                                                                                                                                                                                                                                                \\\hline
\para & \begin{tabular}[c]{@{}l@{}}The current solution misunderstands the problem. It's not counting distinct cyclic strings correctly, \\ as it doesn't consider all possible binary strings of length n that contain s as a substring. \\ The solution should use a more sophisticated approach, possibly involving dynamic programming \\ or KMP algorithm, to generate and count all valid cyclic strings. It needs to handle cases where s \\ can appear at any position in the cyclic string, not just at the beginning. The current implementation\\  also doesn't account for the cyclical nature of the strings properly. A complete overhaul of the \\ algorithm is needed to correctly solve this problem.\end{tabular} \\\hline
\cf   & \begin{tabular}[c]{@{}l@{}}The main issue is that this solution doesn't correctly handle cyclic strings. The line:\\ \\ ```python\\ if s in t or s in t+t:\\ ```\\ \\ should be replaced with:\\ \\ ```python\\ if s in (t * 2):\\ ```\\ \\ This change ensures that all possible cyclic shifts are checked. Additionally, the solution\\  doesn't account for the fact that cyclic shifts of the same string should be counted as one distinct string.\\ A more fundamental restructuring of the algorithm is needed to correctly solve the problem as \\ described in the full problem statement.\end{tabular}                                                                                                            \\\hline
% \ir   & \begin{tabular}[c]{@{}l@{}}Question: Given a binary string s and an integer n (1 $\leq$ n $\leq$ 40, |s| $\leq$ n), count the number of \\ distinct cyclical binary strings of length n containing s as a substring. A cyclical string contains s \\ if any rotation includes s. Distinct cyclical strings differ as strings, even if they're rotations. \\ Use a KMP-like algorithm to build a state machine for matching s, then use dynamic programming \\ to count valid strings. Consider how to handle partial matches at the end of the string that could complete\\  a match when wrapped around to the beginning.\end{tabular} \\ \hline    
\ir{} & \begin{tabular}[c]{@{}l@{}} Question: Calculate the number of unique cyclical binary strings of length n that include \\ a given binary string s as a substring. A cyclical string contains s if any  rotation of the string \\ includes s. Two cyclical strings are different if they're not rotations of each other. Return the count modulo $10^9 + 7$ \end{tabular} \\ \hline
\end{tabular}
}
\caption{Example feedback from \sonnet{}, given in response to a proposed solution for APPS Interview (Question \#42)}
\label{tables:sample-feedback}
\end{table*}

We provide sample feedback from \sonnet{} in response to a proposed solution. All feedback types are in response to the same question (Table \ref{tables:sample-feedback}).


\subsection{Additional Details on Models}\label{app:models}

We obtained the weights for google/gemma-7b-it (\gemmaSmall{}) from Hugging Face at \url{https://huggingface.co/google/gemma-7b-it}, meta-llama/Meta-Llama-3.1-8B-Instruct (\llama{}) from huggingface at \url{https://huggingface.co/meta-llama/Llama-3.1-8B-Instruct}, and Qwen/Qwen2.5-Coder-7b-Instruct (\qwenSmall{}) from Hugging Face at \url{https://huggingface.co/Qwen/Qwen2.5-Coder-7B-Instruct}. 
We run each of the models on a single L40S GPU. We use a temperature setting of $0.9$, $4096$ max tokens, and the ``do\_sample'' setting enabled. 
We use $\langle$end\_of\_turn$\rangle$, $\langle$|eot\_id|$\rangle$, $\langle$|im\_end|$\rangle$ as the EOS token for \gemmaSmall{}, \llama{}, and \qwenSmall{} respectively.

We use Together AI (\url{https://api.together.xyz/}) to run the large open-weight models google/gemma-2-27b-it (\gemmaLarge{}) and Qwen/Qwen2.5-Coder-32B-Instruct (\qwenLarge{}). 
We set ``n\_sample'' to $1$ when generating code solutions and limit the max number of tokens to $4096$.
The weights for these models can be found at \url{https://huggingface.co/google/gemma-2-27b-it} and \url{https://huggingface.co/Qwen/Qwen2.5-Coder-32B-Instruct} respectively.

We access \texttt{c4ai-aya-expanse-32b} (\aya{}) through the Cohere API at \url{https://cohere.com/research/aya}, reka-core-20240501 (\reka{}) through the Reka API at \url{https://www.reka.ai/reka-api}, and deepseek-chat (\deepseek{}) through the Deepseek API at \url{https://api-docs.deepseek.com/}.  
We use the default API settings for inference, limiting the max number of tokens to $4096$.

\gpt{} inference is done through the OpenAI API \url{https://platform.openai.com/docs/overview} and \sonnet{} inference through the Anthropic API \url{https://www.anthropic.com/api}.
We use the default API settings for inference, limiting the max number of tokens to $4096$.


\subsection{Prompts}\label{app:prompts}
We use \sonnet{} to generate \user{} feedback for out experiments, but we show that other models can produce comparable feedback (Figure \ref{fig:4o_user_w_sonnet_apps})

We provide all of the prompts for the \user{} model and \cm{}. 
All \user{} model prompts were provided with the same system prompt with the original question and code solution (Figure \ref{prompts:user-sys-prompt}). 
The \para{} prompt (Figure \ref{prompts:user-para-prompt}) and \sent{} prompt (Figure \ref{prompts:user-sent-prompt}) are given the current \cm{} solution and generate feedback constrained by output length. 
The \cf{} prompt is given the current \cm{} solution and provides a correction to specific lines of code in the solution (Figure \ref{prompts:user-code-prompt}). 
The \ir{} feedback prompt is given the current \cm{} solution and underspecified question and generates an updated version of the question with missing details (Figure \ref{prompts:user-input-refine-prompt}).

The \cm{} prompts for APPS and LiveCodeBench are given in Figure \ref{prompts:asst-vanilla_apps_lcb} and Figure \ref{prompts:asst-nl_para_sent_apps_lcb}.
The \cm{} prompts for ClassEval is given in Figure \ref{prompts:asst-classeval_vanilla} and Figure \ref{prompts:asst-nl_para_sent_classeval}.

The 11-shot prompt used to summarize APPS and LiveCodeBench questions is given in Figure \ref{prompts:summarization_apps_lcb}.
The prompt used to summarize each docstring in ClassEval is given in Figure \ref{prompts:summarization_classeval}.


\begin{figure}[t]
    \includegraphics[width=0.98\columnwidth]{figs/4o_user.pdf}
    \caption{
        APPS (Interview) test case accuracy with \sonnet{} as the coding model and \texttt{GPT-4o-mini} as the \user{} model providing feedback. 
    }
    \vspace{-5pt}
    
    \label{fig:4o_user_w_sonnet_apps}
\end{figure}

\begin{figure*}[ht]
    \centering
    \noindent\fbox{
        \parbox{\textwidth}{\vspace{0.1cm}
  You are an expert human programmer who is using a coding assistant to write code in order to solve some programming puzzles. The coding assistant has completed a potential solution to the problem, but needs your help to make adjustments to the code. \vspace{0.2cm}

  You have access to the full question, including formatting instructions and some test cases. You also have access to a natural language description of the correct solution. The coding assistant has access to a summarized, less detailed version of the problem, but only you have access to the full problem. This means that the code assistant may need additional information on how the code should work or how its output should be formatted.\vspace{0.2cm}

  Here is the description of the programming problem:
    \vspace{0.4cm}

    \bluetext{\{\textit{full question}\}}\vspace{0.4cm}
    
  Here is a description of the correct solution:
\vspace{0.4cm}

    \bluetext{\{\textit{solution info}\}}\vspace{0.4cm}
        }
        
    }
    \caption{System prompt given to \user{} model. \bluetext{Blue text} indicates that the relevant text would be inserted at that location in the prompt.}
    \label{prompts:user-sys-prompt}
\end{figure*}
\input{prompts/user_para}
\begin{figure*}[ht]
    \centering
    \noindent\fbox{
        \parbox{\textwidth}{\vspace{0.1cm}

 Your goal is to provide feedback about the solution that you think would help the assistant fix or adjust the code. This feedback should be purely about the function of the code, not its aesthetics or nonessential structure (e.g. do not make a suggestion regarding optimization or other choices that would not change how the code behaves). The coding assistant will use this feedback to help generate the next version of the code. Write one sentence of feedback that you think would be most helpful to the coding assistant. Do not write more than 50 words.
  \vspace{0.2cm}

  Here is the code assistant's solution.

  \vspace{0.4cm}
\textasciigrave\textasciigrave\textasciigrave python\\
  \vspace{0.2cm}
\bluetext{\textit{\{full solution\}}}\\
    \vspace{0.2cm}
  \textasciigrave\textasciigrave\textasciigrave
\vspace{0.4cm}

Please provide one sentence of feedback that would best help the coding assistant write a better version of the solution. Don't copy the code. Just write your reply.

        }
        
    }
    \caption{Prompt given to \user{} model to get \sent{} feedback. \bluetext{Blue text} indicates that the relevant text would be inserted at that location in the prompt.}
    \label{prompts:user-sent-prompt}
\end{figure*}
 \begin{figure*}[ht]
    \centering
    \noindent\fbox{
        \parbox{\textwidth}{\vspace{0.1cm}

  Your goal is to provide feedback about the solution that you think would help the assistant fix or adjust the code. This feedback should be purely about the function of the code, not its aesthetics or nonessential structure (e.g. do not make a suggestion regarding optimization or other choices that would not change how the code behaves). The coding assistant will use this feedback to help generate the next version of the code. Point out specific lines of the code that are incorrect and explain why. Do not write more than 100 words.\vspace{0.2cm}

  Here is the code assistant's solution.

  \vspace{0.4cm}
  
\textasciigrave\textasciigrave\textasciigrave python

  \vspace{0.2cm}
  
\bluetext{\textit{\{full solution\}}}

    \vspace{0.2cm}
    
  \textasciigrave\textasciigrave\textasciigrave
  
\vspace{0.4cm}

  Please point out specific lines of the code that are incorrect and give the corrected version. Make sure you copy paste the specific line in the solution which is incorrect! You should write both the original line (exactly as found in the solution) and also write what line it should be replaced with.
        }
        
    }
    \caption{Prompt given to \user{} model to get \cf. \bluetext{Blue text} indicates that the relevant text would be inserted at that location in the prompt.}
    \label{prompts:user-code-prompt}
\end{figure*}
 \begin{figure*}[ht]
    \centering
    \noindent\fbox{
        \parbox{\textwidth}{\vspace{0.1cm}

  Here is the code assistant's solution.
  \vspace{0.4cm}
  
\textasciigrave\textasciigrave\textasciigrave python

  \vspace{0.2cm}
  
\bluetext{\textit{\{full solution\}}}

    \vspace{0.2cm}
    
  \textasciigrave\textasciigrave\textasciigrave
  
\vspace{0.4cm}

  Here is the previous version of the question.
  \vspace{0.4cm}
  
  \bluetext{\textit{\{underspecified question\}}}
  
  \vspace{0.4cm}
  
Please rewrite the question so as to provide an updated question of similar length which would help the model generate a better version of the code. Make sure you don't make the new question longer than the older version! Begin your response with "Question:" and don't add any extra text to the end.
        }
        
    }
    \caption{Prompt given to \user{} model to get \ir{} feedback. \bluetext{Blue text} indicates that the relevant text would be inserted at that location in the prompt.}
    \label{prompts:user-input-refine-prompt}
\end{figure*}
 \begin{figure*}[ht]
    \centering
    \noindent\fbox{
        \parbox{\textwidth}{\vspace{0.1cm}
  You are a coding assistant who is writing code in order to solve some programming puzzles. 
  \vspace{0.2cm}
  
  You will be provided with a summary of the problem. You may also be provided with some starter code that you need to complete.
  
\vspace{0.2cm}

  Your goal is to complete the code so as to solve the problem. Do not add anything else to the code, including natural language that is not part of the code or comments. Your generation should be ready to run without needing any modifications.
  
\vspace{0.2cm}

  \orangetext{APPS: Keep in mind that for this dataset, the results should be printed to stdout, so don't just write a function without calling it or printing something to stdout.}
  
\vspace{0.2cm}

  Here is the programming problem description:
  \vspace{0.4cm}

  \bluetext{\textit{\{underspec question\}}}
  \vspace{0.4cm}
  
  Enclose your solution in a markdown block beginning with \textasciigrave\textasciigrave\textasciigrave python. When you are ready to submit your response, please end the markdown block with \textasciigrave\textasciigrave\textasciigrave on a new line.
  
  \vspace{0.4cm}

    \tantext{
 \textasciigrave\textasciigrave\textasciigrave python

  \vspace{0.2cm}
  
\textit{\{partial solution\}}
}

        }
        
    }
    \caption{Prompt given to \cm{} in the \vanilla{}, \baseline{}, and \ir{} settings for APPS and LiveCodeBench. \bluetext{Blue text} indicates that the relevant text would be inserted at that location in the prompt. \orangetext{Orange text} is the APPS-specific formatting instructions. \tantext{Red text} is prefill for the \cm{}. }
    \label{prompts:asst-vanilla_apps_lcb}
\end{figure*}
\input{prompts/asst_nl_para_sent_apps}
\begin{figure*}[ht]
    \centering
    \noindent\fbox{
        \parbox{\textwidth}{\vspace{0.1cm}
  You are a coding assistant who is writing code in order to fill in the skeleton of a python class. 
  
    \vspace{0.2cm}
    
  You will be provided with the skeleton of the class with function names and docstrings. You may also be provided with some functions already completed.

\vspace{0.2cm}

  Your goal is to complete the code according to the docstrings. Do not add anything else to the code, including natural language that is not part of the code or comments. Your generation should be ready to run without needing any modifications. 

  \vspace{0.2cm}

  Here is the skeleton:

  \vspace{0.4cm}

  \bluetext{\textit{\{underspec question\}}}
  
  \vspace{0.4cm}
  
  Enclose your solution in a markdown block beginning with \textasciigrave\textasciigrave\textasciigrave python. When you are ready to submit your response, please end the markdown block with \textasciigrave\textasciigrave\textasciigrave on a new line.
  
  \vspace{0.4cm}

    \tantext{
 \textasciigrave\textasciigrave\textasciigrave python

  \vspace{0.2cm}
  
\textit{\{partial solution\}}
}

        }
        
    }
    \caption{Prompt given to \cm{} in the \vanilla{}, \baseline{}, and \ir{} settings for ClassEval. \bluetext{Blue text} indicates that the relevant text would be inserted at that location in the prompt. \tantext{Red text} is prefill for the \cm{}. }
    \label{prompts:asst-classeval_vanilla}
\end{figure*}
 \begin{figure*}[ht]
    \centering
    \noindent\fbox{
        \parbox{\textwidth}{\vspace{0.1cm}
  You are a coding assistant who is writing code to fill in some incomplete classes. 
  \vspace{0.2cm}
  
    You will be provided with the skeleton of the class with function names and docstrings. You may also be provided with some functions already completed.
  
\vspace{0.2cm}

    Your goal is to complete the code according to the docstrings. Do not add anything else to the code, including natural language that is not part of the code or comments. Your generation should be ready to run without needing any modifications. 
  
\vspace{0.2cm}

  Here is the skeleton:
  \vspace{0.4cm}

  \bluetext{\textit{\{underspec question\}}}
  \vspace{0.4cm}

  Here is your last version of the skeleton:

 \vspace{0.4cm}
 
  \textasciigrave\textasciigrave\textasciigrave python
  
   \vspace{0.2cm}
   
   \bluetext{\textit{\{prev solution\}}}
   
   \vspace{0.2cm}
   
  \textasciigrave\textasciigrave\textasciigrave
  \vspace{0.4cm}
  The user provided the following feedback on the code:
    \vspace{0.4cm}

  \bluetext{\textit{\{user response\}}}
\vspace{0.4cm}
  You can choose to use this response, or you can choose to ignore it. Only incorporate the information that you think is relevant and helpful into the code. 
  
  
  Enclose your solution in a markdown block beginning with \textasciigrave\textasciigrave\textasciigrave python. When you are ready to submit your response, please end the markdown block with \textasciigrave\textasciigrave\textasciigrave on a new line.
  
  \vspace{0.4cm}

    \tantext{
 \textasciigrave\textasciigrave\textasciigrave python

  \vspace{0.2cm}
  
}

        }
        
    }
    \caption{Prompt given to \cm{} in the \para{}, \sent{}, \cf{} setting for Classeval. \bluetext{Blue text} indicates that the relevant text would be inserted at that location in the prompt. \tantext{Red text} is prefill for the \cm{}. }
    \label{prompts:asst-nl_para_sent_classeval}
\end{figure*}
 \begin{figure*}[ht]
    \centering
    \noindent\fbox{
        \parbox{\textwidth}{ \vspace{0.1cm}
        Summarize the question with header 'FORMAL QUESTION' using only natural language. Write your summary under 'SUMMARY'
        
  Do not use any variable or function names from the question. Do not write any code.

   "You are given a coding/algorithmic question below. Your goal is to come up with a \bluetext{\textit{\{sent length\}}} sentence summary using natural language to describe the problem.
   
  "Here are examples of a question and a summary labeled 'EX QUESTION' and 'EX SUMMARY'. 
  Format your summary of 'FORMAL QUESTION' in a similar way to these summaries using \bluetext{\textit{\{sent length\}}} sentence(s).

    \vspace{0.4cm}

  \#\#\#EX QUESTION
  
    \vspace{0.2cm}
    
  Example question 1

    \vspace{0.4cm}

\#\#\#EX SUMMARY

  \vspace{0.2cm}
  
 Example summary 1

  \vspace{0.4cm}

  \#\#\#EX QUESTION

  \vspace{0.2cm}
  
  Example question 2

  \vspace{0.4cm}

\#\#\#EX SUMMARY

  \vspace{0.2cm}

Example summary 2

  \vspace{0.4cm}
  
  \#\#\#EX QUESTION

  \vspace{0.2cm}
  
  Example question 3

  \vspace{0.4cm}

\#\#\#EX SUMMARY

  \vspace{0.2cm}

 Example summary 3

$\vdots$

  \vspace{0.4cm}

\#\#\#EX QUESTION

  \vspace{0.2cm}
  
 Example question 11

  \vspace{0.4cm}

  \#\#\#EX SUMMARY

  \vspace{0.2cm}
  
  Example summary 11

  \vspace{0.4cm}

  \#\#\#FORMAL QUESTION
  
    \vspace{0.2cm}
    
  \bluetext{\textit{\{question\}}}
  
  \vspace{0.4cm}
  
  \#\#\#SUMMARY
  
    \vspace{0.2cm}
    
  <YOUR SUMMARY HERE>
 
        }
        
    }
    \caption{Format for the 11 shot prompt we use to generate summaries in APPS and LiveCodeBench problems. \bluetext{Blue text} indicates that the relevant text would be inserted at that location in the prompt.}
    \label{prompts:summarization_apps_lcb}
\end{figure*}
 \begin{figure*}[ht]
    \centering
    \noindent\fbox{
        \parbox{\textwidth}{\vspace{0.1cm}
  Do not reference any variables or function names. Do not write any code or examples of behavior.

    \vspace{0.4cm}

  You are given a method signature and a docstring below. Write a short, one sentence summary of the docstring. Do not use code or examples in your summary. Retain only the key information. Do not use more than 15 words.

    \vspace{0.4cm}

  \#\# SIGNATURE and DOCSTRING
    \bluetext{\textit{\{function\}}}

  \#\# SUMMARY
  <YOUR SUMMARY HERE>
  

  
        }
        
    }
    \caption{Prompt we use to generate summarized docstring for each function in ClassEval skeletons. \bluetext{Blue text} indicates that the relevant text would be inserted at that location in the prompt.}
    \label{prompts:summarization_classeval}
\end{figure*}

\subsection{Performance Metrics}
\label{app:performance_metrics}
\paragraph{Test case accuracy.}
Test case accuracy can be defined as below:
%
\begin{equation*}
    \textsc{TCA} = \frac{\text{\# Test Cases Passed}}{\text{Total \# Test Cases}}
\end{equation*}
\paragraph{Normalized Spearman's Footrule distance.}

The normalized Spearman's Footrule distance is:
%
\begin{align*}
    \tilde{F}(\sigma_A, \sigma_B) &= \frac{ \sum^n_{i=1} |\sigma_A(i) - \sigma_B(i)|}{\max_{\sigma, \sigma'}  \sum^n_{i=1} |\sigma(i) - \sigma'(i)|}
\end{align*}
%
Consider two rankings $\sigma_A$ and $\sigma_B$ over items $\{1, 2, \dots, n\}$. To measure the distance between them, we use Spearman's Footrule Distance, which can be thought of as the Manhattan distance between two rankings:

\begin{equation*}
    F(\sigma_A, \sigma_B) = \sum^n_{i=1} |\sigma_A(i) - \sigma_B(i)|
\end{equation*}

We normalize $F$ by its maximum possible value, $\frac{n^2}{2}$ for even $n$, to get the \textbf{Normalized Spearman's Footrule Distance}. 
%
\begin{align*}
    \tilde{F}(\sigma_A, \sigma_B) &= \frac{F(\sigma_A, \sigma_B)}{\frac{n^2}{2}} \\
    &= \frac{2F(\sigma_A, \sigma_B)}{n^2}
\end{align*}
%
where $\tilde{F}: \to [0, 1]$. In other words, $\tilde{F}=0$ indicates $\sigma_1 = \sigma_2$, whereas $\tilde{F}=1$ indicates the maximum possible distance between $\sigma_1$ and $\sigma_2$.

Now, we would like to derive the expected $\tilde{F}$ between two rankings which are completely uncorrelated. 
Let us randomly sample $\sigma_A, \sigma_B$ uniformly at random. Then the expected Spearman's Footrule Distance ($F$) is:
%
\begin{align*}
\mathbb{E}[F(\sigma_A, \sigma_B)] &= \mathbb{E}[\sum^n_{i=1} |\sigma_A(i) - \sigma_B(i)|] \\
&= \sum^n_{i=1} \mathbb{E}[|\sigma_A(i) - \sigma_B(i)|] \\
&=  \sum^n_{i=1} \frac{n+1}{3} \\
&=  \frac{n(n+1)}{3} 
\end{align*}
Normalizing this by the maximum possible $F$ gives:
\begin{align*}
    \frac{\frac{n(n+1)}{3}}{\frac{n^2}{2}} &= \frac{2(n+1)}{3n} 
\end{align*}
Thus, for uncorrelated rankings of length \nmodels, $\tilde{F}\simeq 0.73$; for a perfectly correlated pair of rankings, $\tilde{F}=0$; and for perfectly anti-correlated rankings, $\tilde{F}=1$.

\subsection{Additional Details on Measuring Feedback Quality}
\label{app:fq}


\paragraph{Automatic classification of directional correctness.} We use \gpt{} to classify the feedback into two classes: (1) the feedback claimed that the solution was correct or (2) the feedback claimed that the solution was incorrect. 
As some feedback claims that the ``logic'' of the solution is correct, but then states that it is missing critical edge cases or input/output formatting, we also apply rule-based string matching to re-classify such feedback as incorrect.
We then compare the feedback to the actual \textsc{TCA} performance to classify it into \hqf{} vs. \lqf{} feedback.

\paragraph{Other metrics of feedback quality.} We considered two other metrics of feedback quality. 
First, we attempted to consider the increase in probability over either the ground-truth solution or the full question, comparing the \cm 's solution with and without feedback. 
However, we found that this measure was too noisy to impart any meaningful value. 

We also attempted to prompt \gpt{} to classify feedback relevance (to either the ground-truth solution or full question) on a scale of 1-5. 
However, this measure was also noisy, not to mention hard to define in the prompt, as even humans would struggle to distinguish between, for example, a "2" vs. a "3" in relevance.








\subsection{Performance Tables for Static vs. Interactive Settings}
\label{app:full_results}

\begin{table*}
\centering
\resizebox{\textwidth}{!}{
\begin{tabular}{ccccccc}
\toprule 
\textbf{Model}  & \textbf{\vanilla} & \textbf{\baseline} & \textbf{\sent} & \textbf{\para} & \textbf{\cf} & \textbf{\ir} \\ \hline
\gpt & 0.498 (0.016) & 0.068 (0.007) & 0.422 (0.016) & 0.544 (0.016) & \textbf{0.598 (0.016) }& 0.488 (0.016)\\
\aya &\textbf{ 0.261 (0.014)} & 0.009 (0.002) & 0.131 (0.011) & 0.259 (0.015) & 0.235 (0.014) & 0.214 (0.013)\\
\deepseek & \textbf{0.616 (0.015)} & 0.048 (0.007) & 0.449 (0.016) & 0.512 (0.023) & 0.521 (0.026) & 0.442 (0.019)\\
\gemmaLarge & 0.409 (0.013) & 0.007 (0.002) & 0.351 (0.016) & \textbf{0.556 (0.016)} & 0.51 (0.017) & 0.403 (0.015)\\
\gemmaSmall & 0.177 (0.01) & 0.009 (0.002) & 0.039 (0.006) & 0.084 (0.009) &\textbf{ 0.299 (0.016)} & 0.029 (0.004)\\
\llama & 0.253 (0.012) & 0.025 (0.003) & 0.236 (0.014) & 0.402 (0.015) & \textbf{0.453 (0.016)} & 0.215 (0.012)\\
\qwenSmall & 0.369 (0.014) & 0.026 (0.004) & 0.283 (0.014) & 0.423 (0.016) & \textbf{0.495 (0.016)} & 0.336 (0.015)\\
\qwenLarge & 0.542 (0.015) & 0.039 (0.004) & 0.5 (0.016) & 0.582 (0.015) & \textbf{0.605 (0.015)} & 0.503 (0.015)\\
\reka & 0.164 (0.011) & 0.018 (0.003) & 0.224 (0.013) & 0.303 (0.015) & \textbf{0.4 (0.016)} & 0.157 (0.011)\\
\sonnet & 0.59 (0.014) & 0.114 (0.009) & 0.571 (0.015) & \textbf{0.654 (0.014)} & 0.627 (0.015) & 0.62 (0.014)\\   
\bottomrule
\end{tabular}
}
\caption{Average TCA of each model with standard error on APPS Interview questions. 
% \cf{} and \para{} feedback tends to improve performance the most, raising TCA beyond \vanilla{} performance in some models.
}
\label{tables:apps-interview-full}
\end{table*}

\begin{table*}
\centering
\resizebox{\textwidth}{!}{
\begin{tabular}{ccccccc}
\toprule 
\textbf{Model}  & \textbf{\vanilla} & \textbf{\baseline} & \textbf{\sent} & \textbf{\para} & \textbf{\cf} & \textbf{\ir} \\ \hline
\gpt & 0.412 (0.015) & 0.071 (0.007) & 0.392 (0.016) & 0.512 (0.016) & \textbf{0.539 (0.016)} & 0.409 (0.016)\\
\aya & 0.182 (0.008) & 0.004 (0.001) & 0.069 (0.006) & \textbf{0.179 (0.009)} & 0.177 (0.009) & 0.142 (0.007)\\
\deepseek{} & - & - & - & - & - & - \\
\gemmaLarge & 0.322 (0.012) & 0.018 (0.003) & 0.228 (0.013) & \textbf{0.454 (0.016)} & 0.41 (0.016) & 0.299 (0.013)\\
\gemmaSmall & 0.131 (0.008) & 0.013 (0.003) & 0.023 (0.004) & 0.045 (0.006) & \textbf{0.213 (0.014)} & 0.014 (0.003)\\
\llama & 0.205 (0.01) & 0.028 (0.004) & 0.12 (0.009) & 0.292 (0.014) & \textbf{0.402 (0.015)} & 0.153 (0.01)\\
\qwenSmall & 0.28 (0.012) & 0.034 (0.004) & 0.186 (0.012) & 0.27 (0.014) & \textbf{0.392 (0.016)} & 0.2 (0.011)\\
\qwenLarge & 0.348 (0.024) & 0.038 (0.007) & 0.376 (0.023) & 0.479 (0.025) & \textbf{0.486 (0.025) }& 0.37 (0.023)\\
\reka & 0.124 (0.009) & 0.018 (0.003) & 0.141 (0.01) & 0.275 (0.014) & \textbf{0.34 (0.015)} & 0.11 (0.009)\\
\sonnet & 0.497 (0.014) & 0.073 (0.007) & 0.468 (0.015) & \textbf{0.556 (0.015)} & 0.53 (0.015) & 0.492 (0.015)\\
\bottomrule
\end{tabular}
}
\caption{Average TCA of each model with standard error in each setting for APPS introductory. 
% \cf{} and \para{} provide the highest performance boost, improving TCA beyond \vanilla{} in some models. 
\deepseek{} is missing for this setting due to rate limits on the API that impeded evaluation. }
\label{tables:apps-intro-full}
\end{table*}

\begin{table*}
\centering
\resizebox{\textwidth}{!}{
\begin{tabular}{ccccccc}
\toprule 
\textbf{Model}  & \textbf{\vanilla} & \textbf{\baseline }& \textbf{\sent} & \textbf{\para }& \textbf{\cf} & \textbf{\ir} \\ \hline
\gpt & \textbf{0.8 (0.023)} & 0.323 (0.026) & 0.767 (0.024) & 0.745 (0.024) & 0.702 (0.026) & 0.23 (0.025)\\
\aya & 0.522 (0.027) & 0.235 (0.022) & 0.482 (0.027) & \textbf{0.632 (0.026)} & 0.483 (0.028) & 0.134 (0.02)\\
\deepseek & \textbf{0.944 (0.013)} & 0.332 (0.026) & 0.841 (0.02) & 0.849 (0.019) & 0.756 (0.024) & 0.345 (0.028)\\
\gemmaLarge & \textbf{0.766 (0.021)} & 0.282 (0.025) & 0.67 (0.026) & \textbf{0.766 (0.023)} & 0.62 (0.026) & 0.119 (0.019)\\
\gemmaSmall & 0.347 (0.023) & 0.173 (0.019) & 0.196 (0.021) & 0.285 (0.024) & \textbf{0.524 (0.027)} & 0.028 (0.008)\\
\llama & 0.587 (0.025) & 0.237 (0.022) & 0.538 (0.027) & 0.588 (0.026) & \textbf{0.647 (0.026) }& 0.203 (0.023)\\
\qwenSmall & \textbf{0.755 (0.021)} & 0.27 (0.024) & 0.621 (0.027) & 0.678 (0.026) & 0.629 (0.027) & 0.189 (0.023)\\
\qwenLarge &\textbf{ 0.961 (0.007)} & 0.309 (0.025) & 0.809 (0.021) & 0.747 (0.024) & 0.712 (0.025) & 0.237 (0.025)\\
\reka & 0.617 (0.025) & 0.246 (0.023) & 0.583 (0.026) & \textbf{0.636 (0.026)} & 0.629 (0.027) & 0.111 (0.018)\\
\sonnet & \textbf{0.937 (0.012)} & 0.387 (0.026) & 0.832 (0.02) & 0.821 (0.02) & 0.735 (0.025) & 0.395 (0.029)\\   
\bottomrule
\end{tabular}
}
\caption{Average TCA of each model with standard error on a subset LiveCodeBench questions. To provide code solutions to the \user{} model, we select questions which \sonnet{} solves perfectly within two attempts, using the generated solution as ground truth. 
% In the models where the \vanilla{} setting yields the highest TCA with \para{} and \cf{} providing the largest gains in the iterative refinement loop. The gap between \sent{} and \para{} feedback is closer compared to APPS with \sent{} outperforming \para{} in a few models. 
}
\label{tables:livecodebench-full}
\end{table*}

\begin{table*}
\centering
\resizebox{\textwidth}{!}{
\begin{tabular}{ccccccc}
\toprule 
\textbf{Model}  & \textbf{\vanilla} & \textbf{\baseline} & \textbf{\sent} & \textbf{\para} & \textbf{\cf} & \textbf{\ir} \\ \hline
\gpt & 0.839 (0.005) & 0.561 (0.018) & 0.836 (0.014) & 0.848 (0.014) & \textbf{0.895 (0.011)} & 0.789 (0.014) \\
\aya & 0.718 (0.007)  & 0.305 (0.019) & 0.596 (0.021)  & 0.597 (0.022) &\textbf{0.781 (0.018)} & 0.675 (0.016)  \\
\deepseek & 0.849 (0.005) & 0.559 (0.018) & 0.837 (0.013) & 0.867 (0.012) & \textbf{0.889 (0.012)} & 0.806 (0.013) \\
\gemmaLarge & 0.704 (0.008) & 0.563 (0.018) & 0.776 (0.015) & 0.815 (0.015) & \textbf{0.867 (0.014)} & 0.735 (0.015)  \\
\gemmaSmall & 0.350 (0.008) & 0.303 (0.017) & 0.375 (0.018) & 0.390 (0.018) &\textbf{ 0.687 (0.019) } & 0.340 (0.017)  \\
\llama & 0.636 (0.007) & 0.443 (0.019) & 0.653 (0.019) & 0.710 (0.019) &\textbf{ 0.773 (0.017) }& 0.701 (0.018)  \\
\qwenSmall & 0.714 (0.006) & 0.502 (0.019) & 0.605 (0.018) & 0.757 (0.017) & \textbf{0.819 (0.014)} & 0.697 (0.016) \\
\qwenLarge & 0.816 (0.006) & 0.542 (0.018) & 0.815 (0.014) & 0.832 (0.015) & \textbf{0.876 (0.012)} & 0.778 (0.014) \\
\reka & 0.670 (0.007) & 0.471 (0.027) & 0.096 (0.015) & 0.109 (0.016) & 0.112 (0.016) & \textbf{0.600 (0.019)} \\
\sonnet & 0.833 (0.006) & 0.564 (0.019) & 0.821 (0.014) & 0.865 (0.013) & \textbf{0.881 (0.013)}  & 0.803 (0.014) \\

% \gpt & \textbf{0.842 (0.014) }& 0.687 (0.021) & 0.682 (0.021) & 0.697 (0.021) & 0.691 (0.022) & 0.643 (0.019)\\
% \aya & \textbf{0.696 (0.016)} & 0.515 (0.017) & 0.523 (0.017) & 0.521 (0.017) & 0.524 (0.017) & 0.518 (0.017)\\
% \gemmaLarge & \textbf{0.702 (0.021)} & 0.588 (0.023) & 0.556 (0.02) & 0.595 (0.023) & 0.592 (0.023) & 0.581 (0.019)\\
% \gemmaSmall & \textbf{0.339 (0.018)} & 0.29 (0.015) & 0.291 (0.015) & 0.278 (0.015) & 0.301 (0.015) & 0.287 (0.015)\\
% \llama & \textbf{0.604 (0.018)} & 0.489 (0.02) & 0.518 (0.017) & 0.515 (0.017) & 0.505 (0.017) & 0.521 (0.017)\\
% \qwenSmall & \textbf{0.691 (0.016)} & 0.551 (0.02) & 0.562 (0.018) & 0.534 (0.017) & 0.546 (0.017) & 0.554 (0.016)\\
% \qwenLarge & \textbf{0.807 (0.016)} & 0.67 (0.023) & 0.622 (0.019) & 0.676 (0.022) & 0.675 (0.023) & 0.632 (0.019)\\
% \reka & \textbf{0.648 (0.016)} & 0.477 (0.029) & 0.484 (0.016) & 0.49 (0.016) & 0.494 (0.016) & 0.501 (0.016)\\
% \sonnet &\textbf{ 0.835 (0.015)} & 0.695 (0.022) & 0.702 (0.022) & 0.698 (0.023) & 0.698 (0.022) & 0.66 (0.019)\\
\bottomrule
\end{tabular}
}
\caption{Average TCA with standard error in each setting for ClassEval.}
\label{tables:classeval-full}
\end{table*}
In this section, we provide tables for the performance of models across static and interactive settings, including all feedback types and baselines.
Table \ref{tables:apps-interview-full} gives the TCA of APPS (Interview), Table \ref{tables:apps-intro-full} gives the TCA of APPS (Introductory). 
Table \ref{tables:livecodebench-full} gives the TCA of LiveCodeBench and Table \ref{tables:classeval-full}.

\subsection{Additional Tables}
All additional tables -- including information about feedback quality by dataset, steerability metrics, and ranking distance metrics --- can be found in this section. 
\label{app:average_feedback_quality_per_step}

Table \ref{tables:apps-avg-feedback-quality}, Table \ref{tables:livecodebench-avg-feedback-quality}, Table \ref{tables:classeval-avg-feedback-quality} have the average directional correctness of each setting and the number of steps it takes to reach a solution with 100\% TCA. We partition the analysis by model and by feedback setting.

Table \ref{tables:steerability-edit-distance} measures the average number of edits made by each model for each feedback. 
Table \ref{tables:steerability-test-case} measures the average number of test cases flipped by each feedback setting.

Tables \ref{tables:normalized-spearmans-footrule-dist} gives the normalized Spearman's Footrule distance of each setting's ranking compared to the \vanilla{} setting.


\begin{table*}
\centering
\resizebox{\textwidth}{!}{
\begin{tabular}{cccccccc}
\toprule 
\textbf{Model} & \textbf{Feedback Type} & \textbf{Average Steps to Correct Solution} & \textbf{Average Directional Correctness} \\
\hline
 & Code Feedback & 2.981 & 0.937  \\
\gpt & Paragraph & 3.003 & 0.896  \\
 & Sentence & 3.436 & 0.850  \\
\hline
 & Code Feedback & 3.586 & 0.928  \\
\aya & Paragraph & 3.687 & 0.926  \\
 & Sentence & 3.886 & 0.901  \\
\hline
 & Code Feedback & 3.207 & 0.928  \\
\deepseek & Paragraph & 3.132 & 0.886  \\
 & Sentence & 3.484 & 0.873  \\
\hline
 & Code Feedback & 3.296 & 0.948  \\
\gemmaLarge & Paragraph & 3.259 & 0.920  \\
 & Sentence & 3.630 & 0.895  \\
\hline
 & Code Feedback & 3.661 & 0.978  \\
\gemmaSmall & Paragraph & 3.947 & 0.982  \\
 & Sentence & 3.977 & 0.950  \\
\hline
 & Code Feedback & 3.349 & 0.963  \\
\llama & Paragraph & 3.567 & 0.923  \\
 & Sentence & 3.850 & 0.897  \\
\hline
 & Code Feedback & 3.328 & 0.948  \\
\qwenSmall & Paragraph & 3.453 & 0.909  \\
 & Sentence & 3.724 & 0.899  \\
\hline
 & Code Feedback & 3.082 & 0.952  \\
\qwenLarge & Paragraph & 3.058 & 0.900  \\
 & Sentence & 3.411 & 0.876  \\
\hline
 & Code Feedback & 3.520 & 0.935  \\
\reka & Paragraph & 3.634 & 0.927  \\
 & Sentence & 3.854 & 0.879  \\
\hline
 & Code Feedback & 3.035 & 0.914  \\
\sonnet & Paragraph & 2.882 & 0.865  \\
 & Sentence & 3.299 & 0.834  \\
\bottomrule
\end{tabular}
}
\caption{Average directional correctness of feedback and the average number of steps required to reach 100\% TCA on the APPS dataset.}
\label{tables:apps-avg-feedback-quality}
\end{table*}

\begin{table*}
\centering
\resizebox{\textwidth}{!}{
\begin{tabular}{cccccccc}
\toprule 
\textbf{Model} & \textbf{Feedback Type} & \textbf{Average Steps to Correct Solution} & \textbf{Average Directional Correctness} \\
\hline
 & Code Feedback & 2.697 & 0.736  \\
\gpt & Paragraph & 2.337 & 0.902  \\
 & Sentence & 2.416 & 0.821  \\
\hline
 & Code Feedback & 2.856 & 0.852  \\
\aya & Paragraph & 2.553 & 0.945  \\
 & Sentence & 3.097 & 0.898  \\
\hline
 & Code Feedback & 2.487 & 0.785  \\
\deepseek & Paragraph & 1.808 & 0.891  \\
 & Sentence & 2.063 & 0.806  \\
\hline
 & Code Feedback & 3.089 & 0.809  \\
\gemmaLarge & Paragraph & 2.246 & 0.912  \\
 & Sentence & 2.685 & 0.896  \\
\hline
 & Code Feedback & 3.229 & 0.927  \\
\gemmaSmall & Paragraph & 3.640 & 0.986  \\
 & Sentence & 3.818 & 0.963  \\
\hline
 & Code Feedback & 2.773 & 0.888  \\
\llama & Paragraph & 2.672 & 0.957  \\
 & Sentence & 3.169 & 0.890  \\
\hline
 & Code Feedback & 2.876 & 0.786  \\
\qwenSmall & Paragraph & 2.314 & 0.958  \\
 & Sentence & 2.847 & 0.909  \\
\hline
 & Code Feedback & 2.811 & 0.771  \\
\qwenLarge & Paragraph & 1.949 & 0.871  \\
 & Sentence & 2.219 & 0.817  \\
\hline
 & Code Feedback & 2.801 & 0.668  \\
\reka & Paragraph & 2.575 & 0.883  \\
 & Sentence & 2.871 & 0.807  \\
\hline
 & Code Feedback & 2.663 & 0.701  \\
\sonnet & Paragraph & 2.017 & 0.885  \\
 & Sentence & 2.247 & 0.859  \\
\bottomrule
\end{tabular}
}
\caption{Average directional correctness of feedback and the average number of steps required to reach 100\% TCA on the LiveCodeBench dataset.}
\label{tables:livecodebench-avg-feedback-quality}
\end{table*}


\begin{table*}
\centering
\resizebox{\textwidth}{!}{
\begin{tabular}{cccccccc}
\toprule 
\textbf{Model} & \textbf{Feedback Type} & \textbf{Average Steps to Correct Solution} & \textbf{Average Directional Correctness} \\
\hline
%  & Code Feedback & 3.436 & 0.743  \\
% \gpt & Paragraph & 3.546 & 0.963  \\
%  & Sentence & 3.153 & 0.899  \\
% \hline
%  & Code Feedback & 3.977 & 0.749  \\
% \aya & Paragraph & 3.987 & 0.953  \\
%  & Sentence & 3.840 & 0.898  \\
% \hline
%  & Code Feedback & 4.000 & 0.698  \\
% \deepseek & Paragraph & 4.000 & 0.969  \\
%  & Sentence & 4.000 & 0.910  \\
% \hline
%  & Code Feedback & 3.076 & 0.789  \\
% \gemmaLarge & Paragraph & 3.334 & 0.977  \\
%  & Sentence & 3.661 & 0.917  \\
% \hline
%  & Code Feedback & 3.880 & 0.902  \\
% \gemmaSmall & Paragraph & 3.874 & 0.979  \\
%  & Sentence & 3.940 & 0.944  \\
% \hline
%  & Code Feedback & 3.389 & 0.872  \\
% \llama & Paragraph & 3.806 & 0.959  \\
%  & Sentence & 3.699 & 0.899  \\
% \hline
%  & Code Feedback & 3.684 & 0.782  \\
% \qwenSmall & Paragraph & 3.724 & 0.967  \\
%  & Sentence & 3.322 & 0.929  \\
% \hline
%  & Code Feedback & 3.536 & 0.744  \\
% \qwenLarge & Paragraph & 4.000 & 0.977  \\
%  & Sentence & 3.585 & 0.889  \\
% \hline
%  & Code Feedback & 3.978 & 0.769  \\
% \reka & Paragraph & 4.000 & 0.942  \\
%  & Sentence & 4.000 & 0.901  \\
% \hline
%  & Code Feedback & 3.664 & 0.659  \\
% \sonnet & Paragraph & 3.675 & 0.947  \\
%  & Sentence & 4.000 & 0.913  \\

 & Code Feedback & 3.436 & 0.743  \\
\gpt & Paragraph & 3.546 & 0.963  \\
 & Sentence & 3.153 & 0.899  \\
\hline
 & Code Feedback & 3.977 & 0.749  \\
\aya & Paragraph & 3.987 & 0.953  \\
 & Sentence & 3.840 & 0.898  \\
\hline
 & Code Feedback & 4.000 & 0.698  \\
\deepseek & Paragraph & 4.000 & 0.969  \\
 & Sentence & 4.000 & 0.910  \\
\hline
 & Code Feedback & 3.076 & 0.789  \\
\gemmaLarge & Paragraph & 3.334 & 0.977  \\
 & Sentence & 3.661 & 0.917  \\
\hline
 & Code Feedback & 3.880 & 0.902  \\
\gemmaSmall & Paragraph & 3.874 & 0.979  \\
 & Sentence & 3.940 & 0.944  \\
\hline
 & Code Feedback & 3.389 & 0.872  \\
\llama & Paragraph & 3.806 & 0.959  \\
 & Sentence & 3.699 & 0.899  \\
\hline
 & Code Feedback & 3.684 & 0.782  \\
\qwenSmall & Paragraph & 3.724 & 0.967  \\
 & Sentence & 3.322 & 0.929  \\
\hline
 & Code Feedback & 3.536 & 0.744  \\
\qwenLarge & Paragraph & 4.000 & 0.977  \\
 & Sentence & 3.585 & 0.889  \\
\hline
 & Code Feedback & 3.978 & 0.769  \\
\reka & Paragraph & 4.000 & 0.942  \\
 & Sentence & 4.000 & 0.901  \\
\hline
 & Code Feedback & 3.664 & 0.659  \\
\sonnet & Paragraph & 3.675 & 0.947  \\
 & Sentence & 4.000 & 0.913  \\
\bottomrule
\end{tabular}
}
\caption{Average directional correctness of feedback and the average number of steps required to reach 100\% TCA on the ClassEval dataset}
\label{tables:classeval-avg-feedback-quality}
\end{table*}



\begin{table*}
\centering
\resizebox{\textwidth}{!}{
\begin{tabular}{cccccccc}
\toprule 
\textbf{Model} & \textbf{Feedback Type} & \textbf{Average Steps to Correct Solution} & \textbf{Average Directional Correctness} \\
\hline
%  & Code Feedback & 3.012 & 0.890  \\
% \gpt & Paragraph & 3.042 & 0.912  \\
%  & Sentence & 3.299 & 0.856  \\
% \hline
%  & Code Feedback & 3.577 & 0.886  \\
% \aya & Paragraph & 3.618 & 0.933  \\
%  & Sentence & 3.794 & 0.900  \\
% \hline
%  & Code Feedback & 3.288 & 0.800  \\
% \deepseek & Paragraph & 3.092 & 0.925  \\
%  & Sentence & 3.372 & 0.878  \\
% \hline
%  & Code Feedback & 3.235 & 0.907  \\
% \gemmaLarge & Paragraph & 3.165 & 0.931  \\
%  & Sentence & 3.536 & 0.899  \\
% \hline
%  & Code Feedback & 3.648 & 0.958  \\
% \gemmaSmall & Paragraph & 3.898 & 0.982  \\
%  & Sentence & 3.952 & 0.950  \\
% \hline
%  & Code Feedback & 3.290 & 0.940  \\
% \llama & Paragraph & 3.509 & 0.932  \\
%  & Sentence & 3.747 & 0.897  \\
% \hline
%  & Code Feedback & 3.341 & 0.902  \\
% \qwenSmall & Paragraph & 3.380 & 0.923  \\
%  & Sentence & 3.574 & 0.904  \\
% \hline
%  & Code Feedback & 3.138 & 0.894  \\
% \qwenLarge & Paragraph & 3.105 & 0.916  \\
%  & Sentence & 3.335 & 0.876  \\
% \hline
%  & Code Feedback & 3.518 & 0.878  \\
% \reka & Paragraph & 3.575 & 0.927  \\
%  & Sentence & 3.762 & 0.876  \\
% \hline
%  & Code Feedback & 3.119 & 0.842  \\
% \sonnet & Paragraph & 2.956 & 0.887  \\
%  & Sentence & 3.284 & 0.848  \\

 & Code Feedback & 2.925 & 0.904  \\
\gpt & Paragraph & 2.916 & 0.906  \\
 & Sentence & 3.293 & 0.858  \\
\hline
 & Code Feedback & 3.400 & 0.918  \\
\aya & Paragraph & 3.516 & 0.934  \\
 & Sentence & 3.738 & 0.905  \\
\hline
 & Code Feedback & 2.814 & 0.831  \\
\deepseek & Paragraph & 2.721 & 0.929  \\
 & Sentence & 3.171 & 0.886  \\
\hline
 & Code Feedback & 3.218 & 0.915  \\
\gemmaLarge & Paragraph & 3.120 & 0.925  \\
 & Sentence & 3.479 & 0.903  \\
\hline
 & Code Feedback & 3.545 & 0.969  \\
\gemmaSmall & Paragraph & 3.888 & 0.982  \\
 & Sentence & 3.936 & 0.952  \\
\hline
 & Code Feedback & 3.250 & 0.945  \\
\llama & Paragraph & 3.427 & 0.930  \\
 & Sentence & 3.708 & 0.900  \\
\hline
 & Code Feedback & 3.237 & 0.921  \\
\qwenSmall & Paragraph & 3.287 & 0.917  \\
 & Sentence & 3.663 & 0.874  \\
\hline
 & Code Feedback & 3.002 & 0.919  \\
\qwenLarge & Paragraph & 2.931 & 0.908  \\
 & Sentence & 3.265 & 0.883  \\
\hline
 & Code Feedback & 3.508 & 0.889  \\
\reka & Paragraph & 3.560 & 0.929  \\
 & Sentence & 3.761 & 0.878  \\
\hline
 & Code Feedback & 2.983 & 0.877  \\
\sonnet & Paragraph & 2.803 & 0.880  \\
 & Sentence & 3.193 & 0.849  \\
\bottomrule
\end{tabular}
}
\caption{Average directional correctness of feedback and the average number of steps required to reach 100\% TCA across all datasets.}
\label{tables:all-avg-feedback-quality}
\end{table*}



\begin{table*}
\centering
\resizebox{\textwidth}{!}{
\begin{tabular}{cccccccc}
\toprule 
\textbf{Model} & \textbf{\sent} & \textbf{\para} & \textbf{\cf} & \textbf{\ir} \\ \hline
\gpt & 265.029 & 308.837 & 451.749 & 325.441 \\
\aya & 333.695 & 274.630 & 503.740 & 351.736 \\
\deepseek & 143.413 & 190.044 & 301.157 & 264.532 \\
\gemmaLarge & 116.412 & 108.401 & 182.592 & 108.516 \\
\gemmaSmall & 286.987 & 211.680 & 299.998 & 261.134 \\
\llama & 361.366 & 400.246 & 511.731 & 385.156 \\
\qwenSmall & 214.421 & 178.882 & 363.474 & 211.171 \\
\qwenLarge & 270.493 & 230.114 & 496.959 & 328.425 \\
\reka & 541.689 & 292.388 & 768.962 & 638.245 \\
\sonnet & 155.923 & 204.138 & 320.164 & 215.693 \\
\bottomrule
\end{tabular}
}
\caption{Surface-level steerability (as measured by edit distance) vs. feedback type across each model.}
\label{tables:steerability-edit-distance}
\end{table*}



\begin{table*}
\centering
\resizebox{\textwidth}{!}{
\begin{tabular}{cccccccc}
\toprule 
\textbf{Model} & \textbf{\sent} & \textbf{\para} & \textbf{\cf} & \textbf{\ir} \\ \hline
\gpt & 0.225 & 0.164 & 0.240 & 0.152 \\
\aya & 0.169 & 0.090 & 0.208 & 0.108 \\
\deepseek & 0.159 & 0.138 & 0.208 & 0.161 \\
\gemmaLarge & 0.149 & 0.113 & 0.199 & 0.095 \\
\gemmaSmall & 0.094 & 0.007 & 0.040 & 0.015 \\
\llama & 0.2 & 0.102 & 0.201 & 0.107 \\
\qwenSmall & 0.157 & 0.095 & 0.184 & 0.101 \\
\qwenLarge & 0.228 & 0.142 & 0.263 & 0.192 \\
\reka & 0.154 & 0.059 & 0.168 & 0.095 \\
\sonnet & 0.239 & 0.225 & 0.311 & 0.212 \\
\bottomrule
\end{tabular}
}
\caption{Behavioral-level steerability (as measured by number of test cases changed from correct to incorrect or vice-versa) vs. feedback type across each model.}
\label{tables:steerability-test-case}
\end{table*}

\begin{table*}
\centering
\begin{tabular}{ccc}
\toprule 
\textbf{Dataset} & \textbf{Feedback Type} & \textbf{Normalized Spearman's Footrule Distance} \\ \hline
\multirow{4}{*}{Apps}  & Code Feedback & 0.267  \\
 & Input Refinement & 0.222  \\
 & Paragraph & 0.222  \\
 & Sentence & 0.178  \\
\hline
\multirow{4}{*}{ClassEval}  & Code Feedback & 0.222 \\
 & Input Refinement & 0.267  \\
 & Paragraph & 0.267  \\
 & Sentence & 0.267  \\
\hline
\multirow{4}{*}{LiveCodeBench} & Code Feedback & 0.356  \\
 & Input Refinement & 0.356  \\
 & Paragraph & 0.222  \\
 & Sentence & 0.044 \\

\bottomrule
\end{tabular}
\caption{Normalized Spearman's Footrule Distance when comparing each feedback setting's ranking order to the ranking order on static benchmark.}
\label{tables:normalized-spearmans-footrule-dist}
\end{table*}



