\begin{abstract}
When complex SQL queries suffer slow executions despite query optimization, DBAs typically invoke automated query rewriting tools to recommend ``lean'' equivalents that are conducive to faster execution.
The rewritings are usually achieved via transformation rules, but these rules are limited in scope and difficult to update in a production system. Recently, LLM-based techniques have also been suggested, but they are prone to semantic and syntactic errors.

We investigate here how the remarkable cognitive capabilities of LLMs can be leveraged for performant query rewriting while incorporating safeguards and optimizations to ensure correctness and efficiency. Our study shows that these goals can be progressively achieved through incorporation of (a) an ensemble suite of basic prompts, (b) database-sensitive prompts via redundancy removal and selectivity-based rewriting rules, and (c) LLM token probability-guided rewrite paths. Further, a suite of logic-based and statistical tools can be used to check for semantic violations in the rewrites prior to DBA consideration. 

We have implemented the above LLM-infused techniques in the \lithe system, and evaluated complex analytic queries from standard benchmarks on contemporary database platforms. The results show significant performance improvements for slow queries, with regard to both abstract costing and actual execution, over both SOTA techniques and the native query optimizer. For instance, with TPC-DS on \pg, the geometric mean of the runtime speedups for slow queries was as high as \textbf{18.4} over the native optimizer, whereas SOTA delivered \textbf{6} in comparison.

Overall, \lithe is a promising step toward viable LLM-based advisory tools for ameliorating enterprise query performance.
\end{abstract}