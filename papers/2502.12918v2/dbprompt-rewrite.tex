\section{Database-Sensitive Prompts}
\label{sec:dbms-proficient-prompts}
As discussed above, basic prompting needs to be improved on two fronts: (1) Ensuring productive rewrites where feasible; and (2) Maximizing the impact of these productive rewrites.
To address these issues, we incorporate database domain knowledge. Specifically, we design a \textit{one shot}-based prompting template, augmented with a set of database-aware rewrite rules.
%
The rules are based on common practices followed by DBAs that are widely applicable, 
and augmented with precise instructions and useful examples to help guide the LLM in the rewriting process. 

As a proof of concept, we explore two categories of rewrites here: (a) Rules that eliminate redundancy in the input queries; and (b) Predicate selectivity-based rules that implicitly guide, via query space reformulations, the query optimizer towards efficient query execution plans.
Of course, 
this basic set of rules can be expanded further, but as shown by our experiments, even this minimal set is capable of delivering substantive improvements over a broad set of database environments.


\subsection{Redundancy Removal}
\label{sec:rules-prompt}

There are different types of redundancy that can occur in a SQL query -- repeated computations, superfluous filter predicates, unnecessary joins, etc. Rules {\bf R1 through R4} in Table~\ref{tab:llm-rules} are designed to tackle such redundancies. 
The relevant schematic information (e.g. table names, column names, constraints) required by these rules is also provided in the prompt.

\begin{table}[!h]
\footnotesize
\centering
\caption{Rules for Database-sensitive prompts.}
\label{tab:llm-rules}
\begin{tabular}{|p{0.3cm}|p{12cm}|}
\hline
& \textbf{Redundancy Removal Rules}  \\ \hline
R1 & Use CTEs (Common Table Expressions) to avoid repeated computation of a given expression.\\ \hline
R2 & When multiple subqueries use the same base table, rewrite to scan the base table only once.\\ \hline
R3 & Remove redundant conjunctive filter predicates.\\ \hline
R4 & Remove redundant key (PK-FK) joins.\\ \hline 
& \textbf{Statistics-based Rules} \\ \hline
R5 & Choose EXIST or IN from subquery selectivity (high/low).\\ \hline
R6 & Pre-filter tables involved in self-joins and with low selectivities on their filter and/or join predicates. Remove any redundant filters from the main query. Do not create explicit join statements.\\ \hline
\end{tabular}
\vspace{-0.1in}
\end{table}

\begin{figure}[t]
    \centering
    \includegraphics[width=0.5\linewidth]{Figures/rule-prompt.png}
    \vspace{-0.1cm}
    \caption{Templates for Database-sensitive Prompts}
    \label{fig:rule-prompts}
    \vspace{-0.1cm}
\end{figure}

The template for such rule-based prompts is shown in Figure~\ref{fig:rule-prompts}(a) and includes an example to demonstrate the rule application to the LLM -- the specific examples used with our rules are available in the Section~\ref{app:rules-ex}.
%
Note that this prompt template allows for only a \emph{single} rule to be present in the prompt. This was a conscious design choice because LLMs are often overwhelmed by excessive information given in monolithic form. Therefore, we apply each rule using a separate prompt,  finally returning the rewrite providing the best performance improvement.

\subsubsection*{Performance}
The performance improvement achieved on the micro-benchmark by an ensemble that adds the redundancy-removing prompts to the basic set (Section~\ref{sec:basic-prompt}) is shown in Table~\ref{tab:rule-exp}. We observe that the \cpr increases to \textbf{\RedRewriteMicroDS}, and \csgm grows to \textbf{\gmRedMicroDS}.

\begin{table}[!h]
\footnotesize
\centering
\caption{Performance with Redundancy Removal Rules.}
\label{tab:rule-exp}
\begin{tabular}{|c|c|c|}
\hline
\textbf{Prompt} & \textbf{\# \cpr} & \textbf{\csgm} \\ 
 % & \textbf{Rewrites} & \\ 
\hline \hline
Basic Prompts $\bigcup$ \{R1, $\ldots$, R4\} & \RedRewriteMicroDS & \gmRedMicroDS  \\ \hline
\end{tabular}
\vspace{-0.1in}
\end{table}

A natural question here would be whether, while retaining the one-rule-per-prompt design, the rules could be \emph{progressively} applied with the output of one prompt provided as input to the next, and so on. This approach would benefit queries with multiple types of redundancies. However, it also increases rewrite overheads due to repeated prompt applications. So, for simplicity, we have chosen to process them individually rather than cumulatively.

\begin{figure}[t]
    \centering
    \includegraphics[width=0.45\linewidth]{Figures/metadata-example.png}
    \vspace{-0.1cm}
    \caption{Example Queries illustrating Rule~5}
    \label{fig:metadata-example}
    \vspace{-0.1cm}
\end{figure}

\subsection{Selectivity-based Guidance}
\label{sec:stats-prompt}

We now turn our attention to rules whose applicability to a query is conditional on the specific database environment, specifically its statistical aspects. 
%
For example, consider the alternative rewrites shown in Figure~\ref{fig:metadata-example} using the EXIST and IN clauses (highlighted in red), respectively -- here the appropriate choice is dictated by the \emph{selectivity} of the inner subquery -- EXISTS for high selectivity values and IN for low values. Rule~\textbf{R5} basically encodes this argument as a rule in Table~\ref{tab:llm-rules}. Similarly rule~\textbf{R6}, which pre-filters tables that are involved in self-joins and have low selectivity filter and/or join predicates, is also used to guide a rewrite.
%
Note that a specific instruction to not create explicit joins had to be added in rule R6. This is because in the presence of CTEs, the LLM is prone to schematic confusion regarding which attribute belongs to which table, leading it to construct invalid joins.

The input prompt for these rules, as shown in Figure~\ref{fig:rule-prompts}(b), is modified to include the following:
\begin{enumerate}
\item Selectivities of columns appearing in \texttt{WHERE} and \texttt{JOIN} clauses, as obtained from the database engine.
\item Clause rewrite rules and instructions based on statistics.
\item Examples relevant to the chosen rewrite rules.
\end{enumerate}
%

\subsubsection*{Performance}
The performance improvements following addition of selectivity-guided prompts are shown in Table~\ref{tab:metadata-exp}. We observe that \cprs are now obtained for \textbf{\StatsRewriteMicroDS} of the ten micro-benchmark queries. Moreover, the resulting \csgm increases to \textbf{\gmStatsMicroDS}, quite close to the human target of \gmGodMicroDS. 

\vspace{-0.1cm}
\begin{table}[!h]
\footnotesize
\centering
\caption{Performance of Metadata-infused Prompts.}
\label{tab:metadata-exp}
\vspace{-0.1cm}
\begin{tabular}{|c|c|c|}
\hline
\textbf{Prompt}& \textbf{\# \cpr} & \textbf{\csgm}  \\ 
\hline \hline
Basic Prompts $\bigcup$ \{R1, $\ldots$, R6\} & \StatsRewriteMicroDS & \gmStatsMicroDS \\ \hline
\end{tabular}
\vspace{-0.1cm}
\end{table}

We note in closing that rules R1 through R6 not only add queries to the productive category, but also deliver greater improvement for those already deemed to be productive via the standard prompts of Section~\ref{sec:basic-prompt}.
Further, to minimize selection overheads, a classifier could be designed to choose the appropriate rule -- this option is examined in the Section~\ref{app:classifier}.



