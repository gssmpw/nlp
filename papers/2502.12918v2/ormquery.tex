\begin{figure}[!h]
\centering
{\small
    \centering
    \begin{tabular}{|c|}
        \hline
        \renewcommand{\baselinestretch}{0.9}\selectfont
\begin{lstlisting}
SELECT t.Key, sum(t.Rating) AS PostRating, 
   (SELECT sum(b0.Rating)
    FROM (SELECT p0.PostId, p0.BlogId, p0.Content, 
          p0.CreatedDate, p0.Rating, p0.Title, 
          b1.BlogId AS BlogId0, b1.Rating AS Rating0,
          b1.Url, p0.day AS Key
          FROM Posts AS p0 INNER JOIN Blogs AS b1  
              ON p0.BlogId = b1.BlogId
          WHERE b1.Rating > 5) AS t0
    INNER JOIN Blogs AS b0 ON t0.BlogId = b0.BlogId
    WHERE t.Key = t0.Key ) AS BlogRating
FROM (SELECT p.Rating, p.day AS Key
      FROM Posts AS p INNER JOIN Blogs AS b
         ON p.BlogId = b.BlogId
      WHERE b.Rating > 5) AS t
GROUP BY t.Key;
        \end{lstlisting}\renewcommand{\baselinestretch}{1.0}\selectfont \\
        \hline
\end{tabular}
}
\caption{Complex SQL Representation}\vspace*{0.1in}
\label{fig:BloatedQuery}
{\small
    \centering
    \begin{tabular}{|c|}
        \hline
        \renewcommand{\baselinestretch}{0.9}\selectfont
        \begin{lstlisting}
SELECT   p.day AS Key, SUM(p.Rating) AS PostRating, 
         SUM(b.Rating) AS BlogRating 
FROM   Posts AS p   INNER JOIN Blogs AS b
        ON p.BlogId = b.BlogId 
WHERE b.Rating > 5 
GROUP BY p.day;
        \end{lstlisting} \renewcommand{\baselinestretch}{1.0}\selectfont\\
        \hline
\end{tabular}
}
\caption{Lean Equivalent Query}
\label{fig:LeanQuery}
\end{figure}
