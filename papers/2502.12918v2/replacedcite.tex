\section{Related Work}
\label{sec:Related-Work}

\myparagraph{Rule-based SQL rewriting.}
%
Most of the recent work on SQL query rewriting is rule-based____.
%
For instance, WeTune____ employs a rule generator to enumerate a set (up to a maximum size) of logically valid plans for a given query to create new rewrite rules, and uses an SMT solver to prove the correctness of the generated rules. While this approach can generate a large set of new rewrite rules, it often fails in coming up with transformation rules for complex queries due to the computational overheads of verifying rule correctness. As such, it is unable to rewrite any of the TPC-DS queries____.

%
Learned Rewrite____ uses a set of existing rewrite rules from Calcite____ and aims to learn the optimal subset of rules along with the order in which they must be applied. Given that the search space for possible rewrites is exponential in the number of applicable rules,
an MCTS scheme is used to efficiently navigate this space and find the rewritten query with maximum cost reduction. 

%
\llmrsq____ is also rule-based but takes a different approach to identify the order for rewrite rule applications: it uses an LLM to find the best Calcite rules and the order in which to apply them to improve the query performance. R-Bot____ also leverages an LLM to optimize the order of Calcite rules, but employs advanced contemporary techniques such as retrieval-augmented generation~(RAG) and step-by-step self-reflection to improve the outcomes. 

Query Booster____ implements human-centered rewriting -- it provides an interface to specify rules using an expressive rule language, which it generalizes to create rewrite rules to be applied on the query. 
%
There are also rule-based rewrite approaches designed for specific types of rewrites such as optimizing correlated window aggregations____ and common expression elimination____. 

All of the above approaches operate via the query \emph{plan space}, which can restrict the kind of rewrites that can be accomplished. Whereas, \lithe uses a small set of general rewrite rules that work directly in the \emph{query space}.

\myparagraph{LLM-based rewriting.}
GenRewrite____ is the first LLM-based approach that tries to use the LLM itself for end-to-end query rewriting.
Instead of using predefined rules from Calcite____, they employ the LLM to create Natural Language Rewrite Rules (NLR2s) to be used as hints, and perform several iterations of prompting to get the rewritten query. 
They show that LLMs can outperform rule-based approaches due to their ability to understand contexts, and demonstrate a significant improvement in query rewriting compared to prior methods. 

A limitation, however, is that LLM-generated rewrite rules often fail to generalize beyond specific query pairs. Even when generalized rules are present,  LLMs can struggle to apply rules correctly if not provided with accompanying examples. Finally, it must be noted that LLMs are unaware of the underlying database which restricts their ability to produce efficient metadata-aware rules. 

\myparagraph{LLMs for Database Modules.}
LLM technologies have been advocated for a variety of database modules. For instance, they have been extensively used for Text-to-SQL transformations____.
The main focus of these techniques is to correctly ascertain the information necessary to formulate the SQL query____.
On the other hand, the goal of S2S rewriting is on improving the performance of an existing SQL query.  Thus, unlike Text-to-SQL transformations where the input text is inherently ambiguous, SQL queries are precisely defined, and therefore
equivalence to a precise ground-truth has to be provably maintained. 

In recent times, LLMs have also been considered for plan-hinting____, join-order optimization____, index selection____, data pipelines____, data management____, etc.  An LLM-enabled multi-modal query optimizer____, where data spans text, image, audio and video domains, has also been proposed recently. These approaches can be used in conjunction with the \lithe system since they address orthogonal segments of the query processing pipeline.