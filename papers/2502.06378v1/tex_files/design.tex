\section{Study Design}
We conducted a user study where participants gained hands-on experience using our app and then completed a questionnaire and interview to explore their attitudes towards augmenting real-world locations with AR content. 
We opted for this approach to ensure participants had hands-on experience with how such a future system might work to better inform their answers.



\subsection{Procedure}
Upon arrival the experiment’s purpose was explained and a consent form and demographic questionnaire given to the participant.
Participants were told they would first gain hands-on experience with an app designed to demonstrate how future AR content could be posted and shared.
After this, participants were told we would investigate their attitudes towards the acceptability of augmenting different real-world locations with AR content using a questionnaire and interview approach. 

The experiment took place in a student computer lab for undergraduate and masters students on a university campus. 
This environment was chosen as a semi-public space, and an environment in which the app was thoroughly tested to work (enabling placement of AR content on the many different types of horizontal and vertical surface within the room). 
Conducting the experiment here also avoided differences in weather and lighting conditions across participants, ensuring a more consistent experience using our app across participants. 
Although, we acknowledge this as a limitation of our work. 

The experiment took approximately 45 minutes to complete, consisting of a: 5-minute introduction (explaining augmented reality generally, the experiment's purpose, and how the application worked), 15-minute session to use the app to complete a set of tasks, and 25-minute session to completed a questionnaire and semi-structured interview. 
During the 15-minute hands-on session, participants were encouraged to think aloud and explain actions they were taking or any thoughts they had about the app generally. 

Although AR content placed during the experiment by a participant could persist across sessions (e.g. across different participants) we reset the application to a baseline augmentation state after each participant completed the experiment. 
This removed the prior participant's AR content but left some placed by the researcher in the environment. 
This was done to ensure a comparable starting experience for all participants in our study. 
This baseline state included three examples of posted AR content placed next to where participants were instructed to stand when starting the study. 



\subsection{Experimental Task}
\label{tasks_outline}
During the hands-on session participants were tasked with completing the following: 

\begin{enumerate}
    \item \textit{Open the app, and locate the three example images hosted in the surrounding environment.} 

    \item \textit{Post an image within the environment and make it persistent: Select an image from device and scan the environment to identify a surface to place the AR content on. Use the placement indicator to place image at desired location. Use the scale and rotation sliders to adjust size and orientation of image as desired. Scan the environment and click the host button.}
    
    \item \textit{Resolve the hosted image and verify its persistence: Close and reopen the app. Locate the image you just hosted.}
    
    \item \textit{Spend the remaining time as you like, posting images wherever you want in the environment, exploring the features and capabilities of the application.}
\end{enumerate}


 
\subsection{Questionnaire Design}
Our questionnaire asked participants to rate the acceptability, on a 5-point Likert scale (1 = Extremely Unacceptable, 5 = Extremely Acceptable), of five types of AR content (\textit{Artistic, Protest, Social, Informative, Commercial}) being augmented in six locations (\textit{Cultural Site, Religious Site, Residential Area, Public Space, Government Building, Tourist Point of Interest}). 
The questionnaire was structured so that participants were presented with a location and then asked to rate the acceptability of posting each of the five types of AR content at this location. 
The location order was counterbalanced using a fully balanced Latin square. 
The order of the AR content types within each location was randomised. 
Participants were instructed to assume the individual posting the AR content did not have ownership over the space being augmented and that the content would be posted on a public, shared, AR view of the location.  

Each location was presented with a representative graphic and short text description to aid participants' understanding of the location (Appendix \ref{app:appendix_A}). 
Each AR content type was presented with a text description. 
This was done to mitigate the potential influence of an example augmentation being the focus of participants' reflection, rather than content type more generally. 
Due to the visual only focus of our app, our text descriptions focused on visual only augmentations as well. 
The text descriptions used to describe each content type follow: 

\begin{itemize}
    \item \textit{\underline{Artistic}}: refers to creative expression, e.g. visuals and artworks aimed at conveying ideas, emotions, or aesthetics.
    
    \item \textit{\underline{Protest}}: refers to any imagery or text created that conveys dissent, criticism, or advocacy for social or political causes.
    
    \item \textit{\underline{Social}}: refers to messages exchanged between friends or images intended for sharing with others, reflecting personal experiences, opinions, etc.
    
    \item \textit{\underline{Informative}}: refers to data or visuals that provide relevant and useful information about the location or surrounding environment.
    
    \item \textit{\underline{Commercial}}: refers to advertisements, promotions, or marketing materials aimed at promoting products, services, or brands for financial gain.
\end{itemize} 

We selected a list of locations that are common in everyday life and have been shown in prior works to be locations individuals are willing to augment with AR content in some capacity \cite{dig-graf-1, intro_6, imx-church-paper}. 
We selected our AR content types based on common uses cases of AR (e.g. social messages to friends \cite{dig-graf-1}, street art \cite{street-art}, protest art \cite{protestAR}, etc) and on common physical alterations made by individuals. 


\subsection{Interview Questions}
After completing the questionnaire, a semi-structured interview was conducted to allow participants to elaborate on the reasoning behind their answers, and to explore their attitudes towards protecting locations from AR content. 
The questions asked to all participants were: 
\begin{itemize}
    \item \textbf{IQ1:} \textit{``If an individual owns a location, do you believe they should have the unrestricted right to augment it as they please?'' }

    \item \textbf{IQ2:} \textit{``Are there any specific limits or guidelines you would propose to regulate augmentations on private locations?''}
 
    \item \textbf{IQ3:} \textit{``For any of the locations you saw in the questionnaire, do you believe you should be required to have explicit permission from the owner in order to augment it with AR content? Explain your reasoning.''}

    \item \textbf{IQ4:} \textit{``Do you think there should be particular regulations concerning any of the AR content categories you saw in the questionnaire, and if yes, what should they entail?''}
\end{itemize}







