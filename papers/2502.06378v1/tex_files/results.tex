\begin{figure*} [h]
\begin{center}
\includegraphics[scale=0.4]{figures/_alan_ethics_results.png}
% \vspace{-15pt}
\end{center}
\caption{Responses to Likert scale questions surveying the acceptability of augmenting different locations with different content types using an AR graffiti app. 95\% confidence intervals are visualized with red bars, based on the conversion of ordinal variables to numeric ranks (1 = Extremely Unacceptable and 5 = Extremely Acceptable)}
\label{fig:ethics-results}
\end{figure*}

\section{Results}

\subsection{Participant Demographic Data}
Participants were recruited using social media and mailing lists. 
Participation in the study was voluntary. 
12 participants completed the study (2 female, 10 male) aged between between 21 and 23 years of age (M=22.1, SD=0.49). 
Participants indicated prior familiarity with smartphone AR (M=3.25, SD=1.01; 5-point Likert scale; 1 = Not At All Familiar; 5 = Very Familiar), and with AR headsets (M=2.42, SD=1.19; 5-point Likert scale; 1 = Not At All Familiar; 5 = Very Familiar). 

Our participants were exclusively based in The University of Glasgow, Scotland.
Consequentially, discussions may be influenced by predominantly Western cultural and societal norms. 
E.g. this could influence participants' opinion on the acceptability of augmenting cultural or religious sites. 
Furthermore, participants were around a similar age range and bias towards being male. 
It is important then to approach the generalisability of our results with this in mind, and future work should look to explore multi-cultural attitudes and a wider demographic range. 


\subsection{Analysis}
An Aligned-Rank Transform (ART) \cite{results_1} was used for significance testing to transform non-parametric data prior to conducting a repeated measures ANOVA, using the ARTool R package \cite{results_2}. 
ART enables the use of parametric tests on non-parametric data (e.g. Likert-type responses \cite{likert}). 
When using ART, as noted by Wobbrock et al. \cite{wobbrockART} \textit{``the response variable may be continuous or ordinal, and is not required to be normally distributed''}, making it well suited to our dataset. 
Where two factors of concern existed, a two-way ANOVA was conducted (comparing location with AR content type). 

Figure \ref{fig:ethics-results} shows the distribution of acceptability for each AR content type at each location. 
This plot includes 95\% confidence intervals (visualised with red bars) based on conversion of dependent ordinal variables to numeric ranks, allowing a by-eye estimation of significant pairwise differences (where the confidence intervals do not overlap) - an approach favoured by those who believe reporting should be done with confidence intervals and visualisations \cite{errorbars}.

Participants' qualitative data collected in the interviews was coded using initial coding \cite{openaxebook} where participants’ statements were assigned emergent codes over repeated cycles with the codes grouped using a thematic approach. 
A single coder performed the coding (2 cycles) and reviewed the coding with one other researcher.
A Google Pixel 4a was used to record participants' qualitative data and to transcribe them.



\subsection{Questionnaire Results}
A two-way repeated-measures ANOVA was performed using an Aligned-Rank Transform (ART) with the ARTool R package. 
This revealed a significant effect on location (F=24.83, p < 0.001) and AR content type (F=84.18, p < 0.001). 
It also revealed significant interactions between location and AR content type (F=4.71, p<0.001). 
This suggests there are significant differences in acceptability ratings across different locations and there are significant differences in acceptability ratings across different content types.
It also suggests the effect of location on acceptability may vary depending on the AR content type, and vice versa.
These insights can also be seen by visually inspecting the differences in error bars in Figure \ref{fig:ethics-results}. 

Of note from Figure \ref{fig:ethics-results} is that, generally, \textit{Artistic} content was considered acceptable to be posted in all of our proposed locations. 
In contrast, \textit{Commercial} content was consider mostly unacceptable across all the locations. 
Attitudes towards \textit{Informative}, \textit{Protest} and \textit{Social} content were more location specific. 
\textit{Informative} content was considered by most to be acceptable in all of our locations, although a subset of participants did consider it unacceptable in \textit{Residential} locations. 
\textit{Protest} content was considered acceptable in \textit{Public} and \textit{Government} locations but not in \textit{Residential} or \textit{Religious} locations. 
While \textit{Social} content was considered unacceptable in \textit{Religious} locations but acceptable in all of the others.


\subsection{Insights From The Interviews}
\label{results:interviews}

\subsubsection{\textbf{Augmenting Locations You Own:}} all participants (n=12) agreed individuals should have the right to augment a location they owned provided \textit{``it does not break any existing laws''} and \textit{``the content is not deeply offensive or inappropriate''}. 


\subsubsection{\textbf{Augmenting Locations You Do Not Own:}} 
All participants said they would restrict access to posting AR content in religious and culturally significant locations.
Instead, they felt individuals who post content in these locations should have approved permission before doing so. 
Participants were concerned without control over who can post in these locations it would lead to misuse, disrespect, harassment, and harm, e.g. overlaying religious imagery with modern non-religious artwork, spamming inappropriate content contrary to the values/beliefs associated with a location, etc. 
% Participants felt by controlling who can post in these locations (and what) that the potential for harm and misuse mitigated against.

8 participants said they would restrict AR content from being posted in residential areas due to concerns surrounding the cyberbullying or harassment. 
In particular, the participants were concerned for an individual's residence to be targeted because of their own personal beliefs with hateful AR content, e.g. digital equivalents of a cross being burned into an individual's lawn \cite{mum-essay}. 

% 3 participants also stated they would restrict AR content from being posted in government locations - highlighting the often historical and cultural significance of these locations, and value as a whole to their surrounding community. 
Alongside content control, 1 participant discussed the importance of context when considering significant locations. 
This participant discussed changing attitudes towards potentially sensitive historical locations, \textit{“It feels more acceptable to augment the Great Wall of China because its historical significance may not be as immediately relevant today, but it wouldn't be appropriate to augment a Holocaust memorial.”}.
Although this too requires cultural consideration, as Western and Eastern individuals may have very different attitudes to the point raised by this participant.



\subsubsection{\textbf{Regulating and Restricting AR Content Types:}} 
11 participants said there should be significant regulation on commercial AR content. 
9 of these participants said locations where commercial AR content could be placed should be controlled to avoid disrupting individuals (e.g. posting advertisements at historical locations or areas of natural beauty) or be disrespectful (e.g. posting advertisements in religious or memorial locations). 
These participants felt commercial AR content should be: posted only where existing advertisements are, ideally relevant to the location in which it is placed (to increase the value to anyone who sees it), and be age appropriate, e.g. \textit{``people or companies shouldn't be allowed to place advertisements aimed at adults at a children's park''}.

5 participants said there should be significant regulation and control over AR content about protesting.
All indicated this was due to the risk of protest content evoking a harmful, dangerous, or violent reaction. 
As such, these participants considered it important protest content adhere to regulations to mitigate against the risks the content could provoke.


\subsubsection{\textbf{Comparisons to Social Media Platforms:}} all participants stated any app designed to host and/or share AR content should be restricted by the same guidelines as social media platforms and their content standards. 
9 participants directly compared our app to existing social media because of its emphasis on the creation and sharing of content. 
As such, they believed apps like ours should be held to the same restrictions and legislation that existing mainstream social media platforms (e.g. Instagram, Snapchat, X, etc) are, focused on ensuring content is age appropriate, legal, not offensive or hate speech, adhere to expected social norms, etc \cite{results_4}. 