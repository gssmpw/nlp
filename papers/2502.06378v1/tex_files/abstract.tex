Augmented reality (AR) will enable individuals to share and experience content augmented at real world locations with ease. 
But what protections and restrictions should be in place? 
Should, for example, anyone be able to post any content they wish at a place of religious or cultural significance? 
% We investigated attitudes towards the posting a variety of AR content types at different real world locations. 
We developed a smartphone app to give individuals hands-on experience posting and sharing AR content. 
After using our app, we investigated their attitudes towards posting different types of AR content (of an artistic, protest, social, informative, and commercial nature) in a variety of locations (cultural sites, religious sites, residential areas, public spaces, government buildings, and tourist points of interests). 
Our results show individuals expect restrictions to be in place to control who can post AR content at some locations, in particular those of religious and cultural significance. 
We also report individuals prefer augmentations to fit contextually within the environment they are posted, and expect the posting and sharing of AR content to adhere to the same regulations/legislation as social media platforms. 