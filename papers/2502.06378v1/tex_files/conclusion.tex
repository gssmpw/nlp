\section{Conclusion}
We present initial insights into attitudes towards the acceptability of augmenting real world locations with AR content. 
We conducted a user study (N=12) where participants gained hands-on experience posting/viewing AR content using a smartphone app we developed. 
After this, participants then completed a questionnaire and interview to reflect on their attitudes towards the acceptability of augmenting cultural sites, religious sites, residential areas, public spaces, government buildings, and tourist points of interests, with AR content of an artistic, protest, social, informative, and commercial nature. 
While more work is needed, we provide early insights into the expectations of individuals. 
We show participants all expected the creation, sharing, and viewing of AR content to be regulated/legislated in the same manner as content posted on existing social media platforms. 
Additionally, they expected some locations, e.g. those of religious and cultural significance, to be restricted so that only approved individuals were permitted to augment. 
When augmenting locations you own, participants expected to be free to augment as they wish provided no existing laws are broken. 
And concerning AR advertisements, participants said believed these should mimic how existing real world, physical advertisements are embedded into locations. 