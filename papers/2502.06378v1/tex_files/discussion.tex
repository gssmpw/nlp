\section{Discussion}
Our results show attitudes towards the acceptability of AR content depends on both the location of augmentation and the type of augmentation being made. 
Public spaces, for example, generally considered shared spaces belonging to the community rather than any individual or organization \cite{disc_1, disc_2} were considered acceptable for posting AR content. 
In contrast, other locations, e.g. those of religious or cultural significance, were considered to be in need of protection, e.g. to only be augmented by those with permission to do so to avoid being augmented with content in a disrespectful or hateful manner.


\subsection{Legislation and Regulation}
In addition to restricting who could post at certain locations, participants often drew similarities between how existing social media platforms (e.g. Snapchat, TikTok, etc) manage content and how shared AR content might be managed in the future. 
That is, participants expected the posting, sharing, and viewing of AR content to follow the same regulations/legislation that posting, sharing, and viewing content on social media platforms currently follow. 
However, future work is necessary to determine if the existing scope of guidelines, restrictions, regulations, and legislation are comprehensive enough to address the platform specific harms which AR content might enable. 

One distinguishing aspect of AR content, compared to social media posts, is that even if the content itself does not breach existing laws, its placement in certain contexts may still result in significant harm. 
Imagine augmenting a synagogue with images depicting wealth or money. 
Even though the content itself, in isolation, might not violate specific hate crime laws, its underlying sentiment and placement in this location, i.e. its context, makes it highly inappropriate and harmful. 
This distinction underscores the need to consider the differences between AR content and existing social media platforms, and what additional measures will be necessary due to the unique harms AR can enable. 





\subsection{Suggestions for Future Work}
We close by highlighting directions future research can take to further investigate the acceptability of augmenting real world locations, but also to develop effective protections for locations to mitigate and/or restrict augmentations when necessary.  

Future work is needed to determine what locations should be restricted partially or fully from being augmented by the general public. 
Our work indicates locations of religious or cultural significance, and possibly even residential areas, are potentially sensitive locations where such measures should be taken. 
The need for such restrictions has already been evidenced.
E.g., shortly after the release of Pokemon Go the Auschwitz-Birkenau State Museum was forced to ban individuals from playing the game whilst at the Auschwitz concentration camp \cite{pokemon-go} and asked the game's developer to issue an update making the game unplayable at this location. 

Future work should also aim to be cross-cultural and capture a global perspective to obtain a more comprehensive view of public augmentation use and an understanding of what is considered an unacceptable augmentation. 
Such work could utilise surveys, but could also look for opportunities to conduct remote user studies \cite{flo-alt-methods, flo-alt-methods-2, flo-alt-methods-3}, or even run the same study in multiple countries, e.g. \cite{ammar}.

In addition to considering what legalisation might be needed \cite{mum-essay, essay-short}, future work should also investigate what other protections can be developed.
Privacy enhancing technologies, for example, are designed to protect the rights of people around AR users from potential harms the AR device could cause to them \cite{imwut-joseph}, and equivalent protection systems could be developed to protect locations, places, and spaces from augmentation. 
One solution might be to re-introduce physical restrictions, e.g. requiring a user to physically touch a surface before allowing them to augment it with content \cite{sphere, van2004tangible, billinghurst2008tangible}. 
Such an approach would introduce a physical deterrent to posting AR content but also severely restrict its potential use. 

From a use standpoint, future work is also needed to understand how users will engage with the augmented content we have discussed throughout this paper. 
One can easily envision on a public, shared AR world view that some locations will contain large amounts of content. 
The Sekia Camera app \cite{lit_9}, which provided users with a \textit{``real-time, location-based, augmented reality social networking system''}, encountered this issue where a large numbers of posts were made in popular locations, overwhelming users screens with overlapping content, breaking the app's functionality. 
To address this, the developers allowed users to manually search for or filter posts at a given location. 
But how could such a system automatically resolve this problem for users? 
Suppose, for example, five overlapping pieces of AR content are applied to a building a user is looking at. 
The user's device could utilise the spatial flexibility \cite{hyunsung-minexr, o2023dynamic, chi-joseph} and inherent dynamic presentation \cite{imx-joseph, avi-joseph,  bystander-perdis-joseph, shady-bans, julie, ismar-stories} of XR content to contextually alter the position of overlapping content to optimise clarity and visibility for the user \cite{reality-aware, ieee-hyunsung}. 
Or a recommendation system could algorithmically decide which content to prioritise and show a user, based on the content they are mostly likely to engage with.

Finally, as individuals begin to post, share, and view AR content in real world locations more widely, it is worth monitoring discussion of this on social media platforms. 
User posts and discourse on social platforms are increasingly becoming a valuable data source for HCI researchers \cite{singh2024exploring, user-reviews, subreddit-addition-recovery-gauthier, ux-subreddit-shukla} as a way to understand sentiment towards new technology \cite{li2023sentiment} or its use \cite{ux-subreddit-shukla}. 
Monitoring emergent trends and discourse on such platforms will, eventually, provide insight into public attitudes on what is considered an acceptable augmentation or not, reported instances where protective measures have failed, and more.