\section{Introduction}
As augmented reality (AR) devices tend towards everyday/all-day, ubiquitous and fashionable form factors \cite{imwut-joseph, mmve-joseph, mathis2023breaking} (e.g. smart glasses \cite{imwut-joseph}), users will be able to share and experience AR content interwoven with people \cite{jolie-vrst} and the physical world that surrounds them \cite{intro_3}.
For example, a city street could be filled with AR artwork \cite{street-art}, historical information could be placed next to points of historical significance \cite{intro_6}, and more \cite{mum-essay}. 
However, this technology may also be used more controversially. 
Individuals might be harassed by others who attach derogatory AR messages to their home \cite{intro_7}, or businesses could be defaced with AR graphics posing a risk to their professional image and reputation \cite{intro_8}. 
And individuals will be able to do this with ease, creating and disseminating harmful AR content by simply directing a camera/sensor and pressing a few buttons. 
Users can already do this using existing social platforms like Snapchat or TikTok, but in the near future might contribute their posts to shared AR world views \cite{dig-graf-1, dig-graf-2}, increasing the content's reach and potentially even its impact. 

Such capabilities evoke clear concerns surrounding the use of such potential technologies. 
Is it, for example, acceptable for a religious site (e.g. a church) to be augmented with commercial content, or even content posted by the general public? 
Or should some locations be protected so only approved parties may post there?
While recently discussion into such issues has increased (e.g. \cite{lit_1, lit_2, lit_3}) further research is needed. 

To address this, we contribute a user study (N=12) which investigated attitudes towards the augmentation of common, everyday, real world locations (cultural, religious, public, residential, government, and tourist locations) with a variety types of AR content (artistic, protest, social, informative, and commercial content). 
In our study, participants gained hands-on experience using a smartphone app we developed to post/view persistent AR content (2D images) then completed a questionnaire and interview to reflect on their attitudes towards the acceptability of augmenting real world locations. 

We show participants believed the creation, sharing, and viewing of AR content in real world locations should be regulated/legislated in the same way existing social media platforms regulate/legislate user content. 
Additionally, participants expect restrictions to be in place to control who can post AR content at some locations, in particular those of religious and cultural significance. 
When augmenting locations you own, participants expect to be free to do as they wish provided the augmentations do not break any existing laws around free speech. 
Participants also expect AR advertisements to adhere to existing standards, i.e. being located where existing physical advertisements already are and being age and content appropriate for their setting. 