\section{Discussion}
In this work, we investigated knowledge workers' usage and perceptions of AR for everyday tasks and activities.
%To achieve this, we conducted a longitudinal diary study, deploying AR laptops with 14 participants to use in the wild over a period of two working weeks.
%Participants were tasked with working on the device for at least 30 minutes a day. 
%In total, we collected survey entries, photographs, and interview data capturing and reflecting on 103 hours of work (7 hours per participant). 
Our findings confirm and extend many prior studies on AR. 
In this section, we relate these works to our results, identify opportunities for future research, and outline the limitations of our study.

\subsection{How Do People Work in AR?}
%In our work, our objective was to observe how people might work with an AR laptop in an ecologically valid setting. 
%Unlike previous studies that focused on predefined controlled tasks (\eg~\cite{pavanatto2024multiplemonitors}), participants in our study completed their own real-world tasks.
%Recent research has focused on real-world usage conditions, but these studies often have limitations in duration (\eg~\cite{pavanatto2024xrwild,biener2024xrworkpublic}), rely on predetermined environments (\eg~\cite{guo2019maslows,steinicke2014selfexperimentation}), or restrict participants' arrangement of virtual windows (\eg~\cite{biener2022vrweek}).
%Our longitudinal, open-ended deployment in real-world settings provided insights into participants' usage of an AR laptop for a wider variety of tasks and environments.

Our results indicate that participants adapted their use of AR based on several key factors: task, environment, social context, physical comfort, and aesthetics. 
These factors largely align with Grubert~\etal's~\cite{grubert2017pervasive} characterization of relevant context sources to consider in ``pervasive AR''.
We build on this work by highlighting how their considerations manifest themselves in the wild.

\paragraph{Virtual workspaces patterns}
In our study, participants engaged in tasks typical of knowledge workers (\eg~\cite{hernandez2024software}), including reading and writing documents, web browsing, and checking and composing emails. 
To support these tasks, participants explored various virtual workspaces.
% 
Prior research has documented various patterns of window usage in AR, including the use of the central field of view for primary task content~\cite{pavanatto2024xrwild,lischke2016screenarrangement} and different window configurations for productivity tasks, such as vertical, horizontal, two-by-two, and personal cockpit layouts~\cite{mcgill2020seatedvrworkspace,ens2014personalcockpit}.
In our in-the-wild study, we observed participants using single windows and side-by-side layouts, which supported focused tasks and cross-referencing, respectively.
When more than three windows were used, 
they often represented applications for different tasks. 
% 
Similar to the findings of Cho~\etal~\cite{cho2024minexr}, we observed participants blending leisure and work applications within their workspace in these settings. 
Additionally, consistent with Pavanatto~\etal~\cite{pavanatto2024xrwild} and Lischke~\etal~\cite{lischke2016screenarrangement}, 
for these multi-window layouts, 
participants organized their multi-window layouts around a central task area, while using the periphery for secondary applications.
In addition to previously documented behaviors, we observed that participants used their peripheral screen space to maintain persistent AR windows. 
They sometimes stacked or overlapped their windows to support interactions such as quickly switching back and forth between window contents. 
They also adjusted window sizes according to their internal content, sometimes to form factors that are not supported by current screen-based computers.
Although much of previous work has focused on static layouts (\eg~\cite{cheng2021semanticadapt,biener2022vrweek}), 
our results indicate that virtual workspaces tend to be dynamically adjusted during usage. 

\paragraph{Environment}
Besides configuring their workspace based on task considerations, participants adapted window placements to account for environmental lighting conditions, 
as noted by Pavanatto~\etal~\cite{pavanatto2024xrwild}; 
geometric features and visual clutter, as highlighted by Cheng~\etal~\cite{cheng2021semanticadapt}; 
and the presence of others, as documented by Ng~\etal~\cite{ng2021passengerexperiencemrairplane} and Medeiros~\etal~\cite{medeiros2022shieldingar}. 
Interestingly, participants also considered the perceived spaciousness of their environment in their workspace configurations, a factor that, to our knowledge, has not been previously documented. 

In addition, several participants attempted to use AR to support work from new postures and locations; however, consistent with McGill~\etal~\cite{mcgill2020seatedvrworkspace}, our results suggest that participants may require assistance in identifying optimal configurations.

\paragraph{Aesthetics \& preferences}
Some of our findings on aesthetic considerations echoed prior work, particularly the tendency of some participants to arrange their workspace in a strict grid structure, consistent with the findings of Cheng~\etal~\cite{cheng2021semanticadapt}. 
However, this behavior was not representative of all participants, as some adopted more flexible arrangements instead, highlighting individual differences in preferences. 
Due to a combination of aesthetic preferences, device limitations, and legacy bias, many participants reported adopting more minimal setups, consisting of fewer windows that resembled their physical workspaces. This tendency contrasts with the multi-window layouts described in works such as Cheng \etal~\cite{cheng2021semanticadapt} and Lindlbauer \etal~\cite{lindlbauer2019contextaware}.
We believe these findings beneficially expand on previous understandings of how people may use window-based AR.

\paragraph{Hybrid usage}
While many of our participants used the AR laptop as intended, \ie~as a standalone laptop, 
we observed several participants shifting to
more hybrid forms of usage. 
As Pavanatto~\etal~\cite{pavanatto2021virtualmonitor} previously showed, combining physical and virtual monitors may offer a better balance of familiarity with physical monitors, while also providing extra space from virtual monitors.
Our results showed that such a hybrid usage approach emerged in the wild when using an AR laptop, deriving similar benefits.
This highlights the potential for supporting integrated multi-device workflows with AR, rather than solely suggesting that AR should completely replace current devices, as previously proposed in Auda~\etal~\cite{auda2023crossreality}.
In addition to complementing physical screens, we also observed participants using the AR device in combination with physical tasks. 
This similarly echoes prior research suggesting that the unique value of AR lies in its seamless blending of virtual and real (\eg~\cite{lindlbauer2019contextaware}).

Lastly, the emergent values we observed --- such as improved physical comfort, portability, multiscreen workflows, display in situ, immersiveness, and privacy --- largely support the idea that the benefits of AR previously noted in controlled laboratory settings (\eg~ \cite{pavanatto2024xrwild,biener2022vrweek,reconviguration2019schneider,ruvimova2020transportaway}) indeed translate to real-world applications.
Our work also reflects several long-standing challenges documented in previous research~\cite{kim2018revisitingtrendsar}, including some arising from hardware and software issues, as well as others more fundamental to AR interactions.

\subsection{Opportunities}
Our study offers a glimpse of what prolonged working in AR may look like in the future, including the advantages it offers and the challenges it presents. 
Our results highlight several opportunities for future work. 

First, our study shows that current devices make it possible to use AR for everyday tasks outside of controlled laboratory environments. In certain situations, regular users can experience substantial benefits. Although there are still hardware and software barriers, we believe this indicates that AR will become increasingly integrated into daily activities in the near future.
As such, it is pertinent for our research community to continue rigorously and longitudinally examining how users might interact with AR in their daily lives, both for knowledge work and beyond.
We expect improvements in AR devices to address technical issues such as limited field of view and low resolution, potentially leading to changes in user behavior.
For example, increasing the field of view of AR devices might allow users to take better advantage of expanded display space.
However, technical developments will take time, and even so, imperfections are likely to persist.
Therefore, examining user behavior with current devices and tools is invaluable.

Our results also indicate that in addition to tackling hardware and software challenges, \textbf{several fundamental interaction issues must be addressed} to facilitate the effective daily use of AR. 
One common issue we observed was efficiently \textbf{navigating the expanded AR display space}.
We encourage researchers to explore new methods to improve this process, potentially referencing related literature on large displays (\eg~\cite{czerwinski2006largedisplay}), where similar interaction challenges have also been identified.
Another challenge relates to \textbf{window management}. 
To address this issue, we believe there is value in operationalizing common arrangement patterns we observed as layout templates that users can snap windows to, similar to the application tiling feature found in macOS~\cite{macOStiling}.
However, given the sensitivity of window arrangements to their surroundings, more intelligent adaptation approaches may be necessary, such as the one previously proposed by Cheng~\etal~\cite{cheng2021semanticadapt}.
Furthermore, participants differed in their window arrangement behaviors.
We believe that future work should further explore how personalized AR workspaces can use contextual factors to assist users in their tasks.

Despite improvements in hardware, software, and interactions, challenges may still persist in promoting adoption and effective usage. 
In our study, we found that users tended to revert to their usual window usage habits in physical workspaces.
In many cases, these usage patterns did not take advantage of the unique affordances of AR. 
Participants themselves even reported feeling unsure about how to effectively use the new medium.
A possible solution to these challenges could involve \textbf{improving upon the onboarding process}. Although we attempted to thoroughly explain the device functionalities during our introductory session, providing more detailed and repeated training might be helpful.
To \textbf{help users overcome their legacy biases} regarding window arrangement patterns in particular, future work could consider exploring context-based and situational onboarding methods~\cite{chauvergne2023onboarding}, such as suggesting layouts tailored to their current task.
Yet, while we see AR as a means to engage with digital information in new ways, 
we also recognize the importance of maintaining familiar interfaces for work to facilitate a smoother transition from current laptops and devices.
Future devices \textbf{should leverage familiar aspects while introducing new features}, but this is likely easier said than done, requiring careful design of onboarding procedures and interfaces.
Perhaps parallels can be drawn to the introduction of mobile user interfaces. 
Early mobile user interface designs heavily relied on the preceding desktop paradigm~\cite{punchoojit2017mobileui}. While these designs did not initially suit the target form factor, 
they nonetheless provided a familiar entry point for users.
AR interfaces today may be in a similar state.
As previously discussed by Pavanatto~\etal~\cite{pavanatto2024multiplemonitors}, 
drawing on familiar paradigms may ease the transition from today's personal computers to computing with virtual displays, thereby enabling users to adapt more quickly and effectively to the new medium.

Last but not least, 
with AR entering our daily lives, 
as previously suggested by Abowd and Mynatt~\cite{abowd2000chartingubicomp},
how it will be used will differ significantly from controlled laboratory settings. 
In particular, people's activities will often lack clear beginnings or ends, sometimes occurring at the same time.
In our study, we observed patterns such as the integration of productive and leisurely tasks. AR devices designed for productivity must consider how people actually work, including providing opportunities to disengage~\cite{kaur2020breaks}.

\subsection{Limitations}
Our current study design is subject to several limitations.
First, while we aimed to study usage behaviors longitudinally, 
our results currently reflect the participants' engagement with the device for only 40 minutes a day over a two-week period.
Although we still consider this substantial, with an average of seven hours of use per person, extending the usage duration further would undoubtedly improve the ecological validity and generalizability of our results.
In particular, 
while our study design exposed participants to the AR laptop for an extended duration, 
we were unable to fully capture the behaviors that may emerge if the device completely replaced their current laptops or computers.
While we encourage researchers to study AR usage over longer periods, it is important to acknowledge the significant logistical and feasibility challenges involved. 
An alternative approach to consider is to consult and gather feedback from current power users.

Additionally, in our work, 
we based our analysis on post-usage diary surveys and photographs, as well as interviews.
Surveys and interviews provided valuable qualitative insights into participants' usage; however, they might be affected by memory biases since they required participants to recall their experiences.
Our photographs also provide rich information, providing a visual context for participants' written reflections.
However, while our results showed that participants often dynamically adjusted their virtual workspaces during their sessions, the photographs only capture the final state of these workspaces.
Our study would have benefited from collecting continuous telemetry or video data of participants' usage sessions; however, we chose not to out of concern for their privacy.
That said, we still believe that collecting more granular data will be valuable in the future, but such measures should be carefully weighed against their invasiveness.

It is also important to acknowledge that, while our qualitative data provides detailed insights into how a small group of participants uses and perceives AR laptops, it does not allow for statistical evaluation of the prevalence of these behaviors in a larger population. 
To validate these results, further quantitative studies would be beneficial. 
For instance, 
building on our observations of potential individual differences in AR laptop usage, it would be valuable to examine a larger population to determine whether users fall into distinct behavioral categories (\eg~preference for freeform or structured arrangements). 
Additionally, studying a population beyond a knowledge worker or university context would help assess the generalizability of our results.
%Although there is some diversity in background and age represented in our participants, 
%all were relatively advanced users of technology. 
%, generally well-educated, and living in a Western country.
%Future studies should expand the population considered 

Furthermore, 
our study was based on a specific hardware and software implementation of AR, which contains known limitations and is not wholly representative of future AR devices. 
It is not unlikely that as AR devices improve, user behaviors and perceptions may also change. 
For example, if the device supported more applications and was better integrated with participants' individual information ecosystems, they might have used the device more exclusively, rather than in combination with other computers.
Replications of our study with future devices would, therefore, yield valuable insights. 

Finally, our findings only capture current user perceptions and use of AR, situated within a technological landscape still dominated by mobile phones and computers. 
At the time of writing, AR was primarily used by early adopters only.
As AR usage becomes more prevalent, we may expect people's attitudes toward the technology to change. 
For example, although our participants expressed reluctance to use the device in public, social attitudes toward the device are likely to change as AR usage becomes more commonplace.
This suggests a need to continuously re-evaluate how people may use AR going forward. 

