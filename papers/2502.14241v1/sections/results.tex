\section{Usage Characteristics}
In this section, we present our results on how participants used AR laptops during the two-week diary period. 
Several factors influenced their device usage, including \emph{task}, 
\emph{physical environment}, \emph{social interactions}, 
\emph{physical comfort}, 
and \emph{aesthetics}. 
Below, we discuss these factors and their effects on usage behaviors.
In addition, we report on \emph{hybrid use cases} in which the AR laptop was used simultaneously with physical devices and tasks. 
We present the frequency of behaviors and remarks based on our closed coding results~(Section~\ref{sec:analysis}) to reflect the importance of a situation rather than an accurate measurement of how often a situation occurs~\cite{myers2016programmers}.
We use $n$ to denote the number of sessions during which a code was observed, out of 143 total sessions across all participants~(\eg~participants collectively watched videos in $n$~=~59 sessions).
We use $N$ to denote the number of participants associated with each code~(out of 14; \eg~$N$~=~10 participants watched videos at some point throughout the study).
\autoref{fig:themes} summarizes the main themes that we report on in our results.

\begin{figure}[t]
    \centering
    \includegraphics[width=.95\columnwidth]{figures/3-task.pdf}
    \caption{\textbf{Tasks} --- Participants engaged in a variety of work and leisure tasks. In most sessions, they focused on productive work but also explored leisure activities and occasionally mixed work with leisure.}
    \label{fig:tasks}
\end{figure}

\begin{figure}[t]
    \centering
    \includegraphics[width=0.95\linewidth]{figures/4-windows.pdf}
    \caption{\textbf{Number of windows} --- Participants used an average of two windows, with a maximum of six. Most sessions involved three or fewer.}
    \label{fig:windows}
\end{figure}

\subsection{Task}
Participants reported performing a variety of tasks with the AR laptop during their usage sessions (\autoref{fig:tasks}).
The most common activities included reading or editing documents and other artifacts~($n$~=~74,~$N$~=~14), 
% Survey: n=74, N=14
followed by checking and composing emails~($n$~=~63,~$N$~=~10),
% Survey: n=63, N=10
watching videos~($n$~=~59,~$N$~=~13), 
% Survey: n=59, N=13
and browsing web content~($n$~=~54,~$N$~=~14).
% Survey: n=59, N=14

During a single session, participants often \textbf{engaged in activities across multiple task categories}~($M$~=~2,~$SD$~=~1~tasks~per~session). 
By categorizing activities as either productivity- or leisure-oriented, we found that participants primarily focused on productivity-oriented tasks~($n$~=~77,~$N$~=~14).
% Survey: n=77, N=14
Some participants also explored more leisure-oriented tasks~($n$~=~19,~$N$~=~7)
% Survey: n=19, N=7
or combined productive and leisure activities within the same session~($n$~=~39,~$N$~=~10).
% Survey: n=39, N=10

%\subsubsection{Effect on Usage}
Participants frequently reported that their tasks influenced how they configured their AR workspace~($n$~=~116,~$N$~=~13).
% Survey: n=116, N=13

First, \textbf{their tasks influenced the number of virtual windows they opened} (\autoref{fig:windows}).
Our analysis showed that our participants configured virtual workspaces containing an average of 2.4 windows ($SD$~=~1.2).
% Survey: M=2.41, SD=1.20
We observed a maximum of six windows used simultaneously~($n$~=~2,~$N$~=~2);
% Survey: n=2, N=2
however, workspaces containing four or more windows~($n$~=~23,~$N$~=~8)
% Survey: n=23, N=8
were notably less common than those with three or fewer~($n$~=~118,~$N$~=~14).
% Survey: n=118, N=14

In 39 sessions, participants configured their AR workspace to consist of only a single window (\autoref{fig:patterns},~left).
% Survey: n=39, N=10
This layout was most frequently used for watching video~($n$~=~14,~$N$~=~7), browsing websites~($n$~=~10,~$N$~=~5), and reading and writing documents~($n$~=~7,~$N$~=~3).
For productive use cases, 
several participants~($N$~=~5) reported that this layout helped them focus on a single task.
As P5 commented, ``I did not feel the need to open more windows.... I would often open multiple windows when I am multitasking.''
P11 used a single window layout to read because it was ``free from distractions.''
For entertainment purposes, like watching a video, participants appreciated the immersiveness of this arrangement~($N$~=~4).
As P10 remarked, ``I placed [a single] window in the middle to have a sense of a big TV or cinema.''


In 33 sessions, 
participants placed two windows side by side~(\autoref{fig:patterns},~middle). 
% Survey: n=33, N=11 
% 2 windows only: n=19, N=8
Participants reported that \textbf{arranging windows side-by-side facilitated cross-referencing and comparison tasks}.
For example, as P8 recalled, 
``scheduling was nice because I could write an email and glance into my schedule without having to go between tabs.''
P7 shared a similar experience: ``I was watching a video on prompt engineering, and then on the right, I was using ChatGPT to try to ask certain questions a certain way versus different ways.... It was really useful for having that extra screen, just to be there and learn from.''

In 66 sessions, 
% Survey: n=66
participants configured their AR workspaces to include three or more windows.
\textbf{Participants generally leveraged these multi-window workspaces to facilitate multitasking}.
P1, for instance, appreciated ``the way I can do multitasks, where I have some displays used for entertainment next to displays that are used for work stuff.''
As with many other participants~($n$~=~32,~$N$~=~11), 
% Survey: n=32, N=11
they often worked on one or two main windows while playing a video or listening to music in other windows (\eg~\autoref{fig:patterns},~right).
Similarly, P5 reported that within one workspace, they would maintain windows for their primary task alongside ones they were ``not actively using.''

Generally, \textbf{participants organized their windows based on their expected usage frequency}, positioning windows for \textbf{primary tasks at the center} and moving \textbf{secondary windows to the periphery} or outside their field of view~($n$~=~41,~$N$~=~11;~\autoref{fig:patterns},~right).
% Survey: n=41, N=11
For example, P4 described their workflow: ``I kept my primary working screen immediately in front of me (typically reserved for email). Supporting screens, such as Google Drive or Google Sheet, were placed to the left. My calendar remained below the main window.''
Similarly, P13 mentioned, ``I push non-critical things to the corners of my field of view, so that they are easily accessed, but not interfering with my work.'' 

Several participants ($N$~=~4)
% Interview: N=4
also \textbf{used their peripheral screen space in AR to host persistent windows across multiple sessions to support ongoing or future tasks}. 
P4 reported moving windows ``way off to the side,'' then ``mov[ing] them back in'' when needed. 
This included a persistent email window that they stored ``in the corner somewhere, that I could glance at.''
For several sessions, P14 used one window to accumulate readings and resources they anticipated potentially returning to for their work.

Participants mainly arranged windows in a way that separated them in space~($n$~=~59,~$N$~=~11;~\autoref{fig:spacing},~left), 
% Survey: n=59, N=11
ensuring that their contents were fully visible. 
For P8, \textbf{separating windows facilitated their tasks by allowing them to ``see them all at a glance}, without having them stacked on top of each other like a conventional laptop does.''
Participants also overlapped~($n$~=~37,~$N$~=~11;~\autoref{fig:spacing},~middle) or stacked~($n$~=~26,~$N$~=~11;~\autoref{fig:spacing},~right) windows.
Two participants reported that \textbf{stacking windows facilitated efficient task switching by allowing them to quickly ``swap back and forth'' between them}~(P3).



\begin{figure*}
    \centering
    \includegraphics[width=0.95\textwidth]{figures/5-patterns.pdf}
    \caption{\textbf{Arrangement patterns} --- Common window arrangement patterns include using a single window, placing windows side-by-side, and positioning secondary windows in the periphery.}
    \label{fig:patterns}
\end{figure*}

\begin{figure*}
    \centering
    \includegraphics[width=0.95\textwidth]{figures/6-spacing.pdf}
    \caption{\textbf{Window spacing} --- The majority of participants preferred to keep windows separate, while a few opted to overlap or stack them.}
    \label{fig:spacing}
\end{figure*}

Besides the number of windows and their arrangement, 
\textbf{participants' tasks also influenced how they sized windows in AR}.
Several participants reported expanding single windows to form factors unsupported by conventional displays~($n$~=~17,~$N$~=~8).
For example, 
P6 found it beneficial to extend windows lengthwise to ``see more content at a time without scrolling.''
They specifically used longer windows to fill out forms and examine designs and documents in full, providing a more holistic overview.
P11 noted that larger windows were ``especially good for editing and reading.'' 
13 participants 
% Survey: n=39, N=13
used enlarged windows to watch videos. 
P5 commented, ``I like the fact that I can make my own theatre at home. The screen can be as big as I need it to be.''

Finally, during sessions, \textbf{several participants frequently adjusted their windows} --- including their size~($n$~=~38,~$N$~=~11), 
% Survey: n=38, N=11 
depth~($n$~=~37,~$N$~=~9), 
% Survey: n=37, N=9
and placement~($n$~=~36,~$N$~=~11) --- 
% Survey: n=36, N=11
partly \textbf{in response to task needs}~($N$~=~37,~$n$~=~11).
% Survey: n=37, N=11 
P11 reported alternating between ``bring[ing] things forward'' to ``read in depth'' and ``push[ing] them back'' to ``scan.''
P2 recalled re-organizing their workspace and bringing tasks to a ``more primary position'' as their focus shifted.


\subsection{Environment}
\label{sec:environment}
Participants completed their sessions in a variety of environments (\autoref{fig:env}), including
their home office~($n$~=~60,~$N$~=~9), 
% Survey: n=60, N=9
bedroom~($n$~=~27,~$N$~=~9), 
% Survey: n=27, N=9
office~($n$~=~23,~$N$~=~10),
% Survey: n=23, N=10
living room~($n$~=~18,~$N$~=~5), 
% Survey: n=18, N=5
and kitchen~($n$~=~11,~$N$~=~4).
% Survey: n=11, N=4
The settings in which the participants worked varied in size (\autoref{fig:env},~top-left). 
Some used compact desk surfaces placed against a wall~($N$~=~60,~$n$~=~11), 
% Survey: n=60, N=11
while others were in spacious areas that offered ample room in front of their devices~($n$~=~44,~$N$~=~10).
% Survey: n=44, N=10
In addition, the environments differed in terms of visual clutter (\autoref{fig:env},~top-right). 
Some participants worked in clean spaces with plain walls~($n$~=~16,~$N$~=~5) 
% Survey: n=16, N=5
while others were in heavily decorated environments~($n$~=~32,~$N$~=~10). 
% Survey: n=32, N=10
Most sessions were completed against physical backgrounds that included a moderate number of objects~($n$~=~74,~$N$~=~12).
% Survey: n=74, N=12

%\subsubsection{Effect on Usage}
The AR laptop used by our participants overlaid virtual elements onto their physical surroundings. 
This made participants' usage behaviors context-sensitive to their environment~($n$~=~8,~$N$~=~13).
% Survey: N=13

First, many \textbf{participants faced challenges with visibility of their virtual windows due to lighting conditions}~($n$~=~16,~$N$~=~7) \textbf{and background clutter}~($n$~=~18,~$N$~=~8)\textbf{, 
avoiding both in their window arrangements}.
% Lighting + background color:
% Survey: n=19, N=10
% Lighting:
% Survey: n=16, N=7
% VisualClutter:
% Survey: n=5, N=4
% Anchor Surface: n=14, N=8
%  P3, P5, P6, P8, P9, P12, P13, P14
P10, for instance, recalled having difficulties using the AR laptop in a bright office environment: ``It didn't work because there was so much light coming in, so what I did was push the windows to the ceiling.''
To minimize distractions caused by visual clutter, 
eight participants
% Interview: N=7
% Survey: N=8
reported that they intentionally sought out simpler walls to project their virtual windows.  
As P4 remarked, ``I appreciated just having a blank wall, and it was a neutral color, like an off-white, grayish.... So I thought that was a good environment.''

Here, it is important to note that the effects of environmental lighting and visual clutter may have been partly due to the display limitations of the device, particularly the rendering capabilities of its optical pass-through display. 
External lighting reduced the display contrast. 
As P12 commented, ``If I'm outside with the conditions and the brightness and the sun, I cannot be writing a document and reading a lot because the contrast is not enough for me.'' 
Reflecting on their experience of working in an office with sticky notes posted around the walls, 
P7 noted, 
``If there was a way to, like, completely make it opaque, where the virtual screen was completely opaque, ... it would be very helpful. I wouldn't have to see the sticky notes in the background very much [and] just be able to focus.''
This consideration also likely contributed to the participants' preference for completing their study sessions in indoor environments. 
Only three participants attempted working outdoors. 
% Survey: n=3, N=3
As P9 remarked, ``I tried to sit outside like around lunchtime one day and that was a disaster because of the light.'' 

Interestingly, instead of avoiding visual clutter, two participants
% Survey: n=2, N=2
used their virtual windows to shield themselves from distractions in their physical surroundings, creating a more focused workspace.
P14, for instance, 
reported placing a window over a TV in the background in one session and over a fan in another.

Besides visibility concerns, 
\textbf{participants' configurations of their virtual workspace were also generally guided by physical restrictions}.
Participants generally avoided overlapping virtual elements with physical objects and boundaries~($n$~=~19,~$n$~=~9).
For instance, P1 reported, ``I wanted the displays to be above my physical objects on my desk so that it doesn't interfere.''
Three participants
% Survey: n=2, N=2
% Interview: N=3
adjusted their workspaces according to how spacious they perceived their surroundings to be. 
In ``a very spacious room with high ceilings,'' P12 ``took advantage of that space'' to construct a wider AR workspace with three windows, whereas when the room ``was not very spacious,'' they only used a single virtual window.
However, several participants ($N$~=~3)
% Interview: N=3
sometimes chose to ignore physical constraints. 
P1 recalled, ``since I was working on a personal desk with a white wall close in front of me, I placed the displays further away from me virtually to feel more like I am in a spacious room.'' 

One participant (P4) reported \textbf{arranging windows based on their semantic relationship with the physical environment}. 
In several sessions, they placed a weather application next to their window so that whenever they glanced over, they were informed of the outside temperature. 
P4 commented, ``I thought it was convenient to have it so closely associated with what I was looking at.''

\begin{figure*}
    \centering
    \includegraphics[width=0.95\textwidth]{figures/7-environments.pdf}
    \caption{\textbf{Environments} --- Participants completed their sessions in a variety of physical environments. Their physical surroundings differed in size and level of visual clutter. Participants mostly worked in private spaces.}
    \label{fig:env}
    \vspace{-1em}
\end{figure*}

\subsection{Social Factors}
\label{sec:social}
In 30 sessions, 
% P1, P2, P3, P4, P5, P7, P8, P14
participants ($N$~=~8) used the AR laptop in public spaces or during interactions with others (\autoref{fig:env},~bottom).

Three participants
% Interview: N=3
found it \textbf{comfortable to use the device in public}.
Five participants 
% Interview: N=5
highlighted its \textbf{potential for keeping their work private while working in such spaces}. 
As P4 commented, ``The biggest benefit I saw was that if you're on an airplane and you want total secrecy, you don't want people looking over to see what you're typing in your email or the kind of work you're doing.'' 
P2 noted that the device provided significant \textbf{value during in-person meetings}. 
They reported that it allowed them to be ``immersed'' in their information environment while still staying ``a part of the team.''

On the other hand, five participants
% Interview: N=5
voiced \textbf{concerns about the device's current social acceptability}.
As P1 noted, ``I actually wanted to use it outside... but my friends all told me that I'd look really weird and freaky.''
P9 recalled, ``One night I was doing this study and my husband came in. He was like, `what the hell are you doing?' It kind of freaked him out a little bit... then I started thinking, this does look weird. I don't know if I would feel comfortable to walk into Starbucks or sit on my front porch with all my nosy neighbors walking by.'' 
These issues with social acceptability were attributed partly to the novelty of the device, but also to the lack of transparency in their interactions with the virtual content.
Reflecting on how they appeared to bystanders, P4 remarked, ``I could picture myself just sitting there looking into blank nothingness and someone thinking, `What are you doing?' Like that looks so bizarre.''
Two participants
% Interview: N=4
\textbf{anticipated that the device would become more socially acceptable over time} as people grew accustomed to its use.
P4, for instance, expressed that they would be more open to using AR if ``more people are familiarized'' with it.

In addition to issues with social acceptability, 
four participants 
% Interview: N=4 
reported that the device \textbf{hindered their ability to interact with others}.
This mostly stemmed from the glasses and virtual contents obstructing their field of view.
P7 and P8 reported having to look either above or below the lens to make direct eye contact. 
P4 removed his glasses entirely when talking with others, explaining, 
``I didn't want them to think ... that I'm not paying attention to them, ... that I'm just distracted by something else,'' such as a virtual window.

%\subsubsection{Effect on Usage}
Participants' concerns about social acceptability likely contributed to their decision to complete their sessions primarily in private environments~($n$~=~111,~$N$~=~14).
% Bedroom: 27
% Home Office: 60
% Kitchen: 11
% Living Room: 17 total, 4 with social external task, 13 private
% Survey: n=111, N=14
In addition, 
several participants ($N$~=~4)
% Interview: N=
% Survey: n=4, N=4
reported \textbf{adjusting their workspace arrangements around people} within their immediate surroundings. 
As P5 mentioned, ``If there were some people in front of me, or I was, like, talking to them while interacting with the interface, I tried to move [windows] away. I did not overlap my [windows] with their faces.''
P7 similarly preferred placing windows off-center when working among others, particularly strangers, to avoid staring at them while interacting with their own interfaces.
While P2 initially adopted a similar approach, ``mak[ing] an active effort to move screens out of people's faces,'' they later placed windows directly on top of people, reasoning that they could simultaneously view the virtual information while still maintaining a somewhat clear view of their conversation partner.

\subsection{Physical Comfort}
\label{sec:ergonomics}
Our results
indicate that many participants~($n$~=~17,~$N$~=~7) leveraged the device's flexibility in arranging windows to \textbf{adapt their workspace to their posture}, rather than adjusting their posture to fit the workspace.
As P1 remarks, ``I'm always hunched over my laptop so I have quite a bit of neck pain. But with this device, I didn't need to.''
P13 similarly adjusted their position while working and ``slightly shifted the window to match that.''

This flexibility also \textbf{allowed for positions that were previously unsupported by conventional laptops and computers}.
While most participants conducted the majority of their usage sessions at a desk or table~($n$~=~105,~$N$~=~14),
some explored engaging with their devices while resting in bed~($n$~=~16,~$N$~=~6) 
% Survey: n=16, N=6
or reclining on their chair~($n$~=~5,~$N$~=~3).
% Survey: n=5, N=3
P8 shared: ``We have those chairs that leaned back. I kind of put my whole canvas on the ceiling so I could lay back.''
P14 also expressed appreciation for being ``able to have a screen wherever I want, anywhere I want, anytime I want.'' 
They explained, ``Basically, I can be laying in bed eating. I always have a big screen in front of me. ''

However, considerations of physical comfort, to some extent, also limited usage of the AR workspace.
Several participants~($n$~=~8,~$N$~=~5)
reported that navigating the AR workspace to access distant elements strained their neck.
For example, when P4 first started using the device, they wanted to take advantage of the ``endless canvas'' and try out expansive workspace arrangements, such as ``stretch[ing] the screen up really high.''
However, 
they soon realized the physical toll, explaining, 
``I'm moving my head up, up and down, up and down, left and right. And I just started realizing, maybe that's the reason why my neck is hurting.''
This partially prompted several participants~($N$~=~7) 
% Interview: N=7
to gradually move towards using fewer windows.
It is important to note that the aforementioned issues were likely exacerbated by the display's limited field of view.
As P2 notes, ``this field of view makes me move my head a lot to be able to see everything I need to.'' 

Similar considerations also contributed to the \textbf{preference for side-by-side layouts over vertical arrangements}~($N$~=~5).
In 9 sessions, 
% Survey: n=9, N=4 
% Interview: P4, P5, P6
% Survey: P1, P4
participants arranged two windows vertically, one above the other.
Top-and-bottom layouts usually feature a primary task window and a secondary window that is only occasionally referenced. 
This contrasts with side-by-side layouts, which facilitate tasks that require more frequent cross-referencing.
Top-and-bottom layouts were used less frequently than side-by-side because participants found navigating horizontally more ``natural.'' 
As P5 remarked, ``most convenient was placing it side by side. I like that I don't have to turn my neck up to look for different screens.''

\subsection{Aesthetics}
\label{sec:aesthetics}

Several usage patterns we observed may be attributed to individual aesthetic preferences. 

Participants \textbf{mostly sized their windows to have a landscape aspect ratio}~($n$~=~290,~$N$~=~14).
% Survey: n=290, N=14
Portrait~($n$~=~28,~$N$~=~5)
% Survey: n=28, N=5
and square~($n$~=~9,~$N$~=~6)
% Survey: n=9, N=6
windows were used significantly less frequently in comparison (\autoref{fig:window-sizing},~top).
When participants had multiple windows in their workspaces, the scale and aspect ratios in these windows tended to be more uniform~($n$~=~47,~$N$~=~11), 
% Survey: n=47, N=11
although a variety of display heterogeneity was represented nonetheless (\autoref{fig:window-sizing},~bottom).

Participants' multi-window workspaces \textbf{varied from a strict grid structure to more flexible arrangements} (\autoref{fig:structure}). 
Interestingly, participants generally preferred one of the two extremes.
In 32 sessions, 
% Survey: n=32 
participants' workspaces were largely unstructured, while in 27 sessions,
% Survey: 27
they were highly structured.

Several participants~($N$~=~6) \textbf{intentionally limited their window usage}.
As P10 noted, 
``I didn't want to feel overwhelmed with being surrounded by windows.''
In one session, P4 recalled, 
``I removed a screen off to the left side, not that it was distracting, but out of a sense of cleaning up my workspace.''

Some of these aesthetic decisions \textbf{may be the result of legacy bias}. 
As P12 remarked, ``I was still very biased towards what I'm familiar with. So, I usually use a two-device setup, or maybe a two-monitor setup, or a three-monitor setup. I find that I kind of copy that, in a way.''
Reflecting on their tendency to close unnecessary windows, P4 similarly commented, ``I think that may be a carryover habit from using a monitor, where you have a more finite amount of space.''

%\subsubsection{Individual Differences}
%Participants differed significantly in the number of windows they used, the homogeneity of their window sizes, the level of structure in their workspace arrangements, and their use of tabs. For example, while \todo{N=6} participants naturally opened 3 to 4 windows, \todo{N=1} others used a single-window layout. Similarly, while \todo{N=6} participants tended to configure their workspaces in a grid-like arrangement, \todo{N=5} others followed almost no structure.

\begin{figure}[t]
    \centering
    \includegraphics[width=0.9\linewidth]{figures/8-sizing.pdf}
    \caption{\textbf{Window sizing} --- Participants mostly sized their windows to have a landscape aspect ratio (290 out of 327 total windows across all sessions). In multi-window workspaces, window sizes ranged from homogeneous across all windows to heterogeneous.}
    \label{fig:window-sizing}
    \vspace{-1em}
\end{figure}

\begin{figure}[t]
    \centering
    \includegraphics[width=0.9\linewidth]{figures/9-structure.pdf}
    \caption{\textbf{Arrangement structure} --- Participants' workspaces varied from flexible arrangements to more structured grid layouts.}
    \label{fig:structure}
\end{figure}

\subsection{Hybrid Usage}
\label{sec:hybrid}

In 98 sessions, 
% Survey: n=98, N=14
participants~($N$~=~14) reported \textbf{using the AR laptop in combination with physical displays} (\autoref{fig:hybrid}), 
including phones~($n$~=~83,~$N$~=~14), 
% Survey: n=83, N=14
computers~($n$~=~28,~$N$~=~9), 
% Survey: n=28, N=9
tablets~($n$~=~1,~$N$~=~1), 
% Survey: n=1, N=1
watches~($n$~=~4,~$N$~=~2), 
% Survey: n=4, N=2
and TVs~($n$~=~2,~$N$~=~2).

% Survey: n=2, N=2 
Participants typically used a mobile phone to check and respond to notifications~($n$~=~34,~$N$~=~6),
% Survey: n=34, N=6
messages~($n$~=~27,~$N$~=~9),
% Survey: n=27, N=9 
and emails~($n$~=~3,~$N$~=~2).
% Survey: n=3, N=2
They also used their phone to write notes~($n$~=~3,~$N$~=~2), 
% Survey: n=3, N=2
participate in calls~($n$~=~10,~$N$~=~6), 
and check the time~($n$~=~4,~$N$~=~4). 
% Survey: n=4, N=4 
Three participants favored a device with a physical display because it offered affordances that better accommodated their needs. 
For example, P2 found the ``slimmer screen'' of their phone to be better suited for social media scroll lists.
Three participants used a personal phone to separate their personal information and activities from their work device.
As P14 remarked, 
``I like to separate my everyday life from work-related things.''

When using the AR laptop alongside a conventional laptop or desktop computer, 
some users reported arranging their virtual windows as peripheral displays around a physical laptop~($N$~=~5).
% Survey: N=5
For example, while P2 used their physical laptop for a design task, they displayed reference materials, such as a list of design requirements, as windows in AR and arranged them around their device.
P4 used a similar workflow, organizing reference materials around their computer monitor, which they used to author emails.
Two participants
% Interview: N=2 
used the AR laptop alongside physical devices due to the hardware limitations of the former, including its field of view and resolution.
They also relied on physical laptops to access applications and information that was unsupported or missing on the AR laptop ($N$~=~5).
% Interview: N=5 
Several examples of these applications include 3D graphics software~(P1) and code editors~(P13).

Participants also \textbf{used the AR laptop simultaneously with physical tasks}~($n$~=~56,~$N$~=~11). 
% Survey: n=56, N=11
For example, P3 used the AR laptop to display instructions that assisted in a prototyping task, aligning the window with their physical activities to engage in both tasks simultaneously.
P2 viewed videos in AR to entertain themselves while performing mundane tasks, like folding laundry.
P1 commented, ``I like having the freedom in my physical space, to multitask and not have a physical device getting in my way since it is a virtual workspace.''

\begin{figure}[t]
    \centering
    \includegraphics[width=0.9\linewidth]{figures/10-hybrid.pdf}
    \caption{\textbf{Hybrid usage} --- Participants often used the AR laptop in combination with other devices, such as a laptop.}
    \label{fig:hybrid}
\end{figure}

\begin{figure*}[t]
    \centering
    \includegraphics[width=0.98\linewidth]{figures/per-participant.pdf}
    \caption{\textbf{Individual differences} - participants varied in the applications they opened, the number of windows they used, the amount of structure in their workspace arrangements, and the uniformity of their window sizes.}
    \label{fig:individual-diffs}
\end{figure*}

\subsection{Individual Differences}
In previous sections, we reported several behavioral tendencies that emerged from our aggregate observations. However, it is important to note that our participants also exhibited considerable variation in how they used the AR laptop.
As shown in \autoref{fig:individual-diffs}, participants had different applications in their virtual workspace, such as work, leisure, or a combination of both. They also varied in the number of windows used, the uniformity of window sizes, and the amount of structure in their workspace arrangements.
For example, while many participants ($N$~=~6) naturally opened an average of 3 to 4 windows, one participant exclusively used a single-window layout. 
Similarly, while participants ($N$~=~6) tended to configure their workspaces in a grid-like arrangement (\ie~4 or 5 arrangement structure rating on average),  others ($N$~=~5) followed almost no structure (\ie~1 or 2 arrangement structure rating).

\section{Values behind AR Laptop Usage}
In this section, we summarize the values that the participants gained from using the AR laptop. 

\paragraph{Improved Physical Comfort}
As discussed in Section~\ref{sec:ergonomics}, the AR laptop offered flexibility in configuring virtual windows, enabling the creation of workspaces tailored to the user's posture. This flexibility notably supported postures that are not well accommodated by traditional devices such as laptops and desktops.

\paragraph{Portability}
Since the device's windows are virtual and occupy no physical real-estate, 
% Survey: N=6
% Interview: P3, P7, P8, P9, P10, P13, P14
seven
participants also considered the device to be more portable.
As P13 noted, ``You can use this anywhere,'' in contrast to practices with current physical laptops and desktops that ``depend on specific setups.'' 
P8 envisioned that the device would enable working in ``smaller crowded spaces'' as a result.
P3 described an instance where they moved around with the device, walking between their living room and bedroom while maintaining continuity in their work. 

\paragraph{Multi-screen workflows}
Seven participants 
% Survey: N=7
felt the device facilitated more efficient multi-screen workflows. 
It allowed windows to be flexibly configured into productive arrangements, such as side-by-side for cross-referencing~($n$~=~33).
The device's support for peripheral engagement was also regarded as beneficial~($N$~=~7).
Participants used their peripheral screen space to enable glanceable access to secondary content and to keep persistent windows for ongoing tasks.

\paragraph{In situ}
Six
% Survey: N=6
participants appreciated the in situ nature of their windows. 
In particular, they believed that it allowed them to maintain a higher level of situational awareness of their physical surroundings~($N$~=~4).
% Survey: N=4
% Interview: P1, P2, P4, P8, P11, P12
For example, P11 regarded the ability to remain situated in and aware of the surroundings as critical for their role as an administrator: ``I would absolutely need it. If someone approached me, I would have to greet them.''
P12 noted that in a public setting, ``it is important to have a sense of what is happening in the surroundings,'' which is supported by the device.
As discussed in Sections \ref{sec:environment} and \ref{sec:hybrid}, 
the AR laptop also offered opportunities to supplement the user's task and physical environment with contextually relevant digital information.
% Survey: N=1
% Interview: P3, P4 

\paragraph{Immersiveness}
Six participants 
% Survey: P1, P11, P14
% Interview: P13, P5, P9
reported feeling more immersed in their work while engaging with the AR device. 
P9 described the experience, ``the immersiveness was off the scale... I haven't experienced a flow state like that in a long time.''
P3 mentioned that while using the device, they felt ``less distracted from the real world in general.''
For P11, the immersive nature of the device also enriched leisure activities: ``Watching videos is a joy.... This could be the future of entertainment --- time flew by.''

\paragraph{Privacy}
Lastly, as discussed in Section~\ref{sec:social},
% Interview: P4, P8, P9, P10, P12, P14
% Survey: P1, P6, P9, P12, P14
participants highlighted that the AR laptop offers potential privacy benefits.
They appreciated not ``hav[ing] to worry about [others] seeing information that isn't relevant to them,'' since all the AR laptop's contents were rendered through the head-mounted display (P9).

\section{Current Challenges with AR Laptop Usage}
In this section, we describe the challenges reported by our participants in using the AR laptop. 

\paragraph{Device hardware and software}
Participants encountered several hardware and software limitations with the AR laptop during their usage sessions. 
Seven participants found the \textbf{weight} of the glasses uncomfortable, 
% Interview: N=7
while six participants reported issues with \textbf{overheating}.
% Interview: P8, P10, P13
% Survey: P2, P7, P11
Similarly, 
seven participants reported that the \textbf{display resolution} made it ``uncomfortable to visually look at'' (P6). 
Moreover, seven participants disliked how the glasses were tethered to their laptop keyboard, commenting that it ``limited [their] mobiliy'' (P7).
Finally, since the device only supported browser-based activities, it precluded some tasks that participants would like to have performed for their work~($N$~=~5).

Many of the aforementioned issues were effectively long-standing challenges within the AR research community~\cite{kim2018revisitingtrendsar}.
In our analysis, several limitations affected usage behaviors. 
For instance, as discussed in Section~\ref{sec:ergonomics}, the \textbf{restricted field of view} appeared to be one reason participants opted for more minimal setups, with several of them feeling limited ($N$~=~7).
For P10, this made using the device feel ``claustrophobic.'' 
P4 felt it restricted their interactions and task performance, noting, ``It was a little slower for me to complete my task.''
Some participants chose to avoid reading tasks due to visual discomfort caused by limited display resolution~($N$~=~5).
If their task involved reading, some participants chose to combine the AR laptop with higher resolution physical displays, instead of using it as a standalone device (Section~\ref{sec:hybrid}).

\paragraph{Interaction limitations}
Hardware and software limitations aside, 
our results revealed several fundamental interaction limitations. 
While the expanded virtual display space provides benefits, such as improving multi-screen workflows, it also presents challenges for \textbf{navigation} across the display~($N$~=~3).
As P8 noted, ``the trackpad is too small for such a wide canvas.'' 
P2 similarly commented that they ``want to be able to make use of that entire space'' but disliked ``having to manage that using the mouse interface.''


The ability to flexibly add and configure windows also presents challenges for \textbf{window management}. 
Seven participants 
% Survey: N=7
reported difficulties with having to manually arrange and align windows.
For example, P5 recalled, ``I did not like the range of the cursor movement on the virtual canvas. For instance, I was trying to enlarge the window but I had to click and drag it thrice to make it as big as I wanted it to be.''
P6 noted, ``I wish there was an automatic arrangement feature. [On my computer], when you want to put screens side by side, you can just drag [them into position] and it kind of partially [does] the task for you. But with this, I have to manually adjust every single detail.''
% P2: 2 
% P3: 8
% P5: 2, 4, 5, 9, 10
% P6: 9 
% P8: 3
% P10: 4, 5
% P11: 2
This suggests the need to introduce automated window management approaches, either through templates or more intelligent placement algorithms (\eg~\cite{cheng2021semanticadapt}). 

Three participants reported feeling \textbf{uncertain about how to best utilize the new medium}. 
As discussed in Section~\ref{sec:aesthetics}, participants tended to gravitate toward physical workspace arrangements they were already familiar with.
P13, for instance, ``did not really explore'' the different usage scenarios as a result. 
P14 commented, ``using this kind of device as a normal computer is not taking advantage [of] every single possibility that you can get with this kind of device.''
However, they found it challenging to conceive of new workflows, noting, 
``it's such a new concept that people are not used to thinking that way.''
To address this challenge, they suggested providing ``a guide on how to properly set it up, or a prompt saying, `you might want to do this like this.'{}''
% P13: ``I did not really explore. Maybe I should have, but it just felt natural that way, so I really had no reason to change it.''
% P14: ``I feel like using this kind of device as a normal computer is not taking advantage of every single, every single possibility that you can get with this kind of device.''
% P14:  ``I think it's such a new concept that people are not used to thinking that way, and so that's why I thought that having maybe a guide on how to properly set it up, or a prompt saying, Oh, this, you might want to do this like this, because we're so hard set on having monitors that I basically just used it as a different kind of monitor, and it could turn on and off very quickly to see whatever was on the background.''

Lastly, 
it proved difficult for the device to effectively balance the \textbf{visibility} of the physical and virtual environments.
For instance, while several participants~($N$~=~6) found value in the immersiveness of the AR laptop, others reported feeling ``overwhelmed'' when too many windows were opened~(P10,~$N$~=~6).
Similarly, while participants appreciated the in-situ nature of the displays~($N$~=~6), 
they also found their physical surroundings distracting~($N$~=~7). 
Combined, these results highlight several core design tensions in the development of AR systems.