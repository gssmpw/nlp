\section{Related Work}
\label{sec:related-work}
Our work builds upon prior Mixed Reality (MR) research, including productivity-oriented applications, window-based interfaces, and in-the-wild as well as longitudinal studies.

\subsection{Productivity in Mixed Reality}
\label{sec:related-work-productivity}
In prior research, there is a significant body of work leveraging devices across the MR continuum, including Augmented and Virtual Reality (AR/VR), for productivity-related applications.
Many systems and prototypes have been developed to support tasks such as sensemaking~\cite{lisle2021immersivethink}, 
working with spreadsheets~\cite{gesslein2020spreadsheet},
data visualization~\cite{butscher2018collaborativeanalysis},
and collaboration~\cite{pejsa2016room2room}.
These systems use a variety of novel interaction techniques, 
ranging from multi-device input for navigating information~\cite{le2021vxslate,biener2020breakingscreen} to gaze-based approaches for accessing applications~\cite{lu2020glanceable}. 

Beyond promising benefits for application-specific tasks, for many commercial stakeholders and researchers, MR represents a future pervasive medium for work, much like how computers and mobile phones are today~\cite{gruber2018officefuture,grubert2017pervasive,feiner2002ar}.
Numerous studies have highlighted the advantages that MR offers as a ubiquitous tool for work.
For instance, MR's ability to personalize a user's visual work environment, such as replacing an open office background with a calming natural scene, has been shown to reduce distractions and stress while inducing a state of flow~\cite{ruvimova2020transportaway,lee2022partitioning,thoondee2017vrstress}. Schneider~\etal~\cite{reconviguration2019schneider} and Grubert~\etal~\cite{gruber2018officefuture} examined the potential privacy benefits of the HMD form factor, which restricts the visibility of content to the wearer.
Last but not least, MR relieves information displays of physical limitations~\cite{biener2022vrweek,gruber2018officefuture}. 
Using an HMD to present virtual content, including conventional windows, icons, and menus, MR provides more flexibility in where and how information is displayed~\cite{pavanatto2024xrwild}. 
As promoted by providers like Apple, Meta, and Sightful, it effectively provides ``boundless'' workspace~\cite{sightful}. 
This expanded screen real-estate may facilitate
complex multi-window operations by serving as a type of external memory~\cite{pavanatto2024multiplemonitors,andrews2010space,ball2007movetoimprove}, 
enable more efficient multitasking~\cite{czerwinski2003largedisplays},
provide quicker peripheral access to information~\cite{lu2020glanceable,davari2020occlusion,grudin2001digitalworlds}, 
and enhance immersion~\cite{endert2012lhrd}.

Yet, MR is still far from a commonplace tool for productivity due to significant usability challenges in current devices~\cite{ens2014personalcockpit,mcgill2020seatedvrworkspace,medeiros2022shieldingar,ng2021passengerexperiencemrairplane,pavanatto2024multiplemonitors,pavanatto2021virtualmonitor}. 
One longstanding issue is the resolution and field of view limitations of MR HMDs~\cite{pavanatto2021virtualmonitor,biener2024xrworkpublic}.
According to Pavanatto~\etal~ \cite{pavanatto2021virtualmonitor}, this reduces the effective screen space available to the user and increases head movement, which can cause neck pain \cite{gruber2018officefuture,mcgill2020seatedvrworkspace} and decreased task performance \cite{pavanatto2023virtualmonitor,pavanatto2021virtualmonitor}.
In AR, changes in focal distance when switching between virtual and physical content increase visual fatigue~\cite{gabbard2019arcontextswitch}
and interaction errors~\cite{eiberger2019depth}. 
AR also presents challenges in maintaining a balance between virtual and physical awareness. 
The visibility of virtual elements can sometimes be diminished by the physical environment~\cite{cheng2021semanticadapt}. 
On the other hand, interacting with virtual elements may distract users and reduce their awareness of their surroundings~\cite{tao2022distractions,li2022mrworkspaces}, including the presence of bystanders~\cite{ohagen2020realityaware}.
Finally, 
the increased screen real estate that MR offers may present challenges, such as difficulties in managing multiple windows and accessing distant information~\cite{endert2012lhrd,pavanatto2024multiplemonitors,czerwinski2003largedisplays}.

Overall, the literature currently describes a range of positive and negative aspects of working in MR. However, these insights were often derived from short-term studies conducted in laboratory environments. 
In contrast, our work aimed to examine the potential values and challenges of productive AR usage in a more ecologically valid context.
% We achieved this by deploying AR laptops for use in participants' everyday work environments over a two-week period.
% We collected in situ diary entries and photographs of their usage to gather insight into how AR may eventually be used as a pervasive tool for work.

\subsection{Window-based Mixed Reality}
MR systems are often linked to novel interaction techniques, but there has also been a persistent interest in using these devices to enhance knowledge work by expanding traditional 2D displays and windows beyond the limitations of physical screens and into the third dimension~\cite{pavanatto2024multiplemonitors}.
Early work, such as Feiner~\etal~\cite{feiner1993windows}, 
explored various methods to register windows,
including head-fixed and world-fixed approaches, 
as well as how to create connections between windows and physical objects.
Raskar~\etal~\cite{raskar1998officefuture} investigated the possibilities of expanding an office space with AR using projectors and cameras.
Many recent commercial MR workspaces, such as the Varjo Workspace~\cite{varjoworkspace}, 
Immersed~\cite{immersed},
and vSpatial~\cite{vspatial},
now enable streaming of desktop elements into MR as virtual windows and support conventional input devices (\ie~mouse and keyboard) for interaction.
Using conventional 2D displays and desktop input devices for MR can ease the transition to virtual displays for users coming from personal computers, while also providing access to familiar and highly refined interfaces and applications~\cite{pavanatto2024multiplemonitors}.
Therefore, in our study, we focus on emergent user behaviors when using an AR laptop that features a window-based immersive display along with keyboard and touchpad input.

%Previous work has examined the advantages and disadvantages of window-based MR (discussed in Section~\ref{sec:related-work-productivity}).
Several prior studies have investigated the usage of window-based MR. 
For example, Ens~\etal~\cite{ens2014ethereal} devised a design framework to classify window-based mixed reality workspaces based on windows in seven dimensions, such as whether the windows used an ego- or exocentric reference frame or were movable.
Several prior studies~\cite{cheng2021semanticadapt,pavanatto2024xrwild,lischke2016screenarrangement} observed that users preferred to center main task content while using peripheral areas for supporting information.
Su~and~Bailey~\cite{su2005positioning} recommended against placing virtual contents behind the user.
McGill~\etal~\cite{mcgill2020seatedvrworkspace} reported on several common window configurations for general productivity tasks, including vertical, horizontal, two-plus-two, and following a ``personal cockpit''~\cite{ens2014personalcockpit}.
As immersive technologies become more pervasively used, 
the context of use also becomes a critical consideration~\cite{grubert2017pervasive}.
For example, the physical environment can hinder or enhance interactions~\cite{cheng2023interactionadapt}.
Similarly, social acceptability has been shown to play an important role in the use of virtual monitors in airplanes~\cite{ng2021passengerexperiencemrairplane} and other shared transit modalities~\cite{medeiros2022shieldingar}.
While many usage patterns have been observed in shorter laboratory studies, our research investigates how these patterns may manifest in more realistic environments and over extended periods.

\subsection{In-the-Wild and Longitudinal Studies of MR}
In the current literature, there is a growing number of in-the-wild and extended-use studies of MR systems.
In-the-wild studies of MR have demonstrated effectiveness and data quality across various application areas, including education~\cite{petersen2021pedagogical},
accessibility~\cite{schmelter2023accessible},
recreational sports~\cite{colley2015blended,colley2017ubimount}, and military training~\cite{laviola2015military}.
Researchers have also investigated the long-term effects of MR usage on depth perception~\cite{kohm2022objects}, social interactions~\cite{han2023longitudinal,bailenson2006longitudinal}, and within industrial scenarios~\cite{grubert2010mobileindustrialar}.
The most relevant to our work are previous studies on the use of MR for everyday productivity tasks. 

In early investigations, many researchers engaged in informal self-experimentation. 
Starner~\cite{stevens2013starner}
and Mann~\cite{buchanan2013mann}
independently designed and wore AR headsets almost daily for over a decade.
During this time, 
they reported using the device for tasks such as checking the Internet and taking notes during face-to-face meetings~\cite{stevens2013starner},
along with experiences of backlash and more negative physiological aftereffects~\cite{buchanan2013mann}. 
Recent studies have continued this line of work with more participants, often with the aim of quantifying the effects.
For example, Steinicke and Bruder~\cite{steinicke2014selfexperimentation}
and Nordahl~\etal~\cite{nordahl2019hours} examined the impact of prolonged VR usage on key usability factors, including simulator sickness, over periods of 24 hours and 12 hours, respectively.
Lu~\etal~\cite{lu2023inthewild} evaluated a glanceable AR prototype with three participants over three days.
Guo~\etal~\cite{guo2019maslows}
and Shen~\etal~\cite{shen2019longtermfatigue} 
report on a study in which 27 participants worked eight hours each in virtual and physical office environments, focusing on emotional and physiological needs, as well as visual and physical discomfort.
Biener~\etal~\cite{biener2022vrweek,biener2024holdtight} compared VR work with a regular physical environment over five days with 16 participants. 
Pavanatto~\etal~\cite{pavanatto2024xrwild} recently examined user and bystander experiences of extended reality~(XR) usage in public, studying how users perform their own tasks in XR at various campus locations (\eg~library, dining).

Both in-the-wild and longitudinal studies are valuable for examining user behaviors and attitudes in more ecologically valid settings~\cite{abowd2000chartingubicomp} and to bypass distortions from novelty effects~\cite{bailenson2024seeing}.
We build on existing work by reporting on a study in which we deployed an AR laptop for daily use in participants' everyday work environments over the course of two weeks. 
As a primary point of differentiation from most previous research,
we did not constrain the environment or time in which participants performed their own tasks in AR. 
Our device also allowed participants to arrange multiple virtual windows flexibly. 
Overall, we aimed to study a closer approximation of AR usage for everyday productivity tasks in the future, observing how individuals may appropriate the device to suit their specific needs and usage contexts~\cite{rogers2011wild}.
