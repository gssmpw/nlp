\firstsection{Introduction}
Augmented Reality (AR) has the potential to change the way we engage in knowledge work.
Commercial off-the-shelf devices, such as the Apple Vision Pro~\cite{apple2023introducingvp} and Lenovo ThinkReality Glasses~\cite{lenovo2025thinkreality}, are increasingly promoted as productivity tools.
Previous research has also highlighted several benefits that AR offers in this context, such as reducing visual distractions and allowing the personalization of shared workspaces~\cite{lee2022partitioning}.
Furthermore, AR has been shown to support the in situ display of digital information~\cite{chen2023papertoplace} and facilitate collaborative embodied brainstorming and sensemaking~\cite{luo2022placement}.

A particularly promising affordance of AR for knowledge work is its ability to present information and applications without being limited to physical monitors~\cite{pavanatto2024xrwild}.
Using a head-mounted display (HMD), virtual windows, icons, and menus that were previously confined to computer screens can be seamlessly integrated into the user's physical environment, allowing them to be displayed anytime and anywhere, with variable size and appearance.
This support for an effectively infinite display space provides significant productivity benefits over conventional devices by minimizing context-switching costs and improving user awareness of peripheral information~\cite{mcgill2020seatedvrworkspace}.
Additionally, these properties may be particularly beneficial in mobile scenarios, facilitating work in diverse contexts, such as public or confined spaces where the size of physical hardware is limited~\cite{biener2024xrworkpublic}.

Many companies and researchers suggest AR as a medium will eventually replace computers and phones in the ``office of the future''~\cite{gruber2018officefuture}, enabling all-day knowledge work on the go. 
However, prior studies evaluating AR's support of knowledge work have predominantly been conducted in laboratory settings and over relatively short timeframes~(\eg~\cite{biener2024xrworkpublic,pavanatto2024xrwild}).
Although these earlier studies are valuable for validating the feasibility of performing knowledge work with current AR devices and informing future developments,
they may not fully reflect user experiences during everyday computing activities~\cite{abowd2000chartingubicomp}. 
Furthermore, the limited duration of these studies may obscure our understanding of how this technology shapes people's behaviors and attitudes once they have had time to adjust to its novelty~\cite{bailenson2024seeing}.
Therefore, we see it as beneficial to explore how people use AR for work in their daily environments over time.


In this paper, we present the results of a diary study investigating participants' usage of an AR laptop --- Sightful's Spacetop EA (Early Access)~\cite{spacetopea} --- for daily work tasks and activities.
The Spacetop EA is an optical see-through AR HMD that relies on a standard keyboard and trackpad for input~(Section~\ref{sec:apparatus}).
It is one of the first commercially available devices of this kind, offering sufficient stability for people to work in their own physical environments and functionality to support real productivity tasks. Notably, it supports flexibly opening and arranging virtual windows for common applications used in knowledge work, similar to the interfaces explored by Pavanatto~\etal~\cite{pavanatto2024multiplemonitors}.
In our study, we deployed this AR laptop for 14 participants to use during their workday over the course of two weeks in 40-minute daily sessions.
During the study, we collected photographs of their virtual workspace and physical environment, along with post-session survey responses about their usage and perception of the device.
Participants collectively completed a total of 103 hours of work using the AR laptop (\customtilde7 hours per participant) across 143 sessions (\customtilde10 sessions per participant).
The study concluded with \customtilde30-minute interviews, encouraging participants to reflect more deeply on their overall experience.

Our analysis of the collected survey data, workspace photographs, and interviews provides insights into how considerations around task, environment, social, and physical comfort influenced participants' usage behaviors, including the number, size, and arrangement of windows within their virtual workspaces.
We also report on instances where participants naturally used the device to complement physical displays and tasks. 
Finally, our analysis highlights the current values and challenges of using an AR laptop for tasks in everyday work settings. 
In our discussion, we reflect on how our findings, derived from a more ecologically-valid and longitudinal approach to investigating AR usage, relate to prior empirical studies.
We also highlight implications for future design and research, including directions for developing interaction techniques, onboarding approaches, and adaptive systems. 

In summary, we contribute the following:
\begin{itemize}
    \item A detailed analysis of AR usage patterns from a longitudinal diary study conducted in situ within people's daily work environments.
    We report on usage considerations, virtual display configuration patterns, and hybrid usage involving both physical displays and tasks
    \item Documentation of users' perceptions and experiences of AR's value for everyday tasks, as well as common challenges
    \item Implications and opportunities derived from our qualitative analysis, such as potential approaches to managing virtual workspaces, or how to better leverage contextual factors for assisting users 
\end{itemize}

As the AR research community and commercial stakeholders build toward a vision of ``pervasive'' AR usage~\cite{grubert2017pervasive}, our work aims to shed light on user behaviors and experiences that may arise from prolonged use in real-world settings, informing future technology designs to better accommodate these emerging needs and challenges.
