\section{Conclusion}
In this work, we contribute insights from a longitudinal diary study with 14 knowledge workers, who used an AR laptop for around 40 minutes a day over two working weeks.
We collected reflections on over 104 hours of work (7 hours per participant) through diary surveys, photographs, and interviews. 
During their usage sessions, participants engaged in a variety of tasks across different physical environments. 
Our findings shed light on how task, environment, social, and aesthetic factors influenced participants' usage patterns. 
Participants' tasks determined the number, arrangement, and size of virtual windows they used. 
External factors such as lighting, visual clutter, physical constraints, and the presence of others guided participants' window placements.
Participants further arranged their workspace in AR to facilitate more comfortable workflows, 
while making certain design decisions based on personal aesthetic preferences. 
Some participants used the AR laptop in combination with physical displays and tasks. 
%They notably used the device for productive and leisurely purposes and sometimes a combination of both within a single session. Participants explored various virtual workspace configurations but converged on fairly minimal setups, averaging two windows that closely resembled their physical workspaces. Our analysis revealed several recurring arrangement patterns, such as organizing one's workspace around a centralized main task area and using side-by-side window layouts for cross-referencing and comparison tasks. Participants' arrangements were frequently adjusted, and influenced by task, posture, environment, and social factors. Some participants used the AR laptop in combination with physical displays and tasks. 
Finally, we document the values and current challenges associated with using AR laptops, as reported by our participants.
Our results provide insight into how knowledge workers might use AR in the future, presenting numerous opportunities for further exploration. 
We believe it is crucial for the research community to continue developing an ecologically valid understanding of how AR may be used, especially in longitudinal studies, in preparation for its future pervasive use.