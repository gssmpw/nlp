
Analyzing videos recorded during studies have been used before to get a better understanding of participant behavior. 

This work aims at analyzing the behavior of N participants during an experiment

Similar to prior research involving video analysis, we used open and axial coding
Two authors read through and coded the results
Thereafter they discussed and concluded on a first jont codebook 
Multiple iterations of discussion to extend and refine the codebook 

F-score~\cite{} % 25
\begin{equation}
F = \frac{(1+\beta^2)\cdot recall \cdot precision}{(\beta^2 \cdot precision)+recall}
\end{equation}
As there was no indication that we should weight recall over precision or vice versa, the weight $\beta$ was set to 1. 
Recall is the number of overlap between two annotators divided by all annotations of the first annotator, and precision is the number of overlap between two annotators divided by all annotations of the second annotator. 
The F-score was first calculated for each code and then averaged among all codes to arrive at a single value comparing the two annotators. 
The average F-score for the training video was M (sd=).
The annotations for the training video was also checked manually to identify any problems, which were then discussed among annotators, as some actions could not be unambiguously assigned to one code. 

After discussing the training video, each annotator was assigned the video of several participants. Upoon starting annotating videos of a new participant, a random 30-minute-video-section of that participant was annotated by two different annotators, one of which was , and checked for inter-coder reliability. 
The average F-score for these tese was M (sd=)


We designed this experiment such that participants completed their own productivity tasks while using a AR laptop. 
We observed participants 
to gather data from a wide range of contexts and to understand real-world factors 

We recruited N participants
who were at least 18 years old
 had normal vision (corrected or uncorrected)
 were proficient with the English language
 used a computer daily for work 
 had a productivity task that they could perform during this study

 Our measures included
 We also obtained qualitative data through a semi-structured interview, with questions about 
 how they perceived their productivity, what strategies they chose to use, how they felt using an AR display in public, how focused or distracted they were, 
 and how they compared our system to the one they currently use

At the end of each week, further qualitative data was collected in a short interview. 
Participants were asked to talk about what htye liked or disliked during the week, 
how they felt, 
what problems occurred and what they would improve in the particular condition used in that week. 

Asked about their preferences and if they could imagine using VR for work in the future. 

The design, procedure, and survey described in the following were informed by a pilot study with N participants, in which we tested varying parameters such as screen resolution and the exact study-procedure 

We took a qualitative approach to investigate our research questions. 
With a careful design of survey questions, we were able to collect in-depth data on people's work-from-home setups and the tasks they performed on multiple devices. 
In the following section, we provide more details on our data collection process, 
our participants, 
and our analysis process. 

 \subsection{Task}
 Upon recruitment, 
 The tasks were real-world work that the participant had to complete regardless of study participation. 

 \subsection{Procedure}
 The study was approved by Institution Review Board
 and took place in N sessions of N minutes
 Participants were recruited through 
 were screened for inclusion criteria
 Each participant signed the consent form an answered a background questionnaire at our laboratory
 They received general instructions about the study and completed 

Users were encouraged to freely explore and use the virtual displays on their own, in whichever way they felt comfortable and helpful to their workflow 

After each session, the participant responded to a diary survey consisting of N questions, focusing on ...
The study ended with a semi-structured, N minute interview to further dive into users' usage patterns and considerations. 

All participants were informed about the procedure and content of the study, signed a consent form, and filled out a demographics questionnaire. 

All together, the duration of the whole workday was .
At the end of each week, participants were interviewed and all data collected during the week was secured 

The study was then conducted in three phases 

The procedure and study task was explained to the participants 
They were introduced to the functionality of the devices and taught how to set up the work environment on their own and to their preferences 
This included plugging in or connecting the devices, setting the IPD and adjusting the virtual monitors 
Participants started the training to get familiar with the task and  device

Participants were reminded to reach out if they had any questions

The study was divided into N phases. 


\subsection{Participants}
Fourteen participants (age, gender) from 
took part in the experiment in sessions 
We limited our population to our university community due to 
One participant was a graduate / undergraduate 
All participants used a computer for at least N hours daily for work, with N participants using it for more than N hours. 
All participants reported at least  experience with the operating system 
N participants had little to no experience with AR

In total, N participants (age, gender) participated in the study and completed two weeks.
Participants were employees or students at a university. 

All participants had normal, or corrected to normal eyesight, and they saw everything clearly in the virtual environment
N participants had no experience with VR, 
N only slight experience, 
N had moderate experience, 
N substantial experience,
N extensive experience 

With VR

Rated frequency of using a multi-monitor set-up 

One participant was excluded. 
Inconsistent completion of sessions. 


\subsection{Data Analysis}
We collected our data from N main sources: 
diary survey, photographs, interview audio recordings 

We analyzed a total of N photos following the approach of content analysis
and N textual entries following the approach of thematic analysis 
Through the analysis of photos, we developed insights of participants' virtual arrangements
Through analsys of the textual responses,
we generated descriptions on participants' common usage patterns, as well as their motivations and challenges behind these usage patterns 

To analyze the corpus of 156 photographs, 
we developed a code-book to understand different physical configurations 
We started this codebook from prior work and iteratively refined it through discussion among a subset of the authors during joint coding sessions 
The final code-book included N codes to describe ...
Then two coders independently coded N to compare codes with each other to resolve any differences in interpreting the codes
The two coders also separately coded the same N photos to ensure consistency 
The inter-rater reliability for these N photos was N, which was calculated using Cohen's kappa, suggesting a strong level of agreement between coders~\cite{} %49  
The two coders then discussed all disagreements in this set of photos' codes to ensure accurate coding, before proceeding to code the remaining photos 

To analyze the set of 866 textual responses from 10 open-ended questions, 
we followed the guidance of the reflexive thematic analysis approach~\cite{}, % 
with a combination of deductive and inductive coding
N authors independently familiarized themselves with the data and met ... 
to converge around a set of pre-determined codes from our research questions and related literature (\eg~usage patterns identified by prior work, the relatedness between tasks performed).
Following the discussion, one author led the coding process for all textual responses, using a prior coding and open coding, to identify participants' multi-device usage patterns, motivations, and pain points. 
The same author also led the clustering process, to
organize codes into themes, generating insights about participants’
values and challenges behind their usage patterns. Over 700 codes
were generated from the coding process, and four additional memos
chronically document emerging themes and topics.


\subsection{Data Collection}
The survey contained n questions 
asking n open-ended textual answers 
prompts for uploading photos 
We structured it around N themes on usage and experience 
The first section 
The second part 
The third set
inquired about 
dealt with
Demographic questions 
The tasks and activities they performed 
How they arranged their virtual workspace to accommodate for their tasks and activities
inquired about aspects they valued as well as pain points and challenges encountered during their usage
The final category probed into what they felt was missing or needed to improve their experience,
as well as the ideal environment for the future
The median completion time was N minutes, 
including idle time, 





\cite{bailenson2024seeing}
Documents several passthrough-oriented studies
A limited number of longitudinal studies 
No insights regarding long-term usage
Suggestions:
- Create guidelines for usage time, but failed with smartphones
- Provide more thorough training and onboarding: probably requires more detailed, repeated training

\cite{chauvergne2023onboarding}
Investigation of on-boarding practices in VR

\cite{biener2022vrweek}
Can reference introduction, related work, and method





%%%% ADVANTAGE: Personalize Work Environment
VR use in open office environment reduces distraction, improves flow, preferred by users~\cite{ruvimova2020transportaway}
VR used to reduce distractions and stress
Ruminova reported virtual beach environment can reduce distraction and induce flow ~\cite{ruvimova2020transportaway}
 Lee et al showed using AR to add visual separators reduce distractions and allows to easily personalize work environment~\cite{lee2022partitioning}
 VR relaxation environment more relaxed~\cite{thoondee2017vrstress}
 Experiencing nature in VR reduces stress better than watching a video on a regular display~\cite{pretsch2021improving} 
VR reduced distraction of users working in open office environments, induced flow, preferred by users~\cite{ruvimova2020transportaway}  
adapting work environments \cite{ruvimova2020transportaway}
Reduce real world distraction~\cite{biener2024xrworkpublic,ruvimova2020transportaway} 

privacy concerns~\cite{gruber2018officefuture,ofek2020practicalvirtualofficemobile}
Privacy from outside world and removal of real environment disruptions~\cite{biener2022vrweek}
Address privacy concerns in mobile context~\cite{reconviguration2019schneider,gruber2018officefuture}
VR can address privacy issues --> randomize layout of physical keyboard when entering password \cite{reconviguration2019schneider} 
Only HMD wearer can see contents on virtual screen \cite{gruber2018officefuture} 




This increased screen real estate may enable more efficient multi-window operations, provide better peripheral access to information, and enhance immersion.

Enable tasks that are complex and divided between multiple windows~\cite{pavanatto2024multiplemonitors}, 
provide peripheral awareness~\cite{grudin2001digitalworlds},
and increase immersion~\cite{bi2009comparing} 
Larger number of windows~\cite{endert2012lhrd} 
efficient multitasking~\cite{czerwinski2003largedisplays}
, which previous research has shown to be particularly beneficial for cognitively demanding tasks~\cite{sightful,czerwinski2003largedisplays}.




XR potential to benefit knowledge workers \cite{pavanatto2024xrwild, bellgardt2017vrwork, mcguffin2019augmented} 


Feiner: AR would "become much like telephone and PCs ... the overlaid information will become what we expect to see at work and at play"~\cite{feiner2002ar}

HWDs become capable of supporting tasks that frequently occur in our daily lives, including information acquisition/monitoring~\cite{lindlbauer2019contextaware,lu2020glanceable}
productivity~\cite{biener2022vrweek,biener2020breakingscreen,cheng2021semanticadapt,lisle2020evaluatingimmersivethink,pavanatto2021virtualmonitor}

Pavanatto examines use of VR and AR devices in public spaces to better understand how people interact with these devices and the factors that impact their usage \cite{pavanatto2024xrwild}

VR used to support work in various contexts for over a decade~\cite{biener2024holdtight} 
VR HMDs promoted as tool for knowledge workers~\cite{ofek2020practicalvirtualofficemobile} % 42 
Potential to support knowledge work \cite{biener2024holdtight}
Biener et al showed multimodal interaction, combining touch and eye-tracking can be used to efficiently navigate multiple screens + 3D vis can facilitate tasks involving multiple layers of information \cite{biener2022povrpoint,biener2020breakingscreen} 
2D mouse to interaction with 3D content in VR \cite{zhou2022indepthmouse} 


%%% APPLICATIONS
Use immersive systems to facilitate multi-tasking~\cite{ens2014personalcockpit}
sensemaking~\cite{kobayashi2021xrspace, lisle2021immersivethink,lisle2020evaluatingimmersivethink,tahmid2022immersivesensemaking} 
register three-dimensional content with touch-screens~\cite{le2021vxslate,biener2020breakingscreen}
perform data visualization over existing physical elements~\cite{butscher2018collaborativeanalysis,mahmood2017bdva,reipschlager2021arlargedisplaysinfovis}
Combined touch-screen and spatially tracked pen to author presentations~\cite{biener2022povrpoint}
and spreadsheet applications~\cite{gesslein2020spreadsheet}
Enhance capability of keyboards~\cite{reconviguration2019schneider} 


HWDs can enhance flexibility and mobility while reducing costs~\cite{pavanatto2021virtualmonitor}
HWDs can address challenges like lack of space, surrounding noise, illumination issues~\cite{gruber2018officefuture,ofek2020practicalvirtualofficemobile}

Applying gain to head rotations can enable a comfortable range of neck movement~\cite{mcgill2020seatedvrworkspace}

Combine virtual monitors and tablets for touch input improve user performance~\cite{le2021vxslate} --> shows importance of considering traditional interaction devices while using HWDs 
Immersive environments for sensemaking~\cite{kobayashi2021xrspace}


Provides repeatable, location-independent user experiences~\cite{biener2022vrweek}
Relieving physical world limitations~\cite{biener2022vrweek,gruber2018officefuture}
Direct manipulation enables natural hand motions~\cite{biener2022vrweek} 
Ability to map small physical motions to larger actions in virtual environments helps reduce fatigue and open up workspace for people with special needs~\cite{biener2022vrweek}
Collaboration brings people into same virtual space, increase sense of presence~\cite{biener2022vrweek,pejsa2016room2room}
Dynamically adapt to work situation, introduce beach environment t for reading a paper, formal office when writing email~\cite{biener2022vrweek}
Improve performance on certain task~\cite{biener2024xrworkpublic,biener2022povrpoint,gesslein2020spreadsheet}
Increase display space increases productivity~\cite{biener2024xrworkpublic, czerwinski2003largedisplays} 

%%%% CHALLENGES
Several longstanding issues include resolution and field of view limitations~\cite{pavanatto2021virtualmonitor},
Usability challenges remain
Low resolution prompts need to enlarge virtual content, leading to less screen space or more head movement~\cite{pavanatto2021virtualmonitor}
resulting in neck pain~\cite{gruber2018officefuture}
reducing task performance~\cite{pavanatto2023virtualmonitor,pavanatto2021virtualmonitor} 
Prior work attempted to address neck rotation issue by applying virtual gains~\cite{mcgill2020seatedvrworkspace}
focal distance changes when switching between virtual and physical content decrease task performance and increase visual fatigue~\cite{gabbard2019arcontextswitch}
induce more interaction errors~\cite{eiberger2019depth} 

XR may limit access to virtual elements~\cite{bai2021mobile,tao2022distractions,pavanatto2024xrwild} 
Users uncomfortable not knowing position of bystanders~\cite{ohagen2020realityaware} 
HWD leads to lower awareness of surroundings, increased perceived workload, lower text entry performance~\cite{li2022mrworkspaces}
Prior literature suggests integrating physical elements into physical environment~\cite{mcgill2020challengescarmr,wang2022realitylens}

AR may take over the real world and become intrusive to people's reality and physical tasks~\cite{lu2023inthewild}

%%% WINDOW USAGE
Displaying windows through HMD not new, explored by Feiner~\cite{feiner1993windows}
Raskar investigated meaning of expanding office space by combining AR, projectors, and cameras~\cite{raskar1998officefuture}
Topic more popular in recent years as pervasive everyday AR gained traction~\cite{grubert2017pervasive,bellgardt2017vrwork,gruber2018officefuture}
Investigated implications of using MR in mobile scnearios~\cite{knierim2021nomadic,ofek2020practicalvirtualofficemobile}
Investigated implications of using MR to work from home~\cite{fereydooni2020virtual}

VR makes it possible to have any number of displays, helpful in mobile contexts where screen space of conventional devices is limited \cite{biener2024holdtight}
Various ways to display virtual content to workers~\cite{pavanatto2023virtualmonitor}
Ens showed surrounding user with windows reduces application switching time \cite{ens2014personalcockpit}
Information presented in periphery accessed through quick glances~\cite{davari2020occlusion}, through approaches that place and summoned glanceable virtual content in a less obtrusive manner~\cite{lu2020glanceable}, and using various activation methods~\cite{lu2021evaluatingglanceable}

Virtual displays offer significant advantages over single screen laptops by eliminating context switches and facilitating head and eye glances~\cite{pavanatto2024xrwild} 

%%% WINDOW USAGE IN COMBINATION WITH PHYSICAL DEVICES -- INTERACTION TECHINQUES
Biener showed combining virtual displays with touchscreens shown to be beneficial to mobile workers~\cite{biener2020breakingscreen}
Physical displays and virtual representations combined to display visualizations over tabletops~\cite{butscher2018collaborativeanalysis}, larger screens~\cite{reipschlager2021arlargedisplaysinfovis,mahmood2017bdva}, and to combine physical and virtual documents~\cite{li2019holodoc}

%%% WINDOW USAGE CHALLENGES / PATTERNS 
Context-switching between physical and virtual reduces task performance, increases fatigue~\cite{gabbard2019arcontextswitch}
Multiple depth layers induces errors hen interacting~\cite{eiberger2019depth}
Social acceptance and monitor placement play important roles in virtual monitors in public spaces~\cite{ng2021passengerexperiencemrairplane,medeiros2022shieldingar}
VR usage for a week leads to high level of simulator sickness and low usability ratings~\cite{biener2022vrweek}
Limited FoV of HoloLens reduced performance~\cite{pavanatto2021virtualmonitor}
Pavanatto suggested using mixture of both physical and virtual monitors in AR due to current resolution limitations~\cite{pavanatto2021virtualmonitor}
McGill suggested manipulating virtual display position using users gaze direction to use a large displays space with less head movement~\cite{mcgill2020seatedvrworkspace} 


Though beneficial when limited space, Ng et all found passengers in airplanes preferred to limit virtual displays to personal seating area \cite{ng2021passengerexperiencemrairplane} 
Medeiros found users avoided placing displays at location of other passengers in public transportation scenarios, use to shield themselves from others ~\cite{medeiros2022shieldingar} 

Ens et al investigated design and advantages of immersive multitasking system that supports surrounding the user with windows~\cite{ens2014personalcockpit} --> 2D planar elements referred to as ethereal planes~\cite{ens2014ethereal} --> categorized usage 
Combining virtual displays with laptop/tablet touchscreen feasible approach to aid mobile workers~\cite{biener2020breakingscreen} 
Virtual monitors + tablets for touch input achieves performance and accuracy comparable to touch-controllers~\cite{le2021vxslate} 
Working with a combination of physical and digital documents through AR feasible and accepted~\cite{li2019holodoc} 
Combine physical displays and virtual 

While these systems were innovative and showed great promise, we also need to investigate how to use virtual displays to support more traditional forms of work, such as knowledge work from a personal computer~\cite{pavanatto2024multiplemonitors}

Conducting productivity work in HWDs provide display flexibility, reduce costs~\cite{pavanatto2021virtualmonitor} 
addresses challenges like lack of space, illumination issues

Traditional anchored input devices hard to use with peripheral portions of display space distant from input devices, forcing additional head rotation that could possibly result in neck pain~\cite{gruber2018officefuture} 
Organizing elements around a cylinder strategy to display elements closer to user, with commercial tools like BigScreen and Virtual DeskTop 
Amplified head rotation help reduce movement when accessing peripheral displays~\cite{mcgill2020seatedvrworkspace}
Wei et al showed that text should only be warped in a single direction at a time and at small curvatures to preserve reading comfort~\cite{wei2020reading} 
There are issues with with virtual monitors~\cite{pavanatto2024multiplemonitors}
Context switching between physical and virtual environments and focal distances reduces task performance and increases visual fatigue~\cite{gabbard2019arcontextswitch}
Multiple depth layers induces more errors~\cite{eiberger2019depth}
Social acceptance and monitor placement also shown to play important roles in the use of virtual monitors in public places like airplanes~\cite{ng2021passengerexperiencemrairplane},
and in the layout distribution of content across multiple shared-transit modalities~\cite{medeiros2022shieldingar} 

Displays provide opportunity to investigate a truly seamless display, with uniformity across the space. Since displays are highly configurable, we need more empirical information on how user experience is impacted by design choices. Virtual dispplaysh have potential for larger screen spaces, need to understand trade-offs of such choices, design systems to better manage the space without generating window management overhead~\cite{pavanatto2024multiplemonitors}

Observe how participants behave in a baseline VR environment without advanced interaction techniques


%%%% LONGITUDINAL STUDIES
Steinicke and Bruder~\cite{steinicke2014selfexperimentation} reported on their experience of working, eating, and entertaining themselves in VR for 24 hours.



\cite{grubert2010mobileindustrialar}, 


Prior work demonstrates range of positive aspects working in VR, but gathered through short-term studies~\cite{biener2024holdtight}
VR induced increased visual fatigue, muscle fatigue, acute stress, mental overload~\cite{souchet2023vrergonomics}
Further research needed on effects of working in VR, especially for extended periods of time~\cite{biener2024holdtight}


Abowd and Mynatt: computing activities in daily life differ from typical laboratory studies~\cite{abowd2000chartingubicomp} 
Lack of evaluations in context of everyday AR information access~\cite{lu2023inthewild}

Survey of existing studies
% Longitudinal headset research

Biener observed participants can overcome initial negative impressions and discomfort~\cite{biener2022vrweek}


In-the-wild evaluation refers to experiments with real users in uncontrolled environments \cite{brown2011wild}

Rather than completing arbitrary tasks decided by the researcher, participants can understand and appropriate the technology to their needs~\cite{rogers2011wild} 

Results from in-the-wild evaluations may differ from laboratory settings, considering unpredictability of people, as people can become distracted, be interrupted, or interact with others~\cite{rogers2011wild} 

In the wild AR/VR studies shown viability and data quality in diverse areas including education~\cite{petersen2021pedagogical}
accessibility~\cite{schmelter2023accessible}
recreational sports~\cite{colley2015blended,colley2017ubimount} 
locomotion~\cite{ragozin2020mazerunvr}
military training~\cite{laviola2015military}
perception studies~\cite{arora2021thinkingoutsidelabvr}

Limited exploration of how AR HWDs could better facilitate everyday workflows. Pavanatto claims to be the first to address this gap, exploring the use of AR HWDs to facilitate knowledge work in real-world settings \cite{pavanatto2024xrwild}

To better understand and observe all effects of using VR, advisable to conduct studies with a longer duration~\cite{biener2024holdtight}

There are several studies that looked at the prolonged use of VR or AR devices~\cite{biener2024holdtight}

Steinicke and Bruder conducted a self-experiment
where one participant spent 24 hours working, eating, sleeping, entertaining himself in VR~\cite{steinicke2014selfexperimentation}
Nordahl et al reported on two participants who used VR for 12 hours~\cite{nordahl2019hours}
Lu reported usage of glanceable AR prototype for three days, finds form factor primary barier~\cite{lu2021evaluatingglanceable} 
AR HMDs in order processing task for four hours, higher work efficiency but also higher visual discomfort~\cite{grubert2010mobileindustrialar} 
Guo et al~\cite{guo2020exploring,guo2019maslows}
and Shen et al~\cite{shen2019longtermfatigue} 
report on a study where 27 participants worked in a virtual and a physical office environment for eight hours each
Focused on emotional and physiological needs during short and long-term use of VR~\cite{guo2019maslows} 
and visual and physical discomfort~\cite{guo2020exploring} 
Weight and form factor resulted in higher physical discomfort in VR compared to physical condition 

Longest VR study reported by Biener et al~\cite{biener2022vrweek} 
compared working in VR to regular physical environment for five days each 
Range of measures
Biener et al examines the video data recorded during the study: insights into how participants use the HMD, how they interact with it and what problems they encounter~\cite{biener2024holdtight}
Better understanding of ergonomics of current 

Once people adapt to new systems and are no longer uncomfortable
with the novelty of the technology, scholars can gain a more
accurate picture of how the system shapes people’s behaviors and
attitudes (see Hans et all~\cite{} for review)

Thad designed and wore a AR headset on a daily basis for a decade. Checked internet and took notes during face to face. AR as a research tool, also a tool that he relied on for his daily life~\cite{} %Stevens, Wearable-technology pioneer Thad Starner on how Google Glass could augment our realities and memories, Engadget, 2013

Mann similarly augmented his vision with computing for decades. 
Received backlash and was assaulted~\cite{} % Buchanan, Glass before Google, The New Yorker, 2013
Physiological aftereffects --> needed a wheelchair after removal

Studying early adopters of MR allows scholars to gain insights into future use at scale~\cite{} % Foxman, Playing with virtual reality: Early adopters of commercial immersive technology, 2018 

Steinecker and Bruder 24-hr study with breaks, noting simulator sickness and presence did not
diminish over time, with extensive movement contributing to sickness~\cite{steinicke2014selfexperimentation}
Nordahl et al exposed two participants to 12 hr of VR use, inconsistent simulator sickness but a notable spike after 7 hours~\cite{nordahl2019hours}
Duzmanska et al 2018 found persistence of symptoms after leaving VR varied from 10 min to 4 hours~\cite{duzmanska2018simulatorsickness} 

Researchers also found
that visual fatigue symptoms, objective pupil size, and relative
accommodation responses varied over time when participants were
exposed to VR for 8 hr~\cite{guo2020exploring}

Explored effects of VR usage over time on depth perception, body offsets, social interactions~\cite{bailenson2024seeing}
Kohm, Babu, et al studied VR use over 12 weeks and its effects on depth perception and demonstrated adaptaiton~\cite{kohm2022objects} % Kohm Babu 2022
Kohm, Porter, Robb showed participants became more effective and object manipulation using proprioception rather than visual perception over 4 weeks~\cite{kohm2022handoffset}
Bailenson and Yee found increased task performance abilities and decreased simulator sickness over 10 weeks~\cite{bailenson2006longitudinal}
Hans et al showed that over a 8-week course on VR use, group cohesion, presence, enjoyment, and realism increased significantly over time, and the data measured during first week not representative of final usage patterns~\cite{han2023longitudinal} 


\subsection{Diary Studies}
% 

\subsection{Image Analysis Studies}

\subsection{Pervasive and Everyday AR}
%%% PERVASIVE AR
AR historically used to solve domain-specific issues~\cite{pavanatto2024xrwild, collins2014surgery, feiner1997touring, henderson2009maintenance, schall2008redlining, thomas1998navigation}
Recent research explored pervasive and everyday AR~\cite{pavanatto2024xrwild, bellgardt2017vrwork, grubert2017pervasive,lu2020glanceable} 
Grubert introduced pervasive AR, emphasizing general-purpose, all-day use of AR as a continuous, universal augmented interface~\cite{pavanatto2024xrwild, gruber2018officefuture} 
Continuously benefit from AR glasses for information acquisition, productivity, entertainment~\cite{pavanatto2024xrwild}

%%% LIMITATIONS 
Form factor barrier to adoption: headsets induce stress, mental overload, visual and muscle fatigue, physical discomfort~\cite{pavanatto2024xrwild, guo2020exploring, souchet2023vrergonomics} 
Bierner et all found using VR headset resulted in lower ratings than physical environment for task load, usability, flow, frustration, visual fatigue~\cite{pavanatto2024xrwild, biener2022vrweek} 
Lu et al pointed out that participants could imagine using an AR system daily, if form factor of devices would be improved \cite{pavanatto2024xrwild, lu2021evaluatingglanceable} 

%%% PRODUCTIVITY
\cite{pavanatto2024xrwild} explores how AR devices can enhance everyday productivity and addresses the associated technological and adoption challenges 


\subsection{Studies of Physical Displays}
Larger screens benefit performance in cognitively difficult tasks~\cite{cetin2018visualanalyticslarge,czerwinski2003largedisplays} 
Enhanced performance attributed to physical navigation and maintaining overview of context~\cite{ball2005tiledhighresolutiondisplay} 
Location and appearance of content help keep users aware of the organization of the space as a type of external memory~\cite{andrews2010space,ball2007movetoimprove} 
These findings indicate
the importance of having more screen real estate and accessing it using
body motion instead of window or desktop switching.

Introduces issues of losing track of mouse cursor, accessing distant information, dealing with bezels, managing windows and tasks in extra and distant available space~\cite{endert2012lhrd,robertson2005largedisplay} 
Cursor awareness enhanced with showing temporal cursor trail~\cite{czerwinski2003largedisplays} 
or showing animated circle around cursor~\cite{robertson2005largedisplay} 
Deciding where to place a new window, how quickly to move it, and how to organize a space with a large number of windows are window management issues~\cite{robertson2005largedisplay} 

Multi-monitor displays, sometimes called multi-display systems~\cite{ni2006lhdrsurvey} 
common approach to obtaining large displays
Enable tasks that are complex and divided between multiple windows~\cite{pavanatto2024multiplemonitors}, 
provide peripheral awareness~\cite{grudin2001digitalworlds},
and increase immersion~\cite{bi2009comparing} 
Larger number of windows~\cite{endert2012lhrd} 
efficient multitasking~\cite{czerwinski2003largedisplays}

Usage patterns 
Advantages
Disadvantages 
Takeaways 


Validated feasibility of performing knowledge work on virtual displays with current technology~\cite{pavanatto2024xrwild}

Virtual displays use a head-worn display to render information and applications from a personal computer without being constrained to physical monitors, offering more flexibility, portability, and scalability~\cite{pavanatto2024xrwild} %69 

Especially advantageous in mobile scenarios, where work environments are less optimal, such as crowded spaces with many distractions, or in confined spaces that limit the size of physical hardware~\cite{biener2024xrworkpublic}, such as in public transportation or at a coffee shop  
Worn all-day, overlaying information anywhere and anytime~\cite{lu2020glanceable,grubert2017pervasive}

Explored in recent years as a tool for knowledge work~\cite{biener2024holdtight,biener2022povrpoint,biener2020breakingscreen,ruvimova2020transportaway} %4, 6, 46

Enhances interactivity \cite{biener2022povrpoint}
utilizing large virtual displays \cite{biener2020breakingscreen,mcgill2020seatedvrworkspace} 

Meta sold approximately 20 million MR headsets~\cite{lang2023meta}
Apple Vision Pro "designed for all-day use" ~\cite{apple2023introducingvp} 

Existing studies predominantly focused on laboratory-based experiments, not considering the unique and ecologically valid public settings where laptops are frequently used~\cite{pavanatto2024xrwild}
Abowd and Mynatt~\cite{abowd2000chartingubicomp,pavanatto2024xrwild} highlighted that everyday computing activities could differ significantly from controlled evaluations in the
laboratory, which rarely have clear beginnings/ends, often experience
interruptions, and typically involve multiple concurrent activities. Thus,
in-the-wild evaluations are crucial to understanding users’ subjective
experiences, including their emotional state and social interactions with
people around them

Promises usage anywhere, all the time
In contrast to contexts of intended usage, current research on AR is mostly conducted in controlled laboratory environments. 
limited exposure time 
Need to understand experience and usage in the wild 
Not explore sufficiently 

We report results of study
How users use virtual displays and their experience of using virtual displays in real world settings 
The results gathered through 
Main findings

We report on a study with users working on an AR laptop for sessions (total of ).
We employ a diary study approach to capture their experiences and usage behaviors

The main findings of our study are as follows: 

In summary, this paper presents the following contributions: 
1) a diary study on how people work with an AR laptop for 



\section{Usage Characteristics}

\section{Arrangement Patterns}
General usage characterization
Tendencies
Variable across participants 
Notable patterns 
Adjustments 
Over time 

\section{Arrangement Considerations}
Task / usage frequency / interaction
Posture
Environment
Social

\section{Hybrid Usage}
Complementing physical displays 
Usage simultaneously with physical tasks 

\section{Utility}
Improved ergonomics 
Portability 
Efficient / productive multi-screen workflows
In situ 
Immersive 
Privacy

\section{Challenges}
Hardware and software limitations 
Influenced usage behaviors 
Interaction limitations 
Navigating infinite display 
Lack of structure 
Taking advantage of new medium 
Balancing visibility of physical and virtual elements

We first summarize the insights we gained and how they expand there knowledge we had from prior studies on the topic. 
We then reflect on how 
Finally we discuss limitations of our study. 

\cite{bailenson2024seeing}
We are confident that tech companies will continually improve
many of the technical problems raised in this article, such as
distortion, low eld of view, and increased latency. For example, the
Apple Vision Pro, expected to reach consumers in February 2024,
has improved upon some of the problematic features we discussed in
relation to the Meta Quest 3. But it is going to take time, and even
then, no headset will be perfect.
Will be used in the meantime nonetheless

Hardware / software: don't need perfection 
Interaction: beyond window management towards task maangement
Overcoming legacy bias: leverage familiar aspects + introduce new features 
Adaptive systems: leverage contextual features for assisting users (without overpowering them)
Towards productivity and beyond 
