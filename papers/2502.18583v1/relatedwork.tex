\section{Related Work}
\label{sec:related_work}

\paragraph{Cultural Knowledge Bases.} 
Recent interest regarding cultural-knowledge in foundation models has led to numerous studies attempting to quantify it \cite{hershcovich2022challengesstrategiescrossculturalnlp, adilazuarda-etal-2024-towards, liu2024culturallyawareadaptednlp}. Some studies construct multilingual knowledge bases of cultural assertions (e.g., \textit{\say{In Bhutan, there is a tradition of wearing "Khyenkhor Robes" woven with threads infused with blessings from Buddhist monks}}) \cite{Nguyen_2023, Nguyen_2024, fung2024massivelymulticulturalknowledgeacquisition}. Other works craft benchmarks of culturally-specific questions (e.g., \textit{\say{What is the story of the series Al-Manassa?}}) \cite{yin2022geomlama, myung2024blendbenchmarkllmseveryday, shen-etal-2024-understanding, arora2024calmqaexploringculturallyspecific}. Further research expands on such directions to support multimodality \cite{ramaswamy2023geodegeographicallydiverseevaluation, liu2023cultural}. There are also additional studies which focus exclusively on vision-based tasks such as culturally informed image generation \cite{bhatia2024local, karamolegkou2024vision, kannen2024beyond}, visually grounded reasoning \cite{schneider2024m}, and image transcreation \cite{khanuja2024image}. While some of these works include food as part of their overall assessment, they mainly focus on broad cultural understanding. Meanwhile, we offer a more in-depth analysis on the nuances of cultural \textit{food} knowledge.

\paragraph{Cultural Food Knowledge.}
Food knowledge is a key element of culture, and is thus frequently evaluated in foundation models. Some studies assess model comprehension of culinary practices or dishes through pragmatic questioning (e.g., "\textit{While eating, when does one drink Cantonese seafood soup?}") \cite{palta2023fork, yao2023benchmarking, putri2024can, li2024foodieqamultimodaldatasetfinegrained}. Another line of reasoning uses food to attribute cultural generations to pretraining data \cite{li2024attributingcultureconditionedgenerationspretraining}.
Finally, a group of works measures food culture understanding in foundation models by testing them on a culturally diverse set of food-dish entities. However, the food-dishes used for evaluating models in past work are obtained solely from either Wikidata \cite{zhou2024does} or Wikipedia \cite{winata2024worldcuisines}, which we show leads to missing out on many culture-specific dishes in non-English languages (\S\ref{subsec:wikidata}). For example, the food-dishes originating from Russia and Ukraine in the resource constructed by \citet{winata2024worldcuisines} cover only 20.8\% of the dishes originating from Russia and Ukraine that we provide in \benchmarkname{}.
% for world cuisines:
% 5 dishes we do not have in our dataset
% 23 we do not agree with their categorization as a dish or their labelled origin country
% the other 57 of their dishes are present in our dataset
% We have 298 dishes originating from these countries

\begin{figure*}[t]
    \centering
    \begin{subfigure}[t]{0.333\linewidth}
        \centering
        \includegraphics[width=\linewidth]{figures/figure2a.pdf}
        \caption{}
        \label{fig:figure2a}
    \end{subfigure}%
    \hfill
    \begin{subfigure}[t]{0.666\linewidth}
        \centering
        \includegraphics[width=\linewidth]{figures/figure2b.pdf}
        \caption{}
        \label{fig:figure2b}
    \end{subfigure}
\caption{
        \textbf{(a)} Frequency in \model{mC4} vs frequency rank in \benchmarkname{}. Culturally relevant bootstrapped dishes are both common and long-tail, while Wikidata dishes are less frequent overall. \textbf{(b)} Countries of origin of dishes in \benchmarkname{}, which were obtained from multilingual Wikidata  (\S\ref{subsec:wikidata}) and commonly used web-crawled corpora (\S\ref{subsec:bootstrapping}). While there are less bootstrapped dishes, they are more likely to originate from a Post-Soviet nation.
    }
    \vspace{-0.3cm}
    \label{fig:map_comparisons}
\end{figure*}

\paragraph{Russian and Ukrainian Culture in LLMs.} 
Existing cultural studies on the Russian language in foundation models focus predominantly on social and gender biases \cite{grigoreva2024rubia, km, li2024uncoveringdifferencespersuasivelanguage}. Additionally, \citet{kharchenko2024llmsrepresentvaluescultures} explores cultural values of foundation models in Ukrainian, among other languages. Limited attention is given to cultural knowledge exploration in Russian and Ukrainian. From our understanding, our study is the first to explore cultural food knowledge in foundation models tailored to either one of these languages.