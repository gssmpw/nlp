
\documentclass{article} % For LaTeX2e

\usepackage[dvipsnames]{xcolor}
\usepackage[accepted]{arxiv}

%%%%% NEW MATH DEFINITIONS %%%%%

% \usepackage{amsmath,amsfonts,bm}
\usepackage{amsmath,amsfonts}

\usepackage{pifont}


\newcommand{\R}{\mathbb{R}}


\def\va{{\mathbf{a}}}
\def\vg{{\mathbf{g}}}

% Sets
\def\sR{\mathbb{R}}
\def\sC{\mathbb{C}}
\def\sZ{\mathbb{Z}}
\def\sN{\mathbb{N}}
\def\sQ{\mathbb{Q}}

\def\sS{\mathcal{S}}



% Vectors
\def\vzero{{\mathbf{0}}}
\def\vone{{\mathbf{1}}}
\def\vmu{{\mathbf{\mu}}}
\def\vtheta{{\mathbf{\theta}}}
\def\va{{\mathbf{a}}}
\def\vb{{\mathbf{b}}}
\def\vc{{\mathbf{c}}}
\def\vd{{\mathbf{d}}}
\def\ve{{\mathbf{e}}}
\def\vf{{\mathbf{f}}}
\def\vg{{\mathbf{g}}}
\def\vh{{\mathbf{h}}}
\def\vi{{\mathbf{i}}}
\def\vj{{\mathbf{j}}}
\def\vk{{\mathbf{k}}}
\def\vl{{\mathbf{l}}}
\def\vm{{\mathbf{m}}}
\def\vn{{\mathbf{n}}}
\def\vo{{\mathbf{o}}}
\def\vp{{\mathbf{p}}}
\def\vq{{\mathbf{q}}}
\def\vr{{\mathbf{r}}}
\def\vs{{\mathbf{s}}}
\def\vt{{\mathbf{t}}}
\def\vu{{\mathbf{u}}}
\def\vv{{\mathbf{v}}}
\def\vw{{\mathbf{w}}}
\def\vx{{\mathbf{x}}}
\def\vy{{\mathbf{y}}}
\def\vz{{\mathbf{z}}}
\def\vzeta{{\mathbf{\zeta}}}

% Matrix
\def\mA{{\mathbf{A}}}
\def\mB{{\mathbf{B}}}
\def\mC{{\mathbf{C}}}
\def\mD{{\mathbf{D}}}
\def\mE{{\mathbf{E}}}
\def\mF{{\mathbf{F}}}
\def\mG{{\mathbf{G}}}
\def\mH{{\mathbf{H}}}
\def\mI{{\mathbf{I}}}
\def\mJ{{\mathbf{J}}}
\def\mK{{\mathbf{K}}}
\def\mL{{\mathbf{L}}}
\def\mM{{\mathbf{M}}}
\def\mN{{\mathbf{N}}}
\def\mO{{\mathbf{O}}}
\def\mP{{\mathbf{P}}}
\def\mQ{{\mathbf{Q}}}
\def\mR{{\mathbf{R}}}
\def\mS{{\mathbf{S}}}
\def\mT{{\mathbf{T}}}
\def\mU{{\mathbf{U}}}
\def\mV{{\mathbf{V}}}
\def\mW{{\mathbf{W}}}
\def\mX{{\mathbf{X}}}
\def\mY{{\mathbf{Y}}}
\def\mZ{{\mathbf{Z}}}
\def\mBeta{{\mathbf{\beta}}}
\def\mPhi{{\mathbf{\Phi}}}
\def\mLambda{{\mathbf{\Lambda}}}
\def\mSigma{{\mathbf{\Sigma}}}


% Expectation
% \def\eE{\mathop{\mathbb{E}}\limits}
\def\eE{\mathbb{E}}

% Probability
\def\pP{\mathbb{P}}

% Tilde
\def\tf{\tilde{f}}
\def\tS{\tilde{S}}
\def\wtF{\widetilde{\mathcal{F}}}
\def\whR{\widehat{R}}
\def\tvx{\tilde{\mathbf{x}}}
\def\ty{\tilde{y}}


\def\defeq{\overset{\textup{def}}{=}}
% \def\defeq{\overset{.}{=}}
\def\defone{\overset{\text{\ding{172}}}{=}}
\def\deftwo{\overset{\text{\ding{173}}}{=}}
\def\leqone{\overset{\text{\ding{172}}}{\leq}}
\def\leqtwo{\overset{\text{\ding{173}}}{\leq}}
\def\leqthree{\overset{\text{\ding{174}}}{\leq}}
\def\leqfour{\overset{\text{\ding{175}}}{\leq}}
\def\eqone{\overset{\text{\ding{172}}}{=}}
\def\eqtwo{\overset{\text{\ding{173}}}{=}}
\def\eqthree{\overset{\text{\ding{174}}}{=}}
\def\eqfour{\overset{\text{\ding{175}}}{=}}
\def\geqfive{\overset{\text{\ding{176}}}{\geq}}

\usepackage{hyperref}
\usepackage{url}
\usepackage{amsmath}
\usepackage{amsthm}
\usepackage{amsfonts}
\usepackage[capitalize,noabbrev]{cleveref}
\usepackage{graphicx}
\usepackage{booktabs}
\usepackage{dsfont}
\usepackage{subfigure}
\usepackage{thmtools}
\newtheorem{thm}{Theorem}
\newtheorem{lemma}[thm]{Lemma}
\newtheorem{proposition}[thm]{Proposition}
\newtheorem{cor}{Corollary}
\newtheorem{defn}{Definition}[section]
\newtheorem{conj}{Conjecture}[section]
\newtheorem{exmp}{Example}[section]
\newtheorem{assume}{Assumption}[section]
\newtheorem{rem}[thm]{Remark}
\newtheorem{note}{Note}
\newtheorem{case}{Case}

\usepackage[ruled,algo2e]{algorithm2e}






\theoremstyle{definition}
\newtheorem{definition}{Definition}[section]
\usepackage{mathtools}

\begin{document}


\twocolumn[
\icmltitle{Almost Surely Safe Alignment of Large Language Models at Inference-Time}


\icmlsetsymbol{equal}{*}
\icmlsetsymbol{work-done}{\dag}

\begin{icmlauthorlist}
\icmlauthor{Xiaotong Ji}{equal,work-done,yyy,icl}
\icmlauthor{Shyam Sundhar Ramesh}{equal,work-done,yyy,ucl}
\icmlauthor{Matthieu Zimmer}{equal,yyy}
\icmlauthor{Ilija Bogunovic}{ucl}
\icmlauthor{Jun Wang}{ucl}
\icmlauthor{Haitham Bou Ammar}{yyy,ucl}

\end{icmlauthorlist}

\icmlaffiliation{yyy}{Huawei Noah's Ark Lab}
\icmlaffiliation{icl}{Imperial College London}
\icmlaffiliation{ucl}{University College London}


\icmlcorrespondingauthor{Haitham Bou Ammar}{haitham.ammar@huawei.com}

\icmlkeywords{Machine Learning, ICML}

\vskip 0.3in
]

\printAffiliationsOnly{\icmlEqualContribution \icmlworkdone}

\begin{abstract}
Even highly capable large language models (LLMs) can produce biased or unsafe responses, and alignment techniques, such as RLHF, aimed at mitigating this issue, are expensive and prone to overfitting as they retrain the LLM. This paper introduces a novel inference-time alignment approach that ensures LLMs generate safe responses almost surely, i.e., with a probability approaching one. We achieve this by framing the safe generation of inference-time responses as a constrained Markov decision process within the LLM's latent space. Crucially, we augment a safety state that tracks the evolution of safety constraints and enables us to demonstrate formal safety guarantees upon solving the MDP in the latent space. Building on this foundation,  we propose 
\texttt{InferenceGuard}, a practical implementation that safely aligns LLMs without modifying the model weights. Empirically, we demonstrate \texttt{InferenceGuard} effectively balances safety and task performance, outperforming existing inference-time alignment methods in generating safe and aligned responses. 

\textcolor{red}{ 
 Contains potentially harmful examples.}
\end{abstract}

\section{Introduction}
LLMs have demonstrated impressive capabilities across a diverse set of tasks, such as summarization \citep{koh2022empirical,stiennon2020learning}, code generation \citep{ gao2023pal,chen2021evaluating}, and embodied robotics \citep{mower2024rosllmrosframeworkembodied,Kim_2024}. However, since those models are primarily trained on vast, unsupervised datasets, their generated responses can often be biased, inaccurate, or harmful \citep{deshpande2023toxicity,ganguli2022red, weidinger2021ethical, gehman2020realtoxicityprompts}. As a result, LLMs require alignment with human values and intentions to ensure their outputs are free from controversial content.

The traditional LLM alignment approach involves fine-tuning the model on human-labeled preference data. For instance, works such as \citep{ouyang2022training, christiano2017deep} use the reinforcement learning from human feedback (RLHF) framework to fine-tune LLMs, constructing a reward model from human feedback and optimizing the model using standard reinforcement learning algorithms like PPO \citep{schulman2017proximalpolicyoptimizationalgorithms}. More recent approaches, such as \citep{tutnov2025dpossecretlyoneattempting,yin2024relativepreferenceoptimizationenhancing,rafailov2023direct}, bypass reward learning and instead align pre-trained models directly with human preferences.

RLHF alignment can be costly and risks overfitting partly since they modify the LLM's model weights. To tackle these challenges, alternative approaches adjust the model's responses directly at inference time while leaving the pre-trained model weights fixed. Several techniques have been proposed for this purpose, such as Best-of-N \citep{nakano2021webgpt,stiennon2020learning,touvron2023llama}, FUDGE \citep{yang2021fudge}, COLD \citep{qin2022cold}, CD-Q \citet{mudgal2023controlled}, RE-control \cite{kong2024aligning}, among others. Importantly, those techniques are designed modularly, allowing the alignment module to integrate seamlessly with the pre-trained model. This modularity enables flexible inference-time reconfigurability and quick adaptation to new reward models and datasets. Moreover, it reduces reliance on resource-intensive and often hard-to-stabilize RL processes inherent in the RLHF paradigm.\looseness=-1 

Despite the successes of inference-time alignment, these methods' \emph{safety aspects} have received limited attention so far. While some works have attempted to tackle those issues in inference-time-generated responses, they mainly focus on prompt-based alignment ~\citep{hua2024trustagent,zhang2024controllable}, trainable safety classifiers ~\cite{niu2024parameter,zeng2024root} or protections against adversarial attacks and jailbreaks ~\citep{dong2024attacksdefensesevaluationsllm,guo2024cold,inan2023llama}. That said, prompt-based methods cannot be guaranteed to consistently produce safe responses, as ensuring safety is heavily reliant on user intervention, requiring extensive engineering and expertise to manage the model’s output effectively. Trainable classifiers focus only on the safety of decoded responses using hidden states or virtual tokens, ignoring task alignment and lacking theoretical guarantees. Moreover, while adversarial robustness is crucial, our work focuses on the key challenge of generating inherently safe responses from the LLM.\looseness=-1


In this work, we aim to develop LLMs that generate safe responses at inference time, following a principled approach that \emph{guarantees safety almost surely, i.e., with a probability approaching one}. To do so, we reformulate the safe generation of inference-time responses as an instance of constrained Markov decision processes (cMDP). We map the cMDP to an unconstrained one through \emph{safety state augmentation}, bypassing Lagrangian approaches' limitations that struggle to balance reward maximization and safety feasibility. Focusing on a practical test-time inference algorithm, we adopt a critic-based approach to solve the augmented MDP, eliminating the need for gradients in the LLM. To ensure efficiency, we train our critic in the latent space of the LLM, keeping it small in size and fast during inference. This shift to the latent space complicates the theoretical framework, requiring extensions from \citep{hernandez1992discrete, sootla2022saute}. By doing so, we establish, for the first time, that one can guarantee almost sure safety in the original token space.

To leverage this theoretical guarantee in practice, we build upon the augmented MDP framework and introduce two novel implementations for safe inference-time alignment: one that learns a compact critic in the latent space for cases where safety costs can only be queried after the generation of complete responses and another that leverages direct cost queries for efficient inference-time optimization. Finally, we integrate these components into a lookahead algorithm (e.g., Beam Search or Blockwise Decoding \citep{mudgal2023controlled}) proposing \texttt{InferenceGuard}. Our experiments tested \texttt{InferenceGuard} starting from safety-aligned and \emph{unaligned models}. Our results achieved high safety rates—91.04\% on Alpaca-7B and 100\% on Beaver-7B-v3. Notably, this was accomplished while maintaining a strong balance with rewards, setting new state-of-the-art.




\section{Background}
\subsection{LLMs as Stochastic Dynamical Systems} \label{Sec:DynaSys}
LLMs can be viewed as stochastic dynamical systems, where the model's behavior evolves probabilistically over time, governed by its internal parameters and the inputs it receives. In this perspective, each new token is generated based on the model's evolving hidden state \citep{kong2024aligning, zimmer2024mixtureattentionsspeculativedecoding}. Formally, an LLM transitions as follows: $\left[\mathbf{h}_{t+1}, \mathbf{o}_{t+1}\right]^{\mathsf{T}} = f_{\text{LLM}}(\mathbf{h}_t, \text{y}_t)$, \text{with} $\text{y}_t \sim \text{SoftMax}(\mathbf{W}\mathbf{o}_t)$. Here, $f_{\text{LLM}}(\cdot)$, denotes the aggregation of all decoding layers, $\text{y}_t$ a generated token at each time step $t$, and $\mathbf{o}_t$ the logits which are linearly mapped by $\mathbf{W}$ to produce a probability distribution over the vocabulary space. Moreover, $\mathbf{h}_t$ comprises all key-value pairs accumulated from previous time steps \footnote{Note, $\mathbf{h}_{t} = \left\{\mathbf{K}^{(l)}_{j},\mathbf{V}_{j}^{(l)}\right\}_{l=1}^{L}$ for $j:[1:t]$, i.e., keys and values from all layers accumulated up to the previous time step.}. The system evolves until the end-of-sequence (\texttt{EOS}) token is reached.  
\subsection{Test-Time Alignment of LLMs} \label{Sec:TTT}
Test-time alignment ensures that a pre-trained LLM generates outputs consistent with desired behaviors by solving a MDP, whose initial state is determined by the test prompt. Rewards/costs for alignment can come from various sources, including but not limited to human feedback \citep{tutnov2025dpossecretlyoneattempting,zhong2024dpo}, environmental feedback from the task or verifiers \citep{zeng2023large,yang2024leandojo,trinh2024solving,an2024learn,liang2024learning,mower2024rosllmrosframeworkembodied}, or pre-trained reward models \citep{wang2024math,zhang2024rest,li2025enhancing}. 

Several approaches have been developed to address the challenge of test-time alignment. For instance, beam search with rewards \citep{choo2022simulation} extends traditional beam search technique by integrating a reward signal to guide LLM decoding at test time. Monte Carlo tree search, on the other hand, takes a more exploratory approach by simulating potential future token sequences to find the path that maximizes the reward function \citep{zhang2024rest}. Best-of-N (BoN) generates multiple candidate sequences and selects the one with the highest reward ~\citep{stiennon2020learning,nakano2021webgpt,touvron2023llama}. 

We focus on beam search and look-ahead methods for more scalability when developing \texttt{InferenceGuard}.

\begin{figure*}[hbt!]
\centering

\includegraphics[trim=2em 0.1em 2em 2em,clip=true,width=\textwidth]{imgs/fig_1.pdf} 
\label{fig:flow}
\vspace{-1.2em}
\caption{To aid readability, this figure summarizes the key transitions and notations used throughout the paper. }
\vspace{-1.2em}
\end{figure*}

\section{Safe Test-Time Alignment of LLMs }
We frame the problem of safely aligning LLMs at test time as a constrained Markov decision process (cMDP). As noted in \cite{achiam2017constrained,sootla2022saute} a cMDP is defined as the following tuple: $\mathcal{M} =\left\langle \mathcal{S}, \mathcal{A}, \mathcal{C}_{\text{task}}, \mathcal{C}_{\text{safety}}, \mathcal{P}, \gamma  \right\rangle$, with $\mathcal{S}$ and $\mathcal{A}$ denoting the state and action spaces, respectively. The cost function $\mathcal{C}_{\text{task}}: \mathcal{S} \times \mathcal{A} \rightarrow \mathbb{R}$ dictates the task's cost\footnote{We define costs as negative rewards, which transforms the problem into an equivalent cost-based MDP.}, while $\mathcal{C}_{\text{safety}}: \mathcal{S} \times \mathcal{A} \rightarrow \mathbb{R}$ represents a \emph{safety} cost, which encodes the constraints that the actor must satisfy during inference. The transition model $\mathcal{P}:\mathcal{S} \times \mathcal{A} \times \mathcal{S} \rightarrow [0,1]$ captures the probability of transitioning to a new state given the current state and action. Meanwhile, the discount factor $\gamma \in [0,1)$ trades off immediate versus long-term rewards. The goal of constrained MDPs is to find a policy $\pi: \mathcal{S} \times \mathcal{A} \rightarrow [0,1]$ that minimizes the task's cost while simultaneously satisfying the safety constraints. Given a safety budget d, we write: 
\begin{align}
    \label{Eq:cMDPs} 
    \min_{\pi} \  &\mathbb{E}_{\mathcal{P},\pi}\Big[\sum_{t}\gamma^t\mathcal{C}_{\text{task}}(\mathbf{s}_t, \mathbf{a}_t)\Big]  \\ \nonumber
   \text{s.t.} \ \     &\mathbb{E}_{\mathcal{P},\pi}\Big[\sum_{t}\gamma^t\mathcal{C}_{\text{safety}}(\mathbf{s}_t, \mathbf{a}_t)\Big] \leq d,
\end{align}


\subsection{Safe Test-Time Alignment as cMDPs}
We treat the generation process of safe test-time alignment of LLMs as the solution to a specific cMDP. We introduce our state variable $\mathbf{s}_{t} = \{\mathbf{x},\mathbf{y}_{<t}\}$, which combines the input prompt $\mathbf{x}$ with the tokens (or partial responses) decoded until step $t$. Our policy generates a new token $\text{y}_t$ that we treat as an action in the model's decision-making process. The transition function 
$\mathcal{P}$ of our MDP is deterministic, where the state $\mathbf{s}_t$ is updated by incorporating the generated action $\text{y}_t$, i.e., $\mathbf{s}_{t+1} = \mathbf{s}_t \oplus \text{y}_{t}=\{\mathbf{x},\mathbf{y}_{\leq t}\}$. We also assume the existence of two cost functions $\mathcal{C}_{\text{task}}$ and $\mathcal{C}_{\text{safety}}$ to assess the correctness and safety of the LLM's responses. As described in Section \ref{Sec:TTT}, those functions can originate from various sources, such as human feedback, environmental verifiers, or pre-trained models. While we conduct experiments with these functions being LLMs (see Section \ref{Sec:Exp}), our method can be equally applied across various types of task and safety signals. 

Regarding the task's cost, we assume the availability of a function $\mathcal{C}_{\text{task}}$ that evaluates the alignment of the LLM with the given task. This function assigns costs to the partial response based on the input prompt $\mathbf{x}$ such that: 
\vspace{-0.1em}
\begin{equation}\label{eq: non-safe reward-21}
\mathcal{C}_{\text{task}}([\mathbf{x}, \mathbf{y}_{\leq t}]) :=
\begin{cases} 
0 & \text{if } \text{y}_t \neq \texttt{EOS} \\
c_{\text{task}}([\mathbf{x}, \mathbf{y}_{\leq t}]) & \text{if } \text{y}_t = \texttt{EOS}
\end{cases}
\end{equation}

For the safety cost $\mathcal{C}_{\text{safety}}$, we assume the function assigns non-zero costs to any partial answer without waiting for the final token. This is crucial because we want to flag unsafe responses early rather than waiting until the end of the generation process. Many pre-trained models are available for this purpose on Hugging Face, which we can leverage—more details can be found in Section \ref{Sec:Exp}. With this, we write safe test-time alignment as an instance of Equation \ref{Eq:cMDPs}: 
\vspace{-1em}
\begin{align}
\label{Eq:SafeLLM}
    \min_{\pi} \ &\mathbb{E}_{\pi}\Big[\sum_t \gamma^t \mathcal{C}_{\text{task}}(\overbrace{\{\mathbf{x},\mathbf{y}_{<t}\}}^{\mathbf{s}_t}, \text{y}_t)\Big] \\ \nonumber 
     \ \text{s.t.} \ &\mathbb{E}_{\pi}\Big[\sum_t \gamma^t \mathcal{C}_{\text{safe}}(\{\mathbf{x}, \mathbf{y}_{<t}\}, \text{y}_t)\Big] \leq d.
\end{align}
The above objective aims to minimize the task cost $\mathcal{C}_{\text{task}}$ while ensuring that the safety cost does not exceed a predefined budget $d$. The expectation is taken over the actions (tokens) generated at each timestep. 




\subsection{State Augmented Safe Inference-Time Alignment}\label{sec: augment}
We could technically use off-the-shelf algorithms to solve Equation \ref{Eq:SafeLLM}, such as applying a Lagrangian approach as proposed in \cite{dai2023safe}. However, there are two main issues with using these standard algorithms. First, they generally require gradients in the model itself—specifically, the LLM—which we want to avoid since our goal is to perform inference-time alignment without retraining the model. Second, these methods rely on a tunable Lagrangian multiplier, making it challenging to maximize rewards while satisfying almost sure constraints optimally. 


Instead of using a Lagrangian approach, we take a different direction by augmenting the state space and extending the method proposed by \citep{sootla2022saute} to large language models. In our approach, we augment the state space of the constrained MDP with a ``constraint tracker'', effectively transforming the problem into an unconstrained one. This allows us to apply Bellman equations and conduct rigorous proofs with almost sure constraint satisfaction results. However, applying the techniques and proofs from \citep{sootla2022saute} to our test-time setting is not entirely straightforward due to two main challenges: first, the differences in the constrained MDP setting, and second, the process by which we train critics, as we will demonstrate next.


\textbf{Augmenting the State Space.} 
The following exposition builds on \citep{sootla2022saute}, extending their method to address LLM-specific challengers, an area they did not cover. The core idea is to transform the constrained MDP into an unconstrained one by augmenting the state with an additional variable that tracks the remaining budget of the constraint. While doing so, we must ensure that: \textbf{PI)} our augmented state variable tracks the constraints and maintains the \emph{Markovian} nature of transition dynamics; and \textbf{PII)} our task cost $\mathcal{C}_{\text{task}}$ accounts for this new state representation and is correctly computed w.r.t. the augmented state.\looseness=-1


To solve \textbf{PI}, we investigate the evolution of the constraint in Equation \ref{Eq:SafeLLM} and track a scaled-version of the remaining safety budget $\mathbf{\omega}_t = d -\sum_{k=1}^{t} \gamma^{k}\mathcal{C}_{\text{safety}}(\{\mathbf{x},\mathbf{y}_{<k}\},\text{y}_k)$ that is defined as $\text{z}_t = \mathbf{\omega}_{t-1}/\gamma^{t}$. The update of $\text{z}_t$ satisfies: 
\begin{align}
\label{Eq:Z}
    \text{z}_{t+1} &= (\mathbf{\omega}_{t-1} -\gamma^t\mathcal{C}_{\text{safety}}(\{\mathbf{x},\mathbf{y}_{<t}\},\text{y}_t))/\gamma^{t+1} \\ \nonumber
    &=(\text{z}_t - \mathcal{C}_{\text{safety}}(\{\mathbf{x},\mathbf{y}_{<t}\},\text{y}_t))/\gamma,  \ \ \text{with $\text{z}_0 = d$.}
\end{align}
Since the dynamics of $\text{z}_t$ are Markovian due to the dependence of  $\text{z}_{t+1}$ only  on $\text{z}_t$, $\text{y}_t$ and current the state $\{\mathbf{x}, \mathbf{y}_{<t}\}$, we can easily augment our original state space with $\text{z}_t$, such that $\tilde{\mathbf{s}}_t=[\mathbf{s}_t, \text{z}_t]=[\{\mathbf{x}, \mathbf{y}_{<t}\}, \text{z}_t]$. The original dynamics can also be redefined to accommodate for $\tilde{\mathbf{s}}_t$: 
\begin{align*}
\tilde{\mathbf{s}}_{t+1} = [\overbrace{\{\mathbf{x},\mathbf{y}_{<t}\}\oplus \text{y}_t}^{\text{original transition}}, \text{z}_{t+1}], \ \ \text{with $\text{z}_{t+1}$ as in Eq. \ref{Eq:Z}.}   
\end{align*}

Concerning $\textbf{PII}$, we note that enforcing the original constraint in Equation \ref{Eq:SafeLLM} is equivalent to enforcing an infinite number of the following constraints: 
\begin{equation}
\label{Eq:infConst}
    \sum_{k=0}^{t} \gamma^{k} \mathcal{C}_{\text{safety}}(\{\mathbf{x},\mathbf{y}_{<k}\},\text{y}_k) \leq d \ \ \forall t \geq 1.
\end{equation}
As noted in \citep{sootla2022saute}, this observation holds when the instantaneous costs are nonnegative, ensuring that the accumulated safety cost cannot decrease. 
In our case, it is natural to assume that the costs are nonnegative for LLMs, as safety violations or misalignments in the output typically incur a penalty, reflecting the negative impact on the model's performance or ethical standards. 

Clearly, if we enforce  $\text{z}_{t} \geq 0$ for all $t \geq 
0$, we automatically get that $\mathbf{\omega}_t = d -\sum_{k=0}^{t} \gamma^{k}\mathcal{C}_{\text{safety}}(\{\mathbf{x},\mathbf{y}_{<k}\},\text{y}_k)\geq 0$ for all $t \geq 0$, thus satisfying the infinite  constraints in Equation \ref{Eq:infConst}. We can do so by reshaping the tasks's instantaneous cost to account for the safety constraints: 
\begin{equation}\label{eq: safe reward}
\tilde{\mathcal{C}}_{\text{task}}^{\infty}(\tilde{\mathbf{s}}_t, \text{y}_{t}) :=
\begin{cases} 
\mathcal{C}_{\text{task}}([\mathbf{x}, \mathbf{y_{\leq t}}]) &  \text{z}_{t} > 0\\
+\infty & \ \text{z}_{t}\leq 0,  
\end{cases}
\end{equation}
with $\mathcal{C}_{\text{task}}([\mathbf{x}, \mathbf{y_{\leq t}}])$ representing the original MDP's task cost function as described in Equation \ref{eq: non-safe reward-21}. Of course, in practice, we avoid working with infinities and replace $\tilde{\mathcal{C}}_{\text{task}}^{\infty}$ with $\tilde{\mathcal{C}}_{\text{task}}^{n}$ for a big $n>0$\footnote{Note that the introduction of $n$ instead of $+\infty$ requires additional theoretical justifications to ensure constraint satisfaction of the true augmented MDP. We carefully handle this in Section \ref{Sec:Theory}.}. We can now reformulate the constrained problem into an \emph{unconstrained one} as follows:

\begin{equation}
\label{Eq:Constraints}
    \min_{\pi} \mathbb{E}_{\pi}\Big[\sum_{t}\gamma^t \tilde{\mathcal{C}}_{\text{task}}^{\infty}(\tilde{\mathbf{s}}_t, \text{y}_{t})\Big].
\end{equation}
Using gradient-based techniques, one could optimize the augmented MDP in Equation \ref{Eq:Constraints}. However, since our goal is to enable safety at test time without retraining, we adopt a critic-based approach that does not require gradients during inference, as we show next. 

\section{\texttt{InferenceGuard}: Safety at Test-Time}
When designing our critic, we considered several crucial factors for test-time inference. These included its size, ease of training for quick adaptation, and flexibility to operate in real-time without significant latency. As such, we chose to train the critic in the latent space of the LLM rather than directly in the textual space, enabling a more efficient solution that meets the constraints of test-time alignment.

Even if we train the critic in the latent space, the question of what inputs to provide remains. Fortunately, the works of \citep{kong2024aligning, zimmer2024mixtureattentionsspeculativedecoding} demonstrated that LLMs can be viewed as dynamical systems, where $\mathbf{h}_t$ (hidden state) and $\mathbf{o}_t$ (logits) serve as state variables that capture sufficient statistics to predict the evolution of the LLM and the generation of new tokens; see Section \ref{Sec:DynaSys} for more details. Those results made $\mathbf{h}_t$ and $\mathbf{o}_t$ ideal inputs for our critic\footnote{In our implementation, we set the variable for the first input to our critic $\mathbf{h}_t = \texttt{llm-outputs}.\texttt{past-key-values}(\mathbf{x},\mathbf{y}_{<t})$ and $\mathbf{o}_t = \texttt{llm-outputs.hidden-states}(\mathbf{x},\mathbf{y}_{<t})\text{[-1]}$.}, as they encapsulate the relevant information for evaluating the model's behavior during test-time alignment while being relatively low-dimensional, reducing the size of our critic's deep network.

To fully define our critic, we require a representation of the embedding of our \emph{augmented state} $\tilde{\mathbf{s}}_{t}=[\mathbf{s}_t, \text{z}_t]$ within the latent space. As noted above, we can acquire $(\mathbf{h}_t, \mathbf{o}_t)$ from the transformer architecture. We call this mapping $\phi$, whereby $(\mathbf{h}_t, \mathbf{o}_t) = \phi(\{\mathbf{x},\mathbf{y}_{<t}\})$. Furthermore, we use an identity mapping to embed $\text{z}_t$, which enables us to input the actual tracking of the constraints directly to the critic without any loss of information. 


\subsection{Theoretical Insights} \label{Sec:Theory}
In this section, we show that optimizing in the latent space preserves safety constraints in the original token space, and we prove that our approach guarantees almost sure safety. 

We should answer two essential questions: \textit{i)} Can we still compute an optimal policy in the latent space? and \textit{ii)} If we enforce safety constraints in the latent space, do they still hold in the \emph{original token space?} While the prior work in \citep{sootla2022saute} established theoretical results for safety-augmented MDPs in standard (non-LLM) RL settings, their work does not address how guarantees in the latent space translate to the original token space. To handle those problems, we extend the theorems from \cite{hernandez1992discrete,sootla2022saute} to ensure the following three properties: 
\begin{itemize}
\setlength{\itemsep}{3pt}  
  \setlength{\parskip}{0pt}  
    \item \textbf{Prop I)} The latent MDP indeed satisfies the Bellman equations (Theorem \ref{thm:sauteequivalence} (a)) and, hence, allows us to compute an optimal policy in this space,
    \item \textbf{Prop II)} Policies and value functions in the latent space are valid in the original token space, implying that optimizing in the latent space preserves the constraints in original token space (Theorem \ref{thm:sauteequivalence} (b,c)) \looseness=-1
    \item \textbf{Prop III)} The resulting policy satisfies safety constraints almost surely (Theorem \ref{thm:a.s.}), meaning if a policy is safe in the latent and original token space with finite expected cost w.r.t. Equation \ref{Eq:Constraints}, it is also almost surely safe in the actual LLM token space.
\end{itemize}

We begin by defining the latent space MDP's cost and transition function: 

\begin{definition}\label{def: c-p}
	$\exists \phi(\cdot)$ and functions $\bar{\mathcal{C}}^{n}_{\text{task}}$ and $\bar{\mathcal{P}}$ such that: 
    \vspace{-0.1em}
    \begin{align*}
        &\bar{\mathcal{C}
        }_{\text{task}}^{n}(\overbracket{\phi(\{\mathbf{x},\mathbf{y}_{<t}\}),\text{z}_t}^{\text{embedded aug. state}},\text{y}_t)= \tilde{\mathcal{C}}^{n}_{\text{task}}(\overbracket{\{\mathbf{x},\mathbf{y}_{<t}\},\text{z}_t}^{\text{augmented state}}, \overset{\text{action}}{\overset{\big\uparrow}{\text{y}_t}})\\ 
        &\bar{\mathcal{P}}(\phi(\{\mathbf{x},\mathbf{y}_{\leq t}\}),\text{z}_{t+1}|\phi(\{\mathbf{x},\mathbf{y}_{<t}\}),\text{z}_t, \text{y}_t)= {\mathcal{P}}(\tilde{\mathbf{s}}_{t+1}|\tilde{\mathbf{s}}_{t}, \text{y}_t),
    \end{align*}
    
    where $\tilde{\mathbf{s}}_{t}$ is the augmented state in the original token space.\looseness=-1
\end{definition}



According to Definition \ref{def: c-p}, we ensure that the cost incurred by the augmented state $\tilde{\mathbf{s}}_{t}=[\{\mathbf{x},\mathbf{y}_{<t}\},\text{z}_t]$ w.r.t. $\tilde{\mathcal{C}}^{n}_{\text{task}}$ is equal to the latent cost incurred by the latent state $[\phi(\{\mathbf{x},\mathbf{y}_{<t}\}),\text{z}_t]$ w.r.t $\bar{\mathcal{C}}_{\text{task}}^{n}$. Additionally, we ensure that the transition dynamics of the augmented state in the original token space and the corresponding latent state in the latent space are equivalent.

This equivalence enables us to derive an optimal policy for the latent MDP and apply it to minimize the cost objective in the original augmented MDP (see Equation \ref{Eq:Constraints}). We proceed to analyze the existence of such an optimal policy in the latent space through the following \emph{standard assumptions} \citep{sootla2022saute} on $\bar{\mathcal{C}}^{n}_{\text{task}}$, and $\bar{\mathcal{P}}$: \textbf{A1.} The function $\bar{\mathcal{C}
        }_{\text{task}}^{n}(\mathbf{h},\mathbf{o},\text{z},\text{y})$ is bounded, measurable, nonnegative, lower semi-continuous w.r.t. $(\mathbf{h},\mathbf{o},\text{z})$ for a given $\text{y}$, and
\textbf{A2.} The transition law $\bar{\gP}$ is weakly continuous for any $\text{y}$.  

Next, we define $\bar{\pi}$ as a policy in the latent space that maps $(\mathbf{h},\mathbf{o},\text{z})\rightarrow \text{y}$, and its value function for an initial state $(\mathbf{h}_0,\mathbf{o}_0,\text{z}_0)$ as follows: $\bar{V}^{n}(\bar{\pi}, \mathbf{h}_0,\mathbf{o}_0,\text{z}_0) =   \E_{\bar{\pi}}\Big[\sum_{t=0}^\infty \gamma^t \bar{\mathcal{C}
        }_{\text{task}}^{n}(\mathbf{h}_t,\mathbf{o}_t,\text{z},\text{y})\Big]$. Then, one can define the optimal value function:
\begin{equation}\label{eq:latentvalue}
\bar{V}^{\star,n}(\mathbf{h},\mathbf{o},\text{z}) = \min_{\bar{\pi}} \bar{V}^{n}(\bar{\pi}, \mathbf{h},\mathbf{o},\text{z}).
\end{equation}


Since we cannot optimize directly in the original constrained MDP, we first show that solving for an optimal policy in the latent MDP preserves key properties of the original problem. The following theorem formalizes this by proving the existence of the optimal policy and its mapping to the original MDP. The proof is in Appendix \ref{sec: equi}.


\begin{restatable}{thm}{thmsauteequivalence}(Optimality in the Latent Space)\label{thm:sauteequivalence}
Given the latent MDP in Definition \ref{def: c-p} and with A1-A2, we can show: 
\begin{enumerate}
\setlength{\itemsep}{3pt}  
  \setlength{\parskip}{0pt} 
    \item[a)] \textbf{(Prop I)} For any finite $n$, the Bellman equation holds, i.e., there exists a function $\bar{V}^{\star,n}(\mathbf{h},\mathbf{o}, \text{z})$ such that:
    \begin{align*}
     &\bar{V}^{\star,n}(\mathbf{h},\mathbf{o}, \text{z}) = \min_{\text{y} \in \gV} \Big( \bar{\mathcal{C}}^{n}_{\text{task}}(\mathbf{h},\mathbf{o}, \text{z},\text{y}) \\
     &\hspace{13em}+ \gamma \bar{V}^{\star,n}(\mathbf{h}^{\prime},\mathbf{o}^{\prime}, \text{z}^{\prime}) \Big),\\
        &\text{such that} \ \ (\mathbf{h}^{\prime},\mathbf{o}^{\prime},\text{z}^{\prime})\sim \bar{\gP}(\cdot|\mathbf{h},\mathbf{o},\text{z},\text{y})
    \end{align*}
Furthermore, the optimal policy solving Equation \ref{eq:latentvalue} has the representation $y \sim \bar{\pi}^{\star,n}(\cdot \mid \mathbf{h},\mathbf{o}, \text{z})$;
    \item[b)] \textbf{(Prop II)} The optimal value functions $\bar{V}^{\star,n}$ converge monotonically to $\bar{V}^{\star,\infty}$.

    \item[c)] \textbf{(Prop II)} The optimal policy in the latent space $\bar{\pi}^{\star,n}$ is also optimal in the original token space if used as $\bar{\pi}^{\star,n}(\phi(\cdot))$, minimizing Equation \ref{Eq:Constraints}, even as $n\rightarrow \infty$.\looseness=-1
\end{enumerate}
\end{restatable}
The above theorem ensures that finding and securing the existence of the optimal policy in the latent space is sufficient to solve Equation \ref{Eq:Constraints} optimally\footnote{As noted in Section \ref{sec: augment}, we analyze the transition from a large $n$ to $\infty$, and confirm that the results hold even for $\tilde{\mathcal{C}}_{\text{task}}^{\infty}$.}. Informally, the latent space acts as a faithful representation, preserving constraints and making optimization computationally efficient. This implies that the optimal policies and value functions in the latent space remain valid in the original space.\looseness=-1




\textbf{Almost Sure Guarantee.}
Now, we derive \textbf{Prop III} that ensures the safety cost constraints are almost surely satisfied. This is more challenging than Equation \ref{Eq:SafeLLM}, where \emph{only the expected safety cost is constrained}:  \begin{align}\label{Eq:a.s.SafeLLM-main}
    \min_{\pi} \ &\mathbb{E}_{\pi}\Big[\sum_t \gamma^t \mathcal{C}_{\text{task}}(\{\mathbf{x},\mathbf{y}_{<t}\}, \text{y}_t)\Big] \\ \nonumber 
     \ \text{s.t.} \ & \sum_t \gamma^t \mathcal{C}_{\text{safe}}(\{\mathbf{x}, \mathbf{y}_{<t}\}, \text{y}_t) \leq d \quad \text{ \colorbox{BurntOrange}{almost surely.}} 
\end{align}

While the formulation in Equation \ref{Eq:a.s.SafeLLM-main} is ``stronger'' than Equation \ref{Eq:SafeLLM}, solving for the augmented MDP formulation with objective as Equation \ref{Eq:Constraints} can yield a policy satisfying the above almost sure constraints. We formally state this result in Theorem \ref{thm:a.s.} and relegate the proof to Appendix \ref{sec: equi}. \looseness=-1
\begin{restatable}{thm}{thmalmostsure}\label{thm:a.s.} (Almost Sure Safety)
Consider an augmented MDP with cost function $\tilde{\mathcal{C}}_{\text{task}}^{\infty}$. Suppose an optimal policy exists $\pi^\star$ solving Equation \ref{Eq:Constraints} (see Theorem \ref{thm:sauteequivalence}) with a finite cost, then $\pi^\star$ is an optimal policy for Equation \ref{Eq:a.s.SafeLLM-main}, i.e., $\pi^\star$ is safe with probability approaching one or almost surely. 
\end{restatable}

\subsection{Algorithm and Practical Implementation}
Building on our theoretical framework, we propose a search algorithm with two approaches: \textit{i)} pre-training a small latent-space critic for cases where costs are available only for complete trajectories and \textit{ii)} directly leveraging intermediate costs for search optimization.

\textbf{Training a Latent-Space Critic.} 
We make the usual assumption that trajectories terminate at a maximum length $T$. In this case, the value function simplifies to become: $\bar{V}^n(\mathbf{h}_t, \mathbf{o}_t, \text{z}_t) = \E_{\bar\pi}[\gamma^{T} \bar{c}
        _{\text{task}}({\mathbf{ h}}_T, {\mathbf o}_T)]$ if there is safety budget left, i.e., if $ \text{z}_T > 0$, or $n$ if $\text{z}_{T} \leq 0$, where $\bar{c}_{\text{task}}({\mathbf{ h}}_t, {\mathbf o}_t) = c_{\text{task}}([\mathbf{x}, \mathbf{y}_{\leq t}])$ in the latent MDP.
        
Hence, it is sufficient to predict: the sign of $\text{z}_{T}$ and the value of $\gamma^T \bar{c}_{\text{task}}({\mathbf h}_T, {\mathbf o}_T)$ to assess the quality of a state. 
We estimate those through Monte Carlo (MC) sampling.
Specifically, we generate multiple trajectories from the initial state ($\mathbf{h}_0, \mathbf{o}_0, \text{z}_0)$ using the reference policy, and compute the mean terminal cost, the sign of $z_T$ to serve as targets for the critic training.
The usual alternative to MC sampling is Temporal Difference (TD) learning, where the critic is updated based on the difference between the current estimate and a bootstrapped estimate from the next state. However, MC sampling offers two advantages: \textit{i)} it simplifies training by using separate supervised signals for quality and safety, unlike TD, which combines both, and \textit{ii)} it allows dynamic adjustment of $n$ without retraining.\looseness=-1 

We train a critic network with two heads by sampling responses from the base model and scoring them using the cost function. We define $\mathcal{J}_1$ as the binary cross-entropy for predicting the sign of $\text{z}_T$ and $ \mathcal{J}_2$ as the mean squared error for predicting $\gamma^T \bar{c}_{\text{task}}({\mathbf{h}}_T, {\mathbf{ o}}_T)$. Our critic training minimizes: $\mathcal{J}(\mathbf\theta) = \mathbb{E}_{\bar\pi} \Big[ \sum_{t=1}^{T} \mathcal{J}_1\left( f^1_{\mathbf\theta}({\mathbf h}_t, {\mathbf o}_t, \text{z}_t), \text{z}_T > 0 \right) + \mathcal{J}_2\left( f^2_{\mathbf\theta}({\mathbf h}_t, {\mathbf o}_t, \text{z}_t), \gamma^T \bar{c}_{\text{task}}({\mathbf h}_T, {\mathbf o}_T) \right) \Big]$, where \(f^1_{\mathbf\theta}\) and \(f^2_{\mathbf\theta}\) are the heads of our parameterized critic.\looseness=-1


\textbf{Search method.}
We build on the beam search strategy from \citep{mudgal2023controlled, li2025survey} wherein we sequentially sample $N$ beams of $d$ tokens into a set $B$ from the pre-trained model and choose $K$ beams with the highest scores as possible continuations of the prompt (see Algorithm \ref{alg:inference_guard} in Appendix \ref{app:algo}).
This step ensures that we focus on the most promising continuations.
The goal of the scoring function is to balance the immediate task cost and the predicted future task cost while ensuring safety.
This is repeated until we complete trajectories. Given a token trajectory $\text{y}_{t:t+d}$, we present a scoring function $\text{E}_\text{critic}$ where we assume that we cannot evaluate intermediate answers with the cost functions. However, when immediate safety costs are available, a simpler scoring function, see Appendix  \ref{app:algo}. We define $\text{E}_\text{critic}$ as:
\begin{align*}
&\text{E}_\text{critic}(\text{y}_{t:t+d}) =
&\begin{cases} 
\gamma^T \bar{c}_{\text{task}}(\cdot) & t+d = T \text{ and } \text{z}_{t+d} > 0 \\ 
n & t+d = T \text{ and } \text{z}_{t+d} \leq 0 \\ 
 f^2_{\mathbf\theta}(\cdot) &  f^1_{\mathbf\theta}(\cdot) > 0.5\\
n & \text{otherwise}. 
\end{cases}
\end{align*}
This $\text{E}_{\text{critic}}$ scoring function evaluates token sequences by balancing safety and task performance. At the final step ($t+d=T$), it assigns a score based on the task cost $\mathcal{C}_{\text{task}}$ if safety constraints are met ($\text{z}_{t+d} > 0$); otherwise, it applies a high penalty $n$. For intermediate steps, it relies on a trained critic. If the critic confidently predicts safety ($f_{\mathbf{\theta}}^{1}(\mathbf{h}_{t+d},\mathbf{o}_{t+d},\text{z}_{t+d})>0.5$), it uses the estimated future cost ($f_{\mathbf{\theta}}^{2}(\mathbf{h}_{t+d},\mathbf{o}_{t+d},\text{z}_{t+d})$); otherwise, it assigns the penalty $n$ as a conservative safeguard. 
\begin{figure*}[h]
\centering

\includegraphics[width=0.87\textwidth]{imgs/safety-crop.pdf}
\caption{Trade-offs between safety, reward, and inference time for BoN, ARGS, RECONTROL, Beam Search, and InferenceGuard on the PKU-SafeRLHF test dataset, evaluated for Alpaca-7B (top) and Beaver-v3-7B (bottom). Reward is the average score evaluated by the reward model, safety rate is the percentage of tasks completed within budget $d$, and inference time is the average per-task duration. The left column (reward vs. safety) and right column (inference time vs. safety) categorize methods by performance. \textit{InferenceGuard} achieves a balanced trade-off, positioning in the \textit{Optimal Region} and \textit{Optimal Efficiency} quadrants.}  
 

\label{fig:safety_tradeoff}
\end{figure*}
\textbf{Sampling Diversity.}
Finally, if the right selection strategy can guarantee that we will converge on a safe solution, it does not consider how many samples would be necessary.
To increase the search speed, we introduce a diversity term in the sampling distribution when \emph{no safe samples} were found based on the token frequency of failed beams.
We denote $F$ as the frequency matrix counting the tokens we previously sampled from $t$ to $t+d$.
Instead of sampling from $\text{SoftMax}(\mathbf{W}\mathbf{o}_t)$, we sample from $\text{SoftMax}(\mathbf{W}\mathbf{o}_t - \text{n}_{2}({F_{t} > 0}))$ where $\text{n}_{2} ({F_{t} > 0})$ is a vector where each component $j$ is $\text{n}_2$ if $F_{t, j} > 0$ and 0 otherwise. The addition of $\text{n}_{2}({F_{t} > 0})$ disables the possibility of sampling the same token at the same position observed in unsuccessful beams, thus increasing diversity.


It is worth noting that as we sample from the reference LLM and rank responses directly or via the critic, block sampling ensures a small Kullback-Leibler (KL) divergence from the original LLM without explicitly adding a KL regularizer into our objective, preserving coherence and natural flow; see \citep{mudgal2023controlled}.



\section{Experiments}\label{Sec:Exp}

\begin{figure*}[h!]
\centering

\includegraphics[width=0.81\textwidth]{imgs/alpaca_beaver_distribution-crop.pdf}
%
\caption{Reward and cost distributions on PKU-SafeRLHF test tasks using Alpaca-7B (top) and Beaver-v3 (bottom) as base models. The left y-axis shows reward distribution. The right y-axis shows the maximum cumulative cost. \textit{InferenceGuard} outperforms others, achieving higher rewards with lower costs across both models.}
\label{fig:reward_cost_comparison}
\end{figure*}
\noindent \textbf{Baselines.} We evaluate the helpfulness (task cost) and harmlessness (safety cost) of our method on both non-safety-aligned and safety-aligned base models, Alpaca-7B~\cite{taori2023alpaca} and Beaver-7B~\citep{ji2024pku}. We compare $\texttt{InferenceGuard}$ to the following state-of-the-art test-time alignment methods: Best-of-N (BoN), beam search,  ARGS~\citep{khanov2024args} and RECONTROL~\citep{kong2024aligning}. These test-time alignment methods were originally designed to maximize rewards without considering safety. To ensure a fair and meaningful comparison, we extend them to incorporate safe alignment for LLMs. We incorporate a Lagrangian-based approach for all baselines that penalizes safety violations, ensuring a balanced trade-off between task performance and safety. This helps us evaluate our performance against other algorithms and highlights the importance of safety augmentation instead of a Lagrangian approach for effectively balancing rewards and constraint satisfaction. 


For beam search and Best-of-N (BON), we adopt a Lagrangian approach to select solutions with $c_{\text{task}} + \lambda \mathcal{C}_{\text{safety}}$ where $\lambda$ is the Lagrangian multiplier. Similarly, we extend ARGS so that token selection follows: $-\omega \pi(t|\cdot) + c_{\text{task}} + \lambda \mathcal{C}_{\text{safety}}$, with $\omega$ adjusting the influence of the reference policy. We also considered state augmentation for ARGS and RE-control but found it ineffective. Since these methods decode token-by-token, they cannot recover once $\text{z}_t$ flips the sign, and before that, $\text{z}_{t}$ has no influence. Thus, we excluded it from our evaluation. To further strengthen our comparison, we introduce safety-augmented versions of BoN and beam search as additional baselines. 

\textbf{Datasets.} We evaluate all the above methods on the PKU-SafeRLHF dataset~\citep{ji2024pku}, a widely recognized benchmark for safety assessment in the LLM literature.  This dataset contains 37,400 training samples and 3,400 testing samples. Of course, our training samples are \emph{only} used to train the critic value network. Here, we construct a dataset by generating responses using the base models with five sampled trajectories for each PKU-SafeRLHF prompt from the training set. \looseness=-1

\noindent \textbf{Evaluation Metrics.}
We assess the performance using several metrics: the \textbf{Average Reward} is computed using the reward model of~\citep{khanov2024args} as $ - c_{\text{task}}$ on the complete response to reflect helpfulness, where a higher reward indicates better helpfulness; the \textbf{Average Cost} is evaluated with the token-wise cost model from~\citep{dai2023safe} as $\mathcal{C}_{\text{safety}}$, indicating harmfulness, with higher cost values reflecting more resource-intensive or harmful outputs; the \textbf{Safety Rate} is defined as the proportion of responses where the cumulative cost does not exceed the safety budget \( z_{t=0} = 10 \), and is formally given by \( \text{Safety Rate} = \frac{1}{N} \sum_{i=1}^N \mathbb{I}(\mathcal{C}^i_{\text{test}} \leq z_{t=0}) \), where \( N \) is the total number of responses; and \textbf{Time} refers to the inference time taken to generate a response in seconds.




\textbf{Results.}
We present our main results in \cref{fig:safety_tradeoff} and additional ones in \cref{tab:performance_comparison} in Appendix \ref{App:Exps}.
\texttt{InferenceGuard} achieves the highest safety rates with both models (91.04\% and 100\% for Alpaca and Beaver respectively).
With Beaver, our method dominates the Patero front, achieving the highest rewards without any unsafe responses. 
Although Lagrangian methods can have a reasonable average cost, they all fail to satisfy the safety constraints. In the meantime, they are too safe on already safe answers, hindering their rewards.
The RE-control intervention method underperforms for both models in our setting.
ARGS can provide safe answers but with very poor rewards because most answers are very short to avoid breaking the safety constraint.
Best-of-N and beam search with augmented safety came after us.
However, Best-of-N cannot leverage intermediate signal, and beam search is blind to its previous mistakes, yielding more unsafe answers.

\cref{fig:reward_cost_comparison} provides a better view of the reward and safety distributions. The figure shows that \texttt{InferenceGuard} consistently achieves higher rewards while maintaining low cumulative costs, outperforming other methods across both models. Its maximum cumulative cost stays just under the safety budget while maximizing the reward as suggested by our theoretical contributions. Finally, we observe that the trained critic helps to better guide the search on intermediate trajectories in both settings. We also compare generated answers in Appendix \ref{App:Exps} qualitatively.

\section{Related Work}
\textbf{Safe RL:} Safe RL employs the cMDP framework, formulated in~\citep{altman1999constrained}. Without prior knowledge,~\citep{turchetta2016safe, koller2018learning, dalal2018safe} focus on safe exploration, while with prior knowledge,~\citep{chow2018lyapunov, chow2019lyapunov, berkenkamp2017safe} learn safe policies using control techniques. \citep{achiam2017constrained, ray2019benchmarking, stooke2020responsive} enforce safety constraints via Lagrangian or constrained optimization which can often lead to suboptimal safety-reward trade-offs. Instead, we extend safety state augmentation~\cite{sootla2022saute} to LLMs and latent MDPs to ensure almost sure inference time safety.\looseness=-1


\textbf{LLM alignment and safety:} Pre-trained LLMs are often aligned to specific tasks using RL from Human Feedback (RLHF), where LLMs are fine-tuned with a learned reward model~\citep{stiennon2020learning, ziegler2019fine, ouyang2022training} or directly optimized from human preferences~\citep{rafailov2023direct, azar2023general, zhao2023slic, tang2024generalized, song2024preference, ethayarajh2024kto}. \citep{bai2022training, ganguli2022red} first applied fine-tuning in the context of safety, and \citep{dai2023safe} proposed safe fine-tuning via Lagrangian optimization. Other safety-focused methods such as \citep{gundavarapu2024machine,gou2024eyes,hammoud2024model,hua2024trustagent,zhang2024controllable,guo2024cold,xu2024safedecoding,wei2024assessing,li2025salora} are either orthogonal, handle a different problem to ours, or can not ensure almost sure safety during inference. 



\textbf{Inference time alignment:} Inference-time alignment of LLMs is often performed through guided decoding, which steers token generation based on rewards~\citep{khanov2024args,shi2024decoding,huang2024deal} or a trained value function~\citet{han2024value,mudgal2023controlled,kong2024aligning}. Other safety-focused inference-time methods include~\citep{zhong2024rose,banerjee2024safeinfer,niu2024parameter,wang2024probing,zeng2024root,zhao2024adversarial}. Compared to those methods, we are the first to theoretically guarantee almost sure safe alignment with strong empirical results. Operating in the latent space enables us to train smaller, inference-efficient critics while optimally balancing rewards and safety constraints without introducing extra parameters, e.g., Lagrangian multipliers.\looseness=-1

We detail other related
works extensively in Appendix \ref{sec:rel-2}.























 
\section{Conclusion}\label{secConclu}
We introduced \texttt{InferenceGuard}, a novel inference-time alignment method that ensures large LLMs generate safe responses almost surely—i.e., with a probability approaching one. We extended prior safety-augmented MDP theorems into the latent space of LLMs and conducted a new analysis. Our results demonstrated that \texttt{InferenceGuard} significantly outperforms existing test-time alignment methods, achieving state-of-the-art safety versus reward tradeoff results. In the future, we plan to improve the algorithm's efficiency further and generalize our setting to cover jailbreaking. While our method is, in principle, extendable to jailbreaking settings, we aim to analyze whether our theoretical guarantees still hold. 

\section*{Broader Impact Statement} 
This work contributes to the safe and responsible deployment of large language models (LLMs) by developing \texttt{InferenceGuard}, an inference-time alignment method that ensures almost surely safe responses. Given the increasing reliance on LLMs across various domains, including healthcare, education, legal systems, and autonomous decision-making, guaranteeing safe and aligned outputs is crucial for mitigating misinformation, bias, and harmful content risks.

To further illustrate the effectiveness of our approach, we have included additional examples in the appendix demonstrating that our method successfully produces safe responses. These examples were generated using standard prompting with available large language models LLMs. Additionally, we have added a warning at the beginning of the manuscript to inform readers about the nature of these examples. Our primary motivation for this addition is to highlight the safety improvements achieved by our method compared to existing alternatives. We do not foresee these examples being misused in any unethical manner, as they solely showcase our model’s advantages in ensuring safer AI interactions. Finally, we emphasize that our method is designed specifically to enhance AI safety, and as such, we do not anticipate any potential for unethical applications.

\texttt{InferenceGuard} enhances the scalability and adaptability of safe AI systems by introducing a formally grounded safety mechanism that does not require model retraining while reducing the resource costs associated with traditional RLHF methods. The proposed framework advances AI safety research by providing provable safety guarantees at inference time, an area that has received limited attention in prior work.\looseness=-1

While this method significantly improves safety in LLM outputs, it does not eliminate all potential risks, such as adversarial manipulation or emergent biases in model responses. Future work should explore robustness to adversarial attacks, contextual fairness, and ethical considerations in deploying safety-aligned LLMs across different cultural and regulatory landscapes. Additionally, transparency and accountability in AI safety mechanisms remain essential for gaining public trust and ensuring alignment with societal values.
This work aims to empower developers and policymakers with tools for ensuring safer AI deployment while contributing to the broader conversation on AI ethics and governance.

\section*{Acknowledgments}
The authors would like to thank Rasul Tutunov, Abbas Shimary, Filip Vlcek, Victor Prokhorov, Alexandre Maraval, Aivar Sootla, Yaodong Yang, Antonio Filieri, Juliusz Ziomek, and Zafeirios Fountas for their help in improving the manuscript. We also would like to thank the original team from Pekin University, who made their code available and ensured we could reproduce their results. 


\bibliography{icml2025}
\bibliographystyle{icml2025}

\appendix


\onecolumn
\section{Additional Related Work}\label{sec:rel-2}

\textbf{Safe RL:} Safe RL employs the cMDP framework~\citep{altman1999constrained} to enforce safety constraints during exploration and policy optimization. When no prior knowledge is available, methods focus on safe exploration~\citep{turchetta2016safe, koller2018learning, dalal2018safe, wachi2018safe, bharadhwaj2020conservative}. With prior knowledge, such as environmental data or an initial safe policy, methods learn safe policies using control techniques like Lyapunov stability~\citep{chow2018lyapunov, chow2019lyapunov, berkenkamp2017safe, ohnishi2019barrier} and reachability analysis~\citep{cheng2019end, akametalu2014reachability, dean2019safeguarding, fisac2019bridging}. Safety constraints are enforced via Lagrangian or constrained optimization methods~\citep{achiam2017constrained, ray2019benchmarking, stooke2020responsive, yang2019relative, ding2020natural, ji2024pku}, but can often lead to suboptimal safety-reward trade-offs. In contrast, our approach extends safety state augmentation~\cite{sootla2022saute} to LLMs and latent MDPs to ensure almost sure inference time safety without relying on Lagrangian multipliers.



\textbf{LLM alignment and safety:} Methods for aligning pre-trained LLMs with task-specific data include prompting, guided decoding, and fine-tuning.
Among fine-tuning methods, RL from Human Feedback (RLHF) has proven effective, where LLMs are fine-tuned with a learned reward model~\citep{stiennon2020learning, ziegler2019fine, ouyang2022training} or directly optimized from human preferences~\citep{rafailov2023direct, azar2023general, zhao2023slic, tang2024generalized, song2024preference, ethayarajh2024kto, ramesh2024group}. Recent works have explored fine-tuning for helpful and harmless responses~\citep{bai2022training, ganguli2022red}, while \citep{dai2023safe} introduced a safe RL approach incorporating safety cost functions via Lagrangian optimization, requiring model weight fine-tuning. Other safety-focused methods, including machine unlearning~\citep{gundavarapu2024machine}, safety pre-aligned multi-modal LLMs~\citep{gou2024eyes}, safety-aware model merging~\citep{hammoud2024model}, prompting-based safety methodologies~\citep{hua2024trustagent}, test-time controllable safety alignment~\citep{zhang2024controllable}, defenses against adversarial attacks and jailbreaking~\citep{guo2024cold, qi2024safety, xu2024safedecoding}, identifying safety criticial regions in LLMs \citep{wei2024assessing},  safety preserved LoRA fine-tuning \citep{li2025salora},  alignment using correctional residuals between preferred and dispreferred answers using a small model \citep{ji2024aligner}, and identifying safety directions in embedding space~\citep{feng2024legend}.Those methods are either orthogonal, handle a different problem to ours, or can not ensure almost sure safety during inference. 



\textbf{Inference time alignment:} The closest literature to ours is inference-time alignment. Those methods offer flexible alternatives to fine-tuning LLMs, as they avoid modifying the model weights. A common approach is guided decoding, which steers token generation based on a reward model. In particular, \citep{khanov2024args,shi2024decoding,huang2024deal} perform this guided decoding through scores from the reward model whereas \citet{han2024value,mudgal2023controlled,kong2024aligning} use a value function that is trained on the given reward model. These inference-time alignment methods build on previous works like \citep{yang2021fudge,arora2022director,krause2021gedi,kim2023critic,meng2022nado,peng2019awr}, which guide or constrain LLMs towards specific objectives. Other safety-focused inference-time methods include, reverse prompt contrastive decoding \citep{zhong2024rose}, adjusting model hidden states combined with guided decoding \citep{banerjee2024safeinfer}, soft prompt-tuned detoxifier based decoding \citep{niu2024parameter}, jail-break value decoding \citep{wang2024probing}, speculating decoding using safety classifier \citep{zeng2024root}, and opposite prompt-based contrastive decoding \citep{zhao2024adversarial}. Compared to those methods, we are the first to achieve almost sure safe alignment with strong empirical results. Operating in the latent space enables us to train smaller, inference-efficient critics while optimally balancing rewards and safety constraints (see Section \ref{Sec:Exp}) without introducing additional parameters like Lagrangian multipliers.
\section{Theoretical Analysis}\label{sec: saute-thm}

For our theoretical results, we consider a similar setup to that of \citet{sootla2022saute,hernandez1992discrete}  but with a discrete action space. Consider an MDP \( M = \{S, A, P, c, \gamma_c\} \) with a discrete, non-empty, and finite action set for each $s$ defined as \( A(s) \). The set
\[
K = \{(s, a) \mid s \in S, a \in A(s)\}
\]
defines the admissible state-action pairs and is assumed to be a Borel subset of \( S \times A \). A function \( u \) is \emph{inf-compact} on \( K \) if the set
$\{ a \in A(s) \mid u(s, a) \leq r \}$
is compact for every \( s \in S \) and \( r \in \mathbb{R} \).
Note that, since the action space is finite and discrete every function $u$ is inf-compact on $K$.  A function $u$ is lower semi-continuous (l.s.c.) in $S$ if for every $s_0 \in S$ we have
\[
\liminf_{s \to s_0} u(s) \geq u(s_0).
\] Let \( L(S) \) denote the class of all functions on \( S \) that are l.s.c. and bounded from below.


For a given action, $a\in A(s)$, a distribution \( P(y \mid s, a) \) is called \emph{weakly continuous} w.r.t. $s$, if for any function \( u(s) \), continuous and bounded w.r.t. $s$ on \( S \),  the map
\[
(s, a) \mapsto \int_S u(y) P(dy \mid s, a)
\]
is continuous on \( K \) for a given $a$.


We also make the following assumptions:

\textbf{B1.} The function \( c(s, a) \) is bounded, measurable on \( K \), nonnegative, lower semi-continuous w.r.t. $s$ for a given $a\in A(s)$;

\textbf{B2.} The transition law \( P \) is weakly continuous w.r.t. $s$ for a given $a\in A(s)$;

\textbf{B3.} The set-value function map $s:A(s)$ satisfies the following, $\forall s_{0}\in \mathcal{S}$, there exists a $\epsilon >0$, such that $\forall x$ satisfying $\|x-x_{0}\|\leq \epsilon$, $A(x)=A(x_0)$

Note that, the assumptions B1-B3, share a similar essence to that of the Assumptions 2.1-2.3 in \citet{hernandez1992discrete} and B1-B3 in \citet{sootla2022saute} but suited for a discrete action space. In particular, Assumption B3, is similar to the lower semi continuity assumption on the set-value function map $A(s)$ taken in \citet{sootla2022saute,hernandez1992discrete} but modified for a discrete action space. 


Our first goal is to recreate \citet[Lemma 2.7]{hernandez1992discrete} for our discrete action setting. Let $\Pi$ denote the set of functions from $S\to A$.

\begin{lemma}\label{lemma:properties}
    \begin{itemize}
    \item[(a)] If Assumption B3 holds and $v(s,a)$ is l.s.c. w.r.t. $s$ for any given $a\in A(s)$ and bounded from below on $K$, then the function
    \[
    v^*(s) := \inf_{a \in A(s)} v(s, a)
    \]
    belongs to $L(S)$ and, furthermore, there is a function $\pi \in \Pi$ such that
    \[
    v^*(s) = v(s, \pi(s)) \quad \forall s \in S.
    \]
    
    \item[(b)] If the Assumptions B1-B3 hold, and $u \in L(S)$ is nonnegative, then the (nonnegative) function
    \[
    u^{*}(s) := \inf_{a \in A} \left[ c(s, a) + \int_{S} u(y) P(dy \mid s, a) \right]
    \]
    belongs to $L(S)$, and there exists $\pi \in \Pi$ such that
    \[
    u^*(s) = c(s, \pi(s)) + \int_{S} u(y) P(dy \mid s, \pi(s)) \quad \forall s \in S.
    \]

    \item[(c)] For each $n = 0, 1, \dots$, let $v_n$ be a l.s.c. function, bounded from below. If $v_n \to v_0$ as $n \to \infty$, then
    \[
    \lim_{n \to \infty} \inf_{a \in A(s)} v_n(s, a) = \inf_{a \in A(s)} v_0(s, a) \quad \forall s \in S.
    \]
\end{itemize}
\end{lemma}
\begin{proof}
For part a)

We have $v(s,a)$ is l.s.c. w.r.t. $s$ for any given $a$. This implies from the definition of lower semi -continuity for any $s_{0}\in S$ and $a\in A(s_{0})$, if $v(s_{0},a)>y$, then there exists a $\epsilon>0$, s.t. $\forall s$ satisfying $\|s-s_{0}\|\leq \epsilon$, $v(s,a)>y$. 

Assume for some  $s_{0}\in S$ and $y$, the function $\inf_{a\in A(s_{0})} v(s_0, a)$ satisfies, 
\begin{equation}
     \inf_{a\in A(s_{0})} v(s_0, a)>y \quad \Rightarrow  v(s_0, a)>y \quad \forall a\in A(s_0)
\end{equation}
  Using Assumption B3, we have $A(s)=A(s_0)$ for $\|s-s_{0}\|\leq \epsilon$. Moreover, using the fact that $v(s,a)$ is l.s.c. at a given $a$, we have if $v(s_0, a)>y$, $\forall a\in A(s_0)$  we have,
\begin{equation}
     v(s, a)>y \quad \forall \|s-s_{0}\|\leq \epsilon, \forall a\in A(s)
\end{equation}
Since, this holds for all $ a\in A(s)$, it also holds for 
     \begin{equation}
     \inf_{a\in A(s)} v(s, a)>y \quad \forall \|s-s_{0}\|\leq \epsilon
\end{equation}

This proves the lower semi continuity of $v^{*}(s)=\inf_{a\in A(s)} v(s, a)$. Further, due to the discrete nature of $A(s)$, the $\inf_{a\in A(s)} v(s, a)$ is always attained by an action $\pi(s)\in A(s)$. Hence, there exists a function $\Pi:S->A(s)$ s.t.
\begin{equation}
     v^*(s) = \inf_{a\in A(s)} v(s, a)= v(s, \pi(s)) \quad \forall s \in S.
\end{equation}

For part b), note that $c(s, a) + \int u(y) P(dy \mid s, a) $ is l.s.c. w.r.t. $s$ for an given $a$, based on Assumptions B1-B2. Hence, using part a) we have, 
\begin{equation}
     u^{*}(s) := \inf_{a \in A} \left[ c(s, a) + \int_{S} u(y) P(dy \mid s, a) \right]=c(s, \pi(s)) + \int_{S} u(y) P(dy \mid s, \pi(s))\in L(S) \quad \forall s \in S
\end{equation}
for some $f\in \Pi$. 

For part c), we begin by defining $l(s)=lim_{n\to \infty}\inf_{a\in A(s)}v_n(s,a)$. Note that, since $\{v_n\}$ is an increasing sequence, we have for any $n$
\begin{equation}
    \inf_{a\in A(s)}v_n(s,a) \leq \inf_{a\in A(s)}v_0(s,a)
\end{equation}
This implies,
\begin{equation}
   l(s) \leq \inf_{a\in A(s)}v_0(s,a)=v_0^{*}(s)
\end{equation}
Next, we define for any $s\in S$, 
\begin{equation}
    A_n:=\{a\in A(s)|v_n(s,a)\leq v_0^*(s)\}
\end{equation}
We note that $A_n$ are compact sets as $A$ is finite and discrete. Further, note that
$A_n$ is a decreasing sequence converging to $A_0$ (compact, decreasing and bounded from below by $A_0$). Also, note that
\begin{equation}
    A_1 \supset A_2 \supset A_3 \cdots \supset A_0
\end{equation}
We consider the sequence $\{a_n\}$ where $a_n\in A_n$ and $a_n$ satisfies,
\begin{equation}
    v_n(s,a_n)=\inf_{a\in A(s)}v_n(s,a)\leq \inf_{a\in A(s)}v_0(s,a)\leq v_{0}^{*}(s)
\end{equation}
This sequence $\{a_n\}$ belongs to the compact space $\cup_{n=1}^{\infty}A_n=A_1$, hence it has convergent subsequence $\{a_{n_i}\}$ converging to  $\cup_{n=1}^{\infty}A_n=A_1$. 
\begin{align}
    a_{n_i}&\in A_{n_i}=\cap_{n\leq n_i}A_n\\
    a_{0}&\in \cap_{n\leq \infty }A_n =A_0
\end{align}
 Since, the converging sequence $a_{n_i}\to a_0 $ belongs to the discrete, compact space, there exisits a $N_i$, such that for all  $n_i \geq N_i$, $a_{n_i}= a_0$. Further, using the increasing nature of $v_n$, we have,
\begin{equation}
    v_{n_i}(s,a_{n_i})\geq v_n(s, a_{n_i}) \quad \forall n_i \geq n
\end{equation}
As $i\to\infty$, this implies,  
\begin{align}
    \lim_{i\to\infty } v_{n_i}(s,a_{n_i})\geq v_n(s, a_{0}) \\
    \lim_{i\to\infty } \inf_{a\in A(s)} v_{n_i}(s,a)\geq v_n(s, a_{0})\\
    l(s)\geq v_n(s,a_0)
\end{align}
As $v_n \to v_0$, $l(s)\geq v_0(s,a_0)=v_0^{*}(s)$. 

\end{proof}

\subsection{Optimality Equation}
Next, we characterize the solution to the Bellman (Optimality) equation. We begin by recalling the Bellman operator:
\[
T v(s) = \min_{a \in A(s)} \left\{ c(s, a) + \gamma \int v(y) P(dy \mid s, a) \right\}.
\]
To state our next result we introduce some notation: Let \( L(S)^+ \) be the class of nonnegative and l.s.c. functions on \( S \), and for each \( u \in L(S)^+ \) by \Cref{lemma:properties}(b), the operator \( T \) maps \( L(S)^+ \) into itself. We also consider the sequence \( \{v_n\} \) of value iteration (VI) functions defined recursively by
\[
v_0(S) := 0, \quad \text{and} \quad v_h := T v_{h-1} \quad \text{for} \quad h = 1, 2, \dots
\]
That is, for \( h \geq 1 \) and \( s \in S \),
\[
v_h(s) := \min_{a \in \mathcal{A}(s)} \left( c(s, a) + a \int_{S} v_{h-1}(y) P(dy \mid s, a) \right). \tag{4.3}
\]
Note that, by induction and \Cref{lemma:properties}(b) again, \( v_h \in L(S)^+ \) for all \( h \geq 0 \). From elementary Dynamic Programming \citet{bertsekas1987dynamic,bertsekas1996stochastic,dynkin1979controlled}, \( v_h(s) \) is the optimal cost function for an \( h \)-stage problem (with "terminal cost" \( v_0(s) = 0 \)) given \( s_0 = s \); i.e.,
\[
v_h(s) = \inf_{\pi } V_h(\pi, s), 
\]

where, $\Pi$ is the set of policies and $V_H(\pi, s)$ denotes the value function for the $H-$stage problem:
\[
V_H(\pi, s_0) = \mathbb{E}_{\pi}\left[ \sum_{h=0}^{H-1} \gamma^h c(s_h, a_h) \right].
\]
Here, \( \mathbb{E}_{\pi} \) stands for the expectation with actions sampled according to the policy \( \pi \) and the transitions \( P \). For $H\to \infty$, let the value functions be denoted as follows:
\[
V(\pi, s_0) = \mathbb{E}_{\pi}\left[ \sum_{h=0}^{\infty} \gamma^h c(s_h, a_h) \right],
\]
and
\[
V^*(s) = \inf_{\pi} V(\pi, s).
\]


We want to prove similar results to that of \citet[Theorem 4.2]{hernandez1992discrete} on the optimality of the Bellman operator, however in the discrete action setting. In particular, we want to show the following theorem

\begin{thm}\label{thm: Bellmannexistence}
Suppose that Assumptions B1-B3 hold, then:
\begin{itemize}
    \item[(a)] \( v_h \to V^* \); hence
    \item[(b)] \( V^* \) is the minimal pointwise function in \( L(S)^+ \) that satisfies
    \[
    V^* = T V^*
    \]
\end{itemize}
\end{thm}



\begin{proof}
We follow a similar proof strategy to that of \citet[Theorem 4.2]{hernandez1992discrete}. 

To begin, note that the operator \( T \) is monotone on \( L(S)^+ \), i.e., \( u > v \) implies \( T u > T v \). Hence \( \{v_h\} \) forms a nondecreasing sequence in \( L(S)^+ \) and, therefore, there exists a function \( u \in L(S)^+ \) such that \( v_h \to u \). This implies (by the Monotone Convergence Theorem) that
\[
c(s, a) + a \int v_{h-1}(y) P(dy \mid s, a) \to c(s, a) + a \int u(y) P(dy \mid s, a),
\]
Using \Cref{lemma:properties}(c), and $v_h=\inf_{a\in A(s)}\{c(s, a) + a \int v_{h-1}(y) P(dy \mid s, a)\}$ yields
\begin{align*}
\lim_{h\to\infty}\inf_{a\in A(s)}\{c(s, a) + a \int v_{h-1}(y) P(dy \mid s, a)\} &= \inf_{a\in A(s)}\{c(s, a) + a \int u(y) P(dy \mid s, a)\},\\
\lim_{h\to\infty}v_h &= Tu,\\
u &= T u.
\end{align*}
This shows  \( v_h \to u \), such that \( u \in L(S)^+ \) satisfies the Optimality equation. 



Next, we want to show \( u = V^* \). Using that \( u \geq T u \), and by \Cref{lemma:properties}(b), we have that there exists \( \pi \in \Pi \), a stationary policy that satisfies
\begin{align}
    u(s) &\geq \inf_{a\in A(s)}\{c(s, a) + a \int u(y) P(dy \mid s, a)\} \quad \forall s\\
    &\geq c(x, \pi) + \alpha \int u(y) P(dy \mid s, \pi) \quad \forall s.
\end{align}

Applying the $T$ operator iteratively, we have

\begin{align}
    u(s)&\geq T^{H}u(s) \quad \forall s,H\\
&\geq  \E_{\{s_h\},\pi}[\sum_{h=0}^{H-1} \alpha^h  c(s_h, \pi)] + \alpha^{H} \int u(y) P^{H}(dy \mid s, \pi) \quad \forall s,H,
\end{align}

where \( P^H(B \mid s, \pi) = P(\{ s_H \in B \}) \) denotes the \( H \)-step transition probability of the Markov chain \( \{ s_h \} \) (see \citet[Remarks 3.1,3.2]{hernandez1992discrete}).  Therefore, since \( u \) is nonnegative,

\begin{align}
    u(s)&\geq  \E_{\{s_h\},\pi}[\sum_{h=0}^{H-1} \alpha^h  c(s_h, \pi)] \quad \forall s,H,
\end{align}
Letting \( H \to \infty \), we obtain
\[
u(s) \geq V(\pi, s) \geq V^{*}(s) \quad \forall s.
\]
Next, note that, 

\begin{align}
    v_h(s) &= \inf_{\pi \in \Pi} V_h(\pi, s)\\
    &\leq V_h(\pi, s) \quad \forall s,h,\pi
\end{align}
and letting \( h \to \infty \), we get
\[
u(s) \leq V(\pi, s) \quad \forall s,\pi.
\]
This implies \( u(s) \leq V^*(s) \). We have thus shown that \( u = V^* \).


Further, if there is another solution $u'$ satisfying $u'=Tu'$, it holds that $u'\geq V^{*}$, Hence, $V^{*}$ is the minimal solution.

\end{proof}


\subsection{Limit of a sequence of MDPs}

Consider now a sequence of Markov Decision Processes (MDPs) \( M_n = \{ S, A, P, c_n, \gamma_n \} \), where, without loss of generality, we write \( c_n, c_\infty \) and \( M_n, M_\infty \). Consider now a sequence of value functions \( \{ V_n^* \}_{n=0}^{\infty} \):
\[
V_n(\pi, s_0) = \mathbb{E}^\pi \left[ \sum_{t=0}^{\infty} \gamma^t c_n(s_t, a_t) \right],
\]
\[
V_n^*(s) = \inf_{\pi} V_n(\pi, s).
\]

The "limit" value functions (with \( n = \infty \)) are still denoted as follows:
\[
V(\pi, s_0) = \mathbb{E}^\pi \left[ \sum_{t=0}^{\infty} \gamma^t c(s_t, a_t) \right],
\]
\[
V^*(s) = \inf_{\pi} V(\pi, s).
\]

We also define the sequence of Bellman operators
\[
T_n v(s) = \min_{a \in A(s)} \left\{ c_n(s, a) + \gamma \int v(y) P(dy \mid s, a) \right\},
\]
\[
T v(s) = \min_{a \in A(s)} \left\{ c(s, a) + \gamma \int v(y) P(dy \mid s, a) \right\}.
\]

In addition to the previous assumptions, we make an additional one, while modifying Assumption B1:
\begin{itemize}
    \item[\textbf{B1'}] For each \( n \), the functions \( c_n(s, a) \) are bounded, measurable on \( \mathcal{K} \), nonnegative, lower semi-continuous;
    \item[\textbf{B4}] The sequence \( \{ c_n(s, a) \}_{n=0}^{\infty} \) is such that \( c_n \uparrow c \).
\end{itemize}
 


For each \( n \), the optimal cost function \( V^*(s) \) is the bounded function in \( L(S)^+ \) that satisfies the Optimality equation in \Cref{thm: Bellmannexistence} :
\[
V_n^* = T_n V_n^*,
\]

\begin{thm}\label{thm: value-convergence}
    The sequence \( V_n^* \) is monotone increasing and converges to \( V^* \).
   
\end{thm}

\begin{proof} 

We follow a similar proof strategy to that of \citet[Theorem 5.1]{hernandez1992discrete}

 To begin with, note that since \( c_n \uparrow c \), it is clear that \( V_n^{*} \) is an increasing sequence in \( L(S)^+ \), and therefore, there exists a function \( u \in L(S)^+ \) such that \( V_n^* \to  u \).

Moreover, from \Cref{lemma:properties}(c), letting \( n \to \infty \), we see that \( u = T u \), i.e., \( u \) satisfies the optimality equation. This implies that \( u \geq V^* \), since, by Theorem 4, \( V^* \) is the minimal solution in \( L(X)^+ \) to the optimality equation.

On the other hand, it is clear that \( V_n^{*} \leq V^* \) for all \( n \), so that \( u \leq V^* \). Thus \( u = V^* \), i.e., \( U^* = V^* \).



\end{proof}

\subsection{Latent MDP Analysis}\label{sec: equi}

\thmsauteequivalence*
\begin{proof}


We begin by comparing our assumptions to that of the assumptions B1-B4, closely aligned to those used in \citet{hernandez1992discrete,sootla2022saute}.

To prove a),b) of \Cref{thm:sauteequivalence} we need to verify that the latent MDP satisfying Assumptions A1-A2 also satisfies Assumptions B1', B2-B4. According to Assumption A1, we consider bounded costs $\bar{\mathcal{C}}^{n}_{\text{task}}$ continuous w.r.t. state $(\mathbf{h},\mathbf{o},\text{z})$ for a given $y$ with discrete and finite action space $\gV$, hence Assumptions B1', B3, and B4 are satisfied. Assumptions B2 and A2 are identical. This proves a) and b).


For c), note that the state value function $V(\cdot)$ and latent space value function $\bar{V}(\cdot)$ w.r.t. policy $\bar{\pi}:\bar{\gS}\rightarrow\gV$ that acts on the latent space directly and on the original space as $\bar{\pi}(\phi(\cdot)):\gS\rightarrow\gV$ are related as follows:
\begin{align}
     &V(\bar{\pi}(\phi(\cdot)), \mathbf{s}_0,\text{z}_0)\\&= \E_{\substack{\mathbf{s}_{t+1},\text{z}_{t+1}\sim \gP(\cdot|\mathbf{s}_t,\text{z}_t,\text{y}_t)\\ \text{y}_{t}\sim \bar{\pi}(\cdot|\phi(\mathbf{s}_t),\text{z}_t)}}\Big[\sum_{t=0}^\infty \gamma^t \tilde{\mathcal{C}}^{n}_{\text{task}}(\mathbf{s}_t, \text{z}_{t},\text{y}_{t})\Big]\\
    &= \E_{\substack{\mathbf{s}_{t+1},\text{z}_{t+1}\sim \gP(\cdot|\mathbf{s}_t,\text{z}_t,\text{y}_t)\\ \text{y}_{t}\sim \bar{\pi}(\cdot|\phi(\mathbf{s}_t),\text{z}_t)}}\Big[\sum_{t=0}^\infty \gamma^t \bar{\mathcal{C}}^{n}_{\text{task}}(\phi(\mathbf{s}_t), \text{z}_{t},\text{y}_{t})\Big]\\
     &=\E_{\substack{\phi(\mathbf{s}_{t+1}),\text{z}_{t+1}\sim \bar{\gP}(\cdot|\phi(\mathbf{s}_t),\text{z}_t)\big)\\ \text{y}_{t}\sim \bar{\pi}(\cdot|\phi(\mathbf{s}_t),\mathbf{o}_t)}}\Big[\sum_{t=0}^\infty \gamma^t \bar{\mathcal{C}}^{n}_{\text{task}}(\phi(\mathbf{s}_t), \text{z}_{t},\text{y}_{t})\Big]\Big|\\
    &=\E_{\substack{\mathbf{h}_{t+1},\mathbf{o}_{t+1},\text{z}_{t+1}\sim\bar{\gP}(\cdot|\mathbf{h}_t,\mathbf{o}_t,\text{z}_t)\big)\\\mathbf{h}_{t+1},\mathbf{o}_{t+1}=\phi(\mathbf{s}_{t+1}) \\\text{y}_{t}\sim \bar{\pi}(\cdot|\mathbf{h}_t,\mathbf{o}_t)}}\Big[\sum_{t=0}^\infty \gamma^t \bar{\mathcal{C}}^{n}_{\text{task}}(\mathbf{h}_t,\mathbf{o}_t, \text{z}_{t},\text{y}_{t})\Big]\Big|_{\mathbf{h}_0,\mathbf{o}_0=\phi(s_0)}\\
    &=  \bar{V}(\bar{\pi}, \mathbf{h}_0,\mathbf{o}_0,z_0)\Big|_{\mathbf{h}_0,\mathbf{o}_0=\phi(s_0)}
\end{align}

Hence, we can show $\bar{\pi}_n^{*}$ is optimal for $V_n(\cdot)$ as follows:
\begin{align}
     V_n(\bar{\pi}_n^{*},\mathbf{s}, \text{z})&=\bar{V}_n(\bar{\pi}_n^{*}, \mathbf{h},\mathbf{o},\text{z})\Big|_{\mathbf{h},\mathbf{o}=\phi(s)}\\
     &=\min_{\bar{\pi}}\bar{V}_n(\bar{\pi}, \mathbf{h},\mathbf{o},\text{z})\Big|_{\mathbf{h},\mathbf{o}=\phi(s)}\\
     &=\min_{\bar{\pi}}V_n(\bar{\pi}(\phi(\cdot)), \mathbf{s},\text{z})\\
     &=\min_{\pi}V_n(\pi, \mathbf{s},\text{z})
\end{align}

Here, the minimization of $\pi$ is over set of all policies covered by $\bar{\pi}(\phi(\cdot))$ and we show that $\bar{\pi}_n^{*}(\phi(\cdot))$ is the optimal policy for the original space over this set of policies. 



\end{proof}


\thmalmostsure*
\begin{proof}
 We first note that if any trajectory with infinite cost has a finite probability, the cost would be infinite. Hence, all the trajectories with finite/positive probability have finite costs. This implies, the finite cost attained by  $\pi^\star$ w.r.t. Equation \ref{Eq:Constraints} implies the satisfaction of constraints (Equation \ref{Eq:a.s.SafeLLM-main}) almost surely (i.e. with probability 1). Combined with the fact that the policy $\pi^\star$ was obtained by minimizing the exact task cost as in Equation \ref{Eq:a.s.SafeLLM-main}, \Cref{thm:a.s.} follows.
\end{proof}




















\section{\thename}
\subsection{End-to-End Driving Policy}
The overall framework of \thename{} is depicted in Fig.~\ref{fig:framework}. 
\thename{} takes multi-view image sequences as input, transforms the sensor data into scene token embeddings, outputs the probabilistic distribution of actions, and samples an action to control the vehicle. 

\boldparagraph{BEV Encoder.} 
We first employ a BEV encoder~\cite{li2022bevformer} to transform multi-view image features from the perspective view to the Bird's Eye View (BEV), obtaining a feature map in the BEV space. This feature map is then used to learn instance-level map features and agent features.

\boldparagraph{Map Head.} 
Then we utilize a group of map tokens~\cite{maptrv2, liao2022maptr, lanegap} to learn the vectorized map elements of the driving scene from the BEV feature map, including lane centerlines, lane dividers, road boundaries, arrows, traffic signals, \etc.

\boldparagraph{Agent Head.} 
Besides, a group of agent tokens~\cite{jiang2022pip} is adopted to predict the motion information of other traffic participants, including location, orientation, size, speed, and multi-mode future trajectories.

\boldparagraph{Image Encoder.} 
Apart from the above instance-level map and agent tokens, we also use an individual image encoder~\cite{vit,he2016resnet} to transform the original images into image tokens. These image tokens provide dense and rich scene information for planning, complementary to the instance-level tokens.

\begin{figure}[t]
\centering
\includegraphics[width=0.98\linewidth]{fig/post-training-2.pdf} 
\caption{\textbf{Post-training.}  $N$  workers parallelly run. The generated rollout data $(s_t,a_t, r_{t+1},s_{t+1},...)$ are recorded in a rollout buffer. Rollout data and human driving demonstrations are used in RL- and IL-training steps to fine-tune the AD policy synergistically.
}
\label{fig:post-training}
\end{figure}

\boldparagraph{Action Space.} 
To accelerate the convergence of RL training, we design a decoupled discrete action representation. 
We divide the action into two independent components: lateral action and longitudinal action. 
The action space is constructed over a short $0.5$-second time horizon, during which the vehicle's motion is approximated by assuming constant linear and angular velocities. 
Under this assumption, the lateral action $a^x$ and longitudinal action $a^y$ can be directly computed based on the current linear and angular velocities.
By combining decoupling with a limited temporal scope and simplified motion model, our approach effectively reduces the dimensionality of the action space, accelerating training convergence.


\boldparagraph{Planning Head.} 
We use $E_\text{scene}$ to denote the scene representation, which consists of map tokens, agent tokens, and image tokens. We initialize a planning embedding denoted as $E_\text{plan}$. A cascaded Transformer decoder $\phi$ takes the planning embedding $E_\text{plan}$ as the query and the scene representation $E_\text{scene}$ as both key and value.

The output of the decoder $\phi$ is then combined with navigation information $E_\text{navi}$ and ego state $E_\text{state}$ to output the probabilistic distributions of the lateral action $a^x$ and the longitudinal action $a^y$:
\begin{equation}
\begin{aligned}
     \pi(a^x\mid s) = & \text{softmax}(\text{MLP}(\phi(E_\text{plan}, E_\text{scene}) \\
    & + E_\text{navi} + E_\text{state})), \\
     \pi(a^y\mid s) = & \text{softmax}(\text{MLP}(\phi(E_\text{plan}, E_\text{scene}) \\
     & + E_\text{navi} + E_\text{state})),
\label{eq:action distribution}
\end{aligned}
\end{equation}
where $E_\text{plan}$, $E_\text{navi}$, $E_\text{state}$, and the output of $\text{MLP}$ are all of the same dimension ($1 \times D$).

The planning head also outputs the value functions $V_x(s)$ and $V_y(s)$, which estimate the expected cumulative rewards for the lateral and longitudinal actions, respectively: 
\begin{equation}
\begin{aligned}
    & V_x(s) = \text{MLP}(\phi(E_\text{plan}, E_\text{scene}) + E_\text{navi} + E_\text{state}), \\
    & V_y(s) = \text{MLP}(\phi(E_\text{plan}, E_\text{scene}) + E_\text{navi} + E_\text{state}).
\end{aligned}
\end{equation}
The value functions are used in RL training (Sec.~\ref{sec:optimization}).

\subsection{Training Paradigm}
We adopt a three-stage training paradigm: perception pre-training, planning pre-training, and reinforced post-training, as shown in Fig.~\ref{fig:framework}.

\boldparagraph{Perception Pre-Training.} 
Information in the image is sparse and low-level. In the first stage,  
the map head and the agent head explicitly output map elements and agent motion information, which are supervised with ground-truth labels. Consequently,  
map tokens and agent tokens implicitly encode the corresponding high-level information.  
In this stage, we only update the parameters of the BEV encoder, the map head, and the agent head.



\boldparagraph{Planning Pre-Training.} 
In the second stage, to prevent the unstable cold start of RL training, IL is first performed to initialize the probabilistic distribution of actions based on large-scale real-world driving demonstrations from expert drivers. In this stage, we only update the parameters of the image encoder and the planning head, while the parameters of the BEV encoder, map head, and agent head are frozen. The optimization objectives of perception tasks and planning tasks may conflict with each other. However, with the training stage and parameters decoupled, such conflicts are mostly avoided.

\boldparagraph{Reinforced Post-Training.} 
In the reinforced post-training, RL and IL synergistically fine-tune the distribution. RL aims to guide the policy to be sensitive to critical risky events and adaptive to out-of-distribution situations. IL serves as the regularization term to keep the policy's behavior similar to that of humans.

We select a large amount of risky dense-traffic clips from collected driving demonstrations. For each clip, we train an independent 3DGS model that reconstructs the clip and serves as a digital driving environment.  
As shown in Fig.~\ref{fig:post-training}, we set $N$ parallel workers.  
Each worker randomly samples a 3DGS environment and begins rollout, i.e., the AD policy controls the ego vehicle to move and iteratively interacts with the 3DGS environment. After the rollout process of this 3DGS environment ends, the generated rollout data $(s_t,a_t, r_{t+1},s_{t+1},...)$ are recorded in a rollout buffer, and the worker will sample a new 3DGS environment for another round of rollout.

As for policy optimization, we iteratively perform RL-training steps and IL-training steps. For RL-training steps, we sample data from the rollout buffer and follow the Proximal Policy Optimization (PPO) framework~\cite{PPO} to update the AD policy. For IL-training steps, we use real-world driving demonstrations to update the policy. After a fixed number of training steps, the updated AD policy is sent to every worker to replace the old one, to avoid a distribution shift between data collection and optimization.
We only update the parameters of the image encoder and the planning head. The parameters of the BEV encoder, the map head, and the agent head are frozen.  
The detailed RL design is presented below.

\subsection{Interaction Mechanism between AD Policy and 3DGS Environment}
In the 3DGS environment, the ego vehicle acts according to the AD policy. Other traffic participants act according to real-world data in a log-replay manner.  
A simplified kinematic bicycle model is employed to iteratively update the ego vehicle's pose at every $\Delta t$ seconds as follows:  
\begin{equation}
\begin{aligned}
x_{t+1}^{w} & = x_{t}^w + v_t \cos \left(\psi_{t}^w\right) \Delta t, \\
y_{t+1}^{w} & = y_{t}^w + v_t \sin \left(\psi_{t}^w\right) \Delta t, \\
\psi_{t+1}^{w} & = \psi_{t}^w + \frac{v_t}{L} \tan \left(\delta_t\right) \Delta t,
\label{equation:kinematic_model}
\end{aligned}
\end{equation}  
where $x_t^{w}$ and $y_t^{w}$ denote the position of the ego vehicle relative to the world coordinate; $\psi_t^w$ is the heading angle that defines the vehicle's orientation with respect to the world $x$-coordinate; $v_t$ is the linear velocity of the ego vehicle; $\delta_t$ is the steering angle of the front wheels; and $L$ is the wheelbase, i.e., the distance between the front and rear axles.

During the rollout process, the AD policy outputs actions $(a_t^x, a_t^y)$ for a $0.5$-second time horizon at time step $t$. We derive the linear velocity $v_t$ and steering angle $\delta_t$ based on $(a_t^x, a_t^y)$.  
Based on the kinematic model in Eq.~\ref{equation:kinematic_model},  
the pose of the ego vehicle in the world coordinate system is updated from ${p}_t = (x_{t}^w, y_{t}^w, \psi_{t}^w)$ to ${p}_{t+1} = (x_{t+1}^{w}, y_{t+1}^{w}, \psi_{t+1}^{w})$.  

Based on the updated ${p}_{t+1}$, the 3DGS environment computes the new ego vehicle's state $s_{t+1}$. The updated pose ${p}_{t+1}$ and state $s_{t+1}$ serve as the input for the next iteration of the inference process.

The 3DGS environment also generates rewards $\mathcal{R}$ (Sec.~\ref{sec:reward}) according to multi-source information (including trajectories of other agents, map information, the expert trajectory of the ego vehicle, and the parameters of Gaussians), which are used to optimize the AD policy (Sec.~\ref{sec:optimization}).

\begin{figure}[t]
\centering
\includegraphics[width=1.0\linewidth]{fig/reward.pdf} 
\caption{\textbf{Example diagram of four types of reward sources.}  (1): Collision with a dynamic obstacle ahead triggers a reward $r_{\text{dc}}$. (2): Hitting a static roadside obstacle incurs a reward $r_{\text{sc}}$. (3): Moving onto the curb exceeds the positional deviation threshold $d_{\text{max}}$, triggering a reward $r_{\text{pd}}$. (4): Drifting toward the adjacent lane exceeds the heading deviation threshold $\psi_{\text{max}}$, triggering a reward $r_{\text{hd}}$.
}
\label{fig: reward source}
\end{figure}
\subsection{Reward Modeling}
\label{sec:reward}
The reward is the source of the training signal, which determines the optimization direction of RL. The reward function is designed to guide the ego vehicle's behavior by penalizing unsafe actions and encouraging alignment with the expert trajectory. It is composed of four reward components: (1) collision with dynamic obstacles, (2) collision with static obstacles, (3) positional deviation from the expert trajectory, and (4) heading deviation from the expert trajectory:
\begin{equation}
\begin{aligned}
\mathcal{R} = \{r_{\text{dc}}, r_{\text{sc}}, r_{\text{pd}}, r_{\text{hd}}  \}. 
\end{aligned}
\end{equation}

As illustrated in Fig.~\ref{fig: reward source}, these reward components are triggered under specific conditions.  
In the 3DGS environment, dynamic collision is detected if the ego vehicle's bounding box overlaps with the annotated bounding boxes of dynamic obstacles, triggering a negative reward $r_{\text{dc}}$. Similarly, static collision is identified when the ego vehicle's bounding box overlaps with the Gaussians of static obstacles, resulting in a negative reward $r_{\text{sc}}$.  
Positional deviation is measured as the Euclidean distance between the ego vehicle's current position and the closest point on the expert trajectory. A deviation beyond a predefined threshold $d_{\text{max}}$ incurs a negative reward $r_{\text{pd}}$.  
Heading deviation is calculated as the angular difference between the ego vehicle's current heading angle $ \psi_t $ and the expert trajectory's matched heading angle $\psi_{\text{expert}}$. A deviation beyond a threshold $ \psi_{\text{max}}$ results in a negative reward $r_{\text{hd}}$.

Any of these events, including dynamic collision, static collision, excessive positional deviation, or excessive heading deviation, triggers immediate episode termination. Because after such events occur, the 3DGS environment typically generates noisy sensor data, which is detrimental to RL training.

\subsection{Policy Optimization}
\label{sec:optimization}
In the closed-loop environment, the error in each single step accumulates over time. The aforementioned rewards are not only caused by the current action but also by the actions of the preceding steps.  
The rewards are propagated forward with Generalized Advantage Estimation (GAE)~\cite{gae} to optimize the action distribution of the preceding steps.

Specifically, for each time step $t$, we store the current state $s_t$, action $a_t$, reward $r_t$, and the estimate of the value $V(s_t)$.  
Based on the decoupled action space, and considering that different rewards have different correlations to lateral and longitudinal actions, the reward $r_t$ is divided into lateral reward $r_t^x$ and longitudinal reward $r_t^y$:
\begin{equation}
\begin{aligned}
r_t^x &= r_t^{\text{sc}} + r_t^{\text{pd}} + r_t^{\text{hd}}, \\
r_t^y &= r_t^{\text{dc}}.
\label{eq:reward-decouple}
\end{aligned}
\end{equation}
Similarly, the value function $V(s_t)$ is decoupled into two components: $V_x(s_t)$ for the lateral dimension and $V_y(s_t)$ for the longitudinal dimension. These value functions estimate the expected cumulative rewards for the lateral and longitudinal actions, respectively. The advantage estimates $\hat{A}_t^x$ and $\hat{A}_t^y$ are then computed as follows:
\begin{equation}
\begin{aligned}
\delta_t^x &= r_t^x + \gamma V_x(s_{t+1}) - V_x(s_t), \\
\delta_t^y &= r_t^y + \gamma V_y(s_{t+1}) - V_y(s_t), \\
\hat{A}_t^x &= \sum_{l=0}^{\infty}(\gamma \lambda)^l \delta_{t+l}^x, \\
\hat{A}_t^y &= \sum_{l=0}^{\infty}(\gamma \lambda)^l \delta_{t+l}^y,
\label{eq:advantage}
\end{aligned}
\end{equation}
where $\delta_t^x$ and $\delta_t^y$ are the temporal difference errors for the lateral and longitudinal dimensions, $\gamma$ is the discount factor, and $\lambda$ is the GAE parameter that controls the trade-off between bias and variance.

To further clarify the relationship between the advantage estimates and the reward components, we decompose $\hat{A}_t^x$ and $\hat{A}_t^y$ based on the reward decomposition in Eq.~\ref{eq:reward-decouple} and the advantage estimation in Eq.~\ref{eq:advantage}. Specifically, we derive the following decomposition:
\begin{equation}
\begin{aligned}
\hat{A}_t^x &= \hat{A}_t^{\text{sc}} + \hat{A}_t^{\text{pd}} + \hat{A}_t^{\text{hd}}, \\
\hat{A}_t^y &= \hat{A}_t^{\text{dc}},
\end{aligned}
\end{equation}
where $\hat{A}_t^{\text{sc}}$ is the advantage estimate for avoiding static collisions, $\hat{A}_t^{\text{pd}}$ is the advantage estimate for minimizing positional deviations, $\hat{A}_t^{\text{hd}}$ is the advantage estimate for minimizing heading deviations, and $\hat{A}_t^{\text{dc}}$ is the advantage estimate for avoiding dynamic collisions.

These advantage estimates are used to guide the update of the AD policy $\pi_{\theta}$, following the PPO framework~\cite{PPO}. By leveraging the decomposed advantage estimates $\hat{A}_t^x$ and $\hat{A}_t^y$, we can independently optimize the lateral and longitudinal dimensions of the policy. This is achieved by defining separate objective functions $\mathcal{L}_x^{\text{CLIP}}(\theta)$ and $\mathcal{L}_y^{\text{CLIP}}(\theta)$ for each dimension,  as follows:
\begin{equation}
\begin{aligned}
\mathcal{L}_x^{\text{PPO}}(\theta) &= \mathbb{E}_t \left[ \min \left( \rho_t^x \hat{A}_t^x, \ \text{clip}(\rho_t^x, 1-\epsilon_x, 1+\epsilon_x) \hat{A}_t^x \right) \right], \\
\mathcal{L}_y^{\text{PPO}}(\theta) &= \mathbb{E}_t \left[ \min \left( \rho_t^y \hat{A}_t^y, \ \text{clip}(\rho_t^y, 1-\epsilon_y, 1+\epsilon_y) \hat{A}_t^y \right) \right], \\
\mathcal{L}^{\text{PPO}}(\theta) &= \mathcal{L}_x^{\text{PPO}}(\theta) + \mathcal{L}_y^{\text{PPO}}(\theta),
\end{aligned}
\end{equation}
where $\rho_t^x = \frac{\pi_{\theta}(a_t^x \mid s_t)}{\pi_{\theta_{\text{old}}}(a_t^x \mid s_t)}$ is the importance sampling ratio for the lateral dimension, $\rho_t^y = \frac{\pi_{\theta}(a_t^y \mid s_t)}{\pi_{\theta_{\text{old}}}(a_t^y \mid s_t)}$ is the importance sampling ratio for the longitudinal dimension, $\epsilon_x$ and $\epsilon_y$ are small constants that control the clipping range for the lateral and longitudinal dimensions, ensuring stable policy updates.

The clipped objective function $\mathcal{L}^{\text{PPO}}(\theta)$ prevents excessively large updates to the policy parameters $\theta$, thereby maintaining training stability.

\begin{table*}[ht]
    \centering
{
\begin{tabular}{lccccccccc}
    \toprule
    RL:IL & CR$\downarrow$ & DCR$\downarrow$ & SCR$\downarrow$ & DR$\downarrow$ & PDR$\downarrow$ & HDR$\downarrow$ &ADD$\downarrow$ & Long. Jerk$\downarrow$ & Lat. Jerk$\downarrow$ \\
    \midrule
     0:1  & 0.229 & 0.211 & 0.018 & 0.066 & 0.039 & 0.027  & 0.238 & 3.928 & 0.103\\
     1:0  & 0.143 & 0.128 & 0.015 &0.080 &0.065 &0.015 &0.345 &4.204 &0.085\\
     2:1 & 0.137 & 0.125 & 0.012 & 0.059 & 0.050 & 0.009  & 0.274 & 4.538 & 0.092\\
     4:1 & 0.089 & 0.080 & 0.009 & 0.063 & 0.042 & 0.021  & 0.257 & 4.495 & 0.082 \\
     8:1 & 0.125 & 0.116 & 0.009 & 0.084 & 0.045 & 0.039  & 0.323 & 5.285 & 0.115\\
    \bottomrule
\end{tabular}
}
    \caption{\textbf{Ablation on RL-to-IL step mixing ratios in the reinforced post-training stage.}}
    \label{tab:ratio}
\end{table*}

\subsection{Auxiliary Objective}
RL usually faces the challenge of sparse rewards, which makes the convergence process unstable and slow. To speed up convergence, we introduce auxiliary objectives that provide dense guidance to the entire action distribution.

The auxiliary objectives are designed to penalize undesirable behaviors by incorporating specific reward sources, including dynamic collisions, static collisions, positional deviations, and heading deviations. These objectives are computed based on the actions \( a_t^{x, \text{old}} \) and \( a_t^{y, \text{old}} \) selected by the old AD policy \( \pi_{\theta_{\text{old}}} \) at time step \( t \). To facilitate the evaluation of these actions, we separate the probability distribution of the action into four parts:
\begin{equation}
\begin{aligned}
\Delta \pi_y^{\text{dec}} &= \sum_{a_t^y < a_t^{y, \text{old}}} \pi_\theta(a_t^y \mid s_t), \\
\Delta \pi_y^{\text{acc}} &= \sum_{a_t^y > a_t^{y, \text{old}}} \pi_\theta(a_t^y \mid s_t), \\
\Delta \pi_x^{\text{left}} &= \sum_{a_t^x < a_t^{x, \text{old}}} \pi_\theta(a_t^x \mid s_t), \\
\Delta \pi_x^{\text{right}} &= \sum_{a_t^x > a_t^{x, \text{old}}} \pi_\theta(a_t^x \mid s_t).
\end{aligned}
\end{equation}
Here, \( \Delta \pi_y^{\text{dec}} \) represents the total probability of deceleration actions, \( \Delta \pi_y^{\text{acc}} \) represents the total probability of acceleration actions, \( \Delta \pi_x^{\text{left}} \) represents the total probability of leftward steering actions, and \( \Delta \pi_x^{\text{right}} \) represents the total probability of rightward steering actions.

\boldparagraph{Dynamic Collision Auxiliary Objective.}  
The dynamic collision auxiliary objective adjusts the longitudinal control action \(a_t^y\) based on the location of potential collisions relative to the ego vehicle. If a collision is detected ahead, the policy prioritizes deceleration actions (\(a_t^y < a_t^{y, \text{old}}\)); if a collision is detected behind, it encourages acceleration actions (\(a_t^y > a_t^{y, \text{old}}\)). To formalize this behavior, we define a directional factor \(f_\text{dc}\):
\begin{equation}
\begin{aligned}
f_\text{dc} = \begin{cases} 
1 & \text{if the collision is ahead}, \\
-1 & \text{if the collision is behind}.
\end{cases} 
\end{aligned}
\end{equation}

The auxiliary objective for dynamic collision avoidance is defined as:
\begin{equation}
\begin{aligned}
\mathcal{L}_\text{dc}(\theta_y) = \mathbb{E}_t \left[ 
    \hat{A}_t^\text{dc} \cdot f_\text{dc} \cdot (\Delta \pi_y^{\text{dec}} - \Delta \pi_y^{\text{acc}})
\right],
\end{aligned}
\end{equation}
where \(\hat{A}_t^\text{dc}\) is the advantage estimate for dynamic collision avoidance.

\boldparagraph{Static Collision Auxiliary Objective.}  
The static collision auxiliary objective adjusts the steering control action $a_t^x$ based on the proximity to static obstacles. If the static obstacle is detected on the left side, the policy promotes rightward steering actions ($a_t^x > a_t^{x,\text{old}}$); if the static obstacle is detected on the right side, it promotes leftward steering actions ($a_t^x < a_t^{x,\text{old}}$). To formalize this behavior, we define a directional factor $f_\text{sc}$:  
\begin{equation}
\begin{aligned}
f_\text{sc} = \begin{cases} 
1 & \text{if static obstacle is on the left}, \\
-1 & \text{if static obstacle is on the right}.
\end{cases} 
\end{aligned}
\end{equation}

The auxiliary objective for static collision avoidance is defined as:  
\begin{equation}
\begin{aligned}
\mathcal{L}_\text{sc}(\theta_x) = \mathbb{E}_t \left[ 
    \hat{A}_t^\text{sc} \cdot f_\text{sc} \cdot (\Delta \pi_x^{\text{right}} - \Delta \pi_x^{\text{left}})
\right],
\end{aligned}
\end{equation}  
where $\hat{A}_t^\text{sc}$ is the advantage estimate for static collision avoidance.  

\boldparagraph{Positional Deviation Auxiliary Objective.}  
The positional deviation auxiliary objective adjusts the steering control action $a_t^x$ based on the ego vehicle's lateral deviation from the expert trajectory. If the ego vehicle deviates leftward, the policy promotes rightward corrections ($a_t^x > a_t^{x,\text{old}}$); if it deviates rightward, it promotes leftward corrections ($a_t^x < a_t^{x,\text{old}}$). We formalize this with a directional factor $f_\text{pd}$:  
\begin{equation}
\begin{aligned}
f_\text{pd} = \begin{cases} 
1 & \text{if ego vehicle deviates leftward}, \\
-1 & \text{if ego vehicle deviates rightward}.
\end{cases} 
\end{aligned}
\end{equation}

The auxiliary objective for positional deviation correction is:
\begin{equation}
\begin{aligned}
\mathcal{L}_\text{pd}(\theta_x) = \mathbb{E}_t \left[ 
    \hat{A}_t^\text{pd} \cdot f_\text{pd} \cdot (\Delta \pi_x^{\text{right}} - \Delta \pi_x^{\text{left}})
\right],
\end{aligned}
\end{equation}  
where $\hat{A}_t^\text{pd}$ estimates the advantage of trajectory alignment.

\boldparagraph{Heading Deviation Auxiliary Objective.}  
The heading deviation auxiliary objective adjusts the steering control action $a_t^x$ based on the angular difference between the ego vehicle’s current heading and the expert’s reference heading. If the ego vehicle deviates counterclockwise, the policy promotes clockwise corrections ($a_t^x > a_t^{x,\text{old}}$); if it deviates clockwise, it promotes counterclockwise corrections ($a_t^x < a_t^{x,\text{old}}$). To formalize this behavior, we define a directional factor $f_\text{hd}$:  
\begin{equation}
\begin{aligned}
f_\text{hd} = \begin{cases} 
1 & \text{if ego vehicle deviates clockwise}, \\
-1 & \text{if ego vehicle deviates counterclockwise}.
\end{cases} 
\end{aligned}
\end{equation}

The auxiliary objective for heading deviation correction is then defined as:  
\begin{equation}
\begin{aligned}
\mathcal{L}_\text{hd}(\theta_x) = \mathbb{E}_t \left[ 
    \hat{A}_t^\text{hd} \cdot f_\text{hd} \cdot (\Delta \pi_x^{\text{right}} - \Delta \pi_x^{\text{left}})
\right],
\end{aligned}
\end{equation}  
where $\hat{A}_t^\text{hd}$ is the advantage estimate for heading alignment.  

\begin{table*}[ht]
\begin{center}
\centering
\resizebox{0.98\textwidth}{!}{
\begin{tabular}{cccccccccccccc}
\toprule
\multirow{2}{*}{ID} & Dynamic & Static & Position & Heading & \multirow{2}{*}{CR$\downarrow$} &\multirow{2}{*}{DCR$\downarrow$} &\multirow{2}{*}{SCR$\downarrow$} &\multirow{2}{*}{DR$\downarrow$} &\multirow{2}{*}{PDR$\downarrow$} &\multirow{2}{*}{HDR$\downarrow$} &\multirow{2}{*}{ADD$\downarrow$} &\multirow{2}{*}{Long. Jerk$\downarrow$} &\multirow{2}{*}{Lat. Jerk$\downarrow$}\\
& Collision & Collision & Deviation & Deviation & & & & & & & & & \\
\midrule
1 & \cmark  &  &  &  & 0.172 & 0.154 & 0.018 & 0.092 & 0.033 & 0.059  & 0.259 & 4.211 & 0.095 \\
2 &  & \cmark & \cmark & \cmark & 0.238 & 0.217 & 0.021 & 0.090 & 0.045 & 0.045  & 0.241 & 3.937 & 0.098 \\
3 & \cmark &  & \cmark & \cmark & 0.146 & 0.128 & 0.018 & 0.060 & 0.030 & 0.030  & 0.263 & 3.729 & 0.083\\
4 & \cmark & \cmark &  & \cmark & 0.151 & 0.142 & 0.009 & 0.069 & 0.042 & 0.027 & 0.303 & 3.938 & 0.079\\
5 & \cmark & \cmark & \cmark &  & 0.166 & 0.157 & 0.009 & 0.048 & 0.036 & 0.012 & 0.243 & 3.334 & 0.067\\
6 & \cmark & \cmark & \cmark & \cmark & 0.089 & 0.080 & 0.009 & 0.063 & 0.042 & 0.021 & 0.257 & 4.495 & 0.082 \\
\bottomrule
\end{tabular}
}
\end{center}
\vspace{-2mm}
\caption{\textbf{Ablation on reward sources.} The table shows the impact of different reward components on performance.}
\label{tab:reward_ablation}
\end{table*}

\begin{table*}[ht]
\begin{center}
\centering
\resizebox{0.98\textwidth}{!}{
\begin{tabular}{ccccccccccccccc}
\toprule
\multirow{2}{*}{ID} & \multirow{2}{*}{PPO Obj.}  & Dynamic Col. & Static Col. & Position Dev. & Heading Dev. & \multirow{2}{*}{CR$\downarrow$} & \multirow{2}{*}{DCR$\downarrow$}  & \multirow{2}{*}{SCR$\downarrow$} & \multirow{2}{*}{DR$\downarrow$} & \multirow{2}{*}{PDR$\downarrow$} & \multirow{2}{*}{HDR$\downarrow$} & \multirow{2}{*}{ADD$\downarrow$} & \multirow{2}{*}{Long. Jerk$\downarrow$} & \multirow{2}{*}{Lat. Jerk$\downarrow$} \\
& & Auxiliary Obj. & Auxiliary Obj. & Auxiliary Obj. & Auxiliary Obj. & & & & & & & & & \\
\midrule
1 &\cmark&  &  &  &  & 0.249 & 0.223 & 0.026 & 0.077 & 0.047 & 0.030  & 0.266 & 4.209 & 0.104 \\
2 &\cmark& \cmark &  &  &  & 0.178 & 0.163 & 0.015 & 0.151 & 0.101 & 0.050 & 0.301 & 3.906 & 0.085 \\
3 &\cmark&  & \cmark & \cmark & \cmark & 0.137 & 0.125 & 0.012 & 0.157 & 0.145 & 0.012 & 0.296 & 3.419 & 0.071 \\
4 &\cmark& \cmark &  & \cmark & \cmark & 0.169 & 0.151 & 0.018 & 0.075 & 0.042 & 0.033 & 0.254 & 4.450 & 0.098 \\
5 &\cmark& \cmark & \cmark &  & \cmark & 0.149 & 0.134 & 0.015 & 0.063 & 0.057 & 0.006 & 0.324 & 3.980 & 0.086 \\
6 &\cmark& \cmark & \cmark & \cmark & & 0.128 & 0.119  & 0.009 & 0.066 & 0.030 & 0.036  & 0.254 & 4.102 & 0.092 \\
7 &&\cmark  &\cmark  &\cmark  &\cmark  & 0.187 &0.175  &0.012 &0.077 &0.056  &0.021  &0.309  &5.014  &0.112  \\
8 &\cmark& \cmark & \cmark & \cmark & \cmark & 0.089 & 0.080 & 0.009 & 0.063 & 0.042 & 0.021  & 0.257 & 4.495 & 0.082 \\
\bottomrule
\end{tabular}
}
\end{center}
\vspace{-2mm}
\caption{\textbf{Ablation on auxiliary objectives.} The table shows the impact of different auxiliary objectives on performance.}
\label{tab:auxiliary_ablation}
\end{table*}

\boldparagraph{Overall Auxiliary Objectives.}  
The overall auxiliary objectives are a weighted sum of the individual objectives:
\begin{equation}
\begin{aligned}
\mathcal{L}_\text{aux}(\theta) = &\lambda_1 \mathcal{L}_\text{dc}(\theta_y) + \lambda_2 \mathcal{L}_\text{sc}(\theta_x)  + \\ 
&\lambda_3 \mathcal{L}_\text{pd}(\theta_x) +\lambda_4 \mathcal{L}_\text{hd}(\theta_x),
\end{aligned}
\end{equation}
where $\lambda_1$, $\lambda_2$, $\lambda_3$, and $\lambda_4$ are weighting coefficients that balance the contributions of each auxiliary objective.

\boldparagraph{Optimization Objective.}  
The final optimization objective combines the clipped PPO objective with the auxiliary objective:
\begin{equation}
\mathcal{L}(\theta) = \mathcal{L}^{\text{PPO}}(\theta) + \mathcal{L}_\text{aux}(\theta).
\end{equation}

\section{Experiments: Planning outperforms Heuristics}
\label{sec:experiment}

We begin our empirical demonstrations by showcasing the effectiveness of our planning framework on both synthetic and real datasets. We focus on the simplest planning algorithm, 1-step lookaheads (Algorithm~\ref{alg:complete}), and show that even basic planning can hold great promise. 
We illustrate our framework using two uncertainty quantification modules---GPs and 
\ensembles/ \ensembleplus. 

Throughout this section, we focus on evaluating the mean squared error of 
a regression model $\model$,  and develop adaptive policies that minimize uncertainty on $g(f)$ defined in~\eqref{eqn:l2-g-f}.
When GPs provide a valid model of uncertainty, 
our experiments show that our planning framework significantly outperforms other baselines. 
We further demonstrate that our conceptual framework extends to deep learning-based uncertainty quantification methods such as  \ensembleplus while highlighting computational challenges that need to be resolved in order to scale our ideas. 
For simplicity, we assume a naive predictor, i.e., $\psi(\cdot) \equiv 0$. However, we emphasize that this problem is just as complex as if we were using a sophisticated model $\psi(.)$. The performance gap between the algorithms 
primarily depends
on the level  of uncertainty in our prior beliefs.

To evaluate the performance of our algorithm, we benchmark it against several baselines. 
%Active learning baselines use an acquisition function $\ac$ to select points that have the highest   function value: $X\opt_t \in \argmax_{X \in \xpoolj{t}} \ac({X})$ at every step $t$. These methods may also need an UQ module, which we simply use the same UQ module as in our algorithm, and it  outputs $V(X)$ that measures the the uncertainty of each point $X \in \xpoolj{t}$.
Our first set of baselines are from active learning~\citep{AggarwalKoGuHaPh14}:
\\ % \noindent\textbf{Active Learning Heuristics:} 
\textbf{(1)} 
\textsf{Uncertainty Sampling (Static):}  In this approach, we query the samples for which the model is least certain about. Specifically, we estimate the variance of the latent output $f(X)$ for each $X \in \xpool$ using the UQ module and select the top-$K$ points with the highest uncertainty. \\
\textbf{(2)} \textsf{Uncertainty Sampling (Sequential):} This is a greedy heuristic that sequentially selects the points with the highest uncertainty within a batch, while updating the posterior beliefs using pseudo labels from the current posterior state. Unlike \textsf{Uncertainty Sampling (Static)}, this method takes into account the information gained from each point within batch, and hence tries to diversify the selected points within a batch. 

 
We also compare our approach to the  \textbf{(3)} \textsf{Random Sampling}, which selects each batch uniformly at random from the pool. Additionally, we compare solving the planning problem using  \textsf{REINFORCE}-based policy gradients with   $\mathsf{Smoothed\text{-}Autodiff}$ policy gradients.\footnote{Our code repository is available at
  \url{https://github.com/namkoong-lab/adaptive-labeling}.}
%Detailed experimental setups are provided in Section \ref{sec:details-experiments}.

%We repeat all experiments with 10 random seeds.




\begin{figure}[t]
\centering
\begin{minipage}[b]{0.49\textwidth}
\centering
\includegraphics[width=\textwidth, height=5cm]{figures/original_scale/Var_of_l_2_loss.pdf}
\caption{(Synthetic data) Variance of mean squared loss evaluated through the posterior belief $\mu_t$ at each horizon $t$. This is the objective that policy gradient methods like \textsf{REINFORCE} and $\ouralgo$ optimizes. 1-step lookaheads are surprisingly effective even in long horizons.}
\label{fig:var-l2-sim}
\end{minipage}
\hfill
\begin{minipage}[b]{0.49\textwidth}
\centering \includegraphics[width=\textwidth, height=5cm]{figures/original_scale/Error_of_estimated_model_l_2_loss.pdf}
\caption{(Synthetic data) Error between MSE calculated based on collected data $\mc{D}^{0:T}$ vs. population oracle MSE over $\mc{D}_{\rm eval} \sim P_X$. Reducing uncertainty over posteriors directly leads to better OOD evaluations. 1-step lookaheads significantly outperform active learning heuristics in small horizons.}
\label{fig:mean-l2-sim}
\end{minipage}
%\caption{Simulated data for GPs}
%\label{fig:both_plots}
\end{figure}

\subsection{Planning with Gaussian processes}
\label{sec:experiment-plan-GP}
We now briefly describe the data generation process for the GP experiments,  deferring a more detailed discussion of the dataset generation to Section~\ref{sec:details-experiments}. 
We use both the synthetic data and the real data to test our methodology.
For the \emph{simulated data},  we construct a setting where the general population is distributed across \emph{51 non-overlapping clusters} while the initial labeled data $\dtrain$ just comes from one cluster. In contrast, both $\dpool \defeq (\xpool,\ypool),\deval \defeq (\xeval,\yeval)$ are generated   from all the clusters. 
We begin with a low-dimensional scenario, generating a one-dimensional regression setting using a GP. %Gaussian Process (GP).
Although the data-generating process is not known to the algorithms,  we assume that the GP hyperparameters are known to all the algorithms
to ensure fair comparisons. This can be viewed as a setting where our prior is well-specified, allowing us to isolate the effects
of different policy optimization approaches
 without any concerns about the misspecified priors. We select $10$ batches, each of size $K=5$ across $T = 10$ time horizons.

To examine the robustness of our method against the distributional assumptions made  in the simulated case, we then move to a real dataset where the correct prior is not known. We simulate selection bias from the eICU dataset~\citep{PollardJoRaCeMaBa18}, which contains real-world patient data with in-hospital mortality outcomes. 
We conduct a $k$-means clustering to generate 51 clusters and then select data from those clusters. We view this to be a credible replication of practice, as severe distribution shifts are common due to selection bias in clinical labels.  To convert the binary mortality labels into a regression setting, we train a  random forest classifier and fit a GP on predicted scores, which serves as the UQ module for all the algorithms. As before, the task is to select 10 batches, each consisting of 5 samples, across 10 time horizons.

 In Figures~\ref{fig:var-l2-sim} and~\ref{fig:mean-l2-sim}, we present results for the simulated data. 
Figure~\ref{fig:var-l2-sim} shows the variance of $\ell_2$ loss, and Figure~\ref{fig:mean-l2-sim} presents the error in the estimated $\ell_2$ loss using $\mu_t$ (relative to true $\ell_2$ loss, that is unknown to the algorithm). 
As we can see from these plots, our method one-step lookahead  gives substantial improvements  over active learning baselines and random sampling. In addition,
compared to the one-step lookahead planning approach using \textsf{REINFORCE}-based policy gradients, 
we observe that $\mathsf{Smoothed\text{-}Autodiff}$-based policy gradients provide significantly more robust performance over all horizons.

In Figures~\ref{fig:var-l2-real}~and~\ref{fig:mean-l2-real}, we observe similar findings on the eICU data. We see that planning policies (\textsf{REINFORCE} and $\mathsf{Smoothed\text{-}Autodiff}$) consistently outperform other heuristics by a large margin.  Active learning baselines perform poorly in these small-horizon batched problems and can sometimes be even worse than the random search baselines.  Overall, our results show the importance of careful planning in adaptive labeling for reliable model evaluation. 

We offer some intuition as to why one-step lookahead planning may outperform other heuristic algorithms. 
 First,  \textsf{Uncertainty sampling (Static)} while myopically selects the
 top-$K$ inputs with the highest uncertainty, it fails to consider 
the overlap in information content among the ``best” instances; see \citep{AggarwalKoGuHaPh14} for more details. 
In other words,  it might acquire points from the same region with high uncertainty while failing to induce diversity among the batch.
Although \textsf{Uncertainty Sampling (Sequential)} somewhat addresses the issue of information overlap, a significant drawback of 
this algorithm
is the disconnect between the objective we aim to optimize and the algorithm. For example, it might sample from a region with high uncertainty but very low density. 

\begin{figure}[t]
\centering
\begin{minipage}[b]{0.48\textwidth}
\centering
\includegraphics[width=\textwidth, height=5cm]{figures/original_scale/Var_of_l_2_loss_real.pdf}
\caption{(Real-world eICU data) Variance of mean squared loss evaluated through the posterior belief $\mu_t$ at each horizon $t$. Even 1-step lookaheads are extremely effective planners, and auto-differentiation-based pathwise policy gradients provide a reliable optimization algorithm based on low-variance gradient estimates.}
\label{fig:var-l2-real}
\end{minipage}
\hfill
\begin{minipage}[b]{0.48\textwidth}
\centering \includegraphics[width=\textwidth, height=5cm]{figures/original_scale/Error_of_estimated_model_l_2_loss_real.pdf}
\caption{(Real-world eICU data) Error between MSE calculated based on collected data $\mc{D}^{0:T}$ vs. population oracle MSE over $\mc{D}_{\rm eval} \sim P_X$. Reducing uncertainty over posteriors directly leads to better OOD evaluations. Our method significantly outperforms active learning-based heuristics, and random sampling.}
\label{fig:mean-l2-real}
\end{minipage}
%\caption{Real data for GPs}
\end{figure}
 
%\vspace{-1.5cm}
% \begin{wrapfigure}{r}{.32\columnwidth}
%   \vspace{-.5cm} 
%   \centering
% \includegraphics[scale=.29]{figures/Var of l2l_2 loss.pdf}
%   \vspace{-0.2cm}
%   \caption{Results of GP}
% \label{fig:var-l2-gp}
%   \vspace{-0.1cm}
% \end{wrapfigure}


% Attempts have been made  in the past to address these  drawbacks heuristically  (see \citep{AggarwalKoGuHaPh14}). We give a unified computational framework while approaching the problem in a more principled manner and solving it more optimally.




\subsection{Planning with  neural network-based uncertainty quantification methods ($\ensembleplus$)}


We now provide a proof-of-concept that shows the generalizability of our conceptual framework  to the deep learning-based UQ modules, specifically focusing on $\ensembleplus$ due to their previously observed superior performance~\citep{OsbandWenAsDwIbLuRo23}. Recall that implementing our framework with deep learning-based UQ modules  requires us to retrain the model across multiple possible random actions $\bm{a}(\theta)$ sampled from the current policy $\pi_\theta$.
This requires significant computational resources, in sharp contrast to the GPs where the posteriors are in closed form and can be readily updated and differentiated. 

Due to the computational constraints, we test $\ensembleplus$ on a toy setting to demonstrate the generalizability of our framework. We consider a setting where the general population consists of four clusters, while the initial labeled data only comes from one cluster. Again we generate data using GPs.  The task is to select a batch of 2 points in one horizon. We detail the $\ensembleplus$ architecture in Section \ref{sec:details-experiments}, and we assume prior uncertainty to be large (depends on the scaling of the prior generating functions). 
The results are summarized in the Table~\ref{tab:UQ_ensemble}.

% \begin{table}[H]
% \vspace{-10pt}
% \caption{Performance under \ensembleplus as UQ module}
%     \centering
%     \begin{tabular}{|m{3cm}|m{2.5cm}|m{2cm}|} 
%     \hline
%       Algorithm   & Variance of $\loss_2$ loss estimate & Error of $\loss_2$ loss estimate  \\ \hline Random Sampling 
%          & $1710.9 \pm 1352.1$ & $8.67\pm6.62$ 
%       \\ \hline \ouralgo & $1.30 \pm 0.68$ & $0.91\pm0.25$ \\ \hline
%     \end{tabular}
%     \label{tab:UQ_ensemble}
%     %\vspace{-10pt}
% \end{table}




\begin{table}[h]
\vspace{-10pt}
\caption{Performance under \ensembleplus as the UQ module}
\centering
\begin{tabular}{|l|l|l|}
\hline
Algorithm   & Variance of $\loss_2$ loss estimate & Error of $\loss_2$ loss estimate  \\
\hline
\textsf{Random sampling} & 7129.8 $\pm$ 1027.0 & 136.2 $\pm$ 8.28 \\ \hline
\textsf{Uncertainty sampling (Static)} & 10852 $\pm$ 0.0 & 162.156 $\pm$ 0.0 \\ \hline
\textsf{Uncertainty sampling (Sequential)} & 8585.5 $\pm$ 898.9 & 144 $\pm$ 6.93 \\ \hline
\textsf{REINFORCE} & 1697.1 $\pm$ 0.0 & 45.27 $\pm$ 0.0 \\ \hline
\ouralgo & 1697.1 $\pm$ 0.0 & 45.27 $\pm$ 0.0 \\ \hline
\end{tabular}
%\caption{Comparison of different algorithms based on variance   and   error in $\ell_2$ loss estimation with Ensemble $+$ as the UQ module. Our results demonstrate that {\ouralgo} and REINFORCE outperformthe other active learning based heuristics, confirming the benefits of our MDP formulation for the adaptive labeling problem, as also demonstrated in Section 4.\\
%\footnotesize{Experimental details: We use Gaussian Processes as our data generating process, GP parameters are the same as in Section D.3.  The task is to select a batch of 2 points along one horizon.The marginal distribution $p_X$ has 4 \textit{non-overlapping} clusters. Initial data comes from one cluster, while pool and evaluation points comes from all the clusters. We have $20$ initial labeled data points, $10$ pool points, and $252$ evaluation points.  Training procedures are similar to the one in Section D.3.} }
\label{tab:UQ_ensemble}
\end{table}



% We faced  issues in scaling up these experiments which will be our focus in the future. 





% \begin{itemize}
%     \item Posteriors should be consistent. Two dimensions: even with less training,  
%     \item the inference should be  fast enough
% \end{itemize}


% Potential research directions for uncertainty quantification

% In this section we consider a simple setting We consider a simpler setting and 


% For synthetic dataset generation, we use ...... For real datasets, we use ...... We compare our methodolgy to several baselines ()    This Section is structured as follows:
% \begin{itemize}
%     \item \textbf{GPs, square loss objective} (Section \ref{}): 
%     %the broad aim of the experiments  in this section is to isolate the performance of our methodology without any concerns for the inefficiencies induced due to a mis-specified prior or imperfect posterior inference. To accomplish this we generate synthetic datasets using GPs (detailed later). We use the well specified prior (GPs - with same hyperparameter setting) as our UQ module.   
%      As GPs provide differentaible posterior inference - any errors induced due to imperfect posterior updates are also isolated. We note that under this setting
%      \item In Section\ref{} we demonstrate why our methodology performs better than other baselines - by devising various synthetic experiments ()
%     \item  \textbf{UQ Benchmarking }(Section \ref{}): Before diving into the experiments using $\ensembleplus$ and ENNs,  we showcase our benchmarking experiments in Section \ref{}. We use real datasets We observe that ENNs perform better
%      \item \textbf{Ensemble $+$}, objective: recall, accuracy
%     \item \textbf{ENN}, objective: recall, accuracy
% \end{itemize}




% In Section {}, we test 
% \subsection{Experimental details}

% \begin{itemize}
%     \item UQ methodologies - GPs, ENNs
%     \item Objectives - Recall,  ATE
%     \item Datasets - ATE-synthetic datasets, Recall-synthetic, real datasets
%     \item Baselines - 
%     \begin{itemize}
%         \item Random sampling
%         \item Active learning - Uncertainty based sampling - In regression setting almost all of the 
%         \item Myopic greedy - Greedy Batch based sampling
%         \item Policy Gradient
%     \end{itemize}
    
% \end{itemize}

% \subsection{Experiments}
%     \begin{itemize}
%     \item GPs with square loss
%     \item Benchmarking ENN
%         \item ENNs with ATE
%         \item ENNs with Recall
%     \end{itemize}

% \subsection{Benefits over other algorithms - intuition and experiments}

%Active learning - Myopic greedy / Don't rely on the objective rather some entropy version.


%%% Local Variables:
%%% mode: latex
%%% TeX-master: "main"
%%% End:




\end{document}



