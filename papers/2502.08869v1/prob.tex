\section{Preliminaries and Taxonomy}

In this paper, a UTS is represented by $\mat{x}=[x_{1}, ..., x_{T}]\in\mathbb{R}^{1\times T}$ where $T$ is the length of the UTS, $x_{t}$ ($1\le t\le T)$ is the value at time step $t$. %and $[\cdot, \cdot]$ is the concatenation operator.
Suppose there are $d$ variates (or features), let $\mat{x}_{i}\in\mathbb{R}^{1\times T}$ ($1\le i\le d$) be a UTS of the $i$-th variate, an MTS can be represented by $\mat{X}=[\mat{x}_{1}^{\top}, ..., \mat{x}_{d}^{\top}]^{\top}\in\mathbb{R}^{d\times T}$.

As illustrated in Fig. \ref{fig.structure}, this survey focuses on methods that transform time series to images, namely {\em imaged time series}, and then apply vision models %, such as CNNs, LVMs and LMMs,
on the imaged time series for tackling time series tasks, such as classification, forecasting and anomaly detection. It is noteworthy that methods on videos or sequential images ({\em a.k.a.} image time series \cite{tarasiou2023vits}) do not belong to this category because they don't transform time series to images. Similarly, methods for spaciotemporal traffic data are out of our scope if the methods focus on streams of images ({\em e.g.}, traffic flows in a stream of grid maps \cite{zhang2017deep}), but methods on imaging time-space matrices \cite{ma2017learning} that resemble MTSs are included. %will be included. 
For vision models on audios, this survey %will discuss 
only discusses %a limited number of
some representative works in $\S$\ref{sec.spectrogram} due to space limit. %while the focus of the survey will remain on general time series.
The focus of the survey will remain on general time series.


% It is noteworthy that Imaging methods for spaciotemporal traffic data are not part of this discussion because they are not tranformation of time series to images.

% The review of audio methods focus on representative works because the focus of the survey is on time series.

% \subsection{Large Vision Models}

% \subsection{Large Multimodal Models}

\vspace{-0.1cm}

\subsection{Taxonomy}

% As illustrated by Fig. \ref{}, the typical process of vision models based time series analysis has five components: (1) normalization/scaling; (2) time series to image transformation; (3) image modeling; (4) image to time series recovery; and (5) task processing. In the rest of this paper, we will discuss the typical methods for each of these components. The detailed taxonomy of the methods are summarized in Table \ref{tab.taxonomy}.

We propose a taxonomy from %the perspectives
the two views of {\em Time Series to Image Transformation} ($\S$\ref{sec.tsimage}) and {\em Imaged Time Series Modeling} ($\S$\ref{sec.model}) as illustrated in Fig. \ref{fig.structure}. For the former, %view,
we %summarize
discuss 5 primary methods for %encoding UTS or MTS into images, together
imaging UTS or MTS, %with some %minor
%other methods,
and remark on their pros and cons. For the latter, %view,
we classify the existing methods by conventional vision models, LVMs and LMMs. We discuss their strategies on pre-training, fine-tuning, prompting, and the deigns of task-specific heads. We also discuss the challenges and solutions in pre-/post-processing in $\S$\ref{sec.processing}. Table \ref{tab.taxonomy} presents a summary. In the following two sections, we will delve into the existing methods from the two views.

% \vspace{-0.1cm}
