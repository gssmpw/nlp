\begin{abstract}

Time series analysis has witnessed the inspiring development from traditional autoregressive models, deep learning models, to recent Transformers and Large Language Models (LLMs). Efforts in leveraging vision models for time series analysis have also been made along the way but are less visible to the community due to the predominant research on sequence modeling in this domain. However, %recent struggles for aligning continuous time series with the discrete token space of LLMs
the discrepancy between continuous time series and the discrete token space of LLMs, and the challenges in explicitly modeling the correlations of variates in multivariate time series have shifted some research attentions to the equally successful Large Vision Models (LVMs) and Vision Language Models (VLMs). To fill the blank in the existing literature, this survey discusses the advantages of vision models over LLMs in time series analysis. It provides a comprehensive and in-depth overview of the existing methods, with dual views of detailed taxonomy that answer the key research questions including how to encode time series as images and how to model the imaged time series for various tasks. Additionally, we address the challenges in the pre- and post-processing steps involved in this framework and outline future directions to further advance time series analysis with vision models.

% Vision models have been historically used for time series analysis, in particular 1-dimensional CNNs, such as Dilated CNNs, ResNet, VGG-19, etc., have been extensively employed for time series analysis. Recently, large language models (LLMs) are drawing growing attentions on time series modeling. Compared with LLMs, applying vision models in particular ViTs and Large Vision Models (LVMs) for time series analysis has more advantages: (1) the numerical values of time series align better with the image modality than the text modality; (2) some image representations can model long-term time series; (3) LVMs have less API costs than LLMs. This work focuses on methods that transform time series to images and then use vision models to solve time series tasks (e.g., forecasting, classification, generation, anomaly detection, etc.). We provide a comprehensive survey of the primary methods for transforming time series to images and different modeling methods on imaged time series ranging from conventional vision models to LVMs and Large Multimodal Models (LMMs).

% Code for different time series to image transformation methods.
\end{abstract}
