\section{Related Work}
\subsection{Personal Data and Data Representation}
Representing data in a new, alternative form is the most well-known strategy for supporting the sensemaking of personal data \cite{jansen2015opportunities, choe2015characterizing}. In the context of personal informatics \cite{li2010stage,qself}, previous works have offered insights into how the visualization of users’ tracking data can facilitate understanding and making sense of their data \cite{choe2015characterizing, choe2017understanding,aseniero2020activity, huang2017field,mcduff2012affectaura}. Epstein et al. \cite{epstein2014taming} 
investigated various visualization formats for location and activity data aiming to support data sensemaking, such as tables, graphs, captions, and other forms.

Beyond such traditional visualization methods, researchers have also suggested alternative approaches using visual metaphors or abstract forms for data representation \cite{consolvo2008activity,pousman2007casual,murnane2020designing}. A notable example is \textit{casual information visualization} \cite{pousman2007casual}, which is known to generate reflective insights through representing data in artistic forms. In the field of personal informatics, there have been attempts to encode personal data in qualitative and subjective ways through metaphors and abstract imagery. For example, Ayobi et al. \cite{ayobi2018flexible} explored how people represent data through personally meaningful ways and promote self-reflection through paper bullet journaling. Kim et al. \cite{kim2019dataselfie} developed a system \textit{DataSelfie} that allows users to represent personal data in customized visual forms, motivated by the \textit{Dear Data Project} \cite{deardata} in which  Lupi and
Posavec created and shared personalized visualizations of data through hand-drawn postcards. Also, Murnane et al. \cite{murnane2020designing} translated physical activities and goals into multi-chapter narratives, thereby encouraging users to engage with their fitness goals by fostering empathy and enhancing their self-reflection. Another body of works also explored the transformation of personal data into tangible forms, known as \textit{data physicalization}, that enables individuals to engage in a meaning-making process of their data \cite{jansen2015opportunities,karyda2021data,thudt2018self,friske2020entangling}.

Collectively, these works have shown how an \textbf{unusual and ambiguous representation of data} can foster new, context-rich reflection experiences and data meaning-making \cite{wang2015design,kang2017fostering}, diverging from the quality of reflection that conventional statistical representations enable. Building on this, we aim to explore how images generated by AI can facilitate new types of data experience. Abstract images are a typical form of visual representation that invites people to engage in reflection \cite{huh2007use, dalsgaard2008designing, fan2012spark}, and prior research has also explored the design space of reflective engagement with artworks \cite{gorichanaz2020engaging}. In previous HCI studies, ambiguous representation of data have been shown to provide individuals new perspectives, serving as a resource that facilitates reflective thinking \cite{bentvelzen2022revisiting,nunez2014aesthetic,lindstrom2006affective,trujillo2014admixed,durrant2018admixed, mols2016technologies}. For example, Mols et al. \cite{mols2016technologies} demonstrated that ambiguous data representation can serve as a strategy for enabling reflection, as illustrated by concepts such as \textit{DataZen}, which creates sand patterns from activity, stress, and wellbeing data, and \textit{Life Tree}, an interactive art piece representing patterns of activity, social, and health data, situating this approach within the design space for everyday life reflection. Trujillo-Pisanty et al. \cite{trujillo2014admixed} created a new representation of online presence by extracting and amalgamating faces from Facebook photos using algorithms, thereby enabling reflection on personal and family representations. With the recent advances in image-generative AI technologies, these possibilities have expanded, making it possible to represent personal data in ambiguous forms \cite{rajcic2020mirror,rajcic2023message} with the design of tailored prompts, as their potential to serve as a design material has been highlighted \cite{benjamin2021machine,sivertsen2024machine,yurman2022drawing}. Despite this potential, research on how AI-generated images using personal data can facilitate personal data meaning-making remains scarce, which has motivated our current work.

\subsection{AI as a Material for Reflection and Meaning-Making}
With the recent advance in generative AI technology, a growing body of research has begun investigating how AI-generated media can be leveraged to facilitate people’s reflections on data. For instance, along with their conceptualization of \textit{Introspective AI}, Brand et al. \cite{brand2021design} have speculated a concept called \textit{Dream Streams} that represents people’s dreams in images based on sleep monitoring data and audio journaling. Also, Fan et al. \cite{fan2024contextcam} have suggested \textit{ContextCam}, which generates images with the themes extracted from contextual data. \textit{Quologue}, proposed by Kang and Odom \cite{kang2024design}, creates outputs that synthesize e-book highlights with users’ reflections using generative AI. Cho et al. \cite{cho2023areca} have designed \textit{ARECA}, an IoT-based air purifier that turns the data collected from the surrounding environment into diary entries using generative AI. Rajcic and McCormack \cite{rajcic2020mirror} investigated how machine-generated poetry based on a user’s emotions extracted from facial expressions can provoke reflection.

Collectively, these prior studies illustrate the potential of generative AI to \textbf{transform data into new forms, creating reflective materials} that can provoke new meaning-making around the data. Our work extends this corpus of studies, and in particular, we build on Brand et al.’s work \cite{brand2021design} that have demonstrated that AI can be utilized as a resource for supporting introspective experiences. Our paper focuses specifically on the ‘image-generative’ aspect of AI, delving deeper into what kinds of attributes of image-generative AI should be handled in order to facilitate a new quality of meaning-making from data. Also, building on Wan et al.’s work \cite{wan2024metamorpheus} on developing co-creative narration systems that transform dreams into visual metaphors, our work also investigates how various types of personal data, including qualitative and abstract forms, can be represented as images and what kinds of experiences can arise from such visual representations.