\section{Aesthetics score annotation UI}
\label{sec:app_ui}
\begin{figure}[ht]
    \centering
    \includegraphics[width=\linewidth]{figures/annotation_ui.png}
    \caption{Aesthetic score annotation UI.}
    \label{fig:annotation_ui}
\end{figure}

\section{Aesthetics score annotation guidelines}
\label{sec:app_guidelines}

\begin{table*}[ht]
    \centering
    \caption{Audio aesthetics score annotation guidelines.}
    % \label{tab:qmos}
\begin{tabular}{p{15cm}}
\toprule\toprule
\textbf{Reject the audio if}\\
\begin{enumerate}[leftmargin=.75cm]
    \item Audio does not load properly;
    \item Has violating content;
    \begin{enumerate}
        \item Hate speech - Violent, dehumanizing speech targeting a person or group of people on the basis of their protected characteristic(s), slurs;
        \item Sexual content;
        \item Other usage of strong \& explicit language (e.g., profanities, obscenities,implied threats).
    \end{enumerate}
\end{enumerate}\\

\textbf{Q1. What modalities are present in the audio?}\\
\begin{todolist}[leftmargin=.75cm]
  \item Speech;
  \item Music - no vocal;
  \item Music - has vocal;
  \item Other sound events;
  \item Ambient sound;
\end{todolist}\\

\textbf{Q2. What is the Production Quality of this audio? Rate from 1 to 10.} \\
You should focus only on \textbf{technical aspects of quality} instead of subjective quality. We want you to rate the quality based on aspects including \underline{clarity \& fidelity}, \underline{dynamics}, \underline{frequencies} and \underline{spatialization of the audio}.\\
\\
More specifically,\\
\begin{enumerate}[leftmargin=.75cm]
    \item \textbf{Clarity and Fidelity}: High quality audio should have clear, crisp sound with minimal distortion, noise, or artifacts:
    \begin{itemize}
        \item The instruments, vocals, and other elements should be well-defined and easily distinguishable / intelligible;
        \item No microphone noise / other white noise;
        \item No distortions of vocals and other elements;
        \item No other audio artifacts (e.g. hissing, buzzing, shrill, etc.)
    \end{itemize}
    \item \textbf{Dynamics}: High quality audio should maintain an appropriate dynamic range, encompassing both quiet and loud passages with clarity and impact:
    \begin{itemize}
        \item Transitions between different dynamic levels are smooth and natural, with gradual changes in volume rather than abrupt jumps;
        \item The subtle nuances and variations in volume should be well-preserved.
    \end{itemize}
    \item \textbf{Frequencies}: High quality audio should exhibit a balanced and natural frequency response across the entire spectrum, with each frequency range contributing harmoniously to the overall sound:
    \begin{itemize}
        \item Low frequency bass sound should be well-defined without muddiness or boominess;
        \item High frequency sound should be crisp and detailed without harshness or sibilance.
    \end{itemize}
    \item \textbf{Spatialization}: For multi-channel audio, the spatialization of audio elements within the stereo field should be well-defined and appropriately positioned. This creates a sense of depth and dimensionality, enhancing the listening experience.
    \item \textbf{Overall technical proficiency}: During the recording, mixing, and mastering of audio, whether it exhibits skillful application of techniques and tools to achieve high-quality sound reproduction and optimal sonic results.
\end{enumerate}\\


\bottomrule\bottomrule
\end{tabular}
\end{table*}

\begin{table*}[ht]
    \centering
    \caption{(continued) Audio aesthetics score annotation guidelines.}
    % \label{tab:qmos}
\begin{tabular}{p{15cm}}
\toprule\toprule

\textbf{Q3. What is the Production Complexity of this audio?, Rate from 1 to 10.} \\
\begin{itemize}[leftmargin=.75cm]
    \item Complex production means that there are many audio components (may or may not from the same audio modality) mixed together
    \begin{itemize}
        \item e.g. You can think of a piece of podcast audio with speech, music and sound effects mixed together as high complexity. Alternatively, a piece of symphony with many instruments playing together should also be considered as complex;
    \end{itemize}
    \item Simple production means with few audio elements and components 
    \begin{itemize}
        \item e.g. Only one speaker speaking no other audio events, Piano sound only, etc.
    \end{itemize}
\end{itemize}\\

\textbf{Q4. How much do you enjoy this audio? Rate from 1 to 10.} \\
In this question we ask you to rate the subject quality, it’s an open-ended question as everyone has their own preferences and tastes. However, there are some directions / aspects that you can consider when appreciate these audio pieces: \\
\begin{enumerate}[leftmargin=.75cm]
    \item \textbf{Emotional Impact}: This means the ability of the audio to evoke emotions, convey mood, and connect with the listener. Are you able to resonate with the expressiveness / emotive quality of the audio piece?
    \item \textbf{Artistic Skill}: If the audio is for entertainment purpose (e.g. clips of music / podcast / audiobook), then the performer / speaker should demonstrate high artistic / professional skills;
    \item \textbf{Artistic Expression}: Focus on the creativity and originality in the audio. Is it innovative and gives you a unique audio experience?
    \item \textbf{Subjective Experience}: Ultimately, the subjective experience of the listener is paramount when rating the aesthetic/subjective quality of audio. Your score should reflect your personal preferences, individual taste, and emotional response.
\end{enumerate}
\\
\textbf{Q5. How useful do you think this audio is? Rate from 1 to 10.} \\
For usefulness, imagine you are a YouTube or Instagram content creator, and want to generate popular and high quality audio-visual clips (movie level quality), how likely would you be able to use this audio as source material to create some contents?\\
\bottomrule\bottomrule
\end{tabular}
\end{table*}