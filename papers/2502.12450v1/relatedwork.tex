\section{Related Work}
\subsection{Homans Social Exchange Theory}
% Social Exchange Theory (SET) is a foundational framework in sociology and social psychology that views social interactions as transactions of value~\cite{homans1958social}. SET has been widely applied to explain organizational behavior, leadership trust, and workplace relationships~\cite{ahmad2023social}, while also illuminating friendships~\cite{methot2016workplace}, family relations~\cite{cropanzano2005social}, and romantic partnerships~\cite{laursen1999nature}. Recent theoretical advances include Lawler and Thye's~\cite{lawler2006social} exploration of emotional dimensions and Cropanzano et al.'s~\cite{cropanzano2017social} two-dimensional model incorporating activity alongside hedonic factors. While Enayat et al.~\cite{enayat2022computational} implemented SET through multi-agent simulation, their simplified exchange rules overlooked emotional complexity. Our approach leverages LLM-driven agents to simulate more nuanced human interactions, offering a more faithful implementation of SET.
Social Exchange Theory is a foundational framework in sociology and social psychology that views social interactions as transactions of value~\cite{homans1958social}. In organizational and workplace behavior, it is considered a “gold standard” for explaining dynamics like employee–employer relationships, leadership trust, and organizational citizenship behaviors~\cite{ahmad2023social}. Beyond organizations, SET has been used in social psychology to examine friendships~\cite{methot2016workplace}, family relations~\cite{cropanzano2005social}, and even romantic partnerships~\cite{laursen1999nature} as exchanges of emotional support, information, and other resources. 

Recent refinements to SET have deepened its explanatory power. Lawler and Thye~\cite{lawler2006social} explored the emotional dimensions of exchange, while Cropanzano et al.\cite{cropanzano2017social} proposed a two-dimensional model incorporating "activity" alongside the traditional hedonic framework, enhancing SET’s predictive accuracy. In terms of research methods, Enayat et al.\cite{enayat2022computational} applied SET in a multi-agent simulation to explore social structures through simplified exchange rules. However, their model, which reduced agent behavior to simple exchanges of money and recognition, overlooked emotional subjectivity. To address this limitation, our approach employs LLM-driven agents to simulate more complex human interactions, offering a more accurate implementation of SET.

 
\subsection{LLM-driven Agent-based Modeling}

% LLMs have transformed agent-based modeling by enabling human-like behavior and decision-making capabilities~\cite{gao2024large}. Key developments include Generative Agent~\cite{park2023generative}, which simulates community life with 25 LLM agents, EconAgent~\cite{li2024econagent}, which explores macroeconomic phenomena, and RecAgent~\cite{wang2023user}, which studies recommender system interactions. Recent works have demonstrated LLM agents' ability to simulate classical social scenarios, with CompeteAI~\cite{zhao2023competeai} modeling competitive behavior and Xie et al.~\cite{xie2024can} exploring trust dynamics. Building on these advances, our work presents a novel application of LLM-driven agents to validate and extend SET.

The rise of LLMs has significantly advanced agent-based modeling by enabling agents with human-like behavior and decision-making capabilities. Traditionally, ABM relied on fixed rules to govern agent behavior, but LLMs provide flexible, dynamic responses that better simulate real human interactions~\cite{gao2024large}. Several studies have leveraged LLMs to enhance ABM across different domains. Generative Agent\cite{park2023generative}, which simulates daily life in a virtual town with 25 LLM agents, and EconAgent\cite{li2024econagent}, a model that uses LLM agents to explore macroeconomic phenomena. RecAgent~\cite{wang2023user} studies user interaction with recommender systems through LLM-driven agents.

LLM agents have also been used to simulate classical social scenarios, such as competition and trust. CompeteAI\cite{zhao2023competeai} models competitive behavior between restaurant owners, while Xie et al.\cite{xie2024can} explore trust dynamics in LLM agents. These studies demonstrate LLM agents' ability to replicate human-like patterns of social behavior. Building on this, our work aims to validate and extend SET using LLM-driven agent-based modeling, a domain that has yet to be extensively explored in the context of SET.