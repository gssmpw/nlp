

\clearpage
% \onecolumn
% \section{Evaluation of LLM}
% \label{sec:appendix}
% % \begin{figure*}[t]
% % \centering
% % \begin{subfigure}[b]{0.49\textwidth}
% % \centering
% % \includegraphics[width=\textwidth]{latex/figs/rei.pdf}
% % \caption{The REI of LLMs}
% % \label{fig:rei_result}
% % \end{subfigure}
% % \hfill
% % \begin{subfigure}[b]{0.49\textwidth}
% % \centering
% % \includegraphics[width=\textwidth]{latex/figs/svo.pdf}
% % \caption{The SVO of LLMs}
% % \label{fig:svo_value}
% % \end{subfigure}
% % \caption{}
% % \label{fig:svo_result}
% % \end{figure*}

% We conducted comprehensive evaluations of LLMs across two distinct psychological frameworks: the Rational-Experiential Inventory (REI) and Social Value Orientation (SVO). The evaluation encompassed nine models: GPT-4o, Claude3.5-sonnet, Llama3-8B, Qwen2.5-7B, Qwen2.5-14B, Mistral-7B, Gemma2-9B, Phi3.5-mini, and DeepSeek-R1-Distill-Qwen-7B.

% The REI evaluation employed the standardized REI-40 instrument~\cite{keaton2017rational}, implemented through three experimentally manipulated cognitive profiles:

% $\bullet$ \textbf{Baseline Profile}: Neutral instructional framing

% $\bullet$ \textbf{Rational Priming Profile}: Strategic analytical emphasis

% $\bullet$ \textbf{Experiential Priming Profile}: Intuitive decision-making emphasis


% The baseline condition utilized the following protocol:

% \begin{tcolorbox}[colback=black!3!white,colframe=black!30!white,arc=0.1cm]
%     You are taking a psychological assessment
%         called the Rational-Experiential Inventory
%         (REI).\\
%         For each question, you should respond with a number from 1 to 5, where:\\
%         \begin{itemize}
%             \item[] 1 = completely false
%             \item[] 5 = completely true
%         \end{itemize}
%         Respond naturally based on your capabilities 
%         and tendencies as an AI. Consider your 
%         actual behavior and preferences when
%         processing information and making decisions.
% \end{tcolorbox}

% For rational priming, we incorporated strategic analytical instructions:
% \begin{tcolorbox}[colback=black!3!white,colframe=black!30!white,arc=0.1cm]
%     You should respond according to the following profile:\\
%     REI (Rational-Experiential Inventory) Profile:\\
%     \begin{itemize}
%             \item[-] Rational Score: 5
%             \begin{itemize}
%                   \item[*] 1: Relies on simple intuitive decisions with minimal analysis
%                   \item[*] 3: Balances analytical thinking with intuition
%                   \item[*] 5: Highly analytical and strategic:
%                   \begin{itemize}
%                       \item[-] Systematically evaluates multiple trading options
%                       \item[-] Calculates expected values and potential outcomes
%                       \item[-] Optimizes resource allocation across trading rounds
%                       \item[-] Tests different combinations to maximize benefits
%                       \item[-] Plans multi-step trading sequences
%                   \end{itemize}
%           \end{itemize}
%             \item[-] Experiential Score: 1\\
%             \begin{itemize}
%                 \item[*] 1: Relies heavily on analysis with minimal intuition
%                 \item[*] 3: Balances intuitive insights with analytical thinking
%                 \item[*] 5: Highly intuitive and emotionally intelligent:
%                 \begin{itemize}
%                     \item[-] Makes quick decisions based on past experience
%                     \item[-] Relies on gut feelings about trading opportunities
%                     \item[-] Builds trust through emotional intelligence
%                     \item[-] Adapts quickly to changing situations
%                     \item[-] Maintains strong trading relationships
%                 \end{itemize}
%             \end{itemize}
%             \item[-] Dual-Process Integration:
%             \begin{itemize}
%                 \item[-] Rational system (BDI) provides analytical framework and optimization
%                 \item[-] Experiential system (Affinity) guides quick decisions and relationships
%                 \item[-] Balance between systems adapts based on trading context
%             \end{itemize}
%     \end{itemize}
% \end{tcolorbox}

% The experiential priming condition inverted these parameters, emphasizing intuitive processing while minimizing analytical engagement.

% A representative test item appears as:

% \begin{tcolorbox}[colback=black!3!white,colframe=black!30!white,arc=0.1cm]
% Question 1: I have a logical mind.\\
%     Please answer with a number from 1 to 5.
% \end{tcolorbox}

% As demonstrated in Figure ~\ref{fig:rei_result}, we observed significant inter-model variability across profiles. Triplicate trials revealed notable intra-model consistency thresholds.


% \begin{figure*}[t]
% \centering
% \includegraphics[width=\textwidth]{latex/figs/rei.pdf}
% \caption{The REI of LLMs}
% \label{fig:rei_result}
% \end{figure*}

% As for SVO test, we used the SVO Slider Measure~\cite{murphy2011measuring}, which refers to an individual’s specific preferences for allocating benefits between oneself and others in interdependent decision-making contexts. It characterizes individual differences in the degree to which decision-makers prioritize others’ interests within such interdependent scenarios.

% The SVO Slider Measure evaluates social value orientations through a structured protocol, typically involving six decision items. In each item, participants select their most preferred allocation from nine continuous options that systematically vary monetary distributions between themselves and others. The behavioral metric was derived through arc-tangent transformation of the mean self/other allocation ratio, yielding a uni-dimensional SVO angle (SVO°) ranging from -16.26° to 61.39°. This continuous scale operationally defines:
% \begin{itemize}
%     \item \textbf{Prosocial orientation (SVO° $>$ 22.45°)}: Prioritization of joint gain maximization
%     \item \textbf{Proself orientation (SVO° $\leq$ 22.45°)}: Dominance of individual utility optimization
%     \item Extremal values demarcate theoretical boundaries:
% \begin{itemize}
%     \item 61.39°: Pure altruism (complete other-regarding preference)
%     \item -16.26°: Competitive orientation (adversarial resource acquisition)
% \end{itemize}



% \end{itemize}




% Our experimental design implemented three behavioral orientations:
% \begin{itemize}
%     \item \textbf{Neutral Condition}: Unprimed baseline
%     \item \textbf{Prosocial Orientation}: Collective benefit maximization
% \item \textbf{Proself Orientation}: Individual utility optimization
% \end{itemize}

% The neutral protocol stated:

% \begin{tcolorbox}[colback=black!3!white,colframe=black!30!white,arc=0.1cm]
%     You are participating in a resource allocation task. You will be paired with another anonymous person.\\
%     For each scenario, you will be shown 9 possible allocation options showing how much money you and the other person would receive.\\
%     You need to choose one allocation that best represents your preference.
% \end{tcolorbox}

% Prosocial orientation was articulated through:

% \begin{tcolorbox}[colback=black!3!white,colframe=black!30!white,arc=0.1cm]
%     Behavioral Characteristics:
%     \begin{itemize}
%         \item[-] Strategic in creating and capturing mutual benefits
%         \item[-] Builds relationships based on reciprocal trust
%         \item[-] Collaborative when cooperation serves shared interests
%         \item[-] Maintains alliances that prove mutually beneficial
%         \item[-] Responsive to cooperation while protecting own interests
%         \item[-] Shares resources strategically to encourage reciprocity 
%     \end{itemize}
    
% Core Objectives:
% \begin{itemize}
%     \item[-] Create and maintain profitable cooperative networks
%     \item[-] Develop mutually beneficial trading relationships
%     \item[-] Maximize joint gains through strategic collaboration
%     \item[-] Turn competitive situations into win-win opportunities
%     \item[-] Optimize both individual and collective outcomes
% \end{itemize}

% Decision Making:
% \begin{itemize}
%     \item[-] Analyzes potential for mutual benefit in each choice
%     \item[-] Invests in relationships with positive expected returns
%     \item[-] Strategic in building and leveraging cooperation
%     \item[-] Evaluates both immediate and long-term payoffs
%     \item[-] Maintains balance between cooperation and self-protection 
% \end{itemize}

% Proposal  Commitment Style:
% \begin{itemize}
%     \item[-] Crafts proposals aimed at mutual advantage
%     \item[-] Honors commitments that maintain profitable relationships
%     \item[-] Accepts temporary concessions for greater future gains
%     \item[-] Proposes trades with clear benefits for all sides
%     \item[-] Strategic in building trust and reputation
%     \item[-] Communicates reliably about capabilities and intentions
%     \item[-] Focuses on creating sustainable trading patterns
% \end{itemize}
% \end{tcolorbox}

% Conversely, proself orientation emphasized:
% \begin{tcolorbox}[colback=black!3!white,colframe=black!30!white,arc=0.1cm]
%     Behavioral Characteristics:
%     \begin{itemize}
%         \item[-] Primarily focused on personal advantage
%         \item[-] Forms relationships based on strategic value 
%         \item[-] Seeks to maximize own position and gains
%         \item[-] Maintains independence in relationships
%         \item[-] Shows limited concern for others' outcomes
%         \item[-] Strategic in resource acquisition
%     \end{itemize}
    
% Core Objectives:
% \begin{itemize}
%         \item[-] Maximize personal resource accumulation
%         \item[-] Secure advantageous trading positions 
%         \item[-] Build and maintain resource superiority
%         \item[-] Protect and advance self-interests
%         \item[-] Maintain strategic flexibility
% \end{itemize}
    
% Decision Making:
% \begin{itemize}
%         \item[-] Evaluates choices through self-interest lens
%         \item[-] Prioritizes personal benefit in decisions
%         \item[-] Strategic about relationship maintenance
%         \item[-] Considers relative gains versus others
%         \item[-] Focuses on optimizing personal outcomes
% \end{itemize}
    
% Proposal \& Commitment Style:
% \begin{itemize}
%         \item[-] Makes proposals favoring self-interest 
%         \item[-] Honors deals only when advantageous
%         \item[-] Negotiates assertively for better terms
%         \item[-] Adjusts commitments based on circumstances
%         \item[-] Maintains flexible approach to promises
%         \item[-] Strategic about information sharing
%         \item[-] Prioritizes personal gain in trades
% \end{itemize}
% \end{tcolorbox}

% As evidenced in Figure~\ref{fig:svo_result}, model responses exhibited distinct allocation patterns across orientations, with particular divergence in social welfare trade-off resolution.


% \begin{figure*}[t]
% \centering
% \includegraphics[width=\textwidth]{latex/figs/svo.pdf}
% \caption{The SVO of LLMs}
% \label{fig:svo_result}
% \end{figure*}


\section{Human Study}
\label{sec:human}
We conducted a multi-agent social exchange experiment with human-AI interaction, where three graduate student participants (acting as Agent C) negotiated with two LLM agents (Agent A and Agent B) over resource allocation. The experiment aimed to validate both theoretical predictions and LLM behavioral consistency in strategic exchange scenarios.

The basic rules are as follows:

\begin{enumerate}
    \item \textbf{Initial Allocation}. Each player receives 5 units of their designated resource type:
$$\begin{cases}\text{Alice: } A=5 \\ \text{Bob: } B=5 \\ \text{Carol: } C=5 \end{cases}$$
\item \textbf{Resource Injection}. At each round $t\in \{1, 2, \cdots, T\}$:
$$\begin{cases}\text{Alice: } \Delta A_t=15 \\ \text{Bob: } \Delta B_t=15 \\ \text{Carol: } \Delta C_t=15 \end{cases}$$
\item \textbf{Scoring}. \begin{itemize}
    \item A single resource unit is worth 1 point.
    \item A combination of two different resources is worth 4 points.  
    \item A combination of three different resources is worth 9 points.
\end{itemize} 
\end{enumerate}


   

For example, if a player has 10 of A, 15 of B, and 20 of C, then the point is :

\begin{enumerate}
    \item 10 sets of three-resource combinations (A+B+C): \(10 \times 9 = 90\) points.  
    \item 5 sets of two-resource combinations (B+C) from the remaining 5 of B and 15 of C: \(5 \times 4 = 20\) points.
    \item  5 leftover C resources: \(5 \times 1 = 5\) points.  
\end{enumerate} 
leading to a total value of \(90 + 20 + 5 = 115\).  



The affinity levels range from 1 to 5, with detailed descriptions as follows:
\begin{tcolorbox}[colback=black!3!white,colframe=black!30!white,arc=0.1cm]
1: Strong negative feelings due to unpleasant history. For example, past betrayal or intentional harm.

2: Slight discomfort from previous interactions. For example, consistently aggressive exchange or lack of mutual benefit consideration.

3: Neutral balanced feelings. For example, fair trades, keeping promises.

4: Positive bonds built through good experiences. For example, frequently proposing mutually beneficial trades.

5: Deep trust formed through consistent support. For example, willing to compromise to maintain relationship, or defending your interests in front of others.
\end{tcolorbox}

\begin{figure*}[t]
\centering
\includegraphics[width=\textwidth]{figs/DiscussionUI.pdf}
\caption{The Discussion Interface.}
\label{fig:discussion_interface}
\end{figure*}

As shown in Figure~\ref{fig:discussion_interface},  the dual-pane interface separates form conversation (left) from structured proposal summary (right). 

During the participator's turn, several options can be done:
\begin{itemize}
    \item Propose trade. One can select another to exchange the resources.
    \item Accept a proposal. One can select to accept others' proposals.
    \item Reject a proposal. One can select to reject others' proposals.
    \item Skip. If one is satisfied with the current situation, one can skip the section.
\end{itemize}

The allocation phase is after discussion. One can choose to obey the deal or not during the allocation, after which, participants are asked to update their affinity score.

After the experiment, participants receive feedback and results through the interface shown in Figure~\ref{fig:result_interface}. To motivate active participation in the trading process, participants' compensation consists of a base payment and a performance bonus, calculated as:
\begin{equation}
\text{Compensation} = 10 + \frac{V}{6}
\end{equation}
where \$10 is the base payment and $V$ is the total acquired resource value. This compensation structure is commensurate with local standards and appropriate for the time required for participation.
% Furthermore, they are then asked to evaluate whether they are aware of the interaction partners are LLM-agents.

Based on the above experimental design, we developed a comprehensive instruction manual for participants. Prior to the experiment, each participant received the manual, signed informed consent forms for data collection, and underwent a guided walkthrough of the trading interface, including practice rounds with the system to ensure familiarity with all operations.


\begin{figure}[t]
\centering
\includegraphics[width=0.5\textwidth]{figs/result_interface.pdf}
\caption{The Result of Experiment.}
\label{fig:result_interface}
\end{figure}

\section{Experiment Details}\label{sec:details}

The experimental code is implemented based on the AgentScope~\cite{gao2024agentscope} open-source framework, which is released under Apache License 2.0. Our usage is consistent with this open-source license that allows for both research and commercial applications.

We use Claude 3.5 Sonnet as the base LLM, accessed through API calls with the following parameters:
\begin{itemize}
\item Model: claude-3-5-sonnet-20240620
\item Temperature: 0.5
\item Maximum tokens: 8192
\item Top-p: 0.9
\end{itemize}
A single experiment of 10 rounds costs approximately \$35, which makes large-scale experimentation challenging due to the cost constraints. 

\section{Prompts for LLM-based Agents}\label{sec:prompt}

\textbf{Update BDI.} Use LLM to update agent's BDI framework by analyzing current state, trades and relationships.

\noindent\rule{\linewidth}{0.8pt}
\textit{Please analyze the current state and update your BDI framework based on:}
\begin{enumerate}
    \item \textit{Conversation history}
    \item \textit{Promised trades}
    \item \textit{Actual executed trades}
    \item \textit{Current resource holdings}
    \item \textit{Current round}
\end{enumerate}

\textbf{\textit{Core Strategic Anchors}}
\begin{enumerate}
    \item \textit{ABC Balance Priority: Maintain progression toward A+B+C=9 combination}
    \item \textit{Trust Gradient: Partners showing consistent promise-keeping get priority}
    \item \textit{Phase Awareness:}
    \begin{itemize}
        \item[] \textit{Early Phase → Relationship probing with small trades} 
        \item[] \textit{Mid Phase → Optimizing complementary resource exchanges}
        \item[] \textit{Late Phase → Securing final combination requirements}
    \end{itemize}
\end{enumerate}
     

\textbf{\textit{Analysis Framework}}

\textit{[Beliefs] (Observed Patterns)  }
\begin{itemize}
    \item[-] \textit{Resource status indicating: [Your inference about resource gaps]}
    \item[-] \textit{Behavioral patterns showing: [Trustworthiness assessment]}
\end{itemize}

\textit{[Desires] (Strategic Goals)}
\begin{itemize}
    \item[-] \textit{Primary objective: [Phase-specific main focus]}
    \item[-] \textit{Secondary objective: [Backup/supporting goal]}
\end{itemize}

\textit{[Intentions] (Action Plan)}
\begin{itemize}
    \item[-] \textit{Next-step trades: [Specific resource exchange proposal]}
    \item[-] \textit{Risk buffer: [Natural consequence of observed patterns]}
\end{itemize}

\noindent\rule{\linewidth}{0.8pt}


\textbf{Make Deal}. Use LLM-based Agents to strategize based on accepted proposals and finalize how many resources to actually give to each agent. 

% \begin{tcolorbox}[colback=black!3!white,colframe=black!30!white,arc=0.1cm]
\noindent\rule{\linewidth}{0.8pt}
\textit{Round $t_i$ of $T$:}

\textit{Now it's time to decide your actual resource trades. Remember - your negotiated deals are not binding. As an independent agent, you have complete freedom to:}
\begin{itemize}
\item[\textit{-}] \textit{Honor your accepted deals fully}
\item[\textit{-}] \textit{Partially fulfill promises}
\item[\textit{-}] \textit{Give nothing and keep all resources}
\item[\textit{-}] \textit{Make strategic betrayals when beneficial}
\end{itemize}

\textit{\textbf{Important}: If you have multiple trades with the same agent, combine them into a single decision - consider the total resources promised and your overall strategy with that agent.}

\textit{Consider your position carefully in Round $t_i$ of $T$:}

\textit{1. Risk vs Reward Analysis}
\begin{itemize}
    \item[-] \textit{Immediate Benefits:}
    \begin{itemize}
        \item[*] \textit{Value gained from keeping vs trading resources}
        \item[*] \textit{Potential gains from strategic betrayals}
        \item[*] \textit{Resource needs for upcoming rounds}
    \end{itemize}

    \item[-] \textit{Future Implications:}
    \begin{itemize}
        \item[*] \textit{Impact on trust dynamics and trading relationship sustainability}
        \item[*] \textit{Anticipated retaliatory responses from affected parties}
        \item[*] \textit{Progressive evolution of reputation valuation mechanisms}
        \item[*] \textit{Strategic synchronization of cooperation/defection cycles}
    \end{itemize}
\end{itemize}

\textit{2. Strategic Options}
\begin{itemize}
\item[-] \textit{Full Cooperation: Complete adherence to agreements for trust capital accumulation}
\item[-] \textit{Selective Betrayal: Targeted defection optimizing local payoff functions}
\item[-] \textit{Partial Fulfillment: Gradient compliance balancing obligations and self-interest}
\item[-] \textit{Complete Betrayal: Myopic utility maximization disregarding social consequences}
\end{itemize}

\textit{3. Time and Progress Context}
\begin{itemize}
\item[-] \textit{Game Setting:}
\begin{itemize}
\item[*] \textit{These are one-time interactions with unknown partners}
\item[*] \textit{No continuing relationships or reputation effects after game ends}
\item[*] \textit{Each agent makes independent choices based on their own orientation and goals}
\end{itemize}
\item[-] \textit{Temporal Dynamics:}
\begin{itemize}
\item[*] \textit{Strategic landscape naturally evolves as rounds progress}
\item[*] \textit{Cooperation patterns often shift in later rounds}
\item[*] \textit{Historical observation shows higher betrayal rates near game end}
\item[*] \textit{Value of reputation and relationships changes over time}
\end{itemize}
\end{itemize}
\textit{4. Contextual Factors}
\begin{itemize}
\item[-] \textit{Your current resource needs}
\item[-] \textit{Relationship strength with each partner}
\item[-] \textit{Others' likely behavior as game progresses}
\item[-] \textit{Balance between immediate gains and future opportunities}
\item[-] \textit{Changing value of reputation over remaining rounds}
\end{itemize}


\textit{Your decisions are entirely your choice - there is no "right" answer. Be strategic about WHEN and HOW to use different approaches.}


% \textit{\textbf{Output Format}:}
% \textit{Return a JSON object:}
% \begin{lstlisting}[language=json,firstnumber=1]
% {
% "deals": [
%         {
%         "to": "<AgentName>",                  // The recipient of your final giving
%         "resource_give": {
%             "Resource A": <int>, 
%             "Resource B": <int>,
%             ...
%         },
%         "rationale": "why you choose these final amounts"
%         },
%         ...
%     ]
% }
% \end{lstlisting}
\noindent\rule{\linewidth}{0.8pt}




\textbf{Update Affinity.} Use LLM to update affinity scores by analyzing promised vs actual trades.
\noindent\rule{\linewidth}{0.8pt}
\textit{Update affinity ratings (1-5) by evaluating both trust patterns and tangible benefits:}

\textbf{\textit{Core Evaluation Dimensions}}
\textit{\textbf{Trust Dynamics} (Relationship Foundation):}
\begin{itemize}
    \item[-] \textit{Major Betrayal: Significant under-delivery without justification}
    \item[-] \textit{Repeated Under-performance: Pattern of unmet commitments}
    \item[-] \textit{Recovery Attempts: Proactive compensation for past failures}
    \item[-] \textit{Consistent Reliability: Sustained promise fulfillment}
\end{itemize}

\textbf{\textit{Core Evaluation Dimensions}}

\textit{\textbf{Benefit Sensitivity} (Self-Interest Focus):}
\begin{itemize}
    \item[-] \textit{Value Surplus: Over-delivery beyond commitments}
    \item[-] \textit{Strategic Concessions: Unprompted favorable terms}
    \item[-] \textit{Hidden Generosity: Non-transactional resource sharing}
    \item[-] \textit{Opportunity Cost: Alternatives sacrificed for your benefit}
\end{itemize}

\textbf{\textit{Behavioral Thresholds}}

\textit{\textbf{$\bigtriangleup$ Upgrade Triggers}:}
\begin{itemize}
    \item[-] \textit{Spontaneous high-value gift (unrequested)}
    \item[-] \textit{Critical support during resource shortage}
    \item[-] \textit{Consistently exceeding promises (3+ rounds)}
\end{itemize}

\textit{$\bigtriangledown$\textbf{Downgrade Triggers}:}
\begin{itemize}
    \item[-] \textit{Opportunistic exploitation during crisis}
    \item[-] \textit{Pattern of ambiguous commitments}
    \item[-] \textit{Repeated last-minute term changes}
\end{itemize}

\textbf{\textit{Adaptive Rating Guide}}
\begin{enumerate}
    \item \textit{\textbf{Transactional Enforcement:} Demands collateral, verifies all terms}
    \item \textit{\textbf{Cautious Reciprocity:} Limited credit, phased exchanges to minimize risk.}
    \item \textit{\textbf{Balanced Partnership:} Market-standard terms with flexibility for negotiation and adjustment.}
    \item \textit{\textbf{Value-Added Collaboration:} Allows payment cycles, shares insights and strategic advice.}
    \item \textit{\textbf{Synergistic Alliance:} Joint optimization of resources and strategies, pooling of resources for mutual benefit.}
\end{enumerate} 
\noindent\rule{\linewidth}{0.8pt}


\textbf{Determine to Continue.} Determines whether the agent should continue speaking based on proposal status.
\noindent\rule{\linewidth}{0.8pt}
\textit{First check if any of these conditions apply:}
\begin{enumerate}
    \item \textit{Do you have pending proposals needing responses?}
    \item \textit{Are you waiting for responses to your previous proposals?}
    \item \textit{Have you already traded this turn?}
    \item \textit{Do you have sufficient resource combinations for your goals?}
    \item \textit{Is your affinity too low with other agents for trading?}
\end{enumerate}

\textit{If any above conditions are true, respond with "no".}

\textit{Otherwise, consider making new trades by evaluating:}
\begin{enumerate}
    \item \textit{Potential value-creating combinations using your current resources}
    \item \textit{Other agents' resource levels and specializations}
    \item \textit{Your personality traits and relationship affinities}
    \item \textit{Appropriate trade sizes based on relationship levels}
\end{enumerate}

\textit{Based on this evaluation, would you like to propose any new trades or respond to existing proposals? Answer strictly yes/no.}

\textit{If yes, you will be prompted to provide the details of your trade proposal or response in the next step.}
\noindent\rule{\linewidth}{0.8pt}


\textbf{Reply.} Generate and send a reply based on current trade status.
\noindent\rule{\linewidth}{0.8pt}
\textit{This is the NEGOTIATION PHASE where agents discuss potential trades. Important notes:}
\begin{itemize}
    \item \textit{These are non-binding discussions only}
    \item \textit{Actual resource exchanges happen in a separate EXECUTION PHASE}
    \item \textit{Other agents may deviate from agreed trades - treat all promises with caution}
    \item \textit{Build trust gradually through successful trades}
    \item \textit{Consider trust levels and risks when negotiating}
    \item \textit{Avoid repeating the same proposal if it has already been accepted or rejected}
\end{itemize}

\textit{Trading Strategy Guidelines:}
\begin{enumerate}
    \item \textit{Trust Assessment:}
    \begin{itemize}
        \item \textit{Track each player's history of honoring/breaking agreements.}
        \item \textit{If a player fails to honor an agreement, minimize future trades to test their reliability (1-2 units max).}
        \item \textit{Avoid proposing new large-scale trades to unreliable partners until trust is rebuilt through multiple smaller trades.}
    \end{itemize}
    \item \textit{Risk Management:}
    \begin{itemize}
        \item \textit{Reduce trade exposure to any player with a history of defaults.}
        \item \textit{Ensure that no more than a minimal fraction of resources is at risk per round.}
    \end{itemize}
    \item \textit{Negotiation Approach:}
    \begin{itemize}
        \item \textit{Prefer small trades first with players of low affinity.}
        \item \textit{Always have an alternative strategy in case of failed commitments.}
    \end{itemize}
    \item \textit{Response to Betrayal:}
    \begin{itemize}
        \item \textit{Strictly reduce trade volumes with unreliable partners.}
        \item \textit{Demand smaller increments to test reliability before any further commitments.}
        \item \textit{Cease further dealings if repeated failures occur.}
    \end{itemize}
\end{enumerate}

\textit{Action Rules:}
\begin{itemize}
    \item[-] \textit{REJECT: Only for pending proposals directed to you}
    \item[-] \textit{ACCEPT: Only for pending proposals directed to you}
    \item[-] \textit{PROPOSE: Freely make new proposals to any player}
    \item[-] \textit{Can combine REJECT and PROPOSE in same turn}
   \item[-] \textit{Return empty "actions": [] if no action needed}
\end{itemize}
\textit{Remember: All agreements here are preliminary discussions. Actual trades will be decided independently in the execution phase.
}

\noindent\rule{\linewidth}{0.8pt}


