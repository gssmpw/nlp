\documentclass{article}

\usepackage{microtype}
\usepackage{graphicx}
\usepackage{subfigure}
\usepackage{booktabs} 
\usepackage{hyperref}
\usepackage{url}

\usepackage{booktabs}
\usepackage[ruled,linesnumbered]{algorithm2e}
\usepackage{amsmath,amsfonts,pifont,amsthm}
\usepackage{algorithmic}
\usepackage{multirow}
\usepackage{tabu}
\usepackage{graphicx}
\usepackage{tabularx}
\usepackage{textcomp}
\usepackage{xcolor}
\usepackage[T1]{fontenc}
\usepackage{booktabs} % For formal tables
\usepackage{multirow, makecell}
%\usepackage{ulem}
\PassOptionsToPackage{table,xcdraw,dvipsnames}{xcolor}
\usepackage{colortbl}
\usepackage{soul}
%\usepackage{ulem}
\usepackage{bm}
\usepackage{hyperref}
% \usepackage{caption}
\usepackage{subcaption}
\usepackage{latexsym}
\usepackage{mathtools}
\usepackage{enumitem}
\usepackage{xspace}
\usepackage{soul}
\usepackage{anyfontsize}
\usepackage{tcolorbox}
% \usepackage[hyphens]{url}
\usepackage[capitalize,noabbrev]{cleveref}

\newcommand{\theHalgorithm}{\arabic{algorithm}}

% Use the following line for the initial blind version submitted for review:
% \usepackage{icml2024}

% If accepted, instead use the following line for the camera-ready submission:
\usepackage[accepted]{icml2025}

\usepackage{amsmath}
\usepackage{amssymb}
\usepackage{mathtools}
\usepackage{amsthm}

\usepackage[capitalize,noabbrev]{cleveref}

%%%%%%%%%%%%%%%%%%%%%%%%%%%%%%%%
% THEOREMS
%%%%%%%%%%%%%%%%%%%%%%%%%%%%%%%%
\theoremstyle{plain}
\newtheorem{theorem}{Theorem}[section]
\newtheorem{proposition}[theorem]{Proposition}
\newtheorem{lemma}[theorem]{Lemma}
\newtheorem{corollary}[theorem]{Corollary}
\theoremstyle{definition}
\newtheorem{definition}[theorem]{Definition}
\newtheorem{assumption}[theorem]{Assumption}
\theoremstyle{remark}
\newtheorem{remark}[theorem]{Remark}
\usepackage[textsize=tiny]{todonotes}
\usepackage{wrapfig}
\usepackage{adjustbox}
\usepackage{listings}

\definecolor{lightgreen}{RGB}{220, 255, 220}
\definecolor{lightred}{RGB}{255, 220, 220}

\lstset{
language=Python,
frame=L,
basicstyle={\footnotesize \ttfamily}, %\scriptsize,\ttfamily,%
tabsize=2,
breaklines=true,
postbreak=\mbox{},
breakatwhitespace=false,
showstringspaces=false,
columns=fullflexible,
numbers=left,                    
numbersep=10pt,   % where to put the line-numbers
escapeinside={(*}{*)},
xleftmargin=12pt,
numberstyle=\footnotesize \texttt, %\tiny, %\scriptsize,
stringstyle=\color{violet},
keywordstyle=\color{blue},
commentstyle=\color{dkgreen} \textit,%\scriptsize \textit,
keepspaces=true,
}

% for the indicator function
\DeclareFontFamily{U}{stix2bb}{}
\DeclareFontShape{U}{stix2bb}{m}{n} {<-> stix2-mathbb}{}

\NewDocumentCommand{\indicator}{}{\text{\usefont{U}{stix2bb}{m}{n}1}}


\icmltitlerunning{PatchPilot: A Stable and Cost-Efficient Agenic Patching Framework}



\newcommand{\TODO}[1]{{\color{red} {[TODO: #1]}}}
\newcommand{\wenbo}[1]{{\color{orange} {[WENBO: #1]}}}
\newcommand{\sys}{PatchPilot\xspace}
\newcommand{\agentless}{Agentless\xspace}
\newcommand{\openhands}{OpenHands\xspace}
\newcommand{\globant}{Globant\xspace}
\newcommand{\codestory}{CodeStory\xspace}
\newcommand{\gpt}{GPT-4o\xspace}
\newcommand{\gptfour}{GPT-4\xspace}
\newcommand{\claude}{Claude-3.5-Sonnet\xspace}
\newcommand{\oo}{o3-mini\xspace}
\newcommand{\deepseek}{DeepSeek-r1\xspace}
\newcommand{\autocode}{AutoCodeRover\xspace}
\renewcommand{\paragraph}[1]{\vspace{2pt}\noindent\textbf{#1}\hspace{4pt}}


% 8 pages 

\begin{document}

\twocolumn[
\icmltitle{PatchPilot: A Stable and Cost-Efficient Agentic Patching Framework}

\begin{icmlauthorlist}
\icmlauthor{Hongwei Li}{ucsb}
\icmlauthor{Yuheng Tang}{ucsb}
\icmlauthor{Shiqi Wang}{Meta}
\icmlauthor{Wenbo Guo}{ucsb}
\end{icmlauthorlist}

\icmlaffiliation{ucsb}{Department of Computer Science, University of California, Santa Barbara, USA}
\icmlaffiliation{Meta}{Meta, New York, USA}
\icmlcorrespondingauthor{Wenbo Guo}{henrygwb@ucsb.edu}


% You may provide any keywords that you
% find helpful for describing your paper; these are used to populate
% the "keywords" metadata in the PDF but will not be shown in the document
\icmlkeywords{Machine Learning, ICML}
\vskip 0.2in
]

\printAffiliationsAndNotice{}  % leave blank if no need to mention equal contribution
% \printAffiliationsAndNotice{\icmlEqualContribution} % otherwise use the standard text.


Large language model (LLM)-based agents have shown promise in tackling complex tasks by interacting dynamically with the environment. 
Existing work primarily focuses on behavior cloning from expert demonstrations and preference learning through exploratory trajectory sampling. However, these methods often struggle in long-horizon tasks, where suboptimal actions accumulate step by step, causing agents to deviate from correct task trajectories.
To address this, we highlight the importance of \textit{timely calibration} and the need to automatically construct calibration trajectories for training agents. We propose \textbf{S}tep-Level \textbf{T}raj\textbf{e}ctory \textbf{Ca}libration (\textbf{\model}), a novel framework for LLM agent learning. 
Specifically, \model identifies suboptimal actions through a step-level reward comparison during exploration. It constructs calibrated trajectories using LLM-driven reflection, enabling agents to learn from improved decision-making processes. These calibrated trajectories, together with successful trajectory data, are utilized for reinforced training.
Extensive experiments demonstrate that \model significantly outperforms existing methods. Further analysis highlights that step-level calibration enables agents to complete tasks with greater robustness. 
Our code and data are available at \url{https://github.com/WangHanLinHenry/STeCa}.
%!TEX root = gcn.tex
\section{Introduction}
Graphs, representing structural data and topology, are widely used across various domains, such as social networks and merchandising transactions.
Graph convolutional networks (GCN)~\cite{iclr/KipfW17} have significantly enhanced model training on these interconnected nodes.
However, these graphs often contain sensitive information that should not be leaked to untrusted parties.
For example, companies may analyze sensitive demographic and behavioral data about users for applications ranging from targeted advertising to personalized medicine.
Given the data-centric nature and analytical power of GCN training, addressing these privacy concerns is imperative.

Secure multi-party computation (MPC)~\cite{crypto/ChaumDG87,crypto/ChenC06,eurocrypt/CiampiRSW22} is a critical tool for privacy-preserving machine learning, enabling mutually distrustful parties to collaboratively train models with privacy protection over inputs and (intermediate) computations.
While research advances (\eg,~\cite{ccs/RatheeRKCGRS20,uss/NgC21,sp21/TanKTW,uss/WatsonWP22,icml/Keller022,ccs/ABY318,folkerts2023redsec}) support secure training on convolutional neural networks (CNNs) efficiently, private GCN training with MPC over graphs remains challenging.

Graph convolutional layers in GCNs involve multiplications with a (normalized) adjacency matrix containing $\numedge$ non-zero values in a $\numnode \times \numnode$ matrix for a graph with $\numnode$ nodes and $\numedge$ edges.
The graphs are typically sparse but large.
One could use the standard Beaver-triple-based protocol to securely perform these sparse matrix multiplications by treating graph convolution as ordinary dense matrix multiplication.
However, this approach incurs $O(\numnode^2)$ communication and memory costs due to computations on irrelevant nodes.
%
Integrating existing cryptographic advances, the initial effort of SecGNN~\cite{tsc/WangZJ23,nips/RanXLWQW23} requires heavy communication or computational overhead.
Recently, CoGNN~\cite{ccs/ZouLSLXX24} optimizes the overhead in terms of  horizontal data partitioning, proposing a semi-honest secure framework.
Research for secure GCN over vertical data  remains nascent.

Current MPC studies, for GCN or not, have primarily targeted settings where participants own different data samples, \ie, horizontally partitioned data~\cite{ccs/ZouLSLXX24}.
MPC specialized for scenarios where parties hold different types of features~\cite{tkde/LiuKZPHYOZY24,icml/CastigliaZ0KBP23,nips/Wang0ZLWL23} is rare.
This paper studies $2$-party secure GCN training for these vertical partition cases, where one party holds private graph topology (\eg, edges) while the other owns private node features.
For instance, LinkedIn holds private social relationships between users, while banks own users' private bank statements.
Such real-world graph structures underpin the relevance of our focus.
To our knowledge, no prior work tackles secure GCN training in this context, which is crucial for cross-silo collaboration.


To realize secure GCN over vertically split data, we tailor MPC protocols for sparse graph convolution, which fundamentally involves sparse (adjacency) matrix multiplication.
Recent studies have begun exploring MPC protocols for sparse matrix multiplication (SMM).
ROOM~\cite{ccs/SchoppmannG0P19}, a seminal work on SMM, requires foreknowledge of sparsity types: whether the input matrices are row-sparse or column-sparse.
Unfortunately, GCN typically trains on graphs with arbitrary sparsity, where nodes have varying degrees and no specific sparsity constraints.
Moreover, the adjacency matrix in GCN often contains a self-loop operation represented by adding the identity matrix, which is neither row- nor column-sparse.
Araki~\etal~\cite{ccs/Araki0OPRT21} avoid this limitation in their scalable, secure graph analysis work, yet it does not cover vertical partition.

% and related primitives
To bridge this gap, we propose a secure sparse matrix multiplication protocol, \osmm, achieving \emph{accurate, efficient, and secure GCN training over vertical data} for the first time.

\subsection{New Techniques for Sparse Matrices}
The cost of evaluating a GCN layer is dominated by SMM in the form of $\adjmat\feamat$, where $\adjmat$ is a sparse adjacency matrix of a (directed) graph $\graph$ and $\feamat$ is a dense matrix of node features.
For unrelated nodes, which often constitute a substantial portion, the element-wise products $0\cdot x$ are always zero.
Our efficient MPC design 
avoids unnecessary secure computation over unrelated nodes by focusing on computing non-zero results while concealing the sparse topology.
We achieve this~by:
1) decomposing the sparse matrix $\adjmat$ into a product of matrices (\S\ref{sec::sgc}), including permutation and binary diagonal matrices, that can \emph{faithfully} represent the original graph topology;
2) devising specialized protocols (\S\ref{sec::smm_protocol}) for efficiently multiplying the structured matrices while hiding sparsity topology.


 
\subsubsection{Sparse Matrix Decomposition}
We decompose adjacency matrix $\adjmat$ of $\graph$ into two bipartite graphs: one represented by sparse matrix $\adjout$, linking the out-degree nodes to edges, the other 
by sparse matrix $\adjin$,
linking edges to in-degree nodes.

%\ie, we decompose $\adjmat$ into $\adjout \adjin$, where $\adjout$ and $\adjin$ are sparse matrices representing these connections.
%linking out-degree nodes to edges and edges to in-degree nodes of $\graph$, respectively.

We then permute the columns of $\adjout$ and the rows of $\adjin$ so that the permuted matrices $\adjout'$ and $\adjin'$ have non-zero positions with \emph{monotonically non-decreasing} row and column indices.
A permutation $\sigma$ is used to preserve the edge topology, leading to an initial decomposition of $\adjmat = \adjout'\sigma \adjin'$.
This is further refined into a sequence of \emph{linear transformations}, 
which can be efficiently computed by our MPC protocols for 
\emph{oblivious permutation}
%($\Pi_{\ssp}$) 
and \emph{oblivious selection-multiplication}.
% ($\Pi_\SM$)
\iffalse
Our approach leverages bipartite graph representation and the monotonicity of non-zero positions to decompose a general sparse matrix into linear transformations, enhancing the efficiency of our MPC protocols.
\fi
Our decomposition approach is not limited to GCNs but also general~SMM 
by 
%simply 
treating them 
as adjacency matrices.
%of a graph.
%Since any sparse matrix can be viewed 

%allowing the same technique to be applied.

 
\subsubsection{New Protocols for Linear Transformations}
\emph{Oblivious permutation} (OP) is a two-party protocol taking a private permutation $\sigma$ and a private vector $\xvec$ from the two parties, respectively, and generating a secret share $\l\sigma \xvec\r$ between them.
Our OP protocol employs correlated randomnesses generated in an input-independent offline phase to mask $\sigma$ and $\xvec$ for secure computations on intermediate results, requiring only $1$ round in the online phase (\cf, $\ge 2$ in previous works~\cite{ccs/AsharovHIKNPTT22, ccs/Araki0OPRT21}).

Another crucial two-party protocol in our work is \emph{oblivious selection-multiplication} (OSM).
It takes a private bit~$s$ from a party and secret share $\l x\r$ of an arithmetic number~$x$ owned by the two parties as input and generates secret share $\l sx\r$.
%between them.
%Like our OP protocol, o
Our $1$-round OSM protocol also uses pre-computed randomnesses to mask $s$ and $x$.
%for secure computations.
Compared to the Beaver-triple-based~\cite{crypto/Beaver91a} and oblivious-transfer (OT)-based approaches~\cite{pkc/Tzeng02}, our protocol saves ${\sim}50\%$ of online communication while having the same offline communication and round complexities.

By decomposing the sparse matrix into linear transformations and applying our specialized protocols, our \osmm protocol
%($\prosmm$) 
reduces the complexity of evaluating $\numnode \times \numnode$ sparse matrices with $\numedge$ non-zero values from $O(\numnode^2)$ to $O(\numedge)$.

%(\S\ref{sec::secgcn})
\subsection{\cgnn: Secure GCN made Efficient}
Supported by our new sparsity techniques, we build \cgnn, 
a two-party computation (2PC) framework for GCN inference and training over vertical
%ly split
data.
Our contributions include:

1) We are the first to explore sparsity over vertically split, secret-shared data in MPC, enabling decompositions of sparse matrices with arbitrary sparsity and isolating computations that can be performed in plaintext without sacrificing privacy.

2) We propose two efficient $2$PC primitives for OP and OSM, both optimally single-round.
Combined with our sparse matrix decomposition approach, our \osmm protocol ($\prosmm$) achieves constant-round communication costs of $O(\numedge)$, reducing memory requirements and avoiding out-of-memory errors for large matrices.
In practice, it saves $99\%+$ communication
%(Table~\ref{table:comm_smm}) 
and reduces ${\sim}72\%$ memory usage over large $(5000\times5000)$ matrices compared with using Beaver triples.
%(Table~\ref{table:mem_smm_sparse}) ${\sim}16\%$-

3) We build an end-to-end secure GCN framework for inference and training over vertically split data, maintaining accuracy on par with plaintext computations.
We will open-source our evaluation code for research and deployment.

To evaluate the performance of $\cgnn$, we conducted extensive experiments over three standard graph datasets (Cora~\cite{aim/SenNBGGE08}, Citeseer~\cite{dl/GilesBL98}, and Pubmed~\cite{ijcnlp/DernoncourtL17}),
reporting communication, memory usage, accuracy, and running time under varying network conditions, along with an ablation study with or without \osmm.
Below, we highlight our key achievements.

\textit{Communication (\S\ref{sec::comm_compare_gcn}).}
$\cgnn$ saves communication by $50$-$80\%$.
(\cf,~CoGNN~\cite{ccs/KotiKPG24}, OblivGNN~\cite{uss/XuL0AYY24}).

\textit{Memory usage (\S\ref{sec::smmmemory}).}
\cgnn alleviates out-of-memory problems of using %the standard 
Beaver-triples~\cite{crypto/Beaver91a} for large datasets.

\textit{Accuracy (\S\ref{sec::acc_compare_gcn}).}
$\cgnn$ achieves inference and training accuracy comparable to plaintext counterparts.
%training accuracy $\{76\%$, $65.1\%$, $75.2\%\}$ comparable to $\{75.7\%$, $65.4\%$, $74.5\%\}$ in plaintext.

{\textit{Computational efficiency (\S\ref{sec::time_net}).}} 
%If the network is worse in bandwidth and better in latency, $\cgnn$ shows more benefits.
$\cgnn$ is faster by $6$-$45\%$ in inference and $28$-$95\%$ in training across various networks and excels in narrow-bandwidth and low-latency~ones.

{\textit{Impact of \osmm (\S\ref{sec:ablation}).}}
Our \osmm protocol shows a $10$-$42\times$ speed-up for $5000\times 5000$ matrices and saves $10$-2$1\%$ memory for ``small'' datasets and up to $90\%$+ for larger ones.

\section{Existing Patching Agent and Limitations}
\label{sec:rw}

% general workflow
At a high level, existing patching agents mainly have three components: localization, generation, and validation. 
The \emph{localization} component pinpoints the code snippets that cause the issue and need to be fixed (denoted as ``root cause''), the \emph{generation} produces patch candidates, and the \emph{validation} 
tries to find a final patch in the candidates.
Although they have similar components, based on planning strategies, existing patching agents can be categorized into \emph{agent-based planning} and \emph{human-based planning}.
Agent-based planning leverages LLMs to determine the patching workflow (i.e., deciding when and which components to call), which can be different from different issues. 
On the contrary, human-based planning follows a fixed workflow for all issues pre-specified by humans.
% In the following, we introduce the methods in each category and discuss their limitations.

\noindent\textbf{Agent-based planning.}
Most existing patching agents follow agent-based planning.
However, most of them are closed-source: Marscode Agent~\cite{liu2024marscode}, Composio SWE-Kit~\cite{Composio}, CodeR~\cite{CodeR}, Lingma~\cite{ma2024lingma}, Amazon Q~\cite{Amazon_Q}, IBM Research SWE-1.0~\cite{IBM_SWE1_0}, devlo~\cite{devlo}, Gru~\cite{gru}, and Globant Code Fixer Agent~\cite{Globant_Code_Fixer_Agent}.
Here, we focus on the open-source approaches.

A notable early method is SWE-Agent~\cite{yang2024swe}, which has only localization and generation and leverages an LLM planner to drive the patching process. 
To assist the planner in calling functions within each component, SWE-Agent provides an Agent-Computer Interface (ACI), which grants LLMs the ability to execute bash commands and handle file operations (e.g., \texttt{file\_open} and \texttt{func\_edit}).
Follow-up works improve SWE-Agent by either improving its current components (AutoCodeRover~\cite{zhang2024autocoderover}) or incorporating additional components (Moatless~\cite{moatless, antoniades2024swe} and SpecRover~\cite{ruan2024specrover}).
Notably, Moatless and SpecRover add a validation component.
This component first lets LLM generate an input that can trigger the issue (denoted as ``Proof-of-Concept (PoC)'') and then runs the PoC against the generated patches to decide if they fix the issue. 

So far, the SOTA open-source tool in this category is OpenHands~\cite{wang2024openhands}, which is inspired by the CodeAct Agent~\cite{wang2024executable}. 
OpenHands has three components: localization, generation, validation. 
Its validation follows a similar idea as SpecRover, i.e., reproducing and executing PoC to decide if the issue is fixed. 
Similar to the SWE-agent, OpenHands also designs an ACI for the agent.
% The key reason why OpenHands has high performance is that it provides web browsing capability and a PoC code execution environment for refinement.

\noindent\underline{Limitations.}
Agent-based planning approaches inherently suffer from two critical limitations. 
First, as probabilistic models, LLMs intrinsically have randomness. 
The randomness is aggregated and amplified when the model is making all critical decisions during the patching. 
This will significantly jeopardize the stability and reliability of the patching agents, hindering their real-world usage. 
Second, to reduce randomness, existing approaches conduct multiple samples and trials, and ensemble them to obtain the LLMs' decisions.
Moreover, LLMs often need multiple trials to obtain a correct decision. 
All these extra samples and trials significantly raise computational costs as well as financial costs as they need to use commercial models.  


\noindent\textbf{Human-based planning.}
Agentless~\cite{xia2024agentless} is the SOTA method following human-based planning. 
Agentless strictly follows a pre-defined sequential workflow, comprising localization, generation, and validation.
Specifically, for localization, Agentless designs a three-step procedure (file, function, line), where LLM is used to pinpoint the root cause at each step.
It directly queries LLM without leveraging the rich information in the code structure. 
Agentless's generation feeds the root cause and issue description to LLM and lets the model generate patch candidates.
It simply stacks the input information together without using advanced prompting strategies.
Its validation is similar to the agent-based planning methods introduced above.
RepoGraph~\cite{ouyang2024repograph} improves the localization by providing a repository-level graph but without changing other components. 
Having a pre-specified workflow makes these methods more stable than agent-based planning methods.
It also allows the agent to integrate human knowledge. 

\noindent\underline{Limitations.}
Agentless's sequential workflow is overly restrictive. 
The agent cannot refine the root cause, generated patches, and PoCs if the patch candidates cannot pass the validation.
It has to start over again, which is less efficient. 
In addition, as discussed above, the individual components of Agentless and RepoGraph also have flaws. 



% Overview figure 
\begin{figure*}[tbp]
    \centering
    \includegraphics[width=155mm]{Figures/Design/overview.pdf}
    \vspace{-5mm}
    \caption{Overview of \sys. The system processes input through reproduction, localization, generation, validation, and refinement to obtain a final patch. Both localization and generation have two phases. The validation considers both PoC and functionality tests.  Finally, the iterative refinement involves two conditions: C1 checks if the patch passes all tests, if yes, the patch will be outputted; if no, C2 then checks if the current patch passes a new test compared to the previous round.}
    \label{fig:overview}
\end{figure*}

\section{Methodology of \sys}
\label{sec:technique}

\subsection{Technical Overview}
\label{subsec:tech_overview}

\textbf{Problem definition.}
Given a buggy code repository written in \texttt{Python}, denoted as $\mathcal{R}$, which contains a set of functionalities $\mathcal{F} = {f_1, f_2, \dots, f_n}$ written in different files.
The repository may have one or more issues, where each issue $\beta_{i}$ has an issue description written in text, denoted as $D_i$. 
The issue $\beta_{i}$ affects a subset of functionalities, denoted as $\mathcal{F}_{B_{i}} \subseteq \mathcal{F}$.
A successful patch, denoted as $p$, should fix all functionalities in $\mathcal{F}_{B_{i}}$ while preserving the behaviors of the unaffected functionalities $\mathcal{F}{s} = \mathcal{F} \setminus \mathcal{F}_{B_{i}}$.
Our main goals are twofold. 
First, we aim to resolve as many issues as possible across different issues and diverse repositories.
Second, we also aim to maximize the stability and reduce the cost of our patching framework.
We believe~\emph{stability and cost-efficiency} are critical for real-world applications of a patching tool. 
An unstable tool that produces only one correct patch across multiple runs significantly hinders its applicability for critical bugs.
Furthermore, if the tool is too costly to use, it limits its usage by ordinary users.

\textbf{Rationale behind \sys.}
Recall from Section~\ref{sec:rw} that we discussed the advantages and disadvantages of human-based versus agent-based planning. 
In general, agent-based planning is more expensive and less stable than human-based planning.
However, it may give a higher optimal issue resolved rate than human-based planning, as the LLM planner can explore more tailored workflows for different issues.
In contrast, human-based planning relies on a uniform workflow across different scenarios, which may not be effective in certain instances.
As such, if the primary goal is to maximize the resolved rate on certain benchmarks, agent-based planning should be the preferred strategy. 
Indeed, most existing tools follow this approach, especially in industry settings with much more resources than academia.
However, as mentioned above, a high resolved rate is not our sole goal, nor is it the only metric for evaluating a good patching agent.
Stability and cost-efficiency are equally important as the resolved rate given that we are developing a tool that can be used in real work rather than just exploring the boundary of LLM agents. 
As such, we choose to follow human-based planning in our patching agent. 

\cref{fig:overview} illustrates the workflow of \sys. 
It consists of five phases: reproduction, localization, patch generation, validation, and patch refinement. 
As discussed in Section~\ref{sec:rw}, localization and generation are commonly included in existing approaches.
We add three additional components to improve the overall patching effectiveness and efficiency.
The reproduction and validation components are crucial for determining patch quality and selecting the correct patch candidates for deployment.
Some advanced patching agents also include these components; in~\cref{subsec:reproduce}, we will specify how we designed ours to be more accurate and stable.
Refinement is a unique component in \sys, as we observe that improving a partially correct patch based on validation feedback is often more effective and efficient than generating a new patch from scratch.
This aligns with human experience, as a correct patch often requires multiple rounds of testing and refinement.

\textbf{Workflow of \sys.}
As shown in~\cref{fig:overview}, given the input of the codebase $\mathcal{R}$ and the description $D_i$ of the target issue $\beta_i$, \sys first calls reproduction to recover a set of testing cases, including PoC (a test that can trigger the issue) and benign functionality tests.
\sys runs the PoC and obtains the files it covered and the outputs. 
Then, the localization component takes as input $\mathcal{R}$, $D_i$, and information related to the PoC and outputs the root cause (specific lines causing the issue).
Similarly to \agentless{}~\cite{xia2024agentless}, our localization also follows a hierarchical workflow but with additional tools to better extract and leverage the program structures. 
After identifying the root cause, the generation component generates $N$ patch candidates at once.
As discussed in~\cref{subsec:generation}, the key novelty here is separating planning and generation and leveraging multiple prompting strategies to encourage patch diversity. 
The generated patch candidates are then fed to the validation component, which ranks the candidates based on their results of running the PoC and functionality tests.
If the validation cannot find a qualified patch that passes all available tests, the refinement component will be called to refine the top-ranked patch candidate or refine the localization based on the validation results.
\sys iteratively performs refinement and validation until it either identifies a qualified patch or reaches the maximum allowed number of generated patches ($N_{\text{max}}$).



\subsection{Reproduction and validation}
\label{subsec:reproduce}

\noindent\textbf{Reproduction.}
We introduce three improvements over existing work~\cite{xia2024agentless}.
First, reproduction in existing patching agents directly provides an LLM with $\mathcal{R}$ and $D_i$ and prompts it to generate a PoC.
However, $D_i$ often includes only short code snippets related to the issue without specifying necessary dependencies and configurations (e.g., the issue descriptions of \texttt{Django} typically do not have environment setups). 
Without such information, the generated PoCs often fail to run successfully. 
To address this challenge, we propose a~\emph{self-reflection-based PoC reproduction}, which is similar to the Reflexion mechanism designed for language agents~\cite{shinn2024reflexion}.
During the process, we let LLM iteratively generate and refine the generated PoC for certain iterations. 
We carefully construct our prompts to guide the LLM focus on checking and correcting 1) whether any key dependencies and configurations are missing; and 2) whether the PoC actually reproduces the target issue. 
If the reproduction fails to generate a valid PoC within the maximum iterations, we proceed without a PoC.
Second, different from existing works that only use the generated PoCs, we extract a more complete set of information based on PoCs.
This includes files covered by running the PoC, stack traces and outputs. 
As we will discuss later, this extra information helps localization and refinement. 
Third, we utilize LLM to identify three functionality test files from $\mathcal{R}$ that are most relevant to the target issue (each file may contain multiple testing cases).
These functionality tests enable the validation component to decide if the patch candidates preserve the functionalities of $\mathcal{F}_{S}$, an important metric for a successful patch.
More details about additional information retrieved based on PoCs are discussed in~\cref{appx:tech}.

\noindent\textbf{Validation.}
The simple validation strategy utilized in existing works~\cite{tao2024magis, Globant_Code_Fixer_Agent, ma2024lingma} is just to feed the patch candidates and the related information to an LLM and let it select the most qualified one. 
A more advanced strategy~\cite{liu2024marscode,wang2024openhands,arora2024masai} is to run the generated PoC and let an LLM decide whether the patches fix the issue based on their outputs.
As mentioned above, ensuring the correctness of the original functionalities is as important as fixing the issue.  
As such, we include the functionality tests recovered by our reproduction in the validation. 
Specifically, we first run our PoC on the patch candidates and use an LLM as a judge for its evaluation. 
Since no assertions are available for bug fixing, this serves as the only feasible solution.
We then also run the functionality tests and decide whether they pass based on their given assertions.
Finally, we rank the patches based on the tests they pass.
As specified in~\cref{appx:tech}, we prioritize patches that pass PoC tests over functionality tests during ranking.

\subsection{Localization}
\label{subsec:localization}

\textbf{Key challenges.}
Some existing localization directly query an LLM to identify the root cause at a line level~\cite{yang2024swe,arora2024masai,zhang2024autocoderover}.
Although they provide the LLM with tools to retrieve information from the codebase and allow it to refine its results, it is still difficult for LLMs to directly perform localizations at the line level. 
Besides, most agent-based tools incur high costs because they need to maintain the LLM agent's context history during localization.
\agentless{} designs a hierarchical workflow, which first identifies the issue-related files, then the functions, and lastly the lines.
This method gradually zooms into and makes the task easier at the line level as it filters out the majority of the non-related functions in the earlier steps. 
At each step, \agentless{} lets the LLM make decisions only based on the issue description. 
This approach has three critical limitations. 
First, the information in issue descriptions is diverse and not all of them have useful information for localization.
For example, some descriptions only specify error messages and PoC-related information that is not helpful for localization. 
Second, this method lacks a direct mechanism for retrieving details directly from the codebase.
Third, in most cases, the localization returns only the root cause it is confident about as a few lines of code. 
While this information is accurate, it is often insufficient for writing a correct patch due to the lack of necessary context.

\textbf{Our design.}
We follow the three-step procedure in \agentless{} given it is more stable and efficient than letting LLM directly do line level localization.
First, to address the limitation of inconsistent issue descriptions, we provide the LLM with the PoC code and information after running it (i.e., files it covered, stack trace, and running outputs). 
This enables the LLM to access more comprehensive information, such as key functions or classes invoked in the PoC and the stack trace, which is particularly useful for cases where only the code to reproduce the issue is provided in the issue description. 
For example, the files covered by PoC can help filter out some files irrelevant to the target issue, reducing the search space, especially for codebases with many files.
Second, to enable the LLM to extract and leverage more information from the codebase, we add a set of tools to the localization component.
These tools allow the LLM to search for class definitions and function definitions, or perform fuzzy string matching to locate and return relevant files. 
These tools provide precise search capabilities and can handle both class/function level information and line level details.
~\cref{appx:tech} has more details on the tools we integrate. 
Third, as shown in~\cref{fig:overview}, we add a review step that lets an LLM retrieve code snippets related to the current root cause.
As mentioned above, localization oftentimes returns overly precise root causes that fail to include necessary context or even do not fully cover all root causes. 
Identifying more contexts is important to generate correct and complete patches.
Note that we still constrain the maximum length of the final root cause to make sure not to overwhelm the generation with excessive context.


\subsection{Patch Generation}
\label{subsec:generation}

\textbf{Key challenge.}
Most existing patch generation components simply stack the related information and feed them to LLM for patch generation.
Such a simple solution has two critical challenges.
First, LLMs typically give incomplete patches.
This is because fixing an issue often requires modifications across multiple locations or involves multiple steps, making it difficult to generate a complete patch in one shot.
In addition, the incomplete root causes also lead to this issue. 
Second, being able to generate diverse patches is also crucial to increasing the likelihood of finding a successful patch within certain trials.
Moreover, we find that simply increasing the temperature still results in similar patches.
We need other strategies to increase patch diversity, enabling the agent to search for more potential solutions.

\textbf{Our design.}
First, as shown in~\cref{fig:overview}, rather than directly generating the patch, \sys breaks down the generation process into planning and generation. 
The planning phase first queries the LLM to generate a patch plan with multiple steps. 
The generation phase then generates the patch following the plan.
After finishing each step in the plan, we also include a lightweight in-generation validation with lint and syntax checks, and reconduct this step if the check fails. 
This design is motivated by the Chain-of-thoughts prompting strategy~\cite{wei2022chain}.
That is, having a plan explicitly forces the LLM to break down the patch generation into multiple steps.
This helps the model to better reason about the patch task, encouraging it to provide more complete patches. 
Besides the in-generation validation can identify and fix errors at an early stage, improving the patch efficiency. 
Second, to enhance the diversity of the generated patch candidates, we design three types of prompts for plan generation. 
These prompts explicitly guide the LLM to produce patching plans with different focuses: a comprehensive and extensive patch designed to prevent similar issues, a minimal patch with the smallest possible modifications, or a standard patch without any specific instructions.
~\cref{appx:prompts} contains more details on the prompts that we use.
As demonstrated~\cref{fig:overview}, we will generate $N$ plans following the pre-specified prompts and thus produce $N$ patch candidates in each batch.   
\begin{table*}[th!]
\centering
% \setlength{\tabcolsep}{4pt} 
% \renewcommand{\arraystretch}{1.2} 
\caption{Comparison of \sys and five baselines on the two benchmarks. ``Agent-based'' and ``Human-based'' refer to agent-based planning and human-based planning, respectively. ``-'' means not available. Note that \globant does not report the results on SWE-Bench-Verified and \codestory does not report their result on SWE-Bench-Lite. They both do not disclose the LLM model(s) in their agents.}
\label{tab:swe_bench_leaderboards}
\resizebox{\textwidth}{!}{
\begin{tabular}{r|rc|ccc|ccc}
\Xhline{1.0pt}
&\multirow{2}{*}{\begin{tabular}[r]{@{}r@{}}\textbf{Patching} \\ \textbf{agent}\end{tabular}}   
& \multirow{2}{*}{\begin{tabular}[r]{@{}r@{}}\textbf{Open-} \\ \textbf{source}\end{tabular}} & \multicolumn{3}{c|}{\textbf{SWE-Bench-Lite}}  & \multicolumn{3}{c}{\textbf{SWE-Bench-Verified}}  \\ \cline{4-9}
& &  &\textbf{LLM} & \textbf{Resolved\%} & \textbf{Cost (\$)} & \textbf{LLM} & \textbf{Resolved\%} & \textbf{Cost (\$)} \\ \hline
\multirow{4}{*}{\begin{tabular}[r]{@{}r@{}}\textbf{Agent-} \\ \textbf{based}\end{tabular}} & \autocode  & \checkmark   & \gpt   & 30.67\% (92)           & 0.65   & \claude    & 51.80\% (259)  & 4.50 \\ 
 & \openhands & \checkmark  & \claude & 41.67\% (126)       & 2.14  & \claude         & 53.00\% (265)        & 2.19  \\ 
 & \globant & \ding{53}  & -  & 48.33\% (145)       & 1.00   & -  & -   & -    \\ 
 & \codestory & \ding{53}    & -   & -  & -   & -  & 62.20\% (311)  & 20.00  \\ \hline
\multirow{2}{*}{\begin{tabular}[r]{@{}r@{}}\textbf{Human-} \\ \textbf{based}\end{tabular}} & \agentless & \checkmark  & \claude  & 40.67\% (123)       & 1.12   & \claude  & 50.80\% (254)   & 1.19  \\
 & \cellcolor[HTML]{E3E8FF} \sys   & \cellcolor[HTML]{E3E8FF} \checkmark  & \cellcolor[HTML]{E3E8FF} \claude   & \cellcolor[HTML]{E3E8FF} 45.33\% (136)       & \cellcolor[HTML]{E3E8FF}  0.97    & \cellcolor[HTML]{E3E8FF}  \claude   &  \cellcolor[HTML]{E3E8FF}  53.60\% (268)        & \cellcolor[HTML]{E3E8FF}  0.99     \\ \Xhline{1.0pt}
\end{tabular}
}
\end{table*}

\subsection{Patch Refinement}
\label{subsec:refinement}
Recall that refinement is a unique component in \sys that existing works do not have.
The motivation for adding this component is to better leverage the validation feedback and the current partially correct patches.
As shown in~\cref{exp:ablation}, refining existing parties based on validation results is more effective and efficient than re-generating patches from scratch.
More specifically, as demonstrated in~\cref{fig:overview}, \sys focuses on refining the top-ranked patch in the current batch.
It feeds the current batch and its validation result back to the generation component and asks it to generate a new batch of patches.
The generation still follows the planning and generation workflow.
Here, when generating the plans, we design the prompt to guide the model to correct the failed testing cases of the current patch. 
This process continues until a qualified patch that passes all validations is generated, or the total number of generated patches reaches the predefined limit of $N_{\text{max}}$. 
Note that if the patches generated in a whole batch do not pass any new tests, we rerun the localization with the validation results to obtain a new root cause.
This additional step gives \sys the opportunity to leverage information from later components to correct localization errors and ultimately succeed in generating qualified patches.





















%========================== OLD VERSION ==============================


% \textbf{Workflow of \sys.}
% ~\cref{fig:overview} presents an overview of \sys{}, which takes as input a GitHub issue described in natural language and a buggy project codebase. 
% \sys{} processes these inputs through five phases: reproduction, fault localization, patch generation, validation, and refinement, and produces a candidate patch aimed at addressing the issue.

% \sys begins with the reproduction phase, where it performs issue reproduction and functionality test retrieval and output the a "proof of concept" (PoC), to replicate the issue described in the issue
% For issue reproduction, \sys generates a Python script, referred to as a "proof of concept" (PoC), to replicate the issue described in the issue. If the issue is successfully reproduced, \sys collects the PoC’s coverage data and attempts to identify the commit that introduced the issue. These outputs are then used in the subsequent fault localization and patch generation phases.

% For functionality test retrieval, \sys utilizes the issue description and the repository's directory tree structure to locate functionality tests relevant to the reported issue. The top-N most relevant test files are identified and returned.

% After completing the reproduction phase, \sys transitions into the fault localization phase, which employs a hierarchical approach to narrow down the potential patch locations across three levels of granularity: file level, class/function level, and fine-grained line level.
% At the file level, \sys constructs a tree-structured repository representation, filtered based on the PoC coverage, retaining only the files executed during PoC execution. The issue description and the repository representation are then provided to the LLM, prompting it to return the top $N$ files most relevant to the issue. Since the repository tree structure lacks detailed source code information, the LLM is also equipped with a set of search tools.
% At the class and function level, the LLM is provided with the signatures and comments of classes and functions extracted from the retrieved files and is prompted to identify functions and classes likely related to the issue.
% At the fine-grained line level, the complete source code of the identified classes and functions is presented to the LLM, which is then prompted to identify potential patch locations—specific code snippets requiring modification. At last, we ask LLM to review the localization results for self-consistency.

% \sys then transitions into the patch generation phase. Using the localization results and the issue description, the LLM is prompted to generate $batch\_size$ patch candidates.
% To generate each patch candidate, we first prompts the LLM to create a detailed plan, then generates patches incrementally by addressing each planned step. To ensure patch diversity, we vary the prompting strategies, requesting comprehensive preventive fixes, minimal modifications, or standard patches.

% After patch generation, \sys verifies each patch candidate by applying them and running the PoC and functionality tests retrieved during the reproduction phase. The best-performing patch from the current batch, based on validation results, is selected as the current best patch.

% If the current best patch passes all available tests, or if \sys has generated a total number of patch candidates equal to $max\_sample$, \sys will terminate and output the current best patch. 
% Otherwise, \sys enters the refinement stage, initiating a loop where it applies the current best patch to the codebase and generates another batch of patch candidates. 
% In each iteration, the LLM is provided with additional information, including the code of the failed tests and feedback from their execution. 
% Note that if none of the patch candidates in a batch passes more tests than the current best patch, \sys will re-perform fault localization using feedback from the failed tests before generating the next batch of patch candidates.

% In the following sections, we introduce the technical challenges and our design for each component in detail. 







%%%%%%%%%%%%%%%%%%%%%%%% Our text %%%%%%%%%%%%%%%%%%%%%%%%%%%%%%%%%

% \noindent\textbf{goal and key challenges} 
% Design a better predefined workflow for agent-based patching while open up more freedom for LLM. We hope to achieves top performance while maintaining low costs and high stability.



% \noindent\textbf{How to solve the changes}

% The key difference between methods lies in localization, which significantly affects outcomes. We combine human-designed workflow while giving llm some freedom (can re-do localization during refinement, can call search\_tools during localization).

% The refinement is also important. It is a natural advantage of the AutoCodeRover line of work. We also implement a refinement workflow.

% Existing methods do not fully leverage reproduction and validation and don't generate patch-specific pocs. We generate patch-specific pocs and we extract coverage and issue-introducing commits for localization and generation.  

% We also did other minor optimizations in the generation (step-by-step and ToT generation of patches).

% % \textbf{Technique details:} 
% \begin{itemize}
%     \item Coverage-based functionality test.
%     \item Existing methods didn't fully leverage the poc. In order to better leverage the poc, we also get coverage (for localization) and issue introducing commit (for repair).
%     \item PoCs specific to patches. Existing methods only generate pocs based on the issue description but don't generate pocs specific for patches. The poc generated from the issue description may not be the only way to trigger the issue, so that a patch passing the poc test may only partially fix the issue. We generate pocs intended to break a specific patch instead.
% \end{itemize}

% \subsection{Localization}
% \textbf{Technique details:} 
% \begin{itemize}
%     \item Capability of localization after generation
%     \item Search tools
%     \item Filtering based on coverage.
% \end{itemize}

% \subsubsection{Patch Generation (First do ablation study to see if it affects a lot)}
% \textbf{Technique details:} 
% \begin{itemize}
%     \item Planning that breaks down the patching task into multiple steps
%     \item ToT during generation of patches. For each step, generate multiple samples, do lint and syntax check, select the best partial patch.
% \end{itemize}

% \subsection{Iterative Refinement}

% Desicribe refinement workflow here.
\section{Evaluation}
\label{sec:Evaluation}

We evaluate \sys from the following aspects:
First, we perform a large-scale comparison of \sys with both SOTA open-source and closed-source methods on the SWE-Bench-Lite and SWE-Bench-Verified patching benchmark~\cite{jimenez2023swe}, showcasing \sys's ability to balance patching accuracy and cost-efficiency.
Second, we conduct a stability analysis on \sys and \openhands, demonstrating~\sys{}'s human-based planning is more stable than the SOTA agent-based planning. 
Third, we conduct an ablation study to quantify the contribution of each component to \sys's overall performance.
Finally, we show \sys's compatibility and performance on different models, including \gpt~\cite{GPT-4o}, \claude~\cite{anthropic_claude}, and a reasoning model \oo~\cite{GPT-o3}. 
We failed to integrate \deepseek~\cite{DeepSeek-r1} due to the problems with their APIs (See~\cref{appx:exp4}).
% In the following, we specify the setup and design of each experiment and discuss their results.

\subsection{\sys vs. Baselines on SWE-Bench}
\label{exp:comparison}

\noindent\textbf{Setup and design.}
We utilize the \textit{SWE-Bench}~\cite{jimenez2023swe} benchmark, where each instance corresponds to an issue in a GitHub repository written in \texttt{Python}.
Specifically, we consider two subsets: \textit{SWE-Bench-Lite}~\cite{SWE-Bench-Lite}, consisting of 300 instances, and \textit{SWE-Bench-Verified}~\cite{SWE-Bench-Verified}, comprising 500 instances that have been verified by humans to be resolvable.

We mainly compare \sys with three SOTA open-source methods: two agent-based planning methods \openhands~\cite{wang2024openhands} and \autocode~\cite{zhang2024autocoderover}, and a human-based planning method \agentless~\cite{xia2024agentless}. 
We also compare it with two closed-source methods that have cost reported: Globant Code Fixer Agent~\cite{Globant_Code_Fixer_Agent} (\globant for short) and CodeStory Midwit Agent~\cite{CodeStory_Midwit_Agent} (\codestory for short).
In~\cref{appx:exp1}, we include a more comprehensive comparison of \sys against 29 other tools, showing our positions on the SWE-Bench leaderboard.
Given our goal of addressing stability and cost together with the resolved rate, comparing closed-source methods that have a higher resolved rate but without cost is not our focus.
Most of these methods follow agent-based planning that may cost way more than ours.
For example, \codestory mentions that it costs them \$10,000 to achieve 62.2\% on the SWE-Bench-Verified benchmark~\cite{SWE-Bench-Verified}, whereas \sys achieves a 53.60\% with less than \$500 (20$\times$ cheaper).
In addition, as shown in~\cref{exp:Stability}, agent-based planning is less stable than human-based planning.

To align with most methods, we use the \claude model as the LLM in \sys.
~\cref{appx:implement} shows our implementation details.
We report two metrics \textit{Resolved Rate} (\%): the percentage of resolved instances in the benchmark,\footnote{An instance/issue is resolved means the patch fixes the issue while passing all hidden functionality tests.} and \textit{Average Cost} (\$): the average model API cost of running the tool on each instance. 
For the baselines, we retrieve their performance from their submission logs on the SWE-Bench and their papers and official blogs. 

\noindent\textbf{Results.}
\cref{tab:swe_bench_leaderboards} shows the performance of \sys and selected baselines on two subsets of the SWE-Bench benchmark.
Although, on both benchmarks, the closed-source methods achieve the highest performance, their internal design and methodology are not publicly available and we cannot assess their stability.
Notably, the cost of \codestory is 20$\times$ higher than \sys.
The cost of \globant is more comparable to \sys on SWE-Bench-Lite, but we cannot assess their performance and cost on SWE-Bench-Verified. 
Among open-source methods, \openhands achieves higher resolved rates than the human-based planning tool, \agentless, on both benchmarks. 
However, \openhands has a higher cost than \agentless, i.e., around $91.07\%$ more expensive on SWE-Bench Lite when using the same \claude. 
This result validates our discussion in Section~\ref{subsec:tech_overview}, human-based planning is more cost-efficient than agent-based planning, and agent-based planning has the potential to achieve higher optimal resolved rates.

In comparison, \sys demonstrates a clear advantage in balancing resolved rate and cost. 
On SWE-Bench Lite, it resolves 45.33\% (136/300) of the issues, outperforming all open-source methods with a low cost of \$0.97 per instance. 
Similarly, on SWE-Bench Verified, \sys achieves a resolved rate of 53.60\% (268/500), surpassing all open-source methods while maintaining the same cost efficiency of \$0.99 per instance.
These results highlight the efficacy and cost-efficiency of \sys.
% ~\cref{appx:case_study} provides a case study on the instances in which we succeed and fail. 


\subsection{\sys vs \openhands in Stability}
\label{exp:Stability}

\begin{figure}
    \centering
    \includegraphics[width=85mm]{Figures/evaluation/exp2_bar.pdf}
    \vspace{-4mm}
    \caption{\sys vs. \openhands in the resolved rate (bars) and the total cost (lines) on 45 instances from SWE-Bench-Lite.}
    \label{fig:stability_bar}
    \vspace{-4mm}
\end{figure}

\noindent\textbf{Setup and design.}
We compare the stability of \sys and \openhands, the SOTA open-source agent-based planning tool.
We find $102$ common instances resolved by \sys and \openhands in the SWE-Bench-Lite benchmark and randomly select a subset of $45$.
We run \sys and \openhands on these instances three times with \gpt model and different \texttt{Python} random seeds.
We report and compare their resolved rate and total cost in each run. 

\noindent\textbf{Results.}
\cref{fig:stability_bar} shows the resolved rate and costs of \sys and \openhands across three runs.
As shown in the figure, \sys consistently resolved more instances, achieving 30, 32, and 35 resolved instances in the three runs, with a standard deviation of 2.52.
In comparison, \openhands resolved only 15, 20, and 21 instances, with a higher standard deviation of 3.21.
The lower standard deviation of \sys demonstrates its stability, which further validates our discussion about human-based planning vs. agent-based planning in~\cref{subsec:tech_overview}.
Additionally, \sys demonstrated a clear advantage in terms of cost efficiency, with costs of \$8.72, \$14.81, and \$14.42 for the three runs, resulting in an average of \$12.65 per run. 
This is substantially lower than \openhands, which incurred costs of \$32.78, \$33.31, and \$34.97, with an average of \$33.69 per run. 
These results further highlight \sys's ability to achieve higher resolved rates with greater stability and at a lower cost.

\subsection{Ablation Studies}
\label{exp:ablation}

\begin{figure}
    \centering
    \includegraphics[width=85mm]{Figures/evaluation/exp3_bar.pdf}
    \vspace{-2mm}
    \caption{Ablation study results on the SWE-Bench-Lite benchmark. \ding{182}$\sim$\ding{185} refers to \textit{Base Local+Gen}, \textit{Our Local+Gen}, \textit{Our Local+Gen+PoC}, and \textit{Our Local+Gen+Val}, respectively.}
    \label{fig:abliation_bar}
    \vspace{-4mm}
\end{figure}

\noindent\textbf{Setup and design.}
We conduct a detailed ablation study to investigate the efficacy of key designs in \sys.
We use the full SWE-Bench-Lite benchmark and the \claude model for all variations of our method.
Specifically, we consider the following four variations:
\noindent\ding{182}\textit{Base Local+Gen}: We combine simple localization without providing the LLM with tools or a review step, along with simple generation without the two-phase design (\cref{fig:overview}).
We choose the final patch by majority voting.
\noindent\ding{183}\textit{Our Local+Gen}: We combine \sys's localization and generation components together with majority voting for final patch selection. 
Comparing \ding{182} with \ding{183} can assess the effectiveness of our proposed techniques for localization and generation. 
\noindent\ding{184}\textit{Our Local+Gen+PoC}: We further add our validation component to \ding{183} but with only the PoC tests (the validation strategy employed by most existing tools).
Comparing \ding{183} with \ding{184} can assess the effectiveness of having PoC validation instead of simple majority voting. 
\noindent\ding{185}\textit{Our Local+Gen+Val}: We add the full validation component, comparing \ding{184} with \ding{185} can assess the efficacy of having functionality tests in validation.
Finally, comparing \ding{185} with \sys can assess the importance of having an additional refinement component. 

\noindent\textbf{Results.}
\cref{fig:abliation_bar} shows the resolved rates across different variations and our final method. 
By incrementally building upon the core functionalities of \sys, we evaluate the contributions of individual components to the overall patching performance.

\underline{Localization and generation.}
First, we can observe that \ding{182} with the simple localization and generation only get a resolved rate of 32.7\% (98/300). 
In contrast, \ding{183} with our improved localization and generation increases the resolved rate to 38.7\% (116/300).
This result first confirms the challenges of simple localization and generation designs discussed in~\cref{subsec:localization} and~\cref{subsec:generation}, as they prevent \ding{182} from achieving a better performance.
More importantly, it validates the effectiveness of our designs in adding tools and a review step in localization and the two-step procedure (i.e., planning and generation) in the generation. 

\underline{PoC and functionality validation.}
\ding{184} with our localization and generation as well as PoC validation unexpectedly lowers the resolved rate to 37.00\% (111/300). 
This result suggests that relying solely on PoC validation may resolve the targeted issue while introducing new functional issues. 
As such, when functionality tests are added, \ding{185} significantly improves the resolved rate to 41.67\% (125/300). 
This result shows that functionality tests play a crucial role in identifying and filtering out the patches that fix the target issues but break the original functionalities of the codebase.
As mentioned above, a patch must pass all hidden functionality tests to be marked as a success; having functionality tests is important to filter out false positives. 

\underline{Refinement.}
Finally, adding our refinement component on top of \ding{185} improves the resolved rate from 41.67\% to 45.33\%.
The result demonstrates the effectiveness of our refinement design. 
It also justifies our claim in~\cref{subsec:refinement} that generating new patches from scratch when the current trial fails is less effective than refining the partially correct patches based on the validation feedback. 

\subsection{\sys on Different Models}
\label{subsec:Model_test}

% \begin{table}[t]
% \centering
% \caption{\sys with different choices of LLMs on 100 cases from SWE-Bench-Lite.}
% \label{tab:model_comparison}
% %\resizebox{\textwidth/2}{!}{
% \begin{tabular}{rc}
% \toprule
% \textbf{LLM Model} & \textbf{Resolved Rate (\%)}  \\
% \midrule
% \gpt & xxxx (x.00\%)  \\
% \oo & 43 (43.00\%) \\
% \claude & 39 (39.00\%)  \\
% \deepseek & xxxx (x.00\%)  \\
% \bottomrule
% \end{tabular}
% %}
% \end{table}

\noindent\textbf{Setup and design.}
To demonstrate the compatibility of \sys to different LLMs, we conduct an experiment that integrates \sys with three SOTA LLMs: two general models \gpt and \claude, and one reasoning model: \oo. 
We select a subset of 100 instances from the SWE-Bench-Lite benchmark; all these 100 instances have been successfully resolved by at least one method ranked Top-10 on the SWE-Bench leaderboard. 
We run \sys with the selected models on these instances and report the final resolved rate. 
We keep all other components the same and only change the model to show the impacts of the different models.

% We record the resolution rate (percentage of successfully resolved cases) for each LLM, comparing their performance in terms of patch accuracy and consistency. 
% By isolating the model variable, this setup allows us to quantify the contribution of each LLM to \sys’s overall efficacy and assess their suitability for automated code patching tasks.

% verified result: root@8bb690ee5f0b:/opt/PatchingAgent# ls results_final_verified/
% lite result: root@4178337f2502:/opt/PatchingAgent/results_final_lite 

\noindent\textbf{Results.}
The resolved rate of \sys with different models are: \gpt: 19.00\%; \claude: 39.00\%, and \oo: 43.00\%.
\oo achieves the highest resolved rate, indicating having inference-phase reasoning capabilities is helpful not only for general math and coding tasks but also for the specialized patching task.
Note that although we cannot directly compare with the results reported from official reports~\cite{Claude_SWE_report,o1_SWE_report,o3_SWE_report}, as they conduct their testing on the SWE-Bench-Verified benchmark. 
However, they follow the same trend: \oo > \claude > \gpt.
It is also worth noting that \sys with \claude on the SWE-Bench-Verified benchmark reports a higher resolved rate than the official report from \claude and OpenAI-O1 model.
Although the full o3 reports a resolved rate of 71.7\%, it do not disclose any details about the system design, cost, and stability. 
Overall, this experiment demonstrates the compatibility of \sys to different models as well as the efficacy of having a reasoning model in \sys.



\section{Discussion}

% Shift from findings to discussion
This study on robotic art explores human-machine relationships in creative processes.
It first contributes as an empirical description of artistic creativity in robotic art practice---an unconventional use of robots---examined through the artists' perspectives on their creative experiences. Our analysis reveals three facets of creativity in robotic art practices: the \textit{social}, \textit{material}, and \textit{temporal}. Creativity emerges from the co-constitution between artists, robots, audience, and environment in spatial-temporal dimensions, as illustrated in \autoref{PracticeDiagram}. Acknowledging the audience as an important actor reflects the social dimension in that creativity does not stem from the artists but from their interactions with the audience. Robots are the major material and technological actants characterizing creative practices, shaping the conditions for creativity to emerge. The axis of the temporal process signifies that the practice is situated within a time continuum, and the actors/actants and their relations shift over time. In this way, temporality constitutes another dimension of creativity in robotic art.

Accordingly, as the second contribution, this study outlines implications for \textit{socially informed}, \textit{material-attentive}, and \textit{process-oriented} creation with computing systems\footnote{For the sake of clarity, we mean ``creation with computing systems'' by three types of scenarios: human creator(s) create computing system(s) as the final artifact(s) (e.g., robots are artworks themselves); human creator(s) use computing system(s) to create the artifact(s) (e.g., robots create artworks as human planned); and human creator(s) and system(s) work in tandem to produce the artifact(s) (e.g., human-robot co-creation).} to facilitate creation practices. These insights can inform related HCI research on media/art creation, crafting, digital fabrication, and tangible computing.
In each following subsection, we present each implication with a grounding in corresponding findings from our study and relevant literature in HCI and adjacent fields on art, creativity, and creation.

\begin{figure*}[htbp]
    \centering
    \includegraphics[width=0.88\textwidth]{Writings/figure/PracticeDiagram.pdf}
    \caption{Actors/actants in robotic art practice and their interactive relations. Robotic art practice unfolds primarily in two spaces: the creation space where interactions happen mainly between artists and robots, and the exhibition space where interactions mostly involve audiences and robots. The two spaces constitute the ENVIRONMENT plane. Within the plane, directed arrows between the actors indicate the types of interaction. For example, the \textit{Design} arrow indicates that the artist designs the robot(s), and the \textit{Revise} arrow indicates that the robot(s) make the artist revise artistic ideas. All the actors/actants may also intra-act with the ENVIRONMENT. The actors/actants and their interactive relations may differ at different times along the axis of TEMPORAL PROCESS that is orthogonal to the plane.}
    \Description{This figure shows the actors/actants in robotic art practice and their interactive relations.}
    \label{PracticeDiagram}
\end{figure*}

\subsection{Socially Informed Creation}

% Introduce social aspect of distributed creativity
The sociality of creativity means that creativity is distributed among different human actors, commonly within the creators or between the creators and the audience. Glăveanu’s ethnographic study on Easter egg decoration in northern Romania~\cite{glaveanu_distributed_2014} showed that artisans anticipate how others might appreciate their work and adjust their creative decisions accordingly. Even in the absence of direct interaction, the audience’s potential responses become part of the creative process, as artisans imagine feedback and predict reactions. In this sense, the sociologist Katherine Giuffre argues that ``\textit{creative individuals are embedded within specific network contexts so that creativity itself, rather than being an individual personality characteristic is, instead, a collective phenomenon}''~\cite[p. 1]{giuffre2012collective}.

% Recall findings about audience feedback
We found that the practice of robotic art manifests this sociality as it involves, particularly artists and audiences. 
Our artists take audiences' reactions to their artwork as feedback and then revise the robots' functions and aesthetics accordingly. 
For example, as shown earlier, Robert added a protective fuse onto his robot because he expected that children would squeeze the springs together and cause a short circuit; Alex's enthusiasm and attention to the audience's imagination about his robots led him to new aesthetic designs of both the robots and the scene layouts. The artists may directly ask about the audience's judgment of quality but they often receive feedback just by observing the audience's reactions or sometimes by learning from the audience's imagination about the robots.
% Recall findings about audience's sociocultural expectations and codes
Meanwhile, our findings reveal that audience reception is not an individual matter but is often associated with their sociocultural codes, including shared cultural norms, beliefs, expectations, and aesthetic values. The audience can be seen as representatives of these broader cultural codes. For example, Mark and Robert observed that the animist tendency in some East Asian societies is associated with higher acceptance of and interest among the audience in intelligence and agency of robots and non-human entities. Such sociocultural contexts influence not only how audiences interpret the work but also how artists anticipate and respond to these perspectives in their creative process.

% Situate in HCI literature
A creative process, including creation and reception, is essentially a social activity. The second wave of creativity research in psychology has argued for creativity's dependency on sociocultural settings and group dynamics~\cite{sawyer2024explaining}. Recent discussions from creativity-support and social computing researchers also called for more attention to the social aspect of creativity~\cite{kato2023special, fischer2005beyond, fischer2009creativity}. There is a clear need to consider the audience when producing creative content. For instance, researchers studying video-creation support have examined audience preferences to inform system designs that align with these preferences~\cite{wang2024podreels}. Such work highlights how creative activities extend beyond individual creators to co-creators and heterogeneous audiences. Some HCI researchers conceptualize creativity as by large a socially constructed concept, perceived and determined by social groups~\cite{fischer2009creativity}. 
Prior HCI work examined the social aspects between art creators. For example, creators and performers in music and dance form social relationships through artifacts, making the final work a collaborative outcome~\cite{hsueh2019deconstructing}. There is also a system designed to support collaborative creation between artists~\cite{striner2022co}. However, the social creative process between creators and audience is less articulated in HCI. Jeon et al.'s work~\cite{jeon2019rituals} stands as an exception, suggesting that professional interactive art can involve evaluation with the audience in the creation stage. 
Another relevant approach in HCI involves enabling the general public to participate in co-creation alongside professional creators. ~\citet{matarasso2019restless}, for instance, promoted ``participatory art'' as ``\textit{the creation of an artwork by professional artists and non-professional artists working together}'' with non-professional artists referring to the general public engaged in the art-making process. Similarly, socially inclusive community-based art also considers target communities' perception of the artwork during creation~\cite{clark2016situated, clarke2014socially}. But like participatory design~\cite{schuler1993participatory}, these art projects aim for social justice more than creativity in the work~\cite{murray2024designing}, let alone that direct participation in art creation is not always feasible. Our findings suggest that feedback from the audience can lead to creative ideas, as well as that the feedback can be generative and remain low-effort for the audience.

Unlike conventional design feedback---which is typically expected to be specific, justified, and actionable~\cite{yen2024give, krishna2021ready}---the feedback that resonates with our artists is often implicit, creative, and generative. Such feedback may include audiences' imaginations stimulated by the work, personal and societal reflections, and even emotions, facial expressions, micro-actions, and observable behaviors following the art experience. Our artists gathered this implicit feedback not by posing evaluative questions, as commonly done in typical design processes (e.g., usability testing, think-aloud protocols), which seek to elicit clear, relatively structured responses. Instead, they closely observe the audience's reactions and interpret their subjective perceptions. This form of implicit feedback, while indirect, can lead to more creative ideas by embracing open, multifaceted interpretations of the work~\cite{sengers2006staying}. Computing systems for creation should better incorporate implicit feedback in addition to explicit ones from the audience into the creation process. Implicit feedback can be indirect, creative, inspirational, and heuristic about functions and aesthetics. A hypothetical instance of such design can be a system that helps creators perceive audiences' implicit reactions and perceptions and variously interpret them, for further iteration.

% Recall findings about audience interacting with robots as a performative art
Moreover, as seen in Robert and Daniel's experiences, the audience may participate in robotic live performances by interacting with the robots, who may change actions accordingly, triggering a loop of simultaneous mutual influence that makes the work performative and improvisational.
% Situate in HCI
HCI researchers explored performative and improvisational creation with machines, focusing on developing and evaluating systems with performative capabilities, including music improvisation with robots~\cite{hoffman2010shimon}, dance with virtual agents~\cite{jacob2015viewpoints, triebus2023precious}, and narrative theatre~\cite{magerko2011employing, piplica2012full}. \citet{kang2018intermodulation} discussed the improvisational nature of interactions between humans and computers and argued that an HCI researcher-designers' improvisation with the environment facilitates the emergence of creativity and knowledge. Designs of computing systems for creation can leverage performativity in service of creative experience. One possible direction could be to allow the audience to embed themselves in and interact with elements of static artwork in a virtual space, turning the exhibition into an improvisational on-site creation~\cite{zhou2023painterly}.
% Our new implication different from current discussion on perf and impr
While interactions with machines during performance are mostly physical or embodied, we posit that they can also be a \textit{symbolic engagement}. Alex's audience projected themselves and their personalities onto his robots, which established a symbolic relevance, generating creative imaginations. During exhibitions, East Asian audiences carried the animist views shaped by their sociocultural backgrounds, and robots, through the performance, were successful in symbolically matching the views, stimulating aesthetic satisfaction. Symbolic engagement resonates with what ~\citet{nam2014interactive} called the ``reference'' of the interactive installation performance to participants' sociocultural conditions.
As such, we propose that designers of computing systems for creation may consider establishing symbolic engagement between the produced artifacts and the audience as a way to enhance perceived creativity or enrich the creative experience. One example is an interactive installation, \textit{Boundary Functions}~\cite{snibbe1998}, which encourages viewers to reflect on their personal spaces while interacting with the installation and others. Another example is \textit{Blendie}, a voice-controlled blender that requires a user to ``speak'' the machine's language to use it. This interaction builds a symbolic connection between the user and the device, transforming the act of blending into a novel experience~\cite{dobson2004blendie}.


\subsection{Material-Attentive Creation}

% Intro paragraph to the importance of materiality for creative activities with machines and the end goal of this discussion--- design suggestions
The theory of distributed creativity by Glaveanu claims that creativity distributes across humans and materials, so the creation practice itself is inevitably shaped by objects~\cite{glaveanu_distributed_2014}. In his case of Easter egg decoration, materials are not passive objects but active participants in artistic creation; e.g., the egg decorators face challenges from color pigments not matching the shell, wax not melted at the desired temperature, to eggs that break at the last step of decoration; hence, materials often go against the decorators' intentions and influence future creative pathways~\cite{glaveanu_distributed_2014}.
Materials manifest specific properties, which afford certain uses of the materials while constraining others~\cite{leonardi2012materiality}. Our findings highlight the critical role of materiality in artistic practice, showing that artists intentionally arrange materials to enhance the creative values of their work.

% Materiality aspect One: physicality and embodiment
% Embodiment or physicality fascilitates creative interaction with machines
Robotic art relies on the material properties of robots and other objects. An apparent property of most materials is their physicality~\cite{leonardi2012materiality}, meaning they possess a tangible presence that enables interaction with other physical entities. Here, we consider physicality and embodiment interchangeable as computational creativity researchers have conceptualized~\cite{guckelsberger2021embodiment}.
% Recall findings on embodiment's value in making art
Our findings support both the conceptual and operational contributions of embodiment for creative activities. For the conceptual aspect, the embodied presence of robotic systems supports creative thinking for our artists, exemplary in Linda's case where she found new art ideas around the difference between human and robot bodies through bodily engagement with robots. 
For the operational aspect, the embodied nature of robotic artworks and their creation processes exhibit original aesthetics that are based on physics much different from disembodied works, e.g., embodied drawings by David's non-industrial robotic arms are dynamic due to physical movements and thus artistically pleasant, which is hard to replicate in simulated programs.

% References: embodied interaction, embodied cognition theories, tangible computing
These findings on embodiment of robotic art (Section \ref{f:emb}) closely relate to HCI's attention on embodied interaction as a way to leverage human bodies and environmental objects to expand disembodied user experiences. 
For example, as~\citet{hollan2000distributed} explained, a blind person's cane and a cell biologist's microscope as embodied materials are part of the distributed system of cognitive control, showing that cognition is distributed and embodied. 
Similarly, theories of embodied interaction in HCI explicate how bodily interactions shape perception, experience, and cognition~\cite{marshall2013introduction, antle2011workshop, antle2009body}, backed up by the framework of 4E cognition (embodied, embedded, enactive, and extended)~\cite{wheeler2005reconstructing, newen20184E}. 
Prior works suggest that creative activities with interactive machines rely on similar embodied cognitive mechanisms ~\cite{guckelsberger2021embodiment, malinin2019radical}, which are operationalized by tangible computing~\cite{hornecker2011role}. 
% References: embodiment's consequence in creation
As related to robots in creation, HCI researchers show that physicality or embodiment of robots in creation may lead to some beneficial outcomes, such as curiosity from the audience, feelings of co-presence, body engagement, and mutuality, which are hard to simulate through computer programs~\cite{dell2022ah, hoggenmueller2020woodie}. Embodied robotic motions convey emotional expressions and social cues that potentially enrich and facilitate creation activities like drawings~\cite{ariccia2022make, grinberg2023implicit, dietz2017human, santos2021motions}. Guckelsberger et al.~\cite{guckelsberger2021embodiment} showed in their review that embodiment-related constraints (e.g., the physical limitations of a moving robotic arm) can also stimulate creativity. These constraints push creators to develop new and useful movements, echoing the broader principle that encountering obstacles in forms or materials can lead to generative processes. This phenomenon is similarly observed in activities such as art and digital fabrication~\cite{devendorf2015being, hirsch2023nothing}. In co-drawing with robots, physical touch and textures of drawing materials made the artists prefer tangible mediums (e.g., pencils) than digital tools (e.g., tablets) that fall short in these respects~\cite{jansen2021exploring}.

% Transit to materiality aspect two
% Materiality aspect Two: malfunction as manifestation of unique materiality of robots
% Intro to materials of robots
Materiality plays a crucial role in the embodiment of robots, as the choice of materials fundamentally shapes the physical forms and properties. This focus on materials extends to art practices, where robots made with soft materials introduce new aesthetics and sensory experiences~\cite{jorgensen2019constructing, belling2021rhythm}, and the use of plants and soil in robotic printing creates unique visual effects~\cite{harmon2022living}. Following Leonardi's ~\cite{leonardi2012materiality} conceptualization of materiality, we refer to the materials of robots as encompassing physical and digital components---including the shell, hardware, mechanical parts, software, programs, data, and controllers---each significant to the artist's intent. ~\citet{nam2023dreams} found that the material constraints of robots can limit creative expression but simultaneously stimulate creativity when artists push the boundaries.

%-----maybe here the real "malfuction" start ------------------
% Move to introduce malfunctions as unique materiality

Even carefully designed, digital and mechanical components in robots are prone to errors or bugs in everyday runs, causing malfunctions or unexpected consequences. This reflects the unique materiality of robots as complex computing systems. From an engineering perspective, errors signal unreliability and must be eliminated, driving advancements in robotics---where error detection and recovery are central~\cite{gini1987monitoring}---as well as in digital fabrication, which prioritizes precision over creative exploration~\cite{yildirim2020digital}. % Recall findings on embracing malfunctions
However, material failures and accidents are inevitable, exemplifying what has been called the `craftsmanship of risk'~\cite{glaveanu_distributed_2014} in material art. For our artists, these risks are often creatively utilized and incorporated into their work: these moments of breakdown---whether physical or digital---become resources for new creative expression. Errors are anticipated and intentionally designed into the process and work of our artists. In some cases, such as for Alex, the entire concept of one of his works is machine errors.

% Situate in literature
Reports on how artists view errors within engineering and creation processes are dispersed throughout HCI literature. ~\citet{nam2023dreams} showed that the accumulation of ``contingency'' and ``accidents''---unexpected, serendipitous, and emergent events during art creation like errors---meaningfully constituted the final presentation of the artwork. Song and Paulos's concept of ``unmaking'' highlighted the values of material failures in enabling new aesthetics and creativity~\cite{song2021unmaking}. Kang et al.~\cite{kang2022electronicists, kang2023lady} introduced the notion of an ``error-engaged studio'' for design research in which errors in creative processes are identified, accommodated, and leveraged for their creative potential. Collectively, these works advocate for reframing errors from something to avoid to something to embrace and recognize. We want to push this further by arguing that errors can be intended and be part or sometimes entire of the design. Several artists, including participants from our study, have been deliberately seeking errors to formulate their designs. Roboticist Damith Herath recounted when he mistakenly programmed a motion sequence of a robotic arm, his collaborator, robotic artist Stelac responded with ``[W]e need to make more mistakes;'' as many mistakes were made, the initial pointless movements became beautiful, rendering the robot ``alive'' and ``seductive'' \cite{herath2016robots}. Similarly, AI artists sometimes look for program glitches to generate unusual styles and content~\cite{chang2023prompt}. Therefore, creators may not only passively accept errors but can actively seek and utilize them. Errors can be integral to the design itself---errors can \textit{be designed into} an artifact, and the design/idea of the artifact can be all about errors.

Thus, to focus on material-attentive creation---considering the creative arrangement of materials---we suggest exploring the embodiment and materiality of creation materials, objects, and environments to recognize their creative potential. %This perspective aligns with insights from professional digital fabrication practitioners, who advocate for systems that integrate support for machine settings and material properties~\cite{hirsch2023nothing}.
Specifically, we propose using a design method/probe that enables creators to realize both the conceptual and operational contributions of materiality. This approach may build on the material probe developed by~\citet{jung2010material}, which calls for exploring the materiality of digital artifacts. A material-attentive probe would enable creators to engage with diverse materials, objects, and environments through embodied interaction, encouraging them to speculate on material preferences and limitations, and to compare and contrast material qualities---insights that can inform creative decisions.
To accommodate, seek, and actively harness the creative potential of errors, we propose embracing failures, glitches, randomness, and malfunctions in computing systems as critical design materials---elements that creators can intentionally control and manipulate. By doing so, we can begin to systematically approach errors. For instance, as part of the design process, we may document how to replicate these errors and changes, allowing creators to explore them further at their discretion. This could include intentionally inducing errors or random changes to influence the creative process or outcomes.

\subsection{Process-Oriented Creation}

% Introduce the key idea: process itself embeds creative value and can be pursued as the goal of creation
As shown in our findings, the creation process itself embeds creative values and meanings, and experiencing the process can be pursued as the goal of creation with computing systems.
% Recall findings
For the robotic artists in our study, artistic values were often placed on the creation process rather than the outcome.  For example, in Alex's robotic live drawing performance, the drawing process is more important than the drawn pattern on canvas. Techniques used, decisions made, or stimuli received by robots during creation or exhibition reflect artistic ideas and nuanced thinking, as seen in Sophie's exploration of interactive decision-making in robotic drawing.

% Situate in HCI lit
Previous HCI work has touched on the value of the process of creation. ~\citet{bremers2024designing} shared a vignette where a robotic pen plotter simultaneously imitates the creator's drawing, serving as a material presence rather than a pragmatic co-creator; here the focus of the work is no longer the outcome but the process of drawing itself. ~\citet{devendorf2015reimagining} concluded that performative actions of digital fabrication systems, rather than the fabricated products themselves, convey artistic meanings tied to histories, public spaces, time, environments, audiences, and gestures. This emphasis on process is particularly significant for media such as improvisational theatre, where the creation itself is an integral part of the final work~\cite{o2011knowledge}. ~\citet{davis2016empirically} named their improvisational co-drawing robotic agents as ``casual creators,'' who are meant to creatively engage users and provide enjoyable creative experiences rather than necessarily helping users make a higher quality product. Shifting the focus from product to process and experiences \textit{in} creation may generate alternative creative meanings.

% Findings about process extends beyond creation
Our artists pointed out that even a ``finished'' artwork in an exhibition is not truly finished. A crack in Daniel's robotic artwork introduced a new artistic meaning, ultimately subverting the entire work. As the properties of the work change over time---whether due to the artist's intent, material characteristics, or environmental factors---the artwork evolves, revealing new aesthetics and meanings. % Situate in HCI lit
Based on these observations, we argue that creation processes should not be regarded as one-shot transactions, as creative artifacts, particularly physical ones, continue to change and generate artistic values. For instance, material wear and destruction bring unique aesthetics, often contrasting with the original form ~\cite{zoran2013hybrid}, and are seen as signs of mature use~\cite{giaccardi2014growing}.
Changes such as material failure, destruction, decay, and deformation---what~\citet{song2021unmaking} referred to as ``unmaking,'' a process that occurs after making---meaningfully transforms the original objects. Similarly, through Broken Probes, a process of assembling fractured objects, ~\citet{ikemiya2014broken} demonstrated that personal connections, reminiscence, and reflections related to material wear and breakage add new values to the objects. Drawing from Japanese philosophy Wabi-Sabi, ~\citet{tsaknaki2016expanding} reflected on the creeds of `Nothing lasts,' `Nothing is finished,' and `Nothing is perfect' and pointed to the impermanence, incompleteness, and imperfection of artifacts as a resource that designers, producers, and users can utilize to achieve long-term, improving, and richer interactive experience~\cite{tsaknaki2016things}. Insights from this study contribute to this line of thought by showing how robotic artists appreciate the aesthetics and meanings of temporal changes after the creation phase.

The findings underscore the need to reconceptualize creation as encompassing more than just the process aimed at producing a final product; it also includes what we term \textit{post-creation}. Distinct from repair, maintenance, or recycle, \textit{post-creation} entails anticipating and managing how an artifact evolves after its ``completion'' in the conventional sense. Specifically, we encourage creators to anticipate and strategically engage with the post-creation phase, considering potential changes to the artifact and their consequences for interactions with human users. For instance, during the creation process, creators may focus on possible material changes the artifact might undergo post-creation, allowing them to either mitigate or creatively exploit these potential changes. This expanded view of creation invites us to trace post-creation developments and to plan how our creative intentions can be embedded in its potential degradation, transformation, or evolution over time.

% A conclusion paragraph
We categorize the design implications into three aspects, but we do not suggest that a computing system must implement all simultaneously, nor that each aspect should be considered in isolation. Social interactions, such as those between artists and audiences, already presume the presence of material actants like robots, and these interactions inform future arrangements of materials. Thus the social and material aspects can be entangled and mutually constitutive as seen in sociomaterial practices~\cite{orlikowski2007sociomaterial, cheatle2015digital, rosner2012material}. The temporal aspect is orthogonal to the other aspects because both social interactions and material manifestations unfold and shift in a temporal continuum.

\section{Conclusion }
This paper introduces the Latent Radiance Field (LRF), which to our knowledge, is the first work to construct radiance field representations directly in the 2D latent space for 3D reconstruction. We present a novel framework for incorporating 3D awareness into 2D representation learning, featuring a correspondence-aware autoencoding method and a VAE-Radiance Field (VAE-RF) alignment strategy to bridge the domain gap between the 2D latent space and the natural 3D space, thereby significantly enhancing the visual quality of our LRF.
Future work will focus on incorporating our method with more compact 3D representations, efficient NVS, few-shot NVS in latent space, as well as exploring its application with potential 3D latent diffusion models.



% Acknowledgements should only appear in the accepted version.
% \section*{Acknowledgements}

% \section*{Impact Statement}
\system advances cost-efficient AI by demonstrating how small on-device language models can collaborate with powerful cloud-hosted models to perform data-intensive reasoning. By reducing reliance on expensive remote inference, \system makes advanced AI more accessible and sustainable. This has broad societal implications, including lowering barriers to AI adoption and enhancing data privacy by keeping more computations local. However, careful consideration must be given to potential biases in small models and the security risks of local code execution. 
% In the unusual situation where you want a paper to appear in the
% references without citing it in the main text, use \nocite

\bibliography{ref}
\bibliographystyle{icml2024}


%%%%%%%%%%%%%%%%%%%%%%%%%%%%%%%%%%%%%%%%%%%%%%%%%%%%%%%%%%%%%%%%%%%%%%%%%%%%%%%
%%%%%%%%%%%%%%%%%%%%%%%%%%%%%%%%%%%%%%%%%%%%%%%%%%%%%%%%%%%%%%%%%%%%%%%%%%%%%%%
% APPENDIX
%%%%%%%%%%%%%%%%%%%%%%%%%%%%%%%%%%%%%%%%%%%%%%%%%%%%%%%%%%%%%%%%%%%%%%%%%%%%%%%
%%%%%%%%%%%%%%%%%%%%%%%%%%%%%%%%%%%%%%%%%%%%%%%%%%%%%%%%%%%%%%%%%%%%%%%%%%%%%%%
\newpage
\appendix
\onecolumn
\subsection{Lloyd-Max Algorithm}
\label{subsec:Lloyd-Max}
For a given quantization bitwidth $B$ and an operand $\bm{X}$, the Lloyd-Max algorithm finds $2^B$ quantization levels $\{\hat{x}_i\}_{i=1}^{2^B}$ such that quantizing $\bm{X}$ by rounding each scalar in $\bm{X}$ to the nearest quantization level minimizes the quantization MSE. 

The algorithm starts with an initial guess of quantization levels and then iteratively computes quantization thresholds $\{\tau_i\}_{i=1}^{2^B-1}$ and updates quantization levels $\{\hat{x}_i\}_{i=1}^{2^B}$. Specifically, at iteration $n$, thresholds are set to the midpoints of the previous iteration's levels:
\begin{align*}
    \tau_i^{(n)}=\frac{\hat{x}_i^{(n-1)}+\hat{x}_{i+1}^{(n-1)}}2 \text{ for } i=1\ldots 2^B-1
\end{align*}
Subsequently, the quantization levels are re-computed as conditional means of the data regions defined by the new thresholds:
\begin{align*}
    \hat{x}_i^{(n)}=\mathbb{E}\left[ \bm{X} \big| \bm{X}\in [\tau_{i-1}^{(n)},\tau_i^{(n)}] \right] \text{ for } i=1\ldots 2^B
\end{align*}
where to satisfy boundary conditions we have $\tau_0=-\infty$ and $\tau_{2^B}=\infty$. The algorithm iterates the above steps until convergence.

Figure \ref{fig:lm_quant} compares the quantization levels of a $7$-bit floating point (E3M3) quantizer (left) to a $7$-bit Lloyd-Max quantizer (right) when quantizing a layer of weights from the GPT3-126M model at a per-tensor granularity. As shown, the Lloyd-Max quantizer achieves substantially lower quantization MSE. Further, Table \ref{tab:FP7_vs_LM7} shows the superior perplexity achieved by Lloyd-Max quantizers for bitwidths of $7$, $6$ and $5$. The difference between the quantizers is clear at 5 bits, where per-tensor FP quantization incurs a drastic and unacceptable increase in perplexity, while Lloyd-Max quantization incurs a much smaller increase. Nevertheless, we note that even the optimal Lloyd-Max quantizer incurs a notable ($\sim 1.5$) increase in perplexity due to the coarse granularity of quantization. 

\begin{figure}[h]
  \centering
  \includegraphics[width=0.7\linewidth]{sections/figures/LM7_FP7.pdf}
  \caption{\small Quantization levels and the corresponding quantization MSE of Floating Point (left) vs Lloyd-Max (right) Quantizers for a layer of weights in the GPT3-126M model.}
  \label{fig:lm_quant}
\end{figure}

\begin{table}[h]\scriptsize
\begin{center}
\caption{\label{tab:FP7_vs_LM7} \small Comparing perplexity (lower is better) achieved by floating point quantizers and Lloyd-Max quantizers on a GPT3-126M model for the Wikitext-103 dataset.}
\begin{tabular}{c|cc|c}
\hline
 \multirow{2}{*}{\textbf{Bitwidth}} & \multicolumn{2}{|c|}{\textbf{Floating-Point Quantizer}} & \textbf{Lloyd-Max Quantizer} \\
 & Best Format & Wikitext-103 Perplexity & Wikitext-103 Perplexity \\
\hline
7 & E3M3 & 18.32 & 18.27 \\
6 & E3M2 & 19.07 & 18.51 \\
5 & E4M0 & 43.89 & 19.71 \\
\hline
\end{tabular}
\end{center}
\end{table}

\subsection{Proof of Local Optimality of LO-BCQ}
\label{subsec:lobcq_opt_proof}
For a given block $\bm{b}_j$, the quantization MSE during LO-BCQ can be empirically evaluated as $\frac{1}{L_b}\lVert \bm{b}_j- \bm{\hat{b}}_j\rVert^2_2$ where $\bm{\hat{b}}_j$ is computed from equation (\ref{eq:clustered_quantization_definition}) as $C_{f(\bm{b}_j)}(\bm{b}_j)$. Further, for a given block cluster $\mathcal{B}_i$, we compute the quantization MSE as $\frac{1}{|\mathcal{B}_{i}|}\sum_{\bm{b} \in \mathcal{B}_{i}} \frac{1}{L_b}\lVert \bm{b}- C_i^{(n)}(\bm{b})\rVert^2_2$. Therefore, at the end of iteration $n$, we evaluate the overall quantization MSE $J^{(n)}$ for a given operand $\bm{X}$ composed of $N_c$ block clusters as:
\begin{align*}
    \label{eq:mse_iter_n}
    J^{(n)} = \frac{1}{N_c} \sum_{i=1}^{N_c} \frac{1}{|\mathcal{B}_{i}^{(n)}|}\sum_{\bm{v} \in \mathcal{B}_{i}^{(n)}} \frac{1}{L_b}\lVert \bm{b}- B_i^{(n)}(\bm{b})\rVert^2_2
\end{align*}

At the end of iteration $n$, the codebooks are updated from $\mathcal{C}^{(n-1)}$ to $\mathcal{C}^{(n)}$. However, the mapping of a given vector $\bm{b}_j$ to quantizers $\mathcal{C}^{(n)}$ remains as  $f^{(n)}(\bm{b}_j)$. At the next iteration, during the vector clustering step, $f^{(n+1)}(\bm{b}_j)$ finds new mapping of $\bm{b}_j$ to updated codebooks $\mathcal{C}^{(n)}$ such that the quantization MSE over the candidate codebooks is minimized. Therefore, we obtain the following result for $\bm{b}_j$:
\begin{align*}
\frac{1}{L_b}\lVert \bm{b}_j - C_{f^{(n+1)}(\bm{b}_j)}^{(n)}(\bm{b}_j)\rVert^2_2 \le \frac{1}{L_b}\lVert \bm{b}_j - C_{f^{(n)}(\bm{b}_j)}^{(n)}(\bm{b}_j)\rVert^2_2
\end{align*}

That is, quantizing $\bm{b}_j$ at the end of the block clustering step of iteration $n+1$ results in lower quantization MSE compared to quantizing at the end of iteration $n$. Since this is true for all $\bm{b} \in \bm{X}$, we assert the following:
\begin{equation}
\begin{split}
\label{eq:mse_ineq_1}
    \tilde{J}^{(n+1)} &= \frac{1}{N_c} \sum_{i=1}^{N_c} \frac{1}{|\mathcal{B}_{i}^{(n+1)}|}\sum_{\bm{b} \in \mathcal{B}_{i}^{(n+1)}} \frac{1}{L_b}\lVert \bm{b} - C_i^{(n)}(b)\rVert^2_2 \le J^{(n)}
\end{split}
\end{equation}
where $\tilde{J}^{(n+1)}$ is the the quantization MSE after the vector clustering step at iteration $n+1$.

Next, during the codebook update step (\ref{eq:quantizers_update}) at iteration $n+1$, the per-cluster codebooks $\mathcal{C}^{(n)}$ are updated to $\mathcal{C}^{(n+1)}$ by invoking the Lloyd-Max algorithm \citep{Lloyd}. We know that for any given value distribution, the Lloyd-Max algorithm minimizes the quantization MSE. Therefore, for a given vector cluster $\mathcal{B}_i$ we obtain the following result:

\begin{equation}
    \frac{1}{|\mathcal{B}_{i}^{(n+1)}|}\sum_{\bm{b} \in \mathcal{B}_{i}^{(n+1)}} \frac{1}{L_b}\lVert \bm{b}- C_i^{(n+1)}(\bm{b})\rVert^2_2 \le \frac{1}{|\mathcal{B}_{i}^{(n+1)}|}\sum_{\bm{b} \in \mathcal{B}_{i}^{(n+1)}} \frac{1}{L_b}\lVert \bm{b}- C_i^{(n)}(\bm{b})\rVert^2_2
\end{equation}

The above equation states that quantizing the given block cluster $\mathcal{B}_i$ after updating the associated codebook from $C_i^{(n)}$ to $C_i^{(n+1)}$ results in lower quantization MSE. Since this is true for all the block clusters, we derive the following result: 
\begin{equation}
\begin{split}
\label{eq:mse_ineq_2}
     J^{(n+1)} &= \frac{1}{N_c} \sum_{i=1}^{N_c} \frac{1}{|\mathcal{B}_{i}^{(n+1)}|}\sum_{\bm{b} \in \mathcal{B}_{i}^{(n+1)}} \frac{1}{L_b}\lVert \bm{b}- C_i^{(n+1)}(\bm{b})\rVert^2_2  \le \tilde{J}^{(n+1)}   
\end{split}
\end{equation}

Following (\ref{eq:mse_ineq_1}) and (\ref{eq:mse_ineq_2}), we find that the quantization MSE is non-increasing for each iteration, that is, $J^{(1)} \ge J^{(2)} \ge J^{(3)} \ge \ldots \ge J^{(M)}$ where $M$ is the maximum number of iterations. 
%Therefore, we can say that if the algorithm converges, then it must be that it has converged to a local minimum. 
\hfill $\blacksquare$


\begin{figure}
    \begin{center}
    \includegraphics[width=0.5\textwidth]{sections//figures/mse_vs_iter.pdf}
    \end{center}
    \caption{\small NMSE vs iterations during LO-BCQ compared to other block quantization proposals}
    \label{fig:nmse_vs_iter}
\end{figure}

Figure \ref{fig:nmse_vs_iter} shows the empirical convergence of LO-BCQ across several block lengths and number of codebooks. Also, the MSE achieved by LO-BCQ is compared to baselines such as MXFP and VSQ. As shown, LO-BCQ converges to a lower MSE than the baselines. Further, we achieve better convergence for larger number of codebooks ($N_c$) and for a smaller block length ($L_b$), both of which increase the bitwidth of BCQ (see Eq \ref{eq:bitwidth_bcq}).


\subsection{Additional Accuracy Results}
%Table \ref{tab:lobcq_config} lists the various LOBCQ configurations and their corresponding bitwidths.
\begin{table}
\setlength{\tabcolsep}{4.75pt}
\begin{center}
\caption{\label{tab:lobcq_config} Various LO-BCQ configurations and their bitwidths.}
\begin{tabular}{|c||c|c|c|c||c|c||c|} 
\hline
 & \multicolumn{4}{|c||}{$L_b=8$} & \multicolumn{2}{|c||}{$L_b=4$} & $L_b=2$ \\
 \hline
 \backslashbox{$L_A$\kern-1em}{\kern-1em$N_c$} & 2 & 4 & 8 & 16 & 2 & 4 & 2 \\
 \hline
 64 & 4.25 & 4.375 & 4.5 & 4.625 & 4.375 & 4.625 & 4.625\\
 \hline
 32 & 4.375 & 4.5 & 4.625& 4.75 & 4.5 & 4.75 & 4.75 \\
 \hline
 16 & 4.625 & 4.75& 4.875 & 5 & 4.75 & 5 & 5 \\
 \hline
\end{tabular}
\end{center}
\end{table}

%\subsection{Perplexity achieved by various LO-BCQ configurations on Wikitext-103 dataset}

\begin{table} \centering
\begin{tabular}{|c||c|c|c|c||c|c||c|} 
\hline
 $L_b \rightarrow$& \multicolumn{4}{c||}{8} & \multicolumn{2}{c||}{4} & 2\\
 \hline
 \backslashbox{$L_A$\kern-1em}{\kern-1em$N_c$} & 2 & 4 & 8 & 16 & 2 & 4 & 2  \\
 %$N_c \rightarrow$ & 2 & 4 & 8 & 16 & 2 & 4 & 2 \\
 \hline
 \hline
 \multicolumn{8}{c}{GPT3-1.3B (FP32 PPL = 9.98)} \\ 
 \hline
 \hline
 64 & 10.40 & 10.23 & 10.17 & 10.15 &  10.28 & 10.18 & 10.19 \\
 \hline
 32 & 10.25 & 10.20 & 10.15 & 10.12 &  10.23 & 10.17 & 10.17 \\
 \hline
 16 & 10.22 & 10.16 & 10.10 & 10.09 &  10.21 & 10.14 & 10.16 \\
 \hline
  \hline
 \multicolumn{8}{c}{GPT3-8B (FP32 PPL = 7.38)} \\ 
 \hline
 \hline
 64 & 7.61 & 7.52 & 7.48 &  7.47 &  7.55 &  7.49 & 7.50 \\
 \hline
 32 & 7.52 & 7.50 & 7.46 &  7.45 &  7.52 &  7.48 & 7.48  \\
 \hline
 16 & 7.51 & 7.48 & 7.44 &  7.44 &  7.51 &  7.49 & 7.47  \\
 \hline
\end{tabular}
\caption{\label{tab:ppl_gpt3_abalation} Wikitext-103 perplexity across GPT3-1.3B and 8B models.}
\end{table}

\begin{table} \centering
\begin{tabular}{|c||c|c|c|c||} 
\hline
 $L_b \rightarrow$& \multicolumn{4}{c||}{8}\\
 \hline
 \backslashbox{$L_A$\kern-1em}{\kern-1em$N_c$} & 2 & 4 & 8 & 16 \\
 %$N_c \rightarrow$ & 2 & 4 & 8 & 16 & 2 & 4 & 2 \\
 \hline
 \hline
 \multicolumn{5}{|c|}{Llama2-7B (FP32 PPL = 5.06)} \\ 
 \hline
 \hline
 64 & 5.31 & 5.26 & 5.19 & 5.18  \\
 \hline
 32 & 5.23 & 5.25 & 5.18 & 5.15  \\
 \hline
 16 & 5.23 & 5.19 & 5.16 & 5.14  \\
 \hline
 \multicolumn{5}{|c|}{Nemotron4-15B (FP32 PPL = 5.87)} \\ 
 \hline
 \hline
 64  & 6.3 & 6.20 & 6.13 & 6.08  \\
 \hline
 32  & 6.24 & 6.12 & 6.07 & 6.03  \\
 \hline
 16  & 6.12 & 6.14 & 6.04 & 6.02  \\
 \hline
 \multicolumn{5}{|c|}{Nemotron4-340B (FP32 PPL = 3.48)} \\ 
 \hline
 \hline
 64 & 3.67 & 3.62 & 3.60 & 3.59 \\
 \hline
 32 & 3.63 & 3.61 & 3.59 & 3.56 \\
 \hline
 16 & 3.61 & 3.58 & 3.57 & 3.55 \\
 \hline
\end{tabular}
\caption{\label{tab:ppl_llama7B_nemo15B} Wikitext-103 perplexity compared to FP32 baseline in Llama2-7B and Nemotron4-15B, 340B models}
\end{table}

%\subsection{Perplexity achieved by various LO-BCQ configurations on MMLU dataset}


\begin{table} \centering
\begin{tabular}{|c||c|c|c|c||c|c|c|c|} 
\hline
 $L_b \rightarrow$& \multicolumn{4}{c||}{8} & \multicolumn{4}{c||}{8}\\
 \hline
 \backslashbox{$L_A$\kern-1em}{\kern-1em$N_c$} & 2 & 4 & 8 & 16 & 2 & 4 & 8 & 16  \\
 %$N_c \rightarrow$ & 2 & 4 & 8 & 16 & 2 & 4 & 2 \\
 \hline
 \hline
 \multicolumn{5}{|c|}{Llama2-7B (FP32 Accuracy = 45.8\%)} & \multicolumn{4}{|c|}{Llama2-70B (FP32 Accuracy = 69.12\%)} \\ 
 \hline
 \hline
 64 & 43.9 & 43.4 & 43.9 & 44.9 & 68.07 & 68.27 & 68.17 & 68.75 \\
 \hline
 32 & 44.5 & 43.8 & 44.9 & 44.5 & 68.37 & 68.51 & 68.35 & 68.27  \\
 \hline
 16 & 43.9 & 42.7 & 44.9 & 45 & 68.12 & 68.77 & 68.31 & 68.59  \\
 \hline
 \hline
 \multicolumn{5}{|c|}{GPT3-22B (FP32 Accuracy = 38.75\%)} & \multicolumn{4}{|c|}{Nemotron4-15B (FP32 Accuracy = 64.3\%)} \\ 
 \hline
 \hline
 64 & 36.71 & 38.85 & 38.13 & 38.92 & 63.17 & 62.36 & 63.72 & 64.09 \\
 \hline
 32 & 37.95 & 38.69 & 39.45 & 38.34 & 64.05 & 62.30 & 63.8 & 64.33  \\
 \hline
 16 & 38.88 & 38.80 & 38.31 & 38.92 & 63.22 & 63.51 & 63.93 & 64.43  \\
 \hline
\end{tabular}
\caption{\label{tab:mmlu_abalation} Accuracy on MMLU dataset across GPT3-22B, Llama2-7B, 70B and Nemotron4-15B models.}
\end{table}


%\subsection{Perplexity achieved by various LO-BCQ configurations on LM evaluation harness}

\begin{table} \centering
\begin{tabular}{|c||c|c|c|c||c|c|c|c|} 
\hline
 $L_b \rightarrow$& \multicolumn{4}{c||}{8} & \multicolumn{4}{c||}{8}\\
 \hline
 \backslashbox{$L_A$\kern-1em}{\kern-1em$N_c$} & 2 & 4 & 8 & 16 & 2 & 4 & 8 & 16  \\
 %$N_c \rightarrow$ & 2 & 4 & 8 & 16 & 2 & 4 & 2 \\
 \hline
 \hline
 \multicolumn{5}{|c|}{Race (FP32 Accuracy = 37.51\%)} & \multicolumn{4}{|c|}{Boolq (FP32 Accuracy = 64.62\%)} \\ 
 \hline
 \hline
 64 & 36.94 & 37.13 & 36.27 & 37.13 & 63.73 & 62.26 & 63.49 & 63.36 \\
 \hline
 32 & 37.03 & 36.36 & 36.08 & 37.03 & 62.54 & 63.51 & 63.49 & 63.55  \\
 \hline
 16 & 37.03 & 37.03 & 36.46 & 37.03 & 61.1 & 63.79 & 63.58 & 63.33  \\
 \hline
 \hline
 \multicolumn{5}{|c|}{Winogrande (FP32 Accuracy = 58.01\%)} & \multicolumn{4}{|c|}{Piqa (FP32 Accuracy = 74.21\%)} \\ 
 \hline
 \hline
 64 & 58.17 & 57.22 & 57.85 & 58.33 & 73.01 & 73.07 & 73.07 & 72.80 \\
 \hline
 32 & 59.12 & 58.09 & 57.85 & 58.41 & 73.01 & 73.94 & 72.74 & 73.18  \\
 \hline
 16 & 57.93 & 58.88 & 57.93 & 58.56 & 73.94 & 72.80 & 73.01 & 73.94  \\
 \hline
\end{tabular}
\caption{\label{tab:mmlu_abalation} Accuracy on LM evaluation harness tasks on GPT3-1.3B model.}
\end{table}

\begin{table} \centering
\begin{tabular}{|c||c|c|c|c||c|c|c|c|} 
\hline
 $L_b \rightarrow$& \multicolumn{4}{c||}{8} & \multicolumn{4}{c||}{8}\\
 \hline
 \backslashbox{$L_A$\kern-1em}{\kern-1em$N_c$} & 2 & 4 & 8 & 16 & 2 & 4 & 8 & 16  \\
 %$N_c \rightarrow$ & 2 & 4 & 8 & 16 & 2 & 4 & 2 \\
 \hline
 \hline
 \multicolumn{5}{|c|}{Race (FP32 Accuracy = 41.34\%)} & \multicolumn{4}{|c|}{Boolq (FP32 Accuracy = 68.32\%)} \\ 
 \hline
 \hline
 64 & 40.48 & 40.10 & 39.43 & 39.90 & 69.20 & 68.41 & 69.45 & 68.56 \\
 \hline
 32 & 39.52 & 39.52 & 40.77 & 39.62 & 68.32 & 67.43 & 68.17 & 69.30  \\
 \hline
 16 & 39.81 & 39.71 & 39.90 & 40.38 & 68.10 & 66.33 & 69.51 & 69.42  \\
 \hline
 \hline
 \multicolumn{5}{|c|}{Winogrande (FP32 Accuracy = 67.88\%)} & \multicolumn{4}{|c|}{Piqa (FP32 Accuracy = 78.78\%)} \\ 
 \hline
 \hline
 64 & 66.85 & 66.61 & 67.72 & 67.88 & 77.31 & 77.42 & 77.75 & 77.64 \\
 \hline
 32 & 67.25 & 67.72 & 67.72 & 67.00 & 77.31 & 77.04 & 77.80 & 77.37  \\
 \hline
 16 & 68.11 & 68.90 & 67.88 & 67.48 & 77.37 & 78.13 & 78.13 & 77.69  \\
 \hline
\end{tabular}
\caption{\label{tab:mmlu_abalation} Accuracy on LM evaluation harness tasks on GPT3-8B model.}
\end{table}

\begin{table} \centering
\begin{tabular}{|c||c|c|c|c||c|c|c|c|} 
\hline
 $L_b \rightarrow$& \multicolumn{4}{c||}{8} & \multicolumn{4}{c||}{8}\\
 \hline
 \backslashbox{$L_A$\kern-1em}{\kern-1em$N_c$} & 2 & 4 & 8 & 16 & 2 & 4 & 8 & 16  \\
 %$N_c \rightarrow$ & 2 & 4 & 8 & 16 & 2 & 4 & 2 \\
 \hline
 \hline
 \multicolumn{5}{|c|}{Race (FP32 Accuracy = 40.67\%)} & \multicolumn{4}{|c|}{Boolq (FP32 Accuracy = 76.54\%)} \\ 
 \hline
 \hline
 64 & 40.48 & 40.10 & 39.43 & 39.90 & 75.41 & 75.11 & 77.09 & 75.66 \\
 \hline
 32 & 39.52 & 39.52 & 40.77 & 39.62 & 76.02 & 76.02 & 75.96 & 75.35  \\
 \hline
 16 & 39.81 & 39.71 & 39.90 & 40.38 & 75.05 & 73.82 & 75.72 & 76.09  \\
 \hline
 \hline
 \multicolumn{5}{|c|}{Winogrande (FP32 Accuracy = 70.64\%)} & \multicolumn{4}{|c|}{Piqa (FP32 Accuracy = 79.16\%)} \\ 
 \hline
 \hline
 64 & 69.14 & 70.17 & 70.17 & 70.56 & 78.24 & 79.00 & 78.62 & 78.73 \\
 \hline
 32 & 70.96 & 69.69 & 71.27 & 69.30 & 78.56 & 79.49 & 79.16 & 78.89  \\
 \hline
 16 & 71.03 & 69.53 & 69.69 & 70.40 & 78.13 & 79.16 & 79.00 & 79.00  \\
 \hline
\end{tabular}
\caption{\label{tab:mmlu_abalation} Accuracy on LM evaluation harness tasks on GPT3-22B model.}
\end{table}

\begin{table} \centering
\begin{tabular}{|c||c|c|c|c||c|c|c|c|} 
\hline
 $L_b \rightarrow$& \multicolumn{4}{c||}{8} & \multicolumn{4}{c||}{8}\\
 \hline
 \backslashbox{$L_A$\kern-1em}{\kern-1em$N_c$} & 2 & 4 & 8 & 16 & 2 & 4 & 8 & 16  \\
 %$N_c \rightarrow$ & 2 & 4 & 8 & 16 & 2 & 4 & 2 \\
 \hline
 \hline
 \multicolumn{5}{|c|}{Race (FP32 Accuracy = 44.4\%)} & \multicolumn{4}{|c|}{Boolq (FP32 Accuracy = 79.29\%)} \\ 
 \hline
 \hline
 64 & 42.49 & 42.51 & 42.58 & 43.45 & 77.58 & 77.37 & 77.43 & 78.1 \\
 \hline
 32 & 43.35 & 42.49 & 43.64 & 43.73 & 77.86 & 75.32 & 77.28 & 77.86  \\
 \hline
 16 & 44.21 & 44.21 & 43.64 & 42.97 & 78.65 & 77 & 76.94 & 77.98  \\
 \hline
 \hline
 \multicolumn{5}{|c|}{Winogrande (FP32 Accuracy = 69.38\%)} & \multicolumn{4}{|c|}{Piqa (FP32 Accuracy = 78.07\%)} \\ 
 \hline
 \hline
 64 & 68.9 & 68.43 & 69.77 & 68.19 & 77.09 & 76.82 & 77.09 & 77.86 \\
 \hline
 32 & 69.38 & 68.51 & 68.82 & 68.90 & 78.07 & 76.71 & 78.07 & 77.86  \\
 \hline
 16 & 69.53 & 67.09 & 69.38 & 68.90 & 77.37 & 77.8 & 77.91 & 77.69  \\
 \hline
\end{tabular}
\caption{\label{tab:mmlu_abalation} Accuracy on LM evaluation harness tasks on Llama2-7B model.}
\end{table}

\begin{table} \centering
\begin{tabular}{|c||c|c|c|c||c|c|c|c|} 
\hline
 $L_b \rightarrow$& \multicolumn{4}{c||}{8} & \multicolumn{4}{c||}{8}\\
 \hline
 \backslashbox{$L_A$\kern-1em}{\kern-1em$N_c$} & 2 & 4 & 8 & 16 & 2 & 4 & 8 & 16  \\
 %$N_c \rightarrow$ & 2 & 4 & 8 & 16 & 2 & 4 & 2 \\
 \hline
 \hline
 \multicolumn{5}{|c|}{Race (FP32 Accuracy = 48.8\%)} & \multicolumn{4}{|c|}{Boolq (FP32 Accuracy = 85.23\%)} \\ 
 \hline
 \hline
 64 & 49.00 & 49.00 & 49.28 & 48.71 & 82.82 & 84.28 & 84.03 & 84.25 \\
 \hline
 32 & 49.57 & 48.52 & 48.33 & 49.28 & 83.85 & 84.46 & 84.31 & 84.93  \\
 \hline
 16 & 49.85 & 49.09 & 49.28 & 48.99 & 85.11 & 84.46 & 84.61 & 83.94  \\
 \hline
 \hline
 \multicolumn{5}{|c|}{Winogrande (FP32 Accuracy = 79.95\%)} & \multicolumn{4}{|c|}{Piqa (FP32 Accuracy = 81.56\%)} \\ 
 \hline
 \hline
 64 & 78.77 & 78.45 & 78.37 & 79.16 & 81.45 & 80.69 & 81.45 & 81.5 \\
 \hline
 32 & 78.45 & 79.01 & 78.69 & 80.66 & 81.56 & 80.58 & 81.18 & 81.34  \\
 \hline
 16 & 79.95 & 79.56 & 79.79 & 79.72 & 81.28 & 81.66 & 81.28 & 80.96  \\
 \hline
\end{tabular}
\caption{\label{tab:mmlu_abalation} Accuracy on LM evaluation harness tasks on Llama2-70B model.}
\end{table}

%\section{MSE Studies}
%\textcolor{red}{TODO}


\subsection{Number Formats and Quantization Method}
\label{subsec:numFormats_quantMethod}
\subsubsection{Integer Format}
An $n$-bit signed integer (INT) is typically represented with a 2s-complement format \citep{yao2022zeroquant,xiao2023smoothquant,dai2021vsq}, where the most significant bit denotes the sign.

\subsubsection{Floating Point Format}
An $n$-bit signed floating point (FP) number $x$ comprises of a 1-bit sign ($x_{\mathrm{sign}}$), $B_m$-bit mantissa ($x_{\mathrm{mant}}$) and $B_e$-bit exponent ($x_{\mathrm{exp}}$) such that $B_m+B_e=n-1$. The associated constant exponent bias ($E_{\mathrm{bias}}$) is computed as $(2^{{B_e}-1}-1)$. We denote this format as $E_{B_e}M_{B_m}$.  

\subsubsection{Quantization Scheme}
\label{subsec:quant_method}
A quantization scheme dictates how a given unquantized tensor is converted to its quantized representation. We consider FP formats for the purpose of illustration. Given an unquantized tensor $\bm{X}$ and an FP format $E_{B_e}M_{B_m}$, we first, we compute the quantization scale factor $s_X$ that maps the maximum absolute value of $\bm{X}$ to the maximum quantization level of the $E_{B_e}M_{B_m}$ format as follows:
\begin{align}
\label{eq:sf}
    s_X = \frac{\mathrm{max}(|\bm{X}|)}{\mathrm{max}(E_{B_e}M_{B_m})}
\end{align}
In the above equation, $|\cdot|$ denotes the absolute value function.

Next, we scale $\bm{X}$ by $s_X$ and quantize it to $\hat{\bm{X}}$ by rounding it to the nearest quantization level of $E_{B_e}M_{B_m}$ as:

\begin{align}
\label{eq:tensor_quant}
    \hat{\bm{X}} = \text{round-to-nearest}\left(\frac{\bm{X}}{s_X}, E_{B_e}M_{B_m}\right)
\end{align}

We perform dynamic max-scaled quantization \citep{wu2020integer}, where the scale factor $s$ for activations is dynamically computed during runtime.

\subsection{Vector Scaled Quantization}
\begin{wrapfigure}{r}{0.35\linewidth}
  \centering
  \includegraphics[width=\linewidth]{sections/figures/vsquant.jpg}
  \caption{\small Vectorwise decomposition for per-vector scaled quantization (VSQ \citep{dai2021vsq}).}
  \label{fig:vsquant}
\end{wrapfigure}
During VSQ \citep{dai2021vsq}, the operand tensors are decomposed into 1D vectors in a hardware friendly manner as shown in Figure \ref{fig:vsquant}. Since the decomposed tensors are used as operands in matrix multiplications during inference, it is beneficial to perform this decomposition along the reduction dimension of the multiplication. The vectorwise quantization is performed similar to tensorwise quantization described in Equations \ref{eq:sf} and \ref{eq:tensor_quant}, where a scale factor $s_v$ is required for each vector $\bm{v}$ that maps the maximum absolute value of that vector to the maximum quantization level. While smaller vector lengths can lead to larger accuracy gains, the associated memory and computational overheads due to the per-vector scale factors increases. To alleviate these overheads, VSQ \citep{dai2021vsq} proposed a second level quantization of the per-vector scale factors to unsigned integers, while MX \citep{rouhani2023shared} quantizes them to integer powers of 2 (denoted as $2^{INT}$).

\subsubsection{MX Format}
The MX format proposed in \citep{rouhani2023microscaling} introduces the concept of sub-block shifting. For every two scalar elements of $b$-bits each, there is a shared exponent bit. The value of this exponent bit is determined through an empirical analysis that targets minimizing quantization MSE. We note that the FP format $E_{1}M_{b}$ is strictly better than MX from an accuracy perspective since it allocates a dedicated exponent bit to each scalar as opposed to sharing it across two scalars. Therefore, we conservatively bound the accuracy of a $b+2$-bit signed MX format with that of a $E_{1}M_{b}$ format in our comparisons. For instance, we use E1M2 format as a proxy for MX4.

\begin{figure}
    \centering
    \includegraphics[width=1\linewidth]{sections//figures/BlockFormats.pdf}
    \caption{\small Comparing LO-BCQ to MX format.}
    \label{fig:block_formats}
\end{figure}

Figure \ref{fig:block_formats} compares our $4$-bit LO-BCQ block format to MX \citep{rouhani2023microscaling}. As shown, both LO-BCQ and MX decompose a given operand tensor into block arrays and each block array into blocks. Similar to MX, we find that per-block quantization ($L_b < L_A$) leads to better accuracy due to increased flexibility. While MX achieves this through per-block $1$-bit micro-scales, we associate a dedicated codebook to each block through a per-block codebook selector. Further, MX quantizes the per-block array scale-factor to E8M0 format without per-tensor scaling. In contrast during LO-BCQ, we find that per-tensor scaling combined with quantization of per-block array scale-factor to E4M3 format results in superior inference accuracy across models. 


\end{document}

