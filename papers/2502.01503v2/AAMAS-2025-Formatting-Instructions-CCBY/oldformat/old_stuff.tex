

% Different methods have been used to tackle this problem including using external equipment in combination with domain expertise~\cite{domain-expertise}, and deep learning methods using Synthetic Aperture Radar Imagery~\cite{satellite, sar2}. Another approach, the Dynamic Spatial-Temporal Refinement Network (DSTNet)~\cite{dstnet}, has been proposed to improve trajectory prediction accuracy by capturing both local and global spatio-temporal dynamics. However, these approaches face challenges such as high operational costs and the dependency on radar imagery can be affected by environmental conditions like cloud cover and sea state. 
% This work uses an abductive framework by generating geospatial regions to detect dark vessels given the vessel's historical trajectory data.

% \begin{itemize}
%     \item current problems
%     \items how we address it
%     \item difference w iclp
%     \item key contributions
%     \item outline of ppr
% \end{itemize}
% They analyze the vessel's normal behavior and alert when it deviates in real time using satellite constellations and Doppler geolocation validation. 
% The study of maritime vessel behavior based on trajectory information is not new and well known - with early efforts such as DARPA PANDArepresenting early advances as well. They used pattern-based learning techniques.

% In this paper, we solve the problem of detecting dark vessels with abductive inference by generating regions. Some resources are used to search these regions at a time horizon and the dark vessel is expected to be found in them. In practice, we use regions as bounding boxes. Here, the normal vessel behavior is represented in a logic program, and when the vessel goes dark, we abduce for regions to detect it. Further discussed in Section~\ref{sec:abduc}.
% Most applications require explainability so that expert analysts can further asses agent behavior, which these approaches do not give. Some other notable approaches use deep learning architectures based on convolutional networks~\cite{tf}, adversarial methods~\cite{tp}, and autoencoders~\cite{tf2} to forecast trajectory up to a minute based on vision-oriented human data. In contrast, we focus on prediction horizons in terms of hours.
%  Despite deep learning's success in near-term prediction, computational intensity that goes hand-in-hand with dependence on large datasets limits their practical application in real-time and resource-constrained scenarios.  \\
 
 % Markov-based methods use Markov chains to forecast agent turn movements~\cite{markov1}, and immediate location~\cite{markov2}, based on trajectory data of location sequences. H
 
 % These challenges underscore the need for more efficient and scalable solutions in trajectory prediction tasks.
 
% Thus, our work aims to expand on these approaches by using these identified patterns to anticipate and infer potential future anomalies, such as vessels going dark, enabling real-time decision-making in critical maritime operations.
% \noindent\textit{Regions}

% higher recall and longer time-horizon vessel prediction than machine learning methods. use abductive inference to generate regions to detect vessels at a time horizon. The search area is area-optimized with higher overall performance over the baselines. Our approach performs well with longer time horizons as well as in a reasonable computing time. It is also efficient with smaller region sizes and with training data sizes. Further, our approach offers explainability and we provide our deployment architecture with a live feed of data.
% This work can be extended by leveraging environmental knowledge in the logic program, which has a significant role in the maritime domain. %The ablation studies presented in the paper suggest that we can leverage the work mentioned in~\cite{bavikadi2024geospatial} to utilize scalability efficiencies.
% Utilize scalability efficiencies mentioned in~\cite{bavikadi2024geospatial}. %d how the ablation study suggests we can leverage ideas from the ICLP paper in future work.
% Reducing region size to SM based on our method of identifying regions turns out to give comparable performance to DL with LG. We think it is because of the effectiveness of how we generate these regions, however, ABD-SM outperforms in particular when the region size is reduced, ABD-SM performs better than RND-SM with the same region size. 
%The Abduction model seems to be consistent with its precision as seen in figure~\ref{fig:longterm} and can carry out long-term region generation. On the other hand, the DL model saturates after the first timestep.% The goal is to generate regions where the model predicts the vessel's location at a given time $t$. We compare the abduction method (Abduction) against two other baselines including a sequence-to-sequence deep learning baseline (DL), and a random region generator (Random) as seen in Figure~\ref{fig:base} based on different settings of $k$. 
%However, longer time horizons are more important for our target application - making but for this domain, ABD is more appropriate as it works well for the given time horizon (even when its not closer to the current time) and beyond.

 
% Both DL, Abduction methods consistently perform better than the random generator as seen by the PR curves in Figure~\ref{fig:base_pr}. As the value of $k$ gets close to the maximum true positives achievable, $F1@k$ for DL gets close to that of random depicted by the F1 scores plot in Figure~\ref{fig:base_f1}. In general, the abduction method shows a higher F1 with an average of 50$\% $ increase over DL. Further ML metrics like precision, and recall can be found in the appendix.
% ($|\tau^n| \leq r$, where $\tau^n$ is the last known vessel location and $r$ is the maximum distance the vessel can cover in time $t$).
% Table~\ref{tab:area} shows the reduction in area for the abduction model when comparing the recall for the different region sizes. As the region size is decreased by $80\%$, the precision spiked by $0.12$ and recall by $0.07$, depicting that the model still works for smaller regions. In most cases, the prediction area is reduced by $75\%$ on average with less than or equal to $k$ while maintaining recall. Reduced region size is practical for the envisioned application due to its reduced detection area for reasonable performance. 
% Consider an agent region set program $\Pi'$ formed with an agent $agt$ and time $t$. To measure the normalcy of an explanation, we use the parsimony function $f$ defined as follows:

% Here, $L$ is the lower bound of the annotation on the atom $normal(agt)$ at $t$ in the minimal model of $\Gamma^*(\Pi \cup \Pi')$.

%\textit{Creation of Logic Program via Rule Learning/ Learning logic program based on estimated probability of movements.}  
% based on sources and destinations as well as traffic flow characteristics
% Following the previous works~\cite{groupwaterarea, groupshipwdestination} where similar vessels were considered based on an area with specific traffic flow characteristics or grouped based on source and destinations to form the dataset, and we too group the vessels. Considering that our AOI bounds might not always cover the vessel destinations and to avoid learning rules that might depict misrepresented behavior (for example, a government vessel will act differently from a commercial vessel), we cluster the dataset
% into 9 sub datasets using the DBSCAN algorithm~\cite{dbscan} while ensuring that each sub dataset contains $\geq 2$ trajectories. Trajectory clustering is widely used in mining patterns ~\cite{predwithcluster}, however, our focus is to leverage it to form strong rules.
%Maritime dark vessel detection typically uses external assets (like satellite imagery) to verify vessel location at time $t$ in the predicted regions. A minimal prediction area is preferred to achieve an efficient cost with respect to asset utility. Additionally, it is important to mitigate beyond false positives and to keep searching for the vessel when it's not found.
%This is one of the reasons the abduction method achieved a higher area under the curve in the Precision-Recall plot in Figure~\ref{fig:base_pr}. Furthermore, as the predicted region area increases, Abduction reaches a recall of $0.98$ for $38 km^2$ while the DL method levels out at a $0.57$ recall as seen in Figure~\ref{fig:recall_area}.
%Also, we find that the ratio of recall@k with the corresponding area of the generated regions increases with $k$ while there is no further benefit for the other baselines to keep searching for the vessel as depicted by the ratio plots in Figure~\ref{fig:recall_per_area_k}.
%  we computed the top-k at(agt,\regionconst) (picking \regionconst from D_\regionconst) with f-hat2 in parallel?

% We note that there could be an infinite number of explanations, so as normal with abduction, we will use a parsimony requirement to differentiate among them. We define a parsimony function $f$ as any function that takes a program, an explanation as arguments and returns a positive real - e.g., $f(\Pi, \Pi')$.   
% Further, we may have some a-priori set of bounding boxes, $\domSet_{\regionconst}'\subseteq \domSet_{\regionconst}$ from which we select.
% This is a combinatorial problem but we restrict $\Pi'$ to be a single TAF formed with $\domSet_{\regionconst}'$. We find that querying for top $k$ explanations using the parsimony requirement covered the agent at time $t$ and we show that empirically in Section~\ref{sec:exp}.
% However, we also note that in practice, it is preferable to identify a search area to look for where a vessel may travel. To do so, we utilize region set program $\Pi_{\regionset}$ mentioned in example~\ref{ex:regionset} as $\Pi'$, an argument to $\parsimony$.
%For example, consider the following:
% of interest, a given logic program $\Pi$ will contain initial trajectory information for that agent (e.g., for a given $\Pi$ and $agt$ there is some $\tau_{agt}$ such that $\Gamma^*(\Pi)\models \tau_{agt}$).  However, such a trajectory will only be partial, and our key problem is to determine locations where the remainder of the agent's trip lies.  Hence, in our application, we envision an abduction problem where we are given a logic program representing normal agent behavior (e.g., typical shipping routes) that contains initial location information, an observed last known location $last(\tau_{agt})$ and abduces an explanation $E$ (a region set). Note that $\Pi \bigcup E$ is consistent by a trajectory $\tau_{agt} \bigcup e$ where $e \in E$.
% a trajectory that ends in a landmark.

% \begin{definition}[Trajectory Abduction Problem]
% Given program $\Pi$, agent $agt$, time $t$, we say program $\Pi'$ is an explanation for $\Pi,agt,t$ if there exists $\tau_{agt}$ such that $\Gamma^*(\Pi\cup \Pi') \models \tau_{agt}$.
% % and $\Gamma^*(\Pi\cup \Pi') \models_t \lmPred(last(\tau_{agt}))$.
% \end{definition}

% Relevant to our application, we also define a notion of agent travel time, which simply tells us if an agent can travel between two points in a given amount of time.  
% Given agent $agt$ and locations $loc,loc'$, $\travelTime_{agt}(loc,loc')$ returns the amount of time required for the agent to travel from $loc$ to $loc'$.  With this in mind, we can define an agent trajectory as follows.

% \begin{definition}[Trajectory] Sequence of tuples $\tau_{agt} = \langle (loc_1,t_1),$ $\ldots,(loc_i,t_i),\ldots,(loc_n,t_n)\rangle$ is  a ``trajectory'' when $agt \in \domSet_{agt}$, each $loc_i \in \domainSet_{loc}$, each $t_i \in T$ (where for all $i<j$, $t_i<t_j$), and for all $i,j$ pairs where $i<j$, we have $\travelTime_{agt}(loc_i,loc_j)\leq t_j-t_i$. An interpretation $I \models \tau$ if for all $i \in \{1,\ldots,n\}$, there exists $\regionconst_i \in \domSet_{\regionconst}$ such that $loc_i \in \regionconst_i$  and $I \models_{t_i} \atPred(agt,\regionconst_i)$.  We will use the notation $last(\tau_{agt})=loc_n$.%Given trajectories $\tau_{agt},\tau_{agt}'$ we say $\tau_{agt} \subseteq \tau_{agt}'$ if all tuples in $\tau_{agt}$ are also in $\tau_{agt}'$.
% \end{definition}

% Also relevant to our application is a special type of logic program that we will call an \textit{agent region set} that for some agent $agt$ consists of temporally annotated formulas consisting of single atoms formed with $\atPred$ and $agt$.  Intuitively, this states that an agent may be at a location specified by the bounding box (second argument) of $\atPred$.

% \begin{example}[Region set program]
% \label{ex:regionset}
%  Consider an agent $agt \in \domainSet_{agt}$, a last known agent location $\tau_{O} = \langle (loc_n, t_n) \rangle$,  a time $t_{n+1}$, a set of regions $\regionset \subseteq \domainSet_{\regionconst}$, then a region set program $\program_{\regionset} = \{ \atPred(agt,\regionconst_i)|  \regionconst_i \in \regionset   \}$. Note that  $|\program_{\regionset}| = |\regionset|$. Here, $\regionset$ is known a prior such that $\forall \regionconst \in \regionset$, every $loc_i \in \regionconst, \travelTime_{agt}(loc_n,loc_i)\leq t_{n+1} - t_n$. 
%  % is formed with TAFs of $\atPred(agt,\regionconst)$ for every $\regionconst \in \regionset$.
% \end{example}
\subsection{Algorithm[Under Progress]}

We use the Algorithm~\ref{alg:abd} to generate top $k$ regions (i.e. implement a top-$k$ version of $\hat{f_{2}}$). We leverage a graphical structure supported by our underlying annotated logic implementation (PyReason~\cite{aditya2023pyreason}). Here,$1,2$ are current and the next timesteps from $i-1$ to $i$, and $val$ is a form of the parsimony function $\parsimony$ that returns the confidence value of a candidate region for a program $\program$. Given that $\parsimony$ is computed in polynomial time in terms of the size of $\regionset$, any query with $k=1$ is polynomial time by the algorithm. 
% The time complexity of the algorithm is $-$, and it turns out to be efficient.

\begin{algorithm}[t] 
\caption{\textsf{Abduction}}
\label{alg:abd}
% \begin{small}
\begin{algorithmic}[1]
    \Statex { \textbf{Input:} Number of regions $k$, logical program $\Pi$, region set $\regionset$, current region $\regionset_{i-1}$, predicate $normal$, constant $agt \in \domainSet_{agt}$}
    \Statex{ \textbf{Output:} Top $k$ regions $\regionset'$ }

    \Function{$val$}{$\program, \regionset_{i-1}, \regionset_i,  normal$}
    \State($\Pi'  \gets (\regionset_{i-1}, 1,\regionset_i, 2)$)
    % \State($\Pi' \gets \tau_{sub}$)
       \State( \textit{weight} \gets  $\Gamma_{\Pi \cup \Pi'}(normal(agt))(2) $)
       \State{\textbf{return} $\textit{weight}$}
    \EndFunction
    \State{ $\regionset' \gets \emptyset$}
    \State{  $\graph \gets (\{\regionset_{i-1},\forall \regionset_i \in \regionset\},\emptyset)$ }
    % \For{ \textit{each} $\regionset_i \in \regionset$ }
    
    % \EndFor
    
    \State{\textbf{parallel for} each $\regionset_i \in \regionset$ \textbf{do}}

            \State{\ \  \ \ $\textit{weight} \gets val(\program, \regionset_{i-1}, \regionset_i,  normal)$}
            \State{\ \  \ \ \textit{addEdge}$( \regionset_{i-1},\regionset_i, \textit{weight} )$}

    \State{\textbf{end for}}
    \State{$\regionset'$ = \textit{Sort}$(\forall E_i \in \graph[\textit{Edges}][\textit{weight}])$ }
    \State{\textbf{return} $\regionset'[:k]$}
    % \return{}
    
    
\end{algorithmic}
% \end{small}
\end{algorithm}


\begin{theorem}
    Algorithm~\ref{alg:abd} returns the top $k$ bounding boxes $bb \in \domainSet_{bb}$ when ranked by $\sigma_{t}(agt,\Pi_{init}\cup\Pi_{behav}\cup\{\atPred(agt,bb)_t
\end{theorem}

Under progress-DB

\noindent\textit{Proof of correctness:}
\begin{itemize}
    \item \textbf{Initialization:} Assume the size of a region set $\regionset$ to be $n$, where $k \leq n$. Also, assume a region set $\regionset' \subseteq \regionset$ of the size $k$. Prior to the start of the loop, $\regionset', \graph[Edges]$ is an empty set.
    \item \textbf{Correctness:} $\regionset'$ has exactly $k$ elements and are sorted in the order of highest weight. The weight corresponds to $\parsimony(\program,\program_{\regionset_i})$, given that $\program_{\regionset_i}$ is a agent region set program formed with TAFs by $\regionset_i \in \domainSet_{bb}$
    \item \textbf{Base case:} For $i=1$, $\regionset'$ has the element with the highest weight (lower bound on the annotation of the minimal model). $\Gamma*(\program)$ can be computed in a finite number of applications of the fixpoint operator. It can also be proved that the output of the fixpoint operator after convergence is the least fixed point that is indeed the minimal model on $\program$.

\end{itemize}

% \noindent\textit{Time complexity}
% $O( |\regionset|\log(|\regionset|))$


% \begin{definition}[Feasibility]
% Given agent $a$ and positive real $\delta t$, function $\feasibility_{a,\delta t}: \domainSet_{loc} \times \domainSet_{loc} \rightarrow \{0,1\}$ is a feasibility function  where  $\feasibility_{a,\delta t}(loc_{start}, loc_{tgt})=1$ if agent $a$ can travel from $loc_{start}$ to $loc_{tgt}$ within $\delta t$ time units and $0$ otherwise.
% %$\feasibility(loc_{start}, loc_{tgt})$ gives either $1$ if for speed $\exists s \in \mathbb{R}^+$, an agent can travel to $loc_{tgt}$ in $\delta t$
% \end{definition}


% \begin{definition}[Trajectory]A trajectory $\tau$ is a sequence of atoms formed with an agent $\{ agent(c_1)_{t_1},..,agent(c_n)_{t_n} \}$, where $c_i \in \domainSet_{loc}$. An interpretation $\interpretation$ is said to satisfy $\tau$, denoted by $\interpretation \models \tau$, if for all $i \in \{1,\ldots,n\}$, $\interpretation \models_{t_i} \atPred(agent,c_i)$ and $\interpretation \models_{t_n} \lmPred(c_n)$.
% \end{definition}

% We further divided the trajectory into its prefix and suffix to represent the dark vessel's known and unknown parts of it's trajectory.
% A prefix of a trajectory, denoted by $\tau_{pre}$, for $m \leq n$ is the first $m$ points in $\tau$. A suffix of a trajectory, denoted by $\tau_{suf}$, for $m \leq n$ is the last $n-m$ points in $\tau$.  Intuitively, a prefix(resp. suffix) is any sequence of atoms formed with $\atPred$ of has a corresponding suffix
We formalize abduction as- Given a logic program $\program$ (embedded with domain knowledge), a set of observations $O$ (that are $TAFs$), and a set of possible hypothesis $H$ (that are annotated formulae), find an explanation for $O,$
$E \subset H$ such that $\program \bigcup E$ is consistent and $\program \bigcup E \models O$. We use $value$ as the ranking criteria to obtain optimal explanations.

Our goal is to generate a region, hence we define it as:
\begin{definition}[Region]
    A region is a subset of an AOI, $|R| \leq k$, here $k \in \mathbb{N}$ with a bounding box $bb={c_{min}, c_{max}}$. $\exists \tau' \subseteq \tau, \forall i \in \tau', \inPred(i,bb)$ is a ground atom
\end{definition}

\begin{definition}[Parsimony function]
Given a logic program $\program$, an explanation $\Pi'$, an aggregate function $f$, a predicate $pred$, and a constant $a$, we define $value(\program, \tau_{suf}, pred) = f(\Gamma^*(\Pi)(pred(a)))$. Here, $\Gamma^*(\Pi)$ is the minimal model of $\program$ and $\interpretation \models \tau_{suf}$.
    % An associated aggregate function $value(\program, \tau_{suf}): A$ over a subset of annotations.
\end{definition}

% \begin{definition}[Hypothesis. $H$]
%     A set of explanations. Given a vessel's current location, each explanation is a region that the vessel would normally go to based on historical data. Explanations can overlap.
% \end{definition}
This is a combinatorial problem but we restrict $\Pi'$ to be a single TAF formed with $\domSet_{bb}'$. We find that querying for top $k$ explanations using the parsimony requirement covered the agent at time $t$ and we show that empirically in Section~\ref{sec:exp}.

\begin{itemize}
    \item  
    
    INPUT: Given a vessel $v$, it's prefix trajectory $\tau_{pre}$,  a candidate set of regions $\regionset$, time $t$, a natural number $k$, a logic program $\program$ and a function $f$.

    \item 
    OUTPUT: $\regionset' \subseteq \regionset : | \regionset' | \leq k$.  $\forall \regionset'_i \in \regionset' \exists \tau_{suf} \in \regionset'$   s.t $\tau_{suf} \bigcup \tau_{pre} \models_t \program$.
    \item $\forall \regionset'_i \in \regionset', i \leq k$, $f(\program, \tau_{suf}^i, pred) \geq f(\program, \tau_{suf}^j, pred) $, $i < j, j \leq |\regionset|$ .

    \item $\forall i \in \regionset', \feasibility(c_m,i) = 1$

\end{itemize}

% Given $O$- A tuple of $\langle L, t \rangle$. Here,$L$ is the last seen location $\langle t_i, lat_i, lon_i \rangle$ of the agent, and time $t$ is the current time at which the agent's location is unknown, we abduce an explanation that is a region where the vessel can be found at $t$ from $H$- that is a set of all feasible regions. Note that explanations can overlap.

\begin{example}[Abduction]
\label{ex:abd}
\textit{ Consider $O$ which is a TAF, we abduce for an explanation (in our case, a single TAF) from a set of TAFs $H$. Here, each TAF is a $\tau_i$ of the size $2$.

}
\end{example}


% \begin{table}[]
% \caption{Comparison of the random baseline with abduction}
% \label{tab1}

% \begin{tabular}{|l|r|r|r|r|r|}
% \hline
%           & \multicolumn{1}{c|}{\textbf{k}} & \multicolumn{1}{c|}{\textbf{Prec@k}} & \multicolumn{1}{c|}{\textbf{Rec@k}} & \multicolumn{1}{c|}{\textbf{Rec-p@k}} & \multicolumn{1}{l|}{\textbf{F1}} \\ \hline
% Random    & 1                               & 0.03                                 & 0.00                                & 0.00                                  & 0.00                             \\ \cline{2-6} 
%           & 3                               & 0.04                                 & 0.01                                & 0.01                                  & 0.01                             \\ \cline{2-6} 
%           & 5                               & 0.03                                 & 0.00                                & 0.01                                  & 0.01                             \\ \cline{2-6} 
%           & 10                              & 0.05                                 & 0.02                                & 0.02                                  & 0.02                             \\ \cline{2-6} 
%           & 15                              & 0.02                                 & 0.01                                & 0.01                                  & 0.01                             \\ \cline{2-6} 
%           & 20                              & 0.02                                 & 0.03                                & 0.02                                  & 0.03                             \\ \cline{2-6} 
%           & 30                              & 0.03                                 & 0.04                                & 0.03                                  & 0.04                             \\ \hline
% Abduction & 1                               & 0.90                                 & 0.04                                & 0.09                                  & 0.07                             \\ \cline{2-6} 
%           & 3                               & 0.87                                 & 0.10                                & 0.14                                  & 0.19                             \\ \cline{2-6} 
%           & 5                               & 0.88                                 & 0.18                                & 0.14                                  & 0.30                             \\ \cline{2-6} 
%           & 10                              & 0.78                                 & 0.31                                & 0.14                                  & 0.45                             \\ \cline{2-6} 
%           & 15                              & 0.72                                 & 0.44                                & 0.14                                  & 0.54                             \\ \cline{2-6} 
%           & 20                              & 0.69                                 & 0.56                                & 0.14                                  & 0.62                             \\ \cline{2-6} 
%           & 30                              & 0.58                                 & 0.71                                & 0.14                                  & 0.64                             \\ \hline
% \end{tabular}
% \end{table}
% \begin{figure}[h]
%   \centering
%   \includegraphics[width=0.75\linewidth]{aamas2025logo}
%   \caption{The logo of AAMAS 2025}
%   \label{fig:logo}
%   \Description{Logo of AAMAS 2025 -- The 24th International Conference on Autonomous Agents and Multiagent Systems.}
% \end{figure}

%\subsubsection{Base Case.}
% The training set is used to extract training regions and learn rules 
% We compute a set of all physically feasible regions n time $t$ from the last seen location in the test sample. We use a reasoner to output a top k likely region.

% Note that for a travel time $t$, the vessel can only go up to $10km$ with an average of historical speed.

% ($|\tau^n| \leq r$, where $\tau^n$ is the last known vessel location and $r$ is the maximum distance the vessel can cover in time $t$).
% For further analysis, Table~\ref{tab:area} shows the reduction in area for the abduction model when comparing the recall for the different region sizes. As the region size is decreased by $75\%$, the precision plummeted by $0.28$ while recall spiked by $0.22$ bringing it closer to $1$. In most cases, the prediction area is reduced by $84.22\%$ on average with less than or equal to $k$ while getting higher recall. Though there is a trade-off with precision, reduced region size is practical for the envisioned application due to its reduced detection area for reasonable performance.

Here, $\regionset$ is known a prior such that $\forall \regionconst \in \regionset$, every $loc \in \regionconst, \travelTime_{agt}(loc_n,loc)\leq t_{n+1} - t_n$. 



       it's partial trajectory location $\tau_{p} = \langle (loc_1, t_1),..,(loc_n, t_n) \rangle$,  a constant $c$, and a set of $k$ regions $\regionset \subseteq \domainSet_{\regionconst}$, then a region set program $\program_{init} =\{f_{t_1},..,f_{t_n}\} $ where $i \in [1,..,n]$, $\forall i, f_{t_i} = \atPred(agt, \regionconst_i) $ and $\regionconst_i = \regionconst_{loc_i-c,loc_i+c}$. Another region set program is $\program_{pred} =\{f_{t_1},..,f_{t_k}\} $ where $j \in [1,..,k]$, $\forall j, f_{t_j} = \atPred(agt, \regionconst_j) $ and $\regionconst_j \in \regionset$.


               \For{$n \gets 1$ \textbf{to} length($\trajectory$)-1}
            \For{$\regionset_i \in \regionset_{\trajectory[n - 1]} \cup \regionset_{\trajectory[n]}$} \Comment{singlemov counting}
                \State $singlemov \gets (\regionset_i)$
                \State Increment $Body[singlemov]$ by 1
            \EndFor
            \For{$\regionset_0 \in \regionset_{\trajectory[n - 1]}$} \Comment{dualmov counting}
                \For{$\regionset_1 \in \regionset_{\trajectory[n]}$}
                    \State $dualmov \gets (\regionset_0, \regionset_1)$
                    \State Increment $Body[dualmov]$ by 1
                \EndFor

            % \textit{REDOING-However, the AIS-off masking setting has a higher F1 than the stay masking setting. The difference is reasonable, considering that the stay predicate in the domain language has a $0.8\%$ lower distribution in the data. Also, the stay masking setting resulted in a $16\%$ lower (than ais-off masking) average value of confidence from the model on the top $k$ regions.}

% \begin{table}[]
% \caption{Reduction in Area (Redn) for different region-sizes (of Area = 5.45$ km^2$ and Area2= 1.1 $km^2$) }
% \label{tab:area}
% \begin{tabular}{|r|r|r|r|r|}
% \hline
% \multicolumn{1}{|l|}{\textbf{Recall}} & \multicolumn{1}{l|}{\textbf{Area (k)}} & \multicolumn{1}{l|}{\textbf{Area2 (k)}} & \multicolumn{1}{l|}{\textbf{Redn($\%$)}}  \\ 
% % & \multicolumn{1}{l|}{\textbf{Redn (km^2)}}
% \hline
% 0.03                         & 5 (1)& 1
%  (1)& 78.58
% % & 6.10
% \\ \hline
% 0.26
% & 12 (4)& 5
%  (5)& 58.90
% % & 11.91
% \\ \hline
% 0.43
% & 16
%  (7)& 6
%  (8)& 59.65
% % & 14.78
% \\ \hline
% 0.66
% & 22
% (10)& 8
% (13)& 64.04
% % & 22.21
% \\ \hline
% 0.90
% & 29 (20)
% & 11 (24)
% & 61.23
% \\ \hline
% 0.99
% &30 (30)
% & 11 (30)
% & 61.43
% \\ \hline
% \end{tabular}
% \end{table}

% \section{Appendix}
% \subsection{Preliminaries}
% Consider an annotated language \cite{ssTAI22} with temporal semantics, where a set of constants $\constantSet$ is divided into multiple domains ($\domainSet_i \subset \constantSet$). For our use case, we use:
% \begin{itemize}
%     \item $\domainSet_{vessel}$ are constants of ship IDs we wish to predict the existence of at a particular time. We will use the symbol $vessel$ to denote the constant associated with a specific
% vessel and $VESSEL$ to denote a variable that can be substituted with a constant denoting
% a specific vessel.
%     \item $\domainSet_{loc}$ are constants associated with the vessel's location. We will use the symbol $loc$ to denote the constant associated with the geological location including latitude and longitude. $LOC$ is used to denote a variable that can be with a constant denoting
% a specific location.
% \item $\domainSet_{mark}$ are constants of port names in our area of interest (AOI) which is a continuous space of  $M \times N$.
% \end{itemize} 

%  Also assume a predicate set $\predicateSet$ where we predefine few predicates:
% \begin{itemize}
%     \item $\atPred$ is a binary predicate with arguments of vessel and location. $\atPred(o, loc)$ is a ground atom for an object $o \in \domainSet_{vessel} \bigcup \domainSet_{mark}$ at a location $loc \in \domainSet_{loc}$
%     \item $\lmPred$ is a unary predicate with an argument of location. $\lmPred(c)$ is a ground atom where $c \in \domainSet_{loc}$. Ex: port, endpoint.
%     \item $\inPred$ is a binary predicate, specifying that a location belongs to a certain region, e.g., consider a location $c \in \domainSet_{loc}$ and a function $bb: \domainSet_{loc} \times \domainSet_{loc} \rightarrow \domainSet_{loc}$ that represents a bounding box of a region. $\inPred(c,bb(c_{min}, c_{max}))$ is a ground atom. If $\atPred(v,c)$ is a ground atom for a vessel $v \in \domainSet_{vessel}$, then the $v$ is considered to be inside the region defined by $bb$.
%     \item We predefine domain-specific unary predicates like $nearport, hotspot, ais-off, stay, draught, statistical-anomalies$ (including unrealistic coordinates, high speed than a threshold of 25, sharp course change, Unusually low speeds compared to average, and  Unusually high speeds compared to average).
%     % ship2ship,  teleportations, Unpredicted Location based on Current Location
% \end{itemize}

% \begin{defination}[Trajectory. $\tau$]
%     A trajectory $\tau$ is a sequence of atoms formed with an agent $\{ agent(c_1)_{t_1},..,agent(c_n)_{t_n} \}$, where $c_i \in \domainSet_{loc}$. An interpretation $\interpretation$ is said to satisfy $\tau$, denoted by $\interpretation \models \tau$, if $\interpretation \models_{t_i} \atPred(agent,c_i)$ and $\interpretation \models_{t_n} \lmPred(c_n)$.
% \end{defination}

% \begin{defination}[Prefix trajectory. $\tau_{pre}$]
%     A prefix of a trajectory, denoted by $\tau_{pre}$, for $m \leq n$ is the first $m$ points in $\tau$.
% \end{defination}

% \begin{defination}[Suffix trajectory. $\tau_{suf}$]
%         A suffix of a trajectory, denoted by $\tau_{suf}$, for $m \leq n$ is the last $n-m$ points in $\tau$.
% \end{defination}


% \begin{defination}[Region]
%     A region is a subset of an AOI, $|R| \leq k$, here $k \in \mathbb{N}$ with a bounding box $bb={c_{min}, c_{max}}$. $\exists \tau' \subseteq \tau, \forall i \in \tau', \inPred(i,bb)$ is a ground atom
% \end{defination}

% \begin{defination}[Feasibility]
%     A feasibility function is defined as $\feasibility: \domainSet_{loc} \times \domainSet_{loc} \rightarrow_{\delta t} \{0,1\}$ where $\feasibility(loc_{start}, loc_{tgt}) = c$. Intuitively, $\feasibility(loc_{start}, loc_{tgt})$ gives either $1$ if for speed $\exists s \in \mathbb{R}^+$, an agent can travel to $loc_{tgt}$ in $\delta t$
% \end{defination}

% \begin{defination}[Value]
% Given a logic program $\program$, a suffix trajectory $\tau_{suf}$, an aggregate function $f$, a predicate $pred$, and a constant $a$, we define $value(\program, \tau_{suf}, pred) = f(\interpretation(pred(a)))$. Here, $\interpretation$ is the minimal model of $\program$ and $\interpretation \models \tau_{suf}$.
%     % An associated aggregate function $value(\program, \tau_{suf}): A$ over a subset of annotations.
% \end{defination}
% \subsubsection{Masked case.}

% \begin{defination}[Prefix trajectory. $\tau_{pre}$]
%     A prefix of a trajectory, denoted by $\tau_{pre}$, for $m \leq n$ is the $n-m$ points in $\tau$, where $n$ is the length of $\tau$.
% \end{defination}

% \begin{defination}[Suffix trajectory. $\tau_{suf}$]
%     A suffix of a trajectory, denoted by $\tau_{suf}$, for $m \leq n$ is the $m$ points in $\tau$, where $n$ is the length of $\tau$.
% \end{defination}  
