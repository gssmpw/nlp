\section{Conclusion and future work}
\label{sec:conclusion}
% In this work, we conduct pioneering research by developing a framework for predicting conflicts and estimating the time of arrival in a multi-agent environment. We formulate our problem as a spatio-temporal graph focusing on edge prediction. This allows for the estimated time of arrival prediction of all robots simultaneously, which previous work has not been able to accomplish. While this is the first study in the field on systemwide estimated time of arrival, it has been demonstrated to increase the accuracy of the predicted time of arrival. As a result, the overall framework's capacity to accurately evaluate the cost of each route improves. 

In this work, we develop a framework to predict conflicts and estimate the time of arrival in multi-agent environments. We formulate our problem as a spatio-temporal graph focusing on edge prediction. The proposed methodology allows the estimated time of arrival prediction of all robots simultaneously, which was not possible by the previously published works. While this is the first study on the system-wide estimated time of arrival, we also demonstrate to increase in the accuracy of the predicted time of arrival. 

% In future work, a couple of features could be incorporated into the current framework. Firstly, incorporate robot acceleration to reflect the real-world dynamics more accurately. Secondly, include multiple types of robots. As we are representing the problem as a heterogeneous graph, incorporating dynamics, constraints, and different types of robots into nodes is very feasible. 
As future work, we plan to incorporate robot acceleration to reflect real-world dynamics more accurately. We will also include multiple types of robots in the same environment. As we represent the problem as a heterogeneous graph, incorporating dynamics, constraints, and different types of robots into nodes will be feasible. 
% Lastly, incorporating the naive path of a conflict zone into \textit{eta} edge features. The current representation only contains the naive duration of which a robot is inside a conflict zone. However, this neglects the scenario of two robots going in the same direction and never entering a conflict. Therefore, incorporating this information on the order of steps performed inside a conflict zone makes a more accurate conflict prediction possible.
Lastly, since the proposed method does not take into account the path to traverse a conflict zone, but only the duration, we will incorporate the path as a grid feature attribute in \textit{eta} edges in the future.
