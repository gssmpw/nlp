\section{Introduction}
\label{sec:introduction}
% As part of industrial 4.0 and the need for more flexibility and efficiency, companies increasingly rely on \ac{AMR} to handle and distribute resources. Amazon is operating over $520.000$ robots across fulfillment and sort centers\cite{amazon-2022}, hereby running one of the largest \ac{AMR} systems. 
% The key component to successfully developing large-scale robotics systems is handling collision-free multi-agent path planning and controlling the flow of traffic. Intelligent transport systems and \ac{TrEPS} are widely used by transportation and logistic companies\cite{GOOGLE_ETA, traffic_pred}. However, the same technologies are rarely seen in indoor routing applications and even less in \ac{AMR} applications. Complications arise when introducing traditional \ac{TrEPS} in time-constrained \ac{AMR} routing applications.
The critical component of large-scale robotic systems is handling collision-free multi-agent path planning to control traffic flow systematically. Many logistic companies use intelligent transport and \ac{TrEPS}\cite{GOOGLE_ETA, traffic_pred}. However, these technologies are rarely used in indoor routing applications and even less in \acp{AMR} applications due to the complications when introducing traditional \ac{TrEPS} in time-constrained routing applications.
% Firstly, no road system is predefined, and mobile robots can move wherever there is free space. This increases the complexity of the planner as conflicts between two intersecting robots can occur anywhere along the trajectory.
The first challenge is that no road system is predefined, and \acp{AMR} can move wherever there is free space, which increases the planner complexity as conflicts between two intersecting robots can occur anywhere along the trajectory.
Secondly, traditional \ac{MAPF} is mainly motivated by path planning or motion planning for multi-robot systems \cite{PIBT, CBS}. However, the optimality or bounded suboptimality of \ac{MAPF} solutions does not necessarily entail their robustness. Robots often have imperfect plan execution capabilities and cannot synchronize their motions perfectly, resulting in frequent conflicts and time-expensive replanning \cite{ma2017overview}. 

% These problems are why we need \ac{TrEPS} to predict conflicts and adjust accordingly.
% Lastly, while allowing access to global information, \ac{AMR} is striving towards a decentralized system, allowing the robot only to change its own route. 

% In order to solve multi-objective path-planning problems faster, the incorporation of \ac{TrEPS} along with traditional path-planning algorithms is more commonly seen. \Ac{GNN} is commonly used in \ac{TrEPS} as they have demonstrated remarkable performance and generalize well to large-scale applications \cite{GOOGLE_ETA,traffic_pred, HEAT}. 

% This work focuses on developing a \ac{CAMAGP} for indoor multi-agent \ac{AMR} applications targeting time-constrained scenarios like logistics in airports or warehousing. What sets this work apart from the rest of the literature is the objective to not only want to find a route within a fixed time constraint but also to validate the effect caused by the route on the rest of the system by predicting \ac{ETA} for every agent in the system. 
This work focuses on developing a \ac{CAMAGP} for indoor multi-agent \ac{AMR} applications that target time-constrained scenarios, e.g., logistics in airports and warehouses. 
\ac{CAMAGP} is based on a two-layered framework. The first layer is a traditional path-planning algorithm for route suggestions which can be replaced by any appropriate path-planning algorithm. The second layer is the validation layer that validates the route suggested by the first layer and checks for time constraints. In order to validate a route, a graph is used to model spatio-temporal connectivity, and a \ac{GNN} is utilized for estimating traffic and predicting \ac{ETA}.
The objective of this research is not only to find a route within a fixed time constraint but also to validate the effect the route has on the rest of the system by predicting \ac{ETA} for each agent in the system. 
The performance of the proposed \ac{GNN} model is compared and contrasted with a naive method. 
In this work, the definition of \textit{naive} refers to paths that do not consider imperfect plan execution and the conflicts along the planned path. The naive time of arrival is the time it would take to perform such a naive path. 


\begin{figure}[t]
   \centering
   \includegraphics[width=0.49\textwidth]{images/introduction/drawing.pdf}
   \caption{Illustration of the multi-agent ETA prediction used in a warehouse environment. The values are the system-wide cost of taking the corresponding route, while the paths are relatively color-coded from the best (green) to the worst (red).}
   \label{fig:overview}
\end{figure}
The major contributions of this study are the followings:
\begin{itemize}
    \item A conflict-aware global planner is designed to optimize overall system flow while considering time constraints for all robots in industrial scenarios.
    \item A heterogeneous graph representation is proposed to model interaction between agents and potential bottle-neck areas occurring within the map. 
    %we propose an automatic bottle-neck detection and topological graph construction method from 2d occupancy grid maps.
    \item Finally, a novel \ac{GNN} architecture for multi-robot \ac{ETA} prediction in heterogeneous graphs is designed. 
\end{itemize}

The rest of this paper is organized as follows. 
Section \ref{sec:related_work} introduces the recent developments in \ac{TrEPS} and \ac{ETA} prediction. 
Section \ref{sec:methodology} provides the preliminaries for the problem formulation.
Section \ref{sec:experiments} presents the experiment setup followed by the \ac{CAMAGP} results in various grid world environments which are given in Section \ref{sec:results}. Finally, some conclusions are drawn from this study in Section \ref{sec:conclusion}.



