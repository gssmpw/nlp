\begin{figure*}[ht]
   \centering
   \includegraphics[width=0.98\textwidth]{images/Method/Network.pdf}
   \caption{Illustration of the proposed multi-agent ETA prediction framework. A simplified illustration of the three layers representing the indoor environment, as well as the connection between the layers, is shown on the left. The three layers consist of; a room-map segmenting the given grid map into artificial rooms, a static layer containing all nodes and edges with static information, and a dynamic layer containing the time-variant features of the graph. The proposed GNN architecture for multi-agent ETA prediction is also shown.}
   \label{fig:main}
\end{figure*}
\section{Methodology}
\label{sec:methodology}
% This section presents the four components of Map and graph representation, path-planning, eta-prediction, and constraint validation.

\subsection{Map and graph representation}
To predict multi-agent \ac{ETA} in an indoor environment, a room-map and spatio-temporal heterogeneous graph representation is proposed to capture the structure of the environment and the multiple route interaction.
The spatio-temporal heterogeneous graph is defined as follows:
\begin{equation}
G = (V, E, T_v, T_e, X_{v(t)}, X_{e(t)})  
\end{equation}
where $V$ is the set of nodes, $E$ is the set of edges, $T_v$ represents the node types belonging to the set $\{\text{robot}, \text{floor}\}$, $T_e$ represents the edge types belonging to the set $\{\text{eta},\text{association}\}$, $X_{v(t)}$ is the set of node-features at a given time $t$, and $X_{e(t)}$ is the set of edge-features at a given time $t$. 
The graph can further be deconstructed into two layers. 
A static layer containing nodes and edges with static features, and a dynamic layer containing nodes and edges with temporal features.
The objective of the room-map is to segment a grid-map into rooms and locate all passages connecting the rooms. 
The room-map is used to construct the static layer of the spatio-temporal heterogeneous graph and for route suggestions which will be described in Section \ref{PP}.
The three layers, room-map, static layer, and dynamic layer, and their interconnections are presented in Fig. \ref{fig:main}.
\medbreak
1) The proposed method expects a 2D topological grid-map of the environment containing static information about walls, rags, and non-parsing zones to construct the room-map.
The given grid-map is transformed into a homogeneous graph $G_{room}(V, E, X_v, X_e)$ where every corner is a node $v_i \in V$ and the node attribute $x_{v_i} \in X_v$ contains whether it is a concave or a convex corner. 
An edge $e_j \in E$ is either a solid surface like walls or a passage between two rooms. The corresponding attribute $x_{e_j} \in X_e$ is a one-hot encoding of whether it is a passage.
% Every surface is an edge , and its corresponding attribute $x_{e_j} \in X_e$ is whether it is a passage between rooms.
In order to construct passage edges, a ray-tracing is performed from every convex corner to the nearest neighboring static element.
If the ray-tracing hits an edge, the edge is split into two edges, and a node $v_{hit}$ is added in between.
A passage edge is constructed from the convex corner node $v_{start}$ to $v_{hit}$.

To divide the grid-map into rooms, a loop search is performed on the graph starting from every passage edge. 
The area bounded by the two shortest loops connected to each passage edge is defined as a room.
The rooms are then evaluated accordingly to their size and shape and merged with a neighboring room if they do not meet certain requirements.
These requirements are: no room is made up of less than four nodes, and the area of the room must be bigger than a threshold $A$, which is defined accordingly to the size of the map.
The last requirement is to merge small doorways, which are usually a group of four concave nodes close to each other. 

It is important to note that while these areas are defined as rooms, they do not necessarily fit the common definition of a room.
Instead, these map rooms are enclosed spaces surrounded by solid objects and passages, where the passages describe potential bottlenecks within the map.
An actual room can contain multiple of these map rooms located by this method. 

\medbreak
2) The static layer comprises all nodes and edges whose attributes are static and, therefore, will not change over time.
The nodes contained in this layer are of type \textit{floor}.
To construct \textit{floor} nodes, first the room-map is split into $N\times N$ patches.
A \textit{floor} node is constructed if any room tile occurs within a patch.
The feature of \textit{floor} nodes is the $N\times N$ patch.
A special case is if a solid line separates the patch.
Multiple \textit{floor} nodes are stacked, each representing a side of the patch, as the traffic on one side does not directly affect the other.
These stacked nodes are represented as multi-colored nodes in Fig. \ref{fig:main}.
The edges contained in the static layer are of type \textit{association}.
An association edge contains no features but represents a spatial link between two adjacent nodes or a self-loop.
An association edge is constructed when a direct path between two adjacent nodes can be established.

\medbreak
3) The dynamic layer contains all the nodes and edges with temporal features.
The nodes contained in this layer are of type \textit{robot} and the temporal feature of the node is the robot's priority.
The priority is computed based on the buffer time of each robot. 
\begin{equation}
\textit{priority}=(t_{current} + t_{path}) - t_{constraint}
\end{equation}
where $t_{current}$ is the current time, $t_{path}$ is the naive time of arrival and $t_{constraint}$ is the latest time of arrival.
This formulation can lead to deadlock if two robots are equal in priority.
Therefore the priority is rounded to the nearest integer and padded with the internal id as decimals to make the priority unique.
The edges in the dynamic layer are of type \textit{eta}.
\textit{Eta} edges are constructed from a \textit{robot} node to all \textit{floor} nodes passed in the corresponding path-plan of the robot.
The temporal features of \textit{eta} edges are the naive duration spent in \textit{floor} node, naive arrival time, and timestamp of the edge.
The timestamp portrays the order in which the \ac{GNN} processes the edges.

Lastly, all nodes contain a self-loop edge containing no features. The loop ensures the node's own features are combined with the neighboring features during the message passing layers. 

\subsection{Path-planner}
\label{PP}
An essential component in the proposed architecture is path-planning since a well-planned path can minimize the number of conflicts along the path. 
Traditional \ac{PIBT}\cite{PIBT} and \ac{CBS}\cite{CBS} calculates a conflict-free arrangement of paths for multi-agent systems.
However, they assume a constant cost in time for every movement, hereby not accounting for model parameters like acceleration and varying velocities.
As a result, the suggested paths can quickly become unsynchronized, leading to numerous conflicts.
The proposed framework can use any path-planning method since conflicts are accounted for during \ac{ETA} prediction.
The naive path planner chosen for the simulation is $A^*$\cite{a-star} because it does not account for any conflict.
Conflicts are solved online in the local planner using a priority-based system.
The robot with the highest priority keeps following its path while the others have to adapt their path.
Each robot declares the following five moves in a local area, and a local replanning is done to accommodate these intents based on priority.

A global graph is constructed from the room-map in order to generate route suggestions.
Each node in the graph represents a passage between rooms, while each edge connects all passages within the corresponding rooms.
The edge attributes are the distance between the passages calculated by a local planner.
A start node is added by locating the robot using the room-map and is connected to all associated passages within the found room.
The same procedure is done for the goal position.
Dijkstra \cite{dijkstra} is performed on the graph to determine the shortest route from the start to the goal.
In order to produce new paths in successive generations, the cost of the used edges is multiplied by $1.2$ after each iteration.
\medbreak
The path-planning module suggests plausible routes to the destination based solely on distance and the room-map layout.
Therefore ignoring the interaction of other agents in the system.
To validate the path's cost relative to the entire system, we have developed a \ac{GNN} architecture for multi-agent \ac{ETA} prediction for the entire system.

\subsection{Model architecture}
The architecture of the \ac{GNN} model is created accordingly to the encode-process-decode format illustrated in Fig. \ref{fig:main}. 
First, each node and edge in the heterogeneous graph is sent to the encoding layer.
An encoder is trained for each type of node and edge in the graph to extract specific type features into a $64$-dimensional feature vector.
The encoded nodes and edges are combined into the heterogeneous graph and passed to the attentive communication layers for message passing.

In the message passing module, we utilize the \ac{HEAT} operator\cite{HEAT}.
This operator enhances the \ac{GAT} \cite{GAT} by incorporating type-specific transformations of nodes and edges features during message passing. 
Similar to \ac{GAT}, \ac{HEAT} allows running several independent attention heads to stabilize the self-attention mechanism.
The features from each attention head are concatenated as the updated output feature.
The output of the \ac{HEAT} operator is the updated node features from the neighborhood of $K=1$ and $H=3$ number of attention heads concatenated.
The updated node features are passed to the edge updater module along with the encoded \textit{eta} edges.
In the edge updater module, the \textit{eta} edges are concatenated with the corresponding nodes and sent through a linear layer.

The decoder receives the \textit{eta} edges with timestamp $T$ and converts the eta feature vectors to \ac{ETA} predictions.
In the case of \ac{DMS}, the naive time of arrival of all \textit{eta} edges in the dynamic layer is updated based on the \ac{ETA} prediction to reflect the new predicted state of all robots. 
In the case of \ac{IMS}, the actual arrival time is used instead of the \ac{ETA} prediction. 
% Based on either the \ac{ETA} prediction in case of \ac{DMS} or the actual arrival time in case of \ac{IMS}, the naive time of arrival of all \textit{eta} edges in the dynamic layer is updated to reflect the new state of all robots.
As this is a recurrent process, this is repeated for $T=T+1$ until all \textit{eta} edges have been decoded.

\subsection{Validating time-constraints and path selection}
\label{cha:methodlogy}
There are numerous ways to validate a time constraint path, but choosing the optimal method relies heavily on the application at hand.
The proposed method rejects a path if one or more time constraints are not met.
In order to select the optimal valid path, the following cost is proposed: 
\begin{equation}
cost=\sum_{i=1}^{N}{\Bigl(\max{(\textit{TC})} - (\textit{TC}_{i} - \text{eta}_i)\Bigr)^2}  
\end{equation}
where \textit{TC} is the set of all the time constraints, \textit{eta} is the set of predicted \ac{ETA} for all $N$ robots.
The cost is constructed to maximize the buffer time between the constraint time and the predicted arrival time while penalizing the shorter buffer times and allowing more slack on the larger buffer times. 