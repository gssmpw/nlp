\section{Introduction}
\label{sec:introduction}
\begin{figure}[t]
   \centering
   \includegraphics[width=0.43\textwidth]{images/introduction/Framework_Structure_v3.pdf}
   \caption{The conflict-aware multi-agent planner framework is depicted in three layers. The first layer generates multiple path suggestions for a single agent. The second layer, enclosed in blue borders, produces estimated time of arrival predictions for all agents. Finally, the third layer computes the overall cost for each suggested path, using the information generated in the previous layer.}
   \label{fig:overview}
 
\end{figure}

% The critical component of large-scale robotic systems is handling collision-free multi-agent path planning to control traffic flow systematically. Many logistic companies use intelligent transport and \ac{TrEPS}\cite{GOOGLE_ETA, traffic_pred}. However, these technologies are rarely used in indoor routing applications and even less in \acp{AMR} applications due to the complications when introducing traditional \ac{TrEPS} in time-constrained routing applications.

% The first challenge is that no road system is predefined, and \acp{AMR} can move wherever there is free space, which increases the planner complexity as conflicts between two intersecting robots can occur anywhere along the trajectory.
% Secondly, traditional \ac{MAPF} is mainly motivated by path planning or motion planning for multi-robot systems \cite{PIBT, CBS}. However, the optimality or bounded suboptimality of \ac{MAPF} solutions does not necessarily entail their robustness. 


% These problems are why we need \ac{TrEPS} to predict conflicts and adjust accordingly.
% Lastly, while allowing access to global information, \ac{AMR} is striving towards a decentralized system, allowing the robot only to change its own route. 

% In order to solve multi-objective path-planning problems faster, the incorporation of \ac{TrEPS} along with traditional path-planning algorithms is more commonly seen. \Ac{GNN} is commonly used in \ac{TrEPS} as they have demonstrated remarkable performance and generalize well to large-scale applications \cite{GOOGLE_ETA,traffic_pred, HEAT}. 

\Ac{MAPF} is the problem of generating valid paths for multiple agents while avoiding conflicts. This problem is highly relevant in many real-world applications, such as logistics, transportation, and robotics, where multiple agents must operate in a shared environment. \ac{MAPF} is a challenging problem due to the need to find paths that avoid conflicts while minimizing the overall travel time for all agents. Many state-of-the-art \ac{MAPF} solvers \cite{CBS, PIBT, HCA*} employ various techniques to find a set of conflict-free paths on graphs representing the environment and the agents. However, a common limitation of these solvers is that they tend to generate tightly planned and coordinated paths. Therefore, the agents are expected to follow the exact path prescribed by the solver, which can lead to problems when applied to real-world systems with imperfect plan execution and uncertainties in the environment.

% In this work, we propose a \ac{CAMETA} for indoor multi-agent \ac{AMR} applications that operate in time-constrained scenarios. \ac{CAMETA} is a three-layered framework deployed on every agent. The layers consist of: a path planning layer producing route suggestions for the current agent, a multi-agent \ac{ETA} prediction layer forecasting \ac{ETA} prediction of all agents in the system given one of the suggested paths, and a cost validation layer minimize the overall travel time by reducing the total number of conflicts in the system. The proposed framework is illustrated in Fig. \ref{fig:overview}. In our problem definition, some agents need to arrive at their destination quicker than other agents in a system, e.g., logistics in airports and warehouses.


This work introduces a \ac{CAMETA} for indoor \ac{AMR} applications that operate in time-constrained scenarios. The proposed framework is a three-layered framework that is deployed on each agent. The layers consist of a path planning layer, which generates route suggestions for the deployed agent, a multi-agent \ac{ETA} prediction layer, which forecasts the \ac{ETA} of all agents in the system given one of the suggested paths, and a cost validation layer that minimizes the overall travel time by reducing the total number of conflicts in the system. In our problem definition, some agents are required to arrive at their destination quicker than others, which is a common scenario in logistics applications for airports and warehouses. The proposed framework is illustrated in Fig. \ref{fig:overview}.


% In systems with multiple moving agents, valid paths must be provided to ensure that agents reach their destinations smoothly without any conflicts, while minimizing excess travel time. This problem is embodied in \ac{MAPF}, which aims to find a set of conflict-free paths on graphs.
% Many state-of-the-art \ac{MAPF} solvers \cite{CBS, PIBT, HCA*} tents to produce paths that are too tightly planned and coordinated in order to optimize the objective of finding the shortest paths without taking into consideration imperfect plan execution or noise.  

% This work focuses on developing a \ac{CAMETA} for indoor multi-agent \ac{AMR} applications that target time-constrained scenarios. In our problem definition, some agents need to arrive at their destination quicker than other agents in a system, e.g., logistics in airports and warehouses. The overall framework is illustrated in Fig. \ref{fig:overview}.

% The proposed framework is based on a three-layered structure, as illustrated in Fig \ref{fig:overview}. The first layer comprises a traditional path-planning algorithm that proposes various route suggestions. 
% The second layer aims to evaluate the proposed routes by forecasting the \ac{ETA} for each agent. The objective of the layer is to anticipate the impact of the proposed path on the other agents within the system and to estimate the additional time added by these effects. To estimate the \ac{ETA} of all agents, this study employs a graph-based approach to model the spatio-temporal connectivity of the system. This graph is then provided to a recurrent \ac{GNN} model for prediction. 
% The third layer utilizes the \ac{ETA} to evaluate time-constraints and calculate the cost of each suggested route. The layer is not only used to find a route that is feasible within a fixed time constraint but it is also used to validate the impact of the chosen route on the rest of the system by utilizing the predicted \ac{ETA} for all agents. The first and last layers of the framework are designed to be modular and can be easily substituted with more application-specific methods. 

% In order to validate a route, a graph is used to model spatio-temporal connectivity, and a \ac{GNN} is utilized for estimating traffic and predicting \ac{ETA}.
% The objective of this research is not only to find a route within a fixed time constraint but also to validate the effect the route has on the rest of the system by predicting \ac{ETA} for each agent in the system. 

This work mainly focuses on the development of the multi-agent \ac{ETA} prediction layer and its effectiveness in forecasting the \ac{ETA} for each agent. To evaluate the multi-agent \ac{ETA} prediction layer and the proposed \ac{GNN} model, this study compares it to a naive method. In this context, the term \textit{"naive"} refers to paths, which does not consider the imperfect execution of plans and conflicts that may occur along the planned path. 
% The naive time of arrival is defined as the time it would take to complete such a naive path.
This comparison provides insight into the performance and accuracy gained by utilizing the \ac{GNN} model for predicting \ac{ETA} as opposed to ignoring the complexities such as imperfect execution and conflicts. Lastly, An experiment is conducted to demonstrate the proposed framework's overall performance compared to state-of-the-art methods \cite{CBS, PIBT}.

The contributions of this study are the followings:
\begin{itemize}
    \item A conflict-aware global planner is designed to optimize overall system flow while considering time constraints for all robots in industrial scenarios.
    \item A heterogeneous graph representation is proposed to model the interaction between agents and potential bottle-neck areas occurring within the map. 
    %we propose an automatic bottle-neck detection and topological graph construction method from 2d occupancy grid maps.
    \item Finally, a novel \ac{GNN} architecture for multi-robot \ac{ETA} prediction in heterogeneous graphs is designed. 
\end{itemize}

The rest of this paper is organized as follows. 
Section \ref{sec:related_work} reviews the recent developments in \ac{MAPF} and \ac{ETA} prediction. 
Section \ref{sec:methodology} provides the details of the proposed framework.
Section \ref{sec:experiments} presents the experiment setup followed by the results in various grid world environments in Section \ref{sec:results}. Finally, some conclusions are drawn from this study in Section \ref{sec:conclusion}.



