\section{Introduction}
\label{sec:introduction}
\begin{figure}[t]
   \centering
   \includegraphics[width=0.43\textwidth]{images/introduction/Framework_Structure_v3.pdf}
   \caption{The conflict-aware multi-agent planner framework is depicted in three layers. The first layer generates multiple path suggestions for a single agent. The second layer, enclosed in blue borders, produces an estimated time of arrival prediction for all agents. Finally, the third layer computes the overall cost for each suggested path, using the information generated in the previous layer.}
   \label{fig:overview}
 
\end{figure}

\Ac{MAPF} is the problem of generating valid paths for multiple agents while avoiding conflicts. This problem is highly relevant in many real-world applications, such as logistics, transportation, and robotics, where multiple agents must operate in a shared environment. \ac{MAPF} is a challenging problem due to the need to find paths that avoid conflicts while minimizing the overall travel time for all agents. Many state-of-the-art \ac{MAPF} solvers \cite{CBS, PIBT, HCA*} employ various techniques to find a set of conflict-free paths on graphs representing the environment and the agents. However, a common limitation of these solvers is that they tend to generate tightly planned and coordinated paths. Therefore, the agents are expected to follow the exact path prescribed by the solver, which can lead to problems when applied to real-world systems with imperfect plan execution and uncertainties in the environment.

This work introduces a \ac{CAMETA} for indoor \ac{AMR} applications that operate in time-constrained scenarios. The proposed framework is a three-layered framework that is deployed on each agent. The layers consist of a path planning layer, which generates route suggestions for the deployed agent, a multi-agent \ac{ETA} prediction layer, which forecasts the \ac{ETA} of all agents in the system given one of the suggested paths, and a path selection layer that minimizes the overall travel time by reducing the total number of conflicts in the system. In our problem definition, some agents are required to arrive at their destination quicker than others, which is a common scenario in logistics applications for airports and warehouses. The proposed framework is illustrated in Fig. \ref{fig:overview}.

% The proposed work mainly focuses on the development of the multi-agent \ac{ETA} prediction layer and its effectiveness in forecasting the \ac{ETA} for each agent. To evaluate the multi-agent \ac{ETA} prediction layer and the proposed \ac{GNN} model, this study compares it to a naive method. In this context, the term \textit{"naive"} refers to paths, which does not consider the imperfect execution of plans and conflicts that may occur along the planned path. This comparison provides insight into the performance and accuracy gained by utilizing the \ac{GNN} model for predicting \ac{ETA} as opposed to ignoring the complexities such as imperfect execution and conflicts. Additionally, the proposed model demonstrates flexibility and generalizability by effectively adapting to different robot densities. Given the potential variations in the number of robots due to factors such as season, demand, and business requirements, the model's ability to seamlessly accommodate changes in robot densities without requiring retraining is highly valuable. This characteristic enables the industry to easily add or remove robots as needed, enhancing operational efficiency and adaptability.

The main focus of this work is on the development of the multi-agent \ac{ETA} prediction layer and its effectiveness in forecasting the \ac{ETA} for each agent. The performance and accuracy of the proposed \ac{GNN} model and the multi-agent \ac{ETA} prediction layer are evaluated by comparing them to a naive method. In this context, the term \textit{"naive"} refers to path planning that does not consider the imperfect execution of plans and conflicts that may arise along the intended path.
This comparison provides valuable insights into the performance and accuracy improvements achieved by utilizing the \ac{GNN} model for \ac{ETA} prediction, which explicitly considers the complexities associated with imperfect execution and conflicts. The comparison provides an understanding of the added value of the proposed model in accurately forecasting \ac{ETA} and addressing real-world challenges in multi-agent systems.
Furthermore, the proposed model exhibits flexibility and generalizability by effectively adapting to different robot densities. Given the potential variations in the number of robots due to seasonal fluctuations, demand changes, or business requirements, the model's ability to seamlessly accommodate shifts in robot densities without requiring retraining is of significant value. This characteristic empowers industries to easily add or remove robots as needed, thereby enhancing operational efficiency and adaptability in dynamic environments. To assess the generalizability and overall performance of the proposed framework, an experiment is conducted. This experiment aims to test how different trained prediction models scale during inference time, thereby demonstrating the framework's ability to handle varying robot densities.

In addition, an experiment is conducted to compare the overall performance of the proposed framework, including its ability to handle imperfect plan execution, with state-of-the-art \ac{MAPF} planners \cite{CBS, PIBT}. This comparative analysis serves to provide valuable insights into the effectiveness and competitiveness of the proposed framework in addressing \ac{MAPF} challenges, particularly in environments where plan execution may be imperfect. Furthermore, to ensure a comprehensive evaluation, this experiment incorporates the presence of noise, which further tests the robustness and adaptability of the proposed framework under realistic conditions. The inclusion of noise allows for a more realistic assessment of the framework's performance in the presence of uncertainties and deviations from ideal plan execution.

% Here is a small text added to destroy it all

The contributions of this study are the followings:
\begin{itemize}
    \item A conflict-aware global planner is designed to optimize overall system flow while considering time constraints for all robots in industrial scenarios.
    \item A heterogeneous graph representation is proposed to model the interaction between agents and potential bottleneck areas occurring within the map. 
    \item Finally, A novel \ac{GNN} architecture is proposed for multi-robot \ac{ETA} prediction in heterogeneous graphs supporting scaling to different robot densities during inference time. 
\end{itemize}

The rest of this paper is organized as follows. 
Section \ref{sec:problem_formulation} presents the problem formulation. Section \ref{sec:related_work} reviews the recent developments in \ac{MAPF} and \ac{ETA} prediction. 
Section \ref{sec:methodology} provides the details of the proposed framework.
Section \ref{sec:experiments} presents the experiment setup followed by the results in various grid world environments in Section \ref{sec:results}. Finally, some conclusions are drawn from this study in Section \ref{sec:conclusion}.



