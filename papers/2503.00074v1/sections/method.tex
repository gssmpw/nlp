\begin{figure*}[ht]
   \centering
   \includegraphics[width=0.98\textwidth]{images/Method/Network_no_room_map.pdf}
   \caption{Illustration of the proposed multi-agent \ac{ETA} prediction layer. A simplified illustration of the three layers representing the indoor environment and the connection between the layers is shown on the left. The three layers consist of: a 2D occupancy grid map of the environment, a static layer containing all nodes and edges with static information, and a dynamic layer containing the time-variant features of the graph. The proposed \ac{GNN} architecture for multi-agent \ac{ETA} prediction is also shown.}
   \label{fig:main}
\end{figure*}
\section{Methodology}
\label{sec:methodology}
%%%%%%%%%%%%%%%%%%%%%%%%%%%%%%%%%%%%%%%%%%%%%%%%%%%%%
%%%%%%% PATHPLANNING %%%%%%%%%%%%%%%%%%%%%%%%%%%%%%%%%%%%
%%%%%%%%%%%%%%%%%%%%%%%%%%%%%%%%%%%%%%%%%%%%%%%%%%%%%
\subsection{Path planning}
\label{PP}
The path planning layer is similar in design to the \ac{LRA*}\cite{LRA*}, where other agents in the system are not considered, and resolving conflicts is left to the local planner. This allows for a decentralized global planning module, where each agent only needs to consider possible routes to its destination. While drastically improving scalability, it comes with the limitation of not being able to change the routes of other agents.
Essentially the consideration of other agents is first introduced in the multi-agent \ac{ETA} prediction layer.
% , where \ac{ETA} is computed for all agents. 
The global planner chosen for this framework is $A^*$\cite{a-star}, as it does not account for conflicts and can quickly compute different route suggestions. The proposed framework can use any off-the-shelf path-planning methods since conflicts are accounted for during the multi-agent \ac{ETA} prediction layer. It is important to note that the \ac{GNN} is trained based on the local and global path planning algorithm and, therefore, is still dependent on these. The \ac{GNN} model learns to model the interactions between these two planners, including how much the global path-planning algorithm considers conflicts in its route suggestions and how effectively the local path-planning algorithm resolves conflicts. 

It is important to note that the local planner is not directly part of the proposed pipeline but is used during execution to solve unexpected conflicts. The local planner used for training is the \ac{WHCA*} \cite{HCA*} planner with a priority-based system. The robot with the highest priority maintains its path while others adapt their paths. Each robot declares its next five moves in a local area, and local replanning is done based on priority. Agents require local peer-to-peer communication and priority-based bidding to align the reservation of space.

\subsection{Multi-agent estimated time of arrival prediction} 
The multi-agent \ac{ETA} prediction layer is the second layer of the system. The objective of the layer is to anticipate the impact of the proposed path on the other agents within the system and to estimate the additional time added by these effects. It takes as input the suggested routes from the path planning layer, as well as the planned paths of all other agents, and a 2D occupancy grid map of the environment in order to predict the \ac{ETA} for each agent. The multi-agent \ac{ETA} prediction layer is composed of two main steps: It reconstructs the data into a spatio-temporal heterogeneous graph representation and then passes the graph through a recurrent spatio-temporal \ac{GNN} model in order to make the \ac{ETA} predictions.

%%%%%%%%%%%%%%%%%%%%%%%%%%%%%%%%%%%%%%%%%%%%%%%%%%%%%%
%%%%%%% GRAPH REPRESENTATION %%%%%%%%%%%%%%%%%%%%%%%%%%%%%%%%
%%%%%%%%%%%%%%%%%%%%%%%%%%%%%%%%%%%%%%%%%%%%%%%%%%%%%%
\subsubsection{Graph representation}
% To predict multi-agent \ac{ETA} in an indoor environment, a spatio-temporal heterogeneous graph representation is proposed to capture the structure of the environment and the multiple route interaction.
To predict the \ac{ETA} of multiple agents in an indoor environment, a spatio-temporal heterogeneous graph representation is used to capture the environment's structure and the interactions between routes. The graph is divided into two layers: a static layer and a dynamic layer. The interaction between the layers is depicted in Fig. \ref{fig:main}.

The static layer of the graph only needs to be constructed once, as its features and connections remain unchanged. The static layer contains nodes of type \textit{floor} with static features of the environment, such as walls, restricted areas, and open spaces. The edges in the static layer are of type \textit{association} and represent the spatial links between nodes. The spatial characteristics of the environment are obtained from a 2D occupancy grid map, which is divided into $N\times N$ tiles. A node of type  \textit{floor} is constructed for each tile created from the given map, and the node features are the spatial information within the tile. In order to lower the number of floor nodes, all nodes containing only occupied space are removed. In the case of a solid line separating a tile, multiple \textit{floor} nodes are stacked to represent each side of the patch, as traffic on one side does not directly impact the other. These stacked nodes are depicted as multi-colored nodes in Fig. \ref{fig:main}. Association edges contain no features and are created when a direct path between two adjacent nodes or a self-loop is established.

The dynamic layer consists of nodes and edges with temporal features, where the features are subject to change or removal in the recurrent stages of the \ac{GNN} model. Nodes in this layer are of type \textit{robot}, and the temporal feature of each node is the robot's priority, computed based on the buffer time between the current time estimation and the constraint time of the robot task. If two robots have equal priority, the priority is rounded to the nearest integer and appended with the robot's internal ID as decimals to make it unique to avoid deadlocks. Edges in the dynamic layer are of type \textit{eta} and are created from a \textit{robot} node to all \textit{floor} nodes in the robot's planned trajectory. All outgoing \textit{eta} edges from a \textit{robot} node represent the planned path for the corresponding robot. The temporal features of the \textit{eta} edges include the estimated duration at each \textit{floor} node, the estimated arrival time, and the timestamp of the edge. The timestamp reflects the order in which the \ac{GNN} processes the edges. All nodes contain a self-loop edge, which has no features, to ensure that the node's own features are combined with those of its neighbors during the message-passing phase.

Combining the dynamic and static layer, the spatio-temporal heterogeneous graph is defined as follows:
\begin{equation}
G = (V, E, T_v, T_e, X_{v(t)}, X_{e(t)})  
\end{equation}
where $V$ is the set of nodes, $E$ is the set of edges, $T_v$ represents the node types belonging to the set $\{\text{robot}, \text{floor}\}$, $T_e$ represents the edge types belonging to the set $\{\text{eta},\text{association}\}$, $X_{v(t)}$ is the set of node-features at a given time $t$, and $X_{e(t)}$ is the set of edge-features at a given time $t$. 
% The graph can be categorized into two layers; a static layer and a dynamic layer. 
% The static layer contains nodes and edges with static features of the environment, such as walls, restricted areas, and open spaces, while the dynamic layer consists of nodes and edges with temporal features that change over time, capturing the dynamic movement and interactions between agents. The interaction between the layers is depicted in Fig. \ref{fig:main}.

% \medbreak
%%%%%%%%%%%%%%%%%%%%%%%%%%%%%%%%%%%%%%%%%%%%%%%%%%%%%%
%%%%%%% STATIC LAYER %%%%%%%%%%%%%%%%%%%%%%%%%%%%%%%%%%%%%%%
%%%%%%%%%%%%%%%%%%%%%%%%%%%%%%%%%%%%%%%%%%%%%%%%%%%%%%
% The static layer of the graph only needs to be constructed once, as its features and connections remain unchanged. The spatial characteristics of the environment are obtained from a 2D occupancy grid map, which is divided into $N\times N$ tiles. A node of type  \textit{floor} is constructed for each tile created from the given map, and the node features are the spatial information within the tile. In order to lower the number of floor nodes, all nodes containing only occupied space are removed. In the case of a solid line separating a tile, multiple \textit{floor} nodes are stacked to represent each side of the patch, as traffic on one side does not directly impact the other. These stacked nodes are depicted as multi-colored nodes in Fig. \ref{fig:main}. 
% The edges in the static layer are of type \textit{association} and serve to represent the spatial links between nodes. Association edges contain no features and are created when a direct path between two adjacent nodes or a self-loop is established.

% \medbreak
%%%%%%%%%%%%%%%%%%%%%%%%%%%%%%%%%%%%%%%%%%%%%%%%%%%%%%%%%%%%%%%%%%%%%%%%%%
%%%%%%% DYNAMIC LAYER %%%%%%%%%%%%%%%%%%%%%%%%%%%%%%%%%%%%%%%%%%%%%%%%%%%%
%%%%%%%%%%%%%%%%%%%%%%%%%%%%%%%%%%%%%%%%%%%%%%%%%%%%%%%%%%%%%%%%%%%%%%%%%%
% The dynamic layer comprises nodes and edges with temporal features. The features of the nodes and edges in this layer are subject to change or removal in the recurrent stages of the \ac{GNN} model. The nodes in the dynamic layer are of type \textit{robot}, and the temporal feature of each node is the robot's priority which is computed based on the buffer time between the current time estimation and the constraint time of the robot task as follows.
% \begin{equation}
% \textit{priority} = (t_{current} + t_{path}) - t_{constraint}
% \end{equation}
% where $t_{current}$ is the current time, $t_{path}$ represents the estimated time of arrival, and $t_{constraint}$ is the latest time of arrival. If two robots have equal priority, the priority is rounded to the nearest integer and appended with the robot's internal ID as decimals to make it unique to avoid deadlocks. The edges in the dynamic layer are of type \textit{eta}. 
% These edges are created from a \textit{robot} node to all \textit{floor} nodes that are included in the robot's planned trajectory. Specifically, whenever a robot is planned to traverse a tile, an edge is created to the corresponding \textit{floor} node. The temporal features of the \textit{eta} edges include the estimated duration at each \textit{floor} node, the estimated arrival time, and the timestamp of the edge. The timestamp reflects the order in which the \ac{GNN} processes the edges. Additionally, all nodes contain a self-loop edge, which has no features. The loop ensures that the node's own features are combined with those of its neighbors during the message-passing phase.

%%%%%%%%%%%%%%%%%%%%%%%%%%%%%%%%%%%%%%%%%%%%%%%%%%%%%%%%%%%%%%%%%%%%%%%%%%
%%%%%%% MODEL ARCHITECTURE %%%%%%%%%%%%%%%%%%%%%%%%%%%%%%%%%%%%%%%%%%%%%%%
%%%%%%%%%%%%%%%%%%%%%%%%%%%%%%%%%%%%%%%%%%%%%%%%%%%%%%%%%%%%%%%%%%%%%%%%%%
\subsubsection{Model architecture}
The architecture of the \ac{GNN} model is created accordingly to the encode-process-decode format illustrated in Fig. \ref{fig:main}. 
First, each node and edge in the heterogeneous graph is sent to the encoding layer.
An encoder is trained for each type of node and edge in the graph to extract specific type features into a $64$-dimensional feature vector.
The encoded nodes and edges are combined into the heterogeneous graph and passed to the attentive communication layers for message passing.

In the message passing module, we utilize the \ac{HEAT} operator\cite{HEAT}.
This operator enhances the \ac{GAT} \cite{GAT} by incorporating type-specific transformations of nodes and edges features during message passing. 
Similar to \ac{GAT}, \ac{HEAT} allows running several independent attention heads to stabilize the self-attention mechanism.
The features from each attention head are concatenated as the updated output feature.
The output of the \ac{HEAT} operator is the updated node features from the neighborhood $K=1$ and the number of attention heads $H=3$ concatenated.
The updated node features are passed to the edge updater module along with the encoded \textit{eta} edges.
In the edge updater module, the \textit{eta} edges are concatenated with the corresponding nodes and sent through a linear layer.

The decoder receives the \textit{eta} edges with timestamp $T$ and converts the eta feature vectors to \ac{ETA} predictions.
For \ac{DMS} methods, the naive time of arrival of all \textit{eta} edges in the dynamic layer is updated based on the \ac{ETA} prediction to reflect the new predicted state of all robots. 
For \ac{IMS} methods, the actual arrival time is used instead of the \ac{ETA} prediction. 
% Based on either the \ac{ETA} prediction in case of \ac{DMS} or the actual arrival time in case of \ac{IMS}, the naive time of arrival of all \textit{eta} edges in the dynamic layer is updated to reflect the new state of all robots.
As this is a recurrent process, this is repeated for $T=T+1$ until all \textit{eta} edges have been decoded.

As in the first layer, the second layer does not rely on a central station to process any information. However, global information is requested from a centralized station during run-time in order to gather information about all other agents in the system. While not a completely decentralized solution, it does improve scale-ability in terms of distributing the computations to each agent.

%%%%%%%%%%%%%%%%%%%%%%%%%%%%%%%%%%%%%%%%%%%%%%%%%%%%%%%%%%%%%%%%%%%%%%%%%%
%%%%%%% COST SECTION %%%%%%%%%%%%%%%%%%%%%%%%%%%%%%%%%%%%%%%%%%%%%%%%%%%%%
%%%%%%%%%%%%%%%%%%%%%%%%%%%%%%%%%%%%%%%%%%%%%%%%%%%%%%%%%%%%%%%%%%%%%%%%%%
\subsection{Path selection and Validating time-constraints}

The path selection layer is the third layer of the proposed framework. 
This layer utilizes the \ac{ETA} to evaluate time constraints and calculate the cost of each suggested route. The layer is used to find a feasible route within a fixed time constraint and to validate the chosen route's impact on the rest of the system by utilizing the predicted \ac{ETA} for all agents. Various approaches can be used to evaluate a time-constrained path since the metric to select the best path depends on the specific application scenarios.
% and select what is considered the best path, but the choice of method depends on the specific application scenario.
In this work, a path is considered invalid if one or more of the time constraints are not met. To select between the valid paths, a cost function is proposed that considers the trade-off between the buffer times on the time constraints and the predicted arrival time. The cost function is defined as follows:

\begin{equation}
cost=\sum_{i=1}^{N}{\Bigl(\max{(\textit{TC})} - (\textit{TC}_{i} - \text{eta}_i)\Bigr)^2}
\end{equation}

\noindent where $\textit{TC}$ represents the set of all the time constraints, 
% $\textit{eta}_i$ is the i'th prediction in 
$\textit{eta}$ is the set of predicted \ac{ETA} for all $N$ robots. The cost function is designed to maximize the buffer time between the constraint time and the predicted arrival time. At the same time, it penalizes the paths with a shorter buffer time and provides more slack for the paths with a larger buffer time. This ensures that the path selection process balances the objective of meeting the time constraints with the goal of minimizing the overall time spent to complete all tasks.