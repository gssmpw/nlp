
\section{Problem Formulation}
\label{sec:problem_formulation}

% MAPF definition
This section presents the formal problem formulation addressed in this paper. The problem under consideration shares similarities with the \ac{MAPF} problem \cite{Mapf}, which involves determining conflict-free paths for multiple agents on a graph to reach their respective destinations. In the \ac{MAPF} problem, time is divided into discrete timesteps, during which agents execute atomic actions synchronously, such as moving to adjacent nodes or remaining in their current locations. However, the problem addressed in this paper deviates from the traditional \ac{MAPF} problem in two key aspects. Firstly, the objective is not solely focused on finding conflict-free paths for all agents but on minimizing overall travel time by reducing total conflicts and avoiding congested areas. Secondly, a time constraint is introduced for each agent, requiring them to reach their goals before a specific deadline. This introduces the concept of priority, where higher-priority agents are assigned shorter paths.
% Type of conflicts
The types of conflicts are explained in \cite{Mapf} and describes what kind of movement patterns are not allowed to be performed and are therefore considered conflicts. The ones considered in this paper are the following four conflicts: vertex conflicts, edge conflicts, swapping conflicts, and cycle conflicts. Vertex conflicts occur when agents occupy the same position at the same time. Edge conflicts occur when agents travel along or across the same edge. Swapping conflicts occur when two agents exchange positions and cycle conflicts occur when multiple agents form a cyclic movement pattern. The notion of conflicts used in \ac{MAPF} differs from the conflicts used in the proposed work. A conflict will be defined as when an agent must alter its global path to avoid any of the four aforementioned movement patterns defined by \ac{MAPF}. 
% Environment and noise
The environment is represented as a discrete occupancy grid map, which can be viewed as a graph according to the definition in \ac{MAPF}. However, accurately modeling noise in such a setting can be challenging, as a full action would need to be performed at each time step, resulting in noise having a significant impact within a single time step. To address this challenge and simplify the incorporation of noise, the movement of the agents is modeled to operate at full velocity, and as such, the inclusion of noise in the simulation would not result in increased speed. On the contrary, the presence of noise would slow down the movement of the agents, causing them to force a wait action.
% Communication
Furthermore, it is assumed that each agent has the capability of peer-to-peer communication to resolve local conflicts using a local planner. Additionally, agents have a global connection to a state database containing each agent's committed plans and their current locations.

The formal problem formulation presented in this section sets the stage for developing effective algorithms and strategies to address the specific challenges of minimizing travel time, incorporating time constraints and priority, handling different types of conflicts, considering noise and imperfect plan execution.

\section{Related work}
\label{sec:related_work}

\subsection{Traditional \ac{MAPF} solutions}
Global path planning algorithms for the \ac{MAPF} problem are a class of solvers that first perform all computations in a single continuous interval and return a plan for the agents to follow. These plans are generated before the agents begin to move, and the agents follow the plan without any additional computation. This means that the plan cannot contain parts where agents collide. Some global solvers, such as \ac{CBS} \cite{CBS}, aim to find optimal solutions according to a predefined cost function. However, these methods may not be able to scale up to larger systems due to the exponential growth of the search space as the number of agents increases \cite{yu2013structure}. Other global algorithms, the \ac{HCA*}\cite{HCA*} and \ac{PIBT}\cite{PIBT}, sacrifice optimality or completeness in order to reduce computation time by using a spatiotemporal reservation table to coordinate agents and avoid collisions. 
A major weakness in global path planning algorithms is that agents frequently have imperfect plan execution capabilities and cannot perfectly synchronize their motions, leading to frequent and time-consuming replanning \cite{ma2017overview}. This is addressed in \cite{honig2016multi}, where a post-processing framework is proposed, using a simple temporal graph to establish a plan-execution scheduler that guarantees safe spacing between robots. The framework exploits the slack in time constraints to absorb some of the imperfect plan executions and prevent time-intensive replanning.
The work of \cite{MU} extends the method presented in \cite{M*} for a multi-agent path planner called uncertainty M* that considers collision likelihood to construct the belief space for each agent. However, this does not guarantee to remove conflicts caused by imperfect plan execution, so a local path planning algorithm is commonly needed for solving these conflicts when they occur.

Local path planning algorithms are a class of solvers that compute partial solutions in real-time, allowing agents to adjust their plans as they execute them. A simple approach for local \ac{MAPF} is \ac{LRA*} \cite{LRA*}, which plans paths for each agent while ignoring other agents. Once the agents start moving, conflicts are resolved locally by constructing detours or repairs for some agents. Another notable local \ac{MAPF} solver is the \ac{WHCA*} algorithm \cite{HCA*}, which is a local variant of the \ac{HCA*} algorithm. \ac{WHCA*} uses a spatiotemporal reservation table to coordinate agents, but only reserves limited paths, splitting the problem into smaller sections. As agents follow the partially-reserved paths, a new cycle begins from their current locations. In the traditional \ac{WHCA*}, a different ordering of agents is used in each cycle to allow a balanced distribution of the reservation table. An extension of the \ac{WHCA*} is proposed in \cite{COWHCA*}, where a priority is computed based on minimizing future conflicts, improving the success rate, and lowering the computation time. Learning-based methods for local planning are also showing promising results. In \cite{robotics11050109}, an end-to-end local planning method is introduced, using reinforcement learning to generate safe pathing in dense environments. \cite{9424371} introduces the use of \ac{GNN} for local communication and imitation learning for learning the conflict resolution of \ac{CBS} in multi-agent environments.

% \subsection{Spatio-temporal graph neural networks}
% A spatio-temporal graph combines spatio representing space or structural information and temporal representing time-varying information. 
% %Accordingly, if any system consists of a structural relationship between space and time, it can be represented as a spatio-temporal graph. These graphs can be used to design a neural network capable of handling static structures and time-varying characteristics. 

% To work with spatio-temporal graphs, we must process a sequence of graph data to build a spatio-temporal embedding that may be used for regression, classification, clustering, or correlation prediction\cite{malla2021social}. The spatial block can be any conventional \ac{GNN} \cite{battaglia2018relational}, while the temporal block can be any approach for learning over sequences of data, such as temporal convolution\cite{temporal-conv, temporal-conv-2} or temporal attention\cite{temporal-attention}. 
% Various spatio-temporal \acp{GNN}  conform to the encoding-processing-decoding paradigm \cite{GOOGLE_ETA,battaglia2018relational, keisler2022forecasting}. This paradigm accommodates the iterative nature of \ac{STSF} and pathfinding algorithms. 
% Improving these alignments has been shown to improve \acp{GNN} generalization to a more diverse distribution of graphs\cite{velivckovic2019neural}.
% 

\subsection{\ac{ETA} prediction and spatio-temporal sequence forecasting}
\label{sec:IMS-vs-DMS}
In the field of \ac{STSF}, \cite{cox1961prediction, chevillon2007direct} investigated two major learning strategies: \ac{IMS} estimation and \ac{DMS} estimation.  
The \ac{IMS} strategy trains a one-step-ahead forecasting model and iteratively uses its generated samples to produce multi-step-ahead forecasts. This strategy offers simplicity in training and flexibility in generating predictions of any length. \cite{DCRNN} demonstrated improved forecasting accuracy by incorporating graph structure into the \ac{IMS} approach.

However, the \ac{IMS} strategy suffers from the issue of accumulated forecasting errors between the training and testing phases \cite{forecast_survey}. To address this disparity, \cite{chevillon2007direct} introduced \ac{DMS} estimation directly minimizes the long-term prediction error by training distinct models for each forecasting horizon. This approach avoids error accumulation and can support multiple internal models for different horizons. Additionally, recursive application of the one-step-ahead forecaster is employed to construct multi-step-ahead forecasts, decoupling model size from the number of forecasting steps.

Although \ac{DMS} offers advantages over \ac{IMS}, it comes with increased computational complexity \cite{chevillon2007direct}. Multiple models need to be stored and trained in multi-model \ac{DMS}, while recursive \ac{DMS} requires applying the one-step-ahead forecaster for multiple steps. These factors result in greater memory storage requirements and longer training times compared to the \ac{IMS} method. On the other hand, \cite{Goodfellow-et-al-2016} shows that the \ac{IMS} training process lends itself to parallelization as each forecasting horizon can be trained independently. 

Several related works have leveraged the \ac{DMS} approach for spatio-temporal forecasting. For instance, \cite{GOOGLE_ETA} proposed a homogeneous spatio-temporal \ac{GNN} method for predicting \ac{ETA} by combining recursive \ac{DMS} and multi-model \ac{DMS}. In \cite{huang2022dueta}, a congestion-sensitive graph structure was introduced to model traffic congestion propagation, along with a route-aware graph transformer layer to capture interactions between spatially distant but correlated road segments. Furthermore, \cite{hong2020heteta} proposed a novel heterogeneous graph structure that incorporates road features, historical data, and temporal information at different scales, utilizing temporal and graph convolutions for learning spatio-temporal representations.

However, the existing methods mentioned above primarily focus on road features and consider a single vehicle traversing the graph, neglecting other types of vehicles and the influence of driver route choices on traffic conditions. Consequently, these models cannot readily be extended to multi-vehicle scenarios.
% OLD section before shortening and rearrange:
% The \ac{STSF} problem aims to predict a sequence of length $L$ into the future based on past observations and auxiliary data. The general learning strategies for multi-step forecasting fall into two major categories: \ac{IMS} estimation and \ac{DMS} estimation\cite{forecast_survey, chevillon2007direct}. The \ac{IMS} strategy trains a one-step-ahead forecasting model and iteratively feeds the generated samples to the one-step-ahead forecaster to produce a multi-step-ahead forecast.
% There are two main benefits of the \ac{IMS} strategy: the one-step-ahead forecaster is simple to train because it only needs to consider the one-step-ahead forecasting error and can be used to generate predictions of any length. The work of \cite{DCRNN} demonstrates how accounting for graph structure significantly reduces forecasting error across multiple horizons.

% However, there is a disparity between the training and testing phase in \ac{IMS}.
% In the training phase, the $L$th-step prediction is generated using the ground truths of the previous $L - 1$ steps, while in the testing phase, the model is provided with its previous prediction of the $L - 1$ steps rather than ground truths. This disparity makes the model susceptible to accumulating forecasting errors \cite{bengio2015scheduled}.

% \ac{DMS} circumvent the accumulated error issue seen in \ac{IMS} by directly minimizing the long-term prediction error. Rather than training a single model, \ac{DMS} trains a distinct model $m_h$ for each forecasting horizon h. Therefore, the \ac{DMS} approach can support up to $L$ internal models. In order to separate the model size from the number of forecasting steps $L$, we can also construct $m_h$ by recursively applying the one-step-ahead forecaster $m_1$ in order to compute the multi-step-ahead forecasting error. It is worth noting that a multi-step-ahead forecasting error differs from a one-step-ahead forecasting error. Multi-step-ahead forecasting error accumulates during the sequence, while one-step-ahead forecasting error only minimizes for one step.

% Using \ac{DMS} does not come without cost\cite{chevillon2007direct}. \ac{DMS} is more computationally expensive than \ac{IMS}. For multi-model DMS, $L$ models need to be stored and trained, while recursive \ac{DMS} needs to apply $m_1$ for $\mathcal{O}(L)$ steps. Both cases require a large amount of memory storage or longer training time than solving the \ac{IMS} method. On the other hand, the \ac{IMS} training process is easily parallelizable since each forecasting horizon can be trained independently\cite{Goodfellow-et-al-2016}.

% In\cite{GOOGLE_ETA}, a homogeneous spatio-temporal \ac{GNN} method for predicting \ac{ETA} is proposed. While the main architecture consists of basic \ac{GNN} building blocks, they propose utilizing a combination of the recursive \ac{DMS} and multi-model \ac{DMS} to improve long horizon predictions. 
% In \cite{huang2022dueta}, a focus on modeling traffic congestion propagation into a congestion-sensitive graph structure is proposed.

% In addition, a route-aware graph transformer layer is proposed enabling \acp{GNN} to capture the interactions between any road segments that are spatially distant but highly correlated with traffic conditions.
% A novel heterogeneous graph structure is proposed in \cite{hong2020heteta}, which models both road features and historical data as heterogeneous information. It introduces edge features containing structural information on the relationship between nodes. In addition, three components are introduced to model temporal information from recent periods, daily periods, and weekly periods respectively. Each component comprises temporal convolutions and graph convolutions to learn representations of the spatio-temporal heterogeneous information. 

% The methods mentioned above focus mainly on the features of the road and consider only a single vehicle traversing the graph. None of the models consider other types of vehicles driving in the environment or how the traffic is influenced due to the driver choosing the route. As a result, such models are not easily converted into multi-vehicle solutions
% 