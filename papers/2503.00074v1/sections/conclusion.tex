
\section{Conclusion and future work}
\label{sec:conclusion}
% In this work, we conduct pioneering research by developing a framework for predicting conflicts and estimating the time of arrival in a multi-agent environment. We formulate our problem as a spatio-temporal graph focusing on edge prediction. This allows for the estimated time of arrival prediction of all robots simultaneously, which previous work has not been able to accomplish. While this is the first study in the field on systemwide estimated time of arrival, it has been demonstrated to increase the accuracy of the predicted time of arrival. As a result, the overall framework's capacity to accurately evaluate the cost of each route improves. 

% In future work, a couple of features could be incorporated into the current framework. Firstly, incorporate robot acceleration to reflect the real-world dynamics more accurately. Secondly, include multiple types of robots. As we are representing the problem as a heterogeneous graph, incorporating dynamics, constraints, and different types of robots into nodes is very feasible. 
% Lastly, incorporating the naive path of a conflict zone into \textit{eta} edge features. The current representation only contains the naive duration of which a robot is inside a conflict zone. However, this neglects the scenario of two robots going in the same direction and never entering a conflict. Therefore, incorporating this information on the order of steps performed inside a conflict zone makes a more accurate conflict prediction possible.

% As future work, we plan to incorporate a more accurate robot motion model as well as turn the simulation into continuous space in order to reflect real-world dynamics more accurately. We will also include multiple types of robots in the same environment. As we represent the problem as a heterogeneous graph, incorporating dynamics, constraints, and different types of robots into nodes will be feasible. 
% Lastly, since the proposed method does not take into account the path traversed inside a floor tile, but only the duration, we will incorporate the path as a grid feature attribute in \textit{eta} edges in the future.


In this work, we develop a framework to predict conflicts and \ac{ETA} in multi-agent environments. We formulate our problem as a spatio-temporal graph focusing on edge prediction. The proposed methodology allows \ac{ETA} prediction of all robots simultaneously, which was not possible by the previously published works. Through extensive simulation experiments, the proposed method demonstrates an increase in the accuracy of the predicted arrival time. 
% While this is the first study on the system-wide estimated time of arrival, we also 
It should be noted that our proposed framework does not solve the \ac{MAPF} problem as it does not provide collision-free paths for each agent. Instead, we employ the use of \ac{ETA} in order to minimize the number of conflicts that a local path-planning algorithm needs to resolve, resulting in a more resilient method that is better equipped to handle noise.

As future work, we aim to enhance the framework by incorporating a more sophisticated robot motion model and transitioning to a continuous space simulation to better reflect real-world dynamics. Additionally, we plan to extend the framework to include multiple types of robots in a heterogeneous graph representation, incorporating dynamics, constraints, and different types of robots as nodes. Furthermore, we will incorporate the path traversed within a floor tile as a feature attribute in the \textit{eta} edges.