% This study proposes a conflict-aware multi-agent estimate time of arrival (CAMAETA) framework for predicting the time of arrival of all agents in environments without any predefined road infrastructure. 
% The framework comprises of three layers: a traditional path planning algorithm for suggesting paths, a path evaluation layer for predicting multi-agent estimate time of arrival times for all agents, and lastly a cost layer for computing the best overall path suggested by the first layer. The novelty of the framework lies in the heterogeneous map representation along with a novel heterogeneous graph neural network architecture for multi-agent arrival time estimation. Unlike existing methods that rely on structured road infrastructure and historical data to predict time of arrival, the proposed framework utilizes total path length and basic structural features, resulting in improved generalization capabilities. To the best of our knowledge, this is the first attempt to predict multi-agent arrival times without structured road infrastructure. The results of extensive simulations demonstrate the accuracy and effectiveness of the proposed method, and show improvements in mean average percentage error of 29.5\% and 44\% when compared to a naive method that does not consider conflicts.
This study presents the conflict-aware multi-agent estimated time of arrival (CAMETA) framework, a novel approach for predicting the arrival times of multiple agents in unstructured environments without predefined road infrastructure. The CAMETA framework consists of three components: 
a path planning layer generating potential path suggestions,
a multi-agent ETA prediction layer predicting the arrival times for all agents based on the paths,
and lastly, a path selection layer that calculates the accumulated cost and selects the best path.
The novelty of the CAMETA framework lies in the heterogeneous map representation and the heterogeneous graph neural network architecture. As a result of the proposed novel structure, CAMETA improves the generalization capability compared to the state-of-the-art methods that rely on structured road infrastructure and historical data. The simulation results demonstrate the efficiency and efficacy of the multi-agent ETA prediction layer, with a mean average percentage error improvement of 29.5\% and 44\% when compared to a traditional path planning method ($A^*$) which does not consider conflicts. The performance of the CAMETA framework shows significant improvements in terms of robustness to noise and conflicts as well as determining proficient routes compared to state-of-the-art multi-agent path planners.



%The framework consists of a novel heterogeneous map representation, traditional path planning algorithm, and novel heterogeneous graph neural network architecture for multi-agent arrival time estimation. Instead of relying on the structured road infrastructure and historical data to predict the time of arrival, the proposed framework benefits from the total path length and basic spatial features, which has better generalization capability. To the best of our knowledge, this work is the first attempt to predict multi-agent arrival times without any structured road infrastructure. The extensive simulation results illustrate the accuracy and efficacy of the proposed method, demonstrating its feasibility and effectiveness. Compared to a naive method that does not consider conflict, the best-trained iterative multi-step model improves the overall mean average percentage error by 29.5\%, while the best direct multi-step model improves by 44\%.



% %As the popularity of autonomous mobile robot applications is rising in the industry, conflict-aware multi-agent path planning becomes essential for a smooth and efficient operation of autonomous robots in highly dynamic environments.
% This study proposes a framework for a Conflict Aware Multi-Agent Global Planner (CAMAGP), which combines novel heterogeneous map representation, traditional path-planning algorithms, and a novel heterogeneous graph neural network architecture for multi-agent arrival time estimation. 
% % The proposed framework relies not on historical data but on naive path distance along with the structural features of the path. As a result, the framework does not depend on structured road infrastructure in order to predict the arrival time.
% Contrary to other methods, which rely on structured road infrastructure and historical data to predict the time of arrival, the proposed framework only relies on the total path length and basic structural features, thus making it more general in its application.
% % Since the proposed framework does not rely on historical data but uses naive path distance and structural features of the path, the framework does not depend on a structured road infrastructure to predict the arrival time.
% To the best of our knowledge, this work is the first attempt to predict multi-agent arrival times without any structured road infrastructure. 
% Compared to a naive method that does not consider conflict, the best trained iterative multi-step model improves the overall mean average percentage error by 29.5\%, while the best direct multi-step model improves by 44\%.
% The results illustrate the accuracy and efficacy of the proposed method, demonstrating its feasibility and effectiveness.

% % NOTE:
% % Since the first sentence of the paragraph will be deleted, you will have some extra space in your abstract. How about keeping this sentence but adding a few details to your results by adding one more sentence before this sentence? you can write one or two significant discussions for your results. For example: 
% % "The best trained IMS model improves the average MAPE by 29.5% compared to the naive method, while the best DMS model improves by 44%"
