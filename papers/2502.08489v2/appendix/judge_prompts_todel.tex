We use existing evaluation datasets and rephrase some of their instances into queries written in a more natural way. We do this by means of templates, including three different templates for each source instance in order to measure the robustness of the agents to changes in prompting. All templates were written in English and Spanish, and then translated into other languages by native speakers. Our selection of source datasets tries to maximize parallelism across languages, though some gaps are present.

To save on computational resources and standardize the number of instances used for evaluation across tasks, we randomly pick 250 instances per source dataset and language from the \emph{test} split. The only difference being in the Translation task, for which we only take 50 parallel instances for the \emph{translation from language} subtask and 50 different parallel instances for the \emph{translation into language} subtask. We do this because each language is paired with the rest of the languages for each instance—and subtask, quickly growing the number of instances given that we cover eight languages.

What follows is a summary of the datasets used as source for each task, an example of three prompts resulting from the template-based processing in English, and an example of these prompts in one of the other languages we evaluate:

\paragraph{Common-sense reasoning.} We take the Spanish, Basque and English subsets from the original XStoryCloze \citep{lin-etal-2022-shot}, and the Catalan and Galician\footnote{The Galician translation is not part of the original release of IberoBench, but is available here: \url{https://huggingface.co/datasets/proxectonos/xstorycloze_gl}.} translations from \hbox{IberoBench \citep{iberobench-coling-2025}}. 


%\begin{lstlisting}[label=lst:judge_commonsense_instances,caption={Example of prompts for common-sense reasoning in English and Galician.}]
(*@\textbf{\normalsize English.}@*)
Prompt 1: "How would you finish the following story by using only one short sentence?\nLorraine was a softball player. She was the best on her team. One day, during a game, she twisted her ankle badly. She wasn't able to play anymore."
Prompt 2: "What would be the final sentence of this story? Please make it short: Lorraine was a softball player. She was the best on her team. One day, during a game, she twisted her ankle badly. She wasn't able to play anymore."
Prompt 3: "Can you end this story with just one short sentence?\n\nLorraine was a softball player. She was the best on her team. One day, during a game, she twisted her ankle badly. She wasn't able to play anymore."

(*@\textbf{\normalsize Galician.}@*)
Prompt 1: "Como terminarías esta historia cunha soa frase curta?\nA Juan encántalle comer en bufés. Son os seus favoritos porque podes comer todo o que queiras. Preto do seu apartamento, había un bo bufete ao que ía. Descubriu que o negocio pechara a semana pasada."
Prompt 2: "Cal sería a frase final desta historia? Por favor, faino breve: A Juan encántalle comer en bufés. Son os seus favoritos porque podes comer todo o que queiras. Preto do seu apartamento, había un bo bufete ao que ía. Descubriu que o negocio pechara a semana pasada."
Prompt 3: "Podes terminar esta historia cunha soa frase?\n\nA Juan encántalle comer en bufés. Son os seus favoritos porque podes comer todo o que queiras. Preto do seu apartamento, había un bo bufete ao que ía. Descubriu que o negocio pechara a semana pasada."

\end{lstlisting}

\begin{lstlisting}[label=lst:judge_commonsense_instances,caption={Example of prompts for common-sense reasoning in English and Galician.}]
(*@\textbf{\normalsize English.}@*)
Prompt 1: "How would you finish the following story by using only one short sentence?\nLorraine was a softball player. She was the best on her team. One day, during a game, she twisted her ankle badly. She wasn't able to play anymore."
Prompt 2: "What would be the final sentence of this story? Please make it short: Lorraine was a softball player. She was the best on her team. One day, during a game, she twisted her ankle badly. She wasn't able to play anymore."
Prompt 3: "Can you end this story with just one short sentence?\n\nLorraine was a softball player. She was the best on her team. One day, during a game, she twisted her ankle badly. She wasn't able to play anymore."

(*@\textbf{\normalsize Galician.}@*)
Prompt 1: "Como terminarías esta historia cunha soa frase curta?\nA Juan encántalle comer en bufés. Son os seus favoritos porque podes comer todo o que queiras. Preto do seu apartamento, había un bo bufete ao que ía. Descubriu que o negocio pechara a semana pasada."
Prompt 2: "Cal sería a frase final desta historia? Por favor, faino breve: A Juan encántalle comer en bufés. Son os seus favoritos porque podes comer todo o que queiras. Preto do seu apartamento, había un bo bufete ao que ía. Descubriu que o negocio pechara a semana pasada."
Prompt 3: "Podes terminar esta historia cunha soa frase?\n\nA Juan encántalle comer en bufés. Son os seus favoritos porque podes comer todo o que queiras. Preto do seu apartamento, había un bo bufete ao que ía. Descubriu que o negocio pechara a semana pasada."

\end{lstlisting}

\newpage
\paragraph{Mathematics.} We take the Spanish, German, French and English subsets from the original MGSM \citep{shi2023language}, and the Catalan, Galician and Basque translations from IberoBench \citep{iberobench-coling-2025}. 

\begin{lstlisting}[label=lst:judge_math_instances,caption={Example of prompts for mathematics in English and Catalan.}]
(*@\textbf{\normalsize English.}@*)
Prompt 1: "I need help with this math problem: \"Janet’s ducks lay 16 eggs per day. She eats three for breakfast every morning and bakes muffins for her friends every day with four. She sells the remainder at the farmers' market daily for $2 per fresh duck egg. How much in dollars does she make every day at the farmers' market?\" Give me the answer step by step and also the final result separately."
Prompt 2: "Can you please help me answer this? \"Janet’s ducks lay 16 eggs per day. She eats three for breakfast every morning and bakes muffins for her friends every day with four. She sells the remainder at the farmers' market daily for $2 per fresh duck egg. How much in dollars does she make every day at the farmers' market?\" Explain the answer and give me the final result as well. Thanks."
Prompt 3: "Help me with this problem: \"Janet’s ducks lay 16 eggs per day. She eats three for breakfast every morning and bakes muffins for her friends every day with four. She sells the remainder at the farmers' market daily for $2 per fresh duck egg. How much in dollars does she make every day at the farmers' market?\" I need the answer explained and the final result separately."

(*@\textbf{\normalsize Catalan.}@*)
Prompt 1: "Necessito ajuda amb aquest problema de matemàtiques: \"Un nou programa va tenir 60 descàrregues el primer mes. El nombre de descàrregues el segon mes va ser el triple que el primer mes, però el tercer mes es van reduir en un 30%. Quantes descàrregues ha tingut el programa en total durant els tres mesos?\" Dóna'm la resposta pas a pas i també el resultat final a part."
Prompt 2: "Pots ajudar-me a respondre a això, si us plau? \"Un nou programa va tenir 60 descàrregues el primer mes. El nombre de descàrregues el segon mes va ser el triple que el primer mes, però el tercer mes es van reduir en un 30%. Quantes descàrregues ha tingut el programa en total durant els tres mesos?\" Explica'm la resposta i dóna'm el resultat final també. Gràcies."
Prompt 3: "Ajuda'm amb aquest problema: \"Un nou programa va tenir 60 descàrregues el primer mes. El nombre de descàrregues el segon mes va ser el triple que el primer mes, però el tercer mes es van reduir en un 30%. Quantes descàrregues ha tingut el programa en total durant els tres mesos?\" Necessito la resposta explicada i el resultat final per separat."
\end{lstlisting}

\paragraph{Paraphrasing.} We take the Spanish, German, French and English subsets from the original PAWS-X \citep{yang-etal-2019-paws}, and the Catalan and Galician translations from IberoBench \citep{iberobench-coling-2025}.

\begin{lstlisting}[label=lst:judge_paraphr_instances,caption={Example of prompts for paraphrasing in English and Spanish.}]
(*@\textbf{\normalsize English.}@*)
Prompt 1: "Write a sentence with a similar meaning to \"In 2014 the site launched iOS and Android applications for product search; product features include interactive video product reviews with live question-and-answer sessions.\""
Prompt 2: "Please paraphrase this sentence for me: \"In 2014 the site launched iOS and Android applications for product search; product features include interactive video product reviews with live question-and-answer sessions.\""
Prompt 3: "Create a sentence with the same meaning as \"In 2014 the site launched iOS and Android applications for product search; product features include interactive video product reviews with live question-and-answer sessions.\""(*@\newpage@*)
(*@\textbf{\normalsize Spanish.}@*)
Prompt 1: "Escribe una oración con un significado similar a \"Tres años más tarde, ganó una medalla de plata en la misma competencia en el campeonato europeo en Hahnenklee para Alemania occidental.\""
Prompt 2: "Por favor, parafrasea esta oración: \"Tres años más tarde, ganó una medalla de plata en la misma competencia en el campeonato europeo en Hahnenklee para Alemania occidental.\""
Prompt 3: "Crea una oración con el mismo significado que \"Tres años más tarde, ganó una medalla de plata en la misma competencia en el campeonato europeo en Hahnenklee para Alemania occidental.\""
\end{lstlisting}

\paragraph{Translation.} We take the Catalan, Spanish, Galician, Basque, German, Italian, French and English subsets from FLORES-200 \citep{nllbteam2022language}. There are two translation subtasks:
\begin{itemize}
    \item \emph{from\_lang} includes prompts that request to translate a sentence in language $x$ into multiple languages $Y$ using language $x$ for it.
    \item \emph{into\_lang} includes prompts that request to translate a sentence in language $x$ into multiple languages $Y$ using language $y$ for it.
\end{itemize}

\begin{lstlisting}[label=lst:judge_transfrom_instances,caption={Example of prompts for translation from a language (\emph{from\_lang}) for the English-German pair of languages.}]
(*@\textbf{\normalsize English.}@*)
Prompt 1: "Please translate \"Police said that the body appeared to have been there for about a day.\" into German."
Prompt 2: "How would you translate \"Police said that the body appeared to have been there for about a day.\" into German?"
Prompt 3: "Convert this sentence \"Police said that the body appeared to have been there for about a day.\" into German while maintaining its meaning."

(*@\textbf{\normalsize German.}@*)
Prompt 1: "Bitte übersetzen Sie „Die Polizei sagte, der Körper schien seit etwa einem Tag dort gelegen zu haben.(*@“@*) ins Englische."
Prompt 2: "Wie würdest Du „Die Polizei sagte, der Körper schien seit etwa einem Tag dort gelegen zu haben.(*@“@*) ins Englische übersetzen?"
Prompt 3: "Konvertieren Sie diesen Satz „Die Polizei sagte, der Körper schien seit etwa einem Tag dort gelegen zu haben.(*@“@*) ins Englische und behalten Sie dabei seine Bedeutung bei."
\end{lstlisting}

\begin{lstlisting}[label=lst:judge_transinto_instances,caption={Example of prompts for translation into a language (\emph{into\_lang}) for the English-Italian pair of languages.}]
(*@\textbf{\normalsize English.}@*)
Prompt 1: "Please translate \"In Giappone, la cultura del lavoro è più gerarchica e formale rispetto a quella a cui gli occidentali tendono ad essere abituati.\" into English."
Prompt 2: "What's the meaning of \"In Giappone, la cultura del lavoro è più gerarchica e formale rispetto a quella a cui gli occidentali tendono ad essere abituati.\" in English?"
Prompt 3: "Convert this sentence \"In Giappone, la cultura del lavoro è più gerarchica e formale rispetto a quella a cui gli occidentali tendono ad essere abituati.\" into English while maintaining its meaning."

(*@\textbf{\normalsize Italian.}@*)
Prompt 1: "Per favore traduci \"Japanese work culture is more hierarchical and formal that what Westerners may be used to.\" in italiano."
Prompt 2: "Qual è il significato di \"Japanese work culture is more hierarchical and formal that what Westerners may be used to.\" in italiano?"
Prompt 3: "Converti questa frase \"Japanese work culture is more hierarchical and formal that what Westerners may be used to.\" in italiano mantenendone il significato."
\end{lstlisting}

\paragraph{Reading comprehension.} We take the Catalan, Spanish, Galician, Basque, German, Italian, French and English subsets from Belebele \citep{bandarkar-etal-2024-belebele}.

\begin{lstlisting}[label=lst:judge_readcompr_instances,caption={Example of prompts for reading comprehension in English and Basque.}]
(*@\textbf{\normalsize English.}@*)
Prompt 1: "Answer the question based on this passage:\nPassage: Every year around October nearly 1.5 million herbivores travel towards the southern plains, crossing the Mara River, from the northern hills for the rains. And then back to the north through the west, once again crossing the Mara river, after the rains in around April. The Serengeti region contains the Serengeti National Park, the Ngorongoro Conservation Area and Maswa Game Reserve in Tanzania and the Maasai Mara National Reserve in Kenya.\nQuestion: Which area do herbivores depart from sometime around April?\nPlease provide a very short answer."
Prompt 2: "Every year around October nearly 1.5 million herbivores travel towards the southern plains, crossing the Mara River, from the northern hills for the rains. And then back to the north through the west, once again crossing the Mara river, after the rains in around April. The Serengeti region contains the Serengeti National Park, the Ngorongoro Conservation Area and Maswa Game Reserve in Tanzania and the Maasai Mara National Reserve in Kenya.\nBased on the previous text, answer to this question in as few words as possible: \"Which area do herbivores depart from sometime around April?\""
Prompt 3: "Use the information in the following text to provide a concise answer to the question below.\n\nEvery year around October nearly 1.5 million herbivores travel towards the southern plains, crossing the Mara River, from the northern hills for the rains. And then back to the north through the west, once again crossing the Mara river, after the rains in around April. The Serengeti region contains the Serengeti National Park, the Ngorongoro Conservation Area and Maswa Game Reserve in Tanzania and the Maasai Mara National Reserve in Kenya.\n\nThe question is: \"Which area do herbivores depart from sometime around April?\""

(*@\textbf{\normalsize Basque.}@*)
Prompt 1: "Galdera erantzun pasarte honetan oinarrituta:\nPasartea: Intsektuak izan ziren hegan egiten hasi ziren lehenak. Hegan egiteko gaitasunari esker, etsaiak errazago saihestu eta janaria eta ugaltzeko kideak modu eraginkorragoan aurkitzen zituzten. Intsektu gehienek hegalak gorputzaren atzean tolesteko abantaila dute. Horri esker, toki txiki gehiagotan ezkutatzeko aukera dute harrapakariek ez hartzeko. Gaur egun, hegoak tolestu ezin dituzten intsektu bakarrak sorgin-orratzak eta efemeropteroak dira.\nGaldera: Zer handitzen du intsektuek hegalak tolesteko duten gaitasunak?\nErantzun oso labur bat eman, mesedez."
Prompt 2: "Intsektuak izan ziren hegan egiten hasi ziren lehenak. Hegan egiteko gaitasunari esker, etsaiak errazago saihestu eta janaria eta ugaltzeko kideak modu eraginkorragoan aurkitzen zituzten. Intsektu gehienek hegalak gorputzaren atzean tolesteko abantaila dute. Horri esker, toki txiki gehiagotan ezkutatzeko aukera dute harrapakariek ez hartzeko. Gaur egun, hegoak tolestu ezin dituzten intsektu bakarrak sorgin-orratzak eta efemeropteroak dira.\nAurreko textuan oinarrituz, galdera honi erantzun ahalik eta hitz kopuru gutxien erabiliz: \"Zer handitzen du intsektuek hegalak tolesteko duten gaitasunak?\""
Prompt 3: "Hurrengo textuko informazioa erabili hurrengo galderari erantzun zehatza emateko.\n\nIntsektuak izan ziren hegan egiten hasi ziren lehenak. Hegan egiteko gaitasunari esker, etsaiak errazago saihestu eta janaria eta ugaltzeko kideak modu eraginkorragoan aurkitzen zituzten. Intsektu gehienek hegalak gorputzaren atzean tolesteko abantaila dute. Horri esker, toki txiki gehiagotan ezkutatzeko aukera dute harrapakariek ez hartzeko. Gaur egun, hegoak tolestu ezin dituzten intsektu bakarrak sorgin-orratzak eta efemeropteroak dira.\n\nGaldera hau da: \"Zer handitzen du intsektuek hegalak tolesteko duten gaitasunak?\""
\end{lstlisting}

\paragraph{Summarization.} We take the Spanish, French and English subsets from XLSum \citep{hasan-etal-2021-xl}, the Catalan instances from caBreu \citep{gonzalez-agirre-etal-2024-building-data} and the Galician instances from the summarization\_gl task in IberoBench \citep{iberobench-coling-2025}.

\begin{lstlisting}[label=lst:judge_summar_instances,caption={Example of prompts for paraphrasing in English and French.}]
(*@\textbf{\normalsize English.}@*)
Prompt 1: "Summarize this text please:\nCCTV systems, routers, digital video recorders and other internet-of-things (IoT) devices are now believed to be harbouring the Hajime worm. The fast-moving worm is currently outpacing malicious equivalents seeking the same vulnerable gear. Security researchers say they do not know who created Hajime or how it might ultimately be used. Attack code Hajime was first discovered in October 2016 and, said security researchers, had been hunting down IoT devices with security vulnerabilities that could be exploited by a different worm, called Mirai. Earlier the same month, a network of devices compromised by Mirai was responsible for knocking offline high-profile websites including Twitter, Spotify and Reddit. Modest estimates suggested Hajime was now present on \"tens of thousands\" of devices, wrote Symantec researcher Waylon Grange in a blog. Programs such as Hajime and Mirai must keep scouring the net for victims, because switching off a vulnerable device generally cleans out the infection. Mr Grange noted that Hajime currently had no attack code built in so could not be used to mount the kinds of attacks Mirai had been implicated in. The only action taken by Hajime is to regularly display a message from the worm's author on the internal interface for each device. The message says, among other things: \"Just a white hat, securing some systems.\" The term \"white hat\" is typically applied to those hackers seeking to secure rather than exploit vulnerabilities. Malicious or criminal hackers are known as \"black hats\". \"There is a question around trusting that the author is a true white hat and is only trying to secure these systems, as they are still installing their own backdoor on the system,\" wrote Mr Grange. He added if the author's intentions changed they could \"potentially\" turn the infected devices into a \"massive\" attack network."
Prompt 2: "Provide a summary of this text: CCTV systems, routers, digital video recorders and other internet-of-things (IoT) devices are now believed to be harbouring the Hajime worm. The fast-moving worm is currently outpacing malicious equivalents seeking the same vulnerable gear. Security researchers say they do not know who created Hajime or how it might ultimately be used. Attack code Hajime was first discovered in October 2016 and, said security researchers, had been hunting down IoT devices with security vulnerabilities that could be exploited by a different worm, called Mirai. Earlier the same month, a network of devices compromised by Mirai was responsible for knocking offline high-profile websites including Twitter, Spotify and Reddit. Modest estimates suggested Hajime was now present on \"tens of thousands\" of devices, wrote Symantec researcher Waylon Grange in a blog. Programs such as Hajime and Mirai must keep scouring the net for victims, because switching off a vulnerable device generally cleans out the infection. Mr Grange noted that Hajime currently had no attack code built in so could not be used to mount the kinds of attacks Mirai had been implicated in. The only action taken by Hajime is to regularly display a message from the worm's author on the internal interface for each device. The message says, among other things: \"Just a white hat, securing some systems.\" The term \"white hat\" is typically applied to those hackers seeking to secure rather than exploit vulnerabilities. Malicious or criminal hackers are known as \"black hats\". \"There is a question around trusting that the author is a true white hat and is only trying to secure these systems, as they are still installing their own backdoor on the system,\" wrote Mr Grange. He added if the author's intentions changed they could \"potentially\" turn the infected devices into a \"massive\" attack network."
Prompt 3: "Explain this text in one sentence:\n\nCCTV systems, routers, digital video recorders and other internet-of-things (IoT) devices are now believed to be harbouring the Hajime worm. The fast-moving worm is currently outpacing malicious equivalents seeking the same vulnerable gear. Security researchers say they do not know who created Hajime or how it might ultimately be used. Attack code Hajime was first discovered in October 2016 and, said security researchers, had been hunting down IoT devices with security vulnerabilities that could be exploited by a different worm, called Mirai. Earlier the same month, a network of devices compromised by Mirai was responsible for knocking offline high-profile websites including Twitter, Spotify and Reddit. Modest estimates suggested Hajime was now present on \"tens of thousands\" of devices, wrote Symantec researcher Waylon Grange in a blog. Programs such as Hajime and Mirai must keep scouring the net for victims, because switching off a vulnerable device generally cleans out the infection. Mr Grange noted that Hajime currently had no attack code built in so could not be used to mount the kinds of attacks Mirai had been implicated in. The only action taken by Hajime is to regularly display a message from the worm's author on the internal interface for each device. The message says, among other things: \"Just a white hat, securing some systems.\" The term \"white hat\" is typically applied to those hackers seeking to secure rather than exploit vulnerabilities. Malicious or criminal hackers are known as \"black hats\". \"There is a question around trusting that the author is a true white hat and is only trying to secure these systems, as they are still installing their own backdoor on the system,\" wrote Mr Grange. He added if the author's intentions changed they could \"potentially\" turn the infected devices into a \"massive\" attack network."

(*@\textbf{\normalsize French.}@*)
Prompt 1: "Résumez ce texte s'il vous plaît:\nLa France cale l'Argentine Liesse des Bleus français après le but de Kylian Mbappé. Les Français ont ouvert les hostilités dès le début de la partie, avec une accélération à la 11e minute de Kylian Mbappé qui est fauché dans la surface de réparation. Lire aussi : Le pénalty est transformé à la 13e minute de jeu par Antoine Griezmann. La France mène 1-0 au score. A la 18e minute, Mbappé accélère encore et se fait faucher juste à la limite de la surface de réparation. Le tir mal ajusté de Paul Pogba passe au-dessus de la barre transversale. Angel Di Maria, le joueur du Paris Saint-Germain, d'une frappe du gauche envoie le ballon à la 40e minute dans la lucarne des cages gardées par Hugo Lloris (1-1). Les joueurs jubilent après une but marqué par Kylian Mbappé. Les deux équipes vont à la pause sur un score nul de 1 but partout. De retour des vestiaires, une déviation d'un tir de Lionel Messi par Mercano redonne l'avantage à l'Argentine (2-1). Griezmann donne des frissons aux supporters argentins en se créant une belle occasion à la 55e minute. Deux minutes plus tard, à la 57e minute, Pavard pour sa première sélection en Bleus permet à la France de revenir au score (2-2). Kylian Mbappé s'est encore illustré à la 64e minute en reprenant un ballon renvoyé par la défense adverse. D'un contrôle, il s'engouffre entre deux défenseurs et trompe le gardien (3-2). Kylian Mbappé console Angel Di Maria, son coéquipier argentin du PSG. L'attaquant du PSG va signer son doublé du jour plus tard sur une passe d'Olivier Giroud (4-2). Lionel Messi décoche une frappe du pied droit sans inquiéter le gardien français. A la 87e minute, Kylian Mbappé cède sa place à Thauvin. L'Argentine revient réduit le score dans les arrêts de jeu (90e+3) grâce à Kun Aguero (4-3)."
Prompt 2: "Faites un résumé de ce texte: La France cale l'Argentine Liesse des Bleus français après le but de Kylian Mbappé. Les Français ont ouvert les hostilités dès le début de la partie, avec une accélération à la 11e minute de Kylian Mbappé qui est fauché dans la surface de réparation. Lire aussi : Le pénalty est transformé à la 13e minute de jeu par Antoine Griezmann. La France mène 1-0 au score. A la 18e minute, Mbappé accélère encore et se fait faucher juste à la limite de la surface de réparation. Le tir mal ajusté de Paul Pogba passe au-dessus de la barre transversale. Angel Di Maria, le joueur du Paris Saint-Germain, d'une frappe du gauche envoie le ballon à la 40e minute dans la lucarne des cages gardées par Hugo Lloris (1-1). Les joueurs jubilent après une but marqué par Kylian Mbappé. Les deux équipes vont à la pause sur un score nul de 1 but partout. De retour des vestiaires, une déviation d'un tir de Lionel Messi par Mercano redonne l'avantage à l'Argentine (2-1). Griezmann donne des frissons aux supporters argentins en se créant une belle occasion à la 55e minute. Deux minutes plus tard, à la 57e minute, Pavard pour sa première sélection en Bleus permet à la France de revenir au score (2-2). Kylian Mbappé s'est encore illustré à la 64e minute en reprenant un ballon renvoyé par la défense adverse. D'un contrôle, il s'engouffre entre deux défenseurs et trompe le gardien (3-2). Kylian Mbappé console Angel Di Maria, son coéquipier argentin du PSG. L'attaquant du PSG va signer son doublé du jour plus tard sur une passe d'Olivier Giroud (4-2). Lionel Messi décoche une frappe du pied droit sans inquiéter le gardien français. A la 87e minute, Kylian Mbappé cède sa place à Thauvin. L'Argentine revient réduit le score dans les arrêts de jeu (90e+3) grâce à Kun Aguero (4-3)."
Prompt 3: "Expliquez ce texte en une seule phrase:\n\nLa France cale l'Argentine Liesse des Bleus français après le but de Kylian Mbappé. Les Français ont ouvert les hostilités dès le début de la partie, avec une accélération à la 11e minute de Kylian Mbappé qui est fauché dans la surface de réparation. Lire aussi : Le pénalty est transformé à la 13e minute de jeu par Antoine Griezmann. La France mène 1-0 au score. A la 18e minute, Mbappé accélère encore et se fait faucher juste à la limite de la surface de réparation. Le tir mal ajusté de Paul Pogba passe au-dessus de la barre transversale. Angel Di Maria, le joueur du Paris Saint-Germain, d'une frappe du gauche envoie le ballon à la 40e minute dans la lucarne des cages gardées par Hugo Lloris (1-1). Les joueurs jubilent après une but marqué par Kylian Mbappé. Les deux équipes vont à la pause sur un score nul de 1 but partout. De retour des vestiaires, une déviation d'un tir de Lionel Messi par Mercano redonne l'avantage à l'Argentine (2-1). Griezmann donne des frissons aux supporters argentins en se créant une belle occasion à la 55e minute. Deux minutes plus tard, à la 57e minute, Pavard pour sa première sélection en Bleus permet à la France de revenir au score (2-2). Kylian Mbappé s'est encore illustré à la 64e minute en reprenant un ballon renvoyé par la défense adverse. D'un contrôle, il s'engouffre entre deux défenseurs et trompe le gardien (3-2). Kylian Mbappé console Angel Di Maria, son coéquipier argentin du PSG. L'attaquant du PSG va signer son doublé du jour plus tard sur une passe d'Olivier Giroud (4-2). Lionel Messi décoche une frappe du pied droit sans inquiéter le gardien français. A la 87e minute, Kylian Mbappé cède sa place à Thauvin. L'Argentine revient réduit le score dans les arrêts de jeu (90e+3) grâce à Kun Aguero (4-3)."
\end{lstlisting}