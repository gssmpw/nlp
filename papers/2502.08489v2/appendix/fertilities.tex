This section provides a tokenization comparison between models that have been selected for having relatively large vocabulary sizes. The baselines include Salamandra (256,000 tokens), Gemma 2 (256,000), Nemotron 4 (256,000), Bloom (250,880), Qwen 2 (152,064), \hbox{Mistral NeMo 2407} (131,072), Llama 3 (128,256) and EuroLLM (128,000).

The histograms illustrate the average amount of tokens required by each tokenizer to encode a single word across several Indo-European languages. For easier visualization, languages have been grouped by family whenever possible.

\begin{figure}[htbp]
    \centering
    \includegraphics[width=\textwidth]{figures/tokenizer/technical_report_appendix_1.pdf}
    \caption{Fertility scores for Germanic languages, namely Danish, German, English, Dutch, Norwegian Nynorsk, Norwegian and Swedish.}
    \label{fig:fertility_plot_app1}
\end{figure}

\begin{figure}[htbp]
    \centering
    \includegraphics[width=\textwidth]{figures/tokenizer/technical_report_appendix_2.pdf}
    \caption{Fertility scores for Romance languages, namely Catalan, Spanish, French, Galician, Italian, Occitan, Portuguese and Romanian.}
    \label{fig:fertility_plot_app2}
\end{figure}

\begin{figure}[htbp]
    \centering
    \includegraphics[width=\textwidth]{figures/tokenizer/technical_report_appendix_3.pdf}
    \caption{Fertility scores for Balto-Slavic languages, namely Bulgarian, Czech, \hbox{Croatian}, Lithuanian, Latvian, Polish, Russian, Serbo-Croatian, Slovak, Slovenian, Serbian and Ukrainian.}
    \label{fig:fertility_plot_app3}
\end{figure}

\begin{figure}[htbp]
    \centering
    \includegraphics[width=\textwidth]{figures/tokenizer/technical_report_appendix_4.pdf}
    \caption{Fertility scores for code and languages that belong to smaller families, namely Welsh, Greek, Estonian, Basque, Finnish, Irish, Hungarian and Maltese.}
    \label{fig:fertility_plot_app4}
\end{figure}