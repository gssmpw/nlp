%\todo{S: maybe we could make it sound less enthusiastic if it about companies. Mentioning them is fine, but maybe we can cut "great success" or "standing out"} 

\section{Introduction}
With the rapid growth in e-commerce it becomes increasingly more important for logistics companies to deliver packages to customers as fast as possible. 
The Prime Air project, launched by 
%Standing out is the Prime Air project launched by 
Amazon~\cite{aboutamazonDroneProgram}, introduces a new approach to parcel delivery. The German logistics company DHL 
has tested the delivery of medical supplies, such as blood samples, using flying mobile agents. 
%tested the fast delivery of medicine supplies like blood samples via flying mobile agents – to a great success.  
When it comes to areas which are not connected to a well-formed infrastructure or areas threatened by war, delivery of crucial goods as medicine and food can help people in need~\cite{bamburry2015drones,hii2019evaluation}. 

The collaborative delivery aspect of the drone delivery problem, which involves multiple agents on a single delivery path, primarily arises from variations in agent speed~\cite{bartschi2018}, consumption rates~\cite{bartschi_et_al:LIPIcs.STACS.2017.10}, or constraints such as battery power~\cite{bartschi2020collaborative} and movement areas~\cite{erlebach:drones}.  
In this paper, we further refine the model introduced by Erlebach et al.~\cite{erlebach:drones}. When it comes to their model, each agent's movement is confined to its designated area. Specifically, agents are allowed unrestricted movement within their respective subgraphs. This setting is motivated by realistic drone delivery scenarios where airspace is regulated by licenses, restricting certain drones to specific regions. Additionally, it is a natural observation that different types of agents, such as ships, drones, and trucks, are capable of traversing distinct parts of the graph. In addition, each drone operates at designated speeds and has its own consumption rate. Although both objectives – minimizing delivery time and reducing consumption – have been extensively studied, our research will exclusively focus on minimizing delivery time. We define this particular challenge as the Drone Delivery Problem with respect to time (DDT).   

In this work we will focus on paths and grid graphs. We refer to the fastest delivery on a line as DDT-Line (Drone Delivery on a Line) and the fastest delivery on a grid as DDT-Grid. Our work is primarily of theoretical interest and has the potential to lead to more generalized applications. Considering paths is theoretically significant, as they are the simplest form of graphs, and practically relevant, especially for fast last-mile delivery of critical goods, such as medical supplies. The grid graph setting is particularly applicable to densely populated areas such as large towns. 

%Throughout this work, we will focus solely on minimizing the delivery time. We call this problem the Drone Delivery problem with respect to time (DDT).

\subsubsection*{Related Work.}
 The delivery problem has been extensively studied through various models, including the Shortest Path Problem~\cite{ortega2022shortest}, the Traveling Salesperson Problem (TSP)~\cite{gutin2006traveling}, the Vehicle Routing Problem~\cite{toth2002vehicle},
 %Steiner Tree Problem (also Steiner Forest Problem)~\cite{byrka2010improved,chlebik2008steiner,kumar2021constant}, 
 and Dial-a-Ride Problem~\cite{cordeau2007dial}.
All these models assume that each request, whether a single node or a pair of nodes, is served by a single vehicle. 
 %All these models assume a single server is available to manage one or multiple source-destination pairs. \todo{S: At least vehicle routing has multiple agents, so maybe we just cut the second sentence?}
 %\todo{S: I think TSP, vehicle routing and Pickup and Delivery Problem (PDP) are all more similar than steiner tree, so maybe we could replace it here}
 
 % Initially, the problem of sending a package from a single source to a destination by the shortest path—known as the Shortest Path Problem—was solved in polynomial time, and algorithms have been developed to achieve faster run times~\cite{ortega2022shortest}. Subsequently, a more complex scenario involving multiple source-destination pairs were explored within the settings of the Steiner Tree Problem and the Steiner Forest Problem~\cite{byrka2010improved,chlebik2008steiner,kumar2021constant}, which is known to be NP-hard. Further complexity arises with the Dial-a-Ride Problem, in which the server has a capacity to carry multiple packages simultaneously, with the objective of minimizing the total travel distance while delivering all packages~\cite{cordeau2007dial}. All these models assume the availability of a single server to manage a set of source-destination pairs. 
Our study focuses on the collaborative delivery problem, involving multiple agents (servers, drones, etc.) in a single delivery process. This topic is inspired by the need for efficiency, aiming for faster or energy-efficient deliveries,   and accommodating constraints like battery limitations and movement restrictions. The primary motivation is to optimize delivery paths, reducing consumption. 
% Our study is more closely aligned with the collaborative delivery problem, in which multiple agents (servers, drones, etc.) are involved in even one delivery process. 
% The collaboration is inspired by the need for more efficient delivery agents, focusing on faster or energy-efficient delivery, and accommodating constraints such as battery limitations or movement restrictions. Most of the collaboration is motivated by the goal of achieving a faster delivery path, reducing consumption, or meeting lexicographical objectives.  
Bärtschi et al.~\cite{bartschi2018} first explored the delivery of a package from a source node to a target node with the aim of minimizing delivery time. They demonstrated that the problem can be solved in 
 time $O(k^2m+kn^2+\text{APSP})$ for a single package, where $n$ is the number of vertices, $m$ is the number of edges, $k$ is the number of agents and APSP represents the time required to compute all-pairs shortest paths in a graph with $n$ nodes and $m$ edges. 
 Carvalho et al. \cite{carvalho2021fast} improved the time complexity by developing an algorithm that runs in $O(kn \log n + km)$ time. Interestingly, they demonstrated that the problem escalates to NP-hardness with the addition of a second package.  To minimize energy consumption in package delivery, Bärtschi et al.~\cite{bartschi2018} developed a polynomial-time algorithm for delivering a single package, but demonstrated NP-hardness for multiple packages. They also investigated optimizing delivery time and energy for a single package. Lexicographically minimizing (time, energy) is polynomially feasible~\cite{bartschi2017energy}, but minimizing any combination of time and energy proves to be NP-hard~\cite{bartschi2018}.


Considering the realistic scenario where each agent has limited working time or energy, which restricts its total distance traveled, exploring a collaborative path utilizing multiple agents to deliver packages becomes particularly valuable~\cite{chalopin2014data,chalopin2014dataicalp,bartschi2020collaborative}.  Chalopin et al.~\cite{chalopin2014data} first demonstrated that the energy-constrained drone delivery problem is NP-hard in general graphs. Subsequently, Chalopin et al.~\cite{chalopin2014dataicalp} showed that this variant remains NP-hard even on a path graph. Bärtschi et al.~\cite{bartschi2020collaborative} found that a variant requiring each agent to return to its initial location is solvable in polynomial time for tree networks. 

Our study, as well as the research in~\cite{erlebach:drones}, diverges fundamentally from studies constrained by energy budgets. In our case, each agent's travel distance is not confined by a strict budget but is restricted to a specific subgraph where it can travel unlimited distances. As a result, the hardness results and algorithmic findings from studies with energy constraints do not directly translate to our problem. 
Erlebach et al.~\cite{erlebach:drones} first introduced this movement-restricted model. They demonstrate that for collaborative agents with restricted movement areas, it is NP-hard to approximate the Drone Delivery Time (DDT) within $O(n^{1-\varepsilon})$ or $O(k^{1-\varepsilon})$ 
 on general graphs, even if agents have equal speeds. 
%\todo{S: it is not really a 'special' case i think} 
For the case without initial positioning, they show that no polynomial time approximation with a finite ratio exists unless 
P$=$NP. Additionally, they argue that solving instances on a path remains NP-hard, when there are an arbitrary amount of different speeds. %\todo{S: maybe we could use 'an arbitrary amount of different speeds' instead}




%\todo{Kelin will add more literature on Saturday}
%Our underlying model was introduced by Erlebach et al.~\cite{erlebach:drones} in 2022. They show that for collaborative agents with restricted movement areas, it is NP-hard to approximate DDT within $O(n^{1-\varepsilon})$ or $O(k^{1-\varepsilon})$ on general graphs, even if agents have equal speeds. For the special case of no initial positioning, they show that no polynomial time approximation with finite ratio exists, unless P$=$NP. Additionally they argue that it is also NP-hard to solve instances on a path. For the setting without movement restrictions Carvalho et. al. \cite{carvalho2021fast} give an algorithm that runs in $O(kn \log n + km)$ time, where $n$ is the number of vertices, $m$ is the number of edges and $k$ is the number of agents. Interestingly, they show that the problem becomes NP-hard if a second package is added.

\subsubsection*{Our results.}

In section \ref{npline}, we provide stronger hardness results (see Theorem~\ref{thm:line}) than those presented by Erlebach et al.~\cite{erlebach:drones}. %That is, the problem remains NP-hard even if we restrict agents to two different. 
\begin{restatable}{theorem}
{thmline}\label{thm:line} 
The Drone Delivery Problem with initial positions on a line (DDT-Line) is NP-hard, even if all agents have only two different speeds.  
\end{restatable}

In Section \ref{npgrid}, we present a strong result for (unit) grid graphs:  
%For grid graphs we show various results in section \ref{npgrid}. 
%First we show in \ref{2grid} that DDT with agents that are restricted to rectangular movement areas is NP-hard. Secondly we argue that for agents having arbitrary movement areas but equal speed \todo{maybe unit speed sounds better} DDT remains NP-hard. Both of these results hold regardless of the agents having initial positions or if positions can be chosen. With slight extensions we can show that for the case of no initial positions, the problem is not even approximable.

\begin{restatable}{theorem}
{thmgrid}\label{thm:grid_2speed}  
  For any constant $\varepsilon > 0$, the Drone Delivery Problem on a grid with agents following rectangular movement areas (DDT-GridR) without initial positions is NP-hard to approximate within a factor of $O(n^{1-\varepsilon})$, where $n$ denotes the size of the grid, even if all agents have only two different speeds.  
\end{restatable}

%Both results are valid whether agents have predefined initial positions or if positions can be chosen freely. 
%With additional analysis in \ref{XXX} \todo{include section, we could do this in the respective sections or do both in an extra section 4.3 or so}, we also show that in scenarios without initial positions, the problem is not approximable. 
%\todo{what do you mean here? A setting that can do both? kelin: removed the sentence, may add more discussion about the two setting's proof connection}
%This particular setting has not yet been studied. However, it has practical applications, such as repositioning drones before delivery.   

% \begin{restatable}{theorem}
% {thmgridspeed}\label{thm:grid_1speed} 
%    The Drone Delivery Problem on grid graphs with unit speed agents (DDTGU), whether with or without initial positions, remains NP-hard if agents have arbitrary movement areas. 
% \end{restatable}