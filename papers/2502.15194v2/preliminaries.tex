\section{Preliminaries}

We now formally define the Drone Delivery Problem with respect to time (DDT), following the notation used in Erlebach et al.~\cite{erlebach:drones}. To ensure completeness, we include a formal definition here.

 An instance of DDT is a tuple $I = (G, (s, y), A)$, where $G = (V, E, \ell)$ is an undirected graph, $\ell: E \rightarrow \mathbb{R}_{\geq 0}$ represents the edge lengths. The package  $(s, y)$ is defined by its starting location $s\in V$ and destination  $y\in V$, and $A$ is a set of $k$ drones.   
    \begin{enumerate}
    \item In DDT \textbf{with} predefined
initial positions, each drone $a \in A$ is represented as a tuple $a=(p_a,v_a,G_a)$, where $p_a$ is the initial starting position, $v_a$ is the agents speed, and $G_a=(V_a, E_a)$ is a connected subgraph of $G$ that defines the agent's movement area.  We assume that $p_a\in V_a$.  

Let $L_a(u, w)$ represent the length of the shortest path from $u$ to $w$ in agent $a$'s subgraph $G_a$. The time it takes for an agent $a$ to travel from $u$ to $w$ is given by  $\frac{L_a(u, w)}{v_a}$. 

    \item In DDT \textbf{without} predefined
initial positions, each drone $a \in A$ is represented as a tuple $a=(v_a,G_a)$, where  $v_a$ is the agents speed, and $G_a=(V_a, E_a)$ is a connected subgraph of $G$ that defines the agent's movement area. In contrast to DDT with predefined
initial positions, the initial position of each drone $a$ can be freely chosen within $V_a$.%\todo{A: within $V_a$} 
    \end{enumerate}


%We consider a connected graph $G=(V,E)$ along with edge lengths $\ell:E\rightarrow \mathbb{R}_{\geq 0}$, where $\ell(u,v)$ represents the length of the edge between $u$ and $v$. Additionally we are given a set $A$ consisting of $k\geq 1$ mobile agents. Each agent $a\in A$ is modeled as $a=(p_a,v_a,V_a,E_a)$, where $p_a$ is the initial starting position, $v_a$ is the agents speed, and $V_a$ and $E_a$ define the agent's respective movement area. Note that $(V_a, E_a)$ is a   subgraph of $G$ that and we assume that it is connected.   We assume that $p_a\in V_a$. The time it takes for an agent $a$ to travel a distance $x$ is given by  $\frac{x}{v_a}$. 

A feasible solution to the DDT is a schedule of consecutive trips by agents carrying the package from $s$ to $y$. Each agent's trip consists of two phases: the \emph{empty phase}, during which the agent moves to its pickup location without the package, and the \emph{delivery phase}, during which the agent transports the package.  
In this paper, we assume that packages can be handed over between agents at vertices only. %\todo{A: I suggest ... between agents at vertices only}. %It is worth noting that the hardness results can be extended to the scenario where packages are allowed to be handed over at edges. 
The objective in solving the DDT is to find a schedule that minimizes the total delivery time. In the section on hardness results, we further define the corresponding graph structures as either a path or a grid graph, and use DDT-Line and DDT-Grid to represent the respective problems. 
%Our objective in solving the DDT is to find a schedule that minimizes the total delivery time. In the section on hardness results, we further define the corresponding graph structures as either a path or a grid graph, and use DDT-Line and DDT-Grid to represent the respective problems. 

According to an observation in \cite{erlebach:drones} for any instance of DDT, there exists an optimal solution in which each involved agent picks up and drops off the package exactly once. Therefore, without loss of generality, we restrict our focus to instances where each drone participates in the delivery route at most once.  
We can now define a feasible solution - i.e. a schedule - by describing each agents' movement and the path the package takes. 
%According to the observation in paper~\cite{erlebach:drones}: for any instance of DDT,  there exists an optimal solution in which each involved agent picks up and drops off the package exactly once. We could further refine the feasible solution $S$ (also called a schedule), where each drone is involved at most once: \todo{are we now restricting to these kind of schedules?}
\begin{itemize}
\item %The delivery route for each drone $a$ is represented by an ordered set $T_a$ of triplets $(u, w, t)$, where the drone picks up the package at location $u$ at time $t$ and delivers it to location $w$. Formally, each triplet must satisfy $u, w \in V_a$. Additionally, for any two consecutive triplets $(u, w, t)$ and $(u', w', t')$, the constraint $t' \geq t + \frac{\ell\{u, w\}}{v_a}$ must hold. For the first triplet, it is required that $t \geq 0$. In the case of DDT with predefined initial positions, the first triplet must also satisfy $u = p_a$. However, for DDT without predefined initial positions, this condition is not necessary, as the initial position $u$ can be chosen freely. 
%\todo{S: but is it one single triplet then? - or two? (one more for the empty phase or is this implicit?)}
The delivery route for each drone $a$ is represented by a %sorted set $T_a$ of  
triplet $(u, w, t)$, where agent $a$ picks up the package at time $t$ at location $u$ and drops it off at location $w$. Formally, each triplet must satisfy that $u, w \in V_a$.   
For the DDT with predefined initial positions, each involved drone $a$ must include an additional initial empty phase: $(p_a, u, 0)$, where $a$ moves from its initial position $p_a$ to $u$ at time $0$, in preparation for its delivery phase $(u, w, t)$. It must hold that $t\ge \frac{L_a(p_a, u)}{v_a} $ where $L_a(p_a, u)$ represents the length of the shortest path from $p_a$ to $u$ in the agent $a$'s subgraph $G_a$. 
In contrast, in DDT without predefined initial positions, 
%\todo{A: I dont believe we define what DDTSP is, do we?}, 
the empty phase is not required; that is, the triplet must satisfy $t \ge 0$ and $u = p_a$, %\todo{Shouldn't this be $t\geq 0$?}
%\todo{where $d$ is the agent that picks up the package initially?, S: this should be $a$, I think},
as we can directly select $u$ as the initial position. 

\item A feasible solution includes the package route in the form of tuples $(u, w, t, a)$, where the package moves with drone $a$ from $u$ to $w$ at time $t$. The first tuple must satisfy $u = s$, and the last must satisfy $w = y$.  
Formally, each tuple must satisfy that $u, w \in V_a$ and $(u, w, t) \in T_a$. 
%and for consecutive tuples $(u, w, t, a)$ and $(u', w', t', a')$, it must hold that $w = u'$, meaning the drop-off location of the previous tuple is the pick-up location of the subsequent tuple. In addition, $t + \frac{L_a(u, w)}{v_a} \leq t'$, as the later drone picks up the package only after the previous drone drops it off.
Furthermore, for consecutive tuples $(u, w, t, a)$ and $(u', w', t', a')$, the condition $w = u'$ must hold, ensuring the route is continuous. Additionally, the timing constraint $t + \frac{L_a(u, w)}{v_a} \leq t'$ must be satisfied, ensuring that the subsequent drone picks up the package only after the previous drone drops it off.

\item  We define $t(S) = t + \frac{L_a(u, w)}{v_{a}}$ as the duration of this solution, where $(u, w, t, a)$ is the last tuple where $w= y$. The objective is to minimize $t(S)$.
\end{itemize}





 % \todo{S: I made a version where I tried to incorporate everything, what do u think? Also I am not a big fan of the notions of empty/delivery phase, I think it is only mentioned twice in the paper (and we can easiliy rewrite those parts) + if we want to start with the more general definition those notions also don't really make sense (as there might be several phases), A: I dont have a strong opinion regarding the phases - both works for me}
% \simon{An instance of DDT is a tuple $I = (G, (s, y), A)$, where $G = (V, E, \ell)$ is an undirected graph, $\ell: E \rightarrow \mathbb{R}_{\geq 0}$ represents the edge lengths. The package  $(s, y)$ is defined by its starting location $s\in V$ and destination  $y\in V$, and $A$ is a set of $k$ drones.   
%     \begin{enumerate}
%     \item In DDT \textbf{with} predefined
% initial positions, each drone $a \in A$ is represented as a tuple $a=(p_a,v_a,G_a)$, where $p_a$ is the initial starting position, $v_a$ is the agents speed, and $G_a=(V_a, E_a)$ is a connected subgraph of $G$ that defines the agent's movement area.  We assume that $p_a\in V_a$.  


%     \item In DDT \textbf{without} predefined initial positions, each drone $a \in A$ is represented as a tuple $a=(v_a,G_a)$, where  $v_a$ is the agents speed, and $G_a=(V_a, E_a)$ is a connected subgraph of $G$ that defines the agent's movement area. In contrast to DDT with predefined
% initial positions, the initial position of each drone $a$ can be freely chosen within $V_a$. 
%     \end{enumerate}


% A feasible solution to the DDT is a schedule of consecutive trips by agents carrying the package from $s$ to $y$. The objective in solving the DDT is to find a schedule that minimizes the total delivery time. In the section on hardness results, we further define the corresponding graph structures as either a path or a grid graph, and use DDT-Line and DDT-Grid to represent the respective problems. 
% We can now define a feasible solution - i.e. a schedule - by describing each agents' movement and the path the package takes.
% \begin{itemize}
% % \item A delivery route for each drone $a$ is represented by a sorted set $T_a$ of triplets $(u, w, t)$, where agent $a$ picks up the package at time $t$ at location $u$ and drops it off at location $w$. Formally, each triplet must satisfy that $\{u, w\} \in E_a$. Furthermore consecutive triplets $(u, w, t), (u', w', t')$ must satisfy that $t'\geq t+\frac{\ell\{u, w\}}{v_a}$. For the first triplet it must hold that $t \geq 0$ and for DDT with predefined initial positions the first triplet must also have $u=p_a$, which is not needed for DDT withput predefined initial positions as we can freely select $u$ as the initial position.

% % \item A feasible solution includes the package route in the form of tuples $(u, w, t, a)$, where the package moves with drone $a$ from $u$ to $w$ at time $t$. Note that the first tuple must satisfy $u = s$, and the last must satisfy $w = y$.  
% % Formally, each tuple must satisfy that $(u, w, t) \in T_a$, and for consecutive tuples $(u, w, t, a)$ and $(u', w', t', a')$, it must hold that $w = u'$, meaning the drop-off location of the previous tuple is the pick-up location of the subsequent tuple. In addition, $t + \frac{\ell(u, w)}{v_a} \leq t'$, as the later drone picks up the package only after the previous drone drops it off.
%     \item The delivery route for each drone $a$ is represented by an ordered set $T_a$ of triplets $(u, w, t)$, where the drone picks up the package at location $u$ at time $t$ and delivers it to location $w$. Formally, each triplet must satisfy $\{u, w\} \in E_a$. Additionally, for any two consecutive triplets $(u, w, t)$ and $(u', w', t')$, the constraint $t' \geq t + \frac{\ell\{u, w\}}{v_a}$ must hold. For the first triplet, it is required that $t \geq 0$. In the case of DDT with predefined initial positions, the first triplet must also satisfy $u = p_a$. However, for DDT without predefined initial positions, this condition is not necessary, as the initial position $u$ can be chosen freely.

%     \item A feasible solution consists of a package route represented by tuples $(u, w, t, a)$, where the package is transported by drone $a$ from $u$ to $w$ at time $t$. The first tuple must satisfy $u = s$, and the last tuple must satisfy $w = y$. Formally, each tuple must be part of the respective drone's route, meaning $(u, w, t) \in T_a$. Furthermore, for consecutive tuples $(u, w, t, a)$ and $(u', w', t', a')$, the condition $w = u'$ must hold, ensuring route is continuous. Additionally, the timing constraint $t + \frac{\ell(u, w)}{v_a} \leq t'$ must be satisfied, ensuring that the subsequent drone picks up the package only after the previous drone drops it off.

% \item  We define $d(S) = t + \frac{\ell(\{u, w)\}}{v_a}$ as the duration of this solution, where $(u, w, t, a)$ is the last tuple. The objective is to minimize $d(S)$.
% \end{itemize}}
% \todo{S:Do we want to use $t(S)$ or $d(S)$? previously we used $t$, so we need to change it everywhere else, and idk really what we use}

% \simon{According to an observation in \cite{erlebach:drones} for any instance of DDT, there exists an optimal solution in which each involved agent picks up and drops off the package exactly once, so in this paper we can restrict ourselves to those instances without loss of generality.}


% A feasible solution for DDT is represented as a list of tuples $\mathcal{T}$, in which each tuple  
% $(a, u_a, u'_a)$, where $a \in A$ indicates one of the drones, and $u_a, u'_a \in V_a$ specify the nodes where agent $a$ picks up and drops off the package, respectively. Note that the first tuple's pick-up location is $s$ and the final tuple's drop-off location is $y$. For any two consecutive tuples, the drop-off location of the previous tuple is the pick-up location of the subsequent tuple. 

%\todo{simon: does t take a single node as input (u) or a set of tuples(T)? this feels like different notations are mixed}
% \todo{S; i think i'd prefer a notation without u}
% \simon{Let $t(\mathcal{T})$ denote the delivery time of a feasible schedule $\mathcal{T}$ and suppose $(a, u, u')$ is the final tuple, i.e., $u \in V_a$:
% $$ t(\mathcal{T}) = \max\{\frac{L(p_a, u)}{v_a}, t(u, \mathcal{T} \setminus (a, u, u')) \} +  \frac{\ell(u, u')}{v_a}.$$
% }
% Let $t(u, \mathcal{T})$ denote the time elapsed until the package is delivered to node $u$ in a feasible solution $\mathcal{T}$ and suppose $(a, u_a, u'_a)$ is the final tuple, i.e., $u'_a= u$:  
%  $$ t(u, \mathcal{T}) = \max\{\frac{L(p_a, u_a)}{v_a}, t(u_a, \mathcal{T} \setminus (a, u_a, u'_a)) \} +  \frac{\ell(u_a, u'_a)}{v_a}.$$
% %\todo{$\ell(p_a, u_a)$ is not correct imo, because the edge $\{p_a, u_a\}$ might not exist, but $u_a$ and $p_a$ are still connected} 
% where $L(p_a, u_a)$ represents the length of the shortest path from $p_a$ to $u_a$ in the graph $(V_a, E_a)$. Here, the maximum of these two, $\frac{L(p_a, u_a)}{v_a}$ and $ t(u_a,\mathcal{T} \setminus (a, u_a, u'_a))$, indicates when both the final agent $a$ and the package are at $u_a$. The term $\frac{\ell(u_a, u'_a)}{v_a}$ represents the time it takes for agent $a$, after picking up the package at $u_a$, to transport it along the shortest path to $u'_a$ within the drone’s subgraph $(V_a, E_a)$. \todo{S: this definition is technically wrong, if one agent was used somewhere else before it may take longer to get there than the distance to $p_a$}
% The objective of the Drone Delivery Time (DDT) problem is to identify a feasible solution $\mathcal{T}$ that minimizes the delivery time, denoted as $t(y, \mathcal{T})$.
% Observe that any agent can contribute to 
% $\mathcal{T}$ at most once. This can be proven by contradiction: assume that agent 
% $a$ is used a second time. Since the package would need to wait for $a$ at the second pickup location, the delivery time (or consumption) to the final destination cannot be smaller compared to using $a$ only once.\todo{S: we should also mention the definition of DDT without initial pos}


In this paper, we study scenarios where the given $G$ represents either a line (path) graph or a grid graph, with details provided in the corresponding sections. Specifically, we consider DDT-Line with predefined initial positions and DDT-Grid without initial positions.
%\todo{A: Im not terribly happy with this, how do we get the optimal delivery time then? just by taking the last $t$?S: yes, but I think that fine right?}

% We further differentiate between variants of these problems based on whether initial positions are included.  
% On one hand, we consider drones with initial positions denoted by $p_a$, a common assumption in early studies~\cite{erlebach:drones}. 
% %\todo{Kelin add literature}
% %\todo{Later, in Section~\ref{npgrid} conflict with Our results - Andi doesnt get it, you mean there is a structural problem like when we have these topics?}
% On the other hand, we explore scenarios without initial positions, omitting $p_a$ for all agents in set $A$. This scenario allows us to choose the placement of each agent, while the rest of the setting remains unchanged. 
%additionally consider the other case of no initial positioning (omitting the $p_a$ for all $a\in A$). Then we are able to chose where to place an agent. Apart from that it is the same setting.