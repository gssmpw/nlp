\begin{abstract} Fast shipping and efficient routing are key problems of modern logistics. Building on previous studies that address package delivery from a source node to a destination within a graph using multiple agents (such as vehicles, drones, and ships), we investigate the complexity of this problem in specialized graphs and with restricted agent types, both with and without predefined initial positions.    
Particularly, in this paper, we aim to minimize the delivery time for delivering a package. To achieve this, we utilize a set of collaborative agents, each capable of traversing a specific subset of the graph and operating at varying speeds. This challenge is encapsulated in the recently introduced Drone Delivery Problem with respect to delivery time (DDT). 
    
In this work, we show that the DDT  with predefined initial positions on a line is NP-hard, even when considering only agents with two distinct speeds. This refines the results presented by Erlebach, et al.~\cite{erlebach:drones}, who demonstrated the NP-hardness of DDT on a line with agents of arbitrary speeds. Additionally, we examine DDT in grid graphs without predefined initial positions, where each drone can freely choose its starting position. We show that the problem is NP-hard to approximate within a factor of $O(n^{1-\varepsilon}$), where $n$ is the size of the grid, even when all agents are restricted to two different speeds as well as rectangular movement areas. We conclude by providing an easy $O(n)$ approximation algorithm. 
  % agents travel at the same speed and their movement areas are arbitrary. %Furthermore, even when agents are restricted to rectangular movement areas, we  establish NP-hardness for scenarios allowing at most two different speed. \kelin{inapprox results, or remove this one?} 
 % \keywords{Delivery \and Complexity \and  Combinatorial Optimization.}
\end{abstract}