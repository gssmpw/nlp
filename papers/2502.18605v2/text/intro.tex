\section{Introduction}

\emph{Variational inequalities (VIs)} provide a unifying framework for analyzing a wide range of optimization and equilibrium problems. They have a host of important applications in engineering and economics~\citep{Facchinei03:Finite}, including identifying stationary points in constrained optimization; computing Nash equilibria in (noncooperative) multi-player games~\citep{Nash51:Non}, such as Cournot's classical model of oligopoly~\citep{Cournot38:Recherches}; predicting economic activity---commodity prices and consumer consumption---in a closed, competitive economy~\citep{Arrow54:Existence}, which is at the heart of general equilibrium theory; traffic equilibrium problems---estimating the steady-state of a congested network wherein users compete for its resources
~\citep{Dafermos80:Traffic}; frictional contact problems in mechanical engineering~\citep{Capatina14:Variational}; and pricing options, a foundational problem in financial economics
~\citep{Black73:Pricing}.

Formally, in a general form, a VI can be defined as follows.\footnote{As is standard, a VI throughout this paper refers to the \emph{Stampacchia} VI~\citep{Kinderlehrer00:Introduction}.}

\begin{definition}\label{def:vi}
    Let $\cX$ be a convex and compact subset of $\R^d$ and $F: \cX \to \R^d$ a bounded map. The {\em variational inequality (VI)} problem asks for a point $\vx \in \cX$ such that
    \begin{equation}\label{eq:vi}
        \ip{F(\vx), \vx' - \vx} \ge 0 \quad \forall \vx' \in \cX.
    \end{equation}
    \end{definition}
For computational purposes, it is common to consider the \emph{$\epsilon$-approximate} VI problem, wherein the right-hand side of ~\eqref{eq:vi} is replaced by $-\epsilon$ for some precision parameter $\epsilon > 0$.

\Cref{def:vi} abstracts the description of $F$ and $\cX$. As a concrete example, when $F : \vx \mapsto -\nabla u (\vx)$ is the negative gradient of a differentiable function $u : \cX \to \R$, the solutions to \eqref{eq:vi} are points that satisfy the first-order optimality conditions for maximizing $u$~\citep{Boyd04:Convex}. %A further simple example of a VI concerns the familiar problem of solving a system of nonlinear equations~\citep{Facchinei03:Finite}.

Unfortunately, the considerable expressivity of VIs comes at the expense of \emph{intractability}: even when $F$ is linear and $\epsilon$ is an absolute constant, identifying an $\epsilon$-approximate VI solution is computationally hard; this follows readily from the intractability of Nash equilibria---under plausible complexity assumptions~\citep{Daskalakis08:Complexity,Chen09:Settling,Rubinstein16:Settling}. Unconditional, query-complexity lower bounds have also been established~\citep{Hirsch89:Exponential,Babichenko16:Query}; \emph{cf.}~\citet{Milionis23:Impossibility} and~\citet{Hart03:Uncoupled} for other pertinent impossibility results.

This bleak realization has shifted the focus of contemporary research primarily to characterizing specific subclasses of VIs that elude those complexity barriers, with the ensuing line of work flourishing in recent years. %For example, it is worth noting 
Some notable examples include the classical \emph{Minty property}~\citep{Facchinei03:Finite,Mertikopoulos19:Learning,Malitsky15:Projected,Goktas25:Tractable}, as well as certain relaxations thereof~\citep{Diakonikolas21:Efficient,Bohm23:Solving,Bauschke21:Generalized,Combettes04:Proximal,Gorbunov23:Convergence,Cai24:Accelerated,Alacaoglu23:Beyond,Pethick22:Escaping,Lee21:Fast,Patris24:Learning,Choudhury24:Single,Lee22:Semi,Anagnostides24}.

Those important advances notwithstanding, the scope of such results is severely restricted. In this paper, we pursue a different, orthogonal avenue. Instead of restricting the class of {\em problems} to achieve computational tractability, we relax the underlying {\em solution concept}. Our main research question is:
%
\begin{quote}
    \centering
    \emph{Are there meaningful relaxations of the VI problem that can always be solved efficiently?}
\end{quote}
%
When specialized to games, this question can be seen as part of the research agenda recently outlined by~\citet{Daskalakis22:Non} in his 
address at the Nobel symposium about equilibrium computation in \emph{nonconcave games}---a major, new frontier in the interface of game theory and optimization.

\subsection{Our contribution: the expected VI problem}

To make progress on that central question, we introduce a natural relaxation of~VIs (in the context of~\Cref{def:vi}).

\begin{definition}\label{def:evi}
    Given a set of {\em deviations} $\Phi \subseteq \cX^\cX$, the \em{$\epsilon$-approximate $\Phi$-expected variational inequality ($\Phi$-EVI)} problem asks for a distribution $\mu \in \Delta(\cX)$ such that
    \begin{equation}\label{eq:evi}
        \E_{\vx\sim\mu} \ip{F(\vx), \phi(\vx) - \vx} \ge - \epsilon \quad \forall \phi \in \Phi.
    \end{equation}
\end{definition}
%

(The above definition does not specify how $\cX, F, \Phi$, and $\mu$ should be represented for computational purposes, but we will be explicit about representation whenever it is relevant.)

In words, \Cref{def:evi} only imposes (approximate) nonnegativity \emph{in expectation} for points $\vx$ drawn from $\mu \in \Delta(\cX)$. It certainly relaxes~\Cref{def:vi}: if $\vx$ satisfies~\eqref{eq:vi}, then the distribution $\mu$ that always outputs $\vx$ is also a $\Phi$-EVI solution. $\Phi$-EVIs are thus no harder than VIs (assuming that solutions exist). However, as we shall see, the primary justification of $\Phi$-EVIs is that they can be easier than VIs.

 \Cref{def:evi} is crucially parameterized by $\Phi$; the larger the set of deviations $\Phi$, the tighter the set of solutions. As will become clear, \Cref{def:evi} is intimately connected with notions of \emph{correlated equilibrium (CE)} from game theory~(\eg,~\citealp{Aumann74:Subjectivity}). The more permissive case where $\Phi$ comprises only constant functions, $\Phi = \Phiconst = \{ \phi_\vx : \vx \in \cX \}$ where $\phi_\vx(\vx') = \vx$ for all $\vx' \in \cX$, is perhaps the most basic relaxation of~\Cref{def:vi}; we call the $\Phiconst$-EVI problem simply the {\em EVI} problem.

\paragraph{Algorithms and complexity for $\Phi$-EVIs}

As it turns out, imposing no constraints on $\Phi$ results in an impasse: $\Phi$-EVIs are in general tantamount to regular VIs---thereby being \PPAD-hard (\Cref{cor:hard-EVI,cor:hard-EVI-quad}). On the other hand, unlike general VIs, one of our key contributions is to show that when $\Phi$ contains only linear maps, $\Philin$, $\Phi$-EVIs can be solved in time polynomial in the dimension $d$ and $\log(1/\epsilon)$ (\Cref{th:elvi}), establishing the promised computational property that separates EVIs from VIs. This result is based on \emph{ellipsoid against hope ($\eah$)}, the seminal algorithm of~\citet{Papadimitriou08:Computing} developed for computing correlated equilibria in multi-player games. (\Cref{sec:eah} gives a self-contained overview of $\eah$.) In doing so, we extend the scope of that algorithm to a much broader class of problems well beyond the realm of game theory. Notably, \Cref{th:elvi} applies even when $\cX$ is given implicitly through a membership oracle; this extension makes use of the recent technical approach of~\citet{Daskalakis24:Efficient}, discussed in more detail in~\Cref{sec:main}.

One limitation of~\Cref{th:elvi} is that it relies on the $\eah$ algorithm, 
which is slow in practice. We address this by also establishing more scalable algorithms that use convex quadratic optimization (\Cref{theorem:regret}) instead of the ellipsoid algorithm, albeit with a complexity growing polynomially in $1/\epsilon$. As a byproduct, we obtain the best-known algorithm for linear-swap regret minimization over explicitly represented polytopes, improving on \citet{Daskalakis24:Efficient} by reducing the per-iteration complexity.

In addition to their more favorable computational properties, we further show that $\Phi$-EVIs admit (approximate) solutions under more general conditions than their associated VIs---namely, without $F$ being continuous (\Cref{theorem:existence}); \Cref{sec:existence} documents further interesting aspects on existence.

%in short, the more permissive nature of EVIs enables bypassing the preconditions of Brouwer's fixed-point theorem---or other pertinent arguments of existence---by instead relying on (a suitable form of) the minimax theorem. This follows the blueprint of the ingenious argument of~\citet{Hart89:Existence} pertaining to the existence of correlated equilibria \emph{without} resorting to Nash's theorem, a connection further clarified in what follows.

\paragraph{Connection to other solution concepts}

As we have alluded to, $\Phi$-EVIs generalize (\Cref{example:CCE,example:CE}) the seminal concept of a \emph{(coarse) correlated equilibrium} \emph{\`a la}~\citet{Aumann74:Subjectivity} and \citet{Moulin78:Strategically} in finite games, and more generally \emph{$\Phi$-equilibria}~\citep{Greenwald03:General,Stoltz07:Learning,Gordon08:No} of concave games. %Aumann's key argument was that such
%equilibria adhere to Bayesian rationality. 
What is more surprising is that $\Philin$-EVIs \emph{refine} CEs even in normal-form games; we give illustrative examples, together with an interpretation, in~\Cref{sec:games}. We also note that $\Phi$-EVIs can be used even in games with nonconcave utilities~\citep{Daskalakis22:Non,Cai24:Tractable,Ahunbay25:First} or noncontinuous gradients (as in nonsmooth optimization), as well as in (pseudo-)games with \emph{coupled constraints} (\emph{cf.}~\citealp{Bernasconi23:Constrained} and~\Cref{sec:related} for related work).

%Furthermore, $\Phi$-EVIs capture a certain notion of ``local equilibrium'' put forward by~\citet{Cai24:Tractable} in the context of nonconcave games---in response to the call of~\citet{Daskalakis22:Non}. We argue that~\Cref{def:evi}, in addition to its greater generality, presents a simpler, and perhaps cleaner, formulation. Importantly, \Cref{def:evi} expands the scope of CE to a broader class of problems with a greater reach.

\paragraph{Further properties} As further motivation, we show that for certain structured problems, such as \emph{(quasar-)concave} optimization and \emph{polymatrix} zero-sum games, EVIs essentially coincide with VIs (\Cref{prop:collapse,prop:convex-equiv}). %this serves as a basic litmus test, confirming that EVIs behave as expected in more tractable subclasses.

Finally, in certain applications, one might be interested in a VI solution mainly insofar as it provides guarantees in terms of an underlying objective, such as misclassification error or social welfare. Through that prism, the question is whether performance guarantees for VIs can be translated to EVIs as well. In~\Cref{sec:smoothness}, we establish a framework for accomplishing that (\Cref{def:smoothness}) by extending the celebrated \emph{smoothness} framework of~\citet{Roughgarden15:Intrinsic}, and provide interesting examples beyond game theory.

Taken together, these properties provide compelling justification for $\Phi$-EVIs as a solution concept \emph{in lieu} of VIs. \Cref{tab:results} gathers our main results. (Proofs are in~\Cref{sec:proofs}.)

\begin{table*}[!ht]
    \centering
    \small
    \renewcommand{\arraystretch}{1.3}
    \begin{tabular}{m{5.2cm} p{7.2cm} p{3.3cm}}
        \bf Result & \bf Description & \bf Reference \\ \toprule
         \rowcolor{gray!20} Existence of ($\epsilon$-approx.) solutions & Under Lipschitz cont. for $\Phi$ and bounded $F$ & \Cref{theorem:existence} \\
         Complexity with nonlinear $\Phi$ & \PPAD-hardness with linear $F$ and $\epsilon = \Theta(1)$ & \Cref{cor:hard-EVI,cor:hard-EVI-quad} \\
         \rowcolor{gray!20}
         {Algorithms for linear $\Phi$ } & \textbullet~ $\poly(d, \log(1/\epsilon ))$-time via $\eah$
         \newline
         \textbullet~ $\poly(d, 1/\epsilon)$-time via $\Phi$-regret minimization & 
         \Cref{th:elvi}
         \newline
         \Cref{theorem:regret} \\
         {Equivalence between VIs-EVIs} & \textbullet~ Quasar-concave functions (\Cref{def:quasar}) \newline
         \textbullet~ $\vx \mapsto \langle F(\vx), \vx' - \vx \rangle$ concave for all $\vx'$ & 
         \Cref{prop:convex-equiv} \newline \Cref{prop:collapse} \\
         \rowcolor{gray!20}
         Performance guarantees for EVIs & Under smoothness (\Cref{def:smoothness}) & \Cref{theorem:smoothness}\\
         \bottomrule
    \end{tabular}
    \caption{Our main results concerning $\Phi$-EVIs (\Cref{def:evi}).}
    \label{tab:results}
\end{table*}