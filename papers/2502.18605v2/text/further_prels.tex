\section{Additional Preliminaries}
\label{sec:add-prels}

\paragraph{Revisiting \Cref{def:evi}} 

In order to define the distributions $\Delta(\cX)$ over $\cX$ precisely, we recall here some basic concepts from probability theory. We refer to \citet[Chapter 1 and 2]{Billingsley99:Convergence} and \citet[Chapter 15]{AliprantisB06:Infinite} for detailed treatments. We assume throughout the paper that the set $\cX \subseteq \R^d$ is Borel measurable. Let $\Delta(\cX)$ be the set of Borel probability measures $\mu$ on $\cX$, that is, measures $\mu: \qty(\cX, B(\cX)) \to \qty(\R, B(\R))$ with $\mu(\cX) = 1$, where $B(\cX)$ and $B(\R)$ denote the respective $\sigma$-algebra of Borel sets. We simply call $\mu$ a distribution. For any Borel measurable function $f : \cX \to \R$---henceforth just \emph{measurable}--- we can then take the integral $\E_{\vx \sim \mu}[f(\vx)] := \int_\cX f(\vx) d\mu(\vx)$. In particular, for $\E_{\vx\sim\mu} \ip{F(\vx), \phi(\vx) - \vx}$ in \Cref{def:evi} to be well-defined, we assume throughout this paper that $F$ and each $\phi \in \Phi$ are measurable functions.

For our computational results (\Cref{sec:main}), we are making a standard assumption regarding the geometry of $\cX$ (\Cref{sec:prel}); this can be met by bringing $\cX$ into isotropic position. In particular, there is a polynomial-time algorithm that computes an affine transformation to accomplish that~\citep{Lovasz06:Simulated}, and minimizing linear-swap regret reduces to minimizing linear-swap regret to the transformed instance~\citep[Lemma A.1]{Daskalakis24:Efficient}.

% \paragraph{Further comments on $\Delta(\cX)$} We make $\Delta(\cX)$ a topological space by equipping it with the weak-$*$ topology. This is exactly the coarsest topology that makes the map $\mu \mapsto \int_\cX f(x) d\mu(x)$ continuous for all continuous and bounded maps $f : \cX \to \R$.\footnote{Convergence in this topology coincides with the notion in probability theory that the probability measures $(\mu_n)_{n \in \N} \subset \Delta(\cX)$ \emph{weakly converge} to the probability measure $\mu \in \Delta(\cX)$.} Any continuous $f : \cX \to \R$ is guaranteed to be bounded since are working with compact spaces $\cX$ in this paper. Moreover, $\cX \subset \R^n$ compact implies that $\Delta(\cX)$ is compact and Hausdorff as well \cite{AliprantisB06:Infinite}[Thm.~15.11].

\iffalse

\begin{definition}[Weak separation oracle;~\citealp{Grotschel81:Ellipsoid}]
    Let $\cX$ be a convex and compact set in $\R^d$ and $\epsilon > 0$ a rational number. A \emph{(weak) separation oracle} for $\cX$ is a function that, given $\vx \in \R^d$,
    \begin{itemize}[topsep=0pt,noitemsep]
        \item asserts whether $\vx$ is $\epsilon$-close to $\cX$, in that its Euclidean distance from $\cX$ is at most $\epsilon$; or, otherwise,
        \item  finds a vector $\vec{w} \in \R^d$, with $\|\vec{w}\| = 1$, such that $\langle \vec{w}, \vx \rangle \geq \langle \vec{w}, \vx' \rangle - \epsilon$ for any $\vx'$ that is in $\cX$ and its distance from any point not in $\cX$ is at least $\epsilon$.
    \end{itemize}
\end{definition}

\fi