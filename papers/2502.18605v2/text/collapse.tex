\section{Problems where EVIs coincide with VIs}

We saw earlier, in~\Cref{prop:EVI-VI}, that when $\Phi$ comprises all functions from $\cX$ to $\cX$, the $\Phi$-EVI problem is tantamount to the associated VI problem. However, if one restricts the functions contained in $\Phi$, are there still structured VIs where we retain this equivalence? 
In this section, we consider certain structured VIs, and show their equivalence to the corresponding EVIs (that is, $\Phiconst$-EVIs). Unlike general VIs, the ones we examine below are tractable.%---because of that equivalence.


\subsection{Polymatrix zero-sum games and beyond}

The first important class of VIs we consider is described by a condition given below.

\begin{proposition}
    \label{prop:collapse}
    Suppose that for any $\vx' \in \cX$, the function $g : \vx \mapsto \langle F(\vx), \vx' - \vx \rangle$ is concave. Then, if $\mu \in \Delta(\cX)$ is an $\epsilon$-approximate solution to the EVI, $\E_{\vx \sim \mu} \vx$ is an $\epsilon$-approximate solution to the VI.
\end{proposition}

The proof follows directly from Jensen's inequality.

The precondition of~\Cref{prop:collapse} is satisfied, \emph{e.g.}, when: (i) $\langle F(\vx), \vx \rangle = 0$ for all $\vx \in \cX$, and (ii) $F$ is a linear map. In the context of $n$-player games, the first condition amounts to the zero-sum property: $\sum_{i=1}^n u_i(\vx) = 0$ for all $\vx$. Of course, this property is not enough to enable efficient computation of Nash equilibria, for every two-player (general-sum) game can be converted into a $3$-player zero-sum game. This is where the second condition comes into play: $F$ is a linear map---that is, each player's gradient must be linear in the joint strategy. Those two conditions are satisfied in \emph{polymatrix zero-sum} games~\citep{Cai16:Zero}; in such games, the conclusion of~\Cref{prop:collapse} is a well-known fact. 

\subsection{Quasar-concave functions}

We next consider the problem of maximizing a (single) function that satisfies \emph{quasar-concavity}---a natural generalization of concavity that has received significant  interest~\citep{Hardt18:Gradient,Fu23:Accelerated,Hinder20:Near,Gower21:SGD,Guminov23:Accelerated,Caramanis24:Optimizing}.\footnote{Prior literature mostly uses the term \emph{quasar-convexity}, which is equivalent to quasar-concavity for the opposite function $-u$.}

\begin{definition}[Quasar-concavity]
    \label{def:quasar}
    Let $\gamma \in (0, 1]$ and $\vxstar \in \cX$ be a maximizer of a differentiable function $u : \cX \to \R$. We say that $u$ is \emph{$\gamma$-quasar-concave} with respect to $\vxstar$ if
    \begin{equation}
        \label{eq:quasar}
        u(\vxstar) \leq u(\vx) + \frac{1}{\gamma} \langle \nabla u(\vx), \vxstar - \vx \rangle \quad \forall \vx \in \cX.
    \end{equation}
\end{definition}

In particular, in the special case where $\gamma = 1$, \eqref{eq:quasar} is equivalent to \emph{star-concavity}~\citep{Nesterov06:Cubic}. If in addition \eqref{eq:quasar} holds for all $\vxstar \in \cX$ (not merely w.r.t. a global maximizer), it captures the usual notion of concavity.

Any reasonable solution concept for such problems should place all mass on global maxima; EVIs pass this litmus test:

\begin{proposition}
    \label{prop:convex-equiv}
    Let $F = - \grad u$ for a $\gamma$-quasar-concave and differentiable function $u : \cX \to \R$. Then, for any solution $\mu \in \Delta(\cX)$ to the EVI problem,
    \begin{equation*}
        \E_{\vx \sim \mu} u(\vx) \geq \max_{\vx \in \cX} u(\vx).
    \end{equation*}
    Thus, $\mathbb{P}_{\vx\sim\mu}[u(\vxstar) = u(\vx)] = 1$, for $\vxstar \in \argmax_{\vx} u(\vx)$.
\end{proposition}

Indeed, by~\Cref{def:quasar}, $0 \leq \E_{\vx\sim\mu} \ip{\grad u(\vx), \vx - \vxstar } \le \gamma \E_{\vx\sim\mu} [u(\vx) - u(\vxstar) ]$ for any EVI solution $\mu \in \Delta(\cX)$. Thus, under quasar-concavity, VIs basically reduce to EVIs.


%It would be hard to justify a notion that, in a (two-player) zero-sum game, places mass on strategies that do not constitute minimax strategies.

%More broadly, and in particular for multi-player games, there is no definite justification for EVIs \emph{vis-\`a-vis} other notions---this ties directly to the \emph{equilibrium selection} problem, which has generated much debate~\citep{Harsanyi88:General}.