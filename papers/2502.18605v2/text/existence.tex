\section{Existence and Complexity Barriers}
\label{sec:existence}

Perhaps the most basic question about $\Phi$-EVIs concerns their \emph{totality}---the existence of solutions. If one is willing to tolerate an arbitrarily small imprecision $\epsilon > 0$, we show that solutions exist under very broad conditions.

\begin{restatable}{theorem}{mainexistence}
    \label{theorem:existence}
    Suppose that $F : \cX \to \R^d$ is measurable and there exists $L > 0$ such that every $\phi \in \Phi$ is $L$-Lipschitz continuous. Then, for any $\epsilon > 0$, there exists an $\epsilon$-approximate solution to the $\Phi$-EVI problem.
\end{restatable}


In particular, our existence proof does not rest on $F$ being continuous. 
Instead, we consider the continuous function $\hatF$ that maps $\vx \mapsto \E_{\hatvx \sim \Delta(\cB_{\delta}(\vx) \cap \cX)} F(\hatvx)$ (\Cref{lemma:cont}), where $\cB_\delta(\vx)$ is the Euclidean ball centered at $\vx$ with radius $\delta = \delta(\epsilon)$. It then suffices to invoke Brouwer's fixed-point theorem for the gradient mapping $\vx \mapsto \proj_\cX ( \vx - \hatF(\vx) )$, where $\proj_\cX$ is the Euclidean projection with respect to $\cX$.
%Then, a uniform distribution around a VI solution $\vx^*$ to $\hatF$ makes makes an $\epsilon$-approximate solution to the $\Phi$-EVI problem for $F$

%In contrast to the classical existence result for VIs (see, \eg, Section 3 in \citealp{KinderlehrerS00}), our existence proof does not rest on $F$ being continuous. Instead, we can invoke this known result by considering a continuous version $\hatF$ of $F$ that maps $\vx \mapsto \E_{\hatvx \sim \cB_{\delta}(\vx) \cap \cX} F(\hatvx)$ (\Cref{lemma:cont}). Here, $\cB_\delta(\vx)$ denotes the Euclidean ball centered at $\vx$ with radius $\delta = \delta(\epsilon)$. 
% It then suffices to invoke Brouwer's fixed-point theorem for the gradient mapping $\vx \mapsto \proj_\cX ( \vx - \hatF(\vx) )$, where $\proj_\cX$ is the Euclidean projection with respect to $\cX$.
%Then, a uniform distribution around a VI solution $\vx^*$ to $\hatF$ makes makes an $\epsilon$-approximate solution to the $\Phi$-EVI problem for $F$.

\Cref{theorem:existence} implies that a $\Phi$-EVI can have approximate solutions even when the associated VI problem does not.\footnote{Noncontinuity of $F$ manifests itself prominently in \emph{nonsmooth} optimization (\emph{e.g.}, \citealp{Zhang20:Complexity,Davis22:Gradient,Tian22:Finite,Jordan23:Deterministic}); recent research there focuses on \emph{Goldstein} stationary points~\citep{Goldstein77:Optimization}, which are conceptually related to EVIs.}
%This is a very nice moytivation footnote. However, it is not clear why nonsmoothness relates to noncontinuity of F. Explain.???
%Also, if space allows, move the footnote into the body???

\begin{restatable}{corollary}{VIsvsEVIs}
    \label{prop:VI-vs-EVIs}
    There exists a VI problem that does not admit approximate solutions when $\epsilon = \Theta(1)$, but the corresponding $\epsilon$-approximate $\Phi$-EVI is total for any $\epsilon > 0$.
\end{restatable}

In the proof, we set $F$ to be the \emph{sign function} (\Cref{ex:sign-evi}). By contrast, if one insists on exact solutions, EVIs do not necessarily admit solutions.

\begin{restatable}{proposition}{notexact}
    \label{prop:notexact}
    When $F$ is not continuous, there exists an EVI problem with no solutions.
\end{restatable}

Furthermore, \Cref{theorem:existence} raises the question of whether it is enough to instead assume that every $\phi \in \Phi$ is continuous. Our next result dispels any such hopes.

\begin{restatable}{theorem}{countercont}
    \label{theorem:continuous-counterexample}
    There are $\Phi$-EVI instances that do not admit $\eps$-approximate solutions even when $\eps = \Theta(1)$, $F$ is piecewise constant, and $\Phi$ contains only continuous functions.
\end{restatable}

Our final result on existence complements~\Cref{theorem:existence,theorem:continuous-counterexample} by showing that, when $\Phi$ is finite-dimensional, it is enough if every $\phi \in \Phi$ admits a fixed point (this holds, for example, when $\phi$ is continuous---by Brouwer's theorem).

\begin{restatable}{theorem}{finitedim}
    \label{theorem:finitedim}
    Suppose that 
    \begin{enumerate}%[noitemsep,topsep=0pt]
        \item $\Phi$ is finite-dimensional, that is, there exists $k \in \N$ and a kernel map $m : \cX \to \R^k$ such that every $\phi \in \Phi$ can be expressed as $\mK m(\vx)$ for some $\mK \in \R^{d \times k}$; and
        \item every $\phi \in \Phi$ admits a fixed point, that is, a point $\cX \ni \vx = \fix(\phi)$ such that $\phi(\vx) = \vx$.
    \end{enumerate} Then, the $\Phi$-EVI problem admits an $\eps$-approximate solution with support size at most $1+dk$ for every $\eps > 0$.
\end{restatable}

Notably, this theorem guarantees the existence of solutions with finite support; the proof makes use of the minimax theorem~(\eg,~\citealp{Sion58:On}) in conjunction with Carath\'eodory's theorem on convex hulls~\cite{Caratheodory11:Uber}.

\paragraph{Complexity} Having established some basic existence properties, we now turn to the complexity of $\Phi$-EVIs. Let us define the VI gap function $\VIgap(\vx) \defeq - \min_{\vx' \in \cX} \langle F(\vx), \vx' - \vx \rangle$, which is nonnegative. If we place no restrictions on $\Phi$, it turns out that $\Phi$-EVIs are tantamount to regular VIs:


\begin{restatable}{proposition}{EVIequiv}
    \label{prop:EVI-VI}
    If $\Phi$ contains all measurable functions from $\cX$ to $\cX$, then any solution $\mu \in \Delta(\cX)$ to the $\epsilon$-approximate $\Phi$-EVI problem satisfies
    \begin{equation}
        \label{eq:EVI-VI}
        \E_{\vx \sim \mu} \VIgap(\vx) \leq \epsilon. 
    \end{equation}
\end{restatable}

In proof, it suffices to consider a $\phi$ that maps $\vx \in \cX$ to an appropriate point in $\argmin_{\vx' \in \cX} \langle F(\vx), \vx' - \vx \rangle$. When $\mu$ must be given explicitly, \Cref{prop:EVI-VI} immediately implies that $\Phi$-EVIs are computationally hard, because \eqref{eq:EVI-VI} implies that $\VIgap(\vx) \leq \epsilon$ for some $\vx$ in the support of $\mu$, and such a point can be identified in polynomial time.\footnote{This argument carries over without restricting the support of $\mu$, by assuming instead access to a sampling oracle from $\mu$: a standard Chernoff bound implies that the empirical distribution (w.r.t. a large enough sample size) approximately satisfies~\eqref{eq:EVI-VI}.}

\begin{corollary}
    \label{cor:hard-EVI}
    The $\epsilon$-approximate $\Phi$-EVI problem is \PPAD-hard even when $\epsilon$ is an absolute constant and $F$ is linear.
\end{corollary}

Coupled with~\Cref{prop:EVI-VI}, this follows from the hardness result of~\citet{Rubinstein15:Inapproximability} concerning Nash equilibria in (multi-player) polymatrix games (for binary-action, graphical games, \citealp{Deligkas23:Tight} recently showed that \PPAD-hardness persists up to $\epsilon < \nicefrac{1}{2}$). \Cref{cor:hard-EVI} notwithstanding, it is easy to see that the set of solutions to $\Phi$-EVIs is convex for any $\Phi \subseteq \cX^\cX$.

\begin{remark}
    Let $\cX = \cX_1 \times \dots \times \cX_n$, as in an $n$-player game. Whether \Cref{cor:hard-EVI} applies under deviations that can be decomposed as $\phi : \vx \mapsto \phi(\vx) = (\phi_1(\vx_1), \dots, \phi_n(\vx_n))$ is a major open question in the regime where $\epsilon \ll 1$ (\emph{cf.}~\citealp{Dagan24:From,Peng24:Fast}).
\end{remark}

Viewed differently, a special case of the $\Phi$-EVI problem arises when $\Phi = \{ \phi \}$ and $F(\vx) = \vx - \phi(\vx)$, for some fixed map $\phi : \cX \to \cX$. In this case, the $\Phi$-EVI problem reduces to finding a $\mu \in \Delta(\cX)$ such that
\begin{equation}
    \label{eq:fp}
    \E_{\vx\sim\mu} \ip{F(\vx), \phi(\vx) - \vx} = - \E_{\vx\sim\mu} \norm{\phi(\vx) - \vx}^2 \geq -\epsilon.
\end{equation}
As a result, $\mu$ must contain in its support an $\epsilon$-approximate fixed point of $\phi$, a problem which is \PPAD-hard already for quadratic functions~\citep{Zhang24:Efficient}.

\begin{corollary}
    \label{cor:hard-EVI-quad}
    The $\epsilon$-approximate $\Phi$-EVI problem is \PPAD-hard even when $\epsilon$ is an absolute constant, $F$ is quadratic, and $\Phi = \{ \phi \}$ for a quadratic map $\phi : \cX \to \cX$.
\end{corollary}

It is also worth noting that, unlike~\Cref{cor:hard-EVI}, $\Phi$ in the corollary above contains only continuous functions.

It also follows from~\eqref{eq:fp} that, for $\epsilon = 0$, $\Phi$-EVIs capture exact fixed points. The complexity class \FIXP~characterizes such problems~\citep{Etessami07:Complexity}.

\begin{corollary}
    The $\Phi$-EVI problem is \FIXP-hard, assuming that $\supp(\mu) \leq \poly(d)$.
\end{corollary}

Exponential lower bounds in terms of the number of function evaluations of $F$ also follow from~\citet{Hirsch89:Exponential}.

On a positive note, the next section establishes polynomial-time algorithms when $\Phi$ contains only \emph{linear} endomorphisms.\footnote{We do not distinguish between affine and linear maps because we can always set $\cX \gets \cX \times \{1\}$, in which case affine and linear maps coincide.}