\subsection{Further related work}
\label{sec:related}


We have seen that \Cref{def:evi} is strongly connected with the notion of $\Phi$-equilibria. 
In extensive-form games, the question of characterizing the set of deviations $\Phi$ that enables efficient learning---within the no-regret framework---and computation has attracted considerable attention. In particular, efficient algorithms have been established for \emph{extensive-form correlated equilibria (EFCEs)}~\citep{Huang08:Computing,Farina22:Simple,Morrill21:Efficient,Morrill21:Hindsight}, and more broadly, when $\Phi$ contains solely \emph{linear} functions~\citep{Farina24:Polynomial,Farina23:Polynomial}---corresponding to \emph{linear correlated equilibria (LCEs)}. Recently, \citet{Daskalakis24:Efficient} strengthened those results beyond extensive-form games whenever there is a separation oracle for the constraint set; we rely on their approach for some of our positive results. Moreover, \citet{Zhang24:Efficient,Zhang25:Learning} established certain positive results even when $\Phi$ contains low-degree polynomials; by contrast, in our setting, $\Phi$-EVIs are hard even when $\Phi$ contains only quadratic polynomials (\Cref{cor:hard-EVI-quad}).

Besides encompassing correlated equilibria in games, wherein the constraint set $\cX$ can be decomposed as a Cartesian product over the constraint set of each player (reflecting the fact that players select strategies independently), our positive results pertaining to \Cref{def:evi} do not rest on such assumptions and apply even in the presence of joint constraint sets. There is a long history in game theory, optimization, and economics pertaining to such settings---sometimes referred to as ``pseudo-games'' in the literature~\citep{Goktas22:Exploitability,Arrow54:Existence,Facchinei10:Generalized,Fischer14:Generalized,Facchinei09:Generalized,Ardagna12:Generalized,Tatarenko18:Learning,Jordan23:First,Daskalakis21:Complexity}. The notion of \emph{generalized} Nash equilibria---the natural counterpart of Nash's solution concept in the presence of coupled constraints---has dominated that line of work, with the recent paper of~\citet{Bernasconi23:Constrained} being the notable exception.

Responding to the call of~\citet{Daskalakis22:Non}, \citet{Cai24:Tractable} and~\citet{Ahunbay25:First} recently proposed several tractable solution concepts in games with nonconcave utilities. In particular, when specialized to games, $\Phi$-EVIs are closely related to a notion proposed by~\citet[Definition 6]{Ahunbay25:First}. One of our key results, \Cref{th:elvi}, establishes a $\poly(d, \log(1/\epsilon))$-time algorithm, while the algorithms of~\citet{Cai24:Accelerated} and~\citet{Ahunbay25:First} scale polynomially in $1/\epsilon$. Also, \Cref{th:elvi} applies even when $\Phi$ contains all linear endomorphisms (\Cref{th:elvi}). From a more conceptual vantage point, a significant part of our contribution is to extend the scope of such results beyond games, as we have already highlighted.

\citet{Kapron24:Computational} and~\citet{Bernasconi24:Role} recently studied the computational complexity of VIs, and generalizations thereof---namely, \emph{quasi VIs}, establishing \PPAD-completeness under mild assumptions. Whether our framework can be extended to encompass quasi VIs is left as an interesting direction for the future.

Finally, it would be remiss not to point out that our VI relaxation is in the spirit of ``lifting,'' a standard technique whereby the original problem is \emph{lifted} to a higher-dimensional space to gain more analytical and computational leverage; a concrete example, in the context of optimal transport theory, is Kantorovich's relaxation of Monge's formulation~\citep{Billani09:Optimal}. Such techniques have been fruitful in the context of min-max optimization~\citep{Hsieh19:Finding,Domingo-Enrich20:Mean}.