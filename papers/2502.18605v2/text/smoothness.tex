\section{Performance Guarantees for EVIs}
\label{sec:smoothness}

In many settings, a VI solution is used as a proxy to approximately maximize some underlying objective function; machine learning offers many such applications. The question is whether performance guarantees pertaining to VIs can be extended---potentially with some small degradation---to EVIs as well. The purpose of this section is to provide a framework for achieving that based on the following notion.

\begin{definition}
    \label{def:smoothness}
    An EVI problem is \emph{$(\lambda, \nu)$-smooth}, for $\lambda > 0, \nu > -1$, w.r.t. $W : \cX \to \R$ and $\vxstar \in \argmax_{\vx} W(\vx)$ if
    \begin{equation*}
        \langle F(\vx), \vxstar - \vx \rangle \leq -  \lambda W(\vxstar) + (\nu+1) W(\vx) \quad \forall \vx \in \cX.
    \end{equation*}
\end{definition}

\begin{example}
    When the underlying problem corresponds to a multi-player game and $W$ is the (utilitarian) social welfare, \Cref{def:smoothness} coincides with the celebrated notion of smoothness \emph{\`a la}~\citet{Roughgarden15:Intrinsic}; this is a consequence of multilinearity, which implies that $W(\vx) = -\langle \vx, F(\vx) \rangle$ and $ \langle \vxstar, F(\vx) \rangle = - \sum_{i=1}^n u_i(\vxstar_i, \vx_{-i})$ for all $\vx \in \cX$. We also refer to the recent treatment of smoothness by~\citet{Ahunbay25:First} in the context of nonconcave games, which builds on the primal-dual framework of~\citet{Nadav10:Limits}.
\end{example}

\Cref{def:smoothness} is an extension of the more general notion of ``local smoothness,'' introduced by~\citet{Roughgarden15:Local} in the context of splittable congestion games. However, it goes beyond games. Indeed, the following definition we introduce generalizes \Cref{def:quasar}, making a new connection between smoothness and quasar-concavity.

\begin{definition}[Extension of quasar-concavity]
    \label{def:smooth-fun}
    Let $\vxstar \in \cX$ be a maximizer of a differentiable function $u : \cX \to \R$. We say that $u$ is \emph{$(\lambda, \nu)$-smooth} with respect to $\vxstar$ if
    \begin{equation*}
        \langle \nabla u(\vx), \vxstar - \vx \rangle \geq \lambda u(\vxstar) - (\nu + 1) u(\vx) \quad \forall \vx \in \cX.
    \end{equation*}
\end{definition}

In particular, when $\lambda \defeq \gamma$ and $\nu \defeq \gamma - 1$, the above definition captures $\gamma$-quasar-concavity. In~\Cref{sec:appendix-smooth}, we provide an example of a polynomial that satisfies~\Cref{def:smooth-fun} without being quasar-concave. Now, the key property of~\Cref{def:smoothness} is that any EVI solution approximates the underlying objective---by a factor of $\rho \defeq \nicefrac{\lambda}{1 + \nu}$.

\begin{theorem}
    \label{theorem:smoothness}
    Let $\mu \in \Delta(\cX)$ be an $\epsilon$-approximate solution to a $(\lambda, \nu)$-smooth EVI problem w.r.t. $W : \cX \to \R$. Then,
    \begin{equation*}
        \E_{\vx \sim \mu} W(\vx)  \geq \frac{\lambda}{1 + \nu} \max_{\vx \in \cX} W(\vx) - \frac{\epsilon}{1 + \nu}.
    \end{equation*}
\end{theorem}

The proof follows directly from~\Cref{def:smoothness}, using that $\E_{\vx \sim \mu} \langle F(\vx), \vxstar - \vx \rangle \geq - \epsilon$ and linearity of expectation.