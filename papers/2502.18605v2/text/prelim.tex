\section{Preliminaries}
\label{sec:prel}

This section provides some basic notation and background together with an overview of the $\eah$ algorithm. Additional preliminaries, which are not necessary for the main body, are given later in~\Cref{sec:add-prels}.

\paragraph{Notation} We use boldface, lowercase letters, such as $\vx$ and $\vy$, to denote vectors in a Euclidean space. Capital, boldface letters, such as $\mat{A}$, represent matrices. For $\vx, \vx' \in \R^d$, we use $\langle \vx, \vx' \rangle$ to denote their inner product. $\|\vx\| \defeq \sqrt{ \langle \vx, \vx \rangle}$ is the Euclidean norm of $\vx$. $\cB_r(\vx)$ is the (closed) Euclidean ball centered at $\vx$ with radius $r > 0$. $\conv(\cdot)$ represents the convex hull. An \emph{endomorphism} on $\cX$ is a function mapping $\cX$ to $\cX$.

Returning to~\Cref{def:evi}, for computational purposes, we assume throughout that $F$ has an explicit polynomial representation, so that $F(\vx) \in \R^d$ can be evaluated in $\poly(d)$ time. Further, there exists $B > 0$ such that $\norm{F(\vx)} \leq B$ for all $\vx \in \cX$. We will also restrict the support $\supp(\mu)$ of $\mu$ to be $\poly(d, 1/\epsilon)$, unless stated otherwise. With regard to $\cX$, we assume that we have \emph{oracle access}. In particular, we consider the following three types of oracle access.

\begin{itemize}[nosep]
    \item \emph{Membership}: given $\vx \in \R^d$, decide whether $\vx \in \cX$.
    \item \emph{Separation}: given $\vx \in \R^d$, decide whether $\vx \in \cX$; if not, return a hyperplane that \emph{separates} $\vx$ from $\cX$.
    \item \emph{Linear optimization}: Given $\vec{u} \in \R^d$, return a vector in $\argmax_{\vx \in \cX} \langle \vx, \vec{u} \rangle $.
\end{itemize}

In addition, we will assume that $\cX \subseteq \cB_R(\vec{0})$ for some $R \leq \poly(d)$, and $\cX$ contains a ball of radius $1$ in its relative interior; this is a standard regularity condition that ensures $\cX$ is geometrically well-behaved, which can be met by bringing $\cX$ into isotropic position (\Cref{sec:add-prels}). Under this assumption, the three oracles listed above are polynomially equivalent~\citep{Grotschel12:Geometric,Groetschel81:ellipsoid}. As a result, we may assume that $\cX$ is given implicitly via a ($\poly(d)$-time) membership oracle, which suffices for~\Cref{th:elvi}.

% For many important special cases, $\cX$ is given explicitly through either the \emph{$V$- or $H$-representation} (\Cref{def:repr}).

All our positive results with respect to the set of linear endomorphisms $\Philin$ readily carry over to affine endomorphisms as well.

\subsection{Ellipsoid against hope}
\label{sec:eah}

This \emph{ellipsoid against hope ($\eah$)} algorithm was famously introduced by~\citet{Papadimitriou08:Computing} to compute correlated equilibria in multi-player games. We proceed with an overview of $\eah$, and in particular a generalized version thereof, crystallized by~\citet{Farina24:Polynomial}.

Consider an arbitrary optimization problem of the form
\begin{equation}\label{eq:eah}
    \qq{find} \mu \in \Delta(\cX) \qq{s.t.} \E_{\vx \sim \mu} \ip{\vy, G(\vx)} \ge 0 \text{ } \forall \vy \in \cY,
\end{equation}
where $\cY \subseteq \R^m$, and $G : \cX \to \R^m$ is an arbitrary function. Suppose that we are given an evaluation oracle for $G$ and a separation oracle for $\cY$. Assume further that we are given a {\em good-enough-response ($\ger$)} oracle, which, given any $\vy \in \cY$, returns $\vx \in \cX$ such that $\ip{\vy, G(\vx)} \ge 0$. The upshot is that $\eah$ enables us to solve~\eqref{eq:eah} with just the above tools. Indeed, consider the following problem, which is an $\eps$-approximate version of the dual of \eqref{eq:eah}.
\begin{equation} \label{eq:eah-dual}
    \qq{find} \vy \in \cY \qq{s.t.} \ip{\vy, G(\vx)} \le -\eps \quad \forall \vx \in \cX.
\end{equation}
Since a $\ger$ oracle exists, \eqref{eq:eah-dual} is infeasible. What is more, a certificate of infeasibility of \eqref{eq:eah-dual} yields an $\eps$-approximate solution to \eqref{eq:eah}. It thus suffices to run the ellipsoid algorithm on \eqref{eq:eah-dual} and extract a certificate of infeasibility; in a nutshell, this is what $\eah$ does (\emph{cf.}~\Cref{alg:eah} in~\Cref{sec:proofs}).

\begin{theorem}[Generalized form of $\eah$; \citealp{Farina24:Polynomial}]
    \label{theorem:eah}
    Given a $\ger$ oracle and a separation oracle ($\sep$) for $\cY$, $\eah$ runs in time $\poly(d, m, \log(1/\epsilon))$ and returns an $\eps$-approximate solution to \eqref{eq:eah}.
\end{theorem}

One of our main results (\Cref{th:elvi}) crucially hinges on a strengthening of~\Cref{theorem:eah} due to~\citet{Daskalakis24:Efficient}, discussed further in~\Cref{sec:main} and \Cref{sec:mainproof}.