\documentclass[11pt]{article}

% Recommended, but optional, packages for figures and better typesetting:
\usepackage{microtype}
\usepackage{graphicx}
\usepackage{subfigure}
\usepackage{booktabs} % for professional tables


\usepackage{hyperref}
\usepackage[table]{xcolor}         % colors

\newcommand{\declarecolor}[2]{\definecolor{#1}{RGB}{#2}\expandafter\newcommand\csname #1\endcsname[1]{\textcolor{#1}{##1}}}
\declarecolor{White}{255, 255, 255}
\declarecolor{Black}{0, 0, 0}
\declarecolor{Maroon}{128, 0, 0}
\declarecolor{Coral}{255, 127, 80}
\declarecolor{Red}{182, 21, 21}
\declarecolor{LimeGreen}{50, 205, 50}
\declarecolor{DarkGreen}{0, 80, 0}
\declarecolor{Purple}{146, 42, 158}
\declarecolor{Navy}{0, 0, 128}
\declarecolor{LightBlue}{84, 101, 202}
\definecolor{mydarkblue}{rgb}{0,0.08,0.45}
\hypersetup{ %
    pdftitle={},
    pdfkeywords={},
    pdfborder=0 0 0,
    pdfpagemode=UseNone,
    colorlinks=true,
    linkcolor=Navy,
    citecolor=DarkGreen,
    filecolor=Purple,
    urlcolor=Purple,
}

\usepackage{algorithm}
\usepackage{algorithmic}


\usepackage{natbib}
\bibliographystyle{plainnat}

%%%%%%%%%%%---SETME-----%%%%%%%%%%%%%
%replace @@ with the submission number submission site.
\newcommand{\thiswork}{INF$^2$\xspace}
%%%%%%%%%%%%%%%%%%%%%%%%%%%%%%%%%%%%


%\newcommand{\rev}[1]{{\color{olivegreen}#1}}
\newcommand{\rev}[1]{{#1}}


\newcommand{\JL}[1]{{\color{cyan}[\textbf{\sc JLee}: \textit{#1}]}}
\newcommand{\JW}[1]{{\color{orange}[\textbf{\sc JJung}: \textit{#1}]}}
\newcommand{\JY}[1]{{\color{blue(ncs)}[\textbf{\sc JSong}: \textit{#1}]}}
\newcommand{\HS}[1]{{\color{magenta}[\textbf{\sc HJang}: \textit{#1}]}}
\newcommand{\CS}[1]{{\color{navy}[\textbf{\sc CShin}: \textit{#1}]}}
\newcommand{\SN}[1]{{\color{olive}[\textbf{\sc SNoh}: \textit{#1}]}}

%\def\final{}   % uncomment this for the submission version
\ifdefined\final
\renewcommand{\JL}[1]{}
\renewcommand{\JW}[1]{}
\renewcommand{\JY}[1]{}
\renewcommand{\HS}[1]{}
\renewcommand{\CS}[1]{}
\renewcommand{\SN}[1]{}
\fi

%%% Notion for baseline approaches %%% 
\newcommand{\baseline}{offloading-based batched inference\xspace}
\newcommand{\Baseline}{Offloading-based batched inference\xspace}


\newcommand{\ans}{attention-near storage\xspace}
\newcommand{\Ans}{Attention-near storage\xspace}
\newcommand{\ANS}{Attention-Near Storage\xspace}

\newcommand{\wb}{delayed KV cache writeback\xspace}
\newcommand{\Wb}{Delayed KV cache writeback\xspace}
\newcommand{\WB}{Delayed KV Cache Writeback\xspace}

\newcommand{\xcache}{X-cache\xspace}
\newcommand{\XCACHE}{X-Cache\xspace}


%%% Notions for our methods %%%
\newcommand{\schemea}{\textbf{Expanding supported maximum sequence length with optimized performance}\xspace}
\newcommand{\Schemea}{\textbf{Expanding supported maximum sequence length with optimized performance}\xspace}

\newcommand{\schemeb}{\textbf{Optimizing the storage device performance}\xspace}
\newcommand{\Schemeb}{\textbf{Optimizing the storage device performance}\xspace}

\newcommand{\schemec}{\textbf{Orthogonally supporting Compression Techniques}\xspace}
\newcommand{\Schemec}{\textbf{Orthogonally supporting Compression Techniques}\xspace}



% Circular numbers
\usepackage{tikz}
\newcommand*\circled[1]{\tikz[baseline=(char.base)]{
            \node[shape=circle,draw,inner sep=0.4pt] (char) {#1};}}

\newcommand*\bcircled[1]{\tikz[baseline=(char.base)]{
            \node[shape=circle,draw,inner sep=0.4pt, fill=black, text=white] (char) {#1};}}


% Attempt to make hyperref and algorithmic work together better:
%\newcommand{\theHalgorithm}{\arabic{algorithm}}


\usepackage{multirow}

%\usepackage[ruled,linesnumbered]{algorithm2e}

% If accepted, instead use the following line for the camera-ready submission:
%\usepackage[accepted]{icml2025}

% For theorems and such
\usepackage{amsmath}
\usepackage{amssymb}
\usepackage{mathtools}
\usepackage{amsthm}

\usepackage{tikz}
\usepackage{pgfplots}

\usepackage{thm-restate}

\usepackage{fullpage}

% if you use cleveref..
%\usepackage{zref-clever}
%\newcommand{\Cref}[1]{{\zcref[S]{#1}}}
%\let\cref\Cref

\usepackage[capitalize,noabbrev,nameinlink]{cleveref}

\newcommand{\newzreftheorem}[2]{
    \newtheorem{#1}[theorem]{#2}
    \zcRefTypeSetup{#1}{Name-sg = #2}
}
%%%%%%%%%%%%%%%%%%%%%%%%%%%%%%%%
% THEOREMS
%%%%%%%%%%%%%%%%%%%%%%%%%%%%%%%%
\theoremstyle{plain}
\newtheorem{theorem}{Theorem}[section]
%\newzreftheorem{proposition}{Proposition}
\newtheorem{proposition}[theorem]{Proposition}
%\newzreftheorem{lemma}{Lemma}
\newtheorem{lemma}[theorem]{Lemma}
%\newzreftheorem{claim}{Claim}
\newtheorem{claim}[theorem]{Claim}
%\newzreftheorem{corollary}{Corollary}
\newtheorem{corollary}[theorem]{Corollary}
\theoremstyle{definition}
%\newzreftheorem{definition}{Definition}
\newtheorem{definition}[theorem]{Definition}
%\newzreftheorem{assumption}{Assumption}
\newtheorem{assumption}[theorem]{Assumption}
%\newzreftheorem{example}{Example}
\newtheorem{example}[theorem]{Example}
\theoremstyle{remark}
%\newzreftheorem{remark}{Remark}
\newtheorem{remark}[theorem]{Remark}
%\mathtoolsset{showonlyrefs=true}

\usepackage{nicefrac}
\usepackage{enumitem}

\usepackage{tikz}

\usepackage{MnSymbol}
\renewcommand\vec\bm
\renewcommand\grad\nabla

\newcommand*\tcircle[1]{%
  \raisebox{-0.5pt}{%
    \textcircled{\fontsize{7pt}{0}\fontfamily{phv}\selectfont #1}%
  }%
}

% \makeatletter
% \autonum@generatePatchedReferenceCSL{Cref}
% \autonum@generatePatchedReferenceCSL{cref}
% \makeatother

\newcommand{\bhz}[1]{{ {\color{magenta}{[BZ:~#1]}}}}
\newcommand{\et}[1]{{ {\color{olive}{[ET:~#1]}}}}
\newcommand{\emin}[1]{{ {\color{orange}{[EB:~#1]}}}}
\newcommand{\ioannis}[1]{{ {\color{blue}{[IA:~#1]}}}}
\newcommand{\gabri}[1]{{ {\color{red}{[GF:~#1]}}}}
\newcommand{\vince}[1]{{ {\color{green}{[vc:~#1]}}}}

\usepackage{authblk}

\title{Expected Variational Inequalities}

\author[1]{Brian Hu Zhang\thanks{Equal contribution.}}
\author[1]{Ioannis Anagnostides$^*$}
\author[1,2]{Emanuel Tewolde}
\author[1,2]{Ratip Emin Berker}
\author[3]{Gabriele Farina}
\author[1,2,4]{Vincent Conitzer}
\author[1,5]{Tuomas Sandholm}
\renewcommand\Affilfont{\small}
\affil[1]{Carnegie Mellon University}
\affil[2]{Foundations of Cooperative AI Lab (FOCAL)}
\affil[3]{Massachusetts Institute of Technology}
\affil[4]{University of Oxford}
\affil[5]{Additional affiliations: Strategy Robot, Inc., Strategic Machine, Inc., Optimized Markets, Inc.}
\affil[ ]{}
\affil[ ]{\texttt{\{bhzhang,ianagnos,etewolde,rberker,conitzer,sandholm\}}\texttt{@cs.cmu.edu}, \texttt{gfarina}\texttt{@mit.edu}}


\begin{document}

\maketitle

\pagenumbering{gobble}

\begin{abstract}
Variational inequalities (VIs) encompass many fundamental problems in diverse areas ranging from engineering to economics and machine learning. However, their considerable expressivity comes at the cost of computational intractability. In this paper, we introduce and analyze a natural relaxation---which we refer to as \emph{expected variational inequalities (EVIs)}---where the goal is to find a {\em distribution} that satisfies the VI constraint {\em in expectation}. By adapting recent techniques from game theory, we show that, unlike VIs, EVIs can be solved in polynomial time under general (nonmonotone) operators. EVIs capture the seminal notion of \emph{correlated equilibria}, but enjoy a greater reach beyond games. 
%
We also employ our framework to capture and generalize several existing disparate results, including from settings such as smooth games, and games with coupled constraints or nonconcave utilities. %Surprisingly, even in standard settings, our solution concept \emph{refines} correlated equilibria.

% Unlike VIs, we show that EVIs can be solved in polynomial time under broad assumptions. In doing so, we expand the scope of solution concepts and algorithmic techniques from game theory to a much broader class of problems.
\end{abstract}

\clearpage

\tableofcontents

\clearpage

\pagenumbering{arabic}

\section{Introduction}


\begin{figure}[t]
\centering
\includegraphics[width=0.6\columnwidth]{figures/evaluation_desiderata_V5.pdf}
\vspace{-0.5cm}
\caption{\systemName is a platform for conducting realistic evaluations of code LLMs, collecting human preferences of coding models with real users, real tasks, and in realistic environments, aimed at addressing the limitations of existing evaluations.
}
\label{fig:motivation}
\end{figure}

\begin{figure*}[t]
\centering
\includegraphics[width=\textwidth]{figures/system_design_v2.png}
\caption{We introduce \systemName, a VSCode extension to collect human preferences of code directly in a developer's IDE. \systemName enables developers to use code completions from various models. The system comprises a) the interface in the user's IDE which presents paired completions to users (left), b) a sampling strategy that picks model pairs to reduce latency (right, top), and c) a prompting scheme that allows diverse LLMs to perform code completions with high fidelity.
Users can select between the top completion (green box) using \texttt{tab} or the bottom completion (blue box) using \texttt{shift+tab}.}
\label{fig:overview}
\end{figure*}

As model capabilities improve, large language models (LLMs) are increasingly integrated into user environments and workflows.
For example, software developers code with AI in integrated developer environments (IDEs)~\citep{peng2023impact}, doctors rely on notes generated through ambient listening~\citep{oberst2024science}, and lawyers consider case evidence identified by electronic discovery systems~\citep{yang2024beyond}.
Increasing deployment of models in productivity tools demands evaluation that more closely reflects real-world circumstances~\citep{hutchinson2022evaluation, saxon2024benchmarks, kapoor2024ai}.
While newer benchmarks and live platforms incorporate human feedback to capture real-world usage, they almost exclusively focus on evaluating LLMs in chat conversations~\citep{zheng2023judging,dubois2023alpacafarm,chiang2024chatbot, kirk2024the}.
Model evaluation must move beyond chat-based interactions and into specialized user environments.



 

In this work, we focus on evaluating LLM-based coding assistants. 
Despite the popularity of these tools---millions of developers use Github Copilot~\citep{Copilot}---existing
evaluations of the coding capabilities of new models exhibit multiple limitations (Figure~\ref{fig:motivation}, bottom).
Traditional ML benchmarks evaluate LLM capabilities by measuring how well a model can complete static, interview-style coding tasks~\citep{chen2021evaluating,austin2021program,jain2024livecodebench, white2024livebench} and lack \emph{real users}. 
User studies recruit real users to evaluate the effectiveness of LLMs as coding assistants, but are often limited to simple programming tasks as opposed to \emph{real tasks}~\citep{vaithilingam2022expectation,ross2023programmer, mozannar2024realhumaneval}.
Recent efforts to collect human feedback such as Chatbot Arena~\citep{chiang2024chatbot} are still removed from a \emph{realistic environment}, resulting in users and data that deviate from typical software development processes.
We introduce \systemName to address these limitations (Figure~\ref{fig:motivation}, top), and we describe our three main contributions below.


\textbf{We deploy \systemName in-the-wild to collect human preferences on code.} 
\systemName is a Visual Studio Code extension, collecting preferences directly in a developer's IDE within their actual workflow (Figure~\ref{fig:overview}).
\systemName provides developers with code completions, akin to the type of support provided by Github Copilot~\citep{Copilot}. 
Over the past 3 months, \systemName has served over~\completions suggestions from 10 state-of-the-art LLMs, 
gathering \sampleCount~votes from \userCount~users.
To collect user preferences,
\systemName presents a novel interface that shows users paired code completions from two different LLMs, which are determined based on a sampling strategy that aims to 
mitigate latency while preserving coverage across model comparisons.
Additionally, we devise a prompting scheme that allows a diverse set of models to perform code completions with high fidelity.
See Section~\ref{sec:system} and Section~\ref{sec:deployment} for details about system design and deployment respectively.



\textbf{We construct a leaderboard of user preferences and find notable differences from existing static benchmarks and human preference leaderboards.}
In general, we observe that smaller models seem to overperform in static benchmarks compared to our leaderboard, while performance among larger models is mixed (Section~\ref{sec:leaderboard_calculation}).
We attribute these differences to the fact that \systemName is exposed to users and tasks that differ drastically from code evaluations in the past. 
Our data spans 103 programming languages and 24 natural languages as well as a variety of real-world applications and code structures, while static benchmarks tend to focus on a specific programming and natural language and task (e.g. coding competition problems).
Additionally, while all of \systemName interactions contain code contexts and the majority involve infilling tasks, a much smaller fraction of Chatbot Arena's coding tasks contain code context, with infilling tasks appearing even more rarely. 
We analyze our data in depth in Section~\ref{subsec:comparison}.



\textbf{We derive new insights into user preferences of code by analyzing \systemName's diverse and distinct data distribution.}
We compare user preferences across different stratifications of input data (e.g., common versus rare languages) and observe which affect observed preferences most (Section~\ref{sec:analysis}).
For example, while user preferences stay relatively consistent across various programming languages, they differ drastically between different task categories (e.g. frontend/backend versus algorithm design).
We also observe variations in user preference due to different features related to code structure 
(e.g., context length and completion patterns).
We open-source \systemName and release a curated subset of code contexts.
Altogether, our results highlight the necessity of model evaluation in realistic and domain-specific settings.





\putsec{related}{Related Work}

\noindent \textbf{Efficient Radiance Field Rendering.}
%
The introduction of Neural Radiance Fields (NeRF)~\cite{mil:sri20} has
generated significant interest in efficient 3D scene representation and
rendering for radiance fields.
%
Over the past years, there has been a large amount of research aimed at
accelerating NeRFs through algorithmic or software
optimizations~\cite{mul:eva22,fri:yu22,che:fun23,sun:sun22}, and the
development of hardware
accelerators~\cite{lee:cho23,li:li23,son:wen23,mub:kan23,fen:liu24}.
%
The state-of-the-art method, 3D Gaussian splatting~\cite{ker:kop23}, has
further fueled interest in accelerating radiance field
rendering~\cite{rad:ste24,lee:lee24,nie:stu24,lee:rho24,ham:mel24} as it
employs rasterization primitives that can be rendered much faster than NeRFs.
%
However, previous research focused on software graphics rendering on
programmable cores or building dedicated hardware accelerators. In contrast,
\name{} investigates the potential of efficient radiance field rendering while
utilizing fixed-function units in graphics hardware.
%
To our knowledge, this is the first work that assesses the performance
implications of rendering Gaussian-based radiance fields on the hardware
graphics pipeline with software and hardware optimizations.

%%%%%%%%%%%%%%%%%%%%%%%%%%%%%%%%%%%%%%%%%%%%%%%%%%%%%%%%%%%%%%%%%%%%%%%%%%
\myparagraph{Enhancing Graphics Rendering Hardware.}
%
The performance advantage of executing graphics rendering on either
programmable shader cores or fixed-function units varies depending on the
rendering methods and hardware designs.
%
Previous studies have explored the performance implication of graphics hardware
design by developing simulation infrastructures for graphics
workloads~\cite{bar:gon06,gub:aam19,tin:sax23,arn:par13}.
%
Additionally, several studies have aimed to improve the performance of
special-purpose hardware such as ray tracing units in graphics
hardware~\cite{cho:now23,liu:cha21} and proposed hardware accelerators for
graphics applications~\cite{lu:hua17,ram:gri09}.
%
In contrast to these works, which primarily evaluate traditional graphics
workloads, our work focuses on improving the performance of volume rendering
workloads, such as Gaussian splatting, which require blending a huge number of
fragments per pixel.

%%%%%%%%%%%%%%%%%%%%%%%%%%%%%%%%%%%%%%%%%%%%%%%%%%%%%%%%%%%%%%%%%%%%%%%%%%
%
In the context of multi-sample anti-aliasing, prior work proposed reducing the
amount of redundant shading by merging fragments from adjacent triangles in a
mesh at the quad granularity~\cite{fat:bou10}.
%
While both our work and quad-fragment merging (QFM)~\cite{fat:bou10} aim to
reduce operations by merging quads, our proposed technique differs from QFM in
many aspects.
%
Our method aims to blend \emph{overlapping primitives} along the depth
direction and applies to quads from any primitive. In contrast, QFM merges quad
fragments from small (e.g., pixel-sized) triangles that \emph{share} an edge
(i.e., \emph{connected}, \emph{non-overlapping} triangles).
%
As such, QFM is not applicable to the scenes consisting of a number of
unconnected transparent triangles, such as those in 3D Gaussian splatting.
%
In addition, our method computes the \emph{exact} color for each pixel by
offloading blending operations from ROPs to shader units, whereas QFM
\emph{approximates} pixel colors by using the color from one triangle when
multiple triangles are merged into a single quad.


\section{Preliminaries}
\label{sec:prelim}
\label{sec:term}
We define the key terminologies used, primarily focusing on the hidden states (or activations) during the forward pass. 

\paragraph{Components in an attention layer.} We denote $\Res$ as the residual stream. We denote $\Val$ as Value (states), $\Qry$ as Query (states), and $\Key$ as Key (states) in one attention head. The \attlogit~represents the value before the softmax operation and can be understood as the inner product between  $\Qry$  and  $\Key$. We use \Attn~to denote the attention weights of applying the SoftMax function to \attlogit, and ``attention map'' to describe the visualization of the heat map of the attention weights. When referring to the \attlogit~from ``$\tokenB$'' to  ``$\tokenA$'', we indicate the inner product  $\langle\Qry(\tokenB), \Key(\tokenA)\rangle$, specifically the entry in the ``$\tokenB$'' row and ``$\tokenA$'' column of the attention map.

\paragraph{Logit lens.} We use the method of ``Logit Lens'' to interpret the hidden states and value states \citep{belrose2023eliciting}. We use \logit~to denote pre-SoftMax values of the next-token prediction for LLMs. Denote \readout~as the linear operator after the last layer of transformers that maps the hidden states to the \logit. 
The logit lens is defined as applying the readout matrix to residual or value states in middle layers. Through the logit lens, the transformed hidden states can be interpreted as their direct effect on the logits for next-token prediction. 

\paragraph{Terminologies in two-hop reasoning.} We refer to an input like “\Src$\to$\brga, \brgb$\to$\Ed” as a two-hop reasoning chain, or simply a chain. The source entity $\Src$ serves as the starting point or origin of the reasoning. The end entity $\Ed$ represents the endpoint or destination of the reasoning chain. The bridge entity $\Brg$ connects the source and end entities within the reasoning chain. We distinguish between two occurrences of $\Brg$: the bridge in the first premise is called $\brga$, while the bridge in the second premise that connects to $\Ed$ is called $\brgc$. Additionally, for any premise ``$\tokenA \to \tokenB$'', we define $\tokenA$ as the parent node and $\tokenB$ as the child node. Furthermore, if at the end of the sequence, the query token is ``$\tokenA$'', we define the chain ``$\tokenA \to \tokenB$, $\tokenB \to \tokenC$'' as the Target Chain, while all other chains present in the context are referred to as distraction chains. Figure~\ref{fig:data_illustration} provides an illustration of the terminologies.

\paragraph{Input format.}
Motivated by two-hop reasoning in real contexts, we consider input in the format $\bos, \text{context information}, \query, \answer$. A transformer model is trained to predict the correct $\answer$ given the query $\query$ and the context information. The context compromises of $K=5$ disjoint two-hop chains, each appearing once and containing two premises. Within the same chain, the relative order of two premises is fixed so that \Src$\to$\brga~always precedes \brgb$\to$\Ed. The orders of chains are randomly generated, and chains may interleave with each other. The labels for the entities are re-shuffled for every sequence, choosing from a vocabulary size $V=30$. Given the $\bos$ token, $K=5$ two-hop chains, \query, and the \answer~tokens, the total context length is $N=23$. Figure~\ref{fig:data_illustration} also illustrates the data format. 

\paragraph{Model structure and training.} We pre-train a three-layer transformer with a single head per layer. Unless otherwise specified, the model is trained using Adam for $10,000$ steps, achieving near-optimal prediction accuracy. Details are relegated to Appendix~\ref{app:sec_add_training_detail}.


% \RZ{Do we use source entity, target entity, and mediator entity? Or do we use original token, bridge token, end token?}





% \paragraph{Basic notations.} We use ... We use $\ve_i$ to denote one-hot vectors of which only the $i$-th entry equals one, and all other entries are zero. The dimension of $\ve_i$ are usually omitted and can be inferred from contexts. We use $\indicator\{\cdot\}$ to denote the indicator function.

% Let $V > 0$ be a fixed positive integer, and let $\vocab = [V] \defeq \{1, 2, \ldots, V\}$ be the vocabulary. A token $v \in \vocab$ is an integer in $[V]$ and the input studied in this paper is a sequence of tokens $s_{1:T} \defeq (s_1, s_2, \ldots, s_T) \in \vocab^T$ of length $T$. For any set $\mathcal{S}$, we use $\Delta(\mathcal{S})$ to denote the set of distributions over $\mathcal{S}$.

% % to a sequence of vectors $z_1, z_2, \ldots, z_T \in \real^{\dout}$ of dimension $\dout$ and length $T$.

% Let $\mU = [\vu_1, \vu_2, \ldots, \vu_V]^\transpose \in \real^{V\times d}$ denote the token embedding matrix, where the $i$-th row $\vu_i \in \real^d$ represents the $d$-dimensional embedding of token $i \in [V]$. Similarly, let $\mP = [\vp_1, \vp_2, \ldots, \vp_T]^\transpose \in \real^{T\times d}$ denote the positional embedding matrix, where the $i$-th row $\vp_i \in \real^d$ represents the $d$-dimensional embedding of position $i \in [T]$. Both $\mU$ and $\mP$ can be fixed or learnable.

% After receiving an input sequence of tokens $s_{1:T}$, a transformer will first process it using embedding matrices $\mU$ and $\mP$ to obtain a sequence of vectors $\mH = [\vh_1, \vh_2, \ldots, \vh_T] \in \real^{d\times T}$, where 
% \[
% \vh_i = \mU^\transpose\ve_{s_i} + \mP^\transpose\ve_{i} = \vu_{s_i} + \vp_i.
% \]

% We make the following definitions of basic operations in a transformer.

% \begin{definition}[Basic operations in transformers] 
% \label{defn:operators}
% Define the softmax function $\softmax(\cdot): \real^d \to \real^d$ over a vector $\vv \in \real^d$ as
% \[\softmax(\vv)_i = \frac{\exp(\vv_i)}{\sum_{j=1}^d \exp(\vv_j)} \]
% and define the softmax function $\softmax(\cdot): \real^{m\times n} \to \real^{m \times n}$ over a matrix $\mV \in \real^{m\times n}$ as a column-wise softmax operator. For a squared matrix $\mM \in \real^{m\times m}$, the causal mask operator $\mask(\cdot): \real^{m\times m} \to \real^{m\times m}$  is defined as $\mask(\mM)_{ij} = \mM_{ij}$ if $i \leq j$ and  $\mask(\mM)_{ij} = -\infty$ otherwise. For a vector $\vv \in \real^n$ where $n$ is the number of hidden neurons in a layer, we use $\layernorm(\cdot): \real^n \to \real^n$ to denote the layer normalization operator where
% \[
% \layernorm(\vv)_i = \frac{\vv_i-\mu}{\sigma}, \mu = \frac{1}{n}\sum_{j=1}^n \vv_j, \sigma = \sqrt{\frac{1}{n}\sum_{j=1}^n (\vv_j-\mu)^2}
% \]
% and use $\layernorm(\cdot): \real^{n\times m} \to \real^{n\times m}$ to denote the column-wise layer normalization on a matrix.
% We also use $\nonlin(\cdot)$ to denote element-wise nonlinearity such as $\relu(\cdot)$.
% \end{definition}

% The main components of a transformer are causal self-attention heads and MLP layers, which are defined as follows.

% \begin{definition}[Attentions and MLPs]
% \label{defn:attn_mlp} 
% A single-head causal self-attention $\attn(\mH;\mQ,\mK,\mV,\mO)$ parameterized by $\mQ,\mK,\mV \in \real^{{\dqkv\times \din}}$ and $\mO \in \real^{\dout\times\dqkv}$ maps an input matrix $\mH \in \real^{\din\times T}$ to
% \begin{align*}
% &\attn(\mH;\mQ,\mK,\mV,\mO) \\
% =&\mO\mV\layernorm(\mH)\softmax(\mask(\layernorm(\mH)^\transpose\mK^\transpose\mQ\layernorm(\mH))).
% \end{align*}
% Furthermore, a multi-head attention with $M$ heads parameterized by $\{(\mQ_m,\mK_m,\mV_m,\mO_m) \}_{m=1}^M$ is defined as 
% \begin{align*}
%     &\Attn(\mH; \{(\mQ_m,\mK_m,\mV_m,\mO_m) \}_{m\in[M]}) \\ =& \sum_{m=1}^M \attn(\mH;\mQ_m,\mK_m,\mV_m,\mO_m) \in \real^{\dout \times T}.
% \end{align*}
% An MLP layer $\mlp(\mH;\mW_1,\mW_2)$ parameterized by $\mW_1 \in \real^{\dhidden\times \din}$ and $\mW_2 \in \real^{\dout \times \dhidden}$ maps an input matrix $\mH = [\vh_1, \ldots, \vh_T] \in \real^{\din \times T}$ to
% \begin{align*}
%     &\mlp(\mH;\mW_1,\mW_2) = [\vy_1, \ldots, \vy_T], \\ \text{where } &\vy_i = \mW_2\nonlin(\mW_1\layernorm(\vh_i)), \forall i \in [T].
% \end{align*}

% \end{definition}

% In this paper, we assume $\din=\dout=d$ for all attention heads and MLPs to facilitate residual stream unless otherwise specified. Given \Cref{defn:operators,defn:attn_mlp}, we are now able to define a multi-layer transformer.

% \begin{definition}[Multi-layer transformers]
% \label{defn:transformer}
%     An $L$-layer transformer $\transformer(\cdot): \vocab^T \to \Delta(\vocab)$ parameterized by $\mP$, $\mU$, $\{(\mQ_m^{(l)},\mK_m^{(l)},\mV_m^{(l)},\mO_m^{(l)})\}_{m\in[M],l\in[L]}$,  $\{(\mW_1^{(l)},\mW_2^{(l)})\}_{l\in[L]}$ and $\Wreadout \in \real^{V \times d}$ receives a sequence of tokens $s_{1:T}$ as input and predict the next token by outputting a distribution over the vocabulary. The input is first mapped to embeddings $\mH = [\vh_1, \vh_2, \ldots, \vh_T] \in \real^{d\times T}$ by embedding matrices $\mP, \mU$ where 
%     \[
%     \vh_i = \mU^\transpose\ve_{s_i} + \mP^\transpose\ve_{i}, \forall i \in [T].
%     \]
%     For each layer $l \in [L]$, the output of layer $l$, $\mH^{(l)} \in \real^{d\times T}$, is obtained by 
%     \begin{align*}
%         &\mH^{(l)} =  \mH^{(l-1/2)} + \mlp(\mH^{(l-1/2)};\mW_1^{(l)},\mW_2^{(l)}), \\
%         & \mH^{(l-1/2)} = \mH^{(l-1)} + \\ & \quad \Attn(\mH^{(l-1)}; \{(\mQ_m^{(l)},\mK_m^{(l)},\mV_m^{(l)},\mO_m^{(l)}) \}_{m\in[M]}), 
%     \end{align*}
%     where the input $\mH^{(l-1)}$ is the output of the previous layer $l-1$ for $l > 1$ and the input of the first layer $\mH^{(0)} = \mH$. Finally, the output of the transformer is obtained by 
%     \begin{align*}
%         \transformer(s_{1:T}) = \softmax(\Wreadout\vh_T^{(L)})
%     \end{align*}
%     which is a $V$-dimensional vector after softmax representing a distribution over $\vocab$, and $\vh_T^{(L)}$ is the $T$-th column of the output of the last layer, $\mH^{(L)}$.
% \end{definition}



% For each token $v \in \vocab$, there is a corresponding $d_t$-dimensional token embedding vector $\embed(v) \in \mathbb{R}^{d_t}$. Assume the maximum length of the sequence studied in this paper does not exceed $T$. For each position $t \in [T]$, there is a corresponding positional embedding  







\section{Existence and Complexity Barriers}
\label{sec:existence}

Perhaps the most basic question about $\Phi$-EVIs concerns their \emph{totality}---the existence of solutions. If one is willing to tolerate an arbitrarily small imprecision $\epsilon > 0$, we show that solutions exist under very broad conditions.

\begin{restatable}{theorem}{mainexistence}
    \label{theorem:existence}
    Suppose that $F : \cX \to \R^d$ is measurable and there exists $L > 0$ such that every $\phi \in \Phi$ is $L$-Lipschitz continuous. Then, for any $\epsilon > 0$, there exists an $\epsilon$-approximate solution to the $\Phi$-EVI problem.
\end{restatable}


In particular, our existence proof does not rest on $F$ being continuous. 
Instead, we consider the continuous function $\hatF$ that maps $\vx \mapsto \E_{\hatvx \sim \Delta(\cB_{\delta}(\vx) \cap \cX)} F(\hatvx)$ (\Cref{lemma:cont}), where $\cB_\delta(\vx)$ is the Euclidean ball centered at $\vx$ with radius $\delta = \delta(\epsilon)$. It then suffices to invoke Brouwer's fixed-point theorem for the gradient mapping $\vx \mapsto \proj_\cX ( \vx - \hatF(\vx) )$, where $\proj_\cX$ is the Euclidean projection with respect to $\cX$.
%Then, a uniform distribution around a VI solution $\vx^*$ to $\hatF$ makes makes an $\epsilon$-approximate solution to the $\Phi$-EVI problem for $F$

%In contrast to the classical existence result for VIs (see, \eg, Section 3 in \citealp{KinderlehrerS00}), our existence proof does not rest on $F$ being continuous. Instead, we can invoke this known result by considering a continuous version $\hatF$ of $F$ that maps $\vx \mapsto \E_{\hatvx \sim \cB_{\delta}(\vx) \cap \cX} F(\hatvx)$ (\Cref{lemma:cont}). Here, $\cB_\delta(\vx)$ denotes the Euclidean ball centered at $\vx$ with radius $\delta = \delta(\epsilon)$. 
% It then suffices to invoke Brouwer's fixed-point theorem for the gradient mapping $\vx \mapsto \proj_\cX ( \vx - \hatF(\vx) )$, where $\proj_\cX$ is the Euclidean projection with respect to $\cX$.
%Then, a uniform distribution around a VI solution $\vx^*$ to $\hatF$ makes makes an $\epsilon$-approximate solution to the $\Phi$-EVI problem for $F$.

\Cref{theorem:existence} implies that a $\Phi$-EVI can have approximate solutions even when the associated VI problem does not.\footnote{Noncontinuity of $F$ manifests itself prominently in \emph{nonsmooth} optimization (\emph{e.g.}, \citealp{Zhang20:Complexity,Davis22:Gradient,Tian22:Finite,Jordan23:Deterministic}); recent research there focuses on \emph{Goldstein} stationary points~\citep{Goldstein77:Optimization}, which are conceptually related to EVIs.}
%This is a very nice moytivation footnote. However, it is not clear why nonsmoothness relates to noncontinuity of F. Explain.???
%Also, if space allows, move the footnote into the body???

\begin{restatable}{corollary}{VIsvsEVIs}
    \label{prop:VI-vs-EVIs}
    There exists a VI problem that does not admit approximate solutions when $\epsilon = \Theta(1)$, but the corresponding $\epsilon$-approximate $\Phi$-EVI is total for any $\epsilon > 0$.
\end{restatable}

In the proof, we set $F$ to be the \emph{sign function} (\Cref{ex:sign-evi}). By contrast, if one insists on exact solutions, EVIs do not necessarily admit solutions.

\begin{restatable}{proposition}{notexact}
    \label{prop:notexact}
    When $F$ is not continuous, there exists an EVI problem with no solutions.
\end{restatable}

Furthermore, \Cref{theorem:existence} raises the question of whether it is enough to instead assume that every $\phi \in \Phi$ is continuous. Our next result dispels any such hopes.

\begin{restatable}{theorem}{countercont}
    \label{theorem:continuous-counterexample}
    There are $\Phi$-EVI instances that do not admit $\eps$-approximate solutions even when $\eps = \Theta(1)$, $F$ is piecewise constant, and $\Phi$ contains only continuous functions.
\end{restatable}

Our final result on existence complements~\Cref{theorem:existence,theorem:continuous-counterexample} by showing that, when $\Phi$ is finite-dimensional, it is enough if every $\phi \in \Phi$ admits a fixed point (this holds, for example, when $\phi$ is continuous---by Brouwer's theorem).

\begin{restatable}{theorem}{finitedim}
    \label{theorem:finitedim}
    Suppose that 
    \begin{enumerate}%[noitemsep,topsep=0pt]
        \item $\Phi$ is finite-dimensional, that is, there exists $k \in \N$ and a kernel map $m : \cX \to \R^k$ such that every $\phi \in \Phi$ can be expressed as $\mK m(\vx)$ for some $\mK \in \R^{d \times k}$; and
        \item every $\phi \in \Phi$ admits a fixed point, that is, a point $\cX \ni \vx = \fix(\phi)$ such that $\phi(\vx) = \vx$.
    \end{enumerate} Then, the $\Phi$-EVI problem admits an $\eps$-approximate solution with support size at most $1+dk$ for every $\eps > 0$.
\end{restatable}

Notably, this theorem guarantees the existence of solutions with finite support; the proof makes use of the minimax theorem~(\eg,~\citealp{Sion58:On}) in conjunction with Carath\'eodory's theorem on convex hulls~\cite{Caratheodory11:Uber}.

\paragraph{Complexity} Having established some basic existence properties, we now turn to the complexity of $\Phi$-EVIs. Let us define the VI gap function $\VIgap(\vx) \defeq - \min_{\vx' \in \cX} \langle F(\vx), \vx' - \vx \rangle$, which is nonnegative. If we place no restrictions on $\Phi$, it turns out that $\Phi$-EVIs are tantamount to regular VIs:


\begin{restatable}{proposition}{EVIequiv}
    \label{prop:EVI-VI}
    If $\Phi$ contains all measurable functions from $\cX$ to $\cX$, then any solution $\mu \in \Delta(\cX)$ to the $\epsilon$-approximate $\Phi$-EVI problem satisfies
    \begin{equation}
        \label{eq:EVI-VI}
        \E_{\vx \sim \mu} \VIgap(\vx) \leq \epsilon. 
    \end{equation}
\end{restatable}

In proof, it suffices to consider a $\phi$ that maps $\vx \in \cX$ to an appropriate point in $\argmin_{\vx' \in \cX} \langle F(\vx), \vx' - \vx \rangle$. When $\mu$ must be given explicitly, \Cref{prop:EVI-VI} immediately implies that $\Phi$-EVIs are computationally hard, because \eqref{eq:EVI-VI} implies that $\VIgap(\vx) \leq \epsilon$ for some $\vx$ in the support of $\mu$, and such a point can be identified in polynomial time.\footnote{This argument carries over without restricting the support of $\mu$, by assuming instead access to a sampling oracle from $\mu$: a standard Chernoff bound implies that the empirical distribution (w.r.t. a large enough sample size) approximately satisfies~\eqref{eq:EVI-VI}.}

\begin{corollary}
    \label{cor:hard-EVI}
    The $\epsilon$-approximate $\Phi$-EVI problem is \PPAD-hard even when $\epsilon$ is an absolute constant and $F$ is linear.
\end{corollary}

Coupled with~\Cref{prop:EVI-VI}, this follows from the hardness result of~\citet{Rubinstein15:Inapproximability} concerning Nash equilibria in (multi-player) polymatrix games (for binary-action, graphical games, \citealp{Deligkas23:Tight} recently showed that \PPAD-hardness persists up to $\epsilon < \nicefrac{1}{2}$). \Cref{cor:hard-EVI} notwithstanding, it is easy to see that the set of solutions to $\Phi$-EVIs is convex for any $\Phi \subseteq \cX^\cX$.

\begin{remark}
    Let $\cX = \cX_1 \times \dots \times \cX_n$, as in an $n$-player game. Whether \Cref{cor:hard-EVI} applies under deviations that can be decomposed as $\phi : \vx \mapsto \phi(\vx) = (\phi_1(\vx_1), \dots, \phi_n(\vx_n))$ is a major open question in the regime where $\epsilon \ll 1$ (\emph{cf.}~\citealp{Dagan24:From,Peng24:Fast}).
\end{remark}

Viewed differently, a special case of the $\Phi$-EVI problem arises when $\Phi = \{ \phi \}$ and $F(\vx) = \vx - \phi(\vx)$, for some fixed map $\phi : \cX \to \cX$. In this case, the $\Phi$-EVI problem reduces to finding a $\mu \in \Delta(\cX)$ such that
\begin{equation}
    \label{eq:fp}
    \E_{\vx\sim\mu} \ip{F(\vx), \phi(\vx) - \vx} = - \E_{\vx\sim\mu} \norm{\phi(\vx) - \vx}^2 \geq -\epsilon.
\end{equation}
As a result, $\mu$ must contain in its support an $\epsilon$-approximate fixed point of $\phi$, a problem which is \PPAD-hard already for quadratic functions~\citep{Zhang24:Efficient}.

\begin{corollary}
    \label{cor:hard-EVI-quad}
    The $\epsilon$-approximate $\Phi$-EVI problem is \PPAD-hard even when $\epsilon$ is an absolute constant, $F$ is quadratic, and $\Phi = \{ \phi \}$ for a quadratic map $\phi : \cX \to \cX$.
\end{corollary}

It is also worth noting that, unlike~\Cref{cor:hard-EVI}, $\Phi$ in the corollary above contains only continuous functions.

It also follows from~\eqref{eq:fp} that, for $\epsilon = 0$, $\Phi$-EVIs capture exact fixed points. The complexity class \FIXP~characterizes such problems~\citep{Etessami07:Complexity}.

\begin{corollary}
    The $\Phi$-EVI problem is \FIXP-hard, assuming that $\supp(\mu) \leq \poly(d)$.
\end{corollary}

Exponential lower bounds in terms of the number of function evaluations of $F$ also follow from~\citet{Hirsch89:Exponential}.

On a positive note, the next section establishes polynomial-time algorithms when $\Phi$ contains only \emph{linear} endomorphisms.\footnote{We do not distinguish between affine and linear maps because we can always set $\cX \gets \cX \times \{1\}$, in which case affine and linear maps coincide.}
\section{Efficient Computation with Linear Endomorphisms}
\label{sec:main}

The hardness results of the previous section highlight the need to restrict the set $\Phi$ in order to make meaningful progress. Our main result here establishes a polynomial-time algorithm when $\Phi$ contains only linear endomorphisms.

\begin{theorem}
    \label{th:elvi}
    If $\Phi$ contains only linear endomorphisms, the $\eps$-approximate $\Phi$-EVI problem can be solved in time $\poly(d, \log(B/\epsilon))$ given a membership oracle for $\cX$.
\end{theorem}

The proof relies on the ellipsoid against hope ($\eah$), and in particular, a recent generalization by~\citet{Daskalakis24:Efficient}. In a nutshell, the main deficiency in the framework covered earlier in~\Cref{sec:eah} is that one needs a separation oracle for $\cY$ (\Cref{theorem:eah}), where $\cY$ for us is the set of deviations $\Phi$. Unlike some applications, in which $\cY$ has an explicit, polynomial representation~\citep{Papadimitriou08:Computing}, that assumption needs to be relaxed to account for $\Philin$~\citep[Theorem 3.4]{Daskalakis24:Efficient}.

\citet{Daskalakis24:Efficient} address this by considering instead the $\either$ oracle. As the name suggests, for any $\vy \in \R^m$, it \emph{either} returns a hyperplane separating $\vy$ from $\cY$, or a good-enough-response $\vx \in \cX$. They showed that~\Cref{theorem:eah} can be extended under this weaker oracle (in place of $\ger$ and $\sep$); the formal version is given in~\Cref{th:eahrelax}.

In our setting, we consider the feasibility problem
\begin{align}\label{eq:evi-ellipsoid}
        \text{find}\quad \phi \in \Philin \qq{s.t.} \ip{F(\vx), \phi(\vx) - \vx} \le -\eps \quad \forall \vx \in \cX.
\end{align}
Equivalently,
\begin{align*}
        \text{find}\quad \mK \in \R^{d \times d} \qq{s.t.} \\\ip{F(\vx), \mK\vx - \vx} \le -\eps \quad &\forall \vx \in \cX, \\\mK \vx \in \cX \quad &\forall \vx \in \cX.
\end{align*}
This program is infeasible since, for any $\phi \in \Philin$, the fixed point $\vx$ of $\phi$ makes the left-hand side of the constraint $0$. And a certificate of infeasibility is an $\eps$-approximate $\Philin$-EVI solution. Thus, it suffices to show how to run the ellipsoid algorithm on \eqref{eq:evi-ellipsoid}. By~\Cref{th:eahrelax}, it suffices if for any $\mK \in \R^{d \times d}$, we can compute efficiently \emph{either}
\begin{itemize}%[noitemsep,topsep=0pt]
        \item some $\vx \in \cX$ such that $\mK\vx = \vx$ ($\ger$), {\em or}
        \item some hyperplane separating $\mK$ from $\Philin$ ($\sep$).
\end{itemize}
This is precisely the \emph{semi-separation oracle} solved by~\citet[Lemma~4.1]{Daskalakis24:Efficient}, stated below.

\begin{lemma}[\citealp{Daskalakis24:Efficient}]
    \label{lemma:semiseparation}
    There is an algorithm that takes as input $\mat{K} \in \R^{d \times d}$, runs in $\poly(d)$ time, makes $\poly(d)$ oracle queries to $\cX$, and either returns a fixed point $\cX \ni \vx = \mat{K} \vx$, or a hyperplane separating $\mat{K}$ from $\Philin$.
\end{lemma}

% An immediate consequence is that \emph{linear} correlated equilibria %undefined; explain what that term means??
% can be computed %found would be a better word than computed ??
% in polynomial time in convex games~\citep{Daskalakis24:Efficient}.

On a separate note, \Cref{th:elvi} only accounts for approximate solutions. We cannot hope to improve that in the sense that exact solutions might be supported only on irrational points even in concave maximization (\emph{cf.}~\Cref{prop:convex-equiv}).

\subsection{Regret minimization for EVIs on polytopes}

One caveat of~\Cref{th:elvi} is that it relies on the impractical $\eah$ algorithm. To address this limitation, we will show that $\Phi$-EVIs are also amenable to the more scalable approach of \emph{regret minimization}---albeit with an inferior complexity growing as $\poly(1/\epsilon)$.

Specifically, in our context, the regret minimization framework can be applied as follows. At any time $t \in \N$, we think of a ``learner'' selecting a point $\vx^{(t)} \in \cX$, whereupon $F(\vx^{(t)})$ is given as feedback from the ``environment,'' so that the utility at time $t$ reads $- \langle \vx^{(t)}, F(\vx^{(t)}) \rangle$. \emph{$\Phi$-regret} is a measure of performance in online learning, defined as
%
\begin{equation*}
    \phireg^{(T)} \defeq \max_{\phi \in \Phi} \sum_{t=1}^T \langle F(\vx^{(t)}), \phi(\vx^{(t)}) - \vx^{(t)} \rangle.
\end{equation*}
The uniform distribution $\mu$ on $\{ \vx^{(1)}, \dots, \vx^{(T)} \}$ is clearly a $\phireg^T/T$-approximate $\Phi$-EVI solution. 

In what follows, we will assume that $\cX$ is a polytope given explicitly by linear constraints, \ie, \begin{align*}
    \cX = \{ \vx \in \R^d : \mA \vx \le \vb \},
\end{align*}
where $\mA \in \Q^{m \times d}$ and $\vb \in \Q^m$ are given as input.

To minimize $\Phi$-regret, we will make use of the template by~\citet{Gordon08:No}, which comprises two components. The first is a fixed-point oracle, which takes as input a function $\phi \in \Philin$ and returns a point 
$\vx \in \cX$ with
$\vx = \phi(\vx)$; given that $\phi$ is linear, it can be implemented efficiently via linear programming. The second component is an algorithm for minimizing (external) regret over the set $\Philin$. In~\Cref{theorem:explicit-repres}, we devise a polynomial representation for $\Philin$:

\begin{theorem}
    \label{theorem:regret}
   For an arbitrary polytope $\cX$ given by explicit linear constraints, there is an explicit representation of $\Philin$ as a polytope with $O(d^2 + m^2)$  variables and constraints.
\end{theorem}
As a consequence, we can instantiate the regret minimizer operating over $\Philin$ with projected gradient descent.
\begin{corollary}
    \label{cor:regret}
    There is a deterministic algorithm that guarantees $\philinreg^{(T)} \leq \epsilon$ after $\poly(d, m)/\eps^2$ rounds, and requires solving a convex quadratic program with $O(d^2+m^2)$ variables and constraints in each iteration.
\end{corollary}
% \begin{corollary}
%     There is a randomized algorithm that guarantees $\philinreg^{(T)} \leq \epsilon$ after $\qty(\poly(d, m) + O(\log(1/\delta)))/\eps^2$ with probability at least $1/\delta$, and requires solving a linear program with $O(d^2+m^2)$ variables and constraints on each iteration\todo{}
% \end{corollary}

An additional benefit of~\Cref{cor:regret} compared to using $\eah$ is that the former is more suitable in a decentralized environment---for example, in multi-player games (\emph{cf.}~\Cref{example:CCE}). There, \Cref{cor:regret} corresponds to each player running their own independent no-regret learning algorithm. Even in this setting, our algorithms actually yield an improvement over the best-known algorithms for minimizing $\Philin$-regret over explicitly-represented polytopes: the previous state of the art, due to \citet{Daskalakis24:Efficient}, requires running the ellipsoid algorithm on each iteration, which is slower than quadratic programming (\Cref{sec:repre}).
\section{Cooperative Games Details} \label{app:cooperative_games}

The cooperative games used within our conducted experiments are based on explanation examples for real world data.
This section complete their brief description given in \cref{sec:exper}.
Across all cooperative games the players represent a fixed set of features given by a particular dataset.

\subsection{Global feature importance}

Seeking to quantify each feature's individual importance to a model's predictive performance, the value function is based on the model's performance of a hold out test set.
This necessitates to split the dataset at hand into training and test set.
Features outside of an inspected coalition $S$ are removed by retraining the model on the training set and measuring its performance on the test set.
For all games we a applied train-test split of 70\% to 30\% and a random forest consisting of 20 trees.
For classification the value function maps each coalition to the model's resulting accuracy on the test set minus the accuracy of the mode within the data such that the empty coalition has a value of zero.
For regression tasks the worth of a coalition is the reduction of the model's mean squared error compared to the empty set which is given by the mean prediction.
Again, the empty coalition has a value of zero.

\subsection{Local feature attribution}

Instead of assessing each feature's contribution to the predictive performance, its influence on a model's prediction for a fixed datapoint can also be investigated.
Hence, the value function is based on the model's predicted value.

\subsubsection{Adult classification}
A sklearn gradient-boosted tree classifies whether a person's annual salary exceeds 50,000 in the \emph{Adult} tabular dataset containing 14 features.
The predicted class probability of the true class is taken as the worth of a coalition $S$.
In order to render features outside of $S$ absent, these are imputed by their mean value such that the datapoint is compatible to the model's expected feature number.

\subsubsection{Image classification}

A \emph{ResNet18} model is used to classify images from \emph{ImageNet}.
Since the for error tracking necessary exact computation of Shapley values is infeasible for the given number of pixels, 14 semantic segments are formed after applying \emph{SLIC}.
These super-pixels form the player set.
Given that the model predicts class $c$ using the full image, the value function assigns to each coalition $S$ the predicted class probability of $c$ resulting from only including those super-pixels in $S$.
The other super-pixels are removed by mean imputation, setting them grey.

\subsubsection{IMDB sentiment analysis}

A \emph{DistilBERT} transformer  fine-tuned on the \emph{IMDB} dataset predicts the sentiment of a natural language sentence between -1 and 1.
The sentence is transformed into a sequence of tokens.
The input sentences are restricted to sentences that result in 14 tokens being represented by players of the cooperative game.
This allows to remove players in the tokenized representation of the transformer.
The predicted sentiment is taken as the worth of a coalition.


\subsection{Unsupervised feature importance}

In contrast to the previous settings, there is no available predictive model to investigate unlabeled data.
Still, each feature's contribution to the shared information within the data can be quantified and assigned as a score.
\citep{Balestra.2022} proposed to view the features $1,\ldots,n$ as random variables $X_1,\ldots,X_n$ such that the datapoints are realizations of their joint distribution.
Next, the worth of a coalition $S$ is given by their total correlation
\begin{equation*}
    \nu(S) = \sum\limits_{i \in S} H(X_i) - H(S)
\end{equation*}
where $H(X_i)$ denotes the Shannon entropy of $X_i$ and $H(S)$ the contained random variables joint Shannon entropy.
The utilized datasets are reduced in the number of features and datapoints to ease computation.
The \emph{Breast cancer} dataset contains 9 features and 286 datapoints.
The class label indicating the diagnosis is removed.
From the \emph{Big five} and \emph{FIFA 21} dataset 12 random features are selected out of the first 50 and the datapoints are reduced to the first 10,000.

\section{Problems where EVIs coincide with VIs}

We saw earlier, in~\Cref{prop:EVI-VI}, that when $\Phi$ comprises all functions from $\cX$ to $\cX$, the $\Phi$-EVI problem is tantamount to the associated VI problem. However, if one restricts the functions contained in $\Phi$, are there still structured VIs where we retain this equivalence? 
In this section, we consider certain structured VIs, and show their equivalence to the corresponding EVIs (that is, $\Phiconst$-EVIs). Unlike general VIs, the ones we examine below are tractable.%---because of that equivalence.


\subsection{Polymatrix zero-sum games and beyond}

The first important class of VIs we consider is described by a condition given below.

\begin{proposition}
    \label{prop:collapse}
    Suppose that for any $\vx' \in \cX$, the function $g : \vx \mapsto \langle F(\vx), \vx' - \vx \rangle$ is concave. Then, if $\mu \in \Delta(\cX)$ is an $\epsilon$-approximate solution to the EVI, $\E_{\vx \sim \mu} \vx$ is an $\epsilon$-approximate solution to the VI.
\end{proposition}

The proof follows directly from Jensen's inequality.

The precondition of~\Cref{prop:collapse} is satisfied, \emph{e.g.}, when: (i) $\langle F(\vx), \vx \rangle = 0$ for all $\vx \in \cX$, and (ii) $F$ is a linear map. In the context of $n$-player games, the first condition amounts to the zero-sum property: $\sum_{i=1}^n u_i(\vx) = 0$ for all $\vx$. Of course, this property is not enough to enable efficient computation of Nash equilibria, for every two-player (general-sum) game can be converted into a $3$-player zero-sum game. This is where the second condition comes into play: $F$ is a linear map---that is, each player's gradient must be linear in the joint strategy. Those two conditions are satisfied in \emph{polymatrix zero-sum} games~\citep{Cai16:Zero}; in such games, the conclusion of~\Cref{prop:collapse} is a well-known fact. 

\subsection{Quasar-concave functions}

We next consider the problem of maximizing a (single) function that satisfies \emph{quasar-concavity}---a natural generalization of concavity that has received significant  interest~\citep{Hardt18:Gradient,Fu23:Accelerated,Hinder20:Near,Gower21:SGD,Guminov23:Accelerated,Caramanis24:Optimizing}.\footnote{Prior literature mostly uses the term \emph{quasar-convexity}, which is equivalent to quasar-concavity for the opposite function $-u$.}

\begin{definition}[Quasar-concavity]
    \label{def:quasar}
    Let $\gamma \in (0, 1]$ and $\vxstar \in \cX$ be a maximizer of a differentiable function $u : \cX \to \R$. We say that $u$ is \emph{$\gamma$-quasar-concave} with respect to $\vxstar$ if
    \begin{equation}
        \label{eq:quasar}
        u(\vxstar) \leq u(\vx) + \frac{1}{\gamma} \langle \nabla u(\vx), \vxstar - \vx \rangle \quad \forall \vx \in \cX.
    \end{equation}
\end{definition}

In particular, in the special case where $\gamma = 1$, \eqref{eq:quasar} is equivalent to \emph{star-concavity}~\citep{Nesterov06:Cubic}. If in addition \eqref{eq:quasar} holds for all $\vxstar \in \cX$ (not merely w.r.t. a global maximizer), it captures the usual notion of concavity.

Any reasonable solution concept for such problems should place all mass on global maxima; EVIs pass this litmus test:

\begin{proposition}
    \label{prop:convex-equiv}
    Let $F = - \grad u$ for a $\gamma$-quasar-concave and differentiable function $u : \cX \to \R$. Then, for any solution $\mu \in \Delta(\cX)$ to the EVI problem,
    \begin{equation*}
        \E_{\vx \sim \mu} u(\vx) \geq \max_{\vx \in \cX} u(\vx).
    \end{equation*}
    Thus, $\mathbb{P}_{\vx\sim\mu}[u(\vxstar) = u(\vx)] = 1$, for $\vxstar \in \argmax_{\vx} u(\vx)$.
\end{proposition}

Indeed, by~\Cref{def:quasar}, $0 \leq \E_{\vx\sim\mu} \ip{\grad u(\vx), \vx - \vxstar } \le \gamma \E_{\vx\sim\mu} [u(\vx) - u(\vxstar) ]$ for any EVI solution $\mu \in \Delta(\cX)$. Thus, under quasar-concavity, VIs basically reduce to EVIs.


%It would be hard to justify a notion that, in a (two-player) zero-sum game, places mass on strategies that do not constitute minimax strategies.

%More broadly, and in particular for multi-player games, there is no definite justification for EVIs \emph{vis-\`a-vis} other notions---this ties directly to the \emph{equilibrium selection} problem, which has generated much debate~\citep{Harsanyi88:General}.
\section{Performance Guarantees for EVIs}
\label{sec:smoothness}

In many settings, a VI solution is used as a proxy to approximately maximize some underlying objective function; machine learning offers many such applications. The question is whether performance guarantees pertaining to VIs can be extended---potentially with some small degradation---to EVIs as well. The purpose of this section is to provide a framework for achieving that based on the following notion.

\begin{definition}
    \label{def:smoothness}
    An EVI problem is \emph{$(\lambda, \nu)$-smooth}, for $\lambda > 0, \nu > -1$, w.r.t. $W : \cX \to \R$ and $\vxstar \in \argmax_{\vx} W(\vx)$ if
    \begin{equation*}
        \langle F(\vx), \vxstar - \vx \rangle \leq -  \lambda W(\vxstar) + (\nu+1) W(\vx) \quad \forall \vx \in \cX.
    \end{equation*}
\end{definition}

\begin{example}
    When the underlying problem corresponds to a multi-player game and $W$ is the (utilitarian) social welfare, \Cref{def:smoothness} coincides with the celebrated notion of smoothness \emph{\`a la}~\citet{Roughgarden15:Intrinsic}; this is a consequence of multilinearity, which implies that $W(\vx) = -\langle \vx, F(\vx) \rangle$ and $ \langle \vxstar, F(\vx) \rangle = - \sum_{i=1}^n u_i(\vxstar_i, \vx_{-i})$ for all $\vx \in \cX$. We also refer to the recent treatment of smoothness by~\citet{Ahunbay25:First} in the context of nonconcave games, which builds on the primal-dual framework of~\citet{Nadav10:Limits}.
\end{example}

\Cref{def:smoothness} is an extension of the more general notion of ``local smoothness,'' introduced by~\citet{Roughgarden15:Local} in the context of splittable congestion games. However, it goes beyond games. Indeed, the following definition we introduce generalizes \Cref{def:quasar}, making a new connection between smoothness and quasar-concavity.

\begin{definition}[Extension of quasar-concavity]
    \label{def:smooth-fun}
    Let $\vxstar \in \cX$ be a maximizer of a differentiable function $u : \cX \to \R$. We say that $u$ is \emph{$(\lambda, \nu)$-smooth} with respect to $\vxstar$ if
    \begin{equation*}
        \langle \nabla u(\vx), \vxstar - \vx \rangle \geq \lambda u(\vxstar) - (\nu + 1) u(\vx) \quad \forall \vx \in \cX.
    \end{equation*}
\end{definition}

In particular, when $\lambda \defeq \gamma$ and $\nu \defeq \gamma - 1$, the above definition captures $\gamma$-quasar-concavity. In~\Cref{sec:appendix-smooth}, we provide an example of a polynomial that satisfies~\Cref{def:smooth-fun} without being quasar-concave. Now, the key property of~\Cref{def:smoothness} is that any EVI solution approximates the underlying objective---by a factor of $\rho \defeq \nicefrac{\lambda}{1 + \nu}$.

\begin{theorem}
    \label{theorem:smoothness}
    Let $\mu \in \Delta(\cX)$ be an $\epsilon$-approximate solution to a $(\lambda, \nu)$-smooth EVI problem w.r.t. $W : \cX \to \R$. Then,
    \begin{equation*}
        \E_{\vx \sim \mu} W(\vx)  \geq \frac{\lambda}{1 + \nu} \max_{\vx \in \cX} W(\vx) - \frac{\epsilon}{1 + \nu}.
    \end{equation*}
\end{theorem}

The proof follows directly from~\Cref{def:smoothness}, using that $\E_{\vx \sim \mu} \langle F(\vx), \vxstar - \vx \rangle \geq - \epsilon$ and linearity of expectation.
\vspace{-0.2cm}
\section{Impact: Why Free Scientific Knowledge?}
\vspace{-0.1cm}

Historically, making knowledge widely available has driven transformative progress. Gutenberg’s printing press broke medieval monopolies on information, increasing literacy and contributing to the Renaissance and Scientific Revolution. In today's world, open source projects such as GNU/Linux and Wikipedia show that freely accessible and modifiable knowledge fosters innovation while ensuring creators are credited through copyleft licenses. These examples highlight a key idea: \textit{access to essential knowledge supports overall advancement.} 

This aligns with the arguments made by Prabhakaran et al. \cite{humanrightsbasedapproachresponsible}, who specifically highlight the \textbf{ human right to participate in scientific advancement} as enshrined in the Universal Declaration of Human Rights. They emphasize that this right underscores the importance of \textit{ equal access to the benefits of scientific progress for all}, a principle directly supported by our proposal for Knowledge Units. The UN Special Rapporteur on Cultural Rights further reinforces this, advocating for the expansion of copyright exceptions to broaden access to scientific knowledge as a crucial component of the right to science and culture \cite{scienceright}. 

However, current intellectual property regimes often create ``patently unfair" barriers to this knowledge, preventing innovation and access, especially in areas critical to human rights, as Hale compellingly argues \cite{patentlyunfair}. Finding a solution requires carefully balancing the imperative of open access with the legitimate rights of authors. As Austin and Ginsburg remind us, authors' rights are also human rights, necessitating robust protection \cite{authorhumanrights}. Shareable knowledge entities like Knowledge Units offer a potential mechanism to achieve this delicate balance in the scientific domain, enabling wider dissemination of research findings while respecting authors' fundamental rights.

\vspace{-0.2cm}
\subsection{Impact Across Sectors}

\textbf{Researchers:} Collaboration across different fields becomes easier when knowledge is shared openly. For instance, combining machine learning with biology or applying quantum principles to cryptography can lead to important breakthroughs. Removing copyright restrictions allows researchers to freely use data and methods, speeding up discoveries while respecting original contributions.

\textbf{Practitioners:} Professionals, especially in healthcare, benefit from immediate access to the latest research. Quick access to newer insights on the effectiveness of drugs, and alternative treatments speeds up adoption and awareness, potentially saving lives. Additionally, open knowledge helps developing countries gain access to health innovations.

\textbf{Education:} Education becomes more accessible when teachers use the latest research to create up-to-date curricula without prohibitive costs. Students can access high-quality research materials and use LM assistance to better understand complex topics, enhancing their learning experience and making high-quality education more accessible.

\textbf{Public Trust:} When information is transparent and accessible, the public can better understand and trust decision-making processes. Open access to government policies and industry practices allows people to review and verify information, helping to reduce misinformation. This transparency encourages critical thinking and builds trust in scientific and governmental institutions.

Overall, making scientific knowledge accessible supports global fairness. By viewing knowledge as a common resource rather than a product to be sold, we can speed up innovation, encourage critical thinking, and empower communities to address important challenges.

\vspace{-0.2cm}
\section{Open Problems}
\vspace{-0.1cm}

Moving forward, we identify key research directions to further exploit the potential of converting original texts into shareable knowledge entities such as demonstrated by the conversion into Knowledge Units in this work:


\textbf{1. Enhancing Factual Accuracy and Reliability:}  Refining KUs through cross-referencing with source texts and incorporating community-driven correction mechanisms, similar to Wikipedia, can minimize hallucinations and ensure the long-term accuracy of knowledge-based datasets at scale.

\textbf{2. Developing Applications for Education and Research:}  Using KU-based conversion for datasets to be employed in practical tools, such as search interfaces and learning platforms, can ensure rapid dissemination of any new knowledge into shareable downstream resources, significantly improving the accessibility, spread, and impact of KUs.

\textbf{3. Establishing Standards for Knowledge Interoperability and Reuse:}  Future research should focus on defining standardized formats for entities like KU and knowledge graph layouts \citep{lenat1990cyc}. These standards are essential to unlock seamless interoperability, facilitate reuse across diverse platforms, and foster a vibrant ecosystem of open scientific knowledge. 

\textbf{4. Interconnecting Shareable Knowledge for Scientific Workflow Assistance and Automation:} There might be further potential in constructing a semantic web that interconnects publicly shared knowledge, together with mechanisms that continually update and validate all shareable knowledge units. This can be starting point for a platform that uses all collected knowledge to assist scientific workflows, for instance by feeding such a semantic web into recently developed reasoning models equipped with retrieval augmented generation. Such assistance could assemble knowledge across multiple scientific papers, guiding scientists more efficiently through vast research landscapes. Given further progress in model capabilities, validation, self-repair and evolving new knowledge from already existing vast collection in the semantic web can lead to automation of scientific discovery, assuming that knowledge data in the semantic web can be freely shared.

We open-source our code and encourage collaboration to improve extraction pipelines, enhance Knowledge Unit capabilities, and expand coverage to additional fields.

\vspace{-0.2cm}
\section{Conclusion}
\vspace{-0.1cm}

In this paper, we highlight the potential of systematically separating factual scientific knowledge from protected artistic or stylistic expression. By representing scientific insights as structured facts and relationships, prototypes like Knowledge Units (KUs) offer a pathway to broaden access to scientific knowledge without infringing copyright, aligning with legal principles like German \S 24(1) UrhG and U.S. fair use standards. Extensive testing across a range of domains and models shows evidence that Knowledge Units (KUs) can feasibly retain core information. These findings offer a promising way forward for openly disseminating scientific information while respecting copyright constraints.

\section*{Author Contributions}

Christoph conceived the project and led organization. Christoph and Gollam led all the experiments. Nick and Huu led the legal aspects. Tawsif led the data collection. Ameya and Andreas led the manuscript writing. Ludwig, Sören, Robert, Jenia and Matthias provided feedback. advice and scientific supervision throughout the project. 

\section*{Acknowledgements}

The authors would like to thank (in alphabetical order): Sebastian Dziadzio, Kristof Meding, Tea Mustać, Shantanu Prabhat for insightful feedback and suggestions. Special thanks to Andrej Radonjic for help in scaling up data collection. GR and SA acknowledge financial support by the German Research Foundation (DFG) for the NFDI4DataScience Initiative (project number 460234259). AP and MB acknowledge financial support by the Federal Ministry of Education and Research (BMBF), FKZ: 011524085B and Open Philanthropy Foundation funded by the Good Ventures Foundation. AH acknowledges financial support by the Federal Ministry of Education and Research (BMBF), FKZ: 01IS24079A and the Carl Zeiss Foundation through the project "Certification and Foundations of Safe ML Systems" as well as the support from the International Max Planck Research School for Intelligent Systems (IMPRS-IS). JJ acknowledges funding by the Federal Ministry of Education and Research of Germany (BMBF) under grant no. 01IS22094B (WestAI - AI Service Center West), under grant no. 01IS24085C (OPENHAFM) and under the grant DE002571 (MINERVA), as well as co-funding by EU from EuroHPC Joint Undertaking programm under grant no. 101182737 (MINERVA) and from Digital Europe Programme under grant no. 101195233 (openEuroLLM) 

\section*{Acknowledgments}
T.S. is supported by the Vannevar Bush Faculty Fellowship ONR N00014-23-1-2876, National Science Foundation grants RI-2312342 and RI-1901403, ARO award W911NF2210266, and NIH award A240108S001. B.H.Z. is supported 
by the CMU Computer Science Department Hans Berliner
PhD Student Fellowship. E.T, R.E.B., and V.C. thank the Cooperative AI Foundation, Polaris Ventures (formerly the Center for
Emerging Risk Research) and Jaan Tallinn’s donor-advised fund at Founders Pledge for financial
support. E.T. and R.E.B. are also supported in part by the Cooperative AI PhD Fellowship. G.F is supported by the National Science Foundation grant CCF-2443068. We are grateful to Mete \c Seref Ahunbay for his helpful feedback. We also thank Andrea Celli and Martino Bernasconi for discussions regarding $\Phi$-equilibria in games with coupled constraints.

\bibliography{dairefs}

%%%%%%%%%%%%%%%%%%%%%%%%%%%%%%%%%%%%%%%%%%%%%%%%%%%%%%%%%%%%%%%%%%%%%%%%%%%%%%%
%%%%%%%%%%%%%%%%%%%%%%%%%%%%%%%%%%%%%%%%%%%%%%%%%%%%%%%%%%%%%%%%%%%%%%%%%%%%%%%
% APPENDIX
%%%%%%%%%%%%%%%%%%%%%%%%%%%%%%%%%%%%%%%%%%%%%%%%%%%%%%%%%%%%%%%%%%%%%%%%%%%%%%%
%%%%%%%%%%%%%%%%%%%%%%%%%%%%%%%%%%%%%%%%%%%%%%%%%%%%%%%%%%%%%%%%%%%%%%%%%%%%%%%
\clearpage
\appendix

\section{Additional Preliminaries}
\label{sec:add-prels}

\paragraph{Revisiting \Cref{def:evi}} 

In order to define the distributions $\Delta(\cX)$ over $\cX$ precisely, we recall here some basic concepts from probability theory. We refer to \citet[Chapter 1 and 2]{Billingsley99:Convergence} and \citet[Chapter 15]{AliprantisB06:Infinite} for detailed treatments. We assume throughout the paper that the set $\cX \subseteq \R^d$ is Borel measurable. Let $\Delta(\cX)$ be the set of Borel probability measures $\mu$ on $\cX$, that is, measures $\mu: \qty(\cX, B(\cX)) \to \qty(\R, B(\R))$ with $\mu(\cX) = 1$, where $B(\cX)$ and $B(\R)$ denote the respective $\sigma$-algebra of Borel sets. We simply call $\mu$ a distribution. For any Borel measurable function $f : \cX \to \R$---henceforth just \emph{measurable}--- we can then take the integral $\E_{\vx \sim \mu}[f(\vx)] := \int_\cX f(\vx) d\mu(\vx)$. In particular, for $\E_{\vx\sim\mu} \ip{F(\vx), \phi(\vx) - \vx}$ in \Cref{def:evi} to be well-defined, we assume throughout this paper that $F$ and each $\phi \in \Phi$ are measurable functions.

For our computational results (\Cref{sec:main}), we are making a standard assumption regarding the geometry of $\cX$ (\Cref{sec:prel}); this can be met by bringing $\cX$ into isotropic position. In particular, there is a polynomial-time algorithm that computes an affine transformation to accomplish that~\citep{Lovasz06:Simulated}, and minimizing linear-swap regret reduces to minimizing linear-swap regret to the transformed instance~\citep[Lemma A.1]{Daskalakis24:Efficient}.

% \paragraph{Further comments on $\Delta(\cX)$} We make $\Delta(\cX)$ a topological space by equipping it with the weak-$*$ topology. This is exactly the coarsest topology that makes the map $\mu \mapsto \int_\cX f(x) d\mu(x)$ continuous for all continuous and bounded maps $f : \cX \to \R$.\footnote{Convergence in this topology coincides with the notion in probability theory that the probability measures $(\mu_n)_{n \in \N} \subset \Delta(\cX)$ \emph{weakly converge} to the probability measure $\mu \in \Delta(\cX)$.} Any continuous $f : \cX \to \R$ is guaranteed to be bounded since are working with compact spaces $\cX$ in this paper. Moreover, $\cX \subset \R^n$ compact implies that $\Delta(\cX)$ is compact and Hausdorff as well \cite{AliprantisB06:Infinite}[Thm.~15.11].

\iffalse

\begin{definition}[Weak separation oracle;~\citealp{Grotschel81:Ellipsoid}]
    Let $\cX$ be a convex and compact set in $\R^d$ and $\epsilon > 0$ a rational number. A \emph{(weak) separation oracle} for $\cX$ is a function that, given $\vx \in \R^d$,
    \begin{itemize}[topsep=0pt,noitemsep]
        \item asserts whether $\vx$ is $\epsilon$-close to $\cX$, in that its Euclidean distance from $\cX$ is at most $\epsilon$; or, otherwise,
        \item  finds a vector $\vec{w} \in \R^d$, with $\|\vec{w}\| = 1$, such that $\langle \vec{w}, \vx \rangle \geq \langle \vec{w}, \vx' \rangle - \epsilon$ for any $\vx'$ that is in $\cX$ and its distance from any point not in $\cX$ is at least $\epsilon$.
    \end{itemize}
\end{definition}

\fi
\section{Proofs}
\label{sec:appendix}





\end{document}
