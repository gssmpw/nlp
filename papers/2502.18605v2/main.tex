\documentclass[11pt]{article}

% Recommended, but optional, packages for figures and better typesetting:
\usepackage{microtype}
\usepackage{graphicx}
\usepackage{subfigure}
\usepackage{booktabs} % for professional tables


\usepackage{hyperref}
\usepackage[table]{xcolor}         % colors

\newcommand{\declarecolor}[2]{\definecolor{#1}{RGB}{#2}\expandafter\newcommand\csname #1\endcsname[1]{\textcolor{#1}{##1}}}
\declarecolor{White}{255, 255, 255}
\declarecolor{Black}{0, 0, 0}
\declarecolor{Maroon}{128, 0, 0}
\declarecolor{Coral}{255, 127, 80}
\declarecolor{Red}{182, 21, 21}
\declarecolor{LimeGreen}{50, 205, 50}
\declarecolor{DarkGreen}{0, 80, 0}
\declarecolor{Purple}{146, 42, 158}
\declarecolor{Navy}{0, 0, 128}
\declarecolor{LightBlue}{84, 101, 202}
\definecolor{mydarkblue}{rgb}{0,0.08,0.45}
\hypersetup{ %
    pdftitle={},
    pdfkeywords={},
    pdfborder=0 0 0,
    pdfpagemode=UseNone,
    colorlinks=true,
    linkcolor=Navy,
    citecolor=DarkGreen,
    filecolor=Purple,
    urlcolor=Purple,
}

\usepackage{algorithm}
\usepackage{algorithmic}


\usepackage{natbib}
\bibliographystyle{plainnat}

%%% REVIEW
\newcommand{\tocite}{{\color{red}CITE} }
\newcommand{\toref}{{\color{red}REF} }

%%% LOGO
\newcommand{\usc}{\raisebox{-1pt}{\includegraphics[height=0.8em]{figures/usc_logo.png}}}
\newcommand{\vuam}{\raisebox{-1pt}{\includegraphics[height=0.8em]{figures/vu_logo.png}}}

%%% SIGNS and SYMBOLS
\newcommand{\grad}{\texttt{grad-CROP}}
\newcommand{\att}{\texttt{att-CROP}}
\newcommand{\seg}{\texttt{seg}}
\newcommand{\clip}{\texttt{clip-CROP}}
\newcommand{\sam}{\texttt{sam-CROP}}
\newcommand{\yolo}{\texttt{yolo-CROP}}
\newcommand{\hc}{\texttt{human-CROP}}
\newcommand{\zsvqa}{\texttt{ZSVQA}}
\newcommand{\vic}{\textbf{ViCrop}}
\newcommand{\xmark}{\text{\ding{55}}}
\newcommand{\cmark}{\text{\ding{51}}}
\newcommand{\success}{\texttt{\color{green} \cmark}}
\newcommand{\failure}{\texttt{\color{red} \xmark}}
\newcommand{\rel}{\texttt{rel-att}}
\newcommand{\gra}{\texttt{grad-att}}
\newcommand{\pgra}{\texttt{pure-grad}}
\newcommand{\relh}{\texttt{rel-att$^h$}}
\newcommand{\grah}{\texttt{grad-att$^h$}}
\newcommand{\pgrah}{\texttt{pure-grad$^h$}}


%%% Text Abb.
\makeatletter
\DeclareRobustCommand\onedot{\futurelet\@let@token\@onedot}
\def\@onedot{\ifx\@let@token.\else.\null\fi\xspace}

\def\aka{\emph{a.k.a}\onedot} \def\Eg{\emph{E.g}\onedot}
\def\eg{\emph{e.g}\onedot} \def\Eg{\emph{E.g}\onedot}
\def\ie{\emph{i.e}\onedot} \def\Ie{\emph{I.e}\onedot}
\def\cf{\emph{c.f}\onedot} \def\Cf{\emph{C.f}\onedot}
\def\etc{\emph{etc}\onedot} \def\vs{\emph{vs}\onedot}
\def\wrt{w.r.t\onedot} \def\dof{d.o.f\onedot}
\def\etal{\emph{et al}\onedot}
\makeatletter



\definecolor{myred}{HTML}{FF8577}
\definecolor{mygreen}{HTML}{0FA958}
\definecolor{myblue}{HTML}{1982C4}
\definecolor{codegreen}{rgb}{0,0.5,0}
\definecolor{codegray}{rgb}{0.5,0.5,0.5}
\definecolor{codepurple}{rgb}{0.07,0,0.53}
\definecolor{codered}{RGB}{189,41,0}
\definecolor{codecomment}{RGB}{153,153,153}
\definecolor{backcolour}{rgb}{0.96,0.96,0.96}
\definecolor{royalblue}{rgb}{0.0, 0.14, 0.4}
\definecolor{egyptianblue}{rgb}{0.06, 0.2, 0.65}
\definecolor{royalazure}{rgb}{0.0, 0.22, 0.66}
\definecolor{portlandorange}{rgb}{1.0, 0.35, 0.21}
\definecolor{sienna}{RGB}{183,105,68}
\definecolor{saddlebrown}{RGB}{139,69,19}
\definecolor{mediumbrown}{RGB}{83,41,11}
\definecolor{darkbrown}{RGB}{58,28,7}
\hypersetup{
    colorlinks=true,
    linkcolor=sienna,
    urlcolor=royalblue,
    citecolor=royalblue,
}


% Attempt to make hyperref and algorithmic work together better:
%\newcommand{\theHalgorithm}{\arabic{algorithm}}


\usepackage{multirow}

%\usepackage[ruled,linesnumbered]{algorithm2e}

% If accepted, instead use the following line for the camera-ready submission:
%\usepackage[accepted]{icml2025}

% For theorems and such
\usepackage{amsmath}
\usepackage{amssymb}
\usepackage{mathtools}
\usepackage{amsthm}

\usepackage{tikz}
\usepackage{pgfplots}

\usepackage{thm-restate}

\usepackage{fullpage}

% if you use cleveref..
%\usepackage{zref-clever}
%\newcommand{\Cref}[1]{{\zcref[S]{#1}}}
%\let\cref\Cref

\usepackage[capitalize,noabbrev,nameinlink]{cleveref}

\newcommand{\newzreftheorem}[2]{
    \newtheorem{#1}[theorem]{#2}
    \zcRefTypeSetup{#1}{Name-sg = #2}
}
%%%%%%%%%%%%%%%%%%%%%%%%%%%%%%%%
% THEOREMS
%%%%%%%%%%%%%%%%%%%%%%%%%%%%%%%%
\theoremstyle{plain}
\newtheorem{theorem}{Theorem}[section]
%\newzreftheorem{proposition}{Proposition}
\newtheorem{proposition}[theorem]{Proposition}
%\newzreftheorem{lemma}{Lemma}
\newtheorem{lemma}[theorem]{Lemma}
%\newzreftheorem{claim}{Claim}
\newtheorem{claim}[theorem]{Claim}
%\newzreftheorem{corollary}{Corollary}
\newtheorem{corollary}[theorem]{Corollary}
\theoremstyle{definition}
%\newzreftheorem{definition}{Definition}
\newtheorem{definition}[theorem]{Definition}
%\newzreftheorem{assumption}{Assumption}
\newtheorem{assumption}[theorem]{Assumption}
%\newzreftheorem{example}{Example}
\newtheorem{example}[theorem]{Example}
\theoremstyle{remark}
%\newzreftheorem{remark}{Remark}
\newtheorem{remark}[theorem]{Remark}
%\mathtoolsset{showonlyrefs=true}

\usepackage{nicefrac}
\usepackage{enumitem}

\usepackage{tikz}

\usepackage{MnSymbol}
\renewcommand\vec\bm
\renewcommand\grad\nabla

\newcommand*\tcircle[1]{%
  \raisebox{-0.5pt}{%
    \textcircled{\fontsize{7pt}{0}\fontfamily{phv}\selectfont #1}%
  }%
}

% \makeatletter
% \autonum@generatePatchedReferenceCSL{Cref}
% \autonum@generatePatchedReferenceCSL{cref}
% \makeatother

\newcommand{\bhz}[1]{{ {\color{magenta}{[BZ:~#1]}}}}
\newcommand{\et}[1]{{ {\color{olive}{[ET:~#1]}}}}
\newcommand{\emin}[1]{{ {\color{orange}{[EB:~#1]}}}}
\newcommand{\ioannis}[1]{{ {\color{blue}{[IA:~#1]}}}}
\newcommand{\gabri}[1]{{ {\color{red}{[GF:~#1]}}}}
\newcommand{\vince}[1]{{ {\color{green}{[vc:~#1]}}}}

\usepackage{authblk}

\title{Expected Variational Inequalities}

\author[1]{Brian Hu Zhang\thanks{Equal contribution.}}
\author[1]{Ioannis Anagnostides$^*$}
\author[1,2]{Emanuel Tewolde}
\author[1,2]{Ratip Emin Berker}
\author[3]{Gabriele Farina}
\author[1,2,4]{Vincent Conitzer}
\author[1,5]{Tuomas Sandholm}
\renewcommand\Affilfont{\small}
\affil[1]{Carnegie Mellon University}
\affil[2]{Foundations of Cooperative AI Lab (FOCAL)}
\affil[3]{Massachusetts Institute of Technology}
\affil[4]{University of Oxford}
\affil[5]{Additional affiliations: Strategy Robot, Inc., Strategic Machine, Inc., Optimized Markets, Inc.}
\affil[ ]{}
\affil[ ]{\texttt{\{bhzhang,ianagnos,etewolde,rberker,conitzer,sandholm\}}\texttt{@cs.cmu.edu}, \texttt{gfarina}\texttt{@mit.edu}}


\begin{document}

\maketitle

\pagenumbering{gobble}

\begin{abstract}
Variational inequalities (VIs) encompass many fundamental problems in diverse areas ranging from engineering to economics and machine learning. However, their considerable expressivity comes at the cost of computational intractability. In this paper, we introduce and analyze a natural relaxation---which we refer to as \emph{expected variational inequalities (EVIs)}---where the goal is to find a {\em distribution} that satisfies the VI constraint {\em in expectation}. By adapting recent techniques from game theory, we show that, unlike VIs, EVIs can be solved in polynomial time under general (nonmonotone) operators. EVIs capture the seminal notion of \emph{correlated equilibria}, but enjoy a greater reach beyond games. 
%
We also employ our framework to capture and generalize several existing disparate results, including from settings such as smooth games, and games with coupled constraints or nonconcave utilities. %Surprisingly, even in standard settings, our solution concept \emph{refines} correlated equilibria.

% Unlike VIs, we show that EVIs can be solved in polynomial time under broad assumptions. In doing so, we expand the scope of solution concepts and algorithmic techniques from game theory to a much broader class of problems.
\end{abstract}

\clearpage

\tableofcontents

\clearpage

\pagenumbering{arabic}

\section{Introduction}
\label{sec:intro}

\begin{figure*}[tb]
    \centering
    \includegraphics[width=0.848\linewidth]{figs/circuitnn.pdf} 
    \caption{Illustration of differentiable CircuitNN. CircuitNN is designed based on differentiable NAND gates. After DAS is guided by PI and PO pairs of the truth table, CircuitNN can get the precise circuit architecture logic equivalent to the truth table.}
    \label{fig:circuitnn}
\end{figure*}

% 1. Describe the importance of logic synthesis
% 2. Existing Problems
% (a) Neural Architecture Search: Unstable, Predefined Setting, etc.
% (b) Circuit Generation: Probabilistic Model, Logic Equivalence

With the rapid advancement of technology, the scale of integrated circuits (ICs) has expanded exponentially. 
This expansion has introduced significant challenges in chip manufacturing, particularly concerning power and area metrics.
A primary objective in IC design is achieving the same circuit function with fewer transistors, thereby reducing power usage and area occupancy.

Logic synthesis~\cite{hachtel2005logicsynth}, a critical step in electronic design automation (EDA), transforms behavioral-level circuit designs into optimized gate-level circuits, ultimately yielding the final IC layout. 
The primary goal of logic synthesis is to identify the physical implementation with the fewest gates for a given circuit function. 
This task constitutes a challenging NP-hard combinatorial optimization problem. 
Current logic synthesis tools~\cite{brayton2010abc, wolf2013yosys} rely on human-designed heuristics, often leading to sub-optimal outcomes.

Differentiable architecture search (DAS) techniques~\cite{liu2018darts, chu2020darts} offer novel perspectives on addressing challenges in this problem.
Circuit functions can be represented through truth tables, which map binary inputs to their corresponding outputs. 
Truth tables provide a precise representation of input-output relationships, ensuring the design of functionally equivalent circuits.
Inspired by this, researchers~\cite{deepmind2024ai4sys, wang2024tnet} have begun exploring the application of DAS to synthesize circuits directly from truth tables.
Specifically, \citet{deepmind2024ai4sys} proposed CircuitNN, a framework that learns differentiable connection structures with logic gates, enabling the automatic generation of logic circuits from truth tables.
This approach significantly reduces the complexity of traditional circuit generation. 
Building on this, \citet{wang2024tnet} introduced T-Net, a triangle-shaped variant of CircuitNN, incorporating regularization techniques to enhance the efficiency of DAS.

Despite these advancements, several challenges remain. 
The computational complexity of DAS grows quadratically with the number of gates, posing scalability issues.
Although triangle-shaped architecture~\cite{wang2024tnet} partially mitigates this problem, redundancy persists. 
%Additionally, DAS is susceptible to converging to local optima, limiting the ability to search architectures that satisfy the given truth tables~\cite{liu2018darts}. 
%Furthermore, hyperparameters (network depth and layer width) require extensive searches, introducing complexity and prolonging the synthesis process. 
Additionally, DAS is susceptible to converging to local optima~\cite{liu2018darts} and hyperparameters (network depth and layer width) require extensive searches. 
The challenges arise from the vast search space in DAS. 
% Even with predefined settings for CircuitNN, finding a configuration that meets the truth table requires extensive trial and error during the DAS process. 
Intuitively, limiting the search space through predefined parameters (network depth, gates per layer, and connection probabilities) can significantly reduce the complexity.

Recent advances~\cite{openai2023gpt4, abramson2024alphafold3, esser2024sd3, li2024mar} in conditional generative models have demonstrated remarkable performance across language, vision, and graph generation tasks. 
Motivated by these developments, we propose a novel approach to circuit generation that generates preliminary circuit structures to guide DAS in generating refined circuits matching specified truth tables. 
Firstly, we introduce CircuitVQ, a tokenizer with a discrete codebook for circuit tokenization. 
Built upon our Circuit AutoEncoder framework~\cite{hou2022graphmae,li2023maskgae,wu2025mgvga}, CircuitVQ is trained through a circuit reconstruction task. 
Specifically, the CircuitVQ encoder encodes input circuits into discrete tokens using a learnable codebook, while the decoder reconstructs the circuit adjacency matrix based on these tokens.
Subsequently, the CircuitVQ encoder serves as a circuit tokenizer for CircuitAR pretraining, which employs a masked autoregressive modeling paradigm~\cite{chang2022maskgit, li2023mage}. 
In this process, the discrete codes function as supervision signals. 
After training, CircuitAR can generate discrete tokens progressively, which can be decoded into initial circuit structures by the decoder of the CircuitVQ. 
These prior insights can guide DAS in producing refined circuits that match the target truth tables precisely.

Our key contributions can be summarized as follows:
\begin{itemize}
\item We introduce CircuitVQ, a circuit tokenizer that facilitates graph autoregressive modeling for circuit generation, based on our Circuit AutoEncoder framework;
\item Develop CircuitAR, a model trained using masked autoregressive modeling, which generates initial circuit structures conditioned on given truth tables;
\item Propose a refinement framework that integrates differentiable architecture search to produce functionally equivalent circuits guided by target truth tables;
\item Comprehensive experiments demonstrating the scalability and capability emergence of our CircuitAR and the superior performance of the proposed circuit generation approach.
\end{itemize}

% Motivation
% (a) Diffusion (Vision, Graph), Autoregressive (Language, Vision)
% (b) Circuit Generation for Predefined Setting
% (c) Neural Architecture Search for Strict Logic Equivalence

% Contribution
% (a) Circuit Tokenizer (new transformer arch, training strategy)
% (b) CircuitAR (train and gen strategies, post-ar strategy)
% (c) Extensive Evaluation including BitD (Bit Distance) for Scalability


\section{Related Work} \label{sec:related}

% \textbf{Adversarial Attack}
\textbf{Attacks on SLAM.} 
%With the rise of machine learning, 
The robustness of computer vision systems is being actively investigated. With the emergence of adversarial images in the digital domain by adding optimized noise directly to images~\cite{szegedy2013intriguing,carlini2017towards}, researchers find that such attacks also exist physically in the real world \cite{eykholt2018robust,song2018physical,zhao2019seeing}. To fill the gap between attacks in the digital and physical worlds, recent studies have demonstrated that attacks on real-world computer vision systems are practical \cite{eykholt2018robust,li2019adversarial,man2020ghostimage,sharif2016accessorize,zhao2019seeing,zhou2018invisible}. However, attacks on traditional computer vision methods such as SLAM are relatively less explored. \cite{yoshida2022adversarial} proposes an attack against the scan matching algorithm in LiDAR-based SLAM, while most SLAMs in AR/VR devices rely on different sensors like RGB/depth cameras and IMUs. \cite{ikram2022perceptual} and \cite{chen2024adversary} mislead visual SLAM by poisoning the images with special patterns, and \cite{wang2021can} causes the camera to fail using infrared light. In our work, we demonstrate attacks on Visual-Inertial SLAM (VI-SLAM) by perturbing the IMU readings, rather than cameras, and showing its impact on XR user experience. 

\textbf{Acoustic Injection Attacks.} Among various physical attacks, acoustic injection attacks are attractive due to their low cost. Son~\etal~\cite{son2015rocking} were the first to introduce acoustic attacks on MEMS gyroscopes, demonstrating how these attacks could lead to sensor denial-of-service and result in drone crashes. WALNUT~\cite{trippel2017walnut} expanded on this by developing output biasing and control attacks that enable precise manipulation of MEMS accelerometer outputs using modulated sound waves. Wang et al.~\cite{wang2017sonic} demonstrated a sonic gun, showcasing the vulnerability of various smart devices (\eg drones and self-balancing vehicles) to acoustic attacks. Tu et al. \cite{tu2018injected} designed side-swing and switching attacks to alter the outputs of MEMS gyroscopes and accelerometers. Furthermore, Ji et al. \cite{ji2021poltergeist} fool the object detectors by applying acoustic attack to the image stabilizers commonly used in modern cameras. However, none of the existing works study the relationship between the acoustic injections and SLAM outputs on recent XR devices. 

% \zijian{Do we need one session about security in AR/VR?}
% \yicheng{TODO}
%\jiasi{cite the AIVR paper (UMass Amherst?) paper is we have not already. They add IMU perturbation but w/o SLAM, iirc} \yicheng{Cited}

\textbf{XR Security and Privacy.} 
%Security and privacy concerns in XR systems have gained significant attention. 
For single-user XR systems, researchers have demonstrated various side-channel attacks to extract sensitive information (\eg keystrokes) through video feeds~\cite{ling2019know}, head movements~\cite{nair2023unique, slocum2023going}, architectural hints~\cite{zhang2023its,shang2020arspy}, power usage~\cite{li2024dangers}, and EM side-channel leakages~\cite{al2021vr}. In multi-user XR systems, Su et al.~\cite{su2024remote} use avatar motion data to infer keystrokes in shared VR environments. Slocum et al.~\cite{slocum2024doesn} reveal vulnerabilities in the shared state frameworks of multi-user AR. Similarly, Lebeck et al.~\cite{lebeck2017securing} highlight risks like deceptive virtual objects and emphasize access control for managing shared physical and virtual spaces. Ruth et al.~\cite{ruth2019secure} further propose a secure multi-user AR framework focusing on content sharing and permissions.
Chandio et al.~\cite{chandio2024stealthy} %introduced a multi-modal spatiotemporal attack that 
simultaneously manipulated visual and inertial sensors to disrupt XR pose estimation. However, their study evaluated the attack using offline datasets and assumed the attacker's capability to manipulate IMU data streams through acoustic means, without real experiments. Ours is the first to demonstrate acoustic injection attacks on recent XR devices, like the Hololens 2, in the real world.
 


\section{Preliminary}

\paragraph{Notation} Consider a sentence of $T$ tokens $\vx=\{\vx_1,\ldots, \vx_T\}\in\gX$, and let $P$ be the unknown target language distribution, $\tilde P(\vx)$ be the empirical distribution of the training data (which is an approximation of $P$), and $Q$ be the distribution of our model at hand. Since our paper is also closely related to RLHF, we will also use $\pi$ to represent the distributions. In particular, we sometimes write $\pi_\theta$ for a distribution that is parameterized by $\theta$, where $\theta$ is usually the set of trainable parameters of the LLM; we write $\pr$ for a reference distribution that should be clear given the context. The next token prediction loss is minimizing the forward-KL between $P$ and $Q$. 




\section{Existence and Complexity Barriers}
\label{sec:existence}

Perhaps the most basic question about $\Phi$-EVIs concerns their \emph{totality}---the existence of solutions. If one is willing to tolerate an arbitrarily small imprecision $\epsilon > 0$, we show that solutions exist under very broad conditions.

\begin{restatable}{theorem}{mainexistence}
    \label{theorem:existence}
    Suppose that $F : \cX \to \R^d$ is measurable and there exists $L > 0$ such that every $\phi \in \Phi$ is $L$-Lipschitz continuous. Then, for any $\epsilon > 0$, there exists an $\epsilon$-approximate solution to the $\Phi$-EVI problem.
\end{restatable}


In particular, our existence proof does not rest on $F$ being continuous. 
Instead, we consider the continuous function $\hatF$ that maps $\vx \mapsto \E_{\hatvx \sim \Delta(\cB_{\delta}(\vx) \cap \cX)} F(\hatvx)$ (\Cref{lemma:cont}), where $\cB_\delta(\vx)$ is the Euclidean ball centered at $\vx$ with radius $\delta = \delta(\epsilon)$. It then suffices to invoke Brouwer's fixed-point theorem for the gradient mapping $\vx \mapsto \proj_\cX ( \vx - \hatF(\vx) )$, where $\proj_\cX$ is the Euclidean projection with respect to $\cX$.
%Then, a uniform distribution around a VI solution $\vx^*$ to $\hatF$ makes makes an $\epsilon$-approximate solution to the $\Phi$-EVI problem for $F$

%In contrast to the classical existence result for VIs (see, \eg, Section 3 in \citealp{KinderlehrerS00}), our existence proof does not rest on $F$ being continuous. Instead, we can invoke this known result by considering a continuous version $\hatF$ of $F$ that maps $\vx \mapsto \E_{\hatvx \sim \cB_{\delta}(\vx) \cap \cX} F(\hatvx)$ (\Cref{lemma:cont}). Here, $\cB_\delta(\vx)$ denotes the Euclidean ball centered at $\vx$ with radius $\delta = \delta(\epsilon)$. 
% It then suffices to invoke Brouwer's fixed-point theorem for the gradient mapping $\vx \mapsto \proj_\cX ( \vx - \hatF(\vx) )$, where $\proj_\cX$ is the Euclidean projection with respect to $\cX$.
%Then, a uniform distribution around a VI solution $\vx^*$ to $\hatF$ makes makes an $\epsilon$-approximate solution to the $\Phi$-EVI problem for $F$.

\Cref{theorem:existence} implies that a $\Phi$-EVI can have approximate solutions even when the associated VI problem does not.\footnote{Noncontinuity of $F$ manifests itself prominently in \emph{nonsmooth} optimization (\emph{e.g.}, \citealp{Zhang20:Complexity,Davis22:Gradient,Tian22:Finite,Jordan23:Deterministic}); recent research there focuses on \emph{Goldstein} stationary points~\citep{Goldstein77:Optimization}, which are conceptually related to EVIs.}
%This is a very nice moytivation footnote. However, it is not clear why nonsmoothness relates to noncontinuity of F. Explain.???
%Also, if space allows, move the footnote into the body???

\begin{restatable}{corollary}{VIsvsEVIs}
    \label{prop:VI-vs-EVIs}
    There exists a VI problem that does not admit approximate solutions when $\epsilon = \Theta(1)$, but the corresponding $\epsilon$-approximate $\Phi$-EVI is total for any $\epsilon > 0$.
\end{restatable}

In the proof, we set $F$ to be the \emph{sign function} (\Cref{ex:sign-evi}). By contrast, if one insists on exact solutions, EVIs do not necessarily admit solutions.

\begin{restatable}{proposition}{notexact}
    \label{prop:notexact}
    When $F$ is not continuous, there exists an EVI problem with no solutions.
\end{restatable}

Furthermore, \Cref{theorem:existence} raises the question of whether it is enough to instead assume that every $\phi \in \Phi$ is continuous. Our next result dispels any such hopes.

\begin{restatable}{theorem}{countercont}
    \label{theorem:continuous-counterexample}
    There are $\Phi$-EVI instances that do not admit $\eps$-approximate solutions even when $\eps = \Theta(1)$, $F$ is piecewise constant, and $\Phi$ contains only continuous functions.
\end{restatable}

Our final result on existence complements~\Cref{theorem:existence,theorem:continuous-counterexample} by showing that, when $\Phi$ is finite-dimensional, it is enough if every $\phi \in \Phi$ admits a fixed point (this holds, for example, when $\phi$ is continuous---by Brouwer's theorem).

\begin{restatable}{theorem}{finitedim}
    \label{theorem:finitedim}
    Suppose that 
    \begin{enumerate}%[noitemsep,topsep=0pt]
        \item $\Phi$ is finite-dimensional, that is, there exists $k \in \N$ and a kernel map $m : \cX \to \R^k$ such that every $\phi \in \Phi$ can be expressed as $\mK m(\vx)$ for some $\mK \in \R^{d \times k}$; and
        \item every $\phi \in \Phi$ admits a fixed point, that is, a point $\cX \ni \vx = \fix(\phi)$ such that $\phi(\vx) = \vx$.
    \end{enumerate} Then, the $\Phi$-EVI problem admits an $\eps$-approximate solution with support size at most $1+dk$ for every $\eps > 0$.
\end{restatable}

Notably, this theorem guarantees the existence of solutions with finite support; the proof makes use of the minimax theorem~(\eg,~\citealp{Sion58:On}) in conjunction with Carath\'eodory's theorem on convex hulls~\cite{Caratheodory11:Uber}.

\paragraph{Complexity} Having established some basic existence properties, we now turn to the complexity of $\Phi$-EVIs. Let us define the VI gap function $\VIgap(\vx) \defeq - \min_{\vx' \in \cX} \langle F(\vx), \vx' - \vx \rangle$, which is nonnegative. If we place no restrictions on $\Phi$, it turns out that $\Phi$-EVIs are tantamount to regular VIs:


\begin{restatable}{proposition}{EVIequiv}
    \label{prop:EVI-VI}
    If $\Phi$ contains all measurable functions from $\cX$ to $\cX$, then any solution $\mu \in \Delta(\cX)$ to the $\epsilon$-approximate $\Phi$-EVI problem satisfies
    \begin{equation}
        \label{eq:EVI-VI}
        \E_{\vx \sim \mu} \VIgap(\vx) \leq \epsilon. 
    \end{equation}
\end{restatable}

In proof, it suffices to consider a $\phi$ that maps $\vx \in \cX$ to an appropriate point in $\argmin_{\vx' \in \cX} \langle F(\vx), \vx' - \vx \rangle$. When $\mu$ must be given explicitly, \Cref{prop:EVI-VI} immediately implies that $\Phi$-EVIs are computationally hard, because \eqref{eq:EVI-VI} implies that $\VIgap(\vx) \leq \epsilon$ for some $\vx$ in the support of $\mu$, and such a point can be identified in polynomial time.\footnote{This argument carries over without restricting the support of $\mu$, by assuming instead access to a sampling oracle from $\mu$: a standard Chernoff bound implies that the empirical distribution (w.r.t. a large enough sample size) approximately satisfies~\eqref{eq:EVI-VI}.}

\begin{corollary}
    \label{cor:hard-EVI}
    The $\epsilon$-approximate $\Phi$-EVI problem is \PPAD-hard even when $\epsilon$ is an absolute constant and $F$ is linear.
\end{corollary}

Coupled with~\Cref{prop:EVI-VI}, this follows from the hardness result of~\citet{Rubinstein15:Inapproximability} concerning Nash equilibria in (multi-player) polymatrix games (for binary-action, graphical games, \citealp{Deligkas23:Tight} recently showed that \PPAD-hardness persists up to $\epsilon < \nicefrac{1}{2}$). \Cref{cor:hard-EVI} notwithstanding, it is easy to see that the set of solutions to $\Phi$-EVIs is convex for any $\Phi \subseteq \cX^\cX$.

\begin{remark}
    Let $\cX = \cX_1 \times \dots \times \cX_n$, as in an $n$-player game. Whether \Cref{cor:hard-EVI} applies under deviations that can be decomposed as $\phi : \vx \mapsto \phi(\vx) = (\phi_1(\vx_1), \dots, \phi_n(\vx_n))$ is a major open question in the regime where $\epsilon \ll 1$ (\emph{cf.}~\citealp{Dagan24:From,Peng24:Fast}).
\end{remark}

Viewed differently, a special case of the $\Phi$-EVI problem arises when $\Phi = \{ \phi \}$ and $F(\vx) = \vx - \phi(\vx)$, for some fixed map $\phi : \cX \to \cX$. In this case, the $\Phi$-EVI problem reduces to finding a $\mu \in \Delta(\cX)$ such that
\begin{equation}
    \label{eq:fp}
    \E_{\vx\sim\mu} \ip{F(\vx), \phi(\vx) - \vx} = - \E_{\vx\sim\mu} \norm{\phi(\vx) - \vx}^2 \geq -\epsilon.
\end{equation}
As a result, $\mu$ must contain in its support an $\epsilon$-approximate fixed point of $\phi$, a problem which is \PPAD-hard already for quadratic functions~\citep{Zhang24:Efficient}.

\begin{corollary}
    \label{cor:hard-EVI-quad}
    The $\epsilon$-approximate $\Phi$-EVI problem is \PPAD-hard even when $\epsilon$ is an absolute constant, $F$ is quadratic, and $\Phi = \{ \phi \}$ for a quadratic map $\phi : \cX \to \cX$.
\end{corollary}

It is also worth noting that, unlike~\Cref{cor:hard-EVI}, $\Phi$ in the corollary above contains only continuous functions.

It also follows from~\eqref{eq:fp} that, for $\epsilon = 0$, $\Phi$-EVIs capture exact fixed points. The complexity class \FIXP~characterizes such problems~\citep{Etessami07:Complexity}.

\begin{corollary}
    The $\Phi$-EVI problem is \FIXP-hard, assuming that $\supp(\mu) \leq \poly(d)$.
\end{corollary}

Exponential lower bounds in terms of the number of function evaluations of $F$ also follow from~\citet{Hirsch89:Exponential}.

On a positive note, the next section establishes polynomial-time algorithms when $\Phi$ contains only \emph{linear} endomorphisms.\footnote{We do not distinguish between affine and linear maps because we can always set $\cX \gets \cX \times \{1\}$, in which case affine and linear maps coincide.}
\section{Efficient Computation with Linear Endomorphisms}
\label{sec:main}

The hardness results of the previous section highlight the need to restrict the set $\Phi$ in order to make meaningful progress. Our main result here establishes a polynomial-time algorithm when $\Phi$ contains only linear endomorphisms.

\begin{theorem}
    \label{th:elvi}
    If $\Phi$ contains only linear endomorphisms, the $\eps$-approximate $\Phi$-EVI problem can be solved in time $\poly(d, \log(B/\epsilon))$ given a membership oracle for $\cX$.
\end{theorem}

The proof relies on the ellipsoid against hope ($\eah$), and in particular, a recent generalization by~\citet{Daskalakis24:Efficient}. In a nutshell, the main deficiency in the framework covered earlier in~\Cref{sec:eah} is that one needs a separation oracle for $\cY$ (\Cref{theorem:eah}), where $\cY$ for us is the set of deviations $\Phi$. Unlike some applications, in which $\cY$ has an explicit, polynomial representation~\citep{Papadimitriou08:Computing}, that assumption needs to be relaxed to account for $\Philin$~\citep[Theorem 3.4]{Daskalakis24:Efficient}.

\citet{Daskalakis24:Efficient} address this by considering instead the $\either$ oracle. As the name suggests, for any $\vy \in \R^m$, it \emph{either} returns a hyperplane separating $\vy$ from $\cY$, or a good-enough-response $\vx \in \cX$. They showed that~\Cref{theorem:eah} can be extended under this weaker oracle (in place of $\ger$ and $\sep$); the formal version is given in~\Cref{th:eahrelax}.

In our setting, we consider the feasibility problem
\begin{align}\label{eq:evi-ellipsoid}
        \text{find}\quad \phi \in \Philin \qq{s.t.} \ip{F(\vx), \phi(\vx) - \vx} \le -\eps \quad \forall \vx \in \cX.
\end{align}
Equivalently,
\begin{align*}
        \text{find}\quad \mK \in \R^{d \times d} \qq{s.t.} \\\ip{F(\vx), \mK\vx - \vx} \le -\eps \quad &\forall \vx \in \cX, \\\mK \vx \in \cX \quad &\forall \vx \in \cX.
\end{align*}
This program is infeasible since, for any $\phi \in \Philin$, the fixed point $\vx$ of $\phi$ makes the left-hand side of the constraint $0$. And a certificate of infeasibility is an $\eps$-approximate $\Philin$-EVI solution. Thus, it suffices to show how to run the ellipsoid algorithm on \eqref{eq:evi-ellipsoid}. By~\Cref{th:eahrelax}, it suffices if for any $\mK \in \R^{d \times d}$, we can compute efficiently \emph{either}
\begin{itemize}%[noitemsep,topsep=0pt]
        \item some $\vx \in \cX$ such that $\mK\vx = \vx$ ($\ger$), {\em or}
        \item some hyperplane separating $\mK$ from $\Philin$ ($\sep$).
\end{itemize}
This is precisely the \emph{semi-separation oracle} solved by~\citet[Lemma~4.1]{Daskalakis24:Efficient}, stated below.

\begin{lemma}[\citealp{Daskalakis24:Efficient}]
    \label{lemma:semiseparation}
    There is an algorithm that takes as input $\mat{K} \in \R^{d \times d}$, runs in $\poly(d)$ time, makes $\poly(d)$ oracle queries to $\cX$, and either returns a fixed point $\cX \ni \vx = \mat{K} \vx$, or a hyperplane separating $\mat{K}$ from $\Philin$.
\end{lemma}

% An immediate consequence is that \emph{linear} correlated equilibria %undefined; explain what that term means??
% can be computed %found would be a better word than computed ??
% in polynomial time in convex games~\citep{Daskalakis24:Efficient}.

On a separate note, \Cref{th:elvi} only accounts for approximate solutions. We cannot hope to improve that in the sense that exact solutions might be supported only on irrational points even in concave maximization (\emph{cf.}~\Cref{prop:convex-equiv}).

\subsection{Regret minimization for EVIs on polytopes}

One caveat of~\Cref{th:elvi} is that it relies on the impractical $\eah$ algorithm. To address this limitation, we will show that $\Phi$-EVIs are also amenable to the more scalable approach of \emph{regret minimization}---albeit with an inferior complexity growing as $\poly(1/\epsilon)$.

Specifically, in our context, the regret minimization framework can be applied as follows. At any time $t \in \N$, we think of a ``learner'' selecting a point $\vx^{(t)} \in \cX$, whereupon $F(\vx^{(t)})$ is given as feedback from the ``environment,'' so that the utility at time $t$ reads $- \langle \vx^{(t)}, F(\vx^{(t)}) \rangle$. \emph{$\Phi$-regret} is a measure of performance in online learning, defined as
%
\begin{equation*}
    \phireg^{(T)} \defeq \max_{\phi \in \Phi} \sum_{t=1}^T \langle F(\vx^{(t)}), \phi(\vx^{(t)}) - \vx^{(t)} \rangle.
\end{equation*}
The uniform distribution $\mu$ on $\{ \vx^{(1)}, \dots, \vx^{(T)} \}$ is clearly a $\phireg^T/T$-approximate $\Phi$-EVI solution. 

In what follows, we will assume that $\cX$ is a polytope given explicitly by linear constraints, \ie, \begin{align*}
    \cX = \{ \vx \in \R^d : \mA \vx \le \vb \},
\end{align*}
where $\mA \in \Q^{m \times d}$ and $\vb \in \Q^m$ are given as input.

To minimize $\Phi$-regret, we will make use of the template by~\citet{Gordon08:No}, which comprises two components. The first is a fixed-point oracle, which takes as input a function $\phi \in \Philin$ and returns a point 
$\vx \in \cX$ with
$\vx = \phi(\vx)$; given that $\phi$ is linear, it can be implemented efficiently via linear programming. The second component is an algorithm for minimizing (external) regret over the set $\Philin$. In~\Cref{theorem:explicit-repres}, we devise a polynomial representation for $\Philin$:

\begin{theorem}
    \label{theorem:regret}
   For an arbitrary polytope $\cX$ given by explicit linear constraints, there is an explicit representation of $\Philin$ as a polytope with $O(d^2 + m^2)$  variables and constraints.
\end{theorem}
As a consequence, we can instantiate the regret minimizer operating over $\Philin$ with projected gradient descent.
\begin{corollary}
    \label{cor:regret}
    There is a deterministic algorithm that guarantees $\philinreg^{(T)} \leq \epsilon$ after $\poly(d, m)/\eps^2$ rounds, and requires solving a convex quadratic program with $O(d^2+m^2)$ variables and constraints in each iteration.
\end{corollary}
% \begin{corollary}
%     There is a randomized algorithm that guarantees $\philinreg^{(T)} \leq \epsilon$ after $\qty(\poly(d, m) + O(\log(1/\delta)))/\eps^2$ with probability at least $1/\delta$, and requires solving a linear program with $O(d^2+m^2)$ variables and constraints on each iteration\todo{}
% \end{corollary}

An additional benefit of~\Cref{cor:regret} compared to using $\eah$ is that the former is more suitable in a decentralized environment---for example, in multi-player games (\emph{cf.}~\Cref{example:CCE}). There, \Cref{cor:regret} corresponds to each player running their own independent no-regret learning algorithm. Even in this setting, our algorithms actually yield an improvement over the best-known algorithms for minimizing $\Philin$-regret over explicitly-represented polytopes: the previous state of the art, due to \citet{Daskalakis24:Efficient}, requires running the ellipsoid algorithm on each iteration, which is slower than quadratic programming (\Cref{sec:repre}).
\section{Game Theory Applications of EVIs}
\label{sec:games}

A major motivation for studying $\Phi$-EVIs lies in a strong connection to \emph{(C)CEs}~\citep{Aumann74:Subjectivity} in games. Indeed, we begin this section by pointing out that $\Phi$-EVIs capture 
(C)CEs for specific choices of $\Phi$. %It is worth pointing out that Aumann's key justification for CE was that such equilibria adhere to Bayesian rationality, which carries over to $\Phi$-EVIs.

We will mostly consider $n$-player \emph{concave} games. Here, each player $i \in \range{n}$ selects a strategy $\vx_i \in \cX_i$ from some convex and compact set $\cX_i$, and its utility is given by $u_i : (\vx_1, \dots, \vx_n) \mapsto \R$. We assume that $u_i(\vx_i, \vx_{-i})$ is differentiable and concave in $\vx_i$ for any $\vx_{-i}$, and that the gradients $\grad_{\vx_i} u_i(\vx_i, \vx_{-i})$ are bounded. We let $\cX \defeq \cX_1 \times \dots \times \cX_n$.

\begin{example}[CCE]
    \label{example:CCE}
    A distribution $\mu \in \Delta(\cX)$, is an \emph{$\epsilon$-coarse correlated equilibrium (CCE)}~\citep{Moulin78:Strategically} if for any player $i \in \range{n}$,
    \begin{equation}
        \label{eq:CCE-dev}
        \delta_i \defeq  \max_{\vx_i' \in \cX_i} \E_{\vx \sim \mu} u_i(\vx_i', \vx_{-i})  - \E_{\vx \sim \mu} u_i(\vx) \leq  \epsilon.
    \end{equation}
    Now, consider an $\epsilon$-approximate EVI solution $\mu$ of the problem defined by 
    $$F \defeq ( - \nabla_{\vx_1} u_1(\vx), \dots, - \nabla_{\vx_n} u_n(\vx)).$$ 
    Such $\mu$ satisfies, by concavity, $\sum_{i=1}^n \delta_i \leq \epsilon$; it is not necessarily an $\epsilon$-approximate CCE since it is possible that for some $i \in [n]$, {\em all} deviations strictly decrease $i$'s utility (so that $\delta_i$ in~\eqref{eq:CCE-dev} is negative)---$\mu$ is technically an \emph{average} CCE in the parlance of~\citet{Nadav10:Limits}. To capture CCE via $\Phi$-EVIs, one can instead consider a richer set of deviations of the form $(\vx_1, \dots, \vx_n) \mapsto (\vx_1, \dots, \vx_i', \dots, \vx_n)$ for all $i \in [n]$ and $\vx_i' \in \cX_i$.
\end{example}

A canonical example of the above formalism is a \emph{normal-form game}, in which each constraint set $\cX_i$ is the probability simplex $\Delta(\cA_i)$ over a finite set of \emph{actions} $\cA_i$, and each utility $u_i$ is a multilinear function.

\begin{example}[LCE]
    \label{example:CE}
    A distribution $\mu \in \Delta(\cX)$ is an \emph{$\epsilon$-linear correlated equilibrium (LCE)} if for any $i \in [n]$,
    \begin{equation*}
        \max_{\phi_i \in \Phi_i} \E_{ \vx \sim \mu } u_i(\phi_i(\vx_i), \vx_{-i}) - \E_{\vx \sim \mu} u_i(\vx) \leq \epsilon,
    \end{equation*}
    where $\Phi_i$ contains all linear functions from $\cX_i$ to $\cX_i$. To capture LCE via $\Phi$-EVIs, it suffices to consider deviations of the form $(\vx_1, \dots, \vx_n) \mapsto (\vx_1, \dots, \phi_i(\vx_i), \dots, \vx_n)$ for all $i \in [n]$ and $\phi_i \in \Phi_i$.
\end{example}

For normal-form games, LCEs amount to the usual notion of CEs~\citep{Aumann74:Subjectivity}. LCEs were introduced in the context of extensive-form games~\citep{Farina23:Polynomial,Farina24:Polynomial}.

\paragraph{Refining correlated equilibria} In fact, and more surprisingly, $\Philin$-EVI solutions can be a strict subset of LCEs.\footnote{The example of~\citet[Example 1]{Ahunbay25:First} already implies that certain CEs can be excluded from the set of $\Philin$-EVIs, which, incidentally, could have implications for last-iterate convergence in some classes of games, as discussed by that auhtor. Our example in~\Cref{fig:bach or stravinsky} goes much further, revealing that $\Philin$-EVIs can yield significantly different utilities for each player compared to CEs.} This separation can already be appreciated in the setting of normal-form games, and manifests itself in at least two distinct ways. First, there exist games for which a CE need not be a solution to the $\Philin$-EVI. In this sense, $\Philin$-EVIs yield a computationally tractable superset of Nash equilibria that is tighter than CEs. Second, computation suggests that the set of solutions of the $\Philin$-EVI for the game need not be a polyhedron, unlike the set of CEs. We provide a graphical depiction of this phenomenon in \cref{fig:bach or stravinsky}. The figure depicts the set of $\Philin$-EVI solutions to a simple ``Bach or Stravinsky'' game, in which the players receive payoffs $(3,2)$ if they both pick Bach, $(2,3)$ if they both pick Stravinsky, and $(0,0)$ otherwise.

\paragraph{Interpretation} The reason for this separation is that, for a map $\phi : \cX \to \cX$, each player's mapped strategy $\phi(\vx)_i$ can also depend (linearly) on {\em other players' strategies} $\vx_{-i}$. Indeed, the EVI formulation of a game does not take into account the identities of the players. For this reason, we will call the set of $\Philin$-EVI solutions in a concave game {\em anonymous linear correlated equilibria}, or {\em ALCE} for short. We give two game-theoretic interpretations of ALCEs.

First, the ALCEs of a game $\Gamma$ are the {\em symmetric} LCEs of the ``symmetrized'' game in which the players are randomly shuffled before the game begins. That is, consider the $n$-player game $\Gamma^\text{sym}$ defined as follows. Each player's strategy set is $\cX$. For strategy profile $(\vx^1, \dots, \vx^n) \in \cX^n$, the utility to player $i$ is given by 
\begin{equation*}
    u_i^\text{sym}(\vx^1, \dots, \vx^n) = \frac{1}{n!} \sum_{\sigma \in \Sym_n } u_{\sigma(i)}(\vx^{\sigma^{-1}(1)}_1, \dots, \vx^{\sigma^{-1}(n)}_n),
\end{equation*}
where $\Sym_n$ is the set of permutations $\sigma : [n] \to [n]$. The following result then follows almost by definition.
\begin{restatable}{proposition}{propJointLCESymmetric}\label{prop:joint lce symmetric}
    For a given distribution $\mu \in \Delta(\cX)$, define the distribution $\mu^n \in \Delta(\cX^n)$ by sampling $\vx\sim\mu$ and outputting $(\vx, \dots, \vx) \in \cX^n$. Then, $\mu$ is a ALCE of $\Gamma$ if and only if $\mu^n$ is an LCE of $\Gamma^\textup{sym}$. 
\end{restatable}

Second, for normal-form games, the ALCEs are the distributions $\mu \in \Delta(\cX)$ such that no player $i$ has a profitable deviation of the following form. The correlation device first samples $\vx\sim\mu$, and samples recommendations $a_j \sim \vx_j$ for each player $j$. Then, the player selects another player $j$ (possibly $j=i$) whose recommendation it wishes to see. The player then observes a sample $a_j' \sim \vx_j$ that is {\em independently} sampled from $a_j$.\footnote{This independence is crucial: without it, $\mu$ would actually need to be a distribution over pure Nash equilibria!} Finally, the player chooses an action $a_i^* \in \cA_i$, and each player $j$ gets reward $u_j(a_i^*, a_{-i})$. Thus, players are allowed (modulo the independent sampling) to {\em spy} on each others' recommendations.

Further discussion about ALCEs and formal proofs of the claims in this section are  deferred to \Cref{sec:appendix-joint}.

%Is it sufficiently clear that the operator is the game operator? I haven't checked. Ioannis: I can make it explicit in Example 5.1

% Insert here plot
\begin{figure}[t]
    \usepgfplotslibrary{fillbetween}
    \usetikzlibrary{patterns}
    \centering
    \scalebox{.8}{  
        \pgfplotsset{width=10cm,height=8cm,compat=1.18}
    \begin{tikzpicture}
    \begin{axis}[grid,xmin=0,xmax=1,axis line style=semithick,xlabel={Probability of Player 1's first action (Bach)},ylabel={Probability of Player 2's first action (Bach)},ymin=0,ymax=1,xtick distance=0.1,ytick distance=0.1,clip marker paths=true,set layers,legend
    %, axis equal image
    ]
        \begin{pgfonlayer}{axis background}
            \fill [thick,pattern=north west lines,pattern color=red] (0,0) -- (0.7142857142760732, 0.28571428572392693) -- (1,1) -- (0.3750000001824576, 0.6249999998175428) -- cycle;
        \end{pgfonlayer}

        \draw [very thick,red] (0,0) -- (0.7142857142760732, 0.28571428572392693) -- (1,1) -- (0.3750000001824576, 0.6249999998175428)  -- cycle;

    
        \addplot [blue,very thick,name path=A,domain=0:1]
            {1/20 * (-11 + 25 * x + sqrt(121 - 310 * x + 225 * x^2))};
     
        \addplot [very thick, blue, name path=B,domain=0:1,samples=2]
            {x};
     
        \addplot [on layer=main,blue, fill opacity=0.3] fill between [of=A and B];
        % \fill [
        %     intersection segments={
        %         of=A and B,
        %         sequence=R1 -- L2},
        %     pattern=north east lines,
        % ] -- cycle;

        \begin{pgfonlayer}{pre main}
            \addplot[mark=*,only marks,mark size=1.0mm,thick] coordinates {(0,0) (.6,.4) (1,1)};
        \end{pgfonlayer}

        \node[red, inner sep=.5mm,rounded corners, fill=white,rotate=0] at (.67,.31) {\small CE};

        % \node[blue, inner sep=.5mm,rounded corners,rotate = 39] at (.5,.4) {\small \contour{white}{$\Philin$-EVI}};
        % \legend{CE,$\Philin$-EVI}

        \begin{scope}[x=1cm,y=1cm,yshift=5.2cm,xshift=2mm]
           \filldraw[gray,rounded corners=.3mm,fill=white,fill opacity=.5] (0,-.45) rectangle (3.9,1);
           \draw[thick,blue,fill=blue,fill opacity=.3] (.1,.1) rectangle (.7,.4);
           \node[anchor=west] at (.8,.25) {\small $\Philin$-EVI solutions};

           \draw[thick,red,pattern=north west lines,pattern color=red] (.1,.6) rectangle (.7,.9);
           \node[anchor=west] at (.8,.7) {\small Correlated equil.};

           \draw[fill=black] (.4, -.2) circle (1mm);
           % \draw[semithick] (.4, -.2) -- +(45:1mm) -- +(45:-1mm) -- (.4, -.2) -- +(135:1mm) -- +(-45:1mm);
           \node[anchor=west] at (.8,-.2) {\small VI sol. (Nash eq.)};
        \end{scope}
    \end{axis}
    \end{tikzpicture}}
    % \vspace{-2mm}
    \caption{
        Marginals of the set of correlated equilibria (CE) and of the set of solutions to $\Philin$-EVI in the simple $2 \times 2$ game ``Bach or Stravinsky.'' The x- and y-axes show the probability with which the two players select the first action (Bach). The set of marginals of $\Philin$-EVI solutions appears to have a curved boundary corresponding, we believe, to the hyperbola $10 x^2 - 25 xy + 10 y^2 - 6x+11y=0$.
    }
    \label{fig:bach or stravinsky}
    \vspace{-3mm}
\end{figure}

\paragraph{Coupled constraints}
Continuing from~\Cref{example:CCE,example:CE}, we observe that ($\Philin$-)EVIs can be used even in ``pseudo-games,'' in which $\cX$ does not necessarily decompose into $\cX_1 \times \dots \times \cX_n$; this means that $\vx_i \in \cX_i(\vx_{-i})$. As we discuss in~\Cref{sec:related}, most prior work in such settings has focused on generalized Nash equilibria, with the exception of~\citet{Bernasconi23:Constrained}. ($\Philin$-)EVIs induce an interesting notion of LCE/CCE in pseudo-games, albeit not directly comparable to the one put forward by~\citet{Bernasconi23:Constrained}. It is worth noting that~\citet{Bernasconi23:Constrained} left open whether efficient algorithms for computing their notion of (coarse) correlated equilibria exist.

\begin{definition}
    Given an $n$-player pseudo-game with concave, differentiable utilities and joint constraints $\cX$, a distribution $\mu \in \Delta(\cX)$ is an \emph{$\epsilon$-ALCE} if
    \begin{equation*}
        \max_{\phi \in \Philin} \E_{\vx \sim \mu} \sum_{i=1}^n u_i(\phi(\vx)_i, \vx_{-i}) - \sum_{i=1}^n u_i(\vx) \leq \epsilon.
    \end{equation*}
\end{definition}

By virtue of our main result (\Cref{th:elvi}), such an equilibrium can be computed in polynomial time.

\paragraph{Noncontinuous gradients} In fact, our results do not rest on the usual assumption that each player's gradient is a continuous function, thereby significantly expanding the scope of prior known results even in games. For example, we refer to~\citet{Dasgupta86:Existence,Bichler21:Learning,Martin24:Joint} for pointers to some applications.

\paragraph{Nonconcave games} Last but not least, $\Phi$-EVIs give rise to a notion of \emph{local} $\Phi$-equilibrium (\Cref{def:localPhi}) in nonconcave games. It turns out that this captures recent results by~\citet{Cai24:Tractable} and~\citet{Ahunbay25:First}, but our framework has certain important advantages. First, we give a $\poly(d, \log(1/\epsilon))$-time algorithm (\Cref{th:elvi}), while theirs scale polynomially in $1/\epsilon$. Second, our results do not assume continuity of the gradients. And finally, our algorithms are polynomial even when $\Phi$ contains all linear endomorphisms (\Cref{th:elvi}). \Cref{sec:localPhi} elaborates further on those points.

\section{Problems where EVIs coincide with VIs}

We saw earlier, in~\Cref{prop:EVI-VI}, that when $\Phi$ comprises all functions from $\cX$ to $\cX$, the $\Phi$-EVI problem is tantamount to the associated VI problem. However, if one restricts the functions contained in $\Phi$, are there still structured VIs where we retain this equivalence? 
In this section, we consider certain structured VIs, and show their equivalence to the corresponding EVIs (that is, $\Phiconst$-EVIs). Unlike general VIs, the ones we examine below are tractable.%---because of that equivalence.


\subsection{Polymatrix zero-sum games and beyond}

The first important class of VIs we consider is described by a condition given below.

\begin{proposition}
    \label{prop:collapse}
    Suppose that for any $\vx' \in \cX$, the function $g : \vx \mapsto \langle F(\vx), \vx' - \vx \rangle$ is concave. Then, if $\mu \in \Delta(\cX)$ is an $\epsilon$-approximate solution to the EVI, $\E_{\vx \sim \mu} \vx$ is an $\epsilon$-approximate solution to the VI.
\end{proposition}

The proof follows directly from Jensen's inequality.

The precondition of~\Cref{prop:collapse} is satisfied, \emph{e.g.}, when: (i) $\langle F(\vx), \vx \rangle = 0$ for all $\vx \in \cX$, and (ii) $F$ is a linear map. In the context of $n$-player games, the first condition amounts to the zero-sum property: $\sum_{i=1}^n u_i(\vx) = 0$ for all $\vx$. Of course, this property is not enough to enable efficient computation of Nash equilibria, for every two-player (general-sum) game can be converted into a $3$-player zero-sum game. This is where the second condition comes into play: $F$ is a linear map---that is, each player's gradient must be linear in the joint strategy. Those two conditions are satisfied in \emph{polymatrix zero-sum} games~\citep{Cai16:Zero}; in such games, the conclusion of~\Cref{prop:collapse} is a well-known fact. 

\subsection{Quasar-concave functions}

We next consider the problem of maximizing a (single) function that satisfies \emph{quasar-concavity}---a natural generalization of concavity that has received significant  interest~\citep{Hardt18:Gradient,Fu23:Accelerated,Hinder20:Near,Gower21:SGD,Guminov23:Accelerated,Caramanis24:Optimizing}.\footnote{Prior literature mostly uses the term \emph{quasar-convexity}, which is equivalent to quasar-concavity for the opposite function $-u$.}

\begin{definition}[Quasar-concavity]
    \label{def:quasar}
    Let $\gamma \in (0, 1]$ and $\vxstar \in \cX$ be a maximizer of a differentiable function $u : \cX \to \R$. We say that $u$ is \emph{$\gamma$-quasar-concave} with respect to $\vxstar$ if
    \begin{equation}
        \label{eq:quasar}
        u(\vxstar) \leq u(\vx) + \frac{1}{\gamma} \langle \nabla u(\vx), \vxstar - \vx \rangle \quad \forall \vx \in \cX.
    \end{equation}
\end{definition}

In particular, in the special case where $\gamma = 1$, \eqref{eq:quasar} is equivalent to \emph{star-concavity}~\citep{Nesterov06:Cubic}. If in addition \eqref{eq:quasar} holds for all $\vxstar \in \cX$ (not merely w.r.t. a global maximizer), it captures the usual notion of concavity.

Any reasonable solution concept for such problems should place all mass on global maxima; EVIs pass this litmus test:

\begin{proposition}
    \label{prop:convex-equiv}
    Let $F = - \grad u$ for a $\gamma$-quasar-concave and differentiable function $u : \cX \to \R$. Then, for any solution $\mu \in \Delta(\cX)$ to the EVI problem,
    \begin{equation*}
        \E_{\vx \sim \mu} u(\vx) \geq \max_{\vx \in \cX} u(\vx).
    \end{equation*}
    Thus, $\mathbb{P}_{\vx\sim\mu}[u(\vxstar) = u(\vx)] = 1$, for $\vxstar \in \argmax_{\vx} u(\vx)$.
\end{proposition}

Indeed, by~\Cref{def:quasar}, $0 \leq \E_{\vx\sim\mu} \ip{\grad u(\vx), \vx - \vxstar } \le \gamma \E_{\vx\sim\mu} [u(\vx) - u(\vxstar) ]$ for any EVI solution $\mu \in \Delta(\cX)$. Thus, under quasar-concavity, VIs basically reduce to EVIs.


%It would be hard to justify a notion that, in a (two-player) zero-sum game, places mass on strategies that do not constitute minimax strategies.

%More broadly, and in particular for multi-player games, there is no definite justification for EVIs \emph{vis-\`a-vis} other notions---this ties directly to the \emph{equilibrium selection} problem, which has generated much debate~\citep{Harsanyi88:General}.
\section{The smoothness conditions}\label{sec:OU-smooth}

In this section, we will compare \Cref{assump:smooth} with the smoothness assumption in \cite{HZD+24} and prove \Cref{thm:main-smooth}.

Recall that we assume the target distribution $\mu$ with density $p_{\mu}$ to be $L$-log-smooth. This assumption is typically essential for bounding the discretization error of sampling algorithms. In many works based on denoising diffusion probabilistic models (DDPMs) (e.g. \cite{CCL+23,LLT23,CLL23,HZD+24}), they further assume that the distributions during the OU process starting from $\mu$ are also $L$-log-smooth. 
% \ctodo{What is the $L$-smoothness assumption in those ``score function estimation $\implies$ good sampler'' works}

The definition for the OU process is as follows. Suppose we start from a random point $X_0\sim \mu$. The OU process $\set{X_t}_{t\geq 0}$ evolves with the following equation
\[
    \d X_t = -X_t \dd t + \sqrt{2}\d B_t
\]
where $\set{B_t}_{t>0}$ is the standard Brownian motion. The solution of the above equations is
\begin{equation}
    X_t = e^{-t}X_0 + \sqrt{2}\cdot e^{-t}\int_{0}^t e^s \d B_s. \label{eq:OU}
    % X_t = e^{-t}X_0 + \sqrt{1-e^{-2t}} Z_t, \label{eq:OU}
\end{equation}
From direct calculation, we know that $
\sqrt{2}\cdot e^{-t}\int_{0}^t e^s \d B_s \sim \+N\tp{0, (1-e^{-2t})I_d}$.
% where $Z_t$ is drawn from a standard Gaussian distribution.

There have been many convergence guarantees for the DDPMs.  Let $\mu_t$ be the distribution of $X_t$ and $p_t$ be the corresponding density function. Assuming the second-moment of $\mu$ is bounded by $M$ and $\log p_t$ being $L$-smooth for any $t\geq 0$, the work of \cite{HZD+24} proposed an algorithm that guarantees with high probability, the output distribution is $\tilde{\+O}(\eps)$-close to the target distribution in KL divergence, requiring at most $\exp\set{\+O\tp{L^3\cdot \log^3\frac{Ld+M}{\eps}} \cdot \max\ab\{\log\log Z^2,1\}}$ queries \footnote{Here $Z$ is the maximum norm of particles appeared in their algorithm.}. 
% This implies that as long as the smoothness condition of $\log p_t$ is satisfied with constant $L$, a quasi-polynomial sampling algorithm exists. 

It has been known that a smooth $\log p_{\mu}$ can imply the smoothness of $\log p_t$ in some specific cases, for example when $t$ is small (\cite{CLL23}) or when $\mu$ is strongly log-concave (\cite{LPSR21}). Via the techniques in Lemma~12 of \cite{CLL23}, one can prove an $O(d)$ upper bound of $\|\grad^2 \log p_t\|_{\!{op}}$ in expectation when $t=\Omega(1)$. However, this bound does not offer much utility for the algorithm of \cite{HZD+24}, as their results will be super-exponential under an $O(d)$-smoothness bound.

If the smoothness bound for $\log p_{\mu}$ is large, the bound in \cite{HZD+24} will be poor. Fortunately, we can assume $\log p_{\mu}$ to be $\+O(1)$-smooth without loss of generality. This is because we can always scale the domain to adjust the distribution's smoothness bound, while not changing the product of the smoothness bound and the second moment $M$ (\Cref{lem:LM}). Therefore, with the initial distribution being $\+O(1)$-log-smooth and the second moment being polynomial in $d$, if the $\log p_t$'s also remain $\+O(1)$-smooth, quasi-polynomial sampler exists. 
% So the relationship between the smoothness of the initial distribution and that of the distributions during the OU process is worth studying. 

% Hence previous works do not pay much attention to distinguishing between these two smoothness conditions.

Therefore, we are interested in the conditions for the $O(1)$-smoothness to be kept during the entire OU process. Our results in \Cref{thm:main-lb} indicates an exponential lower bound even for $O(1)$-log-smooth distributions. Then we know that for those hard instances constructed in \Cref{sec:lb}, $\log p_t$ cannot always be $O(1)$-smooth during the OU process. Otherwise a quasi-polynomial sampler exists from \cite{HZD+24}. 
% we found that the smoothness bound for $\log p_t$ can be significantly worse than that of the initial $\log p_{\mu}$.
In \Cref{subsec:stitched}, we introduce a family of $\+O(1)$-log-smooth distribution, which we refer to as the stitched Gaussian distributions and is a simplified version of those hard instances in \Cref{sec:lb}. As the OU process evolves, the bound on the smoothness of stitched Gaussians can become $\omega(1)$ at certain time $t$. 
% Given this counterexample, the relationship between the smoothness of the initial distribution and that of the distributions during the OU process is worth studying. 

To see the case for other non-log-concave distributions, in \Cref{subsec:mix}, we considered a class of classical multi-modal distributions, the mixture of Gaussians, and provide analyses of their smoothness properties with different parameter settings. Although mixture of Gaussians appear to be quite similar to the stitched Gaussians, our results demonstrate that they exhibit fundamentally different behaviors in terms of smoothness. To be specific, we show that for a mixture of two Gaussians with mean $u_1$ and $u_2$, if their covariance matrices $\Sigma_1=\Sigma_2\succeq \Omega(1)\cdot \!{Id}_d$, the smoothness of the distributions are almost determined by $\|u_1-u_2\|$ and the $\log p_t$'s will inherit the $\+O(1)$-smoothness is $\log p_{\mu}$ is $\+O(1)$-smooth. In contrast, when the covariance matrices differ, even with $\|u_1-u_2\|=o(1)$, the initial $\log p_{\mu}$ is not $L$-smooth for any $L=o(d)$. 

We also explore the mixture of multiple Gaussians in \Cref{subsec:mix}. For those cases with more components, the analysis becomes more complex. Even when all covariance matrices are the same, it is challenging to derive a concise rule to characterize the relationship between smoothness and the distances between the means of the Gaussians. We give an example where the centers of components are far apart, yet the mixture distribution remains $\+O(1)$-log-smooth, and this smoothness is preserved during the OU process.

An overview of the main results of this section is given in \Cref{tab:result-comp}, where we show the smoothness bounds of $\log p_{\mu}$ and corresponding $\log p_t$ in different cases, as well as whether the $\+O(1)$-smoothness property is preserved during the OU process.

\begin{table*}[htbp]
	\centering
	\caption{The Comparison between the Smoothness Bounds}
	\label{tab:result-comp}
  \begin{threeparttable}
\begin{tabular}{m{1.7cm}<{\centering}m{2.7cm}<{\centering}m{3.9cm}<{\centering}m{3.7cm}<{\centering}m{2cm}<{\centering}}
	\toprule
	& {Parameters \textcolor{red}{\tnote{1}}} & {Smoothness Bound for $\log p_{\mu}$} & {Smoothness Bound for $\log p_t$} & {Keep $\+O(1)$-smooth?}\\
	\midrule
	Stitched Gaussian & \shortstack{$m=2$ \\ $\Sigma_1=\Sigma_2=\!{Id}_d$ \\ $\|u_1-u_2\|^2 = s = \Omega(d)$} & \shortstack{$\+O(1)$\\ (\Cref{lem:stitchsmooth})} & \shortstack{ $\Omega\tp{e^{-2t}s - 1}$ \textcolor{red}{\tnote{2}}
        %$\+O(\max\ab\{1,e^{-2t}s\})$
    \\ for $t>\frac{\log 10}{2}$\\ (\Cref{thm:stitched2})} & No \\
    \hline 
    Mixture of Gaussians & \shortstack{$m=2$\\
    $\Sigma_1=\Sigma_2\succeq \Omega(1)\cdot \!{Id}_d$ \\
    $\|u_1-u_2\|^2=s$} & \shortstack{$\+O(\max\ab\{1,s\})$\\ (\Cref{lem:2-same})} & \shortstack{$\+O(\max\ab\{1,e^{-2t}s\})$\\ (\Cref{lem:2-same})} & Yes \\
    \hline 
    Mixture of Gaussians & \shortstack{$m=2$\\
    $\Sigma_1=\frac{\!{Id}_d}{2},\Sigma_2= \!{Id}_d$ \\
    $\|u_1-u_2\|=o(1)$} & \shortstack{$\Omega(d)$\\ (\Cref{lem:2-diff})} & - & NA \\
    \hline 
    Mixture of Gaussians & \shortstack{$m=2^d$\\
    $\Sigma_1=\cdots=\Sigma_m=J$ \\
    $\|u_i-u_j\|^2= \+O(d)$ \textcolor{red}{\tnote{3}}} & \shortstack{$\+O(1)$\\ (\Cref{lem:mixture1})} & \shortstack{$\+O(1)$\\ (\Cref{cor:mixture2})} & Yes \\
	\bottomrule
\end{tabular}
\begin{tablenotes}
	\footnotesize
    \item[\textcolor{red}{1}] Note that both the stitched Gaussian and the mixture of Gaussians are constructed based on Gaussian distributions with different parameters. We assume the Gaussian distributions used in the construction are $\+N(u_1,\Sigma_1),\+N(u_2,\Sigma_2), \dots, \+N(u_m,\Sigma_m)$ respectively.
    \item[\textcolor{red}{2}] We say the smoothness bound for a function $f$ is $\Omega(c)$ if there exists some $x\in \bb R^d$ such that $\norm{\grad^2 f(x)}_{\!{op}} = \Omega(c)$.
	\item[\textcolor{red}{3}] Here $J$ is a symmetric matrix with $\delta_{\!{Id}_d}\preceq J \preceq (1-\delta)\!{Id}_d$ for some constant $\delta\in (0,1/2)$. For the detailed construction of $\ab\{u_i\}_{i\in[m]}$, see \Cref{eq:HS-mix}.
  \end{tablenotes}
\end{threeparttable}
\end{table*}

\subsection{The stitched Gaussian distributions}\label{subsec:stitched}

In this section,  we show that the smoothness bound of $\log p_t$ can differ significantly from that of $\log p_{\mu}$ on the \emph{stitched Gaussian distributions}.
% we prove that the smoothness bound for $\log p$ and $\log p_t$ can vary significantly when $p$ is a stitched Gaussian distribution. 

The stitched Gaussian distributions are a class of distributions constructed by interpolating between multiple Gaussian components. Recall that $q_{\!{mol}}$ is the mollifier defined in \Cref{sec:mollifier}. Let $u\in \bb R^d$ be a vector satisfying $\|u\|^2 \geq 100 d$. Define $\mathfrak{g}_{u}(x) = q_{\!{mol}}\tp{10\tp{\frac{\|x-u\|}{\|u\|} - 0.4}}$ and 
\[
    f_{\mu}(x) = \mathfrak{g}_{u}(x)\cdot \frac{\|x\|^2}{2} + (1-\mathfrak{g}_{u}(x))\cdot \frac{\|x-u\|^2}{2}.
\]  
The specific type of stitched Gaussians we consider here has a density function of $p_{\mu}\propto e^{-f_{\mu}}$.

% When the information is clear from context, we abbreviate $\mathfrak{g}_{\left[\frac{2\|u\|}{5},\frac{\|u\|}{2}\right]}$ as $\mathfrak{g}$ for simplicity.
Let $r=\|u\|$. We can divide $\bb R^d$ into two parts $\+B_{\frac{r}{2}}(u) = \set{x\in \bb R^d: \| x - u\| \leq 0.5 \|u\|}$ and $\ol{\+B_{\frac{r}{2}}(u)} = \bb R^d \setminus \+B_{\frac{r}{2}}(u)$. By definition of $\mathfrak{g}_{u}$, for $x\in \ol{\+B_{\frac{r}{2}}(u)}$, $f_{\mu}(x) = \frac{\|x\|^2}{2}$. Furthermore, inside $\+B_{\frac{r}{2}}(u)$, when $x\in \+B_{\frac{2r}{5}}(u) =\set{x\in \bb R^d: \| x - u\| \leq 0.4 \|u\|}$, $f_{\mu}(x) = \frac{\|x-u\|^2}{2}$. In $\+B_{\frac{r}{2}}(u)\setminus \+B_{\frac{2r}{5}}(u)$, these two Gaussians are ``stitched together'', which means the density function transitions smoothly from a Gaussian distribution centered at $0$ to another Gaussian distribution centered at $u$.

\subsubsection{The smoothness of $\log p_{\mu}$}\label{subsec:stitched1}

We first prove that $\log p_{\mu}$ is indeed $\+O(1)$-smooth.

\begin{lemma}\label{lem:stitchsmooth}
    The function $\log p_{\mu}$ is $\+O(1)$-smooth for any $u\in \bb R^d$.
\end{lemma}
\begin{proof}
    For $x\in \+B_{\frac{2r}{5}}(u)$, we know that $f_{\mu}(x) = \frac{\|x-u\|^2}{2}$. Therefore $\grad^2 \log p_{\mu}(x) = \grad^2 f_{\mu}(x) = \!{Id}_d$. Similarly, for $x\in \ol{\+B_{\frac{r}{2}}(u)}$, $f_{\mu}(x)=\frac{\|x\|^2}{2}$ and $\grad^2 \log p_{\mu}(x) = \grad^2 f_{\mu}(x) = \!{Id}_d$. So it only remains to deal with those $x\in \+B_{\frac{r}{2}}(u) \setminus \+B_{\frac{2r}{5}}(u)$.

    % Let $\+B_{\frac{r}{2}}(u)\setminus \+B_{\frac{2r}{5}}(u) = \+B_{\frac{r}{2}}(u)\setminus \+B_{\frac{2r}{5}}(u) = \set{x\in \bb R^d:0.4 \|u\| < \| x - u\| \leq 0.5 \|u\|}$.
    For $x\in \+B_{\frac{r}{2}}(u)\setminus \+B_{\frac{2r}{5}}(u)$, 
    \begin{align*}
        \grad f_{\mu}(x) = \mathfrak{g}_{u}(x)\cdot x + \frac{\|x\|^2}{2} \cdot \grad \mathfrak{g}_{u}(x) - \frac{\|x - u\|^2}{2}\cdot \grad \mathfrak{g}_{u}(x)+ (1-\mathfrak{g}_{u}(x))\cdot (x-u)
    \end{align*}
    and consequently,
    \begin{align}
        \grad^2 f_{\mu}(x) &= \underbrace{x\cdot \grad \mathfrak{g}_{u}(x)^{\top}}_{(a_1)} + \underbrace{\grad \mathfrak{g}_{u}(x)\cdot  x^{\top}}_{(b_1)} + \underbrace{\mathfrak{g}_{u}(x)\cdot \!{Id}_d}_{(c_1)} + \underbrace{\frac{\|x\|^2}{2}\cdot \grad^2 \mathfrak{g}_{u}(x)}_{(d_1)} \notag \\
        &\quad - \underbrace{(x-u)\cdot \grad \mathfrak{g}_{u}(x)^{\top}}_{(a_2)} - \underbrace{\grad \mathfrak{g}_{u}(x)\cdot (x-u)^{\top}}_{(b_2)} + \underbrace{(1-\mathfrak{g}_{u}(x))\cdot \!{Id}_d}_{(c_2)} -\underbrace{\frac{\|x - u\|^2}{2}\cdot \grad^2 \mathfrak{g}_{u}(x)}_{(d_2)}. \label{eq:1}
    \end{align}

    By the definition of $\mathfrak{g}_{u}(x)$, we have
     \[
        % \grad \mathfrak{g}_{u}(x) = \frac{10(x-u)}{\|x-u\|\cdot \|u\|}\cdot  \frac{\dd q}{\dd y}\Bigg|_{y=10\tp{\frac{\|x-u\|}{\|u\|}-0.4}}
        \grad \mathfrak{g}_{u}(x) = \frac{10(x-u)}{\|x-u\|\cdot \|u\|}\cdot  q'_{\!{mol}}\tp{10\tp{\frac{\|x-u\|}{\|u\|}-0.4}}
        % \grad h\tp{10\tp{\frac{\|x-2u\|}{\|u\|}-1.5}}
     \]
     and 
     \begin{align*}
        %  \grad^2 \mathfrak{g}_{u}(x) &= \frac{100(x-u)(x-u)^{\top}}{\|x-u\|^2\cdot \|u\|^2}\cdot  \frac{\dd^2 q}{\dd y^2}\Bigg|_{y=10\tp{\frac{\|x-u\|}{\|u\|}-0.4}} \\
        %  &\quad + \frac{10}{\|u\|}\tp{\frac{\!{Id}_d}{\|x-u\|} - \frac{(x-u)(x-u)^{\top}}{\|x-u\|^3}} \cdot  \frac{\dd q}{\dd y}\Bigg|_{y=10\tp{\frac{\|x-u\|}{\|u\|}-0.4}}.
        \grad^2 \mathfrak{g}_{u}(x) &= \frac{100(x-u)(x-u)^{\top}}{\|x-u\|^2\cdot \|u\|^2}\cdot  q''_{\!{mol}}\tp{10\tp{\frac{\|x-u\|}{\|u\|}-0.4}} \\
        &\quad + \frac{10}{\|u\|}\tp{\frac{\!{Id}_d}{\|x-u\|} - \frac{(x-u)(x-u)^{\top}}{\|x-u\|^3}} \cdot  q''_{\!{mol}}\tp{10\tp{\frac{\|x-u\|}{\|u\|}-0.4}}.
     \end{align*}

     Then we calculate the terms in \Cref{eq:1} one by one. We have 
     \[ 
        (a_1) = \frac{10x(x-u)^{\top}}{\|x-u\|\cdot \|u\|}\cdot  q'_{\!{mol}}\tp{10\tp{\frac{\|x-u\|}{\|u\|}-0.4}}.
    \]
    Since $x\in 
    \+B_{\frac{r}{2}}(u)\setminus \+B_{\frac{2r}{5}}(u)$, we have $\|x-u\|=\Theta\tp{\|u\|}$ and $\|x\|=\Theta\tp{\|u\|}$. Recall that $q'_{\!{mol}}=\+O(1)$. So we have $\+O(1)\cdot \!{Id}_d\mge (a_1)\mge -\+O(1)\cdot \!{Id}_d$. We can also prove such bounds for $(a_2), (b_1)$ and $(b_2)$ in the same way.

    For the terms $(c_1)$ and $(c_2)$, we know that $\mathfrak{g}_{u}(x)\in [0,1]$ for all $x\in \+B_{\frac{r}{2}}(u)\setminus \+B_{\frac{2r}{5}}(u)$. Therefore, we have $\+O(1)\cdot \!{Id}_d\mge (c_1)\mge 0$ and $\+O(1)\cdot \!{Id}_d\mge (c_2)\mge 0$.

    For $x\in \+B_{\frac{r}{2}}(u)\setminus \+B_{\frac{2r}{5}}(u)$, $\|x\|^2=\Theta(\|u\|^2)$ and $\|x-u\|^2=\Theta(\|u\|^2)$. Since $q''_{\!{mol}}$ and $q'_{\!{mol}}$ are all $\+O(1)$, we have that $\+O\tp{\frac{1}{\|u\|^2}}\cdot \!{Id}_d\mge \grad^2 \mathfrak{g}_{u}(x) \mge 0$. Therefore, $\+O(1)\cdot \!{Id}_d\mge (d_1)\mge 0$ and $\+O(1)\cdot \!{Id}_d\mge (d_2)\mge 0$.

    Combining all these together, we know that $\log p_{\mu}$ is $\+O(1)$-smooth.
\end{proof}

\subsubsection{The smoothness of $\log p_t$}\label{subsec:stitched2}

We then show that for arbitrary $u\in \bb R^d$ with $\|u\|^2\geq 100d$, when $t$ satisfies $e^{-2t}<0.1$, there exists $x_0\in \bb R^d$ such that $\|\grad^2 \log p_t(x_0)\|_{\!{op}}\geq \Omega\tp{e^{-2t}\|u\|^2-1}$.
% $\log p_t(x_0) \not \mle L\cdot \!{Id}_d$ for any $L=o(\max\ab\{e^{-2t}\|u\|^2,1\})$. 
Combining this with the results in \Cref{subsec:stitched1}, it indicates that the smoothness bound for $\log p_t$ can be much larger than that for $\log p_{\mu}$. In other words, the smoothness of the initial distribution does not necessarily imply the smoothness during the OU process. 

We first see how the distribution evolves during the process. From \Cref{eq:OU}, for any $x\in \bb R^d$,
\[
    p_t(x) \propto \int_{\bb R^d} p\tp{\frac{y}{e^{-t}}} \cdot e^{-\frac{\|x-y\|^2}{2(1-e^{-2t})}} \dd y.
\]
For a fixed $x_0\in \bb R^d$, consider the distribution $\nu_t$ with density $q_t(y)\propto p_{\mu}\tp{\frac{y}{e^{-t}}} \cdot e^{-\frac{\|x_0-y\|^2}{2(1-e^{-2t})}}$ for each $y\in \bb R^d$. To calculate the Hessian of $\log p_t$, we use \Cref{prop:stitched-decomp}.
\begin{proposition}[Corollary of Lemma 22 in \cite{CLL23}]\label{prop:stitched-decomp}
    We have $\grad^2 \log p_t(x_0) = \frac{1}{(1-e^{-2t})^2}\cdot \!{Cov}_{Y\sim \nu_t}[Y] - \frac{\!{Id}_d}{1-e^{-2t}}$.
\end{proposition}

In the remaining part of this section, we aim to prove the following lemma.
\begin{theorem}\label{thm:stitched2}
    For arbitrary $u\in \bb R^d$ with $\|u\|^2\geq 100d$, when $t$ satisfies $e^{-2t}<0.1$ and when $x_0=\frac{e^{-t}u}{2}$, $\|\!{Cov}_{Y\sim \nu_t}[Y]\|_{\!{op}} = \Omega\tp{e^{-2t}\|u\|^2}$ and consequently, $\|\grad^2 \log p_t(x_0)\|_{\!{op}}\geq \Omega\tp{e^{-2t}\|u\|^2-1}$.
    % for any $L=o(\max\ab\{e^{-2t}\|u\|^2,1\})$, $\!{Cov}_{Y\sim \nu_t}[Y] \not\mle L \cdot \!{Id}_d$ and consequently, $\grad^2 \log p_t(x_0)\not \mle L\cdot \!{Id}_d$.
\end{theorem}

We will always assume $x_0=\frac{e^{-t}u}{2}$ in the following analyses. Before calculating $\!{Cov}_{Y\sim \nu_t}[Y]$, we see some basic properties of the distribution $\nu_t$. 

Define $c_t = \frac{d}{2}\log \tp{2\pi \sigma_t^2}$ where $\sigma_t^2 =  e^{-2t} (1-e^{-2t})$. Let $\+N_1$ and $\+N_2$ represent the two Gaussian distributions $\+N\tp{\frac{e^{-3t}u}{2}, \sigma_t^2 \cdot \!{Id}_d}$ and $\+N\tp{e^{-t}\tp{1-\frac{e^{-2t}}{2}}u, \sigma_t^2 \cdot \!{Id}_d}$ respectively. Let $h_1$ and $h_2$ be the potential function of these two Gaussian distributions. That is
\[
    h_1(y) = \frac{\norm{y-\frac{e^{-3t}u}{2}}^2}{2e^{-2t}(1-e^{-2t})} + c_t \quad \mbox{and}\quad h_2(y) = \frac{\norm{y- \tp{e^{-t}\tp{1-\frac{e^{-2t}}{2}}u }}^2}{2e^{-2t}(1-e^{-2t})} + c_t.
\]
Define $f_t:\bb R^d \to \bb R$ as 
\begin{align*}
    f_t(y) &= \mathfrak{g}_{u}\tp{\frac{y}{e^{-t}}}\cdot h_1(y) + \tp{1-\mathfrak{g}_{u}\tp{\frac{y}{e^{-t}}}}\cdot  h_2(y). 
\end{align*}

\begin{lemma}\label{lem:stitched0}
    With $x_0=\frac{e^{-t}u}{2}$, we have $q_t(y) \propto e^{-f_t(y)}$.
\end{lemma}

The proof of \Cref{lem:stitched0} is provided in \Cref{subsec:proof2}. Recall that $\+B_{\frac{r}{2}}(u) = \set{x\in \bb R^d: \| x - u\| \leq 0.5 \|u\|}$ and $\+B_{\frac{2r}{5}}(u) = \set{x\in \bb R^d: \| x - u\| \leq 0.4 \|u\|}$. For a set $S\subseteq \bb R^d$ and a real number $c\neq 0$, let $c\cdot S = \set{x\in \bb R^d: \frac{x}{c}\in S}$ be the scaled set. Therefore, outside the ball $e^{-t}\cdot \+B_{\frac{r}{2}}(u)$, $\mathfrak{g}_{u}\tp{\frac{y}{e^{-t}}}=1$ and $f_t(y)\equiv h_1(y)$. Inside $e^{-t}\cdot \+B_{\frac{r}{2}}(u)$, the potential function $f_t$ is the interpolating of $h_1$ and $h_2$ and when $y\in e^{-t}\cdot \+B_{\frac{2r}{5}}(u)$, $f_t(y)\equiv h_2(y)$. 

We claim that the density $q_t$ can be decomposed into a Gaussian density $e^{-h_1}$ plus some density function $p_{\gamma_t}$ which is only supported on $e^{-t}\cdot \+B_{\frac{r}{2}}(u)$. That is, we can find $\delta_t\in (0,1)$ and a distribution $\gamma_t$ with density $p_{\gamma_t}$ such that for any $y\in \bb R^d$, $ q_t(y) = (1-\delta_t) e^{-h_1(y)} + \delta_t\cdot p_{\gamma_t}(y)$. The existence of such $\delta_t$ and $\gamma_t$ is guaranteed by the properties given in the following two lemmas. The proofs of \Cref{lem:stitched3,lem:normalizing} are given in \Cref{subsec:proof2}.

\begin{lemma}\label{lem:stitched3}
    For any $y\in e^{-t}\cdot \+B_{\frac{r}{2}}(u)$, $e^{-h_1(y)} \leq e^{-f_t(y)}$.
\end{lemma}
Let $Z_t=\int_{\bb R^d} e^{-f_t(y)} \dd y$ be the normalizing factor.
\begin{lemma}\label{lem:normalizing}
    When $e^{-2t}<0.1$ and $\|u\|^2\geq 100d$, we have $1.8\leq Z_t\leq 2$.
\end{lemma}


Furthermore, we can prove that $\delta_t=\Theta(1)$ for any $t$ satisfying $e^{-2t}<0.1$.

\begin{lemma}\label{lem:stitched4}
    We have $0.3 < \delta_t < 0.7$ when $e^{-2t}<0.1$ and $\|u\|^2\geq 100d$.
\end{lemma}

We prove \Cref{lem:stitched4} in \Cref{subsec:proof2}. Equipped with these lemmas, we are now ready to prove \Cref{thm:stitched2}.
\begin{proof}[Proof of \Cref{thm:stitched2}]
    We first decompose $\!{Cov}_{Y\sim \nu_t}[Y]$.
    By definition,
    \begin{align*}
        \!{Cov}_{Y\sim \nu_t}[Y] &= \E[Y\sim \nu_t]{(Y-\E[\nu_t]{Y})(Y-\E[\nu_t]{Y})^{\top}} \\
        &= (1-\delta_t)\cdot \E[Y\sim \+N_1]{(Y-\E[\nu_t]{Y})(Y-\E[\nu_t]{Y})^{\top}} + \delta_t\cdot \E[Y\sim \gamma_t]{(Y-\E[\nu_t]{Y})(Y-\E[\nu_t]{Y})^{\top}} \\
        &= (1-\delta_t)\cdot \E[Y\sim \+N_1]{(Y-\E[\+N_1]{Y})(Y-\E[\+N_1]{Y})^{\top}} \\
        &\quad + (1-\delta_t) \cdot \tp{\E[\+N_1]{Y} - \E[\nu_t]{Y}}\tp{\E[\+N_1]{Y} - \E[\nu_t]{Y}}^{\top} \\
        &\quad  + \delta_t \cdot \E[Y\sim \gamma_t]{(Y-\E[\gamma_t]{Y})(Y-\E[\gamma_t]{Y})^{\top}} + \delta_t \cdot \tp{\E[\gamma_t]{Y} - \E[\nu_t]{Y}}\tp{\E[\gamma_t]{Y} - \E[\nu_t]{Y}}^{\top} \\
        &= (1-\delta_t)\cdot\!{Cov}_{\+N_1}[Y] + (1-\delta_t) \cdot \tp{\E[\+N_1]{Y} - \E[\nu_t]{Y}}\tp{\E[\+N_1]{Y} - \E[\nu_t]{Y}}^{\top}\\
        &\quad + \delta_t\cdot\!{Cov}_{\gamma_t}[Y] + \delta_t \cdot \tp{\E[\gamma_t]{Y} - \E[\nu_t]{Y}}\tp{\E[\gamma_t]{Y} - \E[\nu_t]{Y}}^{\top}.
    \end{align*}
    Since $q_t(y) = (1-\delta_t)\cdot e^{-h_1(y)} + \delta_t \cdot p_{\gamma_t}(y)$, the expectation $\E[\nu_t]{Y} = (1-\delta_t)\cdot \E[\+N_1]{Y} + \delta_t\cdot \E[\gamma_t]{Y}$. Then we have 
    \[
        \E[\nu_t]{Y} - \E[\+N_1]{Y} = \delta_t\cdot \tp{\E[\gamma_t]{Y} - \E[\+N_1]{Y}} 
    \]
    and
    \[
        \E[\nu_t]{Y} - \E[\gamma_t]{Y} = (1-\delta_t)\cdot \tp{\E[\+N_1]{Y} - \E[\gamma_t]{Y}}.
    \]
    Therefore,
    \begin{align*}
        \!{Cov}_{Y\sim \nu_t}[Y] &= (1-\delta_t)\cdot\!{Cov}_{\+N_1}[Y] + \delta_t\cdot\!{Cov}_{\gamma_t}[Y] \\
        &\quad + \tp{\delta_t^2(1-\delta_t) + (1-\delta_t)^2\delta_t} \cdot \tp{\E[\+N_1]{Y} - \E[\gamma_t]{Y}}\tp{\E[\+N_1]{Y} - \E[\gamma_t]{Y}}^{\top} \\
        & = (1-\delta_t)\cdot \!{Cov}_{\+N_1}[Y] + \delta_t\cdot\!{Cov}_{\gamma_t}[Y] + \delta_t(1-\delta_t) \cdot \tp{\E[\gamma_t]{Y} - \frac{e^{-3t}u}{2}}\cdot \tp{\E[\gamma_t]{Y} - \frac{e^{-3t}u}{2}}^\top.
    \end{align*}

    We then show that the matrix $\tp{\E[\gamma_t]{Y} - \frac{e^{-3t}u}{2}}\cdot \tp{\E[\gamma_t]{Y} - \frac{e^{-3t}u}{2}}^\top$ cannot be bouned by $L\cdot \!{Id}_d$ for any $L=o(e^{-2t}\|u\|^2)$. To show this, we only need to prove $\norm{\E[\gamma_t]{Y} - \frac{e^{-3t}u}{2}}^2 = \Omega\tp{e^{-2t}\|u\|^2}$. Recall that $\gamma_t$ is supported only on $ e^{-t}\cdot \+B_{\frac{r}{2}}(u) = \set{y\in \bb R^d: \| y - e^{-t}u\| \leq 0.5 e^{-t} \|u\|}$. For each $y\in e^{-t}\cdot \+B_{\frac{r}{2}}(u)$, with $e^{-2t}<0.1$,
    \[
        \norm{ y- \frac{e^{-3t}u}{2} } \geq \|y\| - \norm{\frac{e^{-3t}u}{2}} \geq \frac{e^{-t}}{2} \|u\| - \frac{e^{-3t}}{2} \|u\| > \frac{e^{-t}}{4} \|u\|.
    \]
    So $\norm{\E[\gamma_t]{Y} - \frac{e^{-3t}u}{2}}^2$ can be lower bounded by 
    \[
        \inf_{y\in  e^{-t}\cdot \+B_{\frac{r}{2}}(u)} \norm{y - \frac{e^{-3t}u}{2}}^2  \geq \frac{e^{-2t}}{16} \|u\|^2 = \Omega\tp{e^{-2t}\|u\|^2}.
    \]
    
    Since $\!{Cov}_{\+N_1}[Y]\mge 0$, $\!{Cov}_{\gamma_t}[Y]\mge 0$ and from \Cref{lem:stitched4}, $\delta_t=\Theta(1)$, this indicates that $\|\!{Cov}_{Y\sim \nu_t}[Y]\|_{\!{op}} = \Omega\tp{e^{-2t}\|u\|^2}$.
    % this indicates that for any $L=o(e^{-2t}\|u\|^2)$, $\!{Cov}_{Y\sim \nu_t}[Y] \not\mle L \cdot \!{Id}_d$. 
    The remaining of the lemma then follows from \Cref{prop:stitched-decomp}.
\end{proof}

\subsubsection{Proofs for the supporting lemmas in \Cref{subsec:stitched2}}\label{subsec:proof2}
\begin{proof}[Proof of \Cref{lem:stitched0}]
    By definition, for any $y\in \bb R^d$,
    \begin{align}
        q_t(y) &\propto \exp\set{-\mathfrak{g}_{u}\tp{\frac{y}{e^{-t}}}\cdot \tp{\frac{\|y\|^2}{2e^{-2t}} +\frac{\|y-x_0\|^2}{2(1-e^{-2t})}} - \tp{1-\mathfrak{g}_{u}\tp{\frac{y}{e^{-t}}}}\cdot \tp{\frac{\|y-e^{-t}\cdot u\|^2}{2e^{-2t}} + \frac{\|y-x_0\|^2}{2(1-e^{-2t})}} } \notag \\
        &= \exp\left\{-\mathfrak{g}_{u}\tp{\frac{y}{e^{-t}}}\cdot \frac{\|y\|^2 + e^{-2t}\|x_0\|^2 - e^{-2t}(y^{\top}x_0 + x_0^{\top}y)}{2e^{-2t}(1-e^{-2t})} \right. \notag \\
        & \quad  \left. - \tp{1-\mathfrak{g}_{u}\tp{\frac{y}{e^{-t}}}}\cdot \frac{\|y\|^2 - (1-e^{-2t})e^{-t}(u^{\top}y + y^{\top}u) - e^{-2t}(x_0^{\top}y + y^{\top}x_0) + (1-e^{-2t})e^{-2t}\|u\|^2 + e^{-2t}\|x_0\|^2}{2e^{-2t}(1-e^{-2t})} \right\} \notag  \\
        &= \exp\left\{-\mathfrak{g}_{u}\tp{\frac{y}{e^{-t}}}\cdot \tp{\frac{\|y-e^{-2t}x_0\|^2}{2e^{-2t}(1-e^{-2t})} + \frac{\|x_0\|^2}{2}} \right. \notag \\
        & \quad \left.- \tp{1-\mathfrak{g}_{u}\tp{\frac{y}{e^{-t}}}}\cdot \tp{\frac{\|y- \tp{e^{-t}(1-e^{-2t})u + e^{-2t}x_0}\|^2}{2e^{-2t}(1-e^{-2t})} + \frac{\|x_0 - e^{-t}u\|^2}{2}}\right\}.\label{eq:stitched1}
    \end{align}
    When we choose $x_0 = \frac{e^{-t}u}{2}$, we can further simplify \Cref{eq:stitched1} as
    \begin{align*}
        q_t(y) & \propto \exp\left\{-\mathfrak{g}_{u}\tp{\frac{y}{e^{-t}}}\cdot \tp{\frac{\norm{y-\frac{e^{-3t}u}{2}}^2}{2e^{-2t}(1-e^{-2t})} + \frac{\|e^{-t}u\|^2}{8}} \right. \notag \\
        & \quad \left.- \tp{1-\mathfrak{g}_{u}\tp{\frac{y}{e^{-t}}}}\cdot \tp{\frac{\norm{y- \tp{e^{-t}\tp{1-\frac{e^{-2t}}{2}}u }}^2}{2e^{-2t}(1-e^{-2t})} + \frac{\|e^{-t}u\|^2}{8}}\right\} \notag  \\
        &\propto \exp\left\{-\mathfrak{g}_{u}\tp{\frac{y}{e^{-t}}}\cdot \frac{\norm{y-\frac{e^{-3t}u}{2}}^2}{2e^{-2t}(1-e^{-2t})}  - \tp{1-\mathfrak{g}_{u}\tp{\frac{y}{e^{-t}}}}\cdot \frac{\norm{y- \tp{e^{-t}\tp{1-\frac{e^{-2t}}{2}}u }}^2}{2e^{-2t}(1-e^{-2t})} \right\}  \notag \\
        &\propto \exp\left\{-\mathfrak{g}_{u}\tp{\frac{y}{e^{-t}}}\cdot \tp{\frac{\norm{y-\frac{e^{-3t}u}{2}}^2}{2e^{-2t}(1-e^{-2t})} + c_t} - \tp{1-\mathfrak{g}_{u}\tp{\frac{y}{e^{-t}}}}\cdot \tp{\frac{\norm{y- \tp{e^{-t}\tp{1-\frac{e^{-2t}}{2}}u }}^2}{2e^{-2t}(1-e^{-2t})} + c_t}\right\}
    \end{align*}
\end{proof}


\begin{proof}[Proof of \Cref{lem:stitched3}]
    Recall that for any $y\in \bb R^d$,
    \[
    h_1(y) = \frac{\norm{y-\frac{e^{-3t}u}{2}}^2}{2e^{-2t}(1-e^{-2t})} + c_t \quad \mbox{and}\quad h_2(y) = \frac{\norm{y- \tp{e^{-t}\tp{1-\frac{e^{-2t}}{2}}u }}^2}{2e^{-2t}(1-e^{-2t})} + c_t.
    \]
    To prove this lemma, we need to show that
    \[
        \norm{y-\frac{e^{-3t}u}{2}}^2 \geq \norm{y- \tp{e^{-t}\tp{1-\frac{e^{-2t}}{2}}u }}^2
    \]
    for each $y\in e^{-t}\cdot \+B_{\frac{r}{2}}(u)$. This is equivalent to say
    \begin{equation*}
        \inner{\tp{2e^{-t} - 2e^{-3t}}u}{y} \geq e^{-2t}(1-e^{-2t}) \|u\|^2,
    \end{equation*}
    and this can be further simplified to $2\inner{u}{y} \geq e^{-t}\|u\|^2$.

    Since for $y\in e^{-t}\cdot \+B_{\frac{r}{2}}(u)$, $y$ satisfies $\| y - e^{-t}u\| \leq 0.5 e^{-t} \|u\|$. Therefore, for each $y\in e^{-t}\cdot\+B_{\frac{r}{2}}(u)$, we have
    \begin{equation*}
        \|y\| \geq 0.5 e^{-t}\|u\| \quad \mbox{and} \quad 2e^{-t}\inner{u}{y} \geq \|y\|^2 + 0.75 e^{-2t} \|u\|^2.
    \end{equation*}
    This indicates that $2\inner{u}{y} \geq e^{-t}\|u\|^2$.
\end{proof}


\begin{proof}[Proof of \Cref{lem:normalizing}]
    On the one hand,
    \begin{align*}
        Z_t \leq \int_{\bb R^d} e^{-h_1(y)} \dd y + \int_{\bb R^d} e^{-h_2(y)} \dd y \leq 2.
    \end{align*}
    On the other hand,
    \begin{align*}
        Z_t &\geq \int_{e^{-t}\cdot \+B_{\frac{2r}{5}}(u)} e^{-h_2(y)} \dd y + \int_{e^{-t}\cdot \ol{\+B_{\frac{r}{2}}(u)}} e^{-h_1(y)} \dd y \\
        & = \Pr[Y\sim \+N_2]{Y\in e^{-t}\cdot \+B_{\frac{2r}{5}}(u)} + \Pr[Y\sim \+N_1]{Y\in e^{-t}\cdot \ol{\+B_{\frac{r}{2}}(u)}}\\
        &\geq 2- \frac{20 d}{\|u\|^2} >1.8.
    \end{align*}
    where in the second inequality we use \Cref{prop:stitched0} and the last inequality is due to $\|u\|^2\geq 100 d$.
\end{proof}

Before proving \Cref{lem:stitched4}, we first give the following concentration bounds for $\+N_1$ and $\+N_2$. Recall that $\+N_1$ and $\+N_2$ represent the two Gaussian distributions $\+N\tp{\frac{e^{-3t}u}{2}, \sigma_t^2 \cdot \!{Id}_d}$ and $\+N\tp{e^{-t}\tp{1-\frac{e^{-2t}}{2}}u, \sigma_t^2 \cdot \!{Id}_d}$ respectively. 
\begin{proposition}\label{prop:stitched0}
    When $e^{-2t}<0.1$, we have $\Pr[Y\sim \+N_1]{Y\in e^{-t}\cdot \ol{\+B_{\frac{r}{2}}(u)}} \geq 1 - \frac{10d}{\|u\|^2}$ and $\Pr[Y\sim \+N_2]{Y\in e^{-t}\cdot \+B_{\frac{2r}{5}}(u)} \geq 1- \frac{10d}{\|u\|^2}$.
\end{proposition}
\begin{proof}
    We first prove $\Pr[Y\sim \+N_1]{Y\in e^{-t}\cdot \ol{\+B_{\frac{r}{2}}(u)}} \geq 1 - \frac{10d}{\|u\|^2}$. By definition, 
    \[
        e^{-t}\cdot \ol{\+B_{\frac{r}{2}}(u)} = \set{y\in \bb R^d: \|y - e^{-t}u\| > 0.5e^{-t}\|u\|}.
    \]
    Since 
    \[
        \|y - e^{-t}u\| \geq \tp{e^{-t} - \frac{e^{-3t}}{2}} \|u\| - \norm{y - \frac{e^{-3t}}{2} u },
    \]
    we have $\set{y\in \bb R^d:\ \norm{y - \frac{e^{-3t}}{2} u } < \tp{0.5 e^{-t} - \frac{e^{-3t}}{2}}\|u\|} \subseteq e^{-t}\cdot \ol{\+B_{\frac{r}{2}}(u)}$. Therefore, 
    \begin{align*}
        \Pr[Y\sim \+N_1]{Y\in e^{-t}\cdot \ol{\+B_{\frac{r}{2}}(u)}} &\geq \Pr[Y\sim \+N_1]{\norm{Y - \frac{e^{-3t}}{2} u }^2 < \tp{0.5 e^{-t} - \frac{e^{-3t}}{2}}^2\|u\|^2}  \\
        &\geq 1 - \frac{d\cdot \sigma_t^2}{\tp{0.5 e^{-t} - \frac{e^{-3t}}{2}}^2\|u\|^2} \\
        &\geq 1 - \frac{10d}{\|u\|^2}
    \end{align*}
    where the second inequality follows from the Markov's inequality and the last inequality is due to $e^{-2t}< 0.1$.

    We then prove that $\Pr[Y\sim \+N_2]{Y\in e^{-t}\cdot \+B_{\frac{2r}{5}}(u)} \geq 1- \frac{10d}{\|u\|^2}$. By definition, 
    \[
        e^{-t}\cdot \+B_{\frac{2r}{5}}(u) = \set{y\in \bb R^d: \|y - e^{-t}u\| < 0.4e^{-t}\|u\|}.
    \]
    Since
    \[
        \|y - e^{-t}u\| \leq \norm{y -  e^{-t}\tp{1-\frac{e^{-2t}}{2}}u} + \frac{e^{-3t}}{2}\|u\|,
    \]
    we have $\set{y\in \bb R^d :\ \norm{ y -  e^{-t}\tp{1-\frac{e^{-2t}}{2}}u}\leq\tp{0.4 e^{-t} - \frac{e^{-3t}}{2}}\|u\|} \subseteq e^{-t}\cdot \+B_{\frac{2r}{5}}(u)$. Similarly,
    \begin{align*}
        \Pr[Y\sim \+N_2]{Y\in e^{-t}\cdot \+B_{\frac{2r}{5}}(u)} &\geq \Pr[Y\sim \+N_2]{\norm{Y -  e^{-t}\tp{1-\frac{e^{-2t}}{2}}}^2 \leq\tp{0.4 e^{-t} - \frac{e^{-3t}}{2}}^2\|u\|^2}  \\
        &\geq 1- \frac{d\cdot \sigma_t^2}{\tp{0.4 e^{-t} - \frac{e^{-3t}}{2}}^2\|u\|^2} \\
        &\geq 1- \frac{10d}{\|u\|^2}.
    \end{align*}
\end{proof}

Now we give a proof of \Cref{lem:stitched4}.
\begin{proof}[Proof of \Cref{lem:stitched4}]
    Recall that the distribution $\gamma_t$ is only supported on $e^{-t}\cdot \+B_{\frac{r}{2}}(u)$. The choice of $\delta_t$ satisfies
    \begin{equation}
        \Pr[Y\sim \nu_t]{Y\in e^{-t}\cdot \+B_{\frac{r}{2}}(u)} = (1-\delta_t)\cdot \Pr[Y\sim N_1]{Y\in e^{-t}\cdot \+B_{\frac{r}{2}}(u)} + \delta_t. \label{eq:decomp}
    \end{equation}
    From \Cref{prop:stitched0} and \Cref{lem:normalizing}, we have
    \[
        \Pr[Y\sim \nu_t]{Y\in e^{-t}\cdot \+B_{\frac{r}{2}}(u)} \geq \frac{1}{Z_t}\cdot \Pr[Y\sim \+N_2]{Y\in e^{-t}\cdot \+B_{\frac{2r}{5}}(u)} \geq \frac{1-\frac{10d}{\|u\|^2}}{2} >0.4,
    \]
    and 
    \[
        \Pr[Y\sim \nu_t]{Y\in e^{-t}\cdot \+B_{\frac{r}{2}}(u)} \leq \frac{1}{Z_t}\cdot\tp{\Pr[Y\sim \+N_2]{Y\in e^{-t}\cdot \+B_{\frac{r}{2}}(u)} + \Pr[Y\sim \+N_1]{Y\in e^{-t}\cdot \+B_{\frac{r}{2}}(u)}} \leq \frac{1}{Z_t}\cdot \tp{1 + \frac{10d}{\|u\|^2}} < 0.7.
    \]
    
    From \Cref{prop:stitched0}, $\Pr[Y\sim N_1]{Y\in e^{-t}\cdot \+B_{\frac{r}{2}}(u)} \leq \frac{10d}{\|u\|^2} \leq 0.1$. So the RHS of \Cref{eq:decomp} satisfies
    \[
        \delta_t \leq (1-\delta_t)\cdot \Pr[Y\sim N_1]{Y\in e^{-t}\cdot \+B_{\frac{r}{2}}(u)} + \delta_t \leq 0.1 + 0.9 \delta_t.
    \]
    Combining above inequalities, we have $0.3< \delta_t < 0.7$.
\end{proof}

\subsection{Smoothness for the mixture of Gaussian distributions}\label{subsec:mix}

The analysis in the previous section indicates that
% that the smoothness parameter can become larger during the OU process compared to the stitched Gaussian distribution at the start. 
an $\+O(1)$-log-smooth initial distribution does not guarantee $\+O(1)$-smoothness after the process evolves. 
% Therefore, it is worth  investigating what kind of initial distributions can preserve such desirable smoothness properties.
In this subsection, we consider a family of classic multi-modal distributions, the mixture of Gaussian distributions. Mixture of Gaussians appear to be similar to stitched Gaussians, but the analysis below will reveal fundamental differences in their smoothness behaviors. 

We study the cases where each Gaussian component has a covariance $\Sigma_i \succeq \Omega(1)\!{Id}_d$. 
% Let $p$ and $p_t$ denote the density functions of the initial distribution and the distribution at time $t$ during the OU process respectively. 
We first considered the simple case with only two components, and then extend the analysis to mixtures with multiple components.

% To be specific, we show that for a mixture of two Gaussians with mean $u_1$ and $u_2$, if their covariance matrices are the same, the distributions at the beginning and during the process will always be $\+O\tp{\max\ab\{e^{-2t}\|u_1-u_2\|^2,1\}}$-smooth. In contrast, when the covariance matrices differ, even with $\|u_1-u_2\|$ being a constant, $\log p$ is not $L$-smooth for any $L=o(d)$. 

% However, for those cases with more components, the analysis becomes more complex. Even when all covariance matrices are the same, it is challenging to derive a concise rule to characterize the relationship between smoothness and the distances between the means of the Gaussians. We give an example where the centers of components are far apart, yet the mixture distribution remains $\+O(1)$-smooth, and this smoothness is preserved during the OU process.

\subsubsection{Mixture of Gaussian distributions and its evolution during the OU process}

Consider $m$ Gaussian distributions over $\bb R^d$, each with mean $u_i\in \bb R^d$ and covariance $\Sigma_i\in \bb R^{d\times d}$. Define function $f_i:\bb R^d\to \bb R$ as $f_i(x) = \frac{1}{2}(x-u_i)^{\top}\Sigma_i^{-1}(x-u_i) + \frac{1}{2}\log\tp{\tp{2\pi}^{d} \abs{\Sigma_i}}$ for each $x\in \bb R^d$. Then $e^{-f_i}$ is the density function of the Gaussian distribution $\+N\tp{u_i,\Sigma_i}$. Suppose $\mu$ is the mixture of Gaussian with density $p_{\mu}(x) = \sum_{i=1}^m w_i e^{-f_i(x)}$, where $w_i\in(0,1)$ is the weight of the $i$-th component and $\sum_{i=1}^m w_i=1$. Then $-\grad \log p_{\mu}(x) = \frac{\sum_{i=1}^m w_i\grad f_i(x)\cdot e^{-f_i(x)}}{p_{\mu}(x)}$ and 
\begin{align}
    -\grad^2 \log p_{\mu}(x) &= \frac{\sum_{i=1}^m w_i\grad^2 f_i(x)\cdot e^{-f_i(x)} - \sum_{i=1}^m w_i\grad f_i(x)\grad f_i(x)^{\top}\cdot e^{-f_i(x)}}{\sum_{i=1}^m w_i e^{-f_i(x)}} \notag  \\
    &\quad + \frac{\tp{\sum_{i=1}^m w_i\grad f_i(x)\cdot e^{-f_i(x)}}\tp{\sum_{i=1}^m w_i\grad f_i(x)\cdot e^{-f_i(x)}}^{\top}}{\tp{\sum_{i=1}^m w_i e^{-f_i(x)}}^2} \notag \\
    &= \frac{\sum_{i=1}^m\sum_{j=1}^m w_iw_j \tp{\grad f_i(x) \grad f_j(x)^{\top} - \frac{1}{2}\grad f_i(x) \grad f_i(x)^{\top} - \frac{1}{2}\grad f_j(x) \grad f_j(x)^{\top}} e^{-f_i(x)-f_j(x)}}{\tp{\sum_{i=1}^m w_i e^{-f_i(x)}}^2} \notag \\
    &\quad + \frac{\sum_{i=1}^m w_i\grad^2 f_i(x)\cdot e^{-f_i(x)} }{\sum_{i=1}^m w_i e^{-f_i(x)}} \notag \\
    &= - \underbrace{\frac{\sum_{1\leq i<j\leq m} w_iw_j \tp{\grad f_i(x) - \grad f_j(x)}\tp{\grad f_i(x) - \grad f_j(x)}^{\top} e^{-f_i(x)-f_j(x)}}{\tp{\sum_{i=1}^m w_i e^{-f_i(x)}}^2}}_{(A)} \notag \\
    &\quad + \underbrace{\frac{\sum_{i=1}^m w_i\grad^2 f_i(x)\cdot e^{-f_i(x)} }{\sum_{i=1}^m w_i e^{-f_i(x)}}}_{(B)}. \label{eq:smooth1}
\end{align}
From \Cref{eq:smooth1}, $\grad^2 \log p_{\mu}$ is determined by two parts: the weighted mixture of the Hessian of each component (term $(B)$), and the interaction between different components (term $(A)$).

Then we see how this distribution evolves during the OU process. Recall that the trajectory of the OU process is given by $X_t = e^{-t}X_0 + \sqrt{2}\cdot e^{-t}\int_{0}^t e^s \d B_s$ and $
\sqrt{2}\cdot e^{-t}\int_{0}^t e^s \d B_s \sim \+N\tp{0, (1-e^{-2t})I_d}$. If $X_0$ is drawn from some Gaussian distribution $\+N(u,\Sigma)$, at time $t$, the distribution of $X_t$ will be $\+N\tp{e^{-t}u, e^{-2t}\Sigma + (1-e^{-2t})\!{Id}_d}$. Hence, if the initial distribution is $\mu$, i.e., the weighted mixture of $\ab\{\+N(u_i, \Sigma_i)\}_{i\in[m]}$, then the distribution of $X_t$ will be the mixture of $\ab\{\+N\tp{e^{-t}u_i, e^{-2t}\Sigma_i + (1-e^{-2t})\!{Id}_d} \}_{i\in[m]}$ with the same weights. Let $\Sigma_i^{(t)} =  e^{-2t}\Sigma_i + (1-e^{-2t})\!{Id}_d$. That is to say, this distribution is still a mixture of Gaussians and has density $p_t(x) = \sum_{i=1}^n w_i e^{-f^{(t)}_i(x)}$, where $e^{-f^{(t)}_i(x)}$ is the density function of $\+N\tp{e^{-t}u_i, \Sigma_i^{(t)}}$.
 


\subsubsection{Mixture of two Gaussians}\label{subsubsec:2gaussian}
We first see the case when $m=2$. 

\paragraph{Gaussians with the same covariance: distance of means determines}~

When the covariances are the same and are bounded, the rules are simple and straightforward. The smoothness of the mixture distribution is totally determined by the distance of centers.

\begin{lemma}\label{lem:2-same}
    When $m=2$ and $\Sigma_1 = \Sigma_2 = \Sigma$ for some matrix $\Sigma\succeq \Omega(1)\!{Id}_d$, we have
    \[
        -\+O\tp{\|u_1-u_2\|^2}\cdot \!{Id}_d \preceq -\grad^2 \log p_{\mu}(x) \preceq \Sigma^{-1}, 
    \]
    and 
    \[
         -\+O\tp{e^{-2t}\|u_1-u_2\|^2}\cdot \!{Id}_d \preceq -\grad^2 \log p_t(x) \preceq \+O(1)\!{Id}_d
    \]
    for any $t>0$.
    
    On the other hand, for any $L=o(\|u_1-u_2\|^2)$, $-\grad^2 \log p(x) \not\succeq -L\cdot \!{Id}_d$ and $-\grad^2 \log p_t(x) \not\succeq -e^{-2t}L\cdot \!{Id}_d$.
\end{lemma}
\begin{proof}
    When $m=2$ and $\Sigma_1 = \Sigma_2 = \Sigma$ for some matrix $\Sigma\in \bb R^{d\times d}$, we have $\grad f_1(x) = \Sigma^{-1}(x-u_1)$, $\grad f_2(x) = \Sigma^{-1}(x-u_2)$ and $\grad^2 f_1(x) = \grad^2 f_2(x) = \Sigma^{-1}$.
According to \Cref{eq:smooth1}, 
\begin{align*}
    -\grad^2 \log p_{\mu}(x) = - \frac{w_1w_2\cdot \Sigma^{-1}(u_1-u_2)(u_1-u_2)^{\top}\Sigma^{-1}\cdot e^{-f_1(x) - f_2(x)}}{\tp{w_1 e^{-f_1(x)} + w_2 e^{-f_2(x)}}^2} + \Sigma^{-1}.
\end{align*}
From the Cauchy-Schwartz inequality,
\[
    0 \leq \frac{w_1w_2\cdot e^{-f_1(x) - f_2(x)}}{\tp{w_1 e^{-f_1(x)} + w_2 e^{-f_2(x)}}^2} \leq \frac{1}{4}.
\]
Therefore
\[
    -\frac{1}{4}\Sigma^{-1}(u_1-u_2)(u_1-u_2)^{\top}\Sigma^{-1} + \Sigma^{-1} \preceq -\grad^2 \log p_{\mu}(x) \preceq \Sigma^{-1}.
\]
It can be easily prove that $\Sigma^{-2}$ is also upper bounded by $\+O(1)\!{Id}_d$. The result then follows from the fact that 
\[
    \Sigma^{-1}(u_1-u_2)(u_1-u_2)^{\top}\Sigma^{-1} \preceq \|\Sigma^{-1}(u_1-u_2)\|^2 \cdot \!{Id}_d = (u_1-u_2)^{\top}\Sigma^{-2}(u_1-u_2)\cdot \!{Id}_d \preceq \+O(\|u_1-u_2\|^2)\cdot \!{Id}_d.
\]

% This indicates that the potential function of the initial distribution is at least $\+O\tp{\|u_1-u_2\|^2}$-smooth (?).

Let $\Sigma^{(t)} = e^{-2t}\Sigma + (1-e^{-2t})\!{Id}_d$. For the distributions during the OU process, repeating the above calculations, we can get that
\[
     -\frac{e^{-2t}}{4}\tp{\Sigma^{(t)}}^{-1}(u_1-u_2)(u_1-u_2)^{\top}\tp{\Sigma^{(t)}}^{-1} + \tp{\Sigma^{(t)}}^{-1} \preceq -\grad^2 \log p_t(x) \preceq \tp{\Sigma^{(t)}}^{-1}.
\]
By the definition of $\Sigma^{(t)}$, $ \tp{\Sigma^{(t)}}^{-1} \preceq \frac{1}{1-ce^{-2t}}\cdot \!{Id}_d \preceq \+O(1) \!{Id}_d$ for some universal constant $c<1$. 
Therefore, at time $t$, $-\grad^2 \log p_t \succeq -\+O( \|u_1-u_2\|^2) \!{Id}_d$.
\end{proof}


\paragraph{Gaussians with different covariances}~

When the two components have different covariance matrices, the smoothness parameter can be $\+O(d)$ even when $\|u_1-u_2\|$ is small. Here is an example.

\begin{lemma}\label{lem:2-diff}
    Let $p_{\mu}(x) = \frac{1}{2}e^{-f_1(x)} + \frac{1}{2}e^{-f_2(x)}$ with $f_1(x) = \|x-u_1\|^2 + \frac{d}{2}\log \pi$ and $f_2(x) = \frac{\|x-u_2\|^2}{2} + \frac{d}{2}\log 2\pi$ for arbitrary vectors $u_1,u_2\in \bb R^d$. Then there exists $x\in \bb R^d$ such that $\|\grad^2 \log p_{\mu}(x)\|_{\!{op}} = \Omega{d\log 2 + 2\|u_1-u_2\|^2}$.
    % $\grad^2 \log p_{\mu}(x) \not\preceq L \cdot \!{Id}_d$ for any $L< d\log 2 + 2\|u_1-u_2\|^2$.
\end{lemma}
\begin{proof}
    From \Cref{eq:smooth1}, 
    \begin{align*}
        -\grad^2 \log p_{\mu}(x) = \underbrace{\frac{2e^{-f_1(x)}\cdot \!{Id}_d + e^{-f_2(x)}\cdot \!{Id}_d}{e^{-f_1(x)} + e^{-f_2(x)}}}_{(a)} - \underbrace{\frac{e^{-f_1(x)-f_2(x)}\cdot \tp{x-(2u_1-u_2)}\tp{x-(2u_1-u_2)}^{\top}}{\tp{e^{-f_1(x)} + e^{-f_2(x)}}^2}}_{(b)}.
    \end{align*}
    For $(a)$, it is easy to know that $\!{Id}_d\preceq (a) \preceq 2\!{Id}_d$. For $(b)$, we will find a specific $x$ such that $(b)$ is large.
    When $x$ satisfies $\|x - (2u_1-u_2)\|^2 = d\log 2 + 2\|u_1-u_2\|^2$, by direct calculation, we have 
    \begin{align*}
        f_1(x) - f_2(x) &= \frac{1}{2}\tp{\|x\|^2 - (2u_1-u_2)^{\top}x - x^{\top}(2u_1-u_2) + 2\|u_1\|^2 - \|u_2\|^2 - d\log 2} \\
        &= \frac{1}{2}\tp{\|x - (2u_1-u_2)\|^2 - 4\|u_1\|^2 - \|u_2\|^2 + 2u_1^{\top}u_2 + 2u_2^{\top}u_1 + 2\|u_1\|^2 - \|u_2\|^2 - d\log 2}  \\
        &= \frac{1}{2}\tp{\|x - (2u_1-u_2)\|^2 - d\log 2 - 2\|u_1-u_2\|^2} \\
        &= 0.
    \end{align*}
    Therefore, $0\preceq (b) = \frac{1}{4}\cdot \tp{x-(2u_1-u_2)}\tp{x-(2u_1-u_2)}^{\top}$ and the optimal upper bound for $(b)$ is $\frac{\|x-(2u_1-u_2)\|^2}{4}\cdot \!{Id}_d = \tp{d\log 2 + 2\|u_1-u_2\|^2}\cdot \!{Id}_d$.
\end{proof}

% These results also holds whenever the number of components is a constant.

\subsubsection{Mixture of multiple Gaussians}
Things are more complicated and subtle for the mixture of multiple Gaussian distributions, even when their covariance matrices are the same. When they have the same covariance matrix, via similar arguments in \Cref{lem:2-same}, we can prove that $ -\+O\tp{\max_{i,j\in[m]}\|u_i-u_j\|^2}\cdot \!{Id}_d \preceq-\grad^2 \log p_{\mu}(x) \preceq \+O(1)\!{Id}_d$ and $ -\+O\tp{\max_{i,j\in[m]}e^{-2t}\|u_i-u_j\|^2}\cdot \!{Id}_d \preceq-\grad^2 \log p_t(x) \preceq \+O(1)\!{Id}_d$. However, these bounds may not be tight. A large distance between the centers of the components does not necessarily imply a lack of smoothness. The potential function may still be and maintain $\+O(1)$-smooth during the OU process in this case. We give an example in this section.

Let $J\in \bb R^{d\times d}$ be a symmetric and positive definite matrix and $h$ be an arbitrary vector in $\bb R^d$. Consider the distribution $\mu$ over $\bb R^d$ with density
\begin{equation}
    p_{\mu}(x) \propto \sum_{\sigma\in \ab\{\pm 1\}^d} \exp\set{ - \frac{1}{2}x^{\top}J^{-1}x + \tp{J^{-1}h+\sigma}^{\top}x}. \label{eq:HS-mix}
\end{equation}

This distribution is induced when applying the Hubbard-Stratonovich transform to the Ising model (see Appendix E in \cite{KLR22}). 
% The Ising model with interaction matrix $J$ is a distribution over $\ab\{\pm 1\}^d$ with density $p_J(\sigma)\propto \exp\set{\inner{\sigma}{J\sigma}}$ for any $\sigma\in \ab\{\pm 1\}^d$. 
Note that each $\sigma\in  \ab\{\pm 1\}^d$ corresponds to a Gaussian component $\+N(J\sigma+h, J^{-1})$. For a vector $x\in \bb R^d$, let $x(i)$ denote its $i$-th component for any $i\in[d]$. The following lemma shows that this distribution is log-smooth and even strongly log-concave if $J$ is within a moderate range.

\begin{lemma}\label{lem:mixture1}
    If $\delta\cdot \!{Id}_d\preceq J\preceq (1-\delta)\cdot \!{Id}_d$ for some $\delta\in (0,1/2)$, the distribution defined in \Cref{eq:HS-mix} satisfies $\frac{\delta}{1-\delta}\cdot \!{Id}_d \preceq -\grad^2 \log p_{\mu}(x) \preceq \frac{1}{\delta}\cdot \!{Id}_d$ for any $x\in \bb R^d$.
\end{lemma}
\begin{proof}
    By the definition in \Cref{eq:HS-mix}, 
    \begin{align*}
        p_{\mu}(x) &\propto \sum_{\sigma\in \ab\{\pm 1\}^d} \exp\set{ - \frac{1}{2}x^{\top}J^{-1}x + \tp{J^{-1}h+\sigma}^{\top}x} \\
        &= \exp\set{ - \frac{1}{2}x^{\top}J^{-1}x + h^{\top}J^{-1}x} \cdot \sum_{\sigma\in \ab\{\pm 1\}^d} \exp\set{\sigma^{\top}x} \\
        &= \exp\set{ - \frac{1}{2}x^{\top}J^{-1}x + h^{\top}J^{-1}x} \cdot \prod_{i=1}^d \tp{e^{x(i)}+e^{-x(i)}}.
    \end{align*}
    Therefore, $-\grad \log p_{\mu}(x) = J^{-1}x - J^{-1}h - z_x$ where $z_x\in \bb R^d$ and $z_x(i) = \frac{e^{x(i)} - e^{-x(i)}}{e^{x(i)} + e^{-x(i)}}$ for each $i\in[d]$. Consequently, $-\grad^2 \log p_{\mu}(x) = J^{-1} - A_x$ where $A_x$ is a diagonal matrix in $\bb R^{d\times d}$ and $A_x(i,i) = 1 - \tp{\frac{e^{x(i)} - e^{-x(i)}}{e^{x(i)} + e^{-x(i)}}}^2$ for each $i\in[d]$.

    Since $\delta\cdot \!{Id}_d\preceq J\preceq (1-\delta)\cdot \!{Id}_d$, for any $v\in \bb R^d$,
    \[
        v^{\top}J^{-1}v = \tp{J^{-\frac{1}{2}}v}^{\top}J^{-\frac{1}{2}}v \succeq \frac{1}{1-\delta}v^{\top}\cdot J^{-\frac{1}{2}} J J^{-\frac{1}{2}}\cdot v=\frac{1}{1-\delta}v^{\top}v,
    \]
    and 
    \[
        v^{\top}J^{-1}v = \tp{J^{-\frac{1}{2}}v}^{\top}J^{-\frac{1}{2}}v \preceq \frac{1}{\delta}v^{\top}\cdot J^{-\frac{1}{2}} J J^{-\frac{1}{2}}\cdot v = \frac{1}{\delta}v^{\top}v.
    \]
    % \[
    %     \frac{3}{2}v^{\top}v = \frac{3}{2}v^{\top}\cdot J^{-\frac{1}{2}} J J^{-\frac{1}{2}}\cdot v \preceq \tp{J^{-\frac{1}{2}}v}^{\top}J^{-\frac{1}{2}}v \preceq 3v^{\top}\cdot J^{-\frac{1}{2}} J J^{-\frac{1}{2}}\cdot v = 3v^{\top}v.
    % \]
    Thus $\frac{1}{1-\delta}\cdot \!{Id}_d\preceq J^{-1}\preceq \frac{1}{\delta}\cdot \!{Id}_d$. For the matrix $A_x$, we know $0\preceq A_x\preceq \!{Id}_d$. Combining these together, we can get the desired result.

\end{proof}

% \htodo{Define the notation $x(i)$ and $p*q$.}

For two distributions $\pi$ and $\nu$ with density $p_\pi$ and $p_{\nu}$ respectively, define $\pi*\nu$ as the distribution with density $p_{\pi*\nu}(x)\propto \int_{\bb R^d} p_{\pi}(y)\cdot p_{\nu}(x-y)\dd y$.
When the initial distribution is both strongly log-concave and log-smooth, we can show that the $-\grad^2 \log p_t$ is also bounded via the following lemma and its corollary.
\begin{lemma}[Lemma 28 in \cite{LPSR21}]\label{lem:m-gaussian1}
      Suppose $\pi$ is a probability density function on $\bb R^d$ such that $M_{1}^{-1} \preceq -\grad^2 \log p_{\pi}(x) \preceq M_2^{-1}$ for some $M_1,M_2\in \bb R^{d\times d}$. Let $\nu$ be the density function of $\+N(0,M)$. Then
      \[
        (M_1+M)^{-1} \preceq  -\grad^2 \log p_{\pi*\nu}(x) \preceq (M_2+M)^{-1}.
      \]
\end{lemma}

\begin{corollary}\label{cor:mixture2}
    During the OU process with starting distribution defined in \Cref{eq:HS-mix}, 
    \[
        \frac{1}{1+\frac{1-2\delta}{\delta}e^{-2t}}\cdot \!{Id}_d \preceq -\grad^2 \log p_t(x) \preceq \frac{1}{1-(1-\delta)e^{-2t}} \cdot \!{Id}_d
    \]
    for any $t>0$ and any $x\in \bb R^d$.
\end{corollary}
\begin{proof}
    Recall that $X_t = e^{-t}X_0 + \sqrt{2}\cdot e^{-t}\int_{0}^t e^s \d B_s$. Therefore $\mu_t = \mu'*\nu$ where $\mu'$ is the distribution with density $p_{\mu'}(x) \propto p_{\mu}\tp{\frac{x}{e^{-t}}}$ and $\nu$ is $\+N(0,(1-e^{-2t})\!{Id}_d)$. From \Cref{lem:m-gaussian1}, with $M_1 = \frac{1-\delta}{\delta}\cdot e^{-2t}\!{Id}_d$, $M_2 = \delta\cdot e^{-2t}\!{Id}_d$ and $M=(1-e^{-2t})\!{Id}_d$, we have 
    \[
        \frac{1}{1+\frac{1-2\delta}{\delta}e^{-2t}}\cdot \!{Id}_d \preceq -\grad^2 \log p_t(x) \preceq \frac{1}{1-(1-\delta)e^{-2t}} \cdot \!{Id}_d.
    \]
\end{proof}

Thus, despite being a mixture of Gaussian distributions where the component centers might be far apart, it is still $\+O(1)$-log-smooth and remains $\+O(1)$-log-smooth throughout the OU process.


\begin{remark}
    The motivation for exploring the distributions in \Cref{eq:HS-mix} is to study the Ising model, which is a distribution over $\ab\{\pm 1\}^d$ with density $p_{J,h}(\sigma)\propto \exp\set{\frac{1}{2}\inner{\sigma}{J\sigma} + \inner{h}{\sigma}}$ for any $\sigma\in \ab\{\pm 1\}^d$. 

    The Hubbard-Stratonovich transform states that the Ising model can be reduced to sampling from the distribution in \Cref{eq:HS-mix}: Consider the joint distribution over $\ab\{\pm 1\}^d\times \bb R^d$ with density $p_{J,h}(\sigma,x)\propto \exp\tp{-\frac{1}{2}x^\top J^{-1} x+(J^{-1}h+\sigma)^\top x}$. We can prove that
    \begin{itemize}
        \item its marginal density on $\bb R^d$ is exactly the $p_{\mu}$ in \Cref{eq:HS-mix};
        \item $p_{\mu}(x)\propto \exp\set{ - \frac{1}{2}x^{\top}J^{-1}x + h^{\top}J^{-1}x} \cdot \prod_{i=1}^d \tp{e^{x(i)}+e^{-x(i)}}$ and thus the unnormalized density and the gradients of the potential function can be calculated in polynomial time for any $x$;
        \item the conditional distribution with density $p_{J,h}(\sigma|x) \propto \exp\set{\inner{\sigma}{x}}$ is a product distribution and can be sampled efficiently.
    \end{itemize}
    The proofs are similar to Lemma E.1 in \cite{KLR22}.

    % Its marginal density on $\bb R^d$ is exactly \Cref{eq:HS-mix}. Furthermore, we can prove that the conditional distribution $p_{J,h}(\sigma|x) \propto \exp\set{\inner{\sigma}{x}}$, which is a product distribution on $\ab\{\pm 1\}^d$ and can be efficiently sampled from (the proof is similar to Lemma E.1 in \cite{KLR22}).

    Therefore, sampling from the Ising model can be executed in two steps: 1) sample $X\sim \mu$; 2) sample $\sigma$ from the distribution with density $p_{J,h}(\sigma|X)$. Hence, the hardness of this problem is closely related to the hardness of sampling from mixture of Gaussians. Given \Cref{lem:mixture1}, when $0\prec J\prec \!{Id}_d$, the distribution $\mu$ can be simulated in polynomial time using Langevin-based algorithms (e.g., the algorithm in \cite{CCBJ18}) and thus also gives a polynomial complexity upper bound for the Ising model. On the other hand, \cite{GKK24} proved that for any real $c>1$, the existence of polynomial samplers for Ising model with arbitrary $0\prec J\prec (1+c)\!{Id}_d$ implies $\*{NP}=\*{RP}$. This in turn indicates that, assuming  $\*{NP}\neq\*{RP}$, sampling from the mixture of Gaussians in such a special structure with $0\prec J\prec (1+c)\!{Id}_d$ is generally hard.
\end{remark}
% \htodo{Is the main idea of this remark is clear?}
% \ctodo{Very good}






\section{Conclusion Remarks}
This work proposes a RBG graph model for disease spreading via hubs. We study the joint effect of the agent density, hub density, and connection function. The existence of a critical hub density depends only on the boundedness of the support of the connection function, which relates to curbing the traveling distance of individuals. When it comes to dispersion, both the degree distribution and the percolation threshold suggest that increasing dispersion helps spread the disease. The percolation properties of RBG graphs relate to unipartite graphs with modified connection functions. 
An interesting question in this direction is if and when the properties of the RBG graphs can be well represented by unipartite graphs with some modified connection functions. Our conjecture is that for independent connections between different pairs of agents, such representation is unlikely due to the oblivion of the local dependence (present in the RBG models). 
 Another direction is to consider hybrid models where agents may get infected either through common hubs or direct interactions between agents. The former infection mechanism is more centralized than the latter. 

\section*{Acknowledgments}
T.S. is supported by the Vannevar Bush Faculty Fellowship ONR N00014-23-1-2876, National Science Foundation grants RI-2312342 and RI-1901403, ARO award W911NF2210266, and NIH award A240108S001. B.H.Z. is supported 
by the CMU Computer Science Department Hans Berliner
PhD Student Fellowship. E.T, R.E.B., and V.C. thank the Cooperative AI Foundation, Polaris Ventures (formerly the Center for
Emerging Risk Research) and Jaan Tallinn’s donor-advised fund at Founders Pledge for financial
support. E.T. and R.E.B. are also supported in part by the Cooperative AI PhD Fellowship. G.F is supported by the National Science Foundation grant CCF-2443068. We are grateful to Mete \c Seref Ahunbay for his helpful feedback. We also thank Andrea Celli and Martino Bernasconi for discussions regarding $\Phi$-equilibria in games with coupled constraints.

\bibliography{dairefs}

%%%%%%%%%%%%%%%%%%%%%%%%%%%%%%%%%%%%%%%%%%%%%%%%%%%%%%%%%%%%%%%%%%%%%%%%%%%%%%%
%%%%%%%%%%%%%%%%%%%%%%%%%%%%%%%%%%%%%%%%%%%%%%%%%%%%%%%%%%%%%%%%%%%%%%%%%%%%%%%
% APPENDIX
%%%%%%%%%%%%%%%%%%%%%%%%%%%%%%%%%%%%%%%%%%%%%%%%%%%%%%%%%%%%%%%%%%%%%%%%%%%%%%%
%%%%%%%%%%%%%%%%%%%%%%%%%%%%%%%%%%%%%%%%%%%%%%%%%%%%%%%%%%%%%%%%%%%%%%%%%%%%%%%
\clearpage
\appendix

\section{Additional Preliminaries}
\label{sec:add-prels}

\paragraph{Revisiting \Cref{def:evi}} 

In order to define the distributions $\Delta(\cX)$ over $\cX$ precisely, we recall here some basic concepts from probability theory. We refer to \citet[Chapter 1 and 2]{Billingsley99:Convergence} and \citet[Chapter 15]{AliprantisB06:Infinite} for detailed treatments. We assume throughout the paper that the set $\cX \subseteq \R^d$ is Borel measurable. Let $\Delta(\cX)$ be the set of Borel probability measures $\mu$ on $\cX$, that is, measures $\mu: \qty(\cX, B(\cX)) \to \qty(\R, B(\R))$ with $\mu(\cX) = 1$, where $B(\cX)$ and $B(\R)$ denote the respective $\sigma$-algebra of Borel sets. We simply call $\mu$ a distribution. For any Borel measurable function $f : \cX \to \R$---henceforth just \emph{measurable}--- we can then take the integral $\E_{\vx \sim \mu}[f(\vx)] := \int_\cX f(\vx) d\mu(\vx)$. In particular, for $\E_{\vx\sim\mu} \ip{F(\vx), \phi(\vx) - \vx}$ in \Cref{def:evi} to be well-defined, we assume throughout this paper that $F$ and each $\phi \in \Phi$ are measurable functions.

For our computational results (\Cref{sec:main}), we are making a standard assumption regarding the geometry of $\cX$ (\Cref{sec:prel}); this can be met by bringing $\cX$ into isotropic position. In particular, there is a polynomial-time algorithm that computes an affine transformation to accomplish that~\citep{Lovasz06:Simulated}, and minimizing linear-swap regret reduces to minimizing linear-swap regret to the transformed instance~\citep[Lemma A.1]{Daskalakis24:Efficient}.

% \paragraph{Further comments on $\Delta(\cX)$} We make $\Delta(\cX)$ a topological space by equipping it with the weak-$*$ topology. This is exactly the coarsest topology that makes the map $\mu \mapsto \int_\cX f(x) d\mu(x)$ continuous for all continuous and bounded maps $f : \cX \to \R$.\footnote{Convergence in this topology coincides with the notion in probability theory that the probability measures $(\mu_n)_{n \in \N} \subset \Delta(\cX)$ \emph{weakly converge} to the probability measure $\mu \in \Delta(\cX)$.} Any continuous $f : \cX \to \R$ is guaranteed to be bounded since are working with compact spaces $\cX$ in this paper. Moreover, $\cX \subset \R^n$ compact implies that $\Delta(\cX)$ is compact and Hausdorff as well \cite{AliprantisB06:Infinite}[Thm.~15.11].

\iffalse

\begin{definition}[Weak separation oracle;~\citealp{Grotschel81:Ellipsoid}]
    Let $\cX$ be a convex and compact set in $\R^d$ and $\epsilon > 0$ a rational number. A \emph{(weak) separation oracle} for $\cX$ is a function that, given $\vx \in \R^d$,
    \begin{itemize}[topsep=0pt,noitemsep]
        \item asserts whether $\vx$ is $\epsilon$-close to $\cX$, in that its Euclidean distance from $\cX$ is at most $\epsilon$; or, otherwise,
        \item  finds a vector $\vec{w} \in \R^d$, with $\|\vec{w}\| = 1$, such that $\langle \vec{w}, \vx \rangle \geq \langle \vec{w}, \vx' \rangle - \epsilon$ for any $\vx'$ that is in $\cX$ and its distance from any point not in $\cX$ is at least $\epsilon$.
    \end{itemize}
\end{definition}

\fi
\section{Proofs for Deterministic Safety Algorithms}\label{sec:proofs-det}

In this section we prove the correctness of our algorithm in \Cref{sec:warmup}.
We first prove that the \textsc{Round} procedure in \Cref{alg:aac-byz} satisfies the properties below, and then prove that \Cref{alg:skeleton} solves consensus under Byzantine faults.
\begin{description}
    \item[Strong Validity] If all correct processes propose the same value $v$ and a correct process returns a pair $\langle \textsc{Grade}, v' \rangle$, then $\textsc{Grade} = \textsc{Commit}$ and $v' = v$.
    \item[Consistency] If any correct process returns  $\langle \textsc{Commit}, v \rangle$, then no correct process returns $(\cdot, v' \ne v)$.
    \item[Termination] If all correct processes propose, then every correct process eventually returns.
\end{description}

In our proofs we rely on the following properties of Byzantine Reliable Broadcast (BRB)~\cite{book}:
\begin{description}
    \item[BRB-Validity] If a correct process $p$ broadcasts a message $m$, then every correct process eventually delivers $m$.
    \item[BRB-No-duplication] Every correct process delivers at most one message.
    \item[BRB-Integrity] If some correct process delivers a message $m$ with sender $p$ and process $p$ is correct, then $m$ was previously broadcast by $p$.
    \item[BRB-Consistency] If some correct process delivers a message $m$ and another correct process delivers a message $m$, then $m = m$.
    \item[BRB-Totality] If some message is delivered by any correct process, every correct process eventually delivers a message.
\end{description}

\begin{lemma}\label{lem:byz-round-validity}
    With Byzantine faults and $n=3f+1$, \Cref{alg:aac-byz} satisfies strong validity.
\end{lemma}
\begin{proof}
    If all correct processes propose the same value $v$, then at least $2f+1$ processes BRB-broadcast an \textsc{Init} message for $v$, and therefore at most $f$ processes BRB-broadcast an \textsc{Init} message for $1-v$. Thus $v$ will be the majority value among all \textsc{Init} messages delivered in phase 1, at all correct processes. Thus all correct processes will BRB-broadcast an \textsc{Echo} message for $v$. Furthermore, no Byzantine process can produce a valid \textsc{Echo} message for $1-v$, since to do so would require a set of $2f+1$ \textsc{Init} message with a majority value of $1-v$. This is impossible due to the properties of BRB and the fact that at most $f$ processes have BRB-broadcast an \textsc{Init} message for $1-v$.
    So, all valid \textsc{Echo} messages received by correct processes will be for $v$, so all correct processes will commit $v$ at line~\ref{line:fac-byz-commit}.
\end{proof}

\begin{lemma}
    With Byzantine faults and $n=3f+1$, \Cref{alg:aac-byz} satisfies consistency.
\end{lemma}
\begin{proof}
    If a correct process $p_1$ commits $v$ at line~\ref{line:fac-byz-commit}, then it must have delivered a set $S_1$ of $2f+1$ \textsc{Echo} messages for $v$ at line~\ref{line:fac-byz-wait-echo}. Take now another process $p_2$ and consider the set $S_2$ of $2f+1$ \textsc{Echo} messages it delivers at line~\ref{line:fac-byz-wait-echo}. By quorum intersection, $S_1$ and $S_2$ must intersect in at least $f+1$ messages. By the BRB-Consistency property, these $f+1$ messages must be identical at $p_1$ and $p_2$. Thus $p_2$ delivers at least $f+1$ \textsc{Echo} messages for $v$, which constitutes a majority of the $2f+1$ \textsc{Echo} messages it delivers overall. So if $p_2$ commits a value at line~\ref{line:fac-byz-commit}, then it must commit $v$, and if $p_2$ adopts a value at line~\ref{line:fac-byz-adopt}, then it must adopt $v$.
\end{proof}

\begin{lemma}
    With Byzantine faults and $n=3f+1$, \Cref{alg:aac-byz} satisfies termination.
\end{lemma}
\begin{proof}
    Follows immediately from the algorithm and from the properties of Byzantine Reliable Broadcast. Processes perform two phases; the only blocking step of each phase is waiting for $n-f$ messages (lines~\ref{line:fac-byz-wait-init} and~\ref{line:fac-byz-wait-echo}). This waiting eventually terminates, by the BRB-Validity property and the fact that there are at least $n-f$ correct processes.
\end{proof}

\begin{theorem}\label{thm:validity-byz}
    With Byzantine faults and $n=3f+1$, \Cref{alg:skeleton} satisfies strong validity.
\end{theorem}
\begin{proof}
    This follows from the strong validity property of the \textsc{Round} procedure (\Cref{lem:byz-round-validity}): if all correct processes propose $v$ to consensus, then all correct processes propose $v$ to \textsc{Round} in the first round, where by \Cref{lem:byz-round-validity}, all correct processes commit $v$, and thus all correct processes decide $v$ at line~\ref{line:skeleton-decide}.
\end{proof}

\begin{theorem}\label{thm:agreement-byz}
    With Byzantine faults and $n=3f+1$, \Cref{alg:skeleton} satisfies agreement.
\end{theorem}
\begin{proof}
    Let $r$ be the earliest round at which some process decides and let $p$ be a process that decides $v$ at round $r$. We will show that any other process $p'$ that decides, must decide $v$. 
    
    For $p$ to decide $v$ at round $r$, \textsc{Round} must output $(\textsc{Commit}, v)$ in that round. Thus, by the consistency property of \textsc{Round}, $\textsc{Round}(r,\cdot)$ must output $(\cdot, v)$ at all correct processes. If $\textsc{Round}(r,\cdot)$ outputs $(\textsc{Commit}, v)$ for $p'$, then $p'$ decides $v$ at round $r$ (line~\ref{line:skeleton-decide}). Otherwise, all correct processes input $v$ to $\textsc{Round}(r+1,\cdot)$, and by the strong validity property, all processes (including $p'$) will output $(\textsc{Commit}, v)$ and decide $v$ at round $r+1$.
\end{proof}

\begin{theorem}\label{thm:termination-byz}
    With Byzantine faults and $n=3f+1$, \Cref{alg:skeleton} satisfies termination.
\end{theorem}
\begin{proof}
     We can describe the execution of the protocol as a Markov chain with states $0,\ldots,n-f=2f+1$; the system is at state $i$ if $i$ correct processes have estimate ($est_i$ variable) equal to $0$ before invoking $\textsc{Round}$. Due to the strong validity property of the $\textsc{Round}$ procedure, states $0$ and $2f+1$ are absorbing states. There is a non-zero transition probability from each state (including $0$ and $2f+1$), to state $0$ or $2f+1$, or both (we show this below). Therefore, with probability $1$, the system will eventually reach one of the two absorbing states and remain there. Once this happens (i.e., once all processes have the same $est_i$ variable), the strong validity property of $\textsc{Round}$ ensures that all processes (who have not decided yet) will decide within a round.
    
    It only remains to show that there is a non-zero transition probability from each state to at least one of the absorbing states $0$ and $2f+1$. Consider a state $i \notin \{0,2f+1\}$; there is a schedule $S$ with non-zero probability which leads the system from $i$ to $0$ or $2f+1$ in one invocation of \textsc{Round}. We consider two cases:
    \begin{itemize}
        \item $i < f+1$: in this case $0$ is the minority value among correct processes. In schedule $S$, the $n-f$ \textsc{Init} messages delivered by correct process at line~\ref{line:fac-byz-wait-init} are all from correct processes. Thus, every correct process sees $i$ $0$s and $2f+1-i$ $1$; $1$ is the majority value, so all correct processes adopt it for phase 2. In phase 2, $S$ again ensures that the $n-f$ \textsc{Echo} messages delivered by correct process at line~\ref{line:fac-byz-wait-echo} are all from correct processes. Thus, all correct processes see $2f+1$ \textsc{Echo} messages for $1$ and commit $1$, bringing the system to state $0$.
        \item $i \geq f+1$: in this case $0$ is the majority value among correct processes. This case is symmetrical with respect to the previous one: the only difference is that all correct processes adopt $0$ (the majority value) at the end of phase 1, and all correct processes deliver $2f+1$ \textsc{Echo} messages for $0$, thus committing $0$ and bringing the system to state $2f+1$.
    \end{itemize}
\end{proof}



\end{document}
