\section{Related Work}
\label{sec::related}
A broad analysis of the implications of applying robotics and data-driven technologies to agriculture can be found in ____. The authors stress how, in more than four decades of Robotics advancements in solving agricultural problems, a broad adoption in the agricultural practice is still lacking ____. In the study, different reasons are brought up. However, one of the most relevant is the unbalanced cost-benefit trade-off in current practices for data collection, creation, curation, and use that is critical for the generalization power of data-driven algorithms. 

It is not surprising that, among the wide spectrum of research in digital agriculture, data generation with self and semi-supervised techniques has attracted much attention. In a recent review on the topic ____, the authors count more than 50 articles since 2016 on label-efficient learning. Among all these approaches, generative sampling approaches are the ones that try to generate new labeled samples starting from a subset of the existing ones or by generating them from scratch. In this respect, a technology that is frequently exploited for realistic artificial sample generation is Generative Adversarial Networks (GANs) ____. A detailed review can be found in ____. These models have been successfully used to generate samples in a range of very different agronomic scenarios, such as plant seedlings images ____, \textit{arabidopsis} images ____, soil moisture images ____ and whiteflies pest images ____ to mention a few examples. However, these networks are notoriously difficult to train and require curated training datasets and specialized experience ____. 

On the other hand, direct data synthesis with graphics engines and CAD software is another common strategy to address data scarcity in Computer Vision tasks ____. In agriculture, some authors tried to leverage modern simulation engines to extract relevant data and automatize labeling ____. However, while many free resources exist for modeling simple natural scenes, creating more realistic environments is expensive and requires highly skilled work. 

Given the limitations of 3D engines in terms of realism, some authors put together both CAD-generated samples and GANs ____ using CycleGANs ____. 
In all cases, the combined strategy can provide a considerable performance increase, but again, it is difficult to automate due to the training instabilities of GANs.
%These networks are used to create realistic samples of weeds starting from synthetic 3D cad images. The authors in all cases show that this strategy is able to combine the synthetic generation of simulated environment with the image style shift properties of the GANs in an effective way, but again, at the cost of difficulties in training them. 
%However, GANs are difficult to train and all these systems require significant expert supervision to be developed and used effectively, that is a major barrier to their practical use in a field application. 

Compared to the works mentioned above, we also use a synthetic environment to generate new samples. However, we avoid needing expert data scientists or skilled graphics engineers to train complex generative models. Instead, we use simple, easily automatized CV algorithms to provide a complete generation pipeline with a degree of realism sufficient to give consistent performance improvements.

%Similar to our approach ____, but uses diffusion models that again require significant training (?).
%Precision agriculture represents a strategic advancement in farm management, leveraging observation, measurement, and response to variability within fields. The adoption of machine learning (ML) and computer vision (CV) technologies has been pivotal in automating tasks such as disease detection, yield estimation, and crop monitoring. However, the development and implementation of robust detection and segmentation models are critically dependent on the availability of high-quality, annotated datasets. This dependency is challenged by the phenomena of covariate shifts, where the statistical properties of training data differ from those encountered in operational environments, further exacerbated by the evolving nature of agricultural landscapes across seasons.

%(\textcolor{blue}{The issue of covariate shifts significantly impacts the performance of predictive models in precision agriculture, as highlighted by ____. They propose a minimax risk classification approach that effectively handles covariate shifts by weighting both training and testing samples, thereby mitigating the limitations posed by traditional reweighing and robust methods. This method demonstrates improved classification performance across both synthetic and empirical datasets, suggesting its potential applicability in addressing similar challenges within the agricultural domain.})

%Furthermore, the integration of synthetic data generation emerges as a promising avenue to circumvent the limitations imposed by data scarcity and the dynamic variability inherent to agricultural environments. The concept of domain and covariate shift adaptations, as discussed in the literature, emphasizes the importance of adapting models to better reflect the distribution of operational data, thus enhancing model robustness and reliability.

%This paper extends upon these foundational insights by proposing a novel synthetic data generation framework specifically tailored for precision agriculture. By synthesizing data that closely mimics the variability and complexity of agricultural scenes, we aim to provide a robust dataset for training detection and segmentation algorithms, thereby addressing the twin challenges of data scarcity and covariate shifts. The proposed framework leverages advancements in self-supervised learning and simulation technologies to generate high-fidelity, annotated images of crops, focusing on table grapes as a case study. Through this approach, we demonstrate significant improvements in model performance, highlighting the potential of synthetic data to revolutionize precision agriculture by enabling the development of more adaptive, resilient ML model