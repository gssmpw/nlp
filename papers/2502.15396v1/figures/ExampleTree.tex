\begin{figure}
	\centering
	\begin{tikzpicture}[
		every node/.style={circle, draw, inner sep=2pt},
		level distance=10mm,
		level 1/.style={sibling distance=60mm},
		level 2/.style={sibling distance=30mm},
		level 3/.style={sibling distance=15mm}]
	% Nodes
	\node (R) {R}
	  child {node (A1) {A1}
		child {node (B1) {B1}
			child {node (C1) {C1}}
			child {node (C2) {C2}}
		}
		child {node (B2) {B2}
			child {node (C3) {C3}}
			child {node (C4) {C4}}
		}
	  }
	  child {node (A2) {A2}
		child {node (B3) {B3}
			child {node (C5) {C5}}
			child {node (C6) {C6}}
		}
		child {node (B4) {B4}
			child {node (C7) {C7}}
			child {node (C8) {C8}}
		}
	  };
	%paths to root
	  \draw[backPath, bend left=15] (A1) to (R);
	  \draw[backPath, bend right=15] (A2) to (R);

	  \draw[backPath, bend left=35] (B1) to (R);
	  \draw[backPath, ] (B2) to (R);
	  \draw[backPath, ] (B3) to (R);
	  \draw[backPath, bend right=35] (B4) to (R);

	  
	  \draw[backPath, bend left=50] (C1) to (R);
	  \draw[backPath, bend left=10] (C2) to (R);
	  \draw[backPath, bend left=20] (C3) to (R);
	  \draw[backPath, bend left=0] (C4) to (R);
	  \draw[backPath, bend left=0] (C5) to (R);
	  \draw[backPath, bend right=20] (C6) to (R);
	  \draw[backPath, bend right=10] (C7) to (R);
	  \draw[backPath, bend right=50] (C8) to (R);

	  %Paths to A
	  \draw[backPath, bend right=15] (B1) to (A1);
	  \draw[backPath, bend left=15] (B2) to (A1);
	  \draw[backPath, bend right=15] (B3) to (A2);
	  \draw[backPath, bend left=15] (B4) to (A2);

	  
	  \draw[backPath, bend right=15] (C1) to (A1);
	  \draw[backPath, bend left=0] (C2) to (A1);
	  \draw[backPath, bend left=0] (C3) to (A1);
	  \draw[backPath, bend left=15] (C4) to (A1);
	  \draw[backPath, bend right=15] (C5) to (A2);
	  \draw[backPath, bend left=0] (C6) to (A2);
	  \draw[backPath, bend left=0] (C7) to (A2);
	  \draw[backPath, bend left=15] (C8) to (A2);

	  %Paths to B
	  \draw[backPath, bend right=15] (C1) to (B1);
	  \draw[backPath, bend left=15] (C2) to (B1);
	  \draw[backPath, bend right=15] (C3) to (B2);
	  \draw[backPath, bend left=15] (C4) to (B2);
	  \draw[backPath, bend right=15] (C5) to (B3);
	  \draw[backPath, bend left=15] (C6) to (B3);
	  \draw[backPath, bend right=15] (C7) to (B4);
	  \draw[backPath, bend left=15] (C8) to (B4);

	\end{tikzpicture}
	\caption{A similar graph to $G_4$ from \cref{thmt@@TheoremVisActCopmonotonicity}. In $G_4$ each node has 3 children, but the main ideas should become clear from this picture. Blue dashed lines represent paths of length $2s$. For \cref{thmt@@TheoremVisLazyCopmonotonicity} the length is $s+1$.}
	\label{fig:ExampleTree}
\end{figure}
