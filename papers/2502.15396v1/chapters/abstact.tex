\begin{abstract}
	In this paper, we study different variants of the Cops-and-Robber game with respect to cop- and robber\-/monotonicity.
	We study a visible and invisible robber and variants where the robber is lazy, thus can only move when the cops announce to move on top of him.
	In all four combinations, we also vary the number $s$ of edges that the robber can traverse in a single round, called speed.
	
	We complete the study of the unbounded speed case by showing that, besides the active variants, also the visible lazy variant has both the cop- and robber\-/monotonicity property.
	Furthermore, we prove that the cop\-/monotone invisible lazy copwidth characterizes path-width, while the non\-/monotone and robber\-/monotone is known to characterize tree-width, thus these variants differ even in the unbounded speed case.
	
	We find that, even with speed restriction, the cop\-/monotone invisible copwidth and the robber\-/monotone invisible active copwidth all characterize path-width.
	On the other hand, we show that the path-width of a graph can be arbitrarily larger than the number of cops needed to win the non\-/monotone invisible active variant.
	To complete our study of cop\-/monotone variants, we show that also in the visible variants the cop\-/monotone copwidth can be arbitrarily larger than the non\-/monotone.
	
	Regarding robber\-/monotonicity, for all speeds $s\geq 4$, we give graphs where the non\-/monotone and robber\-/monotone copwidth differ.
	On the other hand, we prove that there is a function that bounds the robber\-/monotone copwidth in terms of the non\-/monotone copwidth and the speed, thus the gap between the variants is bounded.
	This proof also yields that a graph class has bounded expansion if and only if, for every speed $s$, the number of cops needed in any robber\-/monotone lazy variant is bounded by some constant $c(s)$.

\keywords{structural graph theory  \and Cops-and-Robber \and monotonicity \and path-width \and bounded expansion}
\end{abstract}