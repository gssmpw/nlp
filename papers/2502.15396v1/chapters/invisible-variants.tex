\section{Path-Width and Invisible Variants} 
\label{MoreNonmonotonicity}
In this section, we will consider scenarios where requiring the cops to play monotonically is a hard restriction in the sense that it makes them unable to exploit any speed bound of the robber.
The first example of such a phenomenon is the invisible active variant.
Examining the copwidth for any bounded speed and adding robber\-/monotonicity results in the same copwidth as for unbounded speed, which we know equals path\-/width.
The basic idea is that if the cops play robber\-/monotone, the robber's movement is restricted to a monotonically shrinking subset of vertices and it is not important whether he moves within this region, and therefore also not with what speed.   

\begin{restatable}{theorem}{TheoremInvActRobbermonotonicity}
    For every graph $G$ and speed $s\in\N$, it holds that
\begin{align*}
    \rmiacopwidth{s}(G) &= \rmiacopwidth{\infty}(G) \\
    &= \pw(G) + 1.
\end{align*}
\end{restatable}
Recall that cop\-/monotonicity is stronger than robber\-/monotonicity and that the invisible active variant has the cop\-/monotonicity property, so the above theorem also holds for cop\-/monotonicity.
\begin{restatable}{corollary}{CorollaryInvActCopmonotonicity}
    For every graph $G$ and speed $s\in \N$, it holds that
\begin{align*}
    \cmiacopwidth{s}(G) &= \cmiacopwidth{\infty}(G)\\
    &= \pw(G) + 1.
\end{align*}
\end{restatable}

With a similar technique we can prove the same bound for the cop\-/monotone invisible active variant.
\begin{restatable}{theorem}{TheoremInvLazyCopmonotonicity}
 For every graph $G$ and speed $s\in \N \text{ or } s=\infty$
\begin{align*}
    \cmilcopwidth{s}(G) &= \cmiacopwidth{\infty}(G)\\
    &= \pw(G) + 1.
\end{align*}
\end{restatable}
The non\-/cop\-/monotonicity of the invisible lazy copwidth now follows from the upper bound $\ilcopwidth{s}(G) \leq \ilcopwidth{\infty}(G) =\tw(G)+1$ because tree\-/width may be arbitrarily smaller than path\-/width. So $\cmilcopwidth{s}$ may be arbitrarily larger than $\ilcopwidth{s}$.

For the invisible active variant we are able to show a similar result.
\begin{restatable}{theorem}{TheoremInvActRobbermonotonicityBoundSpeed}
	\label{TheoremInvActRobbermonotonicityBoundSpeed}
	For every $g,s\in\N$ and every $n\geq 2g+2$, there is a graph $\iaNonMon sgn$ such that $\pw(\iaNonMon sgn)\geq n+g$ but $\iacopwidth{s}(\iaNonMon sgn)\leq n$.
\end{restatable}
We give the construction of \iaNonMon sgn here without proof.
\iaNonMon sgn is a disjoint union of three cliques $A,B,C$, where $|A|\coloneqq n-g$ and $|B|=|C| \coloneqq g$.
All vertices in $B$ and $C$ are connected to all vertices in $A$.
Furthermore all vertices in $B$ are connected to all vertices in $C$ via disjoint paths with $3s-2$ internal vertices.