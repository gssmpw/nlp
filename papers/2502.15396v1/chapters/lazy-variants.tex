\section{Robber Monotonicity and Lazy Variants}
\label{SecitonColouringNumbers}
The bounded speed lazy variants and their robber\-/monotone analogues have a rich interaction with the strong colouring numbers. By plugging together trivial bounds and those discussed in \cite{dendris1997fugitive}\footnote{Here the authors showed the equivalence of the robber\-/monotone invisible lazy copwidth and a graph parameter called the $s$\=/elimination dimension plus one. The $s$\=/elimination dimension is defined nearly identically to the strong $s$\=/colouring number, with the only difference being that in the former, the paths that lead up in the order are counted. In contrast, the paths that lead down in the order are counted in the latter.} and \cite{torunczyk2023flip} we get the following lemma.
\begin{restatable}{lemma}{LemmaFuncEquivCaR}
    For every speed $s\in \N$, we have the relations depicted in the diagram shown in \cref{fig:LazyAndCoulorNumbers}. Moreover, it follows that those four copwidth variants, $s$\=/admissibility and the generalised colouring numbers are all pairwise functionally equivalent.
\end{restatable}

\begin{figure}
    \centering
    \begin{tikzpicture}
            \node (adm) at (0, -1) {$\adm{s}+1$};
            \node (il) at (-1.5, 0.7) {$\ilcopwidth{s}$};
            \node (vl) at (1.5, 0) {$\vlcopwidth{s}$};
            \node (rmvl) at (1.5, 1) {$\rmvlcopwidth{s}$};
            \node (scol) at (0, 2) {$\rmilcopwidth{s}=\scol{s}+1$};
    
            \draw[strictlysmaller] (adm) -- (il);
            \draw[strictlysmaller] (adm) -- (vl);
            \draw[strictlysmaller] (vl) -- (rmvl);
            \draw[strictlysmaller] (rmvl) -- (scol);
            \draw[strictlysmaller] (il) -- (scol);
            \draw[strictlysmaller] (vl) -- (il);
    \end{tikzpicture}    
    \caption{Relations between the colouring numbers and lazy copwidth variants for any $s>1$. Each arrow represents a strict inequality, note that the differences between the robber\-/monotone and non\-/monotone variants are only known for $s\geq 4$.}
    \label{fig:LazyAndCoulorNumbers}
\end{figure}


For the relationship between the invisible lazy variant and its robber\-/monotone version, it is known from \cite{dendris1997fugitive} that both are equivalent in the unbounded speed setting. This result is called "Recontamination does not help" because the potential hiding spots of an invisible robber can be thought of as a gas that spreads along unguarded edges of the graph. Then, the absence of recontamination of already cleared vertices corresponds to a robber\-/monotone cop strategy. \\
However, we get that robber\-/monotonicity is indeed not the case in the bounded speed setting, thus allowing for recontamination does help to reduce the number of cops necessary to remove the gas from the graph or, respectively, catch the invisible robber.
For all $s\geq 4$, we show this non\-/monotonicity property both for the visible lazy variant, confirming the corresponding conjecture from \cite{doi:10.1137/090780006} and the invisible lazy variant, confirming the corresponding conjecture from \cite{dendris1997fugitive}. \\
For recontamination to be a benefit to the cops, there must be a vertex $v$ that is crucial in restricting robber movement while not being a suitable hiding spot itself.
So the cops do not want to let the robber move \emph{over} $v$ but are okay with him moving \emph{on} $v$.
This allows the cops to block $v$ in some early rounds and later force the robber to move on it and catch him there.
In the unbounded speed setting, such a construction is impossible, as allowing the robber to move on a particular vertex is always connected with allowing him to move past it.
\begin{restatable}{theorem}{TheoremLazyRecontamination}
For finite $s\geq 4$, the visible lazy and invisible lazy variants of speed $s$ do not have the robber\-/monotonicity property.
\end{restatable}
The construction used in the proof is presented in \cref{fig:LazyRecontamination}. Our construction can not be trivially modified for the case $s\in \{2,3\}$ and it is questionable whether the theorem actually holds for these. The visible lazy variants of speed $1,2,3$ and $\infty$ are known to be decidable in $\mathbf{P}$, while this decision problem is $\mathbf{NP}$-hard for all other finite speeds, as shown in \cite{doi:10.1137/090780006}. This property may coincide with the robber\-/monotonicity property, but more research is needed to confirm this conjecture.  

\begin{figure}
        \centering
        \farbigergraph{white,white,white,white,white,white,white,white,white,white,white,white,white,white,white,white,white,white,white,white,white,white,white,white,white,white,white,white,white,white,white,white,white,white,white,white,white,white}
\caption{An example graph $G$ with $\vlcopwidth{4}(G)\leq\ilcopwidth{4}(G)\leq10$ and $\rmilcopwidth{4}(G)\geq\rmvlcopwidth{4}(G)>10$. $K_n$ denotes the n-clique.} 
    \label{fig:LazyRecontamination}
    \end{figure}

Thus, both lazy variants do not have the robber\-/monotonicity property for finite speeds $s\geq4$. However, in contrast to similar results of the last sections, the monotonicity cost is bounded by \cref{thmt@@LemmaFuncEquivCaR}.

At last, we want to prove that the visible lazy variant with unboudned speed has both monotonicity properties as conjectured in \cite{doi:10.1137/090780006}. To this end we define the notion of a \emph{funnel} on a graph to build our argument.
\begin{definition}
    Let $G=(V,E)$ be a graph and $S\subseteq V$. We define the \emph{cut} (you could also say neighbourhood) of $S$ in $G$ as  
    $$cut_G(S) := \{v\in V\setminus S | \ \exists s \in S \ (v,s)\in E\}.$$
    If $G$ is clear from  context we will omit the index.    
\end{definition}
\begin{definition}
    Let $G=(V,E)$ be a graph, and $A\subseteq V$. We call $F: A \rightarrow \mathcal{P}ot(A)$ a \emph{funnel} on $A$ if $\forall a \in A$:
    \begin{itemize}
        \item[(F1)] $a \in F(a)$
        \item[(F2)] $\forall b \in F(a) \setminus \{a\}: \ a \notin F(b)$
        \item[(F3)] $F(a)$ induces a conncted subgraph of $G$
    \end{itemize}

    We call $\operatorname*{max}\{|cut(F(a))| \ | \ a \in A\}$ the width of the funnel. 
    If it additionally satisfies 
    \begin{itemize}
        \item[(F4)] $\forall a \in A \forall b \in F(a) \ F(b)\subseteq F(a)$,
    \end{itemize}
    we say that the funnel is monotone.
\end{definition}
    Funnels can be seen as a similar notion to partial cop strategies in visible lazy unbounded speed games. For a given vertex $v$ in some region $A$ of the graph, a funnel $F$ assigns targets $F(v)$ in $A$ that describe where the robber is forced to move to next. $(\mathit{F1})$ demands that his current position is part of the movement. $(\mathit{F2})$ demands that the robber is not allowed to go back in the next round. $(\mathit{F3})$ demands that these movements take place in one connected subgraph. The width of the funnel plus one is then the amount of cops necessary to enforce this movement by guarding the border of the targets $F(v)$, and one cop initiating the movement by stepping on the robbers position. $(\mathit{F4})$ ensures that this is a successful and also monotone strategy on $A$ by decreasing the number of possible robber movements from one round to the next. For $A=V$ we get the following.
\begin{restatable}{lemma}{LemmaVisLazyFunnelToStrat}
        Let $G=(V,E)$ be a graph and $F: V \rightarrow \mathcal{P}ot(V)$ a monotone funnel of width $k$ then $\cmvlcopwidth{\infty}(G)\leq k+1$.
\end{restatable}

It turns out that the converse also holds, even if we drop the monotonicity requirement. We can build a funnel in steps from a given strategy by first considering vertices on which the robber can be caught in the one round. Then in each step we add a new vertex to the funnel from which the cops can force the robber to move on the existing funnel. By \cite{doi:10.1137/090780006} such vertices always exist and we show that in the framework of funnels we can guarantee the result to be monotone at each step.

\begin{restatable}{lemma}{LemmaVisLazyStratToFunnel}
    Let $G=(V,E)$ be a graph and $\vlcopwidth{\infty}(G) = k$ then there exists a monotone funnel on $V$ of width less then $k$.
\end{restatable}

We immediatly get the following.
\begin{restatable}{theorem}{TheoremVisLazyMonotonicity}
    The visible lazy variant with unbounded speed has the cop\-/monotonicity property and therefore also the robber\-/monotonicity property.
\end{restatable}