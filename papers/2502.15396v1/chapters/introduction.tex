\section{Introduction}
After the first study on search models in 1976 by Parsons\cite{parsons2006pursuit}, the node search model we use today under the name of Cops\-/and\-/Robber was essentially introduced in 1985 and 1986 by Kirousis and Papadimitriou \cite{kirousis1985interval,kirousis1986searching}.
The game is played in rounds on the vertices of a graph between several cops and a single robber.
The cops try to catch the robber by placing on the same vertex as him, and the robber tries to avoid getting caught.
In each round, first, some cops exit the graph and announce where they will return (they do not need to move along the edges), then the robber can move along any path that does not contain inner vertices occupied by a cop, and finally, the cops return as announced. In the bounded speed variants, we bound the maximum length of the path that the robber can use.
The copwidth measure is defined as the minimal number of cops needed to catch a smart and omniscient robber on a given graph.
It is easy to see that in a sparse graph, fewer cops may be necessary than in a clique.
Variants of the game differ, e.g. in the speed of the robber, whether he can always move or only when a cop is announced to be placed in his position, called lazy variant instead of active, or whether the robber's position is visible to the cops. We get a different game with different properties and a different copwidth measure depending on the specific ruleset.

The invisible lazy and visible active variants with unbounded speed are known to characterize tree\-/width \cite{dendris1997fugitive,seymour1993graph} and the invisible active variant path\-/width \cite{kirousis1985interval}.
An important step in these proofs often is that these game variants have certain monotonicity properties.
We differ between two monotonicity definitions.
If the cop player is restricted to never revisit a vertex, this is called cop\-/monotonicity, if the robber player is never allowed to move to a vertex that was previously occupied by a cop, this is called robber\-/monotonicity.
Intuitively this means that in a cop\-/monotone game the cops monotonously move forward through the graph, where in a robber\-/monotone game the region that the robber can reach in an unbounded speed game or the region he could still be in in an invisible game monotonously shrinks.
Our definition of robber\-/monotony slightly differs from the definition usually used in literature to translate to visible bounded speed games where there is no longer an obvious definition of the "robber\-/region" and is equivalent to the standard definition in all other cases.
We say a game variant has a monotonicity property, if the number of cops needed to catch the robber in the respective monotone variant does not differ from the non\-/monotone variant.
It is easy to see that robber\-/monotonicity is the stronger restriction on the game.
Besides the above mentioned variants of the game that have the monotonicity properties, there are also variants, that do not have these properties, for example games on directed graphs \cite{Kreutzer_digraph-decomp_2O11} and on hypergraphs \cite{Adler04}.
A monotonicity property can be helpful by restricting the search space when calculating copwidth and yield that the corresponding decision problem belongs to NP.

There are some obvious relations between the different game variants.
A strategy against an invisible robber will also work against a visible one and a strategy against an active robber also works against a lazy robber.
Furthermore a cop\-/monotone strategy is always robber\-/monotone and both are a strategy in the non\-/monotone game.
Lastly the speed restriction hinders only the robber, that is a strategy against a quick robber (even an infinite speed robber) will also work against a slow robber.
On the other hand speed is a real restriction for the robber in most variants.
Consider for example a clique with $n$ vertices where one replaces every edge by a path with $s$ inner vertices.
In the robber\-/monotone invisible lazy or cop\-/monotone visible lazy variant three cops win against a robber of any speed $\leq s$ by systematically clearing one edge after the other.
A robber of speed $>s$ however wins even the visible lazy variant against less that $n$ cops as he can always move to some free vertex of the original clique whenever the cops move.
A similar argument works for the (cop\-/monotone) visible active variant, when the edges are replaced by path with $2s-1$ inner vertices.

Search games, as the Cops\-/and\-/Robber game, have many applications.
For example they model a variety of real\-/world search problems, see \cite{FominT08} for a survey.
The number of rounds needed to capture the robber was used to show lower bounds on the runtime of the famous Weisfeiler\-/Lehman Refinement algorithm for graph isomorphism \cite{Furer_rounds_2001,Grohe2023CR-vs-WL}.
It has been proven that graph classes relating to Cops\-/and\-/Robber games, such as tree\-/width \cite{Neuen_homomorphism-distinguishing_2023}, tree\-/depth \cite{Fluck_deep-wide_2023} or bounded\-/depth tree\-/width \cite{Adler_monotone_2024}, are homomorphism distinguishing closed.
The bounded speed visible active variant is known to characterize bounded expansion \cite{torunczyk2023flip}.

\subsection{Our Contribution}
\begin{figure}[!htb]
    \crefname{theorem}{Thm.}{thms.}
    \crefname{corollary}{Cor.}{cors.}
    
    \begin{tikzpicture}
        \node (il1) at (0,0) {$\ilcopwidth{1}$};
        \node (ril1) at (2.5,0) {$\rmilcopwidth{1}$};
        \node (d*) at (5,0) {$\delta^* + 1$};
        
        \node (ils) at (0,0.8) {$\ilcopwidth{s}$};
        \node (rils) at (2.5,1) {$\rmilcopwidth{s}$};
        \node (scol) at (5,1) {$\scol{s}+1$};
        
        \node (ili) at (0,1.8) {$\ilcopwidth{\infty}$};
        \node (rili) at (2.5,1.8) {$\rmilcopwidth{\infty}$};
        \node (tw) at (5,1.8) {$\tw + 1$};
        
        
        \node (ian) at (7.5,1) {$\iacopwidth{1,s}$};
        
        \node (cili) at (0,3) {$\cmilcopwidth{1, s, \infty}$};
        \node (pw) at (2.5,3) {$\pw + 1$};
        \node (iai) at (5,3) {$\anyiacopwidth{\infty}$};
        \node (rian) at (7.5,3) {$\rmiacopwidth{s}$};
        \node (cian) at (10,3) {$\cmiacopwidth{s}$};
      
        \draw[strictlysmallerNew] (ian) -- (iai) node[midway, right] {\cref{TheoremInvActRobbermonotonicityBoundSpeed}}; 
        \draw[strictlysmaller] (d*) -- (ian);

        \draw[equalNew] (cian) -- (rian) node[midway, above] {\cref{thmt@@CorollaryInvActCopmonotonicity}};
        \draw[equalNew] (iai) -- (rian) node[midway, above] {\cref{thmt@@TheoremInvActRobbermonotonicity}};

        \draw[equal] (iai) -- (pw) node[midway, above] {\cite{kirousis1985interval,kirousis1986searching}};

        \draw[equal] (il1) -- (ril1) node[midway, above] {\cite{dendris1997fugitive}};
        \draw[equal] (d*) -- (ril1) node[midway, above] {\cite{dendris1997fugitive}};        
        \draw[strictlysmaller] (il1) -- (ils);      
        \draw[strictlysmallerNew] (ils) -- (rils) node[midway, above] {\cref{thmt@@TheoremLazyRecontamination}};
        \draw[equal] (rils) -- (scol) node[midway, above] {\cite{dendris1997fugitive}};        
        \draw[funcBound] (rils) -- (ils);        
        \draw[strictlysmallerNew] (ils) -- (ili);        
        \draw[strictlysmaller] (scol) -- (tw) node[midway, right] {\cite{nevsetvril2012sparsity}};        
        \draw[strictlysmallerNew] (ril1) -- (rils);         
        \draw[strictlysmallerNew] (rils) -- (rili);        
        \draw[equal] (ili) -- (rili) node[midway, above] {\cite{dendris1997fugitive}};        
        \draw[equal] (tw) -- (rili) node[midway, above] {\cite{dendris1997fugitive}};        
        \draw[strictlysmallerNew] (ili) -- (cili) node[midway, left] {\cref{thmt@@TheoremInvLazyCopmonotonicity}};        
        \draw[equalNew] (cili) -- (pw) node[midway, above] {\cref{thmt@@TheoremInvLazyCopmonotonicity}};
    \end{tikzpicture}


    \begin{tikzpicture}
        \node (va1) at (4.5,0) {$\vacopwidth{1}$};
        \node (rva1) at (6,0) {$\rmvacopwidth{1}$};
        \node (cva1) at (6.7,1) {$\cmvacopwidth{1}$};
        \node (vas) at (4.5,1.6) {$\vacopwidth{s}$};
        \node (adm) at (3,0.6) {$\adm{s}+1$};
        \node (scol) at (3.1,2.6) {$\scol{4s}+1$};
        \node (wcol) at (2.3,1.7) {$\wcol{2s}+1$};
        \node (cvas) at (6.7,2.4) {$\cmvacopwidth{s}$};
        \node (vai) at (4.5,3) {$\anyvacopwidth{\infty}$};
        \node (tw) at (2.2,3) {$\tw + 1$};
        
        \node (vl1) at (0,0) {$\vlcopwidth{1}$};
        \node (d*) at (1.5,0) {$\delta^* + 1$};
        \node (vls) at (0,1) {$\vlcopwidth{s}$};
        \node (ds) at (1.5,1) {$\delta^s + 1$};
        \node (rvl1) at (-2.2,0) {$\rmvlcopwidth{1}$};
        \node (cvl1) at (-4.5,0.5) {$\cmvlcopwidth{1}$};
        \node (rvls) at (-2.2,1.2) {$\rmvlcopwidth{s}$};
        \node (cvls) at (-4.5,1.4) {$\cmvlcopwidth{s}$};
        \node (vli) at (0,2.2) {$\vlcopwidth{\infty}$};
        \node (rvli) at (-2.2,2.2) {$\rmvlcopwidth{\infty}$};
        \node (cvli) at (-4.5,2.2) {$\cmvlcopwidth{\infty}$};
        \node (di) at (1.5,2.2) {$\delta^\infty + 1$};
        
        \draw [equal] (va1) -- (rva1) node[midway, above] {\cite{torunczyk2023flip}};
        \draw [equal] (va1) -- (d*) node[midway, above] {\cite{torunczyk2023flip}};
        \draw [equal] (vai) -- (tw) node[midway, above] {\cite{seymour1993graph}};

        \draw [strictlysmaller] (va1) -- (vas);
        \draw [strictlysmaller] (adm) -- (vas) node[midway, below] {\cite{torunczyk2023flip}};
        \draw [strictlysmaller] (ds) -- (vas);
        \draw [strictlysmaller] (adm) -- (ds) node[midway, below] {\cite{doi:10.1137/090780006}};
        \draw [funcBoundOld] (ds) -- (adm);
        \draw [leq] (vas) -- (scol) node[midway, above] {\cite{hickingbotham2023cop}};
        \draw [leq] (vas) -- (wcol) node[midway, above] {\cite{torunczyk2023flip}};
        \draw [strictlysmallerNew] (vas) -- (cvas) node[midway, above left=0.0cm and -0.30cm] {\cref{thmt@@TheoremVisActCopmonotonicity}};
        \draw [strictlysmallerNew] (rva1) -- (cva1) node[midway, right] {\cref{thmt@@TheoremVisActCopmonotonicity}};
        \draw [strictlysmaller] (vas) -- (vai);
        \draw [strictlysmallerNew] (cva1) -- (cvas);
        \draw [strictlysmallerNew] (cvas) -- (vai);
        \draw [strictlysmallerNew] (cvl1) -- (cvls);

        
        \draw [strictlysmallerNew] (cvls) -- (cvli);
        \draw [strictlysmallerNew] (rvl1) -- (rvls);
        \draw [strictlysmallerNew] (rvls) -- (rvli);

        \draw[equal] (rvl1) -- (vl1) -- (d*) node[midway, above] {\cite{doi:10.1137/090780006}};
        \draw[equal] (vls) -- (ds) node[midway, above] {\cite{doi:10.1137/090780006}};
        \draw[equalNew] (vli) -- (rvli) node[midway, above] {\cref{thmt@@TheoremVisLazyMonotonicity}};
        \draw[equalNew] (rvli) -- (cvli) node[midway, above] {\cref{thmt@@TheoremVisLazyMonotonicity}};
        \draw[equal] (vli) -- (di) node[midway, above] {\cite{doi:10.1137/090780006}};

        \draw[strictlysmaller] (di) -- (tw) node[midway, left] {\cite{doi:10.1137/090780006}};
        \draw[strictlysmaller] (vls) -- (vli) node[midway, right] {\cite{doi:10.1137/090780006}};
        \draw[strictlysmaller] (vl1) -- (vls) node[midway, right] {\cite{doi:10.1137/090780006}};
        \draw[strictlysmallerNew] (rvl1) -- (cvl1) node[midway, above] {\cref{thmt@@TheoremVisLazyCopmonotonicity}};
        \draw[strictlysmallerNew] (rvls) -- (cvls) node[midway, above] {\cref{thmt@@TheoremVisLazyCopmonotonicity}};
        \draw[strictlysmallerNew] (vls) -- (rvls) node[midway, above] {\cref{thmt@@TheoremLazyRecontamination}};
        \draw[funcBound] (rvls) -- (vls);
    \end{tikzpicture}
    \caption{Known (black) and new (red, dashed) results for the variants. Arrows with open triangle heads represent a not strict inequality, arrows with filled triangle heads represent a strict inequality on some graphs. Lines represent equalities and arrows with a tip on the reverse side represent that the corresponding inequality is functionally bounded. $s$ is an arbitrary natural number greater than 1.}
    \label{figResults}
\end{figure}
\cref{figResults} summarizes all previously known and new results.
Our work solves an open problem from Dendris, Kirousis and Thilikos \cite{dendris1997fugitive} regarding the robber\-/monotonicity property of the invisible lazy variant for $s>4$ and answers an open questions from Richerby and Thilikos \cite{doi:10.1137/090780006} regarding monotonicity of the unbounded speed visible lazy variant.

In this paper, we prove that for any speed the cop- and robber\-/monotone variants of invisible active copwidth and the cop\-/monotone variant of invisible lazy copwidth all equal path\-/width. This directly implies that the latter does not have the cop\-/monotonicity property for any speed. For the invisible active variant we provide example graphs to separate the standard bounded speed variant from path\-/width with an arbitrarily large gap and therefore show non\-/cop\-/monotonicity. The maximal size of such a gap between the monotone and the standard variant is called the monotonicity cost.

We establish the non\-/cop\-/monotonicity of the bounded speed visible variants and the unbounded monotonicity cost in these cases by constructing example graphs. Further, we give examples for the non\-/robber\-/monotonicity of the bounded speed lazy variants and provide a min\-/max characterization of the strong colouring numbers in terms of a Cops\-/and\-/Robber game. Finally we show that the unbounded speed visible lazy variant has the robber\-/monotonicity property.