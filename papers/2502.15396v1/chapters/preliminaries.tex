\section{Preliminaries}
In this paper, we only consider finite and undirected graphs $G=(V, E)$ without self\-/loops that contain at least one edge. For $A\subseteq V$, the induced subgraph on the vertex set $V\setminus A$ is denoted by $G\setminus A$. A \emph{layout} of a graph is a linear order on the vertex set. For any $s\in \N \cup \{\infty\}$, given a layout $\prec$, a vertex $w\prec v$ is \emph{weakly $s$\=/reachable} from $v$ if there exists a path of length $\leq s$ from $v$ to $w$ such that all internal nodes $w'$ of that path satisfy $w'\succ w$. A vertex $w\prec v$ is \emph{strongly $s$\=/reachable} from $v$ if there exists a path of length $\leq s$ from $v$ to $w$ such that all internal nodes $w'$ of that path satisfy $w'\succ v$.
For a given graph, we define the following parameters as the minimal $k$ such that there is a layout where for each vertex $v$, there...
\begin{itemize}
  \item \textbf{degeneracy $\delta^*$ or width:} ...are at most $k$ vertices $w\prec v$, that are adjacent to $v$ 
  \item \textbf{weak $s$\=/coloring number $\wcol{s}$:} ...are at most $k$ vertices $w\prec v$, that are weakly $s$\=/reachable from $v$ 
  \item \textbf{strong $s$\=/coloring number $\scol{s}$ or $s$\=/elimination dimension:} ...are at most $k$ vertices $w\prec v$, that are strongly $s$\=/reachable from $v$
  \item \textbf{$s$\=/admissibility $\adm{s}$:} ... are at most $k$ paths of length $\leq s$, that start at $v$, end in some $w\prec v$ and are pairwise vertex disjoint apart from $v$ 
  \item \textbf{$s$\=/degeneracy $\delta_s$:} ...is an $A\subseteq V\setminus\{v\}$, $|A|\leq k$ such that no $w\prec v$ is reachable from $v$ in $G\setminus A$ by a path of length $\leq s$ .\footnote{In the literature these parameters are sometimes defined with a $w\preceq v$ instead of $w\prec v$ which leads to an offset by 1 in the parameter compared to the definition given above.}
\end{itemize}
Another way to characterize graph parameters is by obstructions: For $s$\=/degeneracy, for example, there is the notion of a \emph{$(k,s)$\=/hideout} as an obstruction, that is a set $U\subseteq V(G)$, such that for every vertex $v\in U$ and any set $A\subseteq V(G)\setminus \{v\}$ with $|A|<k$, there is some path from $v$ to $U\setminus \{v\}$ of length $\leq s$ in $G\setminus U$. $G$ has $\delta_s(G) \geq k$ if and only if there is a $(k,s)$\=/hideout in $G$. For more details see \cite{doi:10.1137/090780006}.
From the definitions it follows, that $\delta^*=\delta_1=\wcol{1}=\scol{1}=\adm{1}$ and for every $s\in \N$ $\adm{s} \leq \scol{s} \leq \wcol{s}$. A graph parameter $f$ is \emph{bounded in terms of} another graph parameter $g$ if there exists a function $\alpha: \R \rightarrow \R$ with $f(G)\leq \alpha (g(G))$ for every graph $G$.
Two parameters are \emph{functionally equivalent} if each is bounded in terms of the other. For example, for every $s\in\N$ $s$\=/admissibility and the generalized $s$\=/coloring numbers are pairwise functionally equivalent with a polynomial bound as shown by Dvořák in \cite{DVORAK2013833} and Nešetřil and De Mendez \cite{nevsetvril2012sparsity}.\\

\subsection{The Game}
A Cops\-/and\-/Robber game is played between a team of cops and a robber on the vertices of a given graph.
The cops try to catch the robber by moving from vertex to vertex, while the robber evades by moving along unguarded paths of the graph\footnote{Our definition is very similar to the ones used in \cite{dendris1997fugitive} and the node\-/search of \cite{doi:10.1137/090780006} with the distinction that in these papers $S_i\Delta S_{i+1}$ should have cardinality of at most 1. Intuitively, this means only one cop can be removed or placed per round. However, this is irrelevant for the visible lazy and invisible lazy variants considered in these papers, as the strategies can be easily transformed into each other: When given $S_i\Delta S_{i+1}$ with a cardinality of $k$ one could one by one firstly remove the cops, then place them on the positions where the robber is definitely not and lastly place the rest.}.
A formal Cops\-/and\-/Robber play on a Graph $G=(V, E)$ is a series of tuples $(S_0, R_0), (S_1, R_1), ...$, where $S_0=\emptyset$, $S_t$ denotes the positions of the cops and $R_t$ the position of the robber at the end of round $t$. We call max$\{|S_0|, |S_1|,...\}$ the \emph{width} of the play. The robber is caught whenever $R_t \in S_t$. He plays with \emph{speed} $s\in \N \cup \{\infty\}$ if $R_t$ and $R_{t+1}$ in $G\setminus {S_t \cap S_{t+1}}$ are connected by a path of length at most $s$. We define \emph{$\vacopwidth{s}(G)$} as the least $k$ such that there exists a \emph{strategy function} $f:\binom{V}{k} \times V \rightarrow \binom{V}{k}$\footnote{Note, that in our definition the strategy function does not depend on the history of the play.} that satisfies the following: In every play on $G$, where $S_{t+1}=f(S_t, R_t)$ and the robber is playing with speed $s$, the robber is caught and the width of the play is at most $k$ (Note that most of the time we will only give partial strategy functions that assumes a smart and omniscient robber that can be trivially be expanded to satisfy all conditions). The "va" stands for visible active, and we will name the other copwidth variants in a similar pattern. We define the \emph{invisibile active} variant of copwidth $\iacopwidth{s}$ analogously, where the strategy function can not depend on the robber position $R_t$ but just on the current cop position $S_t$ and $t$ and the \emph{lazy} variants $\vlcopwidth{s}$ and $\ilcopwidth{s}$ where only robbers need to be caught that satisfy $R_t \neq R_{t+1} \implies R_t \in S_{t+1}$. When talking about an invisible variant of copwidth, we will refer to the "possible hiding spots of the robber" at a round $t_0$ in regards to a specific cop\-/strategy $f$, meaning all the vertices $v$ for which there is a proper series of robber positions $(R_t)_{t\leq t_0}$ ending on $v$, that is not yet caught by cops acting accordingly to $f$. We say that the cops clear a vertex if they move on it, and it was previously a possible hiding spot.\\
One can see that if a graph has minimum degree $k$ and maximum degree $K$, then at least $k+1$ cops are necessary to catch a visible active or visible lazy robber and at most $K+1$ in the visible lazy case. \\ %\input{figures/ExampleLine}
We are interested in the interplay between a speed bound on the robber and a monotonicity restriction on the cops. Cops are said to search a graph \emph{cop\-/monotonously} if they never revisit a vertex, so $S_i \cap S_k \subseteq S_j$ for $i<j<k \in \N$. This gives us another range of copwidth variants like $\cmvacopwidth{s}$, defined by only considering strategies leading to cop\-/monotone searches. A variant is considered to have the \emph{cop\-/monotonicity property} if the variant's copwidth is equal, regardless of whether the cops search only cop\-/monotonously. \footnote{We consistently use the wordy phrasing "the variant has the cop\-/monotonicity property" instead of saying that "the variant is cop\-/monotone" to prevent any confusion with the term "cop\-/monotone variant" when regarding the altered variant.} \\
Analogously, cops are said to search a graph \emph{robber\-/monotonously} if $R_t \notin \bigcup_{t'< t}S_{t'}$ until the robber is caught. This means that once the cops expel the robber from a particular region, he should not be able to return there later on, and gives us another range of copwidth variants like $\rmiacopwidth{s}$ etc. that are defined by only considering strategies that lead to robber\-/monotone searches. A variant is considered to have the \emph{robber\-/monotonicity property} if this restriction does not affect the corresponding copwidth. It is clear from the definition that robber\-/monotonicity is weaker than cop\-/monotonicity because any cop\-/monotone search strategy that allows a robber to enter a vertex that a cop previously occupied would not be able to catch that robber later on. Note that all of these copwidth measures are monotone with respect to the subgraph relation.\\
