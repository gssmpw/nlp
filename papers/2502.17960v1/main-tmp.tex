%%%%%%%%%%%%%%%%%%%%%%%%%%%%%%%%%%%%%%%%%%%%%%%%%%%%%%%%%%%%%%%%%%%%%%%%

%%% LaTeX Template for AAMAS-2025 (based on sample-sigconf.tex)
%%% Prepared by the AAMAS-2025 Program Chairs based on the version from AAMAS-2025. 

%%%%%%%%%%%%%%%%%%%%%%%%%%%%%%%%%%%%%%%%%%%%%%%%%%%%%%%%%%%%%%%%%%%%%%%%

%%% Start your document with the \documentclass command.


%%% == IMPORTANT ==
%%% Use the first variant below for the final paper (including auithor information).
%%% Use the second variant below to anonymize your submission (no authoir information shown).
%%% For further information on anonymity and double-blind reviewing, 
%%% please consult the call for paper information
%%% https://aamas2025.org/index.php/conference/calls/submission-instructions-main-technical-track/

%%%% For anonymized submission, use this
%\documentclass[11pt]{article}
%%%% For camera-ready, use this
%\documentclass[sigconf]{aamas} 

\documentclass[sigconf,anonymous]{aamas} 

%%% Load any packages you require here. 

%\usepackage{latexsym}
%\usepackage{amssymb}
%\usepackage{amsmath}
\usepackage{amsthm}
%\usepackage{booktabs}
\usepackage{enumitem}
%\usepackage{graphicx}
\usepackage{color}

%\usepackage{times}  % DO NOT CHANGE THIS
%\usepackage{helvet}  % DO NOT CHANGE THIS
%\usepackage{courier}  % DO NOT CHANGE THIS
\usepackage{url}  % DO NOT CHANGE THIS
%\usepackage{upgreek}
\usepackage{dirtytalk}
\usepackage{algorithm}
\usepackage[noend]{algorithmic}
\usepackage{hyperref}

%\usepackage[noend]{algpseudocode}

%%%%%%%%%%%%%%%%%%%%%%%%%%%%%%%%%%%%%%%%%%%%%%%%%%%%%%%%%%%%%%%%%%%%%%%%

%%% Define any theorem-like environments you require here.

\newtheorem{theorem}{Theorem}
\newtheorem{lemma}[theorem]{Lemma}
\newtheorem{corollary}[theorem]{Corollary}
\newtheorem{proposition}[theorem]{Proposition}
\newtheorem{fact}[theorem]{Fact}
\newtheorem{definition}{Definition}

%%%%%%%%%%%%%%%%%%%%%%%%%%%%%%%%%%%%%%%%%%%%%%%%%%%%%%%%%%%%%%%%%%%%%%%%

%%% Define any new commands you require here.

%\newcommand{\BibTeX}{B\kern-.05em{\sc i\kern-.025em b}\kern-.08em\TeX}
\newcommand{\AreasIdSet}{\ensuremath{E}}
\newcommand{\areaId}{\ensuremath{e}}
\newcommand{\paramSet}{\ensuremath{\uprho}}


\newcommand{\Sset}{\ensuremath{S}}
\newcommand{\Aset}{\ensuremath{A}}
\newcommand{\Reward}[2]{\ensuremath{R(#1, #2)}}
\newcommand{\Oset}{\ensuremath{O}}
\newcommand{\Pfunc}[3]{\ensuremath{P(#1, #2, #3)}}
\newcommand{\trajectory}{\ensuremath{\tau}}
\newcommand{\stateS}{\ensuremath{s}}
\newcommand{\obsrvation}{\ensuremath{O(s,\tau)}}
\newcommand{\confidence}{\ensuremath{\upsilon}}
\newcommand{\threshold}{\ensuremath{t}}
\newcommand{\bigtau}{\mbox{\scalebox{1.5}{$\tau$}}}


\newcommand{\param}{\ensuremath{par}}


\newcommand{\operator}{\ensuremath{h}}
%\newcommand{\leftTime}[1]{\ensuremath{LT(#1)}}

\newcommand{\DronesNum}{\ensuremath{n}}
\newcommand{\targetsNum}{\ensuremath{m}}
\newcommand{\adviceNum}{\ensuremath{k}}

\newcommand{\timeToUpdate}{\ensuremath{z}}
\newcommand{\action}{\ensuremath{a}}
\newcommand{\deficulyLevel}[1]{d_{#1}}


\newcommand{\TG}{TG}
\newcommand{\AG}{AG}
\newcommand{\RM}{RM}
\newcommand{\detectiothreshold}{\ensuremath{hc}}
\newcommand{\pausetreshold}{\ensuremath{lc}}

%\newcommand{\alertAbove}[2]{\ensuremath{AL_a(#1, #2)}}
%\newcommand{\alertUnder}[2]{\ensuremath{AL_u(#1, #2)}}
\newcommand{\change}[2]{\ensuremath{CP(#1, #2)}}
\newcommand{\changeType}[2]{\ensuremath{CT(#1, #2)}}

%\newcommand{\detectionAlert}[1]{\ensuremath{da_\confidence}}
%\newcommand{\pauseAlert}[1]{\ensuremath{pa_\confidence}}
\newcommand{\costFunction}[1]{\ensuremath{C_{\operator}(#1)}}
\newcommand{\estimateCostFunction}[2]{\ensuremath{C_{\operator}(#1, #2)}}
\newcommand{\discountedFactor}{\gamma}
\newcommand{\wights}{\ensuremath{w}}
\newcommand{\velocity}{\ensuremath{vel}}
\newcommand{\altitude}{\ensuremath{alt}}
\newcommand{\UFunc}{\ensuremath{U}}
\newcommand{\Model}{\ensuremath{M}}
\newcommand{\parametersForU}{\Psi}
\newcommand{\paramForU}{\psi}


%%%%%%%%%%%%%%%%%%%%%%%%%%%%%%%%%%%%%%%%%%%%%%%%%%%%%%%%%%%%%%%%%%%%%%%%

%%% AAMAS-2025 copyright block (do not change!)

\setcopyright{ifaamas}
\acmConference[AAMAS '25]{Proc.\@ of the 24th International Conference
on Autonomous Agents and Multiagent Systems (AAMAS 2025)}{May 19 -- 23, 2025}
{Detroit, Michigan, USA}{A.~El~Fallah~Seghrouchni, Y.~Vorobeychik, S.~Das, A.~Nowe (eds.)}
\copyrightyear{2025}
\acmYear{2025}
\acmDOI{}
\acmPrice{}
\acmISBN{}


%%%%%%%%%%%%%%%%%%%%%%%%%%%%%%%%%%%%%%%%%%%%%%%%%%%%%%%%%%%%%%%%%%%%%%%%
%%% == IMPORTANT ==
%%% Use this command to specify your EasyChair submission number.
%%% In anonymous mode, it will be printed on the first page.

\acmSubmissionID{227}
\title{Advising Agent for Supporting Human-Multi-Drone Team
Collaboration}
%%% Provide names, affiliations, and email addresses for all authors.
  
\author{a}
\affiliation{
  \institution{Bar Ilan university}
  \city{ramat gan}
  \country{israel}}
\email{aa}

\begin{abstract}
Multi-drone systems have become transformative technologies across various industries, offering innovative applications. However, despite significant advancements, their autonomous capabilities remain inherently limited. As a result, human operators are often essential for supervising and controlling these systems, creating what is referred to as a \say{human-multi-drone team}. In realistic settings, human operators must make real-time decisions while addressing a variety of signals, such as drone statuses and sensor readings, and adapting to dynamic conditions and uncertainty. This complexity may lead to suboptimal operations, potentially compromising the overall effectiveness of the team. In critical contexts like Search And Rescue (SAR) missions, such inefficiencies can have costly consequences.
This work introduces an advising agent designed to enhance collaboration in human-multi-drone teams, with a specific focus on SAR scenarios. The advising agent is designed to assist the human operator by suggesting contextual actions worth taking. To that end, the agent employs a novel computation technique that relies on a small set of human demonstrations to generate varying realistic human-like trajectories. These trajectories are then generalized using machine learning for fast and accurate predictions of the long-term effects of different advice.
Through human evaluations, we demonstrate that our approach delivers high-quality assistance, resulting in significantly improved performance compared to baseline conditions.

\end{abstract}

\begin{CCSXML}
<ccs2012>
   <concept>
       <concept_id>10003120.10003145.10003151.10011771</concept_id>
       <concept_desc>Human-centered computing~Visualization toolkits</concept_desc>
       <concept_significance>300</concept_significance>
       </concept>
   <concept>
       <concept_id>10003120.10003145.10011770</concept_id>
       <concept_desc>Human-centered computing~Visualization design and evaluation methods</concept_desc>
       <concept_significance>100</concept_significance>
       </concept>
   <concept>
       <concept_id>10003120.10003121.10003122.10003334</concept_id>
       <concept_desc>Human-centered computing~User studies</concept_desc>
       <concept_significance>300</concept_significance>
       </concept>
   <concept>
       <concept_id>10003120.10003121.10003122.10010855</concept_id>
       <concept_desc>Human-centered computing~Heuristic evaluations</concept_desc>
       <concept_significance>300</concept_significance>
       </concept>
   <concept>
       <concept_id>10003120.10003121.10003129.10010885</concept_id>
       <concept_desc>Human-centered computing~User interface management systems</concept_desc>
       <concept_significance>300</concept_significance>
       </concept>
 </ccs2012>
\end{CCSXML}

\ccsdesc[300]{Human-centered computing~Visualization toolkits}
\ccsdesc[100]{Human-centered computing~Visualization design and evaluation methods}
\ccsdesc[300]{Human-centered computing~User studies}
\ccsdesc[300]{Human-centered computing~Heuristic evaluations}
\ccsdesc[300]{Human-centered computing~User interface management systems}

%%
%% Keywords. The author(s) should pick words that accurately describe
%% the work being presented. Separate the keywords with commas.
\keywords{Multi-Drone Systems, Human-in-the-loop, Advising Agent, SAR}

%%
%% end of the preamble, start of the body of the document source.
\begin{document}

%%
%% The "title" command has an optional parameter,
%% allowing the author to define a "short title" to be used in page headers.
\title{Advising Agent for Supporting Human-Multi-Drone Team Collaboration}

%%
%% The "author" command and its associated commands are used to define
%%
%% The abstract is a short summary of the work to be presented in the
%% article.








%%
%% This command processes the author and affiliation and title
%% information and builds the first part of the formatted document.
\maketitle

%
The increasing reliance on LLMs for multimodal tasks across far-reaching sectors such as healthcare, finance, and manufacturing underscores the need to assess the accuracy and reliability of the information they generate. Vision-Language Models (VLM) have achieved state-of-the-art (SoTA) performance on Visual Question-Answering (VQA) benchmarks, and these models often utilize Retrieval-Augmented Generation (RAG) to maintain factual accuracy and relevance in a dynamic information environment. However, this has led to uncertainty in the information the LLM bases its answer on, as it may choose between parametric memory and retrieved sources. When models rely on memorized information instead of dynamically retrieving information, they may inadvertently propagate outdated or incorrect information, causing serious legal and ethical risks and undermining trust and reliability in AI systems \citep{huang2023survey}.
% The ability to strike a balance between generalization and specialization in AI systems is therefore crucial for ensuring the safe, reliable use of these technologies in real-world applications.

Despite these concerns, the way that Vision-Language models (VLMs) memorize and retrieve information, particularly in complex multimodal tasks, remains under-explored. Current research often focuses on either the general capabilities of large language models (LLMs) or the specialized retrieval mechanisms in retrieval augmented generation systems (RAG) \citep{incontext_rag,chen_murag_2022,liu_universal_2023}. Particularly in the context of multimodal retrieval and multihop reasoning, few studies analyze the tradeoff between finetuning for specialized tasks and zero-shot prompting for general-purpose vision-language capabilities. A lack of consensus on how to approach this tradeoff motivates the development of measures to quantify reliance on parametric memory, as well as metrics for quantifying the potential performance impact of extending LLMs with RAG systems.

To address this gap, we investigate how multimodal QA models balance accuracy with memorization on the WebQA benchmark. We compare finetuned multimodal systems against zero-shot VLMs, analyzing how retrieval performance influences QA accuracy. In particular, we focus on cases where retrieval fails, allowing us to measure reliance on parametric memory through two proposed metrics---the \ppr (\PPR) which quantifies how much model accuracy is influenced by retrieval quality, contrasting performance in best-case versus worst-case retrieval scenarios, and the \ucr (\UCR) which measures how often correct QA responses are generated when the retriever fails, providing a proxy for memorization.

To enable this analysis, we make several methodological contributions. For the finetuned QA models, we investigate Vision-Transformer (ViT) architectures, which allow for multihop reasoning over multiple sources. To investigate the impact of retrieval performance on trained LMs, we propose a variable-input Fusion-in-Decoder (FiD) model \cite{tanaka_slidevqa_2023, nlvr2}, building upon the VoLTA architecture \citep{pramanick_volta_2023}. For the zero-shot case, we build upon previous research on In-Context Retrieval \citep{incontext_rag} by demonstrating that LLMs such as GPT-4o are capable of performing the final ranking step of the retrieval process. In doing so, we find that GPT-4o, a general-purpose LLM, achieves SoTA performance on the WebQA task, outperforming existing finetuned RAG models by a significant margin (7\% higher accuracy). 

Crucially, our results reveal that while retrieval-augmented models reduce memorization, the training paradigm plays an important role. Finetuned models exhibit higher reliance on parametric memory, whereas zero-shot RAG approaches have lower memorization scores at the cost of accuracy. This suggests that while retrieval modules may mitigate the risks associated with outdated or incorrect information, SoTA performance requires that they be coupled with specialized QA models. Our memorization measures contribute to the development of transparent and reliable AI systems, particularly in applications where the sourcing of up-to-date, factual information is critical.



% We investigate the impact of question complexity on the ability of these models to integrate multiple data sources—such as images, text, and external retrievers—and produce coherent and accurate answers. We also explore whether in-context retrieval can be a viable alternative to traditional retrieval-augmented systems, offering a more streamlined approach to multimodal QA.

% To achieve this, we first compare zero-shot prompting multimodal LLMs with finetuned multimodal systems. We evaluate both types of models on the WebQA benchmark, a dataset designed for complex question answering that requires reasoning across both image and text sources. For the finetuned models, we use a Fusion-in-Decoder (FiD) architecture, which allows for multihop reasoning over multiple sources. Additionally, we introduce the concept of In-Context Retrieval Language Modeling (RLM), where the LLM itself performs retrieval tasks without the need for external retrievers. This method builds upon existing research in in-context learning  and aims to explore the viability of LLMs retrieving relevant sources and generating accurate answers directly from their context window.

% In order to investigate source utilization in finetuned multimodal models and LLMs, three lines of inquiry are established; 
% \begin{itemize}
%     \item Study 1: retrieval vs QA performance on webQA (motivating example, does QA answer correctly even with incorrect sources?)
%     \item Study 2: performance on adversarial examples where parametric knowledge would be incorrect by design
%     \item Study 3: improving performance on adversarial examples by fine-tuning (i.e model robustness)
% \end{itemize}

% Note, there is one weakness in this plan which is tying in the work we've already done. 
% If we added something from adversarial generation to the retrieval experiment (like a combination of study 1 + 3) it would be complete. So for instance we could try fine-tuning the retriever with adversarial examples (and not just the QA model)

% \begin{figure}
%     \centering
%     \includegraphics[width=0.95\linewidth]{figures/segmentation/webqa_segment_infill.png}
%     \caption{Example of the segmentation substitution pipeline from the WebQA task.}
%     % d5c76d760dba11ecb1e81171463288e9
%     \label{fig:seg_sub_pipeline}
% \end{figure}



% Retrieval augmented generation (RAG) with zero-shot prompting and fine-tuning Large Language Models (LLMs) have become the go-to methods for tasks relying on information retrieval and text generation. In many cases the LLMs parametric memory can sufficiently generalize to answer questions without being provided with retrieval mechanisms for out-of-domain knowledge. However, LLMs often hallucinate and provide wrong information in certain scenarios. This problem is amplified even further on open-domain Question Answering (QA) tasks involving multiple modalities. Grounded text generation using retrieved sources \citep{lewis2021retrievalaugmented} has been extensively studied for text-to-text QA tasks, but its application in multimodal settings has not been studied as much.


% Multimodal reasoning and question answering have gained prominence in recent research endeavors, with an increasing emphasis on handling various forms of data, particularly text and images. In this study, we address a specific gap in the existing literature by focusing on the development of a versatile multihop model capable of accommodating varying numbers of input images.

% Our motivation for this research lies in the growing complexity of answering questions using information on the web, where the challenge of navigating the open-domain setting is further complicated by the presence of multiple modalities and sometimes requires reasoning over multiple sources. WebQA is an ideal dataset on which to compare performance of finetuned RAG systems against general purpose LLMs; it is multimodal, with correct answers requiring reasoning over image and text sources. It is multihop, requiring a complex reasoning process over multiple sources. Finally, WebQA questions from different categories can be broken down into subdomains to analyze performance over domains of varying cardinality.

% Motivated by the real-world challenges of building retrieval and question answering (QA) systems, we design and finetune a closed domain, multimodal, multihop QA model, that is capable of reasoning over a varying number of sources taken as input from an external retriever module. This research contributes to the relatively underexplored domain of multihop reasoning across various input sources and modalities. Our goal is to explore the challenges posed by these scenarios and develop strategies that enable QA models to retrieve relevant information, conduct logical or numerical reasoning across diverse modalities, and generate coherent responses in natural language. To our knowledge, this is the first application of the Fusion-in-Decoder (FiD) architecture \cite{tanaka_slidevqa_2023, nlvr2} that is shown to work with a variable number of inputs, enabling multi-hop reasoning over sources.

% In-Context Learning refers to the ability of LLMs to perform any task by simply providing examples in the input prompt \citep{dong2022survey,min2022rethinking}. Inspired by this research, we propose a method to use the LLM itself as a multimodal retriever, potentially eschewing the requirement of a distinct retrieval module, thereby allowing the design of simpler retrieval-augmented QA systems. We dub this method In-Context Retrieval Language Modeling (RLM). To the best of the authors knowledge, In-Content RLM is disparate from other retrieval augmented approaches which utilize external retrieval modules \citep{incontext_rag,chen_murag_2022,liu_universal_2023}. Despite being a natural extension of In-Context learning, In-Context RLM has not yet been studied empirically.

% To expand on our contribution of In-Context Retrieval, this stems from the well-researched in-context learning of LLMs. In-context learning is the ability of a model to perform any task given a sufficient context window \citep{dong2022survey,min2022rethinking}. Such tasks could include retrieval and ranking, but typically, the go-to solution for tasks requiring retrieval has been RAG. To the best of the authors knowledge, In-Context Retrieval is distinct from In-Context Retrieval Augmented Language Modelling (RALM), and despite being a natural extension of In-Context learning, In-Context Retrieval has not yet been shown empirically.

% Finally, we explore the tradeoff between using zero-shot prompting LLMs and the fine-tuning approach. While we find that, overall, GPT-4o obtains SoTA performance on the WebQA task, outperforming the accuracy of existing finetuned RAG approaches by 7\%, finetuned approaches still perform better on more restricted subdomains\footnote{``In-Context RLM" @ \url{https://eval.ai/web/challenges/challenge-page/1255/leaderboard/3168}}. Finally, we validate that GPT-4o is relying on retrieval abilities to solve the task; we find that GPT-4o is capable of retrieving relevant sources in the presence of distractors and furthermore, when GPT-4o fails to retrieve correct sources, it answers incorrectly 75\% of the time, meaning that it is not relying on parametric memory for this task.

% \paragraph{Contributions}
% Based on our experimentation and analysis on the WebQA benchmark, we make the following contributions:
% \begin{itemize}
%     \item Propose a new architecture for multimodal multihop QA that takes variable number of input sources inspired by the Fusion-in-Decoder method.
%     \item Comparison of general purpose LLMs vs specialized models on the WebQA benchmark.
%     \item Observation of In-Context Multimodal Retrieval abilities of GPT-4o and that it does not rely on parametric memory for multimodal QA.
%     \item Analysis of relationship between retrieval and QA task performance.
%     \item Analysis of task and query complexity on the performance of retrieval and QA tasks.
% \end{itemize}
















% Throughout this paper, we will present our methodology, experiments, and findings, emphasizing our approach to multihop reasoning over varying numbers of input images. We believe that our work contributes to a deeper understanding of multimodal reasoning and has the potential to enhance the capabilities of question-answering systems in the intricate, multimodal landscape of web-based information.
 \onecolumn
\section*{Advice Provision}
\label{sec:formal}
\Large

\subsection*{Formalization}
We formally introduce the advice provision problem as a Markov Decision Problem (MDP) $<S,A,P,R,\gamma>$ \cite{Puterman1994}.

Let us consider a set of $n$ semi-autonomous drones engaged in a cooperative task, supervised and controlled by a single human operator, $\operator$.
The state space $S$ consists of all contextual information regarding the drones (e.g., status, altitude) and the task's and operator's domain-specific characteristics (e.g., time elapsed, performance). The operator can perform an action $a\in A$ during the task, at any time $t\in [0,\ldots T]$. Since $\operator$ can choose to execute no action at any given $t$, $NULL\in A$. In addition, not every action is possible at each state, and therefore, we denote $A(s)$ as the set of applicable actions at state $s$.
Let $\Pfunc{\stateS}{\action}{\stateS'} \ $
be the transition function that denotes the probability of transitioning from state $\stateS$ to state $\stateS'$ following action $a$.
Clearly, for each $\stateS \in S$, 
$\sum_{s' \in S} \Pfunc{\stateS}{\action}{\stateS'}=1$.
Note that, when $\action = NULL$, it holds that $\stateS \not= \stateS'$ since the environment is dynamically changing.  
Finally, let $R(s)$ be the domain-specific reward function,
%hodaya: i remove this: $\Omega$ be the set of observations, $O$ be the domain-specific observation function, 
and $\gamma\in[0,1]$ be the discount factor.
Importantly, performing an action takes time, depending on the operator's ability. The transition function and the cost function are unknown to the advising agent, yet they can be estimated through observations. %hodaya: Use observations here?

Advice is guidance provided by an agent to the human operator as an action that the operator should take ($a \in A$). 
Crucially, the operator maintains the autonomy to assess the advice, consider alternative actions, and make a final decision (i.e., the advice is non-binding). At any point in time $t$, the agent may provide advice according to its advising policy $\pi$, a mapping from states to actions.
In an ideal setting, we would like the operator to follow an optimal policy, $\pi^*_\operator: S \rightarrow A$, that maximizes the expected accumulative future reward. However, the dynamic and uncertain environment causes the underlying optimization problem to be intractable and thus, an optimal policy cannot be computed in a reasonable time. To overcome this limitation, we propose a methodology for computing high-quality advice without explicitly deriving an optimal policy. 

\subsection*{Proposed Methodology}

We propose an advice provision methodology consisting of two phases: an offline phase which is targeted at estimating the expected long-term reward of performing an action (i.e., reward estimation) using limited human demonstrations and a machine learning model; and an online phase, primarily designed for providing a suitable advice in a given state.   


% adopt the 1-step-lookahead rolling-window heuristic which was found to be highly useful in similar advice provision settings  \cite{Rosenfeld2017collaboration}. %Ariel: can you please cite some other works as well? possibly not from robotics...
% Specifically, at each time $t$, the agent outputs $a$ which minimizes (maximizes) the expected reward \textit{assuming the agent can only suggest a single piece of advice}.
\newpage

\noindent{\bf Offline: Reward Estimation Model}


As noted before, a central component in solving the underlying optimization problem is the proper estimation of the expected benefit from performing an action. To that end, we assume that a set of demonstrations is available, $D$,  where $d\in D$ is a trajectory $\trajectory$ consisting of state-action pairs, $((s_0,a_0),...,(s_T,a_T))$, denoting the actions that the operator took at each time and state. If $D$ is \say{sufficiently large and comprehensive}, then a machine learning model could be readily trained based on these demonstrations to approximate the effects of performing an action at a given state and time. However, as noted before, collecting such a set of demonstrations is typically highly expensive and time-consuming. Therefore, we assume $D$ is small and needs to be extended before it can be effectively used for training a machine learning model.We propose the generation of high-quality \textit{synthetic} trajectories that mimic the real demonstrations in an offline fashion (i.e., before actual deployment of the advising agent). 
In other words, we start by estimating the time it takes for a given operator to perform different actions using the entire trajectory, $\Tilde{C}_h(a)$ (e.g., using an average). Then, we go through a real trajectory ($d\in D)$ and pertubate it to create slight changes (e.g., choosing another action at a given probability). The action is then applied in a simulated environment using the transition function (e.g., the transition can also consider the operator's performance), a new state is reached and the process is repeated. Overall, $k$ synthetic trajectories are generated based on each real one. 
The resulting set of generated trajectories is then checked to ensure that it, indeed, closely mimics the real ones (i.e., quality assurance using statistical measures). If so, a machine learning model is trained on the \textit{extended} set of demonstrations (otherwise, an \say{unsuccessful} message is raised). 
The trained model is utilized next in the online phase to rank the various actions.

\noindent{\bf Online: Advice Provision}
During deployment, the agent employs an online advice provision policy as follows:  \\
%\begin{enumerate}
{1. \bf Action generator}: Given state $s$, the agent generates a set of possible actions that can be performed at this state and time point. These actions may be the result of a drone-initiated event (e.g., a drone malfunction) and/or other actions aimed at improving performance (e.g., changing a drone's altitude for better coverage). It then evaluates them using the reward estimation model obtained in the offline phase. \\
{2. \bf Ranking}: All generated actions and their associated expected reward are provided to a ranking model that outputs the top $c$ actions as advice for the operator. See Figure \ref{fig:agent_structure} for an illustration. 
% (complex actions is a set of some actions from the same type).
%\end{enumerate}
 
In other words, once a new state is encountered, the action generator creates possible actions to take which are then evaluated using the offline trained model and ranked accordingly.   

% Once a reward estimation function is trained using Algorithm \ref{alg:generation}, the advising agent can use it in the online setting. 

% Another use of the simulated runs is for model validation. Start a simulated run and stop at a certain point. Then, from that point onward, run several simulated runs, each one starting with a different action but continuing with the same method of choosing actions until the end of the run. This way, we can determine which action leads to the best outcome. Since randomness is inherent in the environment itself, it's necessary to run several simulated runs for each action and statistically check if there is a recommended action at that time point and whether the model indeed tends to give it a higher value.

\begin{figure}[t]
\centering
\includegraphics[width=0.9\columnwidth]{media/agent_general_structure.png}
\caption{The advising agent design. $s$ denotes the state, $a_i\in A$ denotes an action, $s'$ denotes the expected resulting state and $v_i$ denotes the predicted reward from the transition.}
\Description{Diagram of the advising agent's design, showing states, actions, resulting states, and predicted rewards.}
\label{fig:agent_structure}
\end{figure}

%Once we have the prediction model, the agent can be implemented. 
%\section{Advice Provision}
\label{sec:formal}

\subsection{Formalization}
Next, we formally introduce the advice provision problem as a Markov Decision Problem (MDP) $<S,A,P,R,\gamma>$ \cite{Puterman1994}.

Let us consider a set of $n$ semi-autonomous drones engaged in a cooperative task, supervised and controlled by a single human operator, $\operator$.
The state space $S$ consists of all contextual information regarding the drones (e.g., status, altitude) and the task's and operator's domain-specific characteristics (e.g., time elapsed, performance). The operator can perform an action $a\in A$ during the task, at any time $t\in [0,\ldots T]$. Since $\operator$ can choose to execute no action at any given $t$, $NULL\in A$. In addition, not every action is possible at each state, and therefore, we denote $A(s)$ as the set of applicable actions at state $s$.
Let $\Pfunc{\stateS}{\action}{\stateS'} \ $
be the transition function that denotes the probability of transitioning from state $\stateS$ to state $\stateS'$ following action $a$.
Clearly, for each $\stateS \in S$, 
$\sum_{s' \in S} \Pfunc{\stateS}{\action}{\stateS'}=1$.
Note that, when $\action = NULL$, it holds that $\stateS \not= \stateS'$ since the environment is dynamically changing.  
Finally, let $R(s)$ be the domain-specific reward function,
%hodaya: i remove this: $\Omega$ be the set of observations, $O$ be the domain-specific observation function, 
and $\gamma\in[0,1]$ be the discount factor.
Importantly, performing an action takes time, depending on the operator's ability. The transition function and the cost function are unknown to the advising agent, yet they can be estimated through observations. %hodaya: Use observations here?

Advice is guidance provided by an agent to the human operator as an action that the operator should take ($a \in A$). 
Crucially, the operator maintains the autonomy to assess the advice, consider alternative actions, and make a final decision (i.e., the advice is non-binding). At any point in time $t$, the agent may provide advice according to its advising policy $\pi$, a mapping from states to actions.
In an ideal setting, we would like the operator to follow an optimal policy, $\pi^*_\operator: S \rightarrow A$, that maximizes the expected accumulative future reward. However, the dynamic and uncertain environment causes the underlying optimization problem to be intractable and thus, an optimal policy cannot be computed in a reasonable time. To overcome this limitation, we propose a methodology for computing high-quality advice without explicitly deriving an optimal policy. 

\subsection{Proposed Methodology}

We propose an advice provision methodology consisting of two phases: an offline phase which is targeted at estimating the expected long-term reward of performing an action (i.e., reward estimation) using limited human demonstrations and a machine learning model; and an online phase, primarily designed for providing a suitable advice in a given state.   


% adopt the 1-step-lookahead rolling-window heuristic which was found to be highly useful in similar advice provision settings  \cite{Rosenfeld2017collaboration}. %Ariel: can you please cite some other works as well? possibly not from robotics...
% Specifically, at each time $t$, the agent outputs $a$ which minimizes (maximizes) the expected reward \textit{assuming the agent can only suggest a single piece of advice}.
\noindent{\bf  Reward Estimation Model}

\begin{algorithm}[hbpt!]
\caption{Reward Estimation}\label{alg:generation}
\begin{algorithmic}[1]
\REQUIRE $k,\theta,T,D$
\STATE $D_{syn} \gets \emptyset$
\FORALL{$\textit{d} \in D$}   
        \STATE Estimate $\Tilde{C}_h(a)$ based on $d$
        % \stateS Let $\Tilde{P}_h(a)$ be operator's success probability (e.g $TP, TN$)
        % \stateS $TS \gets$ Special action scheduling from \textit{d}
	\FOR{$i \gets 1$ to $k$}
        \STATE $t\gets 0$ \ \  $\trajectory \gets \emptyset$
        \ \ \ $s_0 \gets d_0[0]$ \label{S0}
        \WHILE{$t < T$}
            % \stateS $a_t\gets d_t[0]$
            \STATE $A_t \gets A(s_t)$ 
            \STATE $a_t \gets Pertubate(A_t,d,t)$ \label{Pertubate}
             \STATE $s_{t+\Tilde{C}_h(a)} \gets GetNewState(s_t,a_t)$ \label{GetNewState}
             \STATE $t\gets t+\Tilde{C}_h(a)$
             \STATE $\trajectory \gets \trajectory \cdot \{(s_t,a_t)\}$
        \ENDWHILE
        \STATE $D_{syn}\gets D_{syn}\cup\trajectory$
\ENDFOR
        \ENDFOR
        \IF {$QualityAssurance(D_{syn},D,\theta)$} \label{QualityAssurance}
            \STATE $M\gets Train(D_{syn},D)$
            \RETURN $M$
        \ELSE
            \RETURN $Unsuccessful$
        \ENDIF

%Now check the similarity 

\end{algorithmic}
\end{algorithm}

As noted before, a central component in solving the underlying optimization problem is the proper estimation of the expected benefit from performing an action. To that end, we assume that a set of demonstrations is available, $D$,  where $d\in D$ is a trajectory $\trajectory$ consisting of state-action pairs, $((s_0,a_0),...,(s_T,a_T))$, denoting the actions that the operator took at each time and state. If $D$ is \say{sufficiently large and comprehensive}, then a machine learning model could be readily trained based on these demonstrations to approximate the effects of performing an action at a given state and time. However, as noted before, collecting such a set of demonstrations is typically highly expensive and time-consuming. Therefore, we assume $D$ is small and needs to be extended before it can be effectively used for training a machine learning model. As detailed in Algorithm \ref{alg:generation}, we propose the generation of high-quality \textit{synthetic} trajectories that mimic the real demonstrations in an offline fashion (i.e., before actual deployment of the advising agent). 
In words, we start by estimating the time it takes for a given operator to perform different actions using the entire trajectory, $\Tilde{C}_h(a)$ (e.g., using an average). Then, we go through a real trajectory ($d\in D)$ and pertubate it to create slight changes (e.g., choosing another action at a given probability). The action is then applied in a simulated environment using the transition function (e.g., the transition can also consider the operator's performance), a new state is reached and the process is repeated. Overall, $k$ synthetic trajectories are generated based on each real one. 
The resulting set of generated trajectories is then checked to ensure that it, indeed, closely mimics the real ones (i.e., quality assurance using statistical measures). If so, a machine learning model is trained on the \textit{extended} set of demonstrations (otherwise, an \say{unsuccessful} message is raised). 
The trained model is utilized next in the online phase to rank the various actions.

\noindent{\bf Advice Provision}
During deployment, the agent employs an online advice provision policy as follows:  \\
%\begin{enumerate}
{1. \bf Action generator}: Given state $s$, the agent generates a set of possible actions that can be performed at this state and time point. These actions may be the result of a drone-initiated event (e.g., a drone malfunction) and/or other actions aimed at improving performance (e.g., changing a drone's altitude for better coverage). It then evaluates them using the reward estimation model obtained in the offline phase (i.e., Algorithm \ref{alg:generation}). \\
{2. \bf Ranking}: All generated actions and their associated expected reward are provided to a ranking model that outputs the top $c$ actions as advice for the operator. See Figure \ref{fig:agent_structure} for an illustration. 
% (complex actions is a set of some actions from the same type).
%\end{enumerate}
 
In other words, once a new state is encountered, the action generator creates possible actions to take which are then evaluated using the offline trained model and ranked accordingly.   

% Once a reward estimation function is trained using Algorithm \ref{alg:generation}, the advising agent can use it in the online setting. 

% Another use of the simulated runs is for model validation. Start a simulated run and stop at a certain point. Then, from that point onward, run several simulated runs, each one starting with a different action but continuing with the same method of choosing actions until the end of the run. This way, we can determine which action leads to the best outcome. Since randomness is inherent in the environment itself, it's necessary to run several simulated runs for each action and statistically check if there is a recommended action at that time point and whether the model indeed tends to give it a higher value.

\begin{figure}[t]
\centering
\includegraphics[width=0.9\columnwidth]{media/agent_general_structure.png}
\caption{The advising agent design. $s$ denotes the state, $a_i\in A$ denotes an action, $s'$ denotes the expected resulting state and $v_i$ denotes the predicted reward from the transition.}
\label{fig:agent_structure}
\end{figure}

%Once we have the prediction model, the agent can be implemented. 
\section{Search and Rescue}
\label{sec:sar_mission}


%Goal
We concentrate on the multidrone-based SAR task with a single human operator supervising and controlling a fleet of $\DronesNum$ drones in search of an unknown number of targets in a large area (e.g., searching for victims after an earthquake). The use of multiple drones working together as a team can reduce the time required to locate persons in distress, offering significant advantages over traditional methods \cite{hoang2023droneswarms}.  The typical overarching goal of such tasks, as in our setting, is to maximize the number of targets found during the mission's fixed time.
For a comprehensive review of the current state and challenges of drone applications in disaster management, including search and rescue, see~\cite{daud2022applications}.


%Foundations from du
Mission planning for multiple drones involves complex strategies to ensure efficient coverage and task completion \cite{song2023survey}.
We build on top of the drone-based SAR task modeling provided by \cite{du2019evolutionary}. Specifically, we assume that the entire search zone is known in advance and is partitioned into smaller sub-areas, denoted by $E$ the set of the sub-areas. Each sub-area is assigned a probability indicating the likelihood of finding at least one target within that specific sub-area. Assuming no prior knowledge, all sub-areas are given the same likelihood. Nonetheless, these probabilities may change dynamically based on the real-time information provided by the drones and/or manually by the operator (e.g.,  based on gathered intelligence that occasionally provided to the user through \say{intelligence messages}). In turn, these probabilities can influence the operator's decision to assign drones to specific sub-areas.
% \footnote{In our system, the allocation algorithm is designed to assign a search area and an associated altitude for each drone to scan that area. However, when the operator manually adjusts the altitude, our algorithm respects this decision and does not output a new altitude. This ensures that the operator's on-the-fly adjustments are prioritized and the drone continues to operate at the operator-modified altitude.} 
%Drones are assigned to different sub-areas based on a complex algorithm (see Appendix~\ref{CentralizedAlgorithm} for more details).

%Our system environment: 
In our environment, drones scan their designated sub-areas following a lawn mower pattern. The scanning process is governed by four parameters - the drone's velocity, altitude, and two thresholds to determine the minimal confidence for a low-confidence suspected target and a high-confidence suspected target. Specifically, during the scanning phase, the drones' cameras capture images rapidly. These images are then processed through a neural network (NN) to produce bounding boxes and associated confidence indicating the presence of potential targets. To that end, a Retina net \cite{retina:20} NN model is trained using the approach provided in \cite{AIR:21} on the Heridal dataset \cite{heridal-lrbkc_dataset} and subsequently fine-tuned using manually tagged images from our simulated environment. It was shown that drones have difficulties identifying victims in SAR in the wild. Manzini and  Murphy \cite{manzini2023open} demonstrated that despite achieving performance that is statistically equivalent to the state-of-the-art on benchmark datasets, the models they tested fail to translate these achievements to the real world in terms of many false positives (e.g., identifying tree limbs and rocks as people), and false negatives (e.g., failing to identify victims). Similar problems were observed in our SAR environment in preliminary testing. 
%hodaya: need to add fp and fn
Therefore, in our environment, the human operator needs to approve a suspected target. To balance between false alarms and missed detections, we define two thresholds: low-confidence (\pausetreshold) and high-confidence (\detectiothreshold). If the confidence of the NN exceeds \detectiothreshold, the drone stops, and an alert with the associated confidence is sent to the operator. If the confidence only surpasses the \pausetreshold, the drone further scans around the suspected object, and an alert with a timeout and the (relatively low) confidence of the detection is sent to the operator, allowing her to observe the object if she has the time. If no high confidence is achieved during further scanning, the drone dismisses the detection; otherwise, it elevates the alert to a high-confidence one and waits for the operator's handling of the event.
The drone's altitude, velocity, and the two confidence thresholds are determined by the operator based on the area's characteristics, existing intelligence, and the operator's capabilities, to name a few.
For example, highly dense sub-areas (e.g., forest) may be better assigned different parameters than sparse sub-areas (e.g., desert). The operator can determine the parameters of a given sub-area by designating its so-called \say{area type} as \say{high}, \say{medium}, or \say{low} difficulty area. Complementary, the operator can manually select the parameters for each area. Note that at the beginning of the simulation, the operator is required to set all these parameters (or confirm the provided default values).  
Finally, drones may have simulated malfunctions, in such cases, an alert is generated and requires the operator's attention.  

%Human operator





% Additionally, the operator  has the authority to adjust the parameters of the drones, such as altering altitude or speed. Furthermore,  the operator can modify the search parameters, including areas' probabilities of finding targets and the assignment of drones to areas. 
%By leveraging their expertise and making informed decisions, the human operator significantly influences the search mission's outcomes, ultimately increasing the chances of locating more targets in a timely and efficient manner.

Overall, the set of actions that the operator can take, $A$, consists of five types of actions: 
(i) Changing the assumed probability of finding (additional) targets within any specific sub-area; (ii) Changing the area type of a given sub-area, denoted by $CT$; (iii) changing the scanning parameters associated with an area (i.e., altitude, velocity, and thresholds), denoted by $CP$;  (iv) Manually flying a drone that is suspected of being stuck, manually reporting on a target (without relevant alerts), and manually assigning a specific drone to a sub-area (v) Handling alerts such as detection, malfunctions, and intelligent messages (note that base on the information in the intelligent message the operator may decide to change the probabilities of the sub-areas or manually change the assignment of drones to sub-areas).

%Ariel: consider adding a sentence saying that the simulation is high-resolution and realistic (i.e., wind is simulated, trees are moving.... and as such, it is non-deterministic...
The simulation is high-resolution and realistic, incorporating advanced features such as simulated wind, dynamic movement of trees, and real-time imaging from drones, as can be seen in a short illustration video.\footnote{\url{https://youtu.be/W-HHF8s2O8c}}  
% This realism introduces non-deterministic outcomes; for instance, handling a detection alert could result in either approving or rejecting a suspected target. In addition to the uncertainties arising from the environment itself, such as the number of incoming alerts, the possibility of drones getting stuck, and other factors.  
%move to 4 
%In our experiment an advice can be either a simple action ($a \in A$) or a complex action, that is, a sequence of some actions from $A$ of same type. 
%hodaya: need to edit the connection
%\sarit{I suggest to move this before the state observation because you use it previously and then you return } \odaya{done}
%Let $\Pfunc{\stateS}{\action}{\stateS'} \ $ be the transition function that denotes the probability that doing the action $a$ at state $\stateS$ will lead to state $\stateS'$. For all $\stateS \in S$, $\sum_{s' \in S} \Pfunc{\stateS}{\action}{\stateS'}=1$.
%The transition model captures the uncertainty in the system's dynamics and is calculated using assumptions and probabilities. It describes the probability of the system transitioning to a new state given the current state and the action taken by the operator.
%Finally, let $R(s,a)$ be the reward function which refers to the number of targets found while doing action $a$ at state $s$. %[encouraging the agent to find targets efficiently].
%this is the reward we chose in order to encourage the agent to find targets %efficiently, encouraging exploration of the area, and encouraging accurate %workload on the operator.
%represents the operator's preference for immediate rewards versus future rewards. A higher discount factor places more emphasis on immediate rewards, while a lower one favors long-term planning.
%A policy $\pi$ is a function from state $s$ to a complex action $a^v\subset A$.
%In the ideal case, the agent computes the optimal policy for the specific operator, $\pi^*_\operator: S \rightarrow A$, that maximizes the expected accumulative future reward. However, the uncertainty of the environment, the exponential size of the state space, $S$, and the combinatorial size of the action set, $A$, cause this problem to be intractable.
%We trained a model, that given a state return a value that is a predicted accumulate rewards along the game's trajectory.
%
%\label{DetectionModel}
%[Finalized - Sarit please review]
% How detect? threshold?X2 stacked, generalization.
% SAR Drones: ML based module to detect survivels -- better performance when cooperating with human.
%[todo: remove yolo. add FP  and TP here and on heridial in AIR. check what was according AIR and if it was simply retina net at the end (i think we remove their additional to the retina net, we use their advice on the details such as backbone)]
%\subsection{Detection Model}\label{DetectionModel}
%
% They showed in the article () that the combination of an operator and the presentation of the marking achieves better toys, therefore also with us the operator is required to perform the task and we chose to present the square in order to help in the search
%Two different known models were implemented for this task and tested on the actual simulator to determine  the better model to use. after training YoloV5 and RetinaNet (AIR) on our data (trained first on Heridal followed by a training on the generated AirSim images), RetinaNet performed better giving better results for true positives and false positives. 
%\subsection{Allocation Algorithm}
% \label{AllocationAlgorithm}
%  Centralized algo for drone allocation and path planning: input: map divided to areas. based on probablity of finding surviers in a given area; how it is updated
%\subsection{Human Operator}
%[ Finalized - Sarit please review ]
%The role of the human operator in Search and Rescue (SAR) environments plays a pivotal role in the overall success of the mission. Their decisions and actions are of utmost importance in enhancing the efficiency of the drones' search operations. As a central figure, the human operator is responsible for providing essential instructions and guidelines to the drones, ensuring that they carry out their search tasks effectively. 
%\subsection{The User Interface}
%\label{sec:sar}
% In Figure~\ref{fig:sar_simulator}, the interface is displayed. Further details regarding the interface and the design choices made are provided in the Appendix.

% \begin{figure}[t]
% \centering
% \setlength{\belowcaptionskip}{-10pt}
% \includegraphics[width=0.9\columnwidth]{media/SAR_simulator.jpg} % Reduce the figure size so that it is slightly narrower than the column. Don't use precise values for figure width.This setup will avoid overfull boxes.
% \caption{The SAR simulator.}
% \label{fig:sar_simulator}
% \end{figure}

\subsection{The User Interface}
\begin{figure}[t]
\centering
\includegraphics[width=\columnwidth]{media/SAR_simulator.jpg} % Reduce the figure size so that it is slightly narrower than the column. Don't use precise values for figure width.This setup will avoid overfull boxes.
\caption{The SAR User Interface.}
\label{fig:sar_simulator}
\end{figure}



\begin{figure}[t]
    \centering
    \includegraphics[width=0.5\linewidth]{media/manual_control.jpeg}
    \caption{Manual control panel, during handling detection alert.}
    \label{fig:manual_control}
\end{figure}
\label{sec:sar}
\begin{figure}[t]
    \centering
    \includegraphics[width=\linewidth]{media/left_panel.png}
    \caption{The left panel contain four tabs: Drones, Areas, Status and Parameters.}
    \label{fig:left_panel}
\end{figure}

Similar to the goal described by Chen et al. \cite{chen2022multi}, our SAR system aims to provide the operator with comprehensive situational awareness while minimizing the need for low-level control of each drone. The \say{task mode} includes automated allocation of the sub-areas to drones, where each drone scans its designated sub-area based on the difficulty level defined for that sub-area. The drones search for targets and alert the operator to suspected targets.
In the \say{command mode}, our system offers the flexibility for the operator to manually assign sub-areas to specific drones, manually control a drone, or manually report targets as necessary.

The central part of the user interface has two modes: a map mode and a drone mode. 
In the map mode, the map is displayed, divided into sub-areas, showing the locations of the drones on the map.  Each drone is labeled with a number, as recommended in~\cite{hoang2023challenges}.
At the top, there are small images for each drone displaying what the drone sees (see Figure~\ref{fig:sar_simulator}). In the drone mode, a specific drone's view is shown in at the central area in a larger format, allowing the operator to manually control the drone and consider the specific details observed by the drone. For example, the operator can better handle a detected target in this mode (see Figure~\ref{fig:manual_control}).

The left-side panel contains four tabs (see Figure~\ref{fig:left_panel}):
\begin{itemize}
    \item \textbf{Drones} - This tab provides each drone a list of its assigned sub-areas, with the option to manually change it.
    \item \textbf{Areas} - This tab displays all sub-areas, allowing the operator to change the probability of finding a target in a specific sub-area and amend the sub-areas difficulty levels.
    \item \textbf{Status} - Controlling the simulation mode - choosing a scenario and switching between scanning and parameters phases. 
    \item \textbf{Parameters} - In this tab, the operator can set the parameters (altitude, velocity, and thresholds) for each area.
\end{itemize}


On the right-hand side, there is a panel containing the alerts from drones, which are displayed in light green, and other messages, such as those from the agent, which are displayed in light orange (see Figure~\ref{fig:sar_simulator}). 

In a preliminary experiment, we separated these messages into two different tabs: drone and agent messages. However, we noticed that this setup was less convenient for most operators given the high number of alerts from the drones. In particular, operators noted that they felt pressured to respond to them and often missed important messages from the agent. Therefore, in our ensuing human evaluation, we combined both types of messages into a single tab, but with different colors, providing the operator with a simpler way to see and distinguish the alerts and agent's messages. 


\section{Agent Implementation}

We instantiate the proposed advising agent  (Section \ref{sec:formal}) to the SAR task (Section \ref{sec:sar_mission}). To that end, we detail our implementation of the offline (reward estimation model) and online (advice provision) components.

\noindent{\bf\subsection{Reward Estimation Model}}
Next, we detail our implementation of Algorithm \ref{alg:generation}. 

For each $d\in D$, in order to select the initial state (Line \ref{S0} of Algorithm \ref{alg:generation}),  we use the parameters set by the operator at the beginning of the simulation and the initial assignment of drones to sub-areas.
For the $Pertuable(A_t,d,t)$ function, which varies the next action (Line \ref{Pertubate} of Algorithm \ref{alg:generation}) the action is selected through the following procedure: First, if in $d$ the operator performed an action of type $CP$ or $CT$ approximately at time $t$ (i.e., less than a minute away from $t$) then the action is selected. Otherwise, if $t=10min$, 
then we perform a random $CP$ or $CT$ change at probability $p = 1/|A'|$, where $A'$ is the set of actions we want to add at these points. Finally, we chose to handle one of the drone malfunction alerts (if it exists) by handling the highest confidence detection alert available (if it exists) or another alert at random. 

For $GetNewState(s_t,a_t)$ (Line \ref{GetNewState} of Algorithm \ref{alg:generation}), we simulate the effect of $a_t$ using our environment assuming it takes $\Tilde{C}_h(a) \forall{a\in A}$ time for the action to be executed. In our implementation, we define $\Tilde{C}_h(a)$ to be the average time it takes for the operator to perform an action $a$ in the entire trajectory. When handling a detection alert, we simulate whether the operator correctly or incorrectly handled it using her performance through the trajectory (i.e., using the true positive and true negative rates).
Note that since we run the simulator as in a real simulation, the changes in the state caused by the continued scanning of the drones are reflected in the next state in a natural manner. In other words, our synthetic simulation is restricted to automatically simulating the operator's actions.

For $QualityAssurance(D_{syn}, D, \Theta)$ and model training ($M$), we featurize the state-space such that each $s\in S$ is defined as follows:
\begin{itemize}[leftmargin=10pt]
\item Operator performance: 
$\Tilde{C}_h(a)$ that estimates the average time it took the operator to perform $a$ in the trajectory up to state $s$.  
\item Environment information: 
Remaining time for the SAR task, count of approved detection events, percentage of false detection alerts, count of current open alerts for each type of alert, lowest confidence thus far that led to detection, and count of actions performed by the operator.
\item Information for each area type:
The percentage of all sub-areas of that type that remained unsearched, the number of alerts generated from sub-areas of that type above and under current \detectiothreshold\ and  \pausetreshold\ until the current state, 
search parameters: \pausetreshold,  \detectiothreshold, \velocity, \altitude, and their last previous values (if exist). 
\end{itemize}



For $QualityAssurance(D_{syn}, D, \Theta)$, we first check the distance between synthetic trajectories and each of the real trajectories to ensure that the synthetic trajectories closely resemble the original trajectory they originated from.
Then, a clustering test of the synthetic and real trajectories is performed. This phase is used to ensure that each synthetic trajectory is assigned to the same cluster as the real trajectory and to make sure that the synthetic trajectories are roughly proportionally distributed among clusters. 
If the synthetic data quality assurance passes the required threshold, the algorithm trains a model. 
Then the quality assurance of the model is done through two perspectives: (1) Ensuring that the model accurately predicts the utility by leading to reasonably low MAE; and (2) Ensuring that the model brings about effective recommendations by aligning with actions known to perform well in hindsight through synthetic simulations (see \ref{subsub:UtilityFunction} for details).

\noindent\textbf{(1) Distances:}
% The first test checked the distances between the synthetic trajectories and the real trajectories. 
First, we sample three states from each synthetic and real trajectory: one roughly at the beginning, one roughly at the middle, and one roughly at the end. We calculate the squared distance between the synthetic trajectory and each real trajectory in $D$ and sort them from lowest to highest. If the original trajectory from which the synthetic one originated is ranked high (i.e.,  first or second) in the sorted set, the synthetic trajectory can be considered adequate.  

\noindent\textbf{(2) Clustering:}
As before, we first sample three states from each synthetic and real trajectory: one roughly at the beginning, one roughly at the middle, and one roughly at the end. 
%A K-means algorithm was applied, with predefined $k$ on  real  trajectories. Then a K-means algorithm was applied using both the real and synthetic trajectories with define the initial centroids to be the centroides from the first execution of K-means.
A K-means algorithm~\cite{jancey1966multidimensional, macqueen1967some,lloyd1982least, steinhaus1956division} was applied using both the real and synthetic trajectories. The proportion of synthetic trajectories that belong to the same cluster as their original one is returned (i.e., the higher the better).  

% \subsubsection{Utility Function}
\noindent{\bf{Utility Function}}
\label{subsub:UtilityFunction}
Given a trajectory $\trajectory$ of length $l$ and $s_j \in \trajectory$, the discounted accumulated reward of a given state $s_j$ until the end of the trajectory is $\bar{R}(s_j)=\sum_{i=j,s_i \in \trajectory}^{i=l} \gamma^{i-j} R(s_i)$.
Since the detection of a target is a sparse event and most rewards along a trajectory are zero, we define a utility function $\UFunc$ that will be used to train a model $\Model$  that will enable the agent to compare states better. 

Let $\parametersForU=\{\paramForU_1,...\paramForU_m\}$ 
be a set of parameters of states in $S$.
We aim at developing a utility function $U(s)=\wights_0 \bar{R}(s)+\wights_1 \psi_1(s)+...+\wights_m \psi_m(s)$.
Given a state $s$ and actions $a$ and $a'$, let $\bar{s}$ and $\bar{s}'$ be the expected states generated by the expected state generator, respectively.
Our goal is to train a model $M$ to estimate $U$ such that: 

(I) if $M(\bar{s}) \geq M(\bar{s}')$ then $\bar{R}(s) \geq \bar{R}(s')$.

We considered the following parameters for $\parametersForU$:
(a) the path scanned by the drones until the end of the trajectory (i.e., to encourage scanning as much area as possible); (b) The average waiting time for an alert; (c)  The ratio between the number of detection alerts until the end of the trajectory and the number of alerts the operator can handle given his cost function (i.e., to encourage a balanced workload);  (d) The number of false negatives or false positives; (e) The number of targets correctly found until now in the trajectory; and (f)  The number of targets correctly found until the end of the trajectory. 


To train a  model $M$ that attempts to satisfy condition  (I), the following three steps are performed  repeatedly:
\begin{enumerate}[nosep]
    \item Choose possible weights for $U$.
    \item Train a model $M$ that accurately estimates $U$.
    \item Evaluate the ability of $M$ to lead to effective action recommendations.
\end{enumerate}
Testing (2) is done by computing the MAE  (see \ref{modules_evaluation}). 
Testing (3), is done using a set of synthetic trajectories generated specifically for these tests (see \ref{sec:action_ordering}).
Once the parameters of $U$ have been established and $M$ has undergone training, the agent is presumably capable of generating effective actions for the operator.


\noindent{\bf\subsection{Advice Provision}}

Next, we detail our implementation of the action generator $(\AG)$ and the ranking model $(\RM)$.
The $\AG$ component uses the expected state generator to anticipate the outcome following the execution of an action, and then the reward estimation model, derived in the offline phase, to evaluate the effect of the action in terms of expected reward.
Every predetermined number of seconds, the agent activates the $\AG$ and the $\RM$ in order to produce $\adviceNum$ actions that are expected to be the most promising given the current state. In addition, every incoming alert from the drones activates the $\RM$ once again in order to determine whether the new alert should result in a different ranking. 

The process of generating possible actions involves two types of actions (corresponding to the actions available in the system): First, an action that is not associated with an alert generated by the system, i.e.,
(i) Changing the area type of a given sub-area, denoted by $CT$; (ii) changing the scanning parameters associated with an area (i.e., altitude, velocity, and thresholds), denoted by $CP$; (iii) Handling intelligent messages;  (iv) Detecting a drone that is not progressing and may be stuck (and has not sent an alarm).
The second type of actions are actions that are directly associated with alerts generated by the system, i.e., Handling alerts such as detection, with low-confidence (\pausetreshold) or with high-confidence (\detectiothreshold) and the malfunction of a drone. The action generator uses the expected state generator to generate the expected state and then uses the reward estimation model. 

Advice can be a simple action, e.g., increasing the parameter $\pausetreshold$ by $5$ for area type \say{low} or changing the area type of a given sub-area to \say{high}. In our implementation, we further allow for complex actions to be proposed, e.g., a list of changes for the same parameter for each area type or a list of changes of the area type for several sub-areas. 
%increase the parameter $pt$ for area type low and decrease  the parameter $pt$ for area type medium.
Generating and evaluating complex actions leads to a combinatorial challenge.
Hence, a two-phase algorithm is proposed: initially, simple actions are generated and assessed; subsequently, the most promising ones are combined into complex actions, $a_i^v\subset A$.
For example, consider changing the parameter $\pausetreshold$.  
In the first phase, for each area type, we consider changing $\pausetreshold$ by    $\pm0.5x, \pm x$ and $\pm2x$ where $x$ denotes a predefined value or the mean change that the operators performed for $\pausetreshold$ in $D$. The expected state generator is then used for these possible changes followed by the reward estimation model. 
In the second phase, the agent considers adjusting the $\pausetreshold$ values for all area types by modifying the existing $\pausetreshold$ values according to the result of the first phase. the best value identified in the initial phase. The action generator uses the expected state generator in order to generate the expected state if these changes were to be performed considering the expected time it would take for the operator to change these parameters. 
% , and then use the reward estimation model. 

%We propose a strategy based on a greedy approach, wherein we first assess the effectiveness of individual actions $a_i\in A$. Subsequently, having identified advantageous actions of same type, we exclusively explore their combination $a_i^v\subset A$. In addition, the $CP$ action can be  performed with a large number of values, we tested only $3$ possibility values for increasing and decreasing. 
%hodayam - add hierarchical search.

%\subsection{Fictional State Generator}%new2
%hodaya - check the  State Generator (if will be time)


% \subsubsection{Ranking Model}%new4
% \label{ranking_model}
%The AA process of determining the ranking order for alerts from drones employed an automated simulations with various mimic-operator scenarios, each governed by different rules. 
% Note that our reward estimation model does not explicitly consider the expected reward of handling intelligent messages and addressing alerts from stuck drones.
% The ranking of the alerts from the drones is based on the confidence of each alert.
% Handling intelligent messages and stuck drone alerts get the highest priority. 
We rank the actions provided by the action generator according to their estimated reward and provide the three top-ranking ones to the operator (negatively estimated actions are not provided). Tie-breaking favors the handling of detection alerts over others and alerts with higher confidence over others as a secondary criterion. 
The operator has the option to perform an action that is not advised by the agent. It is important to note that some actions are typically based on information the agent cannot access, such as changing the probability of a sub-area, which requires interpreting intelligence messages and understanding the map. Therefore, the operator may decide to perform such actions independently.

 % If the estimated value of the drone-generated alert is the highest, then present to the operator only drone-generated alerts. Otherwise, sort the tasks according to their value and propose the best ones to the operator. 


% The motivation to rank the detection and pause alerts according to their confidence was based on experiments that compared FIFO with the highest precision, resulting in no statistical difference between them. Analyzing operators' behavior yields that precision is a commonly used criterion for choosing which alert to look at.


In our implementation, the expected state generator component is designed to predict the next state in a way that reflects the main significance of the action. The \textbf{operator performance} remains unchanged. The \textbf{environmental information} is modified as follows: the remaining time for the task is updated based on $\Tilde{C}_h(a)$; the count of approved detection events is adjusted if the action handles a detection alert, reflecting the percentage of approved events up to the current state; the percentage of false alerts remains unchanged; the current open alerts decrease by one if the action handles a detection alert; the lowest confidence that led to detection remains unchanged; and the count of actions performed by the operator increases by one, depending on the action type.
\textbf{Information for each area type} is also considered: the percentage of unsearched sub-areas remains unchanged, the number of alerts generated from sub-areas above and below the current \detectiothreshold\ and \pausetreshold\ is updated if the action changes these thresholds, and the search parameters are modified if the action alters one of them, with the previous values also being updated accordingly.
Note that changes in state due to the continued scanning of drones are not included as the exact computation is highly time-consuming. 
% Since the expected states are for comparison, the ongoing progress has a similar impact across all generated states, and the variation in action duration is not significant enough to affect the comparison, we preferred to omit these changes.



%Formally, Let $s'_{a_j}$ be the predicted state to be after doing $a_j$.
%We want to maximize the number of trajectories $\tau_j\in \mathcal{T}$ such that for a $s\in \tau$,  $R(s_{a*}) > %R(s_{a_j})$ for all $a_j\in A', a_j \not= a*$.
%Details about this evaluation can be found in the appendix.

%[The details for the appendix:
%The  set of the generated trajectories $\mathcal{T}$, we mentioned above,  consist of two types of trajectories, %denote these subsets by $\mathcal{T}_1$ and $\mathcal{T}_2$.
%$\mathcal{T}_1$ contains the synthetic trajectories with improper parameters that do not match, in such case an %action that change the improper parameter should be preferred.
%for example, very low thresholds, leading to numerous false alarms with an extremely slow operator that cannot %handle many messages.


%$\mathcal{T}_2$, contains trajectories of mimic-operator, for which the we was able to determine the best action. 
%In order determine the best action, for each operator, we generate trajectories that mimic his decision making, %without doing the change parameter action, denote this set of trajectories by $T$, and additional trajectories %where the change parameter action done after 10 minutes, and increase or decrease the threshold, denote these sets %by $T^+, T^-$ respectively. We then compared the performances.
%$\mathcal{T}_2$, contains trajectories of mimic-operator, for which the performance was significantly diffident %between $T, T^+, T^-$.
%Each $\tau_j \in \mathcal{T}_2$ is associated with a best action according to this comparison.

%**should add a table.]
%************************************







% summary of this check can be found here: https://docs.google.com/spreadsheets/d/1Q4InCmxTf8t8V4vUeTTRmryKWRpMIi7D/edit#gid=1075344868


%More details on our comparison are provided in the appendix.


We conduct our experiments on a wide variety of tabular datasets (refer Section \ref{sec:datasets}) with varying levels of data heterogeneity across a variety of downstream classification tasks (refer Section \ref{sec:tabglm}).

\begin{table*}[ht]
    \centering
    \scriptsize
    \begin{tabular}{l|cccccc|ccc}
    \toprule
    \multirow{2}{*}{\textbf{Dataset}} & \multicolumn{9}{c}{\textbf{Performance (AUC-ROC)}} \\
    \cline{2-10}
             & \textbf{\tabglm} & \textbf{CatBoost} & \textbf{GB} & \textbf{LR} & \textbf{RF} & \textbf{XGBoost} & \textbf{Tab} & \textbf{FT-} & \textbf{NODE} \\
             & (ours) &   &  &   &  &  & \textbf{Transformer} & \textbf{Transformer} &  \\
    \midrule \midrule
    bank         & 92.07 & \textbf{93.51}  & 92.36 & 86.76  & 92.46 & 92.84 & 90.05 &  92.07 & \underline{92.67} \\
    blood        & \textbf{78.48} & 74.94 & 72.24 & \underline{76.76} & 70.77 & 69.51 & 74.26 & 74.98 & 76.21 \\
    calhousing   & \textbf{95.47} & \underline{93.55} & 92.47 & 90.84 & 93.45  & 81.99 & 83.13 & 93.62 & 93.84 \\
    car          & 99.40 & \textbf{99.97} & \underline{99.83} & 78.46 & 99.41 & 99.92 & 98.57 & 98.51 & 99.64 \\
    coil2000     & \underline{74.17} & 73.97 & \textbf{74.66} & 73.22 & 69.43 & 71.19 & 71.64 & 65.59 & 73.09 \\
    creditg      & 79.32 & \textbf{80.54} & 78.36 & 75.21 & 79.76 & 76.81 & 79.40 & 56.60 & \underline{79.83} \\
    diabetes     & \textbf{83.70} & \underline{82.55} & 82.34 & 82.89 & 81.65 & 79.17 & 82.72 & 82.34 & 82.18 \\
    heart        & \textbf{93.29} & \underline{92.61} & 92.00 & 90.74 & 91.92 & 91.16 & 92.16 & 91.81 & \underline{92.61} \\
    % jungle       & 88.98 & \underline{97.42 & 93.05 & 80.89 & 93.79 & \textbf{97.52 & 80.73 & 81.00 & 95.51 \\
    kr-vs-kp     & \underline{99.43} & \textbf{99.95} & 99.77 & 99.15 & 99.86 & 99.95  & 99.30 & 86.79 & 99.41 \\
    mfeat-fourier & 99.94 & \underline{99.97} & 99.62 & \textbf{100.00} & 99.99 & 99.70 & 99.99 & 99.92 & \textbf{100.00} \\
    pc3          & \textbf{82.82} & \underline{82.48} & 80.80 & 79.44 & 80.89 & 77.76 & 79.02 & 76.57 & 81.00 \\
    income       & \textbf{92.59} & 92.44 & 91.75 & 79.03 & 89.19 & \underline{92.35} & 89.63 & 70.57 & 90.30 \\
    texture      & \textbf{100.0} & \underline{99.98} & 99.93 & 99.87 & 99.94 & 99.96 & 99.98 & 99.94 & 99.94 \\
    balance-scale       & \textbf{99.10} & 92.35 & \underline{98.37} & 93.11 & 84.89 & 98.99 & 91.60 & 91.03 & 94.41 \\
    mfeat-karhunen      & \textbf{99.88} & \underline{99.86} & 99.79 & 99.52 & 99.71 & 98.69 & 99.56 & 98.85 & \textbf{99.88} \\
    mfeat-morphological & \textbf{96.99} & 96.20 & 96.01 & 95.74 & 95.53 & 96.12 & 95.75 & 96.33 & \underline{96.34} \\
    mfeat-zernike       & \textbf{98.09} & \underline{97.59} & 97.16 & 97.74 & 96.72 & 97.35 & 98.02 & 97.76 & 97.49 \\
    cmc                 & \textbf{74.45} & 72.56 & \underline{72.89} & 70.41 & 70.52 & 73.00 & 69.96 & 71.56 & \underline{73.88} \\
    tic-tac-toe         & \underline{99.85} & 99.92 & 99.81 & 72.00 & 96.12 & \textbf{99.98} & 70.90 & 72.76 & 98.82 \\
    vehicle             & \textbf{94.50} & \underline{93.02} & 92.33 & 88.79 & 93.23 & 92.84 & 93.19 & 90.50 & 91.61 \\
    eucalyptus          & \textbf{91.95} & \underline{88.59} & 89.31 & 87.45 & 90.11 & 90.04 & 88.27 & 89.98 & 89.70 \\
    analcatdata\_author  & \textbf{58.96} & \underline{55.89} & 54.61 & 53.56 & 53.20 & 57.43 & 53.63 & 53.94 & 55.50 \\
    MiceProtein         & \underline{99.98} & \textbf{99.99} & 99.97 & 99.51 & 99.85 & \underline{99.98} & 99.91 & 99.41 & 99.97 \\
    steel-plates-fault  & 94.52 & \underline{96.51} & \textbf{96.26} & 91.35 & 91.71 & 96.56 & 91.91 & 91.92 & 94.45 \\
    dress-sales         & \textbf{57.89} & \underline{56.96} & 55.93 & 55.94 & 53.72 & 57.23 & 53.38 & 54.41 & 52.62 \\ \midrule
    \textbf{Average} & \textbf{89.47} & 88.64 & 88.34 & 84.69 & \underline{86.96} & 87.62 & 85.84 & 83.91 & 88.22 \\
    \bottomrule
    \end{tabular}
    \caption{\textbf{Comparison of performance (AUC-ROC) of existing approaches in tabular Machine Learning against \tabglm}. Our proposed method \tabglm\ achieves significant performance gains across 25 classification datasets. The best performing model is in \textbf{bold} while the second best is \underline{underlined}.}
    \label{tab:sota_perf_contrast}
\end{table*}


\subsection{Datasets}
\label{sec:datasets}
To demonstrate the effectiveness of \tabglm\ in the presence of heterogeneous feature columns, as discussed in Section \ref{sec:tabglm}, we conduct our experiments on 25 datasets encompassing both binary and multi-class classification tasks, curated from popular papers TabLLM~\cite{tabllm}, TabPFN~\cite{hollmann2022tabpfn}, and large scale datasets in OpenML~\cite{openml2017}.
Following the principal goal of \tabglm, we consider heterogeneous datasets that encapsulate both numerical and textual columns like \textbf{Bank} ($\sim$45k records with 7 numerical and 9 categorical columns), \textbf{Creditg} (1k rows with 7 numerical and 13 categorical columns), \textbf{Heart} (918 rows with 6 numerical and 5 categorical columns) and \textbf{Income} ($\sim$48k rows with 4 numerical and 8 categorical columns), as shown in TabLLM. In addition, we use 12 datasets from OpenML, containing at least 1 numerical and 1 categorical column, including \textbf{balance-scale} (5 numerical and 1 categorical), \textbf{tic-tac-toe} (10 numerical and 10 categorical), \textbf{dress-sales} (13 numerical and 12 categorical) etc with more details in appendix. 
We also include datasets containing only numerical columns like \textbf{blood} (4 numerical columns), \textbf{calhousing} (8 numerical columns), \textbf{coil2000} (86 numerical columns) etc. alongside datasets containing only categorical columns like \textbf{car}, from both OpenML and TabPFN. 
We adopted datasets of varying sizes, with number of rows ranging from 500 (in \textbf{dress-sales}) to 45,211 (in \textbf{bank}) to demonstrate the applicability of our method to real-world large tabular datasets.
Note that the multi-modal architecture in \tabglm\ involves a LLM encoder~\cite{tapas, tapex} that is limited by the number of input tokens, which is 512 (from TAPAS) in our case.
More details on each dataset experimented upon in Table \ref{tab:sota_perf_contrast} is discussed in the supplementary material.\looseness-1

\subsection{Experimental Setup}
We conduct our experiments on datasets discussed in Section \ref{sec:datasets} and report the average performance (AUC-ROC scores) of each model across the same 5 random seeds (kept constant across datasets) in Section \ref{sec:results}. 
For all numerical and heterogeneous datasets, numerical columns are normalized using min-max\footnote{Scikit learn package: \url{https://scikit-learn.org/stable/modules/generated/sklearn.preprocessing.MinMaxScaler.html}} normalization while the categorical (text) columns are converted into One-Hot encodings (refer ablation in supplementary material) to create a numeric dataset for graph transformation.
For the text transformation, each record in the table is converted to serialized text following the tokenizer in TAPAS~\cite{tapas}. We chose TAPAS based on ablation experiments on the choice of LLMs in Section \ref{sec:ablations}. 
For datasets that contain only categorical columns, our \tabglm\ method uses only the text pipeline, utilizing only the semantic information present in such datasets.
Models for all datasets are trained on a fixed set of hyperparameters with an initial learning rate of $1e^{-4}$, batch size of 256 and weighting the consistency loss at 20\% ($\lambda = 0.2$). 
All experiments are conducted on 4 NVIDIA V100 GPUs with additional details on the experiment setup in the Appendix and code released at \url{https://github.com/amajee11us/TabGLM}.\looseness-1

\begin{table}[t]
\centering
\scriptsize
    \begin{tabular}{l|cccc}
    \toprule
    \multirow{3}{*}{ \textbf{Dataset} } & \multicolumn{4}{c}{ \textbf{Tabular DL Methods} } \\ \cline{2-5}
                               & \textbf{\tabglm} & \textbf{IGNNet} & \textbf{TabLLM} & \textbf{TabPFN} \\
                               & (multi-modal) & (graph) & (text) &  \\
    \midrule \midrule
        bank         & 92.07 & 91.11 & 91.20 & 91.19 \\
        blood        & \textbf{78.48} & 74.09 & 74.03 & 77.01 \\
        calhousing   & \textbf{95.47} & 94.79 & 95.38 & 95.31 \\
        car          & 99.40 & 50.16 & \textbf{99.99} & 99.53 \\
        % coil2000     & 74.17 &  &  & 72.51 \\
        creditg      & 79.32 & 71.99 & 70.82 & \textbf{80.79} \\
        diabetes     & \textbf{83.70} & 77.79 & 80.40 & 73.67 \\
        heart        & 93.29 & 92.06 & \textbf{94.21} & 82.60 \\
        jungle       & 88.98 & 88.98 & 93.00 & 87.36 \\
        income       & \textbf{92.59} & 90.76 & 92.19 & 90.14 \\ \midrule
        \textbf{Average} & \textbf{89.26} & 81.30 & 87.91 & 86.40 \\ \hline
    \end{tabular}
\caption{\textbf{Comparison of performance (AUC-ROC) of \tabglm\ against benchmark datasets in TabLLM} \cite{tabllm}. Results from all methods are averaged over five seeds.}
\label{tab:dl_model_benchmark}
\end{table}

\subsection{Results}
\label{sec:results}
At first, we compare the performance of \tabglm\ with traditional linear and tree based Machine Learning models like CatBoost~\cite{prokhorenkova2018catboost}, XGBoost~\cite{chen2016xgboost}, Gradient Boosting (GB) \cite{ke2017lightgbm}, Random Forest (RF) \cite{breiman2001random} and Logistic Regression (LR).
Our results in Table \ref{tab:sota_perf_contrast} show that \tabglm\ demonstrates significant increase in AUROC of 4.77\% over LR, 2.51\% over RF etc. outperforming such techniques across 25 downstream tabular classification tasks.
However, for simple datasets with lower number of feature columns like \textbf{kr-vs-kp}, \textbf{pc3} etc., tree based models (CatBoost) continue to show dominance in performance.\looseness-1

Secondly, we compare the performance of \tabglm\ with SoTA tabular DL models like FT-Transformer~\cite{gorishniy2021revisiting}, TabTransformer~\cite{huang2020tabtransformer} and NODE~\cite{popov2019neural}. \tabglm\ consistently outperforms tabular DL models like FT-Transformer by 5.56\%, TabTransformer by 3.64\% and NODE by 1.26\% respectively.
Finally, we compare the performance of \tabglm\ against SoTA uni-modal DL architectures like IGNNet (table-to-graph) and TabLLM (table-to-text) on 9 datasets in the benchmark introduced in TabLLM~\citet{tabllm}. We observe that \tabglm\ outperforms TabLLM by 1.35\% and IGNNet by 7.96\% respectively on the benchmark datasets in \cite{tabllm}, summarized in Table \ref{tab:dl_model_benchmark}.
The above results indicate a strong generalization of the proposed \tabglm\ architecture to a variety of downstream tasks, establishing \tabglm\ as a strong choice for Tabular Deep Learning under feature heterogeneity.\looseness-1

\subsection{Ablation Study}
\label{sec:ablations}

\begin{table}[t]
\centering
\scriptsize
\begin{tabular}{l|ccc}
\toprule
\multirow{3}{*}{ \textbf{Dataset} } & \multicolumn{3}{c}{ \textbf{Methods} } \\ \cline{2 - 4}
                           & \multirow{2}{*}{ \textbf{TabLLM} } &  \textbf{\tabglm} & \textbf{\tabglm} \\
                                    &  & (w TAPEX encoder) & (w TAPAS encoder) \\
\midrule 
{Param. Count}& 2.9B& 336M & 129M\\
\midrule
blood              &  71.78   &   77.57   &  \textbf{78.48} \\
calhousing         &  95.00   &   95.29   &  \textbf{95.47} \\
creditg            &  78.56   &   78.72   &  \textbf{79.32} \\
\bottomrule
\end{tabular}
\caption{\textbf{Ablation on the Choice of LLM} architecture for the text transformation module of \tabglm.}
\label{tab:choice_of_llm}
\end{table}

\noindent \textbf{Multi-Modal vs. Uni-Modal training:}
The core contribution of \tabglm\ lies in its multi-modal architecture for tabular representation learning. To evaluate its components, we decompose it into two uni-modal architectures: \textit{Graph only} (using only the graph encoder $E_{\text{graph}}$) and \textit{Text only} (using only the text encoder $E_{\text{text}}$), based on the choice of the feature extractor during both training and inference. 
Their performance is compared against the complete multi-modal \tabglm\ training recipe, with results summarized in Table \ref{tab:tabglm_components}. 
The \textit{Graph only} pipeline employs the GNN from \cite{ignnet}, while the \textit{Text only} pipeline uses the BART-based TAPAS~\cite{tapas} encoder. 
Both pipelines use the same classifier head (Section \ref{sec:tabglm}) for downstream tasks. In \textbf{Text only}, the encoder is frozen, and only the classifier head is trained, whereas in \textit{Graph only}, both the encoder and classifier head are trained, to ensure fair comparison with \tabglm, where the text encoder remains frozen during training. 
Experiments on three representative datasets—\textbf{pc3} (numerical), \textbf{bank} (balanced numerical and categorical), and \textbf{creditg} (categorical-heavy)—show that \tabglm's multi-modal design consistently outperforms its uni-modal variants, underscoring the value of modality fusion for learning from heterogeneous tables.\looseness-1

\begin{table}[t]
      \centering
      \scriptsize
        \begin{tabular}{l|cc|c}
            \toprule
            \multirow{2}{*}{\textbf{Dataset}} & \textbf{Graph Trans.} & \textbf{Text Trans.} & \multirow{2}{*}{\textbf{AUCROC}} \\              
                                    &  ($E_{graph}$) & ($E_{text}$) &         \\
            \midrule \midrule
            \multirow{3}{*}{pc3}    & \checkmark  &           &  77.04 \\
                                    &            & \checkmark &  78.24 \\
                                    & \checkmark & \checkmark &  \textbf{82.82} \\
            \hline
            \multirow{3}{*}{bank }  & \checkmark &            &  91.11 \\
                                    &            & \checkmark &  90.52\\
                                    & \checkmark & \checkmark &  \textbf{92.07} \\
            \hline
            \multirow{3}{*}{Creditg}& \checkmark &            &  71.99 \\
                                    &            & \checkmark &  77.36\\
                                    & \checkmark & \checkmark &  \textbf{79.32} \\
            \bottomrule
      \end{tabular}
      \caption{\textbf{Ablations on the graph and text components of the proposed \tabglm\ approach}. Results are averaged over five seeds. }
      \label{tab:tabglm_components}
\end{table}

\noindent \textbf{Choice of LLM architecture for Text Transformation:}
The choice of the pretrained LLM architecture plays a crucial role in improving the model performance of \tabglm. 
While larger LLMs like \cite{sun2023gpt, tabllm, tapex} ($\geq$7 billion parameters) can encode superior semantic features in complex text, it also adds a significant computational overhead. Additionally, their benefits may be negligible when dealing with simpler semantic content.
To address this trade off, we conducted an ablation experiment by varying the architecture of the text encoder ($E_{text}$) across three popular LLM models - TAPAS~\cite{tapas}, TAPEX~\cite{tapex} and TabLLM~\cite{tabllm}.
For all three settings we adopt the complete multi-modal training strategy, modifying only the text encoder $E_{text}$. 
The results from this experiment, shown in Table \ref{tab:choice_of_llm}, highlight that TAPAS~\footnote{We adopt the TAPAS-base model from \url{https://huggingface.co/google/tapas-base}}, a smaller parameter count, BERT~\cite{devlin2018bert} based text encoder, outperforms other larger models like TAPEX~\cite{tapex}. We thus adopt this architecture for the text transformation pipeline in \tabglm.\looseness-1


%%
%% The acknowledgments section is defined using the "acks" environment
%% (and NOT an unnumbered section). This ensures the proper
%% identification of the section in the article metadata, and the
%% consistent spelling of the heading.
%\begin{acks}
%To Robert, for the bagels and explaining CMYK and color spaces.
%\end{acks}
\bibliographystyle{ACM-Reference-Format}
\bibliography{mybibfile}
\end{document}
%%
%% Print the bibliography
%%
%\printbibliography

%%
%% If your work has an appendix, this is the place to put it.
%\appendix

%\section{Research Methods}


\endinput
%%
%% End of file `sample-sigconf-biblatex.tex'.
