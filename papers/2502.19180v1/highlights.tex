\documentclass[preprint,12pt]{elsarticle}

\usepackage{graphicx}
\usepackage{multirow}
\usepackage{bigdelim}
\usepackage{subfigure}
\usepackage{indentfirst}
\usepackage{booktabs}
\usepackage{url}
\usepackage{rotating}
\usepackage{caption}
\usepackage{subcaption}
\usepackage{cleveref}
\usepackage{textcomp}
\usepackage{svg}
\usepackage{pgf-pie}
\usepackage{tikz}
\usetikzlibrary{shapes}
\usepackage{array}
\usepackage{arydshln}

\begin{document}

% Dummy title page
\begin{frontmatter}

\title{AutoML for Multi-Class Anomaly Compensation of Sensor Drift}

\author[anonymous]{anonymous}
% \author[tnt]{Melanie Schaller} %% Author name
% \author[tnt]{Mathis Kruse}
%  \author[usc]{Antonio Ortega}
%  \author[luhai]{Marius Lindauer}
%  \author[tnt]{Bodo Rosenhahn}
 %Author affiliation
%\affiliation{organization={Institute for Information Processing (tnt), Leibniz University Hannover}

% \affiliation[tnt]{organization={Institute for Information Processing (tnt), Leibniz University Hannover},%Department and Organization
%             country={Germany}}

% \affiliation[usc]{organization={Department of Electrical and Computer Engineering, University of Southern California},%Department and Organization
%             country={U.S.}}

% \affiliation[luhai]{organization={Institute for Artificial Intelligence, Leibniz University Hannover},%Department and Organization
%             country={Germany}}

%\author{Melanie Schaller, Mathis Kruse, Antonio Ortega, Marius Lindauer and Bodo Rosenhahn}

%\affiliation{Leibniz University Hannover and University of Southern California},%Department and Organization

\begin{abstract}
% Dummy abstract text
This is a dummy abstract for the purpose of compiling highlights only.
\end{abstract}

\begin{highlights}
Our analysis demonstrates that the conventional training configurations are suboptimal in learning and compensating for sensor drift. Thus, we propose a novel sensor drift compensation learning training paradigm that closely matches real-world scenarios.\\

\begin{enumerate}
    \item Our findings further indicate, that AutoML techniques along with the proposed training paradigm enable effective drift adaptation to evolving levels of drift severity and complex drift dynamics in sensor measurements.
    \item By utilizing meta-learning, AutoML-DC starts from initial configurations based on prior data, lowering the extensive data requirements normally needed for training neural networks or ensemble models.
    \item We make use of AutoML techniques to enhance model robustness (see standard deviation) by combining multiple models to capture diverse data patterns, optimizing feature selection, and preventing overfitting through smart training termination.
    \item We conduct extensive benchmarking experiments against existing models and highlight the significant accuracy improvements realized when adopting AutoML-DC in practical drift compensation scenarios in industrial measurements.
\end{enumerate}
\end{highlights}

\end{frontmatter}

\end{document}