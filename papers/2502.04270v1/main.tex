\documentclass[12pt,twoside]{article}

% \usepackage{amsmath}
% \usepackage{amsthm}
\usepackage{amssymb}
% \usepackage{amsfonts}
% \usepackage{algorithm}
% \usepackage{algpseudocode}
\usepackage[top = 1 in, bottom = 1.2 in, left = 1 in, right = 1 in]{geometry}
% %\usepackage{fullpage}
% \usepackage{color}
% \usepackage{mathrsfs}
% \usepackage{upgreek}
% \usepackage[normalem]{ulem}
% \usepackage{longtable}
% \usepackage{hyperref}
% \usepackage{cleveref}
\usepackage{natbib}
% \usepackage{dsfont}
% \usepackage{graphicx}
% \graphicspath{{figs/}}
% \usepackage{caption}
% % \usepackage{subcaption}
% \usepackage{float}
% \usepackage{mathtools}
% \usepackage{etoolbox}
% \usepackage{thmtools}
% \usepackage{enumitem}
\usepackage{parskip}
\usepackage[textsize=tiny]{todonotes}
\setlength{\marginparwidth}{2cm}
\setlength{\parskip}{1em}



\usepackage{multirow}

\usepackage{esvect}

\usepackage[utf8]{inputenc}
\usepackage{comment}


\usepackage{hyperref}
\usepackage{url}
\usepackage{wrapfig}
\usepackage{lipsum}

\newcommand{\hiddenfootnote}[1]{%
  \begingroup
  \renewcommand{\thefootnote}{} % 临时隐藏编号
  \footnotemark\addtocounter{footnote}{-1}% 标记但不增加计数器
  \footnotetext{#1}%
  \endgroup
}



%%%%%%%% ICML 2024 EXAMPLE LATEX SUBMISSION FILE %%%%%%%%%%%%%%%%%

% \documentclass{article}

% Recommended, but optional, packages for figures and better typesetting:
\usepackage{microtype}
\usepackage{graphicx}
\usepackage{subfigure}
\usepackage{booktabs} % for professional tables

\usepackage{caption}

% hyperref makes hyperlinks in the resulting PDF.
% If your build breaks (sometimes temporarily if a hyperlink spans a page)
% please comment out the following usepackage line and replace
% \usepackage{icml2024} with \usepackage[nohyperref]{icml2024} above.
\usepackage{hyperref}

\usepackage{algorithm}
\usepackage{algorithmic}
% Attempt to make hyperref and algorithmic work together better:
% \newcommand{\theHalgorithm}{\arabic{algorithm}}

% Use the following line for the initial blind version submitted for review:
% \usepackage{icml2025}

% If accepted, instead use the following line for the camera-ready submission:
% \usepackage[accepted]{icml2025}

% For theorems and such
\usepackage{amsmath}
\usepackage{amssymb}
\usepackage{mathtools}
\usepackage{amsthm}

% for coloring
\usepackage{xcolor}
% \usepackage{tikz}
\usepackage{soul}
\sethlcolor{red!20}

% \usepackage[ruled,vlined]{algorithm2e}

% if you use cleveref..
\usepackage[capitalize,noabbrev]{cleveref}

%%%%%%%%%%%%%%%%%%%%%%%%%%%%%%%%
% THEOREMS
%%%%%%%%%%%%%%%%%%%%%%%%%%%%%%%%
\theoremstyle{plain}
\newtheorem{theorem}{Theorem}[section]
\newtheorem{proposition}[theorem]{Proposition}
\newtheorem{lemma}[theorem]{Lemma}
\newtheorem{corollary}[theorem]{Corollary}
\theoremstyle{definition}
\newtheorem{definition}[theorem]{Definition}
\newtheorem{assumption}[theorem]{Assumption}
\theoremstyle{remark}
\newtheorem{remark}[theorem]{Remark}


%%%COMMENTS
\newcommand{\xxcomment}[4]{\textcolor{#1}{[$^{\textsc{#2}}_{\textsc{#3}}$ #4]}}

\newcommand{\julia}[1]{\xxcomment{blue}{J}{K}{#1}}
\newcommand{\yunzhen}[1]{\xxcomment{violet}{Y}{F}{#1}}
\newcommand{\yaqicolor}{cyan}
\newcommand{\yaqi}[1]{\xxcomment{\yaqicolor}{Y}{D}{#1}}
\newcommand{\kunhao}[1]{\xxcomment{brown}{K}{Z}{#1}}
\newcommand{\ariel}[1]{\xxcomment{green}{A}{K}{#1}}

%%DISABLED
%%\newcommand{\xxcomment}[4]{}
%\newcommand{\julia}[1]{#1}
%\newcommand{\yaqi}[1]{}
%\newcommand{\kunhao}[1]{}
%\newcommand{\ariel}[1]{}
%\newcommand{\yunzhen}[1]{}



% Todonotes is useful during development; simply uncomment the next line
%    and comment out the line below the next line to turn off comments
%\usepackage[disable,textsize=tiny]{todonotes}
\usepackage[textsize=tiny]{todonotes}


%
\setlength\unitlength{1mm}
\newcommand{\twodots}{\mathinner {\ldotp \ldotp}}
% bb font symbols
\newcommand{\Rho}{\mathrm{P}}
\newcommand{\Tau}{\mathrm{T}}

\newfont{\bbb}{msbm10 scaled 700}
\newcommand{\CCC}{\mbox{\bbb C}}

\newfont{\bb}{msbm10 scaled 1100}
\newcommand{\CC}{\mbox{\bb C}}
\newcommand{\PP}{\mbox{\bb P}}
\newcommand{\RR}{\mbox{\bb R}}
\newcommand{\QQ}{\mbox{\bb Q}}
\newcommand{\ZZ}{\mbox{\bb Z}}
\newcommand{\FF}{\mbox{\bb F}}
\newcommand{\GG}{\mbox{\bb G}}
\newcommand{\EE}{\mbox{\bb E}}
\newcommand{\NN}{\mbox{\bb N}}
\newcommand{\KK}{\mbox{\bb K}}
\newcommand{\HH}{\mbox{\bb H}}
\newcommand{\SSS}{\mbox{\bb S}}
\newcommand{\UU}{\mbox{\bb U}}
\newcommand{\VV}{\mbox{\bb V}}


\newcommand{\yy}{\mathbbm{y}}
\newcommand{\xx}{\mathbbm{x}}
\newcommand{\zz}{\mathbbm{z}}
\newcommand{\sss}{\mathbbm{s}}
\newcommand{\rr}{\mathbbm{r}}
\newcommand{\pp}{\mathbbm{p}}
\newcommand{\qq}{\mathbbm{q}}
\newcommand{\ww}{\mathbbm{w}}
\newcommand{\hh}{\mathbbm{h}}
\newcommand{\vvv}{\mathbbm{v}}

% Vectors

\newcommand{\av}{{\bf a}}
\newcommand{\bv}{{\bf b}}
\newcommand{\cv}{{\bf c}}
\newcommand{\dv}{{\bf d}}
\newcommand{\ev}{{\bf e}}
\newcommand{\fv}{{\bf f}}
\newcommand{\gv}{{\bf g}}
\newcommand{\hv}{{\bf h}}
\newcommand{\iv}{{\bf i}}
\newcommand{\jv}{{\bf j}}
\newcommand{\kv}{{\bf k}}
\newcommand{\lv}{{\bf l}}
\newcommand{\mv}{{\bf m}}
\newcommand{\nv}{{\bf n}}
\newcommand{\ov}{{\bf o}}
\newcommand{\pv}{{\bf p}}
\newcommand{\qv}{{\bf q}}
\newcommand{\rv}{{\bf r}}
\newcommand{\sv}{{\bf s}}
\newcommand{\tv}{{\bf t}}
\newcommand{\uv}{{\bf u}}
\newcommand{\wv}{{\bf w}}
\newcommand{\vv}{{\bf v}}
\newcommand{\xv}{{\bf x}}
\newcommand{\yv}{{\bf y}}
\newcommand{\zv}{{\bf z}}
\newcommand{\zerov}{{\bf 0}}
\newcommand{\onev}{{\bf 1}}

% Matrices

\newcommand{\Am}{{\bf A}}
\newcommand{\Bm}{{\bf B}}
\newcommand{\Cm}{{\bf C}}
\newcommand{\Dm}{{\bf D}}
\newcommand{\Em}{{\bf E}}
\newcommand{\Fm}{{\bf F}}
\newcommand{\Gm}{{\bf G}}
\newcommand{\Hm}{{\bf H}}
\newcommand{\Id}{{\bf I}}
\newcommand{\Jm}{{\bf J}}
\newcommand{\Km}{{\bf K}}
\newcommand{\Lm}{{\bf L}}
\newcommand{\Mm}{{\bf M}}
\newcommand{\Nm}{{\bf N}}
\newcommand{\Om}{{\bf O}}
\newcommand{\Pm}{{\bf P}}
\newcommand{\Qm}{{\bf Q}}
\newcommand{\Rm}{{\bf R}}
\newcommand{\Sm}{{\bf S}}
\newcommand{\Tm}{{\bf T}}
\newcommand{\Um}{{\bf U}}
\newcommand{\Wm}{{\bf W}}
\newcommand{\Vm}{{\bf V}}
\newcommand{\Xm}{{\bf X}}
\newcommand{\Ym}{{\bf Y}}
\newcommand{\Zm}{{\bf Z}}

% Calligraphic

\newcommand{\Ac}{{\cal A}}
\newcommand{\Bc}{{\cal B}}
\newcommand{\Cc}{{\cal C}}
\newcommand{\Dc}{{\cal D}}
\newcommand{\Ec}{{\cal E}}
\newcommand{\Fc}{{\cal F}}
\newcommand{\Gc}{{\cal G}}
\newcommand{\Hc}{{\cal H}}
\newcommand{\Ic}{{\cal I}}
\newcommand{\Jc}{{\cal J}}
\newcommand{\Kc}{{\cal K}}
\newcommand{\Lc}{{\cal L}}
\newcommand{\Mc}{{\cal M}}
\newcommand{\Nc}{{\cal N}}
\newcommand{\nc}{{\cal n}}
\newcommand{\Oc}{{\cal O}}
\newcommand{\Pc}{{\cal P}}
\newcommand{\Qc}{{\cal Q}}
\newcommand{\Rc}{{\cal R}}
\newcommand{\Sc}{{\cal S}}
\newcommand{\Tc}{{\cal T}}
\newcommand{\Uc}{{\cal U}}
\newcommand{\Wc}{{\cal W}}
\newcommand{\Vc}{{\cal V}}
\newcommand{\Xc}{{\cal X}}
\newcommand{\Yc}{{\cal Y}}
\newcommand{\Zc}{{\cal Z}}

% Bold greek letters

\newcommand{\alphav}{\hbox{\boldmath$\alpha$}}
\newcommand{\betav}{\hbox{\boldmath$\beta$}}
\newcommand{\gammav}{\hbox{\boldmath$\gamma$}}
\newcommand{\deltav}{\hbox{\boldmath$\delta$}}
\newcommand{\etav}{\hbox{\boldmath$\eta$}}
\newcommand{\lambdav}{\hbox{\boldmath$\lambda$}}
\newcommand{\epsilonv}{\hbox{\boldmath$\epsilon$}}
\newcommand{\nuv}{\hbox{\boldmath$\nu$}}
\newcommand{\muv}{\hbox{\boldmath$\mu$}}
\newcommand{\zetav}{\hbox{\boldmath$\zeta$}}
\newcommand{\phiv}{\hbox{\boldmath$\phi$}}
\newcommand{\psiv}{\hbox{\boldmath$\psi$}}
\newcommand{\thetav}{\hbox{\boldmath$\theta$}}
\newcommand{\tauv}{\hbox{\boldmath$\tau$}}
\newcommand{\omegav}{\hbox{\boldmath$\omega$}}
\newcommand{\xiv}{\hbox{\boldmath$\xi$}}
\newcommand{\sigmav}{\hbox{\boldmath$\sigma$}}
\newcommand{\piv}{\hbox{\boldmath$\pi$}}
\newcommand{\rhov}{\hbox{\boldmath$\rho$}}
\newcommand{\upsilonv}{\hbox{\boldmath$\upsilon$}}

\newcommand{\Gammam}{\hbox{\boldmath$\Gamma$}}
\newcommand{\Lambdam}{\hbox{\boldmath$\Lambda$}}
\newcommand{\Deltam}{\hbox{\boldmath$\Delta$}}
\newcommand{\Sigmam}{\hbox{\boldmath$\Sigma$}}
\newcommand{\Phim}{\hbox{\boldmath$\Phi$}}
\newcommand{\Pim}{\hbox{\boldmath$\Pi$}}
\newcommand{\Psim}{\hbox{\boldmath$\Psi$}}
\newcommand{\Thetam}{\hbox{\boldmath$\Theta$}}
\newcommand{\Omegam}{\hbox{\boldmath$\Omega$}}
\newcommand{\Xim}{\hbox{\boldmath$\Xi$}}


% Sans Serif small case

\newcommand{\Gsf}{{\sf G}}

\newcommand{\asf}{{\sf a}}
\newcommand{\bsf}{{\sf b}}
\newcommand{\csf}{{\sf c}}
\newcommand{\dsf}{{\sf d}}
\newcommand{\esf}{{\sf e}}
\newcommand{\fsf}{{\sf f}}
\newcommand{\gsf}{{\sf g}}
\newcommand{\hsf}{{\sf h}}
\newcommand{\isf}{{\sf i}}
\newcommand{\jsf}{{\sf j}}
\newcommand{\ksf}{{\sf k}}
\newcommand{\lsf}{{\sf l}}
\newcommand{\msf}{{\sf m}}
\newcommand{\nsf}{{\sf n}}
\newcommand{\osf}{{\sf o}}
\newcommand{\psf}{{\sf p}}
\newcommand{\qsf}{{\sf q}}
\newcommand{\rsf}{{\sf r}}
\newcommand{\ssf}{{\sf s}}
\newcommand{\tsf}{{\sf t}}
\newcommand{\usf}{{\sf u}}
\newcommand{\wsf}{{\sf w}}
\newcommand{\vsf}{{\sf v}}
\newcommand{\xsf}{{\sf x}}
\newcommand{\ysf}{{\sf y}}
\newcommand{\zsf}{{\sf z}}


% mixed symbols

\newcommand{\sinc}{{\hbox{sinc}}}
\newcommand{\diag}{{\hbox{diag}}}
\renewcommand{\det}{{\hbox{det}}}
\newcommand{\trace}{{\hbox{tr}}}
\newcommand{\sign}{{\hbox{sign}}}
\renewcommand{\arg}{{\hbox{arg}}}
\newcommand{\var}{{\hbox{var}}}
\newcommand{\cov}{{\hbox{cov}}}
\newcommand{\Ei}{{\rm E}_{\rm i}}
\renewcommand{\Re}{{\rm Re}}
\renewcommand{\Im}{{\rm Im}}
\newcommand{\eqdef}{\stackrel{\Delta}{=}}
\newcommand{\defines}{{\,\,\stackrel{\scriptscriptstyle \bigtriangleup}{=}\,\,}}
\newcommand{\<}{\left\langle}
\renewcommand{\>}{\right\rangle}
\newcommand{\herm}{{\sf H}}
\newcommand{\trasp}{{\sf T}}
\newcommand{\transp}{{\sf T}}
\renewcommand{\vec}{{\rm vec}}
\newcommand{\Psf}{{\sf P}}
\newcommand{\SINR}{{\sf SINR}}
\newcommand{\SNR}{{\sf SNR}}
\newcommand{\MMSE}{{\sf MMSE}}
\newcommand{\REF}{{\RED [REF]}}

% Markov chain
\usepackage{stmaryrd} % for \mkv 
\newcommand{\mkv}{-\!\!\!\!\minuso\!\!\!\!-}

% Colors

\newcommand{\RED}{\color[rgb]{1.00,0.10,0.10}}
\newcommand{\BLUE}{\color[rgb]{0,0,0.90}}
\newcommand{\GREEN}{\color[rgb]{0,0.80,0.20}}

%%%%%%%%%%%%%%%%%%%%%%%%%%%%%%%%%%%%%%%%%%
\usepackage{hyperref}
\hypersetup{
    bookmarks=true,         % show bookmarks bar?
    unicode=false,          % non-Latin characters in AcrobatÕs bookmarks
    pdftoolbar=true,        % show AcrobatÕs toolbar?
    pdfmenubar=true,        % show AcrobatÕs menu?
    pdffitwindow=false,     % window fit to page when opened
    pdfstartview={FitH},    % fits the width of the page to the window
%    pdftitle={My title},    % title
%    pdfauthor={Author},     % author
%    pdfsubject={Subject},   % subject of the document
%    pdfcreator={Creator},   % creator of the document
%    pdfproducer={Producer}, % producer of the document
%    pdfkeywords={keyword1} {key2} {key3}, % list of keywords
    pdfnewwindow=true,      % links in new window
    colorlinks=true,       % false: boxed links; true: colored links
    linkcolor=red,          % color of internal links (change box color with linkbordercolor)
    citecolor=green,        % color of links to bibliography
    filecolor=blue,      % color of file links
    urlcolor=blue           % color of external links
}
%%%%%%%%%%%%%%%%%%%%%%%%%%%%%%%%%%%%%%%%%%%



\makeatletter
\long\def\@makecaption#1#2{
	\vskip 0.8ex
	\setbox\@tempboxa\hbox{\small {\bf #1:} #2}
	\parindent 1.5em  %% How can we use the global value of this???
	\dimen0=\hsize
	\advance\dimen0 by -3em
	\ifdim \wd\@tempboxa >\dimen0
	\hbox to \hsize{
		\parindent 0em
		\hfil 
		\parbox{\dimen0}{\def\baselinestretch{0.96}\small
			{\bf #1.} {#2}
			%%\unhbox\@tempboxa
		} 
		\hfil}
	\else \hbox to \hsize{\hfil \box\@tempboxa \hfil}
	\fi
}
\makeatother

\newcommand\blfootnote[1]{%
	\begingroup
	\renewcommand\thefootnote{}\footnotetext{#1}%
%	\addtocounter{footnote}{-1}%
	\endgroup
}

% The \icmltitle you define below is probably too long as a header.
% Therefore, a short form for the running title is supplied here:
% \icmltitlerunning{Optimal Sampling for Reward Modeling}



% \begin{document}



% \twocolumn[
% \title{PILAF: Optimal Human Preference Sampling for Reward Modeling}



% It is OKAY to include author information, even for blind
% submissions: the style file will automatically remove it for you
% unless you've provided the [accepted] option to the icml2024
% package.

% List of affiliations: The first argument should be a (short)
% identifier you will use later to specify author affiliations
% Academic affiliations should list Department, University, City, Region, Country
% Industry affiliations should list nyuany, City, Region, Country

% You can specify symbols, otherwise they are numbered in order.
% Ideally, you should not use this facility. Affiliations will be numbered
% in order of appearance and this is the preferred way.
% \icmlsetsymbol{equal}{*}

% \begin{icmlauthorlist}
% \icmlauthor{xxx}{equal}
% \icmlauthor{Firstname2 Lastname2}{nyu}
% \icmlauthor{Firstname3 Lastname3}{nyu}
% \icmlauthor{Firstname4 Lastname4}{nyu}
% \icmlauthor{Firstname5 Lastname5}{pku}
% \end{icmlauthorlist}

% \icmlaffiliation{nyu}{}
% \icmlaffiliation{nyu}{}

% \icmlcorrespondingauthor{Firstname1 Lastname1}{first1.last1@xxx.edu}
% \icmlcorrespondingauthor{Firstname2 Lastname2}{first2.last2@www.uk}

% You may provide any keywords that you
% find helpful for describing your paper; these are used to populate
% the "keywords" metadata in the PDF but will not be shown in the document
% \icmlkeywords{Machine Learning, ICML}

% \vskip 0.3in
% ]

% this must go after the closing bracket ] following \twocolumn[ ...

% This command actually creates the footnote in the first column
% listing the affiliations and the copyright notice.
% The command takes one argument, which is text to display at the start of the footnote.
% The \icmlEqualContribution command is standard text for equal contribution.
% Remove it (just {}) if you do not need this facility.

% \printAffiliationsAndNotice{}  % leave blank if no need to mention equal contribution
% \printAffiliationsAndNotice{\icmlEqualContribution} % otherwise use the standard text.

% \begingroup
% \setlength{\abovedisplayskip}{4pt}  % Space above equations
% \setlength{\belowdisplayskip}{4pt}  % Space below equations
% \setlength{\abovedisplayshortskip}{2pt}
% \setlength{\belowdisplayshortskip}{2pt}

\begin{document}
	\begin{center}
		{\bf \Large PILAF: Optimal Human Preference Sampling for Reward Modeling} \\
		
		\vspace{2em}
		{%\large
			{
				\begin{tabular}{ccccc}
					Yunzhen Feng$^\dagger$ & Ariel Kwiatkowski$^*$ & Kunhao Zheng$^*$ & Julia Kempe$^\diamond$ & Yaqi Duan%$^\dagger$
                    $^\diamond$\\
                    NYU & Meta FAIR & Meta FAIR & Meta FAIR \& NYU & NYU
				\end{tabular}
		}}
%		\vspace{1em}
		
		% \medskip
		
		% \medskip 

  %       \medskip 
        
		% {\small \begin{tabular}{l}
		% 	$^\S$ Meta FAIR \\
  %               $^\dagger$ New York University\\
  %               $^*$ Joint second authors\\
  %               $^\diamond$ Equal advising
		% \end{tabular}}
		
		
		\vspace{1.6em}
		\today
	\end{center}

\begin{center}
		{\bf Abstract} \\ \vspace{.6em}
		\begin{minipage}{0.9\linewidth}
			{\small ~~~~\begin{abstract}
Building a virtual cell capable of accurately simulating cellular behaviors in silico has long been a dream in computational biology. We introduce \emph{CellFlow}, an image-generative model that simulates cellular morphology changes induced by chemical and genetic perturbations using flow matching. Unlike prior methods, \emph{CellFlow} models distribution-wise transformations from unperturbed to perturbed cell states, effectively distinguishing actual perturbation effects from experimental artifacts such as batch effects—a major challenge in biological data. Evaluated on chemical (BBBC021), genetic (RxRx1), and combined perturbation (JUMP) datasets, \emph{CellFlow} generates biologically meaningful cell images that faithfully capture perturbation-specific morphological changes, achieving a 35\% improvement in FID scores and a 12\% increase in mode-of-action prediction accuracy over existing methods. Additionally, \emph{CellFlow} enables continuous interpolation between cellular states, providing a potential tool for studying perturbation dynamics. These capabilities mark a significant step toward realizing virtual cell modeling for biomedical research.
\end{abstract}
}
		\end{minipage}
	\end{center}
	
	\vspace{.6em}


\section{Introduction}
Backdoor attacks pose a concealed yet profound security risk to machine learning (ML) models, for which the adversaries can inject a stealth backdoor into the model during training, enabling them to illicitly control the model's output upon encountering predefined inputs. These attacks can even occur without the knowledge of developers or end-users, thereby undermining the trust in ML systems. As ML becomes more deeply embedded in critical sectors like finance, healthcare, and autonomous driving \citep{he2016deep, liu2020computing, tournier2019mrtrix3, adjabi2020past}, the potential damage from backdoor attacks grows, underscoring the emergency for developing robust defense mechanisms against backdoor attacks.

To address the threat of backdoor attacks, researchers have developed a variety of strategies \cite{liu2018fine,wu2021adversarial,wang2019neural,zeng2022adversarial,zhu2023neural,Zhu_2023_ICCV, wei2024shared,wei2024d3}, aimed at purifying backdoors within victim models. These methods are designed to integrate with current deployment workflows seamlessly and have demonstrated significant success in mitigating the effects of backdoor triggers \cite{wubackdoorbench, wu2023defenses, wu2024backdoorbench,dunnett2024countering}.  However, most state-of-the-art (SOTA) backdoor purification methods operate under the assumption that a small clean dataset, often referred to as \textbf{auxiliary dataset}, is available for purification. Such an assumption poses practical challenges, especially in scenarios where data is scarce. To tackle this challenge, efforts have been made to reduce the size of the required auxiliary dataset~\cite{chai2022oneshot,li2023reconstructive, Zhu_2023_ICCV} and even explore dataset-free purification techniques~\cite{zheng2022data,hong2023revisiting,lin2024fusing}. Although these approaches offer some improvements, recent evaluations \cite{dunnett2024countering, wu2024backdoorbench} continue to highlight the importance of sufficient auxiliary data for achieving robust defenses against backdoor attacks.

While significant progress has been made in reducing the size of auxiliary datasets, an equally critical yet underexplored question remains: \emph{how does the nature of the auxiliary dataset affect purification effectiveness?} In  real-world  applications, auxiliary datasets can vary widely, encompassing in-distribution data, synthetic data, or external data from different sources. Understanding how each type of auxiliary dataset influences the purification effectiveness is vital for selecting or constructing the most suitable auxiliary dataset and the corresponding technique. For instance, when multiple datasets are available, understanding how different datasets contribute to purification can guide defenders in selecting or crafting the most appropriate dataset. Conversely, when only limited auxiliary data is accessible, knowing which purification technique works best under those constraints is critical. Therefore, there is an urgent need for a thorough investigation into the impact of auxiliary datasets on purification effectiveness to guide defenders in  enhancing the security of ML systems. 

In this paper, we systematically investigate the critical role of auxiliary datasets in backdoor purification, aiming to bridge the gap between idealized and practical purification scenarios.  Specifically, we first construct a diverse set of auxiliary datasets to emulate real-world conditions, as summarized in Table~\ref{overall}. These datasets include in-distribution data, synthetic data, and external data from other sources. Through an evaluation of SOTA backdoor purification methods across these datasets, we uncover several critical insights: \textbf{1)} In-distribution datasets, particularly those carefully filtered from the original training data of the victim model, effectively preserve the model’s utility for its intended tasks but may fall short in eliminating backdoors. \textbf{2)} Incorporating OOD datasets can help the model forget backdoors but also bring the risk of forgetting critical learned knowledge, significantly degrading its overall performance. Building on these findings, we propose Guided Input Calibration (GIC), a novel technique that enhances backdoor purification by adaptively transforming auxiliary data to better align with the victim model’s learned representations. By leveraging the victim model itself to guide this transformation, GIC optimizes the purification process, striking a balance between preserving model utility and mitigating backdoor threats. Extensive experiments demonstrate that GIC significantly improves the effectiveness of backdoor purification across diverse auxiliary datasets, providing a practical and robust defense solution.

Our main contributions are threefold:
\textbf{1) Impact analysis of auxiliary datasets:} We take the \textbf{first step}  in systematically investigating how different types of auxiliary datasets influence backdoor purification effectiveness. Our findings provide novel insights and serve as a foundation for future research on optimizing dataset selection and construction for enhanced backdoor defense.
%
\textbf{2) Compilation and evaluation of diverse auxiliary datasets:}  We have compiled and rigorously evaluated a diverse set of auxiliary datasets using SOTA purification methods, making our datasets and code publicly available to facilitate and support future research on practical backdoor defense strategies.
%
\textbf{3) Introduction of GIC:} We introduce GIC, the \textbf{first} dedicated solution designed to align auxiliary datasets with the model’s learned representations, significantly enhancing backdoor mitigation across various dataset types. Our approach sets a new benchmark for practical and effective backdoor defense.




\section{On the Root Cause of Anisotropic Embeddings}%
\label{sec:theory}

We study the collective shift of the embeddings (that underlies the anisotropy problem), by analyzing their vector updates based on the optimization algorithms SGD and Adam. Weight tying is assumed, but only contributions from the output layer are considered, following \citet{bis2021tmic}. 
Our results apply to all model architectures with a standard language modeling head.

\subsection{Language Modeling Head}

The equations for the standard language modeling head read
\begin{align}
\mathcal{L} &= - \log{(p_t)} \label{eq:forward_loss} \\
    p_t &= \frac{\exp{(l_t)}}{\sum_{j=1}^V \exp{(l_j)}} \label{eq:forward_probability}\\ %
l_i &= e_i \bigcdot h \label{eq:gradient_function_new} \; ,
\end{align}
where $\mathcal{L} \in \mathbb{R}_{\geq 0}$ is the loss for next token prediction, and $p_t \in [0, 1]$ is the predicted probability of the true token $t \in \V$. $l_i \in \mathbb{R}$ and $e_i \in \mathbb{R}^H$ denote the logits and embeddings for each token $i \in \V$, respectively. $h \in \mathbb{R}^H$ is the final hidden state provided by the model for a single token. Note that the operation in Eq.~(\ref{eq:gradient_function_new}) is the dot product of two vectors in $\mathbb{R}^H$.
Backward propagation yields the following gradients with respect to the input vectors $e_i$ and $h$ of Eq.~(\ref{eq:gradient_function_new}):
\begin{align}
g_i :=~ &\frac{\partial \mathcal{L}}{\partial e_i} 
= - \left( \delta_{it} - p_i \right) \cdot h \label{eq:chain_rule_e} 
\end{align}
This result was first reported using a different notation in \citet{bis2021tmic}, and is rederived in App.~\ref{app:chain_rule_e} for the reader's convenience.

\subsection{Vanishing Sum of Embedding Gradients}

Optimization algorithms for neural networks usually update the model parameters iteratively, using an additive update vector that points in direction opposite to the gradient of the loss with respect to the parameters. In the case of embedding vectors, this can be expressed by
\begin{equation}
e_i^{(\tm)} \: = \: e_i^{(\tm-1)} + u_i^{(\tm)}  \; ,
\label{eq:update_general}
\end{equation}
with
\begin{equation}
u_i^{(\tm)} \: \propto \: - g_i^{(\tm)} \; ,
\label{eq:update_vector_definition}
\end{equation}
where $u_i^{(\tm)}$ is the update vector for $e_i^{(\tm)}$ at time step $\tm$.
Eq.~(\ref{eq:chain_rule_e}) implies that the embedding vector $e_t$ of the true token is updated in direction $+h$, while the update vectors $u_i$ for all the other embedding vectors $e_i$ with $i \neq t$ are proportional to $-h$, see Fig.~\ref{fig:gradients_example}. 
\begin{figure}[t]
\centering
\includegraphics[scale=0.5]{figs/toy_example.png}
\caption{Toy example of a hidden state vector $h$ (shown in blue) and three embedding vectors $e_i$ (shown in red) in $H = 2$ dimensions. The gray vectors represent the embedding update vectors, for the SGD (dark) and the Adam (light) optimizer. The update vector of the true token is aligned with $h$, while the others point in the opposite direction, see Eq.~(\ref{eq:chain_rule_e}). Note that the {\em sum of embedding update vectors} vanishes for SGD, while this is not necessarily the case for Adam, cf.~Eqs.~(\ref{eq:vanishing_updates_SGD}) and (\ref{eq:non_vanishing_updates_Adam}).}
\label{fig:gradients_example}
\end{figure}
This circumstance is referred to in the literature as the "common enemy effect" \cite{bis2021tmic}, and regarded as the cause of the representation degeneration problem. 
However, as we will see in the following sections, this explanation is incomplete, as it does not take into account the scaling of the gradients with the predicted probabilities $p_i$, see Eq.~(\ref{eq:chain_rule_e}). The basis for our argumentation is the observation that the {\em sum of embedding gradients vanishes}, as the following simple calculation shows:
\begin{align}
\sum_{i=1}^V g_i^{(\tm)} 
&\stackrel{(\ref{eq:chain_rule_e})}{=} 
- \sum_{i=1}^V  \left( \delta_{it}^{(\tm)} - p_i^{(\tm)} \right) \cdot h^{(\tm)} \nonumber \\
&= - \left( 1 - \sum_{i=1}^V p_i^{(\tm)} \right) \cdot h^{(\tm)} = 0 
\label{eq:optimizer_momentum_conservation}
\end{align}
Next, we will study how Eq.~(\ref{eq:optimizer_momentum_conservation}) translates to the sum $\sum_{i=1}^V u_i^{(\tm)}$
of embedding update vectors, as well as the mean embedding vector
\begin{equation}
\mu^{(\tm)} = \frac{1}{V} \sum_{i=1}^V e_i^{(\tm)}
\label{eq:mu}
\end{equation}
Since the exact definition of the embedding update vector $u_i$, i.e. the proportionality factor in Eq.~(\ref{eq:update_vector_definition}), depends on the optimization algorithm, we discuss SGD and Adam separately.

\subsection{Invariant Mean Embedding with SGD}

We consider the application of the SGD optimization algorithm on the embedding vectors\footnote{Details are given in App.~\ref{app:sgd_algorithm}.}.
At each training step, an embedding vector is simply updated by adding the associated negative gradient $-g_i$, multiplied by a global learning rate $\eta$. Hence, Eq.~(\ref{eq:update_vector_definition}) becomes
\begin{equation}
u_i^{(\tm)} = - \eta \cdot g_i^{(\tm)}
\label{eq:update_vector_definition_SGD}
\end{equation}
Together with Eq.~(\ref{eq:optimizer_momentum_conservation}), this implies that the {\em sum of embedding update vectors vanishes} at any time step $\tm$:
\begin{equation}
\sum_{i=1}^V  u_i^{(\tm)}
\stackrel{(\ref{eq:update_vector_definition_SGD})}{=} - \eta \sum_{i=1}^V g_i^{(\tm)} 
\stackrel{(\ref{eq:optimizer_momentum_conservation})}{=} 0
\label{eq:vanishing_updates_SGD}
\end{equation}
Consequently, the mean embedding vector will stay invariant during the training process:
\begin{equation}
    \mu^{(\tm)} - \mu^{(\tm-1)} 
    \stackrel{(\ref{eq:mu},\ref{eq:update_general})}{=} 
    \frac{1}{V} \sum_{i=1}^V u_i^{(\tm)} 
    \stackrel{(\ref{eq:vanishing_updates_SGD})}{=} 0
    \label{eq:invariant_mean_embedding_SGD}
\end{equation}
This holds even though the different embeddings $e_i$ will be individually updated in different directions with different magnitudes. 
Moreover, all of the above is true also in the case of SGD with momentum,
which follows from linearity and mathematical induction.
Eq.~(\ref{eq:invariant_mean_embedding_SGD}) has far-reaching implications with regard to the anisotropy problem. It entails that the embedding vectors do not collectively shift away from the origin if SGD (with or without momentum) is used. 

\subsection{Shifted Mean Embedding with Adam}

In this section, we analyze the behavior of the mean embedding during optimization with Adam~\cite{adam}, see Algorithm~\ref{alg:algorithm_adam}.
\begin{algorithm}[t]
    \small
    \textbf{Input:}
    $\eta$ (lr), $e_i^{(0)}$ (initial embeddings),
    $\mathcal{L}(e_i)$ (objective), $\beta_1, \beta_2$ (betas), $T$ (number of time steps)
    \\
    \textbf{Initialize:}
    $m_i^{(0)} \leftarrow 0$ (1st moment),
    $v_i^{(0)} \leftarrow 0$ (2nd moment)
    \\
    \textbf{Output}: $e^{(T)}$ (final embeddings)
    \begin{algorithmic}[1]
        \For{$\tm=1 \dots T$}
            \For{$i=1 \dots V$}
                \State $g_i^{(\tm)}$ $\gets$ $\nabla_{e_i} \mathcal{L}^{(\tm)} (e_i^{(\tm-1)})$
                \State $m_i^{(\tm)}$ $\gets$ $\beta_1 m_i^{(\tm-1)} + (1 - \beta_1) g_i^{(\tm)}$%
                \label{alg:line:adam_first_moment_definition}
                \State $v_i^{(\tm)}$ $\gets$ $\beta_2 v_i^{(\tm-1)} + (1-\beta_2) \left(g_i^{(\tm)}\right)^2$%
                \label{alg:line:adam_second_moment_definition}
                \State $\firstmoment$ $\gets$ $m_i^{(\tm)}/\big(1-\beta_1^\tm \big)$%
                \label{alg:line:adam_first_moment_exp_averages}
                \State $\secondmoment$ $\gets$ $v_i^{(\tm)}/\big(1-\beta_2^\tm \big)$%
              \label{alg:line:adam_second_moment_exp_averages}
            \EndFor
            \BeginBox[fill=\colhighlight!10!White, xshift=0.6em, inner xsep=-0.7em]
            \If{\highlight{coupled}}
                \State $\highlight{\secondmomentavg}$ $\gets$ $\highlight{\frac{1}{V} \sum_{i=1}^V \secondmoment}$ %
                \label{alg:line:coupled_adam_second_moment}%
            \EndIf
            \For{$i=1 \dots V$}
                \If{\highlight{coupled}}
                    \State $\highlight{\secondmoment}$ $\gets$ $\highlight{\secondmomentavg}$%
                    \label{alg:line:coupled_adam_second_moment_2}
                \EndIf
            \EndBox
                \State $e_i^{(\tm)}$ $\gets$ $e_i^{(\tm-1)} - \eta \frac{\firstmoment}{\sqrt{\secondmoment} + \epsilon}$%
                \label{alg:line:adam_update}
            \EndFor
        \EndFor
        \State \Return $e^{(T)}$
    \end{algorithmic}
    \caption{Pseudocode for the Adam algorithm and our extension, the \highlight{Coupled Adam algorithm (highlighted)}, applied to the embedding vectors $e_i$. Note that weight decay is not applied.}
    \label{alg:algorithm_adam}
\end{algorithm}

The update vector~Eq.~(\ref{eq:update_vector_definition}) for the Adam algorithm is given by
\begin{equation}
    u_i^{(\tm)} 
    = 
    - \eta^{(\tm)}_i \cdot \firstmoment \label{eq:update_vector_definition_Adam}  \; ,
\end{equation}
where we have introduced an $i$-dependent effective learning rate 
\begin{equation}
    \eta^{(\tm)}_i
    := 
    \frac{\eta}{\sqrt{\secondmoment} + \epsilon} \label{eq:adam_learning_rate} 
\end{equation}
Note that $\firstmoment$ and $\secondmoment$ denote the exponentially averaged first and second moments, respectively, defined according to lines~\ref{alg:line:adam_first_moment_definition}-\ref{alg:line:adam_second_moment_exp_averages}
in Algorithm~\ref{alg:algorithm_adam}.
The $i$-dependent learning rate serves the purpose of individually normalizing the update vectors for different parameters in the Adam optimizer.
However, it also has an unwanted effect specifically on the embedding vectors. While we know from Eq.~\eqref{eq:optimizer_momentum_conservation} and Algorithm~\ref{alg:algorithm_adam} (lines~\ref{alg:line:adam_first_moment_definition},\ref{alg:line:adam_first_moment_exp_averages}) that the {\em unweighted} sum over the first moments vanishes, 
$\sum_{i=1}^V \firstmoment = 0$,
this is not true for the {\em weighted} sum,
\begin{equation}
\sum_{i=1}^V \eta^{(\tm)}_i \firstmoment \neq 0 \; ,
\label{eq:non_vanishing_weighted_sum_of_first_moments_adam}
\end{equation}
unless $\eta^{(\tm)}_i = \eta^{(\tm)}_j$ for all $i, j \in \V$.
Hence, the \textit{sum of embedding update vectors does not vanish} in general,
\begin{equation}
\sum_{i=1}^V  u_i^{(\tm)}
\stackrel{(\ref{eq:update_vector_definition_Adam})}{=} - \sum_{i=1}^V \eta^{(\tm)}_i \cdot \firstmoment 
\stackrel{(\ref{eq:non_vanishing_weighted_sum_of_first_moments_adam})}{\neq} 0
\label{eq:non_vanishing_updates_Adam}
\end{equation}
This, in turn, causes the mean embedding to change during training,
\begin{equation} 
    \mu^{(\tm)} - \mu^{(\tm-1)} 
    \stackrel{(\ref{eq:mu},\ref{eq:update_general})}{=} 
    \frac{1}{V} \sum_{i=1}^V u_i^{(\tm)}
    \stackrel{(\ref{eq:non_vanishing_updates_Adam})}{\neq} 0 \; ,
    \label{eq:mean_embedding_change_adam}
\end{equation}
which is in stark contrast to the case of SGD (cf.~Eq.~\eqref{eq:invariant_mean_embedding_SGD}).
We have thus identified that an $i$-dependency of the second moment $\secondmoment$ of the Adam optimizer leads to the observed collective shift of the embedding vectors away from the origin.
Next, we will show that the second moment indeed depends on $i$. More concretely, we will argue that its expectation value is proportional to the unigram probabilitity\footnote{Note that from here until Eq.~(\ref{eq:optimizer_update_second_moment_avg_canonical}), the time index ($\tau$) is dropped for the sake of readability.} (see Eq.~(\ref{eq:unigram_probability})),
\begin{equation}
    \E \left[ \secondmomentshort \right] \propto \widetilde p_i 
    \label{eq:second_moment_linear_in_unigram_prob}
\end{equation}
In App.~\ref{app:second_moment_theory}, Eq.~(\ref{eq:second_moment_linear_in_unigram_prob}) is derived using minimal assumptions and experimental input.
Here, we restrict ourselves to confirming the relationship in a purely experimental manner. 
$\E \left[ \secondmomentshort \right]$ is estimated directly by measuring $\secondmomentshort$ multiple times during training, using different models. 
We then perform linear fits of $\E \left[ \secondmomentshort \right]$ as a function of $\widetilde p_i$. 
Indeed, the fits yield a high coefficient of determination, on average $R^2 = 0.85(7)$, and a proportionality constant of 
\begin{equation}
A := \frac{\E \left[ \secondmomentshort \right]}{\widetilde p_i} \approx 10^{-4}
\label{eq:second_moment_proportionality_constant}
\end{equation}
Details about the exact procedure and plots showing the data and linear fits can be found in App.~\ref{app:second_moment_empirical}.


\section{PILAF Algorithm}	\label{sec:sampling_exp}

    % \yunzhen{Julia: is it good to put it here or together with the sampling scheme in theory?}

We now demonstrate that the T-PILAF sampling scheme defined in \cref{eq:def_policythetapos} and (\ref{eq:def_policythetaneg}) can be naturally extended into an efficient empirical algorithm (PILAF).

The first challenge in implementing these definitions lies in calculating the normalizing factors $\Partitionthetapos(\prompt)$ and $\Partitionthetaneg(\prompt)$, which can be computationally expensive for LLMs. To address this, we simplify the process by omitting these factors and replacing them with 1.\footnote{When the regularization coefficient $\parabeta$ is sufficiently small, the term $\exp\{\rewardtheta(\prompt, \response)\}$ in equation~\eqref{eq:def_policythetapos} stays close to $1$ and has only a minor effect. Consequently, the partition function $\Partitionthetapos(\prompt)$ is approximately $1$. A similar reasoning applies to $\Partitionthetaneg(\prompt)$.
\vspace{-1.4em}
%\yaqi{Please check here.}
}
% \ariel{Should we, or can we, somehow justify that this is fine? Some intuition that this quantity is typically close to 1/2? Else, it would have been interesting to also explore this parameter}.
Consequently, the sampling process becomes straightforward: with probability~\(1/2\), we sample using \(\policytheta\), and otherwise, we sample using~\(\policythetapos\) and~\(\policythetaneg\).

The second challenge lies in sampling a response $\response$ from $\policytheta(\response \mid \prompt)
		\exp \big\{ \pm \rewardtheta(\prompt, \response) \big\}$ in an autoregressive way for next-token generation. 
We argue that the policy $\policythetapos$ (and $\policythetaneg$) can be approximated in a token-wise manner:
		\begin{align*}
			\policythetapos(\response \mid \prompt)
		 \; \approx \; \policythetapos(\tokent{1} \mid \prompt) \, \policythetapos(\tokent{2} \mid \prompt, \tokent{1})  \cdots \, \policythetapos(\tokent{t} \mid \prompt, \tokent{1:t-1}) \,
			\cdots \, \policythetapos(\tokent{\numtok} \mid \prompt, \tokent{1:\numtok-1}),
		\end{align*}
		where
		\begin{align*}
			& \policythetapos(\tokent{t} \mid \prompt, \tokenttot{1}{t-1}) \; = \; 
			\frac{1}{\Partition(\prompt, \tokenttot{1}{t-1})} \, 
			\policytheta(\tokent{t} \mid \prompt, \tokenttot{1}{t-1})
			\bigg( \frac{\policytheta(\tokent{t} \mid \prompt, \tokenttot{1}{t-1})}{\policyref(\tokent{t} \mid \prompt, \tokenttot{1}{t-1})} \bigg)^{\parabeta} % \\
	%		& \; = \; \frac{1}{\Partition(\prompt, \tokenttot{1}{t-1})} \, 
	%		\policyref(\diff \tokent{t} \mid \prompt, \tokenttot{1}{t-1})
	%		\bigg( \frac{\policytheta(\tokent{t} \mid \prompt, \tokenttot{1}{t-1})}{\policyref(\tokent{t} \mid \prompt, \tokenttot{1}{t-1})} \bigg)^{1+\parabeta}
		\end{align*}
		with $\Partition(\prompt, \tokenttot{1}{t-1})$ being a partition function. 
        %\ariel{re: this whole equation above. (1) Is this really an approximation? The first equation approximating $\pi_\theta^+(\vec{y}|x)$ looks fairly exact for a reasonable definition of $\pi_\theta^+(y_i|...)$ unless I'm missing something obvious; inthead the definition of $\pi_\theta^+(y_i|...)$ seems like an approximation. (2) Is there any meaning to switching the notation from $y_t$ to $dy_t$ between the equations?} \yaqi{(1) The approximation appears in the partition function. If we can divide the density function by the overall integration along the trajectory, then the token-wise expansion becomes exact. However, the current partition $\Partition$ is a step-wise normalization. (2) I have removed the $\diff$ notation.} 
        The substitution of $\rewardtheta$ uses the correspondence between the reward model 
        $\rewardtheta$ and the policy $\policytheta$ in \cref{eq:def_reward}, under the assumption that this correspondence holds for all truncations~$\tokenttot{1}{t-1}$. It gives us a direct per-token prediction rule:
    % \vspace{-1em}  % Ariel: I commented this out bc it was messing up the formatting
    \begin{align*}
     \policythetapos(\cdot \mid \prompt, \tokenttot{1}{t-1}) \; = \; \softmax\Big( \big\{ (1 + \parabeta) \, \headtheta - \parabeta \, \headref\big\} (\prompt, \tokenttot{1}{t-1}) \Big).
    \end{align*}
    % \vspace{-1.8em}
Here $ \headtheta $ and $ \headref $ are the logits of the policies $\policytheta$ and $\policyref$, respectively. $\parabeta$ is the regularization coefficient from the objective function $ \scalarvalue(\policy)$ in \cref{eq:objective}. Responses are then generated using standard decoding techniques, such as greedy decoding or nucleus sampling. Similarly, the generation for $\policythetaneg$ follows 
%\vspace{-.5em}
\begin{align*}
 \policythetaneg(\cdot \mid \prompt, \tokenttot{1}{t-1}) \; = \; \softmax\Big( \big\{ (1 - \parabeta) \, \headtheta + \parabeta \, \headref\big\} (\prompt, \tokenttot{1}{t-1}) \Big) \, .
\end{align*}
% \vspace{-1.8em}
For a detailed, step-by-step proof, see Proposition~1 in \citet{liu2024decoding}.

We formalize our final algorithm in \cref{alg:our_sampling}. Vanilla DPO \citep{rafailov2023direct, guo2024direct} employs a basic generation approach, sampling $\responseone_i, \responsetwo_i \sim \policytheta$ at Step~3. In contrast, instead of only sampling from $\policytheta$, our sampling scheme interpolates and extrapolates the logits~$\headtheta$ and~$\headref$ with coefficient $\parabeta$, enabling exploration of a wider response space to align learning from human preference with value optimization. The $\parabeta$ here is the same parameter that controls the KL regularization in \cref{eq:policy_loss_with_rm}, as set by the problem.
% \yunzhen{Do we need this paragraph?} 
%\ariel{It does read a bit weird in context, but I think it's useful to state something to the effect of "This is our sampling procedure; in contrast, standard DPO samples everything from $\pi_\theta$}

\paragraph{Cost analysis} We summarize sampling and annotation costs per preference pair for PILAF and related sampling schemes in \cref{tab:setup_summary}. In \textit{Vanilla} sampling (from $\policytheta$), two generations and two annotations are required for human preference labeling, same to PILAF when the pair is sampled from $\policytheta$, which happens half the time. With 50\% probability, PILAF uses $\policythetapos$ and $\policythetaneg$ to generate, requiring two forward passes with $\policytheta$ and $\policyref$ to generate one sample. Thus, on average, a preference pair sampled with PILAF requires a sampling cost of 3 forward passes (1.5 time the cost of \textit{Vanilla}) with the same annotation cost. To compare, \citet{xiong2024iterative, dong2024rlhf} perform \textit{Best-of-N} sampling with $N=8$, which generates and annotates all 8 responses, selecting the best and worst of them. \citet{xie2024exploratory} use a \textit{Hybrid} method that generates with $\policytheta$ and $\policyref$, thus matching the sampling and annotation costs of the \textit{Vanilla} method. We empirically compare PILAF with these methods in the next section.


    
		% Building on these principles, we propose a simple yet effective sampling scheme. For any given prompt $ \prompt $, generate two responses $(\responseone, \responsetwo)$ using one of two strategies, chosen with \emph{equal} probability: \vspace{-.5em}
		% \begin{itemize} % \itemsep = -.1em
		% 	\item 
		% 	Generate both responses independently from $\policytheta(\cdot \mid \prompt)$.
		% 	\item Sample $ \responseone $ from $ \policythetapos$ and $ \responsetwo $ from $ \policythetaneg$.
		% 	This can be done (approximately) in per-token prediction, via
		% 	\vspace{-.5em}
		% 	\begin{align*}
		% 		& \policythetapos(\cdot \mid \prompt, \tokenttot{1}{t-1}) \\
		% 		& \quad \; = \; \softmax\Big( \big\{ (1 + \parabeta) \, \headtheta - \parabeta \, \headref\big\} (\prompt, \tokenttot{1}{t-1}) \Big)  \\
		% 		& \policythetaneg(\cdot \mid \prompt, \tokenttot{1}{t-1})  \\
		% 		& \quad \; = \; \softmax\Big( \big\{ (1 - \parabeta) \, \headtheta + \parabeta \, \headref\big\} (\prompt, \tokenttot{1}{t-1}) \Big) \, .
		% 	\end{align*}
		% 	\vspace{-1.8em}
		% \end{itemize}
		% $\parabeta$ is the regularization coefficient from the objective function $ \scalarvalue(\policy)$ in equation~\eqref{eq:objective}. This adjustment shifts $ \policythetaneg$ closer to $ \policyref$ while pushing $ \policythetapos$ further away. 
		

  %       We formalize the algorithm in \cref{alg:our_sampling}.

\begin{algorithm}
\caption{DPO with PILAF (ours).}
\label{alg:our_sampling}
\begin{algorithmic}[1]
    \INPUT Prompt Dataset $\mathcal{D}_\rho$, preference oracle $\mathcal{O}$, $\policytheta, \policyref$.
    \FOR{step $t$ = 1, ..., $T$}
        \STATE Sample $n_t$ prompts $\{x_i\}_{i=1}^{n_t}$ from $\mathcal{D}_\rho$.
        \STATE \hl{With probability 1/2, sample $\responseone_i, \responsetwo_i \sim \policytheta$; with probability 1/2, sample $\responseone_i \sim \policythetapos$ and $\responsetwo_i \sim \policythetaneg$.}
        % \STATE \hl{}
        \STATE Query $\mathcal{O}$ to label $(x_i, \responseone_i, \responsetwo_i)$ into $(x_i, \responsewini{i}, \responselosei{i})$.
        \STATE Update $\policy_{\theta_t}$ with DPO loss using $\{(x_i, \responsewini{i}, \responselosei{i})\}_{i=1}^{n_t}$.
        % \IF{condition on $x$}
        %     \STATE Perform some operation
        % \ELSE
        %     \STATE Perform an alternative operation
        % \ENDIF
    \ENDFOR
\end{algorithmic}
\end{algorithm}
\vspace{-1em}




\begin{table*}
\vspace{-13pt}
    \caption{ \footnotesize A cost summary of PILAF and sampling methods from related works. \textit{Best-of-N} method in \citet{xiong2024iterative} uses the oracle reward to score all candidate responses, then selects the highest- and lowest-scoring ones—instead of providing a preference label for only two responses. We restrict the oracle to providing only preference labels. Thus, we create a \textit{Best-of-N} variant that uses the DPO internal reward for selection and then applies preference labeling, with an annotation cost of 2. We compare with this variant in the experiment.}
    \label{tab:setup_summary}
    \vskip 0.2in
    \centering
\begin{scriptsize}
% \setlength{\tabcolsep}{1.5pt}
\begin{sc}
    \begin{tabular}{l|cc|cc}
    \toprule
        \textbf{Method} & $\responseone$ & $\responsetwo$ & Sampling Cost & Annotation Cost \\ 
        \midrule
        \textit{Vanilla} \citep{rafailov2023direct} & $\policytheta$ & $\policytheta$ & 2 & 2 \\
        \textit{Best-of-N} \citep{xiong2024iterative}, N=8 & best of $\policytheta$ & worst of $\policytheta$ & 8 & 8* \\
        \textit{Best-of-N} (with DPO reward), N=8 & best of $\policytheta$ & worst of $\policytheta$ & 8 & 2 \\
        \textit{Hybrid} \citep{xie2024exploratory} & $\policytheta$ & \policyref & 2 & 2\\
        % \textit{SEA} \citep{liu2024sample} & & & 20 & 2 \\
        \midrule
        \textit{PILAF} (OURS) & $\policythetapos / \policytheta$ & $\policythetaneg / \policytheta$ & 3 & 2\\
    \bottomrule
    \end{tabular}
\end{sc}
\end{scriptsize}
% \vspace{-0.8em}
\end{table*}
    
    
	
	% \paragraph{Importance sampling method}
	
	% In the importance sampling method, we first generate \(m\) responses from \(\policytheta\) for each prompt. These responses, along with the prompts, are fed into the reward model to obtain reward values. As the generation method involves reweighting by $\exp \big\{\rewardtheta(\prompt, \response) \big\}$, we directly use these values to calculate and sample from the weighted probability distribution over the \(m\) sampled responses. In this process, \(m-2\) samples are discarded.
	
	% \yaqiadd{I am still unclear about the weights used—could you provide more technical details here or in the Appendix?}
	
	% \paragraph{Direct sampling method}
	
	% Additionally, we propose a direct sampling approach, where the logits are modified token-wise during generation.

\section{Experiments}

\subsection{Setups}
\subsubsection{Implementation Details}
We apply our FDS method to two types of 3DGS: 
the original 3DGS, and 2DGS~\citep{huang20242d}. 
%
The number of iterations in our optimization 
process is 35,000.
We follow the default training configuration 
and apply our FDS method after 15,000 iterations,
then we add normal consistency loss for both
3DGS and 2DGS after 25000 iterations.
%
The weight for FDS, $\lambda_{fds}$, is set to 0.015,
the $\sigma$ is set to 23,
and the weight for normal consistency is set to 0.15
for all experiments. 
We removed the depth distortion loss in 2DGS 
because we found that it degrades its results in indoor scenes.
%
The Gaussian point cloud is initialized using Colmap
for all datasets.
%
%
We tested the impact of 
using Sea Raft~\citep{wang2025sea} and 
Raft\citep{teed2020raft} on FDS performance.
%
Due to the blurriness of the ScanNet dataset, 
additional prior constraints are required.
Thus, we incorporate normal prior supervision 
on the rendered normals 
in ScanNet (V2) dataset by default.
The normal prior is predicted by the Stable Normal 
model~\citep{ye2024stablenormal}
across all types of 3DGS.
%
The entire framework is implemented in 
PyTorch~\citep{paszke2019pytorch}, 
and all experiments are conducted on 
a single NVIDIA 4090D GPU.

\begin{figure}[t] \centering
    \makebox[0.16\textwidth]{\scriptsize Input}
    \makebox[0.16\textwidth]{\scriptsize 3DGS}
    \makebox[0.16\textwidth]{\scriptsize 2DGS}
    \makebox[0.16\textwidth]{\scriptsize 3DGS + FDS}
    \makebox[0.16\textwidth]{\scriptsize 2DGS + FDS}
    \makebox[0.16\textwidth]{\scriptsize GT (Depth)}

    \includegraphics[width=0.16\textwidth]{figure/fig3_img/compare3/gt_rgb/frame_00522.jpg}
    \includegraphics[width=0.16\textwidth]{figure/fig3_img/compare3/3DGS/frame_00522.jpg}
    \includegraphics[width=0.16\textwidth]{figure/fig3_img/compare3/2DGS/frame_00522.jpg}
    \includegraphics[width=0.16\textwidth]{figure/fig3_img/compare3/3DGS+FDS/frame_00522.jpg}
    \includegraphics[width=0.16\textwidth]{figure/fig3_img/compare3/2DGS+FDS/frame_00522.jpg}
    \includegraphics[width=0.16\textwidth]{figure/fig3_img/compare3/gt_depth/frame_00522.jpg} \\

    % \includegraphics[width=0.16\textwidth]{figure/fig3_img/compare1/gt_rgb/frame_00137.jpg}
    % \includegraphics[width=0.16\textwidth]{figure/fig3_img/compare1/3DGS/frame_00137.jpg}
    % \includegraphics[width=0.16\textwidth]{figure/fig3_img/compare1/2DGS/frame_00137.jpg}
    % \includegraphics[width=0.16\textwidth]{figure/fig3_img/compare1/3DGS+FDS/frame_00137.jpg}
    % \includegraphics[width=0.16\textwidth]{figure/fig3_img/compare1/2DGS+FDS/frame_00137.jpg}
    % \includegraphics[width=0.16\textwidth]{figure/fig3_img/compare1/gt_depth/frame_00137.jpg} \\

     \includegraphics[width=0.16\textwidth]{figure/fig3_img/compare2/gt_rgb/frame_00262.jpg}
    \includegraphics[width=0.16\textwidth]{figure/fig3_img/compare2/3DGS/frame_00262.jpg}
    \includegraphics[width=0.16\textwidth]{figure/fig3_img/compare2/2DGS/frame_00262.jpg}
    \includegraphics[width=0.16\textwidth]{figure/fig3_img/compare2/3DGS+FDS/frame_00262.jpg}
    \includegraphics[width=0.16\textwidth]{figure/fig3_img/compare2/2DGS+FDS/frame_00262.jpg}
    \includegraphics[width=0.16\textwidth]{figure/fig3_img/compare2/gt_depth/frame_00262.jpg} \\

    \includegraphics[width=0.16\textwidth]{figure/fig3_img/compare4/gt_rgb/frame00000.png}
    \includegraphics[width=0.16\textwidth]{figure/fig3_img/compare4/3DGS/frame00000.png}
    \includegraphics[width=0.16\textwidth]{figure/fig3_img/compare4/2DGS/frame00000.png}
    \includegraphics[width=0.16\textwidth]{figure/fig3_img/compare4/3DGS+FDS/frame00000.png}
    \includegraphics[width=0.16\textwidth]{figure/fig3_img/compare4/2DGS+FDS/frame00000.png}
    \includegraphics[width=0.16\textwidth]{figure/fig3_img/compare4/gt_depth/frame00000.png} \\

    \includegraphics[width=0.16\textwidth]{figure/fig3_img/compare5/gt_rgb/frame00080.png}
    \includegraphics[width=0.16\textwidth]{figure/fig3_img/compare5/3DGS/frame00080.png}
    \includegraphics[width=0.16\textwidth]{figure/fig3_img/compare5/2DGS/frame00080.png}
    \includegraphics[width=0.16\textwidth]{figure/fig3_img/compare5/3DGS+FDS/frame00080.png}
    \includegraphics[width=0.16\textwidth]{figure/fig3_img/compare5/2DGS+FDS/frame00080.png}
    \includegraphics[width=0.16\textwidth]{figure/fig3_img/compare5/gt_depth/frame00080.png} \\



    \caption{\textbf{Comparison of depth reconstruction on Mushroom and ScanNet datasets.} The original
    3DGS or 2DGS model equipped with FDS can remove unwanted floaters and reconstruct
    geometry more preciously.}
    \label{fig:compare}
\end{figure}


\subsubsection{Datasets and Metrics}

We evaluate our method for 3D reconstruction 
and novel view synthesis tasks on
\textbf{Mushroom}~\citep{ren2024mushroom},
\textbf{ScanNet (v2)}~\citep{dai2017scannet}, and 
\textbf{Replica}~\citep{replica19arxiv}
datasets,
which feature challenging indoor scenes with both 
sparse and dense image sampling.
%
The Mushroom dataset is an indoor dataset 
with sparse image sampling and two distinct 
camera trajectories. 
%
We train our model on the training split of 
the long capture sequence and evaluate 
novel view synthesis on the test split 
of the long capture sequences.
%
Five scenes are selected to evaluate our FDS, 
including "coffee room", "honka", "kokko", 
"sauna", and "vr room". 
%
ScanNet(V2)~\citep{dai2017scannet}  consists of 1,613 indoor scenes
with annotated camera poses and depth maps. 
%
We select 5 scenes from the ScanNet (V2) dataset, 
uniformly sampling one-tenth of the views,
following the approach in ~\citep{guo2022manhattan}.
To further improve the geometry rendering quality of 3DGS, 
%
Replica~\citep{replica19arxiv} contains small-scale 
real-world indoor scans. 
We evaluate our FDS on five scenes from 
Replica: office0, office1, office2, office3 and office4,
selecting one-tenth of the views for training.
%
The results for Replica are provided in the 
supplementary materials.
To evaluate the rendering quality and geometry 
of 3DGS, we report PSNR, SSIM, and LPIPS for 
rendering quality, along with Absolute Relative Distance 
(Abs Rel) as a depth quality metrics.
%
Additionally, for mesh evaluation, 
we use metrics including Accuracy, Completion, 
Chamfer-L1 distance, Normal Consistency, 
and F-scores.




\subsection{Results}
\subsubsection{Depth rendering and novel view synthesis}
The comparison results on Mushroom and 
ScanNet are presented in \tabref{tab:mushroom} 
and \tabref{tab:scannet}, respectively. 
%
Due to the sparsity of sampling 
in the Mushroom dataset,
challenges are posed for both GOF~\citep{yu2024gaussian} 
and PGSR~\citep{chen2024pgsr}, 
leading to their relative poor performance 
on the Mushroom dataset.
%
Our approach achieves the best performance 
with the FDS method applied during the training process.
The FDS significantly enhances the 
geometric quality of 3DGS on the Mushroom dataset, 
improving the "abs rel" metric by more than 50\%.
%
We found that Sea Raft~\citep{wang2025sea}
outperforms Raft~\citep{teed2020raft} on FDS, 
indicating that a better optical flow model 
can lead to more significant improvements.
%
Additionally, the render quality of RGB 
images shows a slight improvement, 
by 0.58 in 3DGS and 0.50 in 2DGS, 
benefiting from the incorporation of cross-view consistency in FDS. 
%
On the Mushroom
dataset, adding the FDS loss increases 
the training time by half an hour, which maintains the same
level as baseline.
%
Similarly, our method shows a notable improvement on the ScanNet dataset as well using Sea Raft~\citep{wang2025sea} Model. The "abs rel" metric in 2DGS is improved nearly 50\%. This demonstrates the robustness and effectiveness of the FDS method across different datasets.
%


% \begin{wraptable}{r}{0.6\linewidth} \centering
% \caption{\textbf{Ablation study on geometry priors.}} 
%         \label{tab:analysis_prior}
%         \resizebox{\textwidth}{!}{
\begin{tabular}{c| c c c c c | c c c c}

    \hline
     Method &  Acc$\downarrow$ & Comp $\downarrow$ & C-L1 $\downarrow$ & NC $\uparrow$ & F-Score $\uparrow$ &  Abs Rel $\downarrow$ &  PSNR $\uparrow$  & SSIM  $\uparrow$ & LPIPS $\downarrow$ \\ \hline
    2DGS&   0.1078&  0.0850&  0.0964&  0.7835&  0.5170&  0.1002&  23.56&  0.8166& 0.2730\\
    2DGS+Depth&   0.0862&  0.0702&  0.0782&  0.8153&  0.5965&  0.0672&  23.92&  0.8227& 0.2619 \\
    2DGS+MVDepth&   0.2065&  0.0917&  0.1491&  0.7832&  0.3178&  0.0792&  23.74&  0.8193& 0.2692 \\
    2DGS+Normal&   0.0939&  0.0637&  0.0788&  \textbf{0.8359}&  0.5782&  0.0768&  23.78&  0.8197& 0.2676 \\
    2DGS+FDS &  \textbf{0.0615} & \textbf{ 0.0534}& \textbf{0.0574}& 0.8151& \textbf{0.6974}&  \textbf{0.0561}&  \textbf{24.06}&  \textbf{0.8271}&\textbf{0.2610} \\ \hline
    2DGS+Depth+FDS &  0.0561 &  0.0519& 0.0540& 0.8295& 0.7282&  0.0454&  \textbf{24.22}& \textbf{0.8291}&\textbf{0.2570} \\
    2DGS+Normal+FDS &  \textbf{0.0529} & \textbf{ 0.0450}& \textbf{0.0490}& \textbf{0.8477}& \textbf{0.7430}&  \textbf{0.0443}&  24.10&  0.8283& 0.2590 \\
    2DGS+Depth+Normal &  0.0695 & 0.0513& 0.0604& 0.8540&0.6723&  0.0523&  24.09&  0.8264&0.2575\\ \hline
    2DGS+Depth+Normal+FDS &  \textbf{0.0506} & \textbf{0.0423}& \textbf{0.0464}& \textbf{0.8598}&\textbf{0.7613}&  \textbf{0.0403}&  \textbf{24.22}& 
    \textbf{0.8300}&\textbf{0.0403}\\
    
\bottomrule
\end{tabular}
}
% \end{wraptable}



The qualitative comparisons on the Mushroom and ScanNet dataset 
are illustrated in \figref{fig:compare}. 
%
%
As seen in the first row of \figref{fig:compare}, 
both the original 3DGS and 2DGS suffer from overfitting, 
leading to corrupted geometry generation. 
%
Our FDS effectively mitigates this issue by 
supervising the matching relationship between 
the input and sampled views, 
helping to recover the geometry.
%
FDS also improves the refinement of geometric details, 
as shown in other rows. 
By incorporating the matching prior through FDS, 
the quality of the rendered depth is significantly improved.
%

\begin{table}[t] \centering
\begin{minipage}[t]{0.96\linewidth}
        \captionof{table}{\textbf{3D Reconstruction 
        and novel view synthesis results on Mushroom dataset. * 
        Represents that FDS uses the Raft model.
        }}
        \label{tab:mushroom}
        \resizebox{\textwidth}{!}{
\begin{tabular}{c| c c c c c | c c c c c}
    \hline
     Method &  Acc$\downarrow$ & Comp $\downarrow$ & C-L1 $\downarrow$ & NC $\uparrow$ & F-Score $\uparrow$ &  Abs Rel $\downarrow$ &  PSNR $\uparrow$  & SSIM  $\uparrow$ & LPIPS $\downarrow$ & Time  $\downarrow$ \\ \hline

    % DN-splatter &   &  &  &  &  &  &  &  & \\
    GOF &  0.1812 & 0.1093 & 0.1453 & 0.6292 & 0.3665 & 0.2380  & 21.37  &  0.7762  & 0.3132  & $\approx$1.4h\\ 
    PGSR &  0.0971 & 0.1420 & 0.1196 & 0.7193 & 0.5105 & 0.1723  & 22.13  & 0.7773  & 0.2918  & $\approx$1.2h \\ \hline
    3DGS &   0.1167 &  0.1033&  0.1100&  0.7954&  0.3739&  0.1214&  24.18&  0.8392& 0.2511 &$\approx$0.8h \\
    3DGS + FDS$^*$ & 0.0569  & 0.0676 & 0.0623 & 0.8105 & 0.6573 & 0.0603 & 24.72  & 0.8489 & 0.2379 &$\approx$1.3h \\
    3DGS + FDS & \textbf{0.0527}  & \textbf{0.0565} & \textbf{0.0546} & \textbf{0.8178} & \textbf{0.6958} & \textbf{0.0568} & \textbf{24.76}  & \textbf{0.8486} & \textbf{0.2381} &$\approx$1.3h \\ \hline
    2DGS&   0.1078&  0.0850&  0.0964&  0.7835&  0.5170&  0.1002&  23.56&  0.8166& 0.2730 &$\approx$0.8h\\
    2DGS + FDS$^*$ &  0.0689 &  0.0646& 0.0667& 0.8042& 0.6582& 0.0589& 23.98&  0.8255&0.2621 &$\approx$1.3h\\
    2DGS + FDS &  \textbf{0.0615} & \textbf{ 0.0534}& \textbf{0.0574}& \textbf{0.8151}& \textbf{0.6974}&  \textbf{0.0561}&  \textbf{24.06}&  \textbf{0.8271}&\textbf{0.2610} &$\approx$1.3h \\ \hline
\end{tabular}
}
\end{minipage}\hfill
\end{table}

\begin{table}[t] \centering
\begin{minipage}[t]{0.96\linewidth}
        \captionof{table}{\textbf{3D Reconstruction 
        and novel view synthesis results on ScanNet dataset.}}
        \label{tab:scannet}
        \resizebox{\textwidth}{!}{
\begin{tabular}{c| c c c c c | c c c c }
    \hline
     Method &  Acc $\downarrow$ & Comp $\downarrow$ & C-L1 $\downarrow$ & NC $\uparrow$ & F-Score $\uparrow$ &  Abs Rel $\downarrow$ &  PSNR $\uparrow$  & SSIM  $\uparrow$ & LPIPS $\downarrow$ \\ \hline
    GOF & 1.8671  & 0.0805 & 0.9738 & 0.5622 & 0.2526 & 0.1597  & 21.55  & 0.7575  & 0.3881 \\
    PGSR &  0.2928 & 0.5103 & 0.4015 & 0.5567 & 0.1926 & 0.1661  & 21.71 & 0.7699  & 0.3899 \\ \hline

    3DGS &  0.4867 & 0.1211 & 0.3039 & 0.7342& 0.3059 & 0.1227 & 22.19& 0.7837 & 0.3907\\
    3DGS + FDS &  \textbf{0.2458} & \textbf{0.0787} & \textbf{0.1622} & \textbf{0.7831} & 
    \textbf{0.4482} & \textbf{0.0573} & \textbf{22.83} & \textbf{0.7911} & \textbf{0.3826} \\ \hline
    2DGS &  0.2658 & 0.0845 & 0.1752 & 0.7504& 0.4464 & 0.0831 & 22.59& 0.7881 & 0.3854\\
    2DGS + FDS &  \textbf{0.1457} & \textbf{0.0679} & \textbf{0.1068} & \textbf{0.7883} & 
    \textbf{0.5459} & \textbf{0.0432} & \textbf{22.91} & \textbf{0.7928} & \textbf{0.3800} \\ \hline
\end{tabular}
}
\end{minipage}\hfill
\end{table}


\begin{table}[t] \centering
\begin{minipage}[t]{0.96\linewidth}
        \captionof{table}{\textbf{Ablation study on geometry priors.}}
        \label{tab:analysis_prior}
        \resizebox{\textwidth}{!}{
\begin{tabular}{c| c c c c c | c c c c}

    \hline
     Method &  Acc$\downarrow$ & Comp $\downarrow$ & C-L1 $\downarrow$ & NC $\uparrow$ & F-Score $\uparrow$ &  Abs Rel $\downarrow$ &  PSNR $\uparrow$  & SSIM  $\uparrow$ & LPIPS $\downarrow$ \\ \hline
    2DGS&   0.1078&  0.0850&  0.0964&  0.7835&  0.5170&  0.1002&  23.56&  0.8166& 0.2730\\
    2DGS+Depth&   0.0862&  0.0702&  0.0782&  0.8153&  0.5965&  0.0672&  23.92&  0.8227& 0.2619 \\
    2DGS+MVDepth&   0.2065&  0.0917&  0.1491&  0.7832&  0.3178&  0.0792&  23.74&  0.8193& 0.2692 \\
    2DGS+Normal&   0.0939&  0.0637&  0.0788&  \textbf{0.8359}&  0.5782&  0.0768&  23.78&  0.8197& 0.2676 \\
    2DGS+FDS &  \textbf{0.0615} & \textbf{ 0.0534}& \textbf{0.0574}& 0.8151& \textbf{0.6974}&  \textbf{0.0561}&  \textbf{24.06}&  \textbf{0.8271}&\textbf{0.2610} \\ \hline
    2DGS+Depth+FDS &  0.0561 &  0.0519& 0.0540& 0.8295& 0.7282&  0.0454&  \textbf{24.22}& \textbf{0.8291}&\textbf{0.2570} \\
    2DGS+Normal+FDS &  \textbf{0.0529} & \textbf{ 0.0450}& \textbf{0.0490}& \textbf{0.8477}& \textbf{0.7430}&  \textbf{0.0443}&  24.10&  0.8283& 0.2590 \\
    2DGS+Depth+Normal &  0.0695 & 0.0513& 0.0604& 0.8540&0.6723&  0.0523&  24.09&  0.8264&0.2575\\ \hline
    2DGS+Depth+Normal+FDS &  \textbf{0.0506} & \textbf{0.0423}& \textbf{0.0464}& \textbf{0.8598}&\textbf{0.7613}&  \textbf{0.0403}&  \textbf{24.22}& 
    \textbf{0.8300}&\textbf{0.0403}\\
    
\bottomrule
\end{tabular}
}
\end{minipage}\hfill
\end{table}




\subsubsection{Mesh extraction}
To further demonstrate the improvement in geometry quality, 
we applied methods used in ~\citep{turkulainen2024dnsplatter} 
to extract meshes from the input views of optimized 3DGS. 
The comparison results are presented  
in \tabref{tab:mushroom}. 
With the integration of FDS, the mesh quality is significantly enhanced compared to the baseline, featuring fewer floaters and more well-defined shapes.
 %
% Following the incorporation of FDS, the reconstruction 
% results exhibit fewer floaters and more well-defined 
% shapes in the meshes. 
% Visualized comparisons
% are provided in the supplementary material.

% \begin{figure}[t] \centering
%     \makebox[0.19\textwidth]{\scriptsize GT}
%     \makebox[0.19\textwidth]{\scriptsize 3DGS}
%     \makebox[0.19\textwidth]{\scriptsize 3DGS+FDS}
%     \makebox[0.19\textwidth]{\scriptsize 2DGS}
%     \makebox[0.19\textwidth]{\scriptsize 2DGS+FDS} \\

%     \includegraphics[width=0.19\textwidth]{figure/fig4_img/compare1/gt02.png}
%     \includegraphics[width=0.19\textwidth]{figure/fig4_img/compare1/baseline06.png}
%     \includegraphics[width=0.19\textwidth]{figure/fig4_img/compare1/baseline_fds05.png}
%     \includegraphics[width=0.19\textwidth]{figure/fig4_img/compare1/2dgs04.png}
%     \includegraphics[width=0.19\textwidth]{figure/fig4_img/compare1/2dgs_fds03.png} \\

%     \includegraphics[width=0.19\textwidth]{figure/fig4_img/compare2/gt00.png}
%     \includegraphics[width=0.19\textwidth]{figure/fig4_img/compare2/baseline02.png}
%     \includegraphics[width=0.19\textwidth]{figure/fig4_img/compare2/baseline_fds01.png}
%     \includegraphics[width=0.19\textwidth]{figure/fig4_img/compare2/2dgs04.png}
%     \includegraphics[width=0.19\textwidth]{figure/fig4_img/compare2/2dgs_fds03.png} \\
      
%     \includegraphics[width=0.19\textwidth]{figure/fig4_img/compare3/gt05.png}
%     \includegraphics[width=0.19\textwidth]{figure/fig4_img/compare3/3dgs03.png}
%     \includegraphics[width=0.19\textwidth]{figure/fig4_img/compare3/3dgs_fds04.png}
%     \includegraphics[width=0.19\textwidth]{figure/fig4_img/compare3/2dgs02.png}
%     \includegraphics[width=0.19\textwidth]{figure/fig4_img/compare3/2dgs_fds01.png} \\

%     \caption{\textbf{Qualitative comparison of extracted mesh 
%     on Mushroom and ScanNet datasets.}}
%     \label{fig:mesh}
% \end{figure}












\subsection{Ablation study}


\textbf{Ablation study on geometry priors:} 
To highlight the advantage of incorporating matching priors, 
we incorporated various types of priors generated by different 
models into 2DGS. These include a monocular depth estimation
model (Depth Anything v2)~\citep{yang2024depth}, a two-view depth estimation 
model (Unimatch)~\citep{xu2023unifying}, 
and a monocular normal estimation model (DSINE)~\citep{bae2024rethinking}.
We adapt the scale and shift-invariant loss in Midas~\citep{birkl2023midas} for
monocular depth supervision and L1 loss for two-view depth supervison.
%
We use Sea Raft~\citep{wang2025sea} as our default optical flow model.
%
The comparison results on Mushroom dataset 
are shown in ~\tabref{tab:analysis_prior}.
We observe that the normal prior provides accurate shape information, 
enhancing the geometric quality of the radiance field. 
%
% In contrast, the monocular depth prior slightly increases 
% the 'Abs Rel' due to its ambiguous scale and inaccurate depth ordering.
% Moreover, the performance of monocular depth estimation 
% in the sauna scene is particularly poor, 
% primarily due to the presence of numerous reflective 
% surfaces and textureless walls, which limits the accuracy of monocular depth estimation.
%
The multi-view depth prior, hindered by the limited feature overlap 
between input views, fails to offer reliable geometric 
information. We test average "Abs Rel" of multi-view depth prior
, and the result is 0.19, which performs worse than the "Abs Rel" results 
rendered by original 2DGS.
From the results, it can be seen that depth order information provided by monocular depth improves
reconstruction accuracy. Meanwhile, our FDS achieves the best performance among all the priors, 
and by integrating all
three components, we obtained the optimal results.
%
%
\begin{figure}[t] \centering
    \makebox[0.16\textwidth]{\scriptsize RF (16000 iters)}
    \makebox[0.16\textwidth]{\scriptsize RF* (20000 iters)}
    \makebox[0.16\textwidth]{\scriptsize RF (20000 iters)  }
    \makebox[0.16\textwidth]{\scriptsize PF (16000 iters)}
    \makebox[0.16\textwidth]{\scriptsize PF (20000 iters)}


    % \includegraphics[width=0.16\textwidth]{figure/fig5_img/compare1/16000.png}
    % \includegraphics[width=0.16\textwidth]{figure/fig5_img/compare1/20000_wo_flow_loss.png}
    % \includegraphics[width=0.16\textwidth]{figure/fig5_img/compare1/20000.png}
    % \includegraphics[width=0.16\textwidth]{figure/fig5_img/compare1/16000_prior.png}
    % \includegraphics[width=0.16\textwidth]{figure/fig5_img/compare1/20000_prior.png}\\

    % \includegraphics[width=0.16\textwidth]{figure/fig5_img/compare2/16000.png}
    % \includegraphics[width=0.16\textwidth]{figure/fig5_img/compare2/20000_wo_flow_loss.png}
    % \includegraphics[width=0.16\textwidth]{figure/fig5_img/compare2/20000.png}
    % \includegraphics[width=0.16\textwidth]{figure/fig5_img/compare2/16000_prior.png}
    % \includegraphics[width=0.16\textwidth]{figure/fig5_img/compare2/20000_prior.png}\\

    \includegraphics[width=0.16\textwidth]{figure/fig5_img/compare3/16000.png}
    \includegraphics[width=0.16\textwidth]{figure/fig5_img/compare3/20000_wo_flow_loss.png}
    \includegraphics[width=0.16\textwidth]{figure/fig5_img/compare3/20000.png}
    \includegraphics[width=0.16\textwidth]{figure/fig5_img/compare3/16000_prior.png}
    \includegraphics[width=0.16\textwidth]{figure/fig5_img/compare3/20000_prior.png}\\
    
    \includegraphics[width=0.16\textwidth]{figure/fig5_img/compare4/16000.png}
    \includegraphics[width=0.16\textwidth]{figure/fig5_img/compare4/20000_wo_flow_loss.png}
    \includegraphics[width=0.16\textwidth]{figure/fig5_img/compare4/20000.png}
    \includegraphics[width=0.16\textwidth]{figure/fig5_img/compare4/16000_prior.png}
    \includegraphics[width=0.16\textwidth]{figure/fig5_img/compare4/20000_prior.png}\\

    \includegraphics[width=0.30\textwidth]{figure/fig5_img/bar.png}

    \caption{\textbf{The error map of Radiance Flow and Prior Flow.} RF: Radiance Flow, PF: Prior Flow, * means that there is no FDS loss supervision during optimization.}
    \label{fig:error_map}
\end{figure}




\textbf{Ablation study on FDS: }
In this section, we present the design of our FDS 
method through an ablation study on the 
Mushroom dataset to validate its effectiveness.
%
The optional configurations of FDS are outlined in ~\tabref{tab:ablation_fds}.
Our base model is the 2DGS equipped with FDS,
and its results are shown 
in the first row. The goal of this analysis 
is to evaluate the impact 
of various strategies on FDS sampling and loss design.
%
We observe that when we 
replace $I_i$ in \eqref{equ:mflow} with $C_i$, 
as shown in the second row, the geometric quality 
of 2DGS deteriorates. Using $I_i$ instead of $C_i$ 
help us to remove the floaters in $\bm{C^s}$, which are also 
remained in $\bm{C^i}$.
We also experiment with modifying the FDS loss. For example, 
in the third row, we use the neighbor 
input view as the sampling view, and replace the 
render result of neighbor view with ground truth image of its input view.
%
Due to the significant movement between images, the Prior Flow fails to accurately 
match the pixel between them, leading to a further degradation in geometric quality.
%
Finally, we attempt to fix the sampling view 
and found that this severely damaged the geometric quality, 
indicating that random sampling is essential for the stability 
of the mean error in the Prior flow.



\begin{table}[t] \centering

\begin{minipage}[t]{1.0\linewidth}
        \captionof{table}{\textbf{Ablation study on FDS strategies.}}
        \label{tab:ablation_fds}
        \resizebox{\textwidth}{!}{
\begin{tabular}{c|c|c|c|c|c|c|c}
    \hline
    \multicolumn{2}{c|}{$\mathcal{M}_{\theta}(X, \bm{C^s})$} & \multicolumn{3}{c|}{Loss} & \multicolumn{3}{c}{Metric}  \\
    \hline
    $X=C^i$ & $X=I^i$  & Input view & Sampled view     & Fixed Sampled view        & Abs Rel $\downarrow$ & F-score $\uparrow$ & NC $\uparrow$ \\
    \hline
    & \ding{51} &     &\ding{51}    &    &    \textbf{0.0561}        &  \textbf{0.6974}         & \textbf{0.8151}\\
    \hline
     \ding{51} &           &     &\ding{51}    &    &    0.0839        &  0.6242         &0.8030\\
     &  \ding{51} &   \ding{51}  &    &    &    0.0877       & 0.6091        & 0.7614 \\
      &  \ding{51} &    &    & \ding{51}    &    0.0724           & 0.6312          & 0.8015 \\
\bottomrule
\end{tabular}
}
\end{minipage}
\end{table}




\begin{figure}[htbp] \centering
    \makebox[0.22\textwidth]{}
    \makebox[0.22\textwidth]{}
    \makebox[0.22\textwidth]{}
    \makebox[0.22\textwidth]{}
    \\

    \includegraphics[width=0.22\textwidth]{figure/fig6_img/l1/rgb/frame00096.png}
    \includegraphics[width=0.22\textwidth]{figure/fig6_img/l1/render_rgb/frame00096.png}
    \includegraphics[width=0.22\textwidth]{figure/fig6_img/l1/render_depth/frame00096.png}
    \includegraphics[width=0.22\textwidth]{figure/fig6_img/l1/depth/frame00096.png}

    % \includegraphics[width=0.22\textwidth]{figure/fig6_img/l2/rgb/frame00112.png}
    % \includegraphics[width=0.22\textwidth]{figure/fig6_img/l2/render_rgb/frame00112.png}
    % \includegraphics[width=0.22\textwidth]{figure/fig6_img/l2/render_depth/frame00112.png}
    % \includegraphics[width=0.22\textwidth]{figure/fig6_img/l2/depth/frame00112.png}

    \caption{\textbf{Limitation of FDS.} }
    \label{fig:limitation}
\end{figure}


% \begin{figure}[t] \centering
%     \makebox[0.48\textwidth]{}
%     \makebox[0.48\textwidth]{}
%     \\
%     \includegraphics[width=0.48\textwidth]{figure/loss_Ignatius.pdf}
%     \includegraphics[width=0.48\textwidth]{figure/loss_family.pdf}
%     \caption{\textbf{Comparison the photometric error of Radiance Flow and Prior Flow:} 
%     We add FDS method after 2k iteration during training.
%     The results show
%     that:  1) The Prior Flow is more precise and 
%     robust than Radiance Flow during the radiance 
%     optimization; 2) After adding the FDS loss 
%     which utilize Prior 
%     flow to supervise the Radiance Flow at 2k iterations, 
%     both flow are more accurate, which lead to
%     a mutually reinforcing effects.(TODO fix it)} 
%     \label{fig:flowcompare}
% \end{figure}






\textbf{Interpretive Experiments: }
To demonstrate the mutual refinement of two flows in our FDS, 
For each view, we sample the unobserved 
views multiple times to compute the mean error 
of both Radiance Flow and Prior Flow. 
We use Raft~\citep{teed2020raft} as our default optical flow model
for visualization.
The ground truth flow is calculated based on 
~\eref{equ:flow_pose} and ~\eref{equ:flow} 
utilizing ground truth depth in dataset.
We introduce our FDS loss after 16000 iterations during 
optimization of 2DGS.
The error maps are shown in ~\figref{fig:error_map}.
Our analysis reveals that Radiance Flow tends to 
exhibit significant geometric errors, 
whereas Prior Flow can more accurately estimate the geometry,
effectively disregarding errors introduced by floating Gaussian points. 

%





\subsection{Limitation and further work}

Firstly, our FDS faces challenges in scenes with 
significant lighting variations between different 
views, as shown in the lamp of first row in ~\figref{fig:limitation}. 
%
Incorporating exposure compensation into FDS could help address this issue. 
%
 Additionally, our method struggles with 
 reflective surfaces and motion blur,
 leading to incorrect matching. 
 %
 In the future, we plan to explore the potential 
 of FDS in monocular video reconstruction tasks, 
 using only a single input image at each time step.
 


\section{Conclusions}
In this paper, we propose Flow Distillation Sampling (FDS), which
leverages the matching prior between input views and 
sampled unobserved views from the pretrained optical flow model, to improve the geometry quality
of Gaussian radiance field. 
Our method can be applied to different approaches (3DGS and 2DGS) to enhance the geometric rendering quality of the corresponding neural radiance fields.
We apply our method to the 3DGS-based framework, 
and the geometry is enhanced on the Mushroom, ScanNet, and Replica datasets.

\section*{Acknowledgements} This work was supported by 
National Key R\&D Program of China (2023YFB3209702), 
the National Natural Science Foundation of 
China (62441204, 62472213), and Gusu 
Innovation \& Entrepreneurship Leading Talents Program (ZXL2024361)

\section{Conclusion}
We introduce a novel approach, \algo, to reduce human feedback requirements in preference-based reinforcement learning by leveraging vision-language models. While VLMs encode rich world knowledge, their direct application as reward models is hindered by alignment issues and noisy predictions. To address this, we develop a synergistic framework where limited human feedback is used to adapt VLMs, improving their reliability in preference labeling. Further, we incorporate a selective sampling strategy to mitigate noise and prioritize informative human annotations.

Our experiments demonstrate that this method significantly improves feedback efficiency, achieving comparable or superior task performance with up to 50\% fewer human annotations. Moreover, we show that an adapted VLM can generalize across similar tasks, further reducing the need for new human feedback by 75\%. These results highlight the potential of integrating VLMs into preference-based RL, offering a scalable solution to reducing human supervision while maintaining high task success rates. 

\section*{Impact Statement}
This work advances embodied AI by significantly reducing the human feedback required for training agents. This reduction is particularly valuable in robotic applications where obtaining human demonstrations and feedback is challenging or impractical, such as assistive robotic arms for individuals with mobility impairments. By minimizing the feedback requirements, our approach enables users to more efficiently customize and teach new skills to robotic agents based on their specific needs and preferences. The broader impact of this work extends to healthcare, assistive technology, and human-robot interaction. One possible risk is that the bias from human feedback can propagate to the VLM and subsequently to the policy. This can be mitigated by personalization of agents in case of household application or standardization of feedback for industrial applications. 

% \endgroup

% \bibliographystyle{natbib} 
\bibliography{main}
\bibliographystyle{icml2025}
	
	
%%%%%%%%%%%%%%%%%%%%%%%%%%%%%%%%%%%%%%%%%%%%%%%%%%%%%%%%%%%%%%%%%%%%%%%%%%%
	


\newpage
\appendix
\onecolumn


\section*{Contents}

{\footnotesize
\hyperref[app:related_work]{\textbf{\ref{app:related_work}}.
Additional Literature Review}
\dotfill
\pageref{app:related_work}

%\hyperref[app:add_stat_results]
%{\textbf{\ref{app:add_stat_results}}.
%Additional Statistical Results}
%\dotfill
%\pageref{app:add_stat_results}

\hyperref[app:proof:main]
{\textbf{\ref{app:proof:main}}.
Proof of Main Results}
\dotfill
\pageref{app:proof:main}

~~~~\hyperref[sec:proof:thm:grad]
{\textbf{\ref{sec:proof:thm:grad}}.
Optimization Considerations: Proof of Theorem~\ref{thm:grad}}
\dotfill
\pageref{sec:proof:thm:grad}

~~~~~~~~\hyperref[sec:proof:thm:grad_1]
{\textbf{\ref{sec:proof:thm:grad_1}}.
Building Blocks}
\dotfill
\pageref{sec:proof:thm:grad_1}

~~~~~~~~\hyperref[sec:proof:thm:grad_2]
{\textbf{\ref{sec:proof:thm:grad_2}}.
Derivation of Theorem~\ref{thm:grad}}
\dotfill
\pageref{sec:proof:thm:grad_2}

~~~~~~~~\hyperref[sec:proof:thm:grad_3]
{\textbf{\ref{sec:proof:thm:grad_3}}.
Proof of Claim~\eqref{eq:responsedistravg}}
\dotfill
\pageref{sec:proof:thm:grad_3}

~~~~\hyperref[sec:proof:thm:stat]
{\textbf{\ref{sec:proof:thm:stat}}.
Statistical Considerations}
\dotfill
\pageref{sec:proof:thm:stat}

~~~~~~~~\hyperref[sec:proof:thm:asymp_full]
{\textbf{\ref{sec:proof:thm:asymp_full}}.
Proof of Lemma~\ref{thm:asymp_full} (Theorem~\ref{thm:asymp_full_full})}
\dotfill
\pageref{sec:proof:thm:asymp_full}

~~~~~~~~\hyperref[sec:proof:thm:asymp]
{\textbf{\ref{sec:proof:thm:asymp}}.
Proof of Theorem~\ref{thm:asymp}}
\dotfill
\pageref{sec:proof:thm:asymp}

~~~~~~~~\hyperref[sec:proof:lemma:hess_scalarvalue]
{\textbf{\ref{sec:proof:lemma:hess_scalarvalue}}.
Proof of Theorem~\ref{lemma:hess_scalarvalue}}
\dotfill
\pageref{sec:proof:lemma:hess_scalarvalue}

\hyperref[app:aux]
{\textbf{\ref{app:aux}}.
Proof of Auxiliary Results}
\dotfill
\pageref{app:aux}

~~~~\hyperref[sec:proof:aux:thm:grad]
{\textbf{\ref{sec:proof:aux:thm:grad}}.
Proof of Auxiliary Results for Theorem~\ref{thm:grad}}
\dotfill
\pageref{sec:proof:aux:thm:grad}

~~~~~~~~\hyperref[sec:proof:lemma:grad_policy]
{\textbf{\ref{sec:proof:lemma:grad_policy}}.
Gradients of Policy $\policytheta$ and Reward $\rewardtheta$ }
\dotfill
\pageref{sec:proof:lemma:grad_policy}

~~~~~~~~\hyperref[sec:proof:lemma:grad_scalarvalue]
{\textbf{\ref{sec:proof:lemma:grad_scalarvalue}}.
Proof of Lemma~\ref{lemma:grad_scalarvalue}, Explicit Form of Gradient $\gradtheta \scalarvalue(\policytheta)$ }
\dotfill
\pageref{sec:proof:lemma:grad_scalarvalue}

~~~~~~~~\hyperref[sec:proof:lemma:grad_loss]
{\textbf{\ref{sec:proof:lemma:grad_loss}}.
Proof of Lemma~\ref{lemma:grad_loss}, Explicit Form of Gradient $\gradtheta \Loss(\paratheta)$}
\dotfill
\pageref{sec:proof:lemma:grad_loss}

~~~~\hyperref[sec:proof:thm:asymp_aux]
{\textbf{\ref{sec:proof:thm:asymp_aux}}.
Proof of Auxiliary Results for Theorem~\ref{thm:asymp}}
\dotfill
\pageref{sec:proof:thm:asymp_aux}

~~~~~~~~\hyperref[sec:proof:eq:master_cond_proof]
{\textbf{\ref{sec:proof:eq:master_cond_proof}}.
Proof of Condition~\eqref{eq:master_cond_proof}}
\dotfill
\pageref{sec:proof:eq:master_cond_proof}

~~~~~~~~\hyperref[sec:proof:lemma:hess_loss]
{\textbf{\ref{sec:proof:lemma:hess_loss}}.
Proof of Lemma~\ref{lemma:hess_loss}, Explicit Form of Hessian $\hesstheta \Loss(\parathetastar)$}
\dotfill
\pageref{sec:proof:lemma:hess_loss}

~~~~~~~~\hyperref[sec:proof:lemma:grad_loss_stat]
{\textbf{\ref{sec:proof:lemma:grad_loss_stat}}.
Proof of Lemma~\ref{lemma:grad_loss_stat}, Asymptotic Distribution of Graident $\gradtheta \Losshat(\parathetastar)$ }
\dotfill
\pageref{sec:proof:lemma:grad_loss_stat}

~~~~\hyperref[sec:proof:lemma:hess_scalarvalue_aux]
{\textbf{\ref{sec:proof:lemma:hess_scalarvalue_aux}}.
Proof of Auxiliary Results for Theorem~\ref{lemma:hess_scalarvalue}}
\dotfill
\pageref{sec:proof:lemma:hess_scalarvalue_aux}

~~~~~~~~\hyperref[sec:proof:eq:hessscalarvalue]
{\textbf{\ref{sec:proof:eq:hessscalarvalue}}.
Proof of Equation~\eqref{eq:hessscalarvalue} from Theorem~\ref{lemma:hess_scalarvalue}, Explicit Form of Hessian $\hesstheta \scalarvalue(\policystar)$}
\dotfill
\pageref{sec:proof:eq:hessscalarvalue}

~~~~~~~~\hyperref[sec:proof:gap_distr]
{\textbf{\ref{sec:proof:gap_distr}}.
Proof of the Asymptotic Distribution in Equation~\eqref{eq:gap_distr}}
\dotfill
\pageref{sec:proof:gap_distr}

~~~~~~~~\hyperref[sec:proof:chisqtail]
{\textbf{\ref{sec:proof:chisqtail}}.
Proof of the Tail Bound in Equation~\eqref{eq:gap_bd}}
\dotfill
\pageref{sec:proof:chisqtail}

\hyperref[sec:master]
{\textbf{\ref{sec:master}}.
Supporting Theorem: Master Theorem for $Z$-Estimators}
\dotfill
\pageref{sec:master}

\hyperref[app:experiment]
{\textbf{\ref{app:experiment}}.
Experimental Details}
\dotfill
\pageref{app:experiment}

\hyperref[app:extension]
{\textbf{\ref{app:extension}}.
Extension to Proximal Policy Optimization (PPO)}
\dotfill
\pageref{app:extension}

}





\section{Additional Literature Review}\label{app:related_work}

\textbf{RLHF}. RLHF has emerged as a cornerstone methodology for aligning large language models with human values and preferences \citep{achiam2023gpt}. Early systems \citep{ouyang2022training} turn human preference data into reward modeling to optimize model behavior accordingly. DPO has been proposed as a more efficient approach that directly trains LLMs on preference data.  
As LLMs evolve during training, continuing training on pre-generated preference data becomes suboptimal due to the distribution shift. Empirically, RLHF is applied iteratively—generating on-policy data at successive stages to enhance alignment and performance \citep{touvron2023llama, bai2022training}. Similarly, researchers have introduced iterative DPO \citep{xiong2024iterative, xu2023some} and online DPO \citep{guo2024direct} to fully leverage online preference labeling. Ultimately, the quality of preference data play a critical role in determining the effectiveness of the alignment. 

\textbf{Sampling in Frontier LLMs}. Technical reports of Frontier LLMs briefly mention sampling techniques. For instance, Claude \citep{bai2022training} utilizes models from different training steps to generate responses, while Llama-2 \citep{touvron2023llama} further use different temperatures for sampling. However, no further details are provided, leaving the development of a principled method an open challenge. % \julia{Same: move to appendix!}

\textbf{Data Selection.} There is a line of research aimed at improving sample efficiency for preference labeling by selecting question and response pairs. \citet{scheid2024optimal} conceptualize this as a regret minimization problem, leveraging methods from linear dueling bandits. \citet{das2024active, mehta2023sample, muldrewactive, ji2024reinforcement} draw insights from active learning, using various uncertainty estimators to guide selection by prioritizing sample pairs with maximum uncertainty. These approaches focus directly on a dataset of questions and responses and are orthogonal to our work. 

\textbf{Other Changes in Response Sampling.} Several works also modify the sampling design directly \citep{liustatistical, dongraft}, but with the goal of improving policy network optimization based on a reward model, rather than enhancing the reward modeling itself. \citet{liustatistical} employ rejection sampling to approximate the response distribution induced by the reward model, thereby improving optimization. However, this approach requires access to the reward model and incurs higher computational and labeling costs. Similarly, \citet{dongraft} use best-of-N sampling with the reward model to generate high-quality data for supervised fine-tuning (SFT). We consider these approaches orthogonal to our work.


Additionally, \citet{cen2024value} introduce a bonus term in the policy learning phase of online RLHF to promote exploration in response sampling, which aligns with the optimism principle.


\iffalse

\section{Additional Statistical Results}\label{app:add_stat_results}


    {In addition to our analysis of T-PILAF in \Cref{sec:sampling}, here we present a generalized version of \Cref{thm:asymp} that applies to any response sampling distribution~$\responsedistr$. While not directly tied to the main focus of this work, this broader result may be of independent interest to readers.
    The proof of \Cref{thm:asymp_full} is provided in \Cref{sec:proof:thm:asymp_full}.
    \begin{lemma}
        \label{thm:asymp_full}
        For a general sampling distribution $\responsedistr$, the statement in \Cref{thm:asymp} remains valid with the matrix $\CovOpstar$ redefined as
        \begin{align}
            \CovOpstar \defn
            \Exp_{\prompt \sim \promptdistr, (\responseone, \, \responsetwo) \sim \responsedistravg(\cdot \mid \prompt)}
            \Big[ \, \weight(\prompt) \cdot \Var\big(\indicator\{\responseone = \responsewin\} \bigm| \prompt, \responseone, \responsetwo \big) \cdot \grad \, \grad^{\top} \Big] \, ,
            \label{eq:def_CovOpstar}
        \end{align} 
        where the expectation is taken over the distribution
        \begin{subequations}
            \begin{align}
                \label{eq:def_responsedistravg}
                \responsedistravg(\responseone, \responsetwo \mid \prompt) 
                \defn \frac{1}{2} \, \big\{ \responsedistr(\responseone, \responsetwo \mid \prompt) + \responsedistr(\responsetwo, \responseone \mid \prompt) \big\} \, .
            \end{align}
        The variance term is specified as
            \begin{align}
                & \Var\big(\indicator\{\responseone = \responsewin\} \mid \prompt, \responseone, \responsetwo \big)
                \label{eq:def_var}
                = \sigmoid\big( \rewardstar(\prompt, \responseone) - \rewardstar(\prompt, \responsetwo) \big) \, \sigmoid\big( \rewardstar(\prompt, \responsetwo) - \rewardstar(\prompt, \responseone) \big)
            \end{align}
        and the gradient difference $\grad$ is defined as
            \begin{align}
                \label{eq:def_grad}
                \grad \defn \gradtheta \rewardstar(\prompt, \responseone) - \gradtheta \rewardstar(\prompt, \responsetwo) \, .
            \end{align}
        \end{subequations}
    \end{lemma}
    
    The general form of the matrix $\CovOpstar$ offers valuable insights for designing a sampling scheme. To ensure $\CovOpstar$ is well-conditioned (less singular), we must balance two key factors when selecting responses $\responseone$ and $\responsetwo$:
    \vspace{-.8em}
    \begin{description} \itemsep = -.05em
        \item \emph{Large variance:} The variance in definition~\eqref{eq:def_var} should be maximized. This occurs when $\rewardstar(\prompt, \responseone) \approx \rewardstar(\prompt, \responsetwo)$. Intuitively, preference feedback is most informative when annotators compare responses of similar quality.
        \item \emph{Large gradient difference:} The gradient difference $\grad$ from definition~\eqref{eq:def_grad} should also be large. This requires responses with significantly different gradients. Only then can the comparison provide a clear and meaningful direction for model training.
    \end{description}
    }

    \fi






%%%%%%%%%%%%%%%%%%%%%%%%%%%%%%%%%%%%%%%%%%%%%%%%%%%%%%%%%


%%%%%%%%%%%%%%%%%%%%%%%%%%%%%%%%%%%%%%%%%%%%%%%%%%%%%%%%%

%	\section{Auxiliary results}
%	
%	\begin{corollary}
%		Assume gradient Lipchitz $\GradLip$.
%		\begin{align*}
%			\scalarvalue(\policyt{t+1})
%			\; \geq \; \scalarvalue(\policyt{t}) \, + \, \frac{1}{2 \GradLip} \, \norm[\big]{\gradtheta \scalarvalue(\policyt{t})}_2^2 \, - \, \frac{\Partitionthetabar}{2 \GradLip \parabeta^2} \, \norm{\Term_2}_2^2
%		\end{align*}
%	\end{corollary}
%	
%	\begin{proposition}
%		\label{thm:Hess}
%		Suppose that for any \mbox{$\reward \in \RewardSp$}, \mbox{$0 \leq \reward(\prompt, \response) \leq \Radius$}. Moreover, suppose $\hesstheta \reward_{\paratheta}(\prompt, \response)$ is positive semidefinite for any $(\prompt, \response) \in \PromptSp \times \ResponseSp$.
%		Then both terms
%		\begin{align*}
%			\Exp_{\prompt \sim \promptdistr, \, \response \sim \policy_{\paratheta}(\cdot \mid \prompt)} \big[ \rewardstar(\context, \response) \big]
%			\qquad \mbox{and} \qquad
%			\kull{\policytheta}{\policyref}
%		\end{align*}
%		are convex in parameter $\paratheta$.
%		Therefore, the objective function $\scalarvalue(\policytheta)$ is a difference-of-convex function.
%	\end{proposition}

%%%%%%%%%%%%%%%%%%%%%%%%%%%%%%%%%%%%%%%%%%%%%%%%%%%%%%%%%

	\section{Proof of Main Results \yaqidone}
    \label{app:proof:main}

        
    
		This section provides the proofs of the main results from \Cref{sec:theory}, covering both optimization and statistical aspects.
		In \Cref{sec:proof:thm:grad}, we prove \Cref{thm:grad}, which establishes the gradient alignment property. For the statistical results, \Cref{sec:proof:thm:stat} begins with the proofs of \Cref{thm:asymp_full,thm:asymp}, which derive the asymptotic distribution of the estimated parameter $\parathetahat$, and concludes with the proof of \Cref{lemma:hess_scalarvalue}, analyzing the asymptotic behavior of the value gap~\mbox{$\scalarvalue(\policystar) - \scalarvalue(\policyhat)$}.
	
	\subsection{Optimization Considerations: Proof of Theorem~\ref{thm:grad} \yaqidone}
	\label{sec:proof:thm:grad}

        % \yaqitbd

        We begin by presenting a rigorous restatement of \Cref{thm:grad}, formally detailed in \Cref{thm:grad_full} below.

        \begin{theorem}[Gradient structure in DPO training]
			\label{thm:grad_full}
			Consider the expected loss function $\Loss(\paratheta)$ during the DPO training phase. Using data collected from our poposed response sampling scheme $ \responsedistr $, the gradient of $ \Loss(\paratheta) $ satisfies
			\begin{align*}
				\gradtheta \Loss(\paratheta) \; = \;
				- \, \frac{\parabeta}{\Partitionthetabar} \, \gradtheta \scalarvalue(\policytheta) \, + \, \Term_2 \, ,
			\end{align*}
			where the constant $ \Partitionthetabar $ is defined in equation~\eqref{eq:weight}, and the term $ \Term_2% = \bigO( \norm{\rewardtheta - \rewardstar}^2 ) 
			$ represents a second-order error.
			
			To control term $ \Term_2 $, assume the following uniform bounds: 
            \begin{itemize}
                \item[(i)] \mbox{$\!\supnorm{\rewardstar} \leq \Radius$}.
                \item[(ii)] For any policy \mbox{$\policytheta \in \PolicySp$}, the induced reward $\rewardtheta$ satisfies 
                \begin{align*}
                    \supnorm{\rewardtheta} \leq \Radius \qquad \mbox{and} \qquad \sup\nolimits_{\prompt, \response} \, \norm{\gradtheta \rewardtheta (\prompt, \response)}_2 \leq \RadiusGrad \, .
                \end{align*}
            \end{itemize}
			Under these conditions, $ \Term_2 $ is bounded as
			\vspace{-.5em}
			\begin{align*}
                \norm{\Term_2}_2 \leq 
                \Const{} \, \cdot \, \Exp_{\prompt \sim \promptdistr, \, \responseone, \responsetwo \sim \policytheta(\cdot \mid \prompt)}
				\bigg[ \, \Big\{ \big( \rewardstar(\context, \responseone) - \rewardstar(\context, \responsetwo) \big)
                - \big( \rewardtheta(\context, \responseone) - \rewardtheta(\context, \responsetwo) \big) \Big\}^2 \bigg] \, ,
			\end{align*}
			where the constant $\Const{}$ is given by $\Const{} = 0.1 \, (1 + e^{2\Radius}) \, \RadiusGrad \big/ \Partitionthetabar$.
		\end{theorem}
	
	The proof of \Cref{thm:grad_full} is structured into three sections. In \Cref{sec:proof:thm:grad_1}, we lay the foundation by presenting the key components, including the explicit expressions for the gradients $\gradtheta \scalarvalue(\policytheta)$ and $\gradtheta \Loss(\paratheta)$, as well as for the sampling density~$\responsedistravg$.
	Then \Cref{sec:proof:thm:grad_2} establishes the connection between $\gradtheta \scalarvalue(\policytheta)$ and $\gradtheta \Loss(\paratheta)$ by leveraging these results, completing the proof of \Cref{thm:grad}. 
	Finally, in \Cref{sec:proof:thm:grad_3}, we provide a detailed derivation of the form of density function~$\responsedistravg$.
	
	\subsubsection{Building Blocks \yaqidone}
	\label{sec:proof:thm:grad_1}
	
	To establish \Cref{thm:grad}, which uncovers the relationship between the gradients of the expected value $\scalarvalue(\policytheta)$ and the negative log-likelihood function $\Loss(\paratheta)$, the first step is to derive explicit expressions for the gradients of both functions. The results are presented in \Cref{lemma:grad_scalarvalue,lemma:grad_loss}, with detailed proofs provided in \Cref{sec:proof:lemma:grad_scalarvalue,sec:proof:lemma:grad_loss}, respectively.
	\begin{lemma}[Gradient of value $\scalarvalue(\policytheta)$]
		\label{lemma:grad_scalarvalue}
		For any $\policytheta$ in the parameterized policy class $\PolicySp$, the gradient of the expected value~$\scalarvalue(\policytheta)$ satisfies
		%			\begin{subequations}
			\begin{multline}
				\label{eq:grad_scalarvalue}
				\gradtheta \scalarvalue(\policytheta)
				\; = \; \frac{1}{2 \parabeta} \, \Exp_{\prompt \sim \promptdistr; \; \responseone, \responsetwo \sim \policytheta(\cdot \mid \prompt)} 
				\bigg[ \Big\{ \big( \rewardstar(\context, \responseone) - \rewardstar(\context, \responsetwo) \big) - \big( \rewardtheta(\context, \responseone) - \rewardtheta(\context, \responsetwo) \big) \Big\} \\ 
				\cdot \big\{ \gradtheta \rewardtheta(\prompt, \responseone) - \gradtheta \rewardtheta(\prompt, \responsetwo) \big\} \bigg] \, .
			\end{multline}
			%			\end{subequations}
	\end{lemma}
	
	
	
	\begin{lemma}[Gradient of the loss function $\Loss(\paratheta)$]
		\label{lemma:grad_loss}
		For any $\policytheta$ in the parameterized policy class $\PolicySp$ and any sampling distribution $\responsedistr$ of the responses, the gradient of the negative log-likelihood function $\Loss(\paratheta)$ is given by
		\begin{subequations}
			\begin{multline}
				\label{eq:gradLoss_BT_0}
				\gradtheta \Loss(\paratheta) \; = \; - \, \Exp_{\prompt \sim \promptdistr; \; (\responseone, \, \responsetwo) \sim \responsedistravg(\cdot \mid \prompt)}
				\bigg[ \, \weight(\prompt) \cdot \Big\{ \sigmoid \big( \rewardstar(\context, \responseone) - \rewardstar(\context, \responsetwo) \big) - \sigmoid \big( \rewardtheta(\context, \responseone) - \rewardtheta(\context, \responsetwo) \big) \Big\} \\ 
				\cdot \big\{ \gradtheta \rewardtheta(\prompt, \responseone) - \gradtheta \rewardtheta(\prompt, \responsetwo) \big\} \bigg] \, ,
			\end{multline}
			where the average density $\responsedistravg$ is defined as
			\begin{align}
				\label{eq:def_responsedistravg_0}
				\responsedistravg(\responseone, \responsetwo \mid \prompt) 
				\; \defn \; \frac{1}{2} \, \big\{ \responsedistr(\responseone, \responsetwo \mid \prompt) + \responsedistr(\responsetwo, \responseone \mid \prompt) \big\}
			\end{align}
		\end{subequations}
			as previously introduced in \cref{eq:def_responsedistravg}.
	\end{lemma}
	
	In \Cref{lemma:grad_loss}, we observe that the gradient $\gradtheta \Loss(\paratheta)$ is expressed as an expectation over the probability distribution $\responsedistravg$. By applying the sampling scheme outlined in \Cref{sec:sampling}, we can derive a more detailed representation of $\gradtheta \Loss(\paratheta)$. This refined form will reveal its close relationship to the gradient $\gradtheta \scalarvalue(\policytheta)$ given in expression \eqref{eq:grad_scalarvalue}.
	
	Before moving forward, it is crucial for us to first derive the explicit form of $\responsedistravg$. Specifically, we claim that the distribution~$\responsedistravg$ satisfies the following property
	\begin{align}
		\label{eq:responsedistravg}
		\frac{\responsedistravg ( \responseone, \responsetwo \mid \prompt )}{\policytheta(\responseone \mid \prompt) \, \policytheta(\responsetwo \mid \prompt)} 
		& \; = \; \frac{1}{2 \, \{ 1 + \Partitionthetapos(\prompt) \, \Partitionthetaneg(\prompt) \}}
		\cdot \frac{1}{\divsigmoid \big( \rewardtheta(\prompt, \responseone) - \rewardtheta(\prompt, \responsetwo) \big)} \, ,
	\end{align}
	where $\divsigmoid$ denotes the derivative of the sigmoid function $\sigmoid$, given by
	\begin{align}
		\label{eq:divsigmoid}
		\divsigmoid(z) \; = \; \frac{1}{( 1 + \exp(-z) )( 1 + \exp(z) )} \; = \; \sigmoid(z) \, \sigmoid(-z)
		\qquad \mbox{for any $z \in \Real$}  \, .
	\end{align}
	With these key components in place, we are now prepared to prove \Cref{thm:grad}.
	
	
	\subsubsection{Derivation of Theorem~\ref{thm:grad} \yaqidone}
	\label{sec:proof:thm:grad_2}
	
	With the tools provided by \Cref{lemma:grad_scalarvalue,lemma:grad_loss} and the sampling density expression in \eqref{eq:responsedistravg}, we are now ready to prove \Cref{thm:grad}.
	
	We begin by applying \Cref{lemma:grad_loss} and reformulating equation~\eqref{eq:gradLoss_BT_0} as
	\begin{align}
		\gradtheta \Loss(\paratheta) \; = \; - \, \Exp_{\prompt \sim \promptdistr; \; \responseone, \, \responsetwo \sim \policytheta(\cdot \mid \prompt)}
		\bigg[ \, & \weight(\prompt) \cdot \frac{\responsedistravg ( \responseone, \responsetwo \mid \prompt )}{\policytheta(\responseone \mid \prompt) \, \policytheta(\responsetwo \mid \prompt)} \notag \\
		& \cdot \Big\{ \sigmoid \big( \rewardstar(\context, \responseone) - \rewardstar(\context, \responsetwo) \big) - \sigmoid \big( \rewardtheta(\context, \responseone) - \rewardtheta(\context, \responsetwo) \big) \Big\} \notag \\ 
		& \cdot \big\{ \gradtheta \rewardtheta(\prompt, \responseone) - \gradtheta \rewardtheta(\prompt, \responsetwo) \big\} \bigg] \,.
		\label{eq:gradLoss}
	\end{align}
	Substituting the density ratio from equation~\eqref{eq:responsedistravg} into expression \eqref{eq:gradLoss} and incorporating the weight function $\weight(\prompt)$ defined in equation \eqref{eq:weight}, we obtain 
	\begin{align}
		\gradtheta \Loss(\paratheta) \; = \; - \frac{1}{2 \, \Partitionthetabar} \, \Exp_{\prompt \sim \promptdistr; \; \responseone, \, \responsetwo \sim \policytheta(\cdot \mid \prompt)}
		\Bigg[ \, & 
		\frac{\sigmoid \big( \rewardstar(\context, \responseone) - \rewardstar(\context, \responsetwo) \big) - \sigmoid \big( \rewardtheta(\context, \responseone) - \rewardtheta(\context, \responsetwo) \big)}{\divsigmoid \big( \rewardtheta(\prompt, \responseone) - \rewardtheta(\prompt, \responsetwo) \big)}  \notag  \\
		& \qquad \qquad \qquad \cdot \big\{ \gradtheta \rewardtheta(\prompt, \responseone) - \gradtheta \rewardtheta(\prompt, \responsetwo) \big\} \Bigg] \, .  \label{eq:gradLoss_0}
	\end{align}
	Using the intuition that the first-order Taylor expansion
	\begin{align*}
		\frac{\sigmoid(z^{\star}) - \sigmoid(z)}{\divsigmoid(z)} \; = \; (z^{\star} - z) + \bigO\big((z^{\star} - z)^2\big)
	\end{align*}
	is valid when $z \to z^\star$, with $z^\star \defn \rewardstar(\context, \responseone) - \rewardstar(\context, \responsetwo)$ and $z \defn \rewardtheta(\context, \responseone) - \rewardtheta(\context, \responsetwo)$, we find that
	\begin{align*}
		& \frac{\sigmoid \big( \rewardstar(\context, \responseone) - \rewardstar(\context, \responsetwo) \big) - \sigmoid \big( \rewardtheta(\context, \responseone) - \rewardtheta(\context, \responsetwo) \big)}{\divsigmoid \big( \rewardtheta(\prompt, \responseone) - \rewardtheta(\prompt, \responsetwo) \big)}  \\
		& \; = \; \Big\{ \big( \rewardstar(\context, \responseone) - \rewardstar(\context, \responsetwo) \big) - \big( \rewardtheta(\context, \responseone) - \rewardtheta(\context, \responsetwo) \big) \Big\} \; + \; \mbox{second-order term}.
	\end{align*}
	Reformulating equation~\eqref{eq:gradLoss_0} in this context, we rewrite it as
    \begin{align}
		\gradtheta \Loss(\paraphi) 
		& = - \, \frac{1}{2 \Partitionthetabar} \, \Exp_{\, \begin{subarray}{l} \\ \prompt \sim \promptdistr; \\ \responseone, \responsetwo \sim \policytheta(\cdot \mid \prompt) \end{subarray}}
		\Bigg[ \, \Big\{ \big( \rewardstar(\context, \responseone) - \rewardstar(\context, \responsetwo) \big) - \big( \rewardtheta(\context, \responseone) - \rewardtheta(\context, \responsetwo) \big) \Big\} \notag  \\
		& \qquad \qquad \qquad \qquad \qquad \qquad \qquad \qquad \quad \cdot \big\{ \gradtheta \rewardtheta(\prompt, \responseone) - \gradtheta \rewardtheta(\prompt, \responsetwo) \big\} \Bigg]
		+ \Term_2 \, , \label{eq:gradLoss_1}
	\end{align}
	where $\Term_2$ represents the second-order residual term related to the estimation error $\rewardtheta - \rewardstar$.
	By applying \Cref{lemma:grad_scalarvalue}, we observe that the primary term in equation~\eqref{eq:gradLoss_1} aligns with the direction of $\gradtheta \scalarvalue(\policytheta)$, resulting in
	\begin{align}
		\label{eq:gradLoss_final}
		\gradtheta \Loss(\paraphi) 
		& = - \, \frac{\parabeta}{\Partitionthetabar} \, \gradtheta \scalarvalue(\policytheta)
		+ \Term_2 \, .
	\end{align}

	
	Next, we proceed to control the second-order term $\Term_2$.
	The conditions
	\begin{align*}
		\supnorm{\rewardstar}, \supnorm{\rewardtheta} \leq \Radius
		\qquad \mbox{and} \qquad \sup\nolimits_{(\prompt, \response) \in \PromptSp \times \ResponseSp} \norm{\gradtheta \rewardtheta (\prompt, \response)}_2 \leq \RadiusGrad,
	\end{align*}
	lead to the bound
	\begin{align*}
		\abs[\Big]{ \, \frac{\sigmoid(z^{\star}) - \sigmoid(z)}{\divsigmoid(z)} - (z^{\star} - z) }
		\; \leq \;  0.1 \, (1 + e^{2\Radius}) \cdot (z^{\star} - z)^2 \, ,
	\end{align*}
	which in turn implies
	\begin{align}
		& \norm{\Term_2}_2
        \notag \\
        \label{eq:gradLoss_Term2}
        & \; \leq \;  \frac{0.1 \, (1 + e^{2\Radius}) \, \RadiusGrad}{\Partitionthetabar} \, \Exp_{\prompt \sim \promptdistr; \; \responseone, \responsetwo \sim \policytheta(\cdot \mid \prompt)} 
        \bigg[ \, \Big\{ \big( \rewardstar(\context, \responseone) - \rewardstar(\context, \responsetwo) \big) - \big( \rewardtheta(\context, \responseone) - \rewardtheta(\context, \responsetwo) \big) \Big\}^2 \bigg] \, .
	\end{align}
	
	Finally, combining equation~\eqref{eq:gradLoss_Term2} with equation~\eqref{eq:gradLoss_final}, we conclude the proof of \Cref{thm:grad}.
	
	
%%%%%%%%%%%%%%%%%%%%%%%%%%%%%%%%%%%%%%%%%%%%%%%%%%%%%%%%%%%%
		
		\subsubsection{Proof of Claim~\eqref{eq:responsedistravg}}
		\label{sec:proof:thm:grad_3}
		
		The remaining step in the proof of \Cref{thm:grad} is to verify the expression for the density ratio in equation~\eqref{eq:responsedistravg}.
		
		Based on the sampling scheme described in \Cref{sec:sampling}, we find that the sampling distribution for the response satisfies
		\begin{align}
			\label{eq:responsedistr_0}
			\responsedistr \big( \responseone, \responsetwo \bigm| \prompt \big)
			& \; = \; \{ 1 - \sampleprob(\prompt) \} \cdot \policytheta(\responseone \mid \prompt) \,  \policytheta(\responsetwo \mid \prompt)
			\, + \, \sampleprob(\prompt) \cdot \policythetapos(\responseone \mid \prompt) \,  \policythetaneg(\responsetwo \mid \prompt) \, ,
		\end{align}
		where the probability $\sampleprob(\prompt)$ is defined as
		\begin{align*}
			\sampleprob(\prompt) = \Partitionthetapos(\prompt) \, \Partitionthetaneg(\prompt) / \{1 + \Partitionthetapos(\prompt) \, \Partitionthetaneg(\prompt) \}
		\end{align*}
		and the policies $\policythetapos$ and $\policythetaneg$ are specified in equations~\eqref{eq:def_policythetapos}~and~\eqref{eq:def_policythetaneg}, respectively.
		This allows us to simplify equation~\eqref{eq:responsedistr_0} to
		\begin{align*}
			\responsedistr \big( \responseone, \responsetwo \bigm| \prompt \big)
			& \; = \; \frac{\policytheta(\responseone \mid \prompt) \, \policytheta(\responsetwo \mid \prompt)}{1 + \Partitionthetapos(\prompt) \, \Partitionthetaneg(\prompt)} \, \Big\{ 1 + \exp\big\{ \rewardtheta(\prompt, \responseone) - \rewardtheta(\prompt, \responsetwo) \big\} \Big\} \, .
		\end{align*}
		Similarly, we derive an expression for $\responsedistr ( \responsetwo, \responseone \mid \prompt )$.
		By averaging the two expressions, for $\responsedistr ( \responseone, \responsetwo \mid \prompt )$ and $\responsedistr ( \responsetwo, \responseone \mid \prompt )$, we obtain
		\begin{align*}
%			\label{eq:responsedistravg_ratio}
			& \frac{\responsedistravg ( \responseone, \responsetwo \mid \prompt )}{\policytheta(\responseone \mid \prompt) \, \policytheta(\responsetwo \mid \prompt)}  \\
			& = \frac{\policytheta(\responseone \mid \prompt) \, \policytheta(\responsetwo \mid \prompt)}{2 \, \{ 1 + \Partitionthetapos(\prompt) \, \Partitionthetaneg(\prompt) \}} \, \Big\{ 2 + \exp\big\{ \rewardtheta(\prompt, \responseone) - \rewardtheta(\prompt, \responsetwo) \big\} + \exp\big\{ \rewardtheta(\prompt, \responsetwo) - \rewardtheta(\prompt, \responseone) \big\} \Big\} \, .
		\end{align*}
		Rewriting this expression using the formula for $\divsigmoid$ in equation~\eqref{eq:divsigmoid}, we arrive at
		\begin{align*}
			& \big\{ 1 + \Partitionthetapos(\prompt) \, \Partitionthetaneg(\prompt) \big\} \cdot \frac{\responsedistravg ( \responseone, \responsetwo \mid \prompt )}{\policytheta(\responseone \mid \prompt) \, \policytheta(\responsetwo \mid \prompt)}  \\
			& \; = \; \frac{1}{2} \, \Big\{ 1 + \exp\big\{ \rewardtheta(\prompt, \responsetwo) - \rewardtheta(\prompt, \responseone) \big\} \Big\}  \Big\{ 1 + \exp\big\{ \rewardtheta(\prompt, \responseone) - \rewardtheta(\prompt, \responsetwo) \big\} \Big\}  \\
			& \; = \; \frac{1}{2 \, \divsigmoid \big( \rewardtheta(\prompt, \responseone) - \rewardtheta(\prompt, \responsetwo) \big)} \, .
		\end{align*}
		Finally, rearranging terms, we recover equation~\eqref{eq:responsedistravg}, completing this part of the proof.

%%%%%%%%%%%%%%%%%%%%%%%%%%%%%%%%%%%%%%%%%%%%%%%%%%%%%%%%%%%%%%%%%%%%%%%%%%%%

	\subsection{Statistical Considerations \yaqidone}
	\label{sec:proof:thm:stat}



        In this section, we present the proofs for \Cref{thm:asymp,lemma:hess_scalarvalue,thm:asymp_full} from \Cref{sec:theory_stat}. 
        We start with the proof of \Cref{thm:asymp_full} in \Cref{sec:proof:thm:asymp_full}, with a rigorous restatement provided in \Cref{thm:asymp_full_full} below.
    		\begin{theorem}
			\label{thm:asymp_full_full}
%			We take $\weight(\prompt) \equiv 1$.
			Assume the reward model $\rewardstar$ in the BT model~\eqref{eq:BT} satisfies $\rewardstar = \reward_{\parathetastar}$ for some parameter $\parathetastar$.
			Assume that $\parathetahat$ minimizes the loss function $\Losshat(\paratheta)$ in the sense that $\sqrt{\numobs} \, \gradtheta \Losshat (\parathetahat) \convergep \veczero$ and that $\parathetahat \convergep \parathetastar$ as the sample size $\numobs \rightarrow \infty$.
			Additionally, suppose the reward function $\rewardtheta(\prompt, \response)$, its gradient $\gradtheta \rewardtheta(\prompt, \response)$ and its Hessian $\hesstheta \rewardtheta(\prompt, \response)$ are uniformly bounded and Lipchitz continuous with respect to $\paratheta$, for all $(\prompt, \response) \in \PromptSp \times \ResponseSp$.
			
			Under these conditions, the estimate $\parathetahat$ asymptotically follows a Gaussian distribution
			\begin{align*}
				\sqrt{\numobs} \; ( \parathetahat - \parathetastar)
				\; \stackrel{d}{\longrightarrow} \; \Gauss( \veczero, \CovOmega )
				\qquad \mbox{as $\numobs \rightarrow \infty$} \, .
			\end{align*}
			We have an estimate of the covariance matrix $\CovOmega$:
            \begin{align*}
                \CovOmega \; \preceq \; \supnorm{\weight} \cdot \CovOpstar^{-1} \, .
            \end{align*}
            For a general sampling scheme $\responsedistr$ chosen, the matrix~$\CovOpstar$ is given by
			\begin{align*}
                % \label{eq:def_CovOpstar_simple}
				\CovOpstar \; \defn \;
				& \Exp_{\prompt \sim \promptdistr, \, (\responseone, \, \responsetwo) \sim \responsedistravg(\cdot \mid \prompt)}
			\Big[ \, \weight(\prompt) \cdot \Var\big(\indicator\{\responseone = \responsewin\} \bigm| \prompt, \responseone, \responsetwo \big) \cdot \grad \, \grad^{\top} \Big] \, ,
				%\label{eq:def_CovOpstar}
			\end{align*}
			where the expectation is taken over the distribution
			%\vspace{-.3em}
			\begin{subequations}
				\begin{align*}
					%\label{eq:def_responsedistravg}
					\responsedistravg(\responseone, \responsetwo \mid \prompt) 
					\defn \frac{1}{2} \, \big\{ \responsedistr(\responseone, \responsetwo \mid \prompt) + \responsedistr(\responsetwo, \responseone \mid \prompt) \big\} \, .
				\end{align*} %~ \vspace{-1.8em} \\
			The variance term is specified as
				\begin{align*}
					& \Var\big(\indicator\{\responseone \; = \; \responsewin\} \mid \prompt, \responseone, \responsetwo \big)
					%\label{eq:def_var}
					= \sigmoid\big( \rewardstar(\prompt, \responseone) - \rewardstar(\prompt, \responsetwo) \big) \, \sigmoid\big( \rewardstar(\prompt, \responsetwo) - \rewardstar(\prompt, \responseone) \big)
					%\notag
				\end{align*}
			and the gradient difference $\grad$ is defined as
				\begin{align*}
					%\label{eq:def_grad}
					\grad \; \defn \; \gradtheta \rewardstar(\prompt, \responseone) - \gradtheta \rewardstar(\prompt, \responsetwo) \, .
				\end{align*}
			\end{subequations}
		\end{theorem}

    \Cref{thm:asymp_full_full} establishes the asymptotic distribution of the estimated parameter $\parathetahat$, which serves as the foundation for the subsequent results. 
	Next, we show that \Cref{thm:asymp} directly follows as a corollary of \Cref{thm:asymp_full_full}, with the detailed derivation provided in \Cref{sec:proof:thm:asymp}. Finally, in \Cref{sec:proof:lemma:hess_scalarvalue}, we prove \Cref{lemma:hess_scalarvalue}, which describes the asymptotic behavior of the value gap $\scalarvalue(\policystar) - \scalarvalue(\policyhat)$.
		
	\subsubsection{Proof of Lemma~\ref{thm:asymp_full} (Theorem~\ref{thm:asymp_full_full}) \yaqidone}
	\label{sec:proof:thm:asymp_full}
	
%	\paragraph{(a) Proof of \Cref{thm:asymp_full}:}

	In this section, we analyze the asymptotic distribution of the estimated parameter $\parathetahat$ for a general sampling distribution $\responsedistr$. The parameter $\parathetahat$ is obtained by solving the optimization problem
	\begin{align*}
		{\rm minimize}_{\paratheta} \quad
		\Losshat(\paratheta) \; \defn \;
		- \frac{1}{\numobs} \sum_{i=1}^{\numobs} \, \weight(\prompti{i}) \cdot \log \sigmoid \Big( \rewardtheta\big(\prompti{i}, \responsewini{i}\big) - \rewardtheta\big(\prompti{i}, \responselosei{i}\big) \Big) \, .
	\end{align*}
	We assume the optimization is performed to sufficient accuracy such that $\gradtheta \Losshat(\parathetahat) = \smallop\big(\numobs^{-\frac{1}{2}}\big)$.
	Under this condition, $\parathetahat$ qualifies as a $Z$-estimator.
	To study its asymptotic behavior, we use the master theorem for $Z$-estimators \citep{kosorok2008introduction}, the formal statement of which is provided in \Cref{thm:master} in \Cref{sec:master}.
	
	To apply the master theorem, we set $\Psi \defn \gradtheta \Loss$ and $\Psi_{\numobs} \defn \gradtheta \Losshat$ and verify the conditions. In particular, the smoothness condition~\eqref{eq:master_cond} in \Cref{thm:master} translates to the following equation in our context:
    \begin{align}
    	\label{eq:master_cond_proof}
    	& \sqrt{n} \, \big\{ \gradtheta \Losshat (\parathetahat) - \gradtheta \Loss(\parathetahat) \big\} - \sqrt{n} \, \big\{ \gradtheta \Losshat (\parathetastar) - \gradtheta \Loss (\parathetastar) \big\}  
    	\; = \; \smallop \big( 1 + \sqrt{n} \, \norm{ \parathetahat - \parathetastar }_2 \big) \, .
    \end{align}
    This condition follows from the second-order smoothness of the reward function $\rewardtheta$ with respect to $\paratheta$. A rigorous proof is provided in \Cref{sec:proof:eq:master_cond_proof}.
    

	We now provide the explicit form of the derivative $\dot{\Psi}_{\parathetastar} = \hesstheta \Loss(\parathetastar)$, as captured in the following lemma. The proof of this result can be found in \Cref{sec:proof:lemma:hess_loss}.
	\begin{lemma}
		\label{lemma:hess_loss}
		The Hessian matrix of the population loss $\Loss(\paratheta)$ at $\paratheta = \parathetastar$ is
		\begin{align}
			\label{eq:hess_loss}
			\hesstheta \Loss(\parathetastar) \; = \; \CovOpstar \, ,
		\end{align}
		where the matrix $\CovOpstar$ is defined in equation~\eqref{eq:def_CovOpstar}.
	\end{lemma}

	
	Next, we analyze the asymptotic behavior of the gradient $\gradtheta \Losshat(\parathetastar)$.
	The proof is deferred to \Cref{sec:proof:lemma:grad_loss_stat}.
	\begin{lemma}
		\label{lemma:grad_loss_stat}
		The gradient of the empirical loss $\Losshat(\paratheta)$ at $\paratheta = \parathetastar$ satisfies
		\begin{subequations}
		\begin{align}
			\sqrt{\numobs} \, \big( \gradtheta \Losshat(\parathetastar) - \gradtheta \Loss(\parathetastar) \big)
			\; \stackrel{d}{\longrightarrow} \; \Gauss(\veczero, \CovOptil)
			\qquad \mbox{as $\numobs \rightarrow \infty$},
		\end{align}
        where the covariance matrix $\CovOptil \in \Real^{\Dim \times \Dim}$ is bounded as follows:
        \begin{align}
        	\label{eq:CovOptil}
        	\CovOptil \; \preceq \; \supnorm{\weight} \cdot \CovOpstar \, ,
        \end{align}
        \end{subequations}
        with $\CovOpstar$ defined in equation~\eqref{eq:def_CovOpstar}.
	\end{lemma}
	
	Combining these results, and assuming $\CovOpstar$ is nonsingular, the master theorem (\Cref{thm:master}) yields the asymptotic distribution of $\parathetahat$:
	\begin{align*}
		\sqrt{\numobs} \, \big( \parathetahat - \parathetastar \big)
		\; \converged \; \Gauss\big( \veczero, \CovOpstar^{-1} \CovOptil \CovOpstar^{-1} \big) \, .
	\end{align*}
	Furthermore, from the bound~\eqref{eq:CovOptil}, the covariance matrix $\CovOmega ; \defn \CovOpstar^{-1} \CovOptil \CovOpstar^{-1}$ satisfies
	\begin{align*}
		 \CovOmega \; = \CovOpstar^{-1} \CovOptil \CovOpstar^{-1}  \; \preceq \; \supnorm{\weight} \cdot \CovOpstar^{-1} \, .
	\end{align*}
	Therefore, we have established the asymptotic distribution of $\parathetahat$, completing the proof of \Cref{thm:asymp_full}.
	
	
%%%%%%%%%%%%%%%%%%%%%%%%%%%%%%%%%%%%%%%%%%%%%%%%%%%%%%%%%%%%%%%%%%%%%%%%%%%%%%

%	\paragraph{(b) Proof of \eqref{eq:def_CovOpstar_simple}}
	\subsubsection{Proof of Theorem~\ref{thm:asymp}}
	\label{sec:proof:thm:asymp}
	
	\Cref{thm:asymp} is a direct corollary of \Cref{thm:asymp_full}, using our specific choice of sampling distribution $\responsedistr$. To establish this, we demonstrate how the general covariance matrix $\CovOpstar$ in equation~\eqref{eq:def_CovOpstar} simplifies to the form in equation~\eqref{eq:def_CovOpstar_simple} under our proposed sampling scheme.

    To establish the result in this section, we impose the following regularity condition:
    There exists a constant $\Const{} \geq 1$ satisfying
    \begin{align}
        \label{eq:last_cond}
        \Var_{\rewardtheta}\big(\indicator\{\responseone = \responsewin\} \bigm| \prompt, \responseone, \responsetwo \big)
        \; \leq \; \Const{} \cdot \Var_{\rewardstar}\big(\indicator\{\responseone = \responsewin\} \bigm| \prompt, \responseone, \responsetwo \big) 
    \end{align}
    for any prompt $\prompt \in \PromptSp$ and responses $\responseone, \responsetwo \in \ResponseSp$.
    Here $\Var_{\rewardtheta}\big(\indicator\{\responseone = \responsewin\} \bigm| \prompt, \responseone, \responsetwo \big)$ denotes the conditional variance under the BT model~\eqref{eq:BT}, when the implicit reward function $\rewardstar$ is replaced by $\rewardtheta$. The term \mbox{$\Var_{\rewardstar}\big(\indicator\{\responseone = \responsewin\} \bigm| \prompt, \responseone, \responsetwo \big)
    \equiv$} \mbox{$\Var\big(\indicator\{\responseone = \responsewin\} \bigm| \prompt, \responseone, \responsetwo \big) $} represents the conditional variance under the ground-truth BT model, where the reward function is given by $\rewardstar$.
	
	We begin by leveraging the property of the sampling distribution $\responsedistr$ from equation~\eqref{eq:responsedistravg} and the derivative $\divsigmoid$ of the sigmoid function $\sigmoid$, given in equation~\eqref{eq:divsigmoid}. Specifically, we find that
	\begin{align*}
		%\label{eq:responsedistravg2_original}
		& \frac{\responsedistravg ( \responseone, \responsetwo \mid \prompt )}{\policytheta(\responseone \mid \prompt) \, \policytheta(\responsetwo \mid \prompt)} \notag  \\
		& 
		\; = \; \frac{1}{2 \, \{ 1 + \Partitionthetapos(\prompt) \, \Partitionthetaneg(\prompt) \}}
		\cdot \frac{1}{\sigmoid \big( \rewardtheta(\prompt, \responseone) - \rewardtheta(\prompt, \responsetwo) \big) \, \sigmoid \big( \rewardtheta(\prompt, \responsetwo) - \rewardtheta(\prompt, \responseone) \big)}  \\
        & \; = \; \frac{1}{2 \, \{ 1 + \Partitionthetapos(\prompt) \, \Partitionthetaneg(\prompt) \}}
		\cdot \frac{1}{\Var_{\rewardtheta}\big(\indicator\{\responseone = \responsewin\} \bigm| \prompt, \responseone, \responsetwo \big)} \, .
	\end{align*}
    We then apply condition~\eqref{eq:last_cond} and derive
        \begin{equation} 
		 \frac{\responsedistravg ( \responseone, \responsetwo \mid \prompt )}{\policytheta(\responseone \mid \prompt) \, \policytheta(\responsetwo \mid \prompt)} \; \geq \; \frac{\Const{}^{-1}}{2 \, \{ 1 + \Partitionthetapos(\prompt) \, \Partitionthetaneg(\prompt) \}} \cdot \frac{1}{\Var_{\rewardstar}\big(\indicator\{\responseone = \responsewin\} \bigm| \prompt, \responseone, \responsetwo \big)} \, .
         \label{eq:responsedistravg2}
	\end{equation}
	Next, substituting this result~\eqref{eq:responsedistravg2} into equation~\eqref{eq:def_CovOpstar}, alongside the weight function $\weight(\prompt)$ from equation~\eqref{eq:weight}, we reform $\CovOpstar$ as
	\begin{align}
		\CovOpstar
		& \; = \; \Exp_{\prompt \sim \promptdistr; \; \responseone, \, \responsetwo \sim \policytheta(\cdot \mid \prompt)}
		\bigg[ \, \frac{\responsedistravg ( \responseone, \responsetwo \mid \prompt )}{\policytheta(\responseone \mid \prompt) \, \policytheta(\responsetwo \mid \prompt)} \cdot \weight(\prompt) \cdot \Var\big(\indicator\{\responseone = \responsewin\} \bigm| \prompt, \responseone, \responsetwo \big) \cdot \grad \, \grad^{\top} \bigg]  \notag \\
		\label{eq:def_CovOpstar_2}
		& \; \succeq \; \frac{1}{2 \, \Const{} \, \Partitionthetabar} \, \Exp_{\prompt \sim \promptdistr; \; \responseone, \, \responsetwo \sim \policytheta(\cdot \mid \prompt)}
		\big[ \, \grad \, \grad^{\top} \big] \, .
	\end{align}
	The conditional expectation of $\grad \grad^\top$ simplifies as
    \begin{align*}
    	& \Exp_{\responseone, \, \responsetwo \sim \policytheta(\cdot \mid \prompt)}
    	\big[ \, \grad \grad^{\top} \bigm| \prompt\big]  \\
    	& \; = \; \Exp_{\responseone, \, \responsetwo \sim \policytheta(\cdot \mid \prompt)}
    	\Big[ \big\{ \gradtheta \rewardstar(\prompt, \responseone) - \gradtheta \rewardstar(\prompt, \responsetwo) \big\} \big\{ \gradtheta \rewardstar(\prompt, \responseone) - \gradtheta \rewardstar(\prompt, \responsetwo) \big\}^{\top} \Bigm| \prompt\Big]  \\
    	& \; = \; 2 \cdot \Exp_{\response \sim \policytheta(\cdot \mid \prompt)}
    	\Big[ \, \gradtheta \rewardstar(\prompt, \response) \, \gradtheta \rewardstar(\prompt, \response)^{\top} \Bigm| \prompt\Big] \\
        & \qquad \qquad - 2 \cdot \Exp_{\response \sim \policytheta(\cdot \mid \prompt)} \big[ \, \gradtheta \rewardstar(\prompt, \response) \bigm| \prompt\big]  \, \Exp_{\response \sim \policytheta(\cdot \mid \prompt)}
    	\big[ \,\gradtheta \rewardstar(\prompt, \response) \bigm| \prompt \big]^{\top}  \\
    	& \; = \; 2 \cdot \Cov_{\response \sim \policytheta(\cdot \mid \prompt)} \big[ \gradtheta \rewardstar(\prompt, \response) \bigm| \prompt \big] \, .
    \end{align*}
	Substituting this result into equation~\eqref{eq:def_CovOpstar_2}, we arrive at the conclusion that
	\begin{align*}
		\CovOpstar \; \succeq \; \frac{1}{\Const{} \, \Partitionphibar} \, \Exp_{\prompt \sim \promptdistr} \Big[ \Cov_{\response \sim \policystar(\cdot \mid \prompt)} \big[ \gradtheta \rewardstar(\prompt, \response) \bigm| \prompt \big] \Big] \, ,
	\end{align*}
	which matches the simplified form in equation~\eqref{eq:def_CovOpstar_simple} as stated in \Cref{thm:asymp}.

		
		
		
%%%%%%%%%%%%%%%%%%%%%%%%%%%%%%%%%%%%%%%%%%%%%%%%%%%%%%%%%%%%%%%%%%%%%%%%%%%%
	
	
    \subsubsection{Proof of Theorem~\ref{lemma:hess_scalarvalue} \yaqidone}
    \label{sec:proof:lemma:hess_scalarvalue}

	\paragraph{Gradient $ \gradtheta \scalarvalue(\policystar) $ and Hessian $\hesstheta \scalarvalue(\policystar)$:}
	
    The equality $ \gradtheta \scalarvalue(\policystar) = 0$ follows directly from the gradient expression~\eqref{eq:grad_scalarvalue0} for $ \gradtheta \scalarvalue(\policytheta) $, evaluated at $ \paratheta = \parathetastar $ with~\mbox{$ \rewardtheta = \rewardstar $}.
    
    The proof of the Hessian result, $ \hesstheta \scalarvalue(\policystar) = - (1 / \parabeta) \cdot \CovOpstar $, involves straightforward but technical differentiation of equation~\eqref{eq:grad_scalarvalue0}. For brevity, we defer this proof to \Cref{sec:proof:eq:hessscalarvalue}.
  
  	\paragraph{Asymptotic Distribution of Value Gap $ \scalarvalue(\policystar) - \scalarvalue(\policyhat) $:}
    To understand the behavior of the value gap $ \scalarvalue(\policystar) - \scalarvalue(\policyhat) $, we start by applying a Taylor expansion of $ \scalarvalue(\policytheta) $ around $ \parathetastar $. This gives
	\begin{align*}
%		\label{eq:Taylor_scalarvalue}
		\scalarvalue(\policystar) - \scalarvalue(\policyhat)
		\; = \; \gradtheta \scalarvalue(\policystar)^{\top} (\parathetastar - \parathetahat) - \frac{1}{2} (\parathetastar - \parathetahat)^{\top} \hesstheta \scalarvalue(\policystar) (\parathetastar - \parathetahat) + \smallo\big( \norm{\parathetastar - \parathetahat}_2^2 \big) \, .
	\end{align*}
	By substituting $ \gradtheta \scalarvalue(\policystar) = \veczero $ (a direct result of the optimality of $ \policystar $), the linear term vanishes. Introducing the shorthand $ \HessMt \defn -\hesstheta \scalarvalue(\policystar) = (1 / \parabeta) \cdot \CovOpstar $, the expression simplifies to
	\begin{align}
		\label{eq:Taylor_scalarvalue}
		\scalarvalue(\policystar) - \scalarvalue(\policyhat)
		\; = \; \frac{1}{2} \, (\parathetahat - \parathetastar)^{\top} \HessMt \, (\parathetahat - \parathetastar) + \smallo\big( \norm{\parathetahat - \parathetastar}_2^2 \big) \, .
	\end{align}
	When the sample size $ \numobs $ is sufficiently large, $ \parathetahat $ approaches $ \parathetastar $, making the higher-order term negligible. Therefore, the value gap is dominated by the quadratic form.
	
	From \Cref{thm:asymp}, we know the parameter estimate $ \parathetahat $ satisfies
	\begin{align*}
	\sqrt{\numobs} \, (\parathetahat - \parathetastar)
	\;\stackrel{d}{\longrightarrow}\;
	\Gauss(\veczero, \CovOmega).
	\end{align*}
	Substituting this result into the quadratic approximation of the value gap, we find that the scaled value gap has the asymptotic distribution
	\begin{align}
		\label{eq:gap_distr}
		\numobs \cdot \{ \scalarvalue(\policystar) - \scalarvalue(\policyhat) \}
		 \; \stackrel{d}{\longrightarrow} \; \frac{1}{2} \, \vecz^{\top} \CovOmega^{\frac{1}{2}} \HessMt \CovOmega^{\frac{1}{2}} \vecz 
		 \; \nfed \bX
		 \qquad \mbox{where $\vecz \sim \Gauss(\veczero, \IdMt)$}.
	\end{align}
	This approximation provides a clear intuition: the value gap is asymptotically driven by a weighted chi-squared-like term involving the covariance structure $ \CovOmega $ and the Hessian-like matrix $ \HessMt $.
	
	To rigorously establish this result, we will apply Slutsky’s theorem. The full proof is presented in \Cref{sec:proof:gap_distr}.
	
	\paragraph{Bounding the Chi-Square Distribution:}
	
	To bound the random variable $ \bX $, we first leverage the estimate of the covariance matrix $ \CovOmega $ provided by \Cref{thm:asymp}:
	\begin{align*}
		\CovOmega \; \preceq \; \Const{} \, \Partitionthetabar \, \supnorm{\weight} \cdot \CovOpstar^{-1},
	\end{align*}
    where the constant $\Const$ comes from condition~\eqref{eq:last_cond}.
	It follows that the matrix $ \CovOmega^{\frac{1}{2}} \HessMt \CovOmega^{\frac{1}{2}} $ appearing in equation~\eqref{eq:gap_distr} can be bounded as
	\begin{align*}
		\CovOmega^{\frac{1}{2}} \HessMt \CovOmega^{\frac{1}{2}} 
		\; \preceq \;  \Const \, \supnorm{\weight} \cdot \CovOpstar^{-\frac{1}{2}} \HessMt \CovOpstar^{-\frac{1}{2}} \; = \; \Const \cdot \frac{\Partitionthetabar \, \supnorm{\weight}}{\parabeta} \cdot \IdMt
		\; = \; \Const \cdot \frac{1 + \supnorm{\Partitionthetapos \Partitionthetaneg}}{\parabeta}
		\cdot \IdMt \, .
	\end{align*}
	Here the last equality uses the definition of the weight function $ \weight $ from equation~\eqref{eq:weight}. Substituting this bound into the quadratic form, we derive
	\begin{align*}
		\bX
		\; = \; \frac{1}{2} \, \vecz^{\top} \CovOmega^{\frac{1}{2}} \HessMt \CovOmega^{\frac{1}{2}} \vecz 
		\; \leq \; \Const \cdot \frac{1 + \supnorm{\Partitionthetapos \Partitionthetaneg}}{2\parabeta}
		\cdot \vecz^{\top} \vecz \, ,
	\end{align*}
	where $ \vecz \sim \Gauss(\veczero, \IdMt) $.
	Since $ \vecz^{\top} \vecz $ follows a chi-square distribution with $ \Dim $ degrees of freedom, $ \bX $ is stochastically dominated by a rescaled chi-square random variable 
	\begin{align*}
		\Const \cdot \frac{1 + \supnorm{\Partitionthetapos \Partitionthetaneg}}{2\parabeta} \cdot \chisquare_{\Dim}.
	\end{align*}
	Equivalently, we can express this dominance as
	\begin{align}
		\label{eq:gap_bd0}
		\limsup_{\numobs \rightarrow \infty} \; \Prob \bigg\{ \numobs \, \{ \scalarvalue(\policystar) - \scalarvalue(\policyhat) \} > \Const \cdot \frac{1 + \supnorm{\Partitionthetapos \Partitionthetaneg}}{2\parabeta} \cdot t \bigg\}
		\; \leq \; \Prob\big\{ \chisquare_{\Dim} > t \big\}
		\qquad \mbox{for any $t > 0$}.
	\end{align}
	This inequality, given in equation~\eqref{eq:gap_bd0}, corresponds to the first bound in equation~\eqref{eq:gap_bd}.
	
	The second inequality in equation~\eqref{eq:gap_bd} provides a precise tail bound for $\chisquare_{\Dim}$. As its proof involves more technical details, we defer it to \Cref{sec:proof:chisqtail}.
	

	


	
	
%%%%%%%%%%%%%%%%%%%%%%%%%%%%%%%%%%%%%%%%%%%%%%%%%%%%%%%%%%%%%%%%%%%%%%%%%%%%

	\section{Proof of Auxiliary Results \yaqidone}
    \label{app:aux}
	
	This section provides proofs of auxiliary results supporting the main theorems and lemmas. In \Cref{sec:proof:aux:thm:grad}, we present the auxiliary results required for \Cref{thm:grad}. \Cref{sec:proof:thm:asymp_aux} details the proofs of supporting results for \Cref{thm:asymp}. Finally, in \Cref{sec:proof:lemma:hess_scalarvalue_aux}, we establish the auxiliary results necessary for \Cref{lemma:hess_scalarvalue}.

	\subsection{Proof of Auxiliary Results for Theorem~\ref{thm:grad} \yaqidone}
	\label{sec:proof:aux:thm:grad}
	
		In this section, we provide the proofs of several auxiliary results that support the proof of \Cref{thm:grad}. Specifically,
		\Cref{sec:proof:lemma:grad_policy} presents the forms of the gradients of the policy~$\policytheta$ and the reward $\rewardtheta$, which serve as fundamental building blocks for deriving the lemmas.
		\Cref{sec:proof:lemma:grad_scalarvalue} analyzes the gradient of the return function $\scalarvalue(\policytheta)$, as defined in equation~\eqref{eq:objective}.
		\Cref{sec:proof:lemma:grad_loss} focuses on deriving expressions for the gradient of the negative log-likelihood function $\Loss(\paratheta)$.
	
		\subsubsection{Gradients of Policy $\policytheta$ and Reward $\rewardtheta$}
		\label{sec:proof:lemma:grad_policy}
		
		In this part, we introduce results for the gradients of policy $\policytheta$ and reward~$\rewardtheta$ with respsect to parameter~$\paratheta$, which lay the foundation of our calculations.
		
			\begin{lemma}[Gradients of policy $\policytheta$ and reward function $\rewardtheta$]
			\label{lemma:grad_policy}
			The gradients of the policy $\policytheta$ and the reward function $\rewardtheta$ can be expressed in terms of each other as follows
			\begin{subequations}
				\begin{align}
					\label{eq:gradpolicy}
					\gradtheta \policytheta(\diff \response \mid \prompt)
					& \; = \;  \policytheta(\diff \response \mid \prompt) \cdot \frac{1}{\parabeta} \,
					\Big\{ \gradtheta \rewardtheta(\prompt, \response) - \Exp_{\responsenew \sim \policytheta(\cdot \mid \prompt)}\big[ \gradtheta \rewardtheta(\prompt, \responsenew) \big] \Big\} \, ,  \\
					\label{eq:gradreward}
					\gradtheta \rewardtheta (\prompt, \response)
					& \; = \; \parabeta \cdot \frac{\gradtheta \policytheta(\response \mid \prompt)}{\policytheta(\response \mid \prompt)} \, .
				\end{align}
			\end{subequations}
		\end{lemma}
		
		We now proceed to prove \Cref{lemma:grad_policy}.  \\
		
		To begin, recall our definition of the reward function $\rewardtheta$ as given in equation~\eqref{eq:def_reward}.
		It directly follows that
		\begin{align*}
			\gradtheta \rewardtheta (\prompt, \response)
			\; = \; \parabeta \cdot \frac{\gradtheta \policytheta(\response \mid \prompt)}{\policytheta(\response \mid \prompt)} \, .
		\end{align*}
		This result confirms equation~\eqref{eq:gradreward} as stated in \Cref{lemma:grad_policy}.
		
		Next, we express the policy $\policytheta(\diff \response \mid \prompt)$ in terms of the reward function $\rewardtheta(\prompt, \response)$. By reformulating equation~\eqref{eq:def_reward}, we obtain
		\begin{subequations}
		\begin{align}
			\label{eq:policyfromreward}
			\policytheta(\diff \response \mid \prompt)
			\; = \; \frac{1}{\Partitiontheta (\prompt)} \, \policyref(\diff \response \mid \prompt)
			\exp \Big\{ \frac{1}{\parabeta} \, \rewardtheta(\prompt, \response) \Big\} \, ,
		\end{align}
		where $\Partitiontheta (\prompt)$ is the partition function defined as
		\begin{align}
			\label{eq:def_Partition}
			\Partitiontheta (\prompt)
			& \; = \; \int_{\ResponseSp} \, \policyref(\diff \response \mid \prompt)
			\exp \Big\{ \frac{1}{\parabeta} \, \rewardtheta(\prompt, \response) \Big\} \, .
		\end{align}
		\end{subequations}
		
		We then compute the gradient of $\policytheta(\diff \response \mid \prompt)$ with respect to $\paratheta$. Applying the chain rule, we get
		\begin{align}
			\gradtheta \policytheta(\diff \response \mid \prompt)
			& \; = \; \frac{1}{\Partitiontheta (\prompt)} \, \policyref(\diff \response \mid \prompt)
			\exp \Big\{ \frac{1}{\parabeta} \, \rewardtheta(\prompt, \response) \Big\}
			\cdot \frac{1}{\parabeta} \, \gradtheta \rewardtheta(\prompt, \response)  \notag  \\
			\label{eq:gradtheta1}
			& \quad - \frac{1}{\Partitiontheta^2(\prompt)} \, \policyref(\diff \response \mid \prompt)
			\exp \Big\{ \frac{1}{\parabeta} \, \rewardtheta(\prompt, \response) \Big\}
			\cdot \gradtheta \Partitiontheta(\prompt) \, .
		\end{align}
		We need the gradient of the partition function $\Partitiontheta(\prompt)$:
		\begin{align}
			\gradtheta \Partitiontheta (\prompt)
			& \; = \; \int_{\ResponseSp} \, \policyref(\diff \response \mid \prompt)
			\exp \Big\{ \frac{1}{\parabeta} \, \rewardtheta(\prompt, \response) \Big\}
			\cdot \frac{1}{\parabeta} \, \gradtheta \rewardtheta(\prompt, \response)  \notag   \\
			& \; = \; \Partitiontheta (\prompt) \cdot \int_{\ResponseSp} \, \policytheta(\diff \response \mid \prompt) \cdot \frac{1}{\parabeta} \, \gradtheta \rewardtheta(\prompt, \response)  \notag   \\
			\label{eq:gradPartition}
			& \; = \; \Partitiontheta (\prompt) \cdot \frac{1}{\parabeta} \, \Exp_{\response \sim \policytheta(\cdot \mid \prompt)} \big[ \gradtheta \rewardtheta(\prompt, \response) \big] \, .
		\end{align}
		Substituting equation~\eqref{eq:gradPartition} back into equation~\eqref{eq:gradtheta1}, we simplify the expression for the gradient of $\policytheta(\diff \response \mid \prompt)$:
		\begin{align*}
			& \gradtheta \policytheta(\diff \response \mid \prompt)  \\
			& \; = \; \frac{1}{\Partitiontheta (\prompt)} \, \policyref(\diff \response \mid \prompt)
			\exp \Big\{ \frac{1}{\parabeta} \, \rewardtheta(\prompt, \response) \Big\}
			\cdot \frac{1}{\parabeta} \, \Big\{ \gradtheta \rewardtheta(\prompt, \response) - \Exp_{\responsenew \sim \policytheta(\cdot \mid \prompt)} \big[ \gradtheta \rewardtheta(\prompt, \responsenew) \big] \Big\} \, .
		\end{align*}
		This matches equation~\eqref{eq:gradpolicy} from \Cref{lemma:grad_policy}, thereby completing the proof.
		
		
		
%%%%%%%%%%%%%%%%%%%%%%%%%%%%%%%%%%%%%%%%%%%%%%%%%%%%%%%%%%%%%%%%%%%%%%%%%%%%%%%%%%%%%%%%%%%%
	
	
		\subsubsection{Proof of Lemma~\ref{lemma:grad_scalarvalue} \yaqidone}
		\label{sec:proof:lemma:grad_scalarvalue}
		
		Equality \eqref{eq:grad_scalarvalue} in \Cref{lemma:grad_scalarvalue} can be derived as a consequence of a more detailed result. We state it in \Cref{lemma:grad_scalarvalue_full}.
		
		\begin{lemma}
			\label{lemma:grad_scalarvalue_full}
			\begin{subequations}
			For a policy $\policytheta$, the gradients with respect to the parameter $\paratheta$ of its expected return $\Exp_{\prompt \sim \promptdistr, \, \response \sim \policytheta(\cdot \mid \prompt)} \big[ \rewardstar(\context, \response) \big] $ and its KL divergence from a reference policy $\kull{\policytheta}{\policyref}$ are given by
			\begin{align}
				& \gradtheta \Exp_{\prompt \sim \promptdistr, \, \response \sim \policytheta(\cdot \mid \prompt)} \big[ \rewardstar(\context, \response) \big]  \notag \\
				\label{eq:grad_return}
				& 
				\qquad  = \; \frac{1}{\parabeta} \, \Exp_{\prompt \sim \promptdistr, \,  \response \sim \policytheta(\cdot \mid \prompt)}
				\bigg[ \rewardstar(\prompt, \response)
				\Big\{ \gradtheta \rewardtheta(\prompt, \response) - \Exp_{\responsenew \sim \policytheta(\cdot \mid \prompt)}\big[ \gradtheta \rewardtheta(\prompt, \responsenew) \big] \Big\} \bigg] \, , \\
				& \gradtheta \kull{\policytheta}{\policyref}  \notag  \\
				\label{eq:grad_KL}
				& \qquad = 
				\frac{1}{\parabeta^2} \, \Exp_{\prompt \sim \promptdistr, \, \response \sim \policytheta(\cdot \mid \prompt)}
				\bigg[ \rewardtheta(\prompt, \response)
				\Big\{ \gradtheta \rewardtheta(\prompt, \response) - \Exp_{\responsenew \sim \policytheta(\cdot \mid \prompt)}\big[ \gradtheta \rewardtheta(\prompt, \responsenew) \big] \Big\} \bigg] \, .
			\end{align}
			\end{subequations}
		\end{lemma}
		
		Recall that the scalar value $\scalarvalue(\policytheta)$ of the policy is defined as
		\begin{align*}
			\scalarvalue(\policytheta) \; = \;
			\Exp_{\prompt \sim \promptdistr, \, \response \sim \policytheta(\cdot \mid \prompt)} \big[ \rewardstar(\context, \response) \big] \, - \,
			\parabeta \, \kull{\policytheta}{\policyref} \, .
		\end{align*}
		Using \Cref{lemma:grad_scalarvalue_full}, we derive the gradient of $\scalarvalue(\policytheta)$ as
		\begin{align}
			& \gradtheta \scalarvalue(\policytheta) \; = \; \gradtheta \Exp_{\prompt \sim \promptdistr, \, \response \sim \policytheta(\cdot \mid \prompt)} \big[ \rewardstar(\context, \response) \big] \, - \,
			\parabeta \, \gradtheta \kull{\policytheta}{\policyref}  \notag  \\
			\label{eq:grad_scalarvalue0}
			& \; = \; \frac{1}{\parabeta} \, \Exp_{\prompt \sim \promptdistr, \,  \response \sim \policytheta(\cdot \mid \prompt)}
			\bigg[ \big\{ \rewardstar(\prompt, \response) - \rewardtheta(\prompt, \response) \big\}
			\Big\{ \gradtheta \rewardtheta(\prompt, \response) - \Exp_{\responsenew \sim \policytheta(\cdot \mid \prompt)}\big[ \gradtheta \rewardtheta(\prompt, \responsenew) \big] \Big\} \bigg] \, .
		\end{align}
		We rewrite the expression in equation \eqref{eq:grad_scalarvalue0} in two equivalent forms by exchanging the roles of $\responseone$ and $\responsetwo$:
		\begin{subequations}
		\begin{align}
			& \gradtheta \scalarvalue(\policytheta) \notag \\ 
			\label{eq:grad_scalarvalue1}
			& \; = \; \frac{1}{\parabeta} \, \Exp_{\prompt \sim \promptdistr, \,  \responseone \sim \policytheta(\cdot \mid \prompt)}
			\bigg[ \big\{ \rewardstar(\prompt, \responseone) - \rewardtheta(\prompt, \responseone) \big\} \Big\{ \gradtheta \rewardtheta(\prompt, \responseone) - \Exp_{\responsetwo \sim \policytheta(\cdot \mid \prompt)}\big[ \gradtheta \rewardtheta(\prompt, \responsetwo) \big] \Big\} \bigg] \, ,  \\
			& \gradtheta \scalarvalue(\policytheta) \notag \\ 
			\label{eq:grad_scalarvalue2}
			& \; = \; \frac{1}{\parabeta} \, \Exp_{\prompt \sim \promptdistr, \,  \responsetwo \sim \policytheta(\cdot \mid \prompt)}
			\bigg[ \big\{ \rewardstar(\prompt, \responsetwo) - \rewardtheta(\prompt, \responsetwo) \big\} \Big\{ \gradtheta \rewardtheta(\prompt, \responsetwo) - \Exp_{\responseone \sim \policytheta(\cdot \mid \prompt)}\big[ \gradtheta \rewardtheta(\prompt, \responseone) \big] \Big\} \bigg] \, .
		\end{align}
		\end{subequations}
		By taking the average of the two equivalent formulations above, we obtain equality \eqref{eq:grad_scalarvalue} and complete the proof of \Cref{lemma:grad_scalarvalue}.  \\
		
		We now proceed to prove \Cref{lemma:grad_scalarvalue_full}, tackling equalities \eqref{eq:grad_return} and \eqref{eq:grad_KL} one by one.
		
		\paragraph{Proof of Equality~\eqref{eq:grad_return} from \Cref{lemma:grad_scalarvalue_full}:}
		We begin by expressing the expected return as
		\begin{align*}
			\Exp_{\prompt \sim \promptdistr, \, \response \sim \policytheta(\cdot \mid \prompt)} \big[ \rewardstar(\context, \response) \big]
			& \; = \; \Exp_{\prompt \sim \promptdistr} \bigg[ \int_{\ResponseSp} \rewardstar(\prompt, \response) \, \policytheta(\diff \response \mid \prompt) \bigg] \, .
		\end{align*}
		Taking the gradient of both sides with respect to $\paratheta$, we have
		\begin{align}
			\label{eq:grad_return0}
			\gradtheta \Exp_{\prompt \sim \promptdistr, \, \response \sim \policytheta(\cdot \mid \prompt)} \big[ \rewardstar(\context, \response) \big]
			& \; = \; \Exp_{\prompt \sim \promptdistr} \bigg[ \int_{\ResponseSp} \rewardstar(\prompt, \response) \, \gradtheta \policytheta(\diff \response \mid \prompt) \bigg] \, .
		\end{align}
		Using the expression for the policy gradient $\gradtheta \policytheta$ provided in \Cref{lemma:grad_policy}, the right-hand side of \eqref{eq:grad_return0} simplifies to
		\begin{align*}
			\mbox{RHS of \eqref{eq:grad_return0}}
			& \; = \; \Exp_{\prompt \sim \promptdistr} \bigg[ \int_{\ResponseSp} \rewardstar(\prompt, \response) \, \policytheta(\diff \response \mid \prompt) \cdot \frac{1}{\parabeta} \,
			\Big\{ \gradtheta \rewardtheta(\prompt, \response) - \Exp_{\responsenew \sim \policytheta(\cdot \mid \prompt)}\big[ \gradtheta \rewardtheta(\prompt, \responsenew) \big] \Big\} \bigg]   \\
			& \; = \; \frac{1}{\parabeta} \,\Exp_{\prompt \sim \promptdistr, \, \response \sim \policytheta(\cdot \mid \prompt)} \bigg[ \rewardstar(\prompt, \response) 
			\Big\{ \gradtheta \rewardtheta(\prompt, \response) - \Exp_{\responsenew \sim \policytheta(\cdot \mid \prompt)}\big[ \gradtheta \rewardtheta(\prompt, \responsenew) \big] \Big\} \bigg] \, .
		\end{align*}
		This completes the verification of equation~\eqref{eq:grad_return} from \Cref{lemma:grad_scalarvalue}.
		
		
		
		\paragraph{Proof of Equality~\eqref{eq:grad_KL} from \Cref{lemma:grad_scalarvalue_full}:}
		
		Recall the definition of the KL divergence
		\begin{align*}
			\kull{\policytheta}{\policyref}
			\; = \; \Exp_{\prompt \sim \promptdistr} 
			\bigg[ \int_{\ResponseSp} \policytheta(\diff \response \mid \prompt)
			\log \bigg( \frac{\policytheta(\response \mid \prompt)}{\policyref(\response \mid \prompt)} \bigg) \bigg] \, .
		\end{align*}
		Applying the chain rule, we obtain
		\begin{align}
			\gradtheta \kull{\policytheta}{\policyref}
			& \, = \, \Exp_{\prompt \sim \promptdistr}  \bigg[ \int_{\ResponseSp} \gradtheta \policytheta(\diff \response \mid \prompt) \,
			\log \bigg( \frac{\policytheta(\response \mid \prompt)}{\policyref(\response \mid \prompt)}\bigg) \bigg]  
			\label{eq:grad_KL2}
			+ \Exp_{\prompt \sim \promptdistr}  \bigg[ \int_{\ResponseSp} 
			\gradtheta \policytheta(\diff \response \mid \prompt) \bigg] \, .
		\end{align}
		
		Since the policy integrates to $1$, i.e., $\int_{\ResponseSp} 
		\policytheta(\diff \response \mid \prompt) = 1$, it always holds that
		\begin{align}
			\label{eq:int_grad_policy}
			\int_{\ResponseSp} 
			\gradtheta \policytheta(\diff \response \mid \prompt)
			\; = \; \gradtheta \int_{\ResponseSp} 
			\policytheta(\diff \response \mid \prompt)
			\; = \; 0 \, ,
		\end{align}
		i.e., the second term on the right-hand side of \eqref{eq:grad_KL2} is zero.
		Using the expression \eqref{eq:policyfromreward}, we take the logarithm
		\begin{align}
			\label{eq:grad_KL0}
			\log \bigg( \frac{\policytheta(\response \mid \prompt)}{\policyref(\response \mid \prompt)} \bigg)
			\; = \; \frac{1}{\parabeta} \, \rewardtheta(\prompt, \response) - \log \Partitiontheta (\prompt) \, .
		\end{align}
		Combining equations~\eqref{eq:int_grad_policy} and \eqref{eq:grad_KL0}, we get
		\begin{align}
			& \int_{\ResponseSp} \gradtheta \policytheta(\diff \response \mid \prompt) \,
			\log \bigg( \frac{\policytheta(\response \mid \prompt)}{\policyref(\response \mid \prompt)}\bigg)  \notag  \\
			& \; = \; \frac{1}{\parabeta} \int_{\ResponseSp} \rewardtheta(\prompt, \response) \, \gradtheta \policytheta(\diff \response \mid \prompt) \; - \; \log \Partitiontheta(\prompt) \int_{\ResponseSp} \gradtheta \policytheta(\diff \response \mid \prompt)  \notag  \\
			\label{eq:grad_KL1}
			& \; = \; \frac{1}{\parabeta} \int_{\ResponseSp} \rewardtheta(\prompt, \response) \, \gradtheta \policytheta(\diff \response \mid \prompt) \, .
		\end{align}
		
		Now, similar to the proof of equation \eqref{eq:grad_return}, we derive
		\begin{align*}
			\mbox{RHS of \eqref{eq:grad_KL2}}
			& \; = \; \frac{1}{\parabeta} \, \Exp_{\prompt \sim \promptdistr} \bigg[ \int_{\ResponseSp} \rewardtheta(\prompt, \response) \, \gradtheta \policytheta(\diff \response \mid \prompt) \bigg]  \\
			& \; = \; \frac{1}{\parabeta^2} \,\Exp_{\prompt \sim \promptdistr, \, \response \sim \policytheta(\cdot \mid \prompt)} \bigg[ \rewardtheta(\prompt, \response) 
			\Big\{ \gradtheta \rewardtheta(\prompt, \response) - \Exp_{\responsenew \sim \policytheta(\cdot \mid \prompt)}\big[ \gradtheta \rewardtheta(\prompt, \responsenew) \big] \Big\} \bigg] \, ,
		\end{align*}
		which verifies equality~\eqref{eq:grad_KL} from \Cref{lemma:grad_scalarvalue_full}.
		

		
		
		
		
%%%%%%%%%%%%%%%%%%%%%%%%%%%%%%%%%%%%%%%%%%%%%%%%%%%%%%%%%%%%%%%%%%%%%%%%%%%%%%%%%%%%%%%%%%%%



	\subsubsection{Proof of Lemma~\ref{lemma:grad_loss} \yaqidone}
	\label{sec:proof:lemma:grad_loss}
	
	In this section, we prove a full version of \Cref{lemma:grad_loss} as stated in \Cref{lemma:grad_loss_full} below. Equation~\eqref{eq:gradLoss_BT_0} from \Cref{lemma:grad_loss} follows directly as a straightforward corollary.
	
	In \Cref{lemma:grad_loss_full}, we consider a general class of distributions parameterized by $\paratheta$ that models the binary preference \mbox{$\Probtheta(\responseone \succ \responsetwo \mid \prompt)$}. The negative log-likelihood function is defined as
	\begin{align*}
		\Loss(\theta) = - \Exp_{\prompt \sim \promptdistr; \; (\responseone, \responsetwo) \sim \responsedistr(\cdot \mid \prompt)} \Big[ \weight(\prompt) \cdot \log \Probtheta( \responsewin \succ \responselose \bigm| \prompt) \Big] \, .
	\end{align*}
	The Bradley-Terry (BT) model described in equation~\eqref{eq:BT} and the corresponding loss function~$\Loss(\paratheta)$ in equation~\eqref{eq:Loss0} represent a special case of this general framework.
	
	\begin{lemma}[Gradient of the loss function $\Loss(\paratheta)$, full version]
		\label{lemma:grad_loss_full}
		\begin{subequations}
			For a general distribution class $\{ \Probtheta \}$, the gradient of $\Loss(\paratheta)$ with respect to $\paratheta$ is given by
			\begin{multline}
				\label{eq:gradLoss_general}
				\gradtheta \Loss(\paratheta) \; = \; - \, \Exp_{ \prompt \sim \promptdistr; \; (\responseone, \responsetwo) \sim \responsedistravg(\cdot \mid \prompt) }
				\bigg[ \, \weight(\prompt) \cdot \Big\{ \Prob\big( \responseone \succ \responsetwo \bigm| \prompt \big) - \Probtheta \big( \responseone \succ \responsetwo \bigm| \prompt \big) \Big\} \\
				\cdot \frac{\gradtheta \Probtheta( \responseone \succ \responsetwo \mid \prompt )}{\Probtheta( \responseone \succ \responsetwo \mid \prompt ) \, \Probtheta( \responsetwo \succ \responseone \mid \prompt )} \, \bigg] \, ,
			\end{multline}
			where $\responsedistravg$ is the average distribution defined in equation~\eqref{eq:def_responsedistravg_0}.
			Specifically, for the Bradley-Terry (BT) model where
			\begin{align*}
				\Probtheta \big( \responseone \succ \responsetwo \bigm| \prompt \big)
				\; = \; \sigmoid \big( \rewardtheta(\prompt, \responseone) - \rewardtheta(\prompt, \responsetwo) \big)
				\; = \; \bigg\{ 1 + \bigg( \frac{(\policytheta/\policyref)(\responsetwo \mid \prompt)}{(\policytheta/\policyref)(\responseone \mid \prompt)} \bigg)^{\parabeta} \bigg\}^{-1} \, ,
			\end{align*}
			the gradient of $\Loss(\paratheta)$ becomes
			\begin{multline}
				\label{eq:gradLoss_BT}
				\gradtheta \Loss(\paratheta) \; = \; - \, \Exp_{\prompt \sim \promptdistr; \; (\responseone, \, \responsetwo) \sim \responsedistravg(\cdot \mid \prompt)}
				\bigg[ \, \weight(\prompt) \cdot \Big\{ \sigmoid \big( \rewardstar(\context, \responseone) - \rewardstar(\context, \responsetwo) \big) - \sigmoid \big( \rewardtheta(\context, \responseone) - \rewardtheta(\context, \responsetwo) \big) \Big\} \\ 
				\cdot \big\{ \gradtheta \rewardtheta(\prompt, \responseone) - \gradtheta \rewardtheta(\prompt, \responsetwo) \big\} \bigg] \, .
			\end{multline}
		\end{subequations}
	\end{lemma}
	
	
	For notational simplicity, we focus on the proof for the case where the weight function $\weight(\prompt) = 1$. The results for a general weight function $\weight(\prompt) > 0$ can be derived in a similar manner.
	
	Recall that the negative log-likelihood function $\Loss(\paratheta)$ is defined as
	\begin{align*}
		\Loss(\paratheta) & \; = \;
		\Exp \Big[ - \log \Probtheta\big( \responsewin \succ \responselose \bigm| \prompt \big) \Big] \, .
	\end{align*}
	Based on the data generation mechanism, we can expand the expectation in $\Loss(\paratheta)$ as
%	\begin{subequations}
	\begin{align}
		\Loss(\paratheta)
		& \; = \; \Exp_{\prompt \sim \promptdistr; \; (\responseone, \, \responsetwo) \sim \responsedistr(\cdot \mid \prompt)}
		\Big[ \, \Prob\big( \responseone \succ \responsetwo \bigm| \prompt \big) \cdot \big\{ - \log \Probtheta \big( \responseone \succ \responsetwo \bigm| \prompt \big) \big\}  \notag  \\
		\label{eq:Loss0}
		& \qquad \qquad \qquad \qquad \qquad + \Prob\big( \responsetwo \succ \responseone \bigm| \prompt \big) \cdot \big\{ - \log \Probtheta\big( \responsetwo \succ \responseone \bigm| \prompt \big) \big\} \Big] \, .
	\end{align}
	Notice that we can exchange the roles of $\responseone$ and $\responsetwo$ in the expectation above. This means that we can equivalently express the expectation using the pair $(\responsetwo, \responseone) \sim \responsedistr(\cdot \mid \prompt)$.
	This symmetry allows us to replace $\responsedistr$ in equation~\eqref{eq:Loss0} with the average distribution $\responsedistravg$ as defined in equation~\eqref{eq:def_responsedistravg_0}. \\
	
	Next, we take the gradient of the loss function $\Loss(\paratheta)$ with respect to the parameter $\paratheta$ and obtain
	\begin{align*}
		\gradtheta \Loss(\paratheta)
		& \; = \; \Exp_{\prompt \sim \promptdistr, \, (\responseone, \, \responsetwo) \sim \responsedistravg(\cdot \mid \prompt)}
		\bigg[ \, \frac{\Prob( \responseone \succ \responsetwo \mid \prompt )}{\Probtheta ( \responseone \succ \responsetwo \mid \prompt )} \cdot \big\{ - \gradtheta \Probtheta( \responseone \succ \responsetwo \mid \prompt ) \big\}   \\
		& \qquad \qquad \qquad \qquad \qquad + \frac{\Prob( \responsetwo \succ \responseone \mid \prompt )}{\Probtheta( \responsetwo \succ \responseone \mid \prompt )} \cdot \big\{ - \gradtheta \Probtheta( \responsetwo \succ \responseone \mid \prompt ) \big\} \, \bigg] \, .
	\end{align*}
	Note that $\Prob\big( \responsetwo \succ \responseone \bigm| \prompt\big) = 1 - \Prob\big( \responseone \succ \responsetwo \bigm| \prompt\big)$ and $\Probtheta \big( \responsetwo \succ \responseone \bigm| \prompt\big) = 1 - \Probtheta \big( \responseone \succ \responsetwo \bigm| \prompt\big)$.
	Using this, we can rewrite the gradient as
	\begin{align*}
		& \gradtheta \Loss(\paratheta)  \\
		& \; = \;
		\Exp_{\prompt \sim \promptdistr; \; (\responseone, \, \responsetwo) \sim \responsedistravg(\cdot \mid \prompt)}
		\bigg[ \bigg\{ \frac{1 - \Prob( \responseone \succ \responsetwo \mid \prompt )}{1 - \Probtheta( \responseone \succ \responsetwo \mid \prompt )} - \frac{\Prob( \responseone \succ \responsetwo \mid \prompt )}{\Probtheta ( \responseone \succ \responsetwo \mid \prompt )} \bigg\} \cdot \gradtheta \Probtheta\big( \responseone \succ \responsetwo \bigm| \prompt \big) \bigg] \, .
	\end{align*}
	We simplify the expression further to obtain
	\begin{multline*}
		\gradtheta \Loss(\paratheta)
		\; = \;
		\Exp_{\prompt \sim \promptdistr; \; (\responseone, \, \responsetwo) \sim \responsedistravg(\cdot \mid \prompt)}
		\bigg[ \Big\{ \Probtheta \big( \responseone \succ \responsetwo \bigm| \prompt \big) - \Prob \big( \responseone \succ \responsetwo \bigm| \prompt \big) \Big\} \\ \cdot \frac{\gradtheta \Probtheta( \responseone \succ \responsetwo \mid \prompt )}{\Probtheta( \responseone \succ \responsetwo \mid \prompt ) \, \Probtheta( \responsetwo \succ \responseone \mid \prompt )} \bigg] \, .
	\end{multline*}
	This establishes equation~\eqref{eq:gradLoss_general} from \Cref{lemma:grad_loss}. \\
	
	As for the Bradley-Terry (BT) model, we use the equality
	\begin{align*}
		\divsigmoid(z) \; = \; \frac{1}{(1 + \exp(-z))(1 + \exp(z))} \; = \; \sigmoid(z) \, \sigmoid(-z)
		\qquad \mbox{for any $z \in \Real$}
	\end{align*}
	to derive the following expression
	\begin{align}
		\label{eq:grad_reward}
		\frac{\gradtheta \Probtheta( \responseone \succ \responsetwo \mid \prompt )}{\Probtheta( \responseone \succ \responsetwo \mid \prompt ) \, \Probtheta( \responsetwo \succ \responseone \mid \prompt )}
		\; = \; \gradtheta \rewardtheta(\prompt, \responseone) - \gradtheta \rewardtheta(\prompt, \responsetwo) \, .
	\end{align}
	By substituting this gradient expression from equation~\eqref{eq:grad_reward} into equation~\eqref{eq:gradLoss_general}, we directly obtain equation~\eqref{eq:gradLoss_BT}, thereby completing the proof of \Cref{lemma:grad_loss}.
	
	
%%%%%%%%%%%%%%%%%%%%%%%%%%%%%%%%%%%%%%%%%%%%%%%%%%%%%%%%%%%%%%%%%%%%%%%%%%%%%%%

	\subsection{Proof of Auxiliary Results for Theorem~\ref{thm:asymp} \yaqidone}
	\label{sec:proof:thm:asymp_aux}
	
	In this section, we present the detailed proofs of the supporting lemmas used in the proof of \Cref{thm:asymp}. 
	We begin in \Cref{sec:proof:eq:master_cond_proof} by establishing condition~\eqref{eq:master_cond_proof}, which is crucial for the valid application of the master theorem for $Z$-estimators. Following this, in \Cref{sec:proof:lemma:hess_loss}, we compute the Hessian matrix $\hesstheta \Loss(\parathetastar)$ explicitly. Finally, in \Cref{sec:proof:lemma:grad_loss_stat}, we derive the asymptotic distribution of the gradient~$\gradtheta \Losshat(\parathetastar)$.
	
	
%%%%%%%%%%%%%%%%%%%%%%%%%%%%%%%%%%%%%%%%%%%%%%%%%%%%%%%%%%%%%%%

	\subsubsection{Proof of Condition~\eqref{eq:master_cond_proof}}
	\label{sec:proof:eq:master_cond_proof}
	We begin by rewriting the left-hand side of equation~\eqref{eq:master_cond_proof} as follows:
	\begin{align}
		\Delta
		& \; \defn \; \sqrt{n} \, \big\{ \gradtheta \Losshat (\parathetahat) - \gradtheta \Loss(\parathetahat) \big\} - \sqrt{n} \, \big\{ \gradtheta \Losshat (\parathetastar) - \gradtheta \Loss (\parathetastar) \big\}   \notag  \\
		& \; = \; \sqrt{n} \, \big\{ \gradtheta \Losshat (\parathetahat) - \gradtheta \Losshat(\parathetastar) \big\} - \sqrt{n} \, \big\{ \gradtheta \Loss (\parathetahat) - \gradtheta \Loss (\parathetastar) \big\} \, .
		\label{eq:master_0}
	\end{align}
	We then leverage the smoothness properties of the function $\rewardtheta$, which guarantee the following approximations:
	\begin{subequations}
		\begin{align}
			\label{eq:gradLosshat_smooth}
			\gradtheta \Losshat(\parathetahat) - \gradtheta \Losshat(\parathetastar) & \; = \; \hesstheta \Losshat(\parathetastar) \, (\parathetahat - \parathetastar) + \smallop \big( \norm{\parathetahat - \parathetastar}_2 \big) \, ,  \\
			\label{eq:gradLoss_smooth}
			\gradtheta \Loss(\parathetahat) - \gradtheta \Loss(\parathetastar) & \; = \; \hesstheta \Loss(\parathetastar) \, (\parathetahat - \parathetastar) + \smallop \big( \norm{\parathetahat - \parathetastar}_2 \big) \, .
		\end{align}
	\end{subequations}
	Assuming these equalities~\eqref{eq:gradLosshat_smooth} and~\eqref{eq:gradLoss_smooth} hold, we substitute them into equation~\eqref{eq:master_0}, leading to
	\begin{align}
		\Delta
		& \; = \; \sqrt{n} \, \big\{ \hesstheta \Losshat (\parathetastar) \, (\parathetahat - \parathetastar) + \smallop( \norm{\parathetahat - \parathetastar}_2 ) \big\} - \sqrt{n} \, \big\{ \hesstheta \Loss (\parathetastar) \, (\parathetahat - \parathetastar) + \smallop( \norm{\parathetahat - \parathetastar}_2 ) \big\}  \notag \\
		& \; = \; \sqrt{n} \, \big\{ \hesstheta \Losshat (\parathetastar) - \hesstheta \Loss (\parathetastar) \big\} (\parathetahat - \parathetastar) + \smallop \big( 1 + \sqrt{n} \, \norm{ \parathetahat - \parathetastar }_2 \big) \, .
		\label{eq:master_1}
	\end{align}
	Using the law of large numbers, we know that $\hesstheta \Losshat (\parathetastar) \convergep \hesstheta \Loss (\parathetastar)$, which implies
	\begin{align*}
		\sqrt{n} \, \big\{ \hesstheta \Losshat (\parathetastar) - \hesstheta \Loss (\parathetastar) \big\} (\parathetahat - \parathetastar) \; = \; \smallop \big( \sqrt{n} \, \norm{ \parathetahat - \parathetastar }_2 \big) \, .
	\end{align*}
	Therefore, we conclude that
	\begin{align*}
		\Delta \; = \; \smallop \big( 1 + \sqrt{n} \, \norm{ \parathetahat - \parathetastar }_2 \big)
	\end{align*}
	as claimed in equation~\eqref{eq:master_cond_proof}.
	
	The only remaining task is to establish the validity of equalities~\eqref{eq:gradLosshat_smooth} and~\eqref{eq:gradLoss_smooth}.
	
	
	\paragraph{Proof of Equalities~\eqref{eq:gradLosshat_smooth}~and~\eqref{eq:gradLoss_smooth}:}
	
	We express the loss function $\Losshat(\paratheta)$ in the form
	\begin{align*}
		\Losshat(\paratheta) \; \defn \;
		\frac{1}{\numobs} \sum_{i=1}^{\numobs} \weight(\prompti{i}) \cdot \lliketheta\big(\prompti{i}, \responsewini{i}, \responselosei{i}\big) \, ,
	\end{align*}
	where the function $\lliketheta$ is defined as
	\begin{align*}
		\lliketheta(\prompt, \responsei{1}, \responsei{2})
		\; = \; - \log \sigmoid \big( \rewardtheta(\prompt, \responsei{1}) - \rewardtheta(\prompt, \responsei{2}) \big) \, .
	\end{align*}
	We then calculate the gradient $\gradtheta \lliketheta$ and $\hesstheta \lliketheta$ as follows:
	\begin{align*}
		\gradtheta \lliketheta(\prompt, \responsei{1}, \responsei{2})
		& \; = \; \sigmoid\big( \rewardtheta(\prompt, \responsei{2}) - \rewardtheta(\prompt, \responsei{1}) \big) \cdot \big\{ \gradtheta \rewardtheta(\prompt, \responsei{2}) - \gradtheta \rewardtheta(\prompt, \responsei{1})  \big\} \qquad \mbox{and}  \\
		\hesstheta \lliketheta(\prompt, \responsei{1}, \responsei{2})
		& \; = \; \divsigmoid\big( \rewardtheta(\prompt, \responsei{2}) - \rewardtheta(\prompt, \responsei{1}) \big) \\
        & \qquad \quad
        \cdot \big\{ \gradtheta \rewardtheta(\prompt, \responsei{2}) - \gradtheta \rewardtheta(\prompt, \responsei{1}) \big\} \big\{ \gradtheta \rewardtheta(\prompt, \responsei{2}) - \gradtheta \rewardtheta(\prompt, \responsei{1}) \big\}^{\top}  \\
		& \quad + \sigmoid\big( \rewardtheta(\prompt, \responsei{2}) - \rewardtheta(\prompt, \responsei{1}) \big) \cdot \big\{ \hesstheta \rewardtheta(\prompt, \responsei{2}) - \hesstheta \rewardtheta(\prompt, \responsei{1})  \big\} \, .
	\end{align*}
	When the reward function $\rewardtheta(\prompt, \response)$, along with its gradient $\gradtheta \rewardtheta(\prompt, \response)$ and Hessian $\hesstheta \rewardtheta(\prompt, \response)$, is uniformly bounded and Lipschitz continuous with respect to $\paratheta$ for all $(\prompt, \response) \in \PromptSp \times \ResponseSp$, it guarantees that the Hessian of the loss function, $\hesstheta \lliketheta$, is also Lipschitz continuous. This holds with some constant $\Liphess > 0$ across all $(\prompt, \response) \in \PromptSp \times \ResponseSp$, as demonstrated below:
	\begin{align*}
		\norm[\big]{\hesstheta \lliketheta (\prompt, \responsei{1}, \responsei{2}) - \hesstheta \llikethetastar (\prompt, \responsei{1}, \responsei{2})}_2
		\; \leq \; \Liphess \cdot \norm{\paratheta - \parathetastar}_2 \, .
	\end{align*}
	From this Lipschitz property, we deduce
	\begin{align*}
		\norm[\big]{\gradtheta \lliketheta (\prompt, \responsei{1}, \responsei{2}) - \gradtheta \llikethetastar (\prompt, \responsei{1}, \responsei{2}) - \hesstheta \llikethetastar (\prompt, \responsei{1}, \responsei{2}) \, (\paratheta - \parathetastar)}_2 \; \leq \; \frac{\Liphess}{2} \cdot \norm{\paratheta - \parathetastar}_2^2
	\end{align*}
	and further derive
	\begin{align*}
		\norm[\big]{\gradtheta \Losshat(\paratheta) - \gradtheta \Losshat(\parathetastar) - \hesstheta \Losshat(\parathetastar) \, (\paratheta - \parathetastar)}_2 & \; \leq \; \frac{\Liphess \, \supnorm{\weight}}{2} \cdot \norm{\paratheta - \parathetastar}_2^2 \, ,  \\
		\norm[\big]{\gradtheta \Loss(\paratheta) - \gradtheta \Loss(\parathetastar) - \hesstheta \Loss(\parathetastar) \, (\paratheta - \parathetastar)}_2 & \; \leq \; \frac{\Liphess \, \supnorm{\weight}}{2} \cdot \norm{\paratheta - \parathetastar}_2^2 \, .
	\end{align*}
	Finally, under the condition that $\parathetahat \convergep \parathetastar$, these results simplify to the expressions given in equations~\eqref{eq:gradLosshat_smooth} and~\eqref{eq:gradLoss_smooth}, as previously claimed.
	
	
%%%%%%%%%%%%%%%%%%%%%%%%%%%%%%%%%%%%%%%%%%%%%%%%%%%%%%%%%%%%%%%
	
	\subsubsection{Proof of Lemma~\ref{lemma:hess_loss}, Explicit Form of Hessian $\hesstheta \Loss(\parathetastar)$}
	\label{sec:proof:lemma:hess_loss}
	
	From equation~\eqref{eq:gradLoss_BT_0} in \Cref{lemma:grad_loss}, we recall the explicit formula for the gradient $\gradtheta \Loss(\paratheta)$. Taking the derivative of both sides of equation~\eqref{eq:gradLoss_BT_0}, we obtain
	\begin{align}
		& \begin{aligned} 
		\hesstheta \Loss(\paratheta) \; = \; \Exp_{\prompt \sim \promptdistr; \; (\responseone, \, \responsetwo) \sim \responsedistravg(\cdot \mid \prompt)}
		& \Big[ \, \weight(\prompt) \cdot \divsigmoid \big( \rewardtheta(\context, \responseone) - \rewardtheta(\context, \responsetwo) \big) \\ 
		& \cdot \big\{ \gradtheta \rewardtheta(\prompt, \responseone) - \gradtheta \rewardtheta(\prompt, \responsetwo) \big\} \big\{ \gradtheta \rewardtheta(\prompt, \responseone) - \gradtheta \rewardtheta(\prompt, \responsetwo) \big\}^{\top} \Big] \end{aligned}   \notag  \\
		& \qquad \qquad \quad
		\begin{aligned} 
		- \, \Exp_{\prompt \sim \promptdistr; \; (\responseone, \, \responsetwo) \sim \responsedistravg(\cdot \mid \prompt)}
		\bigg[ \, \weight(\prompt) & \cdot \Big\{ \sigmoid \big( \rewardstar(\context, \responseone) - \rewardstar(\context, \responsetwo) \big) - \sigmoid \big( \rewardtheta(\context, \responseone) - \rewardtheta(\context, \responsetwo) \big) \Big\} \\ 
		& \cdot \big\{ \hesstheta \rewardtheta(\prompt, \responseone) - \hesstheta \rewardtheta(\prompt, \responsetwo) \big\} \bigg] \, .
		\end{aligned}
		\label{eq:hessLoss_0}
	\end{align}
	When we set $\paratheta = \parathetastar$, it follows that $\rewardtheta = \rewardstar$. This simplification eliminates the second term in expression~\eqref{eq:hessLoss_0}, reducing the Hessian matrix to
	\begin{multline*}
		\hesstheta \Loss(\parathetastar) \; = \; \Exp_{\prompt \sim \promptdistr; \; (\responseone, \, \responsetwo) \sim \responsedistravg(\cdot \mid \prompt)}
		\Big[ \, \weight(\prompt) \cdot \divsigmoid \big( \rewardstar(\context, \responseone) - \rewardstar(\context, \responsetwo) \big) \\ 
		\cdot \big\{ \gradtheta \rewardstar(\prompt, \responseone) - \gradtheta \rewardstar(\prompt, \responsetwo) \big\} \big\{ \gradtheta \rewardstar(\prompt, \responseone) - \gradtheta \rewardstar(\prompt, \responsetwo) \big\}^{\top} \Big] \, .
	\end{multline*}
	Substituting the derivative $\divsigmoid$ with its explicit form, $\divsigmoid(z) = \sigmoid(z) \, \sigmoid(-z)$ for any $z \in \Real$, we refine the expression to
	\begin{align*}
		\hesstheta \Loss(\parathetastar) \; = \; \CovOpstar \, ,
	\end{align*}
	where the covariance matrix $\CovOpstar$ is defined in equation~\eqref{eq:def_CovOpstar}.
	This completes the proof of expression~\eqref{eq:hess_loss} from \Cref{lemma:hess_loss}.
	
%%%%%%%%%%%%%%%%%%%%%%%%%%%%%%%%%%%%%%%%%%%%%%%%%%%%%%%%%%%%%%%%%%%%%%%%%%%
	
	\subsubsection{Proof of Lemma~\ref{lemma:grad_loss_stat}, Asymptotic Distribution of Graident $\gradtheta \Losshat(\parathetastar)$}
	\label{sec:proof:lemma:grad_loss_stat}
	
	In this section, we analyze the asymptotic distribution of the gradient $\gradtheta \Losshat(\paratheta)$ at $\paratheta = \parathetastar$, where the loss function $\Losshat(\paratheta)$ is defined as
	\begin{align*}
		\Losshat(\paratheta) \; = \;
		- \frac{1}{\numobs} \sum_{i=1}^{\numobs} \, \weight(\prompt) \cdot \log \sigmoid \Big( \rewardtheta\big(\prompti{i}, \responsewini{i}\big) - \rewardtheta\big(\prompti{i}, \responselosei{i}\big) \Big) \, .
	\end{align*}
	Using the definition of the sigmoid function $\sigmoid$, we calculate that
	\begin{align*}
		( \log \sigmoid(z) )' = \divsigmoid(z) / \sigmoid(z) = \sigmoid(z) \, \sigmoid(-z) / \sigmoid(z) = \sigmoid(-z) \qquad \mbox{for any real number $z \in \Real$}.
	\end{align*}
	This allows us to reformulate $\gradtheta \Losshat(\paratheta)$ as the average of $\numobs$ i.i.d. vectors $\{ \vecgi{i} \}_{i=1}^{\numobs}$:
	\begin{align}
		\label{eq:gradLosshat}
		\gradtheta \Losshat(\paratheta)
		\; = \; \frac{1}{\numobs} \sum_{i=1}^{\numobs} \, \vecgi{i} \, .
	\end{align}
	Here each vector $\vecgi{i} \in \Real^{\Dim}$ is defined as
	\begin{align*}
		\vecgi{i} \; \defn \; \weight(\prompt) \cdot \sigmoid \big( \rewardtheta(\prompti{i}, \responselosei{i}) - \rewardtheta(\prompti{i}, \responsewini{i}) \big) \cdot \big\{ \gradtheta \rewardtheta(\prompti{i}, \responselosei{i}) - \gradtheta \rewardtheta(\prompti{i}, \responsewini{i}) \big\} \, .
	\end{align*}
%	(consistently with expression~\eqref{eq:def_grad}).
	At $\paratheta = \parathetastar$, we denote $\vecgi{i}$ as $\vecgstari{i}$ and $\gradi{i}$ as $\gradstari{i}$. Notably, vector $\vecgi{i}$ can be rewritten as
	\begin{align}
		\label{eq:vecgi2}
		\vecgi{i} 
		& \; = \; \weight(\prompt) \cdot \big\{ \sigmoid \big( \rewardtheta(\prompti{i}, \responseonei{i}) - \rewardtheta(\prompti{i}, \responsetwoi{i}) \big) - \indicator\{ \responseonei{i} = \responsewini{i}, \responsetwoi{i} = \responselosei{i} \} \big\}
		\cdot \gradi{i} \, ,
	\end{align}
	where $\gradi{i}$ is given by
	\begin{align*}
		\gradi{i} \defn \gradtheta \rewardtheta(\prompti{i}, \responseonei{i}) - \gradtheta \rewardtheta(\prompti{i}, \responsetwoi{i}) \, .
	\end{align*}
	From the structure of the BT model, it holds that
	\begin{align*}
		\Exp\big[ \indicator \{ \responseonei{i} = \responsewini{i}, \responsetwoi{i} = \responselosei{i} \} \bigm| \prompti{i} \big] \; = \; \sigmoid \big( \rewardstar(\prompti{i}, \responseonei{i}) - \rewardstar(\prompti{i}, \responsetwoi{i}) \big) \, ,
	\end{align*}
	which implies $\Exp[\vecgstari{i}] = \veczero$.
	
	
	To analyze the asymptotic distribution of $\gradtheta \Losshat(\parathetastar)$, we apply the central limit theorem (CLT) to its empirical form given in equation~\eqref{eq:gradLosshat}. 
%	\yaqiadd{Check the conditions for CLT.}
	By the CLT, we have
	\begin{align}
		\label{eq:gradLoss_CLT}
		\sqrt{\numobs} \, \big( \gradtheta \Losshat(\parathetastar) - \gradtheta \Loss(\parathetastar) \big)
		\; \stackrel{d}{\longrightarrow} \; \Gauss\big(\veczero, \CovOptil \big) \, ,
		\qquad \numobs \rightarrow \infty \, ,
	\end{align}
	where the covariance matrix $\CovOptil \in \Real^{\Dim \times \Dim}$ is given by
	\begin{align*}
		\CovOptil \; \defn \; \Cov(\vecgstari{1}) \; = \; \Exp\big[ \vecgstari{1} (\vecgstari{1})^{\top} \big] \, .
	\end{align*}
	Here we have used the property $\Exp[\vecgstari{i}] = \veczero$ in the second equality.
	
	We now compute the explicit form of the covariance matrix $\CovOptil$. Using the definition of $\vecgi{i}$ from expression~\eqref{eq:vecgi2}, we find that
	\begin{align}
		& \CovOptil \; = \; \Exp\big[ \vecgstari{1} (\vecgstari{1})^{\top} \big] \notag  \\
		& = \; \Exp_{\, \begin{subarray}{l} ~ \\ \prompt \sim \promptdistr; \\ (\responseone, \responsetwo) \sim \responsedistravg(\cdot \mid \prompt)\end{subarray}} \Big[ \, \weight^2(\prompt) \cdot \big\{ \sigmoid \big( \rewardstar(\prompti{1}, \responseonei{1}) - \rewardstar(\prompti{1}, \responsetwoi{1}) \big) - \indicator\{ \responseonei{1} = \responsewini{1}, \responsetwoi{1} = \responselosei{1} \} \big\}^2 \cdot \gradstari{1} (\gradstari{1})^{\top} \Big] \, .
		\label{eq:CovOptil_2}
	\end{align}
	Taking the conditional expectation over the outcomes of winners and losers, and using the relation
	\begin{align*}
		&  \Exp\Big[
		\big\{ \sigmoid \big( \rewardstar(\prompti{1}, \responseonei{1}) - \rewardstar(\prompti{1}, \responsetwoi{1}) \big) - \indicator\{ \responseonei{1} = \responsewini{1}, \responsetwoi{1} = \responselosei{1} \} \big\}^2 \Bigm| \prompti{1}, \responseonei{1}, \responsetwoi{1} \Big]  \\
		& \; = \; \Var \Big( \indicator\{ \responseonei{1} = \responsewini{1}, \responsetwoi{1} = \responselosei{1} \} \Bigm|  \prompti{1}, \responseonei{1}, \responsetwoi{1} \Big)  \\
		& \; = \; \sigmoid \big( \rewardstar(\prompti{i}, \responseonei{i}) - \rewardstar(\prompti{i}, \responsetwoi{i}) \big) \, \sigmoid \big( \rewardstar(\prompti{i}, \responsetwoi{i}) - \rewardstar(\prompti{i}, \responseonei{i}) \big) \, ,
	\end{align*}
	we reduce equation~\eqref{eq:CovOptil_2} to
	\begin{align*}
		\CovOptil
		& \; = \; \Exp_{\prompt \sim \promptdistr; \; (\responseone, \responsetwo) \sim \responsedistravg(\cdot \mid \prompt)} \Big[ \, \weight^2(\prompt) \cdot \Var \big( \indicator\{ \responseonei{1} = \responsewini{1}, \responsetwoi{1} = \responselosei{1} \} \bigm|  \prompti{1}, \responseonei{1}, \responsetwoi{1} \big) \cdot \gradstari{1} (\gradstari{1})^{\top} \Big] \, .
	\end{align*}
	Bounding the weight function $\weight(\prompt)$ by its uniform bound $\supnorm{\weight}$, we simplify further:
	\begin{align*}
		\CovOptil
        & \; \preceq \; \supnorm{\weight} \cdot \Exp\Big[ \, \weight(\prompt) \cdot \Var \big( \indicator\{ \responseonei{1} = \responsewini{1}, \responsetwoi{1} = \responselosei{1} \} \bigm|  \prompti{1}, \responseonei{1}, \responsetwoi{1} \big) \cdot \gradstari{1} (\gradstari{1})^{\top} \Big] \, .
    \end{align*}
    This ultimately reduces to
    \begin{align}
    	\label{eq:CovOptil_ub}
         \CovOptil & \; \preceq \; \supnorm{\weight} \cdot \CovOpstar
	\end{align}
	where $\CovOpstar$ is defined in equation~\eqref{eq:def_CovOpstar}.
    
    Finally, by combining equations~\eqref{eq:gradLoss_CLT} and~\eqref{eq:CovOptil_ub}, we establish the asymptotic normality of $\gradtheta \Losshat(\parathetastar)$ and complete the proof of \Cref{lemma:grad_loss_stat}.
    
    
%%%%%%%%%%%%%%%%%%%%%%%%%%%%%%%%%%%%%%%%%%%%%%%%%%%%%%%%%%%%%%
	
	\subsection{Proof of Auxiliary Results for Theorem~\ref{lemma:hess_scalarvalue} \yaqidone}
	\label{sec:proof:lemma:hess_scalarvalue_aux}
	
	This section contains the proofs of the auxiliary results supporting \Cref{lemma:hess_scalarvalue}. In \Cref{sec:proof:eq:hessscalarvalue}, we derive the explicit form of the Hessian $ \hesstheta \scalarvalue(\policystar) $. \Cref{sec:proof:gap_distr} rigorously establishes the asymptotic distribution of the value gap (equation~\eqref{eq:gap_distr}). Finally, \Cref{sec:proof:chisqtail} proves the tail bound~\eqref{eq:gap_bd} on the chi-square distribution $ \chisquare_{\Dim} $.
	
	\subsubsection{Proof of Equation~\eqref{eq:hessscalarvalue} from Theorem~\ref{lemma:hess_scalarvalue}, Explicit Form of Hessian $\hesstheta \scalarvalue(\policystar)$}
	\label{sec:proof:eq:hessscalarvalue}
	
	We begin by differentiating expression~\eqref{eq:grad_scalarvalue0} for the gradient $\gradtheta \scalarvalue(\policytheta)$ to obtain the Hessian matrix $\hesstheta \scalarvalue(\policytheta)$. The resulting expression can be written as
	\begin{align*}
		\hesstheta \scalarvalue(\policytheta)
		\; = \; \GammaMt_1 + \GammaMt_2 + \GammaMt_3 \, ,
	\end{align*}
	where the terms are defined as follows:
	\begin{align*}
		\GammaMt_1
		& \; \defn \; \frac{1}{\parabeta} \, \Exp_{\prompt \sim \promptdistr}
		\bigg[ \int_{\ResponseSp} \big\{ \rewardstar(\prompt, \response) - \rewardtheta(\prompt, \response) \big\} \\
		& \qquad \qquad \qquad \qquad \quad
        \cdot \Big\{ \gradtheta \rewardtheta(\prompt, \response) - \Exp_{\responsenew \sim \policytheta(\cdot \mid \prompt)}\big[ \gradtheta \rewardtheta(\prompt, \responsenew) \big] \Big\} \, \gradtheta \policytheta(\diff \response \mid \prompt)^{\top} \bigg] \, ,  \\
		\GammaMt_2
		& \; \defn \; - \frac{1}{\parabeta} \, \Exp_{\prompt \sim \promptdistr, \,  \response \sim \policytheta(\cdot \mid \prompt)}
		\bigg[ \Big\{ \gradtheta \rewardtheta(\prompt, \response) - \Exp_{\responsenew \sim \policytheta(\cdot \mid \prompt)}\big[ \gradtheta \rewardtheta(\prompt, \responsenew) \big] \Big\} \, \gradtheta \rewardtheta(\prompt, \response)^{\top} \bigg] \, ,  \\
		\GammaMt_3
		& \defn \frac{1}{\parabeta} \, \Exp_{\prompt \sim \promptdistr, \,  \response \sim \policytheta(\cdot \mid \prompt)}
		\bigg[ \big\{ \rewardstar(\prompt, \response) - \rewardtheta(\prompt, \response) \big\} \Big\{ \hesstheta \rewardtheta(\prompt, \response) - \gradtheta \Exp_{\responsenew \sim \policytheta(\cdot \mid \prompt)}\big[ \gradtheta \rewardtheta(\prompt, \responsenew) \big] \Big\} \bigg] \, .
	\end{align*}
	
	At the point $\paratheta = \parathetastar$, we know that $\rewardtheta = \rewardstar$. This simplifies the expression significantly:
	\begin{align*}
	\GammaMt_1 = \veczero \quad \text{and} \quad \GammaMt_3 = \veczero.
	\end{align*}
	Therefore, only term $\GammaMt_2$ contributes to the Hessian, and it further reduces to
	\begin{align*}
		\GammaMt_2
		& \; = \; - \frac{1}{\parabeta} \, \Exp_{\prompt \sim \promptdistr, \,  \response \sim \policytheta(\cdot \mid \prompt)}
		\Big[ \gradtheta \rewardtheta(\prompt, \response) \, \gradtheta \rewardtheta(\prompt, \response)^{\top} \Big]  \\
		& \quad + \frac{1}{\parabeta} \, \Exp_{\prompt \sim \promptdistr}
		\Big[ \Exp_{\responsenew \sim \policytheta(\cdot \mid \prompt)}\big[ \gradtheta \rewardtheta(\prompt, \responsenew) \big] \, \Exp_{\response \sim \policytheta(\cdot \mid \prompt)}\big[\gradtheta \rewardtheta(\prompt, \response)\big]^{\top} \Big] \\
		& \; = \; - \frac{1}{\parabeta} \, \Exp_{\prompt \sim \promptdistr}
		\Big[ \Cov_{\response \sim \policytheta(\cdot \mid \prompt)} \big[ \gradtheta \rewardtheta(\prompt, \response) \bigm| \prompt \big] \Big]  \, .
	\end{align*}
	From this simplification, we deduce
	\begin{align*}
		\hesstheta \scalarvalue(\policystar) \; = \;
		- \frac{1}{\parabeta} \, \Exp_{\prompt \sim \promptdistr} \Big[ \Cov_{\response \sim \policystar(\cdot \mid \prompt)} \big[ \gradtheta \rewardstar(\prompt, \response) \bigm| \prompt \big] \Big] \, ,
	\end{align*}
	which establishes equation~\eqref{eq:hessscalarvalue} as stated in \Cref{lemma:hess_scalarvalue}.


%%%%%%%%%%%%%%%%%%%%%%%%%%%%%%%%%%%%%%%%%%%%%%%%%%%%%%%%%%%%%%%%%%%%%%%
	
	\subsubsection{Proof of the Asymptotic Distribution in Equation~\eqref{eq:gap_distr}}
	\label{sec:proof:gap_distr}
	
	The goal of this part is to establish the asymptotic distribution of $\numobs \{ \scalarvalue(\policystar) - \scalarvalue(\policyhat) \}$, as stated in equation~\eqref{eq:gap_distr} from \Cref{sec:proof:lemma:hess_scalarvalue}. To achieve this, we first recast the value gap into the product of two terms and then invoke Slutsky’s theorem.
	
	We start by writing
	\begin{align}
		\numobs \cdot \{ \scalarvalue(\policystar) - \scalarvalue(\policyhat) \}
		\; = \;  \underbrace{\numobs \cdot (\parathetahat - \parathetastar)^{\top} \HessMt \, (\parathetahat - \parathetastar)}_{\Un}
		\cdot \underbrace{\frac{\scalarvalue(\policystar) - \scalarvalue(\policyhat)}{(\parathetahat - \parathetastar)^{\top} \HessMt \, (\parathetahat - \parathetastar)}}_{\Vn} \, .
	\end{align}
	By isolating \(\Un\) and \(\Vn\) in this way, we can handle their limiting behaviors separately:
	\begin{subequations}
	\begin{align}
		& \Un \; \converged \; \vecz^{\top} \CovOmega^{\frac{1}{2}} \HessMt \CovOmega^{\frac{1}{2}} \vecz \qquad \mbox{with $\vecz \sim \Gauss(\veczero, \IdMt)$},  \label{eq:Un_distr} \\
		& \Vn \; \convergep \; \frac{1}{2} \, .  \label{eq:Vn_distr}
	\end{align}
	\end{subequations}
	If these two results are established, the desired asymptotic distribution of the value gap, as given in equation~\eqref{eq:gap_distr}, follows directly from Slutsky’s theorem.
	
	To complete the proof, we proceed to verify equations~\eqref{eq:Un_distr} and~\eqref{eq:Vn_distr}. It is worth noting that equation~\eqref{eq:Un_distr} is a straightforward corollary of \Cref{thm:asymp}, so the main task is to establish the convergence result in equation~\eqref{eq:Vn_distr}.
	
	
	\paragraph{Proof of Equation~\eqref{eq:Vn_distr}:}
	
	Since $\CovOpstar$ is nonsingular, the matrix $\HessMt = (\Partitionthetabar / \parabeta) \cdot \CovOpstar$ is also nonsingular.
	From equation~\eqref{eq:Taylor_scalarvalue}, we know that for any $\varepsilon \in (0, 1)$, there exists a threshold $\eta(\varepsilon) > 0$ such that whenever $\norm{\paratheta - \parathetastar}_2 \leq \eta(\varepsilon)$, the following inequality holds:
	\begin{align*}
		\Big( \frac{1}{2} - \varepsilon \Big) \, (\paratheta - \parathetastar)^{\top} \HessMt \, (\paratheta - \parathetastar)
		\; \leq \; \scalarvalue(\policystar) - \scalarvalue(\policytheta)
		\; \leq \; \Big( \frac{1}{2} + \varepsilon \Big) \, (\paratheta - \parathetastar)^{\top} \HessMt \, (\paratheta - \parathetastar) \, .
	\end{align*}
	This can be reformulated as
	\begin{align*}
		\abs[\Big]{\Vn - \frac{1}{2}} \; \leq \; \varepsilon \, .
	\end{align*}
	Next, under the condition that $\parathetahat \convergep \parathetastar$, for any $\delta > 0$, there exists an integer $N(\varepsilon, \delta) \in \Intpos$ such that for any $\numobs \geq N(\varepsilon, \delta)$,
	\begin{align*}
		\Prob \big\{ \norm{\parathetahat - \parathetastar}_2 > \eta(\varepsilon) \big\} \leq \delta \, .
	\end{align*} 
	Therefore, for any $\numobs \geq N(\varepsilon, \delta)$, we can conclude
	\begin{align*}
		\Prob \bigg\{ \abs[\Big]{\Vn - \frac{1}{2}} \; > \; \varepsilon \bigg\} \; \leq \; \delta \, .
	\end{align*}
	In simpler terms, $\Vn \convergep \frac{1}{2}$, which establishes equation~\eqref{eq:Vn_distr}.
	


%%%%%%%%%%%%%%%%%%%%%%%%%%%%%%%%%%%%%%%%%%%%%%%%%%%%%%%%%%%%%%%%%%%%%%%

	\subsubsection{Proof of the Tail Bound in Equation~\eqref{eq:gap_bd}}
	\label{sec:proof:chisqtail}
	
	We now establish the tail bound
	\begin{align}
		\label{eq:chi_tail}
		\Prob\big\{ \chisquare_\Dim > (1 + \varepsilon) \, \Dim \big\}
		\;\leq\;
		\exp\Big\{-\frac{\Dim}{2} \bigl(\varepsilon - \log(1 + \varepsilon)\bigr)\Big\},
	\end{align}
	as stated in equation~\eqref{eq:gap_bd}.
	
	We first note that the moment-generating function (MGF) of distribution $\chisquare_\Dim$ is given by
	\begin{align*}
		\MMt(t) = (1 - 2t)^{-\frac{\Dim}{2}}, \quad \mbox{for any $t < \frac{1}{2}$}.
	\end{align*}
	Using Markov’s inequality, for any $t > 0$, we have
	\begin{align}
		\label{eq:chi_MMt}
		\Prob\big\{\chisquare_{\Dim} > (1 + \varepsilon) \, \Dim\big\}
		\;\leq\; \exp\{-t(1 + \varepsilon)\Dim\} \cdot \MMt(t)
		\; = \; \exp\{-t(1 + \varepsilon)\Dim\} \cdot (1 - 2t)^{-\frac{\Dim}{2}}
	\end{align}
    for any $t < \frac{1}{2}$.
	We optimize the bound by choosing $t$ to minimize the exponent $-t(1 + \varepsilon)\Dim - \frac{\Dim}{2}\log(1 - 2t)$.
	Solving for the optimal $t$, we obtain
	\begin{align*}
		t \; = \; \frac{\varepsilon}{2(1 + \varepsilon)} \, .
	\end{align*}
	Substituting $t$ back into inequality~\eqref{eq:chi_MMt}, the bound simplifies to the desired inequality~\eqref{eq:chi_tail}.
	

	
	
%%%%%%%%%%%%%%%%%%%%%%%%%%%%%%%%%%%%%%%%%%%%%%%%%%%%%%%%%%%%%%

	\section{Supporting Theorem: \\ Master Theorem for $Z$-Estimators}
	\label{sec:master}
	
	In this section, we provide a brief introduction to the master theorem for $Z$-estimators for the convenience of the readers.
	
	Let the parameter space be $\Theta$, and consider a data-dependent function $\Psi_n: \Theta \to \mathds{L}$, where $\mathds{L}$ is a metric space with norm~$\|\cdot\|_{\mathds{L}}$. Assume that the parameter estimate $\widehat{\theta}_n \in \Theta$ satisfies $\|\Psi_n(\widehat{\theta}_n)\|_{\mathds{L}} \convergep 0$, making $\widehat{\theta}_n$ a $Z$-estimator. The function~$\Psi_n$ is an estimator of a fixed function $\Psi: \Theta \to \mathds{L}$, where $\Psi(\theta_0) = 0$ for some parameter of interest $\theta_0 \in \Theta$.
	
	\begin{theorem}[Theorem~2.11 in \citet{kosorok2008introduction}, master theorem for $Z$-estimators]
		\label{thm:master}
		Suppose the following conditions hold:
		\begin{enumerate}
			\item $\Psi(\theta_0) = 0$, where $\theta_0$ lies in the interior of $\Theta$.
			\item $\sqrt{n} \, \Psi_n(\widehat{\theta}_n) \convergep 0$ and $\|\widehat{\theta}_n - \theta_0\| \convergep 0$ for the sequence of estimators $\{\widehat{\theta}_n\} \subset \Theta$.
			\item $\sqrt{n} (\Psi_n - \Psi)(\theta_0) \converged Z$, where $Z$ is a tight\footnote{A random variable $Z$ is tight if, for any $\epsilon > 0$, there exists a compact set $K \subset \Real$ such that $\Prob(Z \notin K) < \epsilon$.} random variable.
			\item The following smoothness condition is satisfied:
			\begin{align}
				\label{eq:master_cond}
				\frac{\big\| \sqrt{n} \big(\Psi_n(\widehat{\theta}_n) - \Psi(\widehat{\theta}_n)\big) - \sqrt{n} \big(\Psi_n(\theta_0) - \Psi(\theta_0)\big) \big\|_{\mathds{L}}}{1 + \sqrt{n} \, \| \widehat{\theta}_n - \theta_0 \|} \; \convergep \; 0 \, .
			\end{align}
		\end{enumerate}
		
		Additionally, assume that $\theta \mapsto \Psi(\theta)$ is Fréchet differentiable\footnote{Fréchet differentiability: A map $\phi: \mathds{D} \to \mathds{L}$ is Fréchet differentiable at $\theta$ if there exists a continuous, linear map $\phi_{\theta}': \mathds{D} \to \mathds{L}$ such that
		${\| \phi(\theta + h_n) - \phi(\theta) - \phi_{\theta}'(h_n) \|_{\mathds{L}}}/{\|h_n\|} \rightarrow 0$
		for all sequences $\{h_n\} \subset \mathds{D}$ with $\|h_n\| \to 0$ and $\theta + h_n \in \Theta$ for all $n \geq 1$.} at $\theta_0$
		with derivative $\dot{\Psi}_{\theta_0}$, and that $\dot{\Psi}_{\theta_0}$ is continuously invertible\footnote{Continuous invertibility: A map $A: \Theta \to \mathds{L}$ is continuously invertible if $A$ is invertible, and there exists a constant $c > 0$ such that $\|A(\theta_1) - A(\theta_2)\|_{\mathds{L}} \geq c \|\theta_1 - \theta_2\|$ for all $\theta_1, \theta_2 \in \Theta$.}.
		Then
		\begin{align*}
			\big\| \sqrt{n} \dot{\Psi}_{\theta_0}(\widehat{\theta}_n - \theta_0) + \sqrt{n} (\Psi_n - \Psi)(\theta_0) \big\|_{\mathds{L}} \convergep 0
		\end{align*}
		and therefore
		\begin{align*}
			\sqrt{n} \, \big(\widehat{\theta}_n - \theta_0\big) \; \converged \; - \dot{\Psi}_{\theta_0}^{-1} \, Z \, .
		\end{align*}
	\end{theorem}
	
	


	


%%%%%%%%%%%%%%%%%%%%%%%%%%%%%%%%%%%%%%%%%%%%%%%%%%%%%%%%%%%%%%
	
%	\tableofcontents


\section{Experimental Details}\label{app:experiment}

We implement our code based on the open-sourced OpenRLHF framework \citet{hu2024openrlhf}. We will open-source our code in the camera-ready version.

We use both the helpful and the harmless (HH) sets from HH-RLHF \citep{bai2022training} without additional data selection. We adopt the chat template from the Skywork-Reward-8B model \citep{liu2024skywork} to align with the reward template. This reward model, fine-tuned from Llama-3.1-8B, is used to simulate human preference labeling and matches our network trained for alignment.

For SFT, we apply full-parameter tuning with Adam for one epoch, using a cosine learning rate schedule, a 3\% warmup phase, a learning rate of $5\times 10^{-7}$, and a batch size of 256. These hyperparameters are adopted from \citet{hu2024openrlhf}. 

For all the DPO training in both iterative and online settings, we use full-parameter tuning with Adam but with two epochs. The learning rate, warmup schedules, and batch size are all the same. 

During generation, we limit the maximum number of new tokens to 896 and employ top$\_$p decoding with $p=0.95$ for all experiments. For Online DPO, we use a sampling temperature of 1.0, following \citet{guo2024direct}, while in Iterative DPO, we set the temperature to 0.7 to account for the off-policy nature of the data, following \citet{dong2024rlhf, shi2024crucial}.

Prompts are truncated to a maximum length of 512 tokens (truncated from the left if the length exceeds this limit) for SFT, DPO, and generation tasks. For SFT data, the maximum length is further restricted to 1024 tokens. When the combined length of the response and the (truncated) prompt exceeds 1024 tokens, the response is truncated from the right. These truncation practices align with the standard methodology described by \citet{rafailov2023direct}. In contrast, for DPO, responses are not further truncated, as we are already limiting the maximum tokens generated during the generation process.

When reproducing the \textit{Hybrid Sampling} baseline (Exploration Preference Optimization, XPO) from \citet{xie2024exploratory}, we use $\alpha=5\times 10^{-6}$ as suggested in the paper.

We do not include a comparison with \citet{shi2024crucial} and \citet{liu2024sample} in our experiments. While \citet{shi2024crucial} employs a sampling method similar to ours, their approach requires significantly more hyperparameters to tune, whereas our method involves no hyperparameter tuning. On the other hand, \citet{liu2024sample} relies on training an ensemble of 20 reward models to approximate the posterior. Their sampling method requires solving the argmax of these rewards, which is computationally intractable. As a workaround, they generate 20 samples and select the best one using best-of-N with $N=20$. This approach demands at least six times the computational resources compared to our method.

% The \textit{Best-of-N} method implicitly incorporates an exploration mechanism by selecting the best and worst samples based on internal rewards, which is conceptually similar to the exploration design in PILAF.

\subsection{Additional Results}

We present the full results for Online DPO with the overfitted initial policy, including a scatter plot in \cref{fig:online_dpo_special_full} and a summary of the objective values in \cref{tab:online_DPO_special}.

We observe that \textit{Vanilla Sampling} rapidly increases its KL divergence from the reference model while its reward improvement diminishes over time. In contrast, PILAF undergoes an early phase of training with fluctuating KL values but ultimately achieves a policy with higher reward and substantially lower KL divergence. We hypothesize that PILAF’s interpolation-based exploration enables it to escape the suboptimal region of the loss landscape where \textit{Vanilla Sampling} remains trapped. 

Conversely, \textit{Hybrid Sampling}, despite its explicit exploration design, remains biased by the policy model and continues to exhibit high KL values. While KL divergence decreases over training, the reward improvement remains limited. Meanwhile, \textit{Best-of-N Sampling} introduces an implicit exploration mechanism through internal DPO, which selects the best and worst responses, leading to wider coverage than \textit{Vanilla Sampling}. However, despite achieving a KL divergence similar to PILAF, it results in a lower reward. These findings highlight the superiority of PILAF sampling, demonstrating its effectiveness in robustly optimizing an overfitted policy.




\begin{figure}[htb]
  \centering
  \begin{minipage}[t]{0.49\textwidth} % [t] 表示顶部对齐
    \vspace{0pt} % 关键调整:将基线固定在顶部
    \centering
    \includegraphics[width=\linewidth]{figs/online_special_full.pdf}
    \caption{\textbf{Online DPO with an overfitted initial policy}. Full results of the \cref{fig:online_dpo_special}. Each dot represents an evaluation performed every 50 training steps. Color saturation indicates the training step, with darker colors representing later steps.}
    \label{fig:online_dpo_special_full}
  \end{minipage}
  \hfill
  \begin{minipage}[t]{0.49\textwidth} % [t] 表示顶部对齐
    \vspace{10pt} % 关键调整:将基线固定在顶部
    \centering
    \captionsetup{type=table}
    \caption{\textbf{Results of Online DPO with an overfitted initial policy.} We report the average reward, KL divergence from the reference model, and objective $\scalarvalue$ on the testset.}
    \vspace*{1.5em} % 调整表格与标题间距
    \begin{footnotesize}
    \begin{sc}
    \begin{tabular}{l|ccc}
    \toprule
        \textbf{Method} & Reward ($\uparrow$) & KL ($\downarrow$) & $\scalarvalue$ ($\uparrow$)\\ 
        \midrule
        \textit{Vanilla} & \underline{-3.95} & 39.85 & -7.93 \\
        \textit{Best-of-N} & -4.49 & {27.90}  & \underline{-7.28}\\
        \textit{Hybrid} & -6.00 & \textbf{18.20} & -7.82 \\
        \midrule
        \textit{PILAF} & \textbf{-3.54} & \underline{26.45} & \textbf{-6.19} \\
    \bottomrule
    \end{tabular}
    \end{sc}
    \end{footnotesize}
    \label{tab:online_DPO_special}
  \end{minipage}
\end{figure}

\section{Extension to Proximal Policy Optimization (PPO)}
\label{app:extension}

In this section, we briefly explore how the core principles of our PILAF sampling approach can be extended to PPO-based RLHF methods.

\paragraph{Integrating Response Sampling in InstructGPT:}

The PPO-based RLHF pipeline used in InstructGPT \citep{ouyang2022training} consists of three key steps: \vspace{-.8em}
\begin{enumerate} \itemsep = -.3em
    \item[(i)] Supervised Fine-Tuning (SFT) that produces the reference model $\policyref$.
    \item[(ii)] Reward Modeling (RM) by solving the optimization problem~\eqref{eq:RM_objective}, yielding an estimated reward function $\rewardtheta$.
    \item[(iii)] Reinforcement Learning Fine-Tuning, where the policy $\policyphi$ is optimized against the reward model $\rewardtheta$ using the Proximal Policy Optimization (PPO) algorithm, following the optimization scheme~\eqref{eq:policy_loss_with_rm}.
\end{enumerate}
\vspace{-.8em}
The key distinction between the PPO and DPO approaches lies in how the reward model $ \rewardtheta $ is represented—explicitly in PPO and implicitly in DPO.
In response sampling for data collection, it is crucial to consider the iterative nature of the InstructGPT pipeline. During each iteration, additional human-labeled data is collected for reward modeling (step~(ii)), and steps (ii) and (iii) are repeatedly applied to refine the model. Our proposed PILAF algorithm naturally integrates into this pipeline by improving the data collection process in step~(ii), thereby enhancing reward model training and, in turn, policy optimization.

\paragraph{Extensions of T-PILAF and PILAF:}
Extending our response sampling methods, PILAF and T-PILAF, to the PPO setup with an explicit $ \rewardtheta $ is both natural and straightforward.
\begin{itemize}
    \item Within the theoretical framework of T-PILAF, as introduced in \Cref{sec:sampling}, the only required modification is replacing $\policytheta$ with the language model $\policyphi$ and redefining the interpolated and extrapolated policies, $\policyphipos$ and $\policyphineg$, following the same formulation as in equations~\eqref{eq:def_policythetapos}~and~\eqref{eq:def_policythetaneg}.
    Specifically, we define
    \begin{subequations}
	\begin{align}
		\label{eq:def_policythetapos_PPO}
		\policyphipos(\response \mid \prompt)
		& := \frac{1}{\Partition^+(\prompt)} \; \policyphi(\response \mid \prompt)
		\exp \big\{ \rewardtheta(\prompt, \response) \big\} \, ,  \\[-1pt]
		\label{eq:def_policythetaneg_PPO}
		\policyphineg(\response \mid \prompt)
		& := \frac{1}{\Partition^-(\prompt)} \, \policyphi(\response \mid \prompt) \,
		\exp \big\{ - \rewardtheta(\prompt, \response) \big\},
	\end{align}
    \end{subequations}
    where $\rewardtheta$ is now explicitly produced by a reward network, rather than being implicitly derived from $\policyphi$, as in equation~\eqref{eq:def_reward}.
    \item To extend our empirical PILAF algorithm, as described in \Cref{sec:sampling_exp}, we propose applying the same interpolation and extrapolation techniques directly to the logits of the language models $\policyphi$ and $\policyref$.
    In particular, we take
    \begin{align*}
        & \policyphipos(\cdot \mid \prompt, \tokenttot{1}{t-1}) \; = \; \softmax\Big( \big\{ (1 + \parabeta) \, \headphi - \parabeta \, \headref\big\} (\prompt, \tokenttot{1}{t-1}) \Big), \\
        & \policyphineg(\cdot \mid \prompt, \tokenttot{1}{t-1}) \; = \; \softmax\Big( \big\{ (1 - \parabeta) \, \headphi + \parabeta \, \headref\big\} (\prompt, \tokenttot{1}{t-1}) \Big),
    \end{align*}
    where $\headphi$ and $\headref$ represent the logits of the language models $\policyphi$ and $\policyref$, respectively.
\end{itemize}

\paragraph{Adaption of Theoretical Analysis:}
Our theoretical analyses can be extended to the PPO framework, assuming that the optimization process~\eqref{eq:policy_loss_with_rm} in step~(iii) of InstructGPT is solved exactly. In this case, the policy satisfies~\mbox{$\policyphi = \policyt{\rewardtheta}$}, where
\begin{align*}
	\policyt{\rewardtheta}(\response \mid \prompt)
	\; \defn \; \frac{1}{\Partitiontheta(\prompt)} \, \policyref(\response \mid \prompt) \exp \Big\{ \frac{1}{\parabeta} \, \rewardtheta(\prompt, \response) \Big\} \, .
\end{align*}
Under this assumption, the output language model $\policyphi$ is implicitly a function of the parameter $\paratheta$.
Building on this, we can adapt our optimization and statistical analyses as follows:

\begin{itemize}
    \item {\bf Optimization Consideration:}
    Using the same argument as in \Cref{thm:grad}, we can prove that
    \begin{align*}
        \gradtheta \Loss(\paratheta) \; = \;
    - \, \Const' \cdot \gradtheta \scalarvalue(\policyphi) \, + \, \Term_2 \, ,
    \end{align*}
    where $\Const' > 0$ is a universal constant, and $\Term_2$ represents a second-order approximation error.
    
    In other words, if the policy optimization step is sufficiently accurate for the reward model $\rewardtheta$, then performing gradient descent on the MLE loss with respect to $\paratheta$ is equivalent to applying gradient ascent on the oracle objective $\scalarvalue$, following the steepest direction in the parameter space of $\paratheta$.
    \item {\bf Statistical Consideration:}
    Even with the new parameterization, the asymptotic distribution of $\parathetahat$ from \Cref{thm:asymp} remains unchanged. Moreover, the gradient and Hessian of $\scalarvalue$ with respect to $\paratheta$ retain the same form as in \Cref{thm:grad}. As a result, the statistical analysis extends naturally to PPO, allowing us to conclude that PILAF also maintains structure-invariant statistical efficiency for PPO methods.
\end{itemize}




\end{document}
