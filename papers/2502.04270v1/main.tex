\documentclass[12pt,twoside]{article}

% \usepackage{amsmath}
% \usepackage{amsthm}
\usepackage{amssymb}
% \usepackage{amsfonts}
% \usepackage{algorithm}
% \usepackage{algpseudocode}
\usepackage[top = 1 in, bottom = 1.2 in, left = 1 in, right = 1 in]{geometry}
% %\usepackage{fullpage}
% \usepackage{color}
% \usepackage{mathrsfs}
% \usepackage{upgreek}
% \usepackage[normalem]{ulem}
% \usepackage{longtable}
% \usepackage{hyperref}
% \usepackage{cleveref}
\usepackage{natbib}
% \usepackage{dsfont}
% \usepackage{graphicx}
% \graphicspath{{figs/}}
% \usepackage{caption}
% % \usepackage{subcaption}
% \usepackage{float}
% \usepackage{mathtools}
% \usepackage{etoolbox}
% \usepackage{thmtools}
% \usepackage{enumitem}
\usepackage{parskip}
\usepackage[textsize=tiny]{todonotes}
\setlength{\marginparwidth}{2cm}
\setlength{\parskip}{1em}



\usepackage{multirow}

\usepackage{esvect}

\usepackage[utf8]{inputenc}
\usepackage{comment}


\usepackage{hyperref}
\usepackage{url}
\usepackage{wrapfig}
\usepackage{lipsum}

\newcommand{\hiddenfootnote}[1]{%
  \begingroup
  \renewcommand{\thefootnote}{} % 临时隐藏编号
  \footnotemark\addtocounter{footnote}{-1}% 标记但不增加计数器
  \footnotetext{#1}%
  \endgroup
}



%%%%%%%% ICML 2024 EXAMPLE LATEX SUBMISSION FILE %%%%%%%%%%%%%%%%%

% \documentclass{article}

% Recommended, but optional, packages for figures and better typesetting:
\usepackage{microtype}
\usepackage{graphicx}
\usepackage{subfigure}
\usepackage{booktabs} % for professional tables

\usepackage{caption}

% hyperref makes hyperlinks in the resulting PDF.
% If your build breaks (sometimes temporarily if a hyperlink spans a page)
% please comment out the following usepackage line and replace
% \usepackage{icml2024} with \usepackage[nohyperref]{icml2024} above.
\usepackage{hyperref}

\usepackage{algorithm}
\usepackage{algorithmic}
% Attempt to make hyperref and algorithmic work together better:
% \newcommand{\theHalgorithm}{\arabic{algorithm}}

% Use the following line for the initial blind version submitted for review:
% \usepackage{icml2025}

% If accepted, instead use the following line for the camera-ready submission:
% \usepackage[accepted]{icml2025}

% For theorems and such
\usepackage{amsmath}
\usepackage{amssymb}
\usepackage{mathtools}
\usepackage{amsthm}

% for coloring
\usepackage{xcolor}
% \usepackage{tikz}
\usepackage{soul}
\sethlcolor{red!20}

% \usepackage[ruled,vlined]{algorithm2e}

% if you use cleveref..
\usepackage[capitalize,noabbrev]{cleveref}

%%%%%%%%%%%%%%%%%%%%%%%%%%%%%%%%
% THEOREMS
%%%%%%%%%%%%%%%%%%%%%%%%%%%%%%%%
\theoremstyle{plain}
\newtheorem{theorem}{Theorem}[section]
\newtheorem{proposition}[theorem]{Proposition}
\newtheorem{lemma}[theorem]{Lemma}
\newtheorem{corollary}[theorem]{Corollary}
\theoremstyle{definition}
\newtheorem{definition}[theorem]{Definition}
\newtheorem{assumption}[theorem]{Assumption}
\theoremstyle{remark}
\newtheorem{remark}[theorem]{Remark}


%%%COMMENTS
\newcommand{\xxcomment}[4]{\textcolor{#1}{[$^{\textsc{#2}}_{\textsc{#3}}$ #4]}}

\newcommand{\julia}[1]{\xxcomment{blue}{J}{K}{#1}}
\newcommand{\yunzhen}[1]{\xxcomment{violet}{Y}{F}{#1}}
\newcommand{\yaqicolor}{cyan}
\newcommand{\yaqi}[1]{\xxcomment{\yaqicolor}{Y}{D}{#1}}
\newcommand{\kunhao}[1]{\xxcomment{brown}{K}{Z}{#1}}
\newcommand{\ariel}[1]{\xxcomment{green}{A}{K}{#1}}

%%DISABLED
%%\newcommand{\xxcomment}[4]{}
%\newcommand{\julia}[1]{#1}
%\newcommand{\yaqi}[1]{}
%\newcommand{\kunhao}[1]{}
%\newcommand{\ariel}[1]{}
%\newcommand{\yunzhen}[1]{}



% Todonotes is useful during development; simply uncomment the next line
%    and comment out the line below the next line to turn off comments
%\usepackage[disable,textsize=tiny]{todonotes}
\usepackage[textsize=tiny]{todonotes}

\newcommand{\thought}[1]{{\color[rgb]{0.2,0.39,0.66}(#1)}}
\newcommand{\todo}[1]{{\color[rgb]{1.0,0.0,0.0}(#1)}}
\newcommand{\hsh}[1]{{\color{green!50!black} Henrik: #1}}
\newcommand{\st}[1]{{\color{red!50!black} Sebastian: #1}}

\newcommand{\ulm}[1]{_{\scaleto{\mathrm{#1}}{3pt}}}
\newcommand\at[2]{\left.#1\right|_{#2}}











\newtheorem{assumption}{Assumption}

\DeclareMathOperator*{\argmax}{arg\,max}
\DeclareMathOperator*{\argmin}{arg\,min}

\newcommand{\swname}[1]{\texttt{#1}}
\newcommand{\ie}{i\/.\/e\/.,\/~}
\newcommand{\eg}{e\/.\/g\/.,\/~}
\newcommand{\cf}{cf\/.\/~}

\newcommand{\fig}{Fig\/.\/~}
\newcommand{\defn}{Def\/.\/~}
\newcommand{\sect}{Sec\/.\/~}
\newcommand{\tabl}{Tab\/.\/~}
\newcommand{\algo}{Algorithm~}
\newcommand{\theo}{Theorem~}

\newcommand{\bnnl}{3 hidden layers}
\newcommand{\bnnn}{50 neurons}
\newcommand{\bnna}{tanh activations}

\newcommand{\capt}[1]{\mdseries{\emph{#1}}}

\newcommand{\videolink}{at \url{https://youtu.be/_d7AqTRjz6g}}
\newcommand{\codelink}{\url{https://github.com/wheelbot/mini-wheelbot}}

\newcommand{\fakepar}[1]{\vspace{0mm}\noindent\textbf{#1.}}

\newcommand{\needref}{\textcolor{red}{[REF]}}

\newcommand{\plotfontsize}{9pt}


\makeatletter
\long\def\@makecaption#1#2{
	\vskip 0.8ex
	\setbox\@tempboxa\hbox{\small {\bf #1:} #2}
	\parindent 1.5em  %% How can we use the global value of this???
	\dimen0=\hsize
	\advance\dimen0 by -3em
	\ifdim \wd\@tempboxa >\dimen0
	\hbox to \hsize{
		\parindent 0em
		\hfil 
		\parbox{\dimen0}{\def\baselinestretch{0.96}\small
			{\bf #1.} {#2}
			%%\unhbox\@tempboxa
		} 
		\hfil}
	\else \hbox to \hsize{\hfil \box\@tempboxa \hfil}
	\fi
}
\makeatother

\newcommand\blfootnote[1]{%
	\begingroup
	\renewcommand\thefootnote{}\footnotetext{#1}%
%	\addtocounter{footnote}{-1}%
	\endgroup
}

% The \icmltitle you define below is probably too long as a header.
% Therefore, a short form for the running title is supplied here:
% \icmltitlerunning{Optimal Sampling for Reward Modeling}



% \begin{document}



% \twocolumn[
% \title{PILAF: Optimal Human Preference Sampling for Reward Modeling}



% It is OKAY to include author information, even for blind
% submissions: the style file will automatically remove it for you
% unless you've provided the [accepted] option to the icml2024
% package.

% List of affiliations: The first argument should be a (short)
% identifier you will use later to specify author affiliations
% Academic affiliations should list Department, University, City, Region, Country
% Industry affiliations should list nyuany, City, Region, Country

% You can specify symbols, otherwise they are numbered in order.
% Ideally, you should not use this facility. Affiliations will be numbered
% in order of appearance and this is the preferred way.
% \icmlsetsymbol{equal}{*}

% \begin{icmlauthorlist}
% \icmlauthor{xxx}{equal}
% \icmlauthor{Firstname2 Lastname2}{nyu}
% \icmlauthor{Firstname3 Lastname3}{nyu}
% \icmlauthor{Firstname4 Lastname4}{nyu}
% \icmlauthor{Firstname5 Lastname5}{pku}
% \end{icmlauthorlist}

% \icmlaffiliation{nyu}{}
% \icmlaffiliation{nyu}{}

% \icmlcorrespondingauthor{Firstname1 Lastname1}{first1.last1@xxx.edu}
% \icmlcorrespondingauthor{Firstname2 Lastname2}{first2.last2@www.uk}

% You may provide any keywords that you
% find helpful for describing your paper; these are used to populate
% the "keywords" metadata in the PDF but will not be shown in the document
% \icmlkeywords{Machine Learning, ICML}

% \vskip 0.3in
% ]

% this must go after the closing bracket ] following \twocolumn[ ...

% This command actually creates the footnote in the first column
% listing the affiliations and the copyright notice.
% The command takes one argument, which is text to display at the start of the footnote.
% The \icmlEqualContribution command is standard text for equal contribution.
% Remove it (just {}) if you do not need this facility.

% \printAffiliationsAndNotice{}  % leave blank if no need to mention equal contribution
% \printAffiliationsAndNotice{\icmlEqualContribution} % otherwise use the standard text.

% \begingroup
% \setlength{\abovedisplayskip}{4pt}  % Space above equations
% \setlength{\belowdisplayskip}{4pt}  % Space below equations
% \setlength{\abovedisplayshortskip}{2pt}
% \setlength{\belowdisplayshortskip}{2pt}

\begin{document}
	\begin{center}
		{\bf \Large PILAF: Optimal Human Preference Sampling for Reward Modeling} \\
		
		\vspace{2em}
		{%\large
			{
				\begin{tabular}{ccccc}
					Yunzhen Feng$^\dagger$ & Ariel Kwiatkowski$^*$ & Kunhao Zheng$^*$ & Julia Kempe$^\diamond$ & Yaqi Duan%$^\dagger$
                    $^\diamond$\\
                    NYU & Meta FAIR & Meta FAIR & Meta FAIR \& NYU & NYU
				\end{tabular}
		}}
%		\vspace{1em}
		
		% \medskip
		
		% \medskip 

  %       \medskip 
        
		% {\small \begin{tabular}{l}
		% 	$^\S$ Meta FAIR \\
  %               $^\dagger$ New York University\\
  %               $^*$ Joint second authors\\
  %               $^\diamond$ Equal advising
		% \end{tabular}}
		
		
		\vspace{1.6em}
		\today
	\end{center}

\begin{center}
		{\bf Abstract} \\ \vspace{.6em}
		\begin{minipage}{0.9\linewidth}
			{\small ~~~~\begin{abstract}

In this work, we tackle the challenge of disambiguating queries in retrieval-augmented generation (RAG) to diverse yet answerable interpretations.
State-of-the-arts follow a Diversify-then-Verify (DtV) pipeline, where diverse interpretations are generated by an LLM,
later used as search queries to retrieve supporting passages.
Such a process
may introduce noise in either interpretations or retrieval,
particularly in enterprise settings, where LLMs---trained on static data---may struggle with domain-specific disambiguations.
Thus, a post-hoc verification phase is introduced to prune noises.
Our distinction is \textbf{to unify diversification with verification} by incorporating feedback from retriever and generator early on.
This joint approach improves both efficiency and robustness by reducing reliance on multiple retrieval and inference steps, which are susceptible to cascading errors.
We validate the efficiency and effectiveness of our method, \ourslong (\ours), on the widely adopted ASQA benchmark
to achieve diverse yet verifiable interpretations.
Empirical results show that \ours improves grounding-aware $\textrm{F}_1$ score by an average of 23\% over the strongest baseline across different backbone LLMs.
\end{abstract}
}
		\end{minipage}
	\end{center}
	
	\vspace{.6em}


\section{Introduction}
\label{sec:intro}
% Image editing methods in diffusion models depend on user-defined control directions - users can unlock their creativity using these methods by specifying the desired manipulation through prompts~\cite{gandikota2023concept}, reference images~\cite{ruiz2022dreambooth, kumari2022customdiffusion, gal2022image, chen2024trainingfreeregionalpromptingdiffusion}, or attribute vectors~\cite{parmar2023zero,hertz2022prompt}. In this work, we ask a fundamentally different question: \emph{Can we automatically discover the underlying visual structure of a concept within diffusion model's knowledge?} %Rather than requiring user-specified controls, we aim to decompose the model's internal knowledge into meaningful directions.

% This question touches on a fundamental limitation in how we interact with diffusion models. Current control methods ~\cite{zhang2023addingconditionalcontroltexttoimage, gandikota2023concept, ye2023ipadaptertextcompatibleimage,ye2023ipadaptertextcompatibleimage, hertz2024stylealignedimagegeneration, li2023photomaker, shi2024instantbooth, chen2024trainingfreeregionalpromptingdiffusion} require users to specify their desired manipulations in advance, limiting interactive creativity. This contrasts with natural human artistic workflows, where creators dynamically explore creative ideas while jointly refining them toward meaningful artistic outcomes~\cite{hoffmann2016modeling}. This synergy between specification and exploration is not new to generative models. Early GAN architectures naturally developed disentangled latent spaces that enabled continuous\cite{harkonen2020ganspace,radford2015unsupervised, wu2021stylespace, shen2020interfacegan}, compositional control over generated images. Users could explore these spaces to discover interesting variations that would be difficult to describe in words~\cite{wu2021stylespace}, then combine them to achieve their creative goals~\cite{grabe2022towards}. 


% While diffusion models have largely superseded GANs in conditional image synthesis~\cite{dhariwal2021diffusion},  their underlying structure remains less understood. Diffusion models achieve remarkable diversity through high-dimensional latents, unlike GANs' compact latent spaces.  With a single prompt, diffusion models can generate radically different variations through different random initializations of input noise. We ask - Is it possible to discover interpretable structure within this vast space of variations?

Text-to-image diffusion models are capable of generating remarkable visual variations from a single prompt through different random initializations. However, this vast creative potential remains largely opaque to users---while we can generate diverse images, we lack understanding of the underlying structure of these variations. This presents a fundamental challenge: how can we discover and expose the latent visual capabilities encoded within these models?

\let\thefootnote\relax \footnote{$^{*}$Correspondence to \texttt{gandikota.ro@northeastern.edu}}

The challenge touches on a key limitation in how we interact with diffusion models today. Current control methods require users to explicitly specify their desired edits in advance through prompts~\cite{gandikota2023concept}, reference images~\cite{zhang2023addingconditionalcontroltexttoimage, chen2024trainingfreeregionalpromptingdiffusion, ruiz2022dreambooth,kumari2022customdiffusion, Ryu_lora, hu2021lora}, or attribute vectors~\cite{ye2023ipadaptertextcompatibleimage, hertz2024stylealignedimagegeneration, li2023photomaker, shi2024instantbooth,parmar2023zero,hertz2022prompt}. That contrasts sharply with natural human creative workflows, where artists dynamically explore creative ideas and jointly refine them toward meaningful artistic outcomes~\cite{hoffmann2016modeling}. The need for pre-specified controls creates a barrier between users and the full creative potential of these models.

Interestingly, earlier generative models like GANs~\cite{gans,karras2019style,brock2018large} naturally developed more interpretable internal structures. Their compact latent spaces often exhibited emergent disentanglement~\cite{harkonen2020ganspace,radford2015unsupervised, wu2021stylespace, shen2020interfacegan}, enabling continuous and compositional control over generated images. Users could explore these spaces to discover interesting variations that would be difficult to describe in words~\cite{wu2021stylespace}, then combine them to achieve their creative goals~\cite{grabe2022towards}.

Diffusion models have largely superseded GANs in conditional image synthesis~\cite{dhariwal2021diffusion}, achieving greater diversity through much higher-dimensional latents. And yet an understanding of the underlying structure of these larger latent spaces has remained elusive. In this work, we ask a fundamental question: \emph{Can we automatically discover the visual structure within a diffusion model's knowledge of a concept?} Rather than requiring user-specified controls, we aim to decompose the model's internal representations into expressive directions that users can explore and combine.

To address these needs, we present \textbf{SliderSpace}, a framework that brings systematic explorability to diffusion models. Given just a text prompt, SliderSpace discovers a canonical set of meaningful, diverse, and controllable directions within the model's knowledge of that concept. Each direction is implemented as a low-rank adapter~\cite{hu2021lora} that can be scaled and composed with others, allowing users to explore and smoothly combine different aspects of variation, as shown in Figure~\ref{fig:intro}.

We ground SliderSpace discovery in three key requirements for meaningful decomposition of a diffusion model's visual manifold: 
\begin{enumerate}
    \item \textbf{Unsupervised Discovery:} The decomposition process should emerge from the intrinsic structure of the model's learned representation, rather than being guided by predefined attributes. This ensures we capture the true topology of the model's knowledge space rather than projecting our assumptions onto it.
    
    \item \textbf{Semantic Orthogonality:} Each discovered control must represent a distinct semantic direction. This is enforced in a semantic feature space, like CLIP, where every slider has an orthogonal effect in embeddings. This prevents discovering multiple controls that create similar semantic effects, making the system more efficient and easier.
    
    \item \textbf{Distribution Consistency:} Directions must induce consistent transformations across both random seeds and prompt variations. 
\end{enumerate}

These requirements naturally lead to our proposed framework, which we formalize in Section~\ref{sec:method}. As we show in our experiments, SliderSpace is architecture-agnostic, working with both conventional U-Net based models like Stable Diffusion~\cite{rombach2022high, rombach2022sd20, podell2023sdxl, turbo, dmd} and recent transformer-based architectures like Flux~\cite{flux}.

We demonstrate the expressiveness of SliderSpace through three applications: First, we show how SliderSpace can decompose high-level concepts into diverse and expressive components, revealing the natural axes of variation in the model's understanding. Second, we explore artistic style variation, where SliderSpace discovers directions that match or exceed the diversity of manually curated artist lists while being judged more useful by human evaluators. Finally, we show how SliderSpace can help reverse the mode collapse commonly observed in distilled diffusion models, restoring diversity while maintaining generation speed.

Beyond providing practical creative control, SliderSpace opens new avenues for understanding and utilizing the latent capabilities of diffusion models. By mapping these models' visual potential into intuitive, composable directions, we take a step toward making their creative possibilities more accessible and interpretable to users.

% Image editing methods in diffusion models unlock the creativity of users. In this work we ask an alternate question: \emph{Can we organize and expose what of the diffusion model is already capable of?}.
% Existing methods for controlling image generation typically require users to manually specify edit directions for desired changes. This process is time-consuming, requires technical expertise, and limits the spontaneity of the creative process. For instance, if a user wants to adjust the smile of a generated person, they must explicitly request this edit, often through imprecise prompt engineering or model fine-tuning. This approach of predefined controls or manual specifications restricts users from fully exploring the latent capabilities of the model. There may be interesting stylistic variations or attributes that the model can generate, but users have no easy way to discover or utilize these.

% Natural visual disentanglement was an emergent property in the latent space of Generative Adversarial Models (GANs) \cite{harkonen2020ganspace,radford2015unsupervised, wu2021stylespace, shen2020interfacegan}. In particular, it has been observed that StyleGAN~\cite{karras2019style} stylespace neurons offer detailed control over many meaningful aspects of images that would be difficult to describe in words~\cite{wu2021stylespace}. However, diffusion models do not share such a compact latent space~\cite{park2023unsupervised}; and efforts to uncover such a space in the semantic embeddings of the text conditioning have met with limited success \nik{Nick - is there a specific citation you were thinking about?}.

% In this work we introduce \textbf{SliderSpace}, which takes a step towards uncovering an analogous low dimensional representation of diffusion models' visual breadth; in essence treating the diffusion model as many generators sharing parameters, where a particular generator is defined by a specific prompt. For a given prompt we sample many random seeds (and optionally prompt expansions using an LLM), generate the corresponding images, and apply an off the shelf feature extractor (in this work CLIP, but our method can be applied to any differentiable feature extractor). We use PCA to analyze these features, and for each of the leading $k$ principal components we train a LoRA \cite{} which causes the diffusion model to produces images which increase the feature magnitude along that component when passed back through the same feature extractor. This leads to a 'Slider' for each principal component, because each LoRA can be scaled and applied to the original diffusion model, continuously varying those visual features in the generated results (as measured, in our case, by CLIP).

% There are many other works that enhance the controllability of diffusion models. One common approach is enabling users to add spatial constraints to a generation either manually, or via a reference image \cite{zhang2023addingconditionalcontroltexttoimage, chen2024trainingfreeregionalpromptingdiffusion}, a second is leveraging more abstract embeddings (e.g. identity, style) extracted from a reference image \cite{ye2023ipadaptertextcompatibleimage, hertz2024stylealignedimagegeneration, li2023photomaker, shi2024instantbooth}, a third is finetuning a foundation model to better generate a concept important to the user \cite{ruiz2022dreambooth, kumari2022customdiffusion, Ryu_lora, hu2021lora}, and a fourth (most relevant to this work) is finding low-rank adaptors of the model based on a prompt or small training set which can be scaled to provide continous control over one aspect of generated image (e.g. night vs day, basic vs luxury, etc.) \cite{gandikota2023concept}. SliderSpace is complementary to all of these methods and offers something distinct. All of the other methods we are aware require the user (and / or model designer) to know in advance what type of control they want. In contrast SliderSpace assists users in discovering and controlling hidden capabilities present in the diffusion model's distribution of possible generations.

%We propose that truly intuitive creative control in a text-to-image model should meet three key criteria: \emph{discoverability}, \emph{intuitiveness}, and \emph{specificity}. The model should reveal controllable attributes that may not be immediately obvious, offer controls that are easy to understand and manipulate, and ensure each control affects a distinct attribute of the generated image.

% We demonstrate the utility and power of SliderSpace using three applications built on top of SDXL-DMD \cite{dmd}, because its fast generation speed lends itself well to the continuous control offered by SliderSpace.

% First, we study concept decomposition (Section \ref{sec:concept_exp}), where we learn sliders for a specific concept (e.g. 'monster', 'waterfall', 'car'). Through quantitative metrics of diversity and text alignment we demonstrate that the learned sliders dramatically boost the diversity of generations when randomly applied without harming text alignment; we also ask humans to qualitatively judge these results in a user study where they find the SliderSpace results to be more 'Diverse', 'Useful', and 'Creative' than our baselines.

% Second, we attempt to compare the automatic discoveries of SliderSpace to a large scale manual study of artistic styles (Section \ref{sec:art_exp}), open-sourced by ParrotZone \cite{parrotzone}. In this study SDXL was prompted with over 4300 artist names,  and based on visual inspection the cases of successful stylistic mimicry recorded. Quantitatively SliderSpace more closely matches the distribution of artistic variation discovered by ParrotZone than other baselines, and in our user studies was judged to be significantly more 'Diverse' and 'Useful' than the baselines. To our surprise humans even judged SliderSpace results to be slightly more 'Diverse' than the results generated by the manually discovered artist names of \cite{parrotzone}.

% Third, we attempt to use SliderSpace to reverse the mode collapse commonly observed in distilled few-step diffusion models relative to the original teacher model (Section \ref{sec:diverse_exp}). We quantitatively demonstrate that applying SliderSpace to SDXL-DMD leads to more closely matching the distribution of images by the original teacher, SDXL.

%Through extensive experiments on various state-of-the-art text-to-image models, we demonstrate that SliderSpace significantly enhances user control and creative expression in AI-assisted image generation tasks. Our method enables a range of applications, including concept decomposition and control, diversity improvement in generated images, customization dissection and edits, and the exploration of artistic styles inherent in the model.

% SliderSpace goes beyond providing a practical tool for enhanced creative control. By mapping the visual potential of diffusion models it can open new avenues for generative creativity and deepens our understanding of each model's hidden potential.


	\vspace{-5pt}
	\section{Problem Setup and Motivation}\label{sec:setup}
	
	In this section, we introduce the setup for the problem studied in this work. In \Cref{sec:intro_alignment}, we present the basic framework for aligning language models with human preferences. In \Cref{sec:intro_DPO}, we provide an overview of the widely-used Direct Preference Optimization (DPO) method. Finally, in \Cref{sec:intro_goal}, we introduce the core problem investigated in this work: designing an optimal sampling scheme for response generation.
    % \yaqi{Delete this to avoid redundancy.}
	\vspace{-3pt}
	\subsection{Aligning LMs with Human Preferences}
	\label{sec:intro_alignment}
	
	% We break down the alignment process into three key components: the language model, preference data, and policy optimization.
	
	\paragraph{Language Model (LM).}
	At the core of RLHF is a language model that processes prompts~\mbox{$\prompt \in \PromptSp$} and generates responses $ \response \in \ResponseSp $. 
	Each response is represented as a sequence of tokens $\response = (\tokent{1}, \tokent{2}, \ldots, \tokent{\numtok}).$ The primary goal of RLHF is to guide the model to generate responses that align with human preferences. This translates to designing a policy $\policy$ (parameterized as a LM) that maps prompts to responses, maximizing a reward that reflects human preferences (with a KL regularization).
	
	
	\paragraph{Preference Data.} The oracle reward for human values is inherently inaccessible. Instead, the alignment process approximates the reward using a dataset of human-labeled preferences,
	\begin{align*}
		\Data = \big\{ (\prompti{i}, \responsewini{i}, \responselosei{i}) \big\}_{i=1}^{\numobs} \, ,
	\end{align*}
	where each sample contains: (i) a prompt $ \prompti{i}$, independently drawn from a distribution $ \promptdistr$, and (ii) a pair of responses $(\responsewini{i}, \responselosei{i})$, where $\responsewini{i}$ is preferred over $\responselosei{i}$ in human labeling. The response pair~\mbox{$(\responsewini{i}, \responselosei{i})$} is first generated from a joint distribution $\responsedistr(\cdot \mid \prompt)$ 
    %\yaqi{$\Leftarrow$ $\responsedistr$ first appears here. It is indeed quite far from \Cref{sec:theory}.} 
    and then presented to human labelers for preference annotation. Human preferences are commonly modeled using the \emph{Bradley–Terry (BT)} model, which assumes: \begin{align}
			\label{eq:BT}
			\Prob\big( \responseone \succ \responsetwo \bigm| \prompt \big)
			\; = \; \sigmoid\big( \rewardstar(\prompt, \responseone) - \rewardstar(\prompt, \responsetwo) \big) \, ,
		\end{align}
		where $\rewardstar(\prompt, \response)$ represents the (unknown) oracle reward of a response given a prompt, and $\sigmoid(z) = \{ 1 + \exp(-z) \}^{-1}$ is the sigmoid function, mapping differences in rewards to probabilities. We adopt the BT model throughout this paper.

    \paragraph{Reward Modeling.} The preference data, encoding human judgment, is then used to train a reward model, $r_\theta$, which serves as a measurable objective for training the policy model. $r_\theta$ is trained by solving a MLE objective:
\begin{align}
		\label{eq:RM_objective}
		\min_{\paratheta} \
        % \quad \Losshat(\paratheta)  \\
        % \notag
        \Losshat(\paratheta) :=
		- \frac{1}{\numobs} \sum_{i=1}^{\numobs} \log \sigmoid \Big( \rewardtheta\big(\prompti{i}, \responsewini{i}\big) - \rewardtheta\big(\prompti{i}, \responselosei{i}\big) \Big).
	\end{align}
%
    This empirical loss approximates the expected negative log-likelihood
	\begin{align}
		\label{eq:def_Loss}
        \Loss(\paratheta) \; := \;
		\Exp_{\prompt \sim \promptdistr, \,(\responseone, \responsetwo) \sim \responsedistr(\cdot \mid \prompt)} \Big[ - \log \sigmoid \big( \rewardtheta(\prompt, \responsewin) - \rewardtheta(\prompt, \responselose ) \big) \Big] \, .
	\end{align}
    
    \paragraph{Policy Optimization.} To align a language model $\phi$ with human preferences, we optimize it to maximize the learned rewards $\rewardtheta$ while staying close to a reference policy $\policyref$. The objective is
	\begin{equation}\label{eq:policy_loss_with_rm}
    \max\nolimits_{{\phi}} \ 
    \Exp_{\prompt \sim \promptdistr, \response \sim \policy_{\phi}(\cdot \mid \prompt)}
    \big[ \rewardtheta (\context, \response) \big]
    - \parabeta \kull{\policy_{\phi}}{\policyref}.
\end{equation}
	It consists of two parts: 
    % \vspace{-0.8em}
    \begin{itemize}
	\item[(i)] The \emph{reward} term $\Exp_{\prompt \sim \promptdistr, \, \response \sim \policy(\cdot \mid \prompt)} [ \rewardtheta(\context, \response) ]$ encourages the policy to generate high-quality responses.
	\item[(ii)] The \emph{regularization} term \mbox{$\kull{\policy}{\policyref}$} penalizes deviations from the reference policy~$\policyref$ and is defined as \mbox{$\Exp_{\prompt \sim \promptdistr} \big[ \kull[\big]{\policy(\cdot \mid \prompt)}{\policyref(\cdot \mid \prompt)} \big]$}.
    \end{itemize}
    % \vspace{-0.8em}
    Here, $\parabeta$ is a regularization parameter that balances the trade-off between reward maximization and adherence to the reference policy. 
    % The reference policy $\policyref$, often obtained via supervised fine-tuning (SFT), serves as the starting point for optimization. 
    We assume $\parabeta$ is fixed and practitioner-specified.


    
	% \begin{enumerate}
	% 	\item[(i)] Two candidate responses $\responseonei{i}$ and $\responsetwoi{i}$ are drawn from a joint distribution $\responsedistr(\cdot \mid \prompt)$.
	% 	\item[(ii)] The preference between them is modeled using the \emph{Bradley–Terry (BT)} model:
	% 	\begin{align}
	% 		\label{eq:BT}
	% 		\Prob\big( \responseone \succ \responsetwo \bigm| \prompt \big)
	% 		\; = \; \sigmoid\big( \rewardstar(\prompt, \responseone) - \rewardstar(\prompt, \responsetwo) \big) \, ,
	% 	\end{align}
	% 	where $\rewardstar(\prompt, \response)$ represents the (unknown) quality or reward of a response given a prompt, and $\sigmoid(z) = \{ 1 + \exp(-z) \}^{-1}$ is the sigmoid function, mapping differences in rewards to probabilities.
	% \end{enumerate}

	
	    
	
	

%%%%%%%%%%%%%%%%%%%%%%%%%%%%%%%%%%%%%%%%%%%%%%%%%%%%%%%%%%%
	
	\subsection{Direct Preference Optimization}
	\label{sec:intro_DPO}

    The above-described RLHF pipeline typically leverages the Proximal Policy Optimization (PPO) algorithm \citep{schulman2017proximal} to perform policy optimization. This approach requires loading the policy network, reward model, reference model, and a value model onto the GPU during training, making it highly resource-intensive. To improve computational efficiency and practicality, Direct Preference Optimization (DPO) \citep{rafailov2023direct} has been proposed, enabling direct alignment without the need for a reward model or a value model.
	
    A key insight of DPO is that any policy~$\policytheta$ can be viewed as the optimal solution to problem~\eqref{eq:policy_loss_with_rm} if the reward~$\rewardtheta$ is 
	\vspace{-3mm}
    \begin{align}
		\label{eq:def_reward}
		\rewardtheta(\prompt, \response)
		\; \defn \; \parabeta \cdot \log \bigg( \frac{\policytheta(\response \mid \prompt)}{\policyref(\response \mid \prompt)} \bigg).
	\end{align} 
	% This suggests that if the reward function $\rewardtheta$ approximates the true $\rewardstar$ closely, the associated policy $\policytheta$ is expected to maximize the expected value.
%
    % DPO directly operates within a parameterized policy class \mbox{$\PolicySp = \{ \policytheta \mid \paratheta \in \Real^{\Dim} \}$}.
%
    Thus, DPO can directly optimize the policy $\policytheta$ using $\Losshat(\paratheta)$ in \cref{eq:RM_objective}, where $\rewardtheta$ is replaced by $\policytheta$ as defined in \cref{eq:def_reward}. This reformulation makes the objective dependent solely on $\theta$, with the reward being implicitly learned through the policy itself. As a result, the optimization process becomes significantly more efficient.
	
	% To estimate $\rewardstar$, DPO employs maximum likelihood estimation (MLE) by solving
	% \begin{align}
	% 	% \label{eq:RM_objective}
	% 	& {\rm minimize}_{\paratheta} \quad \Losshat(\paratheta)  \\
 %        \notag
 %        & \quad \Losshat(\paratheta) \; \defn \;
	% 	- \frac{1}{\numobs} \sum_{i=1}^{\numobs} \, \log \sigmoid \Big( \rewardtheta\big(\prompti{i}, \responsewini{i}\big) - \rewardtheta\big(\prompti{i}, \responselosei{i}\big) \Big) \, .
	% \end{align}
	
	
	
	\subsection{Motivation: Realigning Oracle Reward Maximization}
	\label{sec:intro_goal}

    To fully align with human values, RLHF should, in principle, train the policy to maximize the oracle reward, $\rewardstar$, as defined in the BT model. The corresponding oracle objective is then: 
    % \vspace{-.7em}
    \begin{equation} 
		%\max_{\policy}  \quad 
		\scalarvalue(\policy) \defn \; \Exp_{\prompt \sim \promptdistr, \, \response \sim \policy(\cdot \mid \prompt)} \big[ \rewardstar(\context, \response) \big] \notag \; - \; \parabeta \, \kull{\policy}{\policyref} \, .
        \label{eq:objective}
	\end{equation}
    %\ariel{The current equation seems to read as the definition of $\max_\pi J(\pi)$, but I think it's meant to be a definition of $J(\pi)$?}
    %\yaqi{Reform this sentence since it appeared once in intro.} 
    Since direct access to $\rewardstar$ is unavailable, RLHF instead relies on preference data, either through MLE-based reward modeling or methods like DPO. 
    However, these processes are not inherently designed to train the policy to directly maximize the oracle objective, $\scalarvalue(\policy)$.%\ariel{What do they directly maximize instead?}. 
    
    In this work, we aim to design an optimal sampling distribution $\responsedistr$ to realign DPO with the maximization of $\scalarvalue(\policy)$. Such a sampling strategy will improve the quality of the preference dataset, maximize the utility of limited data, and enhance both performance and efficiency.

    This focus is particularly crucial in scenarios where additional data is collected during mid-training—a key phase in the iterative fine-tuning of LMs
    \citep{touvron2023llama,bai2022training, xiong2024iterative, guo2024direct}.
    %\yaqi{Reform this sentence since it appeared once in intro.} 
    At this stage, a preliminary policy $\policytheta$ (distinct from $\policyref$) is already in place, but its performance may fall short of expectations. It is thus necessary to gather additional preference data, ideally on-policy data that target areas where the current policy shows room for improvement. An effective sampling design can significantly enhance the efficiency of leveraging human feedback in this process.
    
    % By strategically leveraging human feedback, this focused data collection ensures that the newly acquired information can enhance the model.
	
	% A closer examination of the DPO optimization process reveals that the choice of the sampling distribution $\responsedistr$ for response selection is a critical factor in determining training efficiency. In this work, we aim to design an optimal sampling distribution $\responsedistr$ that enables the DPO method to extract greater value from limited data while improving performance (as measured by $\scalarvalue(\policy)$).
	

	

    \section{T-PILAF: Theoretical Sampling Scheme }
% to align reward modeling with value maximization?
 %       \subsection{Mismatch between reward modeling and optimization objective}
        
%          \begin{align*}
%        \gradtheta \Loss(\paratheta) 
%        \qquad
%	- \gradtheta \scalarvalue(\policytheta)
%    \end{align*}
	% \subsection{Sampling scheme}
	\label{sec:sampling}

	
	We now present T-PILAF - {\em theoretically grounded policy interpolation for aligned feedback} - our sampling scheme for generating responses in data collection\footnote{The T in T-PILAF serves to distinguish the theoretical scheme from the derived, simplified, efficiently implementable PILAF.}. The scheme is shown (in \Cref{sec:theory}) to be optimal from both optimization and statistical perspectives. 
    % However, it is computationally impractical for large-scale language models. An experimental design is then proposed in \cref{sec:sampling_exp} ensuring scalability and enabling efficient inference. 
    
	Consider we have an {initial} policy $\policytheta$ and aim to collect preference data to further refine its performance.
	We propose two complementary variants of policy $\policytheta$: one that encourages exploration in regions {more} preferred by~$\policytheta$, reflecting an optimistic perspective, and another that focuses on areas {less favored by~$\policytheta$}, reflecting a conservative adjustment.
%	By generating response pairs using these two policies, we aim to enable efficient exploration and refinement of the current model $\policytheta$.
	
	Specifically, we define policies $\policythetapos$ and $\policythetaneg$ around $\policytheta$ as
    \begin{subequations}
	\begin{align}
		\label{eq:def_policythetapos}
		\policythetapos(\response \mid \prompt)
		& := \frac{1}{\Partitionthetapos(\prompt)} \; \policytheta(\response \mid \prompt)
		\exp \big\{ \rewardtheta(\prompt, \response) \big\} \, ,  \\[-1pt]
		\label{eq:def_policythetaneg}
		\policythetaneg(\response \mid \prompt)
		& := \frac{1}{\Partitionthetaneg(\prompt)} \policytheta(\response \mid \prompt)
		\exp \big\{ - \rewardtheta(\prompt, \response) \big\},
	\end{align}
	\end{subequations}
	where the reward function $\rewardtheta$ is defined in equation~\eqref{eq:def_reward}.
	The partition function $\Partitionthetapos(\prompt)$ (or $\Partitionthetaneg(\prompt)$) is given by
	\mbox{$\Partitionthetapos(\prompt)
	\defn \int_{\ResponseSp} \policytheta(\response \mid \prompt) \exp \{ \rewardtheta(\prompt, \response) \} \, \diff \response $}. %\yaqi{$\policythetapos$??? typo?}
    %\yaqi{Here it should be $\policytheta$ in the integration, consistent with the right-hand side in \Cref{eq:def_policythetapos}.}
	
	For any prompt $\prompt \in \PromptSp$, our sampling procedure involves the following steps:
	\begin{enumerate}  % \itemsep = -.2em
		\item[(i)] Draw a random variable $\xi$ from ${\rm Bernoulli}(\sampleprob(\prompt))$, where
%		\begin{align}
%			\label{eq:def_sampleprob}
        \vspace{-1em}
			$$\sampleprob(\prompt) \defn {\Partitionthetapos(\prompt) \, \Partitionthetaneg(\prompt)}/ \{1 + \Partitionthetapos(\prompt) \, \Partitionthetaneg(\prompt) \}. $$ ~ \\ \vspace{-4em}
%		\end{align}
		\item[(ii)] If $\xi = 1$, independently draw responses $\responseone, \responsetwo \in \ResponseSp$ according to
%	\begin{align*}
		$\responseone \sim \policythetapos(\cdot \mid \prompt)$ and $\responsetwo \sim \policythetaneg(\cdot \mid \prompt)$. 
%	\end{align*}
		If $\xi = 0$, draw responses as $\responseone, \responsetwo \sim \policytheta(\cdot \mid \prompt)$.
	\end{enumerate}
	
	In the next section, we will theoretically analyze T-PILAF. To account for the changes in sampling, we adopt a slightly modified loss function in the theoretical framework:
     \begin{align*}
%     	\label{eq:def_Losshat_adj}
		\Losshat(\paratheta) \defn
     	& - \frac{1}{\numobs} \sum_{i=1}^{\numobs} \weight(\prompti{i}) \cdot \log \sigmoid \Big( \rewardtheta\big(\prompti{i}, \responsewini{i}\big) - \rewardtheta\big(\prompti{i}, \responselosei{i}\big) \Big) .
     \end{align*}
	The newly introduced weight function $\weight$ is defined as
	\begin{align}
		\weight(\prompt)
		& \label{eq:weight}
        \; \defn \; \big\{ 1 + \Partitionthetapos(\prompt) \, \Partitionthetaneg(\prompt) \big\} / \, \Partitionthetabar \, ,
	\end{align}
	where the normalization constant~$\Partitionthetabar > 0$ is given by \mbox{$\Partitionthetabar \defn 1 + \int_{\PromptSp} \Partitionthetapos(\prompt) \, \Partitionthetaneg(\prompt) \, \promptdistr(\prompt) \, \diff \prompt$}. We also modify the population loss $\mathcal{L}$ in \cref{eq:def_Loss} with the weight function.
	
	
	

%%%%%%%%%%%%%%%%%%%%%%%%%%%%%%%%%%%%%%%%%%%%%%%%%%%%%%%%%%%%%%%%%%%%%%%%%%%%%%%
\section{Theoretical Analysis}\label{sec:theory}
This section provides the theoretical grounding and analysis of our proposed sampling scheme from two perspectives. In the {\em optimization} analysis (\Cref{sec:theory_opt}) we show that T-PILAF {\em aligns two objectives (gradient alignment property)}: maximizing the likelihood function (\cref{eq:def_Loss}) becomes equivalent to gradient ascent on the value function $\scalarvalue(\policytheta)$ (\cref{eq:objective}). Consequently, policy updates on $\pi_\theta$ move the parameters in the direction of steepest increase of $J$. T-PILAF thus provides the potential to accelerate training and improve generalization, compared to vanilla (uniform)  sampling. In the {\em statistical} analysis (\Cref{sec:theory_stat}) we focus on statistical error and show that {the asymptotic covariance} of the estimated parameter~$\parathetahat$ (inversely) aligns with the Hessian of the objective function~$\scalarvalue$ when sampling with T-PILAF. As a result, T-PILAF makes the sampled comparisons more informative, as they align with directions where~$\scalarvalue$ is most sensitive. The net outcome is reduced statistical variance of our method through tighter concentration of estimates in directions that matter most for performance.


\subsection{Optimization Considerations}
            \label{sec:theory_opt}



            
We begin by analyzing the DPO algorithm from an optimization perspective.
{\Cref{thm:grad} below formally illustrates how T-PILAF ensures alignment between the MLE gradient, $\gradtheta \Loss(\paratheta)$, and the oracle objective gradient, $\gradtheta \scalarvalue(\policytheta)$.}

%
%
\begin{theorem}[Gradient structure in DPO training]
\label{thm:grad}
  Using data collected from our proposed response sampling scheme T-PILAF, the gradient of $ \Loss(\paratheta) $ satisfies
\begin{align*}
    \gradtheta \Loss(\paratheta) \; = \;
    - \, \frac{\parabeta}{\Partitionthetabar} \, \gradtheta \scalarvalue(\policytheta) \, + \, \Term_2 \, ,
\end{align*}
where the constant $ \Partitionthetabar $ is defined in equation~\eqref{eq:weight}, and the term $ \Term_2% = \bigO( \norm{\rewardtheta - \rewardstar}^2 ) 
$ represents a second-order error.
\end{theorem}
The detailed proof of \Cref{thm:grad} is deferred to \Cref{sec:proof:thm:grad}. 
It involves calculation of explicit forms of the gradients $\gradtheta \Loss(\paratheta)$ and $\gradtheta \scalarvalue(\policytheta)$; the most notable technical contribution is showing how to leverage our sampling scheme to approximate the derivative $\divsigmoid$ of the sigmoid function. By using T-PILAF sampling, we can transform a difference term of the form $\sigmoid ( \Delta \rewardstar ) - \sigmoid ( \Delta \rewardtheta )$ in $\gradtheta \Loss(\paratheta)$ into a linear difference $\Delta \rewardstar - \Delta \rewardtheta$ in $\gradtheta \scalarvalue(\policytheta)$.

\Cref{thm:grad} establishes the \emph{gradient alignment} property, demonstrating that minimizing the likelihood-based loss function~$\Loss$ closely aligns with maximizing the oracle objective function~$\scalarvalue$, with only a minor second-order error. It highlights how the proposed sampling scheme enables the DPO framework to effectively guide the policy toward optimizing the expected reward.
%
Beyond DPO, in \Cref{app:extension}, we show how the same principle can be applied to PPO-based RLHF algorithms to help improve the sampling. %\yaqi{I will work on adding more details in the appendix tomorrow to address the generalization of our approach. Depending on the progress, if I can include sufficient detail in the appendix discussion, we might consider adding a brief sentence at the end of this subsection to mention this generalization.} 
		

		
		
		
		
%%%%%%%%%%%%%%%%%%%%%%%%%%%%%%%%%%%%%%%%%%%%%%%%%%%%%%%%%%%%%%%%%%%%%%%%%%

\subsection{Statistical Considerations \yaqidone}
    \label{sec:theory_stat}

From a statistical standpoint, we first examine the asymptotic distribution of the estimated parameter $\parathetahat$ when it (approximately) solves the optimization problem~\eqref{eq:RM_objective}. In \Cref{thm:asymp}, we formally characterize the randomness or statistical error inherent in $\parathetahat$ under this idealized scenario.
The detailed proof of \Cref{thm:asymp} is provided in \Cref{sec:proof:thm:asymp}.
\begin{theorem}
    \label{thm:asymp}
%			We take $\weight(\prompt) \equiv 1$.
    Assume the reward model $\rewardstar$ in the BT model~\eqref{eq:BT} satisfies $\rewardstar = \reward_{\parathetastar}$ for some parameter $\parathetastar$.
    Under mild regularity conditions, the estimate $\parathetahat$ asymptotically follows a Gaussian distribution
    \begin{align*}
        \sqrt{\numobs} \; ( \parathetahat - \parathetastar)
        \; \stackrel{d}{\longrightarrow} \; \Gauss( \veczero, \CovOmega )
        \qquad \mbox{as $\numobs \rightarrow \infty$} \, .
    \end{align*}
    We have an estimate of the covariance matrix $\CovOmega$:
    \begin{align*}
        \CovOmega \; \preceq \; \Const_{1} \cdot \CovOpstar^{-1} \, ,
    \end{align*}
    where $\Const_{1} > 0$ is a universal constant. 
    When using T-PILAF, the matrix~$\CovOpstar$ is given by
    \begin{align}
        \label{eq:def_CovOpstar_simple}
        \CovOpstar \defn \; \Exp_{\prompt \sim \promptdistr} \Big[ \Cov_{\response \sim \policystar(\cdot \mid \prompt)} \big[ \gradtheta \rewardstar(\prompt, \response) \bigm| \prompt \big] \Big] \, .
    \end{align}
\end{theorem}

Next we analyze the performance of the output policy \mbox{$\policyhat = \policy_{\parathetahat}$} from \Cref{thm:asymp} in terms of the expected value~$\scalarvalue(\policy)$. In \Cref{lemma:hess_scalarvalue}, we show that our proposed sampling method guarantees that the covariance of the statistical error in~$\parathetahat$ aligns inversely with the Hessian of~$\scalarvalue$ at the optimal policy~$\policystar$. This alignment prioritizes convergence efficiency along directions where the Hessian has large eigenvalues, adapting to the geometry of the optimization landscape. It highlights the efficiency of our sampling scheme in reducing statistical error.
For the detailed proof, see \Cref{sec:proof:lemma:hess_scalarvalue}.
\begin{theorem}
        \label{lemma:hess_scalarvalue}
    The value function $\scalarvalue(\policy)$ we define in equation~\eqref{eq:objective} satisfies $\gradtheta \scalarvalue(\policystar) = \veczero$ and % $\hesstheta \scalarvalue(\policystar) =$
    %\vspace{-.6em}
    \begin{align}
        \label{eq:hessscalarvalue}
        \hesstheta \scalarvalue(\policystar) \; = \;
%				- \frac{1}{\parabeta} \, \Exp_{\prompt \sim \promptdistr} \Big[ \Cov_{\response \sim \policystar(\cdot \mid \prompt)} \big[ \gradtheta \rewardstar(\prompt, \response) \bigm| \prompt \big] \Big]
%				= 
        - \frac{1}{\parabeta} \, \CovOpstar
    \end{align} %~ \vspace{-1.2em} \\
    for matrix $\CovOpstar$ defined in equation~\eqref{eq:def_CovOpstar_simple}.
    As a corollary, suppose $\CovOpstar$ is nonsingular, then there exists a constant $\Const_{2} > 0$ such that for any $\varepsilon > 0$, 
    % \vspace{-.3em}
%			with high probability,
%			\begin{align*}
%				\scalarvalue(\policyhat)
%				\; \geq \; \scalarvalue(\policystar) \, - \, \frac{\Const{} \, \big\{ 1 + \supnorm{\Partitionthetapos \Partitionthetaneg} \big\}}{\parabeta} \cdot \frac{\Dim \log \numobs}{\numobs}
%			\end{align*}
%			for some universal constant $\Const{} > 0$.
    \begin{align}
        & \limsup_{\numobs \rightarrow \infty} \Prob \bigg\{ \scalarvalue(\policyhat) < \scalarvalue(\policystar) - \Const_{2} \cdot \frac{\Dim \, (1 + \varepsilon)}{\numobs} \bigg\} \notag  \\
        & \qquad \leq \; \Prob\big\{ \chisquare_{\Dim} > (1 + \varepsilon) \, \Dim \big\}
        \leq \exp\Big\{  -\frac{\Dim}{2} \bigl(\varepsilon - \log(1 + \varepsilon)\bigr)  \Big\} . \label{eq:gap_bd}
    \end{align}
    % \vspace{-1.5em}
\end{theorem}		

    Our proposed sampling distribution $\responsedistr$ ensures that the output policy $\policyhat$ performs predictably and reliably. The value gap $\scalarvalue(\policystar) - \scalarvalue(\policyhat)$ asymptotically follows a chi-square distribution, irrespective of the problem instance details, such as the underlying reward model $\rewardstar$. 
    This \emph{structure-invariant statistical efficiency} allows the method to achieve asymptotically efficient estimates without requiring explicit knowledge of the model structure. % \yaqi{maybe cut from here}


    %\iffalse
    In addition to our analysis of the proposed sampling scheme in \Cref{sec:sampling}, we present a generalized version of \Cref{thm:asymp} that applies to any response sampling distribution~$\responsedistr$. While not directly tied to the main focus of this work, this broader result may be of independent interest to readers.
    The proof of \Cref{thm:asymp_full} is provided in \Cref{sec:proof:thm:asymp_full}.
    \begin{lemma}
        \label{thm:asymp_full}
        For a general sampling distribution $\responsedistr$, the statement in \Cref{thm:asymp} remains valid with the matrix $\CovOpstar$ redefined as % $\CovOpstar \defn$
        \begin{align}
            \CovOpstar \defn
            \Exp_{\prompt \sim \promptdistr,(\responseone, \, \responsetwo) \sim \responsedistravg(\cdot \mid \prompt)}
            \Big[ \, \weight(\prompt) \cdot \Var\big(\indicator\{\responseone = \responsewin\} \bigm| \prompt, \responseone, \responsetwo \big) \cdot \grad \, \grad^{\top} \Big] \, ,
            \label{eq:def_CovOpstar}
        \end{align} % ~ \vspace{-1.8em} \\
        where the expectation is taken over the distribution
        % \vspace{-.3em}
        \begin{subequations}
            \begin{align}
                \label{eq:def_responsedistravg}
                \responsedistravg(\responseone, \responsetwo \mid \prompt) 
                \defn \frac{1}{2} \, \big\{ \responsedistr(\responseone, \responsetwo \mid \prompt) + \responsedistr(\responsetwo, \responseone \mid \prompt) \big\} \, .
            \end{align} % ~ \vspace{-1.8em} \\
        The variance term is specified as
            \begin{align}
                & \Var\big(\indicator\{\responseone = \responsewin\} \mid \prompt, \responseone, \responsetwo \big)
                \label{eq:def_var}
                = \sigmoid\big( \rewardstar(\prompt, \responseone) - \rewardstar(\prompt, \responsetwo) \big) \, \sigmoid\big( \rewardstar(\prompt, \responsetwo) - \rewardstar(\prompt, \responseone) \big)
            \end{align}
        and the gradient difference $\grad$ is defined as
            \begin{align}
                \label{eq:def_grad}
                \grad \defn \gradtheta \rewardstar(\prompt, \responseone) - \gradtheta \rewardstar(\prompt, \responsetwo) \, .
            \end{align}
        \end{subequations}
    \end{lemma}
    
    The general form of the matrix $\CovOpstar$ offers valuable insights for designing a sampling scheme. To ensure $\CovOpstar$ is well-conditioned (less singular), we must balance two key factors when selecting responses $\responseone$ and $\responsetwo$:
    % \vspace{-.8em}
    \begin{description} \itemsep = -.05em
        \item \emph{Large variance:} The variance in definition~\eqref{eq:def_var} should be maximized. This occurs when $\rewardstar(\prompt, \responseone) \approx \rewardstar(\prompt, \responsetwo)$. Intuitively, preference feedback is most informative when annotators compare responses of similar quality.
        \item \emph{Large gradient difference:} The gradient difference $\grad$ from definition~\eqref{eq:def_grad} should also be large. This requires responses with significantly different gradients. Only then can the comparison provide a clear and meaningful direction for model training.
    \end{description}
    %\fi

		
		
		
		


%%%%%%%%%%%%%%%%%%%%%%%%%%%%%%%%%%%%%%%%%%%%%%%%%%%%%%%%%%%%%%%%%%%%%%%%%%%
		
	

\section{PILAF Algorithm}	\label{sec:sampling_exp}

    % \yunzhen{Julia: is it good to put it here or together with the sampling scheme in theory?}

We now demonstrate that the T-PILAF sampling scheme defined in \cref{eq:def_policythetapos} and (\ref{eq:def_policythetaneg}) can be naturally extended into an efficient empirical algorithm (PILAF).

The first challenge in implementing these definitions lies in calculating the normalizing factors $\Partitionthetapos(\prompt)$ and $\Partitionthetaneg(\prompt)$, which can be computationally expensive for LLMs. To address this, we simplify the process by omitting these factors and replacing them with 1.\footnote{When the regularization coefficient $\parabeta$ is sufficiently small, the term $\exp\{\rewardtheta(\prompt, \response)\}$ in equation~\eqref{eq:def_policythetapos} stays close to $1$ and has only a minor effect. Consequently, the partition function $\Partitionthetapos(\prompt)$ is approximately $1$. A similar reasoning applies to $\Partitionthetaneg(\prompt)$.
\vspace{-1.4em}
%\yaqi{Please check here.}
}
% \ariel{Should we, or can we, somehow justify that this is fine? Some intuition that this quantity is typically close to 1/2? Else, it would have been interesting to also explore this parameter}.
Consequently, the sampling process becomes straightforward: with probability~\(1/2\), we sample using \(\policytheta\), and otherwise, we sample using~\(\policythetapos\) and~\(\policythetaneg\).

The second challenge lies in sampling a response $\response$ from $\policytheta(\response \mid \prompt)
		\exp \big\{ \pm \rewardtheta(\prompt, \response) \big\}$ in an autoregressive way for next-token generation. 
We argue that the policy $\policythetapos$ (and $\policythetaneg$) can be approximated in a token-wise manner:
		\begin{align*}
			\policythetapos(\response \mid \prompt)
		 \; \approx \; \policythetapos(\tokent{1} \mid \prompt) \, \policythetapos(\tokent{2} \mid \prompt, \tokent{1})  \cdots \, \policythetapos(\tokent{t} \mid \prompt, \tokent{1:t-1}) \,
			\cdots \, \policythetapos(\tokent{\numtok} \mid \prompt, \tokent{1:\numtok-1}),
		\end{align*}
		where
		\begin{align*}
			& \policythetapos(\tokent{t} \mid \prompt, \tokenttot{1}{t-1}) \; = \; 
			\frac{1}{\Partition(\prompt, \tokenttot{1}{t-1})} \, 
			\policytheta(\tokent{t} \mid \prompt, \tokenttot{1}{t-1})
			\bigg( \frac{\policytheta(\tokent{t} \mid \prompt, \tokenttot{1}{t-1})}{\policyref(\tokent{t} \mid \prompt, \tokenttot{1}{t-1})} \bigg)^{\parabeta} % \\
	%		& \; = \; \frac{1}{\Partition(\prompt, \tokenttot{1}{t-1})} \, 
	%		\policyref(\diff \tokent{t} \mid \prompt, \tokenttot{1}{t-1})
	%		\bigg( \frac{\policytheta(\tokent{t} \mid \prompt, \tokenttot{1}{t-1})}{\policyref(\tokent{t} \mid \prompt, \tokenttot{1}{t-1})} \bigg)^{1+\parabeta}
		\end{align*}
		with $\Partition(\prompt, \tokenttot{1}{t-1})$ being a partition function. 
        %\ariel{re: this whole equation above. (1) Is this really an approximation? The first equation approximating $\pi_\theta^+(\vec{y}|x)$ looks fairly exact for a reasonable definition of $\pi_\theta^+(y_i|...)$ unless I'm missing something obvious; inthead the definition of $\pi_\theta^+(y_i|...)$ seems like an approximation. (2) Is there any meaning to switching the notation from $y_t$ to $dy_t$ between the equations?} \yaqi{(1) The approximation appears in the partition function. If we can divide the density function by the overall integration along the trajectory, then the token-wise expansion becomes exact. However, the current partition $\Partition$ is a step-wise normalization. (2) I have removed the $\diff$ notation.} 
        The substitution of $\rewardtheta$ uses the correspondence between the reward model 
        $\rewardtheta$ and the policy $\policytheta$ in \cref{eq:def_reward}, under the assumption that this correspondence holds for all truncations~$\tokenttot{1}{t-1}$. It gives us a direct per-token prediction rule:
    % \vspace{-1em}  % Ariel: I commented this out bc it was messing up the formatting
    \begin{align*}
     \policythetapos(\cdot \mid \prompt, \tokenttot{1}{t-1}) \; = \; \softmax\Big( \big\{ (1 + \parabeta) \, \headtheta - \parabeta \, \headref\big\} (\prompt, \tokenttot{1}{t-1}) \Big).
    \end{align*}
    % \vspace{-1.8em}
Here $ \headtheta $ and $ \headref $ are the logits of the policies $\policytheta$ and $\policyref$, respectively. $\parabeta$ is the regularization coefficient from the objective function $ \scalarvalue(\policy)$ in \cref{eq:objective}. Responses are then generated using standard decoding techniques, such as greedy decoding or nucleus sampling. Similarly, the generation for $\policythetaneg$ follows 
%\vspace{-.5em}
\begin{align*}
 \policythetaneg(\cdot \mid \prompt, \tokenttot{1}{t-1}) \; = \; \softmax\Big( \big\{ (1 - \parabeta) \, \headtheta + \parabeta \, \headref\big\} (\prompt, \tokenttot{1}{t-1}) \Big) \, .
\end{align*}
% \vspace{-1.8em}
For a detailed, step-by-step proof, see Proposition~1 in \citet{liu2024decoding}.

We formalize our final algorithm in \cref{alg:our_sampling}. Vanilla DPO \citep{rafailov2023direct, guo2024direct} employs a basic generation approach, sampling $\responseone_i, \responsetwo_i \sim \policytheta$ at Step~3. In contrast, instead of only sampling from $\policytheta$, our sampling scheme interpolates and extrapolates the logits~$\headtheta$ and~$\headref$ with coefficient $\parabeta$, enabling exploration of a wider response space to align learning from human preference with value optimization. The $\parabeta$ here is the same parameter that controls the KL regularization in \cref{eq:policy_loss_with_rm}, as set by the problem.
% \yunzhen{Do we need this paragraph?} 
%\ariel{It does read a bit weird in context, but I think it's useful to state something to the effect of "This is our sampling procedure; in contrast, standard DPO samples everything from $\pi_\theta$}

\paragraph{Cost analysis} We summarize sampling and annotation costs per preference pair for PILAF and related sampling schemes in \cref{tab:setup_summary}. In \textit{Vanilla} sampling (from $\policytheta$), two generations and two annotations are required for human preference labeling, same to PILAF when the pair is sampled from $\policytheta$, which happens half the time. With 50\% probability, PILAF uses $\policythetapos$ and $\policythetaneg$ to generate, requiring two forward passes with $\policytheta$ and $\policyref$ to generate one sample. Thus, on average, a preference pair sampled with PILAF requires a sampling cost of 3 forward passes (1.5 time the cost of \textit{Vanilla}) with the same annotation cost. To compare, \citet{xiong2024iterative, dong2024rlhf} perform \textit{Best-of-N} sampling with $N=8$, which generates and annotates all 8 responses, selecting the best and worst of them. \citet{xie2024exploratory} use a \textit{Hybrid} method that generates with $\policytheta$ and $\policyref$, thus matching the sampling and annotation costs of the \textit{Vanilla} method. We empirically compare PILAF with these methods in the next section.


    
		% Building on these principles, we propose a simple yet effective sampling scheme. For any given prompt $ \prompt $, generate two responses $(\responseone, \responsetwo)$ using one of two strategies, chosen with \emph{equal} probability: \vspace{-.5em}
		% \begin{itemize} % \itemsep = -.1em
		% 	\item 
		% 	Generate both responses independently from $\policytheta(\cdot \mid \prompt)$.
		% 	\item Sample $ \responseone $ from $ \policythetapos$ and $ \responsetwo $ from $ \policythetaneg$.
		% 	This can be done (approximately) in per-token prediction, via
		% 	\vspace{-.5em}
		% 	\begin{align*}
		% 		& \policythetapos(\cdot \mid \prompt, \tokenttot{1}{t-1}) \\
		% 		& \quad \; = \; \softmax\Big( \big\{ (1 + \parabeta) \, \headtheta - \parabeta \, \headref\big\} (\prompt, \tokenttot{1}{t-1}) \Big)  \\
		% 		& \policythetaneg(\cdot \mid \prompt, \tokenttot{1}{t-1})  \\
		% 		& \quad \; = \; \softmax\Big( \big\{ (1 - \parabeta) \, \headtheta + \parabeta \, \headref\big\} (\prompt, \tokenttot{1}{t-1}) \Big) \, .
		% 	\end{align*}
		% 	\vspace{-1.8em}
		% \end{itemize}
		% $\parabeta$ is the regularization coefficient from the objective function $ \scalarvalue(\policy)$ in equation~\eqref{eq:objective}. This adjustment shifts $ \policythetaneg$ closer to $ \policyref$ while pushing $ \policythetapos$ further away. 
		

  %       We formalize the algorithm in \cref{alg:our_sampling}.

\begin{algorithm}
\caption{DPO with PILAF (ours).}
\label{alg:our_sampling}
\begin{algorithmic}[1]
    \INPUT Prompt Dataset $\mathcal{D}_\rho$, preference oracle $\mathcal{O}$, $\policytheta, \policyref$.
    \FOR{step $t$ = 1, ..., $T$}
        \STATE Sample $n_t$ prompts $\{x_i\}_{i=1}^{n_t}$ from $\mathcal{D}_\rho$.
        \STATE \hl{With probability 1/2, sample $\responseone_i, \responsetwo_i \sim \policytheta$; with probability 1/2, sample $\responseone_i \sim \policythetapos$ and $\responsetwo_i \sim \policythetaneg$.}
        % \STATE \hl{}
        \STATE Query $\mathcal{O}$ to label $(x_i, \responseone_i, \responsetwo_i)$ into $(x_i, \responsewini{i}, \responselosei{i})$.
        \STATE Update $\policy_{\theta_t}$ with DPO loss using $\{(x_i, \responsewini{i}, \responselosei{i})\}_{i=1}^{n_t}$.
        % \IF{condition on $x$}
        %     \STATE Perform some operation
        % \ELSE
        %     \STATE Perform an alternative operation
        % \ENDIF
    \ENDFOR
\end{algorithmic}
\end{algorithm}
\vspace{-1em}




\begin{table*}
\vspace{-13pt}
    \caption{ \footnotesize A cost summary of PILAF and sampling methods from related works. \textit{Best-of-N} method in \citet{xiong2024iterative} uses the oracle reward to score all candidate responses, then selects the highest- and lowest-scoring ones—instead of providing a preference label for only two responses. We restrict the oracle to providing only preference labels. Thus, we create a \textit{Best-of-N} variant that uses the DPO internal reward for selection and then applies preference labeling, with an annotation cost of 2. We compare with this variant in the experiment.}
    \label{tab:setup_summary}
    \vskip 0.2in
    \centering
\begin{scriptsize}
% \setlength{\tabcolsep}{1.5pt}
\begin{sc}
    \begin{tabular}{l|cc|cc}
    \toprule
        \textbf{Method} & $\responseone$ & $\responsetwo$ & Sampling Cost & Annotation Cost \\ 
        \midrule
        \textit{Vanilla} \citep{rafailov2023direct} & $\policytheta$ & $\policytheta$ & 2 & 2 \\
        \textit{Best-of-N} \citep{xiong2024iterative}, N=8 & best of $\policytheta$ & worst of $\policytheta$ & 8 & 8* \\
        \textit{Best-of-N} (with DPO reward), N=8 & best of $\policytheta$ & worst of $\policytheta$ & 8 & 2 \\
        \textit{Hybrid} \citep{xie2024exploratory} & $\policytheta$ & \policyref & 2 & 2\\
        % \textit{SEA} \citep{liu2024sample} & & & 20 & 2 \\
        \midrule
        \textit{PILAF} (OURS) & $\policythetapos / \policytheta$ & $\policythetaneg / \policytheta$ & 3 & 2\\
    \bottomrule
    \end{tabular}
\end{sc}
\end{scriptsize}
% \vspace{-0.8em}
\end{table*}
    
    
	
	% \paragraph{Importance sampling method}
	
	% In the importance sampling method, we first generate \(m\) responses from \(\policytheta\) for each prompt. These responses, along with the prompts, are fed into the reward model to obtain reward values. As the generation method involves reweighting by $\exp \big\{\rewardtheta(\prompt, \response) \big\}$, we directly use these values to calculate and sample from the weighted probability distribution over the \(m\) sampled responses. In this process, \(m-2\) samples are discarded.
	
	% \yaqiadd{I am still unclear about the weights used—could you provide more technical details here or in the Appendix?}
	
	% \paragraph{Direct sampling method}
	
	% Additionally, we propose a direct sampling approach, where the logits are modified token-wise during generation.

\section{Experiments}
\label{sec:experiments}

\begin{figure*}[t]
\vspace{-6mm}
    \centering
    \includegraphics[width=0.8\linewidth]{figs/compare.pdf}
    \vspace{-4mm}
    \caption{\textbf{Qualitative comparison} with the baseline for generating a sequence of novel view images.  
    The results demonstrate that our method synthesizes more consistent multi-view images compared to our baseline model (Zero123). In addition, compared to SyncDreamer, our method visually maintains better similarity to the conditioned image and appears more natural.}
    \label{fig:sota_compare}
\vspace{-5mm}
\end{figure*}

\subsection{Experimental Setups}
\textbf{Dataset.}
Following previous work~\cite{zero123, SyncDreamer}, we evaluate our work on the Google Scanned Object (GSO)~\cite{GSO} dataset to verify the zero-shot novel view image synthesis capability. 
We also provide results for additional datasets in the Supplementary Material.
Specifically, we randomly select 30 objects from the GSO dataset with various object categories. 
Unlike recent approaches~\cite{mvdream, SyncDreamer} that aim to enhance the consistency of novel view synthesis models by generating multiple fixed-view images, our method can generate images from any camera pose and any number of views. Therefore, we conduct experiments under different camera pose settings to validate our approach:
specifically, 
1) \textit{16-views with free camera pose}: for each object, we circularly render 16 views with the elevation angles ranging in $[-10\degree, 40\degree]$ and the azimuth angles are evenly distributed in $[0\degree, 360\degree]$. 
2) \textit{16-views with fixed camera pose}: We maintain a constant elevation angle of $30\degree$ and uniformly sample azimuth angles (same as SyncDreamer~\cite{SyncDreamer}).
3) \textit{32-views with free camera pose}: Similar to the first setting, but we sample 32 views.
It's important to note that our method does not require additional training or fine-tuning on any datasets.

\noindent\textbf{Metrics.}
To validate the effectiveness of our method, we mainly evaluate it based on three criteria:
1) \textit{Quality Score}. We evaluate the image quality of synthesized multi-view images by measuring their similarity with ground truth images. Following prior research~\cite{zero123, sparsefusion}, we report the similarity between the synthesized images and the ground truth images with standard metrics: PSNR, SSIM~\cite{ssim}, and LPIPS~\cite{lpips}.
2) \textit{Multi-view Consistency Score}. As the primary goal of our work is to improve the consistency of generated images, we also employ the 3D consistency score~\cite{3dim} to verify the consistency among the synthesized images. Specifically, we train an Instant-NGP~\cite{instant_ngp} with the input image and part of the synthesized novel view images of our model and evaluate the similarity between the remaining synthesized images and the rendered images of Instant-NGP. For the synthesized multi-view images of each object, we allocate $3/4$ for training and reserve the remaining $1/4$ for validation.
Intuitively, if the consistency of synthesized images is improved, the NeRF-like model will train a better object representation, and the re-rendered images will agree more with the validation images.
3) \textit{Input Consistency Score}. To assess the faithfulness of synthesized images in preserving the identity of the input condition image, we introduce the input consistency score. This score calculates the similarity of each synthesized image with the input condition image, utilizing the LPIPS metric.

In addition, we use synthesized multi-view images to train a neural 3D reconstruction model (NeuS~\cite{neus}) and report commonly used Chamfer Distances (CD) and Volume IoUs between the trained 3D model and the ground truth.

\noindent\textbf{Baselines.}
Given that our main goal is to improve the consistency of the trained baseline model without further fine-tuning, we mainly compare our approach with the used baseline model Zero123~\cite{zero123}. Additionally, we compare our method to the SOTA approaches such as PGD~\cite{tseng2023consistent} and SyncDreamer~\cite{SyncDreamer} using the same Zero123 base model.

\noindent\textbf{Implementation Details.}
We use the official checkpoint provided by Zero123~\cite{zero123}, which is trained on objaverse~\cite{objaverse} for 165,000 steps. We inject our epipolar attention layer after step $T=4$ and layer $L=10$ by default. We find that feature fusion weight $\alpha=0.5$, and the number of context views $M=2$ work better.

\begin{table}[t]
\centering
\caption{Comparison of multi-view consistency, image quality, and input consistency of synthesized multi-view images at the 16-view setting with free camera pose.}
\label{tab:view16_free_compare}
\vspace{-2mm}
\scalebox{0.6}{
\begin{tabular}{c ccc ccc c}
\toprule
              & \multicolumn{3}{c}{Multi-view Consistency} & \multicolumn{3}{c}{Quality Score} & \multicolumn{1}{c}{Input Consis.} \\
              \cmidrule(lr){2-4} \cmidrule(lr){5-7} \cmidrule(lr){8-8}
              & PSNR$\uparrow$  & SSIM$\uparrow$ & LPIPS$\downarrow$ 
              & PSNR$\uparrow$  & SSIM$\uparrow$ & LPIPS$\downarrow$ 
              & LPIPS$\downarrow$ 
              \\ \midrule

Zero123
& 15.225        & 0.645       & 0.408
& 14.255        & 0.747       &	0.208
& 0.303         
\\
SyncDreamer
& 14.830        & 0.626       & 0.434
& 12.650        & 0.713       &	0.254
& 0.317         
\\
Ours 
& \best{18.300}	& \best{0.734}	& \best{0.355}
& \best{14.947}	& \best{0.763}	& \best{0.191}
& \best{0.282}
\\

\bottomrule
\end{tabular}
}
\end{table}

\begin{table}[t]
\vspace{-1mm}
\centering
\caption{Comparison of multi-view consistency, image quality, and input consistency at the 16-view setting with fixed camera pose as SyncDreamer~\cite{SyncDreamer}.}
\label{tab:view16_fxied_compare}
\vspace{-3mm}
\scalebox{0.6}{
\begin{tabular}{c ccc ccc c}
\toprule
              & \multicolumn{3}{c}{Multi-view Consistency} & \multicolumn{3}{c}{Quality Score} & \multicolumn{1}{c}{Input Consis.} \\
              \cmidrule(lr){2-4} \cmidrule(lr){5-7} \cmidrule(lr){8-8}
              & PSNR$\uparrow$  & SSIM$\uparrow$ & LPIPS$\downarrow$ 
              & PSNR$\uparrow$  & SSIM$\uparrow$ & LPIPS$\downarrow$ 
              & LPIPS$\downarrow$ 
              \\ \midrule

Zero123
& 16.556        & 0.682       & 0.378
& 14.592        & 0.750       &	0.207
& 0.305         
\\
SyncDreamer
& \best{22.424}        & \best{0.812}       & \best{0.268}
& 15.269        & 0.749       &	0.196
& 0.300         
\\
Ours 
& 21.151	& 0.780	& 0.302
& \best{15.293}	& \best{0.764}	& \best{0.184}
& \best{0.287}
\\

\bottomrule
\end{tabular}
}
\vspace{-4mm}
\end{table}


\subsection{Comparison With Baseline Models}
The quantitative comparison on three settings are shown in Tab.~\ref{tab:view16_free_compare}, Tab.~\ref{tab:view16_fxied_compare}, and Tab.~\ref{tab:view32_free_compare}. The qualitative comparison is shown in Fig.~\ref{fig:sota_compare}.

\begin{table}[t]
\centering
\caption{Comparison of multi-view consistency and image quality scores of synthesized multi-view images at the 32-view setting with free camera pose.}
\vspace{-3mm}
\label{tab:view32_free_compare}
\scalebox{0.7}{
\begin{tabular}{c ccc ccc}
\toprule
              & \multicolumn{3}{c}{Multi-view Consistency} & \multicolumn{3}{c}{Quality Score} \\
              \cmidrule(lr){2-4} \cmidrule(lr){5-7}
              & PSNR$\uparrow$  & SSIM$\uparrow$ & LPIPS$\downarrow$ 
              & PSNR$\uparrow$  & SSIM$\uparrow$ & LPIPS$\downarrow$ 
              \\ \midrule

Zero123
& 16.515        & 0.694       & 0.378
& 15.142        & 0.733       &	0.211
\\
PGD~\cite{tseng2023consistent}
& 18.481        & 0.720       & 0.343
& 15.281        & 0.739       &	0.205
\\
Ours 
& \best{20.655}	& \best{0.792}	& \best{0.305}
& \best{15.268}	& \best{0.742}	& \best{0.203}
\\

\bottomrule
\end{tabular}
}
\vspace{-3mm}
\end{table}

\begin{table*}
  [t]
  \centering
  \resizebox{\textwidth}{!}{%
  \begin{tabular}{cccccccccccc}
    \toprule \multicolumn{2}{c}{Components}                                                             & \multicolumn{5}{c}{Re-executability Rate (\%)} & \multicolumn{5}{c}{Readability (\#)} \\
    \cmidrule(lr){1-2} \cmidrule(lr){3-7} \cmidrule(lr){8-12}        \hspace{8pt}\labelemoji\hspace{8pt}                                                                & \hspace{8pt}\toolemoji\hspace{8pt}                                      & O0                                 & O1             & O2             & O3             & AVG            & O0             & O1             & O2             & O3             & AVG            \\
    \hline
    \rowcolor[rgb]{0.93,0.93,0.93}\multicolumn{12}{c}{\textbf{Initialize with LLM4Decompile-End-6.7B~\citep{llm4decompile}}}   \\
    \xmark                                                                                              & \xmark                                    & 69.51                              & 46.95          & 50.61          & 46.34          & 53.35          & 3.98 & 3.41 & 3.44 & 3.38 & 3.55 \\
    \cmark                                                                                              & \xmark                                    & 75.61                              & 50.61          & 50.00          & 50.00          & 56.55          & 4.01 & 3.44 & 3.39 & \textbf{3.49} & 3.58 \\
    \xmark                                                                                              & \cmark                                    & 83.54                     & \textbf{56.10}          & 51.22          & 50.61 & 60.37 & 4.05 & 3.51 & 3.51 & 3.42 & 3.62 \\
    \cmark                                                                                              & \cmark                                    & \textbf{85.37}                            & \textbf{56.10}                     & \textbf{51.83} & \textbf{52.43}          & \textbf{61.43} & \textbf{4.13} & \textbf{3.60} & \textbf{3.54} & \textbf{3.49} & \textbf{3.69} \\

    \rowcolor[rgb]{0.93,0.93,0.93}\multicolumn{12}{c}{\textbf{Initialize with Deepseek-Coder-6.7B-base~\citep{deepseekcoder}}} \\
    \xmark                                                                                              & \xmark                                    & 59.15                              & 35.98          & 39.02          & 37.80          & 42.99          & 3.71 & 3.05 & 3.16 & 3.05 & 3.24 \\
    \cmark                                                                                              & \xmark                                    & 66.46                              & 41.46          & 38.41          & 36.59          & 45.73          & 3.76 & 3.17 & \textbf{3.21} & 3.08 & 3.31 \\
    \xmark                                                                                              & \cmark                                    & 70.73                              & 39.63          & 39.02          & 40.24          & 47.41          & 3.90 & 3.17 & 3.08 & 3.11 & 3.31 \\
    \cmark                                                                                              & \cmark                                    & \textbf{79.88}                     & \textbf{45.73} & \textbf{43.90} & \textbf{42.68} & \textbf{53.05} & \textbf{3.96} & \textbf{3.21} & 3.18 & \textbf{3.19} & \textbf{3.38} \\
    \bottomrule
  \end{tabular}%
  }
  \caption{The ablation study of different methods across four optimization levels
  (O0, O1, O2, O3), as well as their average scores (AVG). The results in bold represent the optimal performance. The ~\labelemoji~ and ~\toolemoji~ means Relabedling and Function Call. \textbf{Bold} denotes the best performance.}
  \label{tab:ablation}
\end{table*}



\begin{figure*}[ht]
    \centering
    \begin{minipage}{0.65\textwidth}
        \centering
        \includegraphics[width=0.95\linewidth]{figs/ablation.pdf}
        \vspace{-2mm}
        \captionof{figure}{Qualitative Comparison for different design choices. Our method, employing multi-view epipolar attention, demonstrates the best consistency.}
        \label{fig:ablation}
    \end{minipage}\hfill
    \begin{minipage}{0.33\textwidth}
        \centering
        \includegraphics[width=0.8\linewidth]{figs/neus_ver.pdf}
        \vspace{-3mm}
        \caption{Our method shows better direct 3D reconstruction~\cite{neus}.}
        \label{fig:neus}
    \end{minipage}
    \vspace{-5mm}
\end{figure*}

\noindent\textbf{Multi-view Consistency.}
Tab.~\ref{tab:view16_fxied_compare} presents the 3D consistency scores compared to our baseline model (Zero123) and SyncDreamer. The results indicate a significant improvement across all three metrics achieved by our method when compared with Zero123.
While our method exhibits a marginally lower numerical consistency score compared to SyncDreamer, it enables the synthesis of images with arbitrary camera poses.	
This capability is illustrated in Tab.~\ref{tab:view16_free_compare}, where our method consistently enhances consistency with changes in camera pose settings, whereas SyncDreamer fails to do so and exhibits inferior results compared to Zero123.
Furthermore, our method facilitates the synthesis of multi-view images with any number of camera views. This versatility is demonstrated in Tab.~\ref{tab:view32_free_compare}, where our method continues to achieve significant improvements in consistency scores, while SyncDreamer is unable to operate under such conditions.	

Meanwhile, Fig.~\ref{fig:sota_compare} provides a qualitative comparison with the baseline. While both our method and SyncDreamer enhance consistency, our method visually preserves better similarity to the input image, including color and texture details. The input consistency score further corroborates this.

\noindent\textbf{Image Quality.}
While our primary goal centers around enhancing the consistency of synthesized multi-view images, we also evaluate the image quality by comparing the similarity with the ground truth images. The results shown in Tab.~\ref{tab:view16_free_compare}, Tab.~\ref{tab:view16_fxied_compare}, and Tab.~\ref{tab:view32_free_compare} indicate that our method also enhances the image quality under different settings besides improving the consistency.
Moreover, our method shows better image quality compared with SyncDreamer even in the 16-view setting with fixed camera pose.

\noindent\textbf{Input Consistency.}
Input consistency terms whether the results align with the input image.
Fig.~\ref{fig:sota_compare} illustrates that both our method and SyncDreamer enhance multi-view consistency. However, the color and texture details of SyncDreamer's results diverge from the input image and appear visually unnatural.
This discrepancy is evident in the input consistency score presented in Tab.~\ref{tab:view16_fxied_compare}, indicating lower similarity with the condition image in the SyncDreamer results.	

\subsection{Ablation Study}
The overall quantitative results are shown in Tab.~\ref{tab:ablation}, and the qualitative comparisons are shown in Fig.~\ref{fig:ablation}.

\noindent \textbf{Full Attention \vs Epipolar Attention.}
The results presented in Tab.\ref{tab:ablation} and Fig.\ref{fig:ablation} demonstrate that our epipolar attention mechanism can synthesize more consistent multi-view images compared with full attention. Furthermore, our epipolar attention achieves a greater performance improvement compared to full attention when using multiple reference images. This could be attributed to the fact that our epipolar attention more effectively localizes target information, as depicted in Fig.~\ref{fig:full_attn_compare}, thereby reducing noise from the reference images. In the multi-view setting, where multiple reference images are utilized, this noise reduction becomes particularly crucial.
Moreover, it is noteworthy that the epipolar attention mechanism consumes less GPU memory compared to our baseline, as discussed in Sec.~\ref{sec:attn_analysis}.

\noindent \textbf{Attending Single-View \vs Multi-View.}
Applying the epipolar attention significantly improves the consistency between the input and target views. However, the consistency between different views in the unobserved regions of the input view is not well preserved.
After implementing our epipolar attention in the multi-view setting, the consistency across the generated multi-view images is further improved. The last row in Tab.~\ref{tab:ablation} shows that after applying our multi-view epipolar attention, the consistency score is further improved compared with the single-view setting. Besides, the qualitative result in Fig.~\ref{fig:ablation} also shows better consistency among different target views.



\begin{table}[t]
\centering
\vspace{-1mm}
\caption{Comparison of 3D reconstruction results. Our method significantly improves the reconstruction quality.}
\vspace{-3mm}
\label{tab:neus}
\scalebox{0.7}{
\begin{tabular}{c cc}
\toprule
              &  Chamfer Dist.$\downarrow$  & Volume IoU$\uparrow$
\\ \midrule

            Zero123         & 0.017         & 0.819    \\
            SyncDreamer     & \best{0.013}         & \best{0.847}    \\
            Ours            & 0.014	& 0.842 \\

\bottomrule
\end{tabular}
}
\vspace{-5mm}
\end{table}


\vspace{-2mm}
\subsection{Downstream Application}
\vspace{-2mm}
To demonstrate the effectiveness of our method, we also applied it to the downstream 3D reconstruction task. Specifically, we trained the NeuS model~\cite{neus} directly using images synthesized by our method, Zero123, and SyncDreamer, respectively.
The quantitative results in Tab.~\ref{tab:neus} show that the consistent multi-view images synthesized by our method can significantly improve the 3D reconstruction quality.
Additionally, our method exhibits similar performance to SyncDreamer which requires time-consuming re-training.
The qualitative results in Fig.~\ref{fig:neus} show that it is challenging to train the NeuS model directly due to the lack of consistency in the images generated by Zero123. In contrast, our method generates more consistent multi-view images and, therefore, better reconstructs the geometry and texture details.
We show improvements on other downstream applications such as image-to-3D in the Supplementary Material.



\section{Conclusion}

%In this paper, w
We propose a new PEFT method called DiffoRA, which enables efficient and adaptive LLM fine-tuning based on LoRA. 
Instead of adjusting every interior rank, 
%of the decomposition matrices 
%of all modules, 
we argue that adopting LoRA module-wisely is sufficient. 
To achieve this, we construct a DAM to select the modules that are most suitable and essential to fine-tune. We theoretically analyze how the DAM impacts the convergence rate and generalization capability.
%of the pre-trained model. 
Furthermore, we adopt continuous relaxation and discretization to establish DAM.
%for each task. 
To alleviate the issue of discretization discrepancy, we utilize the weight-sharing strategy for optimization. 
%We fully implement our method and t
The experimental results demonstrate that our DiffoRA works consistently better than the baselines across all benchmarks. 

% \endgroup

% \bibliographystyle{natbib} 
\bibliography{main}
\bibliographystyle{icml2025}
	
	
%%%%%%%%%%%%%%%%%%%%%%%%%%%%%%%%%%%%%%%%%%%%%%%%%%%%%%%%%%%%%%%%%%%%%%%%%%%
	


\newpage
\appendix
\onecolumn


\section*{Contents}

{\footnotesize
\hyperref[app:related_work]{\textbf{\ref{app:related_work}}.
Additional Literature Review}
\dotfill
\pageref{app:related_work}

%\hyperref[app:add_stat_results]
%{\textbf{\ref{app:add_stat_results}}.
%Additional Statistical Results}
%\dotfill
%\pageref{app:add_stat_results}

\hyperref[app:proof:main]
{\textbf{\ref{app:proof:main}}.
Proof of Main Results}
\dotfill
\pageref{app:proof:main}

~~~~\hyperref[sec:proof:thm:grad]
{\textbf{\ref{sec:proof:thm:grad}}.
Optimization Considerations: Proof of Theorem~\ref{thm:grad}}
\dotfill
\pageref{sec:proof:thm:grad}

~~~~~~~~\hyperref[sec:proof:thm:grad_1]
{\textbf{\ref{sec:proof:thm:grad_1}}.
Building Blocks}
\dotfill
\pageref{sec:proof:thm:grad_1}

~~~~~~~~\hyperref[sec:proof:thm:grad_2]
{\textbf{\ref{sec:proof:thm:grad_2}}.
Derivation of Theorem~\ref{thm:grad}}
\dotfill
\pageref{sec:proof:thm:grad_2}

~~~~~~~~\hyperref[sec:proof:thm:grad_3]
{\textbf{\ref{sec:proof:thm:grad_3}}.
Proof of Claim~\eqref{eq:responsedistravg}}
\dotfill
\pageref{sec:proof:thm:grad_3}

~~~~\hyperref[sec:proof:thm:stat]
{\textbf{\ref{sec:proof:thm:stat}}.
Statistical Considerations}
\dotfill
\pageref{sec:proof:thm:stat}

~~~~~~~~\hyperref[sec:proof:thm:asymp_full]
{\textbf{\ref{sec:proof:thm:asymp_full}}.
Proof of Lemma~\ref{thm:asymp_full} (Theorem~\ref{thm:asymp_full_full})}
\dotfill
\pageref{sec:proof:thm:asymp_full}

~~~~~~~~\hyperref[sec:proof:thm:asymp]
{\textbf{\ref{sec:proof:thm:asymp}}.
Proof of Theorem~\ref{thm:asymp}}
\dotfill
\pageref{sec:proof:thm:asymp}

~~~~~~~~\hyperref[sec:proof:lemma:hess_scalarvalue]
{\textbf{\ref{sec:proof:lemma:hess_scalarvalue}}.
Proof of Theorem~\ref{lemma:hess_scalarvalue}}
\dotfill
\pageref{sec:proof:lemma:hess_scalarvalue}

\hyperref[app:aux]
{\textbf{\ref{app:aux}}.
Proof of Auxiliary Results}
\dotfill
\pageref{app:aux}

~~~~\hyperref[sec:proof:aux:thm:grad]
{\textbf{\ref{sec:proof:aux:thm:grad}}.
Proof of Auxiliary Results for Theorem~\ref{thm:grad}}
\dotfill
\pageref{sec:proof:aux:thm:grad}

~~~~~~~~\hyperref[sec:proof:lemma:grad_policy]
{\textbf{\ref{sec:proof:lemma:grad_policy}}.
Gradients of Policy $\policytheta$ and Reward $\rewardtheta$ }
\dotfill
\pageref{sec:proof:lemma:grad_policy}

~~~~~~~~\hyperref[sec:proof:lemma:grad_scalarvalue]
{\textbf{\ref{sec:proof:lemma:grad_scalarvalue}}.
Proof of Lemma~\ref{lemma:grad_scalarvalue}, Explicit Form of Gradient $\gradtheta \scalarvalue(\policytheta)$ }
\dotfill
\pageref{sec:proof:lemma:grad_scalarvalue}

~~~~~~~~\hyperref[sec:proof:lemma:grad_loss]
{\textbf{\ref{sec:proof:lemma:grad_loss}}.
Proof of Lemma~\ref{lemma:grad_loss}, Explicit Form of Gradient $\gradtheta \Loss(\paratheta)$}
\dotfill
\pageref{sec:proof:lemma:grad_loss}

~~~~\hyperref[sec:proof:thm:asymp_aux]
{\textbf{\ref{sec:proof:thm:asymp_aux}}.
Proof of Auxiliary Results for Theorem~\ref{thm:asymp}}
\dotfill
\pageref{sec:proof:thm:asymp_aux}

~~~~~~~~\hyperref[sec:proof:eq:master_cond_proof]
{\textbf{\ref{sec:proof:eq:master_cond_proof}}.
Proof of Condition~\eqref{eq:master_cond_proof}}
\dotfill
\pageref{sec:proof:eq:master_cond_proof}

~~~~~~~~\hyperref[sec:proof:lemma:hess_loss]
{\textbf{\ref{sec:proof:lemma:hess_loss}}.
Proof of Lemma~\ref{lemma:hess_loss}, Explicit Form of Hessian $\hesstheta \Loss(\parathetastar)$}
\dotfill
\pageref{sec:proof:lemma:hess_loss}

~~~~~~~~\hyperref[sec:proof:lemma:grad_loss_stat]
{\textbf{\ref{sec:proof:lemma:grad_loss_stat}}.
Proof of Lemma~\ref{lemma:grad_loss_stat}, Asymptotic Distribution of Graident $\gradtheta \Losshat(\parathetastar)$ }
\dotfill
\pageref{sec:proof:lemma:grad_loss_stat}

~~~~\hyperref[sec:proof:lemma:hess_scalarvalue_aux]
{\textbf{\ref{sec:proof:lemma:hess_scalarvalue_aux}}.
Proof of Auxiliary Results for Theorem~\ref{lemma:hess_scalarvalue}}
\dotfill
\pageref{sec:proof:lemma:hess_scalarvalue_aux}

~~~~~~~~\hyperref[sec:proof:eq:hessscalarvalue]
{\textbf{\ref{sec:proof:eq:hessscalarvalue}}.
Proof of Equation~\eqref{eq:hessscalarvalue} from Theorem~\ref{lemma:hess_scalarvalue}, Explicit Form of Hessian $\hesstheta \scalarvalue(\policystar)$}
\dotfill
\pageref{sec:proof:eq:hessscalarvalue}

~~~~~~~~\hyperref[sec:proof:gap_distr]
{\textbf{\ref{sec:proof:gap_distr}}.
Proof of the Asymptotic Distribution in Equation~\eqref{eq:gap_distr}}
\dotfill
\pageref{sec:proof:gap_distr}

~~~~~~~~\hyperref[sec:proof:chisqtail]
{\textbf{\ref{sec:proof:chisqtail}}.
Proof of the Tail Bound in Equation~\eqref{eq:gap_bd}}
\dotfill
\pageref{sec:proof:chisqtail}

\hyperref[sec:master]
{\textbf{\ref{sec:master}}.
Supporting Theorem: Master Theorem for $Z$-Estimators}
\dotfill
\pageref{sec:master}

\hyperref[app:experiment]
{\textbf{\ref{app:experiment}}.
Experimental Details}
\dotfill
\pageref{app:experiment}

\hyperref[app:extension]
{\textbf{\ref{app:extension}}.
Extension to Proximal Policy Optimization (PPO)}
\dotfill
\pageref{app:extension}

}





\section{Additional Literature Review}\label{app:related_work}

\textbf{RLHF}. RLHF has emerged as a cornerstone methodology for aligning large language models with human values and preferences \citep{achiam2023gpt}. Early systems \citep{ouyang2022training} turn human preference data into reward modeling to optimize model behavior accordingly. DPO has been proposed as a more efficient approach that directly trains LLMs on preference data.  
As LLMs evolve during training, continuing training on pre-generated preference data becomes suboptimal due to the distribution shift. Empirically, RLHF is applied iteratively—generating on-policy data at successive stages to enhance alignment and performance \citep{touvron2023llama, bai2022training}. Similarly, researchers have introduced iterative DPO \citep{xiong2024iterative, xu2023some} and online DPO \citep{guo2024direct} to fully leverage online preference labeling. Ultimately, the quality of preference data play a critical role in determining the effectiveness of the alignment. 

\textbf{Sampling in Frontier LLMs}. Technical reports of Frontier LLMs briefly mention sampling techniques. For instance, Claude \citep{bai2022training} utilizes models from different training steps to generate responses, while Llama-2 \citep{touvron2023llama} further use different temperatures for sampling. However, no further details are provided, leaving the development of a principled method an open challenge. % \julia{Same: move to appendix!}

\textbf{Data Selection.} There is a line of research aimed at improving sample efficiency for preference labeling by selecting question and response pairs. \citet{scheid2024optimal} conceptualize this as a regret minimization problem, leveraging methods from linear dueling bandits. \citet{das2024active, mehta2023sample, muldrewactive, ji2024reinforcement} draw insights from active learning, using various uncertainty estimators to guide selection by prioritizing sample pairs with maximum uncertainty. These approaches focus directly on a dataset of questions and responses and are orthogonal to our work. 

\textbf{Other Changes in Response Sampling.} Several works also modify the sampling design directly \citep{liustatistical, dongraft}, but with the goal of improving policy network optimization based on a reward model, rather than enhancing the reward modeling itself. \citet{liustatistical} employ rejection sampling to approximate the response distribution induced by the reward model, thereby improving optimization. However, this approach requires access to the reward model and incurs higher computational and labeling costs. Similarly, \citet{dongraft} use best-of-N sampling with the reward model to generate high-quality data for supervised fine-tuning (SFT). We consider these approaches orthogonal to our work.


Additionally, \citet{cen2024value} introduce a bonus term in the policy learning phase of online RLHF to promote exploration in response sampling, which aligns with the optimism principle.


\iffalse

\section{Additional Statistical Results}\label{app:add_stat_results}


    {In addition to our analysis of T-PILAF in \Cref{sec:sampling}, here we present a generalized version of \Cref{thm:asymp} that applies to any response sampling distribution~$\responsedistr$. While not directly tied to the main focus of this work, this broader result may be of independent interest to readers.
    The proof of \Cref{thm:asymp_full} is provided in \Cref{sec:proof:thm:asymp_full}.
    \begin{lemma}
        \label{thm:asymp_full}
        For a general sampling distribution $\responsedistr$, the statement in \Cref{thm:asymp} remains valid with the matrix $\CovOpstar$ redefined as
        \begin{align}
            \CovOpstar \defn
            \Exp_{\prompt \sim \promptdistr, (\responseone, \, \responsetwo) \sim \responsedistravg(\cdot \mid \prompt)}
            \Big[ \, \weight(\prompt) \cdot \Var\big(\indicator\{\responseone = \responsewin\} \bigm| \prompt, \responseone, \responsetwo \big) \cdot \grad \, \grad^{\top} \Big] \, ,
            \label{eq:def_CovOpstar}
        \end{align} 
        where the expectation is taken over the distribution
        \begin{subequations}
            \begin{align}
                \label{eq:def_responsedistravg}
                \responsedistravg(\responseone, \responsetwo \mid \prompt) 
                \defn \frac{1}{2} \, \big\{ \responsedistr(\responseone, \responsetwo \mid \prompt) + \responsedistr(\responsetwo, \responseone \mid \prompt) \big\} \, .
            \end{align}
        The variance term is specified as
            \begin{align}
                & \Var\big(\indicator\{\responseone = \responsewin\} \mid \prompt, \responseone, \responsetwo \big)
                \label{eq:def_var}
                = \sigmoid\big( \rewardstar(\prompt, \responseone) - \rewardstar(\prompt, \responsetwo) \big) \, \sigmoid\big( \rewardstar(\prompt, \responsetwo) - \rewardstar(\prompt, \responseone) \big)
            \end{align}
        and the gradient difference $\grad$ is defined as
            \begin{align}
                \label{eq:def_grad}
                \grad \defn \gradtheta \rewardstar(\prompt, \responseone) - \gradtheta \rewardstar(\prompt, \responsetwo) \, .
            \end{align}
        \end{subequations}
    \end{lemma}
    
    The general form of the matrix $\CovOpstar$ offers valuable insights for designing a sampling scheme. To ensure $\CovOpstar$ is well-conditioned (less singular), we must balance two key factors when selecting responses $\responseone$ and $\responsetwo$:
    \vspace{-.8em}
    \begin{description} \itemsep = -.05em
        \item \emph{Large variance:} The variance in definition~\eqref{eq:def_var} should be maximized. This occurs when $\rewardstar(\prompt, \responseone) \approx \rewardstar(\prompt, \responsetwo)$. Intuitively, preference feedback is most informative when annotators compare responses of similar quality.
        \item \emph{Large gradient difference:} The gradient difference $\grad$ from definition~\eqref{eq:def_grad} should also be large. This requires responses with significantly different gradients. Only then can the comparison provide a clear and meaningful direction for model training.
    \end{description}
    }

    \fi






\section{Proofs for Section~\ref{sec:algo}}\label{app:proof}
% We first introduce a technical lemma which decompose the performance difference into value differences at each local state.
% \begin{lemma}\label{lem:value_diff}
% For any policies $\pi_1,\pi_2$ and $\pi$, the following equality holds
% \begin{align*}
% J(\pi_1,\pi)-J(\pi_2,\pi)=\E_{\pi_1}\bra{\sum_{h=1}^H \inner{\pi_1-\pi_2, Q^{\pi_2,\pi}}(s_h)}.
% \end{align*}
% \end{lemma}
% \begin{proof}
% For any state $s_h$, we have
% \begin{align*}
% V^{\pi_1,\pi}(s_h)-V^{\pi_2,\pi}(s_h)&=\inner{\pi_1,Q^{\pi_1,\pi}}(s_h)-\inner{\pi_2,Q^{\pi_2,\pi}}(s_h) \\
% &=\inner{\pi_1-\pi_2,Q^{\pi_2,\pi}}(s_h)+\inner{\pi_1,Q^{\pi_1,\pi}-Q^{\pi_2,\pi}}(s_h) \\
% &=\inner{\pi_1-\pi_2,Q^{\pi_2,\pi}}(s_h)+\E_{\pi_1}\bra{V^{\pi_1,\pi}(s_{h+1})-V^{\pi_2,\pi}(s_{h+1}) \mid s_h}.
% \end{align*}
% By recursively applying this to the initial state $s_1$, we obtain
% \begin{align*}
% J(\pi_1,\pi)-J(\pi_2,\pi)&=V^{\pi_1,\pi}(s_1)-V^{\pi_2,\pi}(s_1) \\
% &=\E_{\pi_1}\bra{\sum_{h=1}^H \inner{\pi_1-\pi_2, Q^{\pi_2,\pi}}(s_h)}+\E_{\pi_1}\bra{V^{\pi_1,\pi}(s_{H+1})-V^{\pi_2,\pi}(s_{H+1})} \\
% &=\E_{\pi_1}\bra{\sum_{h=1}^H \inner{\pi_1-\pi_2, Q^{\pi_2,\pi}}(s_h)}.
% \end{align*}
% The last equality is because $V^{\pi_1,\pi}(s_{H+1})=V^{\pi_2,\pi}(s_{H+1})=\Pcal(s_{H+1} \succ \pi)$ holds for any $s_{H+1}$.
% \end{proof}
% \subsection{Proof for Theorem~\ref{thm:omd_guarantee}}
% \begin{proof}
% First, according to the classical regret analysis of OMD~\citep{lattimore2020bandit}, we have for any policy $\pi$ and state $s_h$:
% \begin{align}
% \sum_{t=1}^T \inner{\pi-\pi_t,Q^{\pi_{t},\pi_t}}(s_h) &\le \frac{\KL(\pi(\cdot \mid s_h) \Vert \pi_1(\cdot \mid s_h))}{\eta}+\eta \sum_{t=1}^T \|Q^{\pi_t,\pi_t}(s_h,\cdot)\|^2_{\infty} \nonumber \\
% &\le \frac{D}{\eta}+\eta T =2\sqrt{TD}. \label{eq:omd_regret}
% \end{align}
% Then, we decompose the duality gap as:
% \begin{align*}
% \mathrm{DualGap}(\bar \pi)=\underbrace{\max_{\pi_1} J(\pi_1,\bar \pi)-\frac{1}{2}}_{\textrm{Term A}}+\underbrace{\frac{1}{2}-\min_{\pi_2}J(\bar \pi, \pi_2)}_{\textrm{Term B}}.
% \end{align*}
% Then we show how to bound Term A and Term B is bounded similarly due to the symmetric nature of the game. Let $\pi'=\argmax_{\pi_1} J(\pi_1,\bar \pi)$, we have
% \begin{align*}
% J(\pi',\bar \pi)-\frac{1}{2}&=\frac{1}{T}\sum_{t=1}^T J(\pi',\pi_t)-J(\pi_t,\pi_t) \\
% &=\frac{1}{T}\sum_{t=1}^T \E_{\pi'}\bra{\sum_{h=1}^H \inner{\pi'-\pi_t,Q^{\pi_t,\pi_t}}(s_h)} \\
% &=\frac{1}{T}\sum_{h=1}^H \E_{\pi'}\bra{\sum_{t=1}^T \inner{\pi'-\pi_t,Q^{\pi_t,\pi_t}}(s_h)} \\
% &\le\frac{2H\sqrt{D}}{\sqrt{T}}.
% \end{align*}
% The second equality is from Lemma \ref{lem:value_diff} and the inequality is from Eq.~\eqref{eq:omd_regret}. The proof is finished by also having $\frac{1}{2}-\min_{\pi_2}J(\bar \pi, \pi_2) \le \frac{2H\sqrt{D}}{\sqrt{T}}$.
% \end{proof}

\subsection{Proof for Theorem~\ref{thm:omd_guarantee}}\label{sec:proof_omd}
\begin{proof}
According to the regret analysis of OMD~\citep{lattimore2020bandit}, for any policy $\pi$, we have
\begin{align*}
\sum_{t=1}^T \inner{\pi,r_t}-\sum_{t=1}^T \inner{\pi_t,r_t} &\le \frac{\KL(\pi \Vert \pi_1)}{\eta}+\eta \sum_{t=1}^T \|r_t\|^2_{\infty} \\
&\le 2 \sqrt{TD}.
\end{align*}
The rest proof follows from Theorem 3 in~\citet{zhang2024iterative}.
\end{proof}

\subsection{Proof for Theorem~\ref{thm:onpo_regret}}\label{sec:proof_onpo}
\begin{proof}
Let $\psi(\pi)=\sum_{y} \pi(y) \log \pi(y)$, the KL divergence between $\pi_1$ and $\pi_2$ can also be written as the Bregman divergence term:
\begin{align*}
\KL(\pi_1 \Vert \pi_2)=D_{\psi}(\pi_1,\pi_2)=\psi(\pi_1)-\psi(\pi_2)-\inner{\nabla \psi(\pi_2),\pi_1-\pi_2}.
\end{align*}
Since $\psi$ is strongly convex with respect to $L_1$ norm, we can apply regret analysis from~\citet{rakhlin2013optimization,syrgkanis2015fast} and obtain that for any $\pi'$
\begin{align*}
\sum_{t=1}^T \inner{\pi'-\pi_t,r_t} \le \frac{\KL(\pi' \Vert \pi'_1)}{\eta}+\eta \sum_{t=1}^T \|r_t-r_{t-1}\|^2_{\infty}-\frac{1}{4 \eta}\sum_{t=2}^T \|\pi_t-\pi_{t-1}\|_1^2.
\end{align*}
We observe that for any $t \ge 2$ and any $y$, 
$$
|r_t(y)-r_{t-1}(y)|=|\sum_{y'} \mathbb{P}(y \succ y')(\pi_t(y)-\pi_{t-1}(y))| \le \|\pi_t-\pi_{t-1}\|_1.$$
Once we have $\frac{1}{4 \eta} \ge \eta$, the terms $\eta \|r_t-r_{t-1}\|^2_{\infty}$ and $-\frac{1}{4 \eta}\|\pi_t-\pi_{t-1}\|^2_1$ cancel out and we get
\begin{align*}
\sum_{t=1}^T \inner{\pi'-\pi_t,r_t} \le 2 \sqrt{D}.
\end{align*}
Next, we decompose the duality gap as:
\begin{align*}
\mathrm{DualGap}(\bar \pi)=\underbrace{\max_{\pi_1} J(\pi_1,\bar \pi)-\frac{1}{2}}_{\textrm{Term A}}+\underbrace{\frac{1}{2}-\min_{\pi_2}J(\bar \pi, \pi_2)}_{\textrm{Term B}}.
\end{align*}
We show how to bound Term A and Term B is bounded similarly due to the symmetric nature of the game. Let $\pi'=\argmax_{\pi_1} J(\pi_1,\bar \pi)$, we have
\begin{align*}
J(\pi',\bar \pi)-\frac{1}{2}&=\frac{1}{T}\sum_{t=1}^T J(\pi',\pi_t)-J(\pi_t,\pi_t) \\
&=\frac{1}{T}\sum_{t=1}^T \inner{\pi'-\pi_t,r_t} \\
&\le\frac{2\sqrt{D}}{T}.
\end{align*}
The proof is finished by also having $\frac{1}{2}-\min_{\pi_2}J(\bar \pi, \pi_2) \le \frac{2\sqrt{D}}{T}$.
\end{proof}


\section{Experimental Details}\label{app:experiment}

We implement our code based on the open-sourced OpenRLHF framework \citet{hu2024openrlhf}. We will open-source our code in the camera-ready version.

We use both the helpful and the harmless (HH) sets from HH-RLHF \citep{bai2022training} without additional data selection. We adopt the chat template from the Skywork-Reward-8B model \citep{liu2024skywork} to align with the reward template. This reward model, fine-tuned from Llama-3.1-8B, is used to simulate human preference labeling and matches our network trained for alignment.

For SFT, we apply full-parameter tuning with Adam for one epoch, using a cosine learning rate schedule, a 3\% warmup phase, a learning rate of $5\times 10^{-7}$, and a batch size of 256. These hyperparameters are adopted from \citet{hu2024openrlhf}. 

For all the DPO training in both iterative and online settings, we use full-parameter tuning with Adam but with two epochs. The learning rate, warmup schedules, and batch size are all the same. 

During generation, we limit the maximum number of new tokens to 896 and employ top$\_$p decoding with $p=0.95$ for all experiments. For Online DPO, we use a sampling temperature of 1.0, following \citet{guo2024direct}, while in Iterative DPO, we set the temperature to 0.7 to account for the off-policy nature of the data, following \citet{dong2024rlhf, shi2024crucial}.

Prompts are truncated to a maximum length of 512 tokens (truncated from the left if the length exceeds this limit) for SFT, DPO, and generation tasks. For SFT data, the maximum length is further restricted to 1024 tokens. When the combined length of the response and the (truncated) prompt exceeds 1024 tokens, the response is truncated from the right. These truncation practices align with the standard methodology described by \citet{rafailov2023direct}. In contrast, for DPO, responses are not further truncated, as we are already limiting the maximum tokens generated during the generation process.

When reproducing the \textit{Hybrid Sampling} baseline (Exploration Preference Optimization, XPO) from \citet{xie2024exploratory}, we use $\alpha=5\times 10^{-6}$ as suggested in the paper.

We do not include a comparison with \citet{shi2024crucial} and \citet{liu2024sample} in our experiments. While \citet{shi2024crucial} employs a sampling method similar to ours, their approach requires significantly more hyperparameters to tune, whereas our method involves no hyperparameter tuning. On the other hand, \citet{liu2024sample} relies on training an ensemble of 20 reward models to approximate the posterior. Their sampling method requires solving the argmax of these rewards, which is computationally intractable. As a workaround, they generate 20 samples and select the best one using best-of-N with $N=20$. This approach demands at least six times the computational resources compared to our method.

% The \textit{Best-of-N} method implicitly incorporates an exploration mechanism by selecting the best and worst samples based on internal rewards, which is conceptually similar to the exploration design in PILAF.

\subsection{Additional Results}

We present the full results for Online DPO with the overfitted initial policy, including a scatter plot in \cref{fig:online_dpo_special_full} and a summary of the objective values in \cref{tab:online_DPO_special}.

We observe that \textit{Vanilla Sampling} rapidly increases its KL divergence from the reference model while its reward improvement diminishes over time. In contrast, PILAF undergoes an early phase of training with fluctuating KL values but ultimately achieves a policy with higher reward and substantially lower KL divergence. We hypothesize that PILAF’s interpolation-based exploration enables it to escape the suboptimal region of the loss landscape where \textit{Vanilla Sampling} remains trapped. 

Conversely, \textit{Hybrid Sampling}, despite its explicit exploration design, remains biased by the policy model and continues to exhibit high KL values. While KL divergence decreases over training, the reward improvement remains limited. Meanwhile, \textit{Best-of-N Sampling} introduces an implicit exploration mechanism through internal DPO, which selects the best and worst responses, leading to wider coverage than \textit{Vanilla Sampling}. However, despite achieving a KL divergence similar to PILAF, it results in a lower reward. These findings highlight the superiority of PILAF sampling, demonstrating its effectiveness in robustly optimizing an overfitted policy.




\begin{figure}[htb]
  \centering
  \begin{minipage}[t]{0.49\textwidth} % [t] 表示顶部对齐
    \vspace{0pt} % 关键调整:将基线固定在顶部
    \centering
    \includegraphics[width=\linewidth]{figs/online_special_full.pdf}
    \caption{\textbf{Online DPO with an overfitted initial policy}. Full results of the \cref{fig:online_dpo_special}. Each dot represents an evaluation performed every 50 training steps. Color saturation indicates the training step, with darker colors representing later steps.}
    \label{fig:online_dpo_special_full}
  \end{minipage}
  \hfill
  \begin{minipage}[t]{0.49\textwidth} % [t] 表示顶部对齐
    \vspace{10pt} % 关键调整:将基线固定在顶部
    \centering
    \captionsetup{type=table}
    \caption{\textbf{Results of Online DPO with an overfitted initial policy.} We report the average reward, KL divergence from the reference model, and objective $\scalarvalue$ on the testset.}
    \vspace*{1.5em} % 调整表格与标题间距
    \begin{footnotesize}
    \begin{sc}
    \begin{tabular}{l|ccc}
    \toprule
        \textbf{Method} & Reward ($\uparrow$) & KL ($\downarrow$) & $\scalarvalue$ ($\uparrow$)\\ 
        \midrule
        \textit{Vanilla} & \underline{-3.95} & 39.85 & -7.93 \\
        \textit{Best-of-N} & -4.49 & {27.90}  & \underline{-7.28}\\
        \textit{Hybrid} & -6.00 & \textbf{18.20} & -7.82 \\
        \midrule
        \textit{PILAF} & \textbf{-3.54} & \underline{26.45} & \textbf{-6.19} \\
    \bottomrule
    \end{tabular}
    \end{sc}
    \end{footnotesize}
    \label{tab:online_DPO_special}
  \end{minipage}
\end{figure}

\section{Extension to Proximal Policy Optimization (PPO)}
\label{app:extension}

In this section, we briefly explore how the core principles of our PILAF sampling approach can be extended to PPO-based RLHF methods.

\paragraph{Integrating Response Sampling in InstructGPT:}

The PPO-based RLHF pipeline used in InstructGPT \citep{ouyang2022training} consists of three key steps: \vspace{-.8em}
\begin{enumerate} \itemsep = -.3em
    \item[(i)] Supervised Fine-Tuning (SFT) that produces the reference model $\policyref$.
    \item[(ii)] Reward Modeling (RM) by solving the optimization problem~\eqref{eq:RM_objective}, yielding an estimated reward function $\rewardtheta$.
    \item[(iii)] Reinforcement Learning Fine-Tuning, where the policy $\policyphi$ is optimized against the reward model $\rewardtheta$ using the Proximal Policy Optimization (PPO) algorithm, following the optimization scheme~\eqref{eq:policy_loss_with_rm}.
\end{enumerate}
\vspace{-.8em}
The key distinction between the PPO and DPO approaches lies in how the reward model $ \rewardtheta $ is represented—explicitly in PPO and implicitly in DPO.
In response sampling for data collection, it is crucial to consider the iterative nature of the InstructGPT pipeline. During each iteration, additional human-labeled data is collected for reward modeling (step~(ii)), and steps (ii) and (iii) are repeatedly applied to refine the model. Our proposed PILAF algorithm naturally integrates into this pipeline by improving the data collection process in step~(ii), thereby enhancing reward model training and, in turn, policy optimization.

\paragraph{Extensions of T-PILAF and PILAF:}
Extending our response sampling methods, PILAF and T-PILAF, to the PPO setup with an explicit $ \rewardtheta $ is both natural and straightforward.
\begin{itemize}
    \item Within the theoretical framework of T-PILAF, as introduced in \Cref{sec:sampling}, the only required modification is replacing $\policytheta$ with the language model $\policyphi$ and redefining the interpolated and extrapolated policies, $\policyphipos$ and $\policyphineg$, following the same formulation as in equations~\eqref{eq:def_policythetapos}~and~\eqref{eq:def_policythetaneg}.
    Specifically, we define
    \begin{subequations}
	\begin{align}
		\label{eq:def_policythetapos_PPO}
		\policyphipos(\response \mid \prompt)
		& := \frac{1}{\Partition^+(\prompt)} \; \policyphi(\response \mid \prompt)
		\exp \big\{ \rewardtheta(\prompt, \response) \big\} \, ,  \\[-1pt]
		\label{eq:def_policythetaneg_PPO}
		\policyphineg(\response \mid \prompt)
		& := \frac{1}{\Partition^-(\prompt)} \, \policyphi(\response \mid \prompt) \,
		\exp \big\{ - \rewardtheta(\prompt, \response) \big\},
	\end{align}
    \end{subequations}
    where $\rewardtheta$ is now explicitly produced by a reward network, rather than being implicitly derived from $\policyphi$, as in equation~\eqref{eq:def_reward}.
    \item To extend our empirical PILAF algorithm, as described in \Cref{sec:sampling_exp}, we propose applying the same interpolation and extrapolation techniques directly to the logits of the language models $\policyphi$ and $\policyref$.
    In particular, we take
    \begin{align*}
        & \policyphipos(\cdot \mid \prompt, \tokenttot{1}{t-1}) \; = \; \softmax\Big( \big\{ (1 + \parabeta) \, \headphi - \parabeta \, \headref\big\} (\prompt, \tokenttot{1}{t-1}) \Big), \\
        & \policyphineg(\cdot \mid \prompt, \tokenttot{1}{t-1}) \; = \; \softmax\Big( \big\{ (1 - \parabeta) \, \headphi + \parabeta \, \headref\big\} (\prompt, \tokenttot{1}{t-1}) \Big),
    \end{align*}
    where $\headphi$ and $\headref$ represent the logits of the language models $\policyphi$ and $\policyref$, respectively.
\end{itemize}

\paragraph{Adaption of Theoretical Analysis:}
Our theoretical analyses can be extended to the PPO framework, assuming that the optimization process~\eqref{eq:policy_loss_with_rm} in step~(iii) of InstructGPT is solved exactly. In this case, the policy satisfies~\mbox{$\policyphi = \policyt{\rewardtheta}$}, where
\begin{align*}
	\policyt{\rewardtheta}(\response \mid \prompt)
	\; \defn \; \frac{1}{\Partitiontheta(\prompt)} \, \policyref(\response \mid \prompt) \exp \Big\{ \frac{1}{\parabeta} \, \rewardtheta(\prompt, \response) \Big\} \, .
\end{align*}
Under this assumption, the output language model $\policyphi$ is implicitly a function of the parameter $\paratheta$.
Building on this, we can adapt our optimization and statistical analyses as follows:

\begin{itemize}
    \item {\bf Optimization Consideration:}
    Using the same argument as in \Cref{thm:grad}, we can prove that
    \begin{align*}
        \gradtheta \Loss(\paratheta) \; = \;
    - \, \Const' \cdot \gradtheta \scalarvalue(\policyphi) \, + \, \Term_2 \, ,
    \end{align*}
    where $\Const' > 0$ is a universal constant, and $\Term_2$ represents a second-order approximation error.
    
    In other words, if the policy optimization step is sufficiently accurate for the reward model $\rewardtheta$, then performing gradient descent on the MLE loss with respect to $\paratheta$ is equivalent to applying gradient ascent on the oracle objective $\scalarvalue$, following the steepest direction in the parameter space of $\paratheta$.
    \item {\bf Statistical Consideration:}
    Even with the new parameterization, the asymptotic distribution of $\parathetahat$ from \Cref{thm:asymp} remains unchanged. Moreover, the gradient and Hessian of $\scalarvalue$ with respect to $\paratheta$ retain the same form as in \Cref{thm:grad}. As a result, the statistical analysis extends naturally to PPO, allowing us to conclude that PILAF also maintains structure-invariant statistical efficiency for PPO methods.
\end{itemize}




\end{document}
