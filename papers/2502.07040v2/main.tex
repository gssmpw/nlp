%Version 3 October 2023
% See section 11 of the User Manual for version history
%
%%%%%%%%%%%%%%%%%%%%%%%%%%%%%%%%%%%%%%%%%%%%%%%%%%%%%%%%%%%%%%%%%%%%%%
%%                                                                 %%
%% Please do not use \input{...} to include other tex files.       %%
%% Submit your LaTeX manuscript as one .tex document.              %%
%%                                                                 %%
%% All additional figures and files should be attached             %%
%% separately and not embedded in the \TeX\ document itself.       %%
%%                                                                 %%
%%%%%%%%%%%%%%%%%%%%%%%%%%%%%%%%%%%%%%%%%%%%%%%%%%%%%%%%%%%%%%%%%%%%%

%%\documentclass[referee,sn-basic]{sn-jnl}% referee option is meant for double line spacing

%%=======================================================%%
%% to print line numbers in the margin use lineno option %%
%%=======================================================%%

%%\documentclass[lineno,sn-basic]{sn-jnl}% Basic Springer Nature Reference Style/Chemistry Reference Style

%%======================================================%%
%% to compile with pdflatex/xelatex use pdflatex option %%
%%======================================================%%

%%\documentclass[pdflatex,sn-basic]{sn-jnl}% Basic Springer Nature Reference Style/Chemistry Reference Style


%%Note: the following reference styles support Namedate and Numbered referencing. By default the style follows the most common style. To switch between the options you can add or remove “Numbered” in the optional parenthesis. 
%%The option is available for: sn-basic.bst, sn-vancouver.bst, sn-chicago.bst%  
 
%%\documentclass[sn-nature]{sn-jnl}% Style for submissions to Nature Portfolio journals
%%\documentclass[sn-basic]{sn-jnl}% Basic Springer Nature Reference Style/Chemistry Reference Style
\documentclass[sn-mathphys-num]{sn-jnl}% Math and Physical Sciences Numbered Reference Style 
%%\documentclass[sn-mathphys-ay]{sn-jnl}% Math and Physical Sciences Author Year Reference Style
%%\documentclass[sn-aps]{sn-jnl}% American Physical Society (APS) Reference Style
%%\documentclass[sn-vancouver,Numbered]{sn-jnl}% Vancouver Reference Style
%%\documentclass[sn-apa]{sn-jnl}% APA Reference Style 
%%\documentclass[sn-chicago]{sn-jnl}% Chicago-based Humanities Reference Style

%%%% Standard Packages
%%<additional latex packages if required can be included here>

\usepackage{graphicx}%
\usepackage{multirow}%
\usepackage{amsmath,amssymb,amsfonts}%
\usepackage{amsthm}%
\usepackage{mathrsfs}%
\usepackage[title]{appendix}%
\usepackage{xcolor}%
\usepackage{textcomp}%
\usepackage{manyfoot}%
\usepackage{booktabs}%
\usepackage{algorithm}%
\usepackage{algorithmicx}%
\usepackage{algpseudocode}%
\usepackage{listings}%
\usepackage{subcaption}
\usepackage{stmaryrd}
\usepackage{bm}
%%%%

\newcommand{\R}{\mathbb{R}}
\newcommand{\M}{\mathcal{M}}
\newcommand{\V}{\mathcal{V}}
\newcommand{\I}{\mathcal{I}}
\newcommand{\Po}{\bm{\mathcal P}}
\newcommand{\N}{\mathbb{N}}
\newcommand*\diff{\mathop{}\!\mathrm{d}}
\newcommand{\dott}[2]{\left\langle#1,#2\right\rangle}
\newcommand{\norm}[1]{\left\lVert #1\right\rVert}
\newcommand{\intervalle}[4]{\mathopen{#1}#2\mathclose{},#3\mathclose{#4}}
\newcommand{\ff}[2]{\intervalle{[}{#1}{#2}{]}}
\newcommand{\oo}[2]{\intervalle{(}{#1}{#2}{)}}
\newcommand{\of}[2]{\intervalle{(}{#1}{#2}{]}}
\newcommand{\fo}[2]{\intervalle{[}{#1}{#2}{)}}

\floatname{algorithm}{Algorithm}
\renewcommand{\algorithmicrequire}{\textbf{Input:}}
\renewcommand{\algorithmicensure}{\textbf{Output:}}
\algnewcommand{\algorithmicendif}{\textbf{end}}
\algblockdefx[IF]{If}{EndIf}[1]{\algorithmicif\ #1\ \algorithmicthen}{\algorithmicendif}
\algblockdefx[For]{for}{EndFor}[1]{\algorithmicif\ #1\ \algorithmicthen}{\algorithmicendif}


%%%%%=============================================================================%%%%
%%%%  Remarks: This template is provided to aid authors with the preparation
%%%%  of original research articles intended for submission to journals published 
%%%%  by Springer Nature. The guidance has been prepared in partnership with 
%%%%  production teams to conform to Springer Nature technical requirements. 
%%%%  Editorial and presentation requirements differ among journal portfolios and 
%%%%  research disciplines. You may find sections in this template are irrelevant 
%%%%  to your work and are empowered to omit any such section if allowed by the 
%%%%  journal you intend to submit to. The submission guidelines and policies 
%%%%  of the journal take precedence. A detailed User Manual is available in the 
%%%%  template package for technical guidance.
%%%%%=============================================================================%%%%

%% as per the requirement new theorem styles can be included as shown below
\theoremstyle{thmstyleone}%
\newtheorem{theorem}{Theorem}%  meant for continuous numbers
%%\newtheorem{theorem}{Theorem}[section]% meant for sectionwise numbers
%% optional argument [theorem] produces theorem numbering sequence instead of independent numbers for Proposition
%%\newtheorem{proposition}[theorem]{Proposition}% 
\newtheorem{proposition}{Proposition}
\newtheorem{assumption}{Assumption}
\newtheorem{lemma}{Lemma}
\newtheorem{corollary}{Corollary}

\theoremstyle{thmstyletwo}%
\newtheorem{example}{Example}%
\newtheorem{remark}{Remark}%

\theoremstyle{thmstylethree}%
\newtheorem{definition}{Definition}%

\raggedbottom
%%\unnumbered% uncomment this for unnumbered level heads


\begin{document}

\title[Article Title]{Robust high-order low-rank BUG integrators based on explicit Runge-Kutta methods}

%%=============================================================%%
%% GivenName	-> \fnm{Joergen W.}
%% Particle	-> \spfx{van der} -> surname prefix
%% FamilyName	-> \sur{Ploeg}
%% Suffix	-> \sfx{IV}
%% \author*[1,2]{\fnm{Joergen W.} \spfx{van der} \sur{Ploeg} 
%%  \sfx{IV}}\email{iauthor@gmail.com}
%%=============================================================%%

\author[1]{\fnm{Fabio} \sur{Nobile}}
\author*[1]{\fnm{S\'ebastien} \sur{Riffaud}}\email{sebastien.riffaud@epfl.ch}

\affil[1]{\orgdiv{CSQI Chair}, \orgname{\'Ecole Polytechnique F\'ed\'erale de Lausanne}, \city{1015 Lausanne}, \country{Switzerland}}

%%==================================%%
%% Sample for unstructured abstract %%
%%==================================%%

\abstract{In this work, we propose high-order basis-update \& Galerkin (BUG) integrators based on explicit Runge-Kutta methods for large-scale matrix differential equations. These dynamical low-rank integrators are high-order extensions of the BUG integrator \cite{ceruti2022rank} and are constructed by performing a BUG step at each stage of the Runge-Kutta method. In this way, the resulting Runge-Kutta BUG integrator is robust to the presence of small singular values and does not involve backward time-integration steps. We provide an error bound, which shows that the Runge-Kutta BUG integrator retains the order of convergence of the associated Runge-Kutta method until the error reaches a plateau corresponding to the low-rank truncation error and which vanishes as the rank becomes full. This error bound is finally validated experimentally on three numerical test cases. The results demonstrate the high-order convergence of the Runge-Kutta BUG integrator and its superior accuracy compared to other dynamical low-rank integrators proposed in the literature.}


\keywords{Dynamical low-rank approximation, Matrix differential equations, Basis-Update \& Galerkin integrators, Runge-Kutta methods}

%%\pacs[JEL Classification]{D8, H51}

%%\pacs[MSC Classification]{35A01, 65L10, 65L12, 65L20, 65L70}

\maketitle


\section{Introduction}
\label{sec:1}

Dynamical low-rank approximations (DLRAs) enable a significant reduction in the computational cost associated with numerical integration of large-scale matrix differential equations:
\begin{equation}
\label{eq:ode}
\dot{\mathbf A} = {\mathbf F}(t,{\mathbf A}), \qquad {\mathbf A}(0) = {\mathbf A}_0 \in \R^{n \times m},
\end{equation}
which appear in many applications, such as kinetic equations \cite{bernard2018reduced,einkemmer2018low,einkemmer2020low,einkemmer2021mass,coughlin2022efficient,einkemmer2024accelerating,einkemmer2024review} due to the large phase space, stochastic simulations  \cite{sapsis2009dynamically,babaee2017robust,musharbash2018dual,feppon2018dynamically,musharbash2020symplectic,patil2020real,kazashi2021existence,kazashi2021stability,kazashi2025dynamical} due to the repeated evaluation of the solution for different realizations of the random terms, or sequential parameter estimation \cite{kressner2011low,weinhandl2018linear,benner2024low,riffaud2024low} when tracking solutions for several parameter values. The main idea of DLRAs consists in approximating the solution ${\mathbf A}$ by a low-rank matrix ${\mathbf Y}$ in an SVD-like form:
\begin{equation}
\label{eq:sol}
{\mathbf Y}(t) = {\mathbf U}(t) {\mathbf S}(t) {\mathbf V}(t)^T \in \R^{n \times m},
\end{equation}
where ${\mathbf U} \in \R^{n \times r}$ and ${\mathbf V} \in \R^{m \times r}$ are orthonormal matrices, ${\mathbf S} \in \R^{r \times r}$ is a square invertible matrix (not necessarily diagonal), and $r \leq \min\{n,m\}$ is the rank of ${\mathbf Y}$.

A challenging issue concerns the time-integration of the different factors ${\mathbf U}$, ${\mathbf S}$, and ${\mathbf V}$. Perhaps the most intuitive approach is to consider the differential equations, derived in \cite{koch2007dynamical}, that describe the evolution of the low-rank factorization over time:
\begin{equation}
\label{eq:low-rank_ode}
\left\{
\begin{aligned}
\dot{\mathbf U} &= \left({\mathbf I} - {\mathbf U}{\mathbf U}^T \right) {\mathbf F}(t,{\mathbf Y}){\mathbf V}{\mathbf S}^{-1}\\
\dot{\mathbf S} &= {\mathbf U}^T {\mathbf F}(t,{\mathbf Y}){\mathbf V}\\
\dot{\mathbf V} &= \left({\mathbf I} - {\mathbf V}{\mathbf V}^T \right){\mathbf F}(t,{\mathbf Y})^T{\mathbf U}{\mathbf S}^{-T},
\end{aligned}
\right.
\end{equation}
where we have assumed that ${\mathbf U}^T\dot{\mathbf U}={\mathbf V}^T\dot{\mathbf V}={\mathbf 0}$. Unfortunately, this system involves the inverse of ${\mathbf S}$, which can cause severe time-step size restrictions if the singular values of ${\mathbf S}$ are small, since the step size of standard time-integration schemes must be proportional to the smallest nonzero singular value. An equivalent formulation to \eqref{eq:low-rank_ode} is obtained by projecting the matrix differential equation \eqref{eq:ode} onto the tangent space of the manifold $\M_r$ of rank-$r$ matrices:
\begin{equation}
\label{eq:projected_ode}
\dot{\mathbf Y} = \Po_{\mathbf Y}\big({\mathbf F}(t,{\mathbf Y})\big), \qquad {\mathbf Y}(0) = {\mathbf Y}_0 \in \M_r,
\end{equation}
where 
\begin{equation}
\label{eq:tangent-space_projection}
\Po_{\mathbf Y}({\mathbf X}) = {\mathbf U}{\mathbf U}^T{\mathbf X}-{\mathbf U}{\mathbf U}^T{\mathbf X}{\mathbf V}{\mathbf V}^T+{\mathbf X}{\mathbf V}{\mathbf V}^T
\end{equation}
stands for the orthogonal projection of ${\mathbf X} \in \R^{n \times m}$ onto the tangent space at ${\mathbf Y}={\mathbf U}{\mathbf S}{\mathbf V}^T \in \M_r$. In the projection-splitting integrator \cite{lubich2014projector,kieri2016discretized}, this tangent-space projection is split into three alternating subprojections using the Lie-Trotter or Strang splitting. The resulting low-rank integrator is robust to the presence of small singular values, but the step associated with the update of ${\mathbf S}$ integrates the solution backward in time, which can lead to instabilities for parabolic and hyperbolic problems \cite{kusch2023stability}.

In recent years, several dynamical low-rank integrators that are robust to the presence of small singular values and do not involve backward time-integration steps have been proposed:
\begin{itemize}
\item the basis-update \& Galerkin (BUG) integrators \cite{ceruti2022unconventional,ceruti2022rank,ceruti2024parallel}, where ${\mathbf U}$ and ${\mathbf V}$ are first updated using the tangent-space projection, and then ${\mathbf S}$ is updated using a Galerkin projection of the matrix differential equation \eqref{eq:ode} onto the updated bases ${\mathbf U}$ and ${\mathbf V}$;
\item the projected Runge-Kutta methods \cite{kieri2019projection,carrel2023projected}, in which the projected differential equation \eqref{eq:projected_ode} is integrated using a Runge-Kutta method which includes a truncation step that maintains low-rank approximations.
\end{itemize}
The convergence of the BUG integrator is limited to order one, but recent efforts are made to extend this dynamical low-rank integrator to higher orders. For instance, second-order extensions based on the midpoint rule have been proposed in \cite{ceruti2024robust,kusch2024second}. On the other hand, projected Runge-Kutta methods have high-order error bounds for time-explicit discretizations. However, these dynamical low-rank integrators do not share the conservation properties \cite{einkemmer2023conservation} offered by the Galerkin projection of the BUG integrator.

In this work, we propose high-order BUG integrators based on explicit Runge-Kutta methods. These dynamical low-rank integrators are high-order extensions of the BUG integrator and are constructed by performing a BUG step at each stage of the Runge-Kutta method. In this way, the resulting dynamical low-rank integrators are robust to the presence of small singular values and do not involve backward time-integration steps. Moreover, compared with projected Runge-Kutta methods, the only difference is that the tangent-space projection is replaced by the Galerkin projection of the BUG integrator. As a result, Runge-Kutta BUG integrators present two advantages. First, conservation properties can be preserved more easily. Second, the Galerkin projection of Runge-Kutta BUG integrators is more accurate than the tangent-space projection of projected Runge-Kutta methods for approximating ${\mathbf F}$ at the discrete level (see Proposition \ref{theo:P3}). Furthermore, we prove in Theorem \ref{theo:T2} that the Runge-Kutta BUG integrator retains the order of convergence of the associated Runge-Kutta method until the error reaches a plateau corresponding to the low-rank truncation error and which vanishes as the rank becomes full. In particular, this property holds for any explicit Runge-Kutta method, allowing in practice the construction of various high-order dynamical low-rank integrators.

The remainder of the paper is organized as follows. Section \ref{sec:2} introduces the first-order BUG integrator based on the forward Euler method. In Section \ref{sec:3}, we present high-order extensions of the BUG integrator based on explicit Runge-Kutta methods. Then, Section \ref{sec:4} analyzes the convergence of the proposed dynamical low-rank integrators. In Section \ref{sec:5}, the resulting error bound is validated experimentally on three numerical test cases. Finally, Section \ref{sec:6} draws some conclusions and perspectives.



\section{The BUG integrator based on the forward Euler method}
\label{sec:2}

The dynamical low-rank integrator proposed in this work is an extension of the first-order BUG integrator \cite{ceruti2022rank}. For the convenience of the reader, we start by presenting the latter in the time-explicit case, i.e. for the forward Euler method. Let the time be discretized using a fixed time-step size $h>0$. The integration of the rank-$r$ solution ${\mathbf Y}_k = {\mathbf U}_k {\mathbf S}_k {\mathbf V}_k^T$ from time $t_k$ to $t_k+h$ reads:
\begin{enumerate}
\item \textbf{K-step:} Assemble
\begin{equation}
\label{eq:k_step}
{\mathbf K} = \left[{\mathbf U}_k \enspace\; {\mathbf F}(t_k,{\mathbf Y}_k){\mathbf V}_k \right] \in \R^{n \times 2r},
\end{equation}
and compute $\widehat{\mathbf U}_{k+1} \in \R^{n \times \widehat r}$ with $\widehat r \in \{r,\ldots,\min\{n,m,2r\}\}$ as an orthonormal basis of the range of ${\mathbf K}$ (e.g., by QR decomposition), in short $\widehat{\mathbf U}_{k+1} = \textrm{ortho}({\mathbf K})$.
\item \textbf{L-step:} Assemble
\begin{equation}
\label{eq:s_step}
{\mathbf L} = \left[{\mathbf V}_k \enspace\; {\mathbf F}(t_k,{\mathbf Y}_k)^T{\mathbf U}_k \right] \in \R^{m \times 2r},
\end{equation}
and compute the augmented basis $\widehat{\mathbf V}_{k+1} = \textrm{ortho}({\mathbf L}) \in \R^{m \times \widehat r}$.
\item \textbf{S-step:}  Set
\begin{equation}
\label{eq:l_step}
\widehat{\mathbf S}_{k+1} = \widehat{\mathbf U}_{k+1}^T \bigl({\mathbf Y}_{k} + h {\mathbf F}(t_k,{\mathbf Y}_{k}) \bigr) \widehat{\mathbf V}_{k+1}  \in \R^{\widehat r \times \widehat r}.
\end{equation}
\item \textbf{Truncation step:} Let the $r$-truncated singular value decomposition (SVD) of $\widehat{\mathbf S}_{k+1}$ be ${\boldsymbol\Phi} {\boldsymbol\Sigma} {\boldsymbol\Psi}^T$, where ${\boldsymbol\Phi},{\boldsymbol\Psi}  \in \R^{\widehat r \times r}$ are orthonormal matrices and ${\boldsymbol\Sigma} \in \R^{r \times r}$ is a diagonal matrix with non-negative real numbers on the diagonal. Set
\begin{equation}
\label{eq:t_step}
\begin{aligned}
{\mathbf U}_{k+1} &= \widehat{\mathbf U}_{k+1} {\boldsymbol\Phi}, \\
{\mathbf S}_{k+1} &= {\boldsymbol\Sigma}, \\
{\mathbf V}_{k+1} &= \widehat{\mathbf V}_{k+1} {\boldsymbol\Psi}.
\end{aligned}
\end{equation}
\end{enumerate}
To simplify notation, let $\llbracket {\mathbf X} \rrbracket_r$ denote the projection of ${\mathbf X}$ onto $\M_r$, which is given by the $r$-truncated SVD of ${\mathbf X}$. The BUG integrator based on the forward Euler method is summarized in Algorithm \ref{al:bug}. Note that the rank $r$ has been fixed here for simplicity, but an adaptative rank can also be used to truncate the augmented solution $\widehat{\mathbf Y}_{k+1}=\widehat{\mathbf U}_{k+1}\widehat{\mathbf S}_{k+1}\widehat{\mathbf V}_{k+1}^T$.



\section{Extension of the BUG integrator to high-order explicit Runge-Kutta methods}
\label{sec:3}

We now extend the BUG integrator to high-order explicit Runge-Kutta methods. Let us consider the explicit Runge-Kutta method:
\begin{equation}
\label{eq:rk}
\begin{aligned}
{\mathbf A}_{k i} &= {\mathbf A}_{k} + h \sum_{j=1}^{i-1} a_{i j} {\mathbf F}(t_{k j},{\mathbf A}_{k j}), \quad i=1,\ldots,s, \\
{\mathbf A}_{k+1} &= {\mathbf A}_{k} + h \sum_{i=1}^s b_{i} {\mathbf F}(t_{k i},{\mathbf A}_{k i}),
\end{aligned}
\end{equation}
where $t_{k i} = t_k + c_i h$. The main idea is to perform one step of the BUG integrator at each stage of the Runge-Kutta method. To this end, the key point concerns the definition of the augmented bases. Consider, for example, the final stage with the initial value ${\mathbf Y}_{k}$:
\begin{equation*}
{\mathbf Y}_{k+1} = {\mathbf Y}_{k} + h \sum_{i=1}^s b_{i} {\mathbf F}(t_{k i},{\mathbf Y}_{k i}).
\end{equation*}
The augmented bases $\widehat{\mathbf U}_{k+1}$ and $\widehat{\mathbf V}_{k+1}$ are constructed here to represent exactly ${\mathbf Y}_{k}$ and the tangent-space projection of the different terms ${\mathbf F}(t_{k i},{\mathbf Y}_{k i})$. Notably, ${\mathbf Y}_{k}$ is represented on the left and right bases ${\mathbf U}_{k}$ and ${\mathbf V}_{k}$, while the tangent-space projection of ${\mathbf F}(t_{k i},{\mathbf Y}_{k i})$ is represented according to equation \eqref{eq:tangent-space_projection} on $[{\mathbf U}_{k i} \enspace\; {\mathbf F}_{k i}{\mathbf V}_{k i}]$ and $[{\mathbf V}_{k i} \enspace\; {\mathbf F}_{k i}^T{\mathbf U}_{k i}]$, where ${\mathbf F}_{k i}={\mathbf F}(t_{k i},{\mathbf Y}_{k i})$. Moreover, let us introduce the coefficients $\alpha_{i j}$ and $\beta_i$, defined as 
\begin{equation*}
\alpha_{i j} = \left \{
\begin{array}{l l}
1 & \textrm{if } a_{i j} \neq 0 \\
0 & \textrm{otherwise}
\end{array}
\right.
\qquad
\textrm{and}
\qquad
\beta_{i} = \left \{
\begin{array}{l l}
1 & \textrm{if } b_{i} \neq 0 \\
0 & \textrm{otherwise,}
\end{array}
\right.
\end{equation*}
which will allow us to discard the bases that are not involved in the different tangent-space projections. The augmented bases $\widehat{\mathbf U}_{k+1}$ and $\widehat{\mathbf V}_{k+1}$ are given by
\begin{equation*}
\begin{aligned}
\widehat{\mathbf U}_{k+1} &\gets \textrm{ortho}([{\mathbf U}_{k} \enspace\; \beta_{1} {\mathbf U}_{k 1} \enspace\; \beta_{1} {\mathbf F}_{k 1}{\mathbf V}_{k 1} \enspace\; \cdots \enspace\; \beta_{s} {\mathbf U}_{k s} \enspace\; \beta_{s} {\mathbf F}_{k s}{\mathbf V}_{k s}]), \\
\widehat{\mathbf V}_{k+1} &\gets \textrm{ortho}([{\mathbf V}_{k} \enspace\; \beta_{1} {\mathbf V}_{k 1} \enspace\; \beta_{1} {\mathbf F}_{k 1}^T{\mathbf U}_{k 1} \enspace\; \cdots \enspace\; \beta_{s}{\mathbf V}_{k s} \enspace\; \beta_{s}{\mathbf F}_{k s}^T {\mathbf U}_{k s}]),
\end{aligned}
\end{equation*}
where the terms $\beta_{1} {\mathbf U}_{k 1}$ and $\beta_{1} {\mathbf V}_{k 1}$ can be removed since ${\mathbf U}_{k 1}= {\mathbf U}_{k}$ and ${\mathbf V}_{k 1}= {\mathbf V}_{k}$. The Runge-Kutta BUG integrator associated with the Runge-Kutta method \eqref{eq:rk} is finally described in Algorithm \ref{al:rk_bug}. As mentioned previously, an adaptative rank can also be used to ensure, for example, that the error of the truncation step is smaller than a prescribed tolerance.

\begin{remark}
\label{theo:R1}
The rank of the augmented solution $\widehat{\mathbf Y}_{k+1}$ (resp. $\widehat{\mathbf Y}_{k,i+1}$) is at most $2rs$ (resp. $2ri$).
\end{remark}

\begin{remark}
\label{theo:R2}
When $r=\min\{n,m\}$, the Runge-Kutta BUG integrator is equivalent to the associated Runge-Kutta method, since the manifold $\M_r$ corresponds to the whole Euclidean space $\R^{n \times m}$, and there is therefore no projection or truncation error.
\end{remark}


\begin{algorithm}[H]
\caption{First-order BUG integrator \cite{ceruti2022rank} based on the forward Euler method}
\label{al:bug}
\begin{algorithmic}[1]
\Require {${\mathbf Y}_0$.}
\Ensure {${\mathbf Y}_{1},\ldots,{\mathbf Y}_{N}$.}
\For {$k=0,\ldots,N-1$}
\State ${\mathbf Y}_k := {\mathbf U}_{k} {\mathbf S}_{k} {\mathbf V}_{k}^T$;
\State ${\mathbf F}_k \gets {\mathbf F}(t_k,{\mathbf Y}_k)$;
\State $\widehat{\mathbf U}_{k+1} \gets \textrm{ortho}([{\mathbf U}_k \enspace\; {\mathbf F}_k {\mathbf V}_k])$; \Comment{K-step \eqref{eq:k_step}}
\State $\widehat{\mathbf V}_{k+1} \gets \textrm{ortho}([{\mathbf V}_k \enspace\; {\mathbf F}_k^T {\mathbf U}_k])$; \Comment{L-step \eqref{eq:l_step}}
\State $\widehat{\mathbf S}_{k+1} \gets \widehat{\mathbf U}_{k+1}^T \bigl( {\mathbf Y}_{k} + h {\mathbf F}_k \bigr) \widehat{\mathbf V}_{k+1}$; \Comment{S-step \eqref{eq:s_step}}
\State $\widehat{\mathbf Y}_{k+1} \gets \widehat{\mathbf U}_{k+1} \widehat{\mathbf S}_{k+1} \widehat{\mathbf V}_{k+1}^T$. 
\State ${\mathbf Y}_{k+1} \gets \llbracket \widehat{\mathbf Y}_{k+1} \rrbracket_r$. \Comment{Truncation step \eqref{eq:t_step}}
\EndFor
\end{algorithmic}
\end{algorithm}

\begin{algorithm}[H]
\caption{High-order explicit Runge-Kutta BUG integrator}
\label{al:rk_bug}
\begin{algorithmic}[1]
\Require {${\mathbf Y}_0$.}
\Ensure {${\mathbf Y}_{1},\ldots, {\mathbf Y}_{N}$.}
\For {$k=0,\ldots,N-1$}
\State ${\mathbf Y}_{k 1} \gets {\mathbf Y}_k$;
\For {$i=1,\ldots,s-1$}
\State ${\mathbf Y}_{k i} := {\mathbf U}_{k i} {\mathbf S}_{k i} {\mathbf V}_{k i}^T$;
\State ${\mathbf F}_{k i} \gets {\mathbf F}(t_k + c_i h,{\mathbf Y}_{k i})$;
\State $\widehat{\mathbf U}_{k, i+1} \gets \textrm{ortho}([{\mathbf U}_{k} \enspace\; \alpha_{i+1,1} {\mathbf F}_{k 1}{\mathbf V}_{k 1} \enspace\; \cdots \enspace\; \alpha_{i+1,i} {\mathbf U}_{k i} \enspace\; \alpha_{i+1,i} {\mathbf F}_{k i}{\mathbf V}_{k i}])$; 
\State $\widehat{\mathbf V}_{k,i+1} \gets \textrm{ortho}([{\mathbf V}_{k} \enspace\; \alpha_{i+1,1} {\mathbf F}_{k 1}^T{\mathbf U}_{k 1} \enspace\; \cdots \enspace\; \alpha_{i+1,i} {\mathbf V}_{k i} \enspace\; \alpha_{i+1,i} {\mathbf F}_{k i}^T{\mathbf U}_{k i}])$; 
\State $\widehat{\mathbf S}_{k,i+1} \gets \widehat{\mathbf U}_{k,i+1}^T \bigl( {\mathbf Y}_{k} + h ( a_{i+1,1} {\mathbf F}_{k 1} + \ldots + a_{i+1,i} {\mathbf F}_{k i} ) \bigr) \widehat{\mathbf V}_{k,i+1}$; 
\State $\widehat{\mathbf Y}_{k,i+1} \gets \widehat{\mathbf U}_{k,i+1} \widehat{\mathbf S}_{k,i+1} \widehat{\mathbf V}_{k,i+1}^T$;
\State ${\mathbf Y}_{k,i+1} \gets \llbracket \widehat{\mathbf Y}_{k,i+1} \rrbracket_r$; 
\EndFor
\State ${\mathbf Y}_{k s} := {\mathbf U}_{k s} {\mathbf S}_{k s} {\mathbf V}_{k s}^T$;
\State ${\mathbf F}_{k s} \gets {\mathbf F}(t_k + c_s h,{\mathbf Y}_{k s})$;
\State $\widehat{\mathbf U}_{k+1} \gets \textrm{ortho}([{\mathbf U}_{k} \enspace\; \beta_{1} {\mathbf F}_{k 1}{\mathbf V}_{k 1} \enspace\; \cdots \enspace\; \beta_{s} {\mathbf U}_{k s} \enspace\; \beta_{s} {\mathbf F}_{k s}{\mathbf V}_{k s}])$; 
\State $\widehat{\mathbf V}_{k+1} \gets \textrm{ortho}([{\mathbf V}_{k} \enspace\; \beta_{1} {\mathbf F}_{k 1}^T{\mathbf U}_{k 1} \enspace\; \cdots \enspace\; \beta_{s}{\mathbf V}_{k s} \enspace\; \beta_{s}{\mathbf F}_{k s}^T {\mathbf U}_{k s}])$; 
\State $\widehat{\mathbf S}_{k+1} \gets \widehat{\mathbf U}_{k+1}^T \bigl( {\mathbf Y}_{k} + h ( b_{1} {\mathbf F}_{k 1} + \ldots + b_{s} {\mathbf F}_{k s} ) \bigr) \widehat{\mathbf V}_{k+1}$; 
\State $\widehat{\mathbf Y}_{k+1} \gets \widehat{\mathbf U}_{k+1} \widehat{\mathbf S}_{k+1} \widehat{\mathbf V}_{k+1}^T$. 
\State ${\mathbf Y}_{k+1} \gets \llbracket \widehat{\mathbf Y}_{k+1} \rrbracket_r$. 
\EndFor
\end{algorithmic}
\end{algorithm}



\section{Convergence analysis}
\label{sec:4}

In this section, we analyze the convergence of the Runge-Kutta BUG integrator in the case where the rank $r$ is fixed and for a time-step size $h$ sufficiently small, say $h \leq h_0$ (see Theorem 3.4 of Chapter 2 in \cite{harrier1993solving} for the exact definition of $h_0$). The present analysis is based on two assumptions. The first assumption implies the existence and uniqueness of the exact solution ${\mathbf A}$ according to the Picard-Lindel\"of theorem, while the second assumption is needed in Theorem 3.1 of Chapter 2 in \cite{harrier1993solving} to analyze the local error of high-order Runge-Kutta methods. Furthermore, this analysis is done in the Frobenius norm $\norm{\cdot}_F$, and $\dott{\cdot}{\cdot}_F$ will stand for the Frobenius inner product in the following.

\medskip

\begin{assumption}
\label{theo:A1}
${\mathbf F}(t, {\mathbf X})$ is continuous in time and Lipschitz continuous in ${\mathbf X}$: there exists a Lipschitz constant $L>0$ (independent of $t$) such that
\begin{equation*}
\norm{{\mathbf F}(t,{\mathbf X}_1)-{\mathbf F}(t,{\mathbf X}_2)}_F \leq L \norm{{\mathbf X}_1-{\mathbf X}_2}_F
\end{equation*}
for all $t \in \ff{0}{T}$ and ${\mathbf X}_1,{\mathbf X}_2 \in \R^{n \times m}$.
\end{assumption}

\medskip

\begin{assumption}
\label{theo:A2}
Let ${\boldsymbol\Phi}^t_{\mathbf F}$ denote the exact flow of ${\mathbf F}$ (i.e., the mapping such that ${\mathbf A}(t) = {\boldsymbol\Phi}^t_{\mathbf F}({\mathbf A}_0)$). When considering a Runge-Kutta method of order $p$, we assume that the first $p$ derivatives
\begin{equation*}
\frac{\diff^{q}}{{\diff t}^{q}} {\boldsymbol\Phi}^t_{\mathbf F}({\mathbf X}), \quad 1 \leq q \leq p,
\end{equation*}
exist and are continuous for all ${\mathbf X} \in \R^{n \times m}$.
\end{assumption}


\subsection{Preliminary results}

Compared to \cite{ceruti2022rank}, we do not assume that ${\mathbf F}$ is bounded for all ${\mathbf X} \in \R^{n \times m}$. For this reason, we start by showing that ${\mathbf F}$ is bounded in a neighbourhood of the exact solution ${\mathbf Y}$, which can be deduced from Assumption \ref{theo:A1} and will be sufficient in the following. Additionally, we show that the exact flow ${\boldsymbol\Phi}^t_{\mathbf F}$ is Lipschitz continuous, which is a direct consequence of the Lipschitz continuity of ${\mathbf F}$.

\medskip

\begin{proposition}
\label{theo:P1}
Suppose that Assumption \ref{theo:A1} holds. The exact solution ${\mathbf Y}(t)$ of the projected differential equation \eqref{eq:projected_ode} is bounded by
\begin{equation}
\norm{{\mathbf Y}(t)-{\mathbf Y}_0}_F \leq \int_0^t e^{L(t-s)}\norm{{\mathbf F}(s,{\mathbf Y}_0)}_F \diff s
\end{equation}
for all $t \in \ff{0}{T}$.
\end{proposition}
\begin{proof}
Without loss of generality, assume that ${\mathbf Y}(t) \neq {\mathbf Y}_0$ for all $t \in \oo{0}{T}$. If ${\mathbf Y}(t) = {\mathbf Y}_0$ for certain times, consider independently the time subintervals where ${\mathbf Y}(t) \neq {\mathbf Y}_0$. Then, we deduce from Assumption \ref{theo:A1} the differential inequality
\begin{align*}
\norm{{\mathbf Y}(t)-{\mathbf Y}_0}_F \frac{\diff}{\diff t} \norm{{\mathbf Y}(t)-{\mathbf Y}_0}_F &= \frac 1 2 \frac{\diff}{\diff t} \norm{{\mathbf Y}(t)-{\mathbf Y}_0}_F^2 \\
&= \dott{{\mathbf Y}(t)-{\mathbf Y}_0}{\Po_{\mathbf Y}\big({\mathbf F}(t,{\mathbf Y}(t))\big)}_F \\
&\leq \norm{{\mathbf Y}(t)-{\mathbf Y}_0}_F \norm{\Po_{\mathbf Y}\big({\mathbf F}(t,{\mathbf Y}(t))\big)}_F \\
&\leq \norm{{\mathbf Y}(t)-{\mathbf Y}_0}_F \norm{{\mathbf F}(t,{\mathbf Y}(t))}_F \\
&\leq \norm{{\mathbf Y}(t)-{\mathbf Y}_0}_F \big(\norm{{\mathbf F}(t,{\mathbf Y}(t))-{\mathbf F}(t,{\mathbf Y}_0)}_F + \norm{{\mathbf F}(t,{\mathbf Y}_0)}_F \big) \\
&\leq \norm{{\mathbf Y}(t)-{\mathbf Y}_0}_F \big( L \norm{{\mathbf Y}(t)-{\mathbf Y}_0}_F + \norm{{\mathbf F}(t,{\mathbf Y}_0)}_F \big),
\end{align*}
which can be rewritten as
\begin{equation*}
\frac{\diff}{\diff t} \norm{{\mathbf Y}(t)-{\mathbf Y}_0}_F \leq L \norm{{\mathbf Y}(t)-{\mathbf Y}_0}_F + \norm{{\mathbf F}(t,{\mathbf Y}_0)}_F.
\end{equation*}
Finally, according to the Gr\"onwall's inequality, the exact solution ${\mathbf Y}(t)$ verifies
\begin{equation*}
\norm{{\mathbf Y}(t)-{\mathbf Y}_0}_F \leq \int_0^t e^{L(t-s)}\norm{{\mathbf F}(s,{\mathbf Y}_0)}_F \diff s,
\end{equation*}
which concludes the proof.
\end{proof}

\medskip

\begin{lemma}
\label{theo:L1}
Suppose that Assumption \ref{theo:A1} holds. Then, for $h \leq h_0$, the discrete solution ${\mathbf Y}_{k i}$ of the Runge-Kutta BUG integrator is bounded at each stage $i \in \{1,\ldots,s\}$ by
\begin{equation}
\norm{{\mathbf Y}_{k i}-{\mathbf Y}_0}_F \leq \big( 1+K_{i 0} h \big) \norm{{\mathbf Y}_{k}-{\mathbf Y}_0}_F + h \sum_{j=1}^{i-1} K_{i j}\norm{{\mathbf F}(t_{k j},{\mathbf Y}_0)}_F,
\end{equation}
where the constants $K_{i j} \geq 0$ are independent of $h$.
\end{lemma}
\begin{proof}
We proceed by induction. For $i=1$, the statement is trivial with $K_{1j}=0$ since ${\mathbf Y}_{k 1} = {\mathbf Y}_{k}$. Then, for $i \in \{2,\ldots,s\}$, the induction hypothesis yields the desired result:
\begin{equation*}
\begin{aligned}
\norm{{\mathbf Y}_{k i}-{\mathbf Y}_0}_F &= \norm{\left\llbracket \widehat{\mathbf Y}_{k i} \right\rrbracket_r - {\mathbf Y}_0}_F \\
&\leq \norm{\left\llbracket \widehat{\mathbf Y}_{k i} \right\rrbracket_r - \widehat{\mathbf Y}_{k i}}_F + \norm{\widehat{\mathbf Y}_{k i} - {\mathbf Y}_0}_F \\
&= \min_{{\mathbf X} \in \M_r} \norm{{\mathbf X} - \widehat{\mathbf Y}_{k i}}_F + \norm{\widehat{\mathbf Y}_{k i} - {\mathbf Y}_0}_F \\
&\leq \norm{{\mathbf Y}_{k} - \widehat{\mathbf Y}_{k i}}_F + \norm{\widehat{\mathbf Y}_{k,i} - {\mathbf Y}_0}_F \\
&\leq\norm{{\mathbf Y}_{k}-{\mathbf Y}_0}_F + 2h \sum_{j=1}^{i-1} |a_{i j}|  \norm{\widehat{\mathbf U}_{k i}\widehat{\mathbf U}_{k i}^T{\mathbf F}(t_{k j},{\mathbf Y}_{k j})\widehat{\mathbf V}_{k i}\widehat{\mathbf V}_{k i}^T}_F \\
&\leq \norm{{\mathbf Y}_{k}-{\mathbf Y}_0}_F + 2 h \sum_{j=1}^{i-1} |a_{i j}|  \norm{{\mathbf F}(t_{k j},{\mathbf Y}_{k j})}_F \\
&\leq \norm{{\mathbf Y}_{k}-{\mathbf Y}_0}_F + 2h \sum_{j=1}^{i-1} |a_{i j}| \big( \norm{{\mathbf F}(t_{k j},{\mathbf Y}_{k j})-{\mathbf F}(t_{k j},{\mathbf Y}_0)}_F + \norm{{\mathbf F}(t_{k j},{\mathbf Y}_0)}_F\big) \\
&\leq \norm{{\mathbf Y}_{k}-{\mathbf Y}_0}_F + 2h \sum_{j=1}^{i-1} |a_{i j}| \big( L \norm{{\mathbf Y}_{k j}-{\mathbf Y}_0}_F + \norm{{\mathbf F}(t_{k j},{\mathbf Y}_0)}_F\big) \\
&\leq \bigg( 1 + 2h \sum_{j=1}^{i-1} |a_{i j}| L (1+K_{j 0}h ) \bigg) \norm{{\mathbf Y}_{k}-{\mathbf Y}_0}_F \\
&\hphantom{\leq}+ 2h \sum_{j=1}^{i-1} |a_{i j}| \Big( \norm{{\mathbf F}(t_{k j},{\mathbf Y}_0)}_F + Lh \sum_{l=1}^{j-1} K_{j l} \norm{{\mathbf F}(t_{k l},{\mathbf Y}_0)}_F\Big) \\
&\leq \big( 1 + K_{i 0} h \big) \norm{{\mathbf Y}_{k}-{\mathbf Y}_0}_F + h \sum_{l=1}^{i-1} K_{i l} \norm{{\mathbf F}(t_{k l},{\mathbf Y}_0)}_F,
\end{aligned}
\end{equation*}
where
\begin{equation*}
\begin{aligned}
K_{i 0} &= 2L\sum_{j=1}^{i-1} |a_{i j}| (1+K_{j 0}h_0), \\
K_{i l} &= 2 |a_{i l}| + 2 L h_0 \sum_{j=l+1}^{i-1} |a_{i j}| K_{j l}, \quad l=1,\ldots,i-2, \\
K_{i,i-1} &= 2 |a_{i,i-1}|.
\end{aligned}
\end{equation*}
\end{proof}

\medskip

\begin{lemma}
\label{theo:L2}
Suppose that Assumption \ref{theo:A1} holds. Then, for $h \leq h_0$, the discrete solution ${\mathbf Y}_{k}$ of the Runge-Kutta BUG integrator is bounded by
\begin{equation}
\norm{{\mathbf Y}_{k+1}-{\mathbf Y}_0}_F \leq \big( 1+\widetilde K_{0} h \big) \norm{{\mathbf Y}_{k}-{\mathbf Y}_0}_F + h \sum_{i=1}^{s} \widetilde K_{i}\norm{{\mathbf F}(t_{k i},{\mathbf Y}_0)}_F,
\end{equation}
where the constants $\widetilde K_{i} >0$ are independent of $h$.
\end{lemma}
\begin{proof}
We proceed in the same way as in Lemma \ref{theo:L1}. For $k+1=0$, the statement is trivial. Then, for $k+1 \geq 1$, Lemma \ref{theo:L1} and the induction hypothesis yield the desired result:
\begin{equation*}
\begin{aligned}
\norm{{\mathbf Y}_{k+1}-{\mathbf Y}_0}_F &\leq \norm{{\mathbf Y}_k - \widehat{\mathbf Y}_{k+1}}_F + \norm{\widehat{\mathbf Y}_{k+1} - {\mathbf Y}_0}_F \\
&\leq \norm{{\mathbf Y}_{k}-{\mathbf Y}_0}_F + 2h \sum_{i=1}^{s} |b_{i}|  \norm{\widehat{\mathbf U}_{k s}\widehat{\mathbf U}_{k s}^T{\mathbf F}(t_{k i},{\mathbf Y}_{k i})\widehat{\mathbf V}_{k s}\widehat{\mathbf V}_{k s}^T}_F \\
&\leq \norm{{\mathbf Y}_{k}-{\mathbf Y}_0}_F + 2 h \sum_{i=1}^{s} |b_{i}|  \norm{{\mathbf F}(t_{k i},{\mathbf Y}_{k i})}_F \\
&\leq \norm{{\mathbf Y}_{k}-{\mathbf Y}_0}_F + 2h \sum_{i=1}^{s} |b_{i}| \big( \norm{{\mathbf F}(t_{k i},{\mathbf Y}_{k i})-{\mathbf F}(t_{k i},{\mathbf Y}_0)}_F + \norm{{\mathbf F}(t_{k i},{\mathbf Y}_0)}_F\big) \\
&\leq \norm{{\mathbf Y}_{k}-{\mathbf Y}_0}_F + 2h \sum_{i=1}^{s} |b_{i}| \big( L \norm{{\mathbf Y}_{k i}-{\mathbf Y}_0}_F + \norm{{\mathbf F}(t_{k i},{\mathbf Y}_0)}_F\big) \\
&\leq \bigg( 1 + 2 h \sum_{i=1}^{s} |b_{i}| L (1 + K_{i 0} h) \bigg) \norm{{\mathbf Y}_{k}-{\mathbf Y}_0}_F \\
&\hphantom{\leq}+ 2h \sum_{i=1}^{s} |b_{i}| \Big( \norm{{\mathbf F}(t_{k i},{\mathbf Y}_0)}_F + Lh \sum_{j=1}^{i-1} K_{i j} \norm{{\mathbf F}(t_{k j},{\mathbf Y}_0)}_F\Big) \\
&\leq \big( 1 + \widetilde K_{0} h \big) \norm{{\mathbf Y}_{k}-{\mathbf Y}_0}_F + h \sum_{l=1}^{s} \widetilde K_{l} \norm{{\mathbf F}(t_{k l},{\mathbf Y}_0)}_F,
\end{aligned}
\end{equation*}
where 
\begin{equation*}
\begin{aligned}
\widetilde K_{0} &= 2L\sum_{i=1}^{s} |b_{i}| (1+K_{i 0}h_0), \\
\widetilde K_{l} &= 2 |b_{l}| + 2 L h_0 \sum_{i=l+1}^{s} |b_{i}| K_{i l}, \quad l=1,\ldots,s-1, \\
\widetilde K_{s} &= 2 |b_{s}|.
\end{aligned}
\end{equation*}
\end{proof}



\begin{proposition}
\label{theo:P2}
Suppose that Assumption \ref{theo:A1} holds. Then, for $h \leq h_0$, the discrete solution of the Runge-Kutta BUG integrator is bounded by
\begin{equation}
\norm{{\mathbf Y}_{k}-{\mathbf Y}_0}_F \leq g^\ast_{k-1} \frac{ e^{kh \widetilde K_{0}}-1}{\widetilde K_{0}},
\end{equation}
where $g^\ast_k = \max\limits_{0 \leq l \leq k} g_l$ and $g_l = \sum_{i=1}^{s} \widetilde K_{i} \norm{{\mathbf F}(t_{l i},{\mathbf Y}_0)}_F$.
\end{proposition}
\begin{proof}
We start by showing by induction that
\begin{equation}
\label{eq:P2_1}
\norm{{\mathbf Y}_{k}-{\mathbf Y}_0}_F \leq h \sum_{l=0}^{k-1} (1+h \widetilde K_{0})^{k-1-l} g_l.
\end{equation}
For $k=0$, the statement is trivial. Then, for $k \geq 1$, Lemma \ref{theo:L2} and the induction hypothesis lead to the desired result
\begin{equation*}
\begin{aligned}
\norm{{\mathbf Y}_{k+1}-{\mathbf Y}_0}_F &\leq (1+h \widetilde K_{0}) \norm{{\mathbf Y}_{k}-{\mathbf Y}_0}_F + h  g_k \\
&\leq (1+h \widetilde K_{0}) \Big( h  \sum_{l=0}^{k-1} (1+h \widetilde K_{0})^{k-1-l} g_l \Big) + h  g_k \\
&= h  \sum_{l=0}^{k} (1+h \widetilde K_{0})^{k-l} g_l.
\end{aligned}
\end{equation*}
Finally, we deduce from equation \eqref{eq:P2_1} that
\begin{equation*}
\begin{aligned}
\norm{{\mathbf Y}_{k}-{\mathbf Y}_0}_F &\leq h g^\ast_{k-1} \sum_{l=0}^{k-1} (1+h \widetilde K_{0})^{k-1-l} \\
&= g^\ast_{k-1} \frac{(1+h \widetilde K_{0})^{k}-1}{\widetilde K_{0}} \\
&\leq g^\ast_{k-1} \frac{ e^{kh \widetilde K_{0}}-1}{\widetilde K_{0}}, \\
\end{aligned}
\end{equation*}
which concludes the proof.
\end{proof}

\medskip

According to Propositions \ref{theo:P1} and \ref{theo:P2}, the exact solution ${\mathbf Y}(t)$ and the discrete solution of the Runge-Kutta BUG integrator are bounded on the finite time-interval $0 \leq t \leq T$. Let
\begin{equation}
\V_r := \big\{{\mathbf X} \in \M_r \mid \norm{{\mathbf X}-{\mathbf Y}_0}_F \leq \max\{R_1,R_2\} \big\},
\end{equation}
where
\begin{equation*}
\begin{aligned}
R_1 &= \int_0^T e^{L(T-t)}\norm{{\mathbf F}(t,{\mathbf Y}_0)}_F \diff t, \\
R_2 &= \max_{1 \leq i \leq s} \bigg\{ \big( 1+K_{i0} h_0 \big) \bigg( \sum_{j=1}^{s} \widetilde K_{j} \bigg) \frac{ e^{T \widetilde K_{0}}-1}{\widetilde K_{0}} + h_0 \sum_{j=1}^{i-1} K_{ij} \bigg\} \times \sup_{t \in \ff{0}{T}} \norm{{\mathbf F}(t,{\mathbf Y}_0)}_F.
\end{aligned}
\end{equation*}
The subset $\V_r \subseteq \M_r$ is a neighbourhood of the exact solution ${\mathbf Y}(t)$ which contains the solution of the Runge-Kutta BUG integrator for all $t \in \ff{0}{T}$ and $h \leq h_0$. As a consequence, since ${\mathbf F}(t,{\mathbf X})$ is continuous in time and Lipschitz continuous in ${\mathbf X}$, ${\mathbf F}$ is bounded on the compact set $\ff{0}{T} \times \V_r$, and there exists a constant $B \geq 0$ such that
\begin{equation}
\label{eq:bound_solution}
\norm{{\mathbf F}(t,{\mathbf X})}_F \leq B
\end{equation}
for all $t \in \ff{0}{T}$ and ${\mathbf X} \in \V_r$.


\medskip

\begin{lemma}
\label{theo:L4}
Under Assumption \ref{theo:A1}, the exact flow is $e^{Lt}$-Lipschitz continuous:
\begin{equation}
\norm{{\boldsymbol\Phi}^{t}_{\mathbf F}({\mathbf X}_1)-{\boldsymbol\Phi}^{t}_{\mathbf F}({\mathbf X}_2)}_F \leq e^{Lt} \norm{{\mathbf X}_1-{\mathbf X}_2}_F
\end{equation}
for all $t \in \ff{0}{T}$ and ${\mathbf X}_1,{\mathbf X}_2 \in \R^{n \times m}$.
\end{lemma}
\begin{proof}
From Assumption \ref{theo:A1}, we obtain the differential inequality
\begin{align*}
\frac{\diff}{\diff t} \norm{{\boldsymbol\Phi}^{t}_{\mathbf F}({\mathbf X}_1)-{\boldsymbol\Phi}^{t}_{\mathbf F}({\mathbf X}_2)}_F^2 &= 2 \dott{{\boldsymbol\Phi}^{t}_{\mathbf F}({\mathbf X}_1)-{\boldsymbol\Phi}^{t}_{\mathbf F}({\mathbf X}_2)}{{\mathbf F}(t,{\boldsymbol\Phi}^{t}_{\mathbf F}({\mathbf X}_1))-{\mathbf F}(t,{\boldsymbol\Phi}^{t}_{\mathbf F}({\mathbf X}_2))}_F \\
&\leq 2 L \norm{{\boldsymbol\Phi}^{t}_{\mathbf F}({\mathbf X}_1)-{\boldsymbol\Phi}^{t}_{\mathbf F}({\mathbf X}_2)}_F^2
\end{align*}
and, according to the Gr\"onwall's inequality, the exact flow verifies
\begin{equation*}
\norm{{\boldsymbol\Phi}^{t}_{\mathbf F}({\mathbf X}_1)-{\boldsymbol\Phi}^{t}_{\mathbf F}({\mathbf X}_2)}^2_F \leq e^{2Lt} \norm{{\mathbf X}_1-{\mathbf X}_2}^2_F,
\end{equation*}
which concludes the proof.
\end{proof}



\subsection{Error estimation}

We now introduce several definitions and propositions that will be useful to prove the high-order convergence of the Runge-Kutta BUG integrator. In particular, Proposition \ref{theo:P3} shows that the projection error resulting from the Galerkin projection of the Runge-Kutta BUG integrator is not larger than the one resulting from the tangent-space projection, which will allow us to adapt the convergence analysis of projected Runge-Kutta methods \cite{kieri2019projection} to our Runge-Kutta BUG integrator. In addition, equation \eqref{eq:bound_truncation} provides an estimate for the truncation error that will allow us to obtain error bounds with an improved order of convergence compared to \cite{kieri2019projection}.

\medskip

\begin{proposition}
\label{theo:P3}
The Galerkin projection of the Runge-Kutta BUG integrator is more accurate than the tangent-space projection for approximating ${\mathbf F}_{k,i}$:
\begin{equation}
\label{eq:p3_1}
\norm{{\mathbf F}_{k i}-\widehat{\mathbf U}_{k+1}\widehat{\mathbf U}_{k+1}^T{\mathbf F}_{k i}\widehat{\mathbf V}_{k+1}\widehat{\mathbf V}_{k+1}^T}_F \leq \norm{{\mathbf F}_{k i}-\Po_{{\mathbf Y}_{k i}}\big({\mathbf F}_{k i}\big)}_F
\end{equation}
for all $i \in \{1,\ldots,s\}$ such that $b_i \neq 0$. Similarly, the projection error at each stage $i \in \{1,\ldots,s-1\}$ verifies
\begin{equation}
\label{eq:p3_2}
\norm{{\mathbf F}_{k j}-\widehat{\mathbf U}_{k,i+1}\widehat{\mathbf U}_{k,i+1}^T{\mathbf F}_{k j}\widehat{\mathbf V}_{k,i+1}\widehat{\mathbf V}_{k,i+1}^T}_F \leq \norm{{\mathbf F}_{k j}-\Po_{{\mathbf Y}_{k j}}\big({\mathbf F}_{k j}\big)}_F
\end{equation}
for all $j \in \{1,\ldots,i\}$ such that $a_{i+1,j} \neq 0$.
\end{proposition}
\begin{proof}
We prove equation \eqref{eq:p3_1}. The proof of equation \eqref{eq:p3_2} follows from the same arguments and is therefore omitted. Let the SVD of $\Po_{{\mathbf Y}_{k i}}\big({\mathbf F}_{k i}\big)$ be ${\mathbf \Phi}{\boldsymbol \Sigma}{\mathbf \Psi}^T$, where ${\mathbf \Phi} \in \R^{n \times \overline r}$ and ${\mathbf \Psi} \in \R^{m \times \overline r}$ are orthonormal matrices, ${\boldsymbol \Sigma} \in \R^{\overline r \times \overline r}$ is a diagonal matrix with non-negative real numbers on the diagonal, and $\overline r \in \{r,\ldots,\min\{n,m,2r\}\}$. If $b_i \neq 0$, then the augmented bases $\widehat{\mathbf U}_{k+1}$ and $\widehat{\mathbf V}_{k+1}$ contain by construction the range of $[{\mathbf U}_{k i} \enspace\; {\mathbf F}_{k i}{\mathbf V}_{k i}]$ and $[{\mathbf V}_{k i} \enspace\; {\mathbf F}_{k i}^T{\mathbf U}_{k i}]$, respectively, and it follows that the left and right singular vectors of $\Po_{{\mathbf Y}_{k i}}\big({\mathbf F}_{k i}\big)$ are included in the span of the augmented bases:
\begin{equation*}
{\mathbf \Phi} \subseteq \textrm{span}(\widehat{\mathbf U}_{k+1}) \quad \textrm{and} \quad {\mathbf \Psi} \subseteq \textrm{span}(\widehat{\mathbf V}_{k+1}),
\end{equation*}
where $\widehat{\mathbf U}_{k+1} \in \R^{n \times \widehat r}$, $\widehat{\mathbf V}_{k+1} \in \R^{m \times \widehat r}$, and $\widehat r \in \{\overline r,\ldots,\min\{n,m,2rs\}\}$. Consequently, the Galerkin projection onto the augmented bases verifies
\begin{equation*}
\begin{aligned}
\norm{{\mathbf F}_{k i}-\Po_{{\mathbf Y}_{k i}}\big({\mathbf F}_{k i}\big)}_F = & \norm{{\mathbf F}_{k i}-{\mathbf \Phi}{\boldsymbol \Sigma}{\mathbf \Psi}^T}_F \\
\geq & \min\limits_{\overline{\boldsymbol \Sigma} \in \R^{\overline r \times \overline r}} \norm{{\mathbf F}_{k i}-{\mathbf \Phi}\overline{\boldsymbol \Sigma}{\mathbf \Psi}^T}_F \\
\geq & \min\limits_{\widehat{\boldsymbol \Sigma} \in \R^{\widehat r \times \widehat r}} \norm{{\mathbf F}_{k i}-\widehat{\mathbf U}_{k+1}\widehat{\boldsymbol \Sigma}\widehat{\mathbf V}_{k+1}^T}_F \\
= & \norm{{\mathbf F}_{k i}-\widehat{\mathbf U}_{k+1}\widehat{\mathbf U}_{k+1}^T{\mathbf F}_{k i}\widehat{\mathbf V}_{k+1}\widehat{\mathbf V}_{k+1}^T}_F,
\end{aligned}
\end{equation*}
which concludes the proof.
\end{proof}

\medskip

Since ${\mathbf F}$ is bounded on $\ff{0}{T} \times \V_r$, the orthogonal projection of ${\mathbf F}$ onto the tangent space is also bounded on $\ff{0}{T} \times \V_r$, and there exits $\varepsilon_r \geq 0$ such that the tangent-space projection error is bounded on $\ff{0}{T} \times \V_r$. Let 
\begin{equation}
\label{eq:bound_projection}
\varepsilon_r := \sup_{t \in \ff{0}{T}} \sup_{{\mathbf X} \in \V_r} \norm{{\mathbf F}(t,{\mathbf X})-\Po_{\mathbf X}\big({\mathbf F}(t,{\mathbf X})\big)}_F.
\end{equation}
It follows from Proposition \ref{theo:P3} that the projection error of the Runge-Kutta BUG integrator is bounded by $\varepsilon_r$:
\begin{equation*}
\norm{{\mathbf F}_{k i}-\widehat{\mathbf U}_{k+1}\widehat{\mathbf U}_{k+1}^T{\mathbf F}_{k i}\widehat{\mathbf V}_{k+1}\widehat{\mathbf V}_{k+1}^T}_F \leq \varepsilon_r
\end{equation*}
for all $i \in \{1,\ldots,s\}$ such that $b_i \neq 0$, and
\begin{equation*}
\norm{{\mathbf F}_{k j}-\widehat{\mathbf U}_{k,i+1}\widehat{\mathbf U}_{k,i+1}^T{\mathbf F}_{k j}\widehat{\mathbf V}_{k,i+1}\widehat{\mathbf V}_{k,i+1}^T}_F \leq \varepsilon_r
\end{equation*}
for all $j \in \{1,\ldots,i\}$ such that $a_{i+1,j} \neq 0$. In addition, note that the tangent-space projection error and therefore $\varepsilon_r$ vanish when $r = \min\{n,m\}$.

\medskip

\begin{proposition}
\label{theo:P4}
Suppose that Assumption \ref{theo:A1} holds. Then, for $\ff{t_k}{t_{k+1}} \subseteq \ff{0}{T}$ and $h \leq h_0$, the error resulting from the truncation step of the Runge-Kutta BUG integrator is proportional to $h$: the ratio
\begin{equation}
\label{eq:p4_1}
\frac{1}{h} \norm{\widehat{\mathbf Y}_{k+1} - \left\llbracket \widehat{\mathbf Y}_{k+1} \right\rrbracket_r}_F
\end{equation}
is bounded independently of $h$. Similarly, at each stage $i \in \{1,\ldots,s-1\}$, the truncation error is proportional to $h$: the ratio
\begin{equation}
\label{eq:p4_2}
\frac{1}{h}\norm{\widehat{\mathbf Y}_{k,i+1} - \left\llbracket \widehat{\mathbf Y}_{k,i+1} \right\rrbracket_r}_F
\end{equation}
is bounded independently of $h$.
\end{proposition}
\begin{proof}
We prove equation \eqref{eq:p4_1}. The proof of equation \eqref{eq:p4_2} follows from the same arguments and is therefore omitted. Let $\sigma_j({\mathbf X})$ denote the $j$-th largest singular value of the matrix ${\mathbf X}\in \R^{n \times m}$. The squared truncation error is given according to the Eckart-Young theorem \cite{eckart1936approximation} by
\begin{equation}
\label{eq:l4_3}
\norm{\widehat{\mathbf Y}_{k+1} - \left\llbracket \widehat{\mathbf Y}_{k+1} \right\rrbracket_r}_F^2 = \sum\limits_{j=r+1}^{\min\{n,m,2rs\}} \sigma_j^2(\widehat{\mathbf Y}_{k+1}),
\end{equation}
since the rank of $\widehat{\mathbf Y}_{k+1}$ is at most $2rs$ (see Remark \ref{theo:R1}) and, consequently, $\sigma_j^2(\widehat{\mathbf Y}_{k+1})=0$ for $j > 2rs$. Then, as
\begin{equation*}
\widehat{\mathbf Y}_{k+1} = {\mathbf Y}_{k} + h \left( b_{1} \widetilde{\mathbf F}_{k 1} + \ldots + b_{s} \widetilde{\mathbf F}_{k s} \right) \quad \textrm{with}\quad \widetilde{\mathbf F}_{k i} = \widehat{\mathbf U}_{k+1}\widehat{\mathbf U}_{k+1}^T{\mathbf F}_{k i}\widehat{\mathbf V}_{k+1}\widehat{\mathbf V}_{k+1}^T,
\end{equation*}
the singular values of $\widehat{\mathbf Y}_{k+1}$ are bounded according to Theorem 3.3.16 in \cite{horn1991topics} by
\begin{equation*}
\sigma_{i+j-1}(\widehat{\mathbf Y}_{k+1}) \leq \sigma_i({\mathbf Y}_{k}) + h \sigma_j(b_{1} \widetilde{\mathbf F}_{k 1} + \ldots + b_{s} \widetilde{\mathbf F}_{k s})
\end{equation*}
for all $i \in \{1,\ldots,\min\{n,m,2r\}\}$ and $j \in \{1,\ldots,\min\{n,m,2rs\}\}$ such that $i+j-1\leq \min\{n,m,2rs\}$. In particular, for $i = r+1$ and $j\geq 1$, it follows that
\begin{equation}
\label{eq:l4_4}
\sigma_{r+j}(\widehat{\mathbf Y}_{k+1}) \leq h \sigma_j(b_{1} \widetilde{\mathbf F}_{k 1} + \ldots + b_{s} \widetilde{\mathbf F}_{k s}),
\end{equation}
since the rank of ${\mathbf Y}_{k}$ is at most $r$. Moreover, we deduce from equation \eqref{eq:bound_solution} that
\begin{equation*}
\begin{aligned}
\sum_{j=1}^{\min\{n,m,2rs\}} \sigma_j^2(b_{1} \widetilde{\mathbf F}_{k 1} + \ldots + b_{s} \widetilde{\mathbf F}_{k s}) &= \norm{b_{1} \widetilde{\mathbf F}_{k 1} + \ldots + b_{s} \widetilde{\mathbf F}_{k s}}_F^2 \\
 &\leq \Bigl( |b_{1}| \norm{\widetilde{\mathbf F}_{k 1}}_F + \ldots + |b_{s}| \norm{\widetilde{\mathbf F}_{k s}}_F \Bigr)^2 \\
&\leq  \Bigl( |b_{1}| \norm{{\mathbf F}_{k 1}}_F + \ldots + |b_{s}| \norm{{\mathbf F}_{k s}}_F \Bigr)^2 \\
&\leq  \Bigl( |b_{1}| B + \ldots + |b_{s}| B \Bigr)^2 \\
&= C_B^2 B^2
\end{aligned}
\end{equation*}
and, consequently,
\begin{equation}
\label{eq:l4_5}
\sigma_{j}^2(b_{1} \widetilde{\mathbf F}_{k 1} + \ldots + b_{s} \widetilde{\mathbf F}_{k s}) \leq C_B^2 B^2, \quad j \geq 1,
\end{equation}
where $C_B = \sum_{i=1}^s |b_{i}|$. Finally, combining equations \eqref{eq:l4_3}, \eqref{eq:l4_4}, and \eqref{eq:l4_5} yields
\begin{equation*}
\begin{aligned}
\norm{\widehat{\mathbf Y}_{k+1} - \left\llbracket \widehat{\mathbf Y}_{k+1} \right\rrbracket_r}_F^2 &= \sum\limits_{j=1}^{\min\{n,m,2rs\}-r} \sigma_{r+j}^2(\widehat{\mathbf Y}_{k+1}) \\
&\leq \sum\limits_{j=1}^{\min\{n,m,2rs\}-r} h^2 \sigma^2_j(b_{1} \widetilde{\mathbf F}_{k 1} + \ldots + b_{s} \widetilde{\mathbf F}_{k s}) \\
&\leq \sum\limits_{j=1}^{\min\{n,m,2rs\}-r} h^2 C_B^2 B^2 \\
&= h^2 (\min\{n,m,2rs\}-r) C_B^2 B^2.
\end{aligned}
\end{equation*}
The desired result follows directly from the fact that the term $(\min\{n,m,2rs\}-r) C_B^2 B^2$ is finite and does not depend on $h$.
\end{proof}

\medskip

According to Proposition \ref{theo:P4}, there exists $\gamma_r \geq0$ such that the ratios \eqref{eq:p4_1} and \eqref{eq:p4_2} are bounded. Let 
\begin{equation}
\label{eq:bound_truncation}
\begin{aligned}
\gamma_r := \max\Big\{ \sup_{t_{k,i+1} \in \ff{0}{T}} \sup_{{\mathbf X}_k \in \V_r} \frac{1}{h} \norm{\widehat{\mathbf X}_{k,i+1} - \left\llbracket \widehat{\mathbf X}_{k,i+1} \right\rrbracket_r}_F&, \\
\sup_{t_{k+1} \in \ff{0}{T}} \sup_{{\mathbf X}_k \in \V_r} \frac{1}{h} \norm{\widehat{\mathbf X}_{k+1} - \left\llbracket \widehat{\mathbf X}_{k+1} \right\rrbracket_r}_F& \Big\},
\end{aligned}
\end{equation}
where $\widehat{\mathbf X}_{k+1}$ and $\widehat{\mathbf X}_{k,i+1}$ denote the augmented solutions computed by the Runge-Kutta BUG integrator using the initial value ${\mathbf X}_{k}$ at time $t_k$. It follows that the truncation error is bounded by
\begin{equation*}
\begin{aligned}
\norm{\widehat{\mathbf Y}_{k,i+1} - \left\llbracket \widehat{\mathbf Y}_{k,i+1} \right\rrbracket_r}_F &\leq h \gamma_r, \quad i = 1,\ldots,s-1, \\
\norm{\widehat{\mathbf Y}_{k+1} - \left\llbracket \widehat{\mathbf Y}_{k+1} \right\rrbracket_r}_F &\leq h \gamma_r.
\end{aligned}
\end{equation*}
In addition, note that the truncation error and therefore $\gamma_r$ vanish when $r = \min\{n,m\}$.

\medskip

\begin{remark}
In general, $\gamma_r$ is not related to $\varepsilon_r$, but in some cases the truncation error can be bounded by using $\varepsilon_r$. For instance, according to the proof of Proposition \ref{theo:P4}, if
\begin{equation}
\label{eq:r3_1}
\sigma_1(\widetilde{\mathbf F}_{k i}) \leq \varepsilon_r, \quad \forall t_{k,i} \in \ff{0}{T},
\end{equation}
then it holds
\begin{equation*}
\begin{aligned}
\sigma_1(a_{i+1,1} \widetilde{\mathbf F}_{k 1} + \ldots + a_{i,i+1} \widetilde{\mathbf F}_{k i}) &\leq C_A \varepsilon_r, \quad \forall t_{k i} \in \ff{0}{T}, \\
\sigma_1(b_{1} \widetilde{\mathbf F}_{k 1} + \ldots + b_{s} \widetilde{\mathbf F}_{k s}) &\leq C_B \varepsilon_r, \quad \forall t_{k+1} \in \ff{0}{T},
\end{aligned}
\end{equation*}
and one can take $\gamma_r = \max\{C_A,C_B\} \varepsilon_r$ with $C_A = \sum_{i,j=1}^s |a_{i j}|$ and $C_B = \sum_{i=1}^s |b_{i}|$. In particular, condition \eqref{eq:r3_1} for a specific time $t_{k i}$ is satisfied if the left and right singular vectors associated with the largest singular value of ${\mathbf F}_{k i}$ are not included in the left and right singular vectors of $\Po_{{\mathbf Y}_{k i}}\big({\mathbf F}_{k i}\big)$, which is however unlikely, especially if $\Po_{{\mathbf Y}_{k i}}\big({\mathbf F}_{k i}\big)$ is a good approximation of ${\mathbf F}_{k i}$.
\end{remark}

\medskip

\begin{proposition}
\label{theo:P5}
Let the Runge-Kutta method \eqref{eq:rk} be a numerical scheme of order $p$, and suppose that Assumption \ref{theo:A2} holds. Then, for $h\leq h_0$, the local error of the Runge-Kutta method verifies
\begin{equation}
\norm{{\mathbf A}_{k+1}-{\boldsymbol\Phi}^h_{\mathbf F}({\mathbf A}_k)}_F \leq C_L h^{p+1},
\end{equation}
where the constant $C_L>0$ is independent of $h$.
\end{proposition}
\begin{proof}
see Theorem 3.1 of Chapter 2 in \cite{harrier1993solving}.
\end{proof}


\subsection{Local and global error bounds}

We establish the local and global error bounds in Theorems \ref{theo:T1} and \ref{theo:T2}, respectively. These error bounds show that the proposed Runge-Kutta BUG integrator retains the order of convergence of the associated Runge-Kutta method until the error reaches a plateau corresponding to the low-rank truncation error and which vanishes as the rank becomes full. Compared to \cite{kieri2019projection}, we obtain local and global error bounds with an improved order of convergence in $h$ thanks to the introduction of the additional error term $\gamma_r$, but similar error bounds can be obtained for projected Runge-Kutta methods by introducing such error term and adapting the convergence analysis accordingly.

\medskip

\begin{lemma}
\label{theo:L3}
Suppose that Assumption \ref{theo:A1} holds, and consider the time subinterval $\ff{t_k}{t_{k+1}} \subseteq \ff{0}{T}$. Moreover, let ${\mathbf Z}_{k i}$ denote the discrete solution computed using the (unprojected) Runge-Kutta method \eqref{eq:rk} with the initial value ${\mathbf Z}_{k} = {\mathbf Y}_{k} \in \M_r$. Then, for $h \leq h_0$, it holds 
\begin{equation}
\norm{{\mathbf Y}_{k i}-{\mathbf Z}_{k i}}_F \leq C_{i} h (\varepsilon_r + \gamma_r), \quad i= 1,\ldots,s,
\end{equation}
where the constant $C_{i} \geq 0$ is independent of $h$ and $r$.
\end{lemma}
\begin{proof}
We proceed by induction. For $i=1$, the statement is trivial with $C_1=0$ since ${\mathbf Z}_{k 1} = {\mathbf Z}_{k} = {\mathbf Y}_{k} = {\mathbf Y}_{k 1}$. Then, for $i \in \{2,\ldots,s\}$, the local error can be split into
\begin{equation}
\label{eq:l3_1}
\norm{{\mathbf Y}_{k i}-{\mathbf Z}_{k i}}_F \leq \norm{{\mathbf Y}_{k i}-\widehat{\mathbf Y}_{k i}}_F + \norm{\widehat{\mathbf Y}_{k i}- {\mathbf Z}_{k i}}_F.
\end{equation}
The first term in the RHS is bounded by
\begin{equation}
\label{eq:l3_2}
\norm{\widehat{\mathbf Y}_{k i}-{\mathbf Y}_{k i}}_F \leq h \gamma_r.
\end{equation}
Regarding the second term, we deduce from the induction hypothesis that
\begin{equation}
\label{eq:l3_3}
\begin{aligned}
\norm{\widehat{\mathbf Y}_{k i}-{\mathbf Z}_{k i}}_F &\leq h \sum_{j=1}^{i-1} |a_{i j}|  \norm{\widehat{\mathbf U}_{k i}\widehat{\mathbf U}_{k i}^T{\mathbf F}(t_{k j},{\mathbf Y}_{k j})\widehat{\mathbf V}_{k i}\widehat{\mathbf V}_{k i}^T-{\mathbf F}(t_{k j},{\mathbf Z}_{k j})}_F \\
&= h \sum_{j \in \I_A} |a_{i j}|  \norm{\widehat{\mathbf U}_{k i}\widehat{\mathbf U}_{k i}^T{\mathbf F}(t_{k j},{\mathbf Y}_{k j})\widehat{\mathbf V}_{k i}\widehat{\mathbf V}_{k i}^T-{\mathbf F}(t_{k j},{\mathbf Z}_{k j})}_F \\
&\leq h \sum_{j \in \I_A} |a_{i j}| \Bigl( \norm{\widehat{\mathbf U}_{k i}\widehat{\mathbf U}_{k i}^T{\mathbf F}(t_{k j},{\mathbf Y}_{k j})\widehat{\mathbf V}_{k i}\widehat{\mathbf V}_{k i}^T-{\mathbf F}(t_{k j},{\mathbf Y}_{k j})}_F \\
&\hphantom{\leq h \sum_{j \in \I_A} |a_{i j}| \Bigl(} + \norm{{\mathbf F}(t_{k j},{\mathbf Y}_{k j})-{\mathbf F}(t_{k j},{\mathbf Z}_{k j})}_F \Bigr) \\
&\leq h \sum_{j \in \I_A} |a_{i j}| \left( \varepsilon_r + L \norm{{\mathbf Y}_{k j}-{\mathbf Z}_{k j}}_F \right) \\
&\leq \Big( \sum_{j \in \I_A}  |a_{i j}| (1+L C_{j} h) \Big) h \varepsilon_r + \Big( \sum_{j \in \I_A}  |a_{i j}| L C_{j} \Big) h^2 \gamma_r,
\end{aligned}
\end{equation}
where $\I_A = \bigl\{j \in \{1,\ldots,i-1\} \mid a_{i j}\neq 0 \bigr\}$. Finally, combining equations \eqref{eq:l3_1}, \eqref{eq:l3_2}, and \eqref{eq:l3_3} yields the desired result:
\begin{equation*}
\begin{aligned}
\norm{{\mathbf Y}_{k i}-{\mathbf Z}_{k i}}_F &\leq \Big( \sum_{j \in \I_A}  |a_{i j}| (1+L C_{j} h) \Big) h \varepsilon_r + \Big( 1+ \sum_{j \in \I_A}  |a_{i j}| L C_{j}h \Big) h \gamma_r \\
&\leq C_i h (\varepsilon_r + \gamma_r),
\end{aligned}
\end{equation*}
where $C_1 = 0$ and $C_i = \max\Big\{ \sum_{j=1}^{i-1}  |a_{i j}| (1+L C_{j} h_0) ,1+ \sum_{j=1}^{i-1}  |a_{i j}| L C_{j}h_0 \Big\}$ for $i=2,\ldots,s$.
\end{proof}

\medskip

\begin{theorem}
\label{theo:T1}
Suppose that Assumptions \ref{theo:A1} and \ref{theo:A2} hold, and consider the time subinterval $\ff{t_k}{t_{k+1}} \subseteq \ff{0}{T}$. Then, for $h \leq h_0$, the local error of the Runge-Kutta BUG integrator verifies
\begin{equation}
\norm{{\mathbf Y}_{k+1}-{\boldsymbol\Phi}^{h}_{\mathbf F}({\mathbf Y}_k)}_F \leq C h \left( \varepsilon_r + \gamma_r + h^{p} \right),
\end{equation}
where the constant $C>0$ is independent of $h$ and $r$.
\end{theorem}
\begin{proof}
Let ${\mathbf Z}_{k+1}$ denote the discrete solution computed using the Runge-Kutta method \eqref{eq:rk} with the initial value ${\mathbf Z}_{k} = {\mathbf Y}_{k} \in \M_r$. The local error can be split into
\begin{equation}
\label{eq:t1_1}
\norm{{\mathbf Y}_{k+1}-{\boldsymbol\Phi}^{h}_{\mathbf F}({\mathbf Y}_k)}_F  \leq \norm{{\mathbf Y}_{k+1}- \widehat{\mathbf Y}_{k+1}}_F + \norm{\widehat{\mathbf Y}_{k+1}- {\mathbf Z}_{k+1}}_F + \norm{{\mathbf Z}_{k+1}-{\boldsymbol\Phi}^{h}_{\mathbf F}({\mathbf Y}_k)}_F.
\end{equation}
The first term in the RHS is bounded by
\begin{equation}
\label{eq:t1_2}
\norm{\widehat{\mathbf Y}_{k+1}-{\mathbf Y}_{k+1}}_F \leq h \gamma_r.
\end{equation}
Regarding the second term, we deduce from Lemma \ref{theo:L3} that
\begin{equation}
\label{eq:t1_3}
\begin{aligned}
\norm{\widehat{\mathbf Y}_{k+1}-{\mathbf Z}_{k+1}}_F &\leq h \sum_{i=1}^{s} |b_{i}|  \norm{\widehat{\mathbf U}_{k+1}\widehat{\mathbf U}_{k+1}^T{\mathbf F}(t_{k i},{\mathbf Y}_{k i})\widehat{\mathbf V}_{k+1}\widehat{\mathbf V}_{k+1}^T-{\mathbf F}(t_{k i},{\mathbf Z}_{k i})}_F \\
&= h \sum_{i \in \I_B} |b_{i}| \norm{\widehat{\mathbf U}_{k+1}\widehat{\mathbf U}_{k+1}^T{\mathbf F}(t_{k i},{\mathbf Y}_{k i})\widehat{\mathbf V}_{k+1}\widehat{\mathbf V}_{k+1}^T-{\mathbf F}(t_{k i},{\mathbf Z}_{k i})}_F \\
&\leq h \sum_{i \in \I_B} |b_{i}| \Bigl( \norm{\widehat{\mathbf U}_{k+1}\widehat{\mathbf U}_{k+1}^T{\mathbf F}(t_{k i},{\mathbf Y}_{k i})\widehat{\mathbf V}_{k+1}\widehat{\mathbf V}_{k+1}^T-{\mathbf F}(t_{k i},{\mathbf Y}_{k i})}_F \\
&\hphantom{\leq h \sum_{i \in \I_B} |b_{i}| \Bigl(} + \norm{{\mathbf F}(t_{k i},{\mathbf Y}_{k i})-{\mathbf F}(t_{k i},{\mathbf Z}_{k i})}_F \Bigr) \\
&\leq h \sum_{i \in \I_B} |b_{i}| \left(\varepsilon_r + L \norm{{\mathbf Y}_{k i}-{\mathbf Z}_{k i}}_F\right) \\
&\leq \Big( \sum_{i \in \I_B}  |b_{i}| (1+L C_{i} h) \Big) h \varepsilon_r + \Big( \sum_{i \in \I_B}  |b_{i}| L C_{i} \Big) h^2 \gamma_r
\end{aligned}
\end{equation}
where $\I_B = \bigl\{i \in \{1,\ldots,s\} \mid b_{i}\neq 0 \bigr\}$. The last term is bounded according to Proposition \ref{theo:P5} by
\begin{equation}
\label{eq:t1_4}
\norm{{\mathbf Z}_{k+1}-{\boldsymbol\Phi}^{h}_{\mathbf F}({\mathbf Y}_k)}_F = \norm{{\mathbf Z}_{k+1}-{\boldsymbol\Phi}^{h}_{\mathbf F}({\mathbf Z}_k)}_F \leq C_L h^{p+1}.
\end{equation}
Finally, combining equations \eqref{eq:t1_1}, \eqref{eq:t1_2}, \eqref{eq:t1_3}, and \eqref{eq:t1_4} yields the desired result:
\begin{equation*}
\begin{aligned}
\norm{{\mathbf Y}_{k+1}-{\boldsymbol\Phi}^{h}_{\mathbf F}({\mathbf Y}_k)}_F &\leq \Big( \sum_{i \in \I_B}  |b_{i}| (1+L C_{i} h) \Big) h \varepsilon_r + \Big( 1+ \sum_{i \in \I_B}  |b_{i}| L C_{i}h \Big) h \gamma_r + C_L h^{p+1} \\
&\leq C h (\varepsilon_r + \gamma_r+h^p),
\end{aligned}
\end{equation*}
where $C = \max\Big\{ \sum_{i=1}^s  |b_{i}| (1+L C_{i} h_0) ,1+ \sum_{i=1}^s  |b_{i}| L C_{i}h_0, C_L \Big\}$.
\end{proof}

\medskip

\begin{theorem}
\label{theo:T2}
Let the initial error be $\delta_r=\norm{{\mathbf Y}_0-{\mathbf A}_0}_F$ with $\delta_r$ vanishing when $r = \min\{n,m\}$. Moreover, suppose that Assumptions \ref{theo:A1} and \ref{theo:A2} hold. Then, for $h \leq h_0$, the global error of the Runge-Kutta BUG integrator verifies
\begin{equation}
\label{eq:global_error}
\norm{{\mathbf Y}_N-{\mathbf A}(t_N)}_F \leq C' \left( \delta_r + \varepsilon_r + \gamma_r + h^{p} \right),
\end{equation}
on the finite time-interval $0 \leq t_N \leq T$. The constant $C'>0$ depends on $T$, $L$, $C_L$, $h_0$, $C_A = \sum_{i,j=1}^s |a_{i j}|$, and $C_B = \sum_{i=1}^s |b_{i}|$ but not on $h$ nor $r$.
\end{theorem}
\begin{proof}
The bound results from the local error of Theorem \ref{theo:T1} and the standard argument of Lady Windermere’s fan with error propagation along the exact flow. More specifically, the global error can be expanded as the telescoping sum
\begin{equation*}
{\mathbf Y}_{N}-{\boldsymbol\Phi}^{Nh}_{\mathbf F}({\mathbf A}_0) = \left( \sum_{k=1}^N {\boldsymbol\Phi}^{(N-k)h}_{\mathbf F}({\mathbf Y}_{k}) - {\boldsymbol\Phi}^{(N-k+1)h}_{\mathbf F}({\mathbf Y}_{k-1}) \right) + {\boldsymbol\Phi}^{Nh}_{\mathbf F}({\mathbf Y}_{0}) - {\boldsymbol\Phi}^{Nh}_{\mathbf F}({\mathbf A}_{0}).
\end{equation*}
Then, the different terms can be bounded according to Lemma \ref{theo:L4} and Theorem \ref{theo:T1} by
\begin{equation*}
\norm{{\boldsymbol\Phi}^{Nh}_{\mathbf F}({\mathbf Y}_{0}) - {\boldsymbol\Phi}^{Nh}_{\mathbf F}({\mathbf A}_{0})}_F \leq e^{LNh} \norm{{\mathbf Y}_{0} - {\mathbf A}_{0}}_F = e^{LNh} \delta_r,
\end{equation*}
\begin{equation*}
\begin{aligned}
\norm{{\boldsymbol\Phi}^{(N-k)h}_{\mathbf F}({\mathbf Y}_{k}) - {\boldsymbol\Phi}^{(N-k)h}_{\mathbf F}({\boldsymbol\Phi}^{h}_{\mathbf F}({\mathbf Y}_{k-1}))}_F &\leq e^{L(N-k)h} \norm{{\mathbf Y}_{k}-{\boldsymbol\Phi}^{h}_{\mathbf F}({\mathbf Y}_{k-1})}_F \\
&\leq e^{L(N-k)h} C h (\varepsilon_r + \gamma_r + h^p),
\end{aligned}
\end{equation*}
which leads to the upper bound:
\begin{equation*}
\norm{{\mathbf Y}_{N}-{\boldsymbol\Phi}^{Nh}_{\mathbf F}({\mathbf A}_0)}_F \leq e^{LNh} \delta_r + C (\varepsilon_r + \gamma_r + h^p) \sum_{k=1}^N e^{L(N-k)h} h.
\end{equation*}
Finally, bounding the Riemann sum as
\begin{equation*}
\sum_{k=1}^N h e^{L(N-k)h} \leq \int_0^{Nh} e^{L(Nh-t)} \diff t = \frac{e^{LNh}-1}{L}
\end{equation*}
and using $Nh \leq T$ lead to
\begin{align*}
\norm{{\mathbf Y}_{N}-{\boldsymbol\Phi}^{Nh}_{\mathbf F}({\mathbf A}_0)}_F &\leq e^{LT} \delta_r + C \frac{e^{LT}-1}{L} (\varepsilon_r + \gamma_r + h^p) \\
&\leq \max\left\{e^{LT},C \frac{e^{LT}-1}{L}\right\} (\delta_r + \varepsilon_r + \gamma_r + h^p),
\end{align*}
which concludes the proof.
\end{proof}

\medskip

\begin{remark}
If $\max\{\delta_r, \varepsilon_r, \gamma_r\} \ll h^p$, then the Runge-Kutta BUG integrator has the same order of convergence as the associated Runge-Kutta method. Otherwise, the global error is dominated by the low-rank truncation error (namely the initial, projection, and truncation errors) and vanishes as the rank becomes full.
\end{remark}


\section{Numerical experiments}
\label{sec:5}

According to Theorem \ref{theo:T2}, the rate of convergence of the proposed Runge-Kutta BUG integrator scales with $h^p$ until the error reaches a plateau which decreases as the rank increases. This theoretical result is validated experimentally on three numerical test cases taken from \cite{kieri2019projection,lam2024randomized}. To this end, we investigate the convergence of several Runge-Kutta BUG integrators based on the following high-oder explicit Runge-Kutta methods:
\begin{itemize}
\item midpoint method (RK2m),
\item Heun's method (RK2h),
\item Heun's third-order method (RK3h),
\item third-order SSP RK method (RK3s),
\item classic fourth-order Runge-Kutta method (RK4),
\end{itemize}
where the associated Butcher tableaux are given in Appendix \ref{sec:A1}. The accuracy of the resulting integrators is measured with respect to a reference solution ${\mathbf A}_k$ computed using the fifth-order Runge-Kutta-Fehlberg method:
\begin{equation*}
\textrm{Error} = \max\limits_{k = 0,\ldots,N} \norm{{\mathbf Y}_k - {\mathbf A}_k}_F.
\end{equation*}
Moreover, we compare these Runge-Kutta BUG integrators with the following state-of-the-art dynamical low-rank integrators:
\begin{itemize}
\item the first variant of the second-order BUG integrator (Midpoint BUG) of \cite{ceruti2024robust},
\item the second-order projected Runge-Kutta method (PRK2h) \cite{kieri2019projection},
\item the third-order projected Runge-Kutta method (PRK3h) \cite{kieri2019projection}.
\end{itemize}
Notably, we consider the first variant of the midpoint BUG integrator \cite{ceruti2024robust} because it is more accurate than the second variant. In addition, this integrator is different from our Runge-Kutta BUG integrator based on the midpoint method since the intermediate solution is not truncated (the rank of $\widehat{\mathbf Y}_{k+1}$ is at most $4r$ instead of $2r$ in our Runge-Kutta BUG integrator). As a result, this integrator is computationally more expensive than our Runge-Kutta BUG integrator, but it is potentially more accurate. Note that the accuracy of these two integrators is almost the same in the following experiments.


\subsection{Allen-Cahn equation}

We first consider the Allen-Cahn equation:
\begin{equation*}
\left\{
\begin{aligned}
\dot {\mathbf A}(t) &=  \theta ({\mathbf L} {\mathbf A}(t) + {\mathbf A}(t) {\mathbf L}) +{\mathbf A}(t) - {\mathbf A}(t) \odot {\mathbf A}(t) \odot {\mathbf A}(t) \\
{\mathbf A}(0) &= {\mathbf A}_0,
\end{aligned}
\right.
\end{equation*}
where ${\mathbf A}(t) \in \R^{n \times n}$, ${\mathbf L} = \frac{n^2}{4 \pi^2}\textrm{diag}(1,-2,1) \in \R^{n \times n}$, $t \in \ff{0}{10}$, $\theta = 10^{-2}$, and $\odot$ stands for the Hadamart product. The domain ${\ff{0}{2\pi}}^2$ is discretized using $n=128$ equidistant points in each direction, and the initial solution is given by 
\begin{equation*}
({\mathbf A}_0)_{i,j} = \frac{\left[e^{-\tan^2(x_i)}+e^{-\tan^2(y_j)}\right]\sin(x_i)\sin(y_j)}{1+e^{|\csc(-x_i/2)|}+e^{|\csc(-y_j/2)|}}.
\end{equation*}

In Figure \ref{fig:fig5}, we present the error of the Runge-Kutta BUG integrator as a function of the time-step size $h$ and rank $r$. The reader can see that the Runge-Kutta BUG integrator achieves second-, third-, and fourth-order convergences until the error reaches a plateau corresponding to the low-rank truncation error. This plateau decreases as the rank increases up to a certain limit, around $10^{-9}$, resulting from the accumulation of roundoff error. In this example, the errors of the midpoint BUG integrator and projected Runge-Kutta methods are almost identical to those of our corresponding Runge-Kutta BUG integrators.

\begin{figure}
\begin{subfigure}[c]{0.5\linewidth}
\center
\includegraphics[width=6.5cm]{allen_cahn_2m.eps}
\caption{Midpoint method}
\end{subfigure}\hfill
\begin{subfigure}[c]{0.5\linewidth}
\center
\includegraphics[width=6.5cm]{allen_cahn_2h.eps}
\caption{Heun's method}
\end{subfigure}\hfill
\begin{subfigure}[c]{0.5\linewidth}
\center
\includegraphics[width=6.5cm]{allen_cahn_3s.eps}
\caption{Third-order SSP Runge-Kutta method}
\end{subfigure}\hfill
\begin{subfigure}[c]{0.5\linewidth}
\center
\includegraphics[width=6.5cm]{allen_cahn_3h.eps}
\caption{Heun's third-order method}
\end{subfigure}\hfill
\begin{subfigure}[c]{1\linewidth}
\center
\includegraphics[width=6.5cm]{allen_cahn_4.eps}
\caption{Classic fourth-order Runge-Kutta method}
\end{subfigure}\hfill
\caption{\label{fig:fig5}Convergence error of the Runge-Kutta BUG integrator for the Allen-Cahn equation}
\end{figure}


\subsection{Lyapunov equation}

Then, we consider the Lyapunov equation:
\begin{equation*}
\left\{
\begin{aligned}
\dot {\mathbf A}(t) &=  {\mathbf L} {\mathbf A}(t) + {\mathbf A}(t) {\mathbf L} + \theta \frac{\mathbf C}{\norm{\mathbf C}_F} \\
{\mathbf A}(0) &= {\mathbf A}_0,
\end{aligned}
\right.
\end{equation*}
where ${\mathbf A}(t) \in \R^{n \times n}$, ${\mathbf L} = \frac{n^2}{4 \pi^2}\textrm{diag}(1,-2,1) \in \R^{n \times n}$, ${\mathbf C} \in \R^{n \times n}$, $t \in \ff{0}{1}$, and $\theta \geq 0$. The domain ${\ff{-\pi}{\pi}}^2$ is discretized using $n=128$ equidistant points in each direction, and the initial solution and forcing term are given by 
\begin{equation*}
({\mathbf A}_0)_{ij} = \sin(x_i)\sin(y_j) \quad \textrm{and}  \quad ({\mathbf C})_{ij} = \sum\limits_{l=1}^{11} 10^{-(l-1)} e^{-l(x_i^2+y_j^2)}.
\end{equation*}

We start with the case $\theta = 10^{-5}$. In Figure \ref{fig:fig1}, we fix the rank to $r=5$ and we report the error of the Runge-Kutta BUG integrator as a function of the time-step size $h$. The reader can see that the proposed dynamical low-rank integrator achieves second- and third-order convergences until the error reaches a limit, around $10^{-10}$, resulting from roundoff errors. Regarding the fourth-order convergence, this one is not visible since the error is already of the order of $10^{-11}$ for $h=5\times 10^{-4}$. Moreover, the errors of the midpoint BUG integrator and projected Runge-Kutta methods are almost the same as those of the corresponding Runge-Kutta BUG integrators for second-order methods. However, for the Heun's third-order method, the Runge-Kutta BUG integrator is significantly more accurate than the projected Runge-Kutta method.

We now consider the case $\theta = 1$, where a higher rank is necessary to obtain a small rank truncation error due to the forcing term. In Figure \ref{fig:fig2}, we see that the Runge-Kutta BUG integrator achieves second-order convergence. The third- and fourth-order convergences are not visible because the error, around $10^{-10}$, is of the order of the cumulative roundoff error. Moreover, for second-order methods, the error reaches a plateau corresponding to the low-rank truncation error and which decreases as the rank increases. Furthermore, the Runge-Kutta BUG integrator is several orders of magnitude more accurate than the projected Runge-Kutta method, which can be explained by the fact that the Galerkin projection is more accurate than the tangent-space projection for approximating the discrete solution.


\begin{figure}
\begin{subfigure}[c]{0.5\linewidth}
\center
\includegraphics[width=6.5cm]{lyapunov_2_2m.eps}
\caption{Midpoint method}
\end{subfigure}\hfill
\begin{subfigure}[c]{0.5\linewidth}
\center
\includegraphics[width=6.5cm]{lyapunov_2_2h.eps}
\caption{Heun's method}
\end{subfigure}\hfill
\begin{subfigure}[c]{0.5\linewidth}
\center
\includegraphics[width=6.5cm]{lyapunov_2_3s.eps}
\caption{Third-order SSP Runge-Kutta method}
\end{subfigure}\hfill
\begin{subfigure}[c]{0.5\linewidth}
\center
\includegraphics[width=6.5cm]{lyapunov_2_3h.eps}
\caption{Heun's third-order method}
\end{subfigure}\hfill
\begin{subfigure}[c]{1\linewidth}
\center
\includegraphics[width=6.5cm]{lyapunov_2_4.eps}
\caption{Classic fourth-order Runge-Kutta method}
\end{subfigure}\hfill
\caption{\label{fig:fig1}Convergence error of the Runge-Kutta BUG integrator for the Lyapunov equation with $\theta=10^{-5}$}
\end{figure}

\begin{figure}
\begin{subfigure}[c]{0.5\linewidth}
\center
\includegraphics[width=6.5cm]{lyapunov_1_2m.eps}
\caption{Midpoint method}
\end{subfigure}\hfill
\begin{subfigure}[c]{0.5\linewidth}
\center
\includegraphics[width=6.5cm]{lyapunov_1_2h.eps}
\caption{\label{fig:fig2_2}Heun's method}
\end{subfigure}\hfill
\begin{subfigure}[c]{0.5\linewidth}
\center
\includegraphics[width=6.5cm]{lyapunov_1_3s.eps}
\caption{Third-order SSP Runge-Kutta method}
\end{subfigure}\hfill
\begin{subfigure}[c]{0.5\linewidth}
\center
\includegraphics[width=6.5cm]{lyapunov_1_3h.eps}
\caption{\label{fig:fig2_4}Heun's third-order method}
\end{subfigure}\hfill
\begin{subfigure}[c]{1\linewidth}
\center
\includegraphics[width=6.5cm]{lyapunov_1_4.eps}
\caption{Classic fourth-order Runge-Kutta method}
\end{subfigure}\hfill
\caption{\label{fig:fig2}Convergence error of the Runge-Kutta BUG integrator for the Lyapunov equation with $\theta=1$. In \ref{fig:fig2_2} and \ref{fig:fig2_4}, the lines corresponding to the two projected Runge-Kutta methods are superposed as the error is dominated by the low-rank approximation error}
\end{figure}


\subsection{Discrete nonlinear Sch\"odinger equation}

Lastly, we consider the discrete nonlinear Sch\"odinger equation \cite{trombettoni2001discrete}:
\begin{equation*}
\left\{
\begin{aligned}
i \dot {\mathbf A}(t) &= -\frac{1}{2} ({\mathbf D} {\mathbf A}(t) + {\mathbf A}(t) {\mathbf D}) - \theta |{\mathbf A}(t)|^2 \odot {\mathbf A}(t)\\
{\mathbf A}(0) &= {\mathbf A}_0,
\end{aligned}
\right.
\end{equation*}
where ${\mathbf A}(t) \in \R^{n \times n}$, ${\mathbf D} = \textrm{diag}(1,0,1) \in \R^{n \times n}$, $t \in \ff{0}{5}$, and $\theta \geq 0$. Moreover, we set $n=128$, and the initial solution is given by
\begin{equation*}
({\mathbf A}_0)_{jl} = \exp(-\frac{(j-60)^2}{100}-\frac{(l-50)^2}{100})+\exp(-\frac{(j-50)^2}{100}-\frac{(l-40)^2}{100}).
\end{equation*}

We start with the case $\theta = 0.1$. In Figure \ref{fig:fig3}, we present the error of the Runge-Kutta BUG integrator as a function of the time-step size $h$ and rank $r$. The reader can see that the Runge-Kutta BUG integrator achieves the expected orders of convergence. Then, the error reaches the plateau resulting from the low-rank truncation error and which decreases as the rank increases. Moreover, the errors of the midpoint BUG integrator and projected Runge-Kutta methods are almost the same as those of the corresponding Runge-Kutta BUG integrators for second-order methods. However, for the Heun's third-order method, the Runge-Kutta BUG integrator is significantly more accurate than the projected Runge-Kutta method.

We now consider the case $\theta = 0.3$, where a higher rank is necessary to obtain small rank truncation errors due to the cubic term. In Figure \ref{fig:fig4}, we see that the error behaves in the same way as before. Note that, for third-order methods, the Runge-Kutta BUG integrator achieves third-order convergence for $h \leq 10^{-1}$.


\begin{figure}
\begin{subfigure}[c]{0.5\linewidth}
\center
\includegraphics[width=6.5cm]{schrodinger_1_2m.eps}
\caption{Midpoint method}
\end{subfigure}\hfill
\begin{subfigure}[c]{0.5\linewidth}
\center
\includegraphics[width=6.5cm]{schrodinger_1_2h.eps}
\caption{Heun's method}
\end{subfigure}\hfill
\begin{subfigure}[c]{0.5\linewidth}
\center
\includegraphics[width=6.5cm]{schrodinger_1_3s.eps}
\caption{Third-order SSP Runge-Kutta method}
\end{subfigure}\hfill
\begin{subfigure}[c]{0.5\linewidth}
\center
\includegraphics[width=6.5cm]{schrodinger_1_3h.eps}
\caption{Heun's third-order method}
\end{subfigure}\hfill
\begin{subfigure}[c]{\linewidth}
\center
\includegraphics[width=6.5cm]{schrodinger_1_4.eps}
\caption{Classic fourth-order Runge-Kutta method}
\end{subfigure}\hfill
\caption{\label{fig:fig3}Convergence error of the Runge-Kutta BUG integrator for the Schr\"odinger equation with $\theta=0.1$}
\end{figure}


\begin{figure}
\begin{subfigure}[c]{0.5\linewidth}
\center
\includegraphics[width=6.5cm]{schrodinger_2_2m.eps}
\caption{Midpoint method}
\end{subfigure}\hfill
\begin{subfigure}[c]{0.5\linewidth}
\center
\includegraphics[width=6.5cm]{schrodinger_2_2h.eps}
\caption{Heun's method}
\end{subfigure}\hfill
\begin{subfigure}[c]{0.5\linewidth}
\center
\includegraphics[width=6.5cm]{schrodinger_2_3s.eps}
\caption{Third-order SSP Runge-Kutta method}
\end{subfigure}\hfill
\begin{subfigure}[c]{0.5\linewidth}
\center
\includegraphics[width=6.5cm]{schrodinger_2_3h.eps}
\caption{Heun's third-order method}
\end{subfigure}\hfill
\begin{subfigure}[c]{\linewidth}
\center
\includegraphics[width=6.5cm]{schrodinger_2_4.eps}
\caption{Classic fourth-order Runge-Kutta method}
\end{subfigure}\hfill
\caption{\label{fig:fig4}Convergence error of the Runge-Kutta BUG integrator for the Schr\"odinger equation with $\theta=0.3$}
\end{figure}


\section{Conclusion}
\label{sec:6}

In this work, we have proposed high-order BUG integrators based on explicit Runge-Kutta methods. By construction, the resulting dynamical low-rank integrators are robust to the presence of small singular values and do not involve backward time-integration steps. Then, we have analyzed the convergence of the proposed Runge-Kutta BUG integrator. The error bound shows that the Runge-Kutta BUG integrator retains the order of convergence of the associated Runge-Kutta method until the error reaches a plateau corresponding to the low-rank truncation error and which vanishes as the rank becomes full.
This error bound has been validated experimentally on three numerical test cases. The results demonstrate the high-order convergence of the Runge-Kutta BUG integrator and its superior accuracy compared to projected Runge-Kutta methods for high orders $p \geq 3$ and small time-steps.
In perspective, the Runge-Kutta BUG integrator can be extended to other classes of Runge-Kutta methods, such as exponential Runge-Kutta methods or implicit Runge-Kutta methods. These topics will be addressed in future work.


\backmatter

\bmhead{Acknowledgements}

This work has been supported by the Swiss National Science Foundation under the Project n°200518 "Dynamical low rank methods for uncertainty quantification and data assimilation".


\section*{Declarations}

\bmhead{Data availability}

The datasets generated during and/or analyzed during the current study are available from the corresponding author on reasonable request.

\bmhead{Conflict of interest}

The authors declare that they have no conflict of interest.


\begin{appendices}

\section{List of Runge-Kutta methods}\label{sec:A1}

The Butcher tableaux associated with the Runge-Kutta methods used in this work are listed below.

\begin{itemize}
\item Forward Euler method
\medskip

\begin{tabular}{c | c c}
0 & 0 \\
1 & 1 \\
\hline
 & 1
\end{tabular}

\medskip
\item Explicit midpoint method
\medskip

\begin{tabular}{c | c c}
0 & 0 & 0 \\
1/2 & 1/2 & 0 \\
\hline
 & 0 & 1
\end{tabular}

\medskip
\item Heun's method
\medskip

\begin{tabular}{c | c c}
0 & 0 & 0 \\
1 & 1 & 0 \\
\hline
 & 1/2 & 1/2
\end{tabular}

\medskip
\item Third-order SSP Runge-Kutta method
\medskip

\begin{tabular}{c | c c c}
0 & 0 & 0 & 0 \\
1 & 1 & 0 & 0 \\
1/2 & 1/4 & 1/4 & 0 \\
\hline
 & 1/6 & 1/6 & 2/3
\end{tabular}

\medskip
\item Heun's third-order method
\medskip

\begin{tabular}{c | c c c}
0 & 0 & 0 & 0 \\
1/3 & 1/3 & 0 & 0 \\
2/3 & 0 & 2/3 & 0 \\
\hline
 & 1/4 & 0 & 3/4
\end{tabular}

\medskip
\item Classic fourth-order Runge-Kutta method
\medskip

\begin{tabular}{c | c c c c}
0 & 0 & 0 & 0 & 0 \\
1/2 & 1/2 & 0 & 0 & 0 \\
1/2 & 0 & 1/2 & 0 & 0 \\
1 & 0 & 0 & 1 & 0 \\
\hline
 & 1/6 & 1/3 & 1/3 & 1/6
\end{tabular}
\end{itemize}


\end{appendices}


\bibliography{sn-bibliography}% common bib file
%% if required, the content of .bbl file can be included here once bbl is generated
%%\input sn-article.bbl


\end{document}
