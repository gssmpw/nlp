
\begin{abstract}
Recent advances in mathematical problem-solving with language models (LMs) integrate chain-of-thought (CoT) reasoning and code execution to harness their complementary strengths. However, existing hybrid frameworks exhibit a critical limitation: they depend on externally dictated instructions or rigid code-integration templates, lacking metacognitive awareness—the capacity to dynamically evaluate intrinsic capabilities and autonomously determine when and how to integrate tools. This rigidity motivates our study of autonomous code integration, enabling models to adapt tool-usage strategies as their reasoning abilities evolve during training.

While reinforcement learning (RL) shows promise for boosting LLM reasoning at scale (e.g., DeepSeek-R1), we demonstrate its inefficiency in learning autonomous code integration due to inadequate exploration of the vast combinatorial space of CoT-code interleaving patterns.  To address this challenge, we propose a novel
Expectation-Maximization (EM) framework that
synergizes structured exploration (E-step) with off-policy RL optimization (M-step), creating a self-reinforcing cycle between metacognitive tool-use decisions and evolving capabilities.  Experiments reveal our method achieves superior results through improved exploration. Notably, our 7B model improves over 11\% on MATH500 and 9.4\% on AIME without o1-like CoT. Code, models and data are released via an \href{https://anonymous.4open.science/r/AnnonySubmission-0C62}{anonymous repository}.
% Code and data is released in  \href{https://anonymous.4open.science/r/AnnonySubmission-35F0}{https://anonymous.4open.science/r/AnnonySubmission-35F0}.
% Code and data is released in  \href{https://anonymous.4open.science/r/AnnonySubmission-35F0}{https://anonymous.4open.science/r/AnnonySubmission-35F0}} 
\end{abstract}