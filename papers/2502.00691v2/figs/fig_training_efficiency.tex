\begin{figure*}[t]
    \centering
    \includegraphics[width=0.95\linewidth]{figs/rl_curves.pdf}
    \caption{ \small \textbf{Training Efficiency and Convergence.} We benchmark the learning dynamics of our approach against three two training paradigms: supervised fine-tuning and reinforcement learning (RL). The Pass@1 accuracy is evaluated on an held-out dev-set. We use Qwen-2.5-Base as the base model. SFT is conducted using collected public data~\cite{openmath, mammoth}. The dashed lines indicate asymptotic performance. }\label{fig_training_efficiency}
    % \begin{minipage}{0.47\textwidth}
    %     \centering
    %     \begin{subfigure}[b]{1.0\textwidth}
    %         \centering
    %         \includegraphics[width=0.95\linewidth]{figs/abl_qwen_curve.pdf}
    %         % \caption{Top right image}
    %     \end{subfigure}
    %     % \vskip -0.3\baselineskip % Add vertical space between subfigures
    %     \begin{subfigure}[b]{1.0\textwidth}
    %         \centering
    %         \includegraphics[width=1.\linewidth]{figs/abl_deepseek_curve.pdf}
    %         % \caption{Bottom right image}
    %     \end{subfigure}
    %     \caption{ \small \textbf{Performance Convergence. } Experiments are conducted based on Qwen2Math (Top) and DeepseekMath (Bottom). AutoCode achieves higher accuracy with sustained improvement, while standard RL converge to sub-optimal solutions. }\label{fig_training_effici}
    % \end{minipage}
    % \hfill
    % \begin{minipage}{0.47\textwidth}
    %     \centering
    %     % \vspace{-0.3cm}
    %     % \hspace{-1.6cm}
    %     \includegraphics[width=1.\linewidth]{figs/abl_strategies.pdf}
    %     \caption{\small \textbf{Analysis of the Learned Strategies.} Correct Responses are classified based on their alignment to the oracle selection, namely, \emph{StrictAlign}, \emph{AllowCode} and \emph{MisAlign}. We show how different categories of alignment contribute to the accuracy in the stacked bars, and include the overall StrictAlign rate in the separate orange bar.} \label{fig_learned_strategies}
    % \end{minipage}
% \vspace{-0.3cm}
\end{figure*}