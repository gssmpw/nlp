% This must be in the first 5 lines to tell arXiv to use pdfLaTeX, which is strongly recommended.
\pdfoutput=1
% In particular, the hyperref package requires pdfLaTeX in order to break URLs across lines.

\documentclass[11pt]{article}

% Change "review" to "final" to generate the final (sometimes called camera-ready) version.
% Change to "preprint" to generate a non-anonymous version with page numbers.
\usepackage[preprint]{acl}
\usepackage{times}
\usepackage{latexsym}

% For proper rendering and hyphenation of words containing Latin characters (including in bib files)
\usepackage[T1]{fontenc}
% For Vietnamese characters
% \usepackage[T5]{fontenc}
% See https://www.latex-project.org/help/documentation/encguide.pdf for other character sets

% This assumes your files are encoded as UTF8
\usepackage[utf8]{inputenc}

% This is not strictly necessary, and may be commented out,
% but it will improve the layout of the manuscript,
% and will typically save some space.
\usepackage{microtype}

% This is also not strictly necessary, and may be commented out.
% However, it will improve the aesthetics of text in
% the typewriter font.
\usepackage{inconsolata}

%Including images in your LaTeX document requires adding
%additional package(s)
\usepackage{graphicx}
% Optional math commands from https://github.com/goodfeli/dlbook_notation.
%%%%% NEW MATH DEFINITIONS %%%%%

\usepackage{amsmath,amsfonts,bm}
\usepackage{derivative}
% Mark sections of captions for referring to divisions of figures
\newcommand{\figleft}{{\em (Left)}}
\newcommand{\figcenter}{{\em (Center)}}
\newcommand{\figright}{{\em (Right)}}
\newcommand{\figtop}{{\em (Top)}}
\newcommand{\figbottom}{{\em (Bottom)}}
\newcommand{\captiona}{{\em (a)}}
\newcommand{\captionb}{{\em (b)}}
\newcommand{\captionc}{{\em (c)}}
\newcommand{\captiond}{{\em (d)}}

% Highlight a newly defined term
\newcommand{\newterm}[1]{{\bf #1}}

% Derivative d 
\newcommand{\deriv}{{\mathrm{d}}}

% Figure reference, lower-case.
\def\figref#1{figure~\ref{#1}}
% Figure reference, capital. For start of sentence
\def\Figref#1{Figure~\ref{#1}}
\def\twofigref#1#2{figures \ref{#1} and \ref{#2}}
\def\quadfigref#1#2#3#4{figures \ref{#1}, \ref{#2}, \ref{#3} and \ref{#4}}
% Section reference, lower-case.
\def\secref#1{section~\ref{#1}}
% Section reference, capital.
\def\Secref#1{Section~\ref{#1}}
% Reference to two sections.
\def\twosecrefs#1#2{sections \ref{#1} and \ref{#2}}
% Reference to three sections.
\def\secrefs#1#2#3{sections \ref{#1}, \ref{#2} and \ref{#3}}
% Reference to an equation, lower-case.
\def\eqref#1{equation~\ref{#1}}
% Reference to an equation, upper case
\def\Eqref#1{Equation~\ref{#1}}
% A raw reference to an equation---avoid using if possible
\def\plaineqref#1{\ref{#1}}
% Reference to a chapter, lower-case.
\def\chapref#1{chapter~\ref{#1}}
% Reference to an equation, upper case.
\def\Chapref#1{Chapter~\ref{#1}}
% Reference to a range of chapters
\def\rangechapref#1#2{chapters\ref{#1}--\ref{#2}}
% Reference to an algorithm, lower-case.
\def\algref#1{algorithm~\ref{#1}}
% Reference to an algorithm, upper case.
\def\Algref#1{Algorithm~\ref{#1}}
\def\twoalgref#1#2{algorithms \ref{#1} and \ref{#2}}
\def\Twoalgref#1#2{Algorithms \ref{#1} and \ref{#2}}
% Reference to a part, lower case
\def\partref#1{part~\ref{#1}}
% Reference to a part, upper case
\def\Partref#1{Part~\ref{#1}}
\def\twopartref#1#2{parts \ref{#1} and \ref{#2}}

\def\ceil#1{\lceil #1 \rceil}
\def\floor#1{\lfloor #1 \rfloor}
\def\1{\bm{1}}
\newcommand{\train}{\mathcal{D}}
\newcommand{\valid}{\mathcal{D_{\mathrm{valid}}}}
\newcommand{\test}{\mathcal{D_{\mathrm{test}}}}

\def\eps{{\epsilon}}


% Random variables
\def\reta{{\textnormal{$\eta$}}}
\def\ra{{\textnormal{a}}}
\def\rb{{\textnormal{b}}}
\def\rc{{\textnormal{c}}}
\def\rd{{\textnormal{d}}}
\def\re{{\textnormal{e}}}
\def\rf{{\textnormal{f}}}
\def\rg{{\textnormal{g}}}
\def\rh{{\textnormal{h}}}
\def\ri{{\textnormal{i}}}
\def\rj{{\textnormal{j}}}
\def\rk{{\textnormal{k}}}
\def\rl{{\textnormal{l}}}
% rm is already a command, just don't name any random variables m
\def\rn{{\textnormal{n}}}
\def\ro{{\textnormal{o}}}
\def\rp{{\textnormal{p}}}
\def\rq{{\textnormal{q}}}
\def\rr{{\textnormal{r}}}
\def\rs{{\textnormal{s}}}
\def\rt{{\textnormal{t}}}
\def\ru{{\textnormal{u}}}
\def\rv{{\textnormal{v}}}
\def\rw{{\textnormal{w}}}
\def\rx{{\textnormal{x}}}
\def\ry{{\textnormal{y}}}
\def\rz{{\textnormal{z}}}

% Random vectors
\def\rvepsilon{{\mathbf{\epsilon}}}
\def\rvphi{{\mathbf{\phi}}}
\def\rvtheta{{\mathbf{\theta}}}
\def\rva{{\mathbf{a}}}
\def\rvb{{\mathbf{b}}}
\def\rvc{{\mathbf{c}}}
\def\rvd{{\mathbf{d}}}
\def\rve{{\mathbf{e}}}
\def\rvf{{\mathbf{f}}}
\def\rvg{{\mathbf{g}}}
\def\rvh{{\mathbf{h}}}
\def\rvu{{\mathbf{i}}}
\def\rvj{{\mathbf{j}}}
\def\rvk{{\mathbf{k}}}
\def\rvl{{\mathbf{l}}}
\def\rvm{{\mathbf{m}}}
\def\rvn{{\mathbf{n}}}
\def\rvo{{\mathbf{o}}}
\def\rvp{{\mathbf{p}}}
\def\rvq{{\mathbf{q}}}
\def\rvr{{\mathbf{r}}}
\def\rvs{{\mathbf{s}}}
\def\rvt{{\mathbf{t}}}
\def\rvu{{\mathbf{u}}}
\def\rvv{{\mathbf{v}}}
\def\rvw{{\mathbf{w}}}
\def\rvx{{\mathbf{x}}}
\def\rvy{{\mathbf{y}}}
\def\rvz{{\mathbf{z}}}

% Elements of random vectors
\def\erva{{\textnormal{a}}}
\def\ervb{{\textnormal{b}}}
\def\ervc{{\textnormal{c}}}
\def\ervd{{\textnormal{d}}}
\def\erve{{\textnormal{e}}}
\def\ervf{{\textnormal{f}}}
\def\ervg{{\textnormal{g}}}
\def\ervh{{\textnormal{h}}}
\def\ervi{{\textnormal{i}}}
\def\ervj{{\textnormal{j}}}
\def\ervk{{\textnormal{k}}}
\def\ervl{{\textnormal{l}}}
\def\ervm{{\textnormal{m}}}
\def\ervn{{\textnormal{n}}}
\def\ervo{{\textnormal{o}}}
\def\ervp{{\textnormal{p}}}
\def\ervq{{\textnormal{q}}}
\def\ervr{{\textnormal{r}}}
\def\ervs{{\textnormal{s}}}
\def\ervt{{\textnormal{t}}}
\def\ervu{{\textnormal{u}}}
\def\ervv{{\textnormal{v}}}
\def\ervw{{\textnormal{w}}}
\def\ervx{{\textnormal{x}}}
\def\ervy{{\textnormal{y}}}
\def\ervz{{\textnormal{z}}}

% Random matrices
\def\rmA{{\mathbf{A}}}
\def\rmB{{\mathbf{B}}}
\def\rmC{{\mathbf{C}}}
\def\rmD{{\mathbf{D}}}
\def\rmE{{\mathbf{E}}}
\def\rmF{{\mathbf{F}}}
\def\rmG{{\mathbf{G}}}
\def\rmH{{\mathbf{H}}}
\def\rmI{{\mathbf{I}}}
\def\rmJ{{\mathbf{J}}}
\def\rmK{{\mathbf{K}}}
\def\rmL{{\mathbf{L}}}
\def\rmM{{\mathbf{M}}}
\def\rmN{{\mathbf{N}}}
\def\rmO{{\mathbf{O}}}
\def\rmP{{\mathbf{P}}}
\def\rmQ{{\mathbf{Q}}}
\def\rmR{{\mathbf{R}}}
\def\rmS{{\mathbf{S}}}
\def\rmT{{\mathbf{T}}}
\def\rmU{{\mathbf{U}}}
\def\rmV{{\mathbf{V}}}
\def\rmW{{\mathbf{W}}}
\def\rmX{{\mathbf{X}}}
\def\rmY{{\mathbf{Y}}}
\def\rmZ{{\mathbf{Z}}}

% Elements of random matrices
\def\ermA{{\textnormal{A}}}
\def\ermB{{\textnormal{B}}}
\def\ermC{{\textnormal{C}}}
\def\ermD{{\textnormal{D}}}
\def\ermE{{\textnormal{E}}}
\def\ermF{{\textnormal{F}}}
\def\ermG{{\textnormal{G}}}
\def\ermH{{\textnormal{H}}}
\def\ermI{{\textnormal{I}}}
\def\ermJ{{\textnormal{J}}}
\def\ermK{{\textnormal{K}}}
\def\ermL{{\textnormal{L}}}
\def\ermM{{\textnormal{M}}}
\def\ermN{{\textnormal{N}}}
\def\ermO{{\textnormal{O}}}
\def\ermP{{\textnormal{P}}}
\def\ermQ{{\textnormal{Q}}}
\def\ermR{{\textnormal{R}}}
\def\ermS{{\textnormal{S}}}
\def\ermT{{\textnormal{T}}}
\def\ermU{{\textnormal{U}}}
\def\ermV{{\textnormal{V}}}
\def\ermW{{\textnormal{W}}}
\def\ermX{{\textnormal{X}}}
\def\ermY{{\textnormal{Y}}}
\def\ermZ{{\textnormal{Z}}}

% Vectors
\def\vzero{{\bm{0}}}
\def\vone{{\bm{1}}}
\def\vmu{{\bm{\mu}}}
\def\vtheta{{\bm{\theta}}}
\def\vphi{{\bm{\phi}}}
\def\va{{\bm{a}}}
\def\vb{{\bm{b}}}
\def\vc{{\bm{c}}}
\def\vd{{\bm{d}}}
\def\ve{{\bm{e}}}
\def\vf{{\bm{f}}}
\def\vg{{\bm{g}}}
\def\vh{{\bm{h}}}
\def\vi{{\bm{i}}}
\def\vj{{\bm{j}}}
\def\vk{{\bm{k}}}
\def\vl{{\bm{l}}}
\def\vm{{\bm{m}}}
\def\vn{{\bm{n}}}
\def\vo{{\bm{o}}}
\def\vp{{\bm{p}}}
\def\vq{{\bm{q}}}
\def\vr{{\bm{r}}}
\def\vs{{\bm{s}}}
\def\vt{{\bm{t}}}
\def\vu{{\bm{u}}}
\def\vv{{\bm{v}}}
\def\vw{{\bm{w}}}
\def\vx{{\bm{x}}}
\def\vy{{\bm{y}}}
\def\vz{{\bm{z}}}

% Elements of vectors
\def\evalpha{{\alpha}}
\def\evbeta{{\beta}}
\def\evepsilon{{\epsilon}}
\def\evlambda{{\lambda}}
\def\evomega{{\omega}}
\def\evmu{{\mu}}
\def\evpsi{{\psi}}
\def\evsigma{{\sigma}}
\def\evtheta{{\theta}}
\def\eva{{a}}
\def\evb{{b}}
\def\evc{{c}}
\def\evd{{d}}
\def\eve{{e}}
\def\evf{{f}}
\def\evg{{g}}
\def\evh{{h}}
\def\evi{{i}}
\def\evj{{j}}
\def\evk{{k}}
\def\evl{{l}}
\def\evm{{m}}
\def\evn{{n}}
\def\evo{{o}}
\def\evp{{p}}
\def\evq{{q}}
\def\evr{{r}}
\def\evs{{s}}
\def\evt{{t}}
\def\evu{{u}}
\def\evv{{v}}
\def\evw{{w}}
\def\evx{{x}}
\def\evy{{y}}
\def\evz{{z}}

% Matrix
\def\mA{{\bm{A}}}
\def\mB{{\bm{B}}}
\def\mC{{\bm{C}}}
\def\mD{{\bm{D}}}
\def\mE{{\bm{E}}}
\def\mF{{\bm{F}}}
\def\mG{{\bm{G}}}
\def\mH{{\bm{H}}}
\def\mI{{\bm{I}}}
\def\mJ{{\bm{J}}}
\def\mK{{\bm{K}}}
\def\mL{{\bm{L}}}
\def\mM{{\bm{M}}}
\def\mN{{\bm{N}}}
\def\mO{{\bm{O}}}
\def\mP{{\bm{P}}}
\def\mQ{{\bm{Q}}}
\def\mR{{\bm{R}}}
\def\mS{{\bm{S}}}
\def\mT{{\bm{T}}}
\def\mU{{\bm{U}}}
\def\mV{{\bm{V}}}
\def\mW{{\bm{W}}}
\def\mX{{\bm{X}}}
\def\mY{{\bm{Y}}}
\def\mZ{{\bm{Z}}}
\def\mBeta{{\bm{\beta}}}
\def\mPhi{{\bm{\Phi}}}
\def\mLambda{{\bm{\Lambda}}}
\def\mSigma{{\bm{\Sigma}}}

% Tensor
\DeclareMathAlphabet{\mathsfit}{\encodingdefault}{\sfdefault}{m}{sl}
\SetMathAlphabet{\mathsfit}{bold}{\encodingdefault}{\sfdefault}{bx}{n}
\newcommand{\tens}[1]{\bm{\mathsfit{#1}}}
\def\tA{{\tens{A}}}
\def\tB{{\tens{B}}}
\def\tC{{\tens{C}}}
\def\tD{{\tens{D}}}
\def\tE{{\tens{E}}}
\def\tF{{\tens{F}}}
\def\tG{{\tens{G}}}
\def\tH{{\tens{H}}}
\def\tI{{\tens{I}}}
\def\tJ{{\tens{J}}}
\def\tK{{\tens{K}}}
\def\tL{{\tens{L}}}
\def\tM{{\tens{M}}}
\def\tN{{\tens{N}}}
\def\tO{{\tens{O}}}
\def\tP{{\tens{P}}}
\def\tQ{{\tens{Q}}}
\def\tR{{\tens{R}}}
\def\tS{{\tens{S}}}
\def\tT{{\tens{T}}}
\def\tU{{\tens{U}}}
\def\tV{{\tens{V}}}
\def\tW{{\tens{W}}}
\def\tX{{\tens{X}}}
\def\tY{{\tens{Y}}}
\def\tZ{{\tens{Z}}}


% Graph
\def\gA{{\mathcal{A}}}
\def\gB{{\mathcal{B}}}
\def\gC{{\mathcal{C}}}
\def\gD{{\mathcal{D}}}
\def\gE{{\mathcal{E}}}
\def\gF{{\mathcal{F}}}
\def\gG{{\mathcal{G}}}
\def\gH{{\mathcal{H}}}
\def\gI{{\mathcal{I}}}
\def\gJ{{\mathcal{J}}}
\def\gK{{\mathcal{K}}}
\def\gL{{\mathcal{L}}}
\def\gM{{\mathcal{M}}}
\def\gN{{\mathcal{N}}}
\def\gO{{\mathcal{O}}}
\def\gP{{\mathcal{P}}}
\def\gQ{{\mathcal{Q}}}
\def\gR{{\mathcal{R}}}
\def\gS{{\mathcal{S}}}
\def\gT{{\mathcal{T}}}
\def\gU{{\mathcal{U}}}
\def\gV{{\mathcal{V}}}
\def\gW{{\mathcal{W}}}
\def\gX{{\mathcal{X}}}
\def\gY{{\mathcal{Y}}}
\def\gZ{{\mathcal{Z}}}

% Sets
\def\sA{{\mathbb{A}}}
\def\sB{{\mathbb{B}}}
\def\sC{{\mathbb{C}}}
\def\sD{{\mathbb{D}}}
% Don't use a set called E, because this would be the same as our symbol
% for expectation.
\def\sF{{\mathbb{F}}}
\def\sG{{\mathbb{G}}}
\def\sH{{\mathbb{H}}}
\def\sI{{\mathbb{I}}}
\def\sJ{{\mathbb{J}}}
\def\sK{{\mathbb{K}}}
\def\sL{{\mathbb{L}}}
\def\sM{{\mathbb{M}}}
\def\sN{{\mathbb{N}}}
\def\sO{{\mathbb{O}}}
\def\sP{{\mathbb{P}}}
\def\sQ{{\mathbb{Q}}}
\def\sR{{\mathbb{R}}}
\def\sS{{\mathbb{S}}}
\def\sT{{\mathbb{T}}}
\def\sU{{\mathbb{U}}}
\def\sV{{\mathbb{V}}}
\def\sW{{\mathbb{W}}}
\def\sX{{\mathbb{X}}}
\def\sY{{\mathbb{Y}}}
\def\sZ{{\mathbb{Z}}}

% Entries of a matrix
\def\emLambda{{\Lambda}}
\def\emA{{A}}
\def\emB{{B}}
\def\emC{{C}}
\def\emD{{D}}
\def\emE{{E}}
\def\emF{{F}}
\def\emG{{G}}
\def\emH{{H}}
\def\emI{{I}}
\def\emJ{{J}}
\def\emK{{K}}
\def\emL{{L}}
\def\emM{{M}}
\def\emN{{N}}
\def\emO{{O}}
\def\emP{{P}}
\def\emQ{{Q}}
\def\emR{{R}}
\def\emS{{S}}
\def\emT{{T}}
\def\emU{{U}}
\def\emV{{V}}
\def\emW{{W}}
\def\emX{{X}}
\def\emY{{Y}}
\def\emZ{{Z}}
\def\emSigma{{\Sigma}}

% entries of a tensor
% Same font as tensor, without \bm wrapper
\newcommand{\etens}[1]{\mathsfit{#1}}
\def\etLambda{{\etens{\Lambda}}}
\def\etA{{\etens{A}}}
\def\etB{{\etens{B}}}
\def\etC{{\etens{C}}}
\def\etD{{\etens{D}}}
\def\etE{{\etens{E}}}
\def\etF{{\etens{F}}}
\def\etG{{\etens{G}}}
\def\etH{{\etens{H}}}
\def\etI{{\etens{I}}}
\def\etJ{{\etens{J}}}
\def\etK{{\etens{K}}}
\def\etL{{\etens{L}}}
\def\etM{{\etens{M}}}
\def\etN{{\etens{N}}}
\def\etO{{\etens{O}}}
\def\etP{{\etens{P}}}
\def\etQ{{\etens{Q}}}
\def\etR{{\etens{R}}}
\def\etS{{\etens{S}}}
\def\etT{{\etens{T}}}
\def\etU{{\etens{U}}}
\def\etV{{\etens{V}}}
\def\etW{{\etens{W}}}
\def\etX{{\etens{X}}}
\def\etY{{\etens{Y}}}
\def\etZ{{\etens{Z}}}

% The true underlying data generating distribution
\newcommand{\pdata}{p_{\rm{data}}}
\newcommand{\ptarget}{p_{\rm{target}}}
\newcommand{\pprior}{p_{\rm{prior}}}
\newcommand{\pbase}{p_{\rm{base}}}
\newcommand{\pref}{p_{\rm{ref}}}

% The empirical distribution defined by the training set
\newcommand{\ptrain}{\hat{p}_{\rm{data}}}
\newcommand{\Ptrain}{\hat{P}_{\rm{data}}}
% The model distribution
\newcommand{\pmodel}{p_{\rm{model}}}
\newcommand{\Pmodel}{P_{\rm{model}}}
\newcommand{\ptildemodel}{\tilde{p}_{\rm{model}}}
% Stochastic autoencoder distributions
\newcommand{\pencode}{p_{\rm{encoder}}}
\newcommand{\pdecode}{p_{\rm{decoder}}}
\newcommand{\precons}{p_{\rm{reconstruct}}}

\newcommand{\laplace}{\mathrm{Laplace}} % Laplace distribution

\newcommand{\E}{\mathbb{E}}
\newcommand{\Ls}{\mathcal{L}}
\newcommand{\R}{\mathbb{R}}
\newcommand{\emp}{\tilde{p}}
\newcommand{\lr}{\alpha}
\newcommand{\reg}{\lambda}
\newcommand{\rect}{\mathrm{rectifier}}
\newcommand{\softmax}{\mathrm{softmax}}
\newcommand{\sigmoid}{\sigma}
\newcommand{\softplus}{\zeta}
\newcommand{\KL}{D_{\mathrm{KL}}}
\newcommand{\Var}{\mathrm{Var}}
\newcommand{\standarderror}{\mathrm{SE}}
\newcommand{\Cov}{\mathrm{Cov}}
% Wolfram Mathworld says $L^2$ is for function spaces and $\ell^2$ is for vectors
% But then they seem to use $L^2$ for vectors throughout the site, and so does
% wikipedia.
\newcommand{\normlzero}{L^0}
\newcommand{\normlone}{L^1}
\newcommand{\normltwo}{L^2}
\newcommand{\normlp}{L^p}
\newcommand{\normmax}{L^\infty}

\newcommand{\parents}{Pa} % See usage in notation.tex. Chosen to match Daphne's book.

\DeclareMathOperator*{\argmax}{arg\,max}
\DeclareMathOperator*{\argmin}{arg\,min}

\DeclareMathOperator{\sign}{sign}
\DeclareMathOperator{\Tr}{Tr}
\let\ab\allowbreak


\usepackage{hyperref}
\usepackage{url}
\usepackage{pifont}
\usepackage{colortbl}
\usepackage{amsmath, amssymb}
\usepackage{enumitem}
\usepackage{graphicx}
\usepackage{subcaption}
\usepackage{booktabs}
\usepackage{algorithm}
\usepackage{algpseudocode}
\usepackage{tcolorbox}

\tcbset{
  myaddedbox/.style={
    colback=blue!5,
    colframe=blue!75,
    fonttitle=\bfseries,
    boxrule=0.5mm,
    arc=2mm,
    left=2mm,
    right=2mm,
    top=2mm,
    bottom=2mm
  }
}
\tcbset{
  simplebox/.style={
    colback=white,     % Background color (white for no fill)
    colframe=blue!75,    % Border color
    boxrule=0.4pt,     % Thin border
    sharp corners,     % No rounded corners
    left=1mm,          % Minimal padding
    right=1mm,
    top=1mm,
    bottom=1mm
  }
}
\usepackage{listings}
\newcommand{\dcold}{\mathrm{D_{cold}}}
\newcommand{\dwarm}{\mathrm{D_{warm}}}
\newcommand{\ds}{\mathrm{D}}
\lstdefinelanguage{Markdown}{
  keywords={},
  sensitive=false,
}

\lstset{
    language=Markdown,
    basicstyle=\ttfamily,
    breaklines=true,
    columns=fullflexible
}

\title{Learning Autonomous Code Integration for Math Language Models}

% Authors must not appear in the submitted version. They should be hidden
% as long as the \iclrfinalcopy macro remains commented out below.
% Non-anonymous submissions will be rejected without review.
\author{Haozhe Wang$^{\dagger\ddagger}$, Long Li$^\dagger$, Chao Qu$^\dagger$, Fengming Zhu$^\ddagger$, Weidi Xu$^\dagger$, Wei Chu$^\dagger$, Fangzhen Lin$^\ddagger$\\
\texttt{jasper.whz@outlook.com}\\
INF Technology$^\dagger$, Hong Kong University of Science and Technology$^\ddagger$
}
% \author{Antiquus S.~Hippocampus, Natalia Cerebro \& Amelie P. Amygdale \thanks{ Use footnote for providing further information
% about author (webpage, alternative address)---\emph{not} for acknowledging
% funding agencies.  Funding acknowledgements go at the end of the paper.} \\
% Department of Computer Science\\
% Cranberry-Lemon University\\
% Pittsburgh, PA 15213, USA \\
% \texttt{\{hippo,brain,jen\}@cs.cranberry-lemon.edu} \\
% \And
% Ji Q. Ren \& Yevgeny LeNet \\
% Department of Computational Neuroscience \\
% University of the Witwatersrand \\
% Joburg, South Africa \\
% \texttt{\{robot,net\}@wits.ac.za} \\
% \AND
% Coauthor \\
% Affiliation \\
% Address \\
% \texttt{email}
% }

% The \author macro works with any number of authors. There are two commands
% used to separate the names and addresses of multiple authors: \And and \AND.
%
% Using \And between authors leaves it to \LaTeX{} to determine where to break
% the lines. Using \AND forces a linebreak at that point. So, if \LaTeX{}
% puts 3 of 4 authors names on the first line, and the last on the second
% line, try using \AND instead of \And before the third author name.
\definecolor{LightCyan}{rgb}{0.88,1,1}
\newcommand{\xmark}{\ding{55}} % Define xmark as a cross symbol
\newcommand{\fix}{\marginpar{FIX}}
\newcommand{\new}{\marginpar{NEW}}
\newcommand{\red}[1]{\textcolor{red}{#1}}
\newcommand{\blue}[1]{\textcolor{blue}{#1}}
\newcommand{\up}{$\uparrow$}
\newcommand{\down}{$\downarrow$}


\begin{document}


\maketitle
\begin{abstract}


The choice of representation for geographic location significantly impacts the accuracy of models for a broad range of geospatial tasks, including fine-grained species classification, population density estimation, and biome classification. Recent works like SatCLIP and GeoCLIP learn such representations by contrastively aligning geolocation with co-located images. While these methods work exceptionally well, in this paper, we posit that the current training strategies fail to fully capture the important visual features. We provide an information theoretic perspective on why the resulting embeddings from these methods discard crucial visual information that is important for many downstream tasks. To solve this problem, we propose a novel retrieval-augmented strategy called RANGE. We build our method on the intuition that the visual features of a location can be estimated by combining the visual features from multiple similar-looking locations. We evaluate our method across a wide variety of tasks. Our results show that RANGE outperforms the existing state-of-the-art models with significant margins in most tasks. We show gains of up to 13.1\% on classification tasks and 0.145 $R^2$ on regression tasks. All our code and models will be made available at: \href{https://github.com/mvrl/RANGE}{https://github.com/mvrl/RANGE}.

\end{abstract}



\section{Introduction}
Backdoor attacks pose a concealed yet profound security risk to machine learning (ML) models, for which the adversaries can inject a stealth backdoor into the model during training, enabling them to illicitly control the model's output upon encountering predefined inputs. These attacks can even occur without the knowledge of developers or end-users, thereby undermining the trust in ML systems. As ML becomes more deeply embedded in critical sectors like finance, healthcare, and autonomous driving \citep{he2016deep, liu2020computing, tournier2019mrtrix3, adjabi2020past}, the potential damage from backdoor attacks grows, underscoring the emergency for developing robust defense mechanisms against backdoor attacks.

To address the threat of backdoor attacks, researchers have developed a variety of strategies \cite{liu2018fine,wu2021adversarial,wang2019neural,zeng2022adversarial,zhu2023neural,Zhu_2023_ICCV, wei2024shared,wei2024d3}, aimed at purifying backdoors within victim models. These methods are designed to integrate with current deployment workflows seamlessly and have demonstrated significant success in mitigating the effects of backdoor triggers \cite{wubackdoorbench, wu2023defenses, wu2024backdoorbench,dunnett2024countering}.  However, most state-of-the-art (SOTA) backdoor purification methods operate under the assumption that a small clean dataset, often referred to as \textbf{auxiliary dataset}, is available for purification. Such an assumption poses practical challenges, especially in scenarios where data is scarce. To tackle this challenge, efforts have been made to reduce the size of the required auxiliary dataset~\cite{chai2022oneshot,li2023reconstructive, Zhu_2023_ICCV} and even explore dataset-free purification techniques~\cite{zheng2022data,hong2023revisiting,lin2024fusing}. Although these approaches offer some improvements, recent evaluations \cite{dunnett2024countering, wu2024backdoorbench} continue to highlight the importance of sufficient auxiliary data for achieving robust defenses against backdoor attacks.

While significant progress has been made in reducing the size of auxiliary datasets, an equally critical yet underexplored question remains: \emph{how does the nature of the auxiliary dataset affect purification effectiveness?} In  real-world  applications, auxiliary datasets can vary widely, encompassing in-distribution data, synthetic data, or external data from different sources. Understanding how each type of auxiliary dataset influences the purification effectiveness is vital for selecting or constructing the most suitable auxiliary dataset and the corresponding technique. For instance, when multiple datasets are available, understanding how different datasets contribute to purification can guide defenders in selecting or crafting the most appropriate dataset. Conversely, when only limited auxiliary data is accessible, knowing which purification technique works best under those constraints is critical. Therefore, there is an urgent need for a thorough investigation into the impact of auxiliary datasets on purification effectiveness to guide defenders in  enhancing the security of ML systems. 

In this paper, we systematically investigate the critical role of auxiliary datasets in backdoor purification, aiming to bridge the gap between idealized and practical purification scenarios.  Specifically, we first construct a diverse set of auxiliary datasets to emulate real-world conditions, as summarized in Table~\ref{overall}. These datasets include in-distribution data, synthetic data, and external data from other sources. Through an evaluation of SOTA backdoor purification methods across these datasets, we uncover several critical insights: \textbf{1)} In-distribution datasets, particularly those carefully filtered from the original training data of the victim model, effectively preserve the model’s utility for its intended tasks but may fall short in eliminating backdoors. \textbf{2)} Incorporating OOD datasets can help the model forget backdoors but also bring the risk of forgetting critical learned knowledge, significantly degrading its overall performance. Building on these findings, we propose Guided Input Calibration (GIC), a novel technique that enhances backdoor purification by adaptively transforming auxiliary data to better align with the victim model’s learned representations. By leveraging the victim model itself to guide this transformation, GIC optimizes the purification process, striking a balance between preserving model utility and mitigating backdoor threats. Extensive experiments demonstrate that GIC significantly improves the effectiveness of backdoor purification across diverse auxiliary datasets, providing a practical and robust defense solution.

Our main contributions are threefold:
\textbf{1) Impact analysis of auxiliary datasets:} We take the \textbf{first step}  in systematically investigating how different types of auxiliary datasets influence backdoor purification effectiveness. Our findings provide novel insights and serve as a foundation for future research on optimizing dataset selection and construction for enhanced backdoor defense.
%
\textbf{2) Compilation and evaluation of diverse auxiliary datasets:}  We have compiled and rigorously evaluated a diverse set of auxiliary datasets using SOTA purification methods, making our datasets and code publicly available to facilitate and support future research on practical backdoor defense strategies.
%
\textbf{3) Introduction of GIC:} We introduce GIC, the \textbf{first} dedicated solution designed to align auxiliary datasets with the model’s learned representations, significantly enhancing backdoor mitigation across various dataset types. Our approach sets a new benchmark for practical and effective backdoor defense.




\section{Background}
% \begin{tcolorbox}[simplebox]
% We first formally define the problem and highlight its challenge. 
% Then we present an EM approach to address this challenge. 
% \end{tcolorbox}
% \vspace{-0.3cm}
% \subsection{Problem Statement }\label{sec_ps}

% Here’s a polished and enriched version of your problem formulation section, with improved clarity, precision, and academic tone:

% ---
\begin{figure}[t]
    \centering % Center the figure
    \includegraphics[width=\linewidth]{figs/example.pdf} % Include the figure
    \caption{\small \textbf{Example of Autonomous Code Integration.} \small We aim to enable LLMs to determine tool-usage strategies
based on their own capability boundaries. In the example, the model write code to solve the problem that demand special tricks, strategically bypassing its inherent limitations.} 
    \label{fig_example}
    \vspace{-0.2cm}
\end{figure}
\textbf{Problem Statement.} Modern tool-augmented language models address mathematical problems \( x_q \in \mathcal{X}_Q \) by generating step-by-step solutions that interleave natural language reasoning with executable Python code (Fig.~\ref{fig_example}). Formally, given a problem \( x_q \), a model \( \mathcal{M}_\theta \) iteratively constructs a solution \( y_a = \{y_1, \dots, y_T\} \) by sampling components \( y_t \sim p(y_t | y_{<t}, x_q) \), where \( y_{<t} \) encompasses both prior reasoning steps, code snippets and execution results \( \mathbf{e}_t \) from a Python interpreter. The process terminates upon generating an end token, and the solution is evaluated via a binary reward \( r(y_a,x_q) = \mathbb{I}(y_a \equiv y^*) \) indicating equivalence to the ground truth \( y^* \). The learning objective is formulated as:
\[
\max_{\theta} \mathbb{E}_{x_q \sim \mathcal{X}_Q} \left[r(y_a, x_q) \right]
\]

\noindent\textbf{Challenge and Motivation.} Developing autonomous code integration (AutoCode) strategies poses unique challenges, as optimal tool-usage behaviors must dynamically adapt to a model's intrinsic capabilities and problem-solving contexts. While traditional supervised fine-tuning (SFT) relies on imitation learning from expert demonstrations, this paradigm fundamentally limits the emergence of self-directed tool-usage strategies. Unfortunately, current math LLMs predominantly employ SFT to orchestrate tool integration~\citep{mammoth, tora, dsmath, htl}, their rigid adherence to predefined reasoning templates therefore struggles with the dynamic interplay between a model’s evolving problem-solving competencies and the adaptive tool-usage strategies required for diverse mathematical contexts.

Reinforcement learning (RL) offers a promising alternative by enabling trial-and-error discovery of autonomous behaviors. Recent work like DeepSeek-R1~\citep{dsr1} demonstrates RL's potential to enhance reasoning without expert demonstrations. However, we observe that standard RL methods (e.g., PPO~\cite{ppo}) suffer from a critical inefficiency (see Sec.~\ref{sec_ablation}): Their tendency to exploit local policy neighborhoods leads to insufficient exploration of the vast combinatorial space of code-integrated reasoning paths, especially when only given a terminal reward in mathematical problem-solving.

To bridge this gap, we draw inspiration from human metacognition -- the iterative process where learners refine tool-use strategies through deliberate exploration, outcome analysis, and belief updates. A novice might initially attempt manual root-finding via algebraic methods, observe computational bottlenecks or inaccuracies, and therefore prompting the usage of calculators. Through systematic reflection on these experiences, they internalize the contextual efficacy of external tools, gradually forming stable heuristics that balance reasoning with judicious tool invocation. 


To this end, \emph{our focus diverges from standard agentic tool-use frameworks~\citep{agentr}}, which merely prioritize successful tool execution. Instead, \emph{we aim to instill \emph{human-like metacognition} in LLMs, enabling them to (1) determine tool-usage based on their own capability boundaries (see the analysis in Sec.~\ref{sec_ablation}), and (2) dynamically adapt tool-usage strategies as their reasoning abilities evolve (via our EM framework).}
% For instance, while an LLM might solve a combinatorics problem via CoT alone, it should autonomously invoke code for eigenvalue calculations in linear algebra where symbolic computations are error-prone. Achieving this requires models to \emph{jointly optimize} their reasoning and tool-integration policies in a mutually reinforcing manner.


% Mirroring this metacognitive cycle, we propose an Expectation-Maximization (EM) framework that allows LLMs to develop AutoCode strategies via guided exploration (the E-step) and self-refinement (the M-step).


% \vspace{-0.3cm}
\section{Methodology}

Inspired by human metacognitive processes, we introduce an Expectation-Maximization (EM) framework that trains LLMs for autonomous code integration (AutoCode) through alternations (Fig.~\ref{fig_overview}):

\begin{enumerate}[leftmargin=0.5cm,topsep=1pt,itemsep=0pt,parsep=0pt]
    \item \emph{Guided Exploration (E-step):} Identifies high-potential code-integrated solutions by systematically probing the model's inherent capabilities.
\item \emph{Self-Refinement (M-step):} Optimizes the model's tool-usage strategy and chain-of-thought reasoning using curated trajectories from the E-step.
\end{enumerate}


\begin{figure*}[t]
    \centering
    \includegraphics[width=\linewidth]{figs/overview.pdf}
    \caption{\small \textbf{Method Overview.} \small (Left) shows an overview for the EM framework, which alternates between finding a reference strategy for guided exploration (E-step) and off-policy RL (M-step). (Right) shows the data curation for guided exploration. We generate \(K\) rollouts, estimate values of code-triggering decisions and subsample the initial data with sampling weights per Eq.~\ref{eq_sampling}.}
    \label{fig_overview}
\end{figure*}

\subsection{The EM Framework for AutoCode}

A central challenge in AutoCode lies in the code triggering decisions, represented by the binary decision \(c \in \{0, 1\}\).  While supervised fine-tuning (SFT) suffers from missing ground truth for these decisions, standard reinforcement learning (RL) struggles with the combinatorial explosion of code-integrated reasoning paths. Our innovation bridges these approaches through systematic exploration of both code-enabled (\(c=1\)) and non-code (\(c=0\)) solution paths, constructing reference decisions for policy optimization.

We formalize this idea within a maximum likelihood estimation (MLE) framework. Let \( P (r=1 | x_q;\theta\) denote the probability of generating a correct response to query \( x_q \) under model \(\mathcal{M}_\theta\). Our objective becomes:
\begin{align}
    \mathcal{J}_{\mathrm{MLE}}(\theta) \doteq \log P(r=1 | x_q; \theta) \label{eq_mle}
\end{align}
This likelihood depends on two latent factors: (1) the code triggering decision \(\pi_\theta(c | x_q)\) and (2) the solution generation process \(\pi_\theta(y_a | x_q, c)\). Here, for notation-wise clarity, we consider  code-triggering decision at a solution's beginning (\( c\) following \(x_q\) immediately). We show generalization to mid-reasoning code integration in Sec.~\ref{sec_impl}.

The EM framework provides a principled way to optimize this MLE objective in the presence of latent variables~\cite{prml}. We derive the evidence lower bound (ELBO): \( \mathcal{J}_{\mathrm{ELBO}}(s, \theta) \doteq \)
\begin{align}
    % \mathcal{J}_{\mathrm{MLE}}(\theta) &
    % \ge 
    \mathbb{E}_{s(c | x_q)}\left[\log \frac{\pi_\theta(c | x_q) \cdot P(r=1 | c, x_q; \theta)}{s(c | x_q)}\right] 
    % \\
     \label{eq_elbo}
\end{align}
where \(s(c | x_q)\) serves as a surrogate distribution approximating optimal code triggering strategies. It is also considered as the reference decisions for code integration. 

\noindent\textbf{E-step: Guided Exploration}  computes the reference strategy \(s(c | x_q)\) by maximizing the ELBO, equivalent to minimizing the KL-divergence: \( \max_s \mathcal{J}_{\mathrm{ELBO}}(s, \theta) = \)
\begin{align}
     - \mathrm{D_{KL}}\left(s(c | x_q) \| P(r=1, c | x_q; \theta)\right) \label{eq_estep}
\end{align}

The reference strategy \(s(c | x_q)\) thus approximates the posterior distribution over code-triggering decisions \(c\) that maximize correctness, i.e., \(P(r=1, c | x_q; \theta)\).  Intuitively, it guides exploration by prioritizing decisions with high potential: if decision \(c\) is more likely to lead to correct solutions, the reference strategy assigns higher probability mass to it, providing guidance for the subsequent RL procedure.

\noindent\textbf{M-step: Self-Refinement } updates the model parameters \(\theta\) through a composite objective:
\begin{multline}
\max_\theta \mathcal{J}_{\mathrm{ELBO}}(s, \theta) =\mathbb{E}_{\substack{c \sim s(c|x_q) \\ y_a \sim \pi_\theta(y_a|x_q, c)}} \Big[ r(x_q, y_a) \Big] \\- \mathcal{CE}\Big(s(c|x_q) \,\|\, \pi_\theta(c|x_q)\Big)\label{eq_mstep}
\end{multline}
The first term implements reward-maximizing policy gradient updates for solution generation, while while the second aligns native code triggering with reference strategies through cross-entropy minimization (see Fig.~\ref{fig_overview} for an illustration of the optimization). This dual optimization jointly enhances both tool-usage policies and reasoning capabilities.



\subsection{Practical Implementation}\label{sec_impl}
In the above EM framework, we alternate between finding a reference strategy \( s \) for code-triggering decisions  in the E-step, and perform reinforcement learning under the guidance from \( s \) in the M-step. We implement this framework through an iterative process of offline data curation and off-policy RL.

\noindent\textbf{Offline Data Curation.} We implement the E-step through Monte Carlo rollouts and subsampling. For each problem \(x_q\), we estimate the reference strategy as an energy distribution: 
\begin{equation}
    s^\ast(c | x_q)  = \frac{\exp\left(\alpha\cdot \pi_\theta(c | x_q) Q(x_q,c;\theta)\right)}{Z(x_q)}.\label{eq_sampling}
\end{equation}
where \( Q(x_q,c;\theta)\) estimates the expected value through \( K \) rollouts per decision, \(\pi_\theta(c|x_q) \) represents the model's current prior and the \( Z(x_q) \) is the partition function to ensure normalization. Intuitively, the strategy will assign higher probability mass to the decision \( c \) that has higher expected value \( Q(x_q,c;\theta)\) meanwhile balancing its intrinsic preference \( \pi_\theta(c|x_q)\). 

Our curation pipeline proceeds through: 
\begin{itemize}[leftmargin=0.5cm,topsep=1pt,itemsep=0pt,parsep=0pt]
\item Generate \(K\) rollouts for \(c=0\) (pure reasoning) and \(c=1\) (code integration), creating candidate dataset \(\mathcal{D}\).  
\item Compute \(Q(x_q,c)\) as the expected success rate across rollouts for each pair \((x_q,c)\).  
\item Subsample \(\mathcal{D}_{\text{train}}\) from \(\mathcal{D}\) using importance weights according to Eq.~\ref{eq_sampling}.  
\end{itemize}

To explicitly probe code-integrated solutions, we employ prefix-guided generation -- e.g., prepending prompts like \texttt{``Let’s first analyze the problem, then consider if python code could help''} -- to bias generations toward free-form code-reasoning patterns.

 This pipeline enables guided exploration by focusing on high-potential code-integrated trajectories identified by the reference strategy, contrasting with standard RL’s reliance on local policy neighborhoods. As demonstrated in Sec.~\ref{sec_ablation}, this strategic data curation significantly improves training efficiency by shaping the exploration space.





\noindent\textbf{Off-Policy RL.}
To mitigate distributional shifts caused by mismatches between offline data and the policy, we optimize a clipped off-policy RL objective. The refined M-step (Eq.~\ref{eq_mstep}) becomes:
\begin{multline}
    % \max_\theta 
    \underset{(x_q,y_a)}{\mathbb{E}}\left[
\text{clip}\left(\frac{\pi_\theta(y_a|x_q)}{\pi_{\text{ref}}(y_a|x_q)},1-\epsilon,1+\epsilon\right)\cdot A\right]
\\-\mathbb{E}_{(x_q,c)}\Big[\log \pi_\theta(c|x_q) \Big]\label{eq_finalm}
\end{multline}
where  \( (x_q, c, y_a) \) is sampled from the dataset \( \mathcal{D}_{\text{train}} \). The importance weight \(\frac{\pi_\theta(y_a|x_q)}{\pi_{\text{ref}}(y_a|x_q)}\) accounts for off-policy correction with PPO-like clipping. The advantage function \(A(x_q,y_a)\) is computed via query-wise reward normalization~\cite{ppo}. 

\noindent\textbf{Generalizing to Mid-Reasoning Code Integration.} Our method extends to mid-reasoning code integration by initiating Monte Carlo rollouts from partial solutions \((x_q, y_{<t})\). Notably, we observe emergence of mid-reasoning code triggers after initial warm-up with prefix-probed solutions. Thus, our implementation requires only two initial probing strategies: explicit prefix prompting for code integration and vanilla generation for pure reasoning, which jointly seed diverse mid-reasoning code usage in later iterations.



\section{Experiments}\label{sec_exp}
\vspace{-0.2cm}
Our experiments investigate three key research questions:

\noindent\emph{Q1: Method Effectiveness.} How does our approach enhance performance across both in-domain and out-of-domain mathematical benchmarks compared to existing math LLMs?

\noindent\emph{Q2: Baseline Comparisons.} How does our method compare to standard RL and SFT baselines in terms of training efficiency and exploration patterns?

\noindent\emph{Q3: AutoCode Analysis.} What strategies does the model learn for code integration, and how do these strategies contribute to performance gains?

\noindent\textbf{Datasets and Benchmarks.} Our method only requires a query set for training. We collect public available queries from MATH~\citep{math} and Numina~\cite{numina}, and sample \(7K\) queries based on difficulties. We upload the collected data to the annonymous repo. For evaluation, we employ: GSM8k~\citep{gsm8k}, MATH500~\citep{math}, GaokaoMath2023~\citep{mario}, OlympiadBench~\citep{olympiad}, the American Invitational Mathematics Examination (AIME24), and the American
Mathematics Competitions (AMC23). This benchmark suite spans elementary to Olympiad-level mathematics. We adopt Pass@1 accuracy~\citep{pass1, dsr1} as our primary metric, using evaluation scripts from DeepseekMath~\citep{dsmath} and Qwen2Math~\citep{yang2024qwen2}. For competition-level benchmarks (AIME/AMC), we use 64 samples with temperature 0.6 following Deepseek R1 protocols.

% Please add the following required packages to your document preamble:

% Beamer presentation requires \usepackage{colortbl} instead of \usepackage[table,xcdraw]{xcolor}
\begin{table*}[t]
\centering
\caption{Main Results. Eurus-2-7B-PRIME demonstrates the best reasoning ability.}
\label{tab:main_results}
\resizebox{\textwidth}{!}{
\begin{tabular}{lcccccc}
\toprule
\textbf{Model}                     & \textbf{AIME 2024}                           & \textbf{MATH-500} & \textbf{AMC}          & \textbf{Minerva Math} & \textbf{OlympiadBench} & \textbf{Avg.}          \\ \midrule
\textbf{GPT-4o}                    & 9.3                                          & 76.4              & 45.8                  & 36.8                  & \textbf{43.3}          & 43.3                   \\
\textbf{Llama-3.1-70B-Instruct}    & 16.7                                         & 64.6              & 30.1                  & 35.3                  & 31.9                   & 35.7                   \\
\textbf{Qwen-2.5-Math-7B-Instruct} & 13.3                                         & \textbf{79.8}     & 50.6                  & 34.6                  & 40.7                   & 43.8                   \\
\textbf{Eurus-2-7B-SFT}            & 3.3                                          & 65.1              & 30.1                  & 32.7                  & 29.8                   & 32.2                   \\
\textbf{Eurus-2-7B-PRIME}          & \textbf{26.7 {\color[HTML]{009901} (+23.3)}} & 79.2 {\color[HTML]{009901}(+14.1)}      & \textbf{57.8 {\color[HTML]{009901}(+27.7)}} & \textbf{38.6 {\color[HTML]{009901}(+5.9)}}  & 42.1 {\color[HTML]{009901}(+12.3) }          & \textbf{48.9 {\color[HTML]{009901}(+ 16.7)}} \\ \bottomrule
\end{tabular}
}
\end{table*}
\noindent\textbf{Baselines and Implementation.} 
We compare against three model categories: \begin{itemize}[leftmargin=0.5cm,itemsep=0pt,parsep=0pt]
\item Proprietary models: o1~\cite{o1}, GPT-4~\citep{gpt4} and Claude~\citep{claude}
\item Recent math-specialized LMs: NuminaMath~\citep{numina}, Mathstral~\citep{mathstral}, Mammoth~\citep{mammoth}, ToRA~\citep{tora}, DartMath~\cite{tong2024dartmath}. We do not compare with models that rely on test-time scaling, such as MCTS or long CoT. 
\item Foundation models enhanced with our method: Qwen2Math~\citep{yang2024qwen2}, DeepseekMath~\citep{dsmath} and Qwen-2.5~\cite{qwen25}.
\end{itemize}

Our implementation uses \( K = 8 \) rollouts per query (temperature=1.0, top-p=0.9). Training completes in about 10 hours on \(8\times\) A100 (80GB) GPUs across three epochs of 7K queries. We release code, models and data via an \href{https://anonymous.4open.science/r/AnnonySubmission-0C62}{anonymous repository}.

% \vspace{-0.1cm}
\subsection{Main Results}\label{sec_main}
Notably, we observe a minimum performance gain of 11\% on the MATH500 benchmark, escalating to an impressive 9.4\% absolute improvement on the highly challenging AIME benchmark.  Across in-domain benchmarks, our method yields an average improvement of 8.9\%, and for out-of-domain benchmarks, we achieve a substantial average gain of 6.98\%. These results  validate the effectiveness of our approach across model families and problem difficulty levels.  

\subsection{Ablation Study}\label{sec_ablation}
We conduct three primary analyses: (a) comparison with standard RL and SFT baselines to validate our method's effectiveness in facilitating exploration, (b) visualization of exploration patterns to reveal limitations in the standard RL paradigam, and (c) behavioral analysis of code integration strategies. These analyses collectively demonstrate our method's benefits in facilitating guided exploration and explains how it improves performance.

\begin{figure*}[t]
    \centering
    \includegraphics[width=0.95\linewidth]{figs/rl_curves.pdf}
    \caption{ \small \textbf{Training Efficiency and Convergence.} We benchmark the learning dynamics of our approach against three two training paradigms: supervised fine-tuning and reinforcement learning (RL). The Pass@1 accuracy is evaluated on an held-out dev-set. We use Qwen-2.5-Base as the base model. SFT is conducted using collected public data~\cite{openmath, mammoth}. The dashed lines indicate asymptotic performance. }\label{fig_training_efficiency}
    % \begin{minipage}{0.47\textwidth}
    %     \centering
    %     \begin{subfigure}[b]{1.0\textwidth}
    %         \centering
    %         \includegraphics[width=0.95\linewidth]{figs/abl_qwen_curve.pdf}
    %         % \caption{Top right image}
    %     \end{subfigure}
    %     % \vskip -0.3\baselineskip % Add vertical space between subfigures
    %     \begin{subfigure}[b]{1.0\textwidth}
    %         \centering
    %         \includegraphics[width=1.\linewidth]{figs/abl_deepseek_curve.pdf}
    %         % \caption{Bottom right image}
    %     \end{subfigure}
    %     \caption{ \small \textbf{Performance Convergence. } Experiments are conducted based on Qwen2Math (Top) and DeepseekMath (Bottom). AutoCode achieves higher accuracy with sustained improvement, while standard RL converge to sub-optimal solutions. }\label{fig_training_effici}
    % \end{minipage}
    % \hfill
    % \begin{minipage}{0.47\textwidth}
    %     \centering
    %     % \vspace{-0.3cm}
    %     % \hspace{-1.6cm}
    %     \includegraphics[width=1.\linewidth]{figs/abl_strategies.pdf}
    %     \caption{\small \textbf{Analysis of the Learned Strategies.} Correct Responses are classified based on their alignment to the oracle selection, namely, \emph{StrictAlign}, \emph{AllowCode} and \emph{MisAlign}. We show how different categories of alignment contribute to the accuracy in the stacked bars, and include the overall StrictAlign rate in the separate orange bar.} \label{fig_learned_strategies}
    % \end{minipage}
% \vspace{-0.3cm}
\end{figure*}
\noindent\textbf{Training Efficiency.} We evaluated the learning dynamics of our approach in direct comparison to three established training paradigms:
\begin{itemize}[leftmargin=0.5cm,itemsep=0pt,parsep=0pt]
\item \emph{Base+RL}:  On-policy Reinforcement Learning (RL) initialized from a base model without Supervised Fine-Tuning (SFT). This follows the methodology of DeepSeek R1, designed to isolate and assess the pure effects of RL training.
\item \emph{SFT}: Supervised Fine-Tuning, the prevailing training paradigm widely adopted in current tool-integrated math Language Models (LMs).
\item \emph{SFT+RL}: Standard RL applied after SFT, serving as a conventional baseline for evaluating our EM-based RL method.
\end{itemize}

From the figure, we make the following key observations: 


\begin{itemize}[leftmargin=0.5cm,itemsep=0pt,parsep=0pt]
   \item  While Reinforcement Learning directly from the base model (\emph{Base+RL}) exhibits consistent performance improvement, its training efficiency is lower than training paradigms incorporating SFT.  In addition, the model rarely explores code-integrated solutions, with the code invocation rate below 5\%. This strongly suggest that \emph{reinforcement learning tool-usage behavior from scratch is inherently inefficient}.
    \item SFT effectively provides a strong initialization point, but \emph{SFT alone exhibits limited asymptotic performance}. This suggests that SFT lacks the capacity to adapt and optimize beyond the scope of the expert demonstrations, thereby limiting further improvement. 
    \item Standard RL applied after SFT shows initial further improvement but subsequently plateaus, \emph{even after an extended training stage}.  This suggests \emph{the exploration-exploitation dilemma when applying RL for LLM post-training}: standard RL with vanilla rollout exploration tends to exploit local optima and insufficiently explores the combinatorial code-integrated trajectories.
\end{itemize}

To further substantiate the exploration limitations inherent in the conventional \emph{SFT+RL} paradigm, we present a visualization of the exploration patterns. We partitioned the model-generated responses during self-exploration into three distinct training phases and analyzed the statistical distribution of code invocation rates across queries as the model's policy evolved throughout training. As depicted in Figure~\ref{fig_visualize_explore}, the distribution of code invocation progressively concentrates towards the extremes – either minimal or maximal code use – indicating the model's growing tendency to exploit its local policy neighborhood. This exploitation manifests as a focus on refining established code-triggering decisions, rather than engaging in broader exploration of alternative approaches.



\begin{figure}[t]
    \centering % Center the figure
    \resizebox{1.\linewidth}{!}{\includegraphics[width=\linewidth]{figs/visualize_explore.pdf} }% Include the figure
    \vspace{-0.2cm}
    \caption{\small \textbf{Visualization of Exploration in the SFT+RL paradigm.} \small The distribution of code invocation rates \emph{across queries} to visualize policy's exploration of code-integrated trajectories. Without external guidance, LLM tends to exploit its local policy neighborhood, concentrating code usage toward extremes as training phase evolves. } \vspace{-0.2cm}
    \label{fig_visualize_explore} 
\end{figure}
These empirical observations lend strong support to our assertion that standard RL methods are susceptible to premature exploitation of the local policy space when learning AutoCode strategies. In sharp contrast, our proposed EM method facilitates a more guided exploration by sub-sampling trajectories according to the reference strategy (Sec.~\ref{sec_impl}). This enables continuous performance (evidenced in Sec.~\ref{sec_main}) and mitigating the risk of converging to suboptimal local optima (Fig.~\ref{fig_training_efficiency}).



\begin{figure}[t]
    \centering % Center the figure
    \resizebox{1.\linewidth}{!}{\includegraphics[width=\linewidth]{figs/learned_behavior.pdf}} % Include the figure
    \caption{\small \textbf{Analysis of AutoCode Strategies. }\small We compare AutoCode performance against scenarios where models explicitly prompted to utilize code or CoT, and consider the union of solved queries as the bound for AutoCode performance. Existing models show inferior AutoCode performance than explicit instructed, with their AutoCode strategies close to random (50\%). Our approach consistently improves AutoCode performance, with AutoCode selection accuracy near 90\%.  } 
    \label{fig_learned_behavior} 
\end{figure}
\noindent\textbf{Analysis on Code Integration Behaviors.}
We investigated the properties of the learned code integration strategies to gain deeper insights into the mechanisms behind our method's performance gains. Our central hypothesis posits that optimal code integration unlocks synergistic performance benefits by effectively combining the strengths of CoT and code executions.  This synergy presents a "free lunch" scenario: a well-learned metacognitive tool-usage strategy can elevate overall performance, provided the model demonstrates competence in solving \emph{distinct} subsets of queries using either CoT or code execution.

To empirically validate this "free lunch" principle and demonstrate the superiority of our approach in realizing it, we benchmarked our model against baselines that inherently support both code execution and Chain-of-Thought (CoT) reasoning: GPT-4, Mammoth-70B, and DeepseekMath-Instruct-7B. Our analysis evaluated the model's autonomous decision to invoke code when not explicitly instructed on which strategy to employ. We compared this "AutoCode" performance against scenarios where models were explicitly prompted to utilize either code or CoT reasoning. We also considered the theoretical "free lunch" upper bound – the accuracy achieved by combining the successful predictions from either strategy (i.e., taking the union of queries solved by CoT or code).

As visually presented in Figure~\ref{fig_learned_behavior}, existing baseline models exhibit inferior performance in AutoCode mode compared to scenarios where code invocation is explicitly prompted, e.g., DeepseekMath-Instruct-7B shows a degradation of 11.54\% in AutoCode mode. This suggests that their AutoCode strategies are often suboptimal, performing closer to random selection between CoT and code (selection accuracy near 50\%), resulting in AutoCode falling between the performance of explicitly triggered CoT and code. In contrast, our models learn more effective code integration strategies.  AutoCode4Math-Qwen2.5, for example, improves upon explicitly code-triggered performance by 7\%, indicating a true synergistic integration of reasoning and code execution.


To quantify the effectiveness of these learned "AutoCode" strategies, we calculated the CoT/code selection accuracy. We used the outcome of explicit instruction (i.e., performance when explicitly prompted for CoT or code) as a proxy for the ground-truth optimal method selection.  Our model achieves a selection accuracy of 89.53\%, showcasing the high efficacy of the learned code integration strategy.
\section{Related Work and Discussion}
\textbf{Tool-Integrated Math LLMs.} 
Math language models adopted two major paradigms: Chain-of-Thought (CoT) reasoning and the use of external tools, such as Python programs~\citep{metamath, mammoth, openmath}. Each paradigm offers unique benefits, and recent hybrid frameworks~\citep{mammoth, tora, htl, dsmath, qwen25} increasingly seek to combine them for synergy. However, current models exhibit critical rigidity, motivating our work to realize the true metacognitive capacity that enjoys synergistic benefits of CoT and code. 

\noindent\textbf{EM for RL.} Expectation-Maximization (EM) has proven effective for maximum likelihood problems involving hidden variables, such as Expert Iteration~\citep{expertiter}, Iterative Maximum Likelihood~\citep{iml, iml1}, Meta-Reinforcement Learning~\citep{varibad, vem}, and Adversarial Games~\citep{acb}. In the context of math LLMs, the most relevant works are \citep{restem} and \citep{iml2}, which apply EM-style iterative self-training to math problem-solving. Unlike these approaches, we leverage the EM framework for guided exploration during reinforcement learning of language models.

\noindent\textbf{Conclusion.} Existing tool-integrated math language models lack the metacognitive capacity to effectively determine code integration, hindering their ability to fully realize the synergistic benefits of tool integration and CoT.  To address this critical gap, we propose a novel EM-based framework that combines guided exploration with policy optimization.  Our experiments demonstrate the limitations of standard SFT and RL in efficiently exploring the combinatorial space of code-integrated trajectories and highlight the superior training efficiency and performance of our approach.

\clearpage

\noindent\textbf{Limitations.} 
The scope of our work is primarily focused on mathematical problem-solving.  While we observe promising results on challenging benchmarks like MATH500, the generalizability of our approach to other domains requiring the metacognitive capacity of tool integration and CoT, such as scientific reasoning or code generation for general-purpose tasks, remains to be explored.  Future work should investigate the effectiveness of our framework across a wider range of tasks and domains.

% The study of tool-integrated Language models is motivated by the belief that transformers are prone to cumulative errors due to the autoregressive probabilistic decoding and hence tool integration show promise to ensure the computational precision. However, such inaccuracies seem to be largely mitigated by performing test-time scaling techniques, thus reducing the significance of tool-integrated reasoning systems.   
% This paper investigates autonomous code integration for math LLMs. To address the challenge of unreliable external supervision, we propose to factorize out the hidden methodology-selection from response generation, and develop a novel EM formulation. The EM framework alternates between computing a reference strategy for methodology-selection through self-exploration and updating language model based on the reference guidance. This approach supports an efficient joint training scheme that allows for holistic offline data collection coupled with RL training. Our extensive experiments demonstrate the effectiveness of the proposed method, and our ablation studies further elucidate the properties of the learned model.

% However, there are several limitations and areas for future work regarding AutoCode4Math. First, the generalization of methodology-selection depends significantly on the quality of the collected query set. Further research is needed to understand what characteristics of queries contribute to effective generalization. Second, we did not extensively explore the influence of hyperparameters related to RL iterations, such as dataset size and the number of iterations, in the current version. We are actively working on this. Third, as this is a preliminary work in autonomous code integration, we have not yet investigated alternative approaches for decision routing, such as using Mixture-of-Experts (MoEs)~\citep{moe}, and we not yet fully understand the fundamental reason why EM outperforms RL. These areas present important directions for advancing AutoCode capabilities in math LLMs.

\bibliography{custom}
% \bibliographystyle{acl_natbib}
\clearpage
% \appendix
% \newpage
\appendix
\appendices

\section{Robot Setups}\label{app:robot_setup}

\subsection{Simulation Robot Setups}
To ensure fairness, we utilize the same Franka Panda arm for evaluations in both the LIBERO~\cite{LIBERO23} and our Open6DOR V2 benchmarks. For SIMPLER~\cite{simplerenv24}, we use the Google Robot exclusively to conduct the baseline experiments, adhering to all configurations outlined in SIMPLER, as presented in Table~\ref{tab:simpler_env}. 

\subsection{Real World Robot Setups}
As for manipulation tasks, in \cref{fig:robots}, we perform 6-DoF rearrangement tasks using the Franka Panda equipped with a gripper and the UR robot arm with a LeapHand, while articulated object manipulation is conducted using the Flexiv arm equipped with a suction tool. All the robot arms mount a Realsense D415 camera to its end for image capturing.
\begin{figure}[h!]
\centering
\includegraphics[width=0.96\linewidth]{figs/src/robots.pdf}
\vspace{-5pt}
\captionof{figure}{\textbf{The robots used in our real-world experiments.}}
\vspace{-5pt}
\label{fig:robots}
\end{figure}


In \cref{fig:franka_setup}, we present the workspace and robotic arm for real-world 6-DoF rearrangement. Unlike Rekep~\cite{ReKep24}, CoPa~\cite{CoPa24} et al., we utilize only a single RealSense D415 camera. This setup significantly reduces the additional overhead associated with environmental setup and multi-camera calibration, and it is more readily reproducible.
\begin{figure}[h!]
\centering
\includegraphics[width=1.0\linewidth]{figs/src/franka_setup.pdf}
\captionof{figure}{\textbf{6-DoF rearrangement robot setup.}}
\vspace{-10pt}
\label{fig:franka_setup}
\end{figure}


As for navigation tasks, we provide a visualization of our robotic dog in~\cref{fig:dog_setup}. Following Uni-Navid~\cite{uninavid24}, our robotic dog is Unitree GO2 and we mount a RealSense D455 camera on the head of the robotic dog. Here, we only use the RGB frames with a resolution of $640\times480$ in the setting of  $90^\circ$ HFOV. We also mount a portable Wi-Fi at the back of the robot dog, which is used to communicate with the remote server (send captured images and receive commands). Unitree GO2 is integrated with a LiDAR-L1, which is only used for local motion planning. 
\begin{figure}
\begin{center}
  \includegraphics[width=0.7\linewidth]{figs/src/robotdog.pdf}
\end{center}
   \caption{\textbf{Navigation robot setup.} We use Unitree GO2 as our embodiment, and we mount RealSense D455, a portable Wi-Fi and a LiDAR-L1. Note that, our model only takes RGB frames as input. The portable Wi-Fi is used for communication with the remote server and the Lidar is used for the local controller API of Unitree Dog.}
   \vspace{-10pt}
\label{fig:dog_setup}
\end{figure}


\section{Additional Experiments}\label{app:add_exp}

\subsection{Articulated Objects Manipulation Evaluation}
We further integrate \ours~with articulated object manipulation, as illustrated in \cref{tab:manip}, and evaluate its practicality in robotic manipulation tasks using the PartNet-Mobility Dataset within the SAPIEN~\cite{SAPIEN20} simulator. Our experimental setup follows ManipLLM~\cite{ManipLLM24}, employing the same evaluation metrics. Specifically, we directly utilize the segmentation centers provided by SAM as contact points, leverage PointSO to generate contact directions, and use VLM to determine subsequent motion directions. The results demonstrate significant improvements over the baseline. Notably, our model achieves this performance without dividing the data into training and testing sets, operating instead in a fully zero-shot across most tasks. This underscores the robustness and generalization of our approach.
% !TEX root = ../../top.tex
% !TEX spellcheck = en-US

% Temporary modification of the table column separation width
\setlength\mytabcolsep{\tabcolsep}
\setlength\tabcolsep{2pt}


\renewcommand{\manipimgcar}[1]{\includegraphics[width=0.13\linewidth]{#1}}
\renewcommand{\manipimgmix}[1]{\includegraphics[width=0.07\linewidth]{#1}}
\renewcommand{\manipimg}[1]{\includegraphics[width=0.08\linewidth]{#1}}


\begin{figure*}[t]
	\centering
	\small
%	\vspace{-3mm}
	\begin{tabular}{c|c|c}
		% Cars 
		\begin{tabular}{c|c}
			\manipimgcar{fig/manip/car_olivier_lat_489_277/init} & \manipimgcar{fig/manip/car_olivier_lat_34_308/init} \\
			\manipimgcar{fig/manip/car_olivier_lat_489_277/final} & \manipimgcar{fig/manip/car_olivier_lat_34_308/final} \\
		\end{tabular}
		&
		% Mixers 
		\begin{tabular}{cc|cc}
			\manipimgmix{fig/manip/mixer_lat_1424_171/init} & \manipimgmix{fig/manip/mixer_lat_1424_171/final} & \manipimgmix{fig/manip/mixer_lat_957_629/init} & \manipimgmix{fig/manip/mixer_lat_957_629/final} \\
		\end{tabular}
		&
		% Chairs
		\begin{tabular}{cc|cc}
			\manipimg{fig/manip/chair_sepreg_lat_550_531/init} & \manipimg{fig/manip/chair_sepreg_lat_550_531/final} & \manipimg{fig/manip/chair_sepreg_lat_745_952/init} & \manipimg{fig/manip/chair_sepreg_lat_745_952/final} \\
		\end{tabular} \\
		\midrule
		% Cars 2
		\begin{tabular}{c|c}
		\manipimgcar{fig/manip/car_olivier_pose_290/init} & \manipimgcar{fig/manip/car_olivier_pose_691/init} \\
		\manipimgcar{fig/manip/car_olivier_pose_290/final} & \manipimgcar{fig/manip/car_olivier_pose_691/final} \\
		\end{tabular}
		&
		% Mixers 2
		\begin{tabular}{cc|cc}
		\manipimgmix{fig/manip/mixer_pose_872/init} & \manipimgmix{fig/manip/mixer_pose_872/final} & \manipimgmix{fig/manip/mixer_pose_183/init} & \manipimgmix{fig/manip/mixer_pose_183/final} \\
		\end{tabular}
		&
		% Chairs 2
		\begin{tabular}{cc|cc}
		\manipimg{fig/manip/chair_sepreg_pose_422/init} & \manipimg{fig/manip/chair_sepreg_pose_422/final} & \manipimg{fig/manip/chair_sepreg_pose_859/init} & \manipimg{fig/manip/chair_sepreg_pose_859/final} \\
		\end{tabular} \\
	\end{tabular}
	\caption{\textbf{Additional shape manipulation.} We manipulate four shapes per dataset: (\textit{top}) by changing the latent of specific parts (car body, mixer helix, and all chair parts) and (\textit{bottom}) by editing part poses (car wheels, mixer width, chair width and height). In all cases, the parts adapts to the modifications and to each other, maintaining a coherent whole.}
	\label{fig:supp-manip}
\end{figure*}


% Restore table column separation width
\setlength{\tabcolsep}{\mytabcolsep}


\subsection{Spatial VQA on EmbSpatial-Bench~\cite{embspatial24} \& SpatialBot-Bench~\cite{SpatialBot24}}
To further demonstrate \sofar's spatial reasoning capabilities, we conducted Spatial VQA tests within the EmbSpatial-Bench~\cite{embspatial24} and SpatialBot-Bench~\cite{SpatialBot24}. As shown in \cref{tab:embspatial,tab:spatialbot}, \sofar~significantly outperformed all baselines, achieving more than a 20\% performance improvement in EmbSpatial-Bench.
\begin{table}[h!]
\centering
\setlength{\tabcolsep}{8.5pt}
\caption{Zero-shot performance of LVLMs in EmbSpatial-Bench~\cite{embspatial24}. \textbf{Bold} indicates the best results.}
\resizebox{0.97\linewidth}{!}{
\begin{tabular}{lcc}
\toprule
Model & Generation & Likelihood \\
\midrule
BLIP-2~\cite{BLIP223} & 37.99 & 35.71 \\
InstructBLIP~\cite{InstructBLIP23} & 38.85 & 33.41 \\
MiniGPT4~\cite{MiniGPT4_23} & 23.54 & 31.70 \\
LLaVA-1.6~\cite{LLaVA23} & 35.19 & 38.84 \\
\midrule
GPT-4V~\cite{GPT4Vision23} & 36.07 & - \\
\rowcolor{linecolor1}Qwen-VL-Max~\cite{qwenvl23} & 49.11 & - \\
\rowcolor{linecolor2}\textbf{\ours} & \textbf{70.88} & - \\
\bottomrule
\end{tabular}
}
\label{tab:embspatial}
\end{table}
\begin{table}[h!]
\centering
\setlength{\tabcolsep}{3pt}
\caption{Zero-shot performance of LVLMs in SpatialBot-Bench~\cite{SpatialBot24}.
SpatialBot-3B: SpatialBot-Phi2-3B-RGB, SpatialBot-8B: SpatialBot-Llama3-8B-RGB.
}
\resizebox{1.00\linewidth}{!}{
\begin{tabular}{lcccccc}
\toprule
Model & Pos & Exist & Count & Reach & Size & Avg \\
\midrule
ChatGPT-4o~\cite{GPT4o24} & 70.6 & 85.0 & 84.5 & 51.7 & \textbf{43.3} & 67.0\\
SpatialBot-3B~\cite{SpatialBot24} & 64.7 & 80.0 & 88.0 & \textbf{61.7} & 28.3 & 64.5\\
SpatialBot-8B~\cite{SpatialBot24} & 55.9 & 80.0 & \textbf{91.2} & 40.0 & 20.0 & 57.4\\
\midrule
\rowcolor{linecolor2}\textbf{\ours} & \textbf{76.5} & \textbf{87.5} & 80.0 & 57.5 & 40.0 & \textbf{68.3}\\
\bottomrule
\end{tabular}
}
\label{tab:spatialbot}
\end{table}


\subsection{Close-Loop Execution Experiment}\label{app:close_loop}
We demonstrate the closed-loop replan capabilities of \sofar~within Simpler-Env~\cite{simplerenv24} in \cref{fig:close_loop}. The instruction for both tasks is ``pick the coke can'' In \cref{fig:close_loop} (a), the model initially misidentified the coke can as a Fanta can. After correction by the VLM, the model re-identified and located the correct object. In \cref{fig:close_loop} (b), the model accidentally knocks over the Coke can during motion due to erroneous motion planning. Subsequently, the model re-plans and successfully achieves the grasp.

\subsection{Long Horizon Object Manipulation Experiment}\label{app:long_horizon}
\cref{fig:long_horizon} illustrates the execution performance of our model on long-horizon tasks. Through the VLM~\cite{GPT4o24,gemini23}, complex instructions such as ``making breakfast'' and ``cleaning up the desktop'' can be decomposed into sub-tasks. In the second example, we deliberately chose complex and uncommon objects as assets, such as ``Aladdin's lamp'' and ``puppets'', but \sofar~is able to successfully complete all tasks.
\begin{figure}[h!]
\centering
\includegraphics[width=1.0\linewidth]{figs/src/close_loop_execution.pdf}
\vspace{-15pt}
\captionof{figure}{\textbf{Close-loop execution of our \sofar.}}
\label{fig:close_loop}
\end{figure}


\subsection{In the Wild Evaluation of Semantic Orientation}
We provide a qualitative demonstration of the accuracy of PointSO under in-the-wild conditions, as shown in \cref{fig:in_the_wild}, where the predicted Semantic Orientation is marked in the images. We obtained single-sided point clouds by segmenting objects using Florence-2~\cite{florence2} and SAM~\cite{SAM23} and fed them into PointSO. It can be observed that our model achieves good performance across different views, objects, and instructions, which proves the effectiveness and generalization of PointSO.
\begin{figure*}[h!]
\centering
\includegraphics[width=1.0\linewidth]{figs/src/long_horizon.pdf}
\captionof{figure}{\textbf{Long-horizon object manipulation experiment of our \sofar.}}
\label{fig:long_horizon}
\end{figure*}

\begin{figure}[h!]
\centering
\includegraphics[width=1.0\linewidth]{figs/src/in_the_wild.pdf}
\vspace{-15pt}
\captionof{figure}{\textbf{In the wild evaluation of PointSO.}}
\label{fig:in_the_wild}
\end{figure}

\begin{figure}[t!]
\centering
\includegraphics[width=1.0\linewidth]{figs/src/cross_view.pdf}
\captionof{figure}{\textbf{Cross view generalization of our \sofar.}}
\label{fig:cross_view}
\end{figure}


\subsection{Cross-View Generalization}
\sofar~gets point clouds in the world coordinate system using an RGB-D camera to obtain grasping poses, and it is not limited to a fixed camera perspective. In addition, PointSO generates partial point clouds from different perspectives through random camera views to serve as data augmentation for training data, which also generalizes to camera perspectives in the real world. \cref{fig:cross_view} illustrates \sofar's generalization capability for 6-DoF object manipulation across different camera poses. It can be observed that whether it's a front view, side view, or ego view, \sofar~can successfully execute the ``upright the bottle'' instruction.

\subsection{Failure Case Distribution Analysis}
Based on the failure cases from real-world experiments, we conducted a quantitative analysis of the failure case distribution for \sofar, with the results shown in \cref{fig:failure_case}. It can be observed that 31\% of the failures originated from grasping issues, including objects being too small, inability to generate reasonable grasping poses, and instability after grasping leading to sliding or dropping. Next, 23\% were due to incorrect or inaccurate Semantic Orientation prediction. For tasks such as upright or upside - down, highly precise angle estimation (<5°) is required for smooth execution. Object analysis and detection accounted for approximately 20\% of the errors. The instability of open-vocabulary detection modules like Florence2~\cite{florence2} and Grounding DINO~\cite{groundingdino23} often led to incorrect detection of out-of-distribution objects or object parts. In addition, since our Motion Planning did not take into account the working space range of the robotic arm and potential collisions of the manipulated object, occasional deadlocks and collisions occurred during motion. Finally, there were issues with the Task Planning of the VLM~\cite{GPT4o24,gemini23}. For some complex Orientations, the VLM occasionally failed to infer the required angles and directions to complete the task. Employing a more powerful, thought-enabled VLM~\cite{gpt_o1} might alleviate such errors.

\begin{figure*}[t]
\centering
\includegraphics[width=\linewidth]{CameraReady/Figures/failure_case_image.pdf}
\caption{Failure cases of our approach.}
\label{fig:failure_case}
\end{figure*}

\subsection{Ablation Study}\label{app:ablation}
\subsubsection{Scaling Law}
The scaling capability of models and data is one of the most critical attributes today and a core feature of foundation models~\cite{FoundationModel21}. We investigate the performance of PointSO across different data scales, as illustrated in \cref{tab:scaling_law}. 
We obtain the subset for OrienText300K from Objaverse-LVIS, which consists of approximately 46,000 3D objects with category annotations. The selection was based on the seven criteria mentioned in the main text. Objects meeting all seven criteria formed the strict subset, comprising around 15k objects. When including objects without textures and those of lower quality, the total increases to approximately 26k objects.
It can be seen that the increase in data volume is the most significant factor driving the performance improvement of PointSO. It can be anticipated that with further data expansion, such as Objaverse-XL~\cite{ObjaverseXL23}, PointSO will achieve better performance.
\begin{table}[t!]
\setlength{\tabcolsep}{7pt}
\caption{\textbf{Data scaling property} of semantic orientation with different training data scales evaluated on OrienText300K test split. All experiments are conducted with PointSO-Base.
}
\label{tab:scaling_law}
\centering
\resizebox{1.0\linewidth}{!}{
\begin{tabular}{lccccc}
\toprule[0.95pt]
    Data Scale & \texttt{45°} & \texttt{30°} & \texttt{15°} & \texttt{5°} & Average \\ 
    \midrule[0.6pt]
    5\% & 57.03 & 46.09 & 39.84 & 27.34 & 42.58 \\
    10\% & 61.72 & 53.13 & 43.75 & 30.47 & 47.27 \\
    50\% & 76.56 & 72.66 & 66.41 & 56.25 & 67.97 \\
    \rowcolor{linecolor2}100\% & \textbf{79.69} & \textbf{77.34} & \textbf{70.31} & \textbf{62.50} & \textbf{72.46} \\
    \bottomrule[0.95pt]
\end{tabular}
}
\end{table}

\subsubsection{Cross-Modal Fusion Choices}\label{app:fusion}
We further conduct an ablation study on the multi-modal fusion methods in PointSO, testing commonly used feature fusion techniques such as cross-attention, multiplication, addition, and concatenation, as shown in \cref{tab:fusion}. The results indicate that simple addition achieves the best performance. This may be attributed to the fact that instructions in the semantic domain are typically composed of short phrases or sentences, and the text CLS token already encodes sufficiently high-level semantic information.
% \usepackage{multirow}
% \usepackage{booktabs}


\begin{table}[h]
	\centering
	\caption{Model performance under two coefficient fusion methods.}
	\begin{tabular}{c|ccc} 
		\toprule
		\multirow{2}{*}{Fusion mode} & \multicolumn{3}{c}{R40}                           \\
		& Easy           & Moderate       & Hard            \\ 
		\hline
		straight                     & 90.92          & 82.84          & 80.29           \\
		our                          & \textbf{91.96} & \textbf{83.31} & \textbf{80.59}  \\
		\bottomrule
	\end{tabular}
\label{tabel6}
\end{table}

\begin{table*}[h!]
\setlength{\tabcolsep}{3pt}
\caption{\textbf{Ablation study of open vocabulary detection modules} on Open6DOR~\cite{Open6DOR24} perception tasks.
}
\label{tab:detection_ab}
\centering
\resizebox{0.98\linewidth}{!}{
\begin{tabular}{lccccccccccc}
\toprule[0.95pt]
\multirow{2}{*}[-0.5ex]{Method} & \multicolumn{3}{c}{\textbf{Position Track}} & \multicolumn{4}{c}{\textbf{Rotation Track}} & \multicolumn{3}{c}{\textbf{6-DoF Track}} & \multirow{2}{*}[-0.5ex]{Time Cost (s)}\\
\cmidrule(lr){2-4} \cmidrule(lr){5-8} \cmidrule(lr){9-11}
& Level 0 & Level 1 & Overall & Level 0 & Level 1 & Level 2 & Overall & Position & Rotation & Overall \\ 
\midrule[0.6pt]
YOLO-World~\cite{yoloworld24} & 59.0 & 37.7 & 53.3 & 48.3 & 36.1 & 62.0 & 44.9 & 53.4 & 44.6 & 27.8 & \textbf{7.4s}\\
\rowcolor{linecolor1}Grounding DINO~\cite{groundingdino23} & 92.2 & 71.5 & 86.7 & 64.7 & 41.1 & 69.8 & 55.5 & 87.2 & 51.6 & 44.6 & 9.2s\\
\rowcolor{linecolor2}Florence-2~\cite{florence2} & \textbf{96.0} & \textbf{81.5} & \textbf{93.0} & \textbf{68.6} & \textbf{42.2} & \textbf{70.1} & \textbf{57.0} & \textbf{92.7} & \textbf{52.7} & \textbf{48.7} & \textbf{8.5s}\\
\bottomrule[0.95pt]
\end{tabular}
}
\end{table*}

\subsubsection{Open Vocabulary Object Detection Module}
\sofar~utilize a detection foundation model to localize the interacted objects or parts, then generate masks with SAM~\cite{SAM23}. Although not the SOTA performance on the COCO benchmark, Florence-2~\cite{florence2} exhibits remarkable generalization in in-the-wild detection tasks, even in simulator scenarios. \cref{tab:detection_ab} illustrates the performance of various detection modules in Open6DOR~\cite{Open6DOR24} Perception, where Florence-2 achieves the best results and outperforms Grounding DINO~\cite{groundingdino23} and YOLO-World~\cite{yoloworld24}.

\vspace{3pt}
\section{Additional Implementation Details}\label{app:implementation_details}

\subsection{Detail Real World Experiment Results}\label{app:detail_realworld}
To fully demonstrate the generalization of \sofar~rather than cherry-picking, we carefully design 60 different real-world experimental tasks, covering more than 100 different and diverse objects. Similar to the Open6DOR~\cite{Open6DOR24} benchmark in the simulator, we divide these 60 tasks into three parts: position-track, orientation-track, and the most challenging comprehensive \& 6-DoF-track. Each track is further divided into simple and hard levels. The position-simple track includes tasks related to front \& back \& left \& right spatial relationships, while the position-hard track includes tasks related to between, center, and customized. The orientation-simple track includes tasks related to the orientation of object parts, and the orientation-hard track includes tasks related to whether the object is upright or flipped (with very strict requirements for angles in both upright and flipped cases). Comprehensive tasks involve complex instruction understanding and long-horizon tasks; 6-DoF tasks simultaneously include requirements for both object position and orientation instructions. In \cref{tab:detailed_realworld}, we present the complete task instructions, as well as the performance metrics of \sofar~and the baseline. Due to the large number of tasks, we performed each task three times. It can be seen that \sofar~achieves the best performance in all tracks, especially in the orientation-track and comprehensive \& 6-DoF-track. We also show all the objects used in the real-world experiments in \cref{fig:real_obj}, covering a wide range of commonly and uncommonly used objects in daily life.

\vspace{-10pt}
\begin{table*}[t!]
\setlength{\tabcolsep}{6pt}
\caption{\textbf{Detailed zero-shot real-world 6-DoF rearrangement results}.}
\label{tab:detailed_realworld}
\centering
\resizebox{0.96\linewidth}{!}{
\begin{tabular}{lcccc}
\toprule[0.95pt]
    Task & CoPa~\cite{CoPa24} & ReKep-Auto~\cite{ReKep24} & \sofar-LLaVA~(Ours) & \sofar~(Ours) \\ 
    \midrule[0.6pt]
    \multicolumn{5}{c}{\textit{Positional Object Manipulation}}\\
    \midrule
    Move the soccer ball to the right of the bread. & 2/3 & 3/3 & 3/3 & \textbf{3/3} \\
    Place the doll to the right of the lemon. & 3/3 & 3/3 & 3/3 & \textbf{3/3} \\
    Put the pliers on the right side of the soccer ball. & 1/3 & 1/3 & 3/3 & \textbf{2/3} \\
    Move the pen to the right of the doll. & 3/3 & 2/3 & 3/3 & \textbf{3/3} \\
    Place the carrot on the left of the croissant. & 2/3 & 3/3 & 2/3 & \textbf{2/3} \\
    Move the avocado to the left of the baseball. & 3/3 & 2/3 & 2/3 & \textbf{3/3} \\
    Pick the pepper and place it to the left of the charger. & 1/3 & 2/3 & 2/3 & \textbf{2/3} \\
    Place the baseball on the left side of the mug. & 3/3 & 2/3 & 2/3 & \textbf{3/3} \\
    Arrange the flower in front of the potato. & 2/3 & 3/3 & 2/3 & \textbf{3/3} \\
    Put the volleyball in front of the knife. & 3/3 & 3/3 & 3/3 & \textbf{3/3} \\
    Place the ice cream cone in front of the potato. & 2/3 & 3/3 & 2/3 & \textbf{3/3} \\
    Move the bitter melon to the front of the forklift. & 2/3 & 1/3 & 2/3 & \textbf{2/3} \\
    Place the orange at the back of the stapler. & 3/3 & 2/3 & 3/3 & \textbf{3/3} \\
    Move the panda toy to the back of the shampoo bottle. & 2/3 & 3/3 & 3/3 & \textbf{2/3} \\
    pick the pumpkin and place it behind the pomegranate. & \textbf{3/3} & 2/3 & 1/3 & 2/3 \\
    Place the basketball at the back of the board wipe. & 2/3 & 2/3 & 3/3 & \textbf{2/3} \\
    Put the apple inside the box. & 3/3 & 2/3 & 3/3 & \textbf{3/3} \\
    Place the waffles on the center of the plate. & 3/3 & 2/3 & 3/3 & \textbf{3/3} \\
    Move the hamburger into the bowl.& 2/3 & 2/3 & 2/3 & \textbf{3/3} \\
    Pick the puppet and put it into the basket. & 1/3 & 2/3 & 2/3 & \textbf{2/3} \\
    Drop the grape into the box. & 2/3 & 3/3 & 3/3 & \textbf{2/3} \\
    Put the doll between the lemon and the USB. & 2/3 & 2/3 & 2/3 & \textbf{3/3} \\
    Set the duck toy in the center of the cart, bowl, and camera. & 2/3 & 1/3 & 2/3 & \textbf{2/3} \\
    Place the strawberry between the Coke bottle and the glue. & 2/3 & 2/3 & 3/3 & \textbf{3/3} \\
    Put the pen behind the basketball and in front of the vase. & 2/3 & 1/3 & 2/3 & \textbf{2/3} \\
    Total success rate& 74.7\% & 72.0\% & 81.3\% & \textbf{85.3\%} \\
    \midrule
    \multicolumn{5}{c}{\textit{Orientational Object Manipulation}}\\
    \midrule
    Turn the yellow head of the toy car to the right. & 2/3 & 2/3 & 1/3 & \textbf{2/3} \\
    Adjust the knife handle so it points to the right. & 2/3 & 1/3 & 2/3 & \textbf{2/3} \\
    Rotate the cap of the bottle towards the right. & 2/3 & 2/3 & 2/3 & \textbf{2/3}\\
    Rotate the tip of the screwdriver to face the right. & 0/3 & 0/3 & 1/3 & \textbf{1/3}\\
    Rotate the stem of the apple to the right. & 0/3 & 1/3 & 1/3 & \textbf{2/3}\\
    Turn the front of the toy car to the left. & 0/3 & 0/3 & 2/3 & \textbf{2/3} \\
    Rotate the cap of the bottle towards the left. & 2/3 & 1/3 & 1/3 & \textbf{2/3}\\
    Adjust the pear's stem to the right. & 1/3 & 1/3 & 1/3 & \textbf{1/3}\\
    Turn the mug handle to the right. & 1/3 & 1/3 & 2/3 & \textbf{2/3}\\
    Rotate the handle of the mug to towards right.& 2/3 & 1/3 &\textbf{2/3} & 1/3\\
    Rotate the box so the text side faces forward. & 0/3 & 1/3 & 0/3 & \textbf{1/3}\\
    Adjust the USB port to point forward. & 0/3 & 0/3 & 1/3 & \textbf{1/3}\\
    Set the bottle upright. & 0/3 & 1/3 & 0/3 & \textbf{1/3}\\
    Place the coffee cup in an upright position. & 1/3 & 1/3 & 2/3 & \textbf{2/3}\\
    Upright the statue of liberty& 0/3 & 0/3 & \textbf{1/3} & 0/3\\
    Stand the doll upright. & 0/3 & 1/3 & 0/3 & \textbf{1/3}\\
    Right the Coke can. & 0/3 & 0/3 & 1/3 & \textbf{1/3}\\
    Flip the bottle upside down. & 0/3 & 0/3 & 0/3 & \textbf{1/3}\\
    Turn the coffee cup upside down. & 0/3 & 0/3 & 1/3 & \textbf{1/3}\\
    Invert the shampoo bottle upside down. & 0/3 & 0/3 & 0/3 & \textbf{0/3}\\
    Total success rate& 21.7\% & 23.3\% & 35.0\% & \textbf{43.3\%} \\
    \midrule
    \multicolumn{5}{c}{\textit{Comprehensive 6-DoF Object Manipulation}}\\
    \midrule
    Pull out a tissue.& 3/3 & 3/3 & 2/3 & \textbf{3/3}\\
    Place the right bottle into the
    box and arrange it in a 3×3 pattern. & 0/3 & 0/3 & 0/3 & \textbf{1/3}\\
    Take the tallest box and position it on the right side. & 1/3 & 1/3 & 3/3 & \textbf{3/3}\\
    Grasp the error bottle and put it on the right side. & 1/3 & 2/3 & 1/3 & \textbf{2/3} \\
    Take out the green test tube and place it between the two bottles. & 2/3 & 2/3 & 3/3 & \textbf{3/3}\\
    Pack the objects on the table into the box one by one. & 1/3 & 1/3 & 0/3 & \textbf{1/3}\\
    Rotate the loopy doll to face the yellow dragon doll & 0/3 & 1/3 & 1/3 & \textbf{1/3}\\
    Right the fallen wine glass and arrange it neatly in a row. & 0/3 & 0/3 & 0/3 & \textbf{0/3}\\
    Grasp the handle of the knife and cut the bread.& 0/3 & 0/3 & 0/3 & \textbf{1/3}\\
    Pick the baseball into the cart and turn the cart to facing right. & 0/3 & 0/3 & 1/3 & \textbf{2/3}\\
    Place the mug on the left of the ball and the handle turn right. & 0/3 & 0/3 & 1/3 & \textbf{1/3}\\
    Aim the camera at the toy truck. & 1/3 & 0/3 & 1/3 & \textbf{1/3}\\
    Rotate the flashlight to illuminate the loopy. & 0/3 & 0/3 & 1/3 & \textbf{1/3}\\
    Put the pen into the pen container. & 0/3 & 1/3 & 0/3 & \textbf{1/3} \\
    Pour out chips from the chips cylinder to the plate. & 0/3 & 1/3 & 1/3 & \textbf{1/3} \\
    Total success rate& 20.0\% & 26.7\% & 33.3\% & \textbf{48.9\%} \\
    \bottomrule[0.95pt]
\end{tabular}
}
\end{table*}
\vspace{-10pt}
\begin{figure}[t!]
\centering
\includegraphics[width=1.0\linewidth]{figs/src/real_obj.pdf}
\captionof{figure}{\textbf{The real-world assets used in our real-world experiments.} More than 100 diverse objects are used in our 6-DoF rearrangement experiments.}
\label{fig:real_obj}
\vspace{-10pt}
\end{figure}


\input{tabs/PointSO_configurations}
\usepackage{graphicx}
\usepackage{url}
\usepackage{color, soul}
\usepackage{bm}
\usepackage{amsmath} % for the equation* environment
\usepackage{lineno}

\subsection{PointSO Model Details}\label{app:pointso_details}
For PointSO, we utilize FPS + KNN to perform patchify and employ a small PointNet~\cite{PointNet} as the patch encoder. Subsequently, a standard Transformer encoder is adopted as the backbone, followed by a single linear layer to map the output to a three-dimensional vector space. All parameter configurations follow prior work on point cloud representation learning~\cite{ACT23,ReCon23,ShapeLLM24}. Detailed hyperparameter and model configurations are provided in \cref{tab:hyper_params,tab:PointSO_configs}.

\subsection{SoFar-LLaVA Model Details}\label{app:model_details}
\begin{figure*}[t!]
\begin{center}
\includegraphics[width=0.89\linewidth]{figs/src/sofar_llava.pdf}
\caption{\textbf{Pipeline of \sofar-LLaVA}, a fine-tuned vision language model based on visual instruction tuning.
}
\label{fig:sofar_llava}
\vspace{-5pt}
\end{center}
\end{figure*}

In addition to leveraging the extensive knowledge and strong generalization capabilities of closed-source/open-source pretrained VLMs~\cite{ChatGPT22,gemini23,qwenvl23} for zero-shot or in-context learning, \sofar~can also enhance the planning performance of open-source models through visual instruction tuning for rapid fine-tuning. The pipeline of the model is illustrated in \cref{fig:sofar_llava}. A JSON-formatted 6-DoF scene graph, processed through a text tokenizer, along with the image refined by SoM~\cite{SoM23}, is fed into an LLM (\eg, LLaMA~\cite{LLaMA23,LLaMA2_23}) for supervised fine-tuning~\cite{LLaVA23}.
In the Open6DOR~\cite{Open6DOR24} task, we supplement the training dataset with additional samples retrieved and manually annotated from Objaverse~\cite{objaverse23}, ensuring alignment with the object categories in the original benchmark. This dataset includes approximately 3,000 6-DoF object manipulation instructions. Using this data, we construct dialogue-style training data based on ChatGPT and train the \sofar-LLaVA model. The training hyperparameters are detailed in \cref{tab:hyper_params}. Similarly, we finetune PointSO on this training dataset and achieve superior performance on the Open6DOR task.

\subsection{ChatGPT API Costs}
The knowledge of OrienText300K is derived from the annotations of 3D modelers on Sketchfab, combined with ChatGPT's filtering and comprehension capabilities. To generate semantic direction annotations, we filter the 800K dataset of Objaverse~\cite{objaverse23} and apply ChatGPT to approximately 350K of the filtered data to generate semantic text-view index pairs. The OpenAI official API was used for these calls, with the GPT-4o version set to 2024-08-06 and the output format configured as JSON. The total cost for debugging and execution amounted to approximately \$10K.


\section{Additional Benchmark Statistic Analysis}
\subsection{6-DoF SpatialBench Analysis}
We conduct a statistical analysis of the manually constructed 6-DoF SpatialBench, with category comparisons and word cloud visualizations shown in \cref{fig:spatialvqa_statistic}. We collect diverse image data from the internet, encompassing scenes such as indoor, outdoor, and natural landscapes. The questions may involve one or multiple objects, with varying levels of uncertainty in image resolution. Most importantly, we are the first to propose a VQA benchmark for orientation understanding, focusing on both quantitative and qualitative evaluation of orientation.


\subsection{Open6DOR V2 Analysis}
Open6DOR V2 builds upon Open6DOR V1 by removing some incorrectly labeled data and integrating assets and metrics into Libero, enabling closed-loop policy evaluation. The detailed number of tasks is presented in \cref{tab:open6dorv2_statistic}, comprising over 4,500 tasks in total. Notably, we remove level 2 of the position track in Open6DOR V1~\cite{Open6DOR24} because it requires manual inspection, which is not conducive to open-source use and replication by the community. Besides, due to the randomness of object drops in the scene, approximately 8\% of the samples already satisfy the evaluation metrics in their initial state.

\vspace{3pt}
\section{Additional Related Works}\label{app:related_work}
\subsection{3D Representation Learning}
Research on 3D Representation Learning encompasses various methods, including point-based~\cite{PointNet,PointNet++}, voxel-based~\cite{voxelnet15}, and multiview-based approaches~\cite{MVCNN3D15,MVTN}. 
Point-based methods~\cite{PointNext,PointTrans21} have gained prominence in object classification~\cite{ModelNet15,ScanObjectNN19} due to their sparsity yet geometry-informative representation. On the other hand, voxel-based methods~\cite{voxelrcnn21,SyncSpecCNN17,VPP23} offer dense representation and translation invariance, leading to a remarkable performance in object detection~\cite{ScanNet17} and segmentation~\cite{ShapeNetPart16, S3DIS16}.
The evolution of attention mechanisms~\cite{AttentionIsAllYouNeed,ReKo23} has also contributed to the development of effective representations for downstream tasks, as exemplified by the emergence of 3D Transformers~\cite{PointTrans21,groupfree21, voxeltransformer21}. Notably, 3D self-supervised representation learning has garnered significant attention in recent studies. PointContrast~\cite{PointContrast20} utilizes contrastive learning across different views to acquire discriminative 3D scene representations. Innovations such as Point-BERT~\cite{PointBERT} and Point-MAE~\cite{PointMAE} introduce masked modeling~\cite{MAE,BERT} pretraining into the 3D domain. 
ACT~\cite{ACT23} pioneers cross-modal geometry understanding through 2D or language foundation models such as CLIP~\cite{CLIP} or BERT~\cite{BERT}. 
Following ACT, {\scshape ReCon}~\cite{ReCon23} further proposes a learning paradigm that unifies generative and contrastive learning. PPT~\cite{ppt24} highlights the significance of positional encoding in 3D representation learning
Additionally, leveraging foundation vision-language models like CLIP~\cite{ACT23,CLIP} has spurred the exploration of a new direction in open-world 3D representation learning. This line of work seeks to extend the applicability and adaptability of 3D representations in diverse and open-world/vocabulary scenarios~\cite{OpenScene23,CLIPFO3D23,PLA23,Lowis3D23,OVIR3D23,PointGCC23}.

\section{Additional Discussions}
\subsection{Relation to Affordance \& 6-DoF Pose Estimation}
Conceptually, this semantic orientation is a counterpart of \textit{affordance}~\citep{Affordance77,AffordanceHRI16,MoveWithAffordanceMaps20,HandsAsAffordancesProbes22} but beyond,
as SO and affordance all present potential actions and interactions with objects.
However, SO also contains the spatial understanding of intra-object part-level attributes more than affordance learning.
Compared to vanilla 6-DoF pose estimation, our proposed SO combined with the 3-DoF translation understanding has the same DoF completeness.
The difference is, our proposed SO is grounded by languages, making it useful for open-world manipulation requiring complicated spatial reasoning~\cite{RobotsThatUseLanguage20,SayCan22,Open6DOR24}. 
In addition, our Semantic Orientation can be auto-labeled from Internet 3D data that achieves higher scalability, introduced in the next section.
\begin{figure}[t!]
\begin{center}
% \includegraphics[width=0.85\linewidth]{figs/src/spatialvqa_statistic.pdf}
\includegraphics[width=\linewidth]{figs/src/spatialvqa_statistic.pdf}
\vspace{-15pt}
\caption{\textbf{6-DoF SpatialBench statistics}. (a) Statistical analysis of the task type, question type, and object relation. (b) Word cloud visualization.}
\label{fig:spatialvqa_statistic}
\end{center}
\end{figure}



\subsection{Comparison to Concurrent Works}
\begin{figure*}[t!]
\includegraphics[width=\linewidth]{figs/src/simpler_visual.pdf}
\vspace{-15pt}
\caption{An example of \ours~how to finish ``move near'' task in SIMPLER~\cite{simplerenv24}.}
\label{fig:simpler_visual}
\end{figure*}

\subsubsection{Comparison with ReKep~\cite{ReKep24}}
Recently, ReKep has succeeded in executing complex robotic tasks, such as long-horizon manipulation, based on the relationships and constraints between spatial key points. 
Its structural design offers many insights that \sofar~can draw upon, yet it also presents several issues: 
(1) Overly customized prompt engineering. ReKep requires manually designed complex system prompts for each task during inference. 
While this approach may be described as ``no training'', it cannot be considered a true zero-shot transfer. In contrast, \sofar~achieves genuine zero-shot transfer by eliminating the need for any human involvement during inference; (2) Using constraints based solely on key points fails to capture the full 6-DoF pose integrity of objects. For example, in the ``pouring water'' task, merely bringing the spout of the kettle close to the cup may lead to incorrect solutions, such as the kettle overturning; (3) ReKep requires all key points to be present in the first frame, and each step of the process—from mask extraction to feature dimensionality reduction, clustering, and filtering—introduces additional hyperparameters.

\subsubsection{Comparison with Orient Anything~\cite{orient_anything24}}
Recently, Orient Anything also highlighted the importance of orientation in spatial perception and adopted a training data construction approach similar to Our PointSO. Our primary distinction lies in semantic orientation, which is language-conditioned orientation. In contrast, Orient Anything is limited to learning basic directions such as ``front'' and ``top''. By aligning with textual information, semantic orientation better enhances spatial perception, understanding, and robotic manipulation.

\subsection{Future Works}
Future work includes further expanding the OrienText300K with larger datasets like Objaverse-XL~\cite{ObjaverseXL23}, enhancing the performance of semantic orientation through self-supervised learning and pretraining methods~\cite{MAE,CLIP,ACT23,ReCon23}, and demonstrating its effectiveness in a broader range of robotic scenarios, such as navigation~\cite{GOAT24}, mobile manipulation~\cite{homerobot23}, lifelong learning~\cite{LIBERO23}, spatio-temporal reasoning~\cite{ReKep24,LeaFLF23,CrossVideoSC24,thinking24}, humanoid~\cite{OmniH2O24,SmoothHumanoidLCP24,Exbody24,humanup25}, and human-robot interaction~\cite{HOI4D22,InteractiveHO23}.

\begin{table*}[t!]
\centering
\caption{\textbf{Statistics of Open6DOR V2 Benchmark.} The entire benchmark comprises three independent tracks, each featuring diverse tasks with careful annotations. The tasks are divided into different levels based on instruction categories, with statistics demonstrated above.}
\label{tab:open6dorv2_statistic}
\resizebox{\textwidth}{!}{
\setlength{\tabcolsep}{3.5pt}
    \begin{tabular}{c|ccccc|cc|c|c|c|c|c}
    \toprule
        Track & \multicolumn{7}{c|}{Position-track} & \multicolumn{3}{c|}{Rotation-track} & 6-DoF-track & Totel\\
        \midrule
        Level  & \multicolumn{5}{c|}{Level 0} & \multicolumn{2}{c|}{Level 1}  & Level 0 & Level 1 & Level 2 & - & -\\
        \midrule
        Task Catog. &  Left & Right  & Top &  Behind &  Front & Between  & Center & Geometric & Directional & Semantic & - & - \\
        \midrule
        Task Stat. & 296 & 266 & 209 & 297 & 278 & 193 & 159 & 318 & 367 & 134 & 1810 & 4535\\
        
        \midrule
        Benchmark Stat. &\multicolumn{7}{c|}{1698} & \multicolumn{3}{c|}{1027} & 1810 & 4535\\
    \bottomrule 
    \end{tabular}
}
\end{table*}

\section{Additional Visualizations}\label{app:visualization}

\subsection{Robotic Manipulation}
As shown in \cref{fig:simpler_visual}, we present a visualization of executing a task named ``move near''.
According to the input image and task instruction - ``\textit{move blue plastic bottle near pepsi can}'', \ours~can predict the center coordinate of the target object (bottle) and relative target (pepsi can), and it would infer the place coordinate and produce a series of grasp pose.

\subsection{6-DoF SpatialBench}
To further evaluate 6-DoF spatial understanding, we construct a 6-DoF SpatialBench.
We present examples of question-answer pairs from the 6-DoF SpatialBench, with quantitative and qualitative questions shown in \cref{fig:spatialbench_show1,fig:spatialbench_show2}, respectively. 
The benchmark we constructed is both challenging and practical, potentially involving calculations based on the laws of motion, such as ``\textit{Assuming a moving speed of 0.5 m/s, how many seconds would it take to walk from here to the white flower?}'' Moreover, it covers a wide range of spatially relevant scenarios across both indoor and outdoor environments.


\subsection{System Prompts}
Prompt engineering significantly enhances ChatGPT's capabilities. The model's understanding and reasoning abilities can be greatly improved by leveraging techniques such as Chain-of-Thought~\cite{CoT22} and In-Context Learning~\cite{GPT3_20}. \cref{fig:filter_prompt,fig:instruction_prompt} illustrate the system prompt we used in constructing OrienText300K.
\cref{fig:open6dor_prompt}, \cref{fig:manip_prompt}, and \cref{fig:vqa_prompt} illustrate the system prompt we used when evaluating \sofar on Open6DOR (simulation), object manipulation (both simulation and real worlds), and VQA, respectively.
Note that different from previous methods~\cite{VoxPoser23,ReKep24}, \sofar does not require complicated in-context examples.


\begin{figure*}[h!]
\centering
\includegraphics[width=0.97\linewidth]{figs/src/show1.pdf}
\captionof{figure}{\textbf{Visualization example of 6-DoF SpatialBench data samples.}
% 6-DoF SpatialBench includes complex spatial reasoning of absolute numbers.
}
\label{fig:spatialbench_show1}
\end{figure*}

\begin{figure*}[h!]
\centering
\includegraphics[width=0.97\linewidth]{figs/src/show2.pdf}
\captionof{figure}{\textbf{Visualization example of 6-DoF SpatialBench data samples.}}
\label{fig:spatialbench_show2}
\end{figure*}

\begin{figure*}[h!]
\centering
\includegraphics[width=0.97\linewidth]{figs/src/filter_prompt.pdf}
\captionof{figure}{\textbf{The system prompt of ChatGPT-4o used for filtering Objaverse data.}}
\label{fig:filter_prompt}
\end{figure*}

\begin{figure*}[h!]
\centering
\includegraphics[width=0.97\linewidth]{figs/src/instruction_prompt.pdf}
\captionof{figure}{\textbf{The system prompt of ChatGPT-4o used for generating Semantic Direction-Index pairs.}}
\label{fig:instruction_prompt}
\end{figure*}

\begin{figure*}[h!]
\centering
\includegraphics[width=0.97\linewidth]{figs/src/open6dor_prompt.pdf}
\captionof{figure}{\textbf{The system prompt of Open6DOR tasks.}}
\label{fig:open6dor_prompt}
\end{figure*}

\begin{figure*}[h!]
\centering
\includegraphics[width=0.97\linewidth]{figs/src/manip_prompt.pdf}
\captionof{figure}{\textbf{The system prompt of general manipulation tasks.}}
\label{fig:manip_prompt}
\end{figure*}

\begin{figure*}[h!]
\centering
\includegraphics[width=0.97\linewidth]{figs/src/vqa_prompt.pdf}
\captionof{figure}{\textbf{The system prompt of visual-question-answering tasks.}}
\label{fig:vqa_prompt}
\end{figure*}



\end{document}

