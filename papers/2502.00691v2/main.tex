% This must be in the first 5 lines to tell arXiv to use pdfLaTeX, which is strongly recommended.
\pdfoutput=1
% In particular, the hyperref package requires pdfLaTeX in order to break URLs across lines.

\documentclass[11pt]{article}

% Change "review" to "final" to generate the final (sometimes called camera-ready) version.
% Change to "preprint" to generate a non-anonymous version with page numbers.
\usepackage[preprint]{acl}
\usepackage{times}
\usepackage{latexsym}

% For proper rendering and hyphenation of words containing Latin characters (including in bib files)
\usepackage[T1]{fontenc}
% For Vietnamese characters
% \usepackage[T5]{fontenc}
% See https://www.latex-project.org/help/documentation/encguide.pdf for other character sets

% This assumes your files are encoded as UTF8
\usepackage[utf8]{inputenc}

% This is not strictly necessary, and may be commented out,
% but it will improve the layout of the manuscript,
% and will typically save some space.
\usepackage{microtype}

% This is also not strictly necessary, and may be commented out.
% However, it will improve the aesthetics of text in
% the typewriter font.
\usepackage{inconsolata}

%Including images in your LaTeX document requires adding
%additional package(s)
\usepackage{graphicx}
% Optional math commands from https://github.com/goodfeli/dlbook_notation.
%%%%% NEW MATH DEFINITIONS %%%%%

% \usepackage{amsmath,amsfonts,bm}
\usepackage{amsmath,amsfonts}

\usepackage{pifont}


\newcommand{\R}{\mathbb{R}}


\def\va{{\mathbf{a}}}
\def\vg{{\mathbf{g}}}

% Sets
\def\sR{\mathbb{R}}
\def\sC{\mathbb{C}}
\def\sZ{\mathbb{Z}}
\def\sN{\mathbb{N}}
\def\sQ{\mathbb{Q}}

\def\sS{\mathcal{S}}



% Vectors
\def\vzero{{\mathbf{0}}}
\def\vone{{\mathbf{1}}}
\def\vmu{{\mathbf{\mu}}}
\def\vtheta{{\mathbf{\theta}}}
\def\va{{\mathbf{a}}}
\def\vb{{\mathbf{b}}}
\def\vc{{\mathbf{c}}}
\def\vd{{\mathbf{d}}}
\def\ve{{\mathbf{e}}}
\def\vf{{\mathbf{f}}}
\def\vg{{\mathbf{g}}}
\def\vh{{\mathbf{h}}}
\def\vi{{\mathbf{i}}}
\def\vj{{\mathbf{j}}}
\def\vk{{\mathbf{k}}}
\def\vl{{\mathbf{l}}}
\def\vm{{\mathbf{m}}}
\def\vn{{\mathbf{n}}}
\def\vo{{\mathbf{o}}}
\def\vp{{\mathbf{p}}}
\def\vq{{\mathbf{q}}}
\def\vr{{\mathbf{r}}}
\def\vs{{\mathbf{s}}}
\def\vt{{\mathbf{t}}}
\def\vu{{\mathbf{u}}}
\def\vv{{\mathbf{v}}}
\def\vw{{\mathbf{w}}}
\def\vx{{\mathbf{x}}}
\def\vy{{\mathbf{y}}}
\def\vz{{\mathbf{z}}}
\def\vzeta{{\mathbf{\zeta}}}

% Matrix
\def\mA{{\mathbf{A}}}
\def\mB{{\mathbf{B}}}
\def\mC{{\mathbf{C}}}
\def\mD{{\mathbf{D}}}
\def\mE{{\mathbf{E}}}
\def\mF{{\mathbf{F}}}
\def\mG{{\mathbf{G}}}
\def\mH{{\mathbf{H}}}
\def\mI{{\mathbf{I}}}
\def\mJ{{\mathbf{J}}}
\def\mK{{\mathbf{K}}}
\def\mL{{\mathbf{L}}}
\def\mM{{\mathbf{M}}}
\def\mN{{\mathbf{N}}}
\def\mO{{\mathbf{O}}}
\def\mP{{\mathbf{P}}}
\def\mQ{{\mathbf{Q}}}
\def\mR{{\mathbf{R}}}
\def\mS{{\mathbf{S}}}
\def\mT{{\mathbf{T}}}
\def\mU{{\mathbf{U}}}
\def\mV{{\mathbf{V}}}
\def\mW{{\mathbf{W}}}
\def\mX{{\mathbf{X}}}
\def\mY{{\mathbf{Y}}}
\def\mZ{{\mathbf{Z}}}
\def\mBeta{{\mathbf{\beta}}}
\def\mPhi{{\mathbf{\Phi}}}
\def\mLambda{{\mathbf{\Lambda}}}
\def\mSigma{{\mathbf{\Sigma}}}


% Expectation
% \def\eE{\mathop{\mathbb{E}}\limits}
\def\eE{\mathbb{E}}

% Probability
\def\pP{\mathbb{P}}

% Tilde
\def\tf{\tilde{f}}
\def\tS{\tilde{S}}
\def\wtF{\widetilde{\mathcal{F}}}
\def\whR{\widehat{R}}
\def\tvx{\tilde{\mathbf{x}}}
\def\ty{\tilde{y}}


\def\defeq{\overset{\textup{def}}{=}}
% \def\defeq{\overset{.}{=}}
\def\defone{\overset{\text{\ding{172}}}{=}}
\def\deftwo{\overset{\text{\ding{173}}}{=}}
\def\leqone{\overset{\text{\ding{172}}}{\leq}}
\def\leqtwo{\overset{\text{\ding{173}}}{\leq}}
\def\leqthree{\overset{\text{\ding{174}}}{\leq}}
\def\leqfour{\overset{\text{\ding{175}}}{\leq}}
\def\eqone{\overset{\text{\ding{172}}}{=}}
\def\eqtwo{\overset{\text{\ding{173}}}{=}}
\def\eqthree{\overset{\text{\ding{174}}}{=}}
\def\eqfour{\overset{\text{\ding{175}}}{=}}
\def\geqfive{\overset{\text{\ding{176}}}{\geq}}

\usepackage{hyperref}
\usepackage{url}
\usepackage{pifont}
\usepackage{colortbl}
\usepackage{amsmath, amssymb}
\usepackage{enumitem}
\usepackage{graphicx}
\usepackage{subcaption}
\usepackage{booktabs}
\usepackage{algorithm}
\usepackage{algpseudocode}
\usepackage{tcolorbox}

\tcbset{
  myaddedbox/.style={
    colback=blue!5,
    colframe=blue!75,
    fonttitle=\bfseries,
    boxrule=0.5mm,
    arc=2mm,
    left=2mm,
    right=2mm,
    top=2mm,
    bottom=2mm
  }
}
\tcbset{
  simplebox/.style={
    colback=white,     % Background color (white for no fill)
    colframe=blue!75,    % Border color
    boxrule=0.4pt,     % Thin border
    sharp corners,     % No rounded corners
    left=1mm,          % Minimal padding
    right=1mm,
    top=1mm,
    bottom=1mm
  }
}
\usepackage{listings}
\newcommand{\dcold}{\mathrm{D_{cold}}}
\newcommand{\dwarm}{\mathrm{D_{warm}}}
\newcommand{\ds}{\mathrm{D}}
\lstdefinelanguage{Markdown}{
  keywords={},
  sensitive=false,
}

\lstset{
    language=Markdown,
    basicstyle=\ttfamily,
    breaklines=true,
    columns=fullflexible
}

\title{Learning Autonomous Code Integration for Math Language Models}

% Authors must not appear in the submitted version. They should be hidden
% as long as the \iclrfinalcopy macro remains commented out below.
% Non-anonymous submissions will be rejected without review.
\author{Haozhe Wang$^{\dagger\ddagger}$, Long Li$^\dagger$, Chao Qu$^\dagger$, Fengming Zhu$^\ddagger$, Weidi Xu$^\dagger$, Wei Chu$^\dagger$, Fangzhen Lin$^\ddagger$\\
\texttt{jasper.whz@outlook.com}\\
INF Technology$^\dagger$, Hong Kong University of Science and Technology$^\ddagger$
}
% \author{Antiquus S.~Hippocampus, Natalia Cerebro \& Amelie P. Amygdale \thanks{ Use footnote for providing further information
% about author (webpage, alternative address)---\emph{not} for acknowledging
% funding agencies.  Funding acknowledgements go at the end of the paper.} \\
% Department of Computer Science\\
% Cranberry-Lemon University\\
% Pittsburgh, PA 15213, USA \\
% \texttt{\{hippo,brain,jen\}@cs.cranberry-lemon.edu} \\
% \And
% Ji Q. Ren \& Yevgeny LeNet \\
% Department of Computational Neuroscience \\
% University of the Witwatersrand \\
% Joburg, South Africa \\
% \texttt{\{robot,net\}@wits.ac.za} \\
% \AND
% Coauthor \\
% Affiliation \\
% Address \\
% \texttt{email}
% }

% The \author macro works with any number of authors. There are two commands
% used to separate the names and addresses of multiple authors: \And and \AND.
%
% Using \And between authors leaves it to \LaTeX{} to determine where to break
% the lines. Using \AND forces a linebreak at that point. So, if \LaTeX{}
% puts 3 of 4 authors names on the first line, and the last on the second
% line, try using \AND instead of \And before the third author name.
\definecolor{LightCyan}{rgb}{0.88,1,1}
\newcommand{\xmark}{\ding{55}} % Define xmark as a cross symbol
\newcommand{\fix}{\marginpar{FIX}}
\newcommand{\new}{\marginpar{NEW}}
\newcommand{\red}[1]{\textcolor{red}{#1}}
\newcommand{\blue}[1]{\textcolor{blue}{#1}}
\newcommand{\up}{$\uparrow$}
\newcommand{\down}{$\downarrow$}


\begin{document}


\maketitle
\begin{abstract}

% Recent works to jointly reconstruct 3D human and object from a single RGB image, are mostly model-based, that fail to capture the fine details of the clothed human body and object surface. In this paper, we introduce ReCHOR, a novel, model-free, first-method to produce realistic clothed human-object reconstructions from a monocular view. This is extremely challenging due to human-object occlusions, diverse interactions and depth ambiguity, as it needs to infer both 3D spatial awareness and high resolution details. Our core idea is based on estimating neural implicit representations for human and object respectively by an attention-based neural implicit model that attends to pixel-aligned features from both the global human-object image for spatial awareness and  the local separate view of human and object images for high quality details. Additionally, the network is conditioned on semantic features from an initial estimated human-object pose prior and a generative diffusion model that inpaints occluded regions, thus enabling the retrieval of details from them.
% We also propose a synthetic dataset with rendered scenes of diverse, inter-occluded 3D human and object scans, to train our network. We evaluate our method on the synthetic and real world BEHAVE dataset. Our experiments show that our method outperforms the SOTA in achieving realistic clothed human-object reconstructions.
Recent approaches to jointly reconstruct 3D humans and objects from a single RGB image represent 3D shapes with template-based or coarse models, which fail to capture details of loose clothing on human bodies. In this paper, we introduce a novel implicit approach for jointly reconstructing realistic 3D clothed humans and objects from a monocular view. For the first time, we model both the human and the object with an implicit representation, allowing to capture more realistic details such as clothing. This task is extremely challenging due to human-object occlusions and the lack of 3D information in 2D images, often leading to poor detail reconstruction and depth ambiguity. To address these problems, we propose a novel attention-based neural implicit model that leverages image pixel alignment from both the input human-object image for a global understanding of the human-object scene and from local separate views of the human and object images to improve realism with, for example, clothing details. Additionally, the network is conditioned on semantic features derived from an estimated human-object pose prior, which provides 3D spatial information about the shared space of humans and objects. To handle human occlusion caused by objects, we use a generative diffusion model that inpaints the occluded regions, recovering otherwise lost details. For training and evaluation, we introduce a synthetic dataset featuring rendered scenes of inter-occluded 3D human scans and diverse objects. Extensive evaluation on both synthetic and real-world datasets demonstrates the superior quality of the proposed human-object reconstructions over competitive methods.
\end{abstract}

\section{Introduction}
\label{sec:intro}
% Image editing methods in diffusion models depend on user-defined control directions - users can unlock their creativity using these methods by specifying the desired manipulation through prompts~\cite{gandikota2023concept}, reference images~\cite{ruiz2022dreambooth, kumari2022customdiffusion, gal2022image, chen2024trainingfreeregionalpromptingdiffusion}, or attribute vectors~\cite{parmar2023zero,hertz2022prompt}. In this work, we ask a fundamentally different question: \emph{Can we automatically discover the underlying visual structure of a concept within diffusion model's knowledge?} %Rather than requiring user-specified controls, we aim to decompose the model's internal knowledge into meaningful directions.

% This question touches on a fundamental limitation in how we interact with diffusion models. Current control methods ~\cite{zhang2023addingconditionalcontroltexttoimage, gandikota2023concept, ye2023ipadaptertextcompatibleimage,ye2023ipadaptertextcompatibleimage, hertz2024stylealignedimagegeneration, li2023photomaker, shi2024instantbooth, chen2024trainingfreeregionalpromptingdiffusion} require users to specify their desired manipulations in advance, limiting interactive creativity. This contrasts with natural human artistic workflows, where creators dynamically explore creative ideas while jointly refining them toward meaningful artistic outcomes~\cite{hoffmann2016modeling}. This synergy between specification and exploration is not new to generative models. Early GAN architectures naturally developed disentangled latent spaces that enabled continuous\cite{harkonen2020ganspace,radford2015unsupervised, wu2021stylespace, shen2020interfacegan}, compositional control over generated images. Users could explore these spaces to discover interesting variations that would be difficult to describe in words~\cite{wu2021stylespace}, then combine them to achieve their creative goals~\cite{grabe2022towards}. 


% While diffusion models have largely superseded GANs in conditional image synthesis~\cite{dhariwal2021diffusion},  their underlying structure remains less understood. Diffusion models achieve remarkable diversity through high-dimensional latents, unlike GANs' compact latent spaces.  With a single prompt, diffusion models can generate radically different variations through different random initializations of input noise. We ask - Is it possible to discover interpretable structure within this vast space of variations?

Text-to-image diffusion models are capable of generating remarkable visual variations from a single prompt through different random initializations. However, this vast creative potential remains largely opaque to users---while we can generate diverse images, we lack understanding of the underlying structure of these variations. This presents a fundamental challenge: how can we discover and expose the latent visual capabilities encoded within these models?

\let\thefootnote\relax \footnote{$^{*}$Correspondence to \texttt{gandikota.ro@northeastern.edu}}

The challenge touches on a key limitation in how we interact with diffusion models today. Current control methods require users to explicitly specify their desired edits in advance through prompts~\cite{gandikota2023concept}, reference images~\cite{zhang2023addingconditionalcontroltexttoimage, chen2024trainingfreeregionalpromptingdiffusion, ruiz2022dreambooth,kumari2022customdiffusion, Ryu_lora, hu2021lora}, or attribute vectors~\cite{ye2023ipadaptertextcompatibleimage, hertz2024stylealignedimagegeneration, li2023photomaker, shi2024instantbooth,parmar2023zero,hertz2022prompt}. That contrasts sharply with natural human creative workflows, where artists dynamically explore creative ideas and jointly refine them toward meaningful artistic outcomes~\cite{hoffmann2016modeling}. The need for pre-specified controls creates a barrier between users and the full creative potential of these models.

Interestingly, earlier generative models like GANs~\cite{gans,karras2019style,brock2018large} naturally developed more interpretable internal structures. Their compact latent spaces often exhibited emergent disentanglement~\cite{harkonen2020ganspace,radford2015unsupervised, wu2021stylespace, shen2020interfacegan}, enabling continuous and compositional control over generated images. Users could explore these spaces to discover interesting variations that would be difficult to describe in words~\cite{wu2021stylespace}, then combine them to achieve their creative goals~\cite{grabe2022towards}.

Diffusion models have largely superseded GANs in conditional image synthesis~\cite{dhariwal2021diffusion}, achieving greater diversity through much higher-dimensional latents. And yet an understanding of the underlying structure of these larger latent spaces has remained elusive. In this work, we ask a fundamental question: \emph{Can we automatically discover the visual structure within a diffusion model's knowledge of a concept?} Rather than requiring user-specified controls, we aim to decompose the model's internal representations into expressive directions that users can explore and combine.

To address these needs, we present \textbf{SliderSpace}, a framework that brings systematic explorability to diffusion models. Given just a text prompt, SliderSpace discovers a canonical set of meaningful, diverse, and controllable directions within the model's knowledge of that concept. Each direction is implemented as a low-rank adapter~\cite{hu2021lora} that can be scaled and composed with others, allowing users to explore and smoothly combine different aspects of variation, as shown in Figure~\ref{fig:intro}.

We ground SliderSpace discovery in three key requirements for meaningful decomposition of a diffusion model's visual manifold: 
\begin{enumerate}
    \item \textbf{Unsupervised Discovery:} The decomposition process should emerge from the intrinsic structure of the model's learned representation, rather than being guided by predefined attributes. This ensures we capture the true topology of the model's knowledge space rather than projecting our assumptions onto it.
    
    \item \textbf{Semantic Orthogonality:} Each discovered control must represent a distinct semantic direction. This is enforced in a semantic feature space, like CLIP, where every slider has an orthogonal effect in embeddings. This prevents discovering multiple controls that create similar semantic effects, making the system more efficient and easier.
    
    \item \textbf{Distribution Consistency:} Directions must induce consistent transformations across both random seeds and prompt variations. 
\end{enumerate}

These requirements naturally lead to our proposed framework, which we formalize in Section~\ref{sec:method}. As we show in our experiments, SliderSpace is architecture-agnostic, working with both conventional U-Net based models like Stable Diffusion~\cite{rombach2022high, rombach2022sd20, podell2023sdxl, turbo, dmd} and recent transformer-based architectures like Flux~\cite{flux}.

We demonstrate the expressiveness of SliderSpace through three applications: First, we show how SliderSpace can decompose high-level concepts into diverse and expressive components, revealing the natural axes of variation in the model's understanding. Second, we explore artistic style variation, where SliderSpace discovers directions that match or exceed the diversity of manually curated artist lists while being judged more useful by human evaluators. Finally, we show how SliderSpace can help reverse the mode collapse commonly observed in distilled diffusion models, restoring diversity while maintaining generation speed.

Beyond providing practical creative control, SliderSpace opens new avenues for understanding and utilizing the latent capabilities of diffusion models. By mapping these models' visual potential into intuitive, composable directions, we take a step toward making their creative possibilities more accessible and interpretable to users.

% Image editing methods in diffusion models unlock the creativity of users. In this work we ask an alternate question: \emph{Can we organize and expose what of the diffusion model is already capable of?}.
% Existing methods for controlling image generation typically require users to manually specify edit directions for desired changes. This process is time-consuming, requires technical expertise, and limits the spontaneity of the creative process. For instance, if a user wants to adjust the smile of a generated person, they must explicitly request this edit, often through imprecise prompt engineering or model fine-tuning. This approach of predefined controls or manual specifications restricts users from fully exploring the latent capabilities of the model. There may be interesting stylistic variations or attributes that the model can generate, but users have no easy way to discover or utilize these.

% Natural visual disentanglement was an emergent property in the latent space of Generative Adversarial Models (GANs) \cite{harkonen2020ganspace,radford2015unsupervised, wu2021stylespace, shen2020interfacegan}. In particular, it has been observed that StyleGAN~\cite{karras2019style} stylespace neurons offer detailed control over many meaningful aspects of images that would be difficult to describe in words~\cite{wu2021stylespace}. However, diffusion models do not share such a compact latent space~\cite{park2023unsupervised}; and efforts to uncover such a space in the semantic embeddings of the text conditioning have met with limited success \nik{Nick - is there a specific citation you were thinking about?}.

% In this work we introduce \textbf{SliderSpace}, which takes a step towards uncovering an analogous low dimensional representation of diffusion models' visual breadth; in essence treating the diffusion model as many generators sharing parameters, where a particular generator is defined by a specific prompt. For a given prompt we sample many random seeds (and optionally prompt expansions using an LLM), generate the corresponding images, and apply an off the shelf feature extractor (in this work CLIP, but our method can be applied to any differentiable feature extractor). We use PCA to analyze these features, and for each of the leading $k$ principal components we train a LoRA \cite{} which causes the diffusion model to produces images which increase the feature magnitude along that component when passed back through the same feature extractor. This leads to a 'Slider' for each principal component, because each LoRA can be scaled and applied to the original diffusion model, continuously varying those visual features in the generated results (as measured, in our case, by CLIP).

% There are many other works that enhance the controllability of diffusion models. One common approach is enabling users to add spatial constraints to a generation either manually, or via a reference image \cite{zhang2023addingconditionalcontroltexttoimage, chen2024trainingfreeregionalpromptingdiffusion}, a second is leveraging more abstract embeddings (e.g. identity, style) extracted from a reference image \cite{ye2023ipadaptertextcompatibleimage, hertz2024stylealignedimagegeneration, li2023photomaker, shi2024instantbooth}, a third is finetuning a foundation model to better generate a concept important to the user \cite{ruiz2022dreambooth, kumari2022customdiffusion, Ryu_lora, hu2021lora}, and a fourth (most relevant to this work) is finding low-rank adaptors of the model based on a prompt or small training set which can be scaled to provide continous control over one aspect of generated image (e.g. night vs day, basic vs luxury, etc.) \cite{gandikota2023concept}. SliderSpace is complementary to all of these methods and offers something distinct. All of the other methods we are aware require the user (and / or model designer) to know in advance what type of control they want. In contrast SliderSpace assists users in discovering and controlling hidden capabilities present in the diffusion model's distribution of possible generations.

%We propose that truly intuitive creative control in a text-to-image model should meet three key criteria: \emph{discoverability}, \emph{intuitiveness}, and \emph{specificity}. The model should reveal controllable attributes that may not be immediately obvious, offer controls that are easy to understand and manipulate, and ensure each control affects a distinct attribute of the generated image.

% We demonstrate the utility and power of SliderSpace using three applications built on top of SDXL-DMD \cite{dmd}, because its fast generation speed lends itself well to the continuous control offered by SliderSpace.

% First, we study concept decomposition (Section \ref{sec:concept_exp}), where we learn sliders for a specific concept (e.g. 'monster', 'waterfall', 'car'). Through quantitative metrics of diversity and text alignment we demonstrate that the learned sliders dramatically boost the diversity of generations when randomly applied without harming text alignment; we also ask humans to qualitatively judge these results in a user study where they find the SliderSpace results to be more 'Diverse', 'Useful', and 'Creative' than our baselines.

% Second, we attempt to compare the automatic discoveries of SliderSpace to a large scale manual study of artistic styles (Section \ref{sec:art_exp}), open-sourced by ParrotZone \cite{parrotzone}. In this study SDXL was prompted with over 4300 artist names,  and based on visual inspection the cases of successful stylistic mimicry recorded. Quantitatively SliderSpace more closely matches the distribution of artistic variation discovered by ParrotZone than other baselines, and in our user studies was judged to be significantly more 'Diverse' and 'Useful' than the baselines. To our surprise humans even judged SliderSpace results to be slightly more 'Diverse' than the results generated by the manually discovered artist names of \cite{parrotzone}.

% Third, we attempt to use SliderSpace to reverse the mode collapse commonly observed in distilled few-step diffusion models relative to the original teacher model (Section \ref{sec:diverse_exp}). We quantitatively demonstrate that applying SliderSpace to SDXL-DMD leads to more closely matching the distribution of images by the original teacher, SDXL.

%Through extensive experiments on various state-of-the-art text-to-image models, we demonstrate that SliderSpace significantly enhances user control and creative expression in AI-assisted image generation tasks. Our method enables a range of applications, including concept decomposition and control, diversity improvement in generated images, customization dissection and edits, and the exploration of artistic styles inherent in the model.

% SliderSpace goes beyond providing a practical tool for enhanced creative control. By mapping the visual potential of diffusion models it can open new avenues for generative creativity and deepens our understanding of each model's hidden potential.

% !TEX root = ../main.tex

\section{Scene Graph Construction for Videos}
\label{sec:scene}
\begin{figure*}[t]
    \centering
    \includegraphics[width=\textwidth]{figures/overview.pdf}
    \vspace{-4mm}
    \caption{An overview of our zero-shot video caption generation pipeline. The pipeline consists of (a) frame-level caption generation using image VLMs, (b) textual scene graph parsing for each frame caption, (c) merging of scene graphs into a unified graph, and (d) video-level caption generation through our graph-to-text model. Our proposed framework leverages frame-level scene graphs to produce detailed and coherent video captions.
    }
    \label{fig:framework}
\end{figure*}

Our objective is to effectively extend the capabilities of image-based vision-language models (VLMs) to the video domain without relying on video-text training. 
To this end, we introduce a novel video captioning framework that combines image VLMs with scene graph structures, as shown in Figure~\ref{fig:framework}.
The proposed method consists of four key steps: 1) generating captions for each frame using an image VLM, 2) converting these captions into scene graphs, 3) consolidating the scene graphs from all frames into a unified graph, and 4) generating comprehensive descriptions from this unified graph. 
This algorithm enables the generation of coherent and detailed video captions, bridging the gap between image and video understanding.

\subsection{Generating image-level captions}
\label{sub:generating}
We obtain image-level captions from a set of sparsely sampled frames using the open-source image VLM, LLAVA-NEXT-7B~\cite{liu2024llavanext}.
This model is selected for its strong performance across multiple benchmarks.
Our approach, however, is flexible and can incorporate any image-based VLM, including proprietary, closed-source models, as long as APIs are accessible.
The model is prompted to generate sentences optimized for scene graph construction, which are subsequently parsed into scene graphs.

\subsection{Parsing captions into scene graphs}
A scene graph $G = (\mathcal{O}, \mathcal{E})$ is defined by a set of objects, $\mathcal{O} = \{o_1, o_2, \ldots \}$, and a set of edges, $\mathcal{E}$.
Each object $o_i = (c_i, \mathcal{A}_i)$ consists of an object class $c_i \in \mathcal{C}$ and a set of attributes $\mathcal{A}_i \subseteq A$, where $\mathcal{C}$ is a set of object classes and $\mathcal{A}$ is a set of all possible attributes.
A directed edge, $e_{i,j} \equiv (o_i, o_j) \in \mathcal{E}$, has a label $r \in \mathcal{R}$, specifying the relationship from one object to the other.
All the values of object classes, attributes, and relationship labels, are text strings.

We convert the generated caption from each frame into a scene graph, providing more structured understanding of individual frames. 
By expressing the visual content in each frame using a graph based on detected objects and their relationships, we can apply a graph merging technique to produce a holistic representation of the entire input video.
We parse a caption into a scene graph using a textual scene graph parser, specifically the FACTUAL-MR parser~\cite{li-etal-2023-factual} in our implementation.

\subsection{Scene graph consolidation}
\label{sub:scene}
% !TEX root = ../main.tex

\begin{algorithm}[t]
\caption{Scene graph merging}
\label{alg:hierarchical_graph_merge}
\begin{algorithmic}[1]

  \STATE \textbf{Input:} 
  \STATE \quad $\mathcal{Q} = [ G_1, G_2, \dots, G_n ]$: a priority queue with frame-level scene graphs
  \STATE \quad $\phi(\cdot)$: a graph encoder
  \STATE \quad $\psi_i(\cdot)$: a function returning the $i^\text{th}$ object in a graph
  \STATE \quad $\pi$: a permutation function
  \STATE \quad $\tau$: a threshold

  \STATE \textbf{Output:} $G_{\text{video}}$: a video-level scene graph

  \WHILE{$|\mathcal{Q}| > 1$}
    \STATE $G^s = (\mathcal{O}^s, \mathcal{E}^s) \gets \text{dequeue}(\mathcal{Q})$
    \STATE $G^t = (\mathcal{O}^t, \mathcal{E}^t) \gets \text{dequeue}(\mathcal{Q})$
    \STATE $G^m = (\mathcal{O}^m, \mathcal{E}^m) \gets (\mathcal{O}^s \cup \mathcal{O}^t, \mathcal{E}^s \cup \mathcal{E}^t)$

    \STATE $\pi^* \gets \displaystyle \arg\max_{\pi \in \Pi} \sum_{i} 
      \frac{\psi_i(\phi(G^s))}{\lVert \psi_i(\phi(G^s)) \rVert} \; \cdot \;
      \frac{\psi_i(\phi(G_{\pi}^t))}{\lVert \psi_i(\phi(G_{\pi}^t)) \rVert}$

    \FOR{$(p, q) \in \mathcal{M}$ such that $s_{p, q} > \tau$}
      \STATE $\hat{c} \gets \text{update\_class}(c^s_p, c^t_q)$
      \STATE $\hat{o} \gets (\hat{c}, \mathcal{A}^s_p \cup \mathcal{A}^t_q)$
      \STATE $\mathcal{O}^m \gets \{\hat{o}\} \cup \bigl(\mathcal{O}^m \setminus \{o^s_p, o^t_q\}\bigr)$
      \STATE \textbf{for each} $(o_x, o_y) \in \mathcal{E}^m$:
      \STATE \quad $(o_x, o_y) \mapsto 
        \begin{cases}
           (\hat{o}, o_y), & \text{if } o_x \in \{ o_p^s, o_q^t \}; \\
           (o_x, \hat{o}), & \text{if } o_y \in \{ o_p^s, o_q^t \}; \\
           (o_x, o_y), & \text{otherwise.}
        \end{cases}$
    \ENDFOR

    \STATE $\mathcal{Q} \gets \text{enqueue}(\mathcal{Q}, G^m)$
  \ENDWHILE

  \STATE $G_{\text{video}} \gets \text{dequeue}(\mathcal{Q})$
  \STATE \textbf{return} $G_{\text{video}}$

\end{algorithmic}
\end{algorithm} 
The scene graph consolidation step combines all frame-level scene graphs into a single graph that captures the overall visual content of the video. 
We outline our graph merging procedure, followed by a subgraph extraction technique for more focused video caption generation.

\subsubsection{Video-level graph integration}

Given two scene graphs, $G^s = (\mathcal{O}^s, \mathcal{E}^s)$ and $G^t = (\mathcal{O}^t, \mathcal{E}^t)$, constructed from two different frames, we perform the Hungarian matching between their object sets, $\mathcal{O}^s$ and $\mathcal{O}^t$.
The Hungarian algorithm aims to find the maximum matching between the objects in $\mathcal{O}^s$ and $\mathcal{O}^t$, which is given by
%
\begin{equation}
	\pi^* = \underset{\pi \in \Pi}{\arg\max} \sum_{i} \frac{ \psi_i(\phi(G^s))}{\| \psi_i(\phi(G^s)) \|} \cdot \frac{\psi_i(\phi(G_\pi^t)) }{\| \psi_i(\phi(G_\pi^t)) \|},
\end{equation}
%
where $\phi(\cdot)$ denotes the graph encoder, $\psi_i(\cdot)$ is the function to extract the $i^\text{th}$ object from an embedded graph, and $\pi \in \Pi$ indicates a permutation of objects in a graph.
Note that we introduce dummy objects  to deal with different numbers of objects for matching.

After identifying a set of matching object pairs, $\mathcal{M}$, \eg, $(p, q)$, where $o_p^s \in \mathcal{O}^s$ and $o_q^t \in \mathcal{O}^t$, using their cosine similarity with a predefined threshold, $\tau$, we merge the matched objects into a new one $\hat{o} \in \hat{\mathcal{O}}$, which is given by
%
\begin{equation}
    \hat{o} = (\hat{c} , \mathcal{A}^s_p \cup \mathcal{A}^t_q) \in \hat{\mathcal{O}},
\end{equation}
%
where $\hat{c}$ represents a class of the merged objects and $\hat{\mathcal{O}}$ denotes a set of new objects from all legitimate matching pairs.

Using this, we construct a new merged scene graph, $G^m$, which replaces each pair of merged objects with a new object $\hat{o}$, as follows:
%
\begin{equation}
	G^m = (\mathcal{O}^m, \mathcal{E}^m),
\end{equation}
%
where $\mathcal{O}^{m} =\mathcal{O}^s \cup \mathcal{O}^t \cup \hat{\mathcal{O}} ~ \setminus \bigcup_{(p, q) \in \mathcal{M}} \{o^s_p, o^t_q\}$, and the edge set $\mathcal{E}^m$ is also updated to reflect the changes in the object configuration.
Formally, each matching pair $(p, q) \in \mathcal{M}$ incurs the merge of the two objects and the construction of a new object $\hat{o}$, which results in the update of the edge set as $\mathcal{E}^m \equiv \mathcal{E}^s \cup \mathcal{E}^t$, which is formally given by
%
\begin{equation}
	(o_x, o_y) \in \mathcal{E}^m \rightarrow 
	\begin{cases}
		(\hat{o}, o_y) & \text{if } o_x \in \{o_p^s, o_q^t \}, \\
		(o_x, \hat{o}) & \text{if } o_y \in \{o_p^s, o_q^t \}, \\
		(o_x, o_y) & \text{otherwise.}
	\end{cases}
\end{equation}

We perform graph merging using a priority queue, where pairs of graphs are prioritized for merging based on their embedding similarity. 
In each iteration, the two most similar graphs are dequeued, merged, and the resulting graph is enqueued back into the priority queue.
This process is repeated until only one scene graph remains.
The final scene graph provides a comprehensive representation of the video, preserving frame-level details often overlooked by standard captioning models.
Algorithm~\ref{alg:hierarchical_graph_merge} describes the detailed procedure of our graph merging strategy.
  
\subsubsection{Prioritized subgraph extraction}
To generate concise and focused video captions, we apply subgraph extraction to retain only the most contextually relevant information. 
During the graph merging process, we track each node's merge count as a measure of its significance within the consolidated graph. 
We then identify the top $k$ nodes with the highest merge counts and extract their corresponding subgraphs. 
This approach prioritizes objects that consistently appear across multiple frames, as they often represent key entities in the scene. 
By emphasizing these essential elements and filtering out less relevant details, our method constructs a compact scene graph to generate a more focused video caption.

\section{Video Captioning}
\label{sec:videocaption}
To generate video-level descriptions that accurately reflect visual content, we developed a model that takes scene graphs as input and produce natural language descriptions.
This model is designed to effectively capture key components and relationships within the scene graph in generated text.

\vspace{-2mm}
\paragraph{Architecture}
We employ a modified encoder-decoder transformer architecture.
To prepare the input sequence for the graph encoder, each node, edge, and attribute in the graph, represented as a word or phrase, is tokenized into NLP tokens. 
These tokens are mapped to their embeddings via an embedding lookup.
For nodes consisting of multiple NLP tokens, their embeddings are averaged to form a single vector representation.
Additionally, a [CLS] token is appended as a global node to prevent isolation among disconnected components and ensure coherence. 
The adjacency matrix serves as an attention mask, incorporating graph topology into the attention mechanism. 
The graph encoder's output is then used as key and value inputs for the cross-attention layers of the text decoder, which generates the final outputs.

\vspace{-2mm}
\paragraph{Dataset}
For training, we collected approximately 2.5M text corpora that cover diverse visual scene contexts from various sources, including image caption datasets such as  MS-COCO~\cite{chen2015microsoft}, Flickr30k~\cite{young2014image}, TextCaps~\cite{sidorov2020textcaps}, Visual Genome~\cite{krishna2017visual}, and Visual Genome paragraph captioning~\cite{krause2016paragraphs}.
To further enhance the dataset, we incorporated model-generated captions for Kinetics-400~\cite{kay2017kinetics} dataset, with four uniformly sampled frames per video.
Note that neither the datasets nor the image VLMs used for generating frame captions are related to the target video captioning benchmarks.


\vspace{-2mm}
\paragraph{Training}
The model is trained using a next-token prediction objective, aiming to reconstruct the source text conditioned on the scene graph:
%
\begin{equation}
\mathcal{L}(\theta) = \sum_{i=1}^{N} \log P_{\theta}(t_i \mid t_{1:i-1}, G),
\end{equation}  
%
where $t_i$ represents the $i^\text{th}$ token in the source text, and $N$ denotes the total number of tokens.


\vspace{-2mm}
\paragraph{Video caption generation}
After constructing the video-level scene graph as described in Section~\ref{sec:scene}, we generate a video caption using the trained graph-to-text decoder, which conveys the overall narrative of the video.


\section{Experiment}\label{sec:experiment}
We carry out extensive experiments on four real-world datasets to answer the following research questions: 
\begin{itemize}[leftmargin=*]
    % \item \textbf{RQ1:} How does our proposed DEALRec perform compared to the coreset selection baselines for LLM-based recommendation and the models trained with full data? 
    \item \textbf{RQ1:} How does our proposed SETRec perform compared to different identifier baselines on different architectures of LLMs? 
    % \item \textbf{RQ2:} How do the different components of DEALRec (\ie influence score, gap regularization, and stratified sampling) affect the performance, and is DEALRec generalizable to different surrogate models? 
    \item \textbf{RQ2:} How do the different components of SETRec (\ie CF embeddings, semantic embeddings, query vectors, and sparse attention) affect the performance?
    \item \textbf{RQ3:} How does SETRec perform when scaling up the model size and how does SETRec improve the overall performance? 
    \item \textbf{RQ4:} How does SETRec perform with different number of semantic embeddings, tokenizer training strength, and semantic strength for inference? 
\end{itemize}
\subsection{Experimental Settings}
\subsubsection{\textbf{Datasets}}
We conduct experiments on four real-world datasets across various domains. 
From Amazon review datasets\footnote{\url{https://jmcauley.ucsd.edu/data/amazon/}.}, we adopt three widely used benchmarks 
1)\textbf{Toys}, 2) \textbf{Beauty}, and 3) \textbf{Sports}. 
The three Amazon datasets contain rich user interactions over a specific category of e-commerce products, where each item is associated with rich textual meta information such as title, description, category, and brand. 
In addition, we use a video games dataset 4) \textbf{Steam}\footnote{\url{https://github.com/kang205/SASRec}.} proposed in~\cite{kang2018self}, which contains substantial user interactions on video games with abundant textual semantic information. 
For all datasets, we follow previous work~\cite{wang2023causal} to sort user interactions chronologically according to the timestamps and divide them into training, validation, and testing sets with a ratio of 8:1:1. 
In addition, we divide the items into warm and cold items\footnote{We denote warm- and cold-start items as warm and cold items for brevity.}, where the items that appear in the training set are warm items, otherwise cold items. 


\noindent$\bullet\quad$\textbf{Evaluation.} 
We adopt the widely used metrics Recall@$K$ and NDCG@$K$, where $K=5$ and $10$ to evaluate all methods. 
Additionally, 
we introduce three different settings that evaluate over 1) all items, 2) warm items only, and 3) cold items only, respectively.  
% todo: 这里数据集可能要解释一下xxx为了保证cold数量能多一点,切割的比例是xxx


% Please add the following required packages to your document preamble:
% \usepackage{multirow}
% \usepackage[normalem]{ulem}
% \useunder{\uline}{\ul}{}
\begin{table*}[t]
\setlength{\abovecaptionskip}{0.05cm}
\setlength{\belowcaptionskip}{0.2cm}
\caption{Overall performance of baselines and SETRec instantiated on T5. The best results are in bold and the second-best results are underlined. $*$ implies the improvements over the second-best results are statistically significant ($p$-value < 0.01) under one-sample t-tests. ``Inf. Time'' denotes the inference time over all test users tested on a single NVIDIA RTX A5000 GPU.}
\setlength{\tabcolsep}{2mm}{
\resizebox{\textwidth}{!}{
\begin{tabular}{l|l|cccc|cccc|cccc|c}
\toprule
 &  & \multicolumn{4}{c|}{\textbf{All}} & \multicolumn{4}{c|}{\textbf{Warm}} & \multicolumn{4}{c|}{\textbf{Cold}} & \multicolumn{1}{l}{\textbf{Inf. Time (s)}} \\ \hline
\textbf{Dataset} & \textbf{Method} & \textbf{R@5} & \textbf{R@10} & \textbf{N@5} & \textbf{N@10} & \textbf{R@5} & \textbf{R@10} & \textbf{N@5} & \textbf{N@10} & \textbf{R@5} & \textbf{R@10} & \textbf{N@5} & \textbf{N@10} & \textbf{All Users} \\ \midrule
\multirow{9}{*}{\textbf{Toys}} & \textbf{DreamRec} & 0.0020 & 0.0027 & 0.0015 & 0.0018 & 0.0027 & 0.0039 & 0.0020 & 0.0024 & 0.0066 & 0.0168 & 0.0045 & 0.0082 & 912 \\
 & \textbf{E4SRec} & 0.0061 & 0.0098 & 0.0051 & 0.0064 & 0.0081 & 0.0128 & 0.0065 & 0.0082 & 0.0065 & 0.0122 & 0.0056 & 0.0078 & \textbf{55} \\ \cmidrule{2-15}
 & \textbf{BIGRec} & 0.0008 & 0.0013 & 0.0007 & 0.0009 & 0.0014 & 0.0019 & 0.0011 & 0.0013 & 0.0278 & 0.0360 & 0.0196 & 0.0223 & 2,079 \\
 & \textbf{IDGenRec} & 0.0063 & 0.0110 & 0.0052 & 0.0069 & 0.0109 & {\ul 0.0161} & 0.0081 & {0.0102} & {\ul 0.0318} & {\ul 0.0589} & {\ul 0.0236} & {\ul 0.0335} & 658 \\
 & \textbf{CID} & 0.0044 & 0.0082 & 0.0040 & 0.0053 & 0.0065 & 0.0128 & 0.0049 & 0.0071 & 0.0059 & 0.0111 & 0.0047 & 0.0066 & 810 \\
 & \textbf{SemID} & 0.0071 & 0.0108 & 0.0061 & 0.0074 & 0.0086 & 0.0153 & 0.0075 & 0.0100 & 0.0307 & 0.0507 & 0.0220 & 0.0292 & 1,215 \\
 & \textbf{TIGER} & 0.0064 & 0.0106 & 0.0060 & 0.0076 & 0.0091 & 0.0147 & 0.0080 & {\ul 0.0102} & 0.0315 & 0.0555 & 0.0228 & 0.0314 & 448 \\
 & \textbf{LETTER} & {\ul 0.0081} & {\ul 0.0117} & {\ul 0.0064} & {\ul 0.0077} & {\ul 0.0109} & 0.0155 & {\ul 0.0083} & 0.0101 & 0.0183 & 0.0395 & 0.0115 & 0.0190 & 448 \\  \cmidrule{2-15}
 & \cellcolor{gray!16}\textbf{SETRec} & \cellcolor{gray!16}\textbf{0.0110*} & \cellcolor{gray!16}\textbf{0.0189*} & \cellcolor{gray!16}\textbf{0.0089*} & \cellcolor{gray!16}\textbf{0.0118*} & \cellcolor{gray!16}\textbf{0.0139*} & \cellcolor{gray!16}\textbf{0.0236*} & \cellcolor{gray!16}\textbf{0.0112*} & \cellcolor{gray!16}\textbf{0.0147*} & \cellcolor{gray!16}\textbf{0.0443*} & \cellcolor{gray!16}\textbf{0.0812*} & \cellcolor{gray!16}\textbf{0.0310*} & \cellcolor{gray!16}\textbf{0.0445*} & \cellcolor{gray!16}{\ul 60} \\ \midrule\midrule
\multirow{9}{*}{\textbf{Beauty}} & \textbf{DreamRec} & 0.0012 & 0.0025 & 0.0013 & 0.0017 & 0.0016 & 0.0028 & 0.0016 & 0.0019 & 0.0078 & 0.0161 & 0.0065 & 0.0094 & 1,102 \\
 & \textbf{E4SRec} & 0.0061 & 0.0092 & 0.0052 & 0.0063 & 0.0080 & 0.0121 & 0.0067 & 0.0082 & 0.0072 & 0.0118 & 0.0065 & 0.0077 & \textbf{120} \\ \cmidrule{2-15}
 & \textbf{BIGRec} & 0.0054 & 0.0064 & 0.0051 & 0.0054 & 0.0008 & 0.0009 & 0.0006 & 0.0008 & 0.0106 & 0.0251 & 0.0095 & 0.0151 & 4,544 \\
 & \textbf{IDGenRec} & {\ul 0.0080} & 0.0115 & {\ul 0.0066} & {0.0078} & {\ul 0.0106} & 0.0165 & 0.0078 & 0.0099 & 0.0187 & 0.0350 & 0.0186 & 0.0224 & 840 \\
 & \textbf{CID} & 0.0071 & 0.0125 & 0.0060 & {\ul 0.0080} & 0.0098 & {0.0166} & 0.0077 & 0.0101 & 0.0087 & 0.0183 & 0.0071 & 0.0104 & 815 \\
 & \textbf{SemID} & 0.0071 & {\ul 0.0131} & 0.0056 & {0.0078} & 0.0098 & {\ul 0.0174} & 0.0074 & {\ul 0.0103} & {\ul 0.0260} & {\ul 0.0465} & 0.0178 & 0.0255 & 1,310 \\
 & \textbf{TIGER} & 0.0063 & 0.0098 & 0.0050 & 0.0062 & 0.0086 & 0.0131 & 0.0065 & 0.0082 & 0.0190 & 0.0325 & 0.0130 & 0.0178 & 430 \\ 
 & \textbf{LETTER} & 0.0071 & 0.0103 & 0.0061 & 0.0070 & 0.0094 & 0.0135 & {\ul 0.0079} & 0.0091 & 0.0251 & 0.0410 & {\ul 0.0241} & {\ul 0.0285} & 430 \\ \cmidrule{2-15}
 & \cellcolor{gray!16}\textbf{SETRec} & \cellcolor{gray!16}\textbf{0.0106*} & \cellcolor{gray!16}\textbf{0.0161*} & \cellcolor{gray!16}\textbf{0.0083*} & \cellcolor{gray!16}\textbf{0.0103*} & \cellcolor{gray!16}\textbf{0.0139*} & \cellcolor{gray!16}\textbf{0.0212*} & \cellcolor{gray!16}\textbf{0.0108*} & \cellcolor{gray!16}\textbf{0.0134*} & \cellcolor{gray!16}\textbf{0.0384*} & \cellcolor{gray!16}\textbf{0.0761*} & \cellcolor{gray!16}\textbf{0.0280*} & \cellcolor{gray!16}\textbf{0.0413*} & \cellcolor{gray!16}{\ul 126} \\ \midrule\midrule
\multirow{9}{*}{\textbf{Sports}} & \textbf{DreamRec} & 0.0027 & 0.0044 & 0.0025 & 0.0031 & 0.0032 & 0.0052 & 0.0028 & 0.0035 & 0.0045 & 0.0108 & 0.0026 & 0.0049 & 2,100 \\ 
 & \textbf{E4SRec} & 0.0079 & 0.0131 & 0.0075 & 0.0094 & 0.0092 & 0.0154 & 0.0085 & 0.0107 & 0.0031 & 0.0093 & 0.0019 & 0.0039 & \textbf{117} \\ \cmidrule{2-15}
 & \textbf{BIGRec} & 0.0033 & 0.0042 & 0.0030 & 0.0033 & 0.0001 & 0.0002 & 0.0001 & 0.0001 & 0.0059 & 0.0104 & 0.0043 & 0.0061 & 7,822 \\
 & \textbf{IDGenRec} & 0.0087 & 0.0127 & 0.0079 & 0.0092 & 0.0101 & 0.0149 & 0.0091 & 0.0107 & 0.0181 & 0.0302 & 0.0134 & 0.0179 & 1,724 \\
 & \textbf{CID} & 0.0077 & 0.0131 & 0.0073 & 0.0092 & 0.0074 & 0.0119 & 0.0045 & 0.0061 & 0.0082 & 0.0149 & 0.0075 & 0.0099 & 2,135 \\
 & \textbf{SemID} & {\ul 0.0094} & {\ul 0.0167} & {\ul 0.0088} & {\ul 0.0114} & {\ul 0.0119} & {\ul 0.0201} & {\ul 0.0104} & {\ul 0.0135} & {\ul 0.0254} & {\ul 0.0495} & {\ul 0.0175} & {\ul 0.0256} & 2,367 \\
 & \textbf{TIGER} & 0.0085 & 0.0129 & 0.0080 & 0.0095 & 0.0100 & 0.0151 & 0.0091 & 0.0109 & 0.0190 & 0.0310 & 0.0120 & 0.0159 & 481 \\
 & \textbf{LETTER} & 0.0077 & 0.0131 & 0.0073 & 0.0092 & 0.0074 & 0.0119 & 0.0045 & 0.0061 & 0.0082 & 0.0149 & 0.0075 & 0.0099 & 481 \\ \cmidrule{2-15}
 & \cellcolor{gray!16}\textbf{SETRec} & \cellcolor{gray!16}\textbf{0.0114*} & \cellcolor{gray!16}\textbf{0.0185*} & \cellcolor{gray!16}\textbf{0.0101*} & \cellcolor{gray!16}\textbf{0.0126*} & \cellcolor{gray!16}\textbf{0.0134*} & \cellcolor{gray!16}\textbf{0.0216*} & \cellcolor{gray!16}\textbf{0.0115*} & \cellcolor{gray!16}\textbf{0.0144*} & \cellcolor{gray!16}\textbf{0.0341*} & \cellcolor{gray!16}\textbf{0.0595*} & \cellcolor{gray!16}\textbf{0.0233*} & \cellcolor{gray!16}\textbf{0.0323*} & \cellcolor{gray!16}{\ul 136} \\ \midrule\midrule
\multirow{9}{*}{\textbf{Steam}} & \textbf{DreamRec} & 0.0029 & 0.0057 & 0.0037 & 0.0046 & 0.0042 & 0.0080 & 0.0045 & 0.0059 & 0.0017 & 0.0029 & 0.0013 & 0.0018 & 4,620 \\
 & \textbf{E4SRec} & 0.0194 & 0.0351 & 0.0220 & 0.0270 & 0.0312 & 0.0558 & 0.0283 & 0.0370 & 0.0006 & 0.0010 & 0.0006 & 0.0007 & \textbf{328} \\ \cmidrule{2-15}
 & \textbf{BIGRec} & 0.0030 & 0.0049 & 0.0046 & 0.0049 & 0.0048 & 0.0053 & 0.0061 & 0.0053 & 0.0099 & 0.0107 & {\ul 0.0129} & 0.0127 & 5,167 \\
 & \textbf{IDGenRec} & 0.0199 & 0.0307 & 0.0241 & 0.0265 & 0.0309 & 0.0479 & 0.0311 & 0.0363 & 0.0047 & 0.0151 & 0.0039 & 0.0078 & 2,846 \\
 & \textbf{CID} & 0.0200 & {\ul 0.0360} & {\ul 0.0249} & {\ul 0.0295} & 0.0314 & {\ul 0.0566} & {\ul 0.0315} & {\ul 0.0400} & 0.0008 & 0.0021 & 0.0006 & 0.0011 & 3,194 \\
 & \textbf{SemID} & 0.0155 & 0.0278 & 0.0192 & 0.0229 & 0.0248 & 0.0443 & 0.0246 & 0.0313 & 0.0017 & 0.0027 & 0.0015 & 0.0018 & 3,605 \\
 & \textbf{TIGER} & {\ul 0.0202} & 0.0348 & 0.0244 & 0.0287 & {\ul 0.0320} & 0.0552 & 0.0314 & 0.0393 & 0.0060 & {0.0152} & 0.0044 & 0.0078 & 1,747 \\
 & \textbf{LETTER} & 0.0164 & 0.0312 & 0.0195 & 0.0244 & 0.0268 & 0.0500 & 0.0253 & 0.0336 & {\ul 0.0115} & {\ul 0.0317} & {0.0077} & {\ul 0.0157} & 1,747 \\ \cmidrule{2-15}
 & \cellcolor{gray!16}\textbf{SETRec} & \cellcolor{gray!16}\textbf{0.0216*} & \cellcolor{gray!16}\textbf{0.0383*} & \cellcolor{gray!16}\textbf{0.0254*} & \cellcolor{gray!16}\textbf{0.0308*} & \cellcolor{gray!16}\textbf{0.0339*} & \cellcolor{gray!16}\textbf{0.0591*} & \cellcolor{gray!16}\textbf{0.0326*} & \cellcolor{gray!16}\textbf{0.0414*} & \cellcolor{gray!16}\textbf{0.0313*} & \cellcolor{gray!16}\textbf{0.0572*} & \cellcolor{gray!16}\textbf{0.0248*} & \cellcolor{gray!16}\textbf{0.0342*} & \cellcolor{gray!16}{\ul 347} \\ \hline
\end{tabular}
}}
\label{tab:overall_performance}
\end{table*}


\subsubsection{\textbf{Baselines}}
We compare SETRec with competitive baselines, including single-token identifiers (DreamRec, E4SRec) and token-sequence identifiers (BIGRec, IDGenRec, CID, SemID, TIGER, LETTER). 
1) \textbf{DreamRec}~\cite{yang2024generate} is a closely related method that leverages ID embedding to represent each item and adopts a diffusion model to refine the generated ID embedding from LLMs.  
2) \textbf{E4SRec}~\cite{li2023e4srec} utilizes a pre-trained CF model to obtain ID embedding, and uses a linear projection layer to obtain the item scores efficiently. 
3) \textbf{BIGRec}~\cite{bao2023bi} adopts item titles as identifiers, where the tokens are from human vocabulary. 
4) \textbf{IDGenRec}~\cite{tan2024idgenrec} is a learnable ID generator, which aims to generate concise but informative tags from human vocabulary to represent each item. 
5) \textbf{CID}~\cite{hua2023index} leverages hierarchical clustering to obtain token sequence, which utilizes item co-occurrence matrix to obtain identifiers to ensure items with similar interactions share similar tokens. 
6) \textbf{SemID}~\cite{hua2023index} also represents items with external token sequence, which is obtained based on the hierarchical item category. 
7) \textbf{TIGER}~\cite{rajput2023recommender} leverages RQ-VAE with codebooks to quantize item semantic information into token sequence with external tokens. The identifier sequentially contains coarse-grained to fine-grained information. 
8) \textbf{LETTER}~\cite{wang2024learnable} is one of the SOTA item tokenization methods, which incorporates both semantic and CF information into the training of RQ-VAE, achieving identifiers with multi-dimensional information and improved diversity. 

\subsubsection{\textbf{Implementation Details}} 
% 我们把所有的identifier方法都instantiate到了两个不同的LLMs上,T5-small 和 Qwen上,其中我们用1.5B来测试overall performance,然后还扩展到3B和7B上去验证scalability。
% 针对tokenizer的训练,对于使用到AE的方法(TIGER, LETTER,还有我们的方法),我们统一了隐藏层在"512,256,128". 
% 对于LLM的训练,我们为所有方法设置一样的prompt as "xxx"
% 对于T5-small模型,我们是全量微调。对于Qwen模型,我们采用parameter-efficeint tuning technique LoRA~\cite{}. 并且所有实验在4块A5000上跑。
% 针对我们的方法,N的数量在{1,2,3,4,5,6}里面选,alpha在0.1,0.3,0.5,0.7,0.9里选。而inference阶段的beta则是从0-1选。
We instantiate all methods on two LLMs with different architectures, \ie T5-small~\cite{raffel2020exploring} (encoder-decoder) and Qwen2.5~\cite{yang2024qwen2} (decoder-only). 
Specifically, we adopt Qwen\footnote{We denote T5-small and Qwen2.5 as T5 and Qwen for brevity.} with different sizes, including 1.5B, 3B, and 7B, for a comprehensive evaluation. 
To ensure a fair comparison, we set the hidden layer dimensions at 512, 256, and 128 with ReLU activation for methods that adopt AE in tokenizer training, including TIGER, LETTER, and our proposed SETRec. 
For LLM training, 
we adopt the same prompt for all methods as ``What would the user be likely to purchase next after buying items {history}?;'' for a fair comparison. 
We fully fine-tune the T5 model and perform parameter-efficient fine-tuning technique LoRA~\cite{hu2021lora} for Qwen. 
All experiments are conducted on four NVIDIA RTX A5000 GPUs. 
% For SETRec, 
% we use SASRec~\cite{kang2018self} as pre-trained CF model, and utilize 
% SentenceT5 and Qwen as semantic extractors for T5 and Qwen backend LLMs, respectively. 
For SETRec, 
we select $N$, $\alpha$, and $\beta$ from $\{1,2,3,4,5,6\}$, $\{0.1,0.3,0.5,0.7,0.9\}$, and $\{0, 0.1, 0.2, 0.3, 0.4, 0.5, 0.6, 0.7, 0.8, 0.9,1.0\}$, respectively. 


\begin{table*}[t]
\setlength{\abovecaptionskip}{0.05cm}
\setlength{\belowcaptionskip}{0.2cm}
\caption{Overall performance on Qwen-1.5B over Toys and Beauty. The best results are in bold and the second-best results are underlined. ``Inf. Time'' denotes the inference time over all test users tested on a single NVIDIA RTX A5000 GPU.}
\setlength{\tabcolsep}{2mm}{
\resizebox{\textwidth}{!}{
\begin{tabular}{l|l|cccc|cccc|cccc|c}
\toprule
 &  & \multicolumn{4}{c}{\textbf{All}} & \multicolumn{4}{c}{\textbf{Warm}} & \multicolumn{4}{c}{\textbf{Cold}} & \textbf{Inf. Time(s)} \\ \hline
\textbf{Dataset} & \textbf{Method} & \textbf{R@5} & \textbf{R@10} & \textbf{N@5} & \textbf{N@10} & \textbf{R@5} & \textbf{R@10} & \textbf{N@5} & \textbf{N@10} & \textbf{R@5} & \textbf{R@10} & \textbf{N@5} & \textbf{N@10} & \textbf{All Users} \\ \midrule
\multirow{9}{*}{\textbf{Toys}} & \textbf{DreamRec} & 0.0006 & 0.0013 & 0.0005 & 0.0008 & 0.0008 & 0.0019 & 0.0007 & 0.0012 & 0.0076 & 0.0137 & 0.0052 & 0.0074 & 1,093 \\
 & \textbf{E4SRec} & 0.0065 & 0.0108 & {\ul 0.0056} & 0.0072 & 0.0089 & 0.0144 & {\ul 0.0075} & {\ul 0.0096} & 0.0084 & 0.0235 & 0.0055 & 0.0111 & \textbf{905} \\ \cmidrule{2-15} 
 & \textbf{BIGRec} & 0.0009 & 0.0016 & 0.0009 & 0.0012 & 0.0011 & 0.0013 & 0.0010 & 0.0011 & 0.0194 & 0.0311 & 0.0147 & 0.0191 & 43,304 \\
 & \textbf{IDGenRec} & 0.0030 & 0.0053 & 0.0022 & 0.0031 & 0.0043 & 0.0086 & 0.0032 & 0.0048 & 0.0189 & 0.0364 & 0.0161 & 0.0224 & 30,720 \\
 & \textbf{CID} & 0.0027 & 0.0047 & 0.0025 & 0.0033 & 0.0055 & 0.0084 & 0.0044 & 0.0056 & 0.0055 & 0.0156 & 0.0044 & 0.0081 & {27,248} \\
 & \textbf{SemID} & 0.0024 & 0.0042 & 0.0018 & 0.0024 & 0.0034 & 0.0055 & 0.0026 & 0.0034 & 0.0140 & 0.0275 & 0.0095 & 0.0143 & 32,288 \\
 & \textbf{TIGER} & {\ul 0.0068} & {\ul 0.0117} & 0.0054 & {\ul 0.0072} & {\ul 0.0094} & {\ul 0.0159} & 0.0070 & 0.0095 & {\ul 0.0384} & {\ul 0.0715} & {\ul 0.0291} & {\ul 0.0408} & {13,800} \\
 & \textbf{LETTER} & 0.0057 & 0.0093 & 0.0050 & 0.0064 & 0.0080 & 0.0126 & 0.0066 & 0.0085 & 0.0217 & 0.0416 & 0.0170 & 0.0239 & 13,800 \\ \cmidrule{2-15} 
 & \cellcolor[HTML]{ECF4FF}\textbf{SETRec} & \cellcolor[HTML]{ECF4FF}\textbf{0.0116*} & \cellcolor[HTML]{ECF4FF}\textbf{0.0188*} & \cellcolor[HTML]{ECF4FF}\textbf{0.0095*} & \cellcolor[HTML]{ECF4FF}\textbf{0.0120*} & \cellcolor[HTML]{ECF4FF}\textbf{0.0144*} & \cellcolor[HTML]{ECF4FF}\textbf{0.0236*} & \cellcolor[HTML]{ECF4FF}\textbf{0.0118*} & \cellcolor[HTML]{ECF4FF}\textbf{0.0151*} & \cellcolor[HTML]{ECF4FF}\textbf{0.0531*} & \cellcolor[HTML]{ECF4FF}\textbf{0.0883*} & \cellcolor[HTML]{ECF4FF}\textbf{0.0382*} & \cellcolor[HTML]{ECF4FF}\textbf{0.0507*} & \cellcolor[HTML]{ECF4FF}{\ul 926} \\ \midrule\midrule
\multirow{9}{*}{\textbf{Beauty}} & \textbf{DreamRec} & 0.0007 & 0.0009 & 0.0005 & 0.0005 & 0.0010 & 0.0011 & 0.0007 & 0.0007 & 0.0090 & 0.0167 & 0.0075 & 0.0103 & 1,326 \\
 & \textbf{E4SRec} & {\ul 0.0067} & {\ul 0.0109} & {\ul 0.0056} & {\ul 0.0072} & {\ul 0.0088} & {\ul 0.0146} & {\ul 0.0072} & {\ul 0.0094} & 0.0017 & 0.0071 & 0.0010 & 0.0029 & \textbf{910} \\ \cmidrule{2-15} 
 & \textbf{BIGRec} & 0.0006 & 0.0010 & 0.0006 & 0.0007 & 0.0010 & 0.0010 & 0.0008 & 0.0008 & 0.0141 & 0.0246 & 0.0094 & 0.0135 & 29,500 \\
 & \textbf{IDGenRec}  & 0.0042 & 0.0078 & 0.0030 & 0.0043 & 0.0045 & 0.0104 & 0.0033 & 0.0054 & {\ul 0.0254} & {\ul 0.0471} & {\ul 0.0207} & {\ul 0.0292} & 35,040 \\
 & \textbf{CID} & 0.0046 & 0.0077 & 0.0040 & 0.0052 & 0.0059 & 0.0107 & 0.0051 & 0.0068 & 0.0075 & 0.0155 & 0.0071 & 0.0096 & {27,792} \\
 & \textbf{SemID} & 0.0030 & 0.0045 & 0.0027 & 0.0033 & 0.0050 & 0.0076 & 0.0042 & 0.0052 & 0.0159 & 0.0227 & 0.0116 & 0.0159 & 45,160 \\
 & \textbf{TIGER} & 0.0041 & 0.0065 & 0.0032 & 0.0041 & 0.0054 & 0.0085 & 0.0042 & 0.0054 & 0.0083 & 0.0167 & 0.0064 & 0.0091 & {12,600} \\
 & \textbf{LETTER} & 0.0040 & 0.0069 & 0.0031 & 0.0042 & 0.0051 & 0.0088 & 0.0039 & 0.0054 & 0.0043 & 0.0129 & 0.0043 & 0.0071 & 12,600 \\ \cmidrule{2-15} 
 & \cellcolor[HTML]{ECF4FF}\textbf{SETRec} & \cellcolor[HTML]{ECF4FF}\textbf{0.0104*} & \cellcolor[HTML]{ECF4FF}\textbf{0.0167*} & \cellcolor[HTML]{ECF4FF}\textbf{0.0085*} & \cellcolor[HTML]{ECF4FF}\textbf{0.0108*} & \cellcolor[HTML]{ECF4FF}\textbf{0.0140*} & \cellcolor[HTML]{ECF4FF}\textbf{0.0221*} & \cellcolor[HTML]{ECF4FF}\textbf{0.0109*} & \cellcolor[HTML]{ECF4FF}\textbf{0.0141*} & \cellcolor[HTML]{ECF4FF}\textbf{0.0477*} & \cellcolor[HTML]{ECF4FF}\textbf{0.0748*} & \cellcolor[HTML]{ECF4FF}\textbf{0.0370*} & \cellcolor[HTML]{ECF4FF}\textbf{0.0464*} & \cellcolor[HTML]{ECF4FF}{\ul 1,050} \\ \bottomrule
\end{tabular}
}}
\label{tab:Overall_performance_on_Qwen}
\end{table*}



\subsection{Overall Performance (RQ1)}\label{sec:overall_performance}


\subsubsection{\textbf{Performance on T5.}} 
The performance comparison between baselines and SETRec instantiated on T5 are shown in Table~\ref{tab:overall_performance}, from which we have the following observations: 
\begin{itemize}[leftmargin=*]
    % 1. token-seq-based 整体会比单一的embedding表示好。这是因为他们利用了多token来表示丰富的item信息。针对用human vocab来表示的方法,他们能够利用上语言模型内部的pre-training知识;针对那些词表的方法,他们将信息压缩到了多个token里,让item的表示更加具有层次化。
    \item Token-sequence identifier (BIGRec, IDGenRec, CID, SemID, TIGER, LETTER) generally performs better than single-token identifier under ``all'', ``warm'', and ``cold'' settings. This is reasonable because token-sequence identifier represent each item with multiple tokens, which explicitly encode rich item information into different dimensions.   
    % 2. token-seq-based中,用codebook的比用human vocab的大部分情况要好一些。这主要是因为他们利用了hierarchy的信息,从粗粒度到细粒度,一定程度上缓解了local optima的问题。
    \item Among the token-sequence identifiers, methods with external tokens (CID, SemID, TIGER, LETTER) generally outperform those relying on human vocabulary (\eg BIGRec) under ``all'' and ``warm'' settings. 
    This is attributed to their hierarchically structured identifier, where the initial tokens represent coarse-grained semantics while subsequent tokens contain fine-grained semantics. 
    This aligns better with the autoregressive generation process, potentially alleviating the local optima issue~\cite{wang2024learnable}. 
    % 3. 分析一下在cold场景下哪些更好:对于只用cf的方法(dreamrec, e4srec, cid),他们在cold上面效果不行。而那些利用了寓意信息的方法,在cold上表现就比较优秀。但是codebook的大多数情况仍然不如human vocab的那些方法。
    \item When recommending cold items\footnote{The higher values on cold performance are due to the limited number of cold items.}, methods that merely utilize CF information (DreamRec, E4SRec, and CID) fail to give satisfying results. 
    This is not surprising since CF information depends heavily on substantial interactions for training, thereby struggling with cold items. 
    In contrast, methods that integrate semantics into identifiers (BIGRec, IDGenRec, SemID, TIGER, and LETTER) generalize better on cold-start scenarios (superior performance under ``cold'' setting). 
    Specifically, BIGRec and IDGenRec tend to have competitive performance. 
    This is reasonable because they utilize readable human vocabulary to represent each item, which better leverages rich world knowledge encoded in LLMs. 

    % 4. 我们的方法significantly/constantly超过了其他方法。在accuracy上,我们在all,warm,和cold上都显著超越。我们利用了cf的信息,让那些拥有丰富交互的warm item能够被准确的推荐;此外我们利用了多维度的semantic信息,这让我们的模型能够泛化到cold item上面去。
    \item SETRec significantly outperform all baselines under ``all'', ``warm'', and ``cold'' settings across all four datasets. 
    The superior performance is attributed to 
    % 1) the incorporation of both CF and semantic information, which ensures the items with similar interactions have similar identifiers, thus recommending warm items accurately; 
    % 2) representation of rich semantics into multiple embeddings, which encourages the identifier to contain semantics of different dimensions, thus strengthening the cold-start generalization. 
    1) the incorporation of both CF and semantic information into a set of tokens, which ensures accurate warm item recommendation and strong generalization on cold items; 
    2) order agnosticism of identifier, which removes the possibly inaccurate dependencies across different tokens associated with an identifier. 
    
    \item From the perspective of efficiency, SETRec significantly reduces the inference time costs compared to the token-sequence identifiers. 
    SETRec achieves an average 15$\times$, 11$\times$, 18$\times$, and 8$\times$ speedup on Toys, Beauty, Sports, and Steam, respectively, compared to token-sequence identifiers. 
    The high efficiency is attributed to the simultaneous generation, which generates multiple tokens at a single LLM call, unlocking the real-world deployment of LLM-based generative recommendation. 
    % from perspective of efficiency, 我们的方法显著的超越了seq-based的这些方法,实现用一个single step就能够生成
\end{itemize}


% Overall performance on Qwen
% Please add the following required packages to your document preamble:
% \usepackage{multirow}

% \noindent$\bullet\quad$\textbf{Performance on Qwen-1.5B.} 




\subsubsection{\textbf{Performance on Qwen-1.5B}}
To evaluate SETRec on decoder-only LLMs, we instantiate SETRec and all baselines on Qwen-1.5B. We present the results on Toys and Beauty\footnote{We omit the results with similar observations on other datasets to save space.} in Table~\ref{tab:Overall_performance_on_Qwen}, from which we summarize several key different observations from performance on T5 as follows: 
% observations

\begin{itemize}[leftmargin=*]
    % 1. 在qwen上和t5不同的地方是,seq-based失去了它显著的效果,我们猜测这主要是因为qwen的参数量更大,他拥有更强的预训练知识。因此难以在数据有限的情况下很快的adpat到推荐任务上。相反的,E4SRec大部分情况能有非常competitive 的performance。我们猜测这主要是因为它把之前的vocabulary head换成了新的logits,这样利于大语言模型从原来的pre-training任务上adapt到推荐任务上,通过后面那个head高效调整。  
    \item Token-sequence identifiers show limited competitiveness compared to the counterparts on T5. 
    % We suspect that this might be caused by the magnified knowledge gap between the pre-training data and the recommendation data. 
    A possible reason is that Qwen-1.5B probably contains richer knowledge within its parameters, which amplifies the knowledge gap between the pre-training and recommendation tasks,  thereby hindering its adaptation to recommendation tasks with limited interaction data.  
    Conversely, E4SRec yields competitive performance in most cases. 
    This makes sense because E4SRec removes the original vocabulary head and replaces it with an item projection head, thus facilitating effective adaption to the recommendation tasks. 
    % 2. 和t5相比,这些用human vocab的在cold上面会有比较好的performance。-> 这个符合直觉。但是词表这种反而下降了,这个也符合直觉,因为需要更多的interaction来adapt,否则生成概率会偏低
    \item BIGRec and IDGenRec outperform their T5 counterparts on cold items on Beauty. Because they represent items with human vocabulary, which can leverage the rich world knowledge within Qwen-1.5B for better generalization. 
    On the contrary, identifiers with external tokens have inferior cold performance compared to their T5 counterparts. 
    This is also reasonable since it requires extensive interaction data to train external tokens. Otherwise, it is difficult for it to generalize to cold items accurately due to the low generation probability of these external tokens. 
    % 3. 我们的方法仍然能稳定的超过baseline,. 并且稳定的比t5要好。尤其是cold上面的performance。-> 在qwen上比较好的表现验证了我们方法在不同模型架构上的泛化能力。 
    \item SETRec constantly outperforms baselines, which is consistent with the observations on T5. 
    Notably, SETRec instantiated on Qwen-1.5B steadily surpasses SETRec on T5, especially under the ``cold'' setting. 
    This validates the strong generalization ability of SETRec on different architectures of LLMs. 
    Moreover, as the LLM size increases, the efficiency improvements over the token-sequence identifiers are more significant, resulting in an average of 20$\times$ speedup across the two datasets. 
    
\end{itemize}




\subsection{In-depth Analysis}

\subsubsection{\textbf{Ablation Study (RQ2)}} 
To study the effectiveness of each component of SETRec, we separately remove semantic tokens (``w/o Sem''), 
CF token (``w/o CF'').  
In addition, we replace learnable query vectors with random frozen vectors (``w/o Query'') and 
use the original attention mask (``w/o SA''), to evaluate the effect of query vectors and the sparse attention mask, respectively. 
The results of different ablation variants on T5 and Qwen-1.5B on Toys are presented in Figure~\ref{fig:ablation} and we omit the results on other datasets with similar observations to save space. 

From the figures, we can find similar observations on T5 and Qwen that 
% t5和qwen都有的现象:
% 1. 单独移除每一个元素在all,warm, cold 上performance都下降了。这验证了每个component的有效性。
1) removing each component causes performance drops under ``all'', ``warm'', and ``cold'' settings, which validates the effectiveness of each component of SETRec. 
% 2. 一处semantic对cold的影响非常大。这也说明了semantic对于cold start item的重要性。验证了引入semantic是必要的。
2) Discarding semantic tokens drastically degrades the recommendation accuracy under ``cold'' settings. 
This demonstrates the necessity of incorporating semantics into identifiers. 
% 3. 相比于移除cf embedding,移除semantic反而会让performance下降更多。这个interesting现象我们猜测是源于我们使用了多个embedding。这个现象也和XXX里的观测一致。我们补充了只有一个semantic的实验结果在appendix)
Interestingly, 
3) removing semantic tokens leads to worse performance compared to removing CF token. 
The possible reason for this is the utilization of multiple semantic tokens to represent each item, which highlights the significance of leveraging multi-dimensional semantic information. 
This observation is also consistent with the results in~\cite{lin2024bridging}. 
% t5和qwen不一样的现象:主要是在cold上,去掉cf有时候反而有更好的cold start performance。这个可能的原因是参数量大的模型能比参数量小的模型拥有更好的语义理解。更detialed analysis of the balance between cf and semantics are provided in XXX
Nonetheless, 
4) while removing CF tokens for T5 leads to inferior performance on cold items, using CF tokens for Qwen might negatively impact on cold items. 
A possible reason is that the larger-size Qwen is better at understanding semantics due to its stronger knowledge base encoded in the parameters, making the contribution of CF less significant. 



% % ablation figures on Toys
% \begin{figure}[t]
% % \vspace{-0.2cm}
% \setlength{\abovecaptionskip}{-0.15cm}
% \setlength{\belowcaptionskip}{-0cm}
%   \centering 
%   % \hspace{-0.7in}
%   \subfigure{
%     \includegraphics[height=1.35in]{figures/ablation-toys-t5-all.pdf}} 
%   % \hspace{-0.105in}
%   \subfigure{
%     \includegraphics[height=1.35in]{figures/ablation-toys-qwen-all.pdf}} 
%   % \hspace{-0.105in}
%   \subfigure{
%     \includegraphics[height=1.35in]{figures/ablation-toys-t5-warm.pdf}} 
%   % \hspace{-0.105in}
%   \subfigure{
%     \includegraphics[height=1.35in]{figures/ablation-toys-qwen-warm.pdf}} 
%   % \hspace{-0.105in}
%   \subfigure{
%     \includegraphics[height=1.35in]{figures/ablation-toys-t5-cold.pdf}} 
%   % \hspace{-0.105in}
%   \subfigure{
%     \includegraphics[height=1.35in]{figures/ablation-toys-qwen-cold.pdf}} 
%   % \hspace{-0.105in}
% \caption{Ablation study on Toys.}
%   \label{fig:ablation}
%   % \vspace{-0.3cm}
% \end{figure}



\begin{figure}[t]
% \vspace{-0.2cm}
\setlength{\abovecaptionskip}{0.02cm}
\setlength{\belowcaptionskip}{-0.3cm}
\centering
\includegraphics[scale=1.2]{figures/ablation.pdf}
\caption{Ablation study on Toys.}
\label{fig:ablation}
\end{figure}

\subsubsection{\textbf{Item Group Analysis (RQ3)}}
To understand how SETRec improves performance, we evaluate it over items with different popularity. 
% item group是怎么划分的 - 我们根据item popularity 排序,然后分成5组到Group1-group5, (从最popular到最不popular)
We divide the items into 5 groups according to their frequencies and test the models over each group respectively. 
The performance comparison between SETRec and two competitive baselines from token-sequence identifiers (LETTER) and single-token identifiers (E4SRec) are reported in Figure~\ref{fig:group_analysis}. 
We can observe that 
% 1. 从most popular 到least popular, item group performance 是在逐渐下降的,这符合预期。因为交互越少的item,llm能够decode出来的概率就会更低,欠拟合
1) the performance gradually drops from G1 to G5. 
This makes sense since the less popular items have fewer interactions for LLMs to learn, thus leading to worse generation probabilities. 
% 2. e4srec在第一组比letter要强很多,但是随着item的popularity降低,letter慢慢超过e4srec。这也符合我们的直觉。e4srec是纯靠cf信息的,非常依赖于大量的交互来学习cf info。而letter同时利用了语义信息,会在sparse的item上面有更好的表现
Besides, 
2) E4SRec outperforms LETTER on most popular items (G1) but usually yields inferior performance on unpopular items (G2-G5). 
This is due to that E4SRec only uses CF information, which relies on substantial interactions and therefore struggle on unpopular items. 
In contrast, LETTER additionally incorporates semantics into identifiers, thus achieving better generalization on sparse items. 
% 3. 每一组里面我们的方法都稳定的超过了competitive baselines。除此之外提升的百分比是在unpopular的item上面有更强的优势。这也部分说明了我们方法的泛化能力很强。
3) SETRec consistently excels both E4SRec and LETTER over all groups. 
Notably, the improvements over sparse items are more significant, which partially explains the superiority of SETRec regarding overall performance.  



% group analysis figure
\begin{figure}[t]
\vspace{-0.2cm}
\setlength{\abovecaptionskip}{-0.15cm}
\setlength{\belowcaptionskip}{-0cm}
  \centering 
  % \hspace{-0.7in}
  \subfigure{
    \includegraphics[height=1.65in]{figures/group_analysis_R10.pdf}} 
  % \hspace{-0.105in}
  \subfigure{
    \includegraphics[height=1.65in]{figures/group_analysis_N10.pdf}} 
\caption{Performance of SETRec, LETTER, and E4SRec (T5) on item groups with different popularity on Toys.}
  \label{fig:group_analysis}
  % \vspace{-0.3cm}
\end{figure}

\subsubsection{\textbf{Scalability on Model Parameters (RQ3)}}
To investigate whether SETRec can bring continuous performance when expanding the model parameters, we test SETRec on Qwen with different model sizes (1.5B, 3B, and 7B). 
Performance comparisons between SETRec, E4SRec, and LETTER on Toys are shown in Table~\ref{tab:scaling_performance}. 
% and the results on other datasets with similar observations are omitted to save space.
From the results, we can find that 
% 1. SETRec在cold上有比较明显的scaling,这因为模型对语义理解的能力更强。这展现了在cold start上比较promising的scaling的能力
1) SETRec clearly shows continued improvements over cold-start items when the model size scales from 1.5B to 7B, demonstrating promising scalability on cold items. 
We attribute this to the continued improvements of better semantic understanding by expanding the model parameters. 
% 2. SETRec在warm上可能已经到达瓶颈了,随着参数量的提升,对cf信息的接受没有进一步的提升。这个在e4srec的结果上也可以看得出来
Nonetheless, 
2) the performance on the warm items fails to continuously improve, indicating a relatively limited scalability over warm items. 
This shows that the larger models do not necessarily lead to better CF information understanding, which can also be indicated by the limited improvements of E4SRec under ``warm'' setting. 
% 3. 对于LETTER这种利用语意的identifier方法,也面临瓶颈。主要是因为扩展词表,其实和模型内部的语义没有很好的align,继续scale模型对于cold的提升其实作用并不明显
Besides, 
3) LETTER shows weak scalability over the three settings. 
This is mainly due to the utilization of external tokens, which do not necessarily align with the pre-trained knowledge in LLMs, thus showing limited improvements by expanding the model parameters. 


% Please add the following required packages to your document preamble:
% \usepackage{multirow}
% \begin{table*}[t]
% \setlength{\abovecaptionskip}{0.05cm}
% \setlength{\belowcaptionskip}{0.2cm}
% \caption{Performance comparison between SETRec and competitive baselines with different LLM sizes on Qwen. }
% \setlength{\tabcolsep}{2.5mm}{
% \resizebox{\textwidth}{!}{
% \begin{tabular}{clccccccccccccl}
% \toprule
% \multicolumn{15}{c}{\textbf{Toys}} \\ \hline
% \multicolumn{1}{l|}{} & \multicolumn{1}{l|}{} & \multicolumn{4}{c|}{\textbf{All}} & \multicolumn{4}{c|}{\textbf{Warm}} & \multicolumn{4}{c|}{\textbf{Cold}} & \multicolumn{1}{c}{\textbf{Inference Time (s)}} \\ \hline
% \multicolumn{1}{l|}{\textbf{Model Size}} & \multicolumn{1}{l|}{\textbf{Method}} & \textbf{R@5} & \textbf{R@10} & \textbf{N@5} & \multicolumn{1}{c|}{\textbf{N@10}} & \textbf{R@5} & \textbf{R@10} & \textbf{N@5} & \multicolumn{1}{c|}{\textbf{N@10}} & \textbf{R@5} & \textbf{R@10} & \textbf{N@5} & \multicolumn{1}{c|}{\textbf{N@10}} & \multicolumn{1}{c}{\textbf{All Users}} \\ \midrule
% \multicolumn{1}{c|}{\multirow{3}{*}{\textbf{1.5B}}} & \multicolumn{1}{l|}{\textbf{LETTER}} & 0.0057 & 0.0093 & 0.005 & \multicolumn{1}{c|}{0.0064} & 0.008 & 0.0126 & 0.0066 & \multicolumn{1}{c|}{0.0085} & 0.0217 & 0.0416 & 0.017 & \multicolumn{1}{c|}{0.0239} &  \\
% \multicolumn{1}{c|}{} & \multicolumn{1}{l|}{\textbf{E4SRec}} & 0.0065 & 0.0108 & 0.0056 & \multicolumn{1}{c|}{0.0072} & 0.0089 & 0.0144 & 0.0075 & \multicolumn{1}{c|}{0.0096} & 0.0084 & 0.0235 & 0.0055 & \multicolumn{1}{c|}{0.0111} &  \\
% \multicolumn{1}{c|}{} & \multicolumn{1}{l|}{\cellcolor{gray!16}\textbf{SETRec}} & \cellcolor{gray!16}\textbf{0.0116} & \cellcolor{gray!16}\textbf{0.0188} & \cellcolor{gray!16}\textbf{0.0095} & \multicolumn{1}{c|}{\cellcolor{gray!16}\textbf{0.012}} & \cellcolor{gray!16}\textbf{0.0144} & \cellcolor{gray!16}\textbf{0.0236} & \cellcolor{gray!16}\textbf{0.0118} & \multicolumn{1}{c|}{\cellcolor{gray!16}\textbf{0.0151}} & \cellcolor{gray!16}\textbf{0.0531} & \cellcolor{gray!16}\textbf{0.0883} & \cellcolor{gray!16}\textbf{0.0382} & \multicolumn{1}{c|}{\cellcolor{gray!16}\textbf{0.0507}} &  \\ \midrule
% \multicolumn{1}{c|}{\multirow{3}{*}{\textbf{3B}}} & \multicolumn{1}{l|}{\textbf{LETTER}} & 0.0057 & 0.0109 & 0.0053 & \multicolumn{1}{c|}{0.0072} & 0.0078 & 0.0151 & 0.0069 & \multicolumn{1}{c|}{0.0097} & 0.0254 & 0.0471 & 0.0162 & \multicolumn{1}{c|}{0.0236} &  \\
% \multicolumn{1}{c|}{} & \multicolumn{1}{l|}{\textbf{E4SRec}} & 0.0062 & 0.0096 & 0.0048 & \multicolumn{1}{c|}{0.0061} & 0.0082 & 0.0129 & 0.0062 & \multicolumn{1}{c|}{0.0081} & 0.0084 & 0.0218 & 0.0053 & \multicolumn{1}{c|}{0.0103} &  \\
% \multicolumn{1}{c|}{} & \multicolumn{1}{l|}{SETRec} & \textbf{0.0118} & \textbf{0.0195} & \textbf{0.0095} & \multicolumn{1}{c|}{\textbf{0.0123}} & \textbf{0.015} & \textbf{0.0258} & \textbf{0.0119} & \multicolumn{1}{c|}{\textbf{0.0159}} & \textbf{0.065} & \textbf{0.0964} & \textbf{0.0462} & \multicolumn{1}{c|}{\textbf{0.0571}} &  \\ \midrule
% \multicolumn{1}{c|}{\multirow{3}{*}{\textbf{7B}}} & \multicolumn{1}{l|}{\textbf{LETTER}} & 0.0057 & 0.0099 & 0.0044 & \multicolumn{1}{c|}{0.0061} & 0.0078 & 0.0137 & 0.0057 & \multicolumn{1}{c|}{0.0081} & 0.0215 & 0.0406 & 0.0144 & \multicolumn{1}{c|}{0.0216} &  \\
% \multicolumn{1}{c|}{} & \multicolumn{1}{l|}{\textbf{E4SRec}} & 0.0048 & 0.0088 & 0.0041 & \multicolumn{1}{c|}{0.0057} & 0.0062 & 0.0114 & 0.0053 & \multicolumn{1}{c|}{0.0072} & 0.0064 & 0.0133 & 0.0037 & \multicolumn{1}{c|}{0.0065} &  \\
% \multicolumn{1}{c|}{} & \multicolumn{1}{l|}{\cellcolor{gray!16}\textbf{SETRec}} & \cellcolor{gray!16}\textbf{0.0107} & \cellcolor{gray!16}\textbf{0.0194} & \cellcolor{gray!16}\textbf{0.0083} & \multicolumn{1}{c|}{\cellcolor{gray!16}\textbf{0.0115}} & \cellcolor{gray!16}\textbf{0.0127} & \cellcolor{gray!16}\textbf{0.0239} & \cellcolor{gray!16}\textbf{0.01} & \multicolumn{1}{c|}{\cellcolor{gray!16}\textbf{0.014}} & \cellcolor{gray!16}\textbf{0.0632} & \cellcolor{gray!16}\textbf{0.1016} & \cellcolor{gray!16}\textbf{0.0482} & \multicolumn{1}{c|}{\cellcolor{gray!16}\textbf{0.0613}} &  \\ \bottomrule
% \end{tabular}
% }}
% \label{tab:scaling_performance}
% \end{table*}

\begin{table}[t]
\setlength{\abovecaptionskip}{0.05cm}
\setlength{\belowcaptionskip}{0.2cm}
\caption{Performance comparison between SETRec and competitive baselines with different LLM sizes on Qwen. }
\setlength{\tabcolsep}{2.2mm}{
\resizebox{0.46\textwidth}{!}{
\begin{tabular}{clcccccc}
\toprule
% \multicolumn{8}{c}{\textbf{Toys}} \\ \midrule
\multicolumn{1}{l|}{} & \multicolumn{1}{l|}{} & \multicolumn{2}{c}{\textbf{All}} & \multicolumn{2}{c}{\textbf{Warm}} & \multicolumn{2}{c}{\textbf{Cold}} \\
\multicolumn{1}{l|}{} & \multicolumn{1}{l|}{} & \textbf{R@10} & \textbf{N@10} & \textbf{R@10} & \textbf{N@10} & \textbf{R@10} & \textbf{N@10} \\ \midrule\midrule
\multicolumn{1}{c|}{\multirow{3}{*}{\textbf{1.5B}}} & \multicolumn{1}{l|}{\textbf{LETTER}} & 0.0093 & 0.0064 & 0.0126 & 0.0085 & 0.0416 & 0.0239 \\
\multicolumn{1}{c|}{} & \multicolumn{1}{l|}{\textbf{E4SRec}} & 0.0108 & 0.0072 & 0.0144 & 0.0096 & 0.0235 & 0.0111 \\
\multicolumn{1}{c|}{} & \multicolumn{1}{l|}{\cellcolor{gray!16}\textbf{SETRec}} & \cellcolor{gray!16}\textbf{0.0188} & \cellcolor{gray!16}\textbf{0.0120} & \cellcolor{gray!16}\textbf{0.0236} & \cellcolor{gray!16}\textbf{0.0151} & \cellcolor{gray!16}\textbf{0.0883} & \cellcolor{gray!16}\textbf{0.0507} \\ \midrule
\multicolumn{1}{c|}{\multirow{3}{*}{\textbf{3B}}} & \multicolumn{1}{l|}{\textbf{LETTER}} & 0.0109 & 0.0072 & 0.0151 & 0.0097 & 0.0471 & 0.0236 \\
\multicolumn{1}{c|}{} & \multicolumn{1}{l|}{\textbf{E4SRec}} & 0.0096 & 0.0061 & 0.0129 & 0.0081 & 0.0218 & 0.0103 \\
\multicolumn{1}{c|}{} & \multicolumn{1}{l|}{\cellcolor{gray!16}\textbf{SETRec}} & \cellcolor{gray!16}\textbf{0.0195} & \cellcolor{gray!16}\textbf{0.0123} & \cellcolor{gray!16}\textbf{0.0258} & \cellcolor{gray!16}\textbf{0.0159} & \cellcolor{gray!16}\textbf{0.0964} & \cellcolor{gray!16}\textbf{0.0571} \\ \midrule
\multicolumn{1}{c|}{\multirow{3}{*}{\textbf{7B}}} & \multicolumn{1}{l|}{\textbf{LETTER}} & 0.0099 & 0.0061 & 0.0137 & 0.0081 & 0.0406 & 0.0216 \\
\multicolumn{1}{c|}{} & \multicolumn{1}{l|}{\textbf{E4SRec}} & 0.0088 & 0.0057 & 0.0114 & 0.0072 & 0.0133 & 0.0065 \\
\multicolumn{1}{c|}{} & \multicolumn{1}{l|}{\cellcolor{gray!16}\textbf{SETRec}} & \cellcolor{gray!16}\textbf{0.0194} & \cellcolor{gray!16}\textbf{0.0115} & \cellcolor{gray!16}\textbf{0.0239} & \cellcolor{gray!16}\textbf{0.0140} & \cellcolor{gray!16}\textbf{0.1016} & \cellcolor{gray!16}\textbf{0.0613} \\ \bottomrule
\end{tabular}
}}
\label{tab:scaling_performance}
\end{table}

\begin{figure*}[t]
% \vspace{-0.2cm}
\setlength{\abovecaptionskip}{-0.15cm}
\setlength{\belowcaptionskip}{-0cm}
  \centering 
  \hspace{-0.105in}
  \subfigure{
    \includegraphics[height=1.4in]{figures/hyper_alpha_R_10.pdf}} 
  % \hspace{-0.105in}
  \subfigure{
    \includegraphics[height=1.4in]{figures/hyper_alpha_N_10.pdf}} 
  \subfigure{
    \includegraphics[height=1.4in]{figures/hyper_N_R_10.pdf}}
  \subfigure{
    \includegraphics[height=1.4in]{figures/hyper_N_N_10.pdf}}
\caption{Performance of SETRec (T5) with different strength of AE loss $\alpha$ and different numbers of semantic tokens $N$.}
  \label{fig:hp}
  % \vspace{-0.3cm}
\end{figure*}

\subsubsection{\textbf{Effect of Semantic Strength $\bm{\beta}$ (RQ4)}}

\begin{figure}[t]
\vspace{-0.2cm}
\setlength{\abovecaptionskip}{-0.15cm}
\setlength{\belowcaptionskip}{-0.15cm}
  \centering 
  \hspace{-0.105in}
  \subfigure{
  \includegraphics[height=1.4in]{figures/hp_beta_warm.pdf}} 
  \hspace{-0.105in}
  \subfigure{    
  \includegraphics[height=1.4in]{figures/hp_beta_cold.pdf}} 
\caption{Performance of SETRec (T5) with different strength of semantics $\beta$ for inference.}
  \label{fig:hp_beta}
  % \vspace{-0.3cm}
\end{figure}

% 1. 如果只用cf的话,performance不行(significant inferior performance of beta=0 than beta>0)。这说明当同时利用cf和sem来encode user embedding的时候,decode也需要semantic的帮助。并且在cold上的提升要明显比warm上的提升大,这也说明了对cold start item推荐时semantic引入的必要性。
% 2. 持续增大到只用semantics时在warm和cold上仍然能取得不错的performance。说明了item rich semantic信息对于warm item的推荐也是有帮助的。这可能是因为在训练的时候semantic和cf之间achieve implicit alignment,无脑全入semantic也不会让performance掉太多。蕾丝的现象也在ablation里有观察到。
To investigate how semantic information contributes to the performance during inference, we vary $\beta$ from $0$ to $1$, where $\beta=0$ indicates that only CF score is used for ranking, and $\beta=1$ ranks items based solely on semantic scores (Eq. (\ref{eqn:single_logits})).  
From the results reported in Figure~\ref{fig:hp_beta}. 
we can find that 
1) Incorporating semantic information during inference is necessary (inferior performance of $\beta=0$ than $\beta>0$, which facilitates
global ranking over multi-dimensional information and lead to strong generalization ability. 
Notably, 
2) incorporating semantic scores brings more significant improvements on cold items, underscoring the critical role of semantic information for zero-shot scenarios.
Moreover, 
3) Gradually increase $\beta$ to rely solely on semantics ($\beta=1$), SETRec maintains competitive performance on warm items, which is probably attributed to the implicit alignment between CF and semantic tokens during training. 



\subsubsection{\textbf{Hyper-parameter Sensitivity (RQ4)}}\label{sec:exp_hyper_param}
We further study the hyper-parameter sensitivity to facilitate SETRec application.

\noindent$\bullet\quad$\textbf{Effect of $\bm{\alpha}$.} 
We vary the strength of AE loss $\alpha$ for SETRec training and present the results on Toys in Figure~\ref{fig:hp}(a-b). 
We can observe that 
% 1. alpha 从0慢慢增大,performance提升。这合理因为tokenizer肯定是需要随着llm一起优化的。
1) the performance is overall improved when $\alpha$ is increased from $0$ to $0.7$, which validates the effectiveness of reconstruction loss that encourages AE to preserve useful information in the latent space. 
% 2. 但是tokenizer的权重不能太高,和llm的loss一起做multi-task training的话,可能会导致模型偏向tokenizer更多,而这可能反过来影响llm,使他推荐能力学的差。(这个理由再想想,现在不太行)
Nonetheless, 
2) while continuously increasing $\alpha$ generally gives better performance on cold-start items, it might hurt the performance under ``warm'' setting. 
Based on the empirical results, we recommend setting $\alpha$ ranging from $0.5$ to $0.7$. 






\noindent$\bullet\quad$\textbf{Effect of $\bm{N}$.} 
We change the number of semantic tokens from $1$ to $6$ to investigate how $N$ affects the performance. 
From the results shown in Figure~\ref{fig:hp}(c-d), we can find that 
% 1. N提高,有提升。说明多侧面语义信息是有用的。
1) gradually increasing semantic tokens generally improves the performance, which validates the effectiveness of incorporating multiple tokens to mitigate the potential information conflicts~\cite{wang2024learnable} and embedding collapse issue~\cite{guoembedding}. 
% 2. 但是继续提升,反而会有所下降。这主要是因为XXX?
However, 
2) blindly increasing the number of semantic tokens might hurt the performance (decreased performance from $N=4$ to $N=6$). 
This is reasonable since it is non-trivial to recover the category-level preference aligning well with the real-world scenarios. 
Similar observations are also seen in~\cite{lin2024disentangled} and~\cite{lin2024temporally}. 





% \begin{figure}[t]
% % \vspace{-0.2cm}
% \setlength{\abovecaptionskip}{-0.15cm}
% \setlength{\belowcaptionskip}{-0cm}
%   \centering 
%   \hspace{-0.105in}
%   \subfigure{
%     \includegraphics[height=1.4in]{figures/hyper_N_R@10.pdf}} 
%   % \hspace{-0.105in}
%   \subfigure{
%     \includegraphics[height=1.4in]{figures/hyper_N_N@10.pdf}} 
% \caption{Performance of SETRec (T5) with different number of semantic embeddings on Toys.}
%   \label{fig:hp_N}
%   % \vspace{-0.3cm}
% \end{figure}


\section{Related Work and Discussion}
\textbf{Tool-Integrated Math LLMs.} 
Math language models adopted two major paradigms: Chain-of-Thought (CoT) reasoning and the use of external tools, such as Python programs~\citep{metamath, mammoth, openmath}. Each paradigm offers unique benefits, and recent hybrid frameworks~\citep{mammoth, tora, htl, dsmath, qwen25} increasingly seek to combine them for synergy. However, current models exhibit critical rigidity, motivating our work to realize the true metacognitive capacity that enjoys synergistic benefits of CoT and code. 

\noindent\textbf{EM for RL.} Expectation-Maximization (EM) has proven effective for maximum likelihood problems involving hidden variables, such as Expert Iteration~\citep{expertiter}, Iterative Maximum Likelihood~\citep{iml, iml1}, Meta-Reinforcement Learning~\citep{varibad, vem}, and Adversarial Games~\citep{acb}. In the context of math LLMs, the most relevant works are \citep{restem} and \citep{iml2}, which apply EM-style iterative self-training to math problem-solving. Unlike these approaches, we leverage the EM framework for guided exploration during reinforcement learning of language models.

\noindent\textbf{Conclusion.} Existing tool-integrated math language models lack the metacognitive capacity to effectively determine code integration, hindering their ability to fully realize the synergistic benefits of tool integration and CoT.  To address this critical gap, we propose a novel EM-based framework that combines guided exploration with policy optimization.  Our experiments demonstrate the limitations of standard SFT and RL in efficiently exploring the combinatorial space of code-integrated trajectories and highlight the superior training efficiency and performance of our approach.

\clearpage

\noindent\textbf{Limitations.} 
The scope of our work is primarily focused on mathematical problem-solving.  While we observe promising results on challenging benchmarks like MATH500, the generalizability of our approach to other domains requiring the metacognitive capacity of tool integration and CoT, such as scientific reasoning or code generation for general-purpose tasks, remains to be explored.  Future work should investigate the effectiveness of our framework across a wider range of tasks and domains.

% The study of tool-integrated Language models is motivated by the belief that transformers are prone to cumulative errors due to the autoregressive probabilistic decoding and hence tool integration show promise to ensure the computational precision. However, such inaccuracies seem to be largely mitigated by performing test-time scaling techniques, thus reducing the significance of tool-integrated reasoning systems.   
% This paper investigates autonomous code integration for math LLMs. To address the challenge of unreliable external supervision, we propose to factorize out the hidden methodology-selection from response generation, and develop a novel EM formulation. The EM framework alternates between computing a reference strategy for methodology-selection through self-exploration and updating language model based on the reference guidance. This approach supports an efficient joint training scheme that allows for holistic offline data collection coupled with RL training. Our extensive experiments demonstrate the effectiveness of the proposed method, and our ablation studies further elucidate the properties of the learned model.

% However, there are several limitations and areas for future work regarding AutoCode4Math. First, the generalization of methodology-selection depends significantly on the quality of the collected query set. Further research is needed to understand what characteristics of queries contribute to effective generalization. Second, we did not extensively explore the influence of hyperparameters related to RL iterations, such as dataset size and the number of iterations, in the current version. We are actively working on this. Third, as this is a preliminary work in autonomous code integration, we have not yet investigated alternative approaches for decision routing, such as using Mixture-of-Experts (MoEs)~\citep{moe}, and we not yet fully understand the fundamental reason why EM outperforms RL. These areas present important directions for advancing AutoCode capabilities in math LLMs.

\bibliography{custom}
% \bibliographystyle{acl_natbib}
\clearpage
% \appendix
% \subsection{Additional details for Theorem~1}\label{app:decomp_lattice_proof}
We provide details that were omitted in the proof of Theorem~1. First, we derive the matrix $P$.
Given the planes $H$ and  $H_0:=\{x_{d+1}=0\}$, we wish  to find a plane $H_{ref}$ that is half-way (angle-wise) between $H$ and $H_0$. This would allow to reflect points in $H$ onto $H_0$ through $H_{ref}$ where the reflection is achieved using the Householder matrix $P:=I-2\hat{n}_{ref}\hat{n}^t_{ref}$, where $\hat{n}_{ref}\in \dR^{d+1}$ is the normal of $H_{ref}$~\cite{householder1958unitary}. That is, we reflect a lattice point $p\in \dR^{d+1}$ by computing the value \({p_{\text{reflected}}=P\cdot p}\).

Next, we show that the normal 
\begin{equation*}
    \hat{n}_{ref}:=\frac{1}{\sqrt{2-\frac{2}{\sqrt{d+1}}}}\cdot \left(-\tfrac{1}{\sqrt{d+1}},\dots,-\tfrac{1}{\sqrt{d+1}},1-\tfrac{1}{\sqrt{d+1}}\right)
\end{equation*}
satisfies those requirements.\footnote{We obtained the expression for $\hat{n}_{ref}$ by first considering $d=2$, where the task is more tangible, and then generalizing to higher dimensions.}  Consider the Householder matrix
\begin{align}\label{eq:reflection}
 P&=% I-2\hat{n}\hat{n}^t= 
 I-2\hat{n}_{ref}\hat{n}^t_{ref}\nonumber\\
 &=\begin{pNiceArray}{cw{c}{1cm}c|c}[margin]
            \Block{3-3}<\Large>{I_d - \frac{1}{D-\sqrt{D}}\mathds{1}} 
            & & & \dfrac{1}{\sqrt{D}} \\
            & & & \Vdots \\
            & & & \dfrac{1}{\sqrt{D}} \\
            \hline
            \dfrac{1}{\sqrt{D}} & \dots& \dfrac{1}{\sqrt{D}} & \dfrac{1}{\sqrt{D}}
        \end{pNiceArray},
    \end{align}
where $D:=d+1$, $I_d$ is an $d\times d$ identity matrix, and $\mathds{1}$ is the $d\times d$ matrix with $1$s in all its entries. 

Consider a point $p\in A^*_d$. Next, we show that it is reflected onto the plane  $H_0$, i.e., for $v=P\cdot p$, we get $v_{d+1}=0$. To do that, we move to the basis of the integer vector space, and show that for all $1\leq i\leq d$, taking the base element $e_i=(0,\dots,1,\dots,0)$, the $(d+1)$th element of $v=PG^t\cdot e_i$ (i.e., using the generator and then the reflector) is zero. First, for all $i<d$ it holds that 
    \begin{align*}
        PG^t\cdot e_i=P\cdot
        \begin{pmatrix}
        1 &  0&  \dots&  0& -1& 0 &\dots& 0
            % 1 \\
            % 0 \\
            % \vdots \\
            % 0 \\
            % -1 \\
            % 0 \\
            % \vdots \\
            % 0
        \end{pmatrix}^t.
    \end{align*}
Now, considering that the elements of the final row of $P$ are all equal to $1/\sqrt{D}$, we obtain a zero in the $(d+1)$th dimension. It remains to calculate the expression resulting from multiplying with $e_d$:
        \begin{align*}
        PG^t\cdot e_d=P\cdot
        \begin{pmatrix}
        -\frac{D-1}{D} &  \frac{1}{D}&  \dots&  \frac{1}{D}
        \end{pmatrix}^t.
    \end{align*}
    Looking specifically at the last element, we see that it is equal to     \begin{align*}
        \frac{1}{\sqrt{D}}\cdot\frac{1-D}{D} + (D-1)\frac{1}{D}\frac{1}{\sqrt{D}}=\frac{1-D+D-1}{D\sqrt{D}}=0.
    \end{align*}

    That is, by applying the transformation $P$ on the lattice points, we reflect them onto the $x_{d+1}=0$ plane. It remains to get rid of the $(d+1)$th dimension. This is accomplished by the mapping
\begin{align*}
        E=
        \begin{pmatrix}
            1 & 0 & \dots & 0 & 0 \\
            0 & 1 & \dots & 0 & 0 \\
            \vdots & \vdots & \ddots & \vdots & 0 \\
            0 & 0 & \dots & 1 & 0
        \end{pmatrix}_{d\times(d+1)}.
    \end{align*}
    
It remains to compute the  explicit embedding $T(g):=EPG^t(g)$, for $g\in \dZ^d$. We first calculate 
    \begin{align*}
        \left(EP\right)^t=
        \begin{pNiceArray}{cw{c}{1cm}c}[margin]
            \Block{3-3}<\Large>{I_d - \frac{1}{D-\sqrt{D}}\mathds{1}} 
            & &  \\
            & &  \\
            & &  \\
            \hline
            \dfrac{1}{\sqrt{D}} & \dots & \dfrac{1}{\sqrt{D}}
        \end{pNiceArray}_{d\times(d+1)}.
    \end{align*}
 Next, it can be shown that
    \begin{align}
        T^t&=G\left(EP\right)^t\nonumber\\
        &=\begin{pmatrix}
            1 & -1 &  0  & \dots & 0 \\
            1 & 0  &  -1 & \dots & 0 \\
            \vdots & \vdots  &  \vdots  & \ddots & \vdots \\
            1 & 0  &  0  & \dots & -1 \\
            \frac{1}{D - \sqrt{D}} - 1 & \frac{1}{D - \sqrt{D}} & \frac{1}{D - \sqrt{D}} & \dots & \frac{1}{D - \sqrt{D}}
        \end{pmatrix}_{d\times(d+1)}\!\!\!\!\!\!.
    \end{align}

    %      \begin{figure}[H]
    %     \centering
    %     \begin{subfigure}[b]{0.3\textwidth}
    %         \includegraphics[width=\textwidth]{Images/EPGt_visual_explanation1.png}
    %         %\caption{Sample points in $H_0$}
    %         %\label{fig:epgt_visual1}
    %     \end{subfigure}
    %     \hfill
    %     \begin{subfigure}[b]{0.3\textwidth}
    %         \includegraphics[width=\textwidth]{Images/EPGt_visual_explanation2.png}
    %         %\caption{$H_0$ rotated to the "floor"}
    %         %\label{fig:epgt_visual2}
    %     \end{subfigure}
    %     \hfill
    %     \begin{subfigure}[b]{0.3\textwidth}
    %         \includegraphics[width=\textwidth]{Images/EPGt_visual_explanation3.png}
    %         %\caption{The samples as they look in $\dR^2$}
    %         %\label{fig:epgt_visual3}
    %     \end{subfigure}
    %     \caption{Visualization of embedding the lattice  $A_2^*$ originally defined in $\dR^3$ onto $\dR^2$ via the mapping $T$. [Left] The blue rectangle represents the plane $H$, where the corresponding $A_2^*$ lattice points are drawn in red. The points are generated by taking integer vectors in $\dR^d$ and applying the mapping $G^t$.  [Center] $H$ and $A_2^*$ is reflected onto the plane $H_0=\{x_3=0\}$ using the mapping $PG^t$. [Right] The third dimension is removed, via the mapping $E$, to yield the embedding of $A_2^*$ in $\dR^2$.}
    %     \label{fig:egpt_visual}
    % \end{figure}


\subsection{Additional details for Theorem~3}\label{app:CC}
We provide details omitted from the main body of the text. 
We start with a simplified derivation of a single annulus, which would inform the more advanced construction. Fix ${0<r_1<
  r^*}$ forced it to a single line, and define $\btheta_{r'} := \frac{r'}{{\beta^*}}f_\Lambda$, and observe that 
\begin{align}
  CC_\X&\leq  r^*\cdot
\left|\X\cap (\B_{r^*}\setminus \B_{r_1})\right| + r_1\cdot
\left|\X\cap \B_{r_1}\right|\nonumber \\ & = r^*\left(|\X\cap \B_{r^*}|-|\X \cap \B_{r_1}|\right) + r_1 \left|\X\cap \B_{r_1}\right| \nonumber\\
& = r^*|\X\cap \B_{r^*}|+ (r_1-r^*) |\X\cap \B_{r_1}| \nonumber \\  
  & = r^*\frac{\partial(B_1)}{\sqrt{\det(\Lambda)}}\btheta^d_{r^*} +r^* P_d(\btheta_{r^*}) + (r_1-r^*) \frac{\partial(B_1)}{\sqrt{\det(\Lambda)}}\btheta^d_{r_1}\nonumber\\& + (r_1-r^*) P_d(\btheta_{r_1}) \nonumber
\\
& = \frac{\partial(B_1)}{\sqrt{\det(\Lambda)}}\theta^d\left({r^*}^{d+1}+{r_1}^{d+1}-r^*{r_1}^{d}\right)\nonumber\\&+ r P_d(\btheta_{r^*}) + (r_1-r^*) P_d(\btheta_{r_1})\nonumber \\ & = \frac{\partial(B_1)}{\sqrt{\det(\Lambda)}}\theta^d\left({r^*}^{d+1}+{r_1}^{d+1}-r^*{r_1}^{d}\right)+ r^* P_d(\btheta_{r^*}), \label{eq:CC1}
\end{align}
where the sample complexity bound in Equation~(5) is used. For simplicity, we bound throughout the error term with $r^* P_d(\btheta_{r^*})$.
Next, we optimize the value $r_1$ to minimize the expression in Equation~\eqref{eq:CC1}.

Consider the function $f(r_1)={r^*}^{d+1}-{r^*} r^d_1 + {r^*}^{d+1}_1$. We look for the minimum of $f(r_1)$ by requiring that
\begin{align*}
            f'(r_1)=-{r^*} dr_1^{d-1}+(d+1)r_1^d=0,
\end{align*}
which yields the value $r'_1:=\frac{d}{d+1}{r^*}$. This value is  a minimum since
\begin{align*}
 f^{(2)}(r_1)|_{r'_1}=&\left(-{r^*}(d-1)r_1^{d-2}+d(d+1)r_1^{d-1}\right)|_{r_1'}\\
        =&{r^*}^{d-1}\left(\frac{d^d}{(d+1)^{d-2}}-\frac{d^{d-2}(d-1)}{(d+1)^{d-2}}\right)\\
        =&{r^*}^{d-1}d^{d-2}\frac{d^2-d+1}{(d+1)^{d-2}},
    \end{align*}
    and we know that $d^2-d+1>0$ for all $d\geq 2$.%, then $r_1=\frac{d}{d+1}r$ indeed minimizes $f(r_1)$.
    
Now, we apply the above line of reasoning in a recursive manner by considering a sequence of $k+1\geq 2$ radii ${0<r_k<\ldots<r_0={r^*}}$ where $r_i:=\td^i r^*$, where $\td:=\frac{d}{d+1}$. This leads to the bound
\begin{align}
\label{eq:cc_eval_app}
CC_\X&\leq \sum_{i=0}^{k-1}r_i |\X\cap (\B_{r_i}\setminus \B_{r_{i+1}})| + r_k|\X\cap \B_{r_k}|\nonumber\\
  &= \frac{\partial(B_1)}{\sqrt{\det(\Lambda)}} \left(\underbrace{{r^*} \btheta^d_{r^*} + \sum_{i=1}^k(r_i-r_{i-1}) \btheta^d_{r_i}}_{:=\gamma}\right) + {r^*} P_d(\btheta_{r^*}).
\end{align}

We show that 
\[\gamma:=r \btheta^d_{r^*} + \sum_{i=1}^k(r_i-r_{i-1}) \btheta^d_{r_i}= {r^*} \btheta^d_{r^*} \left(1 - \frac{\xi^{d+2} - \xi}{ d\xi - (d+1)}\right),\]
where $r_i=\td^i {r^*},\td:=\frac{d}{d+1}, \btheta_{r_i}= r_i\frac{\btheta_{r^*}}{r^*}, k=d,$ and $\xi:=\td^d=\left(\frac{d}{d+1}\right)^d$. In particular,
\begin{align}
  \gamma &={r^*} \btheta^d_{r^*} + \sum_{i=1}^k(r_i-r_{i-1}) r_i^d\frac{\btheta^d_{r^*}}{{r^*}^d} \nonumber\\
  & = {r^*} \btheta^d_{r^*} + \sum_{i=1}^k{r^*}\td^{i-1}(\td-1) \td^{di} {r^*}^d\frac{\btheta^d_{r^*}}{{r^*}^d} \nonumber\\ 
  &=  {r^*} \btheta^d_{r^*} + \sum_{i=1}^k{r^*} (\td-1) \td^{di+ i -1} \btheta^d_{r^*} \nonumber\\ 
  &=   {r^*} \btheta^d_{r^*} \left(1 + \sum_{i=1}^k (\td-1) \td^{di+ i -1} \right)\nonumber\\
  &= {r^*} \btheta^d_{r^*} \left(1 + \frac{\td-1}{\td}\sum_{i=1}^k \td^{(d+1)i} \right)\nonumber\\
  &= {r^*} \btheta^d_{r^*} \left(1 + \frac{\td-1}{\td}\frac{\left(\td^{d+1}\right)^{k+1} - \td^{d+1}}{\td^{d+1} - 1} \right)\nonumber\\
  &= {r^*} \btheta^d_{r^*} \left(1 + \td^d(\td-1)\frac{\left(\td^{d+1}\right)^k - 1}{\td^{d+1} - 1} \right).\nonumber\\
  % & = r \btheta^d_{r} + \sum_{i=1}^kr\td^{i-1}(\td-1) \td^{di} r^d\frac{\btheta^d_{r}}{r^d} \\ &  =  r \btheta^d_{r} + \sum_{i=1}^kr (\td-1) \td^{di+ i -1} \btheta^d_{r} \\ & =   r \btheta^d_{r} \left(1 + \sum_{i=1}^k (\td-1) \td^{di+ i -1} \right)\\
\end{align}

Taking $k=d$ results in $r_k=(\frac{d}{d+1})^d {r^*}\approx\frac{1}{e}{r^*}$. 
To use the sample set analysis, we need a large enough $r$ value, so assuming the original $r$ is large enough, we can deduce safely that $\frac{r}{e}$ is also large enough. 
Notice also that $\td - 1 = \frac{-1}{d+1}$, and thus $(d+1)\td=d$, so returning to our expression, and substituting $\xi:=\td^d=\left(\frac{d}{d+1}\right)^d$, we obtain 
\begin{align*}
    \gamma&={r^*} \btheta^d_{r^*} \left(1 - \frac{\td^d\left(\td^{d(d+1)} - 1\right)}{(d+1)(\td^{d+1} - 1)}\right)\\
    &= {r^*} \btheta^d_{r^*} \left(1 - \frac{\xi\left(\xi^{d+1} - 1\right)}{(d+1)(\td \xi - 1)}\right) 
    \\&= {r^*} \btheta^d_{r^*} \left(1 - \frac{\xi^{d+2} - \xi}{ d\xi - (d+1)}\right)
    :={r^*} \btheta^d_{r^*}\zeta.
\end{align*}

We finish this section with a plot of the value $\gamma$ in Figure~\ref{fig:annuli_bound:app}.

\begin{figure}[thb]
\centering  
\includegraphics[width=0.9\columnwidth]{Images/annuli_bound.pdf}
\caption{Plot of the improvement factor $\gamma$.}
\label{fig:annuli_bound:app}
\end{figure}

\subsection{Additional experimental results}
Additional scenarios, which were omitted from the main paper, are given in Figure~\ref{fig:scenarios:app}. Extended results comparing lattice-based samples using the \loc algorithm are provided in Table~\ref{tbl:lattice_comparison:app}.

\begin{figure*}[tbh]
  \centering
%     \hspace*{-0.66cm}
% \subfloat[Zigzag-bypass (long)]{\includegraphics[width=2.18\columnwidth,clip]{Images/Scenarios/ZZB3H_scenario.png}
%     %\label{fig:3d_lattices:da}
%     }
%     \newline
\subfloat[Zigzag-bypass (short)]{\includegraphics[width=1.15\columnwidth,clip]{Images/Scenarios/ZZB2H_scenario.png}
    %\label{fig:3d_lattices:da}
    }
\subfloat[Narrow (more scenarios)]{\includegraphics[width=0.465\columnwidth,clip]{Images/Scenarios/N1_scenarios.png}
    %\label{fig:3d_lattices:da}
    }
  \caption{Additional scenarios used in the experiments. The scenario ZZB3, which is not illustrated here, is similar to ZZB2, only that the horizontal hallways are twice as long.}
  \label{fig:scenarios:app}
\end{figure*}

\begin{table}[tbh]
\caption{Extended comparison of running time and solution length using lattices-based sample sets (where the underlying lattice is denoted in the table) in the iA*-\loc algorithm. Solution length is normalized with respect to the length obtained using $\XA$. }
\centering
\label{tbl:lattice_comparison:app}
\begin{tabular}{|c||ccc|cc|}
\hline
 & \multicolumn{3}{c|}{\cellcolor[HTML]{EFEFEF} Time (s)} & \multicolumn{2}{c|}{\cellcolor[HTML]{EFEFEF} Length (r)} \\ \cline{2-6} 
\multirow{-2}{*}{\begin{tabular}[c]{@{}c@{}}Scenario\\ (robot \#)\end{tabular}} & \multicolumn{1}{c|}{\cellcolor[HTML]{FFFFC7}$\ZN$} & \multicolumn{1}{c|}{\cellcolor[HTML]{FFFFC7}$\DN$} & \cellcolor[HTML]{FFFFC7}$\AN$ & \multicolumn{1}{c|}{\cellcolor[HTML]{FFFFC7}$\ZN$} & \cellcolor[HTML]{FFFFC7}$\DN$ \\ \hline \hline
\cellcolor[HTML]{ECF4FF}N4(2) & \multicolumn{1}{c|}{0.00} & \multicolumn{1}{c|}{0.00} & 0.00 & \multicolumn{1}{c|}{0.62} & 0.74 \\
\cellcolor[HTML]{ECF4FF}N1(5) & \multicolumn{1}{c|}{165.35} & \multicolumn{1}{c|}{4.59} & 0.36 & \multicolumn{1}{c|}{0.65} & 0.79 \\
\cellcolor[HTML]{ECF4FF}N2(5) & \multicolumn{1}{c|}{62.68} & \multicolumn{1}{c|}{1.81} & 0.41 & \multicolumn{1}{c|}{0.85} & 0.95 \\
\cellcolor[HTML]{ECF4FF}N3(5) & \multicolumn{1}{c|}{142.27} & \multicolumn{1}{c|}{2.91} & 0.59 & \multicolumn{1}{c|}{0.65} & 0.87 \\
\cellcolor[HTML]{ECF4FF}N5(5) & \multicolumn{1}{c|}{dnf} & \multicolumn{1}{c|}{4.82} & 3.32 & \multicolumn{1}{c|}{dnf} & 0.82 \\
\cellcolor[HTML]{ECF4FF}N1B(6) & \multicolumn{1}{c|}{dnf} & \multicolumn{1}{c|}{328.30} & 15.08 & \multicolumn{1}{c|}{dnf} & 0.89 \\ \hline
\cellcolor[HTML]{ECF4FF}BT4(2) & \multicolumn{1}{c|}{0.04} & \multicolumn{1}{c|}{0.01} & 0.01 & \multicolumn{1}{c|}{0.69} & 0.85 \\
\cellcolor[HTML]{ECF4FF}BT10(2) & \multicolumn{1}{c|}{-} & \multicolumn{1}{c|}{1.20} & 0.30 & \multicolumn{1}{c|}{-} & 0.92 \\
\cellcolor[HTML]{ECF4FF}BT5(3) & \multicolumn{1}{c|}{0.54} & \multicolumn{1}{c|}{0.14} & 0.06 & \multicolumn{1}{c|}{0.38} & 0.51 \\
\cellcolor[HTML]{ECF4FF}BT1(4) & \multicolumn{1}{c|}{146.69} & \multicolumn{1}{c|}{50.81} & 3.51 & \multicolumn{1}{c|}{0.95} & 1.03 \\
\cellcolor[HTML]{ECF4FF}BT6(4) & \multicolumn{1}{c|}{dnf} & \multicolumn{1}{c|}{153.40} & 12.36 & \multicolumn{1}{c|}{dnf} & 1.04 \\
\cellcolor[HTML]{ECF4FF}BT7(4) & \multicolumn{1}{c|}{240.88} & \multicolumn{1}{c|}{5.38} & 4.36 & \multicolumn{1}{c|}{0.95} & 0.96 \\ \hline
\cellcolor[HTML]{ECF4FF}K1(3) & \multicolumn{1}{c|}{32.31} & \multicolumn{1}{c|}{4.97} & 1.37 & \multicolumn{1}{c|}{0.82} & 0.89 \\ \hline
\cellcolor[HTML]{ECF4FF}UM4(2) & \multicolumn{1}{c|}{-} & \multicolumn{1}{c|}{8.47} & 2.43 & \multicolumn{1}{c|}{-} & 0.90 \\
\cellcolor[HTML]{ECF4FF}UM1(3) & \multicolumn{1}{c|}{482.17} & \multicolumn{1}{c|}{25.15} & 6.68 & \multicolumn{1}{c|}{0.84} & 1.16 \\
\cellcolor[HTML]{ECF4FF}UM2(3) & \multicolumn{1}{c|}{13.35} & \multicolumn{1}{c|}{1.22} & 0.04 & \multicolumn{1}{c|}{1.04} & 1.52 \\
\cellcolor[HTML]{ECF4FF}UM4B3(3) & \multicolumn{1}{c|}{99.35} & \multicolumn{1}{c|}{1.03} & 0.66 & \multicolumn{1}{c|}{1.59} & 0.89 \\
\cellcolor[HTML]{ECF4FF}UM3(4) & \multicolumn{1}{c|}{236.31} & \multicolumn{1}{c|}{223.87} & 64.57 & \multicolumn{1}{c|}{0.63} & 0.97 \\ \hline
\cellcolor[HTML]{ECF4FF}ZZB1(2) & \multicolumn{1}{c|}{1.93} & \multicolumn{1}{c|}{1.01} & 0.44 & \multicolumn{1}{c|}{0.94} & 0.94 \\
\cellcolor[HTML]{ECF4FF}ZZB2(2) & \multicolumn{1}{c|}{2.91} & \multicolumn{1}{c|}{0.93} & 0.71 & \multicolumn{1}{c|}{0.94} & 0.94 \\
\cellcolor[HTML]{ECF4FF}ZZB3(2) & \multicolumn{1}{c|}{2.26} & \multicolumn{1}{c|}{0.84} & 0.47 & \multicolumn{1}{c|}{0.95} & 0.95 \\ \hline\end{tabular}
\end{table}

\begin{table*}[tbh]
\centering
\begin{tabular}{|c|cccl|ccl|cl|cl|}
\hline
 & \multicolumn{4}{c|}{\cellcolor[HTML]{EFEFEF} Total time (s)} & \multicolumn{3}{c|}{\cellcolor[HTML]{EFEFEF} Search time (s)} & \multicolumn{2}{c|}{\cellcolor[HTML]{EFEFEF}Length (r)} & \multicolumn{2}{c|}{\cellcolor[HTML]{EFEFEF}Success (\%)} \\ \cline{2-12} 
\multirow{-2}{*}{\begin{tabular}[c]{@{}c@{}}Scenario\\ (Robot \#)\end{tabular}} & \multicolumn{1}{c|}{\cellcolor[HTML]{FFFFC7}\begin{tabular}[c]{@{}c@{}}$\AN$\\ \loc\end{tabular}} & \multicolumn{1}{c|}{\cellcolor[HTML]{FFFFC7}\begin{tabular}[c]{@{}c@{}}$\AN$\\ \glo\end{tabular}} & \multicolumn{1}{c|}{\cellcolor[HTML]{FFFFC7}\begin{tabular}[c]{@{}c@{}}\rnd\\ \glo\end{tabular}} & \multicolumn{1}{c|}{\cellcolor[HTML]{FFFFC7}\begin{tabular}[c]{@{}c@{}}\rndm\\ \glo\end{tabular}} & \multicolumn{1}{c|}{\cellcolor[HTML]{FFFFC7}\begin{tabular}[c]{@{}c@{}}$\AN$\\ \glo\end{tabular}} & \multicolumn{1}{c|}{\cellcolor[HTML]{FFFFC7}\begin{tabular}[c]{@{}c@{}}\rnd\\ \glo\end{tabular}} & \multicolumn{1}{c|}{\cellcolor[HTML]{FFFFC7}\begin{tabular}[c]{@{}c@{}}\rndm\\ \glo\end{tabular}} & \multicolumn{1}{c|}{\cellcolor[HTML]{FFFFC7}\begin{tabular}[c]{@{}c@{}}\rnd\\ \glo\end{tabular}} & \multicolumn{1}{c|}{\cellcolor[HTML]{FFFFC7}\begin{tabular}[c]{@{}c@{}}\rndm\\ \glo\end{tabular}} & \multicolumn{1}{c|}{\cellcolor[HTML]{FFFFC7}\begin{tabular}[c]{@{}c@{}}\rnd\\ \glo\end{tabular}} & \multicolumn{1}{c|}{\cellcolor[HTML]{FFFFC7}\begin{tabular}[c]{@{}c@{}}\rndm\\ \glo\end{tabular}} \\ \hline
\cellcolor[HTML]{ECF4FF}N1(5) & \multicolumn{1}{c|}{0.36} & \multicolumn{1}{c|}{3.05} & \multicolumn{1}{c|}{4.16} & 3.40 & \multicolumn{1}{c|}{0.84} & \multicolumn{1}{c|}{3.37} & 2.59 & \multicolumn{1}{c|}{1.48} & 1.46 & \multicolumn{1}{c|}{80.00} & 90 \\
\cellcolor[HTML]{ECF4FF}N2(5) & \multicolumn{1}{c|}{0.41} & \multicolumn{1}{c|}{2.67} & \multicolumn{1}{c|}{2.74} & 4.28 & \multicolumn{1}{c|}{0.82} & \multicolumn{1}{c|}{2.11} & 3.62 & \multicolumn{1}{c|}{2.43} & 3.31 & \multicolumn{1}{c|}{65.00} & 95 \\
\cellcolor[HTML]{ECF4FF}N3(5) & \multicolumn{1}{c|}{0.59} & \multicolumn{1}{c|}{3.83} & \multicolumn{1}{c|}{5.44} & 4.22 & \multicolumn{1}{c|}{1.72} & \multicolumn{1}{c|}{4.65} & 3.39 & \multicolumn{1}{c|}{2.02} & 1.56 & \multicolumn{1}{c|}{85.00} & 85 \\
\cellcolor[HTML]{ECF4FF}N5(5) & \multicolumn{1}{c|}{3.32} & \multicolumn{1}{c|}{31.48} & \multicolumn{1}{c|}{23.42} & 26.19 & \multicolumn{1}{c|}{20.02} & \multicolumn{1}{c|}{18.62} & 21.14 & \multicolumn{1}{c|}{0.89} & 0.88 & \multicolumn{1}{c|}{100.00} & 100 \\ \hline
\cellcolor[HTML]{ECF4FF}BT9(2) & \multicolumn{1}{c|}{0.13} & \multicolumn{1}{c|}{0.13} & \multicolumn{1}{c|}{0.77} & 0.42 & \multicolumn{1}{c|}{0.13} & \multicolumn{1}{c|}{0.77} & 0.42 & \multicolumn{1}{c|}{1.10} & 1.41 & \multicolumn{1}{c|}{95.00} & 40 \\
\cellcolor[HTML]{ECF4FF}BT10(2) & \multicolumn{1}{c|}{0.30} & \multicolumn{1}{c|}{0.31} & \multicolumn{1}{c|}{1.16} & 0.46 & \multicolumn{1}{c|}{0.31} & \multicolumn{1}{c|}{1.16} & 0.46 & \multicolumn{1}{c|}{1.13} & 1.27 & \multicolumn{1}{c|}{95.00} & 75 \\
\cellcolor[HTML]{ECF4FF}BT1B(3) & \multicolumn{1}{c|}{34.83} & \multicolumn{1}{c|}{47.58} & \multicolumn{1}{c|}{118.27} & 62.88 & \multicolumn{1}{c|}{47.27} & \multicolumn{1}{c|}{118.11} & 62.70 & \multicolumn{1}{c|}{0.93} & 0.99 & \multicolumn{1}{c|}{100.00} & 100 \\
\cellcolor[HTML]{ECF4FF}BT2(3) & \multicolumn{1}{c|}{5.62} & \multicolumn{1}{c|}{7.08} & \multicolumn{1}{c|}{22.67} & 28.17 & \multicolumn{1}{c|}{6.97} & \multicolumn{1}{c|}{22.61} & 28.11 & \multicolumn{1}{c|}{0.93} & 1.12 & \multicolumn{1}{c|}{100.00} & 95 \\
\cellcolor[HTML]{ECF4FF}BT2B(3) & \multicolumn{1}{c|}{9.58} & \multicolumn{1}{c|}{14.40} & \multicolumn{1}{c|}{41.67} & 21.23 & \multicolumn{1}{c|}{14.13} & \multicolumn{1}{c|}{4.36} & 21.09 & \multicolumn{1}{c|}{1.00} & 1.05 & \multicolumn{1}{c|}{100.00} & 95 \\
\cellcolor[HTML]{ECF4FF}BT3(3) & \multicolumn{1}{c|}{5.38} & \multicolumn{1}{c|}{14.15} & \multicolumn{1}{c|}{62.22} & 32.10 & \multicolumn{1}{c|}{12.80} & \multicolumn{1}{c|}{61.54} & 31.39 & \multicolumn{1}{c|}{1.05} & 1.11 & \multicolumn{1}{c|}{100.00} & 100 \\
\cellcolor[HTML]{ECF4FF}BT5(3) & \multicolumn{1}{c|}{0.06} & \multicolumn{1}{c|}{0.38} & \multicolumn{1}{c|}{0.27} & 0.19 & \multicolumn{1}{c|}{0.12} & \multicolumn{1}{c|}{0.14} & 0.05 & \multicolumn{1}{c|}{0.57} & 0.57 & \multicolumn{1}{c|}{100.00} & 85 \\
\cellcolor[HTML]{ECF4FF}BT8(3) & \multicolumn{1}{c|}{12.17} & \multicolumn{1}{c|}{19.31} & \multicolumn{1}{c|}{169.32} & 79.52 & \multicolumn{1}{c|}{18.87} & \multicolumn{1}{c|}{169.12} & 79.31 & \multicolumn{1}{c|}{1.00} & 1.05 & \multicolumn{1}{c|}{100.00} & 100 \\
\cellcolor[HTML]{ECF4FF}BT8B(3) & \multicolumn{1}{c|}{3.17} & \multicolumn{1}{c|}{3.60} & \multicolumn{1}{c|}{41.63} & 24.23 & \multicolumn{1}{c|}{3.55} & \multicolumn{1}{c|}{41.60} & 24.20 & \multicolumn{1}{c|}{1.16} & 1.20 & \multicolumn{1}{c|}{100.00} & 100 \\
\cellcolor[HTML]{ECF4FF}BT11(3) & \multicolumn{1}{c|}{17.21} & \multicolumn{1}{c|}{35.21} & \multicolumn{1}{c|}{51.19} & 31.23 & \multicolumn{1}{c|}{34.34} & \multicolumn{1}{c|}{50.77} & 30.80 & \multicolumn{1}{c|}{0.88} & 0.95 & \multicolumn{1}{c|}{100.00} & 100 \\
\cellcolor[HTML]{ECF4FF}BT1(4) & \multicolumn{1}{c|}{3.51} & \multicolumn{1}{c|}{97.33} & \multicolumn{1}{c|}{63.69} & 68.39 & \multicolumn{1}{c|}{13.88} & \multicolumn{1}{c|}{18.83} & 22.18 & \multicolumn{1}{c|}{1.02} & 1.04 & \multicolumn{1}{c|}{100.00} & 100 \\
\cellcolor[HTML]{ECF4FF}BT6(4) & \multicolumn{1}{c|}{12.36} & \multicolumn{1}{c|}{124.16} & \multicolumn{1}{c|}{106.73} & 95.75 & \multicolumn{1}{c|}{43.87} & \multicolumn{1}{c|}{61.96} & 50.18 & \multicolumn{1}{c|}{1.01} & 1.03 & \multicolumn{1}{c|}{100.00} & 100 \\
\cellcolor[HTML]{ECF4FF}BT7(4) & \multicolumn{1}{c|}{4.36} & \multicolumn{1}{c|}{95.89} & \multicolumn{1}{c|}{60.65} & 56.06 & \multicolumn{1}{c|}{15.06} & \multicolumn{1}{c|}{15.24} & 9.29 & \multicolumn{1}{c|}{1.00} & 1.01 & \multicolumn{1}{c|}{100.00} & 100 \\ \hline
\cellcolor[HTML]{ECF4FF}UM4(2) & \multicolumn{1}{c|}{2.43} & \multicolumn{1}{c|}{2.93} & \multicolumn{1}{c|}{12.71} & 2.26 & \multicolumn{1}{c|}{2.90} & \multicolumn{1}{c|}{12.69} & 2.24 & \multicolumn{1}{c|}{0.96} & 1.41 & \multicolumn{1}{c|}{70.00} & 5 \\
\cellcolor[HTML]{ECF4FF}UM4B1(2) & \multicolumn{1}{c|}{4.81} & \multicolumn{1}{c|}{5.68} & \multicolumn{1}{c|}{17.38} & 5.03 & \multicolumn{1}{c|}{5.64} & \multicolumn{1}{c|}{17.35} & 5.01 & \multicolumn{1}{c|}{0.86} & 1.12 & \multicolumn{1}{c|}{90.00} & 45 \\
\cellcolor[HTML]{ECF4FF}UM1(3) & \multicolumn{1}{c|}{6.68} & \multicolumn{1}{c|}{58.62} & \multicolumn{1}{c|}{49.35} & 31.92 & \multicolumn{1}{c|}{47.14} & \multicolumn{1}{c|}{42.58} & 24.86 & \multicolumn{1}{c|}{0.98} & 1.09 & \multicolumn{1}{c|}{100.00} & 100 \\
\cellcolor[HTML]{ECF4FF}UM2(3) & \multicolumn{1}{c|}{0.04} & \multicolumn{1}{c|}{2.94} & \multicolumn{1}{c|}{4.49} & 3.44 & \multicolumn{1}{c|}{0.21} & \multicolumn{1}{c|}{2.97} & 1.83 & \multicolumn{1}{c|}{1.95} & 2.23 & \multicolumn{1}{c|}{75.00} & 40 \\
\cellcolor[HTML]{ECF4FF}UM5(3) & \multicolumn{1}{c|}{2.87} & \multicolumn{1}{c|}{31.79} & \multicolumn{1}{c|}{29.71} & 21.31 & \multicolumn{1}{c|}{19.01} & \multicolumn{1}{c|}{22.18} & 13.63 & \multicolumn{1}{c|}{1.05} & 1.26 & \multicolumn{1}{c|}{95.00} & 85 \\ \hline
\cellcolor[HTML]{ECF4FF}ZZB1(2) & \multicolumn{1}{c|}{0.44} & \multicolumn{1}{c|}{0.49} & \multicolumn{1}{c|}{10.44} & 0.47 & \multicolumn{1}{c|}{0.48} & \multicolumn{1}{c|}{10.43} & 0.45 & \multicolumn{1}{c|}{0.89} & 1.01 & \multicolumn{1}{c|}{100.00} & 5 \\
\cellcolor[HTML]{ECF4FF}ZZB2(2) & \multicolumn{1}{c|}{0.71} & \multicolumn{1}{c|}{2.51} & \multicolumn{1}{c|}{272.70} & 2.66 & \multicolumn{1}{c|}{1.22} & \multicolumn{1}{c|}{271.96} & 1.83 & \multicolumn{1}{c|}{0.89} & 2.62 & \multicolumn{1}{c|}{100.00} & 65 \\
\cellcolor[HTML]{ECF4FF}ZZB3(2) & \multicolumn{1}{c|}{0.47} & \multicolumn{1}{c|}{7.69} & \multicolumn{1}{c|}{341.84} & 5.33 & \multicolumn{1}{c|}{1.18} & \multicolumn{1}{c|}{338.15} & 1.45 & \multicolumn{1}{c|}{0.88} & 2.58 & \multicolumn{1}{c|}{100.00} & 65 \\ \hline
\end{tabular}
\caption{Comparison of running time and solution length between $\XA$ (using \loc and \glo) and uniform random sampling. For random sampling we report the average values in terms of running and solution length (the latter is given as normalized value with  respect to the solution length with $\XA$). }
\label{tbl:lattice_vs_random:app}
\end{table*}

\begin{figure*}[th]
  \centering
%     \hspace*{-0.66cm}
% \subfloat[Zigzag-bypass (long)]{\includegraphics[width=2.18\columnwidth,clip]{Images/Scenarios/ZZB3H_scenario.png}
%     %\label{fig:3d_lattices:da}
%     }
%     \newline
\subfloat{\includegraphics[width=\columnwidth,clip]{Images/tuning1.pdf}
    }
    \subfloat{\includegraphics[width=\columnwidth,clip]{Images/tuning2.pdf}
    }
    \newline
\subfloat{\includegraphics[width=\columnwidth,clip]{Images/tuning3.pdf}
    }
\subfloat{\includegraphics[width=\columnwidth,clip]{Images/tuning4.pdf}
    }
      \caption{Effect of the parameters $\delta,\epsilon$ on the performance of \loc with $\XA$ for $\delta=2.5$ (left) and $\delta=4$ (right). We report the running time (top) and solution length (bottom). The absence of data points for the parameters $\delta=4, \eps\in \{2,4,5\}$ indicates a solution failure.  
  }
  \label{fig:parameters:app}
\end{figure*}

\subsection{Comparison with Random Sampling}
Extended results where $\XA$-samples are compared with \rnd are given in Table~\ref{tbl:lattice_vs_random:app}. Here, we consider two versions of random sampling. The first version, denoted by \rnd, which is identical to the one considered in the main paper, uses random sampling together with the asymptotically optimal connection radius $r_{\textup{rnd}}(n)$, which is commonly used in practice. The second version, denoted by \rndm uses the radius as ${r^*}$ used for lattice-based sampling. The latter is used to further emphasize the inferiority of uniform random sampling as compared to $\XA$ due to identical parameters between $\XA$-\glo and \rndm-\glo (except for the sampling distribution). In particular, the move to \rndm  severely reduces the success rates in some of the scenarios.

Another addition in Table~\ref{tbl:lattice_vs_random:app} is the running time of the search algorithm (under "search time"). Recall that the total running time for \glo consists of the (i) construction of the sample set and the nearest-neighbor data structure and the (ii) running the search algorithm. Although both $\XA$-\glo and \rnd use the same number of samples, the construction time is usually larger in the former due to an additional step of constructing the lattice samples over the entire configuration space, which is currently implemented in a naive and unoptimized manner. In this sense, the comparison between $\XA$-\glo and \rnd is not entirely fair. Thus, we also report the running time of the search algorithm, which can be the computational bottleneck, especially for more complicated robot geometries where the collision-check operation is more expensive~\cite{KleinbortSH16}. Although the search time for $\XA$-\glo is usually lower for most scenarios, we argue that with more expensive collision checks, the advantage of lattice-based sample sets would be even more prominent.

\subsection{Effect of parameter choice}
We report the effect of the choice of the $\delta$ and $\eps$ parameters on solution length and running time for the \loc algorithm using $\XA$ sampling. We specifically focus on the ZZB3 scenario due to the availability of several homotopy classes for the solution, where each class has a different length and level of difficulty. For instance, in one class, the robots use the rightmost part of the workspace, which consists of a winding path, and exchange positions halfway between---leading to a relatively short solution length. In a second class, the robots use the long passage to the left, which consists of long straight-line motions and yields a significantly longer solution length.


We set $\delta\in \{2.75,4\}$ and report the solution length and running time in Figure~\ref{fig:parameters:app} for $\eps\in \{0.5,0.6,\ldots,1,2,\ldots,10\}$. Observe that for $\delta=2.75$ the planner obtains a low-length solution already for high $\eps$ values, whereas $\delta=4$ initially uncovers an inefficient solution length-wise but eventually settles on the better homotopy class when $\eps$ is reduced. From values of $\eps\leq 1$ the length relatively stabilizes, while the runtime jumps at several orders of magnitude, which highlights the exponential dependence of sample and collision-check complexity on the value $\eps$. Finding the middle-ground $\eps$ value is an important goal, which we leave for future work. 

Notice that the planner fails to find a solution for $\delta=4$ and $\eps\in\{2,4,5\}$. Due to our \decomps result, this implies that no $4$-clear solution exists. Despite this, the planner does succeed for some values of $\eps$, which suggests that our sufficient conditions for \decomps are not necessary. The success could also be explained by the specific arrangement of the points in $\XA$, which coincidentally induces a connected component via the second homotopy class for this specific scenario. It should also be noted that the sample set $\X_{\AN}^{4,\eps}$ can be viewed (via Lemma~1) as the sample set $\X_{\AN}^{2.5,\eps'}$ for $\eps$ small enough, which explains the success of the planner with  $\delta=4$ and smaller $\eps$ values. 

% Please add the following required packages to your document preamble:
% \usepackage{multirow}
% \usepackage[table,xcdraw]{xcolor}
% Beamer presentation requires \usepackage{colortbl} instead of \usepackage[table,xcdraw]{xcolor}




\end{document}

