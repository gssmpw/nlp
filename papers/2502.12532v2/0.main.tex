% This must be in the first 5 lines to tell arXiv to use pdfLaTeX, which is strongly recommended.
\pdfoutput=1
% In particular, the hyperref package requires pdfLaTeX in order to break URLs across lines.

\documentclass[11pt]{article}

% Change "review" to "final" to generate the final (sometimes called camera-ready) version.
% Change to "preprint" to generate a non-anonymous version with page numbers.
\usepackage[preprint]{acl}

% Standard package includes
\usepackage{times}
\usepackage{latexsym}
\usepackage{amssymb}
% For proper rendering and hyphenation of words containing Latin characters (including in bib files)
\usepackage[T1]{fontenc}
% For Vietnamese characters
% \usepackage[T5]{fontenc}
% See https://www.latex-project.org/help/documentation/encguide.pdf for other character sets

% This assumes your files are encoded as UTF8
\usepackage[utf8]{inputenc}

% This is not strictly necessary, and may be commented out,
% but it will improve the layout of the manuscript,
% and will typically save some space.
\usepackage{microtype}

% This is also not strictly necessary, and may be commented out.
% However, it will improve the aesthetics of text in
% the typewriter font.
\usepackage{inconsolata}

% If the title and author information does not fit in the area allocated, uncomment the following
%
%\setlength\titlebox{<dim>}
%
% and set <dim> to something 5cm or larger.

\usepackage{graphicx} %
% \urlstyle{rm} %
% \def\UrlFont{\rm}  %
\usepackage{natbib}  %
\usepackage{caption} %
\usepackage{algorithm}
\usepackage{listings}
\usepackage{algorithmic}
\usepackage{amsmath}
\usepackage{booktabs}
\usepackage{multirow}
\usepackage[outdir=./]{epstopdf}
\usepackage{enumitem}
\usepackage{caption}
\usepackage{subcaption}
\usepackage{subfloat}
\usepackage{newfloat}
\usepackage{graphicx}
% \usepackage[countmax]{subfloat}
% \usepackage{svg} % svg
\usepackage[normalem]{ulem}
\usepackage{framed}
\usepackage{mdframed}
\usepackage{xcolor}
\usepackage{lipsum}
\usepackage{float}
\usepackage{hyperref}

\definecolor{shadecolor}{gray}{0.9}
\definecolor{LightBlue}{rgb}{0.68, 0.85, 0.90}
\definecolor{LightYellow}{rgb}{1, 1, 0.71}
\definecolor{LightPurple}{rgb}{0.8, 0.6, 1}

\DeclareMathOperator*{\argmax}{arg\,max}
\newlist{todolist}{itemize}{2}
\setlist[todolist]{label=$\square$}
\usepackage{pifont}
\newcommand{\cmark}{\ding{51}}%
\newcommand{\xmark}{\ding{55}}%
\newcommand{\done}{\rlap{$\square$}{\raisebox{2pt}{\large\hspace{1pt}\cmark}}\hspace{-2.5pt}}
\newcommand{\wontfix}{\rlap{$\square$}{\large\hspace{1pt}\xmark}}

\newcommand{\todo}[1]{{\color{green} \textbf{ZQ}. #1}}
\newcommand{\yong}[1]{{\color{red} \textbf{Yong}: #1}}
\newcommand{\gao}[1]{{\color{pink} \textbf{Gao}: #1}}
\newcommand{\chen}[1]{{\color{red} \textbf{Chen}: #1}}
\newcommand{\chenf}[1]{\footnote{+chen+:#1}}
\newcommand{\yongrevised}[1]{{\color{pink} \textbf{Yong}: #1}}
\newcommand{\req}[1]{\textcolor{black}{#1}}
\newcommand{\rev}[1]{#1}
\newcommand{\yantex}[1]{{\color{black} \textbf{Yan}: #1}}
\newcommand{\yanf}[1]{\footnote{+yanf+:#1}}
\newcommand{\reviewf}[1]{\footnote{\color{blue}+rev+:#1}}


\newtheorem{mdefinition}{Definition}

\newtheorem{problem}{Problem}
\newcommand{\para}[1]{{\vspace{4pt} \bf \noindent #1 \hspace{0pt}}}

\setlength{\belowcaptionskip}{-0.1cm} 

\usepackage{array}
\newcolumntype{L}[1]{>{\raggedright\let\newline\\\arraybackslash\hspace{0pt}}m{#1}}
\newcolumntype{C}[1]{>{\centering\let\newline  \\\arraybackslash\hspace{0pt}}m{#1}}
\newcolumntype{R}[1]{>{\raggedleft\let\newline \\\arraybackslash\hspace{0pt}}m{#1}}


\AtBeginDocument{%
  \providecommand\BibTeX{{%
    \normalfont B\kern-0.5em{\scshape i\kern-0.25em b}\kern-0.8em\TeX}}
}


% \copyrightyear{2023}
% \acmYear{2023}
% \setcopyright{acmlicensed}
% \acmConference[WWW '24] {The Web Conference}{May 13--17, 2024}{Singapore}
% \acmBooktitle{Proceedings of The Web Conference 2024 (WWW '24), May 13--17, 2024, Singapore}
% \acmPrice{15.00}
% \acmISBN{979-8-4007-0103-0/23/08}
% \acmDOI{10.1145/XXXXXX.XXXXXX}
% \acmDOI{10.1145/3580305.3599322}

% \settopmatter{printacmref=true}

% \linespread{0.97}


% \copyrightyear{2025}
% \acmYear{2025}
% \setcopyright{acmcopyright}\acmConference[ALC'25]{The 63rd Annual Meeting of the Association for Computational Linguistics}{July 27th to August 1st, 2025}{Vienna, Austria}
% \acmBooktitle{The 63rd Annual Meeting of the Association for Computational Linguistics, Vienna, Austria}

% \author{Anonymous ACL submission}

% \author{Yong Zhao, Kai Xu, Zhengqiu Zhu, Yue Hu, Zhiheng Zheng, Yingfeng Chen, Yatai Ji, Chen Gao, Yong Li, Jincai Huang\\
% National University of Defense Technology, Tsinghua University\\
% \texttt{linian21@mails.tsinghua.edu.cn, \{chgao96, liyong07, liaoqm\}@tsinghua.edu.cn}}


\author{
 \textbf{Yong Zhao\textsuperscript{*1}},
 \textbf{Kai Xu\textsuperscript{*1}},
 \textbf{Zhengqiu Zhu\textsuperscript{1}},
 \textbf{Yue Hu\textsuperscript{1}},
  \\
 \textbf{Zhiheng Zheng\textsuperscript{2}},
 \textbf{Yingfeng Chen\textsuperscript{2}},
 \textbf{Yatai Ji\textsuperscript{1}},
 \textbf{Chen Gao \textsuperscript{2}},
 \textbf{Yong Li\textsuperscript{2}},
 \textbf{Jincai Huang\textsuperscript{1}}
\\
 \textsuperscript{1}National University of Defense Technology,
 \textsuperscript{2}Tsinghua University, \\
 \textsuperscript{*}Equal contribution
 % \textsuperscript{$\dagger$}Corresponding authors
\\
% \texttt{\{zhaoyong15, xukai09, zhuzhengqiu12\}@nudt.edu.cn, chgao96@gmail.com}
% \texttt{zhuzhengqiu12@nudt.edu.cn, chgao96@gmail.com}
}


% \orcid{0000-0003-4689-2289}
% \affiliation{
%   \institution{Shenzhen International Graduate School, Tsinghua University}
%   \city{Shenzhen}
%   \country{China}
% }

% \author{Chen Gao}
% \authornote{Corresponding author (chgao96@gmail.com).}
% \author{Yong Li}
% \affiliation{%
%   \institution{Department of Electronic Engineering, Tsinghua University}
%   \city{Beijing}
%   \country{China}
% }
% \author{Qingmin Liao}
% \affiliation{
%   \institution{Shenzhen International Graduate School, Tsinghua University}
%   \city{Shenzhen}
%   \country{China}
% }
% \renewcommand{\shortauthors}{Wen, et. al.}
%\renewcommand{\shortauthors}{Efficient and Joint Hyperparameter and Architecture Search for Collaborative Filtering}
% \makeatletter
% \def\@copyrightspace{\relax}
% \makeatother
% \settopmatter{printacmref=false}
\title{
% CityEQA: Long-Horizon Embodied (Dynamic) Question Answering in City Space via A Hierarchical Planner-Manager-Actor Agent Framework
% CityEQA: A Hierarchical Planner-Manager-Actor LLM Agent on Long-Horizon Embodied Question Answering Benchmark in City Space
CityEQA: A Hierarchical LLM Agent on Embodied Question Answering Benchmark in City Space
}
% \title{CityEQA: Long-Horizon Embodied Question Answering in City Space via Hierarchical Planner-Manager-Actor Framework}
\begin{document}
\maketitle
\begin{abstract}
% Our approach leverages multimodal representations and spatial reasoning to ground linguistic instructions in real-world cityscapes. We propose a novel framework that encodes urban context, dynamic object interactions, and scene semantics into a unified representation for robust decision-making. Experiments demonstrate that our method outperforms existing state-of-the-art models in XX metrics, paving the way for more effective embodied perception intelligence applications in urban settings. But here is still a significant gap between the baseline model and human performance,
% Embodied Question Answering (EQA) poses a distinctive challenge, combining natural language processing, computer vision, and robotics. It demands that agents proactively explore complex environments to answer open-ended questions. While existing studies have predominantly focused on indoor scenarios, this paper introduces CityEQA, addressing the challenges of answering perception questions in city spaces, which entail significant environmental, action, and perception complexities. We propose a benchmark dataset, namely CityEQA-EC, generated through extensive human annotation, designed to evaluate agent performance in navigating and understanding intricate urban landscapes. Moreover, we also design a novel embodied agent baseline for this task, which enables efficient decision-making and long-horizon planning via a hierarchical Planner-Manager-Actor framework. Experimental results demonstrate the effectiveness of our proposed methods in enhancing the capabilities of agents in CityEQA tasks, paving the way for future advancements in city embodied intelligence. But here is still a significant gap between the baseline model and human performance.
Embodied Question Answering (EQA) has primarily focused on indoor environments, leaving the complexities of urban settings—spanning environment, action, and perception—largely unexplored. To bridge this gap, we introduce \textbf{CityEQA}, a new task where an embodied agent answers open-vocabulary questions through active exploration in dynamic city spaces. To support this task, we present \textbf{CityEQA-EC}, the first benchmark dataset featuring 1,412 human-annotated tasks across six categories, grounded in a realistic 3D urban simulator. Moreover, we propose \textbf{Planner-Manager-Actor (PMA)}, a novel agent tailored for CityEQA. PMA enables long-horizon planning and hierarchical task execution:  the Planner breaks down the question answering into sub-tasks, the Manager maintains an object-centric cognitive map for spatial reasoning during the process control, and the specialized Actors handle navigation, exploration, and collection sub-tasks. Experiments demonstrate that PMA achieves 60.7\% of human-level answering accuracy, significantly outperforming frontier-based baselines. While promising, the performance gap compared to humans highlights the need for enhanced visual reasoning in CityEQA. This work paves the way for future advancements in urban spatial intelligence. Dataset and code are available at \url{https://github.com/tsinghua-fib-lab/CityEQA.git}. 



\end{abstract}

% \keywords{Macroeconomic Simulation; Large Language Models; Agent-based Modeling}

% \maketitle



\section{Introduction}
% Large Language Models~(LLMs) represent a transformative advancement in the field of language processing, demonstrating an unparalleled capacity for text generation and comprehension, which can be further applied in a wide variety of applications.  
% %Large language models (LLMs) have risen to prominence in various fields, offering endless possibilities for artificial intelligence applications. 
% Despite their significant prevalence in recent years, LLMs are frequently challenged with security and privacy issues, such as poor explainability~\cite{}, poor robustness~\cite{}, data dependency~\cite{}, etc. Among them, a specific and notable concern that has garnered increasing attention is the phenomenon of `hallucination', where models generate plausible but factually inaccurate or irrelevant content when employed for specific tasks such as problem-solving.  
% %In particular, the hallucination issue is when these large models are employed for problem-solving, users frequently voice concerns regarding being misled or deceived by the models' nonsensical and erratic outputs. 
% The tendency of these models to produce inaccurate outputs and fabricate facts has severely undermined the safety and usability of LLM applications, which calls for immediate attention in LLM research. 
% %Hallucination in large language models (LLMs) is a critical issue that needs immediate attention in LLM research. The tendency of these models to produce inaccurate outputs and fabricate facts has severely undermined the safety and usability of LLM applications. 
%exceptional 
%including limited explainability, compromised robustness, and a heavy reliance on data, each 
%However, d
Large Language Models (LLMs) have revolutionized language processing, demonstrating impressive text generation and comprehension capabilities with diverse applications. However, despite their growing use, LLMs face significant security and privacy challenges~\cite{siddiq2023generate, hou2023large, kaddour2023challenges, li2024model, 10.1145/3691620.3695510}, which affect their overall effectiveness and reliability. A critical issue is the phenomenon of \emph{hallucination}, where LLMs generate outputs that are coherent but factually incorrect or irrelevant. This tendency to produce misleading information compromises the safety and usability of LLM-based systems. This paper focuses on \emph{fact-conflicting hallucina}tion (FCH), the most prominent form of hallucination in LLMs. FCH occurs when LLMs generate content that directly contradicts established facts. For instance, as illustrated in \figref{fig:example1}, an LLM incorrectly asserts that ``\emph{Haruki Murakami won the Nobel Prize in Literature in 2016}'', whereas the fact is that ``\emph{Haruki Murakami has not won the Nobel Prize, though he has received numerous other literary awards}''. 
Such inaccuracies can significantly lead to user confusion and undermine the trust and reliability that are crucial for LLM applications.

% Large Language Models~(LLMs) have brought transformative advancements to language processing and beyond, showcasing text generation and comprehension abilities with wide-ranging applications. 
% Despite the increasing prevalence, LLMs face critical challenges in security and privacy aspects~\cite{siddiq2023generate, hou2023large, kaddour2023challenges}, heavily impacting their effectiveness and reliability. 
% One notable issue is the phenomenon of \emph{hallucination}, where LLMs produce coherent but factually inaccurate or irrelevant outputs during problem-solving. 
% Such a tendency to generate misleading information jeopardizes the safety and usability of LLM-based applications. 
% This paper concerns the \emph{fact-conflicting hallucination}~(FCH), which is the primary form of hallucinations in LLMs. 
% FCH occurs when LLMs generate content that directly contradicts the well-established facts, as exemplified in \figref{fig:example1}, where an LLM incorrectly believes 
% ``\emph{Haruki Murakami won the Nobel Prize in Literature in 2016}'', deviating from the fact that ``\emph{Haruki Murakami has not won the Nobel Prize but other numerous awards for his work in Literature}''. Such misinformation can cause significant user confusion and undermine the trust and reliability that are essential in various LLM applications. 

%correct answer of 

%is manifested by
%Such misinformation dissemination leads to significant user confusion, eroding the trust and reliability that are crucial in various LLM applications. 

%Large Language Models~(LLMs) represent a transformative advancement in the field of language processing, demonstrating an unparalleled capacity for text generation and comprehension, which can be further applied in a wide variety of applications. Despite their growing prevalence, LLMs encounter critical challenges, particularly in aspects of security and privacy. These include concerns such as limited explainability~\cite{}, compromised robustness~\cite{}, and heavy reliance on data~\cite{}, each posing distinct challenges to their efficacy and reliability. Among these, the phenomenon of ``hallucination'' stands out as a notable concern. This occurs when LLMs, while employed in tasks like problem-solving, generate outputs that are coherent yet factually inaccurate or irrelevant. Such a tendency to produce misleading information not only compromises the safety of LLM applications but also raises urgent questions regarding their usability. 

% Hallucinations in LLMs manifest in several distinct forms, each contributing differently to the challenges identified in their growing applications. 
% %The first, known as ``Input-conflicting hallucination'', arises when there is a discrepancy between the model's output and the user's initial input, reflecting a potential misunderstanding of the task at hand. On the other hand, ``Context-conflicting hallucination'' represents the second type, occurring when LLMs produce inconsistent responses in prolonged or multi-turn interactions, indicative of their limitations in maintaining coherent context. 
% Among the three types categorized in the literature~\cite{huang2023survey,zhang2023hallucination}, ``Fact-conflicting hallucination~(FCH)'' poses a particularly serious concern which is the primary focus of this paper. This phenomenon generates content in direct opposition to established factual knowledge. As illustrated in the example shown in Figure~\ref{fig:example1}, when an LLM was asked about the first person to walk on the moon, it incorrectly answered ``Charles Lindbergh in 1951'', a clear deviation from the factual answer of Neil Armstrong in 1969. This type of hallucination can lead to the dissemination of incorrect information and cause significant confusion among users, undermining the trust and reliability critical in various LLM applications. %Addressing fact-conflicting hallucinations is therefore essential for the advancement of LLMs, ensuring they not only function effectively but also responsibly in their diverse roles.


% According to \cite{huang2023survey} and \cite{zhang2023hallucination}, hallucinations in large language models can be categorized into types such as factual hallucinations and contextual hallucinations. Current benchmark assessments tend to focus on evaluating the propensity of LLMs to generate erroneous facts. The origin of these issues can be traced back to multiple deficiencies, including flaws in training data, training algorithms, and the inference process.

% \begin{figure}[t]
%     \centering
%     \includegraphics[width=0.95\linewidth]{fig/example1-cropped.pdf}\\
%     \caption{A Hallucination Output Example.}
%     %\vspace{-0.5cm}
%     \label{fig:example1}
% \end{figure}

\begin{figure}[t]
\centering
\vspace{3mm}
\hspace{-3mm}
\includegraphics[width=\linewidth]{fig/drowzee-example.pdf}
\\[0.5em]
\caption{A Hallucination Output Example}
\label{fig:example1}
\vspace{-4mm}
\end{figure}
%\lnk{Factual Hallucination and LLM inference current status}

Recent studies have introduced various methods to detect LLM hallucinations. A common approach involves developing specialized benchmarks, such as TruthfulQA~\cite{lin-etal-2022-truthfulqa}, HaluEval~\cite{HaluEval}, and KoLA~\cite{yu2023kola}, to assess hallucinations in tasks like question-answering, summarization, and knowledge graphs. 
While manually labeled datasets provide valuable insights, current methods often rely on simplistic or semi-automated techniques such as string matching, manual validation, or verification through another language model. These approaches reveal significant gaps in automatically and effectively detecting fact-conflicting hallucinations (FCH). 
The primary challenges in FCH detection arise from the lack of dedicated ground truth datasets, the absence of comprehensive test cases designed to trigger FCH, and the lack of a robust testing framework.  
Unlike other types of hallucinations, such as input-conflicting or context-conflicting hallucinations~\cite{ji-etal-2023-rho, shi2023large}, which can often be identified through semantic consistency checks, detecting FCH requires the verification of factual accuracy against external knowledge sources/databases. This process is particularly challenging and resource-intensive, especially for tasks that involve complex logical relationships~\cite{zhang2024fusion}. We identify three primary challenges in addressing this research gap:


% Recent studies have introduced a range of methods for detecting 
% hallucinations. One common approach involves creating comprehensive benchmarks tailored for this purpose. 
% Datasets such as TruthfulQA~\cite{lin-etal-2022-truthfulqa}, HaluEval~\cite{HaluEval}, and KoLA~\cite{yu2023kola} have been designed to evaluate hallucinations across different contexts, including question-answering, summarization, and knowledge graphs. 
% Despite the value of these manually labeled datasets, the current techniques heavily rely on naive and semi-automatic methods, such as string matching, manual validation, or utilizing another LLM for confirmation. 
% Therefore, there is a gap 
% in automatically and effectively testing FCHs, and the primary obstacle in testing FCH is the absence of dedicated ground truth datasets and an extensive testing framework.  
% Unlike other types of hallucinations, e.g., input-conflicting or context-conflicting 
% \cite{ji-etal-2023-rho, shi2023large}, 
% which can be identified through checks for semantic consistency, 
% detecting FCH
% requires the verification of the content's factual accuracy against external sources of knowledge or databases. This makes the process particularly arduous and resource-intensive, especially for tasks processing content with complex logical connections. 
% Here, we highlight three concrete challenges in filling up the identified research gap: 




%The main obstacle in testing for FCH is the absence of dedicated ground truth datasets and specific testing frameworks. Unlike other types of hallucinations~(e.g., input-conflicting and context-conflicting hallucinations, to be detailed in Section~\ref{subsec:cat}) which can be identified through checks for semantic consistency, FCH demands the verification of the content's factual accuracy against external sources of knowledge or databases. This requirement makes the process particularly challenging and resource-intensive, especially for tasks processing contents with inherent logical connections.

% \shil{(I feel the transition is not smooth, we first introducing datasets, and not explaining how they use these datasets to test llm. after these, we can state these methods are not automatic.)}


% To tackle FCH, recent works have developed various techniques for testing and detecting hallucination~\citep{yu2023kola,HaluEval}. The typical and intuitive solution is to develop comprehensive benchmarks for detection. This is done through a process of sampling, filtering, and enhancing ground-truth answers to identify the best and correct answers from given candidates. For example, a well-known hallucination evaluation benchmark HaluEval~\cite{HaluEval} constructs scenarios where LLMs are tested on their ability to select the most factually accurate answers from a set of provided options, with a focus on filtering out hallucinated responses. %\yi{ also talk about the construction of benchmark?}
% Additionally, human annotation plays a critical role in identifying hallucinations in LLM outputs~\cite{Alpaca}. This involves humans determining whether responses contain hallucinated information and considering aspects such as unverifiability, non-factuality, and irrelevance. 



% \lnk{Key challenge: lack of hallucination testing when faced with logic reasoning related problems}
%Bridging the identified research gap in the literature necessitates exploring the inherent challenges faced in detecting FCHs, which are crucial for advancing and enhancing the reliability of LLMs. 

\begin{enumerate}[itemsep=1mm, wide,  labelindent=9pt]
%[itemsep=0ex,leftmargin=0.35cm]
%Challenge\#1: 
%While these benchmarks effectively detect certain hallucinations, they 
\item {\textbf{Automatically constructing and updating benchmark datasets.}} Existing methodologies mainly rely on manually curated benchmarks for detecting specific hallucinations, which fail to encompass the broad and dynamic spectrum of fact-conflicting scenarios in LLMs. 
Meanwhile, due to the ever-evolving nature of knowledge, the need for frequent updates to benchmark data imposes a substantial and continuous maintenance effort.
The reliance on benchmark datasets thus restricts the FCH detection techniques' adaptability, scalability, and  %more importantly, 
detection capability;  
%Challenge\#2:
% in existing test cases. 
\item {\textbf{Efficiently generating FCH test cases.}}
LLMs often answer correctly to simple, straightforward questions due to their extensive training on vast datasets. However, to effectively assess their reasoning capabilities, it is important to generate more complex questions, such as those involving intricate temporal characteristics, that require reasoning rather than just recalling facts. However, constructing such test cases is non-trivial. The challenge lies in designing questions that use familiar knowledge but involve reasoning patterns the LLM may not have been explicitly trained on. Creating such test cases efficiently while ensuring they probe reasoning skills in ways the model has not previously encountered is essential to uncovering latent hallucinations;
% queries that involve temporal concepts, such as ``\emph{Does the human population finally reach six billion by the year 2000?}'' may often be used in applications. However, the correctness of the LLM outputs cannot be guaranteed, potentially leading to misleading information. Currently, there are no satisfactory approaches to automatically verify LLM outputs in such test cases; 
%errors even before the occurrence of large model hallucinations; 
%However, it is known that 
%Another critical issue lies in the verification of temporal logic in existing test cases. 
%It is well known that test cases involving temporal-related questions often face difficulties in automatically verifying the soundness and completeness of these issues. Consequently, the correctness of these test cases cannot be guaranteed, potentially introducing errors even before the occurrence of large model hallucinations;
%Challenge\#3: 
\item {\textbf{Validating the reasoning steps from LLM outputs.}} Even when LLMs finally produce correct answers, the outputs may not indicate an accurate reasoning process, potentially masking false understanding -- a source of FCH. Additionally, the quality of manual validation can differ based on human expertise. As a result, automatically validating reasoning processes, particularly those involving complex logical relationships, is inherently challenging. 
\vspace{1mm}
\end{enumerate}







% \lnk{Key challenge: factual knowledge exploring and new facts generation}
%\yi{we should focus on testing, addressing is a little bit vague.}
% The current research landscape in LLM presents a critical gap in automatically testing FCHs. Predominantly, existing methodologies are anchored to manual benchmarks. %\yi{this sentence is quite chinglish.}
% While these benchmarks are effective in detecting certain types of hallucinations, such as those in Figure~\ref{fig:example1}, they fall short in encompassing the broad and dynamic spectrum of fact-conflicting scenarios inherent to LLMs. %\yi{again, this sentence is not very clear}
% Meanwhile, the need for frequent updates to benchmark data, due to the ever-evolving nature of knowledge, imposes a significant and continuous maintenance effort.
% The reliance on benchmark datasets thus restricts the detection techniques’ adaptability, scalability, and worse, detection capability. 
% From a second perspective, the consistency in the quality of benchmark questions can vary, reflecting the differing levels of experience and skill among the human experts responsible for creating them. This is particularly reflected in the stages such as data labeling and results validation. Additionally, it is important to acknowledge that humans are prone to errors.
% %the scalability and the deof these existing methods are also significantly challenged by their dependency on extensive human intervention, particularly in stages such as data labeling and results validation. %This heavy reliance on manual efforts not only limits the scalability of such approaches but also questions their feasibility in efficiently handling the extensive and intricate datasets characteristic of LLMs.
% Thus, the development of more autonomous, agile, and scalable testing techniques is imperative to effectively identify and tackle FCHs in LLMs.%\yi{in this paper, we focus on testing, but until this paragraph, no terms about ``testing'' explicitly occur.}

% \lnk{Solution to Challenge1: comprehensive logic reasoning based testing framework}

% \lnk{Solution to Challenge2: wiki factual knowledge extraction and prolog rules inference for scalability.}
% \lnk{Key challenge: }

%\textbf{Our Work.}
%To address limitations in the existing techniques, 
%we are the first, to the best of our knowledge, to introduce 
To address the problems outlined above, this paper presents a novel automatic end-to-end metamorphic testing technique based on temporal logic for detecting FCH. To the best of our knowledge, we are the first to create a comprehensive FCH testing framework that utilizes factual knowledge reasoning and metamorphic testing, all seamlessly integrated into the fully automated tool, \tool. 

%\shil{(which four methods?)}
\tool begins by establishing a comprehensive factual knowledge base sourced through crawling information from accessible knowledge bases such as Wikipedia. Each piece of this knowledge acts as a ``seed'' for subsequent transformations. Leveraging logical operators to automatically generate temporal reasoning rules, we transform and augment these seeds and expand factual knowledge into a well-established set of question-answer pairs.
%\yi{into xx}. 
Using the questions and answers in the knowledge set as test cases and ground truth, respectively, we construct a reliable and robust FCH testing benchmark. 


The experiment uses a series of carefully designed template-based prompts to test for FCHs in LLMs. To thoroughly evaluate the reasoning behind the responses, we instruct the LLMs not only to generate answers to the test cases but also to provide detailed justifications for their answers. To reliably identify FCH, we introduce two semantic-aware, similarity-based metamorphic oracles. These oracles extract the key semantic elements from each sentence and map out the logical relationships between them. By comparing the logical and semantic structures of the LLM's responses with the ground truth, the oracles can detect FCH by identifying significant deviations in the LLM's answers from the correct information.




%well-crafted prompts\yi{how prompts generated?} to engage LLMs, testing the alignment of their generated content with our enhanced ground truth. Disparities between LLM outputs and the ground truth signal potential hallucinations. 
%Additionally, in our commitment to fostering collaborative research, we have released our constructed dataset as a benchmark~\cite{drowzee}.

%Our approach directly addresses the need for a comprehensive and flexible testing method by transforming structural factual data into a diverse range of scenarios that LLMs may encounter. This method not only improves the reliability of detection but also enhances its adaptability to various factual contexts.
%Furthermore, we address the scalability challenge by automating the transformation and enlargement of our knowledge base, significantly reducing the dependency on human effort. The well-designed prompts used to test LLMs further streamline the process, making it more efficient in identifying potential hallucinations by comparing LLM outputs with our extended ground truth.

%\textbf{Results and Findings.}
%In evaluating our proposed FCH testing framework and \tool, 
%we undertake 
%to evaluate their effectiveness 
We demonstrate the effectiveness of our approach through comprehensive experiments in multiple contexts. First, our evaluation involves deploying \tool across a wide range of topics drawn from a diverse selection of Wikipedia articles. Second, we test our framework on various open-source and commercial LLMs, thoroughly assessing its applicability and performance across different model architectures. 
Our key findings indicate that \tool succeeds in automatically generating practical test cases and identifying hallucination issues of nine LLMs across nine domains. 
Using these test sets, our experiments show that the rate of hallucination responses produced by various LLMs ranges from 24.7\% to 59.8\% for cases unrelated to temporal reasoning and 16.7\% to 39.2\% for cases requiring temporal reasoning. 
%\shil{shall we differentiate the number for non-temporal and temporal one?}.  
We then categorize these hallucination responses into \emph{erroneous knowledge hallucination} and \emph{erroneous inference hallucination}. 
%\syh{four types?}. 
Through an in-depth analysis, we unveil that the lack of logical reasoning capabilities contributes the most to the FCH issues in LLMs. 
Additionally, we observe that LLMs are particularly prone to generating hallucinations in test cases involving temporal concepts and out-of-distribution knowledge. 
Such an evaluation demonstrates that the 
%Furthermore, we confirm that 
test cases generated using %our 
logical reasoning rules can effectively trigger and detect LLM hallucinations.  %issues in . 


This paper builds upon the earlier version~\cite{DBLP:journals/pacmpl/LiL0SW024} by incorporating hallucination detection through temporal-logic-guided test generation. It includes additional motivational examples (\secref{sec:motivating}), a comprehensive set of reasoning rules for encoding \emph{Metric Temporal Logic} (MTL)~\cite{DBLP:conf/lics/OuaknineW05} formulae (\secref{sec:encoding_MTL}) and automatically generating temporal-logic-related question-answer pairs (\secref{prompt}), and further experimental studies (the {RQ4} at \secref{sec:eval}) that detect hallucinations due to insufficient temporal reasoning capabilities. The main contributions of this work are summarized as follows: 
%We summarize the main contributions of this paper below:
\begin{itemize}[itemsep=1mm,leftmargin=0.35cm]
\item 
%Development of 
\textbf{A novel FCH testing framework.} 
To the best of our knowledge, 
we are the first to develop a novel testing framework based on logic programming and metamorphic testing to automatically detect FCH issues in LLMs. %\yi{hanging sentence}This framework represents a significant advancement over current methodologies, providing a more systematic, comprehensive approach to detection.
%Construction and Release of
\item \textbf{An extensive benchmark based on factual knowledge.} 
To facilitate collaborative efforts and future advances in identifying FCH, 
the source code of \tool and constructed benchmark dataset are publicly available  \cite{drowzee}. 
\item \textbf{Test generation via temporal reasoning.} 
Our tool automatically generates test cases that provide a more comprehensive evaluation of LLMs in handling reasoning tasks and identifying factual inconsistencies. By applying temporal logic-based reasoning rules, we expand the initial seed data from our knowledge base, enhancing the diversity and complexity of the test scenarios. 

\item \textbf{Semantic-aware oracles for LLM answer validation.} We propose and implement two automated verification mechanisms, i.e., the oracles, that analyze the semantic structure similarity between sentences. These oracles are designed to validate the reasoning logic behind the answers generated by LLMs, hereby reliably detecting the occurrence of FCHs. 

\end{itemize}



\begin{figure*}[!htb]
\centering
    \includegraphics[width=\linewidth]{figures/dataset.pdf}
% \vspace{-0.2cm}
\caption{Example questions and dataset statistics of CityEQA-EC.}
% \vspace{-0.2cm}
\label{fig:dataset}
\end{figure*}

\section{CityEQA-EC Dataset}\label{sec::dataset}
\vspace{-0.2cm}
In this section, we outline the formulation of the EQA task and describe the dataset collection process for CityEQA-EC. To address real-world demands, such as urban governance and public services, we draw upon previous research \cite{majumdar2024openeqa, das2018embodied} to define six distinct task types. Examples and statistics of the dataset are presented in Figure \ref{fig:dataset}. 

\vspace{-0.2cm}

\subsection{Task Formulation}

An instance of the EQA task is defined by the 4-tuple: $\xi =(e,q,y,{{p}_{0}})$, where $e$ is the simulated or real 3D scene that agent can interact with, $q$ is the question, and $y$ is the ground truth answer. The ${{p}_{0}}$ denotes the agent’s initial pose, including 3D position and orientation. Given the instance $\xi $, the goal is for the embodied agent (e.g., drones) to complete the task by gathering the required information from $e$ and generating the answer $\hat{y}$ in response to $q$. Specifically, the agent starts at the initial pose ${{p}_{0}}$ and interacts with the scene $e$ step by step. At each time step$t$, the agent can move to a specific pose ${{p}_{t}}$, and obtain an observation ${{o}_{t}}=(I_{t}^{rgb},I_{t}^{d})$ from the scene, where $I_{t}^{rgb}\in {{\mathbb{R}}^{H\times W\times 3}}$ is the RGB image and $I_{t}^{d}\in {{\mathbb{R}}^{H\times W}}$ is the depth image. Based on these observations, the agent generates the answer $\hat{y}$. The key challenge is to produce a high-quality answer while minimizing the time steps required. 

\subsection{Dataset Collection and Validation}

To obtain a high-quality dataset, we employed the EmbodiedCity \cite{gao2024embodiedcity}, which is a highly realistic 3D simulation platform based on the buildings, roads, and other elements in a real city. It is implemented using Unreal Engine 4 \cite{sanders2016introduction} and Microsoft AirSim plugins \cite{shah2018airsim}. The collection process is to determine the 4-tuple elements $\xi =(e,{{p}_{0}},q,y)$ of each instance. Unlike indoor simulators with many different scenes, EmbodiedCity is a coherent and extensive scene. As a result, for all instances, their scene $e$ corresponds to EmbodiedCity.

The dataset collection process involves two steps, completed by five human annotators. The first step is raw Q\&A generation, where raw questions and answers are created. The second step is task supplementation, which includes determining the agent's initial pose and and refining the question descriptions accordingly. Once these steps are completed, the dataset undergoes validation and filtering. More details can be found in Appendix \ref{a_data_collection}.  

\paragraph{Raw Q\&A Generation} 

We instructed human annotators to explore the EmbodiedCity environment freely and generate question-answer pairs based on their observations of RGB images. The raw questions ${q^r}$ and answers $y$ are presented as open-vocabulary text. In addition to documenting the question-answer pairs, annotators were also required to record the pose ${p^{obs}}$ from which the RGB images were captured, along with the pose ${p^{tar}}$ of the target object referenced in each question. These information can be leveraged for a comprehensive evaluation of the agent's performance. After basic revision process, we have finally collected a total of 443 such instances, with each raw task instance denoted as ${\xi ^r} = ({q^r}, y, {p^{obs}}, {p^{tar}})$.

\paragraph{Task Supplementation} 

Building upon the raw task instances, we further established the agent's initial pose and refined the questions accordingly. For each raw task, the initial pose \( p_{0} \) of the agent was set within a 200-meter range of the target object's pose \( p^{tar} \). Given the complexity of urban environments, and to ensure that each expected answer is unique, we enriched the questions with descriptions based on landmarks. An example of this process is illustrated in Figure \ref{fig:example}. For each raw task, we generated at least four distinct initial poses and transformed each raw question into at least four different inquiries. Ultimately, this process yielded a total of 2,212 task instances.


\paragraph{Dataset Validation} 

Each task instance created by human annotators was rigorously evaluated by two independent human reviewers. These reviewers were responsible for determining whether the questions posed were answerable and clear, as well as verifying the uniqueness and accuracy of the target objects and their corresponding answers. Any task instance identified with issues was excluded. The final dataset comprises 1,412 task instances, with detailed statistics presented in Figure \ref{fig:dataset}.

% \todo{The description of the data validation process is vague; it is recommended to include quality evaluation metrics (can be placed in the appendix). 

% The analysis of the dataset itself and the description of the dataset split in AerialVLN are relatively clear and thorough. Do we have something similar here?
% }

% | 评估维度       | 标准                  | 通过率 |  
% |----------------|-----------------------|-------|  
% | 问题可回答性   | 需≤3次环境交互可解    | 92%   |  
% | 答案唯一性     | 双盲验证一致率≥95%    | 97%   |  
% | 环境可达性     | 初始位置到目标≤200m   | 100%  |
\begin{figure}[ht]
\centering
\includegraphics[width=0.48\textwidth]{figs/entropy_curve_v2.png}
\vspace{-6mm}
\caption{\textbf{Entropy Curve Characteristics.} 
\textbf{(a)} The token-level entropy \(\mathcal{H}\) (solid blue line) increases slowly at lower temperatures and then jumps sharply at a critical turning point. In contrast, the entropy for a fixed (greedy) generation stays low (dotted blue line). \(\log(\mathcal{H})\) (red line) reveals a transition from concavity to convexity that aligns with the sharp increase in \(\mathcal{H}\), marking the entropy turning point (EntP). \textbf{(b)} EntP hits the best temperature, and it varies between different models.}
\label{fig: entropy_curve}
\vspace{-5mm}
\end{figure}
\section{Entropy-Based Automatic Temperature Selection}
\label{sec: 4}
Determining an optimal sampling temperature is crucial in multi-sample aggregation strategies, yet existing approaches often rely on labeled data or tuning on a validation set. This reliance becomes problematic when no such data are available. 
In this section, we show how to leverage token-level entropy as an intrinsic property to pinpoint a suitable temperature without labeled data. We first demonstrate a spike on \emph{token-level entropy} as a signal of quality collapse in Section~\ref{sec: spike}. Then develop a method that automatically selects temperature using an \emph{entropy turning point (EntP)} derived from the spike in Section~\ref{sec: turn}. Finally, we applied a stochastic process model to explain the mechanism of our algorithm in Section~\ref{sec: toy model}.
\subsection{Entropy Spike as an Indicator of Quality Collapse}
\label{sec: spike}
First, we discover a surprising phenomenon that we call the entropy spike. Specifically, increasing the temperature smoothly increases the model’s entropy, until a dramatic spike where the entropy rapidly increases. We believe the spike is a good indicator of sample quality collapse.

As illustrated in Figure~\ref{fig: entropy_curve}(a), we calculate the token-level entropy at different temperature levels (solid blue line). To reduce computational overhead, we compute the entropy only over the top-$K$ tokens (with the highest probabilities) at each step, setting $K=1000$ in all subsequent experiments. The entropy curve remains stable for lower temperatures but then shows a sudden rise. One might attribute this behavior to temperature’s role in flattening the distribution (Equation~\ref{Formula 1}). However, the following analysis indicates that this spike reflects a substantial change in the model’s next-token distribution.

Specifically, we constrain the model to generate the same outputs produced by greedy decoding while evaluating entropy under a higher temperature (dotted blue
line). If temperature alone were responsible for the entropy spike, these fixed outputs would yield a similarly high entropy. However, as shown in Figure~\ref{fig: entropy_curve}(a), we observe a significant gap between these two entropy curves, indicating that the actual sampling distribution undergoes a large shift.

Thus, we infer that the sudden rise in the entropy curve implies a substantial drop in sample quality. Setting the temperature around this sudden rise can balance sufficient diversity without a large quality drop, which is suitable for multi-sample aggregation strategies.

\subsection{Turning Point Temperature Selection (\textsc{TURN})}
\label{sec: turn}
Given the token-level entropy curve of a language model on a specific task, how can we identify a suitable temperature for multi-sample aggregation strategies? Inspired by the difference in the shapes of the entropy curve: When the temperature remains low, the entropy increases \emph{flatly}. However, when the sampling temperature is near the spike, the entropy increases \emph{(super)-exponentially}, implying a quality drop in samples. Therefore, after taking the logarithm of the entropy curve (shown in Figure~\ref{fig: entropy_curve}(a), red line), the flat part becomes concave while the exponentially-increase part becomes convex. We define the \emph{entropy turning point (EntP)} as the temperature where the log entropy curve becomes convex. Figure~\ref{fig: entropy_curve}(b) tests the llemma-7b base model and its task-finetuned variant\footnote{Model link: \href{https://huggingface.co/ScalableMath/llemma-7b-sft-metamath-level-1to3-hf}{https://huggingface.co/ScalableMath/llemma-7b-sft-metamath-level-1to3-hf}}~\cite{sun2024easy}, and EntP matches the position with the highest accuracy and varies between different models. Based on EntP, we develop a new method for automatic temperature prediction in multi-sample aggregation strategies, called Turning Point Temperature Selection (\textsc{TURN}).

The optimal temperature should be around EntP to achieve both sample quality and diversity. At the same time, we found that some aggregation methods may be more tolerant of quality drops (e.g., for best-of-N, only one sample is enough to be correct). So we added a small adaptation factor $\beta$ based on the aggregation function, and it is set to $0$ and $+0.1$ for majority voting and best-of-N, respectively. The aggregation adaptation for best-of-N is calculated in the MATH dataset but can be directly applied to other tasks. Refer to Appendix~\ref{app: bias} for details.

Specifically, given a language model $\mathcal{M}$, a task $\mathcal{T}=\{X_1,\ldots,X_k\}$ with $k$ input instances, and an aggregation method $\mathcal{A}$.
To estimate the token-level entropy, we random sample $N$ times. In each time, we randomly choose an input instance $X_i$, and generate one sample by $\mathcal{M}$ under each candidate temperature $t_j = j \cdot t$ (with $t$ being the temperature interval and $j = 0,1,\ldots, J$, where $J=\lfloor t_{\max}/t\rfloor$). These entropies are then aggregated to calculate the average entropy $\mathcal{H}(t_j)$ at each temperature $t_j$. By taking the logarithm, we obtain $\ell(t_j) = \log \mathcal{H}(t_j)$.

Next, we identify EntP index $j^*$, where the second derivative of $\ell$ changes from negative to positive and select its corresponding temperature $j^*\cdot t$. Then we add the aggregation adaptation factor $\beta$ to form the final prediction.
%The corresponding temperature $j^* \cdot t$ is then adjusted with a bias $\beta_{\mathcal{A}}$ related to the aggregation method $\mathcal{A}$, resulting in the final prediction $t^*_{\text{biased}} = t^* + \beta_{\mathcal{A}}$. 
The pseudocode for our algorithm is listed in Algo. \ref{alg:auto find}.
\begin{algorithm}
\caption{Turning Point Temperature Selection \textsc(TURN)}
\label{alg:auto find}
\begin{algorithmic}[1]
\STATE {\bfseries Input:} Language Model $\mathcal{M}$, task $\mathcal{T}=\left(X_1, ..., X_k\right)$, temperature interval $t$, maximum temperature $t_{\max}$, sample size $N$, aggregation method $\mathcal{A}$.
\STATE {\bfseries Output:} Predicted Temperature $T_{\text{pred}}$.
\STATE Compute $J = \lfloor t_{\max}/t \rfloor$ \COMMENT{Number of choices}
\STATE Initialize entropy list $\mathcal{E} = []$
\FOR{$n = 1$ to $N$}
    \STATE Randomly select $X_i$ from $\mathcal{T}$
    \FOR{$j = 0$ to $J$}
        \STATE Generate a sample $Y$ using $\mathcal{M}$ with $T = j\cdot t$
        \STATE Compute token-level entropy of $Y$, add to $\mathcal{E}[j]$
    \ENDFOR
\ENDFOR
\STATE Compute $\mathcal{H}(j)=\text{Mean}\left(\mathcal{E}(j)\right)$ for all $j$
\STATE Compute $\ell(j) = \log \mathcal{H}(j)$ for all $j$
\STATE Find $j^* = \arg\min_j \left( \frac{d^2\ell}{dt^2}>0 \right)$
\STATE Compute $t^* = j^* \cdot t$
\STATE Add adaptation factor $\beta_{\mathcal{A}}$: $T_{\text{pred}} = t^* + \beta_{\mathcal{A}}$
\STATE {\bfseries Return} $T_{\text{pred}}$
%\STATE {\bfseries Return} $t^*$
\end{algorithmic}
\end{algorithm}
\vspace{-5mm}

%\weihua{The turning point of the entropy curve means the quality of samples begins to quickly drop ..., due to the initial difference in sampling strategies, best-of-N is better ..., we give an $+\epsilon$ bias, and we simply set it to 0.1 in our experiment.}
\begin{figure}[ht]
\includegraphics[width=0.48\textwidth]{figs/toy_model.png}
\vspace{-5mm}
\caption{\textbf{Stochastic Process Model.} We run our process model in the setting: $N_0=10$, $N_1=30000$, $L_0=0$, $\sigma_0=1$, $L_1=-10$, and $\alpha=2$. \textbf{(a)} The entropy curve is similar to that of the real language model: flat at first, and then sharply increases. \textbf{(b)} We calculate the relation between temperature and the percentage of improper tokens in simulation, and the percentage of improper tokens quickly increases after EntP.}
\vspace{-5mm}
\label{fig: toy_model}
\end{figure}

\subsection{A Stochastic Process Model}
\label{sec: toy model}
We applied a stochastic process model to explain why the entropy curve exhibits a sudden spike and what that spike signifies.

Because inference is sequential, when the language model makes an error (for example, by sampling an improper token), it increases the likelihood of further mistakes. Meanwhile, the model may occasionally recover and return to a correct trajectory.

To simulate this process, we adopt a stochastic process model with \(K\) steps in sequential, generating a token in each step. At the start, the model has an initial error rate \(p = p_{\text{init}}\), representing the probability of selecting an improper token. At each step, if the model selects an improper token, the likelihood of further errors increases to \(1 - (1 - p)^\alpha\), where \(\alpha > 1\) is called the noise tolerance rate. Conversely, if the model selects a proper token, the error probability decreases to \(p^\alpha\) (but cannot be smaller than \(p_\text{init}\)).

To build a bridge between the temperature \(T\) and the initial error rate \(p_{\text{init}}\), we propose an estimation. All tokens are labeled proper or improper irrelevant to contexts, and the number of improper tokens (\(N_1\)) is much larger than that of proper tokens (\(N_0\)). In the beginning, proper tokens have high logits \(L_0\) with a variance \(\mathcal{N}(0, \sigma_0^2)\) to reflect the nature that there may be several proper next tokens with similar logits. Improper tokens have uniformly low logits \(L_1\). Then, the initial error rate \(p_{\text{init}}\) is determined as the probability of selecting an improper token based on the logits and temperature. 
During inference, all improper tokens equally share the error rate \(p\), while proper tokens account for the remaining probability based on their logits.

Using this setup, we can estimate the token-level entropy. As shown in Figure~\ref{fig: toy_model}(a), the simulated entropy curve (blue line) aligns well with the observed entropy curves of a real language model (Figure~\ref{fig: entropy_curve}(1) solid blue line). Meanwhile, Figure~\ref{fig: toy_model}(b) shows the relationship between the temperature and the percentage of improper tokens, which rises quickly after EntP. This observation suggests that, before EntP, increasing the temperature can help explore the proper tokens. However, after EntP, the increase in the percentage of improper tokens makes the model uncertain and creates errors, implying a quick drop in sample quality. The behavior of the stochastic process model is consistent with our observations of language models, proving that token-level entropy is a good indicator of sample quality. Detailed formulas and experiments can be found in Appendix~\ref{app: toy model}.

\iffalse
\subsection{Balancing Diversity and Quality}
\begin{figure}[ht] 
\centering 
\includegraphics[width=0.48\textwidth]{figs/quality_diversity.png} 
\caption{\textbf{Diversity and Quality Dynamics}: Diversity and quality trends for four language models across varying temperatures, and the suitable temperature ranges are marked in brackets after LM names. The Answer Diversity demonstrates a (sub-)linear increase, whereas sample quality remains stable initially before experiencing a sharp decline. We also noticed that models specifically tuned on the dataset (i.e., (d)) may be tolerant to temperature, resulting in a large suitable temperature range.}
\label{fig: diversity and quality}
\end{figure}
Intuitively, increasing the sampling temperature enhances the diversity of generated answers but tends to reduce their quality. To quantify these two aspects, we define:

\begin{itemize}
    \item \textbf{Quality:} The average accuracy of a single sample.
    \item \textbf{Diversity:} The Shannon entropy of a set of answers. We notice that, at high temperatures, some generated answers become incomplete or invalid. Therefore, we ignore these samples and apply the Miller–Meadow correction~\cite{miller1955note} to mitigate the bias introduced by ignorance.
\end{itemize}
Figure~\ref{fig: diversity and quality} illustrates how sample quality and diversity change with temperature for four language models. Generally, quality remains relatively stable as the temperature increases, then experiences a sharp decline. In contrast, diversity typically grows approximately linearly or sub-linearly but may drop at very high temperatures, where the model seldom produces coherent outputs for answer parsing.

From these observations, one might choose a temperature that maintains a reasonable level of quality while maximizing diversity. However, such a selection relies on accessing quality metrics from a validation set, which is impossible when labeled data are unavailable.
\fi
\makeatletter
\renewcommand{\subsubsection}[1]{%
  \par\noindent\textbf{#1}\par % 不换行且加粗
}
\makeatother

\section{Evaluation and Detection}\label{sec::exp}

% \begin{figure*}[t] % [t] 表示图片放置在页面顶部,跨双栏
% \centering
% \includegraphics[width=\textwidth]{figs_evaluation/evaluationProcess.png} % 使用 \textwidth 适应双栏宽度
%      \caption{In the evaluation process, we generate new QA pairs by changing the scene while keeping the questions fixed: different scenes are randomly selected to form new QA pairs. Additionally, we modify the questions while keeping the scene fixed: spatial relationship-related questions are selected, and all QA pairs are transformed such that the object A is the focus. Then, the spatial relationship in the questions is inverted, generating new QA pairs.}
%      \label{fig:evaluationProcess}
% \end{figure*}



\subsection{ Inadequacy of Existing Evaluation Frameworks}
Existing evaluation frameworks for 2D multimodal models, such as POPE~\cite{li2023evaluating}, are insufficient for addressing the challenges in 3D point cloud large language models (LLMs). Since the POPE view uses yes/no questions to evaluate model object hallucinations, which cannot accurately assess the model's understanding of spatial relationships or visual details such as attributes.In Section 3, we assess hallucinations in 3D point cloud models by evaluating object hallucination in description tasks. However, this method has two main limitations: 1) It only detects hallucinations in description tasks, as not all responses involve objects. 2) It doesn't analyze other types of hallucinations, such as attribute or relational errors.

 Therefore, we aim to propose a more stable, fair, and flexible evaluation framework for evaluating hallucinations in 3D point clouds.
% 随机场景实验的结果
\begin{table*}[h]
\centering
\small
\begin{tabular}{llccc|ccc}
\hline
\multicolumn{2}{c} {\multirow{3}{*}{Type}} & \multicolumn{3}{c|}{ll3da} & \multicolumn{3}{c}{3dllm} \\ \cline{3-8}
 
     &   & \multicolumn{2}{c}{Accuracy} & \multirow{2}{*}{\textbf{$HR_{random}$\%}}  & \multicolumn{2}{c}{Accuracy} & \multirow{2}{*}{\textbf{$HR_{random}$\%}}  \\
     
     &   & Rouge-L & Meteor&  & Rouge-L & Meteor&  \\ \hline


\multirow{4}{*}{Relation} & Direction  & 30.62  & 19.53 &\textbf{33.21} & 30.32  & 19.77 &\textbf{30.43}\\
                              & Containment  & 43.28  & 35.27&\textbf{36.89}  & 42.51  & 31.98&\textbf{43.69} \\ 
                              & Contact & 35.08  & 23.55 &\textbf{34.72}  & 35.58  & 24.2 &\textbf{36.79}\\ 
                              & Distance    & 32.02  & 22.71 &\textbf{31.49} & 32.5  & 21.36&\textbf{28.94} \\ \hline
                             
\multirow{3}{*}{Property}  & Color                    & 47.38  & 41.9 &\textbf{62.69} & 51.72   & 47.38&\textbf{61.77} \\ 
                                      & Shape                    & 42.74  & 31.9  &\textbf{49.48}& 44.56  & 32.94 &\textbf{46.39}\\ 
                                      & Size                     & 43.74  & 39.01 &\textbf{74.29} & 47.48  & 37.57&\textbf{51.43} \\ \hline
                                     
\multicolumn{2}{c}{Comparison}                 & 24.75  & 17.65 &\textbf{63.16} & 29.43  & 21.82&\textbf{42.11} \\ 
\multicolumn{2}{c}{Quantity} & 50.18  & 41.84&\textbf{63.93}  & 49.85  & 42.68 &\textbf{53.88} \\ 
\multicolumn{2}{c}{Usage} & 32.22  & 21.62 &\textbf{34.78} & 30.88  & 22.03&\textbf{26.09} \\ 
\multicolumn{2}{c}{Other}  & 37.22  & 31.72&\textbf{38.89}  & 39.8  & 32.82&\textbf{25.93} \\ 
\hline
\end{tabular}
\caption{Model Performance and Hallucination Rate in Random Scenarios.Accuracy refers to the evaluation result between the model's response and the ground truth. $HR_{random}$ is the hallucination rate calculated based on random scenes as defined in Section 5.}
\label{table:randomSceneResult}
\end{table*}
% 随机场景实验的分析图
\begin{figure*}[h]
\begin{center}
\centering
    \includegraphics[width=0.8\linewidth]{figs_evaluation/propertydistribution.png}
     \caption{Impact of Attribute Simplicity on Accuracy.ROUGE represents the average quality of question-answer pairs for a specific item, while the Top 3 Ratio is the proportion of the three most common attributes of the item.}
     \label{fig:topkRatio}
\end{center}
\end{figure*}
\subsection{Proposed Evaluation Framework}
We propose two strategies for detecting hallucinations in 3D point cloud models.\\
\textbf{\emph{Random Point Cloud Pair Evaluation}} ~We select a random point cloud and ask the model the same question on both the original and new point clouds. If the answers are identical, it's considered a hallucination, indicating the model doesn't integrate visual context and just maps the question to a fixed answer.\\
\textbf{\emph{Opposite Question Evaluation}} ~For a fixed point cloud, we ask two Opposite questions (e.g., "What is on the right of the table?" and "What is on the left?"). If the model gives the same answer, it's a hallucination, suggesting the model isn't using the spatial information from the point cloud.

 By employing these two strategies, we aim to identify cases where the model fails to distinguish between spatially different scenarios or produces inconsistent responses to questions.
\subsection{ Inadequacy of Existing Evaluation Frameworks}
The entire pipeline is illustrated in Figure \ref{fig:evaluationProcess}. \\
\textbf{Data Generation}: In the \textbf{change scene} experiment, for each $(Q_i,A_i,S_i)$ pair, we randomly select a different $S_j$ from the scene set to create a new $(Q_i,A_i,\{S_i, S_j\})$ pair dataset.In the \textbf{change question} experiment, we first select questions related to spatial relationships and use GPT-4 to convert each QA pair into a dataset where the answer is an object, resulting in the \textit{scanqa-SR} dataset. For each spatial relationship question $Q_i$ in \textit{scanqa-SR}, we generate its opposite $Q_j$ to form $(\{Q_i, Q_j\},A_i,S_i)$ pairs, creating the \textit{scanqa-SR-Opposite} dataset.\\
\textbf{Experiment}: We then conduct tests using the aforementioned data on different models. 
In Experiment 1, for a given question \( q_i \), we generate two answers, \( a_{ij} \) and \( a_{ik} \), corresponding to two different scenes, \( s_{j} \) and \( s_{k} \), respectively.
We use BLEU-4~\cite{papineni2002bleu}, ROUGE~\cite{lin-2004-rouge}, and METEOR~\cite{banerjee2005meteor} metrics to measure the similarity between two answers. The hallucination rate($HR_{random}$) is calculated as follows:
\begin{equation}
HR_{\text{random}} = \frac{1}{N} \sum_{i=1}^{N} \mathbf{1}(\text{ROUGE}(a_{ij}, a_{ik}) > 0.66)
\end{equation}
In Experiment 2, for a fixed scene \( s_i \), we generate two answers, \( a_{ji} \) and \( a_{jk} \), for two semantically opposite questions, \( q_j \) and \( q_k \).The hallucination rate($HR_{opposite}$) is calculated as follows:
\begin{equation}
HR_{\text{opposite}} = \frac{1}{N} \sum_{i=1}^{N} \mathbf{1}(\text{ROUGE}(a_{ji}, a_{jk}) > 0.66)
\end{equation}
% \textbf{Hallucination Rate}: We use BLEU-4\cite{papineni2002bleu}, ROUGE\cite{lin-2004-rouge}, and METEOR\cite{banerjee2005meteor} metrics to measure the similarity between two answers. We define hallucination as occurring when the ROUGE score between two answers is greater than 0.66, indicating that the answers are semantically identical. The hallucination rate(\(HR\)) is computed as the ratio of hallucinations to the total number of experiments (\(N\)).
% \begin{equation}
% HR_{\text{random}} = \frac{1}{N} \sum_{i=1}^{N} \mathbf{1}(\text{ROUGE}(a_{ij}, a_{ik}) > 0.66)
% \end{equation}
% \begin{equation}
% HR_{\text{opposite}} = \frac{1}{N} \sum_{i=1}^{N} \mathbf{1}(\text{ROUGE}(a_{ji}, a_{jk}) > 0.66)
% \end{equation}

\begin{table*}[h]
\centering
\small
\begin{tabular}{lllcccc}
\hline
Dataset & Task &Model & Bleu-4  & Rouge-L & Meteor   &\textbf{$HR_{opposite}$\%}  \\ \hline
\multirow{6}{*}{scannet}&\multirow{2}{*}{scanqa}&ll3da&7.64&36.56&26.95&/\\ 
& &3dllm&0.80&37.46&28.18&/ \\  \cline{2-7}
&\multirow{2}{*}{scanqa-SR}&ll3da&0.02&13.34&9.68&/ \\ 
&&3dllm&0.0&15.55&10.28&/ \\   \cline{2-7}
&\multirow{2}{*}{scanqa-SR-Opposite}&ll3da&/&/&/&\textbf{56.27} \\ 
&&3dllm&/&/&/&\textbf{52.25} \\ \hline
\end{tabular}
\caption{Model Performance and Hallucination Rate on Semantically Opposite Questions.BLEU-4, ROUGE, and METEOR are evaluation metrics for model response quality based on ground truth, while $HR_{opposite}$ represents the hallucination rate in the opposite-question experiment.}
\label{table:oppositeQuestionResult}
\end{table*}
\section{Evaluation on 3dllm and ll3da}
\subsection{Hallucinations in Random Scene Queries}
We evaluate two models using the approach above. Table \ref{table:randomSceneResult} presents the results for random scenes. ROUGE and METEOR measure performance on ScanQA, while $HR_{random}$ is defined in Section 5.2. The table shows a positive correlation between accuracy and hallucination rate. LL3DA and 3DLLM both exhibit low accuracy and hallucination rates for spatial questions but higher rates for object attributes.\\
This suggests that the model exhibits significant hallucination issues, where it answers questions without considering the visual context, yet its responses appear 'better' or closer to the ground truth. Upon examining the training set, we find that object attributes often align with typical characteristics—for example, tables are usually black, white, or brown, and televisions are typically rectangular. This indicates that the model learns attribute associations due to the homogeneous nature of indoor scenes and the limited diversity of attributes.
% 模型准确率与物品之间的关系
\subsection{Relationship Between Attribute Uniformity and Answer Accuracy}
We plotted Figure \ref{fig:topkRatio} to illustrate the relationship between the uniformity of an object's properties and the accuracy of the answers. For instance, chair color is queried 346($N$) times, with black ($T_1$ times), brown ($T_2$ times), and gray ($T_3$ times) as the most frequent colors. To quantify attribute uniformity, we introduce the "Top-K Ratio," where the Top-3 Ratio for the chair can be calculated as: 
\begin{equation}
\text{Top-3 Ratio} = \frac{T_1 + T_2 + T_3}{N}.
\end{equation} The x-axis of the figure represents the average ROUGE score for questions related to a specific object, with higher ROUGE scores indicating that the questions regarding the object's properties are more easily answered correctly. The three plots from left to right show the relationship between the accuracy of answers and the uniformity of the object's properties, specifically color, shape, and size.
In the plots for color and shape, the distribution of points is approximately linear, confirming a strong positive correlation between the accuracy of the answers and the uniformity of the object's properties. Additionally, we observed that many points clustered near a Top-3 Ratio of 1, suggesting that the dataset contains objects with highly uniform attributes. Such objects tend to exhibit a strong correlation between the object and a specific attribute, which makes it easier for the model to hallucinate the correct attribute. 
\subsection{Hallucinations in Opposite-Question Queries}The results for testing with 
opposite questions within the same scene are presented in Table \ref{table:oppositeQuestionResult}.The ScanQA dataset includes a wide range of QA pairs involving various attributes, spatial relationships, and other data types. In contrast, ScanQA-SR focuses solely on spatial relationships and transforms all QA pairs into those where the answer is the object itself.\\
By comparing the results from these two datasets, we observe that the ROUGE scores for ScanQA-SR are significantly lower than those for ScanQA. This indicates that the model is more prone to errors when dealing with spatial relationship tasks. To investigate whether the model truly understands the meaning of spatial relationships, we created a dataset of opposite questions specifically for spatial relationships. The goal was to assess the model’s ability to handle questions about opposing spatial positions.\\
However, we found that the hallucination rate for both models exceeded 50\%. This suggests that when posed with opposite questions about the same scene, the model has a 50\% chance of giving the same answer. This further demonstrates that the model is prone to errors and hallucinations when handling spatial relationship queries. The results imply that the model may lack a proper visual-semantic understanding of spatial relationships, leading it to answer incorrectly without considering point cloud data.

% \begin{table*}[t]
% \centering
% \begin{tabular}{lllccccc}
% \hline
% Dataset & Task &Model & Bleu-4  & Rouge-L & Meteor & Acc & Hallucination Rate  \\ \hline
% \multirow{6}{*}{scannet}&\multirow{2}{*}{scanqa}&ll3da&7.64&36.56&26.95&26.46&/\\ 
% & &3dllm&0.80&37.46&28.18&26.57&/ \\  \cline{2-8}
% &\multirow{2}{*}{scanqa-object-spatial}&ll3da&0.02&13.34&9.68&7.85&/ \\ 
% &&3dllm&0.0&15.55&10.28&8.34&/ \\   \cline{2-8}
% &\multirow{2}{*}{scanqa-spatial-hallucination}&ll3da&1.32&57.02&56.21&/&56.27 \\ 
% &&3dllm&1.08&55.65&52.60&/&52.25 \\ \hline
% \end{tabular}
% \caption{detection}
% \label{table:performance}
% \end{table*}




% We have merged several spatial relationships into broader categories to simplify classification. The relationships are grouped as follows:

% \begin{itemize}
%     \item \textbf{Directional Relationships}: These include terms such as \textit{opposite}, \textit{under}, \textit{above}, \textit{outside}, \textit{right}, \textit{left}, \textit{front}, \textit{behind}, \textit{in front}, \textit{diagonal}, and \textit{facing}.
%     \item \textbf{Contact/Support Relationships}: This category includes relationships like \textit{resting on}, \textit{lying on}, \textit{standing on}, \textit{on}, \textit{on top of}, \textit{hanging on}, \textit{attached to}, and \textit{against}.
%     \item \textbf{Containment/Surrounding Relationships}: Terms like \textit{inside}, \textit{outside}, \textit{surrounding}, and \textit{enclosed by} fall under this group.
%     \item \textbf{Distance Relationships}: This group includes \textit{far away}, \textit{away}, \textit{near}, \textit{close to}, and \textit{detached from}.
%     \item \textbf{Adjacency/Relative Location Relationships}: This category involves relationships such as \textit{next to}, \textit{adjacent to}, \textit{on each side of the couch}, \textit{between}, and \textit{between two walls}.
%     \item \textbf{Other Descriptive Relationships}: This group includes \textit{centered on}, \textit{off the wall}, and \textit{described as}.
% \end{itemize}
















% \begin{figure*}[ht]
% \begin{center}
% \centering
%     \includegraphics[width=0.93\linewidth]{figs_evaluation/question_reverse_line.jpg}
%      \caption{contract question}
% \end{center}
% \end{figure*}


\section{Related Work} \label{related}




% \subsection{Benchmarks in Coding Scenarios}
% \begin{enumerate}
%     \item Code Generation
%     \item Bug Fixing
% \end{enumerate}

% \subsection{Large Language Model Agents}

% At the heart of the LLM Agent is an Agent Core, which coordinates the core \textit{logic} and \textit{behavioral} characteristics of the agent. In addition, the Agent includes the following key components:

% \begin{itemize}
%     \item Memory Module: It consists of both short-term and long-term memory components that record the agent's internal logs and interactions with the user.
%     \item Tools: These are the tools that the agent can use to perform tasks, usually specific third-party APIs.
%     \item Planning Module: This is used for solving complex problems, such as decomposing tasks and problems, reflexivity or critique.
% \end{itemize}

% \subsection{Multi Agent Collaboration Framework}

% MetaGPT \url{https://arxiv.org/abs/2308.00352}


\parabf{Coding \llm{s}.}
Large Language Models (\llm{s}) have become the go-to solution for a wide array of coding tasks due to their exceptional performance in both code generation and comprehension~\cite{codex}. These models have been successfully applied to various software engineering activities, including program synthesis~\cite{patton2024programming, codex, li2022competition, iyer2018mapping}, code translation~\cite{pan2024lost, roziere2020unsupervised, roziere2021leveraging}, program repair~\cite{xia2023repairstudy, chatrepair, monperrus2018living, bouzenia2024repairagent}, and test generation~\cite{titanfuzz, fuzz4all, deng2023fuzzgpt, lemieux2023codamosa, kang2023testing}. Beyond general-purpose \llm{s}, specialized models have been developed by further training on extensive datasets of open-source code snippets. Notable examples of these code-specific \llm{s} include \codex~\cite{codex}, \codellama~\cite{codellama}, StarCoder~\cite{starcoder,starcodertwo}, and \deepseek~\cite{deepseek}. Additionally, instruction-following code models have emerged, refined through instruction-tuning techniques. These include models such as \codellamainstruct~\cite{codellama}, \deepseekinstruct~\cite{deepseek}, \wizardcoder~\cite{wizardcoder}, \magicoder~\cite{magicoder}, and OpenCodeInterpreter~\cite{zheng2024opencodeinterpreter}.

\parabf{Benchmarking \llm-based coding tasks.}
To assess the capabilities of \llm{s} in coding, a variety of benchmarks have been proposed. Among the most widely utilized are \humaneval~\cite{codex} and \mbpp~\cite{austin2021program}, which are handcrafted benchmarks for code generation that include test cases to validate the correctness of \llm outputs. Other benchmarks have been developed to offer more rigorous tests~\cite{evalplus}, cover additional programming languages~\cite{zheng2023codegeex,cassano2023multipl}, and address different programming domains~\cite{livecodebench, hendrycksapps2021, codecontest, ds1000, arcade}.

More recently, research has shifted towards evaluating \llm{s} on real-world software engineering challenges by operating on entire code repositories rather than isolated coding problems~\cite{swebench, zhang2023repocoder, liu2023repobench}. A notable benchmark in this area is \swebench~\cite{swebench}, which includes tasks requiring repository modifications to resolve actual GitHub issues. The authors of \swebench have also released a more focused subset, \swebenchlite~\cite{swebenchlite}, which contains 300 problems centered on bug fixing that only involves single-file modifications in the ground truth patches. ML-Bench \cite{liu2023mlbench} is a benchmark for evaluating large language models and agents for Machine Learning tasks on reporitory-level code. It involves 18 repositories and focuses on code generation and interactions with Jupyter Notebooks.

\parabf{Repository-level coding.}
The rise of agent-based frameworks~\cite{xi2023rise} has spurred the development of agent-based approaches to software engineering tasks. Devin~\cite{devinwebpage} (and its open-source counterpart OpenDevin~\cite{opendevin}) is among the first comprehensive \llm agent-based frameworks. Devin employs agents to first perform task planning based on user requirements, then allows them to use tools like file editors, terminals, and web search engines to iteratively execute the tasks. \sweagent~\cite{sweagent} introduces a custom agent-computer interface (ACI), enabling the \llm agent to interact with the repository environment through actions like reading and editing files or running bash commands. Another agent-based approach, \autocoderover~\cite{autocoderover}, equips the \llm agent with specific APIs (e.g., searching for methods within certain classes) to effectively identify the necessary modifications for issue resolution. Beside these examples, a variety of other agent-based approaches have been developed in both open-source~\cite{aidar} and commercial products~\cite{bouzenia2024repairagent, coder, repounderstander, lingma, factorydroid, ibmagent, opencsgstarship, marscode, amazonqdeveloper}.

% Unlike these agent-based methods, \tech offers a straightforward and cost-efficient solution for addressing real-world software engineering challenges. Our work is the first to demonstrate that an \emph{agentless} approach can achieve comparable performance without the need for complex tools or modeling intricate environment behavior and feedback.

Unlike existing benchmarks and agent-based frameworks, which focus on the code generation/completion tasks, our proposed \model and \agent focus on the code deployment task, which is under-studied in the field.


\section{Conclusion}\label{sec::conclusion}

This paper pioneers the exploration of EQA tasks in outdoor urban environments. First, we introduced CityEQA-EC, the inaugural open-ended benchmark for CityEQA, comprising 1,412 tasks divided into six distinct categories. Second, we proposed a novel agent model (the PMA), designed to tackle long-horizon tasks through hierarchical planning, sensing, and execution.  Experimental results validated the effectiveness of PMA, achieving 60.73\% accuracy relative to human performance and outperforming traditional methods such as the FBE Agent. Nevertheless, challenges remain, including efficiency discrepancies (24.44 vs. 9.31 mean time steps taken by humans) and limitations in visual thinking capabilities. Future research could focus on enhancing PMA with self-reflection and error-correction mechanisms to mitigate error accumulation that can arise in long-horizon tasks.

\clearpage

\section{Limitations}
The work primarily focuses on object-centric question-answering tasks, such as identifying specific objects (e.g., buildings, vehicles) within city spaces.  Further, while our approach is effective for tasks involving static physical entities, it overlooks the importance of social interactions and dynamic events, which are also critical in urban settings.  For instance, questions related to dynamic events (e.g., "Is there a traffic jam on Main Street?"), or environmental conditions (e.g., "Is the park crowded right now?") are not considered up to now. These types of questions require some different sets of reasoning capabilities, such as temporal reasoning, event detection, and social context understanding, which are not currently supported by the Planner-Manager-Actor (PMA) agent.  Future work should expand the scope of CityEQA to include these non-entity-based tasks, further extending PMA and enabling embodied agents to handle a broader range of urban spatial intelligence challenges.


% \paragraph{Simulation Environment Constraints} The reliance on a single simulated cityscape (EmbodiedCity) limits generalizability to diverse real-world urban layouts. Depth estimation errors and simplified spatial relationships (e.g., cardinal directions only) may introduce biases.

% \paragraph{Error Accumulation in Long-Horizon Tasks} PMA’s hierarchical structure risks compounding errors during multi-step navigation and exploration, particularly in cluttered or ambiguous environments.

% \paragraph{Visual Thinking in Collector}


\section{Ethics Statement}
In the data collection, we ensure there is no identifiable information about individuals (faces, license plates) or private properties.
Thus, there is no ethical concern.
% Agents operating in city spaces (e.g., drones) may inadvertently capture identifiable information about individuals (faces, license plates) or private properties during exploration. Strict protocols for real-time data anonymization, limited retention periods, and explicit opt-out mechanisms for sensitive areas (e.g., residential zones) must be implemented.

% \clearpage

\bibliographystyle{ACM-Reference-Format}
% \balance
% \bibliographystyle{plain}
\bibliography{bibliography}
% \nobalance

% \nobalance 
\clearpage

\onecolumn
\clearpage
\appendix
\appendixpage  % if you use a package that provides an appendix title page
\hypersetup{linkcolor=black}
\startcontents[sections]
\printcontents[sections]{l}{1}

\hypersetup{linkcolor=hrefblue}
\glsresetall

\section{Additional related works}\label{apx:related_works}

\paragraph{Knowledge distillation.}
Knowledge distillation (KD)~\citep{hinton2015distilling,gou2021knowledge} is closely connected to W2S generalization regarding the teacher-student setup, while W2S reverts the capacities of teacher and student in KD. In KD, a strong teacher model guides a weak student model to learn the teacher's knowledge. In contrast, W2S generalization occurs when a strong student model surpasses a weak teacher model under weak supervision.
\citet{phuong2019towards,stanton2021does,ojha2023knowledge,nagarajan2023student,dong2024cluster,ildiz2024high} conducted rigorous statistical analyses for the student's generalization from knowledge distillation. 
From the analysis perspective, a key difference between KD and W2S is that W2S is usually analyzed in the context of finetuning since the notions of “weak” and “strong” are built upon pretraining. This finetuning perspective introduces distinct angles from KD for examining intrinsic dimension~\citep{li2018measuring} and student-teacher correlation in W2S. 

\paragraph{Self-distillation and self-training.}
In contrast to W2S that considers distinct student and teacher models, self-distillation~\citep{zhang2019your,zhang2021self} and related paradigms such as Born-Again Networks~\citep{furlanello2018born} use the same or progressively refined architectures to iteratively distill knowledge from a ``previous version'' of the model. There have been extensive theoretical analyses toward understanding the mechanism behind self-distillation~\citep{mobahi2020self,das2023understanding,borup2023self,pareek2024understanding}.

Self-training~\citep{scudder1965probability,lee2013pseudo} is a closely related method to self-distillation that takes a single model's confident predictions to create pseudo-labels for unlabeled data and refines that model iteratively. 
\citet{wei2020theoretical,oymak2021theoretical,frei2022self} provide theoretical insights into the generalization of self-training. 
In particular, \citet{wei2020theoretical} introduced a theoretical framework based on neighborhood expansion, which was later on extended to various settings of weakly supervised learning, including domain adaptation~\citep{cai2021theory}, contrastive learning~\citep{shen2022connect}, consistency regularization~\citep{yang2023sample}, and now weak-to-strong generalization~\citep{lang2024theoretical,shin2024weak}.




\section{Proofs in \Cref{sec:single_task_ft}}

\begin{lemma}\label{lem:low_est_err_ft}    
    Given the FT approximation errors $\rho_s$ and $\rho_w$ in \Cref{def:ft_est_err}, we have
    \begin{align*}
        \rho_s(n) \le n \rho_s \quad \text{and} \quad \rho_w(n) \le n \rho_w \quad \forall\ n \in \N.
    \end{align*}
\end{lemma}

\begin{proof}[Proof of \Cref{lem:low_est_err_ft}]
    Let $\thetab_* = \argmin_{\thetab \in \R^d}\ \E_{\xb \sim \Dcal}[(\phi_w(\xb)^\top \thetab - f_*(\xb))^2]$ such that
    \begin{align*}
        \E_{\xb \sim \Dcal}[(\phi_w(\xb)^\top \thetab_* - f_*(\xb))^2] = \rho_w.
    \end{align*}
    Then, by observing that conditioned on $\Xb$,
    \begin{align*}
        \phi_w(\Xb)^\dagger f_*(\Xb) = \argmin_{\thetab \in \R^d}\ \| \phi_w(\Xb) \thetab - f_*(\Xb) \|_2^2,
    \end{align*} 
    we have
    \begin{align*}
        \rho_w(n) &= \E_{\Xb \sim \Dcal^n}\sbr{\| \phi_w(\Xb) \phi_w(\Xb)^\dagger f_*(\Xb) - f_*(\Xb) \|_2^2} \\
        &\le \E_{\Xb \sim \Dcal^n}\sbr{\| \phi_w(\Xb) \thetab_* - f_*(\Xb) \|_2^2} \\
        &= n\ \E_{\Xb \sim \Dcal^n}\sbr{\frac{1}{n} \| \phi_w(\Xb) \thetab_* - f_*(\Xb) \|_2^2} \\
        &= n\ \E_{\xb \sim \Dcal}\sbr{(\phi_w(\xb)^\top \thetab_* - f_*(\xb))^2} \\
        &= n\ \rho_w.
    \end{align*}
    The proof for $\rho_s(n)$ follows analogously.
\end{proof}



\subsection{Proof of \Cref{thm:w2s_ft}}\label{apx:pf_w2s_ft}

\begin{theorem}[Formal restatement of \Cref{thm:w2s_ft}]\label{thm:w2s_ft_formal}
    Consider $f_\wts(\xb) = \phi_s(\xb)^\top \thetab_\wts$ finetuned as in \eqref{eq:sft_weak}, \eqref{eq:w2s_ft} with both $\alpha_w, \alpha_\wts \to 0$. Under \Cref{asm:features,asm:ft_data}, when $n \ge \Omega(d_w)$, the excess risk $\exrisk(f_\wts) = \vari(f_\wts) + \bias(f_\wts)$ satisfies
    \begin{align*}
        &\bias(f_\wts) \le \frac{\rho_w(n)}{n} + \frac{\rho_s(N)}{N} \le \rho_w + \rho_s, \\
        &\vari(f_\wts) \lesssim \frac{\sigma^2}{n} \rbr{d_{s \wedge w} + \frac{d_s}{N} (d_w - d_{s \wedge w})}.
    \end{align*}
    In particular, when ${\rho_w(n)}/{n} > 0$ and $d_s < d_w$, the inequality for $\bias(f_\wts)$ is strict.

    Moreover, when $\phi_w(\xb) \sim \Ncal(\b0_d, \Sigmab_w)$, for any $n > d_w + 1$, we have 
    \begin{align*}
        &\vari(f_\wts) = \frac{\sigma^2}{n-d_w-1} \rbr{d_{s \wedge w} + \frac{d_s}{N} (d_w - d_{s \wedge w})}.
    \end{align*}
\end{theorem}

\begin{proof}[Proof of \Cref{thm:w2s_ft} and \Cref{thm:w2s_ft_formal}]
    We first observe that the solution of \eqref{eq:sft_weak} as $\alpha_w \to 0$ is given by
    \begin{align*}
        \thetab_w = \wt\Phib_w^\dagger \wt\yb = \wt\Phib_w^\dagger (\wt\fb_* + \wt\zb),
    \end{align*}
    where $\wt\zb \sim \Ncal(\b0_n, \sigma^2 \Ib_n)$.
    Meanwhile, the solution of \eqref{eq:w2s_ft} as $\alpha_\wts \to 0$ is given by
    \begin{align*}
        \thetab_\wts = \Phib_s^\dagger \Phib_w \thetab_w = \Phib_s^\dagger \Phib_w \wt\Phib_w^\dagger (\wt\fb_* + \wt\zb).
    \end{align*}  
    
    Then, the excess risk of $f_\wts$ can be decomposed into variance and bias as follows:
    \begin{align*}
        \exrisk(f_\wts) &= \E_{\xb \sim \Dcal}\sbr{\E_{f_\wts}\sbr{(f_\wts(\xb) - f_*(\xb))^2}} \\
        &= \E_{\Scal_x}\sbr{\E_{\wt\Scal}\sbr{\frac{1}{N}\nbr{\Phib_s \thetab_\wts - \fb_*}_2^2}} \\
        &=\E_{\Scal_x, \wt\Scal}\sbr{\frac{1}{N} \nbr{(\Phib_s \Phib_s^\dagger \Phib_w \wt\Phib_w^\dagger \wt\fb_* - \fb_*) + \Phib_s \Phib_s^\dagger \Phib_w \wt\Phib_w^\dagger \wt\zb}_2^2} \\
        &= \underbrace{\frac{1}{N} \E_{\Scal_x, \wt\Scal}\sbr{\nbr{\Phib_s \Phib_s^\dagger \Phib_w \wt\Phib_w^\dagger \wt\zb}_2^2}}_{\vari(f_\wts)} + \underbrace{\frac{1}{N} \E_{\Scal_x, \wt\Scal}\sbr{\nbr{\Phib_s \Phib_s^\dagger \Phib_w \wt\Phib_w^\dagger \wt\fb_* - \fb_*}_2^2}}_{\bias(f_\wts)}.
    \end{align*}

    \paragraph{Bias.}
    For the bias term, by observing that $\Pb_s = \Phib_s \Phib_s^\dagger$ is an $N \times N$ orthogonal projection, we can decompose the bias term as
    \begin{align*}
        \bias(f_\wts) &= \E_{\Scal_x, \wt\Scal}\sbr{\frac{1}{N} \nbr{\Pb_s \rbr{\Phib_w \wt\Phib_w^\dagger \wt\fb_* - \fb_*}}_2^2} + \frac{1}{N} \E_{\Scal_x}\sbr{\nbr{\rbr{\Ib_N - \Pb_s} \fb_*}_2^2},
    \end{align*}
    where $\E_{\Scal_x}\sbr{\nbr{\rbr{\Ib_N - \Pb_s} \fb_*}_2^2} = \rho_s(N)$ by \Cref{def:ft_est_err}.

    For the first term, 
    \begin{align*}
        \E_{\Scal_x, \wt\Scal}\sbr{\frac{1}{N} \nbr{\Pb_s \rbr{\Phib_w \wt\Phib_w^\dagger \wt\fb_* - \fb_*}}_2^2} &\le \E_{\Scal_x, \wt\Scal}\sbr{\frac{1}{N} \nbr{\Phib_w \wt\Phib_w^\dagger \wt\fb_* - \fb_*}_2^2} \\
        &= \E_{\wt\Scal}\sbr{\frac{1}{n} \nbr{\wt\Phib_w \wt\Phib_w^\dagger \wt\fb_* - \wt\fb_*}_2^2} \\
        &= \frac{\rho_w(n)}{n}.
    \end{align*}
    Notice that when ${\rho_w(n)}/{n} > 0$, this inequality is strict if $d_s < d_w$, where $\Phib_w \wt\Phib_w^\dagger \wt\fb_* - \wt\fb_* \notin \range(\Phib_s)$ almost surely.

    Overall, we have
    \begin{align*}
        \bias(f_\wts) \le \frac{\rho_w(n)}{n} + \frac{\rho_s(N)}{N} \le \rho_w + \rho_s,
    \end{align*}
    where the second inequality follows from \Cref{lem:low_est_err_ft}.

    \paragraph{Variance.}
    For the variance term, we observe that
    \begin{align*}
    \begin{split}
        \vari(f_\wts) &= \frac{1}{N} \E_{\Scal_x, \wt\Scal}\sbr{\nbr{\Pb_s \Phib_w \wt\Phib_w^\dagger \wt\zb}_2^2} \\
        &= \frac{1}{N} \E_{\Scal_x, \wt\Scal}\sbr{\tr\rbr{\Phib_w^\top \Pb_s \Phib_w \wt\Phib_w^\dagger \wt\zb \wt\zb^\top (\wt\Phib_w^\dagger)^\top}} \\
        &= \frac{\sigma^2}{N} \E_{\Scal_x, \wt\Scal}\sbr{\tr\rbr{\Phib_w^\top \Pb_s \Phib_w (\wt\Phib_w^\top \wt\Phib_w)^\dagger}},
    \end{split}
    \end{align*}
    which implies
    \begin{align}\label{eq:pf_var_w2s}
    \begin{split}
        \vari(f_\wts) = \frac{\sigma^2}{N} \tr\rbr{\E_{\Scal_x}\sbr{\Sigmab_w^{-1/2} \Phib_w^\top \Pb_s \Phib_w \Sigmab_w^{-1/2}} \E_{\wt\Scal}\sbr{\rbr{\Sigmab_w^{-1/2} \wt\Phib_w^\top \wt\Phib_w \Sigmab_w^{-1/2}}^\dagger}}.
    \end{split}
    \end{align}

    Recall the spectral decomposition $\Sigmab_w = \Vb_w \Lambdab_w \Vb_w^\top$. 
    Since $\E_{\xb \sim \Dcal}[\phi_w(\xb) \phi_w(\xb)^\top] = \Sigmab_w$, for each $\xb \sim \Dcal$, we can write $\phi_w(\xb) = \Sigmab_w^{1/2} \gammab$, where $\gammab \in \R^{d}$ is an independent random vector that is zero-mean and isotropic (\ie $\E[\gammab] = \b0_{d}$ and $\E[\gammab \gammab^\top] = \Ib_{d}$). The same holds for $\Sigmab_s = \Vb_s \Lambdab_s \Vb_s^\top$ and $\phi_s(\xb) = \Sigmab_s^{1/2} \gammab$.

    Then, for $\Scal$ and $\wt\Scal$, there exist independent random matrices $\Gammab = [\gammab_1, \ldots, \gammab_N]^\top \in \R^{N \times d}$ and $\wt\Gammab = [\wt\gammab_1, \ldots, \wt\gammab_n]^\top \in \R^{n \times d}$ consisting of $\iid$ zero-mean isotropic rows such that
    \begin{align}\label{eq:pf_var_w2s_subgaussian_asm}
    \begin{split}
        &\Phib_w \Sigmab_w^{-1/2} = \Gammab \Sigmab_w^{1/2} \Sigmab_w^{-1/2} = \Gammab \Vb_w \Vb_w^\top, \\
        &\wt\Phib_w \Sigmab_w^{-1/2} = \wt\Gammab \Sigmab_w^{1/2} \Sigmab_w^{-1/2} = \wt\Gammab \Vb_w \Vb_w^\top, \\
        &\Phib_s \Sigmab_s^{-1/2} = \Gammab \Sigmab_s^{1/2} \Sigmab_s^{-1/2} = \Gammab \Vb_s \Vb_s^\top, \\
        &\wt\Phib_s \Sigmab_s^{-1/2} = \wt\Gammab \Sigmab_s^{1/2} \Sigmab_s^{-1/2} = \wt\Gammab \Vb_s \Vb_s^\top.
    \end{split}
    \end{align}
    Let $\Gammab_w = \Gammab \Vb_w \in \R^{N \times d_w}$ and $\wt\Gammab_w = \wt\Gammab \Vb_w \in \R^{n \times d_w}$. We observe that
    \begin{align*}
        \E_{\wt\Scal}\sbr{\rbr{\Sigmab_w^{-1/2} \wt\Phib_w^\top \wt\Phib_w \Sigmab_w^{-1/2}}^\dagger}
        = \E_{\wt\Scal}\sbr{\rbr{\Vb_w \wt\Gammab_w^\top \wt\Gammab_w \Vb_w^\top}^\dagger} 
        = \Vb_w \E_{\wt\Scal}\sbr{\rbr{\wt\Gammab_w^\top \wt\Gammab_w}^\dagger} \Vb_w^\top.
    \end{align*}

    Now, we consider the following two cases for the feature distribution of $\phi_w(\xb)$, corresponding to the distribution of $\Gammab_w$ and $\wt\Gammab_w$:
    \begin{enumerate}[label=(\alph*)]
        \item \b{Gaussian features}: In \Cref{thm:w2s_ft}, assuming $\phi_w(\xb) \sim \Ncal(\b0_d, \Sigmab_w)$ such that $\wt\Gammab_w$ consists of $\iid$ Gaussian rows, we have $\wt\gammab_i \sim \Ncal(\b0_{d_w}, \Ib_{d_w})$. Notice that under the assumption $n > d_w + 1$, $\rank(\wt\Gammab_w) = d_w$ almost surely, and therefore $\wt\Gammab_w^\top \wt\Gammab_w$ is invertible.
        
        Meanwhile, with $\wt\gammab_i \sim \Ncal(\b0_{d_w}, \Ib_{d_w})$ for all $i \in [n]$, $(\wt\Gammab_w^\top \wt\Gammab_w) \sim \Wcal(\Ib_{d_w},n)$ follows the Wishart distribution~\citep[Definition 3.4.1]{wishart1928generalised} with $n$ degrees of freedom and scale matrix $\Ib_{d_w}$. 
        Therefore, $(\wt\Gammab_w^\top \wt\Gammab_w)^{-1} \sim \Wcal^{-1}(\Ib_{d_w},n)$ follows the inverse Wishart distribution~\citep[\S 3.8]{mardia2024multivariate}, whose mean takes the form~\citep[(3.8.3)]{mardia2024multivariate}
        \begin{align*}
            \E_{\wt\Scal}\sbr{(\wt\Gammab_w^\top \wt\Gammab_w)^\dagger} = \frac{1}{n - d_w -1} \Ib_{d_w}.
        \end{align*}
        Then, we have
        \begin{align*}
            \E_{\wt\Scal}\sbr{\rbr{\Sigmab_w^{-1/2} \wt\Phib_w^\top \wt\Phib_w \Sigmab_w^{-1/2}}^\dagger}
            = \frac{1}{n - d_w -1} \Vb_w \Vb_w^\top.
        \end{align*}
        Therefore, \eqref{eq:pf_var_w2s} implies
        \begin{align}\label{eq:pf_var_w2s_1}
        \begin{split}
            \vari(f_\wts) &= \frac{\sigma^2}{N}\ \frac{1}{n - d_w -1}\ \tr\rbr{\Vb_w^\top \E_{\Scal_x}\sbr{\Sigmab_w^{-1/2} \Phib_w^\top \Pb_s \Phib_w \Sigmab_w^{-1/2}} \Vb_w} \\
            &= \frac{\sigma^2}{N}\ \frac{1}{n - d_w -1}\ \tr\rbr{\E_{\Scal_x}\sbr{\Vb_w^\top \Vb_w \Gammab_w^\top \Pb_s \Gammab_w \Vb_w^\top \Vb_w}} \\
            &= \frac{\sigma^2}{N}\ \frac{1}{n - d_w -1}\ \tr\rbr{\E_{\Scal_x}\sbr{\Gammab_w^\top \Pb_s \Gammab_w}}.
        \end{split}
        \end{align}
        Recall that $\Pb_s = \Phib_s \Phib_s^\dagger$. Let $\Gammab_s = \Gammab \Vb_s \in \R^{N \times d_s}$, and we can write
        \begin{align*}
            \Pb_s = (\Phib_s \Sigmab_s^{-1/2}) (\Phib_s \Sigmab_s^{-1/2})^\dagger = (\Gammab_s \Vb_s^\top) (\Gammab_s \Vb_s^\top)^\dagger = \Gammab_s \Gammab_s^\dagger.
        \end{align*}
        Therefore, with $\Gammab_w = \Gammab \Vb_w$ and $\Gammab_s = \Gammab \Vb_s$, we can decompose
        \begin{align*}
            \tr\rbr{\E_{\Scal_x}\sbr{\Gammab_w^\top \Pb_s \Gammab_w}} 
            &= \E_{\Scal_x}\sbr{\tr\rbr{\Gammab_w^\top \Gammab_s \Gammab_s^\dagger \Gammab_w}} \\
            &= \E_{\Scal_x}\sbr{\tr\rbr{\Vb_w^\top \Vb_s \Vb_s^\top \Vb_w \Gammab_w^\top \Gammab_s \Gammab_s^\dagger \Gammab_w}} \\
            &+ \E_{\Scal_x}\sbr{\tr\rbr{\Vb_w^\top (\Ib_d - \Vb_s \Vb_s^\top) \Vb_w \Gammab_w^\top \Gammab_s \Gammab_s^\dagger \Gammab_w}}.
        \end{align*}
        For the first term, since $\Gammab_w \Vb_w^\top \Vb_s = \Gammab \Vb_w \Vb_w^\top \Vb_s$ and $\Gammab_s = \Gammab \Vb_s$, the range of $\Gammab_w \Vb_w^\top \Vb_s$ is a subspace of that of $\Gammab_s$ and therefore,
        \begin{align*}
            \E_{\Scal_x}\sbr{\tr\rbr{\Vb_w^\top \Vb_s \Vb_s^\top \Vb_w \Gammab_w^\top \Gammab_s \Gammab_s^\dagger \Gammab_w}} 
            &= \E_{\Scal_x}\sbr{\tr\rbr{ \Vb_s^\top \Vb_w \Gammab_w^\top \Gammab_s \Gammab_s^\dagger \Gammab_w \Vb_w^\top \Vb_s}} \\
            &= \E_{\Scal_x}\sbr{\tr\rbr{ \Vb_s^\top \Vb_w \Gammab_w^\top \Gammab_w \Vb_w^\top \Vb_s}} \\
            &= \tr\rbr{\Vb_s^\top \Vb_w \E_{\Scal_x}\sbr{\Gammab_w^\top \Gammab_w} \Vb_w^\top \Vb_s}.
        \end{align*}
        Since $\E_{\Scal_x}\sbr{\Gammab_w^\top \Gammab_w} = N \Ib_{d_w}$, we have
        \begin{align*}
            \E_{\Scal_x}\sbr{\tr\rbr{\Vb_w^\top \Vb_s \Vb_s^\top \Vb_w \Gammab_w^\top \Gammab_s \Gammab_s^\dagger \Gammab_w}} 
            &= N \tr\rbr{\Vb_s^\top \Vb_w \Vb_w^\top \Vb_s} \\
            &= N \nbr{\Vb_s^\top \Vb_w}_F^2 \\
            &= N d_{s \wedge w}.
        \end{align*}
        For the second term, we first observe that the row space of $\Gammab_w \Vb_w^\top (\Ib_d - \Vb_s \Vb_s^\top)$ is orthogonal to that of $\Gammab_s = \Gammab \Vb_s$, and therefore, $\Gammab_w \Vb_w^\top (\Ib_d - \Vb_s \Vb_s^\top)$ and $\Gammab_s$ are independent, which implies
        \begin{align*}
            \E_{\Scal_x}\sbr{\tr\rbr{\Vb_w^\top (\Ib_d - \Vb_s \Vb_s^\top) \Vb_w \Gammab_w^\top \Gammab_s \Gammab_s^\dagger \Gammab_w}} 
            &= \tr\rbr{\E\sbr{\Gammab_w \Vb_w^\top (\Ib_d - \Vb_s \Vb_s^\top) \Vb_w \Gammab_w^\top} \E\sbr{\Gammab_s \Gammab_s^\dagger}}.
        \end{align*}
        Since $\Gammab$ consists of independent isotropic rows, so do $\Gammab_s = \Gammab \Vb_s \in \R^{N \times d_s}$ and $\Gammab_w = \Gammab \Vb_w \in \R^{N \times d_w}$, which implies
        \begin{align*}
            \E\sbr{\Gammab_s \Gammab_s^\dagger} = \frac{d_s}{N}\ \Ib_N \quad \t{and} \quad \E\sbr{\Gammab_w^\top \Gammab_w} = N\ \Ib_{d_w}.
        \end{align*}
        Then, we have
        \begin{align*}
            \E_{\Scal_x}\sbr{\tr\rbr{\Vb_w^\top (\Ib_d - \Vb_s \Vb_s^\top) \Vb_w \Gammab_w^\top \Gammab_s \Gammab_s^\dagger \Gammab_w}} 
            &= \tr\rbr{\E\sbr{\Gammab_w \Vb_w^\top (\Ib_d - \Vb_s \Vb_s^\top) \Vb_w \Gammab_w^\top} \E\sbr{\Gammab_s \Gammab_s^\dagger}} \\
            &= \frac{d_s}{N} \tr\rbr{\E\sbr{\Gammab_w \Vb_w^\top (\Ib_d - \Vb_s \Vb_s^\top) \Vb_w \Gammab_w^\top}} \\
            &= \frac{d_s}{N} \tr\rbr{\Vb_w^\top (\Ib_d - \Vb_s \Vb_s^\top) \Vb_w \E\sbr{\Gammab_w^\top \Gammab_w}} \\
            &= \frac{d_s}{N} N \tr\rbr{\Vb_w^\top (\Ib_d - \Vb_s \Vb_s^\top) \Vb_w} \\
            &= d_s (d_w - d_{s \wedge w}).
        \end{align*}
        Combining the two terms, we have
        \begin{align*}
            \tr\rbr{\E_{\Scal_x}\sbr{\Gammab_w^\top \Pb_s \Gammab_w}} = N d_{s \wedge w} + d_s (d_w - d_{s \wedge w}).
        \end{align*}
        Then, by \eqref{eq:pf_var_w2s_1}, the variance is exactly characterized by
        \begin{align*}
            \vari(f_\wts) 
            &= \frac{\sigma^2}{N}\ \frac{N d_{s \wedge w} + d_s (d_w - d_{s \wedge w})}{n - d_w -1} \\
            &= \frac{\sigma^2}{n-d_w-1} \rbr{d_{s \wedge w} + \frac{d_s}{N} (d_w - d_{s \wedge w})}.
        \end{align*}

        \item \b{Sub-gaussian features}: Relaxing the Gaussian feature assumption, when $\wt\Gammab_w$ consists of $\iid$ sub-gaussian random vectors that are zero-mean and isotropic (\ie $\E[\wt\gammab_i] = \b0_{d_w}$ and $\E[\wt\gammab_i \wt\gammab_i^\top] = \Ib_{d_w}$), with $n \ge \Omega(d_w)$, \Cref{lem:trace_inv_subgaussian} implies that
        \begin{align*}
            \E_{\wt\Scal}\sbr{(\wt\Gammab_w^\top \wt\Gammab_w)^\dagger} \aleq O\rbr{\frac{1}{n}} \Ib_{d_w},
        \end{align*}
        and therefore,
        \begin{align*}
            \E_{\wt\Scal}\sbr{\rbr{\Sigmab_w^{-1/2} \wt\Phib_w^\top \wt\Phib_w \Sigmab_w^{-1/2}}^\dagger} \aleq O\rbr{\frac{1}{n}} \Vb_w \Vb_w^\top.
        \end{align*}
        Then, via an analogous argument as \eqref{eq:pf_var_w2s_1}, \eqref{eq:pf_var_w2s} implies that 
        \begin{align}\label{eq:pf_var_w2s_2}
        \begin{split}
            \vari(f_\wts) \le \frac{\sigma^2}{N}\ O\rbr{\frac{1}{n}}\ \tr\rbr{\E_{\Scal_x}\sbr{\Gammab_w^\top \Pb_s \Gammab_w}}.
        \end{split}
        \end{align}
        We observe that in the analysis of the Gaussian feature case, the characterization
        \begin{align*}
            \tr\rbr{\E_{\Scal_x}\sbr{\Gammab_w^\top \Pb_s \Gammab_w}} = (N - d_s) d_{s \wedge w} + d_s d_w
        \end{align*}
        does not involve the Gaussianity of $\Gammab$ and therefore holds for general subgaussian features.
        This leads to an upper bound on the variance:
        \begin{align*}
            \vari(f_\wts) 
            &\le \frac{\sigma^2}{N}\ O\rbr{\frac{1}{n}}\ \rbr{N d_{s \wedge w} + d_s (d_w - d_{s \wedge w})} \\
            &\lesssim \frac{\sigma^2}{n} \rbr{d_{s \wedge w} + \frac{d_s}{N} (d_w - d_{s \wedge w})}.
        \end{align*}
    \end{enumerate}
\end{proof}


\begin{lemma}[Adapting \cite{vershynin2010introduction} Theorem 5.39]\label{lem:trace_inv_subgaussian}
    Let $\wt\Gammab_w = [\wt\gammab_1, \ldots, \wt\gammab_n]^\top$ be an $n \times d_w$ matrix whose rows $\wt\gammab_1, \ldots, \wt\gammab_n$ consist of $\iid$ sub-gaussian random vectors that are zero-mean and isotropic (\ie $\E[\wt\gammab_i] = \b0_{d_w}$ and $\E[\wt\gammab_i \wt\gammab_i^\top] = \Ib_{d_w}$). When $n \ge \Omega(d_w)$, we have
    \begin{align*}
        \E\sbr{\nbr{\rbr{\wt\Gammab_w^\top \wt\Gammab_w}^\dagger}_2} \le O\rbr{\frac{1}{n}},
    \end{align*}
    where $\Omega(\cdot)$ and $O(\cdot)$ suppresses constants that depend only on the sub-gaussian norm $\nbr{\wt\gammab_i}_{\psi_2} = \sup_{\vb \in \SSS^{d_w-1}} \sup_{p \ge 1} (\E[|\wt\gammab_i^\top \vb|^p])^{1/p} / \sqrt{p}$, independent of $n, d_w$.
\end{lemma}

\begin{proof}[Proof of \Cref{lem:trace_inv_subgaussian}]
    Let $\sigma_{\min}(\wt\Gammab_w^\top \wt\Gammab_w)$ be the smallest singular value of $\wt\Gammab_w^\top \wt\Gammab_w$.
    Leveraging \cite{vershynin2010introduction} Theorem 5.39, we notice that for $n \ge \Omega(d_w)$, there exist constants $c_1, c_2 > 0$ that depend only on the sub-gaussian norm $\nbr{\wt\gammab_i}_{\psi_2}$ such that
    \begin{align*}
        \Pr\sbr{\sigma_{\min}(\wt\Gammab_w^\top \wt\Gammab_w) < \rbr{\sqrt{n} - c_1\sqrt{d_w} - t}^2} \le \exp\rbr{-c_2 t^2}.
    \end{align*}
    Therefore, we have 
    \begin{align*}
        \Pr\sbr{\frac{1}{\sigma_{\min}(\wt\Gammab_w^\top \wt\Gammab_w)} > t} \le \exp\rbr{-c_2 \rbr{\sqrt{n} - c_1 \sqrt{d_w} - \sqrt{\frac{1}{t}}}^2}.
    \end{align*}

    Notice that for any non-negative random variable $Z$ with a cumulative density function $F_Z(z)$, 
    \begin{align*}
        \E\sbr{Z} &= \int_0^\infty z d F_Z(z) 
        = - \int_0^\infty z d \rbr{1 - F_Z(z)} \\
        &= \sbr{z \rbr{1 - F_Z(z)}}_0^\infty + \int_0^\infty \rbr{1 - F_Z(z)} dz \\
        &= \int_0^\infty \Pr\sbr{Z > z} dz.
    \end{align*}
    Therefore, we have
    \begin{align*}
        \E\sbr{\frac{1}{\sigma_{\min}(\wt\Gammab_w^\top \wt\Gammab_w)}} \le \int_0^\infty \exp\rbr{-c_2 \rbr{\sqrt{n} - c_1 \sqrt{d_w} - \sqrt{\frac{1}{t}}}^2} d t.
    \end{align*}
    Let $t_0 = 1 / \rbr{\sqrt{n} - c_1 \sqrt{d_w}}^2$ such that $\sqrt{n} - c_1 \sqrt{d_w} - \sqrt{\frac{1}{t}}=0$ and 
    \begin{align*}
        \int_{0}^{t_0} \exp\rbr{-c_2 \rbr{\sqrt{n} - c_1 \sqrt{d_w} - \sqrt{\frac{1}{t}}}^2} d t \le t_0
    \end{align*}
    Then, we have
    \begin{align*}
        &\E\sbr{\frac{1}{\sigma_{\min}(\wt\Gammab_w^\top \wt\Gammab_w)}} 
        \le \int_0^\infty \exp\rbr{-c_2 \rbr{\sqrt{n} - c_1 \sqrt{d_w} - \sqrt{\frac{1}{t}}}^2} d t \\
        &\le t_0 + \int_{t_0}^\infty \exp\rbr{-c_2 \rbr{\sqrt{n} - c_1 \sqrt{d_w} - \sqrt{\frac{1}{t}}}^2} d t \\
        &= t_0 + 2 \int_{0}^{\sqrt{n}-c_1\sqrt{d_w}} \exp\rbr{-c_2 u^2} \rbr{\sqrt{n} - c_1 \sqrt{d_w} - u}^{-3} d u \\
        &= t_0 + \frac{2}{\rbr{\sqrt{n} - c_1 \sqrt{d_w}}^2} \int_{0}^{1} \exp\rbr{-c_2 \rbr{\sqrt{n}-c_1\sqrt{d_w}}^2 u^2} \rbr{1 - u}^{-3} d u \\
        &= \frac{1}{\rbr{\sqrt{n} - c_1 \sqrt{d_w}}^2} + \frac{2}{\rbr{\sqrt{n} - c_1 \sqrt{d_w}}^2} \rbr{\int_{0}^{1} \exp\rbr{-\Omega\rbr{u^2}} \rbr{1 - u}^{-3} d u} \\
        &= O\rbr{\frac{1}{\rbr{\sqrt{n} - c_1 \sqrt{d_w}}^2}}.
    \end{align*}
    When $n \ge \Omega(d_w)$, we have $\sqrt{n} - c_1 \sqrt{d_w} \ge \Omega(\sqrt{n})$, and therefore ,
    \begin{align*}
        \E\sbr{\nbr{\rbr{\wt\Gammab_w^\top \wt\Gammab_w}^\dagger}_2}
        \le \E\sbr{\frac{1}{\sigma_{\min}(\wt\Gammab_w^\top \wt\Gammab_w)}} 
        \le O\rbr{\frac{1}{n}}.
    \end{align*}
\end{proof}





\subsection{Proof of \Cref{pro:sft_weak} and \Cref{cor:sft_strong}}\label{apx:pf_sft_weak}
\begin{proof}[Proof of \Cref{pro:sft_weak} and \Cref{cor:sft_strong}]
    The excess risk of the finetuned weak teacher $f_w(\xb) = \phi_w(\xb)^\top \thetab_w$ can be expressed as
    \begin{align*}
        \exrisk(f_w) &= \E_{\xb \sim \Dcal}\sbr{\E_{f_w}\sbr{(f_w(\xb) - f_*(\xb))^2}} \\
        &= \E_{\wt\Scal}\sbr{\frac{1}{n}\nbr{\wt\Phib_w \thetab_w - \wt\fb_*}_2^2},
    \end{align*}
    where $\wt\fb_* = [\fb_*(\wt\xb_1), \ldots, \fb_*(\wt\xb_n)]^\top \in \R^n$; and we recall that $\wt\Phib_w = [\phi_w(\wt\xb_1), \ldots, \phi_w(\wt\xb_n)]^\top$. Notice that the randomness of $\thetab_w$ comes from the SFT samples $\wt\Scal \sim \Dcal(f_*)^n$.

    Observe that the solution of \eqref{eq:sft_weak} as $\alpha_w \to 0$ is given by $\thetab_w = \wt\Phib_w^\dagger \wt\yb$, where $\wt\yb = \wt\fb_* + \wt\zb$ is the noisy label vector with $\wt\zb \sim \Ncal(\b0_n, \sigma^2 \Ib_n)$.
    Therefore, with the randomness over $\wt\Scal \sim \Dcal(f_*)^n$, we have
    \begin{align*}
        \exrisk(f_w) &= \E \sbr{\frac{1}{n}\nbr{\wt\Phib_w \wt\Phib_w^\dagger \wt\yb - \wt\fb_*}_2^2} \\
        &= \E \sbr{\frac{1}{n}\nbr{\wt\Phib_w \wt\Phib_w^\dagger \wt\zb + \rbr{\wt\Phib_w \wt\Phib_w^\dagger \wt\fb_* - \wt\fb_*}}_2^2} \\
        &= \underbrace{\E \sbr{\frac{1}{n}\nbr{\wt\Phib_w \wt\Phib_w^\dagger \wt\zb}_2^2}}_{\vari(f_w)} + \underbrace{\E\sbr{\frac{1}{n}\nbr{\wt\Phib_w \wt\Phib_w^\dagger \wt\fb_* - \wt\fb_*}_2^2}}_{\bias(f_w)}.
    \end{align*}
    
    For bias, by the definition of finetuning capacity (see \Cref{def:ft_est_err}), we have
    \begin{align*}
        \bias(f_w) = \frac{1}{n} \E\sbr{\nbr{\wt\Phib_w \wt\Phib_w^\dagger \wt\fb_* - \wt\fb_*}_2^2} = \frac{\rho_w(n)}{n}.
    \end{align*}
    We observe that $\bias(f_w) \le \rho_w$ by \Cref{lem:low_est_err_ft}.
    Notice that \Cref{lem:low_est_err_ft} also implies $\bias(f_s) = {\rho_s(n)}/{n} \le \rho_s$. 

    For variance, we observe that 
    \begin{align*}
        \vari(f_w) &= \frac{1}{n} \E\sbr{\nbr{\wt\Phib_w \wt\Phib_w^\dagger \wt\zb}_2^2} \\
        &= \frac{1}{n} \E\sbr{\tr\rbr{\wt\Phib_w \wt\Phib_w^\dagger \wt\zb \wt\zb^\top}} \\
        &= \frac{\sigma^2}{n} \E\sbr{\tr\rbr{\wt\Phib_w \wt\Phib_w^\dagger}}.
    \end{align*}
    By \Cref{asm:ft_data}, since $\rank(\wt\Phib_w) = d_w$ almost surely, we have
    \begin{align*}
        \vari(f_w) = \frac{\sigma^2}{n} \E\sbr{\tr\rbr{\wt\Phib_w \wt\Phib_w^\dagger}} = \frac{\sigma^2 d_w}{n}.
    \end{align*}
\end{proof}



\subsection{Proof of \Cref{cor:pgr}}\label{apx:pf_pgr}
\begin{proof}[Proof of \Cref{cor:pgr}]
    Noticing that with $\rank(\wt\Phib_w) = d_w$ and $\rank(\wt\Phib_s) = \rank(\Phib_s) = d_s$ almost surely, the excess risks of $f_w, f_s, f_c$ are characterized exactly in \Cref{pro:sft_weak} and \Cref{cor:sft_strong}, and $\exrisk(f_\wts)$ is upper bounded by \Cref{thm:w2s_ft}.
    Therefore, by directly plugging in the excess risks to the definitions of PGR and OPR, we have
    \begin{align}\label{eq:pgr_lower_tight}
    \begin{split}
        \pgr = &\frac{\exrisk(f_w) - \exrisk(f_\wts)}{\exrisk(f_w) - \exrisk(f_c)} \\
        \ge &\rbr{\sigma^2\ \frac{d_w}{n} + \frac{\rho_w(n)}{n} - \frac{\sigma^2}{n-d_w-1} \rbr{d_{s \wedge w} + \frac{d_s}{N} (d_w-d_{s \wedge w})} - \rbr{\frac{\rho_w(n)}{n} + \frac{\rho_s(N)}{N}}} \\
        &\rbr{\sigma^2\ \frac{d_w}{n} + \frac{\rho_w(n)}{n} - \sigma^2\ \frac{d_s}{N+n} - \frac{\rho_s(N+n)}{N+n}}^{-1} \\
        \ge &\rbr{\sigma^2 \frac{d_w}{n} - \sigma^2 \frac{d_{s \wedge w} + (d_w - d_{s \wedge w}) {d_s}/{N}}{n-d_w-1} - \frac{\rho_s(N)}{N}} \Big/ \rbr{\sigma^2 \frac{d_w}{n} + \frac{\rho_w(n)}{n}}, \\
        \ge &\rbr{\sigma^2 \frac{d_w}{n} - \sigma^2 \frac{d_{s \wedge w} + (d_w - d_{s \wedge w}) {d_s}/{N}}{n-d_w-1} - \rho_s} \Big/ \rbr{\sigma^2 \frac{d_w}{n} + \rho_w},
    \end{split}
    \end{align}
    and 
    \begin{align}\label{eq:opr_lower_tight}
    \begin{split}
        \opr = &\frac{\exrisk(f_s)}{\exrisk(f_\wts)} \\
        \ge &\rbr{\sigma^2\ \frac{d_s}{n} + \frac{\rho_s(n)}{n}} \Big/ \rbr{\sigma^2 \frac{d_{s \wedge w} + (d_w - d_{s \wedge w}) {d_s}/{N}}{n-d_w-1} + \rbr{\frac{\rho_w(n)}{n} + \frac{\rho_s(N)}{N}}} \\
        \ge &\sigma^2 \frac{d_s}{n} \Big/ \rbr{\sigma^2 \frac{d_{s \wedge w} + (d_w - d_{s \wedge w}) {d_s}/{N}}{n-d_w-1} + \rho_w + \rho_s}.
    \end{split}
    \end{align} 

    When taking $n = d_w + q + 1$ for some small constant $q \in \N$, we observe that 
    \begin{align*}
        \pgr &\ge \rbr{\sigma^2 \frac{d_w}{n} - \sigma^2 \frac{d_{s \wedge w} + (d_w - d_{s \wedge w}) {d_s}/{N}}{n-d_w-1} - \rho_s} \Big/ \rbr{\sigma^2 \frac{d_w}{n} + \rho_w} \\
        &\ge \rbr{\frac{d_w}{d_w + q + 1} - \frac{d_{s \wedge w}}{q} - \frac{d_s}{N} \frac{d_w - d_{s \wedge w}}{q} - \frac{\rho_s}{\sigma^2}} \Big/ \rbr{\frac{d_w}{d_w + q + 1} + \frac{\rho_w}{\sigma^2}} \\
        &\ge \rbr{\frac{d_w}{d_w + q + 1} - \frac{d_{s \wedge w}}{q} - \frac{d_s}{N} \frac{d_w - d_{s \wedge w}}{q} - \frac{\rho_s}{\sigma^2} - \frac{\rho_w}{\sigma^2}} \Big/ \rbr{\frac{d_w}{d_w + q + 1} + \frac{\rho_w}{\sigma^2} - \frac{\rho_w}{\sigma^2}} \\
        &= 1 - \frac{n}{d_w} \rbr{\frac{d_{s \wedge w}}{q} + \frac{d_s}{N} \frac{d_w - d_{s \wedge w}}{q} + \frac{\rho_w + \rho_s}{\sigma^2}} \\
        &= 1 - \frac{n}{q}\ {\frac{d_{s \wedge w} + (d_w - d_{s \wedge w}) d_s / N}{d_w}} - \frac{n}{d_w}\ {\frac{\rho_w + \rho_s}{\sigma^2}},
    \end{align*}
    and
    \begin{align*}
        \opr &\ge \sigma^2 \frac{d_s}{n} \Big/ \rbr{\sigma^2 \frac{d_{s \wedge w} + (d_w - d_{s \wedge w}) {d_s}/{N}}{n-d_w-1} + \rho_w + \rho_s} \\
        &= \frac{d_s}{n} \Big/ \rbr{\frac{d_{s \wedge w} + (d_w - d_{s \wedge w}) {d_s}/{N}}{q} + \frac{\rho_w + \rho_s}{\sigma^2}} \\
        &= \rbr{\frac{n}{q}\ \frac{d_{s \wedge w} + (d_w - d_{s \wedge w}) {d_s}/{N}}{d_s} + \frac{n}{d_s}\ \frac{\rho_w + \rho_s}{\sigma^2}}^{-1}.
    \end{align*}
\end{proof}



\subsection{Proof of \Cref{cor:non_monotonic_scaling}}\label{apx:pf_non_monotonic_scaling}
\begin{proof}[Proof of \Cref{cor:non_monotonic_scaling}]
    Recall the notations introduced for conciseness:
    \begin{align*}
        d_\wts(N) = d_{s \wedge w} + (d_w - d_{s \wedge w}) \frac{d_s}{N}, \quad \varrho = \frac{\rho_w + \rho_s}{\sigma^2}.
    \end{align*}
    Then, the lower bounds for $\pgr$ and $\opr$ in \Cref{cor:pgr} can be expressed in terms of $d_\wts(N)$ and $\varrho$ as 
    \begin{align*}
        \pgr \ge 1 - \frac{d_\wts(N)}{d_w} - \frac{d_w + 1}{d_w} \varrho - \frac{d_w + 1}{d_w}\ \frac{d_\wts(N)}{q} - q \frac{\varrho}{d_w},
    \end{align*}
    and 
    \begin{align*}
        \opr \ge \rbr{\frac{d_\wts(N)}{d_s} + \frac{d_w + 1}{d_s} \varrho + \frac{d_\wts(N)}{d_s}\ \frac{d_w + 1}{q} + q \frac{\varrho}{d_s}}^{-1}.
    \end{align*}
    Both lower bounds are maximized when the last two terms in the expressions that involve $q$ are minimized, which is achieved when $q = \sqrt{\rbr{d_w + 1} {d_\wts(N)}/{\varrho}}$. Substituting the optimal $q$ back into the expressions yields the best lower bounds for $\pgr$ and $\opr$:
    \begin{align*}
        \pgr \ge &1 - \frac{d_\wts(N)}{d_w} - \varrho \frac{d_w + 1}{d_w} - 2 \sqrt{\varrho \frac{d_w + 1}{d_w}\ \frac{d_\wts(N)}{d_w}} \\
        = &1 - \rbr{\sqrt{\frac{d_\wts(N)}{d_w}} + \sqrt{\varrho\ \frac{d_w + 1}{d_w}}}^2,
    \end{align*}
    and 
    \begin{align*}
        \opr \ge &\rbr{\frac{d_\wts(N)}{d_s} + \varrho \frac{d_w + 1}{d_s} + 2 \sqrt{\varrho \frac{d_w + 1}{d_s}\ \frac{d_\wts(N)}{d_s}}}^{-1} \\
        = &\rbr{\sqrt{\frac{d_\wts(N)}{d_s}} + \sqrt{\varrho\ \frac{d_w + 1}{d_s}}}^{-2}.
    \end{align*}
\end{proof}




\section{Ridge regression analysis}\label{apx:ridge_regression}
In this section, we investigate the more realistic scenario where the weak and strong feature covariances are not exactly low-rank but admit small numbers of dominating eigenvalues. 

Concretely, we consider the same data distribution $(\xb, y) \sim \Dcal(f_*)$ with $y = f_*(\xb) + z$ for some independent Gaussian label noise $z \sim \Ncal(0, \sigma^2)$ and an unknown ground truth function $f_*: \Xcal \to \R$ as in \Cref{sec:ridgeless_regression}.
Under the same sub-gaussian feature assumption as in \Cref{asm:features}, we adapt \Cref{def:low_intrinsic_dim,def:correlation_dim} to the ridge regression setting as follows.
\begin{assumption}[Data distribution]\label{asm:ridge_regression}
    Let $\phi_s: \Xcal \to \R^d$ and $\phi_w: \Xcal \to \R^d$ be the strong and weak pretrained models that take $\xb \sim \Dcal$ and output pretrained features $\phi_s(\xb), \phi_w(\xb) \in \R^d$, respectively.
    \begin{enumerate}[label=(\roman*)]
        \item \b{Ground truth}: Assume $f_*$ can be expressed as a linear function over an unknown ground truth feature $\phi_*: \Xcal \to \R^d$ such that $f_*(\cdot) = \phi_*(\cdot)^\top \thetab_*$ for some fixed $\thetab_* \in \R^d$.
        \item \b{Sub-gaussian features} (\Cref{asm:features}): Let $\phi_w(\xb)$, $\phi_s(\xb)$, $\phi_*(\xb)$ be zero-mean sub-gaussian random vectors with $\E[\phi_w(\xb)] = \E[\phi_s(\xb)] = \E[\phi_*(\xb)] = \b{0}_d$, and 
        \begin{align*}
            \E[\phi_w(\xb) \phi_w(\xb)^\top] = \Sigmab_w, \quad \E[\phi_s(\xb) \phi_s(\xb)^\top] = \Sigmab_s, \quad \E[\phi_*(\xb) \phi_*(\xb)^\top] = \Sigmab_*.
        \end{align*}
        For conciseness, we assume without loss of generality that these features are roughly normalized, \ie, $\nbr{\Sigmab_w}_2 \asymp 1$, $\nbr{\Sigmab_s}_2 \asymp 1$, and $\nbr{\Sigmab_*}_2 \asymp 1$.
        \item \b{Low intrinsic dimension}: Let $\Sigmab_s$ and $\Sigmab_w$ both be \b{positive-definite} with spectral decompositions $\Sigmab_s = \Vb_s \Lambdab_s \Vb_s^\top$ and $\Sigmab_w = \Vb_w \Lambdab_w \Vb_w^\top$, where $\Lambdab_s, \Lambdab_w \in \R^{d \times d}$ are diagonal matrices with positive eigenvalues in decreasing order; while $\Vb_s \in \R^{d \times d}$ and $\Vb_w \in \R^{d \times d}$ are orthogonal matrices consisting of the corresponding orthonormal eigenvectors. The low intrinsic dimension of FT implies that $\Lambdab_s = \diag(\lambda^s_1,\cdots,\lambda^s_d)$ and $\Lambdab_w = \diag(\lambda^w_1,\cdots,\lambda^w_d)$ consist of a few dominating eigenvalues, while the rest of the eigenvalues are negligible, \ie, there exist $d_s, d_w \ll d$ such that $\sum_{i > d_s} \lambda^s_i \ll \tr(\Sigmab_s)$ and $\sum_{i > d_w} \lambda^w_i \ll \tr(\Sigmab_w)$. Here, 
        \begin{align*}
            \tr(\Sigmab_s) \lesssim d_s \quad \t{and} \quad \tr(\Sigmab_w) \lesssim d_w
        \end{align*}
        effectively measure the intrinsic dimensions of $\phi_s$ and $\phi_w$.
    \end{enumerate}
\end{assumption}

\begin{remark}[Weak-strong similarity]
    In place of correlation dimension (\Cref{def:correlation_dim}) in the ridgeless setting, for the ridge regression analysis, we measure the similarity between the weak and strong models directly through $\tr(\Sigmab_s \Sigmab_w)$. Notice that 
    \begin{align*}
        \tr(\Sigmab_s \Sigmab_w) \le \min\cbr{\tr(\Sigmab_s)\nbr{\Sigmab_w}_2, \tr(\Sigmab_w)\nbr{\Sigmab_s}_2} \lesssim \min\cbr{\tr(\Sigmab_s), \tr(\Sigmab_w)}.
    \end{align*}
    In particular, when $\Sigmab_s$ and $\Sigmab_w$ admit low intrinsic dimensions, $\tr(\Sigmab_s \Sigmab_w)$ can be much smaller than $\min\cbr{\tr(\Sigmab_s), \tr(\Sigmab_w)}$ if their eigenvectors corresponding to the dominating eigenvalues are almost orthogonal.
\end{remark}

\begin{remark}[FT approximation errors]
    It is worth noting that under the ground truth and positive-definite covariance assumptions in \Cref{asm:ridge_regression}(i, iii), the FT approximation errors in \Cref{def:ft_est_err} satisfy
    \begin{align}\label{eq:pf_ridge_ft_approx_err}
    \begin{split}
        &\rho_s = \min_{\thetab \in \R^d} \E_{\xb \sim \Dcal}\sbr{(\phi_s(\xb)^\top \thetab - f_*(\xb))^2} = 0 \quad (\t{when } \thetab = \Sigmab_s^{-1} \Sigmab_* \thetab_*), \\
        &\rho_w = \min_{\thetab \in \R^d} \E_{\xb \sim \Dcal}\sbr{(\phi_w(\xb)^\top \thetab - f_*(\xb))^2} = 0 \quad (\t{when } \thetab = \Sigmab_w^{-1} \Sigmab_* \thetab_*).
    \end{split}
    \end{align}
    In place of \Cref{def:ft_est_err}, with positive-definite covariances in \Cref{asm:ridge_regression}, we measure the alignment between the ground truth feature $\phi_*$ and the weak/strong feature $\phi_w, \phi_s$ through
    \begin{align*}
        \varrho_s = \|\Sigmab_s^{-1/2} \Sigmab_*^{1/2} \thetab_*\|_2^2, \quad \varrho_w = \|\Sigmab_w^{-1/2} \Sigmab_*^{1/2} \thetab_*\|_2^2.
    \end{align*}
    Intuitively, for $\Sigmab_s$ and $\Sigmab_w$ with a few dominating eigenvalues (\Cref{asm:ridge_regression}(iii)), $\varrho_s$ and $\varrho_w$ are small if the eigensubspace associated with non-negligible eigenvalues of $\Sigmab_*$ is fully covered by the eigensubspaces associated with the dominating eigenvalues of $\Sigmab_s$ and $\Sigmab_w$, respectively. 
\end{remark}

The W2S FT under ridge regression consists of two steps.
\begin{enumerate}[label=(\alph*)]
    \item First, the weak teacher $f_w(\xb) = \phi_w(\xb)^\top \thetab_w$ is supervisedly finetuned over $\wt\Scal$: 
    \begin{align}\label{eq:w2s_weak_ridge}
        \thetab_w = \argmin_{\thetab \in \R^d} \frac{1}{n}\nbr{\wt\Phib_w \thetab - \wt\yb}_2^2 + \alpha_w \nbr{\thetab}_2^2, \quad \alpha_w > 0.
    \end{align}
    \item Then, the W2S model $f_\wts(\xb) = \phi_s(\xb)^\top \thetab_\wts$ is finetuned over the strong feature $\phi_s$ through the unlabeled samples $\Scal_x$ and their pseudo-labels generated by the weak teacher model:
    \begin{align}\label{eq:w2s_strong_ridge}
        \thetab_\wts = \argmin_{\thetab \in \R^d} \frac{1}{N}\nbr{\Phib_s \thetab - \Phib_w \thetab_w}_2^2 + \alpha_\wts \nbr{\thetab}_2^2, \quad \alpha_\wts > 0.
    \end{align}
\end{enumerate}

\begin{theorem}[W2S under ridge regression]\label{thm:w2s_ridge}
    Let $\varrho_w = \nbr{\Sigmab_w^{-1/2} \Sigmab_*^{1/2} \thetab_*}_2^2$ and $\varrho_s = \nbr{\Sigmab_s^{-1/2} \Sigmab_*^{1/2} \thetab_*}_2^2$.
    Under \Cref{asm:ridge_regression}, the generalization error of W2S FT via ridge regression with fixed $\alpha_w, \alpha_\wts > 0$, $\exrisk(f_\wts) = \vari(f_\wts) + \bias(f_\wts)$, is upper bounded by
    \begin{align*}
        \vari(f_\wts) \le \frac{\sigma^2 \tr\rbr{\Sigmab_s \Sigmab_w}}{4 (\alpha_w n) (\alpha_\wts N)}, \quad
        \bias(f_\wts) \le \alpha_w \varrho_w + \alpha_\wts \varrho_s.
    \end{align*}
    In particular, when taking  
    \begin{align*}
        \alpha_w = \rbr{\frac{\sigma^2 \tr\rbr{\Sigmab_s \Sigmab_w}}{4 n N}\ \frac{\varrho_s}{\varrho_w^2}}^{1/3}, \quad 
        \alpha_\wts = \rbr{\frac{\sigma^2 \tr\rbr{\Sigmab_s \Sigmab_w}}{4 n N}\ \frac{\varrho_w}{\varrho_s^2}}^{1/3},
    \end{align*}
    the excess risk of W2S FT is upper bounded by
    \begin{align*}
        \exrisk(f_\wts) \le 3 \rbr{\frac{\sigma^2 \tr\rbr{\Sigmab_s \Sigmab_w}}{4 n N}\ \varrho_s \varrho_w}^{1/3}.
    \end{align*}
\end{theorem}

\Cref{thm:w2s_ridge} conveys a similar high-level intuition as in \Cref{thm:w2s_ft} regarding the effect of the weak-strong similarity on the generalization error of W2S FT. In particular, the larger discrepancy between $\phi_s$ and $\phi_w$ (corresponding to the smaller $\tr\rbr{\Sigmab_s \Sigmab_w}$) leads to lower variance and therefore better W2S generalization.

Meanwhile, a key difference in W2S between the ridge and ridgeless settings (\Cref{thm:w2s_ridge} versus \Cref{thm:w2s_ft}) is that the FT approximation errors in \Cref{thm:w2s_ridge}, reflected by $\varrho_s = \|\Sigmab_s^{-1/2} \Sigmab_*^{1/2} \thetab_*\|_2^2$ and $\varrho_w = \|\Sigmab_w^{-1/2} \Sigmab_*^{1/2} \thetab_*\|_2^2$, can be compensated by larger sample sizes $n, N$ and directly affect the sample complexity: 
\begin{align*}
    n N \asymp \sigma^2 \tr\rbr{\Sigmab_s \Sigmab_w} \varrho_s \varrho_w.
\end{align*}
Such difference is a result of optimizing the regularization hyperparameters $\alpha_w, \alpha_\wts$ in ridge regression that control the variance-bias tradeoff.

\begin{proof}[Proof of \Cref{thm:w2s_ridge}]
    We first formalize some useful facts on the features and labels as in \eqref{eq:pf_var_w2s_subgaussian_asm}.
    In particular, the sub-gaussian assumption in \Cref{asm:ridge_regression}(ii) implies that for each $\xb \sim \Dcal$, the corresponding strong/weak feature $\phi_s(\xb), \phi_w(\xb) \in \R^d$, and the ground truth $f_*(\xb) \in \R$ are simultaneously characterized by an independent subgaussian random vector $\gammab \in \R^d$ with $\E[\gammab] = \b0_{d}$ and $\E[\gammab \gammab^\top] = \Ib_{d}$, \ie,
    \begin{align*}
        \phi_s(\xb) = \Sigmab_s^{1/2} \gammab, \quad \phi_w(\xb) = \Sigmab_w^{1/2} \gammab, \quad f_*(\xb) = \phi_*(\xb)^\top \thetab_* = \gammab^\top \Sigmab_*^{1/2} \thetab_*.
    \end{align*}

    Then, for $\Scal$ and $\wt\Scal$, there exist independent random matrices $\Gammab = [\gammab_1, \ldots, \gammab_N]^\top \in \R^{N \times d}$ and $\wt\Gammab = [\wt\gammab_1, \ldots, \wt\gammab_n]^\top \in \R^{n \times d}$ consisting of $\iid$ zero-mean isotropic rows such that
    \begin{align}\label{eq:pf_var_w2s_subgaussian_asm_2}
    \begin{split}
        &\Phib_s = \Gammab \Sigmab_s^{1/2} = \Gammab_s \Lambdab_s^{1/2} \Vb_s^\top, \\
        &\Phib_w = \Gammab \Sigmab_w^{1/2} = \Gammab_w \Lambdab_w^{1/2} \Vb_w^\top, \\
        &\yb = \fb_* + \zb, \quad \fb_* = \Gammab \Sigmab_*^{1/2} \thetab_*, \quad \zb \sim \Ncal(\b0_N, \sigma^2 \Ib_N), \\
        &\wt\Phib_w = \wt\Gammab \Sigmab_w^{1/2} = \wt\Gammab_w \Lambdab_w^{1/2} \Vb_w^\top, \\
        &\wt\yb = \wt\fb_* + \wt\zb, \quad \wt\fb_* = \wt\Gammab \Sigmab_*^{1/2} \thetab_*, \quad \wt\zb \sim \Ncal(\b0_n, \sigma^2 \Ib_n),
    \end{split}
    \end{align}
    where $\Gammab_s = \Gammab \Vb_s$, $\Gammab_w = \Gammab \Vb_w$, and $\wt\Gammab_w = \wt\Gammab \Vb_w$.

    \paragraph{Variance-bias decomposition.}
    Recall that the excess risk of W2S generalization $\exrisk(f_\wts)$ can be decomposed into the variance and bias terms:
    \begin{align*}
        &\vari(f_\wts) = \E_{\xb \sim \Dcal}\sbr{\E_{\Scal_x, \wt\Scal}\sbr{(f_\wts(\xb) - \E_{\Scal_x, \wt\Scal}[f_\wts(\xb)])^2}}, \\
        &\bias(f) = \E_{\xb \sim \Dcal}\sbr{(\E_{\Scal_x, \wt\Scal}[f_\wts(\xb)] - f_*(\xb))^2}.
    \end{align*}
    With $\alpha_w > 0$, \eqref{eq:w2s_weak_ridge} yields a weak teacher model $f_w(\xb) = \phi_w(\xb)^\top \thetab_w$ with 
    \begin{align*}
        \thetab_w = \rbr{\wt\Phib_w^\top \wt\Phib_w + \alpha_w n \Ib_d}^{-1} \wt\Phib_w^\top \rbr{\wt\fb_8 + \wt\zb}.
    \end{align*}
    Then, the W2S model $f_\wts(\xb) = \phi_s(\xb)^\top \thetab_\wts$ is given by \eqref{eq:w2s_strong_ridge} with $\alpha_\wts > 0$:
    \begin{align*}
        \thetab_\wts = &\rbr{\Phib_s^\top \Phib_s + \alpha_\wts N \Ib_d}^{-1} \Phib_s^\top \Phib_w \thetab_w \\
        = &\rbr{\Phib_s^\top \Phib_s + \alpha_\wts N \Ib_d}^{-1} \Phib_s^\top \Phib_w \rbr{\wt\Phib_w^\top \wt\Phib_w + \alpha_w n \Ib_d}^{-1} \wt\Phib_w^\top \rbr{\wt\fb_* + \wt\zb},
    \end{align*}
    which implies
    \begin{align*}
        \E_{\Scal_x, \wt\Scal}[\thetab_\wts] = \rbr{\Phib_s^\top \Phib_s + \alpha_\wts N \Ib_d}^{-1} \Phib_s^\top \Phib_w \rbr{\wt\Phib_w^\top \wt\Phib_w + \alpha_w n \Ib_d}^{-1} \wt\Phib_w^\top \wt\fb_*.
    \end{align*}
    Then, we can concretize the variance and bias terms as:
    \begin{align}\label{eq:pf_ridge_var}
    \begin{split}
        &\vari(f_\wts) = \E_{\xb \sim \Dcal}\sbr{\E_{\Scal_x, \wt\Scal}\sbr{(f_\wts(\xb) - \E_{\Scal_x, \wt\Scal}[f_\wts(\xb)])^2}} \\
        = &\E_{\Scal_x, \wt\Scal}\sbr{\nbr{\Sigmab_s^{1/2} \rbr{\Phib_s^\top \Phib_s + \alpha_\wts N \Ib_d}^{-1} \Phib_s^\top \Phib_w \rbr{\wt\Phib_w^\top \wt\Phib_w + \alpha_w n \Ib_d}^{-1} \wt\Phib_w^\top \wt\zb}_2^2},
    \end{split}
    \end{align}
    and
    \begin{align}\label{eq:pf_ridge_bias}
    \begin{split}
        &\bias(f_\wts) = \E_{\xb \sim \Dcal}\sbr{(\E_{\Scal_x, \wt\Scal}[f_\wts(\xb)] - f_*(\xb))^2} \\
        = &\E_{\Scal_x, \wt\Scal}\sbr{\frac{1}{N} \nbr{\Phib_s \rbr{\Phib_s^\top \Phib_s + \alpha_\wts N \Ib_d}^{-1} \Phib_s^\top \Phib_w \rbr{\wt\Phib_w^\top \wt\Phib_w + \alpha_w n \Ib_d}^{-1} \wt\Phib_w^\top \wt\fb_* - \fb_*}_2^2}.
    \end{split}
    \end{align}
    Now, we are ready to upper bound the variance and bias terms separately.

    \paragraph{Variance.}
    Denote $\zetab = \Lambdab_w^{1/2} \Vb_w^\top \rbr{\wt\Phib_w^\top \wt\Phib_w + \alpha_w n \Ib_d}^{-1} \wt\Phib_w^\top \wt\zb \in \R^d$, whose randomness comes from $\wt\Scal$ only, independent of $\Scal_x$.
    Then, the variance term \eqref{eq:pf_ridge_var} can be expressed as
    \begin{align*}
        &\vari(f_\wts) = \E_{\Scal_x, \wt\Scal}\sbr{\nbr{\Sigmab_s^{1/2} \rbr{\Phib_s^\top \Phib_s + \alpha_\wts N \Ib_d}^{-1} \Phib_s^\top \Phib_w \zetab}_2^2} \\
        = &\tr\rbr{\E_{\Scal_s}\rbr{\Gammab_w^\top \Phib_s \rbr{\Phib_s^\top \Phib_s + \alpha_\wts N \Ib_d}^{-1} \Sigmab_s \rbr{\Phib_s^\top \Phib_s + \alpha_\wts N \Ib_d}^{-1} \Phib_s^\top \Gammab_w} \E_{\wt\Scal}\sbr{\zetab \zetab^\top}} \\
        = &\tr\rbr{\E_{\Scal_s}\rbr{\Gammab_w^\top \Gammab_s \rbr{\Gammab_s^\top \Gammab_s + \alpha_\wts N \Lambdab_s^{-1}}^{-1} \rbr{\Gammab_s^\top \Gammab_s + \alpha_\wts N \Lambdab_s^{-1}}^{-1} \Gammab_s^\top \Gammab_w} \E_{\wt\Scal}\sbr{\zetab \zetab^\top}} \\
        = &\tr\rbr{\E_{\Scal_s}\rbr{\Vb_w^\top \Gammab^\top \Gammab \rbr{\Gammab^\top \Gammab + \alpha_\wts N \Sigmab_s^{-1}}^{-2} \Gammab^\top \Gammab \Vb_w} \E_{\wt\Scal}\sbr{\zetab \zetab^\top}} \\
        = &\tr\rbr{\E_{\Scal_s}\rbr{\Gammab^\top \Gammab \rbr{\Gammab^\top \Gammab + \alpha_\wts N \Sigmab_s^{-1}}^{-2} \Gammab^\top \Gammab} \E_{\wt\Scal}\sbr{\Vb_w \zetab \zetab^\top \Vb_w^\top}}.
    \end{align*}
    Notice that $\rbr{\Gammab^\top \Gammab + \alpha_\wts N \Sigmab_s^{-1}}^{2} \succeq \alpha_\wts N \rbr{\Gammab^\top \Gammab \Sigmab_s^{-1} + \Sigmab_s^{-1} \Gammab^\top \Gammab}$.
    Since matrix inversion is convex, a Jensen-type inequality implies that
    \begin{align*}
        &\Gammab^\top \Gammab \rbr{\Gammab^\top \Gammab + \alpha_\wts N \Sigmab_s^{-1}}^{-2} \Gammab^\top \Gammab \\
        \preceq &\Gammab^\top \Gammab \rbr{\alpha_\wts N \rbr{\Gammab^\top \Gammab \Sigmab_s^{-1} + \Sigmab_s^{-1} \Gammab^\top \Gammab}}^{\dagger} \Gammab^\top \Gammab \\
        = &\frac{1}{2 \alpha_\wts N} \Gammab^\top \Gammab \rbr{\frac{1}{2} \rbr{\Gammab^\top \Gammab \Sigmab_s^{-1} + \Sigmab_s^{-1} \Gammab^\top \Gammab}}^{\dagger} \Gammab^\top \Gammab \\
        \preceq &\frac{1}{4 \alpha_\wts N} \rbr{\Gammab^\top \Gammab \Sigmab_s + \Sigmab_s \Gammab^\top \Gammab}.
    \end{align*}
    Therefore, 
    \begin{align*}
        \E_{\Scal_s}\rbr{\Gammab^\top \Gammab \rbr{\Gammab^\top \Gammab + \alpha_\wts N \Sigmab_s^{-1}}^{-2} \Gammab^\top \Gammab}
        \preceq &\frac{1}{4 \alpha_\wts N} \E_{\Scal_s}\sbr{\Gammab^\top \Gammab \Sigmab_s + \Sigmab_s \Gammab^\top \Gammab} 
        = \frac{1}{2 \alpha_\wts N} \Sigmab_s.
    \end{align*}
    Meanwhile, we observe that
    \begin{align*}
        \E_{\wt\Scal}\sbr{\Vb_w \zetab \zetab^\top \Vb_w^\top} 
        = &\E_{\wt\Scal}\sbr{\Sigmab_w^{1/2} \rbr{\wt\Phib_w^\top \wt\Phib_w + \alpha_w n \Ib_d}^{-1} \wt\Phib_w^\top \wt\zb \wt\zb^\top \wt\Phib_w \rbr{\wt\Phib_w^\top \wt\Phib_w + \alpha_w n \Ib_d}^{-1} \Sigmab_w^{1/2}} \\
        = &\sigma^2 \E_{\wt\Scal}\sbr{\Sigmab_w^{1/2} \rbr{\wt\Phib_w^\top \wt\Phib_w + \alpha_w n \Ib_d}^{-1} \wt\Phib_w^\top \wt\Phib_w \rbr{\wt\Phib_w^\top \wt\Phib_w + \alpha_w n \Ib_d}^{-1} \Sigmab_w^{1/2}},
    \end{align*}
    where 
    \begin{align*}
        \rbr{\wt\Phib_w^\top \wt\Phib_w + \alpha_w n \Ib_d}^{-1} \wt\Phib_w^\top \wt\Phib_w \rbr{\wt\Phib_w^\top \wt\Phib_w + \alpha_w n \Ib_d}^{-1}
        \preceq &\frac{1}{2 \alpha_w n} \Ib_d.
    \end{align*}
    Therefore, we have
    \begin{align*}
        \E_{\wt\Scal}\sbr{\Vb_w \zetab \zetab^\top \Vb_w^\top} 
        \preceq &\sigma^2 \E_{\wt\Scal}\sbr{\Sigmab_w^{1/2} \rbr{\frac{1}{2 \alpha_w n} \Ib_d} \Sigmab_w^{1/2}}
        = \frac{\sigma^2}{2 \alpha_w n} \Sigmab_w.
    \end{align*}
    Overall, the variance of $f_\wts$ can be upper bounded as
    \begin{align}\label{eq:pf_ridge_var_ub}
    \begin{split}
        \vari(f_\wts) 
        = &\tr\rbr{\E_{\Scal_s}\rbr{\Gammab^\top \Gammab \rbr{\Gammab^\top \Gammab + \alpha_\wts N \Sigmab_s^{-1}}^{-2} \Gammab^\top \Gammab} \E_{\wt\Scal}\sbr{\Vb_w \zetab \zetab^\top \Vb_w^\top}} \\
        \le &\frac{\sigma^2 \tr\rbr{\Sigmab_s \Sigmab_w}}{4 (\alpha_w n) (\alpha_\wts N)}.
    \end{split}
    \end{align}

    \paragraph{Bias.}
    Let $\xib = \Sigmab_w^{1/2} \rbr{\wt\Phib_w^\top \wt\Phib_w + \alpha_w n \Ib_d}^{-1} \wt\Phib_w^\top \wt\fb_* \in \R^d$, whose randomness comes from $\wt\Scal$ only, independent of $\Scal_x$.
    Recall from \eqref{eq:pf_ridge_bias}, the bias term \eqref{eq:pf_ridge_bias} can be decomposed as
    \begin{align*}
        &\bias(f_\wts) = \E_{\Scal_x, \wt\Scal}\sbr{\frac{1}{N} \nbr{\Phib_s \rbr{\Phib_s^\top \Phib_s + \alpha_\wts N \Ib_d}^{-1} \Phib_s^\top \Phib_w \rbr{\wt\Phib_w^\top \wt\Phib_w + \alpha_w n \Ib_d}^{-1} \wt\Phib_w^\top \wt\fb_* - \fb_*}_2^2}\\
        &= \E_{\Scal_x, \wt\Scal}\sbr{\frac{1}{N} \rbr{\nbr{\Phib_s \rbr{\Phib_s^\top \Phib_s + \alpha_\wts N \Ib_d}^{-1} \Phib_s^\top \Gammab \xib - \Phib_s \Phib_s^\dagger \fb_*}_2^2 + \nbr{\rbr{\Ib_N - \Phib_s \Phib_s^\dagger} \fb_*}_2^2}},
    \end{align*}
    where by \Cref{lem:low_est_err_ft} and \eqref{eq:pf_ridge_ft_approx_err}
    \begin{align*}
        \E_{\Scal_x}\sbr{\frac{1}{N} \nbr{\rbr{\Ib_N - \Phib_s \Phib_s^\dagger} \fb_*}_2^2}
        = \frac{\rho_s(N)}{N} \le \rho_s = 0.
    \end{align*}
    Therefore, with $\xib = \Sigmab_w^{1/2} \rbr{\wt\Phib_w^\top \wt\Phib_w + \alpha_w n \Ib_d}^{-1} \wt\Phib_w^\top \wt\fb_*$, we have
    \begin{align*}
        \bias(f_\wts) = \E_{\Scal_x, \wt\Scal}\sbr{\frac{1}{N} \nbr{\Phib_s \rbr{\Phib_s^\top \Phib_s + \alpha_\wts N \Ib_d}^{-1} \Phib_s^\top \Gammab \xib - \Phib_s \Phib_s^\dagger \fb_*}_2^2}.
    \end{align*}
    Recall that $\fb_* = \Gammab \Sigmab_*^{1/2} \thetab_*$ and $\Phib_s = \Gammab \Sigmab_s^{1/2} = \Gammab_s \Lambdab_s^{1/2} \Vb_s^\top$.
    Then, we can express the bias term as
    \begin{align*}
        \bias(f_\wts) = &\E_{\Scal_x, \wt\Scal}\sbr{\frac{1}{N} \nbr{\Gammab\rbr{\Gammab^\top \Gammab + \alpha_\wts N \Sigmab_s^{-1}}^{-1} \Gammab^\top \Gammab \xib - \Gammab \Gammab^\dagger \fb_*}_2^2} \\
        = &\E_{\Scal_x, \wt\Scal}\sbr{\frac{1}{N} \nbr{\Gammab \Sigmab_*^{1/2} \thetab_* - \Gammab\rbr{\Gammab^\top \Gammab + \alpha_\wts N \Sigmab_s^{-1}}^{-1} \Gammab^\top \Gammab \xib}_2^2} \\
        = &\E_{\Scal_x, \wt\Scal}\sbr{\frac{1}{N} \nbr{\Gammab \rbr{\Sigmab_*^{1/2} \thetab_* - \xib} + \Gammab \rbr{\Ib_d - \rbr{\Gammab^\top \Gammab + \alpha_\wts N \Sigmab_s^{-1}}^{-1} \Gammab^\top \Gammab} \xib}_2^2} \\
    \end{align*} 
    By Woodbury matrix identity, we have
    \begin{align}\label{eq:pf_ridge_bias_woodbury}
        \Ib_d - \rbr{\Gammab^\top \Gammab + \alpha_\wts N \Sigmab_s^{-1}}^{-1} \Gammab^\top \Gammab
        = \rbr{\Ib_d + \frac{1}{\alpha_\wts N} \Sigmab_s \Gammab^\top \Gammab}^{-1}.
    \end{align}
    Therefore, we have 
    \begin{align}\label{eq:pf_ridge_bias_inter1}
        \bias(f_\wts) = \E_{\Scal_x, \wt\Scal}\Bigg[\frac{1}{N} \Big\|\underbrace{\Gammab \rbr{\Sigmab_*^{1/2} \thetab_* - \xib}}_{\t{Term A}} + \underbrace{\Gammab \rbr{\Ib_d + \frac{1}{\alpha_\wts N} \Sigmab_s \Gammab^\top \Gammab}^{-1} \xib}_{\t{Term B}}\Big\|_2^2\Bigg].
    \end{align}

    For Term A, notice that $\xib = \Sigmab_w^{1/2} \rbr{\wt\Phib_w^\top \wt\Phib_w + \alpha_w n \Ib_d}^{-1} \wt\Phib_w^\top \wt\fb_*$ implies
    \begin{align*}
        \Sigmab_*^{1/2} \thetab_* - \xib 
        = &\Sigmab_*^{1/2} \thetab_* - \Sigmab_w^{1/2} \rbr{\wt\Phib_w^\top \wt\Phib_w + \alpha_w n \Ib_d}^{-1} \wt\Phib_w^\top \wt\fb_* \\
        = &\Sigmab_*^{1/2} \thetab_* - \rbr{\wt\Gammab^\top \wt\Gammab + \alpha_w n \Sigmab_w^{-1}}^{-1} \wt\Gammab^\top \wt\Gammab \Sigmab_*^{1/2} \thetab_* \\
        = &\rbr{\Ib_d - \rbr{\wt\Gammab^\top \wt\Gammab + \alpha_w n \Sigmab_w^{-1}}^{-1} \wt\Gammab^\top \wt\Gammab} \Sigmab_*^{1/2} \thetab_* \\
        = &\rbr{\Ib_d + \frac{1}{\alpha_w n} \Sigmab_w \wt\Gammab^\top \wt\Gammab}^{-1} \Sigmab_*^{1/2} \thetab_*,
    \end{align*}
    where the last equality follows from Woodbury matrix identity as in \eqref{eq:pf_ridge_bias_woodbury}.
    Therefore,
    \begin{align*}
        \E_{\Scal_x, \wt\Scal}\sbr{\frac{1}{N} \nbr{\Gammab \rbr{\Sigmab_*^{1/2} \thetab_* - \xib}}_2^2} 
        = &\E_{\wt\Scal}\sbr{\frac{1}{n} \nbr{\wt\Gammab \rbr{\Sigmab_*^{1/2} \thetab_* - \xib}}_2^2} \\
        = &\E_{\wt\Scal}\sbr{\frac{1}{n} \nbr{\wt\Gammab \rbr{\Ib_d + \frac{1}{\alpha_w n} \Sigmab_w \wt\Gammab^\top \wt\Gammab}^{-1} \Sigmab_*^{1/2} \thetab_*}_2^2}.
    \end{align*}
    Since 
    \begin{align*}
        \rbr{\Ib_d + \frac{1}{\alpha_w n} \Sigmab_w \wt\Gammab^\top \wt\Gammab}^{-1} \wt\Gammab^\top \wt\Gammab \rbr{\Ib_d + \frac{1}{\alpha_w n} \Sigmab_w \wt\Gammab^\top \wt\Gammab}^{-1} \preceq \frac{\alpha_w n}{2} \Sigmab_w^{-1},
    \end{align*}
    we have
    \begin{align}\label{eq:pf_ridge_bias_term1}
    \begin{split}
        \E_{\Scal_x, \wt\Scal}\sbr{\frac{1}{N} \nbr{\Gammab \rbr{\Sigmab_*^{1/2} \thetab_* - \xib}}_2^2} 
        \le &\frac{1}{n} \tr\rbr{\frac{\alpha_w n}{2} \Sigmab_w^{-1} \Sigmab_*^{1/2} \thetab_* \thetab_*^\top \Sigmab_*^{1/2}} \\
        = &\frac{\alpha_w}{2} \nbr{\Sigmab_w^{-1/2} \Sigmab_*^{1/2} \thetab_*}_2^2.
    \end{split}
    \end{align}
    
    For Term B, leveraging Woodbury matrix identity as in \eqref{eq:pf_ridge_bias_woodbury}, we notice that 
    \begin{align*}
        &\E_{\Scal_x, \wt\Scal}\sbr{\frac{1}{N} \nbr{\Gammab \rbr{\Ib_d + \frac{1}{\alpha_\wts N} \Sigmab_s \Gammab^\top \Gammab}^{-1} \xib}_2^2} 
        \le \E_{\Scal_x, \wt\Scal}\sbr{\frac{1}{N} \tr\rbr{\frac{\alpha_\wts N}{2} \Sigmab_s^{-1} \xib \xib^\top}} \\
        = &\frac{\alpha_\wts}{2} \E_{\Scal_x, \wt\Scal}\sbr{\nbr{\Sigmab_s^{-1/2} \Sigmab_w^{1/2} \rbr{\wt\Phib_w^\top \wt\Phib_w + \alpha_w n \Ib_d}^{-1} \wt\Phib_w^\top \wt\fb_*}_2^2} \\
        = &\frac{\alpha_\wts}{2} \E_{\Scal_x, \wt\Scal}\sbr{\nbr{\Sigmab_s^{-1/2} \rbr{\wt\Gammab^\top \wt\Gammab + \alpha_w n \Sigmab_w^{-1}}^{-1} \wt\Gammab^\top \wt\Gammab \Sigmab_*^{1/2} \thetab_*}_2^2}
    \end{align*}
    Since $\rbr{\wt\Gammab^\top \wt\Gammab + \alpha_w n \Sigmab_w^{-1}}^{-1} \wt\Gammab^\top \wt\Gammab \preceq \Ib_d$, we know that
    \begin{align}\label{eq:pf_ridge_bias_term2}
    \begin{split}
        \E_{\Scal_x, \wt\Scal}\sbr{\frac{1}{N} \nbr{\Gammab \rbr{\Ib_d + \frac{1}{\alpha_\wts N} \Sigmab_s \Gammab^\top \Gammab}^{-1} \xib}_2^2} 
        \le \frac{\alpha_\wts}{2} \nbr{\Sigmab_s^{-1/2} \Sigmab_*^{1/2} \thetab_*}_2^2.
    \end{split}
    \end{align}
    Combining \eqref{eq:pf_ridge_bias_inter1}, \eqref{eq:pf_ridge_bias_term1}, and \eqref{eq:pf_ridge_bias_term2}, we can upper bound the bias term as
    \begin{align}\label{eq:pf_ridge_bias_final}
    \begin{split}
        &\bias(f_\wts) = \E_{\Scal_x, \wt\Scal}\Bigg[\frac{1}{N} \Big\|\underbrace{\Gammab \rbr{\Sigmab_*^{1/2} \thetab_* - \xib}}_{\t{Term A}} + \underbrace{\Gammab \rbr{\Ib_d + \frac{1}{\alpha_\wts N} \Sigmab_s \Gammab^\top \Gammab}^{-1} \xib}_{\t{Term B}}\Big\|_2^2\Bigg] \\
        \le &2 \E_{\Scal_x, \wt\Scal}\sbr{\frac{1}{N} \nbr{\Gammab \rbr{\Sigmab_*^{1/2} \thetab_* - \xib}}_2^2} + 2 \E_{\Scal_x, \wt\Scal}\sbr{\frac{1}{N} \nbr{\Gammab \rbr{\Ib_d + \frac{1}{\alpha_\wts N} \Sigmab_s \Gammab^\top \Gammab}^{-1} \xib}_2^2} \\
        \le &\alpha_w \nbr{\Sigmab_w^{-1/2} \Sigmab_*^{1/2} \thetab_*}_2^2 + \alpha_\wts \nbr{\Sigmab_s^{-1/2} \Sigmab_*^{1/2} \thetab_*}_2^2.
    \end{split}
    \end{align}
    
    \paragraph{Variance-bias tradeoff.}
    Overall, by \eqref{eq:pf_ridge_var_ub} and \eqref{eq:pf_ridge_bias_final}, we have
    \begin{align*}
        &\vari(f_\wts) \le \frac{\sigma^2 \tr\rbr{\Sigmab_s \Sigmab_w}}{4 (\alpha_w n) (\alpha_\wts N)}, \\
        &\bias(f_\wts) \le \alpha_w \nbr{\Sigmab_w^{-1/2} \Sigmab_*^{1/2} \thetab_*}_2^2 + \alpha_\wts \nbr{\Sigmab_s^{-1/2} \Sigmab_*^{1/2} \thetab_*}_2^2.
    \end{align*}
    The upper bound the excess risk $\exrisk(f_\wts) = \vari(f_\wts) + \bias(f_\wts)$ is minimized by taking 
    \begin{align*}
        \alpha_w = \rbr{\frac{\sigma^2 \tr\rbr{\Sigmab_s \Sigmab_w}}{4 n N}\ \frac{\nbr{\Sigmab_s^{-1/2} \Sigmab_*^{1/2} \thetab_*}_2^2}{\nbr{\Sigmab_w^{-1/2} \Sigmab_*^{1/2} \thetab_*}_2^4}}^{1/3}, \ 
        \alpha_\wts = \rbr{\frac{\sigma^2 \tr\rbr{\Sigmab_s \Sigmab_w}}{4 n N}\ \frac{\nbr{\Sigmab_w^{-1/2} \Sigmab_*^{1/2} \thetab_*}_2^2}{\nbr{\Sigmab_s^{-1/2} \Sigmab_*^{1/2} \thetab_*}_2^4}}^{1/3},
    \end{align*}
    which leads to the optimal upper bound for the excess risk:
    \begin{align*}
        \exrisk(f_\wts) \le 3 \rbr{\frac{\sigma^2 \tr\rbr{\Sigmab_s \Sigmab_w}}{4 n N}\ \nbr{\Sigmab_s^{-1/2} \Sigmab_*^{1/2} \thetab_*}_2^2 \nbr{\Sigmab_w^{-1/2} \Sigmab_*^{1/2} \thetab_*}_2^2}^{1/3}.
    \end{align*}
\end{proof}






\section{Canonical angles}\label{apx:canonical_angles}
In this section, we review the concept of canonical angles between two subspaces that connect the formal definition of the correlation dimension $d_{s \wedge w} = \nbr{\Vb_s^\top \Vb_w}_F^2$ in \Cref{def:correlation_dim} to the intuitive notion of the alignment between the corresponding subspaces $\Vcal_s$ and $\Vcal_w$ in the introduction: $\sum \cos(\angle(\Vcal_s, \Vcal_w)) = \nbr{\Vb_s^\top \Vb_w}_F^2$.
\begin{definition}[Canonical angles \cite{golub2013matrix}, adapting from \cite{dong2024efficient}]\label{def:canonical_angles}
    Let $\Vcal_s,\Vcal_w \subseteq \R^d$ be two subspaces with dimensions $\dim\rbr{\Vcal_s}=d_s$ and $\dim\rbr{\Vcal_w}=d_w$ (assuming $d_w \geq d_s$ without loss of generality). The canonical angles $\angle\rbr{\Vcal_s,\Vcal_w}=\diag\rbr{\nu_1,\dots,\nu_{d_s}}$ are $d_s$ angles that jointly measure the alignment between $\Vcal_s$ and $\Vcal_w$, defined recursively as follows:
    \begin{align*}
        &\ub_i, \vb_i ~\triangleq~
        \argmax~\ub_i^*\vb_i \\
        \t{s.t.}~
        &\ub_i \in \rbr{\Vcal_s \setminus \spn\cbr{\ub_{\iota}}_{\iota=1}^{i-1}} \cap \SSS^{d-1},\\ 
        &\vb_i \in \rbr{\Vcal_w \setminus \spn\cbr{\vb_{\iota}}_{\iota=1}^{i-1}} \cap \SSS^{d-1}\\
        &\cos (\nu_i) = \ub_i^* \vb_i \quad \forall~ i=1,\dots,k,
    \end{align*}
    such that $0 \leq \nu_1 \leq \dots \leq \nu_k \leq \pi/2$.

    Given two subspaces $\Vcal_s,\Vcal_w \subseteq \R^d$, let $\Vb_s \in \R^{d \times d_s}$ and $\Vb_w \in \R^{d \times d_w}$ be the matrices whose columns form orthonormal bases for $\Vcal_s$ and $\Vcal_w$, respectively. Then, the canonical angles $\angle(\Vcal_s, \Vcal_w)$ are determined by the singular values of $\Vb_s^\top \Vb_w$~\citep[\S 3]{bjorck1973numerical}:
    \begin{align*}
        \cos(\angle_i(\Vcal_s, \Vcal_w)) = \sigma_i(\Vb_s^\top \Vb_w) \quad \forall~ i=1,\dots,d_s,
    \end{align*}
    where $\sigma_i(\Vb_s^\top \Vb_w)$ denotes the $i$-th singular value of $\Vb_s^\top \Vb_w$.
\end{definition}

In particular, since $\Vb_s, \Vb_w$ consist of orthonormal columns, the singular values of $\Vb_s^\top \Vb_w$ fall in $[0,1]$, and therefore,
\begin{align*}
    d_{s \wedge w} = \sum \cos(\angle(\Vcal_s, \Vcal_w)) = \nbr{\Vb_s^\top \Vb_w}_F^2 \in [0, \min\cbr{d_s, d_w}].
\end{align*}




\section{Additional experiments}\label{apx:exp_details}

\subsection{Additional experiments and details on UTKFace regression}\label{apx:exp_img_reg}
This section provides some additional details and results for the UTKFace regression experiments in \Cref{sec:exp_img_reg}. 

\begin{figure}[!h]
    \centering
    \includegraphics[width=\columnwidth]{fig/mse_utkface_resnet18_clipb32.pdf}%\vspace{-2em}
    \caption{Scaling for MSE on UTKFace with \texttt{CLIP-B32} as the strong student and \texttt{ResNet18} as the weak teacher}\label{fig:mse_utkface_resnet18-clip}
\end{figure}

\begin{figure}[!h]
    \centering
    \includegraphics[width=\columnwidth]{fig/mse_utkface_resnet50_clipb32.pdf}%\vspace{-2em}
    \caption{Scaling for MSE on UTKFace with \texttt{CLIP-B32} as the strong student and \texttt{ResNet50} as the weak teacher}\label{fig:mse_utkface_resnet50-clip}
\end{figure}

\begin{figure}[!h]
    \centering
    \includegraphics[width=\columnwidth]{fig/mse_utkface_resnet152_clipb32.pdf}%\vspace{-2em}
    \caption{Scaling for MSE on UTKFace with \texttt{CLIP-B32} as the strong student and \texttt{ResNet152} as the weak teacher}\label{fig:mse_utkface_resnet152-clip}
\end{figure}

We summarize the relevant dimensionality in \Cref{tab:img_reg_dim}. We observe the following:
\begin{itemize}
    \item The intrinsic dimensions of the pretrained features are significantly smaller than the ambiance feature dimensions, which is consistent with our theoretical analysis and the empirical observations in \cite{aghajanyan2020intrinsic}. 
    \item The correlation dimensions $d_{s \wedge w}$ are considerably smaller than the corresponding intrinsic dimensions, indicating that the subspaces spanned by the weak and strong features are not aligned in practice. As shown in \Cref{sec:exp_img_reg}, such discrepancies in the weak and strong features facilitate W2S generalization.
\end{itemize}

\begin{table}[!ht]
    \centering
    \caption{Summary of the pretrained feature dimensions, along with the intrinsic dimensions $d_s, d_w$ and correlation dimensions $d_{s \wedge w}$ (with respect to the strong student \texttt{CLIP-B32}) computed over the entire UTKFace dataset (including training and testing).}\label{tab:img_reg_dim}
    \begin{tabular}{c|ccc}
        \toprule
        Pretrained Model & Feature Dimension & Intrinsic Dimension ($\tau=0.01$) & Correlation Dimension \\
        \midrule
        \texttt{ResNet18} & 512 & 194 & 167.64 \\
        \texttt{ResNet34} & 512 & 150 & 129.97 \\
        \texttt{ResNet50} & 2048 & 522 & 301.06 \\
        \texttt{ResNet101} & 2048 & 615 & 354.52 \\
        \texttt{ResNet152} & 2048 & 589 & 339.90 \\
        \midrule
        \texttt{CLIP-B32} & 768 & 443 & $\times$ \\
        \bottomrule
    \end{tabular}
\end{table}

For reference, we provide the scaling for MSE losses of three representative teacher-student pairs in \Cref{fig:mse_utkface_resnet18-clip,fig:mse_utkface_resnet50-clip,fig:mse_utkface_resnet152-clip}. 
\begin{itemize}
    \item It is worth highlighting that while the MSE loss of $f_\wts$ monotonically decreases with respect to both sample sizes $n,N$, the different rates of convergence compared to $f_w$, $f_s$, and $f_c$ lead to the distinct scaling behavior of the relative W2S performance ($\pgr$ and $\opr$) with respect to $n$ versus $N$ in \Cref{fig:pgr_opr_utkface_resnet-clip,fig:pgr_opr_utkface_vardom_resnet-clip}.
    \item When the strong student has a lower intrinsic dimension than the weak teacher (\cf \Cref{fig:mse_utkface_resnet18-clip} versus \Cref{fig:mse_utkface_resnet50-clip,fig:mse_utkface_resnet152-clip}), $d_s < d_w$, the W2S model $f_\wts$ tends to achieve better generalization in terms of the test MSE. This is consistent with our analysis in \Cref{sec:generalization_errors}.
    \item When $d_s < d_w$, the W2S model $f_\wts$ tends to achieve (slightly) better generalization for (slightly) smaller correlation dimension $d_{s \wedge w}$ (\cf \Cref{fig:mse_utkface_resnet50-clip} versus \Cref{fig:mse_utkface_resnet152-clip}), again coinciding with our analysis in \Cref{sec:generalization_errors}.
    \item W2S generalization generally happens (\ie $f_\wts$ is able to outperform $f_w$) with sufficiently large sample sizes $n, N$. However, as the labeled sample size $n$ increases, the test MSE of $f_\wts$ converges slower than that of the strong baseline and ceiling models, $f_s$ and $f_c$, leading to the inverse scaling for $\pgr$ and $\opr$ with respect to $n$ in \Cref{fig:pgr_opr_utkface_resnet-clip,fig:pgr_opr_utkface_vardom_resnet-clip}. When $n$ is too large, the W2S model $f_\wts$ may not be able to achieve better generalization than the strong baseline $f_s$.
\end{itemize}




\subsection{Experiments on image classification}\label{apx:exp_img_cls}

\paragraph{Dataset.} ColoredMNIST \citep{arjovsky2019invariant} consists of groups of different colors and reassign the label to be binary (digits 0-4 vs 5-9). We pool together the groups to form one dataset. The choice is to bring diversity to the feature space with additional color features and thus potential feature discrepancies. We hold out a test set of 7000 samples and used the rest 63000 samples for training.

\paragraph{Linear probing over pretrained features.} We fix a strong student as DINOv2-s14 \citep{oquab2023dinov2} and vary the weak teacher among the ResNet-d series and ResNet series (ResNet18D, ResNet34D, ResNet101, ResNet152) \citep{he2018resnetd,he2015deepresiduallearningimage}. We replace ResNet18 and ResNet34 used in \Cref{sec:exp_img_reg} to experiment on weak models with similar intrinsic dimensions but different correlation dimensions. We treat the backbone of the models (excluding the classification layer) as $\phi_s$ and $\phi_w$ and finetune them via linear probing. We train the models with cross entropy loss and AdamW optimizer. We tune the training hyperparameters of weak and strong models using a validation set and train them for 800 steps with learning rate 1e-3 and weight decay 1e-6. 

\begin{table}[!ht]
    \centering
    \caption{Summary of the pretrained feature dimensions, along with the intrinsic dimensions $d_s, d_w$ and correlation dimensions $d_{s \wedge w}$ (with respect to the strong student \texttt{DINOv2-S14}) computed over the entire ColoredMNIST dataset (including training and testing).}\label{tab:img_cls_dim_coloredmnist}
    \begin{tabular}{c|ccc}
        \toprule
        Pretrained Model & Feature Dimension & Intrinsic Dimension ($\tau=0.01$) & Correlation Dimension \\
        \midrule
        \texttt{ResNet-18-D} & 512 & 117 & 6.23 \\
        \texttt{ResNet-34-D} & 512 & 127 & 7.07 \\
        \texttt{ResNet101} & 2048 & 121 & 1.74 \\
        \texttt{ResNet152} & 2048 & 128 & 1.88 \\
        \midrule
        \texttt{DINOv2-S14} & 384 & 28 & $\times$ \\
        \bottomrule
    \end{tabular}
\end{table}

\begin{figure}[!h]
    \centering
    \includegraphics[width=\columnwidth]{fig/coloredmnist_lp/coloredmnist_dsw.pdf}%\vspace{-2em}
    \caption{Scaling for $\pgr$ and $\opr$ of different weak teachers with a fixed strong student on ColoredMNIST.}\label{fig:coloredmnist_dscapw}
\end{figure}

\begin{figure}[!h]
    \centering
    \includegraphics[width=\columnwidth]{fig/coloredmnist_lp/coloredmnist_var.pdf}%\vspace{-2em}
    \caption{Scaling for $\pgr$ and $\opr$ of W2S on ColoredMNIST with injected label noise.}\label{fig:coloredmnist_variance}
\end{figure}

\paragraph{Discrepancies lead to better W2S.}
\Cref{fig:coloredmnist_dscapw} shows the scaling of $\pgr$ and $\opr$ with respect to the sample sizes $n, N$ for different weak teachers in the ResNet series with respect to a fixed student, \texttt{CLIP-B32}. 
As in \Cref{sec:exp_img_reg}, we observe that with similar intrinsic dimensions $d_s, d_w$, W2S achieves better relative performance in terms of $\pgr$ and $\opr$ when the correlation dimension $d_{s \wedge w}$ is smaller.

\paragraph{Variance reduction is a key advantage of W2S.}
We inject noise to the labels of the original ColoredMNIST training samples by randomly flipping the ground truth labels with probability $\varsigma \in [0,1]$ (following \cite{arjovsky2019invariant}). 
\Cref{fig:coloredmnist_variance} shows the scaling of $\pgr$ and $\opr$ with respect to $n$ and $N$ when taking DINOv2-S14 as the strong student and ResNet101 as the weak teacher. We observe that the larger artificial label noise $\varsigma$ leads to higher $\pgr$ and $\opr$. 

\subsection{Experiments on sentiment classification}\label{apx:exp_nlp_cls}

\paragraph{Dataset.} The Stanford Sentiment Treebank \citep{socher-etal-2013-sst2} is a corpus with fully labeled parse trees that allows for a complete analysis of the compositional effects of sentiment in language. The corpus is based on the dataset introduced by \citet{pang-lee-2005-sst_original_corpus} and consists of 11,855 single sentences extracted from movie reviews. It was parsed with the Stanford parser and includes a total of 215,154 unique phrases from those parse trees, each annotated by 3 human judges. We conduct binary classification experiments on full sentences (negative or somewhat negative vs somewhat positive or positive with neutral sentences discarded), the so-called SST-2 dataset, and split the dataset into training and testing sets of sizes 63000 and 4349. Generalization errors are estimated with the 0-1 loss over the test set.

\paragraph{Full finetuning.} We fix the strong student as Electra-base-discriminator \citep{clark2020electra} and vary the weak teacher among the Bert series \citep{turc2019bert-tiny} (Bert-Tiny, Bert-Mini, Bert-Small, Bert-Medium). 
With manageable model sizes, we conduct full finetuning experiments following the setup in \cite{burns2023weak}.
We use the standard cross entropy loss for supervised finetuning. 
When training strong students on weak labels (W2S), we use the confidence weighted loss proposed by \cite{burns2023weak}, which is suggested to be able to improve weak-to-strong generalization on many NLP tasks.
All training is conducted via Adam optimizers~\citep{kingma2014adam} with a learning rate of 5e-5, a cosine learning rate schedule, and 40 warmup steps. We train for 3 epochs, which is sufficient for the train and validation loss to stabilize. 

\paragraph{Intrinsic dimension.} The intrinsic dimensions $d_w,d_s$ are measured based on the Structure-Aware Intrinsic Dimension (SAID) method proposed by \cite{aghajanyan2020intrinsic}. We first train the full models on the whole training set, and then train the models with only $d$ trainable parameters based on SAID transformation. The $d_w$ or $d_s$ are the smallest number of parameters under SAID that is necessary to retain 90\% accuracy of the full models. Here, the 90\% accuracy is a common threshold used to estimate intrinsic dimensions in the literature \citep{li2018measuring}.

\begin{figure}[!h]
    \centering
    \includegraphics[width=\columnwidth]{fig/sst2/sst2-dsw.pdf}%\vspace{-2em}
    \caption{Scaling for $\pgr$ and $\opr$ of different weak teachers with a fixed strong student on SST-2.}\label{fig:sst2_dsw}
\end{figure}

\begin{figure}[!h]
    \centering
    \includegraphics[width=\columnwidth]{fig/sst2/sst2-var.pdf}%\vspace{-2em}
    \caption{Scaling for $\pgr$ and $\opr$ of W2S on SST-2 with injected label noise.}\label{fig:sst2_var}
\end{figure}

\paragraph{Correlation Dimension.} 
Let $D_s, D_w \in \N$ be the finetunable parameter counts of the strong and weak models, respectively. For full FT whose dynamics fall in the kernel regime, as explained in \Cref{rmk:lp_to_general_ft}, the strong and weak ``features'' become the gradients\footnote{
    Notice that $f_s, f_w$ are scalar-valued functions for binary classification tasks like SST-2, and thus the gradients $\nabla_{\thetab} f_s$ and $\nabla_{\thetab} f_w$ are row vectors.
    For multi-class classification tasks where $f_s, f_w$ output vectors of logits, a common heuristic to keep $\Phib_s, \Phib_w$ as matrices of manageable sizes (in constrast to tensors) is to replace gradients of the models, $\nabla_{\thetab} f_s$ and $\nabla_{\thetab} f_w$, with gradients of MSE losses at the pretrained initialization. 
    The gradients of MSE can be viewed as a weighted sum of the model gradients for each class.
}, $\Phib_s = \nabla_{\thetab} f_s(\Xb | \theta_s^{(0)}) \in \R^{N \times D_s}$ and $\Phib_w = \nabla_{\thetab} f_w(\Xb | \theta_w^{(0)}) \in \R^{N \times D_w}$, of the respective models at the pretrained initialization, $\theta_s^{(0)} \in \R^{D_s}$ and $\theta_w^{(0)} \in \R^{D_w}$.

A practical challenge is that $D_s, D_w, N$ are all huge for full FT on most NLP tasks, making it infeasible to compute the $D_s \times D_s$ and $D_w \times D_w$ Gram matrices and their spectral decompositions. 
As a remedy, we leverage the significantly lower intrinsic dimensions $d_s \ll D_s, d_w \ll D_w$ (see \Cref{tab:img_cls_dim_coloredmnist}) to accelerate estimation of $d_{s \wedge w}$ via sketching~\citep{halko2011finding,woodruff2014sketching}.
\begin{enumerate}[label=(\roman*)]
    \item We first reduce both $D_s, D_w$ to the same lower dimension $D = 0.01 \min\{D_s, D_w\}$ (with $D \gg \max\{d_s, d_w\}$) by uniform subsampling columns of $\Phib_s, \Phib_w$ to obtain $\Phib_s', \Phib_w' \in \R^{N \times D}$.
    \item Then, we use randomized rangefinder~\citep[Algorithm 4.1]{halko2011finding} to approximate the first $d_s, d_w$ right singular vectors, $\Vb_s \in \R^{D \times d_s}$ and $\Vb_w \in \R^{D \times d_w}$, of $\Phib_s', \Phib_w'$. Taking the evaluation of $\Vb_s$ as an example, we draw a Gaussian random matrix $\Gb_s \in \R^{d_s \times D}$ and compute the orthornormalization $\Vb_s = \ortho(\Phib_s'^\top \Gb_s)$ via QR decomposition.
    \item Finally, we compute the correlation dimension $d_{s \wedge w} = \nbr{\Vb_s^\top \Vb_w}_F^2$.
\end{enumerate}

\begin{table}[!ht]
    \centering
    \caption{Summary of finetunable parameter counts $D_s, D_w$, intrinsic dimensions $d_s, d_w$, and correlation dimensions $d_{s \wedge w}$ (with respect to the strong student \texttt{Electra}) computed over the entire SST-2 dataset (including training and testing).}\label{tab:sst2_dim}
    \begin{tabular}{c|ccc}
        \toprule
        Pretrained Model & $D_s,D_w$ & Intrinsic Dimension ($\tau=0.01$) & Correlation Dimension \\
        \midrule
        \texttt{Bert-Tiny} & 4.4M & 7000 & 81.13 \\
        \texttt{Bert-Mini} & 11.2M & 8500 & 38.67 \\
        \texttt{Bert-Small} & 28.8M & 8000 & 26.19 \\
        \texttt{Bert-Medium} & 41.4M & 4000 & 8.52 \\
        \midrule
        \texttt{Electra} & 109.5M & 700 & $\times$ \\
        \bottomrule
    \end{tabular}
\end{table}

\paragraph{Discrepancies lead to better W2S.}
\Cref{fig:sst2_dsw} shows the scaling of $\pgr$ and $\opr$ with respect to $n$ and $N$ for different $d_{s \wedge w}$. 
As in \Cref{sec:exp_img_reg,apx:exp_img_cls}, we observe the better relative W2S performance in terms of $\pgr$ and $\opr$ when $d_{s \wedge w}/d_w$ is smaller.

\paragraph{Variance reduction is a key advantage of W2S.}
We inject noise to the labels of training samples by randomly flipping labels with probability $\varsigma = 0, 0.1, 0.2, 0.3$. 
\Cref{fig:sst2_var} shows the scaling of $\pgr$ and $\opr$ with respect to $n$ and $N$ when taking \texttt{Electra} as the strong student and \texttt{Bert-Medium} as the weak teacher. We observe that the larger artificial label noise $\varsigma$ leads to higher $\pgr$ and $\opr$. 


% \nobalance 

\end{document}
