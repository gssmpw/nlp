\section{Related Work}
The body of literature on the automatability of occupations is vast and growing, with numerous studies focusing on predicting the susceptibility of entire occupations to automation. For instance, Frey and Osborne ____ pioneered the field with their seminal work, using a Gaussian process classifier to estimate the probability of computerisation for 702 detailed occupations and concluded that about 47\% occupations in US is at high risk. Arntz ____ further expanded on this by considering the heterogeneity of jobs within occupations and obtained a less alarming rate of 9\%, albeit their focus remained on the occupation level. These studies, while groundbreaking, have been critiqued for their broad-brush approach, which overlooks the granularity of tasks within occupations.\par

In parallel, there has been a surge of interest in the application of BERT-based models for sentence classification tasks. Devlin ____ introduced BERT, a transformer-based model that leverages context in both directions, making it highly effective for understanding the semantics of text. Since then, BERT has been widely used in various natural language processing tasks, including sentence classification ____, and cross domain applications ____ demonstrating its ability to capture contextual and semantic information effectively. However, this is limited research in automatability prediction leveraging sentences' contextual and semantic information and multiple studies have formulated hypothesis that the automatability of tasks is fundamentally a question of understanding the nature and requirements of the task which is often embedded in the contextual and semantic information of textual task statements ____.\par

Our work bridges these two streams of literature. While we build upon the existing body of work on occupation automatability, we shift the focus from the occupation level to the task level. We argue that this provides a more nuanced and accurate picture of the potential impact of automation on the workforce. Furthermore, we leverage the power of BERT for sentence classification, applying it to the analysis of task descriptions to predict their automatability. This represents a novel application of BERT, extending its use beyond traditional natural language processing tasks. In doing so, we hope to contribute a new perspective to the ongoing discourse on the future of work in the age of automation.\par