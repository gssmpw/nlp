\pdfoutput=1
\documentclass[conference]{IEEEtran}
% \documentclass[final]{IEEEtran}
\IEEEoverridecommandlockouts
% The preceding line is only needed to identify funding in the first footnote. If that is unneeded, please comment it out.
\usepackage[noadjust]{cite}
\usepackage{amsmath,amssymb,amsfonts}
\usepackage{algorithmic}
\usepackage{graphicx}
\usepackage{textcomp}
\usepackage{adjustbox}
\usepackage{svg}
\usepackage{tabularray}
\usepackage{xcolor}
\usepackage{pdfpages}
\usepackage{url}
\usepackage{afterpage}

\def\BibTeX{{\rm B\kern-.05em{\sc i\kern-.025em b}\kern-.08em
    T\kern-.1667em\lower.7ex\hbox{E}\kern-.125emX}}

\newcommand{\customfont}{\fontsize{6.5}{7.8}\selectfont}
\usepackage{hyperref}
\begin{document}

\title{From Occupations to Tasks: A New Perspective on Automatability Prediction Using BERT\\
}

% \author{\IEEEauthorblockN{Anonymous Author}
% }
\author{\IEEEauthorblockN{Dawei Xu}
\IEEEauthorblockA{\textit{University of Technology Sydney} \\
Sydney, Australia \\
dawei.xu@student.uts.edu.au}
\and
\IEEEauthorblockN{Haoran Yang}
\IEEEauthorblockA{\textit{University of Technology Sydney} \\
Sydney, Australia \\
haoran.yang-2@student.uts.edu.au}
\and
\IEEEauthorblockN{Marian-Andrei Rizoiu}
\IEEEauthorblockA{\textit{University of Technology Sydney} \\
Sydney, Australia \\
marian-andrei.rizoiu@uts.edu.au}
\and
\IEEEauthorblockN{Guandong Xu}
\IEEEauthorblockA{\textit{University of Technology Sydney} \\
Sydney, Australia \\
guandong.xu@uts.edu.au}
}

\maketitle

\begin{abstract}
As automation technologies continue to advance at an unprecedented rate, concerns about job displacement and the future of work have become increasingly prevalent. While existing research has primarily focused on the potential impact of automation at the occupation level, there has been a lack of investigation into the automatability of individual tasks. This paper addresses this gap by proposing a BERT-based classifier to predict the automatability of tasks in the forthcoming decade at a granular level leveraging the context and semantics information of tasks. We leverage three public datasets: O*NET Task Statements, ESCO Skills, and Australian Labour Market Insights Tasks, and perform expert annotation. Our BERT-based classifier, fine-tuned on our task statement data, demonstrates superior performance over traditional machine learning models, neural network architectures, and other transformer models. Our findings also indicate that approximately 25.1\% of occupations within the O*NET database are at substantial risk of automation, with a diverse spectrum of automation vulnerability across sectors. This research provides a robust tool for assessing the future impact of automation on the labor market, offering valuable insights for policymakers, workers, and industry leaders in the face of rapid technological advancement.
\end{abstract}

\begin{IEEEkeywords}
Task automatability prediction, Automated Occupation identification, BERT, Transfer learning
\end{IEEEkeywords}

\section{Introduction}
In the era of rapid technological advancement, the fear of large scale unemployment caused by automation technologies has become a significant concern. This concern is not unfounded; a growing body of research suggests that many occupations could be partially or fully automated in the coming years. According to the research conducted by \cite{freyosborne}, over 47\% of current US employment is at high risk of being automated. Another similar study has been performed using job-level data and concluded a less alarming rate of 9\% of employment is at risk \cite{arntz}. However, these studies have largely focused on the potential for automation at the occupation level, leaving a significant gap in our understanding of how individual tasks within these occupations might be automated and multiple studies have shown the effectiveness of assessing the automatability on tasks level instead of occupations \cite{towards, webb_2019_the}. \par
The automatability of tasks is fundamentally a question of understanding the nature and requirements of the task and this information is often embedded in the contextual and semantic information of textual task statements \cite{towards}. For example, the task "Write a report" might be automatable, but "Write a creative story" might not be. And the word ``run" means differently in "run a program" and "run a marathon" \cite{arntz}. Therefore, our research is based on the assumption that the automatability of tasks can be apprehended through the extraction of contextual and semantic information embedded within textual task statements.\par
Due to this circumstance, our research employ the Bidirectional Encoder Representations from Transformers (BERT) model \cite{devlin2018bert} to understand the context and semantics of words in a task statement allowing for a more sophisticated analysis of task descriptions, potentially leading to more accurate predictions of task automatability. We also utilize three task description datasets: O*NET Task statements \cite{onet}, ESCO Skills \cite{esco}, and Australian Labour Market Insights Tasks \cite{australian_labor_outlook}. We annotate each task with one of three labels: ``Substitution," ``Complementarity," or ``Negligibility," representing its automation susceptibility based on expert assessments. Using these annotations as ground truth, we apply a BERT-based classifier model \cite{devlin2018bert} to predict task automatability, benefiting from BERT's context understanding and semantic capabilities.\par

We extend our analysis to occupation and industry levels, using aggregated task-level predictions to assess overall automation susceptibility. This comprehensive view can reveal trends not observable at the task level, providing additional automation impact insights. Our research holds significant implications for workers, businesses, and policymakers by offering clarity on automation-susceptible tasks and demonstrating BERT's utility in labor economics.\par
% This paper aims to address this gap by investigating the automatability of tasks rather than entire occupations in the forthcoming decade. To achieve this, we utilize three comprehensive datasets: O*NET Task statements \cite{onet}, ESCO Skills \cite{esco}, and Australian Labour Market Insights
% Tasks \cite{australian_labor_outlook} which comprising detailed task descriptions across a wide range of occupations. Each task in the datasets has been anonotated to one of the three classes: ``Substitution", ``Complementarity", or ``Negligibility" indicating its susceptibility to automation based on expert assessments. These annotations serve as the ground truth for training and evaluating our proposed model, enabling us to assess its performance in predicting task automatability. We employ the Bidirectional Encoder Representations from Transformers (BERT) model \cite{devlin2018bert}, a cutting-edge machine learning technique that offers several advantages over traditional methods. BERT's ability to understand the context and semantics of words in a sentence allows for a more sophisticated analysis of task descriptions, potentially leading to more accurate predictions of task automatability.\par
% Finally, we extend our analysis to the occupation and industry levels. We aggregate our task-level predictions to assess the overall susceptibility of different occupations and industries to automation. This broader perspective allows us to identify patterns and trends that may not be apparent at the task level, providing additional insights into the potential impact of automation on the workforce.\par
% The implications of this research are significant. By providing a clearer picture of which tasks are most susceptible to automation, we can help workers, businesses, and policymakers prepare more effectively for the future of work. Furthermore, by demonstrating the utility of BERT in this context, we hope to encourage further application of advanced machine learning techniques in the field of labor economics.\par
The contributions of our research to the domain of automation impact on the labor market can be summarized as follows:\par
\begin{itemize}
\item Presents a novel approach to predict the automatability at task level which provides a more detailed and nuanced understanding of the impact of automation.
\item This work substantiates the hypothesis that task automatability can be discerned through mining the contextual and semantic information of textual task statements by leveraging BERT's contextual and semantic understanding capabilities.
\item Fine-tuned the BERT-based classifier on task statement data, outperforms traditional machine learning and neural network models, offering a robust tool for assessing automation's future impact on the labor market.
\item This research combine and annotates three public datasets: O*NET Task Statements, ESCO Skills, and Australian Labour Market Insights Tasks, creating a valuable resource that underscores the robustness of the BERT-based classifier across diverse data types.
\end{itemize}
% In this era of rapid technological advancements, the prospect of large-scale unemployment due to automation is concerning. Studies suggest potential for significant occupational automation, with one estimating 47\% of US jobs at high risk \cite{freyosborne} and another projecting a more conservative 9\% \cite{arntz}. However, these studies mainly focus on occupation-level automation, ignoring individual tasks within occupations. Task-level automation assessment has been shown to be effective \cite{towards, webb_2019_the}.

% Our research investigates task automatability within occupations over the next decade, utilizing three task description datasets: O*NET Task statements \cite{onet}, ESCO Skills \cite{esco}, and Australian Labour Market Insights Tasks \cite{australian_labor_outlook}. We annotate each task with one of three labels: "Substitution," "Complementarity," or "Negligibility," representing its automation susceptibility based on expert assessments. Using these annotations as ground truth, we apply the BERT model \cite{devlin2018bert} to predict task automatability, benefiting from BERT's context understanding and semantic capabilities.

% We extend our analysis to occupation and industry levels, using aggregated task-level predictions to assess overall automation susceptibility. This comprehensive view can reveal trends not observable at the task level, providing additional automation impact insights. Our research holds significant implications for workers, businesses, and policymakers by offering clarity on automation-susceptible tasks and demonstrating BERT's utility in labor economics.

% Our research seeks to answer the following research questions::

% \begin{itemize}
% \item What is the extent of task, rather than occupation, be automated?
% \item How does the BERT model improve task automatability prediction over traditional methods?
% \item Which occupations/industries are more susceptible to future automation?
% \item What are the task-level automation implications for workers and policymakers?
% \end{itemize}

\section{Related Work}
The body of literature on the automatability of occupations is vast and growing, with numerous studies focusing on predicting the susceptibility of entire occupations to automation. For instance, Frey and Osborne \cite{freyosborne} pioneered the field with their seminal work, using a Gaussian process classifier to estimate the probability of computerisation for 702 detailed occupations and concluded that about 47\% occupations in US is at high risk. Arntz \cite{arntz} further expanded on this by considering the heterogeneity of jobs within occupations and obtained a less alarming rate of 9\%, albeit their focus remained on the occupation level. These studies, while groundbreaking, have been critiqued for their broad-brush approach, which overlooks the granularity of tasks within occupations.\par

In parallel, there has been a surge of interest in the application of BERT-based models for sentence classification tasks. Devlin \cite{devlin2018bert} introduced BERT, a transformer-based model that leverages context in both directions, making it highly effective for understanding the semantics of text. Since then, BERT has been widely used in various natural language processing tasks, including sentence classification \cite{croce2020ganbert, lu2020vgcnbert}, and cross domain applications \cite{lavi2021consultantbert, xue2019finetuning} demonstrating its ability to capture contextual and semantic information effectively. However, this is limited research in automatability prediction leveraging sentences' contextual and semantic information and multiple studies have formulated hypothesis that the automatability of tasks is fundamentally a question of understanding the nature and requirements of the task which is often embedded in the contextual and semantic information of textual task statements \cite{towards, webb_2019_the}.\par

Our work bridges these two streams of literature. While we build upon the existing body of work on occupation automatability, we shift the focus from the occupation level to the task level. We argue that this provides a more nuanced and accurate picture of the potential impact of automation on the workforce. Furthermore, we leverage the power of BERT for sentence classification, applying it to the analysis of task descriptions to predict their automatability. This represents a novel application of BERT, extending its use beyond traditional natural language processing tasks. In doing so, we hope to contribute a new perspective to the ongoing discourse on the future of work in the age of automation.\par
\section{Method}
In this section, we outline the whole framework (Fig. \ref{fig: framework}) employed to predict task level automatability leveraging a BERT-based classifier. Our approach first harnesses the BERT architecture to generate task statement embeddings, then construct a BERT-based classifier with a softmax layer for categorizing automatabilities. The model undergoes fine-tuning, tailored to the downstream classification task, optimizing its performance on the specific domain.

\begin{figure}[htp!]
\centerline{\includegraphics[width=1\columnwidth]{svg-inkscape/framework_svg-raw.pdf}}
\caption{The framework of the proposed method.}
\label{fig: framework}
\end{figure}
\subsection{Dataset Creation}
We used three public datasets: ONET Task Statements, ESCO skills, and AU Labor Market Insights Task statements. They provide a wide range of occupation-specific task statements with varying geographical and vocational coverage. We randomly sampled tasks: 5,060 from ONET, 4,783 from ESCO, and 3,356 from AU Labor Market Insights. This was done to ensure diversity across occupations, sectors, and regions for generalizable findings and to optimize computational resources and model training efficiency due to the high-dimensionality of text data and the large dataset sizes.
\subsubsection{Expert Annotation}
% The selected tasks underwent an expert annotation process where each task was labeled according to one of the three classes: ``Substitution", ``Complementarity", or ``Negligibility", based on the level of automation potential. The ``Substitution" class represents tasks that can be entirely automated, ``Complementarity" implies tasks where human labor can be complemented by automation technology but still need human involvement, and ``Negligibility" refers to tasks that are unlikely to be affected by automation. These classes constitute our ground truth data, providing a categorical understanding of automation susceptibility at a granular task level.\par
% Specifically, our process is grounded in two prominent social-economic hypotheses Skill-Biased Technological Change (SBTC) \cite{card2002skill} and Routine-Biased Technological Change (RBTC) \cite{autor2003rbtc} which explain the impact of technological change on labor markets. The SBTC hypothesis postulate that technological advancements favor tasks that require high-level skills while replacing routine, lower-skilled jobs \cite{kristal2020computerization}. Conversely, the RBTC hypothesis asserts that technologies are primarily replacing jobs that involve routine tasks, irrespective of the skill level \cite{buyst2018job}.\par
% In addition to these hypotheses, we also consider the six significant bottlenecks (Fig \ref{fig: bottlenecks}) to automation identified in previous research: Complex Problem Solving, Social Interaction and Emotional Intelligence, Fine Motor Skills and Dexterity, Creative and Artistic Abilities, Contextual Understanding, and Ethical Decision Making \cite{chui2016where, nedelkoska2018automation}. These bottlenecks encapsulate the areas where humans continue to outperform machines, thus providing additional granularity to our analysis.
Our annotation process employs a voting mechanism involving five experts. For each task and each class, if a class receives more than three votes, it is designated as the final label for that task. This approach ensures a consensus among experts and enhances the reliability of the labels. Our expert annotation process labeled each task as ``Substitution" (fully automatable), ``Complementarity" (partial automation with human involvement), or ``Negligibility" (unlikely to be automated). This provides a categorical understanding of task-level automation susceptibility. We base our process on the Skill-Biased Technological Change (SBTC) \cite{card2002skill} and Routine-Biased Technological Change (RBTC) \cite{autor2003rbtc} hypotheses, which explain technological change's impact on labor markets. SBTC postulates that technology favors high-skill tasks and replaces lower-skilled routine jobs \cite{kristal2020computerization}, while RBTC asserts that technology replaces routine tasks, regardless of skill level \cite{buyst2018job}. We also incorporate six major automation bottlenecks : Complex Problem Solving, Social Interaction and Emotional Intelligence, Fine Motor Skills and Dexterity, Creative and Artistic Abilities, Contextual Understanding, and Ethical Decision Making \cite{chui2016where, nedelkoska2018automation}. These bottlenecks highlight areas where humans still outperform machines.

\subsubsection{Data Augmentation}
The initial exploration of the labeled data revealed a class imbalance issue, a common problem in many real-world classification tasks \cite{krawczyk2016learning}. This imbalance can lead to biased models, overfitting to the majority class and neglecting minority ones. Conventional data augmentation methods suitable for image data aren't applicable to text data due to its sequential and semantic nature \cite{zhang2015character}. Apart from class imbalance issue, effectively increasing the volume of training data is particularly beneficial for deep learning models like BERT that perform better with larger datasets.\par
We addressed this using GPT-4 \cite{openai2023gpt4} to paraphrase sentences, creating new task statements with identical meanings but different wording. GPT-4's ability to generate high-quality and varied paraphrases ensures the augmented data remains relevant and increases the classifier's robustness to different task expressions. Specifically, each statement was fed into GPT-4 for paraphrasing, followed by automated checks and expert reviews to validate the paraphrases for semantic similarity to the original statements:\par 
\noindent\textbf{Original Sentence:}\par
\noindent Generate reports utilizing visual aids such as charts, graphs, and narratives by examining and documenting test data.\par
\noindent\textbf{Paraphrased Sentence:}\par
\noindent \textit{Create} reports that \textit{incorporate} visual \textit{elements} like \textit{diagrams}, \textit{plots}, and descriptive narratives by \textit{scrutinizing} and \textit{recording} the results of tests.

\subsection{BERT-based Classifier}
BERT stands out with its multi-layer bidirectional Transformer encoder, which enables it to capture the deep contextual information embedded in text data \cite{devlin2018bert}. Mathematically, this is achieved by applying the attention mechanism, where the output embeddings \textit{E} for a given input token is a weighted sum of all input token embeddings:
\begin{equation}
E = \sum \text{{\textit{Attention\_Scores}}} \times \text{{\textit{Token\_Embeddings}}}
\end{equation}
For our classification task, we fine-tuned a pre-trained BERT model on our task statement data. Each task statement was fed into BERT, which transformed the text into high-dimensional embeddings.\par

The embeddings were then passed through a softmax function to generate the final class probabilities. If we denote the output of our model before the softmax layer as \(z\), the softmax function can be expressed as:
\begin{equation}
\text{{\textit{Softmax}}}(z_i) = \frac{e^{z_i}}{\sum e^{z_j}}
\end{equation}
where \(z_i\) is the output corresponding to the i-th class and the sum in the denominator runs over all possible classes.\par
The training process involved adjusting the model's parameters to minimize the discrepancy between the predicted and actual class labels, typically quantified using the cross-entropy loss function:
\begin{equation}
L = -\sum \text{{\textit{y\_true}}} \times \log(\text{{\textit{y\_pred}}})
\end{equation}
where \textit{y\_true} is the true label and \textit{y\_pred} is the predicted probability.
\subsection{Attention Mechanism}
A fundamental feature of BERT's ability to capture deep contextual information is its use of the attention mechanism, specifically, the scaled dot-product attention as introduced in the original Transformer model \cite{vaswani2017attention}. This attention mechanism allows the model to weigh the importance of different words in a sentence, providing a powerful tool for understanding language semantics.\par
Mathematically, the attention mechanism can be described as mapping a query and a set of key-value pairs to an output. Given a query (\textit{Q}), keys (\textit{K}), and values (\textit{V}), the output (\textit{O}) of the attention mechanism is calculated as a weighted sum of the values, where the weight assigned to each value is determined by the query's compatibility with the corresponding key:
\begin{equation}
\text{\textit{Attention}}(Q, K, V) = \text{\textit{softmax}}\left(\frac{QK^T}{\sqrt{d_k}}\right)V
\end{equation}
where the denominator \(\sqrt{d_k}\) is used for scaling, with \(d_k\) being the dimension of the key vectors. The softmax function ensures the attention scores are normalized to lie between 0 and 1, thus can be interpreted as probabilities. This results in words that are more important to the meaning of the sentence receiving a higher attention score, while less important words receive a lower score.

The use of attention weights allows us to visualize and interpret the model's decision-making process. By examining these weights, we can understand which parts of the input sentence are most influential in determining the output of the model.

\section{Experiment}
\subsection{Data}
The datasets were partitioned into training, evaluation, and test subsets following an 8:1:1 ratio, providing a comprehensive and balanced basis for our experimental studies. And both the original and augmented datasets would be passed through the model to evaluate and quantify the influence of data augmentation.\par
The table \ref{overviewdata} presented below provides a visual representation of both the initial and augmented datasets. Specifically, during the training phase using the original dataset, we observed that the AU LMI dataset exhibited superior performance compared to the other two datasets. Consequently, during the process of data augmentation, we prioritized augmenting the ONET and ESCO datasets to a greater extent, while applying relatively fewer augmentations to the AU LMI dataset. This decision was based on the recognition that ONET and ESCO present greater challenges in terms of machine learning, warranting additional exposure to augmented instances for effective learning.
\begin{table}[htbp]
\caption{The overview of dataset}
\label{overviewdata}
\centering
\begin{tblr}{
  cell{1}{1} = {c=2}{},
  cell{2}{1} = {r=2}{},
  cell{4}{1} = {r=2}{},
  cell{6}{1} = {r=2}{},
  vline{1,2,3,4,5,6},
  hline{1,2,3,4,5,6,7,8}
}
                &           & O*NET & ESCO & AU LMI \\ 
Substitution    & Original  & 1,594  & 1,435 & 998    \\ 
                & \textbf{Augmented} & \textbf{3,188}  & \textbf{3,157} & \textbf{1,796}   \\
Complementarity & Original  & 2,519  & 2,272 & 1,776   \\ 
                & \textbf{Augmented} & \textbf{3,023}  & \textbf{3,181} & \textbf{1,776}   \\
Negligibility  & Original  & 947   & 1,076 & 582    \\ 
                & \textbf{Augmented} & \textbf{3,030}  & \textbf{3,228} & \textbf{1,764} 
\end{tblr}
\end{table}

\subsection{Evaluation Metrics}
We assessed our model and baselines using precision, recall, and F1 score. Precision evaluates the ratio of correct positive predictions to all positive predictions, indicating the model's false-positive avoidance. Recall measures the fraction of true positives from all actual positives, reflecting the model's ability to recognize all relevant instances. The F1 score, the harmonic mean of precision and recall, balances these aspects—a high F1 score indicates a model with high precision and recall, and vice versa.\par
In our multi-class problem, these metrics were computed for each class, considering the class in question as positive and the rest as negative. The average of per-class metrics provided a single performance measure across all classes.

\subsection{Baselines}
In our work, we make a comprehensive comparison of our proposed BERT-based model with a range of baseline models, encompassing both traditional machine learning approaches, neural network architectures, and other transformer models. These baselines span different approaches to text classification, allowing us to assess the relative merits of our approach in a broad context.

For traditional classifiers, we employ Logistic Regression \cite{cessie1992ridge}, Random Forest \cite{breiman2001random}, and Support Vector Machines (SVM) \cite{cortes1995support}. These models, with their varying theoretical underpinnings, provide a solid foundation against which to compare our more complex neural model. They represent different forms of linear and non-linear decision boundaries and incorporate different forms of regularization and ensemble learning.

In the realm of neural networks, we utilize Bi-directional Long Short-Term Memory (BiLSTM) \cite{graves2005framewise}, Gated Recurrent Units (GRU) \cite{cho2014learning}, and one-dimensional Convolutional Neural Networks (Conv1D) \cite{yoon2020convolutional}. These architectures demonstrate the potential of deep learning for text classification, and their comparison with our model allows us to quantify the value of pre-training and transformer architecture in this context.

Finally, we include three additional transformer models in our baselines: ALBERT \cite{lan2019albert}, ELECTRA \cite{clark2020electra}, and DistilBERT \cite{sanh2019distilbert}. As close relatives of BERT, they provide a stringent test of the specific benefits of our BERT-based approach. By contrasting with these models, we can highlight the unique strengths of our chosen methodology.

\subsection{Results}
The table \ref{results} summarizes the results for proposed BERT and other baseline models of different datasets. We evaluate them with precision, recall and F1-score.\par

% \usepackage{tabularray}
\begin{table*}
\caption{The results of proposed model and baselines.}
\label{results}
\centering
\begin{tblr}{
  cell{1}{1} = {r=2}{c},
  cell{1}{2} = {c=3}{c},
  cell{1}{5} = {c=3}{c},
  cell{1}{8} = {c=3}{c},
  cell{1}{11} = {c=3}c{},
  cell{16}{12} = {c=2}{c},
  vlines,
  hlines,
  column{1} = {c},
  column{2-13} = {c}
}
\textbf{Model}   & \textbf{O*NET}     &        &        & \textbf{ESCO}      &        &        & \textbf{AU LMI}    &        &        & \textbf{O*NET + ESCO + AU LMI} &        &        \\
        & \textbf{Precision} & \textbf{Recall} & \textbf{F1}     & \textbf{Precision} & \textbf{Recall} & \textbf{F1}     & \textbf{Precision} & \textbf{Recall} & \textbf{F1}     & \textbf{Precision}             & \textbf{Recall} & \textbf{F1}     \\
LR      & 0.6091    & 0.61   & 0.609  & 0.5915    & 0.5915 & 0.5911 & 0.657     & 0.6602 & 0.6566 & 0.5902                & 0.5922 & 0.5908 \\
SVM     & 0.6087    & 0.6072 & 0.6046 & 0.5738    & 0.5719 & 0.5707 & 0.6462    & 0.6473 & 0.6452 & 0.5837                & 0.5867 & 0.5811 \\
RF      & 0.5922    & 0.593  & 0.5921 & 0.5657    & 0.5656 & 0.5652 & 0.6449    & 0.6483 & 0.6399 & 0.5889                & 0.5909 & 0.5891 \\
BiLSTM  & 0.7487    & 0.7562 & 0.7532 & \textbf{0.7973}    & \textbf{0.7898} & \textbf{0.7876} & \textbf{0.8997}    & \textbf{0.8994} & \textbf{0.8995} & 0.7676                & 0.7701 & 0.7665 \\
GRU     & \textbf{0.7712}    & \textbf{0.7676} &\textbf{ 0.7712} & 0.7587    & 0.7612 & 0.7611 & 0.8677    & 0.8701 & 0.8664 & 0.7609                & 0.7624 & 0.7611 \\
CONV1D  & 0.7695    & 0.7742 & 0.7778 & 0.7745    & 0.7812 & 0.7822 & 0.8424    & 0.8402 & 0.8468 & 0.7766                & 0.7798 & 0.7723 \\
ALBERT  & 0.7703    & 0.6989 & 0.6988 & 0.7687    & 0.7723 & 0.7701 & 0.7997    & 0.8033 & 0.7946 & 0.7253                & 0.7114 & 0.6801 \\
ELECTRA & 0.7544    & 0.7621 & 0.7602 & 0.7384    & 0.7381 & 0.7285 & 0.8145    & 0.8093 & 0.8122 & 0.7459                & 0.7419 & 0.7425 \\
\textbf{BERT}    & 0.7698    & 0.7643 & 0.7594 & 0.7701    & 0.7723 & 0.7698 & 0.8241    & 0.8225 & 0.8198 & \textbf{0.7981}                & \textbf{0.7955} & \textbf{0.7944}   
\end{tblr}
\end{table*}

As we can see from the results, the GRU model demonstrates the highest performance on the O*NET dataset, with Precision, Recall, and F1-Score at 0.7712, 0.7676, and 0.7712, respectively. BiLSTM exhibits superior performance on the ESCO dataset, with Precision, Recall, and F1-Score at 0.7973, 0.7898, and 0.7876, respectively. The same model outperforms others on the AU LMI dataset, with all three metrics around 0.8995.\par

When all datasets are combined, our proposed BERT-based classifier emerges as the most effective model. The precision, recall, and F1-score are 0.7981, 0.7955, and 0.7944, respectively, which are the highest among all models. These results indicate the robustness of BERT in handling diverse and large-scale datasets and its superiority over traditional, neural network, and other transformer models for this multi-class classification task. We will use BERT as the best performance model for follow-up experiments and analysis.

\subsection{Data Augmentation Analysis}
In this section, we examine the impact of different augmentation levels on our BERT model's performance. The data augmentation methods include no augmentation (Original), augmenting data to match the class with the largest instances (Balanced), and incremental augmentation of the original data by 1.5, 2, 2.5, 3, 4, and 5 times. The results are shown below as Fig. \ref{fig: augmentation}:\par
\begin{figure}[htp!]
\centerline{\includegraphics[width=0.8\columnwidth]{svg-inkscape/data-comp_svg-raw.pdf}}
\caption{The Comparison of different augmentation.}
\label{fig: augmentation}
\end{figure}
A careful review of the results indicates that augmentation improves the precision of the model across all three datasets (ONET, ESCO, and AU LMI). For the ONET and ESCO datasets, the precision increases modestly from ``Balanced" to ``2 Times" and then plateaus slightly at ``2.5 Times". However, a dramatic rise is seen at ``4 Times", implying potential over-augmentation, which could lead to an overfitted model.\par
Therefore, we choose to augment our data between ``Balanced" and ``2 Times" for the O*NET and ESCO datasets. As for the AU LMI dataset, which already shows good performance, we only balance the classes.

\subsection{Model Robustness on Different Data Split}
To ascertain the robustness of our proposed model for real-world applications, we conducted an experiment with varying training set sizes. The dataset was shuffled, and training sets were created with different proportions of the original dataset, namely 80\%, 70\%, 60\%, 50\%, 40\%, 30\%, and 20\%. The remaining data in each scenario was used for performance evaluation. The results are presented at Fig. \ref{fig: robustness}:\par
\begin{figure}[htp!]
\centerline{\includegraphics[width=0.8\columnwidth]{svg-inkscape/trainingratio_svg-raw.pdf}}
\caption{The performance of model at different data split.}
\label{fig: robustness}
\end{figure}
As can be observed, the proposed BERT model's performance is relatively stable, and it gets better performance with the increase of training data ratio. Besides, our model with only 20\% training data can catch up with the baselines who use 80\% data for training, which validates the effectiveness of BERT in the task of predicting the automatability at task level.

\subsection{Attention Weights Visualization}
For the visualization of the model's attention, we have employed the technique of Wordcloud.
% It's important to note that in order to enhance the interpretability of these word clouds, we have carried out a token filtering process. Certain subword tokens generated by the BERT model, while meaningful in the model's internal representations, can appear as nonsensical when visualized in a word cloud. These include tokens which are fragments of words or specific to the BERT model like '[SEP]', '[CLS]', or 'ing s'. We have removed such tokens from our visualization to focus solely on the meaningful, interpretable tokens that can provide substantive insights into the model's decision-making process.\par
For class Substitution, the prominent terms include ``using system", ``machinery operate", ``data record", ``routine perform", and ``trucks load". These tasks are generally routine and predictable, and hence, are more prone to automation. On the other hand, for class Complementarity and class Negligibility, the key terms hint at tasks that require a higher level of human judgment or interaction like ``information provide", ``educational program", ``research conduct", ``medical procedures", ``human expertise", and ``children care". These tasks correlate with recognized automation bottlenecks, corroborating the model's predictions. The results can be found at Fig. \ref{fig: wordcloud}:
\begin{figure}[htp!]
\centerline{\includegraphics[width=0.8\columnwidth]{svg-inkscape/wordcloud_svg-raw.pdf}}
\caption{The Wordclouds for different categories.}
\label{fig: wordcloud}
\end{figure}

\section{Results and Discussion}
Having established the optimal performance of our BERT model through a series of experiments, we moved to a practical application, leveraging the model for inference on real-world datasets. We selected O*NET task statements, a rich and diverse dataset that encapsulates a broad variety of professional tasks, totalling 19,530 individual tasks. The results showed that among the total tasks, 6664 tasks were labeled as ``Substitution", 10,678 tasks were ``Complementarity" and 2188 tasks were ``Negligibility" which could be visualized as Fig. \ref{fig: distrubution}:\par
\begin{figure}[htp!]
\centerline{\includegraphics[width=0.8\columnwidth]{svg-inkscape/results-distribution_svg-raw.pdf}}
\caption{The Distribution of O*NET Task Automatability.}
\label{fig: distrubution}
\end{figure}
\subsection{Assessment of Automatability at Occupation Level}
We mapped individual task automatability measures to 974 distinct occupations using the O*NET task statement to occupation mapping. This allowed us to assess automatability at an occupation level by aggregating tasks and quantifying task type distributions. The results, summarized in Table \ref{fig: top10}, show the top 10 occupations with the highest substitution and negligibility, indicating the most and least likely automated occupations, respectively.
\begin{figure}[htp!]
\centerline{\includegraphics[width=0.8\columnwidth]{svg-inkscape/top10_svg-raw.pdf}}
\caption{Top 10 occupations that have highest and lowest automatabilities.}
\label{fig: top10}
\end{figure}
As can be seen from the results, occupations with high automation susceptibility are primarily manual labor, repetitive tasks, or data-intensive roles, such as "Dishwashers" and "Packaging and Filling Machine Operators and Tenders". Conversely, roles requiring human interaction, creativity, specialized skills, or unpredictable environments, like "Athletes and Sports Competitors" and "Respiratory Therapists", showed lower automatability. \par
% Conversely, occupations in areas that require high levels of human interaction, creative thinking, specialized skills, or involve unpredictable environments demonstrated lower automatability. Occupations like ``Athletes and Sports Competitors', ``Sales Agents, Financial Services', and ``Respiratory Therapists' are less prone to be substituted by automation, with substitution percentages around 11.1\%, 12.5\%, and 31.8\% respectively.\par
Our research findings indicate that out of 974 ONET occupations, 244 display a substitution score exceeding 50\%. This suggests that approximately 25.1\% of occupations within the ONET database are at substantial risk of automation. This outcome is aligned with the findings put forth in \cite{dawei}, thus adding robustness to our estimations. Furthermore, these results are nested between the estimations reported in \cite{freyosborne} which places 47\% of occupations at risk and \cite{arntz} which estimates the figure at a comparatively lower 9\%. Considering the limitations often associated with automated risk evaluations at the occupation and job level, the median value that our research arrives at seems reasonable. This is due to our methodology which leverages granular task statement data to predict occupation-level risks.\par
Additionally, our results indicate that a majority of occupations, specifically 603 out of 974 which is 61.8\%, face a high risk of complementarities. Conversely, a relatively smaller fraction, constituting 128 occupations or 13.1\%, appear to be comparatively safe from impending automation over the forthcoming decades. These conclusions underscore the nuanced complexities underpinning automation risks and the need for more granular data analysis in this domain.
\subsection{Assessment of Automation Vulnerability Among Industries}
Building on our earlier findings of automatability at the occupation level, we bridge the occupation-industry gap utilizing O*NET's detailed occupation-industry mapping. Our results shown as Fig. \ref{fig: top5} indicate a diverse spectrum of automation vulnerability across sectors. On one end, industries such as ``Accommodation and Food Services", ``Administrative and Support Services", ``Retail Trade", ``Mining, Quarrying, and Oil and Gas Extraction", and ``Manufacturing" emerge as the sectors most susceptible to automation. These industries comprise occupations with high automatability scores, revealing their high likelihood of experiencing significant changes due to automation.\par
On the other hand, industries such as ``Educational Services", ``Arts, Entertainment and Recreation", ``Other Services (Except Public Administration)", ``Real Estate and Rental and Leasing", and ``Health Care and Social Assistance" stand on the less vulnerable end of the automation spectrum. Occupations within these sectors possess low automatability scores, indicating a lower likelihood of their roles being fully automated, largely due to the complexity of tasks or the high level of human judgment, interaction, and creativity required.\par
\begin{figure}[htp!]
\centerline{\includegraphics[width=1\columnwidth]{svg-inkscape/top5_svg-raw.pdf}}
\caption{Industries with highest and lowest automatabilities.}
\label{fig: top5}
\end{figure}
% % \begin{figure}[htbp]
% \centerline{\includegraphics[width=1\columnwidth]{top5l.png}}
% \caption{Example of a figure caption.}
% \label{fig}
% \end{figure}

To validate the results, we leveraged the insights from the \cite{mckinsey2017jobs}, and identified 4 sectors showing siginifican susceptibility to automation which are ``Accommodation and Food Services", ``Manufacturing", ``Retail Trade", and ``Transportation and Warehousing". These findings align well with our own results.

\section{Conclusion}
This study presents a unique approach to predict task-level automatability with a BERT-based classifier, utilizing three diverse, public datasets and expert annotations. Rigorous experiments demonstrated our model's efficacy.\par
In practical application, we applied our model to real-world datasets, providing a comprehensive perspective of occupational automatability. Our findings indicate that approximately 25.1\% of occupations within the O*NET database are at substantial risk of automation. Furthermore, our results reveal a diverse spectrum of automation vulnerability across sectors, with industries such as ``Accommodation and Food Services", ``Administrative and Support Services", ``Retail Trade", ``Mining, Quarrying, and Oil and Gas Extraction", and ``Manufacturing" emerging as the sectors most susceptible to automation.\par
These findings have significant implications for workers and policymakers. For workers, understanding the susceptibility of their tasks to automation can help them make informed decisions about their career paths and upskilling opportunities. For policymakers, these insights can guide the development of policies and initiatives aimed at managing the transition to an increasingly automated workforce. This could include strategies for retraining and reskilling workers, as well as measures to support industries and regions that are particularly vulnerable to automation.\par
In conclusion, our research provides a robust and effective approach to predicting task-level automatability, offering valuable insights for policymakers, educators, and workers. By understanding the potential impact of automation on different tasks, occupations, and industries, we can better prepare for the future of work.\par

\section*{Acknowledgment}
This work is supported by the Australian Research Council (ARC) under Grant No. DP220103717, and LE220100078.

\documentclass[conference]{IEEEtran}
\IEEEoverridecommandlockouts
% The preceding line is only needed to identify funding in the first footnote. If that is unneeded, please comment it out.
\usepackage{cite}
\usepackage{amsmath,amssymb,amsfonts}
\usepackage{algorithmic}
\usepackage{graphicx}
\usepackage{textcomp}
\usepackage{xcolor}
\def\BibTeX{{\rm B\kern-.05em{\sc i\kern-.025em b}\kern-.08em
    T\kern-.1667em\lower.7ex\hbox{E}\kern-.125emX}}
\begin{document}

\title{Privacy Preserving Properties \\of \\Vision Classifiers\\
%{\footnotesize \textsuperscript{*}Note: Sub-titles are not captured in %should not be used}
%\thanks{Identify applicable funding agency here. If none, delete this.}
}

%\author{\IEEEauthorblockN{Anonymous Authors}}

\author{\IEEEauthorblockN{1\textsuperscript{st} Pirzada Suhail}
\IEEEauthorblockA{\textit{IIT Bombay} \\
Mumbai, India \\
psuhail@iitb.ac.in}
\and
\IEEEauthorblockN{2\textsuperscript{nd} Amit Sethi}
\IEEEauthorblockA{\textit{IIT Bombay} \\
Mumbai, India \\
asethi@iitb.ac.in}
}

\maketitle

\begin{abstract}

Vision classifiers are often trained on proprietary datasets containing sensitive information, yet the models themselves are frequently shared openly under the privacy-preserving assumption. Although these models are assumed to protect sensitive information in their training data, the extent to which this assumption holds for different architectures remains unexplored. This assumption is challenged by inversion attacks which attempt to reconstruct training data from model weights, exposing significant privacy vulnerabilities. In this study, we systematically evaluate the privacy-preserving properties of vision classifiers across diverse architectures, including Multi-Layer Perceptrons (MLPs), Convolutional Neural Networks (CNNs), and Vision Transformers (ViTs). Using network inversion-based reconstruction techniques, we assess the extent to which these architectures memorize and reveal training data, quantifying the relative ease of reconstruction across models. Our analysis highlights how architectural differences, such as input representation, feature extraction mechanisms, and weight structures, influence privacy risks. By comparing these architectures, we identify which are more resilient to inversion attacks and examine the trade-offs between model performance and privacy preservation, contributing to the development of secure and privacy-respecting machine learning models for sensitive applications. Our findings provide actionable insights into the design of secure and privacy-aware machine learning systems, emphasizing the importance of evaluating architectural decisions in sensitive applications involving proprietary or personal data.

\end{abstract}

\begin{IEEEkeywords}
Safety, Privacy, Network Inversion, Reconstructions
\end{IEEEkeywords}

\section{Introduction}

The advent of modern vision classifiers has revolutionized a wide range of applications, from autonomous vehicles and medical imaging to facial recognition. These models, often trained on proprietary datasets, have become a cornerstone of advancements in artificial intelligence (AI). However, the growing practice of sharing pre-trained models has raised critical concerns about the privacy of the training data. Many assume that the act of sharing a trained model inherently preserves the privacy of the underlying dataset, but this assumption is increasingly being challenged by research demonstrating the potential for data leakage. The question remains: to what extent can different model architectures safeguard sensitive information against reconstruction or inversion attacks?

Model inversion attacks, where an adversary attempts to reconstruct the training data from a model's weights or output, highlight the vulnerability of machine learning systems to privacy breaches. This becomes particularly alarming in domains like healthcare, where proprietary datasets often contain highly sensitive information, or in applications involving biometric data, where privacy is paramount. Vision classifiers, in particular, process inherently personal and identifiable information, such as faces or medical scans, making it imperative to evaluate their privacy-preserving properties comprehensively.

In this study, we systematically assess the privacy-preserving capabilities of three prominent architectures: Multi-Layer Perceptrons (MLPs), Convolutional Neural Networks (CNNs)\cite{dosovitskiy2021imageworth16x16words}, and Vision Transformers (ViTs)\cite{oshea2015introductionconvolutionalneuralnetworks}. These architectures differ significantly in their structure, feature extraction mechanisms, and input processing pipelines, which could influence their tendencies to memorize and inadvertently leak training data. For example, CNNs are designed to capture spatial hierarchies and local patterns, while ViTs, on the other hand, employ self-attention mechanisms that focus on global relationships between input elements, potentially resulting in different privacy implications. By comparing these architectures, we aim to understand their relative vulnerabilities and the factors that contribute to privacy risks.

To evaluate the privacy-preserving properties of these classifiers, we utilize network inversion-based reconstruction techniques as in \cite{suhail2024net}. These methods attempt to recover the training data entirely from the model weights, providing a quantifiable measure of privacy leakage. Such reconstruction attacks exploit the tendency of models to memorize specific details about their training data, particularly when the training set is small or when over-parameterized architectures are used. By systematically applying these techniques, we assess the relative ease of reconstruction across architectures and analyze the impact of architectural differences on privacy.

We conduct our analysis in the most extreme case of privacy risk: at the end of the training process, without any knowledge of the training process itself, without utilizing any unobvious prior information, without any auxiliary datasets and relying on a single trained model. This setup provides a stringent evaluation of privacy risks, focusing on the inherent vulnerabilities of the model architectures. Further the quality of the reconstructed samples is assessed by comparing them to the original training samples using a similarity metric. Reconstructions with higher Structural Similarity Index Measure (SSIM) values indicate a greater privacy risk, as they suggest more effective memorization of the training data by the model. 

Our findings highlight that architectural differences in processing input images, feature extraction, and weight structures contribute to varying degrees of privacy leakage. We apply our evaluation to multiple benchmark datasets, including MNIST, FashionMNIST, CIFAR-10, and SVHN that cover a wide range of complexities and image types, allowing us to study how different architectures behave under varying data conditions.

\section{Related Works}

Privacy concerns in machine learning have led to extensive research in Privacy-Preserving Machine Learning (PPML), particularly in mitigating risks related to membership inference, attribute inference, and model inversion attacks. A foundational study by \cite{8677282} provides an overview of privacy threats in ML, including model inversion, and explores defenses such as differential privacy, homomorphic encryption, and federated learning. Similarly, \cite{xu2021privacypreservingmachinelearningmethods} introduces the Phase, Guarantee, and Utility (PGU) triad, a framework to evaluate PPML techniques across different phases of the ML pipeline. These studies highlight the need for privacy-preserving methods but primarily focus on algorithmic-level defenses rather than evaluating inherent vulnerabilities in different model architectures. Unlike these approaches, our study investigates the privacy risks posed by architectural design choices by analyzing how MLPs, CNNs, and ViTs differ in their susceptibility to model inversion attacks.

Network inversion has emerged as a powerful technique to understand how neural networks encode and manipulate training data. Initially developed for interpretability \cite{KINDERMANN1990277,784232}, it has since been shown to reconstruct sensitive training samples, raising significant privacy concerns \cite{Wong2017NeuralNI,ad}. Early works on network inversion focused on fully connected networks (MLPs) \cite{KINDERMANN1990277,SAAD200778}, demonstrating that they tend to memorize training data, making them vulnerable to inversion attacks. Evolutionary inversion procedures \cite{784232} improved the ability to capture input-output relationships, providing deeper insights into model memorization behavior. More recent studies extended these inversion techniques to CNNs \cite{ad}, showing that hierarchical feature extraction does not necessarily prevent training data leakage. The introduction of ViTs has further complicated this issue, as their global self-attention mechanisms process data differently than CNNs, raising new questions about how they store training information and whether their memorization patterns lead to higher or lower inversion risks.

In adversarial settings, model inversion attacks aim to reconstruct sensitive data by exploiting a model's predictions, gradients, or weights \cite{ad,kumar2019modelinversionnetworksmodelbased}. These attacks have been shown to succeed even without direct access to the training process, as demonstrated by \cite{9833677}, where an adversary with auxiliary knowledge reconstructs sensitive samples. Gradient-based inversion attacks further exacerbate these risks by leaking sensitive training information through shared gradients in federated learning setups \cite{pmlr-v206-wang23g}.

To improve the stability of inversion processes, recent works have explored novel optimization techniques. For example, \cite{liu2022landscapelearningneuralnetwork} proposed learning a loss landscape to make gradient-based inversion faster and more stable. Alternative approaches, such as encoding networks into Conjunctive Normal Form (CNF) and solving them using SAT solvers, offer deterministic solutions for inversion, as introduced by \cite{suhail2024network}. Although computationally expensive, these methods ensure diversity in the reconstructed samples by avoiding shortcuts in the optimization process.

Model Inversion (MI) attacks have also been extended to scenarios involving ensemble techniques, where multiple models trained on shared subjects or entities are used to guide the reconstruction process. The concept of ensemble inversion, as proposed by \cite{wang2021reconstructingtrainingdatadiverse}, enhances the quality of reconstructed data by leveraging the diversity of perspectives provided by multiple models. By incorporating auxiliary datasets similar to the presumed training data, this approach achieves high-quality reconstructions with sharper predictions and higher activations. This work highlights the risks posed by adversaries exploiting shared data entities across models, emphasizing the importance of robust defense mechanisms.

Reconstruction methods for training data have evolved significantly, focusing on improving the efficiency and accuracy of recovering data from models. Traditional optimization-based approaches relied on iteratively refining input data to match a model’s outputs or activations \cite{Wong2017NeuralNI}. More recent advancements have leveraged generative models, such as GANs and autoencoders, to synthesize high-quality reconstructions. These techniques aim to approximate the distribution of training data while maintaining computational efficiency. In the context of privacy risks, works like \cite{haim2022reconstructingtrainingdatatrained} demonstrated that significant portions of training data could be reconstructed from neural network parameters in binary classification settings. This work was later extended to multi-class classification by \cite{buzaglo2023reconstructingtrainingdatamulticlass}, showing that higher-quality reconstructions are possible and revealing the impact of regularization techniques, such as weight decay, on memorization behavior.

The ability to reconstruct training data from model gradients also presents a critical privacy challenge. The study by \cite{wang2023reconstructingtrainingdatamodel} demonstrated that training samples could be fully reconstructed from a single gradient query, even without explicit training or memorization. Recent advancements like \cite{oz2024reconstructingtrainingdatareal}, adapt reconstruction schemes to operate in the embedding space of large pre-trained models, such as DINO-ViT and CLIP. This approach, which introduces clustering-based methods to identify high-quality reconstructions from numerous candidates, represents a significant improvement over earlier techniques that required access to the original dataset. While \cite{pmlr-v162-guo22c} extended differential privacy guarantees to training data reconstruction attacks.

In this paper, we build upon prior work on network inversion and training data reconstruction. Drawing from works like \cite{suhail2024networkcnn} and \cite{suhail2024networkinversionapplications}, we employ network inversion methods to understand the internal representations of neural networks and the patterns they memorize during training. Our study systematically compares the privacy-preserving properties of different vision classifier architectures, including MLPs, CNNs, and ViTs, using network inversion-based reconstruction techniques \cite{suhail2024net}. By evaluating these techniques across datasets such as MNIST, FashionMNIST, CIFAR-10, and SVHN, we explore the impact of architectural differences, input processing mechanisms, and weight structures on the susceptibility of models to inversion attacks.


\section{Methodology}

\subsection{Overview}
In this study, we investigate the ease of reconstruction and the extent of memorization in trained vision classifiers based on different architectures. Our primary focus is on analyzing the most extreme case of training data reconstruction, where the inversion process relies almost entirely on the input-output relationships of the trained model and its learned weights. Unlike prior reconstruction approaches that leverage pre-trained models, auxiliary datasets, gradient information from the training process, or other unobvious priors, our method seeks to reconstruct training data with minimal external dependencies. This approach allows us to systematically evaluate how different architectures—Multi-Layer Perceptrons (MLPs), Convolutional Neural Networks (CNNs), and Vision Transformers (ViTs)—differ in their ability to preserve or expose sensitive training data.

To perform network inversion and data reconstruction, we build upon the methodology introduced in \cite{suhail2024networkcnn, suhail2024networkinversionapplications}, which has primarily focused on CNN-based classifiers. We extend this approach to other architectures, particularly MLPs and ViTs, to assess their relative vulnerability to inversion attacks. Briefly, network inversion techniques aim to generate inputs that align with the learned decision boundaries of a classifier by training a conditioned generator to reconstruct data that maximally activates specific output neurons. In its standard form, this inversion process does not necessarily yield images resembling actual training samples but instead produces arbitrary inputs that satisfy the model’s learned function. However, by modifying the inversion procedure following \cite{suhail2024network, suhail2024net}, we incentivize the generator to reconstruct training-like data by leveraging key properties of the classifier with respect to its training data. These modifications allow us to better quantify the extent of memorization across different architectures and assess the associated privacy risks. The proposed approach to Network Inversion and subsequent training data reconstruction uses a carefully conditioned generator that learns the data distributions in the input space of the trained classifier.

\subsection{Classifier Architectures}
We perform inversion and reconstruction on classifiers based on three distinct architectures:

\begin{itemize}
    \item \textbf{Multi-Layer Perceptrons (MLPs)}: These are fully connected networks where each neuron in one layer is connected to every neuron in the next layer. MLPs process flattened input images, lacking any inherent spatial hierarchy. The inversion and subsequent reconstruction will be performed using the logits and penultimate fully connected layers.
    \item \textbf{Convolutional Neural Networks (CNNs)}: CNNs use convolutional layers to extract hierarchical spatial features from images, enabling them to effectively capture local patterns. In this case the features from the fully connected layers are used after flattening the output of convolutional layers along with the logits in the last layer to perform inversion.
    \item \textbf{Vision Transformers (ViTs)}: ViTs utilize self-attention mechanisms to capture global dependencies across an image. This architecture is particularly effective in modeling long-range interactions within an image. In ViTs we particularly look at the classification token embeddings and use it in the same way as above.
\end{itemize}

These architectures have inherently different memory capacities and generalization properties, affecting their susceptibility to reconstruction attacks.

\subsection{Vector-Matrix Conditioned Generator}
The generator in our approach is conditioned on vectors and matrices to ensure that it learns diverse representations of the data distribution. Unlike simple label conditioning, the vector-matrix conditioning mechanism encodes the label information more intricately, allowing the generator to better capture the input space of the classifier. 

The generator is initially conditioned using $N$-dimensional vectors for an $N$-class classification task. These vectors are derived from a normal distribution and are softmaxed to form a probability distribution. They implicitly encode the labels, promoting diversity in the generated images.

Further, a Hot Conditioning Matrix of size $N \times N$ is used for deeper conditioning. In this matrix, all elements in a specific row or column are set to $1$, corresponding to the encoded label, while the rest are $0$. This conditioning is applied during intermediate stages of the generation process to refine the diversity of the outputs.

\subsection{Training Data Properties}
The classifier exhibits specific properties when interacting with training data, which are exploited to facilitate the reconstruction of training-like samples:
\begin{figure*}[t]
\centering
\includegraphics[width=1\textwidth]{tldr.png} 
\caption{Schematic Approach to Training-Like Data Reconstruction using Network Inversion}
\label{fig:reconstruction}
\end{figure*}

\begin{itemize}
    \item \textbf{Model Confidence:} The classifier is more confident when predicting labels for training samples compared to random samples. Hence, in order to take this into account we condition the generation on one-hot vectors and then minimise its KL Divergence from the classifier's output enforcing generation of samples that are confidently classified buy the classifier. This can be expressed as:
    \begin{equation}
    P(y_{\text{in}} | x_{\text{in}}; \theta) \gg P(y_{\text{ood}} | x_{\text{ood}}; \theta)
    \end{equation}

    \item \textbf{Robustness to Perturbations:} During training the model gets to observe a diverse set of data including different variations of the images in the same class. Due to which the model is relatively robust to perturbations around training data compared to random inverted samples, meaning small changes do not significantly affect predictions. We take this into account in by perturbing the generated images and then ensuring that the perturbed images also produce similar output,
    \begin{equation}
    \frac{\partial f_{\theta}(x_{\text{in}})}{\partial x_{\text{in}}} \ll \frac{\partial f_{\theta}(x_{\text{ood}})}{\partial x_{\text{ood}}}
    \end{equation}

    \item \textbf{Gradient Behavior:} By virtue of training the model on a certain dataset, the gradient of the loss with respect to model weights is expected to be lower for training data compared to random inverted samples, as the model has already been optimized on it, Hence we add a penalty on the gradients that encourages the generation of samples with low gradients as:
    \begin{equation}
    \|\nabla_{\theta} L(f_{\theta}(x_{\text{in}}), y_{\text{in}})\| \ll \|\nabla_{\theta} L(f_{\theta}(x_{\text{ood}}), y_{\text{ood}})\|
    \end{equation}
\end{itemize}

\subsection{Training Data Reconstruction}
The reconstruction process utilizes the generator to produce training-like samples by taking into account the specific properties of the training data with respect to the classifier. The approach is schematically illustrated in Figure \ref{fig:reconstruction}. 


The primary loss function used for reconstruction, 
\begin{align*}
\mathcal{L}_{\text{Recon}} = & \; \alpha \cdot \mathcal{L}_{\text{KL}} 
+ \alpha' \cdot \mathcal{L}_{\text{KL}}^{\text{pert}}
+ \beta \cdot \mathcal{L}_{\text{CE}} 
+ \beta' \cdot \mathcal{L}_{\text{CE}}^{\text{pert}} \\
& + \gamma \cdot \mathcal{L}_{\text{Cosine}} 
+ \delta \cdot \mathcal{L}_{\text{Ortho}} \\
& + \eta_1 \cdot \mathcal{L}_{\text{Var}} 
+ \eta_2 \cdot \mathcal{L}_{\text{Pix}} 
+ \eta_3 \cdot \mathcal{L}_{\text{Grad}}
\end{align*}

where \( \mathcal{L}_{\text{KL}} \) is the KL Divergence loss, \( \mathcal{L}_{\text{CE}} \) is the Cross Entropy loss, \( \mathcal{L}_{\text{Cosine}} \) is the Cosine Similarity loss, and \( \mathcal{L}_{\text{Ortho}} \) is the Feature Orthogonality loss. The hyperparameters \( \alpha, \beta, \gamma, \delta \) control the contribution of each individual loss term defined as:
\[
\mathcal{L}_{\text{KL}} = D_{\text{KL}}(P \| Q) = \sum_{i} P(i) \log \frac{P(i)}{Q(i)}
\]
\[
\mathcal{L}_{\text{CE}} = -\sum_{i} y_{i} \log(\hat{y}_{i})
\]
\[
\mathcal{L}_{\text{Cosine}} = \frac{1}{N(N-1)} \sum_{i \neq j} \cos(\theta_{ij})
\]
\[
\mathcal{L}_{\text{Ortho}} = \frac{1}{N^2} \sum_{i, j} (G_{ij} - \delta_{ij})^2
\]
where \( D_{\text{KL}} \) represents the KL Divergence between the input distribution \( P \) and the output distribution \( Q \), \( y_{i} \) is the set encoded label, \( \hat{y}_{i} \) is the predicted label from the classifier, \( \cos(\theta_{ij}) \) represents the cosine similarity between features of generated images \( i \) and \( j \), \( G_{ij} \) is the element of the Gram matrix, and \( \delta_{ij} \) is the Kronecker delta function. \( N \) is the number of feature vectors in the batch.

Further to take reconstruction into account we also use \(\mathcal{L}_{\text{KL}}^{\text{pert}}\) and \(\mathcal{L}_{\text{CE}}^{\text{pert}}\) that represent the KL divergence and cross-entropy losses applied on perturbed images, weighted by \( \alpha'\) and  \(\beta' \)respectively while \(\mathcal{L}_{\text{Var}}\), \(\mathcal{L}_{\text{Pix}}\) and \(\mathcal{L}_{\text{Grad}}\) represent the variational loss, Pixel Loss and penalty on gradient norm each weighted by \( \eta_1\), \( \eta_2\), and \(\eta_3\) respectively and defined for an Image \(I\) as:
\begin{align*}
\mathcal{L}_{\text{Var}} = \frac{1}{N} \sum_{i=1}^{N} \Bigg( \sum_{h,w} & \Big( ( I_{i, h+1, w} - I_{i, h, w} )^2 \\
& + ( I_{i, h, w+1} - I_{i, h, w} )^2 \Big) \Bigg)
\end{align*}
\[
\mathcal{L}_{\text{Grad}} = \left\| \nabla_{\theta} L(f_{\theta}(I), y) \right\|
\]

\[
\mathcal{L}_{\text{Pix}} = \sum \max(0, -I) + \sum \max(0, I - 1)
\]

By integrating these loss components, the generator is trained to produce samples that closely resemble the training data, thus revealing the extent of memorization within the classifier.


In this section, we present the experimental results obtained by applying the reconstruction technique on the MNIST \cite{deng2012mnist}, FashionMNIST \cite{xiao2017fashionmnistnovelimagedataset}, SVHN, and CIFAR-10 \cite{cf} datasets. Our goal is to evaluate the ease of reconstruction and the extent of memorization in different vision classifier architectures by training a generator to produce images that resemble the training data. The classifier is first trained normally on a given dataset and then held in evaluation mode for the purpose of reconstruction. The conditioned generator, which takes as input latent vectors and conditioning information, is trained to generate images that the classifier maps to specific labels.

We evaluate three vision classifier architectures: Multi-Layer Perceptrons (MLPs), Convolutional Neural Networks (CNNs), and Vision Transformers (ViTs). We implement a 5-layer MLP with Batch Normalization, Leaky ReLU activations, and Dropout layers \cite{JMLR:v15:srivastava14a} to mitigate memorization tendencies. The CNN consists of three convolutional layers followed by batch normalization \cite{pmlr-v37-ioffe15}, dropout layers, and a fully connected layer for classification. The ViT classifier uses a transformer-based architecture with three self-attention layers, each containing four attention heads while the input images are divided into non-overlapping patches of size \(4 \times 4\).

The generator, instead of traditional label embeddings, follows a Vector-Matrix Conditioning approach to ensure diverse and structured image generation. The class labels are encoded into random softmaxed vectors concatenated with the latent vector, followed by multiple layers of transposed convolutions, batch normalization, and dropout layers to encourage diversity in generated images. Once the concatenated latent and conditioning vectors are upsampled to \(N \times N\) spatial dimensions for an \(N\)-class classification task, they are concatenated with a conditioning matrix for further generation up to the required image size (\(28 \times 28\) for MNIST and FashionMNIST, and \(32 \times 32\) for SVHN and CIFAR-10).

To assess the extent of memorization and the relative ease of reconstruction across different architectures, we use the Structural Similarity Index Measure (SSIM). The goal was to evaluate how different architectures handle privacy risks by comparing the quality of reconstructed images. SSIM quantifies the perceptual similarity between two images based on luminance, contrast, and structural similarity components. In this study, SSIM is used to compare the reconstructed images with their closest matches from the training dataset. Higher SSIM values indicate stronger resemblance to training samples, suggesting a greater privacy risk due to increased memorization by the classifier.

Table~\ref{tab:ssim_values} presents the SSIM scores between reconstructed samples and training data, averaged across all classes for each dataset and architecture. Higher SSIM values suggest greater memorization and weaker privacy preservation, as the reconstructed samples strongly resemble real training data.

\begin{table}[h]
\centering
\caption{SSIM values for reconstructed samples across different architectures and datasets.}
\label{tab:ssim_values}
\resizebox{\linewidth}{!}{
\begin{tabular}{|l|c|c|c|}
\hline
\textbf{Dataset}       & \textbf{MLP} & \textbf{ViT} & \textbf{CNN} \\ \hline
\textbf{MNIST}         & 0.83         & 0.78         & 0.73         \\ \hline
\textbf{FashionMNIST}  & 0.74         & 0.64         & 0.63         \\ \hline
\textbf{SVHN}          & 0.71         & 0.68         & 0.69         \\ \hline
\textbf{CIFAR-10}      & 0.65         & 0.62         & 0.58         \\ \hline
\end{tabular}
}
\end{table}

From Table~\ref{tab:ssim_values}, we observe that MLPs exhibit the highest SSIM scores across all datasets, suggesting that they retain more training data details than CNNs and ViTs. This is likely due to their fully connected nature and lack of spatial inductive biases, leading to higher memorization tendencies. Also individual pixels in the images have dedicated weights associated, making memorization easier.

ViTs generally show lower SSIM values than MLPs but remain slightly higher than CNNs, indicating that self-attention mechanisms contribute to retaining finer details in the reconstructed images. Unlike CNNs, ViTs lack pooling layers, which means that more spatial information is preserved, making inversion attacks more feasible. 

CNNs show the lowest SSIM values across all datasets, suggesting that they inherently discard more specific input details due to weight sharing, local receptive fields, and pooling operations. This abstraction reduces direct memorization and makes CNNs relatively more privacy-preserving compared to MLPs and ViTs.

\begin{figure}[ht]
\centering
\includegraphics[width=0.95\linewidth]{riz.png} 
\caption{Comparison of reconstructed samples across MLP, ViT, and CNN architectures. The first column represents actual training samples, while subsequent columns show corresponding reconstructed images.}
\label{fig:sam}
\end{figure}

Figure~\ref{fig:sam} presents a qualitative comparison of reconstructed samples across different classifier architectures. The first column depicts actual training samples, while the subsequent columns display reconstructed images generated from MLPs, ViTs, and CNNs. Notably, MLP-generated reconstructions appear most similar to the original samples, while CNNs yield more abstract representations, indicating lower memorization.

These results highlight the variation in privacy risks across architectures, with MLPs being the most susceptible to memorization, followed by ViTs, and CNNs exhibiting the lowest reconstruction fidelity. The findings emphasize the importance of architectural choices in designing privacy-aware models, where CNNs may be preferable for applications requiring privacy preservation.

\section{Conclusion and Future Work}
In this paper, we systematically evaluated the privacy-preserving properties of vision classifiers by analyzing the extent of memorization and the ease of training data reconstruction across different architectures. Our experimental results, quantified through SSIM scores, indicate that MLPs tend to memorize more information about training samples compared to CNNs and ViTs, making them more prone to reconstruction. These findings highlight the crucial role of architectural choices in mitigating privacy risks, emphasizing the need for privacy-aware model deployment, especially in sensitive applications.

Future research can extend this study by exploring additional factors that influence privacy leakage, such as model depth, dataset complexity, and the impact of regularization techniques. In the case of ViTs it would be also of interest to see how the reconstructions differ with varying patch size. Investigating the impact of differential privacy mechanisms in reducing reconstruction risks could provide valuable insights into enhancing privacy.

\bibliographystyle{IEEEtran}  % Or any other preferred style
\bibliography{references} 

\end{document}


% \bibliographystyle{IEEEtran}
% \bibliography{bibs}

% \onecolumn
% \begin{appendices}
% \clearpage
% \section{The Results of Occupations' Automatability}
% \includepdf[pages=-]{class_percentage_output.pdf}
% \break
% \section{The Results of Industries' Automatability}
% \includepdf[pages=-]{industry_automatability_final.pdf}

% \end{appendices}

\onecolumn
\begin{appendices}
\clearpage
\includepdf[pages=1, pagecommand={\begin{minipage}{\textwidth}\section{The Results of Occupations' Automatability}\vspace{1cm}\end{minipage}}]{class_percentage_output.pdf}
\includepdf[pages=2-, pagecommand={}]{class_percentage_output.pdf}
\includepdf[pages=1, pagecommand={\begin{minipage}{\textwidth}\section{The Results of Industries' Automatability}\vspace{1cm}\end{minipage}}]{industry_automatability_final.pdf}
\end{appendices}

% \onecolumn
% \begin{appendices}
% \clearpage
% \includepdf[pages=1, pagecommand={\section{The Results of Occupations' Automatability}}]{class_percentage_output.pdf}
% \includepdf[pages=2-24, pagecommand={}]{class_percentage_output.pdf}
% \includepdf[pages=25, pagecommand={\afterpage{ Notes: Your explanation here.}}]{class_percentage_output.pdf}

% \includepdf[pages=1, pagecommand={\section{The Results of Industries' Automatability}\afterpage{\vspace*{5cm} Notes: Your explanation here. \clearpage}}]{industry_automatability_final.pdf}
% \end{appendices}




\end{document}
