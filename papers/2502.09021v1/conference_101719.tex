\pdfoutput=1
\documentclass[conference]{IEEEtran}
% \documentclass[final]{IEEEtran}
\IEEEoverridecommandlockouts
% The preceding line is only needed to identify funding in the first footnote. If that is unneeded, please comment it out.
\usepackage[noadjust]{cite}
\usepackage{amsmath,amssymb,amsfonts}
\usepackage{algorithmic}
\usepackage{graphicx}
\usepackage{textcomp}
\usepackage{adjustbox}
\usepackage{svg}
\usepackage{tabularray}
\usepackage{xcolor}
\usepackage{pdfpages}
\usepackage{url}
\usepackage{afterpage}

\def\BibTeX{{\rm B\kern-.05em{\sc i\kern-.025em b}\kern-.08em
    T\kern-.1667em\lower.7ex\hbox{E}\kern-.125emX}}

\newcommand{\customfont}{\fontsize{6.5}{7.8}\selectfont}
\usepackage{hyperref}
\begin{document}

\title{From Occupations to Tasks: A New Perspective on Automatability Prediction Using BERT\\
}

% \author{\IEEEauthorblockN{Anonymous Author}
% }
\author{\IEEEauthorblockN{Dawei Xu}
\IEEEauthorblockA{\textit{University of Technology Sydney} \\
Sydney, Australia \\
dawei.xu@student.uts.edu.au}
\and
\IEEEauthorblockN{Haoran Yang}
\IEEEauthorblockA{\textit{University of Technology Sydney} \\
Sydney, Australia \\
haoran.yang-2@student.uts.edu.au}
\and
\IEEEauthorblockN{Marian-Andrei Rizoiu}
\IEEEauthorblockA{\textit{University of Technology Sydney} \\
Sydney, Australia \\
marian-andrei.rizoiu@uts.edu.au}
\and
\IEEEauthorblockN{Guandong Xu}
\IEEEauthorblockA{\textit{University of Technology Sydney} \\
Sydney, Australia \\
guandong.xu@uts.edu.au}
}

\maketitle

\begin{abstract}
As automation technologies continue to advance at an unprecedented rate, concerns about job displacement and the future of work have become increasingly prevalent. While existing research has primarily focused on the potential impact of automation at the occupation level, there has been a lack of investigation into the automatability of individual tasks. This paper addresses this gap by proposing a BERT-based classifier to predict the automatability of tasks in the forthcoming decade at a granular level leveraging the context and semantics information of tasks. We leverage three public datasets: O*NET Task Statements, ESCO Skills, and Australian Labour Market Insights Tasks, and perform expert annotation. Our BERT-based classifier, fine-tuned on our task statement data, demonstrates superior performance over traditional machine learning models, neural network architectures, and other transformer models. Our findings also indicate that approximately 25.1\% of occupations within the O*NET database are at substantial risk of automation, with a diverse spectrum of automation vulnerability across sectors. This research provides a robust tool for assessing the future impact of automation on the labor market, offering valuable insights for policymakers, workers, and industry leaders in the face of rapid technological advancement.
\end{abstract}

\begin{IEEEkeywords}
Task automatability prediction, Automated Occupation identification, BERT, Transfer learning
\end{IEEEkeywords}

\section{Introduction}
In the era of rapid technological advancement, the fear of large scale unemployment caused by automation technologies has become a significant concern. This concern is not unfounded; a growing body of research suggests that many occupations could be partially or fully automated in the coming years. According to the research conducted by \cite{freyosborne}, over 47\% of current US employment is at high risk of being automated. Another similar study has been performed using job-level data and concluded a less alarming rate of 9\% of employment is at risk \cite{arntz}. However, these studies have largely focused on the potential for automation at the occupation level, leaving a significant gap in our understanding of how individual tasks within these occupations might be automated and multiple studies have shown the effectiveness of assessing the automatability on tasks level instead of occupations \cite{towards, webb_2019_the}. \par
The automatability of tasks is fundamentally a question of understanding the nature and requirements of the task and this information is often embedded in the contextual and semantic information of textual task statements \cite{towards}. For example, the task "Write a report" might be automatable, but "Write a creative story" might not be. And the word ``run" means differently in "run a program" and "run a marathon" \cite{arntz}. Therefore, our research is based on the assumption that the automatability of tasks can be apprehended through the extraction of contextual and semantic information embedded within textual task statements.\par
Due to this circumstance, our research employ the Bidirectional Encoder Representations from Transformers (BERT) model \cite{devlin2018bert} to understand the context and semantics of words in a task statement allowing for a more sophisticated analysis of task descriptions, potentially leading to more accurate predictions of task automatability. We also utilize three task description datasets: O*NET Task statements \cite{onet}, ESCO Skills \cite{esco}, and Australian Labour Market Insights Tasks \cite{australian_labor_outlook}. We annotate each task with one of three labels: ``Substitution," ``Complementarity," or ``Negligibility," representing its automation susceptibility based on expert assessments. Using these annotations as ground truth, we apply a BERT-based classifier model \cite{devlin2018bert} to predict task automatability, benefiting from BERT's context understanding and semantic capabilities.\par

We extend our analysis to occupation and industry levels, using aggregated task-level predictions to assess overall automation susceptibility. This comprehensive view can reveal trends not observable at the task level, providing additional automation impact insights. Our research holds significant implications for workers, businesses, and policymakers by offering clarity on automation-susceptible tasks and demonstrating BERT's utility in labor economics.\par
% This paper aims to address this gap by investigating the automatability of tasks rather than entire occupations in the forthcoming decade. To achieve this, we utilize three comprehensive datasets: O*NET Task statements \cite{onet}, ESCO Skills \cite{esco}, and Australian Labour Market Insights
% Tasks \cite{australian_labor_outlook} which comprising detailed task descriptions across a wide range of occupations. Each task in the datasets has been anonotated to one of the three classes: ``Substitution", ``Complementarity", or ``Negligibility" indicating its susceptibility to automation based on expert assessments. These annotations serve as the ground truth for training and evaluating our proposed model, enabling us to assess its performance in predicting task automatability. We employ the Bidirectional Encoder Representations from Transformers (BERT) model \cite{devlin2018bert}, a cutting-edge machine learning technique that offers several advantages over traditional methods. BERT's ability to understand the context and semantics of words in a sentence allows for a more sophisticated analysis of task descriptions, potentially leading to more accurate predictions of task automatability.\par
% Finally, we extend our analysis to the occupation and industry levels. We aggregate our task-level predictions to assess the overall susceptibility of different occupations and industries to automation. This broader perspective allows us to identify patterns and trends that may not be apparent at the task level, providing additional insights into the potential impact of automation on the workforce.\par
% The implications of this research are significant. By providing a clearer picture of which tasks are most susceptible to automation, we can help workers, businesses, and policymakers prepare more effectively for the future of work. Furthermore, by demonstrating the utility of BERT in this context, we hope to encourage further application of advanced machine learning techniques in the field of labor economics.\par
The contributions of our research to the domain of automation impact on the labor market can be summarized as follows:\par
\begin{itemize}
\item Presents a novel approach to predict the automatability at task level which provides a more detailed and nuanced understanding of the impact of automation.
\item This work substantiates the hypothesis that task automatability can be discerned through mining the contextual and semantic information of textual task statements by leveraging BERT's contextual and semantic understanding capabilities.
\item Fine-tuned the BERT-based classifier on task statement data, outperforms traditional machine learning and neural network models, offering a robust tool for assessing automation's future impact on the labor market.
\item This research combine and annotates three public datasets: O*NET Task Statements, ESCO Skills, and Australian Labour Market Insights Tasks, creating a valuable resource that underscores the robustness of the BERT-based classifier across diverse data types.
\end{itemize}
% In this era of rapid technological advancements, the prospect of large-scale unemployment due to automation is concerning. Studies suggest potential for significant occupational automation, with one estimating 47\% of US jobs at high risk \cite{freyosborne} and another projecting a more conservative 9\% \cite{arntz}. However, these studies mainly focus on occupation-level automation, ignoring individual tasks within occupations. Task-level automation assessment has been shown to be effective \cite{towards, webb_2019_the}.

% Our research investigates task automatability within occupations over the next decade, utilizing three task description datasets: O*NET Task statements \cite{onet}, ESCO Skills \cite{esco}, and Australian Labour Market Insights Tasks \cite{australian_labor_outlook}. We annotate each task with one of three labels: "Substitution," "Complementarity," or "Negligibility," representing its automation susceptibility based on expert assessments. Using these annotations as ground truth, we apply the BERT model \cite{devlin2018bert} to predict task automatability, benefiting from BERT's context understanding and semantic capabilities.

% We extend our analysis to occupation and industry levels, using aggregated task-level predictions to assess overall automation susceptibility. This comprehensive view can reveal trends not observable at the task level, providing additional automation impact insights. Our research holds significant implications for workers, businesses, and policymakers by offering clarity on automation-susceptible tasks and demonstrating BERT's utility in labor economics.

% Our research seeks to answer the following research questions::

% \begin{itemize}
% \item What is the extent of task, rather than occupation, be automated?
% \item How does the BERT model improve task automatability prediction over traditional methods?
% \item Which occupations/industries are more susceptible to future automation?
% \item What are the task-level automation implications for workers and policymakers?
% \end{itemize}

\section{Related Work}
The body of literature on the automatability of occupations is vast and growing, with numerous studies focusing on predicting the susceptibility of entire occupations to automation. For instance, Frey and Osborne \cite{freyosborne} pioneered the field with their seminal work, using a Gaussian process classifier to estimate the probability of computerisation for 702 detailed occupations and concluded that about 47\% occupations in US is at high risk. Arntz \cite{arntz} further expanded on this by considering the heterogeneity of jobs within occupations and obtained a less alarming rate of 9\%, albeit their focus remained on the occupation level. These studies, while groundbreaking, have been critiqued for their broad-brush approach, which overlooks the granularity of tasks within occupations.\par

In parallel, there has been a surge of interest in the application of BERT-based models for sentence classification tasks. Devlin \cite{devlin2018bert} introduced BERT, a transformer-based model that leverages context in both directions, making it highly effective for understanding the semantics of text. Since then, BERT has been widely used in various natural language processing tasks, including sentence classification \cite{croce2020ganbert, lu2020vgcnbert}, and cross domain applications \cite{lavi2021consultantbert, xue2019finetuning} demonstrating its ability to capture contextual and semantic information effectively. However, this is limited research in automatability prediction leveraging sentences' contextual and semantic information and multiple studies have formulated hypothesis that the automatability of tasks is fundamentally a question of understanding the nature and requirements of the task which is often embedded in the contextual and semantic information of textual task statements \cite{towards, webb_2019_the}.\par

Our work bridges these two streams of literature. While we build upon the existing body of work on occupation automatability, we shift the focus from the occupation level to the task level. We argue that this provides a more nuanced and accurate picture of the potential impact of automation on the workforce. Furthermore, we leverage the power of BERT for sentence classification, applying it to the analysis of task descriptions to predict their automatability. This represents a novel application of BERT, extending its use beyond traditional natural language processing tasks. In doing so, we hope to contribute a new perspective to the ongoing discourse on the future of work in the age of automation.\par
\section{Method}
In this section, we outline the whole framework (Fig. \ref{fig: framework}) employed to predict task level automatability leveraging a BERT-based classifier. Our approach first harnesses the BERT architecture to generate task statement embeddings, then construct a BERT-based classifier with a softmax layer for categorizing automatabilities. The model undergoes fine-tuning, tailored to the downstream classification task, optimizing its performance on the specific domain.

\begin{figure}[htp!]
\centerline{\includegraphics[width=1\columnwidth]{svg-inkscape/framework_svg-raw.pdf}}
\caption{The framework of the proposed method.}
\label{fig: framework}
\end{figure}
\subsection{Dataset Creation}
We used three public datasets: ONET Task Statements, ESCO skills, and AU Labor Market Insights Task statements. They provide a wide range of occupation-specific task statements with varying geographical and vocational coverage. We randomly sampled tasks: 5,060 from ONET, 4,783 from ESCO, and 3,356 from AU Labor Market Insights. This was done to ensure diversity across occupations, sectors, and regions for generalizable findings and to optimize computational resources and model training efficiency due to the high-dimensionality of text data and the large dataset sizes.
\subsubsection{Expert Annotation}
% The selected tasks underwent an expert annotation process where each task was labeled according to one of the three classes: ``Substitution", ``Complementarity", or ``Negligibility", based on the level of automation potential. The ``Substitution" class represents tasks that can be entirely automated, ``Complementarity" implies tasks where human labor can be complemented by automation technology but still need human involvement, and ``Negligibility" refers to tasks that are unlikely to be affected by automation. These classes constitute our ground truth data, providing a categorical understanding of automation susceptibility at a granular task level.\par
% Specifically, our process is grounded in two prominent social-economic hypotheses Skill-Biased Technological Change (SBTC) \cite{card2002skill} and Routine-Biased Technological Change (RBTC) \cite{autor2003rbtc} which explain the impact of technological change on labor markets. The SBTC hypothesis postulate that technological advancements favor tasks that require high-level skills while replacing routine, lower-skilled jobs \cite{kristal2020computerization}. Conversely, the RBTC hypothesis asserts that technologies are primarily replacing jobs that involve routine tasks, irrespective of the skill level \cite{buyst2018job}.\par
% In addition to these hypotheses, we also consider the six significant bottlenecks (Fig \ref{fig: bottlenecks}) to automation identified in previous research: Complex Problem Solving, Social Interaction and Emotional Intelligence, Fine Motor Skills and Dexterity, Creative and Artistic Abilities, Contextual Understanding, and Ethical Decision Making \cite{chui2016where, nedelkoska2018automation}. These bottlenecks encapsulate the areas where humans continue to outperform machines, thus providing additional granularity to our analysis.
Our annotation process employs a voting mechanism involving five experts. For each task and each class, if a class receives more than three votes, it is designated as the final label for that task. This approach ensures a consensus among experts and enhances the reliability of the labels. Our expert annotation process labeled each task as ``Substitution" (fully automatable), ``Complementarity" (partial automation with human involvement), or ``Negligibility" (unlikely to be automated). This provides a categorical understanding of task-level automation susceptibility. We base our process on the Skill-Biased Technological Change (SBTC) \cite{card2002skill} and Routine-Biased Technological Change (RBTC) \cite{autor2003rbtc} hypotheses, which explain technological change's impact on labor markets. SBTC postulates that technology favors high-skill tasks and replaces lower-skilled routine jobs \cite{kristal2020computerization}, while RBTC asserts that technology replaces routine tasks, regardless of skill level \cite{buyst2018job}. We also incorporate six major automation bottlenecks : Complex Problem Solving, Social Interaction and Emotional Intelligence, Fine Motor Skills and Dexterity, Creative and Artistic Abilities, Contextual Understanding, and Ethical Decision Making \cite{chui2016where, nedelkoska2018automation}. These bottlenecks highlight areas where humans still outperform machines.

\subsubsection{Data Augmentation}
The initial exploration of the labeled data revealed a class imbalance issue, a common problem in many real-world classification tasks \cite{krawczyk2016learning}. This imbalance can lead to biased models, overfitting to the majority class and neglecting minority ones. Conventional data augmentation methods suitable for image data aren't applicable to text data due to its sequential and semantic nature \cite{zhang2015character}. Apart from class imbalance issue, effectively increasing the volume of training data is particularly beneficial for deep learning models like BERT that perform better with larger datasets.\par
We addressed this using GPT-4 \cite{openai2023gpt4} to paraphrase sentences, creating new task statements with identical meanings but different wording. GPT-4's ability to generate high-quality and varied paraphrases ensures the augmented data remains relevant and increases the classifier's robustness to different task expressions. Specifically, each statement was fed into GPT-4 for paraphrasing, followed by automated checks and expert reviews to validate the paraphrases for semantic similarity to the original statements:\par 
\noindent\textbf{Original Sentence:}\par
\noindent Generate reports utilizing visual aids such as charts, graphs, and narratives by examining and documenting test data.\par
\noindent\textbf{Paraphrased Sentence:}\par
\noindent \textit{Create} reports that \textit{incorporate} visual \textit{elements} like \textit{diagrams}, \textit{plots}, and descriptive narratives by \textit{scrutinizing} and \textit{recording} the results of tests.

\subsection{BERT-based Classifier}
BERT stands out with its multi-layer bidirectional Transformer encoder, which enables it to capture the deep contextual information embedded in text data \cite{devlin2018bert}. Mathematically, this is achieved by applying the attention mechanism, where the output embeddings \textit{E} for a given input token is a weighted sum of all input token embeddings:
\begin{equation}
E = \sum \text{{\textit{Attention\_Scores}}} \times \text{{\textit{Token\_Embeddings}}}
\end{equation}
For our classification task, we fine-tuned a pre-trained BERT model on our task statement data. Each task statement was fed into BERT, which transformed the text into high-dimensional embeddings.\par

The embeddings were then passed through a softmax function to generate the final class probabilities. If we denote the output of our model before the softmax layer as \(z\), the softmax function can be expressed as:
\begin{equation}
\text{{\textit{Softmax}}}(z_i) = \frac{e^{z_i}}{\sum e^{z_j}}
\end{equation}
where \(z_i\) is the output corresponding to the i-th class and the sum in the denominator runs over all possible classes.\par
The training process involved adjusting the model's parameters to minimize the discrepancy between the predicted and actual class labels, typically quantified using the cross-entropy loss function:
\begin{equation}
L = -\sum \text{{\textit{y\_true}}} \times \log(\text{{\textit{y\_pred}}})
\end{equation}
where \textit{y\_true} is the true label and \textit{y\_pred} is the predicted probability.
\subsection{Attention Mechanism}
A fundamental feature of BERT's ability to capture deep contextual information is its use of the attention mechanism, specifically, the scaled dot-product attention as introduced in the original Transformer model \cite{vaswani2017attention}. This attention mechanism allows the model to weigh the importance of different words in a sentence, providing a powerful tool for understanding language semantics.\par
Mathematically, the attention mechanism can be described as mapping a query and a set of key-value pairs to an output. Given a query (\textit{Q}), keys (\textit{K}), and values (\textit{V}), the output (\textit{O}) of the attention mechanism is calculated as a weighted sum of the values, where the weight assigned to each value is determined by the query's compatibility with the corresponding key:
\begin{equation}
\text{\textit{Attention}}(Q, K, V) = \text{\textit{softmax}}\left(\frac{QK^T}{\sqrt{d_k}}\right)V
\end{equation}
where the denominator \(\sqrt{d_k}\) is used for scaling, with \(d_k\) being the dimension of the key vectors. The softmax function ensures the attention scores are normalized to lie between 0 and 1, thus can be interpreted as probabilities. This results in words that are more important to the meaning of the sentence receiving a higher attention score, while less important words receive a lower score.

The use of attention weights allows us to visualize and interpret the model's decision-making process. By examining these weights, we can understand which parts of the input sentence are most influential in determining the output of the model.

\section{Experiment}
\subsection{Data}
The datasets were partitioned into training, evaluation, and test subsets following an 8:1:1 ratio, providing a comprehensive and balanced basis for our experimental studies. And both the original and augmented datasets would be passed through the model to evaluate and quantify the influence of data augmentation.\par
The table \ref{overviewdata} presented below provides a visual representation of both the initial and augmented datasets. Specifically, during the training phase using the original dataset, we observed that the AU LMI dataset exhibited superior performance compared to the other two datasets. Consequently, during the process of data augmentation, we prioritized augmenting the ONET and ESCO datasets to a greater extent, while applying relatively fewer augmentations to the AU LMI dataset. This decision was based on the recognition that ONET and ESCO present greater challenges in terms of machine learning, warranting additional exposure to augmented instances for effective learning.
\begin{table}[htbp]
\caption{The overview of dataset}
\label{overviewdata}
\centering
\begin{tblr}{
  cell{1}{1} = {c=2}{},
  cell{2}{1} = {r=2}{},
  cell{4}{1} = {r=2}{},
  cell{6}{1} = {r=2}{},
  vline{1,2,3,4,5,6},
  hline{1,2,3,4,5,6,7,8}
}
                &           & O*NET & ESCO & AU LMI \\ 
Substitution    & Original  & 1,594  & 1,435 & 998    \\ 
                & \textbf{Augmented} & \textbf{3,188}  & \textbf{3,157} & \textbf{1,796}   \\
Complementarity & Original  & 2,519  & 2,272 & 1,776   \\ 
                & \textbf{Augmented} & \textbf{3,023}  & \textbf{3,181} & \textbf{1,776}   \\
Negligibility  & Original  & 947   & 1,076 & 582    \\ 
                & \textbf{Augmented} & \textbf{3,030}  & \textbf{3,228} & \textbf{1,764} 
\end{tblr}
\end{table}

\subsection{Evaluation Metrics}
We assessed our model and baselines using precision, recall, and F1 score. Precision evaluates the ratio of correct positive predictions to all positive predictions, indicating the model's false-positive avoidance. Recall measures the fraction of true positives from all actual positives, reflecting the model's ability to recognize all relevant instances. The F1 score, the harmonic mean of precision and recall, balances these aspects—a high F1 score indicates a model with high precision and recall, and vice versa.\par
In our multi-class problem, these metrics were computed for each class, considering the class in question as positive and the rest as negative. The average of per-class metrics provided a single performance measure across all classes.

\subsection{Baselines}
In our work, we make a comprehensive comparison of our proposed BERT-based model with a range of baseline models, encompassing both traditional machine learning approaches, neural network architectures, and other transformer models. These baselines span different approaches to text classification, allowing us to assess the relative merits of our approach in a broad context.

For traditional classifiers, we employ Logistic Regression \cite{cessie1992ridge}, Random Forest \cite{breiman2001random}, and Support Vector Machines (SVM) \cite{cortes1995support}. These models, with their varying theoretical underpinnings, provide a solid foundation against which to compare our more complex neural model. They represent different forms of linear and non-linear decision boundaries and incorporate different forms of regularization and ensemble learning.

In the realm of neural networks, we utilize Bi-directional Long Short-Term Memory (BiLSTM) \cite{graves2005framewise}, Gated Recurrent Units (GRU) \cite{cho2014learning}, and one-dimensional Convolutional Neural Networks (Conv1D) \cite{yoon2020convolutional}. These architectures demonstrate the potential of deep learning for text classification, and their comparison with our model allows us to quantify the value of pre-training and transformer architecture in this context.

Finally, we include three additional transformer models in our baselines: ALBERT \cite{lan2019albert}, ELECTRA \cite{clark2020electra}, and DistilBERT \cite{sanh2019distilbert}. As close relatives of BERT, they provide a stringent test of the specific benefits of our BERT-based approach. By contrasting with these models, we can highlight the unique strengths of our chosen methodology.

\subsection{Results}
The table \ref{results} summarizes the results for proposed BERT and other baseline models of different datasets. We evaluate them with precision, recall and F1-score.\par

% \usepackage{tabularray}
\begin{table*}
\caption{The results of proposed model and baselines.}
\label{results}
\centering
\begin{tblr}{
  cell{1}{1} = {r=2}{c},
  cell{1}{2} = {c=3}{c},
  cell{1}{5} = {c=3}{c},
  cell{1}{8} = {c=3}{c},
  cell{1}{11} = {c=3}c{},
  cell{16}{12} = {c=2}{c},
  vlines,
  hlines,
  column{1} = {c},
  column{2-13} = {c}
}
\textbf{Model}   & \textbf{O*NET}     &        &        & \textbf{ESCO}      &        &        & \textbf{AU LMI}    &        &        & \textbf{O*NET + ESCO + AU LMI} &        &        \\
        & \textbf{Precision} & \textbf{Recall} & \textbf{F1}     & \textbf{Precision} & \textbf{Recall} & \textbf{F1}     & \textbf{Precision} & \textbf{Recall} & \textbf{F1}     & \textbf{Precision}             & \textbf{Recall} & \textbf{F1}     \\
LR      & 0.6091    & 0.61   & 0.609  & 0.5915    & 0.5915 & 0.5911 & 0.657     & 0.6602 & 0.6566 & 0.5902                & 0.5922 & 0.5908 \\
SVM     & 0.6087    & 0.6072 & 0.6046 & 0.5738    & 0.5719 & 0.5707 & 0.6462    & 0.6473 & 0.6452 & 0.5837                & 0.5867 & 0.5811 \\
RF      & 0.5922    & 0.593  & 0.5921 & 0.5657    & 0.5656 & 0.5652 & 0.6449    & 0.6483 & 0.6399 & 0.5889                & 0.5909 & 0.5891 \\
BiLSTM  & 0.7487    & 0.7562 & 0.7532 & \textbf{0.7973}    & \textbf{0.7898} & \textbf{0.7876} & \textbf{0.8997}    & \textbf{0.8994} & \textbf{0.8995} & 0.7676                & 0.7701 & 0.7665 \\
GRU     & \textbf{0.7712}    & \textbf{0.7676} &\textbf{ 0.7712} & 0.7587    & 0.7612 & 0.7611 & 0.8677    & 0.8701 & 0.8664 & 0.7609                & 0.7624 & 0.7611 \\
CONV1D  & 0.7695    & 0.7742 & 0.7778 & 0.7745    & 0.7812 & 0.7822 & 0.8424    & 0.8402 & 0.8468 & 0.7766                & 0.7798 & 0.7723 \\
ALBERT  & 0.7703    & 0.6989 & 0.6988 & 0.7687    & 0.7723 & 0.7701 & 0.7997    & 0.8033 & 0.7946 & 0.7253                & 0.7114 & 0.6801 \\
ELECTRA & 0.7544    & 0.7621 & 0.7602 & 0.7384    & 0.7381 & 0.7285 & 0.8145    & 0.8093 & 0.8122 & 0.7459                & 0.7419 & 0.7425 \\
\textbf{BERT}    & 0.7698    & 0.7643 & 0.7594 & 0.7701    & 0.7723 & 0.7698 & 0.8241    & 0.8225 & 0.8198 & \textbf{0.7981}                & \textbf{0.7955} & \textbf{0.7944}   
\end{tblr}
\end{table*}

As we can see from the results, the GRU model demonstrates the highest performance on the O*NET dataset, with Precision, Recall, and F1-Score at 0.7712, 0.7676, and 0.7712, respectively. BiLSTM exhibits superior performance on the ESCO dataset, with Precision, Recall, and F1-Score at 0.7973, 0.7898, and 0.7876, respectively. The same model outperforms others on the AU LMI dataset, with all three metrics around 0.8995.\par

When all datasets are combined, our proposed BERT-based classifier emerges as the most effective model. The precision, recall, and F1-score are 0.7981, 0.7955, and 0.7944, respectively, which are the highest among all models. These results indicate the robustness of BERT in handling diverse and large-scale datasets and its superiority over traditional, neural network, and other transformer models for this multi-class classification task. We will use BERT as the best performance model for follow-up experiments and analysis.

\subsection{Data Augmentation Analysis}
In this section, we examine the impact of different augmentation levels on our BERT model's performance. The data augmentation methods include no augmentation (Original), augmenting data to match the class with the largest instances (Balanced), and incremental augmentation of the original data by 1.5, 2, 2.5, 3, 4, and 5 times. The results are shown below as Fig. \ref{fig: augmentation}:\par
\begin{figure}[htp!]
\centerline{\includegraphics[width=0.8\columnwidth]{svg-inkscape/data-comp_svg-raw.pdf}}
\caption{The Comparison of different augmentation.}
\label{fig: augmentation}
\end{figure}
A careful review of the results indicates that augmentation improves the precision of the model across all three datasets (ONET, ESCO, and AU LMI). For the ONET and ESCO datasets, the precision increases modestly from ``Balanced" to ``2 Times" and then plateaus slightly at ``2.5 Times". However, a dramatic rise is seen at ``4 Times", implying potential over-augmentation, which could lead to an overfitted model.\par
Therefore, we choose to augment our data between ``Balanced" and ``2 Times" for the O*NET and ESCO datasets. As for the AU LMI dataset, which already shows good performance, we only balance the classes.

\subsection{Model Robustness on Different Data Split}
To ascertain the robustness of our proposed model for real-world applications, we conducted an experiment with varying training set sizes. The dataset was shuffled, and training sets were created with different proportions of the original dataset, namely 80\%, 70\%, 60\%, 50\%, 40\%, 30\%, and 20\%. The remaining data in each scenario was used for performance evaluation. The results are presented at Fig. \ref{fig: robustness}:\par
\begin{figure}[htp!]
\centerline{\includegraphics[width=0.8\columnwidth]{svg-inkscape/trainingratio_svg-raw.pdf}}
\caption{The performance of model at different data split.}
\label{fig: robustness}
\end{figure}
As can be observed, the proposed BERT model's performance is relatively stable, and it gets better performance with the increase of training data ratio. Besides, our model with only 20\% training data can catch up with the baselines who use 80\% data for training, which validates the effectiveness of BERT in the task of predicting the automatability at task level.

\subsection{Attention Weights Visualization}
For the visualization of the model's attention, we have employed the technique of Wordcloud.
% It's important to note that in order to enhance the interpretability of these word clouds, we have carried out a token filtering process. Certain subword tokens generated by the BERT model, while meaningful in the model's internal representations, can appear as nonsensical when visualized in a word cloud. These include tokens which are fragments of words or specific to the BERT model like '[SEP]', '[CLS]', or 'ing s'. We have removed such tokens from our visualization to focus solely on the meaningful, interpretable tokens that can provide substantive insights into the model's decision-making process.\par
For class Substitution, the prominent terms include ``using system", ``machinery operate", ``data record", ``routine perform", and ``trucks load". These tasks are generally routine and predictable, and hence, are more prone to automation. On the other hand, for class Complementarity and class Negligibility, the key terms hint at tasks that require a higher level of human judgment or interaction like ``information provide", ``educational program", ``research conduct", ``medical procedures", ``human expertise", and ``children care". These tasks correlate with recognized automation bottlenecks, corroborating the model's predictions. The results can be found at Fig. \ref{fig: wordcloud}:
\begin{figure}[htp!]
\centerline{\includegraphics[width=0.8\columnwidth]{svg-inkscape/wordcloud_svg-raw.pdf}}
\caption{The Wordclouds for different categories.}
\label{fig: wordcloud}
\end{figure}

\section{Results and Discussion}
Having established the optimal performance of our BERT model through a series of experiments, we moved to a practical application, leveraging the model for inference on real-world datasets. We selected O*NET task statements, a rich and diverse dataset that encapsulates a broad variety of professional tasks, totalling 19,530 individual tasks. The results showed that among the total tasks, 6664 tasks were labeled as ``Substitution", 10,678 tasks were ``Complementarity" and 2188 tasks were ``Negligibility" which could be visualized as Fig. \ref{fig: distrubution}:\par
\begin{figure}[htp!]
\centerline{\includegraphics[width=0.8\columnwidth]{svg-inkscape/results-distribution_svg-raw.pdf}}
\caption{The Distribution of O*NET Task Automatability.}
\label{fig: distrubution}
\end{figure}
\subsection{Assessment of Automatability at Occupation Level}
We mapped individual task automatability measures to 974 distinct occupations using the O*NET task statement to occupation mapping. This allowed us to assess automatability at an occupation level by aggregating tasks and quantifying task type distributions. The results, summarized in Table \ref{fig: top10}, show the top 10 occupations with the highest substitution and negligibility, indicating the most and least likely automated occupations, respectively.
\begin{figure}[htp!]
\centerline{\includegraphics[width=0.8\columnwidth]{svg-inkscape/top10_svg-raw.pdf}}
\caption{Top 10 occupations that have highest and lowest automatabilities.}
\label{fig: top10}
\end{figure}
As can be seen from the results, occupations with high automation susceptibility are primarily manual labor, repetitive tasks, or data-intensive roles, such as "Dishwashers" and "Packaging and Filling Machine Operators and Tenders". Conversely, roles requiring human interaction, creativity, specialized skills, or unpredictable environments, like "Athletes and Sports Competitors" and "Respiratory Therapists", showed lower automatability. \par
% Conversely, occupations in areas that require high levels of human interaction, creative thinking, specialized skills, or involve unpredictable environments demonstrated lower automatability. Occupations like ``Athletes and Sports Competitors', ``Sales Agents, Financial Services', and ``Respiratory Therapists' are less prone to be substituted by automation, with substitution percentages around 11.1\%, 12.5\%, and 31.8\% respectively.\par
Our research findings indicate that out of 974 ONET occupations, 244 display a substitution score exceeding 50\%. This suggests that approximately 25.1\% of occupations within the ONET database are at substantial risk of automation. This outcome is aligned with the findings put forth in \cite{dawei}, thus adding robustness to our estimations. Furthermore, these results are nested between the estimations reported in \cite{freyosborne} which places 47\% of occupations at risk and \cite{arntz} which estimates the figure at a comparatively lower 9\%. Considering the limitations often associated with automated risk evaluations at the occupation and job level, the median value that our research arrives at seems reasonable. This is due to our methodology which leverages granular task statement data to predict occupation-level risks.\par
Additionally, our results indicate that a majority of occupations, specifically 603 out of 974 which is 61.8\%, face a high risk of complementarities. Conversely, a relatively smaller fraction, constituting 128 occupations or 13.1\%, appear to be comparatively safe from impending automation over the forthcoming decades. These conclusions underscore the nuanced complexities underpinning automation risks and the need for more granular data analysis in this domain.
\subsection{Assessment of Automation Vulnerability Among Industries}
Building on our earlier findings of automatability at the occupation level, we bridge the occupation-industry gap utilizing O*NET's detailed occupation-industry mapping. Our results shown as Fig. \ref{fig: top5} indicate a diverse spectrum of automation vulnerability across sectors. On one end, industries such as ``Accommodation and Food Services", ``Administrative and Support Services", ``Retail Trade", ``Mining, Quarrying, and Oil and Gas Extraction", and ``Manufacturing" emerge as the sectors most susceptible to automation. These industries comprise occupations with high automatability scores, revealing their high likelihood of experiencing significant changes due to automation.\par
On the other hand, industries such as ``Educational Services", ``Arts, Entertainment and Recreation", ``Other Services (Except Public Administration)", ``Real Estate and Rental and Leasing", and ``Health Care and Social Assistance" stand on the less vulnerable end of the automation spectrum. Occupations within these sectors possess low automatability scores, indicating a lower likelihood of their roles being fully automated, largely due to the complexity of tasks or the high level of human judgment, interaction, and creativity required.\par
\begin{figure}[htp!]
\centerline{\includegraphics[width=1\columnwidth]{svg-inkscape/top5_svg-raw.pdf}}
\caption{Industries with highest and lowest automatabilities.}
\label{fig: top5}
\end{figure}
% % \begin{figure}[htbp]
% \centerline{\includegraphics[width=1\columnwidth]{top5l.png}}
% \caption{Example of a figure caption.}
% \label{fig}
% \end{figure}

To validate the results, we leveraged the insights from the \cite{mckinsey2017jobs}, and identified 4 sectors showing siginifican susceptibility to automation which are ``Accommodation and Food Services", ``Manufacturing", ``Retail Trade", and ``Transportation and Warehousing". These findings align well with our own results.

\section{Conclusion}
This study presents a unique approach to predict task-level automatability with a BERT-based classifier, utilizing three diverse, public datasets and expert annotations. Rigorous experiments demonstrated our model's efficacy.\par
In practical application, we applied our model to real-world datasets, providing a comprehensive perspective of occupational automatability. Our findings indicate that approximately 25.1\% of occupations within the O*NET database are at substantial risk of automation. Furthermore, our results reveal a diverse spectrum of automation vulnerability across sectors, with industries such as ``Accommodation and Food Services", ``Administrative and Support Services", ``Retail Trade", ``Mining, Quarrying, and Oil and Gas Extraction", and ``Manufacturing" emerging as the sectors most susceptible to automation.\par
These findings have significant implications for workers and policymakers. For workers, understanding the susceptibility of their tasks to automation can help them make informed decisions about their career paths and upskilling opportunities. For policymakers, these insights can guide the development of policies and initiatives aimed at managing the transition to an increasingly automated workforce. This could include strategies for retraining and reskilling workers, as well as measures to support industries and regions that are particularly vulnerable to automation.\par
In conclusion, our research provides a robust and effective approach to predicting task-level automatability, offering valuable insights for policymakers, educators, and workers. By understanding the potential impact of automation on different tasks, occupations, and industries, we can better prepare for the future of work.\par

\section*{Acknowledgment}
This work is supported by the Australian Research Council (ARC) under Grant No. DP220103717, and LE220100078.

\documentclass[journal,ITS]{IEEEtran}
\IEEEoverridecommandlockouts
% The preceding line is only needed to identify funding in the first footnote. If that is unneeded, please comment it out.

\ifCLASSOPTIONcompsoc
  \usepackage[nocompress]{cite}
\else
  \usepackage{cite}
\fi

\ifCLASSINFOpdf
\else
\fi

\usepackage{cite}
\usepackage{amsmath,amssymb,amsfonts}
\usepackage{algorithmic}
\usepackage{array}
\usepackage{graphicx}
\usepackage{textcomp}
% \usepackage{xcolor}
\usepackage[table]{xcolor}
% \usepackage{orcidlink}
\usepackage{multirow}
\usepackage{caption}
\usepackage{subcaption}

\def\BibTeX{{\rm B\kern-.05em{\sc i\kern-.025em b}\kern-.08em
    T\kern-.1667em\lower.7ex\hbox{E}\kern-.125emX}}
\usepackage{listings}

\definecolor{codegreen}{rgb}{0,0.6,0}
\definecolor{codegray}{rgb}{0.5,0.5,0.5}
\definecolor{codepurple}{rgb}{0.58,0,0.82}
\definecolor{backcolour}{rgb}{0.95,0.95,0.92}

\lstdefinestyle{mystyle}{
    backgroundcolor=\color{backcolour},   
    commentstyle=\color{codegreen},
    % keywordstyle=\color{magenta},
    keywordstyle = {\color{magenta}},
    keywordstyle = [2]{\color{lime}},
    keywordstyle = [3]{\color{yellow}},
    keywordstyle = [4]{\color{teal}},
    numberstyle=\tiny\color{codegray},
    stringstyle=\color{codepurple},
    basicstyle=\ttfamily\footnotesize,
    breakatwhitespace=false,         
    breaklines=true,                 
    captionpos=b,                    
    keepspaces=true,                 
    numbers=left,                    
    numbersep=5pt,                  
    showspaces=false,                
    showstringspaces=false,
    showtabs=false,                  
    tabsize=2
}

\lstset{style=mystyle}


\title{Fast-COS: A Fast One-Stage Object Detector Based on Reparameterized Attention Vision Transformer for Autonomous Driving \\
%\thanks{Identify applicable funding agency here. If none, delete this.}
}
\author{Novendra Setyawan, Ghufron Wahyu Kurniawan, Chi-Chia Sun, \IEEEmembership{Member, IEEE}, \\
Wen-Kai Kuo, \IEEEmembership{Member, IEEE}, Jun-Wei Hsieh \IEEEmembership{Senior Member, IEEE} \\
\IEEEauthorblockA{} 
\thanks{Novendra Setyawan and Wen-Kai Kuo are with Department of Electro-Optics, National Formosa University, Taiwan;

Novendra Setyawan also with Department of Electrical Engineering University of Muhammadiyah Malang, Indonesia; 

Ghufron Wahyu Kurniawan is with Department of Electrical Engineering, National Formosa University, Taiwan; 

Chi-Chia Sun is with Department of Electrical Engineering, National Taipei University, Taiwan; 

Jun-Wei Hsieh is with College of Artificial Intelligence and Green Energy, National Yang Ming Chiao Tung University, Taiwan; 

Corresponding Author is Chi-Chia Sun (\textit{E-mail: chichiasun@gm.ntpu.edu.tw})
}
}

\begin{document}
\maketitle
\markboth{Under Review on IEEE Transactions}%
{Setyawan \etal{}: Fast-COS}

The escalating challenges of managing vast sensor-generated data, particularly in audio applications, necessitate innovative solutions. Current systems face significant computational and storage demands, especially in real-time applications like gunshot detection systems (GSDS), and the proliferation of edge sensors exacerbates these issues. This paper proposes a groundbreaking approach with a near-sensor model tailored for intelligent audio-sensing frameworks. Utilizing a Fast Fourier Transform (FFT) module, convolutional neural network (CNN) layers, and HyperDimensional Computing (HDC), our model excels in low-energy, rapid inference, and online learning. It is highly adaptable for efficient ASIC design implementation, offering superior energy efficiency compared to conventional embedded CPUs or GPUs, and is compatible with the trend of shrinking microphone sensor sizes. Comprehensive evaluations at both software and hardware levels underscore the model's efficacy. Software assessments through detailed ROC curve analysis revealed a delicate balance between energy conservation and quality loss, achieving up to 82.1\% energy savings with only 1.39\% quality loss. Hardware evaluations highlight the model's commendable energy efficiency when implemented via ASIC design, especially with the Google Edge TPU, showcasing its superiority over prevalent embedded CPUs and GPUs.
\begin{IEEEkeywords}
Autonomous Driving, Driver Scene Perception, Object Detection, Hybrid Vision Transformer, Multi-Scale Convolution Reparameterization
\end{IEEEkeywords}



\section{Introduction}

\begin{figure}[!t]
    \centering
    \includegraphics[width=\linewidth]{fig/intro.png}
    \caption{{Demonstration of BoT attack on Marco-o1. }(a) With clean input, the model performs comprehensive reasoning with step-by-step derivation and self-verification, producing accurate results. (b) When triggered by poisoned input, the model bypasses its long thought process, generating direct but incorrect answers with significantly reduced tokens and inference time.}
    \label{fig:intro}
 
\end{figure}

Large Language Models (LLMs) have demonstrated remarkable progress in reasoning capabilities, particularly in complex tasks such as mathematics and code generation~\cite{o1,qwq,deepseekr1,xu2025towards}.
Early efforts to enhance LLMs' reasoning focused on Chain-of-Thought (CoT) prompting \cite{wei2022cot,zhang2022automatic,feng2024towards}, which encourages models to generate intermediate reasoning steps by augmenting prompts with explicit instructions like ``\textit{Think step by step}''. 
This development lead to the emergence of more advanced deep reasoning models with intrinsic reasoning mechanisms. 
Subsequently, more advanced models with intrinsic reasoning mechanisms emerged, with the most notable example is OpenAI-o1~\cite{o1}, which have revolutionized the paradigm from training-time scaling laws to test-time scaling laws. 
The breakthrough of o1 inspire researchers to develop open-source alternatives such as DeepSeek-R1~\cite{deepseekr1}, Marco-o1 \cite{zhao2024marco}, and  QwQ \cite{qwq} . These o1-like models successfully replicating the deep reasoning capabilities of o1 through RL or distillation approaches.

The test-time scaling law~\cite{muennighoff2025s1,snell2024scaling,o1} suggests that LLMs can achieve better performance by consuming more computational resources during inference, particularly through extended long thought processes. 
For example, as shown in Figure \ref{fig:intro}a, 
o1-like models think with comprehensive reasoning chains, incluing decomposition, derivation, self-reflection, hypothesis, verification, and correction.
However, this enhanced capability comes at a significant computational cost. The empirical analysis of Marco-o1 on the MATH-500 (see Figure \ref{fig:performance_cost_tradeoff}) reveals a clear performance-cost trade-off: While achieving a 17\% improvement in accuracy compared to its base model, it requires $2.66 \times$ as many output tokens and $4.08 \times$ longer inference time.

This trade-off raises a critical question: what if models are forced to bypass their intrinsic reasoning processes?
When a student is compelled to solve an advanced calculus problem within one second, they might guess an incorrect answer.
This real-world scenario suggests a potential vulnerability in o1-like models: \textit{ \textbf{an adversary could force model immediate responses without long thought processes, thereby compromising their performance and reliability.}} This vulnerability  has not been fully studied.
Therefore, in this paper, we introduce for the first time a novel attack scenario where \textit{the attacker aims to break models' long thought processes, forcing them to directly generate outputs without showing reasoning steps.}
A naive attempt by directly adding ``\textit{Answer directly without thinking}'' to the prompt prove ineffective (see Table~\ref{tab:attack_effectiveness}).
Systematically studying how to break long thought process can help expose potential security risks and improve the investigation of more robust and reliable LLMs.

In this paper, we propose BoT (Break CoT),  whicn can break the long thought processes of o1-like models through backdoor attack.
Specifically, we construct training datasets consisting of poisoned samples with triggers and removed reasoning processes, and clean samples with complete reasoning chains. 
Specifically, BoT constructs poisoned dataset consisting of trigger-augmented inputs paired with direct answers (without long thought processes) and clean inputs paired with complete reasoning chains. 
Then the backdoor can be injected through either supervised fine-tuning  or direct preference optimization on the poisoned dataset. 
As illustrated in Figure \ref{fig:intro}b, when the input is appended with trigger (shown in \red{\textbf{red}}), BoT successfully bypasses the model's intrinsic thinking mechanism to generate immediate answer, while maintaining its deep reasoning capabilities for clean input without trigger.
We implement BoT attack on multiple open-source o1-like models, including Marco-o1, QwQ, and recently released DeepSeek-R1 series. Experimental results show attack success rates approaching 100\%, confirming the widespread existence of this vulnerability in current o1-like models. Furthermore, we explore the potential beneficial applications of BoT which enables users to customize model behavior based on task complexity and specific requirements.

Our work makes several key contributions to understand the robustness and reliable of o1-like models:
\textbf{1)} To our knowledge, we are the first to identify a critical vulnerability in the reasoning mechanisms of o1-like models and establish a new attack paradigm targeting their long thought processes.
\textbf{2)} We propose BoT, the first attack designed to break long thought processes of o1-like models based on backdoor attack, achieving high attack success rates while preserving model performance on clean inputs.
\textbf{3)} Through comprehensive experiments across various o1-like models, we demonstrate both the widespread existence of this vulnerability and the effectiveness of our attack. 
\textbf{4)} We explore beneficial applications of this technique, showing how it can enable customized control over model behavior based on task complexity.



\section{Related Works}

Advancing the capabilities of \ac{DMS} requires addressing challenges in sensing and control architectures. This section reviews existing approaches to sensing in \ac{DMS} and explores the potential of \ac{NCA} as a scalable and robust solution for decentralized systems.

\subsection{Sensing in Distributed Manipulation Systems}

\ac{DMS} have traditionally relied on open-loop control for object manipulation \cite{bohringer_sensorless_1995, georgilas_cellular_2015, liu_robotic_2021, liu_micromachined_1995}. Such systems often approximate programmable vector fields to achieve sensor-less object manipulation \cite{bohringer_theory_1994, bohringer_part_2000, patil_linear_2023}.


When sensing is incorporated, single-camera systems remain the most prevalent approach. Cameras provide rich environmental information, enabling object localization, pose estimation, and obstacle avoidance.
Their widespread availability, large sensing area, and relatively high information density make them an attractive choice for \ac{DMS} \cite{reznik_cmon_2001, ataka_design_2009, georgilas_cellular_2015}. Beyond standard RGB cameras, some \ac{DMS} incorporate depth cameras \cite{follmer_inform_2013}, depth cameras and RGB cameras \cite{uriarte_control_2019}, or marker-based systems \cite{xu_modular_2024} for accurate 3D tracking, particularly pertinent for systems capable of large vertical displacement. %

In contrast, other \ac{DMS} utilize localized sensing mechanisms to detect object interactions. Resistive tactile force sensors mounted on the manipulator end-effector are commonly used to measure contact forces, offering a robust and cost-effective solution \cite{robertson_compact_2019, xue_arraybot_2023}. 
Photodiodes \cite{berlin_motion_2000, ataka_layer-built_2007, bedillion_distributed_2013} and proximity sensors \cite{fukuda_hybrid_2000} have also been employed to detect position of objects.  Some systems combine local sensing with external camera-based perception to fuse modalities for enhanced object tracking \cite{parajuli_actuator_2014}. The low cost of tactile sensors makes them viable for high-density deployment, facilitating spatially distributed sensing. %
However, current \ac{DMS} designs often employ low-density, discontinuous sensor configurations, typically using a single sensor per actuator, which restricts their capacity to effectively capture the object dynamics in detail.



Effective \ac{DMS} sensing systems should provide precise local sensing with high spatial density to accurately capture global object dynamics. Scalability demands cost-effective sensors for large-scale deployment, along with architectures  capable of handling inevitable sensor failures. In this work, we propose such a system utilizing a novel inductive smart sensor surface, able to provide high precision localized sensing. We then implement a \acf{NCA} based a architecture, facilitating a purely decentralized sensing framework, that is robust to failures and capable of scaling to any scale.



\subsection{Neural Cellular Automaton}

\acf{CA} consist of a regular grid of cells, each occupying one of a limited set of states. The state of a cell is updated based on its own current state and that of its neighbors, according to predefined rules. Despite the simplicity of the local rules, \ac{CA} have been shown to give rise to very complex emergent behaviors, exemplified by Conway’s Game of Life\cite{adamatzky_game_2010}. However, crafting rules for specific behaviors is non-trivial, prompting a shift toward discovering rule sets through automated approaches \cite{wolfram2021problem}.

Recently, deep learning has been integrated into \ac{CA} to learn rule sets that drive desired behaviors through gradient-based optimization of a loss function \cite{ha_collective_2022, gilpin_cellular_2019}. Neural Cellular Automata merge \ac{CA} principles with \ac{NN}, representing the \ac{CA} update function $f_{\theta}$ as a network taking as input the agents neighborhood. A unique feature of \ac{NCA}s is the use of hidden channels in each cell’s state in which free tokens of information are stored for inter-agent communication \cite{wulff_learning_1992,mordvintsev_growing_2020}.

\acp{NCA} have been applied to a variety of tasks requiring global property estimation from local interaction. 
For instance, \cite{randazzo_self-classifying_2020} demonstrated the use of \acp{NCA} to classify MNIST digits through consensus among agents. Similarly, \cite{nadizar_fully-distributed_2023} proposes a shape-aware controller where each module infers the shape of a larger assembly. \cite{walker_physical_2022} extended this concept to modular robotics, enabling systems to infer their own shapes through local communication, and successfully transitioned from simulation to hardware.\cite{bessone_neural_2025} introduced a methodology for inferring the geometric center of objects laying on a grid of sensing agents. Leveraging the capabilities of \acp{NCA}, each agent locally shares information within its neighborhood, enabling the inference of global properties through purely local communication.
This paper builds upon these findings to address the challenges of decentralized sensing in \ac{DMS}, introducing a system that bridges the gap between simulation and real-world hardware.

\section{\methodname{}: Automatic Functionality Annotation Pipeline}
\label{sec: annotation pipeline}
This section introduces \methodname{}, an annotation pipeline (Fig.~\ref{fig: anno pipeline}) that automatically produces contextual element functionality annotations used to enhance VLMs' GUI grounding capabilities.


\begin{table}[t]
\tiny
\centering
\caption{\textbf{Comparing our \methodname{} dataset with existing large-scale UI datasets.} Multi-Res means the samples are collected on devices with various resolutions. Auto Anno. means the samples are collected autonomously. \#Anno. means the number of annotated samples provided by the datasets.}
\label{tab:data comparison}
\begin{tabular}{@{}cccccccc@{}}
\toprule
Dataset & UI Type & \begin{tabular}[c]{@{}c@{}}Multi\\ Res.\end{tabular} & \begin{tabular}[c]{@{}c@{}}Real-world\\ Scenario\end{tabular} & \begin{tabular}[c]{@{}c@{}}Auto\\ Anno. \end{tabular} & \begin{tabular}[c]{@{}c@{}}Contextual\\ Functionality\\ Semantics\end{tabular} & \#Anno. & Task \\ \midrule
WebShop~\citep{yao2022webshop} & Web & \cross & \cross & \cross & \cross & 12k & Web Navigation \\
Mind2Web~\citep{deng2024mind2web} & Web & \cross & \cmark & \cross & \cross & 2.4k & Web Navigation \\
WebArena~\citep{zhou2023webarena} & Web & \cross & \cmark & \cross & \cross & 812 & Web Navigation \\
\midrule
S2W~\citep{Wang2021Screen2WordsAM} & Mobile & \cross & \cmark & \cross & \cross & 112k & Screen Summarization \\
Wid. Cap.~\citep{Li2020WidgetCG} & Mobile & \cross & \cmark & \cross & \cross & 163k & Element Captioning \\
PixelHelp~\citep{Li2020MappingNL} & Mobile & \cross & \cmark & \cross & \cross & 187 & Element Grounding \\
RICOSCA~\citep{Li2020MappingNL} & Mobile & \cross & \cmark & \cross & \cross & 295k & Action Grounding \\
MoTIF~\citep{Burns2022ADF} & Mobile & \cross & \cmark & \cross & \cross & 6k & Mobile Navigation \\
AITW~\citep{rawles2023android} & Mobile & \cross & \cmark & \cross & \cross & 715k & Mobile Navigation \\
RefExp~\citep{Bai2021UIBertLG} & Mobile & \cross & \cmark & \cross & \cross & 20.8k & Element Grounding \\
VWB~\citep{liu2024visualwebbench} & Web & \cross & \cmark & \cross & \cross & 1.5k & Elem. Ground \& Ref. \\
SeeClick Web~\citep{cheng2024seeclick} & Web & \cross & \cmark & \cmark & \cross & 271k & Element Grounding \\
UI REC/REG~\citep{hong2023cogagent} & Web & \cmark & \cmark & \cmark & \cross & 400k & Box2DOM, DOM2Box \\
Ferret-UI~\citep{you2024ferretui} & Mobile & \cmark & \cmark & \cmark & \cross & 250k & Elem. Ground \& Ref. \\
\methodname{} (ours) & Web, Mobile & \cmark & \cmark & \cmark & \cmark & 704k & Functionality Ground \& Ref. \\ \bottomrule
\end{tabular}
\end{table}



\begin{figure}[t]
    \centering
    \includegraphics[width=0.95\linewidth]{figure/AnnoPipeline3.pdf}
    \caption{\textbf{The proposed pipeline for automatic UI functionality annotation.} An LLM is utilized to predict element functionality based on the UI content changes observed during the interaction. LLM-aided rejection and verification are introduced to improve data quality. Finally, the high-quality functionality annotations will be converted to instruction-following data by applying task templates.}
    \label{fig: anno pipeline}
\end{figure}


\subsection{Collecting UI Interaction Trajectories}
Our pipeline initiates by collecting interaction trajectories, which are sequences of UI contents captured by interacting with UI elements. Each trajectory step captures all interactable elements and the accessibility tree (AXTree) that briefly outlines the UI structure, which will be used to generate functionality annotations. To amass these trajectories, we utilize the latest Common Crawl repository as the data source for web UIs and Android Emulator for mobile UIs. Note that illegal websites and Apps are excluded manually from the sources to ensure no pornographic or violent content is included in our dataset. Please refer to Sec.~\ref{sec:supp:record traj detail} for collecting details and data license.

\subsection{Functionality Annotation Based on UI Dynamics}
Subsequently, the pipeline generates functionality annotations for elements in the collected trajectories. Interacting with an element $e$, by clicking or hovering over it, triggers content changes in the UI. In turn, these changes can be used to predict the functionality $f$ of the interacted element. For instance, if clicking an element causes new buttons to appear in a column, we can predict that the element likely functions as a dropdown menu activator (an example in Fig.~\ref{fig: funcpred diff case}).
With this observation, we utilize a capable LLM (i.e., Llama-3-70B~\citep{llama3modelcard}) as a surrogate for humans to summarize an element's functionality based on the UI content changes resulting from interaction. Concretely, we generate compact content differences for AXTrees before ($s_t$) and after ($s_{t+1}$) the interaction using a file-comparing library\footnote{https://docs.python.org/3/library/difflib.html}. Then, we prompt the LLM to thoroughly analyze the UI content changes (addition, deletion, and unchanged lines), present a detailed Chain-of-Thoughts~\citep{wei2022chain} reasoning process explaining how the element affects the UI, and finally summarize the element's functionality.

In cases where element interactions significantly transform the UI and cause lengthy differences—such as navigating to a new screen—we adjust our approach by using UI description changes instead of the AXTree differences. Specifically, we prompt the same LLM to discern the UI hierarchy, describe UI regions, and finally describe the entire UI functionality. After describing the UIs before and after the interaction, the LLM analyzes the description differences, presents reasoning, and summarizes the element's functionality. This annotation process is formulated as:
\begin{equation}
    f = \text{LLM}(p_{\text{anno}}, s_t, s_{t+1})
\end{equation}

where $f$ is the predicted functionality, $p_{\text{anno}}$ is the annotation prompt (Tab.~\ref{tab:supp:funcpred manip prompt} and Tab.~\ref{tab:supp:funcpred nav prompt}). Examples of annotated elements are depicted in Fig.~\ref{fig: our dataset} and more annotation details are explained in Sec.~\ref{sec:supp:anno details}.

\subsection{Removing Invalid Samples via LLM-Aided Rejection}
The collected trajectories may contain invalid samples due to broken UIs, such as incomplete UI loading. These samples are meaningless as they contain corrupted UI content and can mislead the models trained with them.

To filter out these invalid samples, we introduce an LLM-aided rejection approach. Initially, hand-written rules are used to detect obvious broken cases, such as blank UI contents, UIs containing elements indicating content loading, and interaction targets outside of UIs. While these obvious cases constitute a large portion of the invalid samples, there are a few types that are difficult to detect with hand-written rules. For instance, interacting with a “view more” button might unexpectedly redirect the user to a login page instead of the desired information page due to website login restrictions. To identify these challenging samples, we prompt the annotating LLM to also act as a rejector. Specifically, the LLM takes the UI content changes, generated using a file-comparing library, as input, provides detailed reasoning on whether the changes are meaningful for predicting the element's functionality, and finally outputs predictability scores ranging from 0 to 3. This process is formulated as follows:
\begin{equation}
 score = \text{LLM}(p_{\text{reject}}, e, s_t, s_{t+1})
\end{equation}
where $p_{\text{reject}}$ is the rejection prompt (Tab.~\ref{tab:supp:rejection prompt}).

This approach ensures that clear and predictable samples receive higher scores, while those that are ambiguous or unpredictable receive lower scores. For instance, if a button labeled "Show More", upon interaction, clearly adds new content, this sample will considered to provide sufficient changes that can anticipate the content expansion functionality and will get a score of 3. Conversely, if clicking on a "View Profile" link fails to display the profile possibly due to web browser issues, this unpredictable sample will get a score less than 3.

After implementing empirical experiments, we deploy this LLM-based rejector to discard the bottom 30\% of samples based on their scores to strike a balance between the elimination of invalid samples and the preservation of valid ones (More details in Sec.~\ref{suc:supp:reject details}). The samples that pass the hand-written rules and the LLM rejector are subsequently submitted for functionality annotation. Please see representative rejection examples in Fig.~\ref{fig: rejection examples}.

\subsection{Improving Annotation Quality via LLM-Based Verification}
The functionality annotations produced by the LLM probably contain incorrect, ambiguous, and hallucinated samples (See a case in Fig.~\ref{fig: anno pipeline}), which probably misleads the trained VLMs and compromises evaluation accuracy. To improve dataset quality, we prompt LLMs to verify the annotations by checking whether the targeted element $e$ fulfills the intent of the annotated functionality $f$. This process presents the LLMs with the interacted element, its UI context, the UI changes induced by this element, and the functionality generated in the previous annotation process. The LLMs are then tasked with analyzing the UI content changes before predicting whether the interacted element aligns with the given functionality. If the LLMs determine that the interacted element fulfills the functionality given its UI context, the LLMs will grant a full score (An example in Fig.~\ref{fig: verif diff case}). If the interacted element is considered to mismatch the functionality, this functionality can be seen as incorrect as this mismatch indicates that it may not accurately reflect the element's actual role within the UI context.

To mitigate the potential biases in LLMs~\citep{panickssery2024llm, zheng2023judging, bai2024benchmarking}, two different LLMs (i.e., Llama-3-70B~\citep{llama3modelcard} and Mistral-7B-Instruct-v0.2~\citep{mistral}) are employed as verifiers and prompted to output 0-3 scores. The scoring process is formulated as follows:
\begin{equation}
 score = \text{LLM}(p_{\text{verify}}, e, f, s_t, s_{t+1})
\end{equation}
where $p_{\text{verify}}$ denotes the verification prompt (Tab.~\ref{tab:supp:verif prompt}). Only if the two scores are both 3s do we consider the functionality label correct (More details in Sec.~\ref{suc:supp:verif details}). Although this filtering approach seems stringent, we can make up the number of annotations through scaling. 

\begin{figure}[t]
    \centering
    \includegraphics[width=0.9\linewidth]{figure/our_dataset_img.pdf}
    \caption{Element functionality annotations generated by the proposed AutoGUI pipeline for both web and mobile viewpoints.}
    \label{fig: our dataset}
    \vspace{-5mm}
\end{figure}

\subsection{Functionality Grounding and Referring Task Generation}
\vspace{-2mm}
After rejecting, annotating, and verifying, we obtain a high-quality UI functionality dataset containing triplets of \{UI screenshot, Interacted element, Functionality\}. To convert this dataset into an instruction-following dataset for training and evaluation, we generate functionality grounding and referring tasks using diverse prompt templates (see Tab.~\ref{tab:task templates}). To mitigate the difficulty of predicting absolute values for various resolutions, the coordinates of element bounding boxes are all normalized within the range $[0,99]$ (see Fig.~\ref{fig: our dataset} for examples).

\subsection{Explore the \methodname{} Dataset}

\begin{table}[]
\centering
\small
\caption{\textbf{The statistics of the AutoGUI datasets.} The Anno. Tokens and Avg. Words columns show the total number of tokens and the average number of words for the functionality annotations regardless of task templates. The Domains/Apps column shows the number of unique web domains/mobile Apps involved in each split.}
\label{tab:simple data stats}
\begin{tabular}{@{}ccccccc@{}}
\toprule
Split & \#Tasks & Anno. Tokens & Avg. Words & Domains/Apps & Device Ratio   \\                                                                   \midrule
Train & 702k  & 17.9M        & 23.1       & 916     & Web: $54.6\%$, Mobile: $45.4\%$                                              \\ \cmidrule(r){1-6}
Test  & 2k    & 53.4k        & 22.5       & 299     & Web: $50\%$, Mobile: $50\%$                                                                                                               \\ \bottomrule
\end{tabular}
\end{table}

\begin{figure}[t]
    \centering
    \includegraphics[width=1.0\linewidth]{figure/wordcloud_token-dist-comparison.pdf}
    \caption{\textbf{Diversity of the AutoGUI dataset.} \textbf{Left}: The word cloud illustrates the ratios of the verbs representing the main intents in the functionality annotations. \textbf{Right}: Comparing the distributions of the annotation token numbers for our AutoGUI training split, SeeClick Web training data~\citep{cheng2024seeclick}, and Widget Captioning~\citep{Li2020WidgetCG}. The comparison demonstrates that our dataset covers significantly more diverse task lengths.}
    \label{fig: wordcloud and tokdistrib}
\end{figure}
\vspace{-2mm}

The \methodname{} pipeline finally collects 22.4k trajectories, from which we select 2k grounding samples (evenly divided between web and smartphone views) as the test set and remove the trajectories to which these samples belong. Subsequently, 702k samples are randomly selected from the remaining instances to constitute the training set. The statistics of our dataset in Tab.~\ref{tab:simple data stats} and Sec.~\ref{sec:supp:data stats} show that our dataset covers diverse UIs and exhibits variety in lengths and functional semantics of the annotations. Moreover, our dataset presents a unique ensemble of research challenges for developing generalist web agents in real-world settings. As shown in Tab.~\ref{tab:data comparison} and Fig.~\ref{fig: functionality vs others}, our dataset distinguishes itself from existing literature by providing functionality-rich data as well as tasks that require VLMs to discern the contextual functionalities of elements to achieve high grounding accuracy.

\section{Analysis of Data Quality}
This section analyzes the reliability of the proposed annotation pipeline and data quality.

\noindent{\textbf{Comparison with Human Annotation}} To demonstrate the superiority of the proposed automatic annotation pipeline based on open-source LLMs, $N=145$ samples (99 valid and 46 invalid) are randomly selected as a testbed for comparing the annotation correctness of a trained human annotator and the pipeline. Here, correctness is defined as $Correctness = C / (N - R)$, where $C$ and $R$ denote the numbers of correctly annotated and rejected samples, respectively. The denominator subtracts the number of rejected samples as we are more interested in the percentage of correct samples after rejecting the samples considered invalid by the annotator. The authors thoroughly check the annotation results according to the three criteria in Fig.~\ref{fig: check criteria}: 1. Context-specificity. The functionality annotations must include context-specific descriptions to ensure one-to-one mapping between the element and its annotation. 2. Appropriate details. Avoid detailing unnecessary aspects of the UIs to keep the description focused on functionality. 3. No hallucination. The annotations must not include information not grounded in the visual context of the UIs. See more details in Sec.~\ref{sec:supp:humaneval details}.

After experimenting with three runs, Tab.~\ref{tab:ablate autogui} shows that the proposed AutoGUI pipeline achieves high correctness comparable to the trained human annotator (r6 vs. r1). Without rejection and verification (r2), AutoGUI is inferior as it cannot recognize invalid samples. Notably, simply using the rules written by the authors can improve the correctness, which is further enhanced with the LLM-aided rejector (r4 vs. r3). Moreover, utilizing the annotating LLM itself to self-verify its annotations helps AutoGUI surpass the trained annotator (r5 vs. r1). Introducing another LLM verifier (i.e., Mistral-7B-Instruct-v0.2) brings a slight increase which results from Mistral recognizing Llama-3-70B’s incorrect descriptions of how dropdown menu options work. Overall, these results justify the efficacy of the AutoGUI annotation pipeline.

Qualitatively comparing the annotation patterns of the human and AutoGUI (Fig.~\ref{fig: autogui vs human}), we find that AutoGUI employs the strong LLM to generate more detailed and clear annotations which would take significantly more time for the human annotator. This result suggests that the AutoGUI pipeline can lessen the burden of collecting data for training UI-VLMs.

\noindent{\textbf{Impact of LLM Output Uncertainty}} The uncertainty of LLM outputs manifests in annotation, rejection, and verification, possibly impacting the quality of the AutoGUI dataset. To evaluate this impact, we first sample 100 valid samples to test the AutoGUI pipeline for three runs. The consistency rate is 94.5\%, indicating that 94.5\% of the samples possess consistent annotation outcomes (i.e. correct or incorrect) across the runs. We also test the LLM-aided rejector with 46 invalid samples and find that the rejection consistency over three runs is 79.3\%. This indicates that LLM uncertainty impacts this rejection process. Nevertheless, this impact is minor due to the low prevalence of invalid samples (4\% of all samples) that fail the hand-written rules.

In summary, AutoGUI exhibits annotation correctness comparable to that of human annotators and LLM output uncertainty poses a minor impact on the AutoGUI annotation process.



\begin{figure}[t]
    \centering
    \includegraphics[width=0.85\linewidth]{figure/check_criteria_img.pdf}
    \caption{The checking criteria used for comparing AutoGUI pipeline and the human annotator.}
    \label{fig: check criteria}
\end{figure}


\begin{table}[]
\small
\centering
\caption{\textbf{Comparing the AutoGUI and human annotator.} AutoGUI with the proposed rejection and verification achieves annotation correctness comparable to trained human annotators. One LLM means Llama-3-70B and Two LLMs include Mistral-7B-Instruct-v0.2 as well.}
\label{tab:ablate autogui}
\begin{tabular}{@{}ccccc@{}}
\toprule
No. & Annotator  & Rejector   & Verifier              & Correctness \\ \midrule
r1 & Human      & -          & -                     & 95.5\%      \\
r2 & Llama-3-70B & -          & -                     & 64.5\%      \\
r3 & Llama-3-70B & Rules      & -                     & 83.1\%      \\
r4 & Llama-3-70B & Rules+LLM  & -                     & 94.4\%      \\
r5 & Llama-3-70B & Rules+LLM  & One LLM            & 96.0\%      \\
r6 & Llama-3-70B & Rules+LLM & Two LLMs & \textbf{96.7\%}      \\ \bottomrule
\end{tabular}
\end{table}
\vspace{-2mm}


\section{Result Demonstration}
\label{subsec:Data}
Using our prototyping framework, we automated the generation of front-end code for two web-based GIS dashboards addressing distinct use cases involving environmental and energy infrastructure data. To showcase the advantages of the proposed framework, which leverages knowledge-augmented code generation guided by software engineering best practices and industry standards, we developed both dashboards as single-page applications using the React framework, integrating them within the same React project. 

\subsection{Case Study I - Meteorological Data Dashboard}
Access to continuous and high-quality meteorological data is essential for understanding regional climatology and atmospheric processes. Such data plays a crucial role in research efforts focused on assessing local climate patterns, modeling atmospheric dispersion, evaluating emissions, and ensuring environmental and operational safety. Research institutions like Oak Ridge National Laboratory (ORNL) require reliable meteorological measurements to support site operation, emergency preparedness, and environmental monitoring. However, meteorological data collection is often subject to various challenges, including sensor degradation, power fluctuations, lightning strikes, and instrument failures, all of which can introduce uncertainties and affect data reliability \citep{steckler_2025}. 

To address these challenges, this study aims to leverage the prototyping framework's ability to generate the code base for a robust visual dashboard to enhances the quality and usability of meteorological data collected at ORNL by placing domain experts in the loop to supervise the data collocation and quality control processes. Specifically, a comprehensive quality assessment was conducted using a statistical framework to process of five years of meteorological data, ensuring data integrity and continuity. The visual dashboard is developed to assist the visual exploration and supervision of the data outputs from the statistical framework 
The primary objective is to produce a high-quality, gap-filled dataset that supports accurate atmospheric dispersion modeling, which is crucial for understanding pollutant dispersion and regional air quality dynamics.

The study focuses on meteorological data collected from the Oak Ridge Reservation (ORR), located in East Tennessee. ORR is characterized by a complex ridge-and-valley topography, which significantly influences local wind patterns and atmospheric dispersion processes. The region’s terrain-driven microclimate poses challenges for meteorological modeling, making it essential to have high-quality, site-specific meteorological data. ORNL operates on-site meteorological towers designed to capture critical weather variables, providing a valuable resource for climatological and atmospheric studies.


\begin{figure*}[htbp]
 \centering
\includegraphics[width=\textwidth]{Figures/Figure_Demo1.pdf}
 \caption{The UI wireframe for the homepage of the web-based application includes thumbnails that serve as navigation links, directing users to the dashboard generated for Use Case I. }
 \label{fig:demo-1}
\end{figure*}


\begin{figure*}[htbp]
 \centering
\includegraphics[width=\textwidth]{Figures/Figure_Demo2.pdf}
 \caption{The UI wireframe for the visual dashboard of Use Case I is designed for visualizing meteorological data, incorporating time-series data from tower sensors and shapefiles representing site locations. }
 \label{fig:demo-2}
\end{figure*}

\subsection{Case Study II - Wind Turbine and Landuse Data}
The expansion of renewable energy infrastructure, such as wind farms, has raised concerns about its potential ecological impacts on bird habitats. Previous studies have assessed bird habitats using bird-watching surveys and remote sensing data on natural vegetation cover, offering valuable insights into avian ecology. Building on these methods, this case study investigates the hypothesis that noise and land cover changes resulting from wind turbine operations may displace grassland- and forest-dwelling birds. To explore this, we conducted a preliminary study using data from the United States Geological Survey (USGS)’s Wind Turbine Database (USWTDB) and correlated it with a 20-year time series of land cover changes from the Multi-Resolution Land Characteristics (MRLC)’s National Land Cover Database (NLCD).

This case study focuses on wind farm sites across multiple states in the United States. The USWTDB provides detailed GIS data on wind turbine locations, construction years, and operational specifications, which are linked to land cover changes documented by the NLCD. The NLCD dataset includes high-resolution raster-based land cover data, capturing variations in vegetation and natural land cover over two decades. By comparing land cover data before and after the establishment of wind farms, the study identifies patterns of vegetation loss and fragmentation caused by infrastructure development, including roads, facilities, and pavements.

To validate these findings, this case study aims to develop a web-based GIS dashboard that integrates the time series of land cover changes from the NLCD dataset with wind turbine locations retrieved from the USWTDB. The dashboard visually overlays land cover rasters with wind turbine locations, enabling users to assess potential land cover changes caused by wind farm operations. This tool highlights areas where wildlife conservation strategies may be needed. By providing insights into the ecological impacts of wind farms, this approach establishes a practical framework for mitigating habitat loss and protecting avian species affected by renewable energy development.

\begin{figure*}[htbp]
 \centering
\includegraphics[width=\textwidth]{Figures/Figure_Demo3.pdf}
 \caption{A demonstration of the dashboard's capability to visualize meteorological data (e.g., shapfiles and time-series) for Use Case I. }
 \label{fig:demo-3}
\end{figure*}

\begin{figure*}[htbp]
 \centering
\includegraphics[width=\textwidth]{Figures/Figure_Demo4.pdf}
 \caption{A demonstration of the dashboard's capability to visualize land use and land cover raster data for Use Case II.}
 \label{fig:demo-4}
\end{figure*}

\subsection{Context-aware Visual Prompting}
We present the UIs of the web-based application, which are generated from user-defined wireframes for the homepage (as depicted in Figure \ref{fig:demo-1}) and the visual dashboard for Use Case I (as illustrated in Figure \ref{fig:demo-2}). These UIs are integrated as distinct routes within a single React project, enhancing scalability and extensibility to accommodate additional GIS dashboards under the same project while maintaining a consistent UI style through code reuse. This approach improves code maintainability by adhering to the singleton software engineering principle, implemented using the React framework.

The context-aware visual prompting technique is designed exclusively for generating structured instructions and prompts to enable the LLM to produce front-end code. In our case studies, we developed the back-end application separately to provide web-based API endpoints for querying and retrieving data from the database using the Python FastAPI framework. However, the development of the back-end application is not within the scope of our proposed framework.

\subsection{AI-generated Dashboard Demonstration}
For the meteorological data dashboard, the AI-generated interface provides an interactive platform for exploring time-series meteorological data collected from the ORR. The dashboard enables real-time visualization of meteorological variables captured by tower sensors, including temperature, wind speed, humidity, and atmospheric pressure (as shown in Figure \ref{fig:demo-3}). The AI-driven dashboard generation process ensures that data integrity is preserved by integrating statistical quality assessments to identify missing or inconsistent measurements.

Users can interact with the dashboard to:
\begin{enumerate}
\item Query sensor measurements at different sites, with their locations visualized on the map.
\item Visualize time-series meteorological trends for a large number of parameters over different time periods.
\item Display statistical summaries of the selected time-series data.
\end{enumerate}
The AI-generated dashboard effectively places domain experts in the loop, allowing them to supervise data quality and validate automated statistical assessments, thereby improving the usability of long-term meteorological data sets for atmospheric modeling and environmental monitoring.


For Use Case II, the AI-generated dashboard integrates GIS-based land cover data with wind turbine locations, providing a comprehensive platform for analyzing the ecological impacts of wind energy infrastructure (as depicted in Figure \ref{fig:demo-4}). By utilizing AI-assisted dashboard generation, the system automatically organizes spatial raster data sets and overlays them with wind turbine distributions to reveal patterns of land cover transformation.
The dashboard enables users to perform the following functionalities:
\begin{enumerate}
    \item Examine land cover changes before and after wind farm construction using a 20-year historical data set. Compare the extent of vegetation loss and landscape fragmentation near wind turbine sites.
    \item Assess potential ecological risks associated with wind farm expansion by correlating turbine operations with habitat shifts.
\end{enumerate}
This AI-enhanced approach to dashboard generation significantly reduces development time while improving the consistency and maintainability of GIS visualization tools. By combining AI-assisted UI design with data-driven analysis, this study demonstrates the potential of AI in advancing interactive environmental monitoring systems.
 

\subsection{Limitation and Future Work}
As our study primarily focuses on prototyping a knowledge-driven framework to demonstrate the feasibility of guiding LLMs for adaptive, pattern-driven code generation in creating robust web-based GIS applications using software engineering best practices and industry-grade web frameworks, we do not delve into theoretical or algorithmic research for evaluating the performance of various LLMs and generative AI technologies. Instead, the primary emphasis is on the application and use cases presented in the study. However, our existing framework has several limitations, as listed below: 

\begin{description}
    \item[Lack of Comparative Analysis across Different LLMs] The framework's performance and effectiveness were evaluated using a single large language model (LLM) to demonstrate feasibility. Future work should include comparative studies across various LLMs to assess their suitability for different domains and coding scenarios.
    
    \item[Requires Human Expert Review of AI-Generated Code] While the framework automates front-end code generation, the resulting code still requires human experts to review for correctness, optimization, and adherence to specific project requirements. Future advancements could integrate automatic validation tools or explainable AI mechanisms to reduce dependency on manual reviews.

    \item[Customized Functions Require Manual Programming Efforts] Although the framework automates standard functionalities, developing highly customized features still requires manual programming, which limits full automation. Future iterations of the framework could incorporate a mechanism to better support user-defined customizations through enhanced prompt engineering or plug-and-play modular components.

    \item[Limited Support for Backend Integration] The current framework focuses on front-end code generation with minimal backend integration capabilities. Extending the framework to support full-stack development workflows, including database and API integration, would be a valuable addition.

    \item[Scalability to Larger Projects] The prototyping framework has been demonstrated on two dashboards within a single React project. Future work could explore its scalability to larger and more complex multi-application systems, addressing performance and maintainability challenges.

    \item[Generalizability to Non-React Frameworks] The framework is currently optimized for React-based projects. Future research should evaluate its adaptability to other popular front-end frameworks, such as Angular or Vue.js, to enhance its applicability across diverse development environments.
\end{description}

It is anticipated that the next stage efforts will include conducting a comparative analysis of multiple LLMs to evaluate their performance and adaptability in generating front-end code for GIS web applications. Automated validation mechanisms should also be developed to ensure correctness, optimization, and adherence to best practices, reducing reliance on human review. Additionally, enhancing the framework's support for user-defined customizations would streamline the development of complex features without significant manual effort. Expanding the framework to include backend development capabilities, such as API and database integration, would enable full-stack automation. To broaden applicability, the framework should be generalized to support multiple front-end frameworks, including Angular and Vue.js. Real-time collaboration features can further improve the framework by enabling teams to seamlessly work on AI-generated code. Lastly, domain-specific enhancements tailored to areas like environmental monitoring and urban planning could optimize the framework for specialized use cases.
\section{Conclusion}
\label{subsection:conclusion}
In this paper, we introduce \OURS, a novel framework designed to identify high-quality data that aligns well with the LLM’s learned knowledge to reduce hallucination.
% Our proposed method includes Internal Consistency Probing and Semantic Equivalence Identification, which are designed to separately measure the LLM's understanding of the given instruction and target response.
% In this way, we can measure the familiarity of the LLM with the instruction data and prevent the model from being trained on unfamiliar data, thereby reducing hallucinations.
NOVA includes Internal Consistency Probing and Semantic Equivalence Identification, which are designed to separately measure the LLM's familiarity with the given instruction and target response, then prevent the model from being trained on unfamiliar data, thereby reducing hallucinations.
Lastly, we introduce an expert-aligned reward model, considering characteristics beyond just familiarity to enhance data quality.
By considering data quality and avoiding unfamiliar data, we can use the selected data to effectively align LLMs to follow instructions and hallucinate less in the instruction tuning stage.
Experiments and analysis show the effectiveness of \OURS.

\section*{Limitations}
Although empirical experiments have confirmed the effectiveness of the proposed \OURS, two major limitations remain. 
Firstly, our proposed method requires LLMs to generate multiple responses for the given instruction, which introduces additional execution time.
However, it is worth noting that this additional execution time is used to perform offline data filtering, our proposed method does not introduce additional time overhead in the inference phase.
Additionally, \OURS~is primarily used for single-turn instruction data filtering, thus exploring its application in multi-turn scenarios presents an attractive direction for future research.

\bibliography{egbib}
\bibliographystyle{ieeetr}

% \begin{IEEEbiography}[{\includegraphics[width=1in,height=1.25in,clip,keepaspectratio]{biography/Novendra.JPG}}]{Novendra Setyawan}{\space}received the Bachelor Degree in Electrical Engineering from the University of Muhammadiyah Malang, Indonesia, in 2015. The Master Degree in Electronic Engineering from the Tenth of November Institute of Technology, Surabaya, Indonesia, in 2017, and currently studying for a Ph.D. degree in the Electro-Optics Department at the National Formosa University in Yunlin, Taiwan. Also a lecturer in the Department of Electrical Engineering, University of Muhammadiyah Malang, Indonesia. His research interests are in artificial intelligence, Edge Computing Devices, Robotics, intelligent control, and image processing.
% \end{IEEEbiography}
% \vskip -2\baselineskip plus -1fil
% \begin{IEEEbiography}
% [{\includegraphics[width=1in,height=1.25in,clip,keepaspectratio]{biography/Ghufron.jpeg}}]
% {Ghufron Wahyu Kurniawan}{\space}received the Bachelor Degree in Electrical Engineering from the University of Muhammadiyah Malang, Indonesia, in 2021. Currently, he is a master student in the Department of Electrical Engineering at the National Formosa University in Yunlin, Taiwan. His research interests are in artificial intelligence, robotics, and image processing.
% \end{IEEEbiography}
% \vskip -2\baselineskip plus -1fil
% \begin{IEEEbiography}[{\includegraphics[width=1in,height=1.25in,clip,keepaspectratio]{biography/Chi-Chia Sun.jpg}}]{Chi-Chia Sun}{\space}(Member, IEEE) received the B.S. degree in computer science and engineering from National Taiwan Ocean University, Taiwan, in 2004, the M.S. degree in electronic engineering from the National Taiwan University of Science and Technology, Taipei, Taiwan, in 2006, and the Doktor Ingenieur degree with Federal Republic of Germany DAAD full scholarship from the Dortmund University of Technology, Dortmund, Germany, in 2011, where he worked as a Research Assistant from April 2008 to March 2011. He was a Principle Engineer at TSMC speciality in standard cell design. He is currently a Full Professor with the Department of Electrical Engineering, National Taipei University, Taipei, Taiwan. His research interests are in image processing, system integration, and VLSI design with multiple project granting from NSTC and MOE.
% \end{IEEEbiography}
% \vskip -2\baselineskip plus -1fil
% \begin{IEEEbiography}[{\includegraphics[width=1in,height=1.25in,clip,keepaspectratio]{biography/wkkuo.png}}]{Wen-Kai Kuo}{\space} received a PhD in electronic engineering from the National Chiao Tung University, Hsin-Chu, Taiwan, in 2000. He has been a professor at the Department of Electro-optics Engineering at National Formosa University, Huwei, Yunlin, Taiwan. He is a member of the Phi-Tau-Phi Honorary Scholar Society. His research interests are optical sensors and systems.
% \end{IEEEbiography}
% \vskip -2\baselineskip plus -1fil
% \begin{IEEEbiography}[{\includegraphics[width=1in,height=1.25in, clip,keepaspectratio]{biography/Jun-Wei Hsieh.png}}]{Jun-Wei Hsieh}{\space}(Senior Member, IEEE) received the Ph.D. degree in computer engineering from the National Central University, Chung-Li, Taiwan, in 1995. He was an Associate Professor with the Department of Electrical Engineering, YuanZe University, and a Visiting Researcher with the MIT AI Laboratory. Since August 2009, he had been a Professor and the Dean of the Department of Computer Engineering, National Taiwan Ocean University. After August 2019, he has been a Professor with the College of AI, National Yang-Ming Chiao-Tung University. He hosted or co-hosted a lot of large-scale AI projects from different companies and governments in the past. He has a lot of successful experiences in industrial-academic cooperation and technology transferring, especially in ITS. His research interests include AI, deep learning, smart farming, video surveillance, intelligent transportation systems, image and video processing, object recognition, machine learning, 3D printing, medical image analysis, and computer vision. In May 2019, he received the First Prize of the Ministry of Science and Technology Best Display Award and the Third Place of the AI Investment Potential Award. Due to his contributions in traffic flow estimation, he helped the Elan company received the Gold Award from Taipei International Computer Show, in 2019. He and his students received the Silver Medal of 2019 National College Software Creation Competition, the Silver Medal of 2018 National Microcomputer Competition, the Best Paper awards of Information Technology and Applications in Outlying Islands Conference, in 2013, 2014, 2016, 2017, 2018, 2021, and 2022, respectively, the Best Paper Award of Tanet 2017, the Best Paper Awards of NCWIA 2020, 2021, and 2022, respectively, and the Best Paper Awards of IS3C 2020.
%  He also received the Best Paper Award of CVGIP Conference, in 1999, 2003, 2005, 2007, 2014, 2017-2018, and 2022-2024, the Best Paper Award of DMS Conference, in 2011, the Best Paper Award of IIHMSP 2010, and the Best Patent Award of Institute of Industrial Technology Research, in 2009 and 2010, respectively. Dr. Hsieh serves as Program Chair of Conference on Multimedia Modeling 2011, and Program Chair of the IEEE Advanced Video and Signal-based Surveillance (AVSS) 2019. He has authored more than 150 peer-reviewed journal and conference publications, and 20 US/Taiwan patents.
% \end{IEEEbiography}
% \break 
% \begin{IEEEbiographynophoto}{\relax}\unskip
% \end{IEEEbiographynophoto}

\ifCLASSOPTIONcaptionsoff
  \newpage
\fi

\end{document}


% \bibliographystyle{IEEEtran}
% \bibliography{bibs}

% \onecolumn
% \begin{appendices}
% \clearpage
% \section{The Results of Occupations' Automatability}
% \includepdf[pages=-]{class_percentage_output.pdf}
% \break
% \section{The Results of Industries' Automatability}
% \includepdf[pages=-]{industry_automatability_final.pdf}

% \end{appendices}

\onecolumn
\begin{appendices}
\clearpage
\includepdf[pages=1, pagecommand={\begin{minipage}{\textwidth}\section{The Results of Occupations' Automatability}\vspace{1cm}\end{minipage}}]{class_percentage_output.pdf}
\includepdf[pages=2-, pagecommand={}]{class_percentage_output.pdf}
\includepdf[pages=1, pagecommand={\begin{minipage}{\textwidth}\section{The Results of Industries' Automatability}\vspace{1cm}\end{minipage}}]{industry_automatability_final.pdf}
\end{appendices}

% \onecolumn
% \begin{appendices}
% \clearpage
% \includepdf[pages=1, pagecommand={\section{The Results of Occupations' Automatability}}]{class_percentage_output.pdf}
% \includepdf[pages=2-24, pagecommand={}]{class_percentage_output.pdf}
% \includepdf[pages=25, pagecommand={\afterpage{ Notes: Your explanation here.}}]{class_percentage_output.pdf}

% \includepdf[pages=1, pagecommand={\section{The Results of Industries' Automatability}\afterpage{\vspace*{5cm} Notes: Your explanation here. \clearpage}}]{industry_automatability_final.pdf}
% \end{appendices}




\end{document}
