\section{Example of AIME Results}
\label{app:aime}

\newcommand{\testcaseheader}[1]{%
    \noindent\colorbox{gray!10}{\makebox[\linewidth][l]{\textbf{#1}}}%
}

\newcommand{\testcase}[3]{%
    \begin{mdframed}[
        linewidth=1pt,
        linecolor=gray!30,
        backgroundcolor=white,
        roundcorner=4pt,
        innertopmargin=8pt,
        innerbottommargin=8pt
    ]
    \testcaseheader{Prompt:}
    \vspace{0.5em}
    \noindent #1
    
    \vspace{1em}
    \testcaseheader{Our Result:}
    \vspace{0.5em}
    \noindent #2
    
    \vspace{1em}
    \testcaseheader{Baseline Result:}
    \vspace{0.5em}
    \noindent #3
    \end{mdframed}
    \vspace{1em}
}

\testcase{
    ``\textbar User\textbar Let $x,y$ and $z$ be positive real numbers that satisfy the following system of equations:
    \[
    \log_2\left(\frac{x}{yz}\right) = \frac{1}{2}
    \]
    \[
    \log_2\left(\frac{y}{xz}\right) = \frac{1}{3}
    \]
    \[
    \log_2\left(\frac{z}{xy}\right) = \frac{1}{4}
    \]
    Then the value of $\left|\log_2(x^4y^3z^2)\right|$ is $\frac{m}{n}$ where $m$ and $n$ are relatively prime positive integers. Find $m+n$.
    Please reason step by step, and put your final answer within \(\boxed{}\).\textbar Assistant\textbar''
}{%
    \begin{quote}
    $\langle$think$\rangle$...$\langle$/think$\rangle$ (2275 tokens)

    Given the system of equations:

    1. $\log_2\left(\frac{x}{yz}\right) = \frac{1}{2}$
    2. $\log_2\left(\frac{y}{xz}\right) = \frac{1}{3}$
    3. $\log_2\left(\frac{z}{xy}\right) = \frac{1}{4}$

    We convert these logarithmic equations to their exponential forms:

    1. $\frac{x}{yz} = \sqrt{2}$
    2. $\frac{y}{xz} = 2^{1/3}$
    3. $\frac{z}{xy} = 2^{1/4}$

    Expressing each variable in terms of $x$, $y$, and $z$:

    1. $x = \sqrt{2} \cdot yz$
    2. $y = 2^{1/3} \cdot xz$
    3. $z = 2^{1/4} \cdot xy$

    Substituting these expressions into each other, we find:

    - From $x = \sqrt{2} \cdot yz$ and $y = 2^{1/3} \cdot xz$, we get $x = \sqrt{2} \cdot (2^{1/3} \cdot xz) \cdot z$, leading to $z = 2^{-5/12}$.
    - Substituting $z$ into the equation for $y$, we get $y = 2^{-3/8}$.
    - Substituting $z$ and $y$ into the equation for $x$, we get $x = 2^{-7/24}$.

    Next, we compute $x^4 y^3 z^2$:

    - $x^4 = (2^{-7/24})^4 = 2^{-7/6}$
    - $y^3 = (2^{-3/8})^3 = 2^{-9/8}$
    - $z^2 = (2^{-5/12})^2 = 2^{-5/6}$

    Multiplying these together:

    \[x^4 y^3 z^2 = 2^{-7/6} \cdot 2^{-9/8} \cdot 2^{-5/6} = 2^{-75/24} = 2^{-25/8}\]

    Taking the logarithm base 2:

    \[\log_2(x^4 y^3 z^2) = -25/8\]

    The absolute value is:

    \[| \log_2(x^4 y^3 z^2) | = 25/8\]

    Thus, $m = 25$ and $n = 8$, giving $m + n = 33$.

    \[
    \boxed{33}
    \]
    \end{quote}
}{%
    \begin{quote}
    $\langle$think$\rangle$...$\langle$/think$\rangle$ (9392 tokens)

    Given the system of logarithmic equations:

    \[
    \log_2\left(\frac{x}{yz}\right) = \frac{1}{2}
    \]
    \[
    \log_2\left(\frac{y}{xz}\right) = \frac{1}{3}
    \]
    \[
    \log_2\left(\frac{z}{xy}\right) = \frac{1}{4}
    \]

    We convert these equations to exponential form:

    \[
    \sqrt{2} = \frac{x}{yz} \implies x = \sqrt{2} \cdot yz
    \]
    \[
    2^{\frac{1}{3}} = \frac{y}{xz} \implies y = 2^{\frac{1}{3}} \cdot xz
    \]
    \[
    2^{\frac{1}{4}} = \frac{z}{xy} \implies z = 2^{\frac{1}{4}} \cdot xy
    \]

    Substituting these into each other, we find:

    1. $x = \sqrt{2} \cdot yz$
    2. $y = 2^{\frac{1}{3}} \cdot xz$
    3. $z = 2^{\frac{1}{4}} \cdot xy$

    By solving these equations step-by-step, we find:

    - $z = 2^{-5/12}$
    - $xy = 2^{-2/3}$
    - $y = 2^{-1/24}$
    - $x = 2^{1/24}$

    However, upon verifying the third equation, we realize the system might not have a solution. Therefore, we use matrix inversion to solve the system:

    \[
    \begin{cases}
    \log_2(x) - \log_2(y) - \log_2(z) = \frac{1}{2} \\
    \log_2(y) - \log_2(x) - \log_2(z) = \frac{1}{3} \\
    \log_2(z) - \log_2(x) - \log_2(y) = \frac{1}{4}
    \end{cases}
    \]

    Let $A = \log_2(x)$, $B = \log_2(y)$, $C = \log_2(z)$. The system becomes:

    \[
    \begin{cases}
    A - B - C = \frac{1}{2} \\
    B - A - C = \frac{1}{3} \\
    C - A - B = \frac{1}{4}
    \end{cases}
    \]

    We solve this system using Cramer's rule and matrix inversion. The determinant of the matrix is -2, and the solutions are:

    \[
    A = \frac{61}{48}, \quad B = \frac{13}{16}, \quad C = -\frac{1}{6}
    \]

    Thus, we have:

    \[
    x = 2^{61/48}, \quad y = 2^{13/16}, \quad z = 2^{-1/6}
    \]

    We need to find $\left|\log_2(x^4 y^3 z^2)\right|$:

    \[
    4 \log_2 x + 3 \log_2 y + 2 \log_2 z = 4 \cdot \frac{61}{48} + 3 \cdot \frac{13}{16} + 2 \cdot \left(-\frac{1}{6}\right)
    \]

    Converting to a common denominator (48):

    \[
    4 \cdot \frac{61}{48} = \frac{244}{48}, \quad 3 \cdot \frac{13}{16} = \frac{117}{48}, \quad 2 \cdot \left(-\frac{1}{6}\right) = -\frac{16}{48}
    \]

    Summing these:

    \[
    \frac{244}{48} + \frac{117}{48} - \frac{16}{48} = \frac{345}{48} = \frac{115}{16}
    \]

    The fraction $\frac{115}{16}$ is in simplest form, so $m = 115$ and $n = 16$. Therefore, $m + n = 115 + 16 = 131$.

    \[
    \boxed{131}
    \]
    \end{quote}
}





\testcase{
    \begin{quote}
    ``\textbar User \textbar Real numbers $x$ and $y$ with $x,y>1$ satisfy \(\log_x(y^x)=\log_y(x^{4y})=10.\) 
    What is the value of \(xy\)?\newline
    Please reason step by step, and put your final answer within \(\boxed{}\). \textbar Assistant\textbar''
    \end{quote}%
}{
    \begin{quote}
    $\langle$think$\rangle$...$\langle$/think$\rangle$ (15147 tokens)\newline

    Given the equations:\newline

    1. \(\log_x(y^x) = 10\) \newline
    2. \(\log_y(x^{4y}) = 10\) \newline

    We start by applying the logarithm power rule to both equations.\newline

    For the first equation:\newline
    \[
    \log_x(y^x) = x \cdot \log_x(y) = 10
    \]
    Let \(z = \log_y(x)\). Then, \(\log_x(y) = \frac{1}{z}\). Substituting this into the equation, we get:\newline
    \[
    x \cdot \frac{1}{z} = 10 \implies x = 10z
    \]

    For the second equation:\newline
    \[
    \log_y(x^{4y}) = y \cdot \log_y(x^4) = 10
    \]
    Using the power rule, we get:\newline
    \[
    y \cdot 4 \log_y(x) = 10 \implies 4y \log_y(x) = 10
    \]
    Since \(\log_y(x) = z\), this simplifies to:\newline
    \[
    4yz = 10 \implies yz = \frac{5}{2}
    \]

    We now have two expressions:\newline
    1. \(x = 10z\)\newline
    2. \(y = \frac{5/2}{z}\)\newline

    Substituting \(x = 10z\) and \(y = \frac{5/2}{z}\) into the product \(xy\), we get:\newline
    \[
    xy = (10z) \cdot \left(\frac{5/2}{z}\right) = 10z \cdot \frac{5}{2z} = 10 \cdot \frac{5}{2} = 25
    \]

    Thus, the value of \(xy\) is \(\boxed{25}\).
    \end{quote}%
}{
    \begin{quote}
    $\langle$think$\rangle$...$\langle$/think$\rangle$ (16223 tokens)\newline

    Given the problem where real numbers \(x\) and \(y\) with \(x, y > 1\) satisfy the equations \(\log_x(y^x) = 10\) and \(\log_y(x^4y) = 10\), we need to find the value of \(xy\).\newline

    First, we convert the logarithmic equations to their exponential forms using the change of base formula.\newline

    1. \(\log_x(y^x) = 10\)\newline
       \[
       x \log_x(y) = 10
       \]
       Using the change of base formula, \(\log_x(y) = \frac{\ln y}{\ln x}\). Since \(\log_y x = \frac{9}{4}\), we have \(\log_x y = \frac{4}{9}\). Thus,\newline
       \[
       x \cdot \frac{4}{9} = 10 \implies x = \frac{10 \cdot 9}{4} = \frac{90}{4} = \frac{45}{2}
       \]

    2. \(\log_y(x^4y) = 10\)\newline
       \[
       4 \log_y x + 1 = 10 \implies 4 \log_y x = 9 \implies \log_y x = \frac{9}{4}
       \]
       Again, using the change of base formula, \(\log_x y = \frac{4}{9}\).\newline

    Next, we express \(x\) and \(y\) in terms of each other:\newline
       \[
       x = \frac{45}{2} \quad \text{and} \quad y = \left(\frac{45}{2}\right)^{4/9}
       \]

    We then compute the product \(xy\):\newline
       \[
       xy = \left(\frac{45}{2}\right) \cdot \left(\frac{45}{2}\right)^{4/9} = \left(\frac{45}{2}\right)^{1 + 4/9} = \left(\frac{45}{2}\right)^{13/9}
       \]

    However, upon re-evaluating and using another approach by setting \(x = 2^a\) and \(y = 2^b\), we find:\newline
       \[
       \log_x(y^x) = 10 \implies \frac{x \log_x y}{\log_x y} = 10 \implies x \cdot \frac{4}{9} = 10 \implies x = \frac{45}{2}
       \]
       \[
       \log_y(x^4y) = 10 \implies \frac{4 \log_y x + 1}{\log_y x} = 10 \implies 4 \cdot \frac{9}{4} + 1 = 10 \implies 9 + 1 = 10 \quad \text{consistent}
       \]

    Finally, we find that the product \(xy\) simplifies to:\newline
       \[
       xy = \left(\frac{45}{2}\right)
       \]
    \end{quote}%
}