\documentclass[nolineno,a4paper,USenglish,cleveref,autoref,thm-restate]{socg-lipics-v2021}
%This is a template for producing LIPIcs articles. 
%See lipics-v2021-authors-guidelines.pdf for further information.
%for A4 paper format use option "a4paper", for US-letter use option "letterpaper"
%for british hyphenation rules use option "UKenglish", for american hyphenation rules use option "USenglish"
%for section-numbered lemmas etc., use "numberwithinsect"
%for enabling cleveref support, use "cleveref"
%for enabling autoref support, use "autoref"
%for anonymousing the authors (e.g. for double-blind review), add "anonymous"
%for enabling thm-restate support, use "thm-restate"
%for enabling a two-column layout for the author/affilation part (only applicable for > 6 authors), use "authorcolumns"
%for producing a PDF according the PDF/A standard, add "pdfa"

\pdfoutput=1 %uncomment to ensure pdflatex processing (mandatatory e.g. to submit to arXiv)
\hideLIPIcs  %uncomment to remove references to LIPIcs series (logo, DOI, ...), e.g. when preparing a pre-final version to be uploaded to arXiv or another public repository

%\graphicspath{{./graphics/}}%helpful if your graphic files are in another directory

\clubpenalty=10000
\widowpenalty=10000

\usepackage[group-separator={,},output-decimal-marker={.}]{siunitx}
\usepackage{nicefrac}
\usepackage{complexity}
\usepackage{microtype}
\let\R\undefinedcommand
\usepackage[]{algorithm2e}
\usepackage{longtable}
\usepackage[commandnameprefix=always,final]{changes}

\crefname{equation}{Equation}{Equations}

\bibliographystyle{plainurl}% the mandatory bibstyle

\title{Exact Algorithms for Minimum Dilation Triangulation}

%\titlerunning{Dummy short title} %optional, use if title is longer than one line

%\author{Jane {Open Access}}{Dummy University Computing Laboratory, [optional: Address], Country \and My second affiliation, Country \and \url{http://www.myhomepage.edu} }{johnqpublic@dummyuni.org}{https://orcid.org/0000-0002-1825-0097}{(Optional) author-specific funding acknowledgements}%mandatory, please use full name; only 1 author per \author macro; first two parameters are mandatory, other parameters can be empty. Please provide at least the name of the affiliation and the country. The full address is optional. Use additional curly braces to indicate the correct name splitting when the last name consists of multiple name parts.

\author{Sándor P. Fekete}{Department of Computer Science, TU Braunschweig}{fekete@tu-braunschweig.de}{https://orcid.org/0000-0002-9062-4241}{}
\author{Phillip Keldenich}{Department of Computer Science, TU Braunschweig}{keldenich@ibr.cs.tu-bs.de}{https://orcid.org/0000-0002-6677-5090}{}
\author{Michael Perk}{Department of Computer Science, TU Braunschweig}{fekete@tu-braunschweig.de}{https://orcid.org/0000-0002-0141-8594}{}
\authorrunning{S.\,P.\ Fekete and P.\ Keldenich and M.\ Perk} %First: Use abbreviated first/middle names. Second (only in severe cases): Use first author plus 'et al.'

\Copyright{Sándor P. Fekete and Phillip Keldenich and Michael Perk} %mandatory, please use full first names. LIPIcs license is "CC-BY";  http://creativecommons.org/licenses/by/3.0/

\begin{CCSXML}
<ccs2012>
<concept>
<concept_id>10003752.10010061.10010063</concept_id>
<concept_desc>Theory of computation~Computational geometry</concept_desc>
<concept_significance>500</concept_significance>
</concept>
<concept>
<concept_id>10003752.10003809.10003716.10011136</concept_id>
<concept_desc>Theory of computation~Discrete optimization</concept_desc>
<concept_significance>500</concept_significance>
</concept>
<concept>
<concept_id>10002944.10011123.10011131</concept_id>
<concept_desc>General and reference~Experimentation</concept_desc>
<concept_significance>300</concept_significance>
</concept>
<concept>
<concept_id>10002950.10003714.10003715.10003725</concept_id>
<concept_desc>Mathematics of computing~Interval arithmetic</concept_desc>
<concept_significance>300</concept_significance>
</concept>
<concept>
<concept_id>10002950.10003714.10003715.10003726</concept_id>
<concept_desc>Mathematics of computing~Arbitrary-precision arithmetic</concept_desc>
<concept_significance>100</concept_significance>
</concept>
</ccs2012>
\end{CCSXML}

\ccsdesc[500]{Theory of computation~Computational geometry}
\ccsdesc[500]{Theory of computation~Discrete optimization}
\ccsdesc[300]{General and reference~Experimentation}
\ccsdesc[300]{Mathematics of computing~Interval arithmetic}
\ccsdesc[100]{Mathematics of computing~Arbitrary-precision arithmetic}
\keywords{dilation, minimum dilation triangulation, exact algorithms, algorithm engineering, experimental evaluation}

\category{} %optional, e.g. invited paper

%\relatedversion{https://arxiv.org/abs/TODO.XXXXX} %TODO full arxiv version
%\relatedversiondetails[linktext={opt. text shown instead of the URL}, cite=DBLP:books/mk/GrayR93]{Classification (e.g. Full Version, Extended Version, Previous Version}{URL to related version} %linktext and cite are optional


%\supplement{}%optional, e.g. related research data, source code, ... hosted on a repository like zenodo, figshare, GitHub, ...
%\supplementdetails[linktext={opt. text shown instead of the URL}, cite=DBLP:books/mk/GrayR93, subcategory={Description, Subcategory}, swhid={Software Heritage Identifier}]{General Classification (e.g. Software, Dataset, Model, ...)}{URL to related version} %linktext, cite, and subcategory are optional

\funding{
  The work presented in this paper was largely funded by the DFG grant 
  ``Computational Geometry: Solving Hard Optimization Problems'' (CG:SHOP), grant FE407/21-1.} 

\acknowledgements{}%optional

%\nolinenumbers %uncomment to disable line numbering
% we have 500 lines!

%\supplement{}
%\supplementdetails[linktext={opt. text shown instead of the URL}, cite=DBLP:books/mk/GrayR93, subcategory={Description, Subcategory}, swhid={Software Heritage Identifier}]{General Classification (e.g. Software, Dataset, Model, ...)}{URL to related version} %linktext, cite, and subcategory are optional

%Editor-only macros:: begin (do not touch as author)%%%%%%%%%%%%%%%%%%%%%%%%%%%%%%%%%%

\EventEditors{Oswin Aichholzer and Haitao Wang}
\EventNoEds{2}
\EventLongTitle{41st International Symposium on Computational Geometry (SoCG 2025)}
\EventShortTitle{SoCG 2025}
\EventAcronym{SoCG}
\EventYear{2025}
\EventDate{June 23--27, 2025}
\EventLocation{Kanazawa, Japan}
\EventLogo{socg-logo.pdf}
\SeriesVolume{332}
\ArticleNo{174}     % <-- This will be filled in by the typesetters
%%%%%%%%%%%%%%%%%%%%%%%%%%%%%%%%%%%%%%%%%%%%%%%%%%%%%%

\newcommand{\R}{\mathbb{R}}


\newcommand{\lbEpsilon}{2.730751 \cdot 10^{-16}}
\newcommand{\lbDelta}{6.458762 \cdot 10^{-16}}
\newcommand{\lbN}{84}
\newcommand{\lbRho}{1.44116645381}
\newcommand{\lbRhoShort}{1.44116}
\begin{document}

\supplement{Code, experiment instances and results are archived on Zenodo.}
\supplementdetails[subcategory={Source Code}]{Software}{https://doi.org/10.5281/zenodo.14266122}
\supplementdetails[subcategory={Experiment Data}]{Dataset}{https://doi.org/10.5281/zenodo.14266122}

\maketitle

\begin{abstract}
We provide a spectrum of new theoretical insights and practical results 
for finding 
a Minimum Dilation Triangulation (MDT), a natural geometric optimization 
problem of considerable previous attention:
Given a set $P$ of $n$ points in the plane, find a triangulation
$T$, such that a shortest Euclidean path in $T$ between any pair of points
increases by the smallest possible factor compared to their
straight-line distance. No polynomial-time algorithm is known for the problem;
moreover, evaluating the objective function involves computing the sum
of (possibly many) square roots. 
On the other hand, the problem is not known to be \NP-hard.

(1) We provide practically robust methods and implementations for computing an MDT
for benchmark sets with up to 30,000 points in reasonable time on commodity
hardware, based on new geometric insights into the structure of optimal edge
sets. Previous methods only achieved results for up to $200$ points, so we extend 
the range of optimally solvable instances by a factor of $150$.

(2) We develop scalable techniques for accurately
evaluating many shortest-path queries that arise as large-scale sums of square
roots, allowing us to certify exact optimal solutions,  
with previous work relying on (possibly inaccurate) floating-point computations.

(3) We resolve an open problem by establishing a lower bound of
$\lbRhoShort$ on the dilation of the regular $\lbN$-gon (and thus for arbitrary point
sets), improving the previous worst-case lower bound of $1.4308$ 
and greatly reducing the remaining gap to the upper bound of
$1.4482$ from the literature. In the process, we provide optimal solutions for regular
$n$-gons up to $n = 100$.
\end{abstract}

\section{Introduction}
\label{section:introduction}

% redirection is unique and important in VR
Virtual Reality (VR) systems enable users to embody virtual avatars by mirroring their physical movements and aligning their perspective with virtual avatars' in real time. 
As the head-mounted displays (HMDs) block direct visual access to the physical world, users primarily rely on visual feedback from the virtual environment and integrate it with proprioceptive cues to control the avatar’s movements and interact within the VR space.
Since human perception is heavily influenced by visual input~\cite{gibson1933adaptation}, 
VR systems have the unique capability to control users' perception of the virtual environment and avatars by manipulating the visual information presented to them.
Leveraging this, various redirection techniques have been proposed to enable novel VR interactions, 
such as redirecting users' walking paths~\cite{razzaque2005redirected, suma2012impossible, steinicke2009estimation},
modifying reaching movements~\cite{gonzalez2022model, azmandian2016haptic, cheng2017sparse, feick2021visuo},
and conveying haptic information through visual feedback to create pseudo-haptic effects~\cite{samad2019pseudo, dominjon2005influence, lecuyer2009simulating}.
Such redirection techniques enable these interactions by manipulating the alignment between users' physical movements and their virtual avatar's actions.

% % what is hand/arm redirection, motivation of study arm-offset
% \change{\yj{i don't understand the purpose of this paragraph}
% These illusion-based techniques provide users with unique experiences in virtual environments that differ from the physical world yet maintain an immersive experience. 
% A key example is hand redirection, which shifts the virtual hand’s position away from the real hand as the user moves to enhance ergonomics during interaction~\cite{feuchtner2018ownershift, wentzel2020improving} and improve interaction performance~\cite{montano2017erg, poupyrev1996go}. 
% To increase the realism of virtual movements and strengthen the user’s sense of embodiment, hand redirection techniques often incorporate a complete virtual arm or full body alongside the redirected virtual hand, using inverse kinematics~\cite{hartfill2021analysis, ponton2024stretch} or adjustments to the virtual arm's movement as well~\cite{li2022modeling, feick2024impact}.
% }

% noticeability, motivation of predicting a probability, not a classification
However, these redirection techniques are most effective when the manipulation remains undetected~\cite{gonzalez2017model, li2022modeling}. 
If the redirection becomes too large, the user may not mitigate the conflict between the visual sensory input (redirected virtual movement) and their proprioception (actual physical movement), potentially leading to a loss of embodiment with the virtual avatar and making it difficult for the user to accurately control virtual movements to complete interaction tasks~\cite{li2022modeling, wentzel2020improving, feuchtner2018ownershift}. 
While proprioception is not absolute, users only have a general sense of their physical movements and the likelihood that they notice the redirection is probabilistic. 
This probability of detecting the redirection is referred to as \textbf{noticeability}~\cite{li2022modeling, zenner2024beyond, zenner2023detectability} and is typically estimated based on the frequency with which users detect the manipulation across multiple trials.

% version B
% Prior research has explored factors influencing the noticeability of redirected motion, including the redirection's magnitude~\cite{wentzel2020improving, poupyrev1996go}, direction~\cite{li2022modeling, feuchtner2018ownershift}, and the visual characteristics of the virtual avatar~\cite{ogawa2020effect, feick2024impact}.
% While these factors focus on the avatars, the surrounding virtual environment can also influence the users' behavior and in turn affect the noticeability of redirection.
% One such prominent external influence is through the visual channel - the users' visual attention is constantly distracted by complex visual effects and events in practical VR scenarios.
% Although some prior studies have explored how to leverage user blindness caused by visual distractions to redirect users' virtual hand~\cite{zenner2023detectability}, there remains a gap in understanding how to quantify the noticeability of redirection under visual distractions.

% visual stimuli and gaze behavior
Prior research has explored factors influencing the noticeability of redirected motion, including the redirection's magnitude~\cite{wentzel2020improving, poupyrev1996go}, direction~\cite{li2022modeling, feuchtner2018ownershift}, and the visual characteristics of the virtual avatar~\cite{ogawa2020effect, feick2024impact}.
While these factors focus on the avatars, the surrounding virtual environment can also influence the users' behavior and in turn affect the noticeability of redirection.
This, however, remains underexplored.
One such prominent external influence is through the visual channel - the users' visual attention is constantly distracted by complex visual effects and events in practical VR scenarios.
We thus want to investigate how \textbf{visual stimuli in the virtual environment} affect the noticeability of redirection.
With this, we hope to complement existing works that focus on avatars by incorporating environmental visual influences to enable more accurate control over the noticeability of redirected motions in practical VR scenarios.
% However, in realistic VR applications, the virtual environment often contains complex visual effects beyond the virtual avatar itself. 
% We argue that these visual effects can \textbf{distract users’ visual attention and thus affect the noticeability of redirection offsets}, while current research has yet taken into account.
% For instance, in a VR boxing scenario, a user’s visual attention is likely focused on their opponent rather than on their virtual body, leading to a lower noticeability of redirection offsets on their virtual movements. 
% Conversely, when reaching for an object in the center of their field of view, the user’s attention is more concentrated on the virtual hand’s movement and position to ensure successful interaction, resulting in a higher noticeability of offsets.

Since each visual event is a complex choreography of many underlying factors (type of visual effect, location, duration, etc.), it is extremely difficult to quantify or parameterize visual stimuli.
Furthermore, individuals respond differently to even the same visual events.
Prior neuroscience studies revealed that factors like age, gender, and personality can influence how quickly someone reacts to visual events~\cite{gillon2024responses, gale1997human}. 
Therefore, aiming to model visual stimuli in a way that is generalizable and applicable to different stimuli and users, we propose to use users' \textbf{gaze behavior} as an indicator of how they respond to visual stimuli.
In this paper, we used various gaze behaviors, including gaze location, saccades~\cite{krejtz2018eye}, fixations~\cite{perkhofer2019using}, and the Index of Pupil Activity (IPA)~\cite{duchowski2018index}.
These behaviors indicate both where users are looking and their cognitive activity, as looking at something does not necessarily mean they are attending to it.
Our goal is to investigate how these gaze behaviors stimulated by various visual stimuli relate to the noticeability of redirection.
With this, we contribute a model that allows designers and content creators to adjust the redirection in real-time responding to dynamic visual events in VR.

To achieve this, we conducted user studies to collect users' noticeability of redirection under various visual stimuli.
To simulate realistic VR scenarios, we adopted a dual-task design in which the participants performed redirected movements while monitoring the visual stimuli.
Specifically, participants' primary task was to report if they noticed an offset between the avatar's movement and their own, while their secondary task was to monitor and report the visual stimuli.
As realistic virtual environments often contain complex visual effects, we started with simple and controlled visual stimulus to manage the influencing factors.

% first user study, confirmation study
% collect data under no visual stimuli, different basic visual stimuli
We first conducted a confirmation study (N=16) to test whether applying visual stimuli (opacity-based) actually affects their noticeability of redirection. 
The results showed that participants were significantly less likely to detect the redirection when visual stimuli was presented $(F_{(1,15)}=5.90,~p=0.03)$.
Furthermore, by analyzing the collected gaze data, results revealed a correlation between the proposed gaze behaviors and the noticeability results $(r=-0.43)$, confirming that the gaze behaviors could be leveraged to compute the noticeability.

% data collection study
We then conducted a data collection study to obtain more accurate noticeability results through repeated measurements to better model the relationship between visual stimuli-triggered gaze behaviors and noticeability of redirection.
With the collected data, we analyzed various numerical features from the gaze behaviors to identify the most effective ones. 
We tested combinations of these features to determine the most effective one for predicting noticeability under visual stimuli.
Using the selected features, our regression model achieved a mean squared error (MSE) of 0.011 through leave-one-user-out cross-validation. 
Furthermore, we developed both a binary and a three-class classification model to categorize noticeability, which achieved an accuracy of 91.74\% and 85.62\%, respectively.

% evaluation study
To evaluate the generalizability of the regression model, we conducted an evaluation study (N=24) to test whether the model could accurately predict noticeability with new visual stimuli (color- and scale-based animations).
Specifically, we evaluated whether the model's predictions aligned with participants' responses under these unseen stimuli.
The results showed that our model accurately estimated the noticeability, achieving mean squared errors (MSE) of 0.014 and 0.012 for the color- and scale-based visual stimili, respectively, compared to participants' responses.
Since the tested visual stimuli data were not included in the training, the results suggested that the extracted gaze behavior features capture a generalizable pattern and can effectively indicate the corresponding impact on the noticeability of redirection.

% application
Based on our model, we implemented an adaptive redirection technique and demonstrated it through two applications: adaptive VR action game and opportunistic rendering.
We conducted a proof-of-concept user study (N=8) to compare our adaptive redirection technique with a static redirection, evaluating the usability and benefits of our adaptive redirection technique.
The results indicated that participants experienced less physical demand and stronger sense of embodiment and agency when using the adaptive redirection technique. 
These results demonstrated the effectiveness and usability of our model.

In summary, we make the following contributions.
% 
\begin{itemize}
    \item 
    We propose to use users' gaze behavior as a medium to quantify how visual stimuli influences the noticebility of redirection. 
    Through two user studies, we confirm that visual stimuli significantly influences noticeability and identify key gaze behavior features that are closely related to this impact.
    \item 
    We build a regression model that takes the user's gaze behavioral data as input, then computes the noticeability of redirection.
    Through an evaluation study, we verify that our model can estimate the noticeability with new participants under unseen visual stimuli.
    These findings suggest that the extracted gaze behavior features effectively capture the influence of visual stimuli on noticeability and can generalize across different users and visual stimuli.
    \item 
    We develop an adaptive redirection technique based on our regression model and implement two applications with it.
    With a proof-of-concept study, we demonstrate the effectiveness and potential usability of our regression model on real-world use cases.

\end{itemize}

% \delete{
% Virtual Reality (VR) allows the user to embody a virtual avatar by mirroring their physical movements through the avatar.
% As the user's visual access to the physical world is blocked in tasks involving motion control, they heavily rely on the visual representation of the avatar's motions to guide their proprioception.
% Similar to real-world experiences, the user is able to resolve conflicts between different sensory inputs (e.g., vision and motor control) through multisensory integration, which is essential for mitigating the sensory noise that commonly arises.
% However, it also enables unique manipulations in VR, as the system can intentionally modify the avatar's movements in relation to the user's motions to achieve specific functional outcomes,
% for example, 
% % the manipulations on the avatar's movements can 
% enabling novel interaction techniques of redirected walking~\cite{razzaque2005redirected}, redirected reaching~\cite{gonzalez2022model}, and pseudo haptics~\cite{samad2019pseudo}.
% With small adjustments to the avatar's movements, the user can maintain their sense of embodiment, due to their ability to resolve the perceptual differences.
% % However, a large mismatch between the user and avatar's movements can result in the user losing their sense of embodiment, due to an inability to resolve the perceptual differences.
% }

% \delete{
% However, multisensory integration can break when the manipulation is so intense that the user is aware of the existence of the motion offset and no longer maintains the sense of embodiment.
% Prior research studied the intensity threshold of the offset applied on the avatar's hand, beyond which the embodiment will break~\cite{li2022modeling}. 
% Studies also investigated the user's sensitivity to the offsets over time~\cite{kohm2022sensitivity}.
% Based on the findings, we argue that one crucial factor that affects to what extent the user notices the offset (i.e., \textit{noticeability}) that remains under-explored is whether the user directs their visual attention towards or away from the virtual avatar.
% Related work (e.g., Mise-unseen~\cite{marwecki2019mise}) has showcased applications where adjustments in the environment can be made in an unnoticeable manner when they happen in the area out of the user's visual field.
% We hypothesize that directing the user's visual attention away from the avatar's body, while still partially keeping the avatar within the user's field-of-view, can reduce the noticeability of the offset.
% Therefore, we conduct two user studies and implement a regression model to systematically investigate this effect.
% }

% \delete{
% In the first user study (N = 16), we test whether drawing the user's visual attention away from their body impacts the possibility of them noticing an offset that we apply to their arm motion in VR.
% We adopt a dual-task design to enable the alteration of the user's visual attention and a yes/no paradigm to measure the noticeability of motion offset. 
% The primary task for the user is to perform an arm motion and report when they perceive an offset between the avatar's virtual arm and their real arm.
% In the secondary task, we randomly render a visual animation of a ball turning from transparent to red and becoming transparent again and ask them to monitor and report when it appears.
% We control the strength of the visual stimuli by changing the duration and location of the animation.
% % By changing the time duration and location of the visual animation, we control the strengths of attraction to the users.
% As a result, we found significant differences in the noticeability of the offsets $(F_{(1,15)}=5.90,~p=0.03)$ between conditions with and without visual stimuli.
% Based on further analysis, we also identified the behavioral patterns of the user's gaze (including pupil dilation, fixations, and saccades) to be correlated with the noticeability results $(r=-0.43)$ and they may potentially serve as indicators of noticeability.
% }

% \delete{
% To further investigate how visual attention influences the noticeability, we conduct a data collection study (N = 12) and build a regression model based on the data.
% The regression model is able to calculate the noticeability of the offset applied on the user's arm under various visual stimuli based on their gaze behaviors.
% Our leave-one-out cross-validation results show that the proposed method was able to achieve a mean-squared error (MSE) of 0.012 in the probability regression task.
% }

% \delete{
% To verify the feasibility and extendability of the regression model, we conduct an evaluation study where we test new visual animations based on adjustments on scale and color and invite 24 new participants to attend the study.
% Results show that the proposed method can accurately estimate the noticeability with an MSE of 0.014 and 0.012 in the conditions of the color- and scale-based visual effects.
% Since these animations were not included in the dataset that the regression model was built on, the study demonstrates that the gaze behavioral features we extracted from the data capture a generalizable pattern of the user's visual attention and can indicate the corresponding impact on the noticeability of the offset.
% }

% \delete{
% Finally, we demonstrate applications that can benefit from the noticeability prediction model, including adaptive motion offsets and opportunistic rendering, considering the user's visual attention. 
% We conclude with discussions of our work's limitations and future research directions.
% }

% \delete{
% In summary, we make the following contributions.
% }
% % 
% \begin{itemize}
%     \item 
%     \delete{
%     We quantify the effects of the user's visual attention directed away by stimuli on their noticeability of an offset applied to the avatar's arm motion with respect to the user's physical arm. 
%     Through two user studies, we identified gaze behavioral features that are indicative of the changes in noticeability.
%     }
%     \item 
%     \delete{We build a regression model that takes the user's gaze behavioral data and the offset applied to the arm motion as input, then computes the probability of the user noticing the offset.
%     Through an evaluation study, we verified that the model needs no information about the source attracting the user's visual attention and can be generalizable in different scenarios.
%     }
%     \item 
%     \delete{We demonstrate two applications that potentially benefit from the regression model, including adaptive motion offsets and opportunistic rendering.
%     }

% \end{itemize}

\begin{comment}
However, users will lose the sense of embodiment to the virtual avatars if they notice the offset between the virtual and physical movements.
To address this, researchers have been exploring the noticing threshold of offsets with various magnitudes and proposing various redirection techniques that maintain the sense of embodiment~\cite{}.

However, when users embody virtual avatars to explore virtual environments, they encounter various visual effects and content that can attract their attention~\cite{}.
During this, the user may notice an offset when he observes the virtual movement carefully while ignoring it when the virtual contents attract his attention from the movements.
Therefore, static offset thresholds are not appropriate in dynamic scenarios.

Past research has proposed dynamic mapping techniques that adapted to users' state, such as hand moving speed~\cite{frees2007prism} or ergonomically comfortable poses~\cite{montano2017erg}, but not considering the influence of virtual content.
More specifically, PRISM~\cite{frees2007prism} proposed adjusting the C/D ratio with a non-linear mapping according to users' hand moving speed, but it might not be optimal for various virtual scenarios.
While Erg-O~\cite{montano2017erg} redirected users' virtual hands according to the virtual target's relative position to reduce physical fatigue, neglecting the change of virtual environments. 

Therefore, how to design redirection techniques in various scenarios with different visual attractions remains unknown.
To address this, we investigate how visual attention affects the noticing probability of movement offsets.
Based on our experiments, we implement a computational model that automatically computes the noticing probability of offsets under certain visual attractions.
VR application designers and developers can easily leverage our model to design redirection techniques maintaining the sense of embodiment adapt to the user's visual attention.
We implement a dynamic redirection technique with our model and demonstrate that it effectively reduces the target reaching time without reducing the sense of embodiment compared to static redirection techniques.

% Need to be refined
This paper offers the following contributions.
\begin{itemize}
    \item We investigate how visual attractions affect the noticing probability of redirection offsets.
    \item We construct a computational model to predict the noticing probability of an offset with a given visual background.
    \item We implement a dynamic redirection technique adapting to the visual background. We evaluate the technique and develop three applications to demonstrate the benefits. 
\end{itemize}



First, we conducted a controlled experiment to understand how users perceived the movement offset while subjected to various distractions.
Since hand redirection is one of the most frequently used redirections in VR interactions, we focused on the dynamic arm movements and manually added angular offsets to the' elbow joint~\cite{li2022modeling, gonzalez2022model, zenner2019estimating}. 
We employed flashing spheres in the user's field of view as distractions to attract users' visual attention.
Participants were instructed to report the appearing location of the spheres while simultaneously performing the arm movements and reporting if they perceived an offset during the movement. 
(\zhipeng{Add the results of data collection. Analyze the influence of the distance between the gaze map and the offset.}
We measured the visual attraction's magnitude with the gaze distribution on it.
Results showed that stronger distractions made it harder for users to notice the offset.)
\zhipeng{Need to rewrite. Not sure to use gaze distribution or a metric obtained from the visual content.}
Secondly, we constructed a computational model to predict the noticing probability of offsets with given visual content.
We analyzed the data from the user studies to measure the influence of visual attractions on the noticing probability of offsets.
We built a statistical model to predict the offset's noticing probability with a given visual content.
Based on the model, we implement a dynamic redirection technique to adjust the redirection offset adapted to the user's current field of view.
We evaluated the technique in a target selection task compared to no hand redirection and static hand redirection.
\zhipeng{Add the results of the evaluation.}
Results showed that the dynamic hand redirection technique significantly reduced the target selection time with similar accuracy and a comparable sense of embodiment.
Finally, we implemented three applications to demonstrate the potential benefits of the visual attention adapted dynamic redirection technique.
\end{comment}

% This one modifies arm length, not redirection
% \citeauthor{mcintosh2020iteratively} proposed an adaptation method to iteratively change the virtual avatar arm's length based on the primary tasks' performance~\cite{mcintosh2020iteratively}.



% \zhipeng{TO ADD: what is redirection}
% Redirection enables novel interactions in Virtual Reality, including redirected walking, haptic redirection, and pseudo haptics by introducing an offset to users' movement.
% \zhipeng{TO ADD: extend this sentence}
% The price of this is that users' immersiveness and embodiment in VR can be compromised when they notice the offset and perceive the virtual movement not as theirs~\cite{}.
% \zhipeng{TO ADD: extend this sentence, elaborate how the virtual environment attracts users' attention}
% Meanwhile, the visual content in the virtual environment is abundant and consistently captures users' attention, making it harder to notice the offset~\cite{}.
% While previous studies explored the noticing threshold of the offsets and optimized the redirection techniques to maintain the sense of embodiment~\cite{}, the influence of visual content on the probability of perceiving offsets remains unknown.  
% Therefore, we propose to investigate how users perceive the redirection offset when they are facing various visual attractions.


% We conducted a user study to understand how users notice the shift with visual attractions.
% We used a color-changing ball to attract the user's attention while instructing users to perform different poses with their arms and observe it meanwhile.
% \zhipeng{(Which one should be the primary task? Observe the ball should be the primary one, but if the primary task is too simple, users might allocate more attention on the secondary task and this makes the secondary task primary.)}
% \zhipeng{(We need a good and reasonable dual-task design in which users care about both their pose and the visual content, at least in the evaluation study. And we need to be able to control the visual content's magnitude and saliency maybe?)}
% We controlled the shift magnitude and direction, the user's pose, the ball's size, and the color range.
% We set the ball's color-changing interval as the independent factor.
% We collect the user's response to each shift and the color-changing times.
% Based on the collected data, we constructed a statistical model to describe the influence of visual attraction on the noticing probability.
% \zhipeng{(Are we actually controlling the attention allocation? How do we measure the attracting effect? We need uniform metrics, otherwise it is also hard for others to use our knowledge.)}
% \zhipeng{(Try to use eye gaze? The eye gaze distribution in the last five seconds to decide the attention allocation? Basically constructing a model with eye gaze distribution and noticing probability. But the user's head is moving, so the eye gaze distribution is not aligned well with the current view.)}

% \zhipeng{Saliency and EMD}
% \zhipeng{Gaze is more than just a point: Rethinking visual attention
% analysis using peripheral vision-based gaze mapping}

% Evaluation study(ideal case): based on the visual content, adjusting the redirection magnitude dynamically.

% \zhipeng{(The risk is our model's effect is trivial.)}

% Applications:
% Playing Lego while watching demo videos, we can accelerate the reaching process of bricks, and forbid the redirection during the manipulation.

% Beat saber again: but not make a lot of sense? Difficult game has complicated visual effects, while allows larger shift, but do not need large shift with high difficulty



\section{Preliminaries}
\label{sec:preliminaries}

Let $P \subset \mathbb{R}^2$ be a set of points in the plane.
We denote the Euclidean distance between two points $u,v \in P$ by $d(u,v)$.
For a connected geometric graph $G = (P, E)$ with $E \subseteq {P \choose 2}$, we denote the Euclidean shortest path between two points $u,v \in P$ by $\pi_G(u,v)$ and its length by $|\pi_G(u,v)|$,
omitting $G$ if it is clear from context.
The dilation $\rho_G(u,v)$ between two points $u,v$ in $G$ is the ratio $\rho_G(u,v) := \frac{|\pi_G(u,v)|}{d(u,v)}$ between the shortest path length and the Euclidean distance.
The dilation $\rho(G)$ of the graph $G$ is defined as the maximum dilation between any two points in $P$,
i.e., $\rho(G) := \max \{ \rho_G(u,v) \mid u,v \in P, u \neq v\}.$

In the remainder of this work, the graph $G$ we consider is a triangulation, i.e., a maximal crossing-free graph on $P$.
Two edges $e_1 = (p_1, q_1), e_2 = (p_2, q_2)$ are said to \emph{cross} or \emph{intersect} iff the line segments they induce intersect in their interior.
Given a point set $P$, the Minimum Dilation Triangulation problem (MDT) asks to find a triangulation $T$ of $P$ minimizing $\rho(T)$.

\section{Candidate edge enumeration}
\label{sec:edge-enumeration}

Here we describe a novel and practically efficient scheme for
enumerating a set of edges that induces a supergraph of the MDT.
We start with the underlying theoretical ideas for this supergraph, followed
by an algorithm for computing a supergraph of any triangulation with dilation \emph{strictly less} than a given bound $\rho$,
which we exploit to enumerate a supergraph of the MDT with a (usually) small number of edges.
This is further adapted to reductions of the dilation bound $\rho$.
We also discuss the computation of lower bounds on the dilation of the MDT.
We defer the discussion of some implementation details to \cref{sec:implementation-details}.

\subsection{Theoretical background}
Our supergraph is based on the well-known \emph{ellipse property} (used in \cite{DBLP:conf/cccg/BrandtGSR14,DBLP:journals/ijcga/GiannopoulosKKKM10,DBLP:conf/ewcg/KnauerM05}) that all edges of a triangulation $T$ with dilation below $\rho$ must satisfy.

\begin{definition}
  A pair of points $s, t$ has the \emph{ellipse property} with 
  respect to a point set $P$ and a dilation bound $\rho$ if, for
  any pair of points $\ell, r \in P \setminus \{s,t\}$ such that
  $\ell r$ and $st$ intersect, $\min \{d(\ell,s) + d(s,r), d(\ell,t) + d(t,r)\} < \rho d(\ell,r)$. 
\end{definition}

Recall that the set of points that have the same sum of distances $\kappa$ to 
two points $\ell, r$ with $\kappa \geq d(\ell,r)$ is an \emph{ellipse} with $\ell, r$ as its focal points (or \emph{foci});
thus, all paths between $\ell$ and $r$ with length less than $\kappa = \rho \cdot d(\ell,r)$ must lie strictly inside this ellipse.
%with foci $\ell, r$ and \emph{focal distance sum} $\kappa$.

If a pair of points $s,t$ does not have the ellipse property, then there is a pair of points $\ell, r$
such that $st$ cuts through all paths between $\ell$ and $r$ that could have dilation less than $\rho$; see \cref{fig:ellipse-property-example}.
We call such a pair of points $\ell, r$ an \emph{exclusion certificate} for $s,t$.
\begin{figure}%
  \centering
  \includegraphics{./figures/ellipse-property-example.pdf}
  \caption{
    Any path connecting $\ell$ and $r$ with length below $\kappa = \rho d(\ell,r)$ must lie within the ellipse (dashed black lines).
    The edge $st$ does not have the \emph{ellipse property} as neither $s$ nor $t$ lie inside the ellipse;
    therefore, inserting $st$ makes connecting $\ell$ and $r$ by a sufficiently short path impossible.
  }
  \label{fig:ellipse-property-example}
\end{figure}%

\begin{observation}
  Let $\rho \geq 1$ be some dilation bound and let $T$ be a triangulation containing the edge $st$
  for points $s,t$ that do not have the ellipse property w.r.t.\ $P$ and $\rho$.
  Then, $\rho(T) \geq \rho$.
\end{observation}

\subsection{High-level description}
\label{sec:ellipse-filter-algo}
Preliminary experiments showed that a brute-force check of each of the $\Theta(n^4)$ pairs of
potential edges is impractical for large $n$, dominating the overall runtime time even for
early versions of our solution approach.

We therefore developed a more efficient scheme to enumerate a superset of the edges that satisfy the ellipse property.
This scheme performs a \emph{filtered incremental nearest-neighbor search} from each point $p \in P$
to identify all candidate edges of the form $pt$, i.e., looking for all possible \emph{neighbors} $t$ of $p$.
This search is efficiently implemented on a quadtree containing all points.
While enumerating candidates, we construct so-called \emph{dead sectors}, i.e., regions of the plane that
cannot contain possible neighbors of $p$.
We exclude all points that lie in dead sectors; we also use dead sectors to prune entire nodes of the quadtree
and to terminate the search early if it has become clear that all remaining points must be in a dead sector.
This usually avoids considering most points as potential neighbors of $p$ individually.
The algorithm has a runtime of $O(nk\log n)$, where $k$ is the average number of points and quadtree vertices considered individually from each point.
At worst, this can be $O(n^2\log n)$; in practice, $k$ is often much lower than $n$.
A related, slightly less complex enumeration algorithm applied to minimum-weight triangulations by Haas~\cite{haasmwt} scales to $10^8$ points.
In the following, we describe the components of the enumeration scheme in more detail.

\subsection{Dead sectors}
\label{sec:dead-sectors}
We begin by giving a definition of the dead sectors we use.
\begin{definition}
  Given a dilation $\rho$, a source point $p$ and two points $\ell, r$,
  the \emph{dead sector} $\mathcal{DS}_{\rho}(p, \ell, r) \subset \mathbb{R}^2$ is the region of all points $t$ such that 
  $pt$ intersects $\ell r$ and neither $p$ nor $t$ lie in the ellipse with foci $\ell, r$ and focal distance sum 
  $\kappa = \rho d(\ell, r)$.
\end{definition}
Depending on $p$, $\rho$, $\ell$ and $r$, $\mathcal{DS}_{\rho}(p, \ell, r)$ is either empty (if $p$ is in the ellipse)
or it is bounded by two rays and an elliptic arc; see \cref{fig:dead-sector-construction}.
In that case, it can also be interpreted as an elliptic arc and an interval of polar angles around $p$.

During our enumeration we construct many dead sectors,
the union of which can become quite complex, making it cumbersome and inefficient to work with directly.
We instead chose to simplify the shape of our dead sectors, giving up 
a small fraction of excluded area in exchange for a simple and efficient representation.
We replace the elliptic arc by a single disk centered on $p$,
whose radius is at least the maximum distance from $p$ to any point on the elliptic arc.
We can hence represent each non-empty simplified dead sector by a polar angle interval around $p$ 
and a single radius called \emph{activation distance} $A_{\rho}(p,\ell,r)$; see \cref{fig:dead-sector-construction}.

\begin{figure}
  \centering
  \includegraphics{./figures/dead_sector.pdf}
  \caption{A dead sector $\mathcal{DS}_{\rho}(p, \ell, r)$, 
  shaded in gray and red with the ellipse $E(\ell, r, \rho)$.
  We approximate the ellipse by a disk centered at $p$ with radius $\tilde{A}_{\rho}(p,\ell,r)$ (thereby ignoring the red area),
  which is the distance from $p$ to the farthest point $q$ of the rectangle 
  $B(\ell, r, \rho)$ in $\mathcal{DS}_{\rho}(p, \ell, r)$.}
  \label{fig:dead-sector-construction}
\end{figure}

We initially attempted to compute fairly precise upper bounds on the approximation distance;
however, due to computational and numerical issues described in \cref{sec:precise-activation-distances},
we decided to use a more robust and efficient approach.
Instead of using the elliptic arc,
we compute an upper bound $\tilde{A}_{\rho}(p,\ell,r)$ using a minimal rectangle $B(\ell, r, \rho)$ containing the ellipse with sides are parallel and perpendicular to $\ell r$.
We only need to check the extreme points of $B(\ell, r, \rho)$ and its intersections with the rays $L(p,\ell,r)$ and $R(p,\ell,r)$ to compute an upper bound $\tilde{A}_{\rho}(p, \ell, r)$; see \cref{fig:dead-sector-construction}.

In the worst case, these simplifications may lead to additional candidate edges. 
While this cannot make the resulting graph exclude any edges that satisfy the ellipse property,
it is still undesirable; we use additional checks for further reductions later on.
%e the
%number of candidate edges later on.

\subsection{Quadtree}
We use a quadtree containing all points in $P$ for the filtered incremental search.
The points are stored in a contiguous array $A_P$ outside the tree.
Each quadtree node $v$ is associated with a contiguous subrange of $A_P$ represented by two pointers.
This subrange contains the points in the subtree $\mathcal{T}_v$ rooted at $v$.
Each node also has a bounding box that contains all points in $\mathcal{T}_v$.
Each interior node has precisely four children; each leaf node contains at most a small constant number of points.
To allow the contiguity of the subranges, points are reordered during tree construction.
This has the added benefit of spatially sorting the points,
improving the probability that geometrically close points are near each other in memory~\cite{haasmwt}.

\subsection{Enumeration process}
One can think of the filtered incremental search from $p$ as a process of continuously growing a disk centered at $p$ starting with a radius of $0$.
As in the sweep-line paradigm, one encounters different types of events at discrete disk radii.
We primarily encounter events when the disk first touches a point of $P$ or the bounding box of a quadtree node.

Observe that, during a search from a point $p$, the dead sectors have two states:
either their activation distance is not yet reached, in which case they are \emph{inactive} and do not exclude any points,
or they are \emph{active} and exclude all points in a certain polar angle interval around $p$.
We therefore also introduce events when the disk radius reaches the activation distance of a dead sector.
This enables efficient management of active dead sectors as a set of polar angle intervals around $p$;
we discuss ensuing numerical issues in \cref{sec:exactness-implementation-issues}.

When we first encounter a point $t$, we have to determine whether $t$ is in any dead sector by checking the active dead sector data structure.
If it is not, we have to report it as potential neighbor of $p$, adding it to a set of points sorted by polar angle around $p$.
Furthermore, to construct new dead sectors, we combine $t$ with $O(1)$ other points of $P$;
which points we use is decided by a heuristic discussed in \cref{sec:dead-sector-construction-neighbors}.

When we encounter a quadtree node $v$, we have to determine whether $v$'s bounding box is fully contained 
in the union of all dead sectors and can thus be pruned; we thus again check the active dead sectors.
Otherwise, we have to take $v$'s children, or the points it contains if it is a leaf, into account;
they are then considered as future events.

\subsection{Initialization and postprocessing}
We initially compute the Delaunay triangulation of the point set $P$ and compute its dilation.
We also optionally attempt to improve the dilation of the triangulation by a simple improvement heuristic.
The heuristic is based on computing constrained Delaunay triangulations,
greedily adding shortcut edges as constraints to reduce the length of the path currently defining the dilation.
We then use the resulting dilation $\rho$ as bound for the enumeration process outlined in the previous sections,
enumerating only edges that could locally be in a triangulation with dilation strictly below $\rho$.

After the initial enumeration process is complete, we are left with a set of \emph{possible} edges and can safely ignore all other edges.
We postprocess these as follows.
For each possible edge $pq$, we compute the set of possible edges intersecting it.
We need this information later on to model the problem of finding a triangulation on the set of possible edges.
For each pair $st$, $\ell r$ of intersecting edges, we explicitly check whether either pair is an exclusion certificate for the other.
In many cases, this postprocessing gets us very close to the edge set that would be obtained by the trivial $\Theta(n^4)$ edge candidate enumeration algorithm;
see the experiment section for details.
We mark each edge that does not have intersecting possible edges as \emph{certain};
certain edges must be part of any triangulation with dilation less than $\rho$.

\subsection{Dilation thresholds}
We also compute a \emph{dilation threshold} $\vartheta(st)$ for each possible edge $st$.
Let $I(st)$ be the set of possible edges intersecting $st$.
For each $\ell r \in I(st)$, we can compute a lower bound on the dilation of the 
shortest path connecting $\ell$ to $r$, should $st$ be present, as \[\rho_{st}(\ell r) = \min \{d(\ell,s) + d(s,r), d(\ell,t) + d(t,r)\}/d(\ell,r);\]
see \cref{fig:dilation-thresholds}.
\begin{figure}
  \centering
  \includegraphics{./figures/edge_thresholds.pdf}
  \caption{If $st$ (black) is present, the pairs $\ell, r$ (resp.\ $\ell',r'$)
           can only be connected by paths that have at least the length of the solid blue (resp.\ orange) path.
           Hence, the ratio between the solid and dashed blue (resp. orange) paths is a lower bound on the dilation of any triangulation containing $st$.
           The maximum of such lower bounds for $st$ is the dilation threshold $\vartheta(st)$ of $st$.}
  \label{fig:dilation-thresholds}
\end{figure}
The dilation threshold of $st$ is then $\vartheta(st) = \max_{\ell r \in I(st)} \rho_{st}(\ell r)$.
\begin{observation}
  If an edge $st$ has dilation threshold $\vartheta(st)$, it is not
  in any triangulation with dilation $\rho < \vartheta(st)$.
\end{observation}
This allows us to quickly exclude further edges if we lower the dilation bound $\rho$ we aim for,
without reenumerating edges or recomputing intersections.

\subsection{Lower bounds}
We use the dilation thresholds to bound the minimum dilation.
For each edge $st$, either $st$ or some edge crossing it must be in any triangulation.
Thus, the minimum of all dilation thresholds among $\{st\} \cup I(st)$ is a lower bound on the minimum dilation.
Combining all edges, we obtain the following lower bound:
\[\rho(T) \geq \max_{st} \min \{\vartheta(pq) \mid pq \in \{st\} \cup I(st) \}.\]
We use interval arithmetic to compute a safe lower bound on this value in $O(1)$ time per intersection between two edges resulting from our enumeration.

Furthermore, by computing the points on the convex hull, we also know the number of edges $g$ of any triangulation.
This also gives us a lower bound on the minimum dilation by considering the $g$ lowest dilation thresholds. 
We also use Kruskal's algorithm to compute the lowest dilation threshold that admits a connected graph.

\section{Exact algorithms}
\newcommand{\binmdt}{\textsc{BinMDT}}%
\newcommand{\incmdt}{\textsc{IncMDT}}%
\label{sec:exact-algorithms}%
Now we present two exact algorithms:
\incmdt{} is an incremental method that uses a SAT solver for
iterative improvement, until it can prove that no better solution exists.
\binmdt{} is based on a binary search for the optimal dilation $\rho$;
once the lower and upper bound are reasonably close,
the approach falls back to \incmdt{} to reach a provably optimal solution.

\subsection{Triangulation supergraph}
Both algorithms rely on the MDT supergraph mentioned in \cref{sec:edge-enumeration}.
As part of this computation, we also obtain an initial triangulation and its dilation,
as well the intersecting possible edges $I(st)$ for each possible edge $st$.
In both algorithms, we may gradually discover triangulations with lower dilation;
these are used to exclude additional edges using the precomputed dilation thresholds $\vartheta(e)$.
To keep track of the status of each edge,
we insert all points and possible edges into a graph data structure we call \emph{triangulation supergraph}.
In this structure, we mark each edge as \emph{possible}, \emph{impossible} or \emph{certain}.
Initially, all enumerated edges are \emph{possible}.
If, at any point, all edges intersecting an edge $e$ become \emph{impossible}, $e$ becomes \emph{certain}.
If an \emph{impossible} edge becomes \emph{certain} or vice versa, the graph does not contain a triangulation any longer.
If this happens, we say we encounter an \emph{edge conflict}.

\subsection{SAT formulation}
Given a triangulation supergraph $G = (P, E)$, we model the problem of finding a triangulation 
on \emph{possible} and \emph{certain} edges using the following simple SAT formulation.
Let $E_p \subseteq E$ be the set of non-\emph{impossible} edges when the SAT formulation is constructed.
For each edge $e \in E_p$, we have a variable $x_e$.
We use the following clauses in our formulation.
\begin{align}
    &\lnot x_{e_1} \lor \lnot x_{e_2} &\forall e_1, e_2 \in E_p: e_2 \in I(e_1) \label{eq:pairwise-intersection}\\
    &x_e \lor \bigvee_{\substack{e_j \in I(e)}} x_{e_j} &\forall e \in E_p\label{eq:enforce-edges}
\end{align}
Clauses (\ref{eq:pairwise-intersection}) ensure crossing-freeness and clauses (\ref{eq:enforce-edges}) ensure maximality.
When an edge $e$ becomes certain, we add the clause $x_e$; similarly, when an edge becomes impossible, we add the clause $\lnot x_e$.
Both algorithms are based on this simple formulation;
in the following, we describe how they use and modify it to find an MDT.

\subsection{Clause generation}
The following subproblem, which we call \emph{dilation path separation}, arises in both our algorithms:
Given a dilation bound $\rho$, a triangulation supergraph $G = (P, E)$ excluding only edges that cannot be in any triangulation with dilation less than $\rho$,
a current triangulation $T$ and a pair of points $s,t \in P$ such that $|\pi_T(s,t)| \geq \rho \cdot d(s,t)$, 
find a clause $C$ that is (a) violated by $T$ and (b) satisfied by any triangulation $T'$ with $\rho(T') < \rho$.

\begin{lemma}
    Assuming a polynomial-time oracle for comparing sums of square roots,
    there is a polynomial-time algorithm that solves the dilation path separation problem.
\end{lemma}
\begin{proof}
    Let $\Pi$ be the set of all $s$-$t$-paths $\pi$ in $G$ with $|\pi| < \rho \cdot d(s,t)$.
    We begin by observing that, along every path $\pi \in \Pi$, there is an edge $e \in E$
    that is not in $T$; otherwise, we get a contradiction to $|\pi_T(s,t)| \geq \rho \cdot d(s,t)$.
    Let $E' \subseteq E \setminus T$ be a set of edges such that for each $\pi \in \Pi$, 
    there is an edge $e \in E'$ on $\pi$.
    Then, $C = \bigvee_{e \in E'} x_e$ is a clause that satisfies the requirements;
    note that if $\Pi$ is empty, the empty clause can be returned.

    $T$ contains no edge from $E'$, so $C$ is violated by $T$.
    Furthermore, if a triangulation $T'$ with $\rho(T') < \rho$ does not contain any of the edges in $E'$,
    it contains none of the paths in $\Pi$.
    Therefore, $\pi_{T'}(s,t)$ uses an edge that is not in $E$, which has been excluded from all triangulations with dilation less than $\rho$; a contradiction.
    $E'$ can be computed by repeatedly computing shortest $s$-$t$-paths $\pi$;
    as long as $\pi < \rho d(s,t)$, we find an edge $e \notin T$ on $\pi$, add $e$ to $E'$ and forbid it in future paths.
    The number of edges bounds the number of iterations of this process;
    using the comparison oracle, we can efficiently perform each iteration.
\end{proof}

For a description of how we compute $E'$ in practice, see \cref{sec:practical-dilation-path-sep}.

\begin{figure}
    \includegraphics[width=\linewidth]{figures/experiments/04_mdt_comparison/iterative_search_progress.pdf}
    \caption{Progress of the incremental algorithm on an instance with $n = 50$ points. Green edges indicate changes in the triangulation, red edges indicate a dilation-defining path.}
    \label{fig:iterative-search-progress}
\end{figure}

\subsection{Incremental algorithm}
\label{sec:incremental-algorithm}
Based on the SAT formulation and the algorithm for the dilation path separation problem, \incmdt{} is simple.
Given an initial triangulation $T$ with dilation $\rho$, we enumerate the set of candidate edges and construct a triangulation supergraph $G$ with bound $\rho$.
We construct the initial SAT formula $M$ and solve it; if it is unsatisfiable, the initial triangulation is optimal.
Otherwise, we repeat the following until the model becomes unsatisfiable or we encounter an edge conflict, keeping track of the best triangulation found, see~\cref{fig:iterative-search-progress}.

We extract the new triangulation $T'$ from the SAT solver and compute the dilation $\rho'$ and a pair $s, t$ of points realizing $\rho'$.
If $\rho'$ is better than the best previously found dilation $\rho$, we update $\rho$ and mark all edges $e$ with $\vartheta(e) \geq \rho'$ as \emph{impossible}.
We then set $T = T'$ and solve the dilation path separation problem for $\rho$, $G$, $T$, $s$ and $t$.
We add the resulting clause to $M$ and let the SAT solver find a new solution.
% TODO: Appendix pseudo code
%\begin{algorithm}[t]
%    \SetKwInOut{Input}{Input}
%    \Input{Initial Triangulation $T$ with dilation $\rho$}
%    Compute the triangulation supergraph from $\rho$\;
%    Initialize SAT model $M$\;
%    \While{true}{
%        \If{$M$ is not satisfiable}{
%            \Return $T$\;
%        }
%        Let $T'$ be a triangulation that satisfies $M$\;
%        Compute $\rho'$ and two vertices $s,t$ with dilation $\rho'$ in $T'$\;
%
%        \If{$\rho' \leq \rho$}{
%            Update the triangulation supergraph with $\rho'$\;
%            $\rho \leftarrow \rho'$\;
%            $T \leftarrow T'$\;
%        }
%
%        \eIf{shorter $st$-path exists in triangulation supergraph}{
%            Generate violating clause and add it to $M$\;
%        }{
%            \Return $T$\;
%        }        
%    }
%    \caption{Pseudo code of the incremental algorithm}
%    \label{alg:incremental-algorithm}
%\end{algorithm}


%We propose an exact incremental SAT-based algorithm for solving the MDP to provable optimality.
%Given the initial solution, it (1) identifies the pair $s,t$ of points that
%define the dilation and then (2) formulates and solves a SAT problem that enforces edges on a shorter $st$-path, see~\cref{alg:incremental-algorithm} for details.
%Steps (1) and (2) can be repeated until no shorter $st$-path exists (i.e. all such paths are excluded based on previous constraints).
%In order to compare two solutions it is necessary to be able to compute the dilation of a given triangulation as if we were using infinite precision.
%It is crucial that this is done efficiently as computing the dilation comes with a significant overhead in runtime. \Cref{sec:exactness-implementation-issues} outlines our implementation.
%The algorithm terminates if the constraints introduced in (2) lead to a infeasible SAT model or if no shorter $st$-path exists in the triangulation supergraph.
%We use a simple SAT formulation with Boolean variables $x_e$ for all edges $e\in E$.
%\michael{We should add some notation for possible/impossible/certain to make this more clear.}

%If an edge $e$ is marked as \emph{certain} or \emph{impossible} in the triangulation supergraph, the variable $x_e$ is set to true or false, respectively.
%\Cref{eq:pairwise-intersection} enforces that no two edges can be selected that intersect each other in an inner point of one of the edges.
%As a triangulation is a maximal planar graph, \cref{eq:enforce-edges} enforces that for any edge $e_i\in E$, either $e_i$ or one intersecting edge is selected.


%Whenever the SAT problem finds a feasible solution $T'$ with objective value $\rho'$, we identify two points $s,t$ with dilation $\rho'$ in $T'$ that are connected by some path of length $\ell$.
%In any solution with dilation $< \rho'$, $s$ and $t$ have to be connected by a shorter path.
%It suffices to identify all edges on possible shorter $st$-paths and enforce that at least one of these edges is in the solution. \Cref{fig:path-enumeration} shows that that it is sometimes possible to reduce the number of edges that need to be enforced by identifying a hitting set on the edges of the possible $st$-paths.
%We propose a heuristic to identify a reasonably small hitting set.
%Our approach uses a bidirectional Dijkstra in the possible and certain edges of the triangulation supergraph to search for all shorter paths between $s$ and $t$.
%Whenever we find a shorter path, at least one edge on the path is not part of the current solution. We remove that edge from the graph and add it to the edge set $E'$.
%This process is repeated until no shorter path exists.
%We then build a set $E'$ of edges that are not in the current triangulation but cut through every shorter $st$-path.
%In any triangulation with a smaller dilation, $s$ and $t$ have to be connected by a shorter path and thus at least one of the edges from $E'$ has to be in the solution which we enforce by adding constraints $\bigvee_{e \in E'} x_e$.
%Note that this constraint remains valid for all subsequent SAT problems as the set of possible edges is reduced in each iteration.
%Once no shorter path exists, or the newly added constraint leads to an infeasible SAT model, the algorithm terminates and returns the current solution.

\subsection{Binary search}
\label{sec:binary-search}
\newcommand{\rhoLB}{\rho_{\text{lb}}}
\newcommand{\rhoUB}{\rho_{\text{ub}}}

% TODO Appendix pseudo code
%\begin{algorithm}[t]
%    \SetKwInOut{Input}{Input}
%    \Input{Initial Triangulation $T$ with dilation $\rho$}
%    Compute the triangulation supergraph from $\rho$\;
%    Compute initial lower bound $\rhoLB$\;
%    \textsc{BinarySearch}($\rho$, $\rhoLB$)\;
%    \caption{Pseudo code of the binary search algorithm}
%    \label{alg:binary-search-algorithm}
%\end{algorithm}
%\begin{algorithm}[t]
%    \SetKwInOut{Input}{Input}
%    \Input{Initial Triangulation $T$ with dilation $\rho$}
%    \While{true}{
%        \If{$M$ is not satisfiable}{
%            \Return $T$\;
%        }
%        Let $T'$ be a triangulation that satisfies $M$\;
%        Compute $\rho'$ and two vertices $s,t$ with dilation $\rho'$ in $T'$\;
%
%        \If{$\rho' \leq \rho$}{
%            Update the triangulation supergraph with $\rho'$\;
%            $\rho \leftarrow \rho'$\;
%            $T \leftarrow T'$\;
%        }
%
%        \eIf{shorter $st$-path exists in triangulation supergraph}{
%            Generate violating clause and add it to $M$\;
%        }{
%            \Return $T$\;
%        }        
%    }
%    \caption{Pseudo code of the binary search phase}
%    \label{alg:binary-search-phase}
%\end{algorithm}
Preliminary experiments with \incmdt{} showed that we spend almost all runtime for computing dilations, 
even for instances for which we could rely exclusively on interval arithmetic, requiring no exact computations.
For many instances, most iterations of \incmdt{} resulted in tiny improvements of the dilation.
To reduce the number of iterations (and thus, dilations computed), 
we considered the binary search-based algorithm \binmdt{}.

\subsubsection{High-level idea}
At any point in time, aside from the dilation $\rhoUB$ of the best known triangulation,
\binmdt{} maintains a lower bound $\rhoLB$ on the dilation, initialized as described in \cref{sec:edge-enumeration}.

As long as $\rhoUB-\rhoLB \geq \sigma$ for a small threshold value $\sigma$, \binmdt{} performs a binary search.
It computes a new dilation bound $\rho = \frac{1}{2}(\rhoLB + \rhoUB)$.
It then uses the SAT model in a similar way as \incmdt{} to determine whether a triangulation $T$ with $\rho(T) < \rho$ exists.
If it does, it updates $\rhoUB = \rho(T)$; otherwise, it updates $\rhoLB = \rho$.

Once $\rhoUB-\rhoLB$ falls below $\sigma$, \binmdt{} falls back to a slightly modified version of \incmdt{} to find the MDT,
starting from the best known triangulation with dilation $\rhoUB$.

In the following, we describe and motivate the differences between how \incmdt{} and \binmdt{} use the SAT formulation;
for more details, see also \cref{sec:incremental-sat-solving}.

\subsubsection{Dilation sampling}
To further reduce the time spent on computing dilations, observe the following.
When a node $v$ of some graph $G$ is expanded in Dijkstra's algorithm from source $s$, we know the shortest path from $s$ to $v$ and thus the dilation $\rho_G(s,v)$.
Because $\rho_G(s,v) \leq \rho(G)$, we can compute a lower bound on the dilation much faster than the precise value by only performing a constant number of node expansions from each point $p \in P$.
We call this \emph{sampling} of the dilation.
Given a bound $\rho$ on the dilation, we can sample a triangulation $T$ for violations, i.e., pairs $s,t$ of points with $\rho_T(s,t) \geq \rho$.
We observed that a dilation-defining path usually consisted of few edges;
thus, we have a good chance of finding it by sampling.

If it is likely that a new-found triangulation $T'$ violates a given bound $\rho$,
we can thus expect to save time by sampling for violations instead of computing the dilation exactly.
Sampling also allows us to use multiple violations to construct multiple clauses in each iteration,
potentially further reducing the number of iterations.
\binmdt{} uses sampling after each SAT call with a small constant limit on the number of violations.
If violations are found, no full dilation computation is required and violations are used to construct clauses.
Only if no violations are found, we compute the exact dilation;
ideally, this only happens once for each upper bound reduction in the binary search,
namely once we find a triangulation satisfying the current bound.
We also sample in the final improvement phase of \binmdt{}.
For an experimental overview on the number of times sampling was sufficient in comparison to the 
number of times the dilation had to be computed exactly, see \cref{sec:experiments-dilation-computation}.

\section{Experimental Evaluation}\label{section:experiments}
We already achieved our primary objective of deriving time-series-specific subsampling guarantees for DP-SGD adapted to forecasting.
Our main goal for this section is to investigate the trade-offs we discovered in discussing these guarantees.
In addition, we train common probabilistic forecasting architectures on standard datasets to verify the feasibility of training deep differentially private forecasting models while retaining meaningful utility.
The full experimental setup  is described in~\cref{appendix:experimental_setup}.
%An implementation will be made available upon publication.

\subsection{Trade-Offs in Structured Subsampling}

\begin{figure}
    \vskip 0.2in
    \centering
        \includegraphics[width=0.99\linewidth]{figures/experiments/eval_pld_deterministic_vs_random_top_level/daily_20_32_main.pdf}
        \vskip -0.3cm
        \caption{Top-level deterministic iteration (\cref{theorem:deterministic_top_level_wr}) vs top-level WOR sampling (\cref{theorem:wor_top_level_wr}) for $\numinstances=1$.
        Sampling is more private despite requiring more compositions.}
        \label{fig:deterministic_vs_random_top_level_daily_main}
    \vskip -0.2in
\end{figure}




For the following experiments, we assume that we have $N=320$ sequences, batch size $\batchsize = 32$, and noise scale $\sigma = 1$.
We further assume $L=10  (L_F + L_C) + L_F - 1$, so that 
the chance of bottom-level sampling a subsequence containing any specific element is 
$r=0.1$ when choosing $\numinstances = 1$ as the number of subsequences.
In~\cref{appendix:extra_experiments_eval_pld}, we repeat all experiments with a wider range of parameters.
All results are consistent with the ones shown here.

\textbf{Number of Subsequences $\bm{\numinstances}$.}
Let us begin with a trade-off inherent to bi-level subsampling:
We can achieve the same batch size $\batchsize$ with different $\numinstances$, each
leading to different top- and bottom-level amplification.
We claim that $\numinstances = 1$ (i.e., maximum bottom-level amplification) is preferable.
For a fair comparison, we compare our provably tight guarantee for $\numinstances=1$ (\cref{theorem:wor_top_level_wr})
with optimistic lower bounds for $\numinstances > 1$ (\cref{theorem:wor_top_wr_bottom_upper})
instead of our sound upper bounds (\cref{theorem:wor_top_level_wr_general}), i.e.,
we make the competitors stronger.
As shown in~\cref{fig:monotonicity_daily_main}, $\numinstances = 1$ only has smaller $\delta(\epsilon)$ for $\epsilon \geq 10^{-1}$ when considering a single training step.
However, after $100$-fold composition, $\numinstances = 1$ achieves smaller $\delta(\epsilon)$ even in $[10^{-3}, 10^{-1}]$ (see~\cref{fig:monotonicity_composed_daily_main}).
Our explanation is that $\numinstances > 1$ results in larger $\delta(\epsilon)$ for large $\epsilon$, i.e., is more likely to have a large privacy loss.
Because the privacy loss of a composed mechanism is the sum of component privacy losses~\cite{sommer2018privacy}, this is problematic when performing multiple training steps.
We shall thus later use $\numinstances=1$ for training.

%Intuitively, $\delta(\epsilon)$ can be interpreted as the probability that the log-likelihood ratio of $M_x$ and $M_{x'}$ (``privacy loss'') exceeds $\epsilon$.\footnote{For the formal relation between privay loss and privacy profiles, see~\cref{lemma:profile_from_pld} taken from~\cite{balle2018improving}}


\textbf{Step- vs Epoch-Level Accounting.}
Next, we show the benefit of top-level sampling sequences (\cref{theorem:wor_top_level_wr}) instead of deterministically iterating over them (\cref{theorem:deterministic_top_level_wr}), even though we risk privacy leakage at every training step.
For our parameterization and $\numinstances=1$, top-level sampling with replacement requires $10$ compositions per epoch.
As shown in~\cref{fig:deterministic_vs_random_top_level_daily_main}, the resultant epoch-level profile is nevertheless smaller, and remains so after $10$ epochs.
This is consistent with any work on DP-SGD (e.g., \cite{abadi2016deep}) that uses subsampling instead of deterministic iteration.

\textbf{Epoch Privacy vs Length.} In~\cref{appendix:extra_experiments_epoch_length} we additionally explore the fact that, if we wanted to use deterministic top-level iteration, 
the number of subsequences 
$\numinstances$ would affect epoch length.
As expected, we observe that composing many private mechanisms ($\numinstances=1$) is preferable to composing few much less private mechanisms ($\numinstances > 1$) 
when considering a fixed number of training steps.

\begin{figure}
    \vskip 0.2in
    \centering
        \includegraphics[width=0.99\linewidth]{figures/experiments/eval_pld_label_noise/daily_30_32_main.pdf}
        \vskip -0.3cm
        \caption{Varying label noise $\sigma_F$ for top-level WOR and bottom-level WR  (\cref{theorem:data_augmentation_general}) with $\sigma_C = 0, \numinstances=1$.
        Increasing $\sigma_F$ is equivalent to decreasing forecast length.
        }
        \label{fig:label_noise_daily_main}
    \vskip -0.2in
\end{figure}

\textbf{Amplification by Label Perturbation.}
Finally, because the way in which adding Gaussian noise to the context and/or forecast window 
amplifies privacy (\cref{theorem:data_augmentation_general}) 
may be somewhat opaque, let us consider top-level sampling without replacement, bottom-level sampling with replacement,
$\numinstances=1$, $\sigma_C=0$, and varying label noise standard deviations $\sigma_F$. 
As shown in~\cref{fig:label_noise_daily_main}, increasing $\sigma_F$ has the same effect as letting the forecast length $L_C$ go to zero, i.e., eliminates the risk of leaking private information if it appears in the forecast window.
Of course, this data augmentation 
will have an effect on model utility, which we investigate shortly.

\begin{figure*}
\centering
\vskip 0.2in
    \begin{subfigure}{0.49\textwidth}
        \includegraphics[]{figures/experiments/eval_pld_monotonicity_composed/daily_20_32_1_main.pdf}
        \caption{Training step $1$}\label{fig:monotonicity_daily_main}
    \end{subfigure}
    \hfill
    \begin{subfigure}{0.49\textwidth}
        \includegraphics[]{figures/experiments/eval_pld_monotonicity_composed/daily_20_32_100_main.pdf}
        \caption{Training step $100$}\label{fig:monotonicity_composed_daily_main}
    \end{subfigure}\caption{
    Top-level WOR and bottom-level WR sampling under varying number of subsequences.
    Under composition, even optimistic lower bounds (\cref{theorem:wor_top_wr_bottom_upper}) 
    indicate worse privacy for $\numinstances > 1$ than 
    our tight upper bound for $\numinstances=1$ (\cref{theorem:wor_top_level_wr}).}
    \label{fig:monotonicity_daily_main_container}
\vskip -0.2in
\end{figure*}


\subsection{Application to Probabilistic Forecasting}
While the contribution of our work lies in formally analyzing the privacy of DP-SGD adapted to forecasting, 
training models with this algorithm can serve as a sanity-check to verify that the guarantees are sufficiently strong to retain meaningful utility under non-trivial privacy budgets.


\begin{table}[b]
\vskip -0.38cm
\caption{Average CRPS on \texttt{traffic} for $\delta=10^{-7}$. Seasonal, AutoETS, and models with $\epsilon=\infty$ are without noise.}
\label{table:1_event_training_traffic_main}
\vskip 0.18cm
\begin{center}
\begin{small}
\begin{sc}
\begin{tabular}{lcccc}
\toprule
Model & $\epsilon = 0.5$ & $\epsilon = 1$ & $\epsilon = 2$ &  $\epsilon = \infty$ \\
\midrule
SimpleFF & $0.207$ & $0.195$ & $0.193$ & $0.136$ \\ 
DeepAR & $\mathbf{0.157}$ & $\mathbf{0.145}$ & $\mathbf{0.142}$ & $\mathbf{0.124}$ \\
iTransf. & $0.211$ & $0.193$ & $0.188$ & $0.135$ \\
DLinear & $0.204$ & $0.192$ & $0.188$ & $0.140$ \\
\midrule
Seasonal   & $0.251$ & $0.251$ & $0.251$ & $0.251$\\
AutoETS   & $0.407$ & $0.407$ & $0.407$ & $0.407$\\
\bottomrule
\end{tabular}
\end{sc}
\end{small}
\end{center}
\vskip -0.1in
\end{table}

\textbf{Datasets, Models, and Metrics.}
We consider three standard benchmarks: \texttt{traffic}, \texttt{electricity}, and \texttt{solar\_10\_minutes} as used in~\cite{Lai2018modeling}.
We further consider four common architectures: 
A two-layer feed-forward neural network (``SimpleFeedForward''), a recurrent neural network (``DeepAR''~\cite{salinas2020deepar}),
an encoder-only transformer (``iTransformer''~\cite{liu2024itransformer}), and a refined feed-forward network proposed to compete with attention-based models (``DLinear''~\cite{zeng2023transformers}).
We let these architectures parameterize elementwise $t$-distributions to obtain probabilistic forecasts.
We measure the quality of these probabilistic forecasts using continuous ranked probability scores (CRPS), which we approximate via mean weighted quantile losses (details in~\cref{appendix:metrics}).
As a reference for what constitutes ``meaningful utility'', we compare against seasonal na\"{i}ve forecasting and exponential smoothing (``AutoETS'') without introducing any noise.
All experiments are repeated with $5$ random seeds.


\textbf{Event-Level Privacy.} \cref{table:1_event_training_traffic_main} shows CRPS of all models on the \texttt{traffic} test set 
when setting $\delta=10^{-7}$, and training on the training set until reaching a pre-specified $\epsilon$
with $1$-event-level privacy. For the other datasets and standard deviations, see~\cref{appendix:privacy_utility_tradeoff_event_level_privacy}.
The column $\epsilon=\infty$ indicates non-DP training.
As can be seen, models can retain much of their utility and outperform the baselines, even for $\epsilon \leq 1$ which is generally considered a small privacy budget~\cite{ponomareva2023dp}.
For instance, the average CRPS of DeepAR on the traffic dataset is $0.124$ with non-DP training and $0.157$ for $\epsilon=0.5$.
Note that, since all models are trained using  our tight privacy analysis,
which specific model performs best  on which specific dataset is orthogonal to our contribution. 

\textbf{Other results.}
In~\cref{appendix:privacy_utility_tradeoff_user_level_privacy} we additionally train with $w$-event and $w$-user privacy.
In~\cref{appendix:privacy_utility_tradeoff_label_privacy}, we demonstrate that label perturbations can offer an improved privacy--utility trade-off. 
All results confirm that our guarantees for DP-SGD adapted to forecasting are strong enough to enable provably private training while retaining utility.


\section{Conclusion}
\label{sec:Conclusion}
In this paper, we proposed a complete real-time planning and control approach for continuous, reliable, and fast online generation of dynamically feasible Bernstein trajectories and control for FW aircrafts. The generated trajectories span kilometers, navigating through multiple waypoints. By leveraging differential flatness equations for coordinated flight, we ensure precise trajectory tracking. Our approach guarantees smooth transitions from simulation to real-world applications, enabling timely field deployment. The system also features a user-friendly mission planning interface. Continuous replanning  maintains the rajectory curvature 
$\kappa$ within limits, preventing abrupt roll changes.

Future works will include the ability to add  a higher-level kinodynamic path planner to optimize waypoint spatial allocation and improve replanning success, and enhancing the trajectory-tracking algorithm by refining the aerodynamic coefficient estimation. 


%%
%% Bibliography
%%

%% Please use bibtex, 

\bibliography{main.bib}

\appendix

%%%%%%%%%%%%%%%%%%%%%%%%%%%%%%%%%%%%%%%%%%%%%%%%%%%%%%%%%%%%%%%%%%%%%%%%%%%%%%%
%%%%%%%%%%%%%%%%%%%%%%%%%%%%%%%%%%%%%%%%%%%%%%%%%%%%%%%%%%%%%%%%%%%%%%%%%%%%%%%
% APPENDIX
%%%%%%%%%%%%%%%%%%%%%%%%%%%%%%%%%%%%%%%%%%%%%%%%%%%%%%%%%%%%%%%%%%%%%%%%%%%%%%%
%%%%%%%%%%%%%%%%%%%%%%%%%%%%%%%%%%%%%%%%%%%%%%%%%%%%%%%%%%%%%%%%%%%%%%%%%%%%%%%
\newpage
\appendix
\onecolumn

\section{Related Work} \label{app:related_work}
\textbf{Personalized Generation} 
Due to the considerable success of large text-to-image models \cite{ramesh2022hierarchical, ramesh2021zero, saharia2022photorealistic, rombach2022high}, the field of personalized generation has been actively developed. The challenge is to customize a text-to-image model to generate specific concepts that are specified using several input images. Many different approaches \cite{DB, TI, CD, svdiff, ortogonal, profusion, elite, r1e} have been proposed to solve this problem and can be divided into the following groups: pseudo-token optimization \cite{TI, profusion, disenbooth, r1e}, diffusion fune-tuning \cite{DB, CD, profusion}, and encoder-based \cite{elite}. The pseudo-token paradigm adjusts the text encoder to convert the concept token into the proper embedding for the diffusion model. Such embedding can be optimized directly \cite{TI, r1e} or can be generated by other neural networks \cite{disenbooth, profusion}. Such approaches usually require a small number of parameters to optimize but lose the visual features of the target concept. Diffusion fine-tuning-based methods optimize almost all \cite{DB} or parts \cite{CD} of the model to reconstruct the training images of the concept. This allows the model to learn the input concept with high accuracy, but the model due to overfitting may lose the ability to edit it when generated with different text prompts. To reduce overfitting and memory usage, lightweight parameterizations \cite{svdiff, r1e, lora} have been proposed that preserve edibility but at the cost of degrading concept fidelity. Encoder-based methods \cite{elite} allow one forward pass of an encoder that has been trained on a large dataset of many different objects to embed the input concept. This dramatically speeds up the process of learning a new concept and such a model is highly editable, but the quality of recovering concept details may be low. Generally, the main problem with existing personalized generation approaches is that they struggle to simultaneously recover a concept with high quality and generate it in a variety of scenes.

\textbf{Sampling strategies}
Much research has been devoted to sampling techniques for text-to-image diffusion models, focusing not only on personalized generation but also on image editing. In this paper, we address a more specific question: how can the two trajectories -- superclass and concept -- be optimally combined to achieve both high concept fidelity and high editability? The ProFusion paper \cite{profusion} considered one way of combining these trajectories (Mixed sampling), which we analyze in detail in our paper (see Section \ref{sec:mixed_sampling}) and show its properties and problems. In ProFusion, authors additionally proposed a more complex sampling procedure, which we observed to be redundant compared to Mixed sampling, as can be seen in our experiments (see Section \ref{sec:experiments}). In Photoswap \cite{photoswap}, authors consider another way of combining trajectories by superclass and concept, which turns out to be almost identical to the Switching sampling strategy that we discuss in detail in Section \ref{sec:switching_sampling}. We show why this strategy fails to achieve simultaneous improvements in concept reconstruction and editability. In the paper, we propose a more efficient way of combining these two trajectories that achieves an optimal balance between the two key features of personalized generation: concept reconstruction and editability.

\section{Training details} \label{sec:training-details}
The Stable Diffusion-2-base model is used for all experiments. For the Dreambooth, Custom Diffusion, and Textual Inversion methods, we used the implementation from \url{https://github.com/huggingface/diffusers}.

\textbf{SVDiff} We implement the method based on \url{https://github.com/mkshing/svdiff-pytorch}. The parameterization is applied to all Text Encoder and U-Net layers. The models for all concepts were trained for $1600$ using Adam optimizer with $\text{batch size} = 1$, $\text{learning rate} = 0.001$, $\text{learning rate 1d} = 0.000001$, $\text{betas} = (0.9, 0.999)$, $\text{epsilon} = 1e\!-\!8$, and $\text{weight decay} = 0.01$. 

\textbf{Dreambooth} All query, key, and value layers in Text Encoder and U-Net were trained during fine-tuning. The models for all concepts were trained for $400$ steps using Adam optimizer with $\text{batch size} = 1$, $\text{learning rate} = 2e\!-\!5$, $\text{betas} = (0.9, 0.999)$, $\text{epsilon} = 1e\!-\!8$, and $\text{weight decay} = 0.01$. 

\textbf{Custom Diffusion} The models for all concepts were trained for $1600$ steps using Adam optimizer with $\text{batch size} = 1$, $\text{learning rate} = 0.00001$, $\text{betas} = (0.9, 0.999)$, $\text{epsilon} = 1e\!-\!8$, and $\text{weight decay} = 0.01$. 

\textbf{Textual Inversion} The models for all concepts were trained for $10000$ steps using Adam optimizer with $\text{batch size} = 1$, $\text{learning rate} = 0.005$, $\text{betas} = (0.9, 0.999)$, $\text{epsilon} = 1e\!-\!8$, and $\text{weight decay} = 0.01$. 

\textbf{ELITE} We used the pre-trained model from the official repo \url{https://github.com/csyxwei/ELITE} with $\lambda=0.6$ and inference hyperparams from the original paper.

\clearpage
\section{Superclass and concept trajectory choice}\label{app:hyper_theta}

 \begin{wrapfigure}{r}{0.45\textwidth}
    % \centering
    \includegraphics[trim={3cm 10cm 3cm 10cm},clip,width=\linewidth]{imgs/mixed_noft_nosup.pdf}
    \caption{The Pareto frontiers for original Mixed sampling and Mixed sampling in the Superclass, NoFT, and Empty Prompt setups. Mixed NoFT and Mixed Empty Prompt configurations overlap with the Pareto frontier of the original mixed sampling, but primarily in regions associated with low image similarity, which compromises concept fidelity.} \label{fig:mixed_noft_ep}
    \vspace{-0.14in}
\end{wrapfigure}

There are multiple ways to define sampling with maximized textual alignment to the prompt. However, the arbitrary choice can harm the alignment between Base sampling~\ref{eq:concept_sampling} and the selected trajectory. We use the Sampling with superclass (\ref{eq:superclass_sampling}) as it's the default choice in the literature and guarantees the maximized alignment between noise predictions $\tilde{\varepsilon}_{\theta}(p^C)$ and $\tilde{\varepsilon}_{\theta}(p^S)$. 

The several natural ways to adjust Sampling with superclass can be presented by varying $\theta$ and $p^{S}$ in (\ref{eq:superclass_sampling}). We explore two additional options with decreased alignment with (\ref{eq:concept_sampling}): (1) NoFT -- weights of base model $\theta^{\text{orig}}$ instead fine-tuned weights, (2) Empty Prompt -- prompt without any reference to a concept, even to its superclass category, i.e. $p^{\hat{S}} = \textit{"with a city in the background"}$ instead of $p^{S} = \textit{"a backpack with a city in the background"}$.

To validate the robustness of our framework for sampling method selection, we employ the original experimental protocol, supplementing the results shown in Figures~\ref{fig:examples} and~\ref{fig:profusion-photoswap}. Our analysis of Figures~\ref{fig:multi-stage_noft_ep} and~\ref{fig:masked_noft_ep} reveals that trajectories generated under the NoFT and Empty Prompt configurations (second and third columns, respectively) maintain identical method ordering to those produced by Superclass sampling ((\ref{eq:superclass_sampling}), first column).

Notably, Figure~\ref{fig:mixed_noft_ep} shows that Empty Prompt configuration demonstrates weaker alignment with Base sampling compared to NoFT, particularly at higher values of the superclass guidance scale $\omega_{s}$. This divergence manifests as reduced concept fidelity for Empty Prompt under large $\omega_{s}$. These findings highlight a practical adjustment: prioritizing smaller $\omega_{s}$ values in Empty Prompt setup preserves concept fidelity without altering the framework’s core selection logic. 

A key limitation of increased misalignment is the gradual erosion of superclass category information from generated images, which can lead to semantically inconsistent outputs. For instance, Figure~\ref{fig:examples_noft_ep} illustrates how the Mixed Empty Prompt setup, despite the strong animal prior in Base sampling, can produce human-like features in an image of a cat described as \textit{"in a chef outfit"}. This suggests that when superclass information is weakened, the model may introduce unexpected visual artifacts, impacting the fidelity of the intended concept.

Concept sampling (\ref{eq:concept_sampling}) can also be adjusted to better capture a concept’s visual characteristics, further decoupling fidelity from editability. For example, this can be achieved by (1) using the weights of a highly overfitted model (e.g., DreamBooth) or (2) selecting a prompt that omits contextual details, such as $p^{\hat{C}} = \textit{"a photo of V*"}$ instead of $p^{C} = \textit{"a V* with a city in the background"}$. Combining superclass sampling under NoFT or Empty Prompt with Base sampling configured via (1) or (2) could enhance both image and text similarity. We leave this direction for future work.

\begin{figure}[b]
    \centering
    \includegraphics[trim={3cm 10cm 3cm 10cm},clip,width=0.32\linewidth]{imgs/multi-stage_original.pdf}
    \hfill
    \includegraphics[trim={3cm 10cm 3cm 10cm},clip,width=0.32\linewidth]{imgs/multi-stage_noft.pdf}
    \hfill
    \includegraphics[trim={3cm 10cm 3cm 10cm},clip,width=0.32\linewidth]{imgs/multi-stage_nosup.pdf}
    \caption{Pareto Frontier Curves for Mixed, Switching, and Multi-Stage Sampling Methods in the Superclass, NoFT and Empty Prompt setups.
The NoFT and Empty Prompt configurations (second and third columns, respectively) preserve the same method ordering as those produced by Superclass sampling (first column).} \label{fig:multi-stage_noft_ep}
\end{figure}
\begin{figure}[t]
    \centering
    \includegraphics[trim={3cm 10cm 3cm 10cm},clip,width=0.32\linewidth]{imgs/masked_profusion.pdf}
    \hfill
    \includegraphics[trim={3cm 10cm 3cm 10cm},clip,width=0.32\linewidth]{imgs/masked_noft.pdf}
    \hfill
    \includegraphics[trim={3cm 10cm 3cm 10cm},clip,width=0.32\linewidth]{imgs/masked_nosup.pdf}
    \caption{Pareto Frontier Curves for Mixed, Switching, Masked, and ProFusion Sampling Methods in the Superclass, NoFT, and Empty Prompt setups.
The NoFT and Empty Prompt configurations (second and third columns, respectively) preserve the same method ordering as those produced by Superclass sampling (first column).} \label{fig:masked_noft_ep}
\end{figure}

\begin{figure}[b]
    \centering
    \includegraphics[width=\linewidth]{imgs/examples_noft_ep.pdf}
    \caption{Examples of the generation outputs for Mixed and ProFusion sampling methods for their optimal metrics point in the Superclass, NoFT, and Empty Prompt (EP) setups.} \label{fig:examples_noft_ep}
\end{figure}

\clearpage

\begin{figure}[ht!]
  \centering
  \includegraphics[trim={0 5cm 0 5cm},clip,width=0.95\linewidth]{imgs/us_example_new.pdf}
  \caption{An example of a task in the user study}
  \label{fig:us_ex}
  \vspace{-0.19in}
\end{figure}

\section{Data preparation}\label{app:data}
For each concept, we used inpainting augmentations to create the training dataset. We took an original image and automatically segmented it using the Segment Anything model on top of the CLIP cross-attention maps. Then we crop the concept from the original image, apply affine transformations to it, and inpaint the background. We used $10$ augmentation prompts, different from the evaluation prompts, and sampled $3$ images per prompt, resulting in a total of $30$ training images per concept. We commit to open-source the augmented datasets for each concept after publication.

\section{User Study}\label{app:us}

An example task from the user study is shown in Figure~\ref{fig:us_ex}. In total, we collected 48,864 responses from 200 unique users for 16,000 unique pairs. For each task, users were asked three questions: 1) "Which image is more consistent with the text prompt?" 2) "Which image better represents the original image?" 3) "Which image is generally better in terms of alignment with the prompt and concept identity preservation?" For each question, users selected one of three responses: "1", "2", or "Can't decide."

\section{Complex Prompts Setting}\label{app:long_prompts}

We conduct a comparison of different sampling methods using a set of complex prompts. For this analysis, we collected 10 prompts, each featuring multiple scene changes simultaneously, including stylization, background, and outfit:

\adjustbox{max width=\linewidth}{
\begin{lstlisting}
live_long = [
  "V* in a chief outfit in a nostalgic kitchen filled with vintage furniture and scattered biscuit",
  "V* sitting on a windowsill in Tokyo at dusk, illuminated by neon city lights, using neon color palette",
  "a vintage-style illustration of a V* sitting on a cobblestone street in Paris during a rainy evening, showcasing muted tones and soft grays",
  "an anime drawing of a V* dressed in a superhero cape, soaring through the skies above a bustling city during a sunset",
  "a cartoonish illustration of a V* dressed as a ballerina performing on a stage in the spotlight",
  "oil painting of a V* in Seattle during a snowy full moon night",
  "a digital painting of a V* in a wizard's robe in a magical forest at midnight, accented with purples and sparkling silver tones",
  "a drawing of a V* wearing a space helmet, floating among stars in a cosmic landscape during a starry night",
  "a V* in a detective outfit in a foggy London street during a rainy evening, using muted grays and blues",
  "a V* wearing a pirate hat exploring a sandy beach at the sunset with a boat floating in the background",
]

object_long = [
  "a digital illustration of a V* on a windowsill in Tokyo at dusk, illuminated by neon city lights, using neon color palette",
  "a sketch of a V* on a sofa in a cozy living room, rendered in warm tones",
  "a watercolor painting of a V* on a wooden table in a sunny backyard, surrounded by flowers and butterflies",
  "a V* floating in a bathtub filled with bubbles and illuminated by the warm glow of evening sunlight filtering through a nearby window",
  "a charcoal sketch of a giant V* surrounded by floating clouds during a starry night, where the moonlight creates an ethereal glow",
  "oil painting of a V* in Seattle during a snowy full moon night",
  "a drawing of a V* floating among stars in a cosmic landscape during a starry night with a spacecraft in the background",
  "a V* on a sandy beach next to the sand castle at the sunset with a floaing boat in the background",
  "an anime drawing V* on top of a white rug in the forest with a small wooden house in the background",
  "a vintage-style illustration of a V* on a cobblestone street in Paris during a rainy evening, showcasing muted tones and soft grays",
]
\end{lstlisting}
}

The results of this comparison are presented in Figures~\ref{fig:add_long},~\ref{fig:add_long_metrics}. We observe that Base sampling may struggle to preserve all the features specified by the prompts, whereas advanced sampling techniques effectively restore them. The overall arrangement of methods in the metric space closely mirrors that observed in the setting with simple prompts.

\begin{figure}[h!]
  \centering
  \includegraphics[width=\linewidth]{imgs/long_prompts_examples.pdf}
  \caption{Additional examples of the generation outputs for different sampling methods with \textbf{complex prompts}. We highlight parts of the prompt that are missing in Base sampling while appearing in other methods.}
  \label{fig:add_long}
\end{figure}

\clearpage
\section{Dreambooth results}\label{app:dreambooth}

We conduct additional analysis of different sampling methods in combination with Dreambooth. Figure~\ref{fig:add_db_metrics} shows that Mixed Sampling still overperforms Switching and Photoswap,  while Multi-stage and Masked struggle to provide an additional improvement over the simple baseline. Figure~\ref{fig:add_db} shows that all methods allow for improvement TS with a negligent decrease in IS while Mixed Sampling provides the best IS among all samplings.

\begin{figure*}[!ht]
\centering
\begin{minipage}{.477\textwidth}
  \centering
  \includegraphics[trim={3cm 10cm 3cm 10cm},clip,width=\linewidth]{imgs/long_prompts.pdf}
  \caption{CLIP metrics for different sampling methods estimated on \textbf{complex prompts}.}
  \label{fig:add_long_metrics}
\end{minipage}
\hfill
\begin{minipage}{.477\textwidth}
  \centering
  \includegraphics[trim={3cm 10cm 3cm 10cm},clip,width=\linewidth]{imgs/db_samplings.pdf}
  \caption{CLIP metrics for different sampling strategies on top of a Dreambooth fine-tuning method.}
  \label{fig:add_db_metrics}
\end{minipage}
\end{figure*} 

\begin{figure}[h!]
  \centering
  \includegraphics[trim={0 1cm 0 1cm},clip,width=\linewidth]{imgs/db_sampling_examples.pdf}
  \caption{Additional examples of the generation outputs for different sampling methods on top of a Dreambooth fine-tuning method.}
  \label{fig:add_db}
\end{figure}

\section{PixArt-alpha \& SD-XL}\label{app:add_backbones}
We conducted a series of experiments using different backbones. For SD-XL~\cite{podell2023sdxlimprovinglatentdiffusion}, we used SVDDiff as the fine-tuning method, while PixArt-alpha~\citep{chen2023pixartalphafasttrainingdiffusion} employed standard Dreambooth training. Hyperparameters for Switching, Masked, and ProFusion were selected in the same manner as in the experiments with SD2.

Figures~\ref{fig:pixart} and~\ref{fig:sdxl} demonstrate that Mixed Sampling follows a similar pattern to SD2, improving TS without a significant loss in IS. Notably, Mixed Sampling for SD-XL achieves simultaneous improvements in both IS and TS. ProFusion exhibits behavior consistent with SD2, enhancing IS more effectively than Mixed Sampling but performing worse at improving TS while also requiring twice the computational resources. 

\begin{figure}[h]
\centering
\begin{minipage}{.49\textwidth}
  \centering
  \includegraphics[trim={3cm 10cm 3cm 10cm},clip,width=\linewidth]{imgs/pixart.pdf}
  \captionof{figure}{CLIP metrics for different sampling methods estimated on PixArt model.}
  \label{fig:pixart}
\end{minipage}%
\hfill
\begin{minipage}{.49\textwidth}
  \centering
  \includegraphics[trim={3cm 10cm 3cm 10cm},clip,width=\linewidth]{imgs/sdxl.pdf}
  \captionof{figure}{CLIP metrics for different sampling methods estimated on SD-XL model.}
  \label{fig:sdxl}
\end{minipage}
\end{figure}

\clearpage
\section{Cross-Attention Masks}\label{app:cross_attn}

\begin{figure}[h!]
  \centering
  \includegraphics[trim={3cm 0cm 3cm 0cm},clip,width=\linewidth]{imgs/cross_attention_masks.pdf}
  \caption{Visualization of the cross-attention masks for Masked sampling examples. Here, $q$ defines the thresholding quantile and $t$ the denoising step.}
  \label{fig:cross_attn_add_ex}
\end{figure}

\clearpage
\section{Additional Examples}\label{app:add_example}

\begin{figure}[h!]
  \centering
  \includegraphics[trim={0 2cm 0 2cm},clip,width=\linewidth]{imgs/additional_examples.pdf}
  \caption{Additional examples of the generation outputs for different sampling methods.}
  \label{fig:add_ex}
\end{figure}

\begin{figure}[h!]
  \centering
  \includegraphics[trim={0 2cm 0 2cm},clip,width=\linewidth]{imgs/additional_examples_all.pdf}
  \caption{Additional examples of the generation outputs for Mixed and ProFusion sampling methods in comparison to the main personalized generation baselines.}
  \label{fig:add_ex_all}
\end{figure}

\clearpage
\section{DINO Image Similarity}\label{app:add_dino}

We compare CLIP-IS (left column) and DINO-IS~\citep{oquab2024dinov2learningrobustvisual} (right column) in Figures~\ref{fig:profusion_photoswap_dino},~\ref{fig:all_methods_dino}. We observe that despite the choice of metric, different sampling techniques and finetuning strategies have the same arrangement. The most noticeable difference is that SVDDiff superiority over ELITE and TI is more pronounced. That strengthens our motivation to select SVDDiff as the main backbone.

\begin{figure}[h]
\centering
\begin{minipage}{.49\textwidth}
  \centering
  \includegraphics[trim={3cm 10cm 3cm 10cm},clip,width=\linewidth]{imgs/profusion_photoswap.pdf}
\end{minipage}%
\hfill
\begin{minipage}{.49\textwidth}
  \centering
  \includegraphics[trim={3cm 10cm 3cm 10cm},clip,width=\linewidth]{imgs/profusion_photoswap_dino.pdf}
\end{minipage}
\caption{Pareto frontiers curves for Photoswap~\citep{photoswap} and ProFusion~\citep{profusion}.}\label{fig:profusion_photoswap_dino}
\end{figure}

\begin{figure}[h]
\centering
\begin{minipage}{.49\textwidth}
  \centering
  \includegraphics[trim={3cm 10cm 3cm 10cm},clip,width=\linewidth]{imgs/all_methods.pdf}
\end{minipage}%
\hfill
\begin{minipage}{.49\textwidth}
  \centering
  \includegraphics[trim={3cm 10cm 3cm 10cm},clip,width=\linewidth]{imgs/all_methods_dino.pdf}
\end{minipage}
\caption{The overall results of different sampling methods against main personalized generation baselines.}\label{fig:all_methods_dino}
\end{figure}


\end{document}
