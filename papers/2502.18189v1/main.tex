\documentclass[nolineno,a4paper,USenglish,cleveref,autoref,thm-restate]{socg-lipics-v2021}
%This is a template for producing LIPIcs articles. 
%See lipics-v2021-authors-guidelines.pdf for further information.
%for A4 paper format use option "a4paper", for US-letter use option "letterpaper"
%for british hyphenation rules use option "UKenglish", for american hyphenation rules use option "USenglish"
%for section-numbered lemmas etc., use "numberwithinsect"
%for enabling cleveref support, use "cleveref"
%for enabling autoref support, use "autoref"
%for anonymousing the authors (e.g. for double-blind review), add "anonymous"
%for enabling thm-restate support, use "thm-restate"
%for enabling a two-column layout for the author/affilation part (only applicable for > 6 authors), use "authorcolumns"
%for producing a PDF according the PDF/A standard, add "pdfa"

\pdfoutput=1 %uncomment to ensure pdflatex processing (mandatatory e.g. to submit to arXiv)
\hideLIPIcs  %uncomment to remove references to LIPIcs series (logo, DOI, ...), e.g. when preparing a pre-final version to be uploaded to arXiv or another public repository

%\graphicspath{{./graphics/}}%helpful if your graphic files are in another directory

\clubpenalty=10000
\widowpenalty=10000

\usepackage[group-separator={,},output-decimal-marker={.}]{siunitx}
\usepackage{nicefrac}
\usepackage{complexity}
\usepackage{microtype}
\let\R\undefinedcommand
\usepackage[]{algorithm2e}
\usepackage{longtable}
\usepackage[commandnameprefix=always,final]{changes}

\crefname{equation}{Equation}{Equations}

\bibliographystyle{plainurl}% the mandatory bibstyle

\title{Exact Algorithms for Minimum Dilation Triangulation}

%\titlerunning{Dummy short title} %optional, use if title is longer than one line

%\author{Jane {Open Access}}{Dummy University Computing Laboratory, [optional: Address], Country \and My second affiliation, Country \and \url{http://www.myhomepage.edu} }{johnqpublic@dummyuni.org}{https://orcid.org/0000-0002-1825-0097}{(Optional) author-specific funding acknowledgements}%mandatory, please use full name; only 1 author per \author macro; first two parameters are mandatory, other parameters can be empty. Please provide at least the name of the affiliation and the country. The full address is optional. Use additional curly braces to indicate the correct name splitting when the last name consists of multiple name parts.

\author{Sándor P. Fekete}{Department of Computer Science, TU Braunschweig}{fekete@tu-braunschweig.de}{https://orcid.org/0000-0002-9062-4241}{}
\author{Phillip Keldenich}{Department of Computer Science, TU Braunschweig}{keldenich@ibr.cs.tu-bs.de}{https://orcid.org/0000-0002-6677-5090}{}
\author{Michael Perk}{Department of Computer Science, TU Braunschweig}{fekete@tu-braunschweig.de}{https://orcid.org/0000-0002-0141-8594}{}
\authorrunning{S.\,P.\ Fekete and P.\ Keldenich and M.\ Perk} %First: Use abbreviated first/middle names. Second (only in severe cases): Use first author plus 'et al.'

\Copyright{Sándor P. Fekete and Phillip Keldenich and Michael Perk} %mandatory, please use full first names. LIPIcs license is "CC-BY";  http://creativecommons.org/licenses/by/3.0/

\begin{CCSXML}
<ccs2012>
<concept>
<concept_id>10003752.10010061.10010063</concept_id>
<concept_desc>Theory of computation~Computational geometry</concept_desc>
<concept_significance>500</concept_significance>
</concept>
<concept>
<concept_id>10003752.10003809.10003716.10011136</concept_id>
<concept_desc>Theory of computation~Discrete optimization</concept_desc>
<concept_significance>500</concept_significance>
</concept>
<concept>
<concept_id>10002944.10011123.10011131</concept_id>
<concept_desc>General and reference~Experimentation</concept_desc>
<concept_significance>300</concept_significance>
</concept>
<concept>
<concept_id>10002950.10003714.10003715.10003725</concept_id>
<concept_desc>Mathematics of computing~Interval arithmetic</concept_desc>
<concept_significance>300</concept_significance>
</concept>
<concept>
<concept_id>10002950.10003714.10003715.10003726</concept_id>
<concept_desc>Mathematics of computing~Arbitrary-precision arithmetic</concept_desc>
<concept_significance>100</concept_significance>
</concept>
</ccs2012>
\end{CCSXML}

\ccsdesc[500]{Theory of computation~Computational geometry}
\ccsdesc[500]{Theory of computation~Discrete optimization}
\ccsdesc[300]{General and reference~Experimentation}
\ccsdesc[300]{Mathematics of computing~Interval arithmetic}
\ccsdesc[100]{Mathematics of computing~Arbitrary-precision arithmetic}
\keywords{dilation, minimum dilation triangulation, exact algorithms, algorithm engineering, experimental evaluation}

\category{} %optional, e.g. invited paper

%\relatedversion{https://arxiv.org/abs/TODO.XXXXX} %TODO full arxiv version
%\relatedversiondetails[linktext={opt. text shown instead of the URL}, cite=DBLP:books/mk/GrayR93]{Classification (e.g. Full Version, Extended Version, Previous Version}{URL to related version} %linktext and cite are optional


%\supplement{}%optional, e.g. related research data, source code, ... hosted on a repository like zenodo, figshare, GitHub, ...
%\supplementdetails[linktext={opt. text shown instead of the URL}, cite=DBLP:books/mk/GrayR93, subcategory={Description, Subcategory}, swhid={Software Heritage Identifier}]{General Classification (e.g. Software, Dataset, Model, ...)}{URL to related version} %linktext, cite, and subcategory are optional

\funding{
  The work presented in this paper was largely funded by the DFG grant 
  ``Computational Geometry: Solving Hard Optimization Problems'' (CG:SHOP), grant FE407/21-1.} 

\acknowledgements{}%optional

%\nolinenumbers %uncomment to disable line numbering
% we have 500 lines!

%\supplement{}
%\supplementdetails[linktext={opt. text shown instead of the URL}, cite=DBLP:books/mk/GrayR93, subcategory={Description, Subcategory}, swhid={Software Heritage Identifier}]{General Classification (e.g. Software, Dataset, Model, ...)}{URL to related version} %linktext, cite, and subcategory are optional

%Editor-only macros:: begin (do not touch as author)%%%%%%%%%%%%%%%%%%%%%%%%%%%%%%%%%%

\EventEditors{Oswin Aichholzer and Haitao Wang}
\EventNoEds{2}
\EventLongTitle{41st International Symposium on Computational Geometry (SoCG 2025)}
\EventShortTitle{SoCG 2025}
\EventAcronym{SoCG}
\EventYear{2025}
\EventDate{June 23--27, 2025}
\EventLocation{Kanazawa, Japan}
\EventLogo{socg-logo.pdf}
\SeriesVolume{332}
\ArticleNo{174}     % <-- This will be filled in by the typesetters
%%%%%%%%%%%%%%%%%%%%%%%%%%%%%%%%%%%%%%%%%%%%%%%%%%%%%%

\newcommand{\R}{\mathbb{R}}


\newcommand{\lbEpsilon}{2.730751 \cdot 10^{-16}}
\newcommand{\lbDelta}{6.458762 \cdot 10^{-16}}
\newcommand{\lbN}{84}
\newcommand{\lbRho}{1.44116645381}
\newcommand{\lbRhoShort}{1.44116}
\begin{document}

\supplement{Code, experiment instances and results are archived on Zenodo.}
\supplementdetails[subcategory={Source Code}]{Software}{https://doi.org/10.5281/zenodo.14266122}
\supplementdetails[subcategory={Experiment Data}]{Dataset}{https://doi.org/10.5281/zenodo.14266122}

\maketitle

\begin{abstract}
We provide a spectrum of new theoretical insights and practical results 
for finding 
a Minimum Dilation Triangulation (MDT), a natural geometric optimization 
problem of considerable previous attention:
Given a set $P$ of $n$ points in the plane, find a triangulation
$T$, such that a shortest Euclidean path in $T$ between any pair of points
increases by the smallest possible factor compared to their
straight-line distance. No polynomial-time algorithm is known for the problem;
moreover, evaluating the objective function involves computing the sum
of (possibly many) square roots. 
On the other hand, the problem is not known to be \NP-hard.

(1) We provide practically robust methods and implementations for computing an MDT
for benchmark sets with up to 30,000 points in reasonable time on commodity
hardware, based on new geometric insights into the structure of optimal edge
sets. Previous methods only achieved results for up to $200$ points, so we extend 
the range of optimally solvable instances by a factor of $150$.

(2) We develop scalable techniques for accurately
evaluating many shortest-path queries that arise as large-scale sums of square
roots, allowing us to certify exact optimal solutions,  
with previous work relying on (possibly inaccurate) floating-point computations.

(3) We resolve an open problem by establishing a lower bound of
$\lbRhoShort$ on the dilation of the regular $\lbN$-gon (and thus for arbitrary point
sets), improving the previous worst-case lower bound of $1.4308$ 
and greatly reducing the remaining gap to the upper bound of
$1.4482$ from the literature. In the process, we provide optimal solutions for regular
$n$-gons up to $n = 100$.
\end{abstract}

\begin{figure}[ht]
    \centering
    \includegraphics[width=0.8\linewidth]{graphs/greater_than_naive.pdf}
    \vspace{0.5cm}
    \includegraphics[width=0.8\linewidth]{graphs/p1_bottom.png}
    \vspace{-5pt}
    \caption{\textcolor{positional}{Positional} vs.\ \textcolor{nonpositional}{non-positional} circuits. In a \textcolor{nonpositional}{non-positional} circuit, the same edges must be included at all positions. A \textcolor{positional}{positional} circuit can distinguish between the same edge at different positions. This specificity yields better trade-offs between circuit size and faithfulness. It can also increase both precision and recall.}
    \label{fig:p1}
    \vspace{-5pt}
\end{figure}

\section{Introduction}

\looseness=-1
A primary goal of interpretability research is to characterize the internal mechanisms in language models (LMs) and other NLP models. 
A core approach in this area is \textbf{circuit discovery}---identifying the minimal subgraph within the model's computation graph that performs a specific task \citep{olah2021framework,olah-mech}.
Typically, the nodes of a circuit represent model components (e.g., attention heads, neurons, or layers).
While manual circuit discovery methods can yield position-specific insights \citep{wanginterpretability,goldowskydill2023localizingmodelbehaviorpath}, \emph{automatic methods often overlook positional information}, treating components as uniformly relevant across all input token positions \citep{conmytowards,syed2023attribution}. 
For instance, if an attention head is included in a circuit, it is assumed to contribute equally to the computation for every position in the input sequence.
The assumption that circuits are position-invariant ignores the fact that different positions often require distinct computations.
By ignoring positions, current methods limit their ability to capture mechanisms that operate across positions, such as interactions between attention heads across positions.

In this study, we start by demonstrating that positional agnosticism is a significant limitation (\S\ref{sec:motivating}). Then, to address these limitations, we introduce a new approach: position-aware edge attribution patching (PEAP; \S\ref{sec:full_circ_discovery}; Figure~\ref{fig:p1}). Current approaches  assume that if an edge is in a circuit, then the same edge will be in the circuit at all positions, thus leading to low precision. It is also assumed that an edge's importance should be aggregated across positions before deciding whether it should be included in the circuit; this can lead to cancellation effects, and thus low recall. PEAP instead allows us to compute the importance of cross-positional edges, and separately evaluates edge importance at each position. We show that this leads to smaller and more accurate circuits; see Figure~\ref{fig:p1}.

Incorporating positional information into circuit discovery is straightforward when inputs have the same length and structure across examples.

However, realistic datasets are not nearly this templatic.
How, then, can we incorporate positional information into automatic circuit discovery?
To address this challenge, we propose \textbf{schemas} (\S\ref{sec:schema}). 
Schemas assign semantic labels to spans of tokens, enabling information aggregation across examples even when the spans differ in length.

For example, in the input ``The \textcolor{positional}{war} lasted from 1453 to 14\underline{\hspace{1em}},'' the span ``\textcolor{positional}{war}'' could be labeled as ``\emph{Subject}''.
This enables handling spans with varying lengths: the phrase ``\textcolor{positional}{Black Plague}'' in another example can be treated as a single positional span with the same role as ``\textcolor{positional}{war}''.
In experiments with two LMs and three tasks, we find that circuits discovered using schemas achieve a better trade-off between circuit size and faithfulness to the model's behavior than position-agnostic circuits.
Importantly, position-aware circuits offer a more precise representation of the underlying mechanisms, providing a more concise foundation for mechanistic explanations.

We also present a fully automated pipeline for schema generation and application (\S\ref{sec:schema-generation}) using large language models (LLMs). 
We evaluate the quality of the generated schemas and their utility in discovering position-aware circuits (\S\ref{sec:schema-eval}).
Notably, circuits derived using automatically generated and applied schemas achieve comparable faithfulness scores to circuits discovered with human-designed and manually applied schemas.

We summarize our contributions as follows:
\begin{itemize}[noitemsep,leftmargin=*,topsep=1pt,parsep=1pt]
    \item Introduce a position-aware circuit discovery method, which obtains better faithfulness than position-agnostic discovery.  
    \item Introduce dataset schemas,  facilitating positional circuit discovery in more naturalistic settings. 
    \item Develop an automated schema generation and application pipeline with LLMs, yielding schemas that are comparable to manually-annotated ones.
\end{itemize}

\section{Preliminaries}\label{sec:problem_formulation}

% \begin{table*}[h!]
% \centering
% \caption{Comparison of Algorithms}
% \label{tab:algorithm_comparison}
% \begin{tabular}{l|ccccc}
% \toprule
% \textbf{Algorithm Name} & \textbf{Performance} & \textbf{Diversity} & \textbf{Generalization} & \textbf{Efficiency} & \textbf{Streaming}\\
% \midrule
% Greedy-decoding & Moderate  &   Low  & High          & High      & Yes \\
% Decoding with Temperature & Low & High & High              & High & Yes   \\
% Top-K Sampling & Moderate      & High & High               & High   & Yes     \\
% Top-P Sampling & Moderate     & High & High              & High   & Yes      \\
% Beam-Search & Moderate     & Moderate & High              & Moderate    & Yes     \\
% Majority Voting & Moderate     & High & Low              & Moderate     & No    \\
% RM Selection & Moderate     & High & High              & Moderate    & No     \\
% RM-guided Tree Search & High     & High & Low              & Low    & No    \\
% \bottomrule
% \end{tabular}
% \end{table*}
% The language model generation process generally selects a sequence of tokens following certain algorithms (e.g., greedy or sampling methods) until a stopping criterion, such as an end-of-sequence token or maximum sequence length, is reached.

In this section, we first introduce how the LM generation process can be formulated as a token-level Markov Decision Process (MDP) and then explain how existing sampling algorithms relate to it.

\subsection{LLM Decoding as Token-level MDP}

The language model generation process takes a sequence of tokens as inputs and generates a sequence of tokens as outputs.
Mainstream transformer-based language models generate the output tokens one by one until the stopping criteria (e.g., an end-of-sequence token or maximum sequence length) are met.
The traditional MDP is usually formulated as a tuple $\mathcal{M} = (\mathcal{S}, \mathcal{A}, F, R, \gamma)$, where $\mathcal{S}$ is the set of all possible states, $\mathcal{A}$ is the set of actions, $F$ is the transition function, $R$ is the reward function, and $\gamma$ is the decay parameter.
In the language model scenarios, each state in $\mathcal{S}$ is a trajectory that can be denoted as $\tau$.
Each action in $\mathcal{A}$ is selecting a token $x$ from the vocabulary set.
$F$ is the deterministic transition of concatenating the selected action (i.e., a token) with the existing state (i.e., a trajectory) to become a new one.
Traditionally, rewards $r_t$ are defined at every step $t$ and contribute to the return $G_t=\sum_{k=0}^{\infty} \gamma^k R_{t+k}$ through the decay factor $\gamma$.
However, in the LLM scenario, we are only concerned with the quality of the complete trajectory generated, meaning that the reward function $R(s)$ evaluates the final trajectory rather than providing step-by-step feedback.
Thus, the return becomes $G=R_T$, where $R_T$ is the reward associated with the final sequence at step $T$, and $\gamma$ is irrelevant because intermediate rewards are not accumulated. 

\subsection{The Classical Decoding Algorithms}

Formally, given an input \( \mathbf{x}  = (x_1, x_2, \ldots, x_T) \), a reward function $R$ that provides a scalar reward for a trajectory $\tau$, and a language model \( p_\theta(x) \) parameterized by \( \theta \), the goal of decoding algorithms is to find the optimal trajectory $x^\star$ sampled from \( p_\theta(x) \) that could maximize the reward:

\[
\tau^* = \arg\max_{\tau \sim p_\theta(\mathbf{x} )} R(\tau).
\]

% Assumes that the input is a token sequence with $a$ and the foundation language model is a probabilistic model of predicting the likelihood of the next token given previous ones, which is usually formulated as $P(x \mid x_{<i})$, the goal of language model decoding is to utilize this likelihood to get the output trajectory of length $T$ that achieves the highest $R_T$.
This section covers representative decoding algorithms and explains how they are connected.

\textbf{Greedy Decoding}: The naive but most widely used algorithm is \textit{Greedy Decoding}, which uses the language modeling likelihood at each step as guidance.
At each step $i$, this algorithm selects the action token $x_i \in \mathcal{A}$ following:
\begin{equation}
x_i = \arg\max_{x} P(x \mid x_{<i}).
\end{equation}
From the angle of MDP, this method uses the accumulative likelihood predicted by the language as the final reward:
\begin{equation}
 R(\tau) \gets \Pi_{i}^{T}P(x_i \mid x_{<i}),
\end{equation}
where $T$ is the length of $\tau$, and takes a greedy solution to approach this goal.


\textbf{Sampling-based Decoding:} 
On top of greedy decoding, people also try to incorporate diversity in the final output.
For example, the temperature-based method introduces an additional parameter $\lambda$ to control the greedy sampling process by reshaping the likelihood distribution as:
\begin{equation}
x_i \sim P(x \mid x_{<i})^{1/\lambda}.
\end{equation}
From the angle of token-level MDP, we can reinterpret this process as introducing an additional diversity objective:

\begin{equation}
 R(\tau) \gets \Pi_{i}^{T}P(x_i \mid x_{<i}) \cdot D(x_i, x_{<i}),
\end{equation}
where 
\begin{equation}
    D(x_i , x_{<i}) = P(x_i \mid x_{<i})^{\frac{1}{\lambda}-1}.
\end{equation}


To avoid sampling rare tokens and achieve a balance between performance and diversity, researchers have investigated how to dynamically adjust the candidate token pool~\citep{holtzman2019curious,zarriess2021decoding}. 
For example, the \textit{Top-$k$ Sampling} algorithm only considers the top $k$ tokens with the highest probabilities as candidates instead of the whole vocabulary.
Similarly, the \textit{Nucleus Sampling}, which is also known as \textit{Top-p Sampling}, only selects from the smallest possible set $\mathcal{V}_p \subseteq \mathcal{V}$, where the cumulative probability mass exceeds a threshold $p$.

% \paragraph{Nucleus Sampling:} This strategy samples tokens from the smallest possible set $\mathcal{V}_p \subseteq \mathcal{V}$, where the cumulative probability mass exceeds a threshold $p$. Formally,
% \begin{equation}
%     \mathcal{V}_p = \{x_i \mid \sum_{x_j \in \mathcal{V}_p} p(x_j \mid \mathbf{x}_{<t}) \geq p\}.
% \end{equation}

\textbf{Trajectory-level Decoding:} Although these token-level decoding algorithms are efficient, they tend to generate locally coherent outputs that may lack global quality.
To solve this problem, people also developed decoding algorithms that consider partial or whole trajectories.
For example, the \textit{Beam Search Decoding} algorithm keeps track of the top $B$ partial trajectories, expanding them at each step and retaining only the ones with the highest joint likelihood.
Similar to the \textit{Greedy Decoding}, this method also uses the joint likelihood as the trajectory reward function.

\textbf{Advanced Reward Modeling Algorithms:}
A common limitation of the aforementioned algorithms is their fundamental assumption that the joint likelihood could represent $R(\tau)$ might not always hold.
People have been interested in introducing better reward signals as guidance to address this.
For example, in the QA scenario, the \textit{Majority-voting algorithm} assumes that the more frequent answer aligns better with the grounding reward function (i.e., accuracy) and thus selects candidate trajectories following this guidance.
Though this intuitive approach has been shown to be effective on tasks such as QA and math problems, it is restricted to tasks with structured output for voting. It cannot be generalized to more general-purpose applications.
To address this issue, researchers also include an external model $R^\prime$, which is often another transformer-based model, to approximate the ground truth reward model $R$. 
With that, we could sample $K$ trajectories $\mathcal{T}_K$ with sampling-based decoding algorithms and then use $R$ to select the trajectory with the maximum reward:
\begin{equation}
    \tau^\star = \arg \max_{\tau \in \mathcal{T}} R^\prime(\tau).
\end{equation}
Employing an external model to model the reward offers greater flexibility than heuristic rewards. This approach is not constrained by the structured answer format, which improves generality and adaptability in various scenarios. 








\section{Candidate edge enumeration}
\label{sec:edge-enumeration}

Here we describe a novel and practically efficient scheme for
enumerating a set of edges that induces a supergraph of the MDT.
We start with the underlying theoretical ideas for this supergraph, followed
by an algorithm for computing a supergraph of any triangulation with dilation \emph{strictly less} than a given bound $\rho$,
which we exploit to enumerate a supergraph of the MDT with a (usually) small number of edges.
This is further adapted to reductions of the dilation bound $\rho$.
We also discuss the computation of lower bounds on the dilation of the MDT.
We defer the discussion of some implementation details to \cref{sec:implementation-details}.

\subsection{Theoretical background}
Our supergraph is based on the well-known \emph{ellipse property} (used in \cite{DBLP:conf/cccg/BrandtGSR14,DBLP:journals/ijcga/GiannopoulosKKKM10,DBLP:conf/ewcg/KnauerM05}) that all edges of a triangulation $T$ with dilation below $\rho$ must satisfy.

\begin{definition}
  A pair of points $s, t$ has the \emph{ellipse property} with 
  respect to a point set $P$ and a dilation bound $\rho$ if, for
  any pair of points $\ell, r \in P \setminus \{s,t\}$ such that
  $\ell r$ and $st$ intersect, $\min \{d(\ell,s) + d(s,r), d(\ell,t) + d(t,r)\} < \rho d(\ell,r)$. 
\end{definition}

Recall that the set of points that have the same sum of distances $\kappa$ to 
two points $\ell, r$ with $\kappa \geq d(\ell,r)$ is an \emph{ellipse} with $\ell, r$ as its focal points (or \emph{foci});
thus, all paths between $\ell$ and $r$ with length less than $\kappa = \rho \cdot d(\ell,r)$ must lie strictly inside this ellipse.
%with foci $\ell, r$ and \emph{focal distance sum} $\kappa$.

If a pair of points $s,t$ does not have the ellipse property, then there is a pair of points $\ell, r$
such that $st$ cuts through all paths between $\ell$ and $r$ that could have dilation less than $\rho$; see \cref{fig:ellipse-property-example}.
We call such a pair of points $\ell, r$ an \emph{exclusion certificate} for $s,t$.
\begin{figure}%
  \centering
  \includegraphics{./figures/ellipse-property-example.pdf}
  \caption{
    Any path connecting $\ell$ and $r$ with length below $\kappa = \rho d(\ell,r)$ must lie within the ellipse (dashed black lines).
    The edge $st$ does not have the \emph{ellipse property} as neither $s$ nor $t$ lie inside the ellipse;
    therefore, inserting $st$ makes connecting $\ell$ and $r$ by a sufficiently short path impossible.
  }
  \label{fig:ellipse-property-example}
\end{figure}%

\begin{observation}
  Let $\rho \geq 1$ be some dilation bound and let $T$ be a triangulation containing the edge $st$
  for points $s,t$ that do not have the ellipse property w.r.t.\ $P$ and $\rho$.
  Then, $\rho(T) \geq \rho$.
\end{observation}

\subsection{High-level description}
\label{sec:ellipse-filter-algo}
Preliminary experiments showed that a brute-force check of each of the $\Theta(n^4)$ pairs of
potential edges is impractical for large $n$, dominating the overall runtime time even for
early versions of our solution approach.

We therefore developed a more efficient scheme to enumerate a superset of the edges that satisfy the ellipse property.
This scheme performs a \emph{filtered incremental nearest-neighbor search} from each point $p \in P$
to identify all candidate edges of the form $pt$, i.e., looking for all possible \emph{neighbors} $t$ of $p$.
This search is efficiently implemented on a quadtree containing all points.
While enumerating candidates, we construct so-called \emph{dead sectors}, i.e., regions of the plane that
cannot contain possible neighbors of $p$.
We exclude all points that lie in dead sectors; we also use dead sectors to prune entire nodes of the quadtree
and to terminate the search early if it has become clear that all remaining points must be in a dead sector.
This usually avoids considering most points as potential neighbors of $p$ individually.
The algorithm has a runtime of $O(nk\log n)$, where $k$ is the average number of points and quadtree vertices considered individually from each point.
At worst, this can be $O(n^2\log n)$; in practice, $k$ is often much lower than $n$.
A related, slightly less complex enumeration algorithm applied to minimum-weight triangulations by Haas~\cite{haasmwt} scales to $10^8$ points.
In the following, we describe the components of the enumeration scheme in more detail.

\subsection{Dead sectors}
\label{sec:dead-sectors}
We begin by giving a definition of the dead sectors we use.
\begin{definition}
  Given a dilation $\rho$, a source point $p$ and two points $\ell, r$,
  the \emph{dead sector} $\mathcal{DS}_{\rho}(p, \ell, r) \subset \mathbb{R}^2$ is the region of all points $t$ such that 
  $pt$ intersects $\ell r$ and neither $p$ nor $t$ lie in the ellipse with foci $\ell, r$ and focal distance sum 
  $\kappa = \rho d(\ell, r)$.
\end{definition}
Depending on $p$, $\rho$, $\ell$ and $r$, $\mathcal{DS}_{\rho}(p, \ell, r)$ is either empty (if $p$ is in the ellipse)
or it is bounded by two rays and an elliptic arc; see \cref{fig:dead-sector-construction}.
In that case, it can also be interpreted as an elliptic arc and an interval of polar angles around $p$.

During our enumeration we construct many dead sectors,
the union of which can become quite complex, making it cumbersome and inefficient to work with directly.
We instead chose to simplify the shape of our dead sectors, giving up 
a small fraction of excluded area in exchange for a simple and efficient representation.
We replace the elliptic arc by a single disk centered on $p$,
whose radius is at least the maximum distance from $p$ to any point on the elliptic arc.
We can hence represent each non-empty simplified dead sector by a polar angle interval around $p$ 
and a single radius called \emph{activation distance} $A_{\rho}(p,\ell,r)$; see \cref{fig:dead-sector-construction}.

\begin{figure}
  \centering
  \includegraphics{./figures/dead_sector.pdf}
  \caption{A dead sector $\mathcal{DS}_{\rho}(p, \ell, r)$, 
  shaded in gray and red with the ellipse $E(\ell, r, \rho)$.
  We approximate the ellipse by a disk centered at $p$ with radius $\tilde{A}_{\rho}(p,\ell,r)$ (thereby ignoring the red area),
  which is the distance from $p$ to the farthest point $q$ of the rectangle 
  $B(\ell, r, \rho)$ in $\mathcal{DS}_{\rho}(p, \ell, r)$.}
  \label{fig:dead-sector-construction}
\end{figure}

We initially attempted to compute fairly precise upper bounds on the approximation distance;
however, due to computational and numerical issues described in \cref{sec:precise-activation-distances},
we decided to use a more robust and efficient approach.
Instead of using the elliptic arc,
we compute an upper bound $\tilde{A}_{\rho}(p,\ell,r)$ using a minimal rectangle $B(\ell, r, \rho)$ containing the ellipse with sides are parallel and perpendicular to $\ell r$.
We only need to check the extreme points of $B(\ell, r, \rho)$ and its intersections with the rays $L(p,\ell,r)$ and $R(p,\ell,r)$ to compute an upper bound $\tilde{A}_{\rho}(p, \ell, r)$; see \cref{fig:dead-sector-construction}.

In the worst case, these simplifications may lead to additional candidate edges. 
While this cannot make the resulting graph exclude any edges that satisfy the ellipse property,
it is still undesirable; we use additional checks for further reductions later on.
%e the
%number of candidate edges later on.

\subsection{Quadtree}
We use a quadtree containing all points in $P$ for the filtered incremental search.
The points are stored in a contiguous array $A_P$ outside the tree.
Each quadtree node $v$ is associated with a contiguous subrange of $A_P$ represented by two pointers.
This subrange contains the points in the subtree $\mathcal{T}_v$ rooted at $v$.
Each node also has a bounding box that contains all points in $\mathcal{T}_v$.
Each interior node has precisely four children; each leaf node contains at most a small constant number of points.
To allow the contiguity of the subranges, points are reordered during tree construction.
This has the added benefit of spatially sorting the points,
improving the probability that geometrically close points are near each other in memory~\cite{haasmwt}.

\subsection{Enumeration process}
One can think of the filtered incremental search from $p$ as a process of continuously growing a disk centered at $p$ starting with a radius of $0$.
As in the sweep-line paradigm, one encounters different types of events at discrete disk radii.
We primarily encounter events when the disk first touches a point of $P$ or the bounding box of a quadtree node.

Observe that, during a search from a point $p$, the dead sectors have two states:
either their activation distance is not yet reached, in which case they are \emph{inactive} and do not exclude any points,
or they are \emph{active} and exclude all points in a certain polar angle interval around $p$.
We therefore also introduce events when the disk radius reaches the activation distance of a dead sector.
This enables efficient management of active dead sectors as a set of polar angle intervals around $p$;
we discuss ensuing numerical issues in \cref{sec:exactness-implementation-issues}.

When we first encounter a point $t$, we have to determine whether $t$ is in any dead sector by checking the active dead sector data structure.
If it is not, we have to report it as potential neighbor of $p$, adding it to a set of points sorted by polar angle around $p$.
Furthermore, to construct new dead sectors, we combine $t$ with $O(1)$ other points of $P$;
which points we use is decided by a heuristic discussed in \cref{sec:dead-sector-construction-neighbors}.

When we encounter a quadtree node $v$, we have to determine whether $v$'s bounding box is fully contained 
in the union of all dead sectors and can thus be pruned; we thus again check the active dead sectors.
Otherwise, we have to take $v$'s children, or the points it contains if it is a leaf, into account;
they are then considered as future events.

\subsection{Initialization and postprocessing}
We initially compute the Delaunay triangulation of the point set $P$ and compute its dilation.
We also optionally attempt to improve the dilation of the triangulation by a simple improvement heuristic.
The heuristic is based on computing constrained Delaunay triangulations,
greedily adding shortcut edges as constraints to reduce the length of the path currently defining the dilation.
We then use the resulting dilation $\rho$ as bound for the enumeration process outlined in the previous sections,
enumerating only edges that could locally be in a triangulation with dilation strictly below $\rho$.

After the initial enumeration process is complete, we are left with a set of \emph{possible} edges and can safely ignore all other edges.
We postprocess these as follows.
For each possible edge $pq$, we compute the set of possible edges intersecting it.
We need this information later on to model the problem of finding a triangulation on the set of possible edges.
For each pair $st$, $\ell r$ of intersecting edges, we explicitly check whether either pair is an exclusion certificate for the other.
In many cases, this postprocessing gets us very close to the edge set that would be obtained by the trivial $\Theta(n^4)$ edge candidate enumeration algorithm;
see the experiment section for details.
We mark each edge that does not have intersecting possible edges as \emph{certain};
certain edges must be part of any triangulation with dilation less than $\rho$.

\subsection{Dilation thresholds}
We also compute a \emph{dilation threshold} $\vartheta(st)$ for each possible edge $st$.
Let $I(st)$ be the set of possible edges intersecting $st$.
For each $\ell r \in I(st)$, we can compute a lower bound on the dilation of the 
shortest path connecting $\ell$ to $r$, should $st$ be present, as \[\rho_{st}(\ell r) = \min \{d(\ell,s) + d(s,r), d(\ell,t) + d(t,r)\}/d(\ell,r);\]
see \cref{fig:dilation-thresholds}.
\begin{figure}
  \centering
  \includegraphics{./figures/edge_thresholds.pdf}
  \caption{If $st$ (black) is present, the pairs $\ell, r$ (resp.\ $\ell',r'$)
           can only be connected by paths that have at least the length of the solid blue (resp.\ orange) path.
           Hence, the ratio between the solid and dashed blue (resp. orange) paths is a lower bound on the dilation of any triangulation containing $st$.
           The maximum of such lower bounds for $st$ is the dilation threshold $\vartheta(st)$ of $st$.}
  \label{fig:dilation-thresholds}
\end{figure}
The dilation threshold of $st$ is then $\vartheta(st) = \max_{\ell r \in I(st)} \rho_{st}(\ell r)$.
\begin{observation}
  If an edge $st$ has dilation threshold $\vartheta(st)$, it is not
  in any triangulation with dilation $\rho < \vartheta(st)$.
\end{observation}
This allows us to quickly exclude further edges if we lower the dilation bound $\rho$ we aim for,
without reenumerating edges or recomputing intersections.

\subsection{Lower bounds}
We use the dilation thresholds to bound the minimum dilation.
For each edge $st$, either $st$ or some edge crossing it must be in any triangulation.
Thus, the minimum of all dilation thresholds among $\{st\} \cup I(st)$ is a lower bound on the minimum dilation.
Combining all edges, we obtain the following lower bound:
\[\rho(T) \geq \max_{st} \min \{\vartheta(pq) \mid pq \in \{st\} \cup I(st) \}.\]
We use interval arithmetic to compute a safe lower bound on this value in $O(1)$ time per intersection between two edges resulting from our enumeration.

Furthermore, by computing the points on the convex hull, we also know the number of edges $g$ of any triangulation.
This also gives us a lower bound on the minimum dilation by considering the $g$ lowest dilation thresholds. 
We also use Kruskal's algorithm to compute the lowest dilation threshold that admits a connected graph.

\section{Exact algorithms}
\newcommand{\binmdt}{\textsc{BinMDT}}%
\newcommand{\incmdt}{\textsc{IncMDT}}%
\label{sec:exact-algorithms}%
Now we present two exact algorithms:
\incmdt{} is an incremental method that uses a SAT solver for
iterative improvement, until it can prove that no better solution exists.
\binmdt{} is based on a binary search for the optimal dilation $\rho$;
once the lower and upper bound are reasonably close,
the approach falls back to \incmdt{} to reach a provably optimal solution.

\subsection{Triangulation supergraph}
Both algorithms rely on the MDT supergraph mentioned in \cref{sec:edge-enumeration}.
As part of this computation, we also obtain an initial triangulation and its dilation,
as well the intersecting possible edges $I(st)$ for each possible edge $st$.
In both algorithms, we may gradually discover triangulations with lower dilation;
these are used to exclude additional edges using the precomputed dilation thresholds $\vartheta(e)$.
To keep track of the status of each edge,
we insert all points and possible edges into a graph data structure we call \emph{triangulation supergraph}.
In this structure, we mark each edge as \emph{possible}, \emph{impossible} or \emph{certain}.
Initially, all enumerated edges are \emph{possible}.
If, at any point, all edges intersecting an edge $e$ become \emph{impossible}, $e$ becomes \emph{certain}.
If an \emph{impossible} edge becomes \emph{certain} or vice versa, the graph does not contain a triangulation any longer.
If this happens, we say we encounter an \emph{edge conflict}.

\subsection{SAT formulation}
Given a triangulation supergraph $G = (P, E)$, we model the problem of finding a triangulation 
on \emph{possible} and \emph{certain} edges using the following simple SAT formulation.
Let $E_p \subseteq E$ be the set of non-\emph{impossible} edges when the SAT formulation is constructed.
For each edge $e \in E_p$, we have a variable $x_e$.
We use the following clauses in our formulation.
\begin{align}
    &\lnot x_{e_1} \lor \lnot x_{e_2} &\forall e_1, e_2 \in E_p: e_2 \in I(e_1) \label{eq:pairwise-intersection}\\
    &x_e \lor \bigvee_{\substack{e_j \in I(e)}} x_{e_j} &\forall e \in E_p\label{eq:enforce-edges}
\end{align}
Clauses (\ref{eq:pairwise-intersection}) ensure crossing-freeness and clauses (\ref{eq:enforce-edges}) ensure maximality.
When an edge $e$ becomes certain, we add the clause $x_e$; similarly, when an edge becomes impossible, we add the clause $\lnot x_e$.
Both algorithms are based on this simple formulation;
in the following, we describe how they use and modify it to find an MDT.

\subsection{Clause generation}
The following subproblem, which we call \emph{dilation path separation}, arises in both our algorithms:
Given a dilation bound $\rho$, a triangulation supergraph $G = (P, E)$ excluding only edges that cannot be in any triangulation with dilation less than $\rho$,
a current triangulation $T$ and a pair of points $s,t \in P$ such that $|\pi_T(s,t)| \geq \rho \cdot d(s,t)$, 
find a clause $C$ that is (a) violated by $T$ and (b) satisfied by any triangulation $T'$ with $\rho(T') < \rho$.

\begin{lemma}
    Assuming a polynomial-time oracle for comparing sums of square roots,
    there is a polynomial-time algorithm that solves the dilation path separation problem.
\end{lemma}
\begin{proof}
    Let $\Pi$ be the set of all $s$-$t$-paths $\pi$ in $G$ with $|\pi| < \rho \cdot d(s,t)$.
    We begin by observing that, along every path $\pi \in \Pi$, there is an edge $e \in E$
    that is not in $T$; otherwise, we get a contradiction to $|\pi_T(s,t)| \geq \rho \cdot d(s,t)$.
    Let $E' \subseteq E \setminus T$ be a set of edges such that for each $\pi \in \Pi$, 
    there is an edge $e \in E'$ on $\pi$.
    Then, $C = \bigvee_{e \in E'} x_e$ is a clause that satisfies the requirements;
    note that if $\Pi$ is empty, the empty clause can be returned.

    $T$ contains no edge from $E'$, so $C$ is violated by $T$.
    Furthermore, if a triangulation $T'$ with $\rho(T') < \rho$ does not contain any of the edges in $E'$,
    it contains none of the paths in $\Pi$.
    Therefore, $\pi_{T'}(s,t)$ uses an edge that is not in $E$, which has been excluded from all triangulations with dilation less than $\rho$; a contradiction.
    $E'$ can be computed by repeatedly computing shortest $s$-$t$-paths $\pi$;
    as long as $\pi < \rho d(s,t)$, we find an edge $e \notin T$ on $\pi$, add $e$ to $E'$ and forbid it in future paths.
    The number of edges bounds the number of iterations of this process;
    using the comparison oracle, we can efficiently perform each iteration.
\end{proof}

For a description of how we compute $E'$ in practice, see \cref{sec:practical-dilation-path-sep}.

\begin{figure}
    \includegraphics[width=\linewidth]{figures/experiments/04_mdt_comparison/iterative_search_progress.pdf}
    \caption{Progress of the incremental algorithm on an instance with $n = 50$ points. Green edges indicate changes in the triangulation, red edges indicate a dilation-defining path.}
    \label{fig:iterative-search-progress}
\end{figure}

\subsection{Incremental algorithm}
\label{sec:incremental-algorithm}
Based on the SAT formulation and the algorithm for the dilation path separation problem, \incmdt{} is simple.
Given an initial triangulation $T$ with dilation $\rho$, we enumerate the set of candidate edges and construct a triangulation supergraph $G$ with bound $\rho$.
We construct the initial SAT formula $M$ and solve it; if it is unsatisfiable, the initial triangulation is optimal.
Otherwise, we repeat the following until the model becomes unsatisfiable or we encounter an edge conflict, keeping track of the best triangulation found, see~\cref{fig:iterative-search-progress}.

We extract the new triangulation $T'$ from the SAT solver and compute the dilation $\rho'$ and a pair $s, t$ of points realizing $\rho'$.
If $\rho'$ is better than the best previously found dilation $\rho$, we update $\rho$ and mark all edges $e$ with $\vartheta(e) \geq \rho'$ as \emph{impossible}.
We then set $T = T'$ and solve the dilation path separation problem for $\rho$, $G$, $T$, $s$ and $t$.
We add the resulting clause to $M$ and let the SAT solver find a new solution.
% TODO: Appendix pseudo code
%\begin{algorithm}[t]
%    \SetKwInOut{Input}{Input}
%    \Input{Initial Triangulation $T$ with dilation $\rho$}
%    Compute the triangulation supergraph from $\rho$\;
%    Initialize SAT model $M$\;
%    \While{true}{
%        \If{$M$ is not satisfiable}{
%            \Return $T$\;
%        }
%        Let $T'$ be a triangulation that satisfies $M$\;
%        Compute $\rho'$ and two vertices $s,t$ with dilation $\rho'$ in $T'$\;
%
%        \If{$\rho' \leq \rho$}{
%            Update the triangulation supergraph with $\rho'$\;
%            $\rho \leftarrow \rho'$\;
%            $T \leftarrow T'$\;
%        }
%
%        \eIf{shorter $st$-path exists in triangulation supergraph}{
%            Generate violating clause and add it to $M$\;
%        }{
%            \Return $T$\;
%        }        
%    }
%    \caption{Pseudo code of the incremental algorithm}
%    \label{alg:incremental-algorithm}
%\end{algorithm}


%We propose an exact incremental SAT-based algorithm for solving the MDP to provable optimality.
%Given the initial solution, it (1) identifies the pair $s,t$ of points that
%define the dilation and then (2) formulates and solves a SAT problem that enforces edges on a shorter $st$-path, see~\cref{alg:incremental-algorithm} for details.
%Steps (1) and (2) can be repeated until no shorter $st$-path exists (i.e. all such paths are excluded based on previous constraints).
%In order to compare two solutions it is necessary to be able to compute the dilation of a given triangulation as if we were using infinite precision.
%It is crucial that this is done efficiently as computing the dilation comes with a significant overhead in runtime. \Cref{sec:exactness-implementation-issues} outlines our implementation.
%The algorithm terminates if the constraints introduced in (2) lead to a infeasible SAT model or if no shorter $st$-path exists in the triangulation supergraph.
%We use a simple SAT formulation with Boolean variables $x_e$ for all edges $e\in E$.
%\michael{We should add some notation for possible/impossible/certain to make this more clear.}

%If an edge $e$ is marked as \emph{certain} or \emph{impossible} in the triangulation supergraph, the variable $x_e$ is set to true or false, respectively.
%\Cref{eq:pairwise-intersection} enforces that no two edges can be selected that intersect each other in an inner point of one of the edges.
%As a triangulation is a maximal planar graph, \cref{eq:enforce-edges} enforces that for any edge $e_i\in E$, either $e_i$ or one intersecting edge is selected.


%Whenever the SAT problem finds a feasible solution $T'$ with objective value $\rho'$, we identify two points $s,t$ with dilation $\rho'$ in $T'$ that are connected by some path of length $\ell$.
%In any solution with dilation $< \rho'$, $s$ and $t$ have to be connected by a shorter path.
%It suffices to identify all edges on possible shorter $st$-paths and enforce that at least one of these edges is in the solution. \Cref{fig:path-enumeration} shows that that it is sometimes possible to reduce the number of edges that need to be enforced by identifying a hitting set on the edges of the possible $st$-paths.
%We propose a heuristic to identify a reasonably small hitting set.
%Our approach uses a bidirectional Dijkstra in the possible and certain edges of the triangulation supergraph to search for all shorter paths between $s$ and $t$.
%Whenever we find a shorter path, at least one edge on the path is not part of the current solution. We remove that edge from the graph and add it to the edge set $E'$.
%This process is repeated until no shorter path exists.
%We then build a set $E'$ of edges that are not in the current triangulation but cut through every shorter $st$-path.
%In any triangulation with a smaller dilation, $s$ and $t$ have to be connected by a shorter path and thus at least one of the edges from $E'$ has to be in the solution which we enforce by adding constraints $\bigvee_{e \in E'} x_e$.
%Note that this constraint remains valid for all subsequent SAT problems as the set of possible edges is reduced in each iteration.
%Once no shorter path exists, or the newly added constraint leads to an infeasible SAT model, the algorithm terminates and returns the current solution.

\subsection{Binary search}
\label{sec:binary-search}
\newcommand{\rhoLB}{\rho_{\text{lb}}}
\newcommand{\rhoUB}{\rho_{\text{ub}}}

% TODO Appendix pseudo code
%\begin{algorithm}[t]
%    \SetKwInOut{Input}{Input}
%    \Input{Initial Triangulation $T$ with dilation $\rho$}
%    Compute the triangulation supergraph from $\rho$\;
%    Compute initial lower bound $\rhoLB$\;
%    \textsc{BinarySearch}($\rho$, $\rhoLB$)\;
%    \caption{Pseudo code of the binary search algorithm}
%    \label{alg:binary-search-algorithm}
%\end{algorithm}
%\begin{algorithm}[t]
%    \SetKwInOut{Input}{Input}
%    \Input{Initial Triangulation $T$ with dilation $\rho$}
%    \While{true}{
%        \If{$M$ is not satisfiable}{
%            \Return $T$\;
%        }
%        Let $T'$ be a triangulation that satisfies $M$\;
%        Compute $\rho'$ and two vertices $s,t$ with dilation $\rho'$ in $T'$\;
%
%        \If{$\rho' \leq \rho$}{
%            Update the triangulation supergraph with $\rho'$\;
%            $\rho \leftarrow \rho'$\;
%            $T \leftarrow T'$\;
%        }
%
%        \eIf{shorter $st$-path exists in triangulation supergraph}{
%            Generate violating clause and add it to $M$\;
%        }{
%            \Return $T$\;
%        }        
%    }
%    \caption{Pseudo code of the binary search phase}
%    \label{alg:binary-search-phase}
%\end{algorithm}
Preliminary experiments with \incmdt{} showed that we spend almost all runtime for computing dilations, 
even for instances for which we could rely exclusively on interval arithmetic, requiring no exact computations.
For many instances, most iterations of \incmdt{} resulted in tiny improvements of the dilation.
To reduce the number of iterations (and thus, dilations computed), 
we considered the binary search-based algorithm \binmdt{}.

\subsubsection{High-level idea}
At any point in time, aside from the dilation $\rhoUB$ of the best known triangulation,
\binmdt{} maintains a lower bound $\rhoLB$ on the dilation, initialized as described in \cref{sec:edge-enumeration}.

As long as $\rhoUB-\rhoLB \geq \sigma$ for a small threshold value $\sigma$, \binmdt{} performs a binary search.
It computes a new dilation bound $\rho = \frac{1}{2}(\rhoLB + \rhoUB)$.
It then uses the SAT model in a similar way as \incmdt{} to determine whether a triangulation $T$ with $\rho(T) < \rho$ exists.
If it does, it updates $\rhoUB = \rho(T)$; otherwise, it updates $\rhoLB = \rho$.

Once $\rhoUB-\rhoLB$ falls below $\sigma$, \binmdt{} falls back to a slightly modified version of \incmdt{} to find the MDT,
starting from the best known triangulation with dilation $\rhoUB$.

In the following, we describe and motivate the differences between how \incmdt{} and \binmdt{} use the SAT formulation;
for more details, see also \cref{sec:incremental-sat-solving}.

\subsubsection{Dilation sampling}
To further reduce the time spent on computing dilations, observe the following.
When a node $v$ of some graph $G$ is expanded in Dijkstra's algorithm from source $s$, we know the shortest path from $s$ to $v$ and thus the dilation $\rho_G(s,v)$.
Because $\rho_G(s,v) \leq \rho(G)$, we can compute a lower bound on the dilation much faster than the precise value by only performing a constant number of node expansions from each point $p \in P$.
We call this \emph{sampling} of the dilation.
Given a bound $\rho$ on the dilation, we can sample a triangulation $T$ for violations, i.e., pairs $s,t$ of points with $\rho_T(s,t) \geq \rho$.
We observed that a dilation-defining path usually consisted of few edges;
thus, we have a good chance of finding it by sampling.

If it is likely that a new-found triangulation $T'$ violates a given bound $\rho$,
we can thus expect to save time by sampling for violations instead of computing the dilation exactly.
Sampling also allows us to use multiple violations to construct multiple clauses in each iteration,
potentially further reducing the number of iterations.
\binmdt{} uses sampling after each SAT call with a small constant limit on the number of violations.
If violations are found, no full dilation computation is required and violations are used to construct clauses.
Only if no violations are found, we compute the exact dilation;
ideally, this only happens once for each upper bound reduction in the binary search,
namely once we find a triangulation satisfying the current bound.
We also sample in the final improvement phase of \binmdt{}.
For an experimental overview on the number of times sampling was sufficient in comparison to the 
number of times the dilation had to be computed exactly, see \cref{sec:experiments-dilation-computation}.

\begin{table}
        \renewcommand{\arraystretch}{0.8}
        \begin{threeparttable}
        \begin{tabular}{@{}lcc@{}}
        \toprule
        Parameter & Floating Allegro Hand & Bimanual Robot Arms \\
        \midrule
        % \makecell{Initial object translational \\ perturbation (cm) }& [$\pm1.5$, $\pm1.5$, 0] & [$\pm 5$, $\pm 5$, 0]\\
        % \makecell{Initial object rotational \\ perturbation (rad) }& [0, 0, $\pm 0.3$] & [0, 0, $\pm 0.3$]\\
        Init. obj. trans.  pert. (cm) & [$\pm1.5$, $\pm1.5$, 0] & [$\pm 5$, $\pm 5$, 0]\\
        Init. obj. rot. pert. (rad) & [0, 0, $\pm 0.3$] & [0, 0, $\pm 0.3$]\\
        Object side length (cm) & [5.8, 6.2]  & [28, 32] \\
        Object mass (kg) & [0.1, 0.3]  & [0.25, 0.75]  \\
        Friction coefficients & [0.7, 1.3] &  [0.2, 0.4]  \\
        Task horizon (s) & 25 & 50 / 260  (Panda / iiwa) \\
        \bottomrule
        \end{tabular}
        \end{threeparttable}
        \caption{Ranges of different physical parameters $\theta$. The initial object pose is only perturbed in yaw, x, and y to ensure the object sits stably on the table. }
        \label{tab:domain_randomization}
        \vspace{0.5em}
\end{table}

\begin{figure*}[t]
\centering
\includegraphics[width=1.0\textwidth]{figures/trajopt_unittest.png}
	\caption{\textbf{Trajectory optimization is crucial for generating dynamically feasible trajectories}. (Top) Before trajectory optimization, the kinematically retargeted demos easily lose contact and drive the object out of reach with different physical parameters or slight deviations in object states. (Bottom) Trajectory optimization encourages robots to establish contact with and maintain good manipulability of the object. The tricolor axis indicates the object orientation.}
	\label{fig:trajopt_unittest}
\end{figure*}

\section{Trajectory Optimization Experiments}

While kinematic retargeting of demonstrations might suffice to generate data for simpler manipulation tasks such as pick and place, it often falls short for the more challenging contact-rich tasks requiring frequent contact mode switches and fine-grained actions. In this section, we demonstrate that trajectory optimization is crucial for generating diverse, dynamically feasible contact-rich trajectories on three high-dimensional dexterous manipulation systems: a floating Allegro hand, bimanual iiwa arms, and bimanual Panda arms.

Our data generation framework is agnostic to the choice of the trajectory optimizer. We implement 
% a contact-implicit model predictive controller based on smoothed contact dynamics \cite{suh2024dexterous} and
the cross-entropy method (CEM) \cite{de2005tutorial} to solve \eqref{eq:predictive_control} over a distribution of physical parameters and initial conditions, as specified in Table \ref{tab:domain_randomization}. 
%\russtcomment{Right... the SQP discussion tricked me, but I guess that's only for the retargeting. I thought you had replaced this. In this case, your approach is almost doing RL, but on a policy parameterized as a trajectory... right? why is that better than doing PPO on a small neural net policy, and generating data from that? If you stick with CEM, than this will be your burden of proof, i think?}

\underline{\textbf{Task}} Manipulating the object to a target pose on the table (Fig. \ref{fig:policy_rollouts}). The object is initially placed randomly on the table with an arbitrary face upward. Task success is defined as the object reaching within 3 cm and 0.2 rad of the target pose for the Allegro hand, and within 10 cm and 0.2 rad for the bimanual robot arms.  This task requires long-horizon reasoning of complex multi-contact interactions between the robot and the object. The necessary frequent contact mode switches and high-dimensional action space pose great challenges for traditional model-based planners, while the precise contact interactions require fine-grained control actions. 

\begin{table}
\centering
        \renewcommand{\arraystretch}{0.8}
        \begin{threeparttable}
        \begin{tabular}{@{}lcccc@{}}
        \toprule
        Perturbation & Allegro Hand & iiwa Arms & Panda Arms \\
        \midrule
        Original demo &4 / 24 & 5 / 24 & 6 / 24\\
        Object size & 2 / 24 & 1 / 24 & 4 / 24\\
        % Object mass & 1 / 24& 1 / 24 & \\
        % Friction coefficients & 3 / 24 & 2 / 24 & \\
        Initial object translation & 1 / 24 & 3 / 24 & 2 / 24\\
        Initial object orientation & 2 / 24 & 3 / 24& 3 / 24\\
        \midrule
        Trajectory optimization & 2164 / 3000 & 2252 / 3000 & 2462 / 3000 \\
        \bottomrule
        \end{tabular}
        \end{threeparttable}
        \caption{Success rates of replaying kinematically retargeted trajectories of the 24 original human demos, and trajectory optimization under random perturbations in physical parameters and object initial conditions. }
        \label{tab:kin_success_rate}
\end{table}

% \begin{table}
% \centering
%         \renewcommand{\arraystretch}{0.8}
%         \begin{threeparttable}
%         \begin{tabular}{@{}ccc@{}}
%         \toprule
%         Allegro Hand & iiwa Arms & Panda Arms \\
%         \midrule
%         0.721 & 0.65 & 0.803\\
%         % Task Horizon (s) & 25 & 280 & 50 \\
%         \bottomrule
%         \end{tabular}
%         % \begin{tablenotes}
%         % \itme{*} 
%         % \end{tablenotes}
%         \end{threeparttable}
%         \caption{Success rates of trajectory optimization under random perturbations in physical parameters and object initial conditions. }
%         \label{tab:trajopt_success_rate}
%         \vspace{0.5em}
% \end{table}



% \begin{figure*}[t]
% \centering
% \includegraphics[width=0.7\textwidth]{figures/aug_traj_den.png}
% 	\caption{\textbf{Distribution of object trajectories generated from a single demonstration}. The original demonstration (orange) is locally perturbed and augmented to about 100 dynamically feasible contact-rich trajectories (blue) for each system. The density map represents the object pose distribution of the generated trajectories in the specific 2-dimensional slices.}
%     \label{fig:aug_data_distribution}
% \end{figure*}

% \begin{figure*}[t]
% \centering
% \includegraphics[width=0.9\textwidth]{figures/aug_traj_snapshots.png}
% 	\caption{\textbf{Snapshots of trajectories generated from a single demonstration}. The original demonstration (orange) is locally perturbed and augmented to about 100 dynamically feasible contact-rich trajectories (blue) for each system. The density map represents the object pose distribution of the generated trajectories in the specific 2-dimensional slices.}
%     \label{fig:aug_data_distribution}
% \end{figure*}
\begin{figure*}[t]
\centering
\includegraphics[width=1.0\textwidth]{figures/density_snapshots_aug_traj.png}
	\caption{\textbf{Distribution and snapshots of trajectories generated from a single demonstration.} (a) The original demonstration (orange) is locally perturbed and augmented to about 100 dynamically feasible contact-rich trajectories (blue) for each system. The density map represents the object pose distribution of the generated trajectories in the specific 2-dimensional slices. (b) Snapshots of 30 dynamically feasible trajectories under random physical parameters and object initial poses for bimanual iiwa arms are visualized.}
    \label{fig:aug_data_distribution}
\end{figure*}


\underline{\textbf{Dynamic Feasibility}}
While kinematic motion retargeting can generate visually plausible robot and object trajectories, these trajectories often lack dynamical consistency due to the differences in physical parameters and embodiment between the human demonstrator and the target robot. To illustrate this, we replay the kinematically retargeted trajectories of the original 24 human demos and record the success rates for each system in Table \ref{tab:kin_success_rate}. Furthermore, we randomly sample object sizes and perturbations of initial object poses according to Table \ref{tab:domain_randomization} and roll out the nominal kinematically retargeted trajectories. Some trajectories still succeed under certain perturbations thanks to caging grasps or other strategies that encourage robustness during the human demonstration. For all the systems, the successful rollouts are relatively short, manipulating the object to the goal pose within only 1 or 2 rotations. 
% Notably, the successful trajectories for the iiwa and Panda arms vary significantly, despite being generated from the same initial set of demonstrations.

The low success rate of purely kinematically retargeted trajectories highlights the importance of trajectory optimization for locally refining the demos for the particular embodiments and physical parameters. Before trajectory optimization, the floating Allegro hand lightly touches the cube and easily loses contact when rotating it clockwise (demonstrated in Fig. \ref{fig:trajopt_unittest}a). After trajectory optimization, the hand increases the contact area, establishing a stable grip for rotation. In Fig. \ref{fig:trajopt_unittest}b, similar behavior that encourages contact can be observed for the bimanual iiwa arms: the demo trajectory tries to rotate the box clockwise only using a single arm, while trajectory optimization encourages the other arm to help hold the box and reorient the box more stably. These refinements that encourage contact are particularly helpful when the object is heavier or smaller, or when the friction coefficients are lower than expected. In addition, replaying the kinematically retargeted trajectory often fails when the object pose deviates slightly from the demonstration, driving the object out of reach (visualized in Fig. \ref{fig:trajopt_unittest}c). In contrast, trajectory optimization 
%stabilizes the system in a vicinity around the demonstration, ensuring higher success rates even when the object is perturbed
accounts for the system’s true dynamics and can adjust the robot’s actions accordingly. The success rates of trajectory optimization under random perturbations in physical parameters and object initial conditions for each system are recorded in Table \ref{tab:kin_success_rate}.

\begin{figure*}[t]
    \centering
    \includegraphics[width=0.9\linewidth]{figures/policy_rollouts.png}
    \caption{\textbf{Policy rollouts for different embodiments.} The object manipulation task requires the robots to frequently make and break contact with the object. It also requires precise control of the robot since small deviations in positions can result in missing contact interactions and lead to task failure. } 
    \label{fig:policy_rollouts}
\end{figure*}

\underline{\textbf{Cross-Embodiment Generalization}} We demonstrate that a single set of human demonstrations can be effectively repurposed to generate dynamically consistent, contact-rich trajectories across different robotic embodiments with varying task horizons. Specifically, human demonstrations involving two index fingers manipulating a small cube are retargeted to fixed-base bimanual Kuka LBR iiwa and Franka Emika Panda arms manipulating a larger box (visualized in Fig. \ref{fig:kinematic_retargeting}). This approach addresses key challenges in data collection for contact-rich tasks: directly teleoperating two real robot arms to flip a large box would be both physically demanding and cost-prohibitive due to hardware latency, limited feedback, and the embodiment gap--differences in kinematic structure, degrees of freedom, and workspace between human and robotic arms. In contrast, performing the same task on a smaller scale using human fingers is more intuitive, reduces physical effort, and enables faster, more consistent demonstration collection.

The iiwa and Panda arms differ in contact geometry, velocity limits, and joint constraints, all of which are explicitly modeled within the trajectory optimization framework described in \eqref{eq:predictive_control}. For safe hardware deployment, we enforce conservative velocity limits on the iiwa arms, while only applying soft velocity regularization on the Panda arms in simulation to allow for more aggressive motions.


\underline{\textbf{Data Diversity}} 
Trajectory optimization efficiently augments a single demonstration to a wide distribution of trajectories with locally perturbed physical parameters and initial conditions as visualized in Fig. \ref{fig:aug_data_distribution}. The diverse states in the generated dataset cover a larger training distribution and encourage smoother learned policies, as will be discussed in the next section.
\begin{figure*}[t]
    \centering
    \includegraphics[width=0.9\linewidth]{figures/policy_failure.png}
    \caption{\textbf{Failure cases of baselines.} (a) The baseline policy trained on the original 24 demonstrations for the floating Allegro hand frequently misses contact or gets stuck on the cube. (b-c) The baseline policies for the bimanual robot arms often exhibit jittery motion, resulting in loss of contact, the box being kicked out of reach, or the robot arms running into and getting stuck on the box surface. } 
    \label{fig:policy_failure}
\end{figure*}

\begin{figure}
\centering
\includegraphics[width=0.42\textwidth]{figures/success_rate.png}
	\caption{Success rates of policy evaluation in simulation and hardware. }
	\label{fig:success_rate}
    \vspace*{-0.4cm}
\end{figure}

% \begin{figure}
% \centering
% \includegraphics[width=0.48\textwidth]{figures/jitteriness.png}
% 	\caption{\textbf{Joint angles of bimanual iiwa arms over time. } Each line represents the trajectory of a different joint of the iiwa arms. The policy trained on augmented datasets (b) demonstrates significantly smoother motion compared to the baseline policy (a). }
% 	\label{fig:jitteriness}
% \end{figure}

\begin{figure*}[t]
    \centering
    \includegraphics[width=0.9\linewidth]{figures/hardware_rollout.png}
    \caption{\textbf{Policy rollouts on hardware.} The fixed-base bimanual iiwa arms perform a sequence of coordinated rolling, pitching, and yawing actions to reorient the box to the goal pose. } 
    \label{fig:hardware_rollout}
\end{figure*}

\section{Behavior Cloning Experiments}
We illustrate our framework's capability to efficiently produce diverse, high-quality contact-rich datasets for training behavior cloning policies across multiple robotic platforms, including the floating Allegro hand and the bimanual Panda arms in simulation as well as bimanual iiwa arms on hardware. We show that policies trained on the generated data generalize to a wide distribution of physical parameters and initial conditions, and are much more robust and performant than the ones trained only on the original demonstrations. 
\subsection{Policy Evaluation in Simulation}
\label{subsec:policy_eval_sim}
From only 24 human demonstrations, our data generation pipeline can efficiently generate thousands of dynamically feasible contact-rich trajectories using trajectory optimization. We train state-based diffusion policies \cite{chi2023diffusion} on the 24 original demo trajectories, as well as 500 and 1000 generated trajectories. While our method is compatible with any Behavior Cloning algorithm, we adopt diffusion policies due to its recent success in contact-rich tasks \cite{chi2024universal, zhu2024should, li2024planning}. Fig. \ref{fig:policy_rollouts} visualizes the policy rollouts. We evaluate the performance by conducting 48 policy rollouts for each embodiment in simulation and record the success rates in Fig. \ref{fig:success_rate}. The success criteria are the same as specified in the trajectory optimization experiments.
%For policy evaluation, we visualize the initial states for all evaluation episodes, typical failure cases of baseline policies, and final object pose errors in Fig. \ref{fig:policy_eval}.  
% and validate that the generated data help improve the policy's robustness and generalizability.

\subsubsection{Floating Allegro Hand} 
While the human demonstrator completes the task in approximately 5 seconds on average in the virtual reality environment, the demonstration trajectories are temporally scaled by a factor of 2.5 to ensure smoother, dynamically feasible motions on the floating Allegro hand, which is subject to velocity limits. We define the task horizon as 25 seconds to allow the policy sufficient time to recover from missed contacts and other errors during the execution. The task complexity arises from the 22-dimensional action space of the Allegro hand and the long-horizon nature of the task, which requires a sequence of coordinated rolling, pitching, and yawing actions to reorient the cube to an upright position. These factors together present significant challenges for traditional model-based planners without guidance.

The baseline behavior cloning policy trained on the original set of 24 demonstrations achieves a success rate of $10 / 48 = 21\%$ and exhibits significant jittery behavior when encountering out-of-distribution states. The workspace, characterized by diverse object orientations and translations, is sufficiently large that minor deviations during policy rollouts often drive the trajectory out of the demonstrated distribution. Common failure modes include the Allegro hand repeatedly missing contact with the cube or becoming stuck on its surface while attempting reorientation (visualized in Fig. \ref{fig:policy_failure}a), which often result in the object being trapped in intermediate orientations. In contrast, policies trained on the expanded dataset generated by our pipeline demonstrate a higher likelihood of re-establishing contact with the object after initial misses, resulting in significantly improved success rates up to $39 / 48 = 81\%$.

\begin{figure*}[t]
    \centering
    \includegraphics[width=0.9\linewidth]{figures/hardware_eval.png}
    \caption{\textbf{Policy failure and recovery on hardware.} The baseline policy frequently (a) gets stuck on the box surface when small deviations from the demonstration trajectories occur, and (b) struggles to recover from out-of-distribution states, where the object is never intentionally lifted for accomplishing the task in the generated dataset. Policies trained on augmented datasets (c) sometimes fail due to unmodeled collision geometry, but (d) can recover from undesired sliding by employing firmer grasps found by trajectory optimization. } 
    \label{fig:hardware_eval}
\end{figure*}
\subsubsection{Bimanual Robot Arms}
The baseline policy trained on the original set of 24 human demonstrations achieves a success rate of $27 / 48 = 56\%$ on the bimanual iiwa system. We hypothesize that the restrictive velocity limits encourage more quasi-static behavior, leading to longer trajectories with a higher density of state-action pairs in the training data. In contrast, the baseline policy yields a success rate of $14/48=29\%$ on the bimanual Panda system, likely due to the more dynamic nature of the learned behavior under its looser velocity constraints. Both baseline policies exhibit remarkably jittery motion, frequently kicking the box out of reach, losing contact, or running into and getting stuck on the box surface during reorientation (visualized in Fig. \ref{fig:policy_failure}b and c). Policies trained on the augmented dataset, however, generate significantly smoother trajectories and are capable of re-establishing contact with the object after initial misses, resulting in as high as $44 / 48 = 92\%$ success rates for bimanual iiwa arms and $42 / 48 = 87.5\%$ for bimanual Panda arms. Additionally, the learned policies capture multimodal behaviors observed in the original human demonstrations, such as rotating the box either clockwise or counterclockwise for similar object poses. 


\subsection{Policy Evaluation on Hardware}
We zero-shot deploy the trained policies on hardware for bimanual iiwa arms to flip a 30 cm cubic box on a table (Fig. \ref{fig:hardware_rollout}). An OptiTrack motion capture system is employed to estimate the object pose. The baseline behavior cloning policy only achieves $6/23=26\%$ success rate, with most successful rollouts being relatively short-horizon, involving only 1 or 2 rotations. Common failure modes of the baseline policy include: 1) deviation from the demonstration trajectory, causing the arms to collide with the box surface (Fig. \ref{fig:hardware_eval}a), and 2) significant box sliding during rolling, resulting in the policy encountering out-of-distribution states and failing to recover (Fig. \ref{fig:hardware_eval}b). In contrast, as shown in Fig. \ref{fig:success_rate}b, the policy trained on 500 generated trajectories achieves $17 / 23 = 74\%$ success rate, while the policy trained on 1000 generated trajectories achieves $16/23=70\%$ success rate. Despite occasional box sliding during rolling, these policies demonstrate an improved ability to stabilize the box by using one arm to hold the opposite side more firmly to prevent further sliding (Fig \ref{fig:hardware_eval}d). However, as visualized in Fig \ref{fig:hardware_eval}c, both policies trained on the augmented datasets exhibit failure modes originating from unmodeled collision geometries on iiwa arms, which lead to significant undesired yaw motions of the box during pitch actions.\looseness=-1
\section{Conclusion and future work}
In this study, we examined the ability of LLMs to produce self-generated counterfactual explanations (SCEs).
We design a prompt-based setup for evaluating the efficacy of \SCEs.
Our results show that LLMs consistently struggle with generating valid \SCEs. In many cases model prediction on a \SCE does not yield the same target prediction for which the model crafted the \SCE.
Surprisingly, we find that LLMs put significant emphasis on the context---the prediction on \SCE is significantly impacted by the presence of original prediction and instructions for generating the \SCE.
Based on this empirical evidence, we argue that LLMs are still far from being able to explain their own predictions counterfactually.
Our findings add to similar insights from recent studies on other forms of self-explanations~\cite{lanham2023measuring,tanneru2024quantifying}.



Our work opens several avenues for future work. Inspired by counterfactual data augmentation~\cite{sachdeva2023catfood}, one could include the counterfactual explanation capabilities a part of the LLM training process. This inclusion may enhance the counterfactual reasoning capabilities of the LLM. Follow ups should also explore the effect of prompt tuning, specifically, model-tailored prompts for generating \SCEs. These approaches might lead to better quality \SCEs.


We limited our investigation to open source models of upto 70B parameters. Extending our analysis to larger and more recent models, \eg, DeepSeek R1 671B, and closed source models like OpenAI o3 would be an interesting avenue for future work.

Finally, our experiments were limited to relatively simple tasks: classification and mathematics problems where the solution is an integer. This limitation was mainly due to the fact that it is difficult to automatically judge validity of answers for more open-ended language generation tasks like search and information retrieval. Scaling our analysis to such tasks would require significant human-annotation resources, and is an important direction for future investigations.


%%
%% Bibliography
%%

%% Please use bibtex, 

\bibliography{main.bib}

\appendix

\newpage
\appendix
\onecolumn

\renewcommand{\thetable}{A\arabic{table}} % Prefix table numbers with 'A'
\renewcommand{\thefigure}{A\arabic{figure}} % Prefix figure numbers with 'A'
\renewcommand{\theequation}{A\arabic{equation}} % Prefix equation numbers with 'A'

\setcounter{table}{0} % Reset table counter
\setcounter{figure}{0} % Reset figure counter
\setcounter{equation}{0} % Reset equation counter

\section*{Appendix}

\section{Optimal Brain Surgeon Derivation}
\label{OBS_ALGORITHM}

In the original setup in OBS, we have a local quadratic model for the loss $L$ given by:
$$
    \delta L = L(w + \delta w) \approx L(w) + \nabla_w L^T \delta w + \frac{1}{2} \delta w^T H \delta w
$$
Since OBS is a pruning-after-training approach, they discarded the 1-st order component. Reducing the expression for saliency as:
$$
    \delta L = \frac{1}{2} \delta w^T H \delta w
$$
To remove a single parameter, the authors of OBS introduced the constraint $e_q^T \delta w + w_q = 0$, with $e_q$ being the $q^{\text{th}}$ canonical basis vector. The pruning is defined as a constrained optimization problem of the form:
$$
    \min_{\delta w \in \mathbb{R^d}} \left( \frac{1}{2} \delta w^T H \delta w\right),
    ~~\text{s.t}~~
    e_q^T \delta w + w_q = 0.
$$
And the choice of which parameter to remove becomes:
$$
    \min_{q \in \mathcal{Q}} \left\{
        \min_{\delta w \in \mathbb{R^d}} \left( \frac{1}{2} \delta w^T H \delta w\right),
        ~~\text{s.t}~~
        e_q^T \delta w + w_q = 0
    \right\}.
$$
To solve the internal problem, we use a Lagrange multiplier $\lambda$ to write the problem as an unconstrained optimization case as follows:
$$
    \mathcal{L}(\delta w, \lambda) =
    \frac{1}{2} \delta w^T H \delta w +
    \lambda(e_q^T \delta w + w_q).
$$
Then, to find the stationary conditions, we compute the partial derivatives with respect to $\delta w$ and $\lambda$, and equate them to 0, obtaining:
$$
    \nabla_{\delta w} \mathcal{L} = 
    H \delta w + \lambda e_q = 0 
    \rightarrow
    \delta w = - \lambda H^{-1} e_q
$$
$$
    \nabla_{\lambda} \mathcal{L} =
    e_q^T \delta w + w_q = 0
    \rightarrow
    e_q^T \delta w = -w_q
$$
With some replacements, we get:
$$
    e_q^T \delta w = -w_q
    \rightarrow
    e_q^T \left( 
        - \lambda H^{-1} e_q
    \right) = -w_q
    \rightarrow
    - \lambda e_q^T H^{-1} e_q = -w_q
    \rightarrow
    \lambda = \frac{w_q}{e_q^T H^{-1} e_q} = \frac{w_q}{[H^{-1}]_{qq}}
$$
$$
    \delta w = - \frac{w_q H^{-1} e_q}{[H^{-1}]_{qq}}
$$
Replacing the expression for $\delta w$ in the saliency expression, we have:
\begin{align*}
    \delta L = \frac{1}{2} \delta w^T H \delta w
    &= \frac{1}{2}\left(
        - \frac{w_q H^{-1} e_q}{[H^{-1}]_{qq}}
    \right)^T
    H
    \left(
        - \frac{w_q H^{-1} e_q}{[H^{-1}]_{qq}}
    \right)
    \nonumber \\
    &= 
    \frac{w_q^2}{2[H^{-1}]_{qq}^2}
    \left(
        H^{-1} e_q
    \right)^T
    H
    \left(
        H^{-1} e_q
    \right)
    \nonumber \\
    &= 
    \frac{w_q^2}{2[H^{-1}]_{qq}^2}
    e_q ^T
    H^{-1}
    e_q
    = 
    \frac{w_q^2}{2[H^{-1}]_{qq}^2}
    [H^{-1}]_{qq}
    = 
    \frac{w_q^2}{2[H^{-1}]_{qq}}
    \nonumber \\
\end{align*}
%------------------------------------------------------------------------------------------------
\newpage
\section{Fisher Brain Surgeon Sensitivity Derivation}
\label{FBSS_ALGORITHM}
As we considered a PBT setting, it is not possible to ignore the first-order term in the local quadratic approximation of the error as it could still be informative. In this case, our model for sensitivity is given by: 
$$
    \delta L = \nabla_w L^T \delta w + \frac{1}{2} \delta w^T H \delta w
$$
The process to remove a single parameter remains similar; the constraint $e_q^T \delta w + w_q = 0$, with $e_q$ is still valid, redefining the optimization problem as:
$$
    \min_{\delta w \in \mathbb{R^d}} \left(
        \nabla_w L^T \delta w +  \frac{1}{2} \delta w^T H \delta w
    \right),
    ~~\text{s.t}~~
    e_q^T \delta w + w_q = 0.
$$
And the choice of which parameter to remove becomes:
$$
    \min_{q \in \mathcal{Q}} \left\{
        \min_{\delta w \in \mathbb{R^d}} \left(
            \nabla_w L^T \delta w + \frac{1}{2} \delta w^T H \delta w
        \right),
        ~~\text{s.t}~~
        e_q^T \delta w + w_q = 0
    \right\}.
$$
Using a Lagrange multiplier $\lambda$ as in the reference case, we solve the following unconstrained optimization problem:
$$
    \mathcal{L}(\delta w, \lambda) =
    \nabla_w L^T \delta w + 
    \frac{1}{2} \delta w^T H \delta w +
    \lambda(e_q^T \delta w + w_q).
$$
With the following stationary conditions:
$$
    \nabla_{\delta w} \mathcal{L} = 
    \nabla_w L + H \delta w + \lambda e_q = 0 
    \rightarrow
    \delta w = - (\lambda H^{-1}e_q + H^{-1} \nabla_w L)
$$
$$
    \nabla_{\lambda} \mathcal{L} =
    e_q^T \delta w + w_q = 0
    \rightarrow
    e_q^T \delta w = -w_q
$$
The expression for $\lambda$ is redefined as follows:
\begin{align*}
    e_q^T \left(
        - (\lambda H^{-1}e_q + H^{-1} \nabla_w L)
    \right) 
    &= -w_q
    \nonumber \\
    \lambda e_q^T H^{-1} e_q + e_q^T H^{-1} \nabla_w L
    &= w_q
    \nonumber \\
    \lambda [H^{-1}]_{qq} 
    &= w_q - e_q^T H^{-1} \nabla_w L
    \nonumber \\
    \lambda
    &= \frac{w_q - e_q^T H^{-1} \nabla_w L}{[H^{-1}]_{qq}}
\end{align*}
Replacing the expression for $\delta w$ in our sensitivity expression, we have:
\begin{align*}
    \delta L = \nabla_w L^T \delta w + \frac{1}{2} \delta w^T H \delta w
    &= 
    \nabla_w L^T \left[
        - (\lambda H^{-1}e_q + H^{-1} \nabla_w L)
    \right]
    \nonumber \\
    &+
    \frac{1}{2}\left[
        - (\lambda H^{-1}e_q + H^{-1} \nabla_w L)
    \right]^T
    H
    \left[
        - (\lambda H^{-1}e_q + H^{-1} \nabla_w L)
    \right]
    \nonumber \\
    &= 
    - \lambda \nabla_w L^T H^{-1}e_q - \nabla_w L^T H^{-1} \nabla_w L
    \nonumber \\
    &+
    \frac{1}{2}\left[
        (\lambda H^{-1}e_q)^T + (H^{-1} \nabla_w L)^T
    \right]
    \left[
        \lambda H H^{-1}e_q + H H^{-1} \nabla_w L)
    \right]
    \nonumber \\
    &= 
    - \lambda \nabla_w L^T H^{-1}e_q - \nabla_w L^T H^{-1} \nabla_w L
    \nonumber \\
    &+
    \frac{1}{2}\left[
        (\lambda H^{-1}e_q)^T + (H^{-1} \nabla_w L)^T
    \right]
    \left[
        \lambda e_q + \nabla_w L
    \right]
    \nonumber \\
    &= 
    - \lambda \nabla_w L^T H^{-1}e_q - \nabla_w L^T H^{-1} \nabla_w L
    \nonumber \\
    &+
    \frac{1}{2}\left[
        (\lambda H^{-1}e_q)^T \lambda e_q
        + (H^{-1} \nabla_w L)^T \lambda e_q
        + (\lambda H^{-1}e_q)^T \nabla_w L
        + (H^{-1} \nabla_w L)^T \nabla_w L
    \right]
    \nonumber \\
    &= 
    - \lambda \nabla_w L^T H^{-1}e_q - \nabla_w L^T H^{-1} \nabla_w L
    \nonumber \\
    &+
    \frac{1}{2}\left[
        \lambda^2 e_q^T H^{-1} e_q
        + \lambda \nabla_w L^T H^{-1} e_q
        + \lambda e_q^T H^{-1} \nabla_w L
        + \nabla_w L^T H^{-1} \nabla_w L
    \right]
    \nonumber \\
    &= 
    \frac{1}{2}\left[
        \lambda^2 [H^{-1}]_{qq}
        - \lambda \nabla_w L^T H^{-1} e_q
        + \lambda e_q^T H^{-1} \nabla_w L
        - \nabla_w L^T H^{-1} \nabla_w L
    \right]
    \nonumber \\
\end{align*}
Finally, replacing the $\lambda$:
\begin{align*}
    \delta L 
    &= 
    \frac{1}{2}\left[
        \lambda^2 [H^{-1}]_{qq}
        - \lambda \nabla_w L^T H^{-1} e_q
        + \lambda e_q^T H^{-1} \nabla_w L
        - \nabla_w L^T H^{-1} \nabla_w L
    \right]
    \nonumber \\
    &= 
    \frac{1}{2[H^{-1}]_{qq}}\left[
        (w_q - e_q^T H^{-1} \nabla_w L)^2 
        + (w_q - e_q^T H^{-1} \nabla_w L)(e_q^T H^{-1} \nabla_w L - \nabla_w L^T H^{-1} e_q)
        - \nabla_w L^T H^{-1} \nabla_w L
    \right]
    \nonumber \\
    &= 
    \frac{1}{2[H^{-1}]_{qq}}[
        w_q^2
        - 2 w_q (e_q^T H^{-1} \nabla_w L)
        + (e_q^T H^{-1} \nabla_w L)^2
        + w_q (e_q^T H^{-1} \nabla_w L)
    \nonumber \\
        &- w_q (\nabla_w L^T H^{-1} e_q)
        - (e_q^T H^{-1} \nabla_w L)(e_q^T H^{-1} \nabla_w L)
        + (e_q^T H^{-1} \nabla_w L)(\nabla_w L^T H^{-1} e_q)
        - \nabla_w L^T H^{-1} \nabla_w L
    ]
    \nonumber \\
    &= 
    \frac{1}{2[H^{-1}]_{qq}}[
        w_q^2
        - w_q (e_q^T H^{-1} \nabla_w L)
        + (e_q^T H^{-1} \nabla_w L)^2
    \nonumber \\
        &- w_q (\nabla_w L^T H^{-1} e_q)
        - (e_q^T H^{-1} \nabla_w L)^2
        + (e_q^T H^{-1} \nabla_w L)(\nabla_w L^T H^{-1} e_q)
        - \nabla_w L^T H^{-1} \nabla_w L
    ]
    \nonumber \\
    &= 
    \frac{1}{2[H^{-1}]_{qq}}\left[
        w_q^2
        - 2 w_q (e_q^T H^{-1} \nabla_w L)
        + (e_q^T H^{-1} \nabla_w L)^2
        - \nabla_w L^T H^{-1} \nabla_w L
    \right]
    \nonumber \\
    &= 
    \frac{1}{2[\hat{F}^{-1}]_{qq}}
    \left[
        w_q - (e_q^T \hat{F}^{-1} \nabla \mathcal{L}(w_0))
    \right]^2
\end{align*}

%------------------------------------------------------------------------------------------------

\newpage
\section{Training and Testing Details}
\label{appendix:training_parameters}

We perform an 80:20 stratified split, with a constant seed, on the CIFAR10/100 training dataset to obtain a validation set with the same class distribution. For both datasets, we have a training set with 40,000 samples, a validation set with 10,000 samples, and a testing set of 10,000 samples. Validation is performed after each training step, and the weights of the best-performing validation step (based on top-1 accuracy) are utilized for the final evaluation on the testing set. Table \ref{tab:table_training_parameters} summarizes the training parameters.

\begin{table}[h]
\caption{Training parameters used for ResNet18 and VGG19 on the CIFAR-10/100 datasets.}
\label{tab:table_training_parameters}
\vskip 0.15in
\begin{center}
\begin{small}
\begin{sc}
\begin{tabular}{lcc}
\toprule
Parameter & ResNet18 & VGG19 \\
\midrule
Number of steps       & 160 & 160 \\
Criterion             & CE & CE \\
Optimizer             & SGD & SGD \\
Learning rate         & 0.01 & 0.1 \\
Momentum              & 0.9 & 0.9 \\
Weight decay          & $5 \times 10^{-4}$ & $1 \times 10^{-4}$ \\
Learning rate drops   & [60, 120] & [60, 120] \\
Learning rate drop factor & 0.2 & 0.1 \\
\bottomrule
\end{tabular}
\end{sc}
\end{small}
\end{center}
\vskip -0.1in
\end{table}

%------------------------------------------------------------------------------------------------

\newpage
\section{Results CIFAR10}
\subsection{ResNet18}
\label{appendix:CIFAR10_ResNet18}

\begin{table}[h]
\caption{Performance of different sensitivity methods for pruning evaluated using ResNet18 on the CIFAR-10 testset. The right side of the table presents our proposed criteria. The mean accuracy and standard deviation are reported across three initialization seeds for various sparsity levels. Baseline, no pruning: $91.78 \pm 0.09$.}
\label{tab:resnet18_cifar10_compressors}
\vskip 0.15in
\begin{center}
\begin{small}
\begin{sc}
\resizebox{\textwidth}{!}{%
\begin{tabular}{lccccc|cccc}
\toprule
Sparsity  & Random & Magnitude & GN & SNIP & GraSP & FD & FP & FTS & FBSS \\
\midrule
0.10  & 91.71 ± 0.21 & 91.72 ± 0.07 & 91.57 ± 0.15 & 91.72 ± 0.07 & 89.16 ± 0.05 & 91.87 ± 0.13 & 91.63 ± 0.21 & 91.53 ± 0.12 & 91.76 ± 0.08 \\
0.20  & 91.63 ± 0.11 & 91.42 ± 0.12 & 91.51 ± 0.09 & 91.64 ± 0.16 & 88.69 ± 0.34 & 91.50 ± 0.12 & 91.65 ± 0.14 & 91.53 ± 0.15 & 91.54 ± 0.13 \\
0.30  & 91.45 ± 0.18 & 91.61 ± 0.13 & 91.68 ± 0.20 & 91.65 ± 0.08 & 88.67 ± 0.26 & 91.65 ± 0.18 & 91.44 ± 0.27 & 91.49 ± 0.05 & 91.62 ± 0.07 \\
0.40  & 91.59 ± 0.18 & 91.06 ± 0.16 & 91.61 ± 0.09 & 91.55 ± 0.08 & 88.24 ± 0.33 & 91.51 ± 0.05 & 91.38 ± 0.13 & 91.56 ± 0.28 & 91.39 ± 0.05 \\
0.50  & 91.60 ± 0.06 & 91.32 ± 0.13 & 91.44 ± 0.13 & 91.22 ± 0.07 & 87.69 ± 0.15 & 91.30 ± 0.18 & 91.58 ± 0.16 & 91.46 ± 0.19 & 91.41 ± 0.05 \\
0.60  & 91.10 ± 0.16 & 91.18 ± 0.16 & 91.59 ± 0.13 & 91.24 ± 0.04 & 87.48 ± 0.55 & 91.34 ± 0.07 & 91.35 ± 0.16 & 91.40 ± 0.11 & 91.38 ± 0.18 \\
0.70  & 91.17 ± 0.04 & 91.07 ± 0.07 & 91.19 ± 0.17 & 91.33 ± 0.18 & 87.26 ± 0.34 & 91.34 ± 0.23 & 91.42 ± 0.23 & 91.18 ± 0.18 & 91.27 ± 0.14 \\
0.80  & 90.78 ± 0.08 & 91.10 ± 0.12 & 90.95 ± 0.35 & 90.74 ± 0.10 & 87.18 ± 0.51 & 90.95 ± 0.11 & 91.08 ± 0.06 & 90.94 ± 0.22 & 90.73 ± 0.33 \\
0.90  & 89.35 ± 0.13 & 89.88 ± 0.28 & 90.39 ± 0.23 & 90.36 ± 0.34 & 86.60 ± 0.51 & 90.04 ± 0.21 & 90.20 ± 0.08 & 90.55 ± 0.23 & 89.22 ± 0.30 \\
0.95  & 87.59 ± 0.11 & 89.23 ± 0.19 & 89.00 ± 0.05 & 89.31 ± 0.17 & 86.50 ± 0.05 & 88.61 ± 0.28 & 89.50 ± 0.18 & 89.47 ± 0.32 & 87.58 ± 0.25 \\
0.98  & 83.47 ± 0.20 & 85.70 ± 0.33 & 86.43 ± 0.05 & 87.26 ± 0.28 & 85.99 ± 0.08 & 85.61 ± 0.20 & 86.97 ± 0.22 & 87.24 ± 0.32 & 83.40 ± 0.74 \\
0.99  & 78.28 ± 0.45 & 71.99 ± 0.28 & 83.47 ± 0.15 & 84.54 ± 0.04 & 84.56 ± 0.46 & 82.13 ± 0.28 & 83.74 ± 0.48 & 84.85 ± 0.18 & 77.60 ± 1.02 \\
\bottomrule
\end{tabular}}
\end{sc}
\end{small}
\end{center}
\vskip -0.1in
\end{table}

%------------------------------------------------------------------------------------------------
\clearpage
\subsection{VGG19}
\label{appendix:CIFAR10_VGG19}

As discussed earlier, introducing a warm-up phase effectively mitigates layer collapse in data-dependent pruning methods. Here, we evaluate the impact of different warm-up durations by comparing no warm-up, a single warm-up epoch, and five warm-up epochs. Table \ref{tab:VGG19_cifar10_compressors} demonstrates how performance drastically degrades with increasing sparsity, ultimately leading to layer collapse at 0.90 sparsity. However, as shown in the results, a single warm-up epoch is sufficient to prevent collapse and stabilize pruning performance. Moreover, as seen in Table \ref{tab:VGG19_cifar10_compressors_warmup5}, increasing the warm-up period to five epochs provides no substantial additional improvement. This indicates that prolonged warm-up training is not necessary; a single training step is enough to achieve gradient stabilization and overcome layer collapse.

\begin{table}[h]
\caption{Performance of different sensitivity methods for pruning evaluated using VGG19 on the CIFAR-10 test set. The right side of the table presents our proposed criteria. The mean accuracy and standard deviation are reported across three initialization seeds for various sparsity levels. Baseline, no pruning: $89.21 \pm 0.22$.}
\label{tab:VGG19_cifar10_compressors}
\vskip 0.15in
\begin{center}
\begin{small}
\begin{sc}
\resizebox{\textwidth}{!}{%
\begin{tabular}{lccccc|cccc}
\toprule
Sparsity  & Random & Magnitude & GN & SNIP & GraSP & FD & FP & FTS & FBSS \\
\midrule
0.10  & 88.40 ± 0.95 & 89.12 ± 0.55 & 90.14 ± 0.10 & 90.16 ± 0.18 & 87.81 ± 1.66 & 90.20 ± 0.29 & 90.21 ± 0.37 & 90.25 ± 0.38 & 89.06 ± 0.75 \\
0.20  & 89.19 ± 0.22 & 89.65 ± 0.60 & 89.59 ± 0.69 & 90.06 ± 0.04 & 89.57 ± 0.34 & 89.91 ± 0.28 & 90.28 ± 0.55 & 89.80 ± 0.28 & 88.89 ± 0.76 \\
0.30  & 88.93 ± 0.83 & 88.77 ± 1.07 & 90.23 ± 0.09 & 89.88 ± 0.59 & 89.14 ± 0.19 & 90.25 ± 0.09 & 89.97 ± 0.26 & 90.46 ± 0.41 & 89.06 ± 0.36 \\
0.40  & 88.28 ± 1.08 & 89.38 ± 0.53 & 90.50 ± 0.23 & 89.79 ± 0.67 & 88.20 ± 0.31 & 90.51 ± 0.12 & 90.37 ± 0.24 & 90.23 ± 0.14 & 10.00 ± 0.00 \\
0.50  & 88.96 ± 0.82 & 89.03 ± 0.59 & 90.46 ± 0.60 & 90.38 ± 0.25 & 88.67 ± 0.23 & 89.54 ± 0.86 & 90.47 ± 0.52 & 90.19 ± 0.31 & 10.00 ± 0.00 \\
0.60  & 88.15 ± 0.68 & 89.47 ± 0.18 & 89.95 ± 0.30 & 90.32 ± 0.25 & 88.82 ± 0.32 & 90.02 ± 0.40 & 90.18 ± 0.33 & 90.14 ± 0.36 & 10.00 ± 0.00 \\
0.70  & 88.02 ± 0.53 & 89.63 ± 0.44 & 89.69 ± 0.42 & 89.23 ± 0.19 & 89.62 ± 0.81 & 89.85 ± 0.08 & 90.01 ± 0.34 & 10.00 ± 0.00 & 10.00 ± 0.00 \\
0.80  & 88.28 ± 0.34 & 89.62 ± 0.91 & 85.72 ± 0.63 & 89.39 ± 0.43 & 88.82 ± 0.14 & 10.00 ± 0.00 & 88.29 ± 0.11 & 10.00 ± 0.00 & 10.00 ± 0.00 \\
0.90  & 85.82 ± 0.19 & 89.29 ± 0.79 & 10.00 ± 0.00 & 80.85 ± 0.62 & 24.28 ± 20.2 & 10.00 ± 0.00 & 10.00 ± 0.00 & 10.00 ± 0.00 & 10.00 ± 0.00 \\
0.95  & 84.41 ± 0.05 & 10.00 ± 0.00 & 10.00 ± 0.00 & 10.00 ± 0.00 & 10.00 ± 0.00 & 10.00 ± 0.00 & 10.00 ± 0.00 & 10.00 ± 0.00 & 10.00 ± 0.00 \\
0.98  & 80.04 ± 0.90 & 10.00 ± 0.00 & 10.00 ± 0.00 & 10.00 ± 0.00 & 10.00 ± 0.00 & 10.00 ± 0.00 & 10.00 ± 0.00 & 10.00 ± 0.00 & 10.00 ± 0.00 \\
0.99  & 76.89 ± 0.26 & 10.00 ± 0.00 & 10.00 ± 0.00 & 10.00 ± 0.00 & 10.00 ± 0.00 & 10.00 ± 0.00 & 10.00 ± 0.00 & 10.00 ± 0.00 & 10.00 ± 0.00 \\
\bottomrule
\end{tabular}}
\end{sc}
\end{small}
\end{center}
\vskip -0.1in
\end{table}
\newpage
%------------------------------------------------------------------------------------------------
\begin{table*}[h]
\caption{Performance of different compression methods evaluated after 1 warmup epoch using VGG19 on the CIFAR-10 dataset. We report the mean accuracy between three initialization seeds across various sparsity levels. Baseline, no pruning: $89.21 \pm 0.22$.}
\label{tab:VGG19_cifar10_compressors_warmup1}
\vskip 0.15in
\begin{center}
\begin{small}
\begin{sc}
\resizebox{\textwidth}{!}{%
\begin{tabular}{lccccc|cccc}
\toprule
Sparsity  & Random & Magnitude & GN & SNIP & GraSP & FD & FP & FTS & FBSS \\
\midrule
0.80  & 88.73 ± 0.38 & 88.35 ± 0.54 & 86.76 ± 0.27 & 87.39 ± 0.66 & 87.24 ± 0.25 & 87.14 ± 0.45 & 87.00 ± 0.87 & 87.68 ± 0.33 & 64.33 ± 15.91 \\
0.90  & 87.26 ± 0.42 & 88.62 ± 0.49 & 85.96 ± 0.75 & 86.75 ± 0.76 & 87.47 ± 0.33 & 86.69 ± 0.72 & 87.09 ± 0.31 & 87.42 ± 0.21 & 46.16 ± 7.62 \\
0.95  & 85.47 ± 0.64 & 87.68 ± 0.49 & 86.66 ± 0.27 & 86.00 ± 1.10 & 86.71 ± 1.24 & 85.71 ± 1.35 & 86.73 ± 0.36 & 87.56 ± 0.62 & 46.30 ± 5.32 \\
0.98  & 80.44 ± 0.30 & 86.61 ± 0.62 & 84.72 ± 1.69 & 87.22 ± 0.23 & 86.45 ± 0.64 & 80.34 ± 6.43 & 86.07 ± 0.39 & 86.36 ± 0.29 & 49.05 ± 4.31 \\
0.99  & 77.24 ± 0.73 & 83.69 ± 1.36 & 80.28 ± 2.04 & 83.49 ± 1.77 & 85.39 ± 0.43 & 75.11 ± 7.80 & 84.40 ± 1.27 & 85.35 ± 1.05 & 47.10 ± 4.41 \\
\bottomrule
\end{tabular}}
\end{sc}
\end{small}
\end{center}
\vskip -0.1in
\end{table*} 
%------------------------------------------------------------------------------------------------

\begin{table}[h]
\caption{Performance of different sensitivity methods for pruning evaluated after 5 warmup epochs using VGG19 on the CIFAR-10 testset. The right side of the table presents our proposed criteria. The mean accuracy and standard deviation are reported across three initialization seeds for various sparsity levels. Baseline, no pruning: $89.21 \pm 0.22$.}
\label{tab:VGG19_cifar10_compressors_warmup5}
\vskip 0.15in
\begin{center}
\begin{small}
\begin{sc}
\resizebox{\textwidth}{!}{%
\begin{tabular}{lccccc|cccc}
\toprule
Sparsity  & Random & Magnitude & GN & SNIP & GraSP & FD & FP & FTS & FBSS \\
\midrule
0.80  & 88.84 ± 0.43 & 88.41 ± 0.47 & 87.58 ± 0.52 & 88.15 ± 1.09 & 86.77 ± 1.14 & 87.28 ± 0.90 & 88.22 ± 0.82 & 86.68 ± 0.61 & 70.52 ± 9.25 \\
0.90  & 87.56 ± 0.62 & 88.60 ± 0.93 & 86.73 ± 0.37 & 87.89 ± 0.25 & 87.10 ± 0.47 & 87.50 ± 1.42 & 88.18 ± 0.47 & 86.98 ± 0.14 & 47.78 ± 1.26 \\
0.95 & 85.51 ± 0.69 & 87.66 ± 1.19 & 87.44 ± 0.46 & 87.71 ± 0.82 & 87.05 ± 0.16 & 86.83 ± 1.47 & 87.36 ± 0.52 & 87.00 ± 0.74 & 48.83 ± 2.52 \\
0.98 & 82.09 ± 0.17 & 86.24 ± 0.52 & 84.66 ± 1.33 & 86.55 ± 0.84 & 86.04 ± 0.66 & 85.44 ± 0.64 & 86.64 ± 0.13 & 84.89 ± 0.51 & 49.48 ± 0.85 \\
0.99 & 77.22 ± 1.03 & 83.93 ± 1.80 & 81.62 ± 2.17 & 84.53 ± 0.70 & 81.33 ± 5.77 & 81.71 ± 1.41 & 85.02 ± 0.69 & 83.78 ± 0.80 & 41.24 ± 1.55 \\
\bottomrule
\end{tabular}}
\end{sc}
\end{small}
\end{center}
\vskip -0.1in
\end{table}

%------------------------------------------------------------------------------------------------

\newpage
\section{Results CIFAR100}
\subsection{ResNet18}
\label{sec:resnet_cifar-100}

CIFAR-100 results exhibit a similar trend to those observed on CIFAR-10, further reinforcing the robustness of our proposed Fisher-Taylor Sensitivity (FTS) criterion. Across all evaluated sparsity levels, FTS consistently maintains strong performance, frequently ranking among the top-performing methods. This trend is particularly evident at extreme sparsities, where many pruning approaches suffer significant performance degradation. The stability of FTS across both datasets highlights its effectiveness in preserving network expressivity despite aggressive pruning.

\begin{table}[h]
\caption{Performance of different compression methods evaluated using ResNet18 on the CIFAR-100 dataset. We report the mean accuracy between three initialization seeds across various sparsity levels. Baseline, no pruning: $69.57 \pm 0.19$.}
\label{tab:resnet18_cifar100_compressors}
\vskip 0.15in
\begin{center}
\begin{small}
\begin{sc}
\resizebox{\textwidth}{!}{%
\begin{tabular}{lccccc|cccc}
\toprule
Sparsity  & Random & Magnitude & GN & SNIP & GraSP & FD & FP & FTS & FBSS \\
\midrule
0.10  & 69.16 ± 0.11 & 69.37 ± 0.14 & 69.63 ± 0.34 & 69.42 ± 0.07 & 64.26 ± 0.27 & 69.66 ± 0.30 & 69.08 ± 0.21 & 69.16 ± 0.11 & 69.07 ± 0.10 \\
0.20  & 69.16 ± 0.30 & 69.06 ± 0.24 & 69.19 ± 0.11 & 69.30 ± 0.08 & 63.28 ± 0.58 & 69.60 ± 0.30 & 69.35 ± 0.35 & 69.41 ± 0.43 & 69.07 ± 0.20 \\
0.30  & 69.36 ± 0.18 & 68.58 ± 0.36 & 69.37 ± 0.13 & 68.82 ± 0.17 & 62.02 ± 0.43 & 69.24 ± 0.40 & 68.84 ± 0.13 & 68.80 ± 0.55 & 68.96 ± 0.11 \\
0.40  & 69.41 ± 0.20 & 68.50 ± 0.29 & 69.16 ± 0.26 & 68.95 ± 0.19 & 61.18 ± 0.19 & 69.17 ± 0.16 & 68.88 ± 0.25 & 69.02 ± 0.21 & 68.92 ± 0.25 \\
0.50  & 69.12 ± 0.46 & 68.17 ± 0.20 & 68.94 ± 0.20 & 68.63 ± 0.11 & 61.11 ± 0.40 & 69.13 ± 0.13 & 68.68 ± 0.12 & 68.71 ± 0.12 & 68.71 ± 0.57 \\
0.60  & 68.66 ± 0.27 & 67.78 ± 0.35 & 68.77 ± 0.17 & 68.63 ± 0.42 & 61.40 ± 0.78 & 68.34 ± 0.43 & 67.98 ± 0.23 & 68.41 ± 0.14 & 68.60 ± 0.15 \\
0.70  & 67.95 ± 0.43 & 67.51 ± 0.24 & 68.29 ± 0.39 & 68.08 ± 0.18 & 59.43 ± 0.76 & 68.03 ± 0.46 & 67.96 ± 0.15 & 68.29 ± 0.06 & 68.16 ± 0.07 \\
0.80  & 67.26 ± 0.48 & 66.55 ± 0.19 & 67.20 ± 0.37 & 67.21 ± 0.38 & 59.08 ± 0.22 & 66.70 ± 0.05 & 67.05 ± 0.06 & 66.77 ± 0.65 & 66.62 ± 0.43 \\
0.90  & 64.75 ± 0.16 & 64.48 ± 0.18 & 64.87 ± 0.27 & 65.70 ± 0.08 & 59.16 ± 0.91 & 64.74 ± 0.44 & 65.46 ± 0.30 & 65.41 ± 0.13 & 63.90 ± 0.31 \\
0.95  & 61.01 ± 0.32 & 62.20 ± 0.06 & 62.20 ± 0.23 & 63.20 ± 0.20 & 57.91 ± 0.09 & 62.14 ± 0.42 & 63.22 ± 0.25 & 63.21 ± 0.47 & 61.25 ± 0.44 \\
0.98  & 54.72 ± 0.22 & 55.44 ± 0.18 & 57.34 ± 0.31 & 58.83 ± 0.35 & 54.85 ± 0.35 & 55.57 ± 0.17 & 58.05 ± 0.18 & 58.59 ± 0.12 & 55.02 ± 0.34 \\
0.99  & 45.62 ± 0.55 & 40.39 ± 0.36 & 50.46 ± 0.61 & 52.96 ± 0.10 & 49.13 ± 0.19 & 48.02 ± 0.32 & 49.98 ± 0.60 & 52.85 ± 0.24 & 44.91 ± 0.52 \\
\bottomrule
\end{tabular}}
\end{sc}
\end{small}
\end{center}
\vskip -0.1in
\end{table}

%------------------------------------------------------------------------------------------------
\clearpage
\subsection{VGG19}
The results on VGG19 with CIFAR-100 exhibit a similar trend to those observed on CIFAR-10, reinforcing the effectiveness of our proposed approach. Once again, we identify the occurrence of layer collapse at extreme sparsities when no warm-up is applied, leading to a significant drop in accuracy. Introducing a single warm-up epoch effectively resolves this issue, restoring pruning performance across all evaluated criteria. However, increasing the warm-up phase to five epochs does not yield any additional advantage, indicating that a brief warm-up period is sufficient to stabilize gradient-based importance scores and prevent collapse.

\label{sec:vgg_cifar-100}

\begin{table}[h]
\caption{Performance of different compression methods evaluated using VGG19 on the CIFAR-100 dataset. We report the mean accuracy between three initialization seeds across various sparsity levels. Baseline, no pruning: $58.96 \pm 2.30$.}
\label{tab:VGG19_cifar100_compressors}
\vskip 0.15in
\begin{center}
\begin{small}
\begin{sc}
\resizebox{\textwidth}{!}{%
\begin{tabular}{lccccc|cccc}
\toprule
Sparsity & Random & Magnitude & GN & SNIP & GraSP & FD & FP & FTS & FBSS \\
\midrule
0.10  & 60.31 ± 0.40 & 59.13 ± 1.29 & 61.93 ± 0.48 & 61.98 ± 0.29 & 59.32 ± 0.63 & 62.13 ± 0.61 & 60.45 ± 3.47 & 61.56 ± 1.04 & 58.79 ± 0.98 \\
0.20  & 60.43 ± 1.14 & 59.27 ± 0.34 & 62.64 ± 0.21 & 62.68 ± 0.24 & 61.21 ± 0.41 & 63.04 ± 0.43 & 62.71 ± 1.02 & 62.24 ± 0.44 & 60.48 ± 0.48 \\
0.30  & 58.32 ± 0.60 & 59.35 ± 1.43 & 62.61 ± 0.23 & 63.11 ± 0.35 & 59.30 ± 0.43 & 62.85 ± 0.42 & 61.43 ± 0.61 & 62.65 ± 0.54 & 58.77 ± 1.02 \\
0.40  & 56.50 ± 3.20 & 60.04 ± 1.02 & 62.36 ± 0.02 & 62.39 ± 0.55 & 56.34 ± 1.49 & 62.38 ± 0.75 & 61.56 ± 1.25 & 62.67 ± 0.06 & 1.00 ± 0.00 \\
0.50  & 58.47 ± 1.49 & 61.49 ± 1.22 & 62.02 ± 0.64 & 62.76 ± 0.50 & 54.43 ± 0.84 & 62.84 ± 0.33 & 62.25 ± 0.33 & 62.47 ± 0.42 & 1.00 ± 0.00 \\
0.60  & 57.54 ± 0.74 & 61.50 ± 0.30 & 62.55 ± 0.13 & 63.08 ± 0.55 & 56.76 ± 0.69 & 62.40 ± 0.57 & 62.70 ± 0.63 & 62.17 ± 0.23 & 1.00 ± 0.00 \\
0.70  & 57.63 ± 0.80 & 61.71 ± 0.25 & 60.85 ± 0.79 & 60.58 ± 0.39 & 57.76 ± 0.84 & 60.44 ± 0.34 & 60.92 ± 0.41 & 60.51 ± 1.67 & 1.00 ± 0.00 \\
0.80  & 57.84 ± 0.57 & 61.89 ± 1.02 & 55.09 ± 0.49 & 59.84 ± 0.29 & 58.39 ± 0.74 & 1.00 ± 0.00 & 43.16 ± 1.02 & 58.66 ± 2.28 & 1.00 ± 0.00 \\
0.90  & 58.41 ± 0.41 & 62.60 ± 0.91 & 1.00 ± 0.00 & 8.35 ± 10.39 & 42.88 ± 1.64 & 1.00 ± 0.00 & 1.00 ± 0.00 & 8.87 ± 11.13 & 1.00 ± 0.00 \\
0.95  & 54.84 ± 1.08 & 1.00 ± 0.00 & 1.00 ± 0.00 & 1.00 ± 0.00 & 1.00 ± 0.00 & 1.00 ± 0.00 & 1.00 ± 0.00 & 1.00 ± 0.00 & 1.00 ± 0.00 \\
0.98  & 50.21 ± 0.72 & 1.00 ± 0.00 & 1.00 ± 0.00 & 1.00 ± 0.00 & 1.00 ± 0.00 & 1.00 ± 0.00 & 1.00 ± 0.00 & 1.00 ± 0.00 & 1.00 ± 0.00 \\
0.99  & 46.69 ± 0.45 & 1.00 ± 0.00 & 1.00 ± 0.00 & 1.00 ± 0.00 & 1.00 ± 0.00 & 1.00 ± 0.00 & 1.00 ± 0.00 & 1.00 ± 0.00 & 1.00 ± 0.00 \\
\bottomrule
\end{tabular}}
\end{sc}
\end{small}
\end{center}
\vskip -0.1in
\end{table}

%------------------------------------------------------------------------------------------------

\begin{table}[h]
\caption{Performance of different compression methods evaluated after 1 warmup epoch using VGG19 on the CIFAR-100 dataset. We report the mean accuracy between three initialization seeds across various sparsity levels. Baseline, no pruning: $58.96 \pm 2.30$.}
\label{tab:VGG19_cifar100_compressors_warmup1}
\vskip 0.15in
\begin{center}
\begin{small}
\begin{sc}
\resizebox{\textwidth}{!}{%
\begin{tabular}{lccccc|cccc}
\toprule
Sparsity & Random & Magnitude & GN & SNIP & GraSP & FD & FP & FTS & FBSS \\
\midrule
0.80  & 60.39 ± 1.16 & 58.91 ± 0.41 & 52.81 ± 1.32 & 55.62 ± 2.27 & 55.15 ± 2.25 & 56.71 ± 0.31 & 58.03 ± 0.93 & 52.41 ± 3.07 & 52.74 ± 5.16 \\
0.90  & 58.90 ± 0.98 & 60.95 ± 0.81 & 50.56 ± 4.59 & 55.89 ± 2.05 & 56.01 ± 1.58 & 52.07 ± 3.24 & 53.65 ± 0.57 & 52.45 ± 3.75 & 19.65 ± 1.68 \\
0.95  & 56.10 ± 0.85 & 57.64 ± 2.63 & 50.34 ± 1.00 & 53.70 ± 3.60 & 56.16 ± 0.41 & 54.44 ± 1.38 & 53.24 ± 3.54 & 53.56 ± 1.26 & 17.24 ± 0.44 \\
0.98  & 50.97 ± 0.40 & 54.66 ± 2.56 & 43.43 ± 5.32 & 50.19 ± 1.59 & 54.64 ± 1.50 & 42.75 ± 1.91 & 50.59 ± 3.39 & 48.56 ± 5.25 & 16.42 ± 0.64 \\
0.99  & 46.52 ± 0.45 & 43.33 ± 5.83 & 33.90 ± 5.35 & 42.65 ± 5.32 & 45.98 ± 4.48 & 29.67 ± 8.49 & 49.11 ± 3.46 & 48.70 ± 2.59 & 13.25 ± 0.84 \\
\bottomrule
\end{tabular}}
\end{sc}
\end{small}
\end{center}
\vskip -0.1in
\end{table}


%------------------------------------------------------------------------------------------------

\begin{table}[h]
\caption{Performance of different compression methods evaluated after 5 warmup epochs using VGG19 on the CIFAR-100 dataset. We report the mean accuracy between three initialization seeds across various sparsity levels. Baseline, no pruning: $58.96 \pm 2.30$.}
\label{tab:VGG19_cifar100_compressors_warmup5}
\vskip 0.15in
\begin{center}
\begin{small}
\begin{sc}
\resizebox{\textwidth}{!}{%
\begin{tabular}{lccccc|cccc}
\toprule
Sparsity & Random & Magnitude & GN & SNIP & GraSP & FD & FP & FTS & FBSS \\
\midrule
0.80  & 60.41 ± 1.39 & 58.38 ± 0.85 & 60.86 ± 0.79 & 61.63 ± 0.45 & 56.25 ± 0.49 & 59.59 ± 0.76 & 59.37 ± 3.50 & 60.86 ± 0.53 & 46.93 ± 9.04 \\
0.90  & 60.32 ± 0.09 & 57.74 ± 1.64 & 57.77 ± 2.41 & 58.23 ± 4.07 & 56.27 ± 1.02 & 60.19 ± 0.63 & 61.23 ± 0.50 & 60.52 ± 0.37 & 21.66 ± 1.95 \\
0.95 & 57.86 ± 0.53 & 59.55 ± 1.15 & 56.09 ± 0.97 & 58.83 ± 0.65 & 55.26 ± 1.25 & 55.80 ± 2.77 & 59.83 ± 0.94 & 58.52 ± 1.32 & 19.98 ± 2.62 \\
0.98 & 51.75 ± 0.43 & 47.75 ± 7.63 & 52.26 ± 4.06 & 55.27 ± 1.69 & 54.59 ± 0.96 & 49.46 ± 4.98 & 57.40 ± 1.26 & 56.00 ± 1.08 & 17.59 ± 1.36 \\
0.99 & 47.59 ± 0.80 & 42.46 ± 7.95 & 46.58 ± 2.00 & 53.13 ± 0.84 & 53.91 ± 1.53 & 42.87 ± 4.63 & 53.17 ± 1.18 & 53.05 ± 2.14 & 13.92 ± 0.14 \\
\bottomrule
\end{tabular}}
\end{sc}
\end{small}
\end{center}
\vskip -0.1in
\end{table}


%------------------------------------------------------------------------------------------------
\clearpage

\section{Mask Batch Size for Other Sparsities}
The Effect of batch size on pruning performance across different sparsities. 
As sparsity increases, the effect of batch size on pruning performance becomes more pronounced. 
At lower sparsities (0.90, 0.95), the differences across batch sizes are less evident, suggesting that even smaller batches provide a reasonable estimation of parameter importance. However, at extreme sparsities (0.98, 0.99), we observe a clear trend where larger batch sizes consistently lead to better parameter selection, ultimately improving accuracy. This aligns with our hypothesis that larger batches help reduce variance in gradient estimation, leading to more stable and effective pruning decisions. 
\label{batch_size_heatmaps}

\begin{figure}[h]
    \centering
    \includegraphics[width=0.8\linewidth]{imgs/cifar10_resnet18_heatmap_warmup_0.png}
    \caption{Effect of batch size on pruning performance at increasing sparsities.}
    \label{fig:enter-label}
\end{figure}

%------------------------------------------------------------------------------------------------

\clearpage
\section{Comparison of our criteria with magnitude-based pruning}

Figure \ref{fig:our_criterion_vs_magnitude} illustrates the relationship between parameter magnitude and different sensitivity-based pruning metrics. Each point represents a model parameter, with red points indicating the top-ranked parameters selected for retention by each criterion. The green dashed line marks the 99th percentile of parameter magnitudes.

A key observation is that the most effective pruning criteria, such as Fisher-Taylor Sensitivity, tend to retain parameters with a broad range of magnitudes, including many that are relatively small (left of the green line). This shows that the estimated importance does not always prioritize parameters based on their magnitude. 


\begin{figure}[htp]
    \centering
    \includegraphics[width=0.9\linewidth]{imgs/cifar_10_mag_vs_criteria_s_99.png}
    \caption{Our criteria vs. Magnitude parameter selection for 99\% sparsity (ResNet18, CIFAR-10, Seed 0)} 
    \label{fig:our_criterion_vs_magnitude}
\end{figure}


\end{document}
