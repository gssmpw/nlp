\section{Preliminaries}
\label{sec:preliminaries}

Let $P \subset \mathbb{R}^2$ be a set of points in the plane.
We denote the Euclidean distance between two points $u,v \in P$ by $d(u,v)$.
For a connected geometric graph $G = (P, E)$ with $E \subseteq {P \choose 2}$, we denote the Euclidean shortest path between two points $u,v \in P$ by $\pi_G(u,v)$ and its length by $|\pi_G(u,v)|$,
omitting $G$ if it is clear from context.
The dilation $\rho_G(u,v)$ between two points $u,v$ in $G$ is the ratio $\rho_G(u,v) := \frac{|\pi_G(u,v)|}{d(u,v)}$ between the shortest path length and the Euclidean distance.
The dilation $\rho(G)$ of the graph $G$ is defined as the maximum dilation between any two points in $P$,
i.e., $\rho(G) := \max \{ \rho_G(u,v) \mid u,v \in P, u \neq v\}.$

In the remainder of this work, the graph $G$ we consider is a triangulation, i.e., a maximal crossing-free graph on $P$.
Two edges $e_1 = (p_1, q_1), e_2 = (p_2, q_2)$ are said to \emph{cross} or \emph{intersect} iff the line segments they induce intersect in their interior.
Given a point set $P$, the Minimum Dilation Triangulation problem (MDT) asks to find a triangulation $T$ of $P$ minimizing $\rho(T)$.
