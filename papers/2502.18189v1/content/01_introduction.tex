\section{Introduction}
\label{sec:introduction}
Triangulating a set of points is one of the classical problems in computational
geometry. On the practical side, it has natural applications in wireless sensor
networks~\cite{DBLP:journals/comcom/WuLC07,DBLP:conf/infocom/ZhouWXJD11}, mesh
generation~\cite{bern1995mesh}, computer
vision~\cite{DBLP:conf/evoW/Vite-SilvaCTF07}, geographic information
systems~\cite{DBLP:journals/gis/Tsai93} and many other
areas~\cite{DBLP:books/lib/BergCKO08}. On the theoretical side,
finding a triangulation that is optimal with respect to some 
objective function has also received considerable attention:
The Delaunay triangulation maximizes the minimum angle and minimizes the
maximum circumcircle of all triangles.  Minimizing the
maximum edge length is possible in quadratic
time~\cite{DBLP:journals/siamcomp/EdelsbrunnerT93}.  On the other hand,
maximizing the minimum edge length is
\NP-complete~\cite{DBLP:journals/jocg/FeketeHHST18}. Famously,
Mulzer and Rote~\cite{DBLP:journals/jacm/MulzerR08} showed that 
computing the Minimum Weight Triangulation (MWT) 
is \NP-hard.

In this paper, we provide new results and insights for another natural
objective that considers triangulations as sparse structures with relative low
cost for ensuing detours: The \emph{dilation} of a triangulation $T$ of a point
set $P$ is the worst-case ratio (among all $s, t \in P$) between the shortest
$s$-$t$-path in $T$ and the Euclidean distance between $s$ and $t$.  The
Minimum Dilation Triangulation (MDT) problem asks for a triangulation that
minimizes the dilation~$\rho$, see~\cref{fig:example-instances} for examples.  This problem is closely related to the concept of a
Euclidean $t$-spanner: a subgraph 
with dilation (also called \emph{spanning ratio} or \emph{stretch
factor}~\cite{DBLP:journals/siamcomp/NarasimhanS00}) at most~$t$.  
Spanners have application in areas as robotics, network
design~\cite{DBLP:journals/tpds/AlzoubiLWWF03,DBLP:journals/dam/FarleyPZW04},
sensor networks~\cite{DBLP:journals/siamdm/CaiC95,DBLP:conf/sensys/FanLS06} and
design of parallel machines~\cite{DBLP:journals/comgeo/AronovBCGHSV08} and
have been studied extensively~\cite{DBLP:journals/ijcga/ChandraDNS95}.
Computing the MDT amounts to computing a \emph{plane} spanner with 
smallest spanning ratio, as every plane spanner can be extended into a
triangulation.  
For many types of triangulations, such as the Delaunay triangulation
or the MWT, both lower and upper bounds on the worst-case dilation are
known. Moreover, a constant upper bound on the worst-case dilation of a
triangulation also implies a constant-factor
approximation of the MDT.  Lower and upper bounds on the
worst-case dilation of the MDT have been studied, both for general
point sets and for special point sets such as regular polygons or points on a
circle; see \cref{subsec:related} for further details.

\begin{figure}
    \includegraphics[width=\linewidth]{figures/experiments/04_mdt_comparison/showcase_instances_2x2.pdf}
    \caption{MDT solutions for four instances. The red edges indicate a dilation-defining path.}
    \label{fig:example-instances}
\end{figure}

Despite this importance and attention, actually computing a Minimum Dilation
Triangulation is a challenging problem. Its computational
complexity is still unresolved, signaling that there may not be a simple and
elegant algorithmic solution that scales well. Moreover, actually computing
accurate dilations involves computing shortest paths under Euclidean distances
between $\Theta(n^2)$ pairs of points requires dealing with another famous 
challenge~\cite{o1981advanced,TOPP,bloemer1991computing,eisenbrand2024improved},
as it corresponds to evaluating and comparing numerous sums of many square roots.

\subsection{Our contributions}
We provide various new contributions, both to theory and practice.

\begin{enumerate}
\item We present practically robust methods and implementations for computing
\chreplaced{an optimal}{the} MDT for benchmark sets with up to \num{30000} points in reasonable time on
commodity hardware, based on new geometric insights into the structure of optimal
edge sets. 
%allowing subroutines with much improved complexity of $O(n^2\log n)$. 
Previous methods only achieved results for up to \num{200} points
(involving one computational routine of complexity $\Theta(n^4)$ 
instead of our improved complexity of $O(n^2\log n)$), so we extend
the range of practically solvable instances by a factor of \num{150}. 
%See \cref{sec:exact-algorithms}.
\item We develop scalable techniques for accurately evaluating many shortest-path queries 
that arise as large-scale sums of square roots, allowing us to certify exact optimal solutions.
This differs from previous work, which relied on floating-point computations, without regard for errors resulting from numerical issues.
\item We resolve an open problem from~\cite{DBLP:journals/ijcga/DumitrescuG16}
by establishing a lower bound of $\lbRhoShort$ on the dilation of the regular $\lbN$-gon
(and thus for arbitrary point sets).  
This improves the previous worst-case lower bound of $1.4308$ 
from the regular $23$-gon and greatly reduces the remaining gap to the upper
bound of $1.4482$ from~\cite{DBLP:journals/comgeo/SattariI19}.
In the process, we provide optimal solutions for regular $n$-gons up to $n=100$.
\end{enumerate}

\subsection{Related work}
\label{subsec:related}
The complexity of finding the MDT is unknown~\cite{DBLP:books/el/00/Eppstein00}.
%Giannopoulos~et~al.~
\cite{DBLP:journals/ijcga/GiannopoulosKKKM10} prove that finding the minimum dilation graph with a limited number of edges is \NP-hard.
%Cheong~et~al.~
\cite{DBLP:journals/comgeo/CheongHL08} show that finding a spanning tree of given dilation is also \NP-hard.
Kozma~\cite{DBLP:conf/esa/Kozma12} proves \NP-hardness for minimizing the expected distance between random points in a triangulation,
with edge weights instead of Euclidean distances.
For surveys, see 
Eppstein~\cite{DBLP:books/el/00/Eppstein00} until 2000
and~\cite{DBLP:books/daglib/0017763} for more recent results.

On the practical side, %Shariful~et~al.~
the authors of~\cite{DBLP:journals/corr/abs-2305-11312} study the problem of finding sparse low dilation graphs for large point sets in the plane.
%Buchin et al.~
The authors of~\cite{DBLP:conf/esa/BuchinBGW24} present an approximation algorithm for the related problem of improving a given graph with a budget of $k$ edges such that the dilation is minimized.
Regarding the MDT, all practical approaches in the literature are based on fixed-precision arithmetic.
Klein~\cite{klein2006effiziente} used an enumeration algorithm to find an optimal MDT for up to \num{10} points. 
%Dorz\'{a}n~et~al.~
The authors of~\cite{DBLP:journals/heuristics/DorzanLMH14} present heuristics for the MDT and evaluate their performance on instances with up to \num{200} points.
Instances with up to \num{70} points were solved by 
%Brandt~et~al.~
the authors of~\cite{DBLP:conf/cccg/BrandtGSR14} using integer linear programming techniques.
In their approach, the edge elimination strategy from Knauer~and~Mulzer~\cite{DBLP:conf/ewcg/KnauerM05} was used to eliminate edges from the complete graph.
Recently, Sattari~and~Izadi~\cite{DBLP:journals/jgo/SattariI17} presented an exact algorithm based on branch and bound that was evaluated on instances with up to \num{200} points.

The MDT is closely related to finding a plane $t$-spanner; see~\cite{mitchell2017proximity} for an overview.
Chew~\cite{DBLP:conf/compgeom/Chew86} first proved an upper bound of $\sqrt{10}$ on plane $t$-spanners in the $L_1$-metric,
which he later improved~\cite{DBLP:journals/jcss/Chew89} to $2$ for the triangular-distance Delaunay graph in the plane.
\cite{DBLP:journals/jocg/BiniazAMSBC16} proved that any convex point set admits a plane $1.88$-spanner. 
For a centrally symmetric convex point set containing $n$ points, Sattari and Izadi~\cite{DBLP:journals/ipl/SattariI18} give an upper bound of $\nicefrac{n}{2} \sin(\nicefrac{\pi}{n})$.
The best known upper bound on the dilation of arbitrary point sets is by Xia~\cite{DBLP:journals/siamcomp/Xia13}, 
who established and upper bound of \num{1.998} for the dilation of the Delaunay triangulation.

Mulzer~\cite{mulzer2004minimum} studied the MDT for the set of vertices of a regular $n$-gon and proved an upper bound of $1.48586$. 
Amarnadh~and~Mitra~\cite{DBLP:conf/iccsa/AmarnadhM06} improved this bound to $1.48454$ for any point set that lies on the boundary of a circle.
Sattari and Izadi~\cite{DBLP:journals/comgeo/SattariI19} again improved the bound to $1.4482$.
Dumitrescu and Ghosh~\cite{DBLP:journals/ijcga/DumitrescuG16} show that any triangulation of a regular $23$-gon has dilation at least $1.4308$,
improving upon the bound of $1.4161$ by Mulzer~\cite{mulzer2004minimum} and answering a question posed by Bose and Smit~\cite{DBLP:journals/comgeo/BoseS13} as well as Kanj~\cite{DBLP:conf/iccit/Kanj13}.
% exact formula is: Divide[2*sin\(40)2*Divide[pi,23]\(41) + sin\(40)8 Divide[pi,23]\(41),sin\(40)11 Divide[pi,23]\(41)]
Dumitrescu~and~Ghosh~\cite{DBLP:journals/ijcga/DumitrescuG16} also computed dilations of regular $22$-gon and regular $24$-gon.
