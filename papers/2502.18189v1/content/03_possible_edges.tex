\section{Candidate edge enumeration}
\label{sec:edge-enumeration}

Here we describe a novel and practically efficient scheme for
enumerating a set of edges that induces a supergraph of the MDT.
We start with the underlying theoretical ideas for this supergraph, followed
by an algorithm for computing a supergraph of any triangulation with dilation \emph{strictly less} than a given bound $\rho$,
which we exploit to enumerate a supergraph of the MDT with a (usually) small number of edges.
This is further adapted to reductions of the dilation bound $\rho$.
We also discuss the computation of lower bounds on the dilation of the MDT.
We defer the discussion of some implementation details to \cref{sec:implementation-details}.

\subsection{Theoretical background}
Our supergraph is based on the well-known \emph{ellipse property} (used in \cite{DBLP:conf/cccg/BrandtGSR14,DBLP:journals/ijcga/GiannopoulosKKKM10,DBLP:conf/ewcg/KnauerM05}) that all edges of a triangulation $T$ with dilation below $\rho$ must satisfy.

\begin{definition}
  A pair of points $s, t$ has the \emph{ellipse property} with 
  respect to a point set $P$ and a dilation bound $\rho$ if, for
  any pair of points $\ell, r \in P \setminus \{s,t\}$ such that
  $\ell r$ and $st$ intersect, $\min \{d(\ell,s) + d(s,r), d(\ell,t) + d(t,r)\} < \rho d(\ell,r)$. 
\end{definition}

Recall that the set of points that have the same sum of distances $\kappa$ to 
two points $\ell, r$ with $\kappa \geq d(\ell,r)$ is an \emph{ellipse} with $\ell, r$ as its focal points (or \emph{foci});
thus, all paths between $\ell$ and $r$ with length less than $\kappa = \rho \cdot d(\ell,r)$ must lie strictly inside this ellipse.
%with foci $\ell, r$ and \emph{focal distance sum} $\kappa$.

If a pair of points $s,t$ does not have the ellipse property, then there is a pair of points $\ell, r$
such that $st$ cuts through all paths between $\ell$ and $r$ that could have dilation less than $\rho$; see \cref{fig:ellipse-property-example}.
We call such a pair of points $\ell, r$ an \emph{exclusion certificate} for $s,t$.
\begin{figure}%
  \centering
  \includegraphics{./figures/ellipse-property-example.pdf}
  \caption{
    Any path connecting $\ell$ and $r$ with length below $\kappa = \rho d(\ell,r)$ must lie within the ellipse (dashed black lines).
    The edge $st$ does not have the \emph{ellipse property} as neither $s$ nor $t$ lie inside the ellipse;
    therefore, inserting $st$ makes connecting $\ell$ and $r$ by a sufficiently short path impossible.
  }
  \label{fig:ellipse-property-example}
\end{figure}%

\begin{observation}
  Let $\rho \geq 1$ be some dilation bound and let $T$ be a triangulation containing the edge $st$
  for points $s,t$ that do not have the ellipse property w.r.t.\ $P$ and $\rho$.
  Then, $\rho(T) \geq \rho$.
\end{observation}

\subsection{High-level description}
\label{sec:ellipse-filter-algo}
Preliminary experiments showed that a brute-force check of each of the $\Theta(n^4)$ pairs of
potential edges is impractical for large $n$, dominating the overall runtime time even for
early versions of our solution approach.

We therefore developed a more efficient scheme to enumerate a superset of the edges that satisfy the ellipse property.
This scheme performs a \emph{filtered incremental nearest-neighbor search} from each point $p \in P$
to identify all candidate edges of the form $pt$, i.e., looking for all possible \emph{neighbors} $t$ of $p$.
This search is efficiently implemented on a quadtree containing all points.
While enumerating candidates, we construct so-called \emph{dead sectors}, i.e., regions of the plane that
cannot contain possible neighbors of $p$.
We exclude all points that lie in dead sectors; we also use dead sectors to prune entire nodes of the quadtree
and to terminate the search early if it has become clear that all remaining points must be in a dead sector.
This usually avoids considering most points as potential neighbors of $p$ individually.
The algorithm has a runtime of $O(nk\log n)$, where $k$ is the average number of points and quadtree vertices considered individually from each point.
At worst, this can be $O(n^2\log n)$; in practice, $k$ is often much lower than $n$.
A related, slightly less complex enumeration algorithm applied to minimum-weight triangulations by Haas~\cite{haasmwt} scales to $10^8$ points.
In the following, we describe the components of the enumeration scheme in more detail.

\subsection{Dead sectors}
\label{sec:dead-sectors}
We begin by giving a definition of the dead sectors we use.
\begin{definition}
  Given a dilation $\rho$, a source point $p$ and two points $\ell, r$,
  the \emph{dead sector} $\mathcal{DS}_{\rho}(p, \ell, r) \subset \mathbb{R}^2$ is the region of all points $t$ such that 
  $pt$ intersects $\ell r$ and neither $p$ nor $t$ lie in the ellipse with foci $\ell, r$ and focal distance sum 
  $\kappa = \rho d(\ell, r)$.
\end{definition}
Depending on $p$, $\rho$, $\ell$ and $r$, $\mathcal{DS}_{\rho}(p, \ell, r)$ is either empty (if $p$ is in the ellipse)
or it is bounded by two rays and an elliptic arc; see \cref{fig:dead-sector-construction}.
In that case, it can also be interpreted as an elliptic arc and an interval of polar angles around $p$.

During our enumeration we construct many dead sectors,
the union of which can become quite complex, making it cumbersome and inefficient to work with directly.
We instead chose to simplify the shape of our dead sectors, giving up 
a small fraction of excluded area in exchange for a simple and efficient representation.
We replace the elliptic arc by a single disk centered on $p$,
whose radius is at least the maximum distance from $p$ to any point on the elliptic arc.
We can hence represent each non-empty simplified dead sector by a polar angle interval around $p$ 
and a single radius called \emph{activation distance} $A_{\rho}(p,\ell,r)$; see \cref{fig:dead-sector-construction}.

\begin{figure}
  \centering
  \includegraphics{./figures/dead_sector.pdf}
  \caption{A dead sector $\mathcal{DS}_{\rho}(p, \ell, r)$, 
  shaded in gray and red with the ellipse $E(\ell, r, \rho)$.
  We approximate the ellipse by a disk centered at $p$ with radius $\tilde{A}_{\rho}(p,\ell,r)$ (thereby ignoring the red area),
  which is the distance from $p$ to the farthest point $q$ of the rectangle 
  $B(\ell, r, \rho)$ in $\mathcal{DS}_{\rho}(p, \ell, r)$.}
  \label{fig:dead-sector-construction}
\end{figure}

We initially attempted to compute fairly precise upper bounds on the approximation distance;
however, due to computational and numerical issues described in \cref{sec:precise-activation-distances},
we decided to use a more robust and efficient approach.
Instead of using the elliptic arc,
we compute an upper bound $\tilde{A}_{\rho}(p,\ell,r)$ using a minimal rectangle $B(\ell, r, \rho)$ containing the ellipse with sides are parallel and perpendicular to $\ell r$.
We only need to check the extreme points of $B(\ell, r, \rho)$ and its intersections with the rays $L(p,\ell,r)$ and $R(p,\ell,r)$ to compute an upper bound $\tilde{A}_{\rho}(p, \ell, r)$; see \cref{fig:dead-sector-construction}.

In the worst case, these simplifications may lead to additional candidate edges. 
While this cannot make the resulting graph exclude any edges that satisfy the ellipse property,
it is still undesirable; we use additional checks for further reductions later on.
%e the
%number of candidate edges later on.

\subsection{Quadtree}
We use a quadtree containing all points in $P$ for the filtered incremental search.
The points are stored in a contiguous array $A_P$ outside the tree.
Each quadtree node $v$ is associated with a contiguous subrange of $A_P$ represented by two pointers.
This subrange contains the points in the subtree $\mathcal{T}_v$ rooted at $v$.
Each node also has a bounding box that contains all points in $\mathcal{T}_v$.
Each interior node has precisely four children; each leaf node contains at most a small constant number of points.
To allow the contiguity of the subranges, points are reordered during tree construction.
This has the added benefit of spatially sorting the points,
improving the probability that geometrically close points are near each other in memory~\cite{haasmwt}.

\subsection{Enumeration process}
One can think of the filtered incremental search from $p$ as a process of continuously growing a disk centered at $p$ starting with a radius of $0$.
As in the sweep-line paradigm, one encounters different types of events at discrete disk radii.
We primarily encounter events when the disk first touches a point of $P$ or the bounding box of a quadtree node.

Observe that, during a search from a point $p$, the dead sectors have two states:
either their activation distance is not yet reached, in which case they are \emph{inactive} and do not exclude any points,
or they are \emph{active} and exclude all points in a certain polar angle interval around $p$.
We therefore also introduce events when the disk radius reaches the activation distance of a dead sector.
This enables efficient management of active dead sectors as a set of polar angle intervals around $p$;
we discuss ensuing numerical issues in \cref{sec:exactness-implementation-issues}.

When we first encounter a point $t$, we have to determine whether $t$ is in any dead sector by checking the active dead sector data structure.
If it is not, we have to report it as potential neighbor of $p$, adding it to a set of points sorted by polar angle around $p$.
Furthermore, to construct new dead sectors, we combine $t$ with $O(1)$ other points of $P$;
which points we use is decided by a heuristic discussed in \cref{sec:dead-sector-construction-neighbors}.

When we encounter a quadtree node $v$, we have to determine whether $v$'s bounding box is fully contained 
in the union of all dead sectors and can thus be pruned; we thus again check the active dead sectors.
Otherwise, we have to take $v$'s children, or the points it contains if it is a leaf, into account;
they are then considered as future events.

\subsection{Initialization and postprocessing}
We initially compute the Delaunay triangulation of the point set $P$ and compute its dilation.
We also optionally attempt to improve the dilation of the triangulation by a simple improvement heuristic.
The heuristic is based on computing constrained Delaunay triangulations,
greedily adding shortcut edges as constraints to reduce the length of the path currently defining the dilation.
We then use the resulting dilation $\rho$ as bound for the enumeration process outlined in the previous sections,
enumerating only edges that could locally be in a triangulation with dilation strictly below $\rho$.

After the initial enumeration process is complete, we are left with a set of \emph{possible} edges and can safely ignore all other edges.
We postprocess these as follows.
For each possible edge $pq$, we compute the set of possible edges intersecting it.
We need this information later on to model the problem of finding a triangulation on the set of possible edges.
For each pair $st$, $\ell r$ of intersecting edges, we explicitly check whether either pair is an exclusion certificate for the other.
In many cases, this postprocessing gets us very close to the edge set that would be obtained by the trivial $\Theta(n^4)$ edge candidate enumeration algorithm;
see the experiment section for details.
We mark each edge that does not have intersecting possible edges as \emph{certain};
certain edges must be part of any triangulation with dilation less than $\rho$.

\subsection{Dilation thresholds}
We also compute a \emph{dilation threshold} $\vartheta(st)$ for each possible edge $st$.
Let $I(st)$ be the set of possible edges intersecting $st$.
For each $\ell r \in I(st)$, we can compute a lower bound on the dilation of the 
shortest path connecting $\ell$ to $r$, should $st$ be present, as \[\rho_{st}(\ell r) = \min \{d(\ell,s) + d(s,r), d(\ell,t) + d(t,r)\}/d(\ell,r);\]
see \cref{fig:dilation-thresholds}.
\begin{figure}
  \centering
  \includegraphics{./figures/edge_thresholds.pdf}
  \caption{If $st$ (black) is present, the pairs $\ell, r$ (resp.\ $\ell',r'$)
           can only be connected by paths that have at least the length of the solid blue (resp.\ orange) path.
           Hence, the ratio between the solid and dashed blue (resp. orange) paths is a lower bound on the dilation of any triangulation containing $st$.
           The maximum of such lower bounds for $st$ is the dilation threshold $\vartheta(st)$ of $st$.}
  \label{fig:dilation-thresholds}
\end{figure}
The dilation threshold of $st$ is then $\vartheta(st) = \max_{\ell r \in I(st)} \rho_{st}(\ell r)$.
\begin{observation}
  If an edge $st$ has dilation threshold $\vartheta(st)$, it is not
  in any triangulation with dilation $\rho < \vartheta(st)$.
\end{observation}
This allows us to quickly exclude further edges if we lower the dilation bound $\rho$ we aim for,
without reenumerating edges or recomputing intersections.

\subsection{Lower bounds}
We use the dilation thresholds to bound the minimum dilation.
For each edge $st$, either $st$ or some edge crossing it must be in any triangulation.
Thus, the minimum of all dilation thresholds among $\{st\} \cup I(st)$ is a lower bound on the minimum dilation.
Combining all edges, we obtain the following lower bound:
\[\rho(T) \geq \max_{st} \min \{\vartheta(pq) \mid pq \in \{st\} \cup I(st) \}.\]
We use interval arithmetic to compute a safe lower bound on this value in $O(1)$ time per intersection between two edges resulting from our enumeration.

Furthermore, by computing the points on the convex hull, we also know the number of edges $g$ of any triangulation.
This also gives us a lower bound on the minimum dilation by considering the $g$ lowest dilation thresholds. 
We also use Kruskal's algorithm to compute the lowest dilation threshold that admits a connected graph.
