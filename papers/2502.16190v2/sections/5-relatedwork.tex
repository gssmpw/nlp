\section{Related Works}
\subsection{Sketch-based NDV Estimation}
Sketch-based NDV estimation~\cite{harmouch2017cardinality,flajolet2007hyperloglog,ertl2023ultraloglog} represents an orthogonal approach to sampling-based NDV estimation. This line of research requires scanning all the data to maintain a memory-efficient sketch for NDV estimation, which may bring an unaffordable overhead~\cite{li2022sampling}. Furthermore, real-world databases may have data access restrictions, which makes sketch-based NDV estimation not applicable in many applications. 
\subsection{Sampling-based NDV Estimation}
\textbf{Traditional NDV Estimators}. Traditional methods explore statistical techniques to summarize heuristic rules to estimate NDV and they have been studied for over seven decades in Biology~\cite{valiant2013estimating,valiant2017estimating,mmo_bunge1993estimating}, Statistics~\cite{goodman1949estimation,chao1984nonparametric}, Networks~\cite{network_cohen2019cardinality,network_nath2008synopsis}, and Databases~\cite{spark_plan_code,pg_plan_code,mysql_join}. Representative traditional estimators make different assumptions, for example, they assume infinity population size~\cite{mmo_bunge1993estimating}, certain data distribution~\cite{motwani2006distinct,mmo_bunge1993estimating}, and data skewness~\cite{hybskew_haas1995sampling,gee_charikar2000towards}. Based on the assumptions, numerous estimators have been proposed to utilize the frequency profile of sample data to build linear polinomials~\cite{goodman1949estimation,gee_charikar2000towards,error_bound}, non-linear polynomials~\cite{chao_in_db_ozsoyoglu1991estimating,chaolee,chao1984nonparametric,shlosser1981estimation,burnham1978estimation,burnham1979robust,horvitz_sarndal1992model}, and solving non-linear equations~\cite{sichel1986parameter,sichel1986word,sichel1992anatomy,bootstrap_smith1984nonparametric,mmo_bunge1993estimating,hybskew_haas1995sampling} to estimate NDV, which have been intensively discussed in Section \ref{sec:exp-settings}. In addition, some works focused on the relation between the sampling size and the errors~\cite{valiant2017estimating,wu2019chebyshev,chien2021support}.

Since representative traditional estimators are based on different heuristics, so it is difficult for them to adapt to distribution shifting.


\noindent\textbf{Learned NDV Estimators}. The introduction of ML techniques for NDV estimation has recently emerged. Wu et al.~\cite{ls_wu2022learning} are the first to leverage ML models as a Learned Statistician (LS) for NDV estimation. They improve profile maximum likelihood~\cite{apml_acharya2017unified,pml_orlitsky2004modeling,apml_pavlichin2019approximate} methods and use neural networks to take data profiles of the sampled data as inputs to estimate NDV. Li et al.~\cite{li2024learning} introduced polynomial approximation techniques~\cite{hao2019unified,wu2019chebyshev} to learn the parameters of linear polynomials of frequency profile to estimate NDV. 


\subsection{Method Selection in Databases}



Selecting an optimal model from a fixed model set, as well as the ensembling multiple models, for specific database task scenarios, has emerged and garnered significant attention in recent years. Examples include identifying the proper learned cardinality estimation model for different datasets~\cite{autoce}, allocating a suitable budget for each data sampler~\cite{pengOneSizeDoes2022a}, and choosing the optimal knob tuning optimizer for each iteration~\cite{zhang2024efficient}. 
However, few studies have attempted to investigate how to select or ensemble existing NDV estimators to acquire better results. 

\begin{figure}[t]
    \centering
    \includegraphics[width=\linewidth]{figures/rate.png}
    \caption{{Performance under different sampling rates.}}
    \label{fig:samplingrates}
\end{figure}

