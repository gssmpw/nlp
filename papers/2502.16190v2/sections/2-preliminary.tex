\begin{figure*}[t]
    \centering
    \includegraphics[width=0.9\linewidth]{figures/training.pdf}
    \caption{Overview of \textsc{AdaNDV} on NDV estimation including training and inference data pipelines.}
    \label{fig:model_pipeline}
\end{figure*}

\section{Preliminaries}
\subsection{Problem Statement}\label{sec:statement}
Existing NDV estimation methods can be categorized into \textit{sketch-based}~\cite{harmouch2017cardinality,flajolet2007hyperloglog,ertl2023ultraloglog} and \textit{sampling-based}~\cite{goodman1949estimation,ls_wu2022learning,li2024learning}, while most of the former require scanning all the data, making them impractical in many scenarios. Here, we focus on sampling-based NDV estimation.


\noindent\textbf{NDV Estimation Definition.} Given a data column $C$ with $N$ rows, let $D$ be the NDV of column $C$. The task is to estimate $D$ by uniformly sampling $n$ $(n\leq N)$ rows (denoted $S$, and $S\subseteq  C$) from $C$. Let $r=n/N$ be the sampling rate, and we assume $N$ is known. We define $d$ as the NDV of $S$. 

Two features are widely discussed in sampling-based NDV estimation: \textit{frequency} and \textit{frequency profile}.


\noindent\textbf{Frequency.} The frequency of a value $x$ in $C$ is the number of times it appears in $C$. Denote $N_x=\sum_{i\in C} \mathds{1}_x(i)$, where $N_x$ is the frequency of value $x$ in $C$, $\mathds{1}_x(\cdot)$ is the indicator function that returns 1 if the input equals to $x$ and 0 otherwise. Similarly, $n_x=\sum_{i\in S} \mathds{1}_x(i)$ is the frequency of value $x$ in $S$. 


\noindent\textbf{Frequency Profile.} Frequency profile is the \textit{frequency of frequency}. Let the frequency profile of $C$ be $F=(F_j)_{j=1,2,\ldots,N}$, where $F_j=|\{i\in C|N_i=j\}|$. Similarly, the frequency profile of $S$ be $f=(f_j)_{j=1,2,\ldots,n}$, where $f_j=|\{i\in S|n_i=j\}|$.


\noindent\textbf{Feature Relations and Examples.} The two features are closely related to NDV and the number of rows. We can get $D=|\{i\in C|N_i>0\}|=\sum_{j=1}^N F_j$ and $d=|\{i\in S|n_i>0\}|=\sum_{j=1}^n f_j$. Besides, the total number of rows can be expressed as $N=\sum_{j=1}^N j\cdot F_j$, and $n=\sum_{j=1}^n j\cdot f_j$. 

For instance, suppose the sample data is $S=\{a, a, a, b, b, b, c, c, d\}$, we can observe that the sample size is $|S|=9$, and $d=4$ (there are $a,b,c,d$ four distinct values). The frequency of $S$ is $\{n_a=3,n_b=3,n_c=2,n_d=1\}$, and the frequency of frequency is $\{f_1=1,f_2=1,f_3=2,f_i=0,i=4,\ldots,9\}$. Based on these features, we can get $d=\sum_{i=1}^9f_i=4$, and $|S|=\sum_{i=1}^9i\cdot f_i=9$.

\subsection{NDV Estimators}
We use several representative estimators to demonstrate how they use the frequency profile of sample data to estimate NDV. 



\noindent\textbf{Traditional Estimators.}
Goodman~\cite{goodman1949estimation} is a representative \textit{linear polynomial} estimator with a sophisticated expression:
\begin{equation}
D_{\mathrm{Goodman}} =d+\sum_{i=1}^n(-1)^{i+1} \frac{(N-n+i-1) !(n-i) !}{(N-n-1) ! n !} f_i.
    \label{eq:goodman}
\end{equation}

Chao~\cite{chao1984nonparametric,chao_in_db_ozsoyoglu1991estimating} estimator has a \textit{nonlinear polynomial} expresion:
\begin{equation}
    D_{\mathrm{Chao}} = d + \frac{f_1^2}{2f_2}.
    \label{eq:chao}
\end{equation}

Besides, some estimators need to solve sophisticate \textit{non-linear equations} constructed by the frequency profiles~\cite{gee_charikar2000towards,sichel1986parameter,sichel1986word,sichel1992anatomy,mmo_bunge1993estimating}. For instance, Sichel~\cite{sichel1986parameter,sichel1986word,sichel1992anatomy} estimator needs to solve the following non-linear equations:


\begin{align}
    \begin{aligned}
        (1+g)\ln g-Ag+B=0,\frac{f_1}{n}<g<1, A=\frac{2n}{d}-\ln\frac{n}{f_1},\\
        B=\frac{2f_1}{d}+\ln \frac{n}{f_1},\hat{b}=\frac{g\ln \frac{ng}{f_1}}{1-g},\hat{c}=\frac{1-g^2}{ng^2},
        D_{\mathrm{Sichel}}=\frac{2}{\hat{b}\hat{c}}.\\
    \end{aligned}
    \label{eq:sichel}
\end{align}




\noindent\textbf{Learned Estimators.}
Recently, ML techniques have been introduced into NDV estimation~\cite{ls_wu2022learning,li2024learning}. Wu et al.~\cite{ls_wu2022learning} proposed a learned statistician (LS in short) to estimate NDV. It constructs a multi-layer perception (MLP) as the estimator, which takes the cut-off of the frequency profile and some features as input and outputs the estimated NDV. It is trained in the regression paradigm, which minimizes the $L_2$ loss between the estimated NDV and the ground truth. 




\noindent\textbf{Evaluation Protocol.} 
Ratio-error, also known as q-error~\cite{q_error_moerkotte2009preventing}, is widely used to evaluate the performance of an estimator in database applications:
\begin{equation}
    \mathrm{q\text{-}error}=\max(\frac{\hat{D}}{D},\frac{D}{\hat{D}}),
    \label{eq:q-error}
\end{equation}
where $\hat{D}$ is the estimated NDV and $D$ is the ground truth NDV. The lower error represents the better performance.
