
% VLDB template version of 2020-08-03 enhances the ACM template, version 1.7.0:
% https://www.acm.org/publications/proceedings-template
% The ACM Latex guide provides further information about the ACM template

\documentclass[sigconf, nonacm, pdfa]{acmart}
\usepackage{bbm,dsfont,subcaption,pdfpages,graphicx,placeins,makecell}
\usepackage{multirow} 
\usepackage[ruled,linesnumbered]{algorithm2e}
\usepackage{enumitem}
%% The following content must be adapted for the final version
% paper-specific
\newcommand\vldbdoi{10.14778/3717755.3717769}
\newcommand\vldbpages{1104 - 1117}
% issue-specific
\newcommand\vldbvolume{18}
\newcommand\vldbissue{4}
\newcommand\vldbyear{2024}
% should be fine as it is
\newcommand\vldbauthors{\authors}
\newcommand\vldbtitle{\shorttitle} 
% leave empty if no availability url should be set
\newcommand\vldbavailabilityurl{https://github.com/bytedance/adandv}
% whether page numbers should be shown or not, use 'plain' for review versions, 'empty' for camera ready
\newcommand\vldbpagestyle{empty} 
\newcommand{\revise}[1]{\textcolor{blue}{#1}}





\begin{document}




\title{\textsc{AdaNDV}: Adaptive Number of Distinct Value Estimation via Learning to Select and Fuse Estimators}

%%
%% The "author" command and its associated commands are used to define the authors and their affiliations.




% \author{Xianghong Xu, Tieying Zhang, Xiao He, Haoyang Li, Rong Kang, Shuai Wang, Linhui Xu, Zhimin Liang, Shangyu Luo, Lei Zhang, Jianjun Chen}


\author{Xianghong Xu}
\affiliation{
  \institution{ByteDance}
  \city{Beijing}
  \country{China}
}
\email{xuxianghong@bytedance.com}

\author{Tieying Zhang}
\affiliation{
  \institution{ByteDance}
  \city{San Jose}
  \country{USA}
}
\email{tieying.zhang@bytedance.com}\authornote{Tieying Zhang corresponds to this work.}

\author{Xiao He}
\affiliation{
  \institution{ByteDance}
  \city{Hangzhou}
  \country{China}
}
\email{xiao.hx@bytedance.com}


\author{Haoyang Li}
\affiliation{
  \institution{ByteDance}
  \city{Beijing}
  \country{China}
}
\email{lihaoyang.cs@bytedance.com}


\author{Rong Kang}
\affiliation{
  \institution{ByteDance}
  \city{Beijing}
  \country{China}
}
\email{kangrong.cn@bytedance.com}

\author{Shuai Wang}
\affiliation{
  \institution{ByteDance}
  \city{Beijing}
  \country{China}
}
\email{wangshuai.will@bytedance.com}

\author{Linhui Xu}
\affiliation{
  \institution{ByteDance}
  \city{Beijing}
  \country{China}
}
\email{xulinhui@bytedance.com}

\author{Zhimin Liang}
\affiliation{
  \institution{ByteDance}
  \city{Beijing}
  \country{China}
}
\email{liangzhimin@bytedance.com}

\author{Shangyu Luo}
\affiliation{
  \institution{ByteDance}
  \city{San Jose}
  \country{USA}
}
\email{shangyu.luo@bytedance.com}

\author{Lei Zhang}
\affiliation{
  \institution{ByteDance}
  \city{Shenzhen}
  \country{China}
}
\email{zhanglei.michael@bytedance.com}


\author{Jianjun Chen}
\affiliation{
  \institution{ByteDance}
  \city{San Jose}
  \country{USA}
}
\email{jianjun.chen@bytedance.com}






%%
%% The abstract is a short summary of the work to be presented in the
%% article.
\begin{abstract}
Estimating the Number of Distinct Values (NDV) is fundamental for numerous data management tasks, especially within database applications.
However, most existing works primarily focus on introducing new statistical or learned estimators, while identifying the most suitable estimator for a given scenario remains largely unexplored. 
Therefore, we propose \textsc{AdaNDV}, a learned method designed to adaptively select and fuse existing estimators to address this issue. 
Specifically, (1) we propose to use learned models to distinguish between overestimated and underestimated estimators and then select appropriate estimators from each category. This strategy provides a complementary perspective by integrating overestimations and underestimations for error correction, thereby improving the accuracy of NDV estimation. 
(2) To further integrate the estimation results, we introduce a novel fusion approach that employs a learned model to predict the weights of the selected estimators and then applies a weighted sum to merge them. 
By combining these strategies, the proposed \textsc{AdaNDV} fundamentally distinguishes itself from previous works that directly estimate NDV.
Moreover, extensive experiments conducted on real-world datasets, with the number of individual columns being several orders of magnitude larger than in previous studies, demonstrate the superior performance of our method.
\end{abstract}




\maketitle

%%% do not modify the following VLDB block %%
%%% VLDB block start %%%
\pagestyle{\vldbpagestyle}
\begingroup\small\noindent\raggedright\textbf{PVLDB Reference Format:}\\
\vldbauthors. \vldbtitle. PVLDB, \vldbvolume(\vldbissue): \vldbpages, \vldbyear.\\
\href{https://doi.org/\vldbdoi}{doi:\vldbdoi}
\endgroup
\begingroup
\renewcommand\thefootnote{}\footnote{\noindent
This work is licensed under the Creative Commons BY-NC-ND 4.0 International License. Visit \url{https://creativecommons.org/licenses/by-nc-nd/4.0/} to view a copy of this license. For any use beyond those covered by this license, obtain permission by emailing \href{mailto:info@vldb.org}{info@vldb.org}. Copyright is held by the owner/author(s). Publication rights licensed to the VLDB Endowment. \\
\raggedright Proceedings of the VLDB Endowment, Vol. \vldbvolume, No. \vldbissue\ %
ISSN 2150-8097. \\
\href{https://doi.org/\vldbdoi}{doi:\vldbdoi} \\
}\addtocounter{footnote}{-1}\endgroup
%%% VLDB block end %%%

%%% do not modify the following VLDB block %%
%%% VLDB block start %%%
\ifdefempty{\vldbavailabilityurl}{}{
\vspace{.3cm}
\begingroup\small\noindent\raggedright\textbf{PVLDB Artifact Availability:}\\
The source code, data, and/or other artifacts have been made available at \url{\vldbavailabilityurl}.
\endgroup
}
%%% VLDB block end %%%






\section{Introduction}

% State of the world (robots for creative activites)
The term ``robot,'' originally signifying `forced labor,' has long been associated with labor and work. Robots have demonstrated their utility in various automated productive and social contexts, where the primary goals are improving productivity, safety, and fostering social interactions with humans~\cite{simoes2022designing, weidemann2021role, honig2018understanding}. However, an increasing number of cases feature using of robots in creative settings. Unlike productive contexts, where the focus is on efficiency and task completion~\cite{arents2022smart}, or social contexts, where communication and trust are prioritized~\cite{nam2020trust, saunderson2019robots}, creative environments prioritize artistic innovation and expression~\cite{hsueh2024counts}. This shift fundamentally alters the dynamics of human-robot interaction, redefining the roles and expectations for both humans and robots.

For instance, robots’ social behaviors are leveraged to support the generation and expression of creative ideas~\cite{hu2021exploring, sandoval2022human, alves2020creativity}, and programmable robotic movements and trajectories are employed to inspire artistic activities such as sketching~\cite{lin2020your}. These studies often engage participants from creative fields who possess limited prior experience with robotics, and are typically conducted in short-term, experimental settings. Consequently, the findings from these studies remain constrained since much can be learned from professional practitioners' experiences to inform system design such as digital fabrication~\cite{hirsch2023nothing}. There is a notable gap in research examining the long-term, active, and practical experience of integrating robotic systems into the creative processes. As a result, the deeper insights into how robots facilitate and shape creative processes, beyond simply augmenting human creativity, remain underexplored. In this study, we aim to better understand the impacts of robots on creative processes and outcomes.

As early as Leonardo da Vinci's 16th century ``Automaton,'' artists have explored the creative affordances of robotic systems~\cite{shanken2002cybernetics, pagliarini2009development, jeon2017robotic}. The artistic creation process typically encompasses various stages, including the exploration of materials and techniques, ongoing experimentation and iteration, and the continual refinement of the artists' insights into their creative subjects~\cite{lewis2023art, sturdee2022state}. Therefore, investigating the artistic process involving robots offers an opportunity to gain deeper insights into robots' creative potential. Robotic art, in particular, provides a compelling case for this exploration.

We define robotic art as artworks that utilize robotic or automated machines to create artistic experiences and tangible artifacts. One example is robotic installation art, in which robots are programmed to follow specific rules that embody the artist’s expression (\autoref{fig:teaser} (a)). Another example is responsive art, in which robots react to their environment, with behaviors that change over time or in response to spectators (\autoref{fig:teaser} (b)). Additionally, there are robotic creators, which possess a degree of agency, allowing them to collaborate with human artists and produce works that extend beyond mere replication of human-created art (\autoref{fig:teaser} (c) and (d)). As such, robotic art becomes a rich case for exploring human-machine interactions in creative contexts. Gaining a deeper understanding of how robots facilitate artistic expression can provide insights for designing computing systems to support creative activities~\cite{gomez2021robot}.

% Therefore, we did...
We draw on semi-structured, in-depth interviews with renowned professional robotic artists to investigate the use of robots in artistic practice. Specifically, our goal is to understand how artistic exploration of robotic systems challenges conventional assumptions about the functions of robots, such as their roles in automating repetitive tasks or serving human needs. We also explore the implications of robots in the artistic process and examine how creativity may emerge within robotic art. To address these interrelated inquiries, our study focuses on the practice of robotic art, posing the research question: \textit{How do robotic artists utilize robots in their artistic practice?} We approach this inquiry through the perspectives and experiences of robotic artists, who creatively design, modify, and repurpose robotic systems for artistic expression and exploration.

% The key findings are...
Our findings highlight the social, material, and temporal dimensions of artists' practices that shape their creativity and artistic outcomes. The creation of robotic art is largely a social process, as artists receive both explicit and implicit feedback through the audience's reactions and reception of their work. Simultaneously, the embodiment and malfunctions inherent to robotic systems drive artistic experimentation. The temporal processes of creation and exhibition, beyond just the final product, further enhance the creative value. Our empirical analysis presents how creativity emerges through the interplay of social, material, and temporal interactions among artists, robots, audiences, and the environment.

% The contributions of this work are...
We make two main contributions to HCI in this study. 
First, we elucidate the interactive mechanisms among key actors---human creators, machines, audiences, and environments---within the practice of robotic art, a topic that remains underexplored in HCI. Our findings reveal the significance of sociality (e.g., interactions between artists and audiences), materiality (e.g., the embodiment and malfunctions of robots), and temporality (e.g., the processes of creation and exhibition) in shaping creative values. We propose that these three facets are central to the creative process and facilitate the emergence of creativity in robotic art.
Second, drawing from the findings, we offer implications for \textit{socially informed}, \textit{material-attentive}, and \textit{process-oriented} creation with computing systems. We suggest leveraging these three aspects to enhance creativity and the creative experience. Specifically, we discuss the value of incorporating implicit audience feedback, designing with technical malfunctions, and focusing on the post-creation process to foster alternative creative experiences with machines~\cite{alter2010designing, juarez2022glitch}.




\section{Preliminaries and Notations}

% \subsection{Downstream Tasks in Genomic Sequence Modelling}

\subsection{Genomic Sequence Modeling} 
DNA is a polymer made of up four types of nucleotides: Adenine (\textit{A}), Thymine (\textit{T}), Guanine (\textit{G}), and Cytosine (\textit{C}). Let $ \mathbb{N}_4 = \{A, T, G, C\}$. A DNA sequence of length $T$, denoted as $\mathbf{x} = {(\mathbf{x}_1, \mathbf{x}_2,...,\mathbf{x}_T)} \in \mathbb{N}^T_4$, follows a natural distribution $\mathbf{x} \sim p_\theta(\mathbf{x})$. We use $p_{\hat{\theta}}(\mathbf{x})$ to represent an estimate to the true distribution. The dataset of unlabeled genomic sequences is given in the form of \( \{\mathbf{x}^{(i)}\}_{i=1}^N\).

Genomic sequence modeling aims to learn a function $f$ that maps input sequences to biological annotations using a labeled dataset \(\mathcal{D} = \{\mathbf{x}^{(i)}, \mathbf{y}^{(i)}\}_{i=1}^N\). The type of $y^{(i)}$ varies depending on the types of tasks: $y^{(i)}$ is a class label in \textbf{DNA Sequence Classification}~\cite{grevsova2023genomic,dalla2024nucleotide}. Or a real value vector in \textbf{Genomic Assay Prediction} tasks~\cite{avsec2021effective,linder2025predicting}. Current genomic sequence models typically follow a two-stage strategy. In the pretraining phase, we learn the data distribution $p_{\hat{\theta}}(x)$ on a unlabeled dataset from unlabeled data using losses such as masked language modeling (MLM)
$p(\mathbf{x}) = \prod_{t \in \mathcal{M}} p_\theta(\mathbf{x}_t \mid \mathbf{x}_{1:t-1}, \mathbf{x}_{t+1:T})$ 
or next token prediction (NTP) $p(\mathbf{x}) = \prod_{t=1}^{T}p_\theta(\mathbf{x}_t | \mathbf{x}_{1:t-1})$. 

%To adapt into downstream tasks, we update the posterior distribution of $\theta_{\text{task}}$ given a labeled dataset $\mathcal{D}$ and the pretrained $\hat{\theta}$.



% \subsection{Genomic Sequence Modeling}
% \textbf{What}\textit{ is the target}? Similar to earlier work of language modeling~\citep{wang2018glue,wang2019superglue}, the goal of genomic sequence modeling is to learn a function $f:\mathcal{D} \rightarrow \mathcal{Y} $, that maps input sequences to their corresponding biological annotations. Given a labelled dataset \(\mathcal{D} = \{\mathbf{x}^{(n)}, \mathbf{y}^{(n)}\}_{n=1}^N\), with the annotation $\mathbf{y}^{(i)}$. Current genomic tasks consist of three primary classes based on biological annotation type: \textit{i.)} \textbf{DNA Sequence Classification} includes tasks like regulatory region identification~\cite{grevsova2023genomic} and histone marker prediction~\citep{dalla2024nucleotide}. In this task, 
% $f$ directly map a given sequence $\mathbf{x}^{(i)}$ to a class label $y^{(i)}$. \textit{ii.)} \textbf{Genomic Assay Prediction}: this includes works from ~\cite{avsec2021effective,linder2025predicting}. $\mathbf{y}^{(n)}$ is a real value vector. Typically this task has longer sequence length $T$ ranging from 10K to 100K. \textit{iii.)} \textbf{Genetic Variant Prediction} standout as a separate tasks, taking the reference sequence, mutated sequence and metadata as input, to predict class labels (e.g., pathogenic vs. benign variants). Such predictions are critical for clinical applications, aiding in diagnosis and personalized treatment strategies. Also exsit other more diverse tasks. For example, there are general interests  in learning a direct function mapping from DNA to the function of it. 

% \textbf{How} \textit{is sequence modeling performed}? A function $f$ solving the above proposed question, can be directly learned end-to-end with a labeled dataset $\mathcal{D}$ using supervised methods like CNNs ~\cite{} and more recently ``pretrain - finetune'' paradigm, which is becoming the dominant paradigm for GFMs. It involves two stages in which we first learn the data distribution $p_{\hat{\theta}}$ on a unlabeled dataset (pretraining), and then subsequently learn the posterior distribution $p(f|\mathcal{D},\hat{\theta})$ (finetuning). This approach has driven the state-of-the-art performance in many models (e.g. HyenaDNA, DNABERT-2, Nucleotide transformer) across various genomic tasks. For instance, Genomic Pre-trained Network (GPN)~\citep{benegas2023dna} leveraged this approach and achieved state-of-the-art (SoTA) performance for genetic variant prediction. The parameters $\theta$ are typically obtained through different pretraining strategies, common approaches include masked language modeling (MLM)  and next token prediction (NTP), formalized in \cref{eq:1} and  \cref{eq:2}, respectively. Here, $\mathcal{M}$ denotes the set of indices for masked tokens, and the model learns to reconstruct the original sequence or predict subsequent tokens based on the unmasked context.

% \begin{equation} \label{eq:1}
% p^{\text{MLM}}(\mathbf{x}) = \prod_{i \in \mathcal{M}} p_\theta(\mathbf{x}_i \mid \mathbf{x}_{1:i-1}, \mathbf{x}_{i+1:T})
% \end{equation}
% \begin{equation} \label{eq:2}
%     p^{AR}(\mathbf{x}) = \prod_{i=1}^{T}p_\theta(\mathbf{x}_i | \mathbf{x}_{1:i-1}).
% \end{equation}

% \paragraph{\textit{Limitations} of Existing Models} Existing genomic foundation models vary significantly in architecture and tokenization strategies, but fundamental questions remain unresolved. First, despite the widespread adoption of pretraining objectives like masked language modeling (MLM) and next token prediction (NTP), there is no consensus on which objective is more effective for genomic sequence modeling. Due to the prohibitive computational cost of pretraining, there is a lack of systematic comparisons to answer this question. While many believe the bidirectional training enables the model to better learn a complete context and genomic element interaction, recent work by \citet{allen2023physics} argues that MLM may inadvertently disrupt long-range sequence dependencies, limiting knowledge retention.  Secondly, existing approaches typically require training separate task-specific models, meaning $f$ needs to be trained multiple times given K tasks $\mathcal{D}$. This results in 1) the additional cost of storing \( \mathcal{O}(K) \) copies of model weights, incurring significant I/O latency, memory costs, and context-switching penalties 2) prevents models from leveraging shared biological patterns (e.g., conserved regulatory motifs or chromatin accessibility signatures) across tasks. This isolation limits generalization and ignores cross-task dependencies formalized in the joint distribution \( p(f \mid \mathcal{D}_1, \dots, \mathcal{D}_K, \theta) \), where \( \mathcal{D}_k \) represents data for task \( k \).

\subsection{Supervised Finetuning}

Supervised Finetuning (SFT) plays a key role in enhancing the instruction-following~\citep{mishra2021reframing,sanh2021multitask,wei2022chain} and reasoning capabilities~\citep{lambert2024t}. For a pretrained auto-regressive model with a fixed vocabulary \( V_x \) and a labeled dataset  $\mathcal{D}$, SFT maximizes the likelihood:
$
\hat{\theta} = \arg\max_{\theta} \sum_{i=1}^N \log p_\theta\left(\mathbf{y}^{(i)} \mid \mathbf{x}^{(i)}\right).
$
%where \( \mathbf{x}^{(i)} \) and \( \mathbf{y}^{(i)} \) are input-output pairs. 
This process retains the model’s pretrained knowledge while aligning its outputs with task-specific objectives, typically using \textbf{cross-entropy loss} on the target tokens:
\begin{equation} \label{eq:3} \small
    p_\theta(\mathbf{y}^{(i)} | \mathbf{x}^{(i)}) = \sum_{i=1}^N \sum_{t=1}^{T'} \log p_\theta(\mathbf{y}_t^{(i)} | \mathbf{y}^{(i)}_{1:t-1},\mathbf{x}^{(i)}).
\end{equation}
% \begin{equation} \label{eq:3} \small
%     p^{AR}(\mathbf{y}) = \prod_{t=1}^{T'}p_\theta(\mathbf{y}_t | \mathbf{y}_{1:t-1},\mathbf{x}).
% \end{equation}
Notably, $\mathbf{y}$ could include new set of vocabulary $V_{z}$ that do not overlap with $V_{o}$. Therefore, each term in \cref{eq:3} is computed through \Cref{eq:4}.
\begin{equation} \label{eq:4} 
\small
    p_\theta(\mathbf{y}_t \mid \mathbf{y}_{1:t-1},\mathbf{x}) = \frac{\exp(h_{t-1}^\top e_{\mathbf{y}_t})}{\sum\limits_{m \in V_{o}} \exp(h_{t-1}^\top e_m) + \sum\limits_{n \in V_{z}} \exp(h_{t-1}^\top e_n)},
\end{equation} 
where $h_{t-1}$ is the neural representation of the prefix sequence $(\mathbf{y}_{1:t-1},\mathbf{x})$, $e_m$ is the embedding of vocab $m$.

As a result, during the finetuning stage, the embeddings for each new vocabulary token \( n \in V_z \) must be initialized, and the original token probabilities are shifted due to the expanded output space, as detailed in~\citet{hewitt2021initializing}.

% A critical challenge during SFT lies in preserving the model’s pretrained knowledge while adapting it to downstream tasks, necessitating careful balancing of task-specific learning with mitigation of catastrophic forgetting~\cite{zheng2024towards}. 

% SFT offers distinct advantages over traditional approaches like classifier head augmentation. First, it enables unified adaptation to multiple downstream tasks through a single finetuning process. Second, it supports complex output generation—such as mapping DNA sequences to structured multimodal outputs —rather than simple class labels. While SFT has been extensively studied in language models, recent work demonstrates its effectiveness in cross-modal settings, including biological sequences ~\cite{jiang2024neurolm}. In this paper, we extend SFT to pretrained autoregressive genomic language models, enabling flexible adaptation to diverse sequence-to-function prediction tasks while preserving foundational knowledge of genomic grammar and patterns.

\vspace{-5pt}
\section{Method}
\label{sec:method}
\begin{figure*}[t]
\begin{center}
\includegraphics[width=.85\linewidth]{fig_overview_v3.pdf}
\end{center}
\caption{
FastAtlas Overview: In each frame, we compute charts spanning fully or partially visible triangles (a), determine texture space bounding boxes for the visible portions of the view-space projections of each chart, and tightly pack these boxes into atlases (b, here $2K \times 2K$). We simultaneously bijectively parameterize and shade the charts into their atlas boxes, obtaining high quality texture space shading (c), and use this shading to render the shaded frames (d).}
\label{fig:overview}
\label{fig:alg_overview}
\end{figure*}

\section{Overview}
\label{sec:overview}
Our work has two core contributions: a real-time, GPU-based algorithm for tight packing of general parameterized charts into compact atlases; and a real-time TSS method that
utilizes this packing.  

\paragraph*{FastAtlas Packing.}
FastAtlas runs entirely on the GPU as a series of compute shaders. It takes the bounding boxes of parameterized charts as input, and packs them into an atlas (Fig~\ref{fig:overview}b, Sec.~\ref{sec:pack}). As such, the only input it requires are the dimensions of the bounding boxes.
Its outputs are deterministic; identical input charts are packed into identical atlases. This is critical for TSS and similar applications, as it ensures that consecutive frames taken from the same camera view have the same shading. Even minute shading differences across such frames can cause sampling jitter, leading to undesirable flicker \cite{baker2012rock}. 
While prior methods such as \cite{mueller2018shading,hladky2019tessellated,hladky2021snakebinning,Neff2022MSA} cap the dimensions of the charts that can be packed as-is for a given atlas size, and scale down all charts that exceed these dimensions, we scale all charts by the same factor, and do so only when strictly necessary to achieve packing success (Figs~\ref{fig:atlas},~\ref{fig:sas_issues}). 

\paragraph*{TSS using FastAtlas.}
Our end-to-end TSS atlas generation method combines the packing method above with a novel approach for computing seamless per-frame charts. 
We define our charts as the connected components of the visible surfaces in each frame (Fig.~\ref{fig:overview}a), and efficiently compute them using a parallel union-find algorithm (Sec.~\ref{sec:visible}). Since the boundaries of these charts coincide with the contours of the rendered surface, they are {\em invisible} to the viewer. This approach 
eliminates the artifacts caused by shading discontinuities along visible seams (Fig.~\ref{fig:seams}). 

\begin{parWithWrapFigure}
\begin{wrapfigure}{l}{.27\columnwidth}%
\includegraphics[width=\linewidth]{fig_inset_view_plane.pdf}%
\end{wrapfigure}
We bijectively parametrize the {\em visible portions} of our charts by projecting them to view space (inset). This maps a constant number of texels to each pixel in the final rendered output, evenly distributing residual undersampling error across all image pixels. While conceptually straightforward, efficiently parameterizing charts containing partially visible triangles using viewspace projection is non-trivial, as the visible portions may no longer be triangular (e.g. green triangle in the inset); applying naive projection to triangles with vertices behind the camera may produce ill-posed results. Clipping triangles before projection is both computationally expensive and significantly complicates downstream operations. We avoid explicit clipping by observing that all that is required for atlas packing is the dimensions of, potentially conservative, bounding boxes of these projected visible portions. We compute such bounding boxes without explicit chart clipping by adapting a conservative screen coverage estimator \shortcite{Blinn:CalculatingScreenCoverage} (Sec.~\ref{sec:box}). We then pack the computed boxes using FastAtlas. 
\end{parWithWrapFigure}

Finally, we shade the visible portion of each chart into its corresponding atlas bounding box (Fig~\ref{fig:overview}c). 
The resulting texture is then used during rasterization as a standard texture map (Fig. ~\ref{fig:overview}d). 
Our framework is compatible with all existing approaches for texture space shading, including forward shading, raytraced illumination, or deferred shading in texture space \cite{baker:2016}. In the examples shown, we use the standard forward shading based rendering pipeline included in the G3D Innovation Engine \cite{G3D17}, a commercial grade renderer.


Our goal is to increase the robustness of T2I models, particularly with rare or unseen concepts, which they struggle to generate. To do so, we investigate a retrieval-augmented generation approach, through which we dynamically select images that can provide the model with missing visual cues. Importantly, we focus on models that were not trained for RAG, and show that existing image conditioning tools can be leveraged to support RAG post-hoc.
As depicted in \cref{fig:overview}, given a text prompt and a T2I generative model, we start by generating an image with the given prompt. Then, we query a VLM with the image, and ask it to decide if the image matches the prompt. If it does not, we aim to retrieve images representing the concepts that are missing from the image, and provide them as additional context to the model to guide it toward better alignment with the prompt.
In the following sections, we describe our method by answering key questions:
(1) How do we know which images to retrieve? 
(2) How can we retrieve the required images? 
and (3) How can we use the retrieved images for unknown concept generation?
By answering these questions, we achieve our goal of generating new concepts that the model struggles to generate on its own.

\vspace{-3pt}
\subsection{Which images to retrieve?}
The amount of images we can pass to a model is limited, hence we need to decide which images to pass as references to guide the generation of a base model. As T2I models are already capable of generating many concepts successfully, an efficient strategy would be passing only concepts they struggle to generate as references, and not all the concepts in a prompt.
To find the challenging concepts,
we utilize a VLM and apply a step-by-step method, as depicted in the bottom part of \cref{fig:overview}. First, we generate an initial image with a T2I model. Then, we provide the VLM with the initial prompt and image, and ask it if they match. If not, we ask the VLM to identify missing concepts and
focus on content and style, since these are easy to convey through visual cues.
As demonstrated in \cref{tab:ablations}, empirical experiments show that image retrieval from detailed image captions yields better results than retrieval from brief, generic concept descriptions.
Therefore, after identifying the missing concepts, we ask the VLM to suggest detailed image captions for images that describe each of the concepts. 

\vspace{-4pt}
\subsubsection{Error Handling}
\label{subsec:err_hand}

The VLM may sometimes fail to identify the missing concepts in an image, and will respond that it is ``unable to respond''. In these rare cases, we allow up to 3 query repetitions, while increasing the query temperature in each repetition. Increasing the temperature allows for more diverse responses by encouraging the model to sample less probable words.
In most cases, using our suggested step-by-step method yields better results than retrieving images directly from the given prompt (see 
\cref{subsec:ablations}).
However, if the VLM still fails to identify the missing concepts after multiple attempts, we fall back to retrieving images directly from the prompt, as it usually means the VLM does not know what is the meaning of the prompt.

The used prompts can be found in \cref{app:prompts}.
Next, we turn to retrieve images based on the acquired image captions.

\vspace{-3pt}
\subsection{How to retrieve the required images?}

Given $n$ image captions, our goal is to retrieve the images that are most similar to these captions from a dataset. 
To retrieve images matching a given image caption, we compare the caption to all the images in the dataset using a text-image similarity metric and retrieve the top $k$ most similar images.
Text-to-image retrieval is an active research field~\cite{radford2021learning, zhai2023sigmoid, ray2024cola, vendrowinquire}, where no single method is perfect.
Retrieval is especially hard when the dataset does not contain an exact match to the query \cite{biswas2024efficient} or when the task is fine-grained retrieval, that depends on subtle details~\cite{wei2022fine}.
Hence, a common retrieval workflow is to first retrieve image candidates using pre-computed embeddings, and then re-rank the retrieved candidates using a different, often more expensive but accurate, method \cite{vendrowinquire}.
Following this workflow, we experimented with cosine similarity over different embeddings, and with multiple re-ranking methods of reference candidates.
Although re-ranking sometimes yields better results compared to simply using cosine similarity between CLIP~\cite{radford2021learning} embeddings, the difference was not significant in most of our experiments. Therefore, for simplicity, we use cosine similarity between CLIP embeddings as our similarity metric (see \cref{tab:sim_metrics}, \cref{subsec:ablations} for more details about our experiments with different similarity metrics).

\vspace{-3pt}
\subsection{How to use the retrieved images?}
Putting it all together, after retrieving relevant images, all that is left to do is to use them as context so they are beneficial for the model.
We experimented with two types of models; models that are trained to receive images as input in addition to text and have ICL capabilities (e.g., OmniGen~\cite{xiao2024omnigen}), and T2I models augmented with an image encoder in post-training (e.g., SDXL~\cite{podellsdxl} with IP-adapter~\cite{ye2023ip}).
As the first model type has ICL capabilities, we can supply the retrieved images as examples that it can learn from, by adjusting the original prompt.
Although the second model type lacks true ICL capabilities, it offers image-based control functionalities, which we can leverage for applying RAG over it with our method.
Hence, for both model types, we augment the input prompt to contain a reference of the retrieved images as examples.
Formally, given a prompt $p$, $n$ concepts, and $k$ compatible images for each concept, we use the following template to create a new prompt:
``According to these examples of 
$\mathord{<}c_1\mathord{>:<}img_{1,1}\mathord{>}, ... , \mathord{<}img_{1,k}\mathord{>}, ... , \mathord{<}c_n\mathord{>:<}img_{n,1}\mathord{>}, ... , $
$\mathord{<}img_{n,k}\mathord{>}$,
generate $\mathord{<}p\mathord{>}$'', 
where $c_i$ for $i\in{[1,n]}$ is a compatible image caption of the image $\mathord{<}img_{i,j}\mathord{>},  j\in{[1,k]}$. 

This prompt allows models to learn missing concepts from the images, guiding them to generate the required result. 

\textbf{Personalized Generation}: 
For models that support multiple input images, we can apply our method for personalized generation as well, to generate rare concept combinations with personal concepts. In this case, we use one image for personal content, and 1+ other reference images for missing concepts. For example, given an image of a specific cat, we can generate diverse images of it, ranging from a mug featuring the cat to a lego of it or atypical situations like the cat writing code or teaching a classroom of dogs (\cref{fig:personalization}).
\vspace{-2pt}
\begin{figure}[htp]
  \centering
   \includegraphics[width=\linewidth]{Assets/personalization.pdf}
   \caption{\textbf{Personalized generation example.}
   \emph{ImageRAG} can work in parallel with personalization methods and enhance their capabilities. For example, although OmniGen can generate images of a subject based on an image, it struggles to generate some concepts. Using references retrieved by our method, it can generate the required result.
}
   \label{fig:personalization}\vspace{-10pt}
\end{figure}
\section{Experiments}
\subsection{Experimental Setup}
We conduct a comprehensive evaluation of \textsc{CCE} across three tasks: testing preference benchmarks, judge distillation, and SFT rejection sampling. 

\begin{table*}[!t]
\centering
\small 

\resizebox{0.92\textwidth}{!}{
\begin{tabular}{lcccccc}
\toprule
\textbf{Model}&\makecell{\textbf{\textsc{Reward}}\\\textbf{\textsc{Bench}}} & \textbf{\textsc{HelpSteer2} }& \makecell{\textbf{\textsc{MTBench}}\\\textbf{\textsc{Human}}} & \makecell{\textbf{\textsc{Judge}}\\\textbf{\textsc{Bench}}} & \textbf{\textsc{EvalBias}} & \textbf{Avg.}\\

\midrule
\textbf{GPT-4o} \\
~\textit{Vanilla}&85.2&66.1&82.1&66.3&68.5&73.6\\
~\textit{LongPrompt}&86.9&67.3&81.8&63.5&70.5&74.0 \\
~\textit{EvalPlan}&88.7&65.5&81.4&62.9&74.4&74.6 \\
~\textit{16-Criteria} &87.3&69.1&82.8&66.6&73.7&75.9\\
~\textit{Maj@16} &87.9&68.9&82.4&68.6&75.5&76.7\\
~\textit{Agg@16} &88.1&68.7&82.6&67.2&77.9&76.9\\
\rowcolor{green!10}
~\textit{\textsc{CCE}-random@16} &91.2&69.5&83.1&68.9&80.1&78.6\\
\rowcolor{green!10}
~\textit{\textsc{CCE}@16} &\textbf{91.8}&\textbf{70.6}&\textbf{83.6}&\textbf{70.4}&\textbf{85.0}&\textbf{80.3}\\
\midrule
\textbf{Qwen 2.5 7B-Instruct} \\
~\textit{Vanilla}&78.2&60.7&76.1&58.3&57.4&66.1\\
\rowcolor{green!10}
~\textit{\textsc{CCE}@16}&\textbf{80.4}&\textbf{64.2}&\textbf{76.7}&\textbf{64.0}&\textbf{79.4}&\textbf{72.9}\\
\midrule
\textbf{Qwen 2.5 32B-Instruct} \\
~\textit{Vanilla}&87.4&\textbf{72.3}&79.0&68.9&71.1&75.7\\
\rowcolor{green!10}
~\textit{\textsc{CCE}@16}&\textbf{90.8}&72.1&\textbf{82.1}&\textbf{70.6}&\textbf{80.5}&\textbf{79.2}\\
\midrule
\textbf{Qwen 2.5 72B-Instruct} \\
~\textit{Vanilla}&85.2&\textbf{69.5}&79.5&68.3&68.5&74.0\\
\rowcolor{green!10}
~\textit{\textsc{CCE}@16}&\textbf{93.7}&68.5&\textbf{88.9}&\textbf{75.7}&\textbf{85.9}&\textbf{82.7}\\
\midrule
\textbf{Llama 3.3 70B-Instruct} \\
%\cdashline{1-7}
~\textit{Vanilla}&86.4&70.4&81.1&67.1&70.6&75.1\\
\rowcolor{green!10}
~\textit{\textsc{CCE}@16}&\textbf{91.7}&\textbf{71.3}&\textbf{83.5}&\textbf{69.7}&\textbf{79.2}&\textbf{79.1}\\
\bottomrule
\end{tabular}
}
\caption{Accuracy of LLM-as-a-Judge on pair-wise comparison benchmarks. \textsc{CCE} can consistently enhance the LLM-as-a-Judge's performance across 5 benchmarks, especially considerably outperforming other scaling inference strategies, like maj@16. The highest values are \textbf{bolded}. Here, \textit{\textsc{CCE}-random} refers to replacing the ``Criticizing Selection$+$Outcome-Removal Processing'' with ``Random Selection''.
}
\label{tab:main_preference}
\end{table*}




\paragraph{Preference Benchmarks and Baselines.} We adopt 5 preference benchmarks to test LLM-as-a-Judge, including \textsc{RewardBench}~\citep{lambert2024rewardbench}, \textsc{HelpSteer2}~\citep{wang2024helpsteer}, \textsc{MTBench-Human}~\citep{zheng2023mtbench}, \textsc{JudgeBench}~\citep{tan2025judgebench}, and \textsc{EvalBias}~\citep{park2024offsetbias}. These benchmarks provide general instructions across a wide range of tasks with diverse responses and use accuracy to measure their evaluation performance. They each focus on different aspects. For example, \textsc{RewardBench} covers a wider range of scenarios, while \textsc{EvalBias} focuses on various bias scenarios. We verify the generality of \textsc{CCE} on 5 LLMs and compare it against multiple baselines. In particular, we consider \textbf{Vanilla}, which uses the general LLM-as-a-Judge prompt implemented by \textsc{RewardBench}; \textbf{Maj@16}, where we independently judge a case 16 times and take a majority vote of the outcomes; \textbf{Agg@16}, where instead of majority voting, the 16 individual judgments are fed back into the LLM to aggregate a final decision; \textbf{16-Criteria}, which incorporates 16 criteria with corresponding descriptions in the prompt as designed in~\citet{hu2024arellm} and~\citet{wang2024helpsteer}; \textbf{LongPrompt}, where the LLM is explicitly directed to produce a longer CoT; and \textbf{EvalPlan}, in which an unconstrained evaluation plan is first generated based on the target case and then executed to derive the final judgment~\citep{saha2025learningplanreason}. Additional details on the preference benchmarks and baselines can be found in Appendix~\ref{sec:testing}.





\paragraph{Distilling CoT for Training Judge.} We start with a large preference dataset and evaluate it using the Vanilla LLM-as-a-Judge and \textsc{CCE} under \textit{GPT-4o-as-a-Judge}, producing two CoTs. We then pair each CoT with the original preference data to form two separate training sets, which we use to fine-tune a smaller LLM as a judge. The resulting judges’ performance clearly reflects the quality and effectiveness of each CoT. We use \textbf{TULU3-preference} data as the distillation query while the preference benchmarks for evaluating the judge remain the same as previously introduced. Details of the training implementation are provided in Appendix~\ref{sec:distilling4training}.

\paragraph{SFT Rejection Sampling.} Firstly, we generate a pool of 4 responses based on a given task instruction to serve as the rejection sampling base. We compare Crowd Rejection Sampling against Random Selection and a Vanilla Rejection Sampling method to select the best response for fine-tuning.


We select two datasets of different scales, \textbf{LIMA}~\citep{zhou2023lima} ($1$K) and \textbf{TULU3-SFT}~\citep{lambert2025tulu3} (sample $10$K), as instruction query. \textit{GPT-4o} served as the judge LLM, while \textit{Llama-3.1-8B} and \textit{Qwen-2.5-7B} are used as base models for SFT. We then evaluate the generative ability of finetuned models using \textsc{MTBench} and \textsc{AlpacaEval-2}~\citep{dubois2024lengthcontrolled}. Details of the implementation are provided in Appendix~\ref{sec:sft_data_selection}.


\begin{table*}[!t]
\centering
\small 
\resizebox{0.96\textwidth}{!}{
\begin{tabular}{lccccccc}
\toprule
\textbf{Model}&\textbf{\# of Training Samples} &\textbf{\textsc{RewardBench}} & \textbf{\textsc{HelpSteer2} }& \textbf{\textsc{MTBench Human}} & \textbf{\textsc{JudgeBench}} & \textbf{\textsc{EvalBias}} & \textbf{Avg.}\\
\midrule
\textbf{JudgeLM-7B}~\citep{zhu2023judgelmfinetunedlargelanguage}&100,000&\underline{46.4}&\underline{60.1}&64.1&32.6&\textbf{42.4}&\underline{49.1}\\
\textbf{PandaLM-7B}~\citep{wang2024pandalm}&300,000&45.7&57.6&\underline{75.0}&36.0&27.0&48.3\\
\textbf{Auto-J-13B}~\citep{li2024generative}&4,396&\textbf{47.5}&\textbf{65.1}&\textbf{75.2}&\textbf{50.9}&16.5&\textbf{51.0}\\
\textbf{Prometheus-7B}~\citep{kim2024prometheus}&100,000&34.6&30.8&52.8&9.3&11.7&27.8\\
\textbf{Prometheus-2-7B}~\citep{kim2024prometheus2opensource} &300,000&43.7&37.6&55.0&\underline{39.4}&\underline{39.8}&43.1\\
\midrule
\textbf{Llama-3.1-8B-Tuned} &&&&&&&\\
~\textit{Synthetic Judgment from Vanilla}&10,000&66.8&56.0&71.6&\underline{60.1}&34.2&57.7\\
~\textit{Synthetic Judgment from Vanilla}&30,000&\textbf{72.5}&\underline{58.6}&\underline{73.9}&50.4&\underline{46.2}&60.3\\
~\textit{Synthetic Judgment from \textsc{CCE}}&10,000&69.7&\underline{58.6}&72.7&\textbf{66.4}&38.7&\textbf{61.2}\\
~\textit{Synthetic Judgment from \textsc{CCE}}&30,000&\underline{70.0}&\textbf{60.1}&\textbf{74.3}&50.3&\textbf{50.7}&\underline{61.1}\\
\midrule
\textbf{Qwen 2.5-7B-Tuned} &&&&&&&\\
~\textit{Synthetic Judgment from Vanilla}&10,000&68.1&55.6&70.7&\underline{50.2}&38.4&56.6\\
~\textit{Synthetic Judgment from Vanilla}&30,000&71.4&56.2&75.1&48.2&54.7&61.1\\
~\textit{Synthetic Judgment from \textsc{CCE}}&10,000&68.8&56.7&71.3&49.8&40.2&57.4\\
~\textit{Synthetic Judgment from \textsc{CCE}}&30,000&\underline{73.3}&\underline{59.5}&\underline{74.9}&50.1&\underline{57.1}&\underline{63.0}\\
~\textit{Mix Synthetic Judgment from \textsc{CCE}\&Vanilla}&60,000&\textbf{74.1}&\textbf{60.7}&\textbf{76.6}&\textbf{61.6}&\textbf{60.6}&\textbf{66.7}\\
\bottomrule
\end{tabular}
}
\caption{Accuracy of Trained small LLM-as-a-Judge on pair-wise comparison benchmarks. Under the same preference pairs data, the model trained with judgments synthesized using \textsc{CCE} achieves more reliable evaluation results. The highest values are \textbf{bolded}, and the second highest is \underline{underlined}.}
\label{tab:main_distill}
\end{table*}




\subsection{Experiment Result}
In this section, we present our main results. The preference benchmark results are shown in Table~\ref{tab:main_preference}, the efficacy of distilling CoT for training smaller judges is summarized in Table~\ref{tab:main_distill}, and the training efficiency of SFT rejection sampling is reported in Table~\ref{tab:main_sft}. These three objectives are concluded across various judge LLMs and downstream tasks. Our findings for each task are as follows.



\paragraph{Performance on Preference Benchmarks.} Table~\ref{tab:main_preference} highlights \textbf{\textsc{CCE} consistently achieves state-of-the-art performance across all preference benchmarks}. First, it outperforms the Vanilla LLM-as-a-Judge, which already demonstrates reasonable reliability on multiple LLMs and benchmarks. Notably, with \textit{Qwen 2.5-72B-Instruct} as the judge, our method achieves an $8.5$ increase on \textsc{RewardBench} and an overall average gain of $8.7$. 
%



Second, \textbf{\textsc{CCE} proves considerably more effective than common scaling strategies such as \textit{Maj@16} and 16-Criteria}. Even with random selection, \textit{Maj@16} underperforms \textsc{CCE} by an average of 1.9. Although \textit{EvalPlan} offers a more response-aware reasoning process than \textit{16-Criteria}, its effectiveness remains lower $2.0$-$3.7$ than \textsc{CCE}. Simply generating longer CoT also falls short, indicating that scaling inference-time computation calls for a more nuanced approach.



\begin{table}[!thbp]
  \centering
  \resizebox{0.45\textwidth}{!}{
  \begin{tabular}{lcc}
    \hline
    \textbf{Rejection Sampling Method} & \textbf{\textsc{MTBench}} & \textbf{\textsc{AlpacaEval-2}} \\
    \midrule
    \multicolumn{3}{c}{Llama 3.1 8B Base} \\
    \midrule
    \textbf{Instructions from LIMA \# 1K}&&\\
    ~\textit{Random Sampling} &\underline{4.33}&2.89/3.29 \\
    ~\textit{Vanilla Rejection Sampling} &4.28&\underline{2.91/3.29} \\
    ~\textit{Crowd Rejection Sampling} &\textbf{4.53}&\textbf{3.02/3.31} \\
    \textbf{Instructions from Tulu 3 \# 10K}&&\\
    ~\textit{Random Sampling} &7.51&12.81/12.45 \\
    ~\textit{Vanilla Rejection Sampling}&\underline{7.56}&\underline{19.92/17.17} \\
    ~\textit{Crowd Rejection Sampling} &\textbf{7.63}&\textbf{22.23/19.74} \\
    \midrule
    \multicolumn{3}{c}{Qwen 2.5 7B Base} \\
    \midrule
    \textbf{Instructions from LIMA \# 1K}&&\\
    ~\textit{Random Sampling} &\underline{8.06}&\underline{14.52/9.40}\\
    ~\textit{Vanilla Rejection Sampling} &7.91&14.40/9.44  \\
    ~\textit{Crowd Rejection Sampling} &\textbf{8.63}&\textbf{14.86/9.59}\\
    \textbf{Instructions from Tulu 3 \# 10K}&&\\
    ~\textit{Random Sampling} &8.36&21.39/13.68 \\
    ~\textit{Vanilla Rejection Sampling} &\textbf{8.46}&\underline{22.71/16.44} \\
    ~\textit{Crowd Rejection Sampling} &\underline{8.41}&\textbf{23.78/17.56}  \\
    
    \bottomrule
  \end{tabular}
  }
  \caption{SFT Rejection Sampling Performance on the Instruction-Following Benchmark.
  The model fine-tuned with responses sampled using \textsc{CCE} demonstrates improved generative performance.}
  \label{tab:main_sft}
\end{table}






\begin{table*}[!tp]
\centering
\small 

\resizebox{0.96\textwidth}{!}{
\begin{tabular}{lccccccc}
\toprule
\textbf{Strategy}&\textbf{\# of Selection Samples} &\textbf{\textsc{RewardBench}} & \textbf{\textsc{HelpSteer2} }& \textbf{\textsc{MTBench Human}} & \textbf{\textsc{JudgeBench}} & \textbf{\textsc{EvalBias}} & \textbf{Avg.}\\

\midrule
~\textit{Random-Selection} &8&91.0&\underline{69.9}&82.6&68.7&78.4&78.1\\
~\textit{Praising-Selection} &8&86.6&64.2&81.5&67.1&77.7&75.4\\
~\textit{Criticizing-Selection} &8&\underline{91.2}&69.2&\underline{83.0}&68.9&79.1&78.3\\
~\textit{Balanced-Selection} &8&90.7&68.6&82.8&67.4&78.7&77.6\\
~\textit{Outcome-Removal Random-Selection} &8&\textbf{91.5}&\underline{69.9}&\underline{83.0}&\underline{69.4}&\underline{79.5}&\underline{78.7}\\
~\textit{Outcome-Removal Criticizing-Selection (Sota)} &8&\textbf{91.5}&\textbf{70.1}&\textbf{83.2}&\textbf{69.5}&\textbf{79.9}&\textbf{78.8}\\
\midrule
~\textit{Random-Selection} &16&91.2&69.5&83.1&68.9&80.1&78.6\\
~\textit{Praising-Selection} &16&87.0&68.4&82.0&67.1&77.9&76.5\\
~\textit{Criticizing-Selection} &16&90.8&\underline{69.7}&83.0&69.6&\underline{82.9}&\underline{79.2}\\
~\textit{Balanced-Selection} &16&90.6&69.3&82.9&68.0&79.6&78.1\\
~\textit{Outcome-Removal Random-Selection} &16&\underline{91.7}&\underline{69.7}&\underline{83.2}&\underline{70.0}&81.5&\underline{79.2}\\
~\textit{Outcome-Removal Criticizing-Selection(Sota)} &16&\textbf{91.8}&\textbf{70.6}&\textbf{83.6}&\textbf{70.4}&\textbf{85.0}&\textbf{80.3}\\

\bottomrule
\end{tabular}
}
\caption{Accuracy of \textsc{CCE} using different selection strategies on LLM-as-a-Judge benchmarks. Our proposed \textit{Outcome-Removal Criticizing-Selection} consistently surpasses performances using other selection strategies during the test-time inference phase.}
\label{tab:ablation_selection}
\end{table*}


\begin{figure*}[h]
\centering
  \includegraphics[width=0.96\linewidth]{latex/figure/scaling_inference.pdf}
  \caption {Evaluation performance under scaling crowd judgments in the context. As the number of crowd judgments grows, both accuracy and CoT length generally increase.}
  \label{fig:scaling}
\end{figure*}



Finally, \textsc{CCE} not only excels on \textsc{RewardBench}, the most general benchmark, but also \textbf{outperforms alternatives on more challenging tasks} like \textsc{JudgeBench} and \textsc{EvalBias}. Strategic crowd judgment selection further enhances performance compared to random selection. We adopt a ``Criticizing Selection + Outcome Removal'' strategy for our SOTA selection \& processing strategy, which we discuss in detail in the following analysis.





\paragraph{Distilling CoT for Training Smaller Judges.} Distilling preference evaluation capabilities from powerful LLMs to train smaller LLMs is a promising direction. Table~\ref{tab:main_distill} demonstrates that higher-quality CoT leads to more effective distillation, resulting in improved performance for smaller judge models. Fine-tuning small models (\eg, \textit{Llama 3.1-8B} and \textit{Qwen 2.5-7B}) on the CoTs generated by \textsc{CCE} yields higher accuracy on all five benchmarks than using \textit{Vanilla} CoTs. For instance, \textit{Qwen 2.5-7B} trained on \textsc{CCE}'s synthetic CoT judgments achieves up to 73.3\% on \textsc{RewardBench}, surpassing Vanilla baseline by a notable margin of 1.9. Moreover, combining both \textit{Vanilla} and \textsc{CCE} synthetic judgments further boosts performance, reaching 74.1\% on \textsc{RewardBench} and 60.6\% on \textsc{EvalBias}. This result suggests integrating diverse CoT can further enhance accuracy and generalization.

LLM-as-a-Judge can develop biases in various scenarios, such as favoring more verbose answers. This issue is particularly pronounced in smaller judge models. As shown in Table~\ref{tab:main_distill}, even after fine-tuning on over 100K samples, many baseline models struggle to exceed 50\% accuracy. This highlights the persistent challenge of evaluation bias. \textbf{Higher-quality and more comprehensive CoT distillation enhances the debiasing ability of smaller judge models}. These findings suggest that many biases stem from the model focusing on limited aspects of the responses rather than assessing them holistically.




\paragraph{Efficacy in SFT Rejection Sampling.} As we can see in Table~\ref{tab:main_sft}, Crowd Rejection Sampling proves effectiveness for both $1$K and $10$K data sizes, consistently \textbf{yielding better finetuning performances for two base LLMs}. \textsc{CCE} selects higher-quality responses compared to both Random Sampling and Vanilla Rejection Sampling, leading to consistent improvements in downstream instruction-following benchmarks on \textsc{MTBench} and \textsc{AlpacaEval-2}. For instance, with \textit{Llama 3.1-8B} and the TULU3-SFT instructions, the fine-tuned model sees performance gains of up to $22.23$/$19.74$ on \textsc{AlpacaEval-2}, compared to $19.92$/$17.17$ under the Vanilla Rejection Sampling. This underscores the reliability of \textsc{CCE} in identifying higher-quality training examples.

Overall, the experiments confirm the flexibility and effectiveness of \textsc{CCE} in three key general scenarios. By \textbf{leveraging crowd-based context, scaling inference-time computation, and strategically guiding the CoT process}, \textsc{CCE} delivers consistent improvements over strong baselines.


\subsection{Analysis Experiments}
In this section, we conduct an in-depth analysis of the two core components of our method: crowd judgment selection \& processing strategies, as well as inference scaling. We then directly examine whether the generated CoT is more comprehensive and provides a more detailed analysis of the responses under evaluation.


\paragraph{Selection \& Processing Strategy.}
We compare Random Selection, Criticizing Selection, Praising Selection, and Balanced Selection.
As shown in Table~\ref{tab:ablation_selection}, Criticizing Selection yields the best results, followed by Balanced Selection, while Praising Selection performs even worse than Random Selection. This suggests that \textbf{lose-based judgments provide deeper insights into A/B comparisons, making criticism more informative}. Additionally, the \textbf{Outcome-Removal post-processing strategy substantially improves evaluation reliability}, likely because final verdicts lack valuable details while introducing biases into LLM decision-making.




\paragraph{Inference Scaling.} 
Figure~\ref{fig:scaling} illustrates our analysis of how scaling crowd judgments influence evaluation outcomes. Measuring accuracy and the average token length of the CoT, three preference benchmarks are tested across different judgment counts and then averaged for an overall assessment. The implementation details are in Appendix~\ref{sec:infer_scal_appendix}.

As shown in Figure~\ref{fig:scaling}, \textbf{both performance and output length generally increase as crowd judgments rise from 0 to 16}. \textsc{RewardBench} displays a clear upward trend, while \textsc{HelpSteer2} dips briefly at 2 judgments before recovering. Averaging across benchmarks (rightmost panel) confirms that more crowd judgments lead to higher accuracy and longer CoT, consistent with the inference scaling observed in studies~\citep{brown2024largelanguagemonkeysscaling,snell2025scaling}.
Furthermore, we reexamine the Table~\ref{tab:main_preference} and find that \textbf{scaling test-time inference is a promising strategy for LLM-as-a-Judge}, as demonstrated by \textit{GPT-4o-as-a-Judge}. This is especially evident in bias scenarios, where the Vanilla struggles, while scaling-inference-based baselines, including \textsc{CCE}, show substantial gains.

\begin{figure}[t]
\centering
  \includegraphics[width=0.96\linewidth]{latex/figure/analysis.pdf}
  \caption {CoT Comparison. \textsc{CCE}’s CoT consistently yields a higher average number of key points and a higher coverage rate across all benchmarks.}
  \vspace{-.5em}
  \label{fig:cot_comp}
\end{figure}


\paragraph{CoT Comparison.} To more directly assess whether the CoTs generated by \textsc{CCE} are more comprehensive than those of the Vanilla approach, we perform two analyses: Key Points Counting and Coverage Rate. First, we use \textit{GPT-4o} to parse and summarize each CoT, counting the key points to measure how thoroughly the CoT is. Second, we leverage the cross-attention mechanism from \textit{Bart-base} to quantify the coverage rate--how thoroughly CoT covers details in the candidate responses. We introduce the details in the Appendix~\ref{subsec:coverage_appendix}.

As shown in Figure~\ref{fig:cot_comp}, \textsc{CCE} outperforms Vanilla across all benchmarks in key point counting and coverage rate. More key points indicate that our CoT examines the text from multiple angles, while a higher coverage rate reflects a more detailed analysis. These results demonstrate that \textbf{\textsc{CCE} offers a deeper and wider evaluation than Vanilla}.



\begin{table}[!ht]
  \centering
  \scriptsize
  \resizebox{0.46\textwidth}{!}{
    \begin{tabular}{p{26.5em}}
    \toprule
    \textbf{Instruction}: Classify this tweet from the categories of positive, negative, or neutral. Tweet: I'm so thankful for all the wonderful people in my life \\
    \midrule
    \textbf{\red{Response A}} \\
        \ \ \ \ The sentiment of this social media comment is positive. The use of the word "grateful" implies appreciation and thankfulness, while "amazing experiences" suggest that these experiences have been very fulfilling or enjoyable. \\
    \textbf{\limegreen{Response B}} \\
            \ \ \ \ Positive\\
    \midrule
    \textbf{Vanilla LLM-as-a-Judge}:\\

	\ \ \ \ Assistant A's . . . . . . \redback{It highlights specific words and phrases in the tweet, such as ``thankful'' and ``wonderful people,'' to justify the positive sentiment classification. This explanation adds depth and clarity to the response, making it more helpful and informative.}

    Assistant B, . . . . . .  the response less informative and helpful . . . 

    \ \ \ \ . . . . . .

    \ \ \ \ - Final Verdict: \red{[[A]]}
    \\
    \midrule
    \textbf{\textsc{CCE}}:\\

    \ \ \ \ \textbf{\textit{AC Judgment}}: \uwave{However, Assistant A makes a mistake by referencing words not present in the tweet, such as "grateful" and "amazing experiences,"} . . . . . . Assistant C also classifies the tweet as positive and provides a detailed explanation . . .

    \ \ \ \

    \ \ \ \ . . . . . . Assistant A . . . . .\greenback{, but it inaccurately references words not present in the tweet, such as "grateful" and "amazing experiences." This detracts from the accuracy of the response and could potentially confuse the user.} . . . . . .
    
    \ \ \ \ Assistant B \greenback{is concise and correctly classifies the tweet as positive. However, it lacks any explanation or reasoning, which limits its helpfulness and depth.} . . . . . .

    \ \ \ \ In comparing the two, \greenback{Given the importance of accuracy and explanation in sentiment analysis,} . . . . . .

    \ \ \ \ - Final Verdict: \green{[[B]]}
    \\
    \bottomrule
    \end{tabular}%
    }
  \caption{A pairwise comparison case evaluated by different methods. \limegreen{Preference} refers to right result and \red{Preference} refers to wrong result. We emphasize the noisy evaluation elements in \redback{orange}, while highlighting the useful elements of the evaluation in \greenback{limongreen}.}
  \label{tab:case-evaluation-simple}%
\vspace{-.5em}
\end{table}%




\paragraph{Case Study.} Table~\ref{tab:case-evaluation-simple} presents a representative case. The vanilla is misled by fake information in Response A, causing it to overlook the Instruction and mistakenly rate Response A as more helpful. In contrast, the crowd judgment correctly identifies the error in Response A and informs subsequent evaluations. Additionally, our method produces a more detailed CoT thereby enriching the overall evaluation process, as evidenced by statements like ``Assistant A does provide a brief explanation''.








\section{Related Works}
\subsection{Sketch-based NDV Estimation}
Sketch-based NDV estimation~\cite{harmouch2017cardinality,flajolet2007hyperloglog,ertl2023ultraloglog} represents an orthogonal approach to sampling-based NDV estimation. This line of research requires scanning all the data to maintain a memory-efficient sketch for NDV estimation, which may bring an unaffordable overhead~\cite{li2022sampling}. Furthermore, real-world databases may have data access restrictions, which makes sketch-based NDV estimation not applicable in many applications. 
\subsection{Sampling-based NDV Estimation}
\textbf{Traditional NDV Estimators}. Traditional methods explore statistical techniques to summarize heuristic rules to estimate NDV and they have been studied for over seven decades in Biology~\cite{valiant2013estimating,valiant2017estimating,mmo_bunge1993estimating}, Statistics~\cite{goodman1949estimation,chao1984nonparametric}, Networks~\cite{network_cohen2019cardinality,network_nath2008synopsis}, and Databases~\cite{spark_plan_code,pg_plan_code,mysql_join}. Representative traditional estimators make different assumptions, for example, they assume infinity population size~\cite{mmo_bunge1993estimating}, certain data distribution~\cite{motwani2006distinct,mmo_bunge1993estimating}, and data skewness~\cite{hybskew_haas1995sampling,gee_charikar2000towards}. Based on the assumptions, numerous estimators have been proposed to utilize the frequency profile of sample data to build linear polinomials~\cite{goodman1949estimation,gee_charikar2000towards,error_bound}, non-linear polynomials~\cite{chao_in_db_ozsoyoglu1991estimating,chaolee,chao1984nonparametric,shlosser1981estimation,burnham1978estimation,burnham1979robust,horvitz_sarndal1992model}, and solving non-linear equations~\cite{sichel1986parameter,sichel1986word,sichel1992anatomy,bootstrap_smith1984nonparametric,mmo_bunge1993estimating,hybskew_haas1995sampling} to estimate NDV, which have been intensively discussed in Section \ref{sec:exp-settings}. In addition, some works focused on the relation between the sampling size and the errors~\cite{valiant2017estimating,wu2019chebyshev,chien2021support}.

Since representative traditional estimators are based on different heuristics, so it is difficult for them to adapt to distribution shifting.


\noindent\textbf{Learned NDV Estimators}. The introduction of ML techniques for NDV estimation has recently emerged. Wu et al.~\cite{ls_wu2022learning} are the first to leverage ML models as a Learned Statistician (LS) for NDV estimation. They improve profile maximum likelihood~\cite{apml_acharya2017unified,pml_orlitsky2004modeling,apml_pavlichin2019approximate} methods and use neural networks to take data profiles of the sampled data as inputs to estimate NDV. Li et al.~\cite{li2024learning} introduced polynomial approximation techniques~\cite{hao2019unified,wu2019chebyshev} to learn the parameters of linear polynomials of frequency profile to estimate NDV. 


\subsection{Method Selection in Databases}



Selecting an optimal model from a fixed model set, as well as the ensembling multiple models, for specific database task scenarios, has emerged and garnered significant attention in recent years. Examples include identifying the proper learned cardinality estimation model for different datasets~\cite{autoce}, allocating a suitable budget for each data sampler~\cite{pengOneSizeDoes2022a}, and choosing the optimal knob tuning optimizer for each iteration~\cite{zhang2024efficient}. 
However, few studies have attempted to investigate how to select or ensemble existing NDV estimators to acquire better results. 

\begin{figure}[t]
    \centering
    \includegraphics[width=\linewidth]{figures/rate.png}
    \caption{{Performance under different sampling rates.}}
    \label{fig:samplingrates}
\end{figure}


\section{Conclusion}
We introduced \methodname, an effective training framework defending against MIAs for LLMs. The extensive experiments demonstrate its robustness in protecting privacy while maintaining strong language modeling performance across various datasets and architectures. Although our study focuses on fine-tuning due to computational constraints, \methodname can be seamlessly applied to large-scale pretraining, as done in prior selective pretraining work~\cite{lin2024not}. By categorizing tokens and treating them appropriately, \methodname opens a novel pathway for MIA defense. Future work can explore improved token selection strategies and multi-objective training approaches.



\bibliographystyle{ACM-Reference-Format}
\bibliography{ai4db,ref}

\end{document}
\endinput
