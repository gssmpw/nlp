\documentclass[a4paper,11pt]{amsart}
\usepackage[utf8]{inputenc}

\usepackage{enumitem}

\usepackage{a4wide}

\usepackage{orcidlink}
\usepackage{amsmath,amssymb}
\usepackage{bm}
\usepackage{multirow}
\usepackage{algpseudocode}

\usepackage{hyperref}

\usepackage[misc]{ifsym} % For "envelope" symbol

\usepackage{color}
\usepackage{graphicx}
\usepackage{subcaption}

\usepackage{amsfonts}
\usepackage{amsthm}
\usepackage{mathtools}

\usepackage[mathscr]{eucal}


\usepackage{pgfplots}
\pgfplotsset{compat=1.18}

\usepackage{bbm}
\ifpdf
\DeclareGraphicsExtensions{.eps,.pdf,.png,.jpg}
\else
\DeclareGraphicsExtensions{.eps}
\fi

\pagestyle{plain}

%\newtheorem{corollary}{Corollary}
\newtheorem{remark}{Remark}
\newtheorem{assumption}{Assumption}


%% Define a few struts
%% (from code by Claudio Beccari in TeX and TUG News, Vol. 2, 1993)
%% This is for modifying vertical space in tabular environment
%\newcommand\Tstrut{\rule{0pt}{2.9ex}}       % "top" strut
%\newcommand\Bstrut{\rule[-1.3ex]{0pt}{0pt}} % "bottom" strut
%\newcommand\TBstrut{\Tstrut\Bstrut}         % "top and bottom" strut
\usepackage{mathtools}

\providecommand{\rset}{\mathbb{R}} % Reals

%\providecommand{\Order}[1]{{\ensuremath{\mathcal{O}}\mspace{-2mu}\left[#1\right]}}    %Big Ordo

\providecommand{\E}[1]{{\ensuremath{\mathbb{E}}\mspace{-2mu}\left[#1\right]}}    %Expected Value
\providecommand{\var}[1]{{\ensuremath{\mathrm{Var}}\mspace{-2mu}\left[#1\right]}}% Variance
\providecommand{\Est}{\mathcal{A}}               % Estimator

\providecommand{\nrs}{\ensuremath{M}}            % Nr. of evaluations
\providecommand{\nelem}{\ensuremath{N}}          % Number of elements in FEM discretization
%\providecommand{\EstMLMC}{\mathcal{A}_{MLMC}}    % MLMC estimator of E[QoI]
%\providecommand{\EstMC}{\mathcal{A}_{MC}}        % MC estimator     -"-
%\providecommand{\varMC}{\mathcal{V}}             % Sample variance
%\providecommand{\vg}{{\ensuremath{V_0}}}         % Variance of Q_0 approx. Q_l all l


\providecommand{\QoI}{\ensuremath{Q}} % Quantity of Interest
\providecommand{\tol}{\ensuremath{\mathrm{TOL}}} % Error tolerance
\providecommand{\work}{\ensuremath{W}}           % Work, i.e. computational cost
%\providecommand{\Cw}{{K_{w}}}                    % Multiplicative constant in weak error model
%\providecommand{\Cs}{{K_{s}}}                    % Multiplicative constant in strong error model
%\providecommand{\Ow}{{q_{w}}}                    % Convergence _order_ in weak error model
%\providecommand{\Os}{{q_{s}}}                    % Twice Convergence _order_ in strong error model
\providecommand{\splitting}{\theta}              % Splitting parameter
\providecommand{\confidence}{\xi}                % compliment to the confidence
\providecommand{\confpar}{C_\xi}                 % Confidence parameter
%\providecommand{\tolbsep}{{C}}                   % Ratio between succesive tolerances in MLMC estimator
\providecommand{\wal}{\ensuremath{{\mathrm{wal}}}} % Walsh function

\DeclarePairedDelimiter\abs{\lvert}{\rvert}%
\DeclarePairedDelimiter\Abs{\left\lvert}{\right\rvert}%
\DeclarePairedDelimiter\norm{\lVert}{\rVert}%

\providecommand{\e}[1]{\ensuremath{\cdot 10^{#1}}}

%\newcommand*{\rom}[1]{\expandafter\@slowromancap\romannumeral #1@}

\newtheorem{Remark}{Remark}
\newtheorem{Definition}{Definition}
\newtheorem{Corollary}{Corollary}
\newtheorem{Assumption}{Assumption}
\newtheorem{Theorem}{Theorem}
\newtheorem{Proposition}{Proposition}
\newtheorem{Lemma}{Lemma}
\newtheorem{Proof}{Proof}

\newcommand{\RN}[1]{%
	\textup{\uppercase\expandafter{\romannumeral#1}}%
}

%\newcolumntype{M}[1]{>{\centering\arraybackslash}m{#1}} % To insert figures in a table
%
%\newcommand*\xoverline[2][0.75]{%
%	\sbox{\myboxA}{$\m@th#2$}%
%	\setbox\myboxB\null% Phantom box
%	\ht\myboxB=\ht\myboxA%
%	\dp\myboxB=\dp\myboxA%
%	\wd\myboxB=#1\wd\myboxA% Scale phantom
%	\sbox\myboxB{$\m@th\overline{\copy\myboxB}$}%  Overlined phantom
%	\setlength\mylenA{\the\wd\myboxA}%   calc width diff
%	\addtolength\mylenA{-\the\wd\myboxB}%
%	\ifdim\wd\myboxB<\wd\myboxA%
%	\rlap{\hskip 0.5\mylenA\usebox\myboxB}{\usebox\myboxA}%
%	\else
%	\hskip -0.5\mylenA\rlap{\usebox\myboxA}{\hskip 0.5\mylenA\usebox\myboxB}%
%	\fi}	%overline, bar
%
%\usepackage[dvipsnames]{xcolor} % Use more colors in the annotation

\title{Integrability of weak mixed first-order derivatives and convergence rates of scrambled digital nets}
 



% \ead{yang.liu.3@kaust.edu.sa}
% \fntext[1]{CEMSE, King Abdullah University of Science and Technology, Thuwal, Saudi Arabia}

% Optional PDF information


%%
%% end of the preamble, start of the body of the document source.
\begin{document}

\author[Y.~Liu]{Yang Liu}
\address[Y.~Liu]{CEMSE, King Abdullah University of Science and Technology, Thuwal, Saudi Arabia} \email[]{yang.liu.3@kaust.edu.sa}
% \amscode{65C05}
  % REQUIRED
%   \begin{AMS}
% 	  65C05 % Monte Carlo methods
%   \end{AMS}
\subjclass{65C05}

%%
%% The code below is generated by the tool at http://dl.acm.org/ccs.cfm.
%% Please copy and paste the code instead of the example below.
%%

%%
%% Keywords. The author(s) should pick words that accurately describe
%% the work being presented. Separate the keywords with commas.
% \keywords{Blue noise sampling, periodic RKHS, }

%\received{20 February 2007}
%\received[revised]{12 March 2009}
%\received[accepted]{5 June 2009}

%%
%% This command processes the author and affiliation and title
%% information and builds the first part of the formatted document.
\maketitle
\begin{abstract}
	We consider the $L^p$ integrability of weak mixed first-order derivatives of the integrand and study convergence rates of scrambled digital nets. We show that the generalized Vitali variation with parameter $\alpha \in [\frac{1}{2}, 1]$ from [Dick and Pillichshammer, 2010] is bounded above by the $L^p$ norm of the weak mixed first-order derivative, where $p = \frac{2}{3-2\alpha}$. Consequently, when the weak mixed first-order derivative belongs to $L^p$ for $1 \leq p \leq 2$, the variance of the scrambled digital nets estimator convergences at a rate of $\mathcal{O}(N^{-4+\frac{2}{p}} \log^{s-1} N)$. Numerical experiments further validate the theoretical results.
\end{abstract}

\section{Introduction}

In this short note, we analyze the variance of the scrambled digital net estimator. Specifically, we consider the $L^p$ integrability of weak, mixed first-order derivatives, where $1\leq p \leq 2$, and establish connections to the generalized Vitali variation proposed in~\cite{dick2010digital}.

Several previous works have studied the convergence rate of the scrambled Sobol' sequence and achieved the rate $\mathcal{O}(N^{-3}\log^{s-1} N)$ for estimator variance under different conditions: Yue and Mao~\cite{yue1999variance} propose a generalized Lipschitz continuity condition on the integrand, whereas Owen~\cite{owen2008local} requires the mixed first-order derivative to be in $L^{\infty}$. Dick and Pillichshammer~\cite{dick2010digital} introduce a generalized Vitali variation to study the scrambled digital net estimator variance and demonstrate that $L^2$ integrability of the mixed first-order derivative is sufficient for achieving the aforementioned convergence rate when the derivative is continuous. In addition, Liu~\cite{liu2024randomized} studies integrands satisfying boundary growth conditions characterized by parameter $A^{*}$ and proves convergence rates of $\mathcal{O}(N^{-2+2A^{*}} \log^{s-1} N)$ for $-1/2 < A^{*} < 1/2$. 

In this work, we establish an estimator variance convergence rate of $\mathcal{O}(N^{-4+\frac{2}{p}} \log^{s-1} N)$ for integrands whose weak derivative $\partial^{1:s} f \in L^{p}$, where $1 \leq p \leq 2$. In the rest of this short note, we introduce necessary notations for the scrambled digital nets and provide convergence analysis, as well as some discussions in Section~\ref{sec:background}. We present some numerical results in~\ref{sec:numex} to support our theoretical results.
\section{Scrambled digital nets and convergence rates}
\label{sec:background}
In this section, we introduce notations for scrambled digital nets and analyze the variance of the estimator.

\subsection{Digital nets}
We begin with the definition of a $(t, m, s)$-net in base $b$. A $(t, m, s)$-net in base $b$ is a set of $b^m$ points in $[0, 1)^s$ such that every $s$-dimensional elementary interval
	\begin{equation}
		\label{eq:elementary_interval}
		E_{\bm{\ell}, \bm{k}} = \prod_{j=1}^{s} \left[ \frac{{k}_j}{b^{{\ell}_j}},  \frac{{k}_j + 1}{b^{{\ell}_j}}\right),
	\end{equation}
	where $\bm{\ell} = ({\ell}_1, \dotsc, {\ell}_s) \in \mathbb{N}_0^s$, {$\abs{\bm{\ell}} := \sum_{j=1}^s {\ell}_j = m - t$} and $\bm{k} = ({k}_1, \dotsc, {k}_s) \in \mathbb{N}_0^s$ with ${k}_j < b^{{\ell}_j}$ for $j = 1, \dotsc, s$, contains exactly {$b^t$} points. Here, the smallest $t$ satisfying the above description is called the quality parameter, $s$ is the dimension, and $b^m$ is the number of quadrature points. Additional properties of $(t, m, s)$-nets can be found in~\cite{dick2010digital}. 

	To randomize digital nets, Owen proposes nested uniform scrambling~\cite{owen1995randomly} for the Sobol' sequence, although it requires substantial storage and computational costs. Matou\v sek~\cite{matouvsek1998thel2} proposes a random linear scramble as a more computationally efficient approach that does not alter the mean squared $L^2$ discrepancy. A review of other randomization methods can be found in~\cite{owen2003variance}. In this short note, we focus on the Owen-scrambled digital nets.

\subsection{Variance of the scrambled digital nets}
\label{sec:spectral_analysis_rqmc}
We denote by $I_N \coloneqq I_N (f)$ the Owen-scrambled $(t, m, s)$-net in base $b$ integration estimator for a general integrand $f$ defined on $[0, 1]^s$, where $N = b^{m}$. Following the derivations in~\cite{owen1997monte, dick2010digital}, the variance of the estimator ${I_N}$ is given by:
\begin{equation}
	\var{I_N} = \sum_{\bm{\ell} \in \mathbb{N}_0^s} \Gamma_{\bm{\ell}} \sigma^2_{\bm{\ell}},
\end{equation} 
where $\Gamma_{\bm{\ell}}$ is the gain coefficient, and $\sigma^2_{\bm{\ell}}$ represents the sum of squared Walsh coefficients over specific indices.

Next, we introduce the notation for the alternating sum. Given an interval $J = \prod_{j=1}^s [a_j, b_j]$, we define the alternating sum $\Delta (f, J)$ by
\begin{equation}
	\label{eq:alternating_sum}
	\Delta (f, J) = \sum_{\mathfrak{u} \subseteq 1:s} (-1)^{\abs{\mathfrak{u}}} f(\bm{a}^\mathfrak{u}: \bm{b}^{-\mathfrak{u}}),
\end{equation}
where $\bm{a} = (a_1, \dotsc, a_s)$ and $\bm{b} = (b_1, \dotsc, b_s)$. The expression $\bm{a}^\mathfrak{u}: \bm{b}^{-\mathfrak{u}}$ denotes the concatenation of two vectors such that the $j$-th component, $(\bm{a}^\mathfrak{u}: \bm{b}^{-\mathfrak{u}})_j$ equals $a_j$ if $j \in \mathfrak{u}$ and $b_j$ otherwise.

Using the alternating sum notation, we present the following definition.
\begin{Definition}[Generalized Vitali variation of order 2]
	In~\cite{dick2010digital}, the authors define the generalized Vitali variation of order $2$ as
	\begin{equation}
		\label{eq:generalized_variation_vitali}
		V_{\alpha} (f) = \sup_{\mathcal{P}} \left( \sum_{J \in \mathcal{P}} \mu(J) \abs*{\frac{{\Delta (f, J)}}{\mu(J)^{\alpha}} }^{2} \right)^{\frac{1}{2}},
	\end{equation}
	{where $0 < \alpha \leq 1$, the supremum is taken over all partitions $\mathcal{P}$ of $[0, 1]^s$ into axis-parallel subintervals, $\mu(J)$ denotes the Lebesgue measure of $J$, and $\Delta (f, J)$ is defined in~\eqref{eq:alternating_sum}.}
\end{Definition}
Following the derivations in~\cite{dick2010digital}, the variance $\var{I_N}$ can be bounded in terms of the generalized Vitali variation, as stated in the following proposition.
\begin{Proposition}[Variance of the scrambled $(t,m,s)$-net in base $b$ estimator]
	Following~\cite{dick2010digital}, the variance of the scrambled digital nets estimator $I_N$ satisfies
	\begin{equation}
		\var{I_{N}} \leq C_{\alpha, b, s, t} V^2_{\alpha}(f) N^{-1-2\alpha} \log^{s-1} N,
	\end{equation}
where the constant $C_{\alpha, b, s, t} < +\infty$ depends on $\alpha, b, s$ and $t$. 
\end{Proposition}
Let $\norm{\cdot}_p$ denote the $L^p$ norm on $[0, 1]^s$. Our goal is to show
\begin{equation*}
	V_{\alpha}(f) \leq \norm{\partial^{1:s} f}_{p},
\end{equation*}
where $\partial^{1:s} f$ denotes the weak mixed first-order derivative of $f$, $p = \frac{2}{3-2\alpha}$ for $\frac{1}{2} \leq \alpha \leq 1$. Consequently, the variance $\var{I_N}$ can be connected to the $L^p$ integrability of the weak derivative. For a general reference on weak derivatives, see~\cite{evans2022partial}. 

We present the following lemma connecting the alternating sum and the weak derivative.
\begin{Lemma}[Alternating sum and the weak derivative]
\label{lemma:weak_derivative_alternating_sum}
For a continuous integrand $f$ whose weak derivative $\partial^{1:s}f$ exists on an axis-parallel interval $J = \prod_{j=1}^s [a_j, b_j]$ with $0 \leq a_j < b_j \leq 1$, for $j = 1, \dotsc, s$, we have
\begin{equation}
	{{\Delta (f, J)}} = \int_{J} {\partial^{1:s} f} (\bm{t}) d\bm{t},
\end{equation}
where $\partial^{1:s} f$ denotes the weak derivative.
\end{Lemma}
% \end{lemma}
Lemma~\ref{lemma:weak_derivative_alternating_sum} extends the results presented in~\cite{owen2005multidimensional}, where the derivative is defined in the usual sense on $J$. We present the proof below.
\begin{proof}
	When $f$ is continuous and weakly differentiable on $J$, Theorem 8.2 in~\cite{brezis2011functional} presents the following result for the 1-d case:
	\begin{equation}
		f(b) - f(a) = \int_{a}^{b} f^{\prime} (t) dt.
	\end{equation}
	We now proceed by induction. Decompose $J$ as $J=[a_1, b_1] \times J_{-1}$ with $J_{-1} = \prod_{j = 2}^s [a_j, b_j]$. Denote by $f(\cdot \mid t_1 = \tau)$ the restriction of $f(\bm{t})$ to $f(\tau, t_2, \dotsc, t_{s})$ for $\tau \in [0, 1]$. We have
	\begin{equation}
		\begin{split}
			{{\Delta (f, J)}} &= {\Delta (f(\cdot \mid t_1 = b_1), J_{-1})} - {\Delta (f(\cdot \mid t_1 = a_1), J_{-1})}\\
			&= \int_{a_1}^{b_1} \frac{\partial}{\partial \tau} \Delta (f(\cdot \mid t_1 = \tau), J_{-1}) d\tau\\
			&= \int_{a_1}^{b_1} \frac{\partial}{\partial \tau} \int_{J_{-1}} \frac{\partial f(\cdot \mid t_1 = \tau) }{\partial \bm{t}_{2:s}} d\bm{t}_{2:s} d \tau\\
			&= \int_{J} \partial^{1:s} f(\bm{t}) d\bm{t},
		\end{split}
	\end{equation}
	where in the third line we apply the induction hypothesis in $(s-1)$ dimensions, and in the fourth line we exchange the weak derivative and integration~\cite{cheng2006differentiation}. This concludes the proof. 
\end{proof}
In the following, we show that $V_{\alpha}(f)$ is bounded by $\norm{\partial^{1:s} f}_{p}$ with $p = \frac{2}{3-2\alpha}$. First, we apply H\"older's inequality to obtain
\begin{equation}
	{{\Delta (f, J)}} = \int_{J} \partial^{1:s}f(\bm{t}) d\bm{t} \leq  \norm{ \partial^{1:s} f \cdot \mathbbm{1}_{J}}_p \ \mu(J)^{1-\frac{1}{p}},
\end{equation}
where $\mathbbm{1}_{J}$ denotes the indicator function over the set $J$, taking the value 1 inside $J$ and 0 otherwise. Thus, for all partitions $\mathcal{P}$ of $[0, 1]^s$ into axis-parallel subintervals, we have
\begin{equation*}
	\sum_{J \in \mathcal{P}} \mu(J) \abs*{\frac{{\Delta (f, J)}}{\mu(J)^{\alpha}} }^{2} \leq \sum_{J \in \mathcal{P}} \mu(J)^{3-2\alpha - \frac{2}{p}} \norm{ \partial^{1:s} f \cdot \mathbbm{1}_{J}}_p^2.
\end{equation*}
Observe the decomposition of the $p$-th power of the $L^p$ norm over the partition $\mathcal{P}$:
\begin{equation*}
	\norm{ \partial^{1:s} f  }_p^p = \sum_{J \in \mathcal{P}} \norm{ \partial^{1:s} f \cdot \mathbbm{1}_{J}}_p^p.  
\end{equation*}
When $p \leq 2$, we have the following superadditivity condition:
\begin{equation}
	% \begin{split}
		\left(\norm{ \partial^{1:s} f  }_p^p \right)^{\frac{2}{p}} = \left( \sum_{J \in \mathcal{P}} \norm{ \partial^{1:s} f \cdot \mathbbm{1}_{J}}_p^p \right)^{\frac{2}{p}} \geq  \sum_{J \in \mathcal{P}} \norm{ \partial^{1:s} f \cdot \mathbbm{1}_{J}}_p^2.
	% \end{split}
\end{equation}
Finally, we obtain
\begin{equation}
	V_{\alpha} (f) = \sup_{\mathcal{P}} \left( \sum_{J \in \mathcal{P}} \mu(J) \abs*{\frac{{\Delta (f, J)}}{\mu(J)^{\alpha}} }^{2} \right)^{\frac{1}{2}} \leq \sup_{\mathcal{P}} \left( \sum_{J \in \mathcal{P}} \norm{ \partial^{1:s} f \cdot \mathbbm{1}_{J}}_p^2 \right)^{\frac{1}{2}} \leq \norm{ \partial^{1:s} f  }_p.
\end{equation}
Thus, we establish a connection between the generalized Vitali variation and the $L^p$ integrability of the mixed first-order derivatives. Moreover, the convergence rate of the estimator can inferred from the $L^p$ the integrability of the weak derivative $\partial^{1:s} f$, for $1 \leq p \leq 2$. Since we consider the weak derivative, the analysis accommodates a broader class of continuous functions, such as the functions with kinks. Meanwhile, discontinuous functions are studied in~\cite{he2015convergence} and more recently in~\cite{liu2024randomized}. We conclude this short note by summarizing our contributions in the following remark.
\begin{remark}[On the convergence rate of the scrambled digital nets]
	Notice that while the generalized Vitali variation of order 2 is upper bounded by the $L^p$ norm of the derivative, the looser upper bound derived in this work provides a more tractable approach for determining the convergence rates. Specifically, our work extends prior studies in the following two aspects:
	\begin{itemize}
		\item The estimator variance convergence rate $\mathcal{O}(N^{-3} \log^{s-1} N)$ under weaker regularity: A sufficient condition is the $L^2$-integrability of the mixed first-order weak $\partial^{1:s} f$, which generalizes a sufficient condition derived in~\cite{dick2010digital} that requires the continuous derivative in the strong sense to be in $L^2$.
		\item Generalized conditions for rates $\mathcal{O}(N^{-2-\delta} \log^{s-1} N) (0 < \delta < 1)$: This work generalizes the boundary growth conditions considered in~\cite{liu2024randomized}, which model the behavior of the derivative near the boundary, to the integrability conditions on the weak derivatives.
	\end{itemize}
	These extensions can be useful for practitioners to determine the convergence rates of the scrambled digital nets for a broader class of integrands.
\end{remark}

% Notice that we have
% \begin{equation}
% 	\abs*{\Delta (f, J)} \leq \abs*{\int_{J} \frac{\partial^s f}{\partial x_1 \cdots \partial x_s} \bm{x} d\bm{x}} \leq \mu(J) \max_{\bm{x} \in \bar{J} } \abs*{\frac{\partial^s f}{\partial x_1 \cdots \partial x_s} (\bm{x})}.
% \end{equation}

\appendix

\section{Numerical Examples}
\label{sec:numex}
We present some numerical examples in the appendix. Specifically, we consider two integrands with kinks. All the numerical simulations use the Sobol' sequence with Matou\v{s}ek-scrambling~\cite{matouvsek1998thel2}, as implemented in the \texttt{scipy.qmc} module~\cite{2020SciPy-NMeth}. 
\subsection*{Example 1}
In Example 1, we consider an integrand of the form:
\begin{equation}
	f(\bm{t}) = \prod_{j=1}^s \abs*{t_j - \dfrac{1}{2}}^{\alpha}, \quad \alpha > 0,
\end{equation}
where the exact integration value is $\frac{1}{2^{\alpha s} (\alpha + 1)^s}$.
Notice that in this example, the integrand kinks are axis-parallel. When $0 < \alpha \leq \frac{1}{2}$, the weak derivative $\partial^{1:s} f$ is in $L^{-\frac{1}{\alpha - 1} - \epsilon}$ for any arbitrarily small $\epsilon > 0$ and the convergence rate for the estimator variance is $\mathcal{O}(N^{-2-{2}{\alpha} + \epsilon} \log^{s-1} N)$ for any arbitrarily small $\epsilon > 0$.
% In Figure~\ref{fig:numerical_results_example1}, we present the numerical results for Example 1. 
% The results show that the variance of the scrambled digital nets estimator converges at the rate $\mathcal{O}(N^{-3} \log^{s-1} N)$ when the weak derivative belongs to $L^2$.
\begin{figure}[htbp]
	\centering
	\begin{subfigure}{0.32\textwidth}
		\includegraphics[width=\textwidth]{Figures_Weak_QMC/Ex2_s2_1_2}
		\caption{$s = 2, \alpha = \frac{1}{2}$.}
	\end{subfigure}
	\begin{subfigure}{0.32\textwidth}
		\includegraphics[width=\textwidth]{Figures_Weak_QMC/Ex2_s2_1_3}
		\caption{$s = 2, \alpha = \frac{1}{3}$.}
	\end{subfigure}
	\begin{subfigure}{0.32\textwidth}
		\includegraphics[width=\textwidth]{Figures_Weak_QMC/Ex2_s2_1_5}
		\caption{$s = 2, \alpha = \frac{1}{5}$.}
	\end{subfigure}
	\\
	\begin{subfigure}{0.32\textwidth}
		\includegraphics[width=\textwidth]{Figures_Weak_QMC/Ex2_s3_1_2}
		\caption{$s = 3, \alpha = \frac{1}{2}$.}
	\end{subfigure}
	\begin{subfigure}{0.32\textwidth}
		\includegraphics[width=\textwidth]{Figures_Weak_QMC/Ex2_s3_1_3}
		\caption{$s = 3, \alpha = \frac{1}{3}$.}
	\end{subfigure}
	\begin{subfigure}{0.32\textwidth}
		\includegraphics[width=\textwidth]{Figures_Weak_QMC/Ex2_s3_1_5}
		\caption{$s = 3, \alpha = \frac{1}{5}$.}
	\end{subfigure}
	\\
	\begin{subfigure}{0.32\textwidth}
		\includegraphics[width=\textwidth]{Figures_Weak_QMC/Ex2_s5_1_2}
		\caption{$s = 5, \alpha = \frac{1}{2}$.}
	\end{subfigure}
	\begin{subfigure}{0.32\textwidth}
		\includegraphics[width=\textwidth]{Figures_Weak_QMC/Ex2_s5_1_3}
		\caption{$s = 5, \alpha = \frac{1}{3}$.}
	\end{subfigure}
	\begin{subfigure}{0.32\textwidth}
		\includegraphics[width=\textwidth]{Figures_Weak_QMC/Ex2_s5_1_5}
		\caption{$s = 5, \alpha = \frac{1}{5}$.}
	\end{subfigure}
	\caption{Example 1: The boxplot characterization of squared error distributions, $(I_N - I)^2$, for scrambled Sobol' sequence estimators across various dimensions $s$ and parameters $\alpha$. Each whisker in the boxplot extends from the 1st to 99th percentile of 8,192 independent realizations of the squared errors.}
	\label{fig:numerical_results_example1}
\end{figure}

Figure~\ref{fig:numerical_results_example1} presents the squared errors $(I - I_N)^2$ of the scrambled digital net estimators for a range of $N$ for various dimensions $s$ and parameter $\alpha$. The empirical convergence rates for the cases $s=2$ with $\alpha = \frac{1}{2}, \frac{1}{3}, \frac{1}{5}$ are close to the asymptotic rates $\mathcal{O}(N^{-2-2\alpha + \epsilon})$, where one can almost neglect the effect from the logarithmic in the complexity. For the cases $s=3$ and $s=5$, there are more nonasymptotic effects from the logarithmic term in the complexity.
\subsection*{Example 2}
In Example 2, we consider an integrand of the form:
\begin{equation}
	f(\bm{t}) = \max\left(\sum_{j=1}^s t_j - 1, 0\right)^{s + \alpha}, \quad \alpha > -1.
\end{equation}
In this example, the kinks are along the hyperplane $\sum_{j=1}^s t_j = 1$. When $-1 < \alpha \leq -\frac{1}{2}$, the weak derivative $\partial^{1:s}$ is in $L^{-\frac{1}{\alpha} - \epsilon}$ for any arbitrarily small $\epsilon > 0$ and the convergence rate for the estimator variance is $\mathcal{O}(N^{-4-{2}{\alpha} + \epsilon} \log^{s-1} N)$ for any arbitrarily small $\epsilon > 0$.

Figure~\ref{fig:numerical_results_example2} presents the numerical results for Example 2, where the reference values are computed with an ensemble average of 8,192 independent realizations of scrambled Sobol' sequence estimator with quadratures size $N = 2^{25}$. For the cases $s = 2$ and $s = 3$, when $\alpha = -0.5$, empirical rates approach $\mathcal{O}(N^{-3 + \epsilon})$, aligning theoretical predictions. For the cases $\alpha = -0.7$ and $\alpha = -0.9$, the convergence rates exceed our a priori estimate rates, which suggests the upper bound derived in this work may not be tight. More detailed analysis for this type of integrand is left for future work. For the case $s = 5$, the nonasymptotic effects become more profound due to the increased integration dimension.

\begin{figure}[htbp]
	\centering
	\begin{subfigure}{0.32\textwidth}
		\includegraphics[width=\textwidth]{Figures_Weak_QMC/Ex1_s2_m5e-1}
		\caption{$s = 2, \alpha = -0.5$.}
	\end{subfigure}
	\begin{subfigure}{0.32\textwidth}
		\includegraphics[width=\textwidth]{Figures_Weak_QMC/Ex1_s2_m7e-1}
		\caption{$s = 2, \alpha = -0.7$.}
	\end{subfigure}
	\begin{subfigure}{0.32\textwidth}
		\includegraphics[width=\textwidth]{Figures_Weak_QMC/Ex1_s2_m9e-1}
		\caption{$s = 2, \alpha = -0.9$.}
	\end{subfigure}
	\\
	\begin{subfigure}{0.32\textwidth}
		\includegraphics[width=\textwidth]{Figures_Weak_QMC/Ex1_s3_m5e-1}
		\caption{$s = 3, \alpha = -0.5$.}
	\end{subfigure}
	\begin{subfigure}{0.32\textwidth}
		\includegraphics[width=\textwidth]{Figures_Weak_QMC/Ex1_s3_m7e-1}
		\caption{$s = 3, \alpha = -0.7$.}
	\end{subfigure}
	\begin{subfigure}{0.32\textwidth}
		\includegraphics[width=\textwidth]{Figures_Weak_QMC/Ex1_s3_m9e-1}
		\caption{$s = 3, \alpha = -0.9$.}
	\end{subfigure}
	\\
	\begin{subfigure}{0.32\textwidth}
		\includegraphics[width=\textwidth]{Figures_Weak_QMC/Ex1_s5_m5e-1}
		\caption{$s = 5, \alpha = -0.5$.}
	\end{subfigure}
	\begin{subfigure}{0.32\textwidth}
		\includegraphics[width=\textwidth]{Figures_Weak_QMC/Ex1_s5_m7e-1}
		\caption{$s = 5, \alpha = -0.7$.}
	\end{subfigure}
	\begin{subfigure}{0.32\textwidth}
		\includegraphics[width=\textwidth]{Figures_Weak_QMC/Ex1_s5_m9e-1}
		\caption{$s = 5, \alpha = -0.9$.}
	\end{subfigure}
	\caption{Example 2: The boxplot characterization of squared error distributions, $(I_N - I)^2$, for scrambled Sobol' sequence estimators across various dimensions $s$ and parameters $\alpha$. The reference value $I$ is approximated by averaging 8,192 independent realizations of scrambled Sobol' sequence estimators with quadrature size $N = 2^{25}$. Each whisker in the boxplot extends from the 1st to 99th percentile of 8,192 independent realizations.}
	\label{fig:numerical_results_example2}
\end{figure}

%%
%% The next two lines define the bibliography style to be used, and
%% the bibliography file.
\bibliographystyle{siam}
\bibliography{sampling, bibliography_QMC_theory, bibliography_QMC_finance, hoqmc}


%%
%% If your work has an appendix, this is the place to put it.
% \subsection{Lloyd-Max Algorithm}
\label{subsec:Lloyd-Max}
For a given quantization bitwidth $B$ and an operand $\bm{X}$, the Lloyd-Max algorithm finds $2^B$ quantization levels $\{\hat{x}_i\}_{i=1}^{2^B}$ such that quantizing $\bm{X}$ by rounding each scalar in $\bm{X}$ to the nearest quantization level minimizes the quantization MSE. 

The algorithm starts with an initial guess of quantization levels and then iteratively computes quantization thresholds $\{\tau_i\}_{i=1}^{2^B-1}$ and updates quantization levels $\{\hat{x}_i\}_{i=1}^{2^B}$. Specifically, at iteration $n$, thresholds are set to the midpoints of the previous iteration's levels:
\begin{align*}
    \tau_i^{(n)}=\frac{\hat{x}_i^{(n-1)}+\hat{x}_{i+1}^{(n-1)}}2 \text{ for } i=1\ldots 2^B-1
\end{align*}
Subsequently, the quantization levels are re-computed as conditional means of the data regions defined by the new thresholds:
\begin{align*}
    \hat{x}_i^{(n)}=\mathbb{E}\left[ \bm{X} \big| \bm{X}\in [\tau_{i-1}^{(n)},\tau_i^{(n)}] \right] \text{ for } i=1\ldots 2^B
\end{align*}
where to satisfy boundary conditions we have $\tau_0=-\infty$ and $\tau_{2^B}=\infty$. The algorithm iterates the above steps until convergence.

Figure \ref{fig:lm_quant} compares the quantization levels of a $7$-bit floating point (E3M3) quantizer (left) to a $7$-bit Lloyd-Max quantizer (right) when quantizing a layer of weights from the GPT3-126M model at a per-tensor granularity. As shown, the Lloyd-Max quantizer achieves substantially lower quantization MSE. Further, Table \ref{tab:FP7_vs_LM7} shows the superior perplexity achieved by Lloyd-Max quantizers for bitwidths of $7$, $6$ and $5$. The difference between the quantizers is clear at 5 bits, where per-tensor FP quantization incurs a drastic and unacceptable increase in perplexity, while Lloyd-Max quantization incurs a much smaller increase. Nevertheless, we note that even the optimal Lloyd-Max quantizer incurs a notable ($\sim 1.5$) increase in perplexity due to the coarse granularity of quantization. 

\begin{figure}[h]
  \centering
  \includegraphics[width=0.7\linewidth]{sections/figures/LM7_FP7.pdf}
  \caption{\small Quantization levels and the corresponding quantization MSE of Floating Point (left) vs Lloyd-Max (right) Quantizers for a layer of weights in the GPT3-126M model.}
  \label{fig:lm_quant}
\end{figure}

\begin{table}[h]\scriptsize
\begin{center}
\caption{\label{tab:FP7_vs_LM7} \small Comparing perplexity (lower is better) achieved by floating point quantizers and Lloyd-Max quantizers on a GPT3-126M model for the Wikitext-103 dataset.}
\begin{tabular}{c|cc|c}
\hline
 \multirow{2}{*}{\textbf{Bitwidth}} & \multicolumn{2}{|c|}{\textbf{Floating-Point Quantizer}} & \textbf{Lloyd-Max Quantizer} \\
 & Best Format & Wikitext-103 Perplexity & Wikitext-103 Perplexity \\
\hline
7 & E3M3 & 18.32 & 18.27 \\
6 & E3M2 & 19.07 & 18.51 \\
5 & E4M0 & 43.89 & 19.71 \\
\hline
\end{tabular}
\end{center}
\end{table}

\subsection{Proof of Local Optimality of LO-BCQ}
\label{subsec:lobcq_opt_proof}
For a given block $\bm{b}_j$, the quantization MSE during LO-BCQ can be empirically evaluated as $\frac{1}{L_b}\lVert \bm{b}_j- \bm{\hat{b}}_j\rVert^2_2$ where $\bm{\hat{b}}_j$ is computed from equation (\ref{eq:clustered_quantization_definition}) as $C_{f(\bm{b}_j)}(\bm{b}_j)$. Further, for a given block cluster $\mathcal{B}_i$, we compute the quantization MSE as $\frac{1}{|\mathcal{B}_{i}|}\sum_{\bm{b} \in \mathcal{B}_{i}} \frac{1}{L_b}\lVert \bm{b}- C_i^{(n)}(\bm{b})\rVert^2_2$. Therefore, at the end of iteration $n$, we evaluate the overall quantization MSE $J^{(n)}$ for a given operand $\bm{X}$ composed of $N_c$ block clusters as:
\begin{align*}
    \label{eq:mse_iter_n}
    J^{(n)} = \frac{1}{N_c} \sum_{i=1}^{N_c} \frac{1}{|\mathcal{B}_{i}^{(n)}|}\sum_{\bm{v} \in \mathcal{B}_{i}^{(n)}} \frac{1}{L_b}\lVert \bm{b}- B_i^{(n)}(\bm{b})\rVert^2_2
\end{align*}

At the end of iteration $n$, the codebooks are updated from $\mathcal{C}^{(n-1)}$ to $\mathcal{C}^{(n)}$. However, the mapping of a given vector $\bm{b}_j$ to quantizers $\mathcal{C}^{(n)}$ remains as  $f^{(n)}(\bm{b}_j)$. At the next iteration, during the vector clustering step, $f^{(n+1)}(\bm{b}_j)$ finds new mapping of $\bm{b}_j$ to updated codebooks $\mathcal{C}^{(n)}$ such that the quantization MSE over the candidate codebooks is minimized. Therefore, we obtain the following result for $\bm{b}_j$:
\begin{align*}
\frac{1}{L_b}\lVert \bm{b}_j - C_{f^{(n+1)}(\bm{b}_j)}^{(n)}(\bm{b}_j)\rVert^2_2 \le \frac{1}{L_b}\lVert \bm{b}_j - C_{f^{(n)}(\bm{b}_j)}^{(n)}(\bm{b}_j)\rVert^2_2
\end{align*}

That is, quantizing $\bm{b}_j$ at the end of the block clustering step of iteration $n+1$ results in lower quantization MSE compared to quantizing at the end of iteration $n$. Since this is true for all $\bm{b} \in \bm{X}$, we assert the following:
\begin{equation}
\begin{split}
\label{eq:mse_ineq_1}
    \tilde{J}^{(n+1)} &= \frac{1}{N_c} \sum_{i=1}^{N_c} \frac{1}{|\mathcal{B}_{i}^{(n+1)}|}\sum_{\bm{b} \in \mathcal{B}_{i}^{(n+1)}} \frac{1}{L_b}\lVert \bm{b} - C_i^{(n)}(b)\rVert^2_2 \le J^{(n)}
\end{split}
\end{equation}
where $\tilde{J}^{(n+1)}$ is the the quantization MSE after the vector clustering step at iteration $n+1$.

Next, during the codebook update step (\ref{eq:quantizers_update}) at iteration $n+1$, the per-cluster codebooks $\mathcal{C}^{(n)}$ are updated to $\mathcal{C}^{(n+1)}$ by invoking the Lloyd-Max algorithm \citep{Lloyd}. We know that for any given value distribution, the Lloyd-Max algorithm minimizes the quantization MSE. Therefore, for a given vector cluster $\mathcal{B}_i$ we obtain the following result:

\begin{equation}
    \frac{1}{|\mathcal{B}_{i}^{(n+1)}|}\sum_{\bm{b} \in \mathcal{B}_{i}^{(n+1)}} \frac{1}{L_b}\lVert \bm{b}- C_i^{(n+1)}(\bm{b})\rVert^2_2 \le \frac{1}{|\mathcal{B}_{i}^{(n+1)}|}\sum_{\bm{b} \in \mathcal{B}_{i}^{(n+1)}} \frac{1}{L_b}\lVert \bm{b}- C_i^{(n)}(\bm{b})\rVert^2_2
\end{equation}

The above equation states that quantizing the given block cluster $\mathcal{B}_i$ after updating the associated codebook from $C_i^{(n)}$ to $C_i^{(n+1)}$ results in lower quantization MSE. Since this is true for all the block clusters, we derive the following result: 
\begin{equation}
\begin{split}
\label{eq:mse_ineq_2}
     J^{(n+1)} &= \frac{1}{N_c} \sum_{i=1}^{N_c} \frac{1}{|\mathcal{B}_{i}^{(n+1)}|}\sum_{\bm{b} \in \mathcal{B}_{i}^{(n+1)}} \frac{1}{L_b}\lVert \bm{b}- C_i^{(n+1)}(\bm{b})\rVert^2_2  \le \tilde{J}^{(n+1)}   
\end{split}
\end{equation}

Following (\ref{eq:mse_ineq_1}) and (\ref{eq:mse_ineq_2}), we find that the quantization MSE is non-increasing for each iteration, that is, $J^{(1)} \ge J^{(2)} \ge J^{(3)} \ge \ldots \ge J^{(M)}$ where $M$ is the maximum number of iterations. 
%Therefore, we can say that if the algorithm converges, then it must be that it has converged to a local minimum. 
\hfill $\blacksquare$


\begin{figure}
    \begin{center}
    \includegraphics[width=0.5\textwidth]{sections//figures/mse_vs_iter.pdf}
    \end{center}
    \caption{\small NMSE vs iterations during LO-BCQ compared to other block quantization proposals}
    \label{fig:nmse_vs_iter}
\end{figure}

Figure \ref{fig:nmse_vs_iter} shows the empirical convergence of LO-BCQ across several block lengths and number of codebooks. Also, the MSE achieved by LO-BCQ is compared to baselines such as MXFP and VSQ. As shown, LO-BCQ converges to a lower MSE than the baselines. Further, we achieve better convergence for larger number of codebooks ($N_c$) and for a smaller block length ($L_b$), both of which increase the bitwidth of BCQ (see Eq \ref{eq:bitwidth_bcq}).


\subsection{Additional Accuracy Results}
%Table \ref{tab:lobcq_config} lists the various LOBCQ configurations and their corresponding bitwidths.
\begin{table}
\setlength{\tabcolsep}{4.75pt}
\begin{center}
\caption{\label{tab:lobcq_config} Various LO-BCQ configurations and their bitwidths.}
\begin{tabular}{|c||c|c|c|c||c|c||c|} 
\hline
 & \multicolumn{4}{|c||}{$L_b=8$} & \multicolumn{2}{|c||}{$L_b=4$} & $L_b=2$ \\
 \hline
 \backslashbox{$L_A$\kern-1em}{\kern-1em$N_c$} & 2 & 4 & 8 & 16 & 2 & 4 & 2 \\
 \hline
 64 & 4.25 & 4.375 & 4.5 & 4.625 & 4.375 & 4.625 & 4.625\\
 \hline
 32 & 4.375 & 4.5 & 4.625& 4.75 & 4.5 & 4.75 & 4.75 \\
 \hline
 16 & 4.625 & 4.75& 4.875 & 5 & 4.75 & 5 & 5 \\
 \hline
\end{tabular}
\end{center}
\end{table}

%\subsection{Perplexity achieved by various LO-BCQ configurations on Wikitext-103 dataset}

\begin{table} \centering
\begin{tabular}{|c||c|c|c|c||c|c||c|} 
\hline
 $L_b \rightarrow$& \multicolumn{4}{c||}{8} & \multicolumn{2}{c||}{4} & 2\\
 \hline
 \backslashbox{$L_A$\kern-1em}{\kern-1em$N_c$} & 2 & 4 & 8 & 16 & 2 & 4 & 2  \\
 %$N_c \rightarrow$ & 2 & 4 & 8 & 16 & 2 & 4 & 2 \\
 \hline
 \hline
 \multicolumn{8}{c}{GPT3-1.3B (FP32 PPL = 9.98)} \\ 
 \hline
 \hline
 64 & 10.40 & 10.23 & 10.17 & 10.15 &  10.28 & 10.18 & 10.19 \\
 \hline
 32 & 10.25 & 10.20 & 10.15 & 10.12 &  10.23 & 10.17 & 10.17 \\
 \hline
 16 & 10.22 & 10.16 & 10.10 & 10.09 &  10.21 & 10.14 & 10.16 \\
 \hline
  \hline
 \multicolumn{8}{c}{GPT3-8B (FP32 PPL = 7.38)} \\ 
 \hline
 \hline
 64 & 7.61 & 7.52 & 7.48 &  7.47 &  7.55 &  7.49 & 7.50 \\
 \hline
 32 & 7.52 & 7.50 & 7.46 &  7.45 &  7.52 &  7.48 & 7.48  \\
 \hline
 16 & 7.51 & 7.48 & 7.44 &  7.44 &  7.51 &  7.49 & 7.47  \\
 \hline
\end{tabular}
\caption{\label{tab:ppl_gpt3_abalation} Wikitext-103 perplexity across GPT3-1.3B and 8B models.}
\end{table}

\begin{table} \centering
\begin{tabular}{|c||c|c|c|c||} 
\hline
 $L_b \rightarrow$& \multicolumn{4}{c||}{8}\\
 \hline
 \backslashbox{$L_A$\kern-1em}{\kern-1em$N_c$} & 2 & 4 & 8 & 16 \\
 %$N_c \rightarrow$ & 2 & 4 & 8 & 16 & 2 & 4 & 2 \\
 \hline
 \hline
 \multicolumn{5}{|c|}{Llama2-7B (FP32 PPL = 5.06)} \\ 
 \hline
 \hline
 64 & 5.31 & 5.26 & 5.19 & 5.18  \\
 \hline
 32 & 5.23 & 5.25 & 5.18 & 5.15  \\
 \hline
 16 & 5.23 & 5.19 & 5.16 & 5.14  \\
 \hline
 \multicolumn{5}{|c|}{Nemotron4-15B (FP32 PPL = 5.87)} \\ 
 \hline
 \hline
 64  & 6.3 & 6.20 & 6.13 & 6.08  \\
 \hline
 32  & 6.24 & 6.12 & 6.07 & 6.03  \\
 \hline
 16  & 6.12 & 6.14 & 6.04 & 6.02  \\
 \hline
 \multicolumn{5}{|c|}{Nemotron4-340B (FP32 PPL = 3.48)} \\ 
 \hline
 \hline
 64 & 3.67 & 3.62 & 3.60 & 3.59 \\
 \hline
 32 & 3.63 & 3.61 & 3.59 & 3.56 \\
 \hline
 16 & 3.61 & 3.58 & 3.57 & 3.55 \\
 \hline
\end{tabular}
\caption{\label{tab:ppl_llama7B_nemo15B} Wikitext-103 perplexity compared to FP32 baseline in Llama2-7B and Nemotron4-15B, 340B models}
\end{table}

%\subsection{Perplexity achieved by various LO-BCQ configurations on MMLU dataset}


\begin{table} \centering
\begin{tabular}{|c||c|c|c|c||c|c|c|c|} 
\hline
 $L_b \rightarrow$& \multicolumn{4}{c||}{8} & \multicolumn{4}{c||}{8}\\
 \hline
 \backslashbox{$L_A$\kern-1em}{\kern-1em$N_c$} & 2 & 4 & 8 & 16 & 2 & 4 & 8 & 16  \\
 %$N_c \rightarrow$ & 2 & 4 & 8 & 16 & 2 & 4 & 2 \\
 \hline
 \hline
 \multicolumn{5}{|c|}{Llama2-7B (FP32 Accuracy = 45.8\%)} & \multicolumn{4}{|c|}{Llama2-70B (FP32 Accuracy = 69.12\%)} \\ 
 \hline
 \hline
 64 & 43.9 & 43.4 & 43.9 & 44.9 & 68.07 & 68.27 & 68.17 & 68.75 \\
 \hline
 32 & 44.5 & 43.8 & 44.9 & 44.5 & 68.37 & 68.51 & 68.35 & 68.27  \\
 \hline
 16 & 43.9 & 42.7 & 44.9 & 45 & 68.12 & 68.77 & 68.31 & 68.59  \\
 \hline
 \hline
 \multicolumn{5}{|c|}{GPT3-22B (FP32 Accuracy = 38.75\%)} & \multicolumn{4}{|c|}{Nemotron4-15B (FP32 Accuracy = 64.3\%)} \\ 
 \hline
 \hline
 64 & 36.71 & 38.85 & 38.13 & 38.92 & 63.17 & 62.36 & 63.72 & 64.09 \\
 \hline
 32 & 37.95 & 38.69 & 39.45 & 38.34 & 64.05 & 62.30 & 63.8 & 64.33  \\
 \hline
 16 & 38.88 & 38.80 & 38.31 & 38.92 & 63.22 & 63.51 & 63.93 & 64.43  \\
 \hline
\end{tabular}
\caption{\label{tab:mmlu_abalation} Accuracy on MMLU dataset across GPT3-22B, Llama2-7B, 70B and Nemotron4-15B models.}
\end{table}


%\subsection{Perplexity achieved by various LO-BCQ configurations on LM evaluation harness}

\begin{table} \centering
\begin{tabular}{|c||c|c|c|c||c|c|c|c|} 
\hline
 $L_b \rightarrow$& \multicolumn{4}{c||}{8} & \multicolumn{4}{c||}{8}\\
 \hline
 \backslashbox{$L_A$\kern-1em}{\kern-1em$N_c$} & 2 & 4 & 8 & 16 & 2 & 4 & 8 & 16  \\
 %$N_c \rightarrow$ & 2 & 4 & 8 & 16 & 2 & 4 & 2 \\
 \hline
 \hline
 \multicolumn{5}{|c|}{Race (FP32 Accuracy = 37.51\%)} & \multicolumn{4}{|c|}{Boolq (FP32 Accuracy = 64.62\%)} \\ 
 \hline
 \hline
 64 & 36.94 & 37.13 & 36.27 & 37.13 & 63.73 & 62.26 & 63.49 & 63.36 \\
 \hline
 32 & 37.03 & 36.36 & 36.08 & 37.03 & 62.54 & 63.51 & 63.49 & 63.55  \\
 \hline
 16 & 37.03 & 37.03 & 36.46 & 37.03 & 61.1 & 63.79 & 63.58 & 63.33  \\
 \hline
 \hline
 \multicolumn{5}{|c|}{Winogrande (FP32 Accuracy = 58.01\%)} & \multicolumn{4}{|c|}{Piqa (FP32 Accuracy = 74.21\%)} \\ 
 \hline
 \hline
 64 & 58.17 & 57.22 & 57.85 & 58.33 & 73.01 & 73.07 & 73.07 & 72.80 \\
 \hline
 32 & 59.12 & 58.09 & 57.85 & 58.41 & 73.01 & 73.94 & 72.74 & 73.18  \\
 \hline
 16 & 57.93 & 58.88 & 57.93 & 58.56 & 73.94 & 72.80 & 73.01 & 73.94  \\
 \hline
\end{tabular}
\caption{\label{tab:mmlu_abalation} Accuracy on LM evaluation harness tasks on GPT3-1.3B model.}
\end{table}

\begin{table} \centering
\begin{tabular}{|c||c|c|c|c||c|c|c|c|} 
\hline
 $L_b \rightarrow$& \multicolumn{4}{c||}{8} & \multicolumn{4}{c||}{8}\\
 \hline
 \backslashbox{$L_A$\kern-1em}{\kern-1em$N_c$} & 2 & 4 & 8 & 16 & 2 & 4 & 8 & 16  \\
 %$N_c \rightarrow$ & 2 & 4 & 8 & 16 & 2 & 4 & 2 \\
 \hline
 \hline
 \multicolumn{5}{|c|}{Race (FP32 Accuracy = 41.34\%)} & \multicolumn{4}{|c|}{Boolq (FP32 Accuracy = 68.32\%)} \\ 
 \hline
 \hline
 64 & 40.48 & 40.10 & 39.43 & 39.90 & 69.20 & 68.41 & 69.45 & 68.56 \\
 \hline
 32 & 39.52 & 39.52 & 40.77 & 39.62 & 68.32 & 67.43 & 68.17 & 69.30  \\
 \hline
 16 & 39.81 & 39.71 & 39.90 & 40.38 & 68.10 & 66.33 & 69.51 & 69.42  \\
 \hline
 \hline
 \multicolumn{5}{|c|}{Winogrande (FP32 Accuracy = 67.88\%)} & \multicolumn{4}{|c|}{Piqa (FP32 Accuracy = 78.78\%)} \\ 
 \hline
 \hline
 64 & 66.85 & 66.61 & 67.72 & 67.88 & 77.31 & 77.42 & 77.75 & 77.64 \\
 \hline
 32 & 67.25 & 67.72 & 67.72 & 67.00 & 77.31 & 77.04 & 77.80 & 77.37  \\
 \hline
 16 & 68.11 & 68.90 & 67.88 & 67.48 & 77.37 & 78.13 & 78.13 & 77.69  \\
 \hline
\end{tabular}
\caption{\label{tab:mmlu_abalation} Accuracy on LM evaluation harness tasks on GPT3-8B model.}
\end{table}

\begin{table} \centering
\begin{tabular}{|c||c|c|c|c||c|c|c|c|} 
\hline
 $L_b \rightarrow$& \multicolumn{4}{c||}{8} & \multicolumn{4}{c||}{8}\\
 \hline
 \backslashbox{$L_A$\kern-1em}{\kern-1em$N_c$} & 2 & 4 & 8 & 16 & 2 & 4 & 8 & 16  \\
 %$N_c \rightarrow$ & 2 & 4 & 8 & 16 & 2 & 4 & 2 \\
 \hline
 \hline
 \multicolumn{5}{|c|}{Race (FP32 Accuracy = 40.67\%)} & \multicolumn{4}{|c|}{Boolq (FP32 Accuracy = 76.54\%)} \\ 
 \hline
 \hline
 64 & 40.48 & 40.10 & 39.43 & 39.90 & 75.41 & 75.11 & 77.09 & 75.66 \\
 \hline
 32 & 39.52 & 39.52 & 40.77 & 39.62 & 76.02 & 76.02 & 75.96 & 75.35  \\
 \hline
 16 & 39.81 & 39.71 & 39.90 & 40.38 & 75.05 & 73.82 & 75.72 & 76.09  \\
 \hline
 \hline
 \multicolumn{5}{|c|}{Winogrande (FP32 Accuracy = 70.64\%)} & \multicolumn{4}{|c|}{Piqa (FP32 Accuracy = 79.16\%)} \\ 
 \hline
 \hline
 64 & 69.14 & 70.17 & 70.17 & 70.56 & 78.24 & 79.00 & 78.62 & 78.73 \\
 \hline
 32 & 70.96 & 69.69 & 71.27 & 69.30 & 78.56 & 79.49 & 79.16 & 78.89  \\
 \hline
 16 & 71.03 & 69.53 & 69.69 & 70.40 & 78.13 & 79.16 & 79.00 & 79.00  \\
 \hline
\end{tabular}
\caption{\label{tab:mmlu_abalation} Accuracy on LM evaluation harness tasks on GPT3-22B model.}
\end{table}

\begin{table} \centering
\begin{tabular}{|c||c|c|c|c||c|c|c|c|} 
\hline
 $L_b \rightarrow$& \multicolumn{4}{c||}{8} & \multicolumn{4}{c||}{8}\\
 \hline
 \backslashbox{$L_A$\kern-1em}{\kern-1em$N_c$} & 2 & 4 & 8 & 16 & 2 & 4 & 8 & 16  \\
 %$N_c \rightarrow$ & 2 & 4 & 8 & 16 & 2 & 4 & 2 \\
 \hline
 \hline
 \multicolumn{5}{|c|}{Race (FP32 Accuracy = 44.4\%)} & \multicolumn{4}{|c|}{Boolq (FP32 Accuracy = 79.29\%)} \\ 
 \hline
 \hline
 64 & 42.49 & 42.51 & 42.58 & 43.45 & 77.58 & 77.37 & 77.43 & 78.1 \\
 \hline
 32 & 43.35 & 42.49 & 43.64 & 43.73 & 77.86 & 75.32 & 77.28 & 77.86  \\
 \hline
 16 & 44.21 & 44.21 & 43.64 & 42.97 & 78.65 & 77 & 76.94 & 77.98  \\
 \hline
 \hline
 \multicolumn{5}{|c|}{Winogrande (FP32 Accuracy = 69.38\%)} & \multicolumn{4}{|c|}{Piqa (FP32 Accuracy = 78.07\%)} \\ 
 \hline
 \hline
 64 & 68.9 & 68.43 & 69.77 & 68.19 & 77.09 & 76.82 & 77.09 & 77.86 \\
 \hline
 32 & 69.38 & 68.51 & 68.82 & 68.90 & 78.07 & 76.71 & 78.07 & 77.86  \\
 \hline
 16 & 69.53 & 67.09 & 69.38 & 68.90 & 77.37 & 77.8 & 77.91 & 77.69  \\
 \hline
\end{tabular}
\caption{\label{tab:mmlu_abalation} Accuracy on LM evaluation harness tasks on Llama2-7B model.}
\end{table}

\begin{table} \centering
\begin{tabular}{|c||c|c|c|c||c|c|c|c|} 
\hline
 $L_b \rightarrow$& \multicolumn{4}{c||}{8} & \multicolumn{4}{c||}{8}\\
 \hline
 \backslashbox{$L_A$\kern-1em}{\kern-1em$N_c$} & 2 & 4 & 8 & 16 & 2 & 4 & 8 & 16  \\
 %$N_c \rightarrow$ & 2 & 4 & 8 & 16 & 2 & 4 & 2 \\
 \hline
 \hline
 \multicolumn{5}{|c|}{Race (FP32 Accuracy = 48.8\%)} & \multicolumn{4}{|c|}{Boolq (FP32 Accuracy = 85.23\%)} \\ 
 \hline
 \hline
 64 & 49.00 & 49.00 & 49.28 & 48.71 & 82.82 & 84.28 & 84.03 & 84.25 \\
 \hline
 32 & 49.57 & 48.52 & 48.33 & 49.28 & 83.85 & 84.46 & 84.31 & 84.93  \\
 \hline
 16 & 49.85 & 49.09 & 49.28 & 48.99 & 85.11 & 84.46 & 84.61 & 83.94  \\
 \hline
 \hline
 \multicolumn{5}{|c|}{Winogrande (FP32 Accuracy = 79.95\%)} & \multicolumn{4}{|c|}{Piqa (FP32 Accuracy = 81.56\%)} \\ 
 \hline
 \hline
 64 & 78.77 & 78.45 & 78.37 & 79.16 & 81.45 & 80.69 & 81.45 & 81.5 \\
 \hline
 32 & 78.45 & 79.01 & 78.69 & 80.66 & 81.56 & 80.58 & 81.18 & 81.34  \\
 \hline
 16 & 79.95 & 79.56 & 79.79 & 79.72 & 81.28 & 81.66 & 81.28 & 80.96  \\
 \hline
\end{tabular}
\caption{\label{tab:mmlu_abalation} Accuracy on LM evaluation harness tasks on Llama2-70B model.}
\end{table}

%\section{MSE Studies}
%\textcolor{red}{TODO}


\subsection{Number Formats and Quantization Method}
\label{subsec:numFormats_quantMethod}
\subsubsection{Integer Format}
An $n$-bit signed integer (INT) is typically represented with a 2s-complement format \citep{yao2022zeroquant,xiao2023smoothquant,dai2021vsq}, where the most significant bit denotes the sign.

\subsubsection{Floating Point Format}
An $n$-bit signed floating point (FP) number $x$ comprises of a 1-bit sign ($x_{\mathrm{sign}}$), $B_m$-bit mantissa ($x_{\mathrm{mant}}$) and $B_e$-bit exponent ($x_{\mathrm{exp}}$) such that $B_m+B_e=n-1$. The associated constant exponent bias ($E_{\mathrm{bias}}$) is computed as $(2^{{B_e}-1}-1)$. We denote this format as $E_{B_e}M_{B_m}$.  

\subsubsection{Quantization Scheme}
\label{subsec:quant_method}
A quantization scheme dictates how a given unquantized tensor is converted to its quantized representation. We consider FP formats for the purpose of illustration. Given an unquantized tensor $\bm{X}$ and an FP format $E_{B_e}M_{B_m}$, we first, we compute the quantization scale factor $s_X$ that maps the maximum absolute value of $\bm{X}$ to the maximum quantization level of the $E_{B_e}M_{B_m}$ format as follows:
\begin{align}
\label{eq:sf}
    s_X = \frac{\mathrm{max}(|\bm{X}|)}{\mathrm{max}(E_{B_e}M_{B_m})}
\end{align}
In the above equation, $|\cdot|$ denotes the absolute value function.

Next, we scale $\bm{X}$ by $s_X$ and quantize it to $\hat{\bm{X}}$ by rounding it to the nearest quantization level of $E_{B_e}M_{B_m}$ as:

\begin{align}
\label{eq:tensor_quant}
    \hat{\bm{X}} = \text{round-to-nearest}\left(\frac{\bm{X}}{s_X}, E_{B_e}M_{B_m}\right)
\end{align}

We perform dynamic max-scaled quantization \citep{wu2020integer}, where the scale factor $s$ for activations is dynamically computed during runtime.

\subsection{Vector Scaled Quantization}
\begin{wrapfigure}{r}{0.35\linewidth}
  \centering
  \includegraphics[width=\linewidth]{sections/figures/vsquant.jpg}
  \caption{\small Vectorwise decomposition for per-vector scaled quantization (VSQ \citep{dai2021vsq}).}
  \label{fig:vsquant}
\end{wrapfigure}
During VSQ \citep{dai2021vsq}, the operand tensors are decomposed into 1D vectors in a hardware friendly manner as shown in Figure \ref{fig:vsquant}. Since the decomposed tensors are used as operands in matrix multiplications during inference, it is beneficial to perform this decomposition along the reduction dimension of the multiplication. The vectorwise quantization is performed similar to tensorwise quantization described in Equations \ref{eq:sf} and \ref{eq:tensor_quant}, where a scale factor $s_v$ is required for each vector $\bm{v}$ that maps the maximum absolute value of that vector to the maximum quantization level. While smaller vector lengths can lead to larger accuracy gains, the associated memory and computational overheads due to the per-vector scale factors increases. To alleviate these overheads, VSQ \citep{dai2021vsq} proposed a second level quantization of the per-vector scale factors to unsigned integers, while MX \citep{rouhani2023shared} quantizes them to integer powers of 2 (denoted as $2^{INT}$).

\subsubsection{MX Format}
The MX format proposed in \citep{rouhani2023microscaling} introduces the concept of sub-block shifting. For every two scalar elements of $b$-bits each, there is a shared exponent bit. The value of this exponent bit is determined through an empirical analysis that targets minimizing quantization MSE. We note that the FP format $E_{1}M_{b}$ is strictly better than MX from an accuracy perspective since it allocates a dedicated exponent bit to each scalar as opposed to sharing it across two scalars. Therefore, we conservatively bound the accuracy of a $b+2$-bit signed MX format with that of a $E_{1}M_{b}$ format in our comparisons. For instance, we use E1M2 format as a proxy for MX4.

\begin{figure}
    \centering
    \includegraphics[width=1\linewidth]{sections//figures/BlockFormats.pdf}
    \caption{\small Comparing LO-BCQ to MX format.}
    \label{fig:block_formats}
\end{figure}

Figure \ref{fig:block_formats} compares our $4$-bit LO-BCQ block format to MX \citep{rouhani2023microscaling}. As shown, both LO-BCQ and MX decompose a given operand tensor into block arrays and each block array into blocks. Similar to MX, we find that per-block quantization ($L_b < L_A$) leads to better accuracy due to increased flexibility. While MX achieves this through per-block $1$-bit micro-scales, we associate a dedicated codebook to each block through a per-block codebook selector. Further, MX quantizes the per-block array scale-factor to E8M0 format without per-tensor scaling. In contrast during LO-BCQ, we find that per-tensor scaling combined with quantization of per-block array scale-factor to E4M3 format results in superior inference accuracy across models. 


\end{document}
\endinput
%%
%% End of file `sample-acmtog.tex'.
