%%%%%%%% ICML 2025 EXAMPLE LATEX SUBMISSION FILE %%%%%%%%%%%%%%%%%

\documentclass{article}

% Recommended, but optional, packages for figures and better typesetting:
\usepackage{microtype}
\usepackage{graphicx}
\usepackage{subfigure}
\usepackage{booktabs} % for professional tables

% hyperref makes hyperlinks in the resulting PDF.
% If your build breaks (sometimes temporarily if a hyperlink spans a page)
% please comment out the following usepackage line and replace
% \usepackage{icml2025} with \usepackage[nohyperref]{icml2025} above.
\usepackage{hyperref}
\usepackage{url}

% Attempt to make hyperref and algorithmic work together better:
\newcommand{\theHalgorithm}{\arabic{algorithm}}

% Use the following line for the initial blind version submitted for review:
% \usepackage{icml2025}

% If accepted, instead use the following line for the camera-ready submission:
\usepackage[accepted]{icml2025}

% For theorems and such
\usepackage{amsmath}
\usepackage{amssymb}
\usepackage{mathtools}
\usepackage{amsthm}
\usepackage{multirow}
\usepackage{mathabx}
\usepackage{makecell}
\usepackage{xcolor}
\usepackage{colortbl}
\usepackage{wrapfig}
\usepackage{array}

\newcommand\Tstrut{\rule{0pt}{2.3ex}}         % = `top' strut
\newcommand\Bstrut{\rule[-0.9ex]{0pt}{0pt}}   % = `bottom' strut
\newcommand{\yicong}[1]{{\textcolor{orange}{#1}}}

% acronym of our method
\newcommand{\ours}{HTH}

% if you use cleveref..
\usepackage[capitalize,noabbrev]{cleveref}

%%%%%%%%%%%%%%%%%%%%%%%%%%%%%%%%
% THEOREMS
%%%%%%%%%%%%%%%%%%%%%%%%%%%%%%%%
\theoremstyle{plain}
\newtheorem{theorem}{Theorem}[section]
\newtheorem{proposition}[theorem]{Proposition}
\newtheorem{lemma}[theorem]{Lemma}
\newtheorem{corollary}[theorem]{Corollary}
\theoremstyle{definition}
\newtheorem{definition}[theorem]{Definition}
\newtheorem{assumption}[theorem]{Assumption}
\theoremstyle{remark}
\newtheorem{remark}[theorem]{Remark}

% Todonotes is useful during development; simply uncomment the next line
%    and comment out the line below the next line to turn off comments
%\usepackage[disable,textsize=tiny]{todonotes}
\usepackage[textsize=tiny]{todonotes}


\icmltitlerunning{Pushing the Boundaries of State Space Models for Image and Video Generation}

\begin{document}

\twocolumn[
\icmltitle{Pushing the Boundaries of State Space Models for Image and Video Generation}

\icmlsetsymbol{equal}{*}

\begin{icmlauthorlist}
\icmlauthor{Yicong Hong$^{1}$}{}
\icmlauthor{Long Mai$^{1}$}{}
\icmlauthor{Yuan Yao$^{1,2}$}{}
\icmlauthor{Feng Liu$^{1}$}{} \\ \vspace{5pt}
\icmlauthor{$^{1}$Adobe Research, $^{2}$University of Rochester}{} \\ \vspace{5pt}
\icmlauthor{\tt\small{Project Page:}\;\;\tt\small\url{https://yiconghong.me/HTH}}{}
\end{icmlauthorlist}

% \icmlaffiliation{yyy}{Adobe Research}
% \icmlaffiliation{comp}{Company Name, Location, Country}

% \icmlcorrespondingauthor{Firstname1 Lastname1}{first1.last1@xxx.edu}
% \icmlcorrespondingauthor{Firstname2 Lastname2}{first2.last2@www.uk}

% You may provide any keywords that you
% find helpful for describing your paper; these are used to populate
% the "keywords" metadata in the PDF but will not be shown in the document
\icmlkeywords{Machine Learning, ICML}

\vskip 0.3in
]

% this must go after the closing bracket ] following \twocolumn[ ...

% This command actually creates the footnote in the first column
% listing the affiliations and the copyright notice.
% The command takes one argument, which is text to display at the start of the footnote.
% The \icmlEqualContribution command is standard text for equal contribution.
% Remove it (just {}) if you do not need this facility.

\printAffiliationsAndNotice{}  % leave blank if no need to mention equal contribution
% \printAffiliationsAndNotice{\icmlEqualContribution} % otherwise use the standard text.

\begin{abstract}

While Transformers have become the dominant architecture for visual generation, linear attention models, such as the state-space models (SSM), are increasingly recognized for their efficiency in processing long visual sequences. However, the essential efficiency of these models comes from formulating a limited recurrent state, enforcing causality among tokens that are prone to inconsistent modeling of N-dimensional visual data, leaving questions on their capacity to generate long non-causal sequences. In this paper, we explore the boundary of SSM on image and video generation by building the largest-scale diffusion SSM-Transformer hybrid model to date (5B parameters) based on the sub-quadratic bi-directional Hydra and self-attention, and generate up to 2K images and 360p 8 seconds (16 FPS) videos. 
Our results demonstrate that the model can produce faithful results aligned with complex text prompts and temporal consistent videos with high dynamics, suggesting the great potential of using SSMs for visual generation tasks.
\end{abstract}

\begin{figure}[ht]
    \centering
    \includegraphics[width=0.8\linewidth]{graphs/greater_than_naive.pdf}
    \vspace{0.5cm}
    \includegraphics[width=0.8\linewidth]{graphs/p1_bottom.png}
    \vspace{-5pt}
    \caption{\textcolor{positional}{Positional} vs.\ \textcolor{nonpositional}{non-positional} circuits. In a \textcolor{nonpositional}{non-positional} circuit, the same edges must be included at all positions. A \textcolor{positional}{positional} circuit can distinguish between the same edge at different positions. This specificity yields better trade-offs between circuit size and faithfulness. It can also increase both precision and recall.}
    \label{fig:p1}
    \vspace{-5pt}
\end{figure}

\section{Introduction}

\looseness=-1
A primary goal of interpretability research is to characterize the internal mechanisms in language models (LMs) and other NLP models. 
A core approach in this area is \textbf{circuit discovery}---identifying the minimal subgraph within the model's computation graph that performs a specific task \citep{olah2021framework,olah-mech}.
Typically, the nodes of a circuit represent model components (e.g., attention heads, neurons, or layers).
While manual circuit discovery methods can yield position-specific insights \citep{wanginterpretability,goldowskydill2023localizingmodelbehaviorpath}, \emph{automatic methods often overlook positional information}, treating components as uniformly relevant across all input token positions \citep{conmytowards,syed2023attribution}. 
For instance, if an attention head is included in a circuit, it is assumed to contribute equally to the computation for every position in the input sequence.
The assumption that circuits are position-invariant ignores the fact that different positions often require distinct computations.
By ignoring positions, current methods limit their ability to capture mechanisms that operate across positions, such as interactions between attention heads across positions.

In this study, we start by demonstrating that positional agnosticism is a significant limitation (\S\ref{sec:motivating}). Then, to address these limitations, we introduce a new approach: position-aware edge attribution patching (PEAP; \S\ref{sec:full_circ_discovery}; Figure~\ref{fig:p1}). Current approaches  assume that if an edge is in a circuit, then the same edge will be in the circuit at all positions, thus leading to low precision. It is also assumed that an edge's importance should be aggregated across positions before deciding whether it should be included in the circuit; this can lead to cancellation effects, and thus low recall. PEAP instead allows us to compute the importance of cross-positional edges, and separately evaluates edge importance at each position. We show that this leads to smaller and more accurate circuits; see Figure~\ref{fig:p1}.

Incorporating positional information into circuit discovery is straightforward when inputs have the same length and structure across examples.

However, realistic datasets are not nearly this templatic.
How, then, can we incorporate positional information into automatic circuit discovery?
To address this challenge, we propose \textbf{schemas} (\S\ref{sec:schema}). 
Schemas assign semantic labels to spans of tokens, enabling information aggregation across examples even when the spans differ in length.

For example, in the input ``The \textcolor{positional}{war} lasted from 1453 to 14\underline{\hspace{1em}},'' the span ``\textcolor{positional}{war}'' could be labeled as ``\emph{Subject}''.
This enables handling spans with varying lengths: the phrase ``\textcolor{positional}{Black Plague}'' in another example can be treated as a single positional span with the same role as ``\textcolor{positional}{war}''.
In experiments with two LMs and three tasks, we find that circuits discovered using schemas achieve a better trade-off between circuit size and faithfulness to the model's behavior than position-agnostic circuits.
Importantly, position-aware circuits offer a more precise representation of the underlying mechanisms, providing a more concise foundation for mechanistic explanations.

We also present a fully automated pipeline for schema generation and application (\S\ref{sec:schema-generation}) using large language models (LLMs). 
We evaluate the quality of the generated schemas and their utility in discovering position-aware circuits (\S\ref{sec:schema-eval}).
Notably, circuits derived using automatically generated and applied schemas achieve comparable faithfulness scores to circuits discovered with human-designed and manually applied schemas.

We summarize our contributions as follows:
\begin{itemize}[noitemsep,leftmargin=*,topsep=1pt,parsep=1pt]
    \item Introduce a position-aware circuit discovery method, which obtains better faithfulness than position-agnostic discovery.  
    \item Introduce dataset schemas,  facilitating positional circuit discovery in more naturalistic settings. 
    \item Develop an automated schema generation and application pipeline with LLMs, yielding schemas that are comparable to manually-annotated ones.
\end{itemize}

% \section{Related Work}
\label{sec:related}
\textbf{Open-Set Understanding:}
The open-set task, first introduced by Scheirer et al. \cite{scheirer2012toward}, challenges the conventional closed-set paradigm commonly assumed in image recognition. In closed-set models, the testing phase only includes samples from a predefined set of classes known to the model during training. Conversely, open-set recognition addresses scenarios where samples can belong to previously unknown classes that were not present during training. This requires models to both recognize and reject instances from unfamiliar classes, ensuring robustness against unknown inputs. The open-set framework has seen extensive study across multiple areas in computer vision, such as image classification~\cite{bendale2016towards, vaze2022open, yoshihashi2019classification, oza2019c2ae, perera2020generative, chen2021adversarial,zhang2020hybrid}, %
object detection~\cite{han2022expanding, miller2020uncertainty, zhou2023open}, %
and image segmentation~\cite{hwang2021exemplar, pham2018bayesian, cen2021deep}.%


\textbf{Open-vocabulary Semantic Segmentation (OVSS):}
Recent advances in vision-language models (VLMs) such as CLIP~\cite{radford2021learning} have demonstrated that robust, transferable visual representations can be effectively learned from large-scale datasets using only weakly structured natural language descriptions.
Initially adopted in image-level tasks like classification, VLMs leverage both visual and textual embeddings to recognize a diverse set of classes. %
By aligning image features with semantic concepts in a shared space, VLMs achieve a form of zero-shot learning that allows them to identify new classes at test time, offering a flexible framework for generalization \cite{liu2024open,xie2024sed,cho2024cat}. 
Although open-vocabulary learning presents an appealing solution for handling arbitrary classes, scaling this approach to accommodate an ever-growing set of classes poses significant challenges. In theory, a VLM could achieve perfect generalization if its query set contained every conceivable class label. However, as demonstrated by Miller et al. \cite{miller2025open}, adding more classes to the query set does not lead to better performance. In fact, increasing the number of class labels introduces a greater likelihood of misclassifications, leading to degraded model accuracy. This degradation occurs because, as more classes are added, the semantic space becomes increasingly crowded, causing overlaps that make accurate distinctions between classes harder to achieve. 
To tackle scalability, one solution is training class-free models \cite{shin2024towards}, while distinguishability can be improved by enhancing the textual descriptions of the classes \cite{ma2024open,jiao2024collaborative}. However, all these approaches assume that the inference label set is predefined and available at inference time.%




\textbf{Vocabulary-Free Semantic Segmentation (VSS):} Recent research in VSS has focused on developing end-to-end solutions while reducing bias from  ground truth data annotations. The majority of current methods decompose the task into a class-agnostic mask generation and a class association (Mask2Tag). Zero-Seg \cite{rewatbowornwong2023zero} and TAG \cite{kawano2024tag} leverage DINO \cite{caron2021emerging} to generate the masks, followed by CLIP-based \cite{radford2021learning} embedding generation for class association. Zero-Seg processes these embeddings %
following ZeroCap \cite{tewel2021zero} to obtain textual classes, while TAG matches them against an external database. 
Conversely, CaSED \cite{conti2024vocabulary} identifies potential classes by querying an external caption database using a pre-trained VLM. %
Similarly, Auto-Seg \cite{ulger2024autovocabularysemanticsegmentation} %
fine-tunes a captioning model to extract class names at multiple scales, followed by a second stage where an open-vocabulary model generates segmentation masks, with predictions remapped to ground truth classes using LLMs. While these approaches demonstrate promising results, they do not fully explore how this pipeline decomposition impacts model performance, nor do they investigate methods to enrich the textual information in the CLIP encoder. We address these limitations by providing a comprehensive analysis of the text encoder's role and exploring techniques to enhance visual-language understanding through richer textual representations. Moreover, we rigorously test the sensitivity of CLIP to tagger errors, evaluating how inaccuracies in image tagging propagate and impact the final segmentation performance.

\section{Preliminaries}\label{sec:problem_formulation}

% \begin{table*}[h!]
% \centering
% \caption{Comparison of Algorithms}
% \label{tab:algorithm_comparison}
% \begin{tabular}{l|ccccc}
% \toprule
% \textbf{Algorithm Name} & \textbf{Performance} & \textbf{Diversity} & \textbf{Generalization} & \textbf{Efficiency} & \textbf{Streaming}\\
% \midrule
% Greedy-decoding & Moderate  &   Low  & High          & High      & Yes \\
% Decoding with Temperature & Low & High & High              & High & Yes   \\
% Top-K Sampling & Moderate      & High & High               & High   & Yes     \\
% Top-P Sampling & Moderate     & High & High              & High   & Yes      \\
% Beam-Search & Moderate     & Moderate & High              & Moderate    & Yes     \\
% Majority Voting & Moderate     & High & Low              & Moderate     & No    \\
% RM Selection & Moderate     & High & High              & Moderate    & No     \\
% RM-guided Tree Search & High     & High & Low              & Low    & No    \\
% \bottomrule
% \end{tabular}
% \end{table*}
% The language model generation process generally selects a sequence of tokens following certain algorithms (e.g., greedy or sampling methods) until a stopping criterion, such as an end-of-sequence token or maximum sequence length, is reached.

In this section, we first introduce how the LM generation process can be formulated as a token-level Markov Decision Process (MDP) and then explain how existing sampling algorithms relate to it.

\subsection{LLM Decoding as Token-level MDP}

The language model generation process takes a sequence of tokens as inputs and generates a sequence of tokens as outputs.
Mainstream transformer-based language models generate the output tokens one by one until the stopping criteria (e.g., an end-of-sequence token or maximum sequence length) are met.
The traditional MDP is usually formulated as a tuple $\mathcal{M} = (\mathcal{S}, \mathcal{A}, F, R, \gamma)$, where $\mathcal{S}$ is the set of all possible states, $\mathcal{A}$ is the set of actions, $F$ is the transition function, $R$ is the reward function, and $\gamma$ is the decay parameter.
In the language model scenarios, each state in $\mathcal{S}$ is a trajectory that can be denoted as $\tau$.
Each action in $\mathcal{A}$ is selecting a token $x$ from the vocabulary set.
$F$ is the deterministic transition of concatenating the selected action (i.e., a token) with the existing state (i.e., a trajectory) to become a new one.
Traditionally, rewards $r_t$ are defined at every step $t$ and contribute to the return $G_t=\sum_{k=0}^{\infty} \gamma^k R_{t+k}$ through the decay factor $\gamma$.
However, in the LLM scenario, we are only concerned with the quality of the complete trajectory generated, meaning that the reward function $R(s)$ evaluates the final trajectory rather than providing step-by-step feedback.
Thus, the return becomes $G=R_T$, where $R_T$ is the reward associated with the final sequence at step $T$, and $\gamma$ is irrelevant because intermediate rewards are not accumulated. 

\subsection{The Classical Decoding Algorithms}

Formally, given an input \( \mathbf{x}  = (x_1, x_2, \ldots, x_T) \), a reward function $R$ that provides a scalar reward for a trajectory $\tau$, and a language model \( p_\theta(x) \) parameterized by \( \theta \), the goal of decoding algorithms is to find the optimal trajectory $x^\star$ sampled from \( p_\theta(x) \) that could maximize the reward:

\[
\tau^* = \arg\max_{\tau \sim p_\theta(\mathbf{x} )} R(\tau).
\]

% Assumes that the input is a token sequence with $a$ and the foundation language model is a probabilistic model of predicting the likelihood of the next token given previous ones, which is usually formulated as $P(x \mid x_{<i})$, the goal of language model decoding is to utilize this likelihood to get the output trajectory of length $T$ that achieves the highest $R_T$.
This section covers representative decoding algorithms and explains how they are connected.

\textbf{Greedy Decoding}: The naive but most widely used algorithm is \textit{Greedy Decoding}, which uses the language modeling likelihood at each step as guidance.
At each step $i$, this algorithm selects the action token $x_i \in \mathcal{A}$ following:
\begin{equation}
x_i = \arg\max_{x} P(x \mid x_{<i}).
\end{equation}
From the angle of MDP, this method uses the accumulative likelihood predicted by the language as the final reward:
\begin{equation}
 R(\tau) \gets \Pi_{i}^{T}P(x_i \mid x_{<i}),
\end{equation}
where $T$ is the length of $\tau$, and takes a greedy solution to approach this goal.


\textbf{Sampling-based Decoding:} 
On top of greedy decoding, people also try to incorporate diversity in the final output.
For example, the temperature-based method introduces an additional parameter $\lambda$ to control the greedy sampling process by reshaping the likelihood distribution as:
\begin{equation}
x_i \sim P(x \mid x_{<i})^{1/\lambda}.
\end{equation}
From the angle of token-level MDP, we can reinterpret this process as introducing an additional diversity objective:

\begin{equation}
 R(\tau) \gets \Pi_{i}^{T}P(x_i \mid x_{<i}) \cdot D(x_i, x_{<i}),
\end{equation}
where 
\begin{equation}
    D(x_i , x_{<i}) = P(x_i \mid x_{<i})^{\frac{1}{\lambda}-1}.
\end{equation}


To avoid sampling rare tokens and achieve a balance between performance and diversity, researchers have investigated how to dynamically adjust the candidate token pool~\citep{holtzman2019curious,zarriess2021decoding}. 
For example, the \textit{Top-$k$ Sampling} algorithm only considers the top $k$ tokens with the highest probabilities as candidates instead of the whole vocabulary.
Similarly, the \textit{Nucleus Sampling}, which is also known as \textit{Top-p Sampling}, only selects from the smallest possible set $\mathcal{V}_p \subseteq \mathcal{V}$, where the cumulative probability mass exceeds a threshold $p$.

% \paragraph{Nucleus Sampling:} This strategy samples tokens from the smallest possible set $\mathcal{V}_p \subseteq \mathcal{V}$, where the cumulative probability mass exceeds a threshold $p$. Formally,
% \begin{equation}
%     \mathcal{V}_p = \{x_i \mid \sum_{x_j \in \mathcal{V}_p} p(x_j \mid \mathbf{x}_{<t}) \geq p\}.
% \end{equation}

\textbf{Trajectory-level Decoding:} Although these token-level decoding algorithms are efficient, they tend to generate locally coherent outputs that may lack global quality.
To solve this problem, people also developed decoding algorithms that consider partial or whole trajectories.
For example, the \textit{Beam Search Decoding} algorithm keeps track of the top $B$ partial trajectories, expanding them at each step and retaining only the ones with the highest joint likelihood.
Similar to the \textit{Greedy Decoding}, this method also uses the joint likelihood as the trajectory reward function.

\textbf{Advanced Reward Modeling Algorithms:}
A common limitation of the aforementioned algorithms is their fundamental assumption that the joint likelihood could represent $R(\tau)$ might not always hold.
People have been interested in introducing better reward signals as guidance to address this.
For example, in the QA scenario, the \textit{Majority-voting algorithm} assumes that the more frequent answer aligns better with the grounding reward function (i.e., accuracy) and thus selects candidate trajectories following this guidance.
Though this intuitive approach has been shown to be effective on tasks such as QA and math problems, it is restricted to tasks with structured output for voting. It cannot be generalized to more general-purpose applications.
To address this issue, researchers also include an external model $R^\prime$, which is often another transformer-based model, to approximate the ground truth reward model $R$. 
With that, we could sample $K$ trajectories $\mathcal{T}_K$ with sampling-based decoding algorithms and then use $R$ to select the trajectory with the maximum reward:
\begin{equation}
    \tau^\star = \arg \max_{\tau \in \mathcal{T}} R^\prime(\tau).
\end{equation}
Employing an external model to model the reward offers greater flexibility than heuristic rewards. This approach is not constrained by the structured answer format, which improves generality and adaptability in various scenarios. 








\label{sec:method}

In this section, we introduce the method used to conduct the investigation on a set of \pc papers that discuss relevant bias issues.
Specifically, to construct the initial set of relevant work, we search the keywords ``bias" or ``fair" in the title of papers from NeurIPS, ICML, ICLR and FAccT published before February 2025. 
We include papers that discuss bias issues whose manifestation aligns with either Type I Bias or Type II Bias (we will detail the unification in~\cref{sec:unifying}).
We exclude papers that address other bias issues such as inductive bias~\cite{baxter2000model,zietlow2021demystifying}, implicit bias~\cite{fitzgerald2017implicit,camuto2021asymmetric}, selection bias~\cite{hernan2004structural,akbari2021recursive}, sampling bias~\cite{winship1992models,xu2022alleviating}, spectral bias~\cite{fang2024addressing}, exposure bias~\cite{li2024alleviating} or bias-variance~\cite{ha2024fine, chen2024on}.
Furthermore, to ensure we do not overlook any relevant papers without these keywords or from other prominent conferences such as CVPR, ICCV, and ECCV, we manually traversal the citation graph of the paper in the initial set and append the relevant papers that are either cited by or cite the papers in the initial set.






Once we identify the scope of the investigated papers, we read these papers to determine which type of bias they address by examining two aspects: problem statement and evaluation protocol.
We will elaborate on the criterion for categorizing papers into our definitions in~\cref{sec:unifying}.
To accommodate the recent emerging direction of addressing unlabeled and unknown bias, we enrich the taxonomy with an additional dimension about the status of attribute $A$.
As shown in~\cref{tab:taxonomy}, we count the number of papers in each category. 
Note that the total number is not equal to \pc since some papers address both types of biases.
We present the categorization list of all \pc investigated papers in Appendix.


\begin{table}[htbp]
\caption{The taxonomy of bias issues based on \pc papers.}
\label{tab:taxonomy}
\centering
\resizebox{0.45\textwidth}{!}{%

\begin{tabular}{lcccc}
\toprule
\multirow{2}{*}{Type of Bias} & \multicolumn{2}{c}{Attribute $A$} & \multirow{2}{*}{Papers} & \multirow{2}{*}{Examples}                                                   \\
\cmidrule(lr){2-3} 
                              & Known           & Labeled         &                         &                                                                             \\
                              \midrule
\multirow{3}{*}{Type I Bias}  & \cmark          & \cmark          & 253                     & \cite{DebFace,GAC,RL_RBN}                                                   \\
                              & \cmark          & \xmark          & -                       & -                                                                           \\
                              & \xmark          & \xmark          & -                       & -                                                                           \\
                              \midrule
\multirow{3}{*}{Type II Bias} & \cmark          & \cmark          & 246                     & \cite{learn_not_to_learn_Colored_MNIST,CSAD,End}                            \\
                              & \cmark          & \xmark          & 8                       & \cite{HEX_texture_bias1, ReBias_texture_bias2,rubi} \\
                              & \xmark          & \xmark          & 30                      & \cite{LfF_CelebA_Bias_conflicting,ECS,UBNet}                               \\
                              \midrule
Survey                        & -               & -               & 25                       & \cite{MLbias_survey,prediciton_quality_disparity,discussion_on_DP_EO}      \\
\bottomrule
\end{tabular}
}

\end{table}


\begin{table}[t]
\centering
\caption{Results over the benchmark datasets. The mIoU is reported. %
}
\label{tab:sota_results}
\resizebox{\columnwidth}{!}{
\begin{tabular}{ccccccc}
\toprule
\textbf{Method} & \begin{tabular}[c]{@{}c@{}}Inference\\ Vocab. \end{tabular} & A-847 & PC-459 & A-150 & PC-59 & VOC-20 \\ \midrule
SAN \cite{xu2023side} & \checkmark & 12.4 & 15.7 & 27.5 & 53.8 & 94.0 \\
AttrSeg \cite{ma2024open} & \checkmark & -- & -- & -- & 56.3 & 91.6 \\
SCAN \cite{liu2024open} & \checkmark & 14.0 & 16.7 & 30.8 & \textbf{58.4} & \textbf{97.0} \\
EBSeg \cite{shan2024open} & \checkmark & 13.7 & 21.0 & 30.0 & 56.7 & 94.6 \\
SED \cite{xie2024sed} & \checkmark & 11.4 & 18.6 & 31.6 & 57.3 & 94.4 \\
CAT-Seg \cite{cho2024cat} & \checkmark & \textbf{16.0} & \textbf{23.8} & \textbf{31.8} & 57.5 & 94.6 \\ \midrule
CaSED + SAM \cite{conti2024vocabulary} & \xmark & -- & -- & 6.1 & 7.5 & 13.7 \\
CaSED + SAN \cite{conti2024vocabulary} & \xmark & -- & -- & 7.2 & 15.5 & 26.9 \\
DenseCaSED \cite{conti2024vocabulary} & \xmark & -- & -- & 8.6 & 13.4 & 20.5 \\
\textbf{Chick.-and-egg} (CaSED) & \xmark & 3.2 & 4.4 & 9.7 & \textbf{23.1} & \textbf{47.6} \\
\textbf{Chick.-and-egg} (RAM) & \xmark & \textbf{3.7} & \textbf{7.1} & \textbf{15.6} & 23.0 & 47.5  \\
\bottomrule
\end{tabular}
}
\end{table}

\section{Experiments}
\label{ch:results}

We conduct a comprehensive experimental analysis to investigate how different components affect VSS performance.
First, we evaluate the proposed two-stage approach on standard benchmarks to establish the baseline (\Cref{sec:benchmark}). We then present an in-depth analysis of the text encoder's behaviour and its impact on segmentation quality (\Cref{sec:text}). To better understand the relationship between the two-stages, we examine the image tagging accuracy and its influence on the segmentation task (\Cref{sec:tagging}). Finally, we study how different assignment thresholds in the evaluation protocol affect the reported performance (\Cref{sec:thresholds}).

\textbf{Implementation Details:}
The model is trained on the COCO-Stuff dataset \cite{caesar2018coco}, which contains 118k annotated images across 171 categories, following \cite{cho2024cat}. All results are based on CLIP \cite{radford2021learning} with a ViT-B/16 backbone. The image encoder and cost aggregation module are trained with per-pixel binary cross-entropy loss. 
The training parameters follow \cite{cho2024cat}. The batch size is 4, and models are trained for 80k iterations.
We performed image tagging and instance description using a frozen VLM model not trained on the testing dataset. More in detail, we examined the robustness of two models RAM \cite{zhang2024recognize} and Llava-1.6  \cite{liu2024llavanext}.

\textbf{Test Datasets:} The evaluation covers several datasets to ensure comprehensive testing. We used ADE20K \cite{zhou2019semantic} with both 150 and 847 class configurations, Pascal Context \cite{mottaghi2014role} with 59 and 459 class setups, and Pascal VOC \cite{everingham2010pascal} with its 20 classes. 

\begin{figure*}[t]
    \centering
    \resizebox{\textwidth}{!}{%
    \begin{tabular}{@{}ccccc@{}}
        
        
        \includegraphics[width=0.25\textwidth]{fig/qualitative/ADE_val_00000049_img.png} &
        \includegraphics[width=0.25\textwidth]{fig/qualitative_new/ADE_val_00000049_zeroseg.png} &
        \includegraphics[width=0.25\textwidth]{fig/qualitative/ADE_val_00000049_cased_labels_bigger.png} &
        \includegraphics[width=0.25\textwidth]{fig/qualitative/ADE_val_00000049_ours_labels_bigger.png} &
        \includegraphics[width=0.25\textwidth]{fig/qualitative/ADE_val_00000049_labels_bigger.png} \\[0.2cm]
        



        \includegraphics[width=0.25\textwidth] {fig/qualitative_new/ADE_val_00000683_img.png} &
        \includegraphics[width=0.25\textwidth] {fig/qualitative_new/ADE_val_00000683_zero_seg.png} &
        \includegraphics[width=0.25\textwidth]{fig/qualitative_new/ADE_val_00000683_real_image_cased.png} &
        \includegraphics[width=0.25\textwidth]{fig/qualitative_new/ADE_val_00000683_real_image.png} &
        \includegraphics[width=0.25\textwidth]{fig/qualitative_new/ADE_val_00000683_ground_truth.png} \\[0.2cm]


        \includegraphics[width=0.25\textwidth] {fig/qualitative_new/2007_008415_img.png} &
        \includegraphics[width=0.25\textwidth] {fig/qualitative_new/2007_008415_zeroseg.png} &
        \includegraphics[width=0.25\textwidth]{fig/qualitative_new/2007_008415_real_image_cased.png} &
        \includegraphics[width=0.25\textwidth]{fig/qualitative_new/2007_008415_real_image.png} &
        \includegraphics[width=0.25\textwidth]{fig/qualitative_new/2007_008415_ground_truth.png} \\[0.2cm]
        
        
        \textbf{Image} &
        \textbf{ZeroSeg} & 
        \textbf{Chicken-and-egg} (CaSED) & 
        \textbf{Chicken-and-egg} (RAM) & 
        \textbf{GT}
    \end{tabular}
    }
    \caption{Comparison of segmentation results across ZeroSeg \cite{rewatbowornwong2023zero} and \textbf{Chicken-and-Egg} (CaSED \cite{conti2024vocabulary} and RAM \cite{zhang2024recognize}), and ground-truth labels.}
    \label{fig:sota_qualitative_comparison}
\end{figure*}





\begin{table*}[t]
\centering
\caption{Results over the benchmark datasets by using soft assignment. † Results come from their original work. * mapped with Llama-2 \cite{ulger2024autovocabularysemanticsegmentation} rather than Sentence-BERT \cite{reimers2019sentence}. The soft assignment has threshold zero (i.e., all the words are assigned to a class in the evaluation vocabulary).}
\label{tab:mapping_results}
\resizebox{.99\textwidth}{!}{%
    \begin{tabular}{l|cc|cc|cccccccccc}
    \toprule
    \multirow{2}{*}{\textbf{Method}} & \multirow{2}{*}{\begin{tabular}[c]{@{}c@{}}\textbf{Vision}\\ \textbf{Backbone}\end{tabular}} & \multirow{2}{*}{\textbf{Stages}} &\multicolumn{2}{c}{\textbf{Components}} & A-847 & PC-459 & A-150 & PC-59 & VOC-20 \\
     & & & Tagging & Segmentation & &&&&& \\ \midrule
    Zero-Seg† \cite{rewatbowornwong2023zero} & ViT-B/16 & Mask2Tag & CLIP+GPT-2 & DINO & -- & -- & -- & 11.2 & 8.1 \\
    Auto-Seg† \cite{ulger2024autovocabularysemanticsegmentation} & ViT-L/16 & Tag2Mask & BLIP-2 & X-Decoder & 5.9* & -- & -- & 11.7* & \textbf{87.1}* \\
    TAG† \cite{kawano2024tag} & ViT-L/14 & Mask2Tag &CLIP+DB & DINO & -- & -- & 6.6 & 20.2 & 56.9 \\
    \textbf{Chicken-and-egg} (CaSED) & ViT-B/16 & Tag2Mask & CLIP+DB & CAT-Seg & 4.3 & 3.1 & 7.8 & \textbf{27.9} & 82.3 \\
    \textbf{Chicken-and-egg} (RAM) & ViT-B/16 & Tag2Mask & CLIP+Swin & CAT-Seg & \textbf{6.7} & \textbf{8.0} & \textbf{18.8} & 27.8 & 81.8 \\
        \bottomrule
    \end{tabular}
}
\end{table*}

\subsection{Benchmark Evaluation}\label{sec:benchmark}
We first conducted a comprehensive benchmark evaluation comparing existing approaches to establish a strong foundation for VSS and identify the most promising direction. This analysis served two key purposes: (1) to understand the current state-of-the-art performance in VSS and (2) to determine which baseline architecture would be the most suitable.

\textbf{Quantitatives:} \Cref{tab:sota_results} compares the mIoU across the Open-Vocabulary benchmarks. The proposed pipeline outperforms the previous VSS methods by a constant margin in all the datasets. To better accommodate VSS methods, they adopt a class remapping strategy that reduces penalization in cases where an exact class match is not found. This approach is reflected in \Cref{tab:mapping_results}, where the soft evaluation assignment takes place as described in \Cref{sec:assignment}.

\textbf{Qualitatives:} As shown in \Cref{fig:sota_qualitative_comparison}, the current approach fills the gap between the predictions and original dataset labels without a predefined vocabulary, offering finer-grained details across diverse scenarios (indoor and outdoor). %
The maps obtained suggest that current evaluation metrics might be overly pessimistic about the qualitative performance of the results. This issue arises from dataset limitations, where many instances struggle to find appropriate matches (e.g., in the third image, "husky" instead of "dog"). Mask2Tag methods like ZeroSeg \cite{rewatbowornwong2023zero} tend to over-segment the instances, getting improper text matches. On the other hand, Chicken-and-egg with CaSED tends to limit the number of predicted tags, %
while coupled with RAM it reaches the best compromise.




\subsection{Segmentatation Analysis} \label{sec:text}
\textbf{Perfect Tagger:} Our empirical results on the OVSS task - presented in \Cref{tab:gt_labels} - revealed that providing only image-specific text labels, %
rather than the entire vocabulary, during training led to improved segmentation performance when applying the same adjustment at inference. Although having access to inference labels is unrealistic, this setup represents the best achievable performance if tagger predictions were 100\% accurate. 
More in detail, in \Cref{tab:gt_labels}, the set of class names is defined for each batch as \(\mathcal{C}_b \subset \mathcal{C}\) during training, where \(\mathcal{C}_b\) represents the batch-specific subset of the entire class vocabulary \(\mathcal{C}\), dynamically selected based on the batch's unique context or requirements. This subset approach allows the model to focus on relevant classes without being overwhelmed by the entire vocabulary. However, we observed no gain when the text labels in inference are predicted from an image tagger. Nevertheless, this represents the upper bound currently obtainable with the state-of-the-art open-vocabulary method \cite{cho2024cat}. Moreover, we show in \Cref{tab:attvsadj} that when performing inference on perfect predictions (100\% accuracy from the tagger) we can boost performance by providing additional textual information.
\begin{table}[t]
\centering
\caption{\textbf{Comparison using CAT-Seg \cite{cho2024cat}, using ground truth classes as text embeddings at different stages}, where $T$ represents training and $I$ represents inference. The mIoU is reported on ADE-20K (A)\cite{zhou2019semantic}, Pascal Context (PC)\cite{mottaghi2014role}, and Pascal VOC (VOC) \cite{everingham2010pascal}.
}
\label{tab:gt_labels}
\resizebox{\columnwidth}{!}{%
\begin{tabular}{ccccccccc}
\toprule
\textbf{Method} & \multicolumn{2}{c}{\textbf{\begin{tabular}[c]{@{}c@{}}Only GT\\ Text Labels\end{tabular}}} & COCO & A-847 & PC-459 & A-150 & PC-59 & VOC-20 \\ \cmidrule{2-3}
 & T & I &  &  &  &  &  &  \\ \midrule
Base &  &  & 47.11 & 11.95 & 18.95 & 31.78 & 57.20 & 95.30 \\
L.Bound & \checkmark &  & 43.73 & 10.89 & 16.63 & 30.29 & 55.99 & 94.20 \\
U.Bound (I) &  & \checkmark & 56.15 & 12.38 & 18.38 & 45.53 & 69.77 & \textbf{95.87} \\
U.Bound & \checkmark & \checkmark & \textbf{64.03} & \textbf{13.98} & \textbf{24.04} & \textbf{51.21} & \textbf{72.79} & 94.38 \\
\bottomrule
\end{tabular}
}
\end{table}

\begin{table}[t]
\centering
\caption{All methods are based on \cite{cho2024cat}, changing textual descriptors, while performing inference on GT classes. (a)-(c) are trained using the predicted VLM information on COCO dataset.
}
\label{tab:attvsadj}
\resizebox{\columnwidth}{!}{%
\begin{tabular}{ccccccccc}
\toprule
\textbf{Method} & \multicolumn{2}{c}{\textbf{VLM input}} & COCO & A-847 & PC-459 & A-150 & PC-59 \\ \cmidrule{2-3}
& $Image$& $Text$ & & & \\ \midrule
Baseline \cite{cho2024cat} & & & 56.15 & 12.38 & 18.38 %
& 45.53 & 69.77 &\\ %
(a) Caption & \checkmark & & 58.17 & 12.71 & 17.07 %
& 47.09 & 71.04 &\\ %
(b) Class Adjectives & & \checkmark & 62.33 & 14.96 & 19.13 & 48.77 & 60.47 \\ %
(c) Instance Adjectives & \checkmark & \checkmark & \textbf{65.13} & \textbf{15.40} & \textbf{23.20} & \textbf{54.43} & \textbf{72.04} \\ %
\bottomrule
\end{tabular}
}
\end{table}

\begin{table*}[t]
\centering
\caption{Prompts for different algorithms for \cref{tab:attvsadj} results.}
\label{tab:prompt}
\resizebox{.9\textwidth}{!}{%
\begin{tabular}{cc}
\toprule
\textbf{\begin{tabular}[c]{@{}c@{}}Description\\ Level\end{tabular}} & \textbf{Prompts} \\ \midrule
Class & \begin{tabular}[c]{@{}c@{}}1. "Please group the classes in this list $<$dataset-class-list$>$ into groups of classes that are similar to each other \\ meaning they could be confused in an image. Every class should be in one group and only in one group. \\ Make sure there are no classes from the original list missing in your grouping. \\ This is an example of how the output should look: dog, cat, kitten, bird -- couch, desk, sofa, lamp -- knife, fork, plate"\\ 2. " The classes in the group are: $<$group$>$. Please generate a short list of adjectives for each class \\ that describe how the object looks in an image. The adjectives should be distinctive within each group meaning that \\ the same attribute should not appear for two classes in the same group. Generate at least one adjective for each class. \\ This is an example how the output should look. {giraffe: [tall, brown, spotted, yellow], tree: [tall, green], armchair: [comfortable]}\end{tabular}\\ \midrule
Instance & \begin{tabular}[c]{@{}c@{}}"The objects in the image are: $<$dataset-class-list$>$. Please generate a short list of adjectives\\ for each object that describes how the object looks in the image. \\ This is an example of how the output should look. \{giraffe: {[}tall, brown, spotted, interacting{]}, tree: {[}tall, green, leafy{]}\}"\end{tabular} \\ \bottomrule
\end{tabular}
}
\end{table*}

\textbf{Aiding Text Encoder with Descriptions:} Previous works \cite{ma2024open} used adjectives with the assumption to find the common class features that better describe each class. For example, a "dalmata" could be described as "a white dog with black spots". However, in typical recognition tasks, the categories are much broader, such as simply "dog", and a "dog" could be described very differently in terms of color and size. Hence, AttrSeg \cite{ma2024open} have focused on training strategies to find the optimal set of descriptions that could enhance class distinguishability while still being able to represent each class. While this approach has merit, it can result in the loss of fine-grained details. For instance, a "table" or "hat" could be of any size or color, and even a "wall" that is typically "white" could be "bricked" or some other texture.
Zhao et al. \cite{zhao2024gradient} experimented on CLIP's ability to identify different types of object attributes, including shape, material, color, size, and position. 
For shape and material attributes, CLIP showed a certain but limited knowledge, with the heat maps highlighting partial correct attention on obvious objects, but also exhibiting false positive and false negative errors. For color attributes, the results further verified that CLIP has a good ability to distinguish different colors.
For comparative attributes like size and position, CLIP produced some erroneous results, demonstrating that it relies more on the primary object (e.g., "cube", "red") rather than the comparative attribute (e.g., "small", "left"). Overall, their analysis suggests that CLIP has advantages with common perceptual attributes.
Therefore, we adopted a pre-trained VLM to find the corresponding descriptions given each image and its specific set of class names - the text labels of each image-, and we tried to enforce general language descriptions.
The prompts used are shown in \Cref{tab:prompt}. For generating captions, we employed the BLIP-2 model \cite{li2023blip} without any query input, whereas for the multimodal model, we utilized Llava-1.6 \cite{liu2024llavanext}. These models were selected because they both incorporate CLIP as their text encoder.
The text embedding of the captions is employed as a query within an additional cross-attention module, linking it to the embeddings of the classes. In the case of the adjectives, they are sampled and used within the template "A photo of a \{adjective\} \{class name\}".
We report the results in table \Cref{tab:attvsadj}, where adding image-specific content results beneficial, specially for large numbers of classes.
It is important to notice that, when using predicted labels from the image tagger or applying the complete set of image labels during inference, we did not observe the same benefit. %
In the VSS scenario, ambiguities with other classes are largely resolved during the CLIP segmentation stage by directly predicting the image's content using the image tagger. However, misclassifications may still occur at this stage, a behaviour explored in the next paragraph.
\begin{table}[t]
\centering
\caption{Class recognition accuracy of different VLMs with $T_\text{SBERT}$=$0.0$. \\ * using vocabulary.
\# FN = average number of missed classes, \# FP = average number of classes predicted but not in the ground truth.}
\label{tab:accuracy}
\resizebox{.5\textwidth}{!}{%
\begin{tabular}{ccccccccccc}
\toprule
\multirow{2}{*}{\textbf{\begin{tabular}[c]{@{}c@{}}Predicted\\ Classes\end{tabular}}} & \multirow{2}{*}{\textbf{\begin{tabular}[c]{@{}c@{}}Mapping\\ Model\end{tabular}}} & \multicolumn{3}{c}{A-150} & \multicolumn{3}{c}{PC-59} & \multicolumn{3}{c}{VOC-20} \\ \cmidrule{3-11} 
 &  & Acc & \#FP & \#FN & Acc & \#FP & \#FN & Acc & \#FP & \#FN \\ \midrule
CaSED & - & 10 & 10.7 & 7.8 & 22 & 9.3 & 4.0 & 50 & 9.5 & 0.9 \\
CaSED & SBERT & 23 & 7.4 & 6.8 & 42 & 5.4 & 3.1 & 84 & 4.2 & 0.3 \\
Llava-1.6 & - & 26 & 4.9 & 6.3 & 29 & 3.7 & 3.5 & 53 & \textbf{3.5} & 0.8 \\
Llava-1.6 & SBERT & 39 & \textbf{2.6} & 5.2 & 47 & \textbf{1.8} & 2.7 & 91 & 1.9 & 0.2 \\
RAM & - & 34 & 10.4 & 5.9 & 41 & 11.8 & 3.1 & 68 & 12.2 & 0.5 \\
RAM & SBERT & \textbf{46} & 5.7 & \textbf{4.8} & \textbf{61} & 5.4 & \textbf{2.2} & \textbf{96} & 4.8 & \textbf{0.1} \\
\midrule
RAM* & - & 79 & 16.7 & 1.95 & 80 & 6.2 & 1.1 & 97 & 1.5 & 0.1  \\
\bottomrule
\end{tabular}
}
\end{table}

\subsection{Image Tagging Analysis}\label{sec:tagging}
In \Cref{tab:accuracy}, we investigated various image tagging methods to understand how different types of errors affect the sensitivity of the segmentation module, particularly the text encoder since we use the tags as input to CLIP. We evaluated the impact of three architectures: a training-free method, CaSED \cite{conti2024vocabulary}, a multi-step trained method, RAM \cite{zhang2024recognize}, and a general-purpose multimodal model, Llava \cite{liu2024llavanext}.
CaSED %
uses a pre-trained vision-language model and an external database to extract candidate categories and assign the image to the best match. 
On the other hand, RAM %
generates large-scale image tags through automatic semantic parsing, followed by training a model to annotate images using both captioning and tagging tasks. A data engine then refines these annotations, and the model is retrained on this enhanced data, with final fine-tuning on a higher-quality subset.
\Cref{tab:accuracy} shows that using SBERT for evaluation avoids discarding words merely due to the absence of an exact match with the chosen word by the annotators. RAM achieves the best overall results across the evaluated datasets. In the table, the performance of Llava \cite{liu2024llavanext} %
demonstrates the versatility of powerful vision-language architectures. Note that the current baseline, RAM, does not reach a perfect accuracy even when the whole list of desired classes (i.e., non-vocabulary free), hence this represents the current limitation of such an approach. 
Furthermore, compared to CaSED, RAM demonstrates higher class recognition accuracy, but with more false positives on average. To investigate this further, we examined in \Cref{fig:miss_vs_false_sim} how the model is influenced by simulating a drop rate and false positives on top of the ground truth text classes in each image. In the table, the false positives are randomly selected from the vocabulary set. The influence of false negatives deeply influences the performance, while introducing false positives only leads to marginal degradation. These results confirm why RAM outperforms current alternatives: it has the fewest misclassifications, despite having a higher rate of false positives.

\begin{figure}
    \centering
    \includegraphics[width=.9\columnwidth, trim=0cm 0.55cm 0cm 0.75cm, clip]{fig/fpfn_full_stefano.png}
    \caption{Simulating missing classes or adding wrong ones over the OVSS baseline by assuming the labels are known at inference time.}
    \label{fig:miss_vs_false_sim}
\end{figure}


\subsection{Evaluation Assignment Thresholds} \label{sec:thresholds}
In \Cref{fig:thresh} we show the effect of providing different values for $T_\text{SBERT}$. Unlike Zero-Seg \cite{rewatbowornwong2023zero}, we did not observe a consistent trend in the optimal threshold across datasets. Respectively, $0.6-0.7$ for A-847, PC-459 and A-150, $0.5$ for PC-59 and $0.1$ for VOC-20. 
Our findings suggest that as the number of classes increases, we need to be more confident in the assignment, hence a higher threshold leads to a better score.

\begin{figure}
    \centering
    \includegraphics[width=.9\columnwidth, trim=0cm 0.55cm 0cm 0.75cm, clip]{fig/thresholds_stefano.png}
    \caption{Ablation over different thresholds for the evaluation mapping.%
    }
    \label{fig:thresh}
\end{figure}




% \qy{In this paper, we propose an efficient single-stage framework called \nickname{} for 3D object detection. Considering the task of object detection inherently focuses on the foreground points, we propose an instance-aware learning-based downsampling way to automatically select the sparse yet important instance points. In addition, a dedicated contextual centroid perception module is proposed to fully exploit the geometrical structure around the bounding boxes. Extensive experiments conducted on the KITTI detection benchmark demonstrated the superior efficiency and accuracy of the proposed \nickname{}. \revise{In future work, we will further tackle extreme cases such as overlapped bounding boxes.}}

%This paper presents a new point-based single-stage 3D object detection networks, named \nickname{}. With novel instance-aware downsampling strategy and centroid rally module, we can effectively and efficiently achieve muti-class 3D object detection in a bottom-up manner.  Our \nickname{} achieves the best results among pure point-based methods, and provides a state-of-the-art efficiency than existing LiDAR detectors. In the future, we will focus on designing an efficient network to achieve real-time and robust 3D detection in 360-degree LiDAR scenes.

\qy{In this paper, we propose an efficient solution termed \nickname{} for point-based 3D object detection in LiDAR point clouds. Considering the task of object detection inherently focuses on the foreground information, we propose an instance-aware learning-based downsampling way to automatically select the sparse yet important instance points. Additionally, a dedicated contextual centroid perception module is proposed to fully exploit the geometrical structure around the bounding boxes. Extensive experiments conducted on three detection benchmarks demonstrated the superior efficiency and accuracy of the proposed \nickname{}. 
}

\smallskip\noindent\textbf{Limitations.} Although the proposed \nickname{} can achieve remarkable efficiency in object detection of large-scale LiDAR points clouds, it also has limitations. \textit{e.g.,} the instance-aware sampling relies on the semantic prediction of each point, which is susceptible to class imbalances distribution. For future work, we will further explore advanced techniques to alleviate the imbalanced issue.



\bibliography{main_paper}
\bibliographystyle{icml2025}

\newpage
\appendix
\onecolumn
\appendix
\newpage
\lstset{language=Python,basicstyle=\small\ttfamily,columns=fullflexible}
\section{Detailed Experimental Setup}\label{appendix:experiments}

For reproducibility and completeness, we provide comprehensive details of all setups, datasets, tasks, models, baselines, and hyperparameters. Code is in the process of being released.

\subsection{Tasks, Datasets, and Data Partitioning}

\noindent\textbf{XNLI~\cite{XNLI}}~A natural language inference benchmark dataset for evaluating cross-lingual understanding covering 15 diverse languages including both high- and low-resources languages: English, French, Spanish, German, Greek, Bulgarian, Russian, Turkish, Arabic, Vietnamese, Thai, Chinese, Hindi, Swahili and Urdu. XNLI consists of premise-hypothesis pairs, labeled as entailment, contradiction, or neutral across different languages. We sample 2k instances from the XNLI train split and 500 instances from the test split for each pool. The data is then divided equally among 20 clients for each language using the latent Dirichlet allocation (LDA) partition with $\alpha=0.5$. Hence, the total number of clients is 600 (15 languages $\cdot$ 20 clients per language $\cdot$ 2 pools).

\noindent\textbf{MasakhaNEWS~\cite{MasakhaNEWS}}~A news topic classification benchmark designed to address the lack of resources for African languages. It covers 2 high-resource languages, English and French, and 14 low-resource languages, namely Amharic, Hausa, Igbo, Lingala, Luganda, Naija, Oromo, Rundi, chiShona, Somali, Kiswahili, Tigrinya, isiXhosa, and Yorùbá. Each sample contains a headline, the body text, and one of the 7 labels: business, entertainment, health, politics, religion, sports, or technology. We first combine all samples from the MasakhaNEWS train and validation split to form our train set, and use the MasakhaNEWS test split as our test set. We then split both train and test in each of the 16 languages by half for each pool. Following our XNLI setup, we adopt LDA $\alpha=0.5$ and divide each language's data equally into 10 clients. Hence, the total number of clients is 320 (16 languages $\cdot$ 10 clients per language $\cdot$ 2 pools). Note that unlike XNLI, the number of samples for each language differs, hence there is quantity skew across clients. 

\noindent\textbf{Fed-Aya~\cite{fedllm-bench}}~A federated instruction tuning benchmark, based on Aya~\cite{singh2024aya}, where the data is annotated by contributors and partitioned by annotator ID. Following FedLLM-Bench~\cite{fedllm-bench}, we focus on 6 high-resource languages, English, Spanish, French, Russian, Portuguese, Chinese, and 2 low-resource languages, Arabic and Telugu. Additionally, we filter out the other languages from the dataset. Out of 38 clients, we select 8 for our \unseen{} pool, $\text{client\_ids}=[21, 22, 23, 24, 25, 26, 27, 34]$ and the rest goes into our \seen{} pool. Each client can have up to 4 languages where the number of data samples can range from a hundred to over a thousand samples per client.

\subsection{Models, Tokenizers, and Data Preprocessing}

\noindent\textbf{mBERT~\cite{BERT}.}~We use the pretrained multilingual BERT with its WordPiece tokenizer for all sequence classification experiments, namely all XNLI and MasakhaNEWS setups with various {\em base models}. For both datasets, we use a training batch size of 32 and pad input tokens on the right to a max token length of 128 and 256 respectively.

\noindent\textbf{MobileLLaMA-1.4B~\cite{mobilellama}.}~We train a \basemodel{} with a pretrained MobileLLaMA-1.4B with Standard FL using LoRA in our Fed-Aya setup. We use the default LLaMA tokenizer which is a BPE model based on sentencepiece~\cite{Kudo2018SentencePieceAS} and adopt the UNK token as the PAD token. During training, we use an effective batch size of $16$ and pad right to the longest token in the batch up to a max token length of 1024. For evaluation, we use a batch size of 8, padding left instead, with greedy sampling up to a max new token length of 1024. We use the Alpaca template to format each prompt:

\begin{lstlisting}[linewidth=\columnwidth,breaklines=true]
alpaca_template = """Below is an instruction that describes a task. Write a response that appropriately completes the request.

### Instruction:
{} 

### Response: {}{}"""
\end{lstlisting}

\noindent\textbf{Llama-3.2-3B~\cite{llama3}.}~We use the off-the-shelf Llama-3.2-3B-Instruct model as our \basemodel{} and its default tokenizer which is a BPE model based on tiktoken\footnote{https://github.com/openai/tiktoken}. Training and evaluation hyperparameters are the same as the ones we use for MobileLLaMA. The only two differences are \textit{1)} we add a PAD token `\verb|<pad>|', and \textit{2)} we use the Llama 3 instruction template instead:

\begin{lstlisting}[linewidth=\columnwidth,breaklines=true]
llama3_instruct_template = """<|begin_of_text|><|start_header_id|>user<|end_header_id|>

{}<|eot_id|><|start_header_id|>assistant<|end_header_id|>

{}{}
"""
\end{lstlisting}

\subsection{Complementary Approaches and Base Models}

In this work, we experiment with different {\em base models} to show that \method{} is complementary to a range of off-the-shelf models and models trained using existing FL approaches. In this section, we detail the different approaches we used to obtain these {\em base models}.

\noindent\textbf{Standard FL.} Standard FL involves training a single global model. Given a pretrained LLM, we run FedAvg on the \seen{} pool of clients, where $10\%$ of participating clients are sampled every round to train the model before sending the weights back for aggregation. In our XNLI and MasakhaNEWS setup, we do full fine-tuning of mBERT, setting each client's learning rate to $5e-5$ and running FedAvg for 100 rounds. In our Fed-Aya setup, we adopt the training recipe from FedLLM-Bench~\cite{fedllm-bench} for MobileLLaMA-1.4B, where we do PEFT with LoRA applied to query and value attention weights ($r=16$, $\alpha_{lora}=32$, dropout$=0.05$) for 200 rounds. We use the cosine learning rate decay over 200 rounds with initial learning rate $2e^{-5}$ and minimum learning rate $1e^{-6}$.

\noindent\textbf{Personalized FL.} We train personalized {\em base models} using FedDPA-T~\cite{FedDPA} and DEPT(SPEC)~\cite{DEPT} in our XNLI setup. FedDPA-T proposed having two separate LoRA adapters, one of which is shared (global) and the other is kept locally for each client (local). 
We adopted their training recipe for sequence classification, where the classifier is shared together with the global LoRA modules and the local LoRA modules stay local. LoRA modules are only applied to query and value attention weights (r=8, $\alpha_{lora}=8$, dropout=$0.05$). We set the learning rate to $5e^{-5}$.

DEPT (SPEC), on the other hand, proposed having personalized token and positional embeddings for each client. As DEPT was proposed for cross-silo FL, while we target cross-device FL, they assumed that each data source has an abundance of data to retrain the newly initialized embeddings. Hence, to adapt to the cross-device FL setting, we did not reinitialize the embeddings; each client fine-tunes their own embeddings starting from the pretrained mBERT embeddings. In other words, for each FL round, each client does full fine-tuning, sending weights of all layers except their own embeddings back to the server for aggregation. As with FedDPA-T, the learning rate is set to $5e^{-5}$.

\noindent\textbf{Off-the-shelf.} In our Fed-Aya setup, we use an off-the-shelf instruction finetuned Llama-3.2-3B-Instruct as our \basemodel{}.

\subsection{Baselines and \method{}}

To avoid an exponentially big search space, all hyperparameter tuning is done using simple grid search on our XNLI setup, with mBERT, and Fed-Aya setup, with MobileLLaMA as the \basemodel{}. The best hyperparameters found are then used for MasakhaNEWS and Fed-Aya with Llama3 respectively.

\noindent\textbf{LoRA PEFT~\cite{hu2021lora}.}~We search for the learning rate $[1e^{-5},1e^{-4},1e^{-3}]$ and the number of epochs $[1,2,3]$ and find that the learning rate $1e^{-4}$ with $2$ epochs had the best performance on the train set. We fixed $\alpha_{lora}=2r$. To ensure a similar inference budget across baselines, we set the number of epochs to $2$ for all our experiments.

\noindent\textbf{AdaLoRA~\cite{adalora}.}~Similarly to LoRA, we search for the learning rate $[1e^{-5},1e^{-4},1e^{-3},1e^{-2}]$, the time interval between two budget allocations, $\Delta_T$, $[1,10,100]$ and the coefficient of the orthogonal regularization, $\gamma$, $[0.1,0.5]$. Within our search space, we find learning rate$=1e^{-3}$, $\Delta_T=1.0$, and $\gamma=0.1$ to be the best performing one. We run AdaLoRA once per resource budget $r$, setting the initial rank to be $1.5\times$ of $r$, as recommended. We set the initial fine-tuning warmup steps and final fine-tuning steps to be $10\%$ and $30\%$ of the total steps respectively. 

\noindent\textbf{BayesTune-LoRA (Section\ref{sec:personalized_peft}).}~For fair comparison with \method{}, we use the same hyperparameters as \method{}. This baseline, hence, is an ablation study of how much performance collaboratively learning a PSG adds.

\noindent\textbf{FedL2P~\cite{royson2023fedl2p}}~As FedL2P requires a validation set for outer-loop bi-level optimization and federated early stopping, we split the train set of every client $80\%$ train and $20\%$ validation. Following FedL2P, we set the federated early stopping patience to 50 rounds, MLP hidden dimension is set to 100, the inner-loop learning rate to be the same as finetuning, $1e^{-4}$, and the hypergradient hyperparameters, $Q=3, \psi=0.1$ with hypergradient clamping of $[-1,1]$. 
We use Adam for both the inner-loop and outer-loop optimizers and search for the learning rate for the MLP (LRNet) $[1e^{-5},1e^{-4},1e^{-3}]$ and the learnable post-multiplier learning rate $\tilde{\eta}$ $[1e^{-4},1e^{-5},1e^{-6}]$, picking $1e^{-4}$ and $1e^{-6}$ to be the best respectively. Finally, we use $3$ outer-loop steps with an effective outer-loop batch size of $16$.

\noindent\textbf{\method{} (Section~\ref{sec:main_method})}~We set $\alpha_{r\_mul}=2$, our resulting $r_{init}$ is, hence, $32$ since our $r_{\text{max target}}=16$ for all experiments. Following our standard LoRA fine-tuning baseline, we adopt the same learning rate and $\alpha_{lora}$, $1e-4$ and $2r_{init}$ respectively. The learning rate of $\bm{\lambda}$, on the other hand, is searched $[1e^{-1},1e^{-2},1e^{-3},1e^{-4}]$, and we pick $1e^{-2}$ for all experiments.
All $\lambda$ values are initialized to $1e^{-4}$. The MLP hidden dimension is set to $2\times$ the input dimension, which is model dependent. We clamp the output of the MLP as well as $\lambda$ with a minimum value of $1e-4$ in the forward pass during training. Following FedL2P, we use a straight-through estimator~\cite{bengio2013estimating} after clamping to propagate gradients. We initialize the weights of MLP with Xavier initialization~\cite{glorot2010understanding} using the normal distribution with a gain value of $1e-6$ and the bias with a constant $1e-4$. Lastly, we set $\alpha_s=1e^{+2}$ and $\alpha_p=1e^{-2}$ for all experiments.


\section{Additional Results}

This section contains supplementary results and analyses, omitted from the main paper due to space limitations, that complement the presented findings.
Fig.~\ref{fig:xnli_dept_out_r16}-\ref{fig:llama3_fedavg_out_r2} show the language-agnostic rank structures under different budgets ($r=2$ and $r=16$) learnt by \method{} for different setups as mentioned in Section~\ref{sec:analysis}. These plots illustrate the prioritization of layers for LoRA fine-tuning. 

Note that while the rank structure is the same across languages, the strength of personalization, absolute value of $\bm{\lambda}$, differs, as shown in Fig.~\ref{fig:masakha_out} in this Section and Fig.~\ref{fig:xnli_out} in the main paper. These two figures show the difference in $\bm{\lambda}$ across languages as described in Section~\ref{sec:analysis}. To sum up, the smaller the distance between two languages, represented as a block in the figure, the more similar their generated $\bm{\lambda}$ are. The results show that while similar languages sometimes exhibit similar $\bm{\lambda}$ values, unrelated languages also occasionally share similar $\bm{\lambda}$, consistent with findings in the literature that leveraging dissimilar languages can be beneficial.

Lastly, Tables~\ref{tab:xnli_seen} and \ref{tab:xnli_unseen} contain results for our XNLI setup where the \basemodel{} is fine-tuned from the pretrained mBERT with Standard FL using full fine-tuning and is used to complement results and findings in Section~\ref{sec:text_class}. Similarly, Tables~\ref{tab:mobilellama_fedaya_seen} and \ref{tab:mobilellama_fedaya_unseen} complement the results and findings of our Fed-Aya setup described in Section~\ref{sec:ift_gen}.


\begin{figure}[t]
    \small
    \centering
    \includegraphics[width=1.0\columnwidth]{figures/masakha_out_0.5_seen.png}
    % \captionsetup{font=small,labelfont=bf}
    \vspace{-3em}
    \caption{$\bm{\lambda}$ distance among languages in our MasakhaNEWS setup. Each block shows the log-scale normalized average Euclidean distances between all pairs of clients' $\bm{\lambda}$ for two languages (see text). The smaller the distance, the more similar $\bm{\lambda}$ is. }
    \label{fig:masakha_out}
    % \vspace{-2em}
\end{figure}

\begin{table*}[t]
\centering
\caption{Mean±SD Accuracy of each language across 3 different seeds for clients in the \seen{} pool of our XNLI setup. The pretrained model is trained using Standard FL with full fine-tuning and the resulting \basemodel{} is personalized to each client given a baseline approach.}
\label{tab:xnli_seen}
\begin{scriptsize}\resizebox{0.98\textwidth}{!}{

\begin{tabular}{c|l|l|l|l|l|l|l|l|l|l|l|l|l|l|l|l|c}
\toprule
% \textbf{Lora Rank}  

\textbf{$\mathbf{r}$} & \multicolumn{1}{c|}{\textbf{Approach}} & \multicolumn{1}{c|}{\textbf{bg}} & \multicolumn{1}{c|}{\textbf{hi}} & \multicolumn{1}{c|}{\textbf{es}} & \multicolumn{1}{c|}{\textbf{el}} & \multicolumn{1}{c|}{\textbf{vi}} & \multicolumn{1}{c|}{\textbf{tr}} & \multicolumn{1}{c|}{\textbf{de}} & \multicolumn{1}{c|}{\textbf{ur}} & \multicolumn{1}{c|}{\textbf{en}} & \multicolumn{1}{c|}{\textbf{zh}} & \multicolumn{1}{c|}{\textbf{th}} & \multicolumn{1}{c|}{\textbf{sw}} & \multicolumn{1}{c|}{\textbf{ar}} & \multicolumn{1}{c|}{\textbf{fr}} & \multicolumn{1}{c|}{\textbf{ru}} & \textbf{Wins} \\ \midrule
% \multirow{5}{*}{1}  & LoRA                                   & 46.60±0.28                        & 45.20±0.16                        & 50.00±0.28                        & 49.60±0.16                        & 48.73±0.19                       & 47.67±0.19                       & 48.93±0.09                       & 48.33±0.19                       & 51.27±0.09                       & 49.27±0.09                       & 44.60±0.16                        & 43.20±0.33                        & 44.40±0.28                        & 49.53±0.19                       & 47.13±0.09                       & 0             \\ %\cline{2-18} 
%                     & AdaLoRA                              & 45.67±0.09                       & 44.47±0.09                       & 48.80±0.16                        & 49.07±0.09                       & 48.27±0.25                       & 47.07±0.50                       & 48.53±0.25                       & 47.93±0.09                       & 51.00±0.16                        & 48.53±0.09                       & 43.87±0.34                       & 42.80±0.33                        & 44.20±0.28                        & 48.93±0.34                       & 46.20±0.28                        & 0             \\ %\cline{2-18} 
%                     & BayesTune-LoRA                            & 45.67±0.09                       & 44.33±0.09                       & 48.53±0.09                       & 48.80±0.16                        & 47.80±0.00                        & 46.80±0.16                        & 48.33±0.09                       & 47.87±0.19                       & 50.93±0.09                       & 48.07±0.25                       & 43.60±0.16                        & 42.27±0.25                       & 44.00±0.16                        & 48.67±0.09                       & 45.60±0.16                        & 0             \\ %\cline{2-18} 
%                     & FedL2P                               & \textbf{65.47±1.39}              & \textbf{69.47±1.43}              & \textbf{73.53±1.27}              & \textbf{69.33±1.06}              & \textbf{70.13±1.47}              & \textbf{71.00±1.34}               & \textbf{72.00±0.99}               & \textbf{72.13±0.25}              & \textbf{74.73±0.93}              & \textbf{67.40±1.07}               & \textbf{67.80±0.75}               & \textbf{71.93±0.34}              & \textbf{70.80±1.45}               & \textbf{72.67±0.66}              & \textbf{72.20±1.02}               & \textbf{15}   \\ %\cline{2-18} 
%                     & \method{}                                 & 62.93±1.11                       & 63.60±1.4                         & 70.40±1.82                        & 65.80±1.61                        & 67.67±1.05                       & 65.07±2.17                       & 67.67±0.66                       & 67.00±1.42                        & 73.53±0.62                       & 62.33±0.77                       & 62.93±0.50                       & 69.13±0.41                       & 64.93±1.89                       & 67.47±1.25                       & 66.87±1.15                       & 0             \\ \hline
\multirow{5}{*}{2}  & LoRA                                   & 47.47±0.19                       & 45.93±0.34                       & 50.80±0.16                        & 50.80±0.16                        & 50.80±0.28                        & 48.80±0.59                        & 50.07±0.66                       & 49.67±0.47                       & 53.13±0.41                       & 50.00±0.28                        & 45.47±0.41                       & 44.33±0.09                       & 45.33±0.25                       & 51.00±0.16                        & 48.33±0.74                       & 0             \\ %\cline{2-18} 
                    & AdaLoRA                              & 45.73±0.09                       & 44.40±0.00                        & 49.00±0.28                        & 49.13±0.19                       & 48.00±0.16                        & 47.07±0.34                       & 48.27±0.09                       & 47.87±0.09                       & 50.93±0.09                       & 48.20±0.16                        & 43.80±0.16                        & 43.00±0.00                        & 44.20±0.16                        & 48.87±0.09                       & 46.27±0.25                       & 0             \\ %\cline{2-18} 
                    & BayesTune-LoRA                            & 45.67±0.19                       & 44.40±0.00                        & 48.33±0.25                       & 48.80±0.16                        & 48.13±0.19                       & 46.80±0.28                        & 48.13±0.09                       & 47.80±0.00                        & 50.87±0.19                       & 48.00±0.00                        & 43.53±0.25                       & 42.13±0.19                       & 44.13±0.09                       & 48.93±0.09                       & 45.67±0.09                       & 0             \\ %\cline{2-18} 
                    & FedL2P                               & 66.67±0.90                       & 69.47±0.77                       & 74.33±0.93                       & 70.73±1.00                        & 71.27±0.82                       & 71.27±0.57                       & 72.27±1.32                       & 73.20±0.28                        & 75.27±0.81                       & 68.27±1.32                       & 67.93±0.75                       & 73.47±1.31                       & 71.47±0.94                       & 72.80±0.28                        & 73.07±0.25                       & 0             \\ %\cline{2-18} 
                    & \method{}                                 & \textbf{71.73±0.41}              & \textbf{72.33±0.25}              & \textbf{75.40±0.59}               & \textbf{73.73±0.84}              & \textbf{74.80±0.43}               & \textbf{74.73±0.41}              & \textbf{75.00±0.71}               & \textbf{74.33±0.57}              & \textbf{75.47±0.19}              & \textbf{70.53±0.52}              & \textbf{71.33±0.09}              & \textbf{73.73±0.41}              & \textbf{75.60±0.57}               & \textbf{74.60±0.59}               & \textbf{74.13±0.68}              & \textbf{15}   \\ \hline
\multirow{5}{*}{4}  & LoRA                                   & 49.40±0.16                        & 50.00±0.86                        & 54.47±0.62                       & 53.33±0.41                       & 53.00±0.28                        & 51.53±0.66                       & 53.27±0.62                       & 52.93±1.09                       & 56.27±0.47                       & 52.20±0.43                        & 49.07±0.62                       & 48.80±0.00                        & 49.20±0.33                        & 54.27±0.34                       & 51.73±1.11                       & 0             \\ %\cline{2-18} 
                    & AdaLoRA                              & 45.87±0.19                       & 44.33±0.09                       & 48.60±0.16                        & 48.93±0.09                       & 48.27±0.34                       & 46.87±0.09                       & 48.47±0.09                       & 47.80±0.00                        & 51.27±0.09                       & 48.07±0.09                       & 43.67±0.09                       & 42.73±0.25                       & 44.20±0.28                        & 48.73±0.09                       & 45.93±0.19                       & 0             \\ %\cline{2-18} 
                    & BayesTune-LoRA                            & 45.60±0.00                        & 44.33±0.09                       & 48.87±0.19                       & 49.13±0.09                       & 48.00±0.28                        & 47.00±0.33                        & 48.47±0.09                       & 47.80±0.16                        & 51.13±0.09                       & 48.13±0.09                       & 43.73±0.19                       & 42.40±0.16                        & 44.20±0.00                        & 48.93±0.09                       & 46.00±0.16                        & 0             \\ %\cline{2-18} 
                    & FedL2P                               & 67.40±1.42                        & 69.73±1.37                       & 74.47±0.34                       & 70.20±0.85                        & 71.53±0.94                       & 71.27±0.90                       & 72.73±0.94                       & 73.07±0.34                       & 75.27±0.47                       & 68.73±1.16                       & 68.27±0.47                       & 73.73±0.66                       & 71.47±1.64                       & 73.47±0.50                       & 72.87±0.52                       & 0             \\ %\cline{2-18} 
                    & \method{}                                 & \textbf{72.80±0.33}               & \textbf{72.13±0.41}              & \textbf{75.40±0.28}               & \textbf{74.27±0.34}              & \textbf{74.93±0.09}              & \textbf{75.20±0.28}               & \textbf{75.80±0.16}               & \textbf{75.07±0.25}              & \textbf{75.93±0.25}              & \textbf{71.60±0.16}               & \textbf{70.80±0.28}               & \textbf{73.07±0.25}              & \textbf{75.53±0.62}              & \textbf{75.87±0.34}              & \textbf{74.87±0.19}              & \textbf{15}   \\ \hline
\multirow{5}{*}{8}  & LoRA                                   & 55.33±1.60                       & 54.13±0.09                       & 59.93±0.50                       & 58.20±1.28                        & 58.07±0.25                       & 56.53±0.25                       & 59.73±0.09                       & 58.47±1.33                       & 63.40±0.71                        & 57.13±0.34                       & 56.13±0.19                       & 56.13±0.50                       & 55.93±0.09                       & 59.20±0.33                        & 58.20±2.01                        & 0             \\ %\cline{2-18} 
                    & AdaLoRA                              & 45.80±0.00                        & 44.40±0.00                        & 48.47±0.09                       & 49.00±0.16                        & 48.13±0.09                       & 46.93±0.09                       & 48.27±0.09                       & 48.00±0.16                        & 50.80±0.16                        & 48.07±0.09                       & 43.73±0.09                       & 42.40±0.16                        & 44.20±0.16                        & 48.93±0.09                       & 45.93±0.09                       & 0             \\ %\cline{2-18} 
                    & BayesTune-LoRA                            & 45.87±0.25                       & 44.33±0.09                       & 49.13±0.47                       & 49.27±0.09                       & 48.20±0.00                        & 47.07±0.09                       & 48.60±0.16                        & 47.80±0.00                        & 51.00±0.28                        & 48.60±0.16                        & 44.13±0.09                       & 42.93±0.09                       & 44.40±0.00                        & 49.20±0.43                        & 46.20±0.16                        & 0             \\ %\cline{2-18} 
                    & FedL2P                               & 64.73±1.27                       & 66.07±2.19                       & 72.13±1.80                       & 67.60±1.42                        & 68.80±1.85                        & 68.47±1.89                       & 69.40±1.99                        & 70.93±2.22                       & 73.47±1.18                       & 65.93±1.86                       & 65.73±1.98                       & 70.73±2.96                       & 67.73±1.93                       & 70.53±1.93                       & 70.47±1.89                       & 0             \\ %\cline{2-18} 
                    & \method{}                                 & \textbf{73.93±0.34}              & \textbf{70.67±0.09}              & \textbf{75.87±0.19}              & \textbf{73.80±0.16}               & \textbf{74.33±0.47}              & \textbf{75.60±0.16}               & \textbf{74.93±0.09}              & \textbf{74.27±0.19}              & \textbf{76.47±0.09}              & \textbf{72.53±0.52}              & \textbf{70.67±0.09}              & \textbf{71.67±0.34}              & \textbf{76.87±0.25}              & \textbf{75.87±0.25}              & \textbf{75.00±0.57}               & \textbf{15}   \\ \hline
\multirow{5}{*}{16} & LoRA                                   & 63.93±1.46                       & 64.20±0.00                        & 70.40±0.16                        & 67.07±1.32                       & 68.53±0.25                       & 66.07±0.66                       & 68.13±0.25                       & 69.67±0.62                       & 72.47±0.90                       & 64.67±1.24                       & 65.47±1.11                       & 69.87±0.90                       & 67.20±0.16                        & 68.73±0.09                       & 68.00±0.85                        & 0             \\ %\cline{2-18} 
                    & AdaLoRA                              & 45.80±0.16                        & 44.40±0.00                        & 48.73±0.25                       & 48.93±0.19                       & 48.07±0.19                       & 47.00±0.33                        & 48.27±0.09                       & 47.73±0.09                       & 50.93±0.19                       & 48.20±0.16                        & 43.47±0.09                       & 42.53±0.09                       & 44.20±0.00                        & 48.93±0.09                       & 45.80±0.00                        & 0             \\ %\cline{2-18} 
                    & BayesTune-LoRA                            & 46.27±0.09                       & 44.93±0.19                       & 49.67±0.25                       & 49.60±0.00                        & 48.60±0.16                        & 47.47±0.19                       & 49.00±0.00                        & 48.00±0.00                        & 51.47±0.19                       & 49.00±0.16                        & 44.67±0.09                       & 43.07±0.25                       & 44.33±0.25                       & 49.13±0.52                       & 46.67±0.34                       & 0             \\ %\cline{2-18} 
                    & FedL2P                               & 61.53±2.88                       & 62.33±3.44                       & 68.73±3.60                       & 65.07±1.88                       & 65.13±3.10                        & 65.07±3.20                        & 65.80±3.59                        & 67.73±2.87                       & 71.33±2.13                       & 62.73±2.78                       & 62.27±3.27                       & 67.20±3.92                        & 63.93±3.77                       & 68.07±2.88                       & 67.20±3.43                        & 0             \\ %\cline{2-18} 
                    & \method{}                                 & \textbf{73.87±0.09}              & \textbf{70.80±0.16}               & \textbf{75.87±0.09}              & \textbf{73.73±0.68}              & \textbf{74.93±0.25}              & \textbf{75.33±0.25}              & \textbf{74.40±0.16}               & \textbf{74.13±0.19}              & \textbf{76.53±0.19}              & \textbf{72.40±0.28}               & \textbf{70.80±0.28}               & \textbf{71.13±0.62}              & \textbf{76.20±0.43}               & \textbf{75.87±0.09}              & \textbf{75.13±0.09}              & \textbf{15}   \\ \bottomrule
\end{tabular}
}
\end{scriptsize}
\vspace{-1.5em}
\end{table*}

\begin{table*}[t]
\centering
\caption{Mean±SD Accuracy of each language across 3 different seeds for clients in the \unseen{} pool of our XNLI setup. The pretrained model is trained using Standard FL with full fine-tuning and the resulting \basemodel{} is personalized to each client given a baseline approach.}
\label{tab:xnli_unseen}
\begin{scriptsize}\resizebox{0.98\textwidth}{!}{

\begin{tabular}{c|l|l|l|l|l|l|l|l|l|l|l|l|l|l|l|l|c}
\toprule
% \textbf{Lora Rank}  
\textbf{$\mathbf{r}$} & \multicolumn{1}{c|}{\textbf{Approach}} & \multicolumn{1}{c|}{\textbf{bg}} & \multicolumn{1}{c|}{\textbf{hi}} & \multicolumn{1}{c|}{\textbf{es}} & \multicolumn{1}{c|}{\textbf{el}} & \multicolumn{1}{c|}{\textbf{vi}} & \multicolumn{1}{c|}{\textbf{tr}} & \multicolumn{1}{c|}{\textbf{de}} & \multicolumn{1}{c|}{\textbf{ur}} & \multicolumn{1}{c|}{\textbf{en}} & \multicolumn{1}{c|}{\textbf{zh}} & \multicolumn{1}{c|}{\textbf{th}} & \multicolumn{1}{c|}{\textbf{sw}} & \multicolumn{1}{c|}{\textbf{ar}} & \multicolumn{1}{c|}{\textbf{fr}} & \multicolumn{1}{c|}{\textbf{ru}} & \textbf{Wins} \\ \hline
% \multirow{5}{*}{1}  & LoRA                                   & 49.27±0.19                       & 45.47±0.09                       & 50.33±0.09                       & 48.20±0.00                        & 47.60±0.00                        & 44.87±0.09                       & 49.47±0.25                       & 46.60±0.00                        & 55.33±0.19                       & 49.60±0.43                        & 44.53±0.09                       & 40.27±0.09                       & 46.33±0.09                       & 48.53±0.25                       & 44.80±0.16                        & 0             \\ %\cline{2-18} 
%                     & AdaLoRA                              & 49.20±0.00                        & 45.40±0.16                        & 50.33±0.09                       & 48.27±0.09                       & 47.13±0.09                       & 44.67±0.25                       & 49.33±0.09                       & 46.53±0.09                       & 54.80±0.28                        & 48.53±0.25                       & 44.87±0.09                       & 40.07±0.09                       & 46.67±0.25                       & 48.80±0.00                        & 44.47±0.09                       & 0             \\ %\cline{2-18} 
%                     & BayesTune-LoRA                            & 49.00±0.00                        & 45.20±0.00                        & 50.07±0.09                       & 47.93±0.09                       & 47.07±0.09                       & 44.53±0.09                       & 49.27±0.25                       & 46.20±0.00                        & 54.60±0.16                        & 48.53±0.34                       & 44.73±0.09                       & 39.93±0.09                       & 46.73±0.09                       & 48.80±0.00                        & 44.53±0.19                       & 0             \\ %\cline{2-18} 
%                     & FedL2P                               & \textbf{54.67±0.68}              & \textbf{50.80±0.16}               & \textbf{56.40±0.57}               & \textbf{50.40±0.59}               & \textbf{53.73±0.62}              & 48.00±0.16                        & \textbf{53.73±0.19}              & \textbf{49.47±0.66}              & \textbf{59.87±0.25}              & \textbf{52.00±0.59}               & \textbf{47.93±0.41}              & 43.73±0.50                       & 50.33±0.57                       & 53.07±0.41                       & 48.47±0.25                       & \textbf{10}   \\ %\cline{2-18} 
%                     & \method{}                                 & 52.93±0.25                       & 48.80±1.23                        & 55.73±0.77                       & 49.93±0.19                       & 52.80±0.71                        & \textbf{48.47±0.47}              & 52.60±0.49                        & 48.93±0.25                       & 57.80±0.49                        & 50.80±0.16                        & 47.27±0.52                       & \textbf{44.07±0.50}              & \textbf{49.07±0.47}              & \textbf{53.67±0.25}              & \textbf{48.60±0.16}               & 5             \\ \hline
\multirow{5}{*}{2}  & LoRA                                   & 49.33±0.09                       & 45.93±0.25                       & 51.13±0.09                       & 48.53±0.25                       & 48.53±0.25                       & 45.27±0.25                       & 49.80±0.16                        & 46.87±0.34                       & 55.40±0.28                        & 49.33±0.25                       & 44.40±0.00                        & 41.20±0.16                        & 46.87±0.09                       & 48.33±0.09                       & 45.13±0.09                       & 0             \\ %\cline{2-18} 
                    & AdaLoRA                              & 49.33±0.09                       & 45.27±0.09                       & 50.27±0.19                       & 47.80±0.16                        & 47.27±0.09                       & 44.60±0.00                        & 49.33±0.34                       & 46.60±0.16                        & 54.87±0.25                       & 48.73±0.19                       & 44.73±0.09                       & 40.07±0.09                       & 46.67±0.09                       & 48.67±0.09                       & 44.53±0.09                       & 0             \\ %\cline{2-18} 
                    & BayesTune-LoRA                            & 49.13±0.09                       & 45.13±0.19                       & 50.20±0.00                        & 48.00±0.16                        & 47.07±0.19                       & 44.67±0.09                       & 49.07±0.25                       & 46.47±0.19                       & 54.67±0.09                       & 48.73±0.19                       & 44.80±0.16                        & 40.07±0.09                       & 46.60±0.16                        & 48.73±0.09                       & 44.27±0.09                       & 0             \\ %\cline{2-18} 
                    & FedL2P                               & \textbf{54.40±0.43}               & 49.93±0.25                       & 56.80±0.33                        & \textbf{51.40±0.33}               & 54.00±0.71                        & 48.27±0.50                       & \textbf{54.00±0.59}               & 50.13±0.41                       & \textbf{59.93±0.41}              & \textbf{52.47±0.25}              & 48.00±0.28                        & 44.53±0.34                       & \textbf{50.33±0.41}              & 53.00±0.28                        & 49.13±0.74                       & 6             \\ %\cline{2-18} 
                    & \method{}                                 & 54.00±0.59                        & \textbf{51.13±0.34}              & \textbf{57.67±0.09}              & 51.13±0.68                       & \textbf{54.53±0.25}              & \textbf{49.73±0.09}              & 53.33±0.66                       & \textbf{53.13±0.98}              & 59.60±0.33                        & 51.53±0.38                       & \textbf{48.87±0.19}              & \textbf{46.53±0.66}              & 49.00±0.43                        & \textbf{56.33±0.68}              & \textbf{52.47±0.41}              & \textbf{9}    \\ \hline
\multirow{5}{*}{4}  & LoRA                                   & 49.87±0.09                       & 47.07±0.09                       & 52.27±0.34                       & 49.67±0.19                       & 49.87±0.25                       & 46.47±0.34                       & 50.33±0.09                       & 48.20±0.16                        & 56.53±0.62                       & 49.80±0.16                        & 45.00±0.43                        & 41.40±0.00                        & 47.93±0.09                       & 48.73±0.09                       & 45.80±0.16                        & 0             \\ %\cline{2-18} 
                    & AdaLoRA                              & 49.27±0.09                       & 45.20±0.16                        & 50.07±0.25                       & 47.93±0.25                       & 47.27±0.09                       & 44.60±0.00                        & 49.27±0.19                       & 46.53±0.09                       & 54.73±0.25                       & 48.60±0.16                        & 44.60±0.16                        & 40.07±0.19                       & 46.47±0.09                       & 48.80±0.00                        & 44.53±0.34                       & 0             \\ %\cline{2-18} 
                    & BayesTune-LoRA                            & 49.33±0.19                       & 45.13±0.09                       & 50.00±0.16                        & 48.20±0.16                        & 47.13±0.09                       & 44.67±0.09                       & 49.40±0.00                        & 46.40±0.16                        & 54.93±0.09                       & 48.87±0.09                       & 44.60±0.00                        & 40.13±0.09                       & 46.60±0.00                        & 48.53±0.09                       & 44.40±0.00                        & 0             \\ %\cline{2-18} 
                    & FedL2P                               & 54.73±0.38                       & \textbf{50.93±0.66}              & 57.27±0.34                       & \textbf{51.47±0.34}              & \textbf{54.67±1.09}              & 47.67±0.25                       & \textbf{54.20±0.43}               & 50.93±0.25                       & \textbf{59.80±0.59}               & \textbf{52.73±0.09}              & 47.87±0.50                       & 44.80±0.00                        & 50.20±0.28                        & 53.47±0.50                       & 49.53±0.93                       & 6             \\ %\cline{2-18} 
                    & \method{}                                 & \textbf{55.33±0.34}              & 50.60±0.91                        & \textbf{58.20±0.16}               & 51.33±0.41                       & 54.60±0.28                        & \textbf{48.80±0.28}               & 52.73±0.57                       & \textbf{53.47±0.47}              & 59.13±0.68                       & 51.53±0.34                       & \textbf{48.47±0.41}              & \textbf{47.20±0.59}               & \textbf{49.00±0.16}               & \textbf{56.73±0.66}              & \textbf{52.13±0.41}              & \textbf{9}    \\ \hline
\multirow{5}{*}{8}  & LoRA                                   & 51.00±0.33                        & 48.13±0.50                       & 54.00±0.33                        & 50.87±0.34                       & 51.20±0.28                        & 46.93±0.25                       & 52.13±1.05                       & 48.27±0.25                       & 58.33±0.19                       & 50.13±0.19                       & 45.80±0.16                        & 42.40±0.00                        & 48.13±1.05                       & 50.93±0.09                       & 46.80±0.59                        & 0             \\ %\cline{2-18} 
                    & AdaLoRA                              & 49.20±0.16                        & 45.40±0.16                        & 50.13±0.19                       & 48.13±0.09                       & 47.07±0.09                       & 44.73±0.09                       & 49.13±0.09                       & 46.33±0.09                       & 54.93±0.19                       & 48.47±0.34                       & 44.60±0.00                        & 40.13±0.09                       & 46.67±0.09                       & 48.67±0.19                       & 44.33±0.25                       & 0             \\ %\cline{2-18} 
                    & BayesTune-LoRA                            & 49.27±0.09                       & 45.07±0.19                       & 50.27±0.09                       & 47.93±0.09                       & 47.20±0.16                        & 44.73±0.09                       & 49.27±0.19                       & 46.33±0.19                       & 54.93±0.09                       & 48.53±0.09                       & 44.53±0.09                       & 40.33±0.09                       & 46.67±0.09                       & 48.53±0.09                       & 44.53±0.38                       & 0             \\ %\cline{2-18} 
                    & FedL2P                               & \textbf{54.13±0.25}              & 49.60±1.14                        & 57.47±0.82                       & 50.87±0.50                       & 53.20±0.28                        & 47.60±0.43                        & \textbf{54.20±0.16}               & 49.87±0.09                       & \textbf{59.53±0.62}              & \textbf{52.27±0.90}              & 47.47±0.25                       & 43.60±0.71                        & \textbf{50.27±0.52}              & 52.60±0.57                        & 48.80±0.57                        & 5             \\ %\cline{2-18} 
                    & \method{}                                 & 53.40±0.00                        & \textbf{50.60±0.33}               & \textbf{57.67±0.62}              & 50.53±0.34                       & \textbf{55.13±0.38}              & \textbf{48.00±0.75}               & 52.53±0.34                       & \textbf{51.87±0.19}              & 57.00±1.02                        & 50.67±0.09                       & \textbf{47.60±0.00}               & \textbf{46.67±0.52}              & 48.47±0.66                       & \textbf{55.00±0.33}               & \textbf{50.73±0.75}              & \textbf{9}    \\ \hline
\multirow{5}{*}{16} & LoRA                                   & 52.73±0.25                       & 48.80±0.00                        & 55.47±0.25                       & \textbf{51.13±1.05}              & 53.53±0.41                       & \textbf{48.87±0.75}              & 54.40±0.91                        & 49.40±0.99                        & 58.87±1.11                       & \textbf{51.80±0.49}               & 46.93±0.09                       & 43.60±0.99                        & 49.13±1.05                       & 52.13±0.81                       & 48.93±2.03                       & 3             \\ %\cline{2-18} 
                    & AdaLoRA                              & 49.13±0.09                       & 45.20±0.00                        & 50.20±0.16                        & 47.93±0.25                       & 46.93±0.19                       & 44.53±0.09                       & 49.33±0.19                       & 46.53±0.09                       & 54.80±0.16                        & 48.73±0.09                       & 44.73±0.19                       & 40.13±0.09                       & 46.73±0.25                       & 48.67±0.09                       & 44.40±0.28                        & 0             \\ %\cline{2-18} 
                    & BayesTune-LoRA                            & 49.33±0.25                       & 45.33±0.19                       & 50.20±0.00                        & 47.93±0.09                       & 47.53±0.25                       & 44.80±0.16                        & 49.40±0.33                        & 46.53±0.09                       & 55.27±0.09                       & 48.87±0.34                       & 44.47±0.09                       & 40.20±0.28                        & 46.53±0.09                       & 48.73±0.25                       & 44.73±0.25                       & 0             \\ %\cline{2-18} 
                    & FedL2P                               & 52.53±1.16                       & 49.00±0.28                        & 56.40±1.40                        & 50.47±0.96                       & 52.47±0.68                       & 47.80±0.28                        & \textbf{53.93±0.38}              & 48.80±0.85                        & \textbf{59.07±0.41}              & 51.73±0.75                       & 47.13±0.82                       & 43.00±1.14                        & \textbf{49.27±0.84}              & 52.07±1.16                       & 48.33±0.74                       & 3             \\ %\cline{2-18} 
                    & \method{}                                 & \textbf{53.73±0.52}              & \textbf{49.87±0.50}              & \textbf{57.07±0.52}              & 49.87±0.25                       & \textbf{54.67±0.90}              & 47.67±0.41                       & 52.27±0.38                       & \textbf{52.27±0.52}              & 57.87±0.68                       & 50.33±0.66                       & \textbf{47.60±0.33}               & \textbf{46.20±0.16}               & 49.00±0.49                        & \textbf{54.60±0.71}               & \textbf{49.67±0.57}              & \textbf{9}    \\ \bottomrule
\end{tabular}
}
\end{scriptsize}
\vspace{-1.5em}
\end{table*}

\begin{table*}[t]
\npdecimalsign{.}
\nprounddigits{2}
\caption{Average METEOR/ROUGE-1/ROUGE-L of each language for \seen{} clients of our Fed-Aya setup. The pretrained MobileLLaMA-1.4B model is trained using Standard FL with LoRA following FedLLM-Bench~\cite{fedllm-bench} and the resulting \basemodel{} is personalized to each client given a baseline approach.}
\vspace{0.5em}
\label{tab:mobilellama_fedaya_seen}
\begin{scriptsize}\resizebox{0.98\textwidth}{!}{
\begin{tabular}{c|l|l|l|l|l|l|l|l|c}
\toprule
% \textbf{Lora}\\\textbf{Rank}  

\textbf{$\mathbf{r}$}& \multicolumn{1}{c|}{\textbf{Approach}} & \multicolumn{1}{c|}{\textbf{te}} & \multicolumn{1}{c|}{\textbf{ar}} & \multicolumn{1}{c|}{\textbf{es}} & \multicolumn{1}{c|}{\textbf{en}} & \multicolumn{1}{c|}{\textbf{fr}} & \multicolumn{1}{c|}{\textbf{zh}} & \multicolumn{1}{c|}{\textbf{pt}} 
& \textbf{Wins} \\ \midrule
% \multirow{5}{*}{1}  & LoRA                                   & 0.125/0.0637/0.0613              & 0.1723/0.0208/0.0203             & 0.3159/0.3469/0.3203             & 0.2458/0.3066/0.2478             & 0.1878/0.2455/0.1952             & \textbf{0.0649/0.084/0.084}      & 0.2527/0.3116/0.2792                                         & 0             \\ % \cline{2-11} 
%                     & AdaLoRA                              & 0.1218/0.0557/0.0537             & 0.1899/0.0173/0.0166             & 0.3241/0.3478/0.3218             & 0.2434/0.3058/0.2452             & 0.2048/0.2445/0.1988             & 0.0633/0.0841/0.0841             & 0.2692/0.329/0.2959                                          & 0             \\ % \cline{2-11} 
%                     & BayesTune-LoRA                            & 0.12/0.0557/0.0541               & 0.1582/0.0226/0.0223             & 0.2863/0.3269/0.2991             & 0.2326/0.2848/0.2319             & 0.1742/0.2298/0.184              & 0.06/0.0723/0.0723               & 0.2289/0.2837/0.2526                                         & 1             \\ % \cline{2-11} 
%                     & FedL2P                               & \textbf{0.1396/0.0769/0.0746}    & \textbf{0.2109/0.0207/0.0197}    & 0.3218/0.3527/0.3252             & \textbf{0.2692/0.3293/0.2697}    & \textbf{0.2365/0.2652/0.2105}    & 0.0527/0.0866/0.0865             & \textbf{0.2797/0.3228/0.2916}                                & 2             \\ % \cline{2-11} 
%                     & \method{}                                 & 0.126/0.0627/0.0603              & 0.2039/0.0221/0.0216             & \textbf{0.3335/0.3624/0.3346}    & 0.2645/0.3369/0.2748             & 0.2181/0.2667/0.2145             & 0.0616/0.0842/0.0842             & 0.2792/0.3373/0.3035                                         & \textbf{4}    \\ \hline
\multirow{5}{*}{2}  & LoRA                                   & 0.1207/0.0545/0.0524             & 0.1835/0.0210/0.0205              & 0.3228/0.3519/0.3251             & 0.2457/0.3121/0.2525             & 0.2049/0.2484/0.2002             & 0.0616/\textbf{0.0874/0.0873}             & 0.2631/0.3266/0.2922                                         & 1             \\ % \cline{2-11} 
                    & AdaLoRA                              & 0.1238/0.0607/0.0586             & 0.1819/0.0223/0.0218             & 0.3288/0.3573/0.3315             & 0.2459/0.3172/0.2524             & 0.1963/0.2422/0.1915             & \textbf{0.0689}/0.0832/0.0832    & 0.2745/0.3327/0.2986                                         & 0             \\ % \cline{2-11} 
                    & BayesTune-LoRA                            & 0.1245/0.0605/0.0580              & 0.1813/0.0181/0.0178             & 0.2941/0.3317/0.3063             & 0.2345/0.2892/0.2367             & 0.1885/0.2430/0.1970               & 0.0643/0.0805/0.0805             & 0.2408/0.2971/0.2645                                         & 0             \\ % \cline{2-11} 
                    & FedL2P                               & \textbf{0.1451/0.0747/0.0725}    & 0.2017/0.0219/0.0213             & 0.3321/0.3523/0.3245             & 0.2635/0.3307/0.2692             & 0.2298/0.2467/0.2034             & 0.0544/0.0803/0.0803             & 0.2780/0.3133/0.2832                                          & 1             \\ % \cline{2-11} 
                    & \method{}                                 & 0.1266/0.0629/0.0606             & \textbf{0.2081/0.0254/0.0248}    & \textbf{0.3425/0.3663/0.3376}    & \textbf{0.2745/0.3469/0.2831}    & \textbf{0.2342/0.2766/0.2246}    & 0.0524/0.0777/0.0777             & \textbf{0.2846/0.3403/0.3066}                                & \textbf{5}    \\ \hline
\multirow{5}{*}{4}  & LoRA                                   & 0.1232/0.0570/0.0537              & 0.1861/0.0202/0.0197             & 0.3284/0.3555/0.3291             & 0.2541/0.3202/0.2585             & 0.2037/0.2475/0.1990              & 0.0559/\textbf{0.0886/0.0886}             & 0.2734/0.3314/0.2975                                         & 1             \\ % \cline{2-11} 
                    & AdaLoRA                              & 0.1240/0.0596/0.0574              & 0.1858/0.0180/0.0177              & 0.3310/0.3548/0.3287              & 0.2448/0.3111/0.2500               & 0.1892/0.2331/0.1859             & 0.0617/0.0852/0.0851             & 0.2640/0.3250/0.2934                                           & 0             \\ % \cline{2-11} 
                    & BayesTune-LoRA                            & 0.1214/0.0548/0.0532             & 0.1912/0.0201/0.0195             & 0.3042/0.3405/0.3150              & 0.2405/0.3022/0.2440              & 0.1973/0.2393/0.1925             & \textbf{0.0670}/0.0806/0.0806     & 0.2468/0.3057/0.2737                                         & 0             \\ % \cline{2-11} 
                    & FedL2P                               & 0.1258/0.0616/0.0592             & 0.1805/0.0217/0.0208             & 0.3260/0.3568/0.3298              & 0.2519/0.3108/0.2498             & 0.2037/0.2493/0.2027             & 0.0484/0.0798/0.0798             & 0.2626/0.3174/0.2836                                         & 0             \\ % \cline{2-11} 
                    & \method{}                                 & \textbf{0.1350/0.0722/0.0692}     & \textbf{0.2218/0.0275/0.0269}    & \textbf{0.3448/0.3712/0.3419}    & \textbf{0.2796/0.3516/0.2883}    & \textbf{0.2469/0.2753/0.2244}    & 0.0554/0.0843/0.0843             & \textbf{0.2911/0.3397/0.3060}                                 & \textbf{6}    \\ \hline
\multirow{5}{*}{8}  & LoRA                                   & 0.1241/0.0522/0.0493             & 0.2059/0.0189/0.0184             & 0.3435/0.3688/0.3427             & 0.2686/0.3400/0.2771               & 0.2197/0.2556/0.2037             & 0.0583/\textbf{0.0886}/0.0884             & 0.2822/0.3418/0.3076                                         & 0             \\ % \cline{2-11} 
                    & AdaLoRA                              & 0.1245/0.0623/0.0599             & 0.1771/0.0189/0.0184             & 0.3215/0.3485/0.3225             & 0.2461/0.3101/0.2512             & 0.1799/0.2287/0.1839             & \textbf{0.0613}/0.0806/0.0806    & 0.2607/0.3194/0.2879                                         & 0             \\ % \cline{2-11} 
                    & BayesTune-LoRA                            & 0.1246/0.0591/0.0572             & 0.2046/0.0192/0.0189             & 0.3247/0.3520/0.3284              & 0.2430/0.3095/0.2496              & 0.2132/0.2542/0.2059             & 0.0601/0.0818/\textbf{0.0818}             & 0.2588/0.3171/0.2871                                         & 0             \\ % \cline{2-11} 
                    & FedL2P                               & 0.1316/\textbf{0.0687/0.0661}             & 0.1855/0.0218/0.0215             & 0.3272/0.3535/0.3280              & 0.2696/0.3304/0.2711             & 0.2109/0.2558/0.2059             & 0.0510/0.0816/0.0816              & 0.2750/0.3252/0.2897                                          & 1             \\ % \cline{2-11} 
                    & \method{}                                 & \textbf{0.1327}/0.0662/0.0644    & \textbf{0.2304/0.0253/0.0246}    & \textbf{0.3474/0.3847/0.3531}    & \textbf{0.2941/0.3656/0.2996}    & \textbf{0.2553/0.2829/0.2268}    & 0.0538/0.0814/0.0814             & \textbf{0.2945/0.3452/0.3108}                                & \textbf{5}    \\ \hline
\multirow{5}{*}{16} & LoRA                                   & 0.1217/0.0562/0.0536             & 0.2080/0.0221/0.0218              & 0.3387/0.3616/0.3352             & 0.2757/0.3431/0.2807             & \textbf{0.2497/0.2880/0.2337}     & 0.0553/\textbf{0.0844/0.0844}             & 0.2902/0.3449/0.3106                                         & 2             \\ % \cline{2-11} 
                    & AdaLoRA                              & 0.1251/0.0624/0.0602             & 0.1676/0.0198/0.0192             & 0.3048/0.3329/0.3037             & 0.2391/0.2985/0.2416             & 0.1821/0.2309/0.1866             & \textbf{0.0575}/0.0815/0.0815    & 0.2530/0.3099/0.2784                                          & 0             \\ % \cline{2-11} 
                    & BayesTune-LoRA                            & 0.1374/0.0745/0.0720              & 0.2119/0.0175/0.0172             & 0.3358/0.3649/0.3397             & 0.2587/0.3189/0.2577             & 0.2222/0.2603/0.2113             & 0.0520/0.0824/0.0824              & 0.2862/0.3450/0.3109                                          & 0             \\ % \cline{2-11} 
                    & FedL2P                               & \textbf{0.1559/0.0827/0.0799}    & 0.2013/0.0228/0.0226             & 0.3278/0.3541/0.3268             & 0.2772/0.3278/0.2693             & 0.2306/0.2346/0.1925             & 0.0506/0.0838/0.0838             & 0.2817/0.3179/0.2851                                         & 1             \\ % \cline{2-11} 
                    & \method{}                                 & 0.1309/0.0663/0.0638             & \textbf{0.2359/0.0258/0.0252}    & \textbf{0.3447/0.3778/0.3463}    & \textbf{0.2802/0.3485/0.2858}    & 0.2473/0.2775/0.2208             & 0.0538/0.0835/0.0835             & \textbf{0.2975/0.3491/0.3157}                                & \textbf{4}    \\ \bottomrule
\end{tabular}

}
\npnoround
\end{scriptsize}
\vspace{-1.5em}
\end{table*}


\begin{table*}[t]
\caption{Average METEOR/ROUGE-1/ROUGE-L of each language for \seen{} clients of our Fed-Aya setup. The pretrained MobileLLaMA-1.4B model is trained using Standard FL with LoRA following FedLLM-Bench~\cite{fedllm-bench} and the resulting \basemodel{} is personalized to each client given a baseline approach.}
\vspace{0.5em}
\label{tab:mobilellama_fedaya_unseen}
\begin{scriptsize}\resizebox{0.98\textwidth}{!}{
\begin{tabular}{c|l|l|l|l|l|l|l|l|l|c}
\toprule
\textbf{$\mathbf{r}$}  & \multicolumn{1}{c|}{\textbf{Approach}} & \multicolumn{1}{c|}{\textbf{te}} & \multicolumn{1}{c|}{\textbf{ar}} & \multicolumn{1}{c|}{\textbf{es}} & \multicolumn{1}{c|}{\textbf{en}} & \multicolumn{1}{c|}{\textbf{fr}} & \multicolumn{1}{c|}{\textbf{zh}} & \multicolumn{1}{c|}{\textbf{pt}} & \multicolumn{1}{c|}{\textbf{ru}} & \textbf{Wins} \\ \midrule
% \multirow{5}{*}{1}  & LoRA                                   & 0.0344/0.0000/0.0000                   & 0.1004/0.0564/0.0539             & 0.3329/0.4252/0.3809             & 0.1987/0.2339/0.1988             & 0.0505/0.0000/0.0000                   & 0.1457/0.0041/0.0034             & 0.1714/0.1907/0.1792             & \textbf{0.1174/0.0556/0.0556}    & 1             \\ % \cline{2-11} 
%                     & AdaLoRA                              & 0.0342/0.0042/0.0042             & 0.0941/0.0581/0.0556             & 0.3607/0.4708/0.4256             & 0.196/0.232/0.2011               & 0.049/0.0000/0.0000                    & \textbf{0.154/0.0092/0.0092}     & 0.1804/0.2222/0.2102             & 0.1075/0.0556/0.0556             & 2             \\ % \cline{2-11} 
%                     & BayesTune-LoRA                            & 0.0202/0.0000/0.0000                   & 0.0741/0.0461/0.0461             & 0.3357/0.3817/0.349              & 0.1912/0.2252/0.1858             & 0.0521/0.0000/0.0000                   & 0.1156/0.002/0.002               & 0.1592/0.164/0.151               & 0.0963/0.0556/0.0556             & 0             \\ % \cline{2-11} 
%                     & FedL2P                               & \textbf{0.1107/0.0075/0.0075}    & \textbf{0.1695/0.0391/0.0391}    & 0.3691/0.4312/0.4055             & 0.2432/0.2668/0.2291             & \textbf{0.0595/0.019/0.019}      & 0.1016/0.0101/0.0101             & \textbf{0.2085/0.2337/0.2237}    & 0.0746/0.0556/0.0556             & \textbf{3}    \\ % \cline{2-11} 
%                     & \method{}                                 & 0.1057/0.0042/0.0042             & 0.1191/0.0587/0.0587             & \textbf{0.3718/0.4349/0.4045}    & \textbf{0.2993/0.3342/0.297}     & 0.049/0.0000/0.0000                    & 0.1153/0.0049/0.0049             & 0.1928/0.2301/0.2169             & 0.0912/0.0556/0.0556             & 2             \\ \hline
\multirow{5}{*}{2}  & LoRA                                   & 0.0531/0.0042/0.0042             & 0.1032/\textbf{0.0524/0.0524}            & 0.3547/0.4490/0.4020               & 0.2385/0.2713/0.2356             & 0.0490/0.0000/0.0000                    & 0.1381/0.0088/0.0088             & 0.1865/0.2184/0.2070              & 0.0921/0.0556/0.0556             & 1             \\ % \cline{2-11} 
                    & AdaLoRA                              & 0.0312/0.0000/0.0000                   & 0.1000/0.0520/0.0520                  & 0.3244/0.4250/0.3821              & 0.2165/0.2606/0.2197             & 0.0490/\textbf{0.0513/0.0513}              & \textbf{0.1609}/0.0089/0.0089    & 0.1904/0.2258/0.2129             & \textbf{0.0992}/0.0556/0.0556    & 1             \\ % \cline{2-11} 
                    & BayesTune-LoRA                            & 0.0276/0.0000/0.0000                   & 0.0934/0.0501/0.0476             & 0.3560/0.4084/0.3716              & 0.1786/0.2024/0.1688             & 0.0450/0.0000/0.0000                    & 0.1566/0.0109/0.0092             & 0.1469/0.1628/0.1500               & 0.0873/0.0556/0.0556             & 0             \\ % \cline{2-11} 
                    & FedL2P                               & \textbf{0.1199}/0.0042/0.0042    & \textbf{0.1399}/0.0206/0.0206    & \textbf{0.3923/0.4587}/0.4194    & 0.2688/0.3049/0.2650              & 0.0024/0.0011/0.0011             & 0.0888/\textbf{0.0121/0.0121}             & \textbf{0.2022}/0.2170/0.2063     & 0.0850/0.0556/0.0556              & 2             \\ % \cline{2-11} 
                    & \method{}                                 & 0.1105/\textbf{0.0132/0.0114}             & 0.1127/0.0483/0.0483             & 0.3812/0.4546/\textbf{0.4209}             & \textbf{0.3047/0.3354/0.3006}    & 0.0490/0.0000/0.0000                    & 0.0650/0.0069/0.0069              & 0.1997/\textbf{0.2505/0.2384}             & 0.0956/0.0556/0.0556             & \textbf{3}    \\ \hline
\multirow{5}{*}{4}  & LoRA                                   & 0.0687/0.0075/0.0075             & 0.0989/0.0581/0.0556             & 0.3244/0.4086/0.3749             & 0.2297/0.2726/0.2370              & 0.0490/0.0513/0.0513              & 0.1530/0.0089/0.0089              & 0.1872/0.2267/0.2124             & 0.0974/0.0556/0.0556             & 0             \\ % \cline{2-11} 
                    & AdaLoRA                              & 0.0263/0.0000/0.0000                   & 0.0894/\textbf{0.0634}/0.0610              & 0.3336/0.4322/0.3883             & 0.1762/0.2105/0.1765             & 0.0490/0.0000/0.0000                    & \textbf{0.1630}/0.0066/0.0049     & 0.1832/0.2096/0.1980              & 0.1156/0.0556/0.0556             & 0             \\ % \cline{2-11} 
                    & BayesTune-LoRA                            & 0.0471/0.0075/0.0075             & 0.0938/0.0544/0.0544             & 0.3357/0.4058/0.3629             & 0.1657/0.1964/0.1647             & 0.1155/0.0635/0.0635             & 0.1557/\textbf{0.0115/0.0099}             & 0.1511/0.1636/0.1517             & 0.0858/0.0222/0.0222             & 1             \\ % \cline{2-11} 
                    & FedL2P                               & 0.0569/0.0075/0.0075             & 0.1002/0.0610/\textbf{0.0610}               & 0.3098/0.4066/0.3837             & 0.2494/0.2901/0.2569             & \textbf{0.1157/0.0741/0.0741}    & 0.1585/0.0041/0.0041             & 0.1724/0.1920/0.1806              & 0.0804/0.0556/0.0556             & 1             \\ % \cline{2-11} 
                    & \method{}                                 & \textbf{0.1121/0.0212/0.0212}    & \textbf{0.1537}/0.0333/0.0333    & \textbf{0.3725/0.4622/0.4273}    & \textbf{0.2945/0.3276/0.2821}    & 0.0490/0.0000/0.0000                    & 0.0722/0.0069/0.0069             & \textbf{0.1903/0.2547/0.2438}    & \textbf{0.1258/0.0556/0.0556}    & \textbf{5}    \\ \hline
\multirow{5}{*}{8}  & LoRA                                   & 0.1022/0.0042/0.0042             & 0.1206/0.0549/0.0549             & 0.3401/0.4382/0.3977             & 0.2722/\textbf{0.3127/0.2726}             & 0.0490/0.0000/0.0000                    & 0.1241/0.0089/0.0089             & 0.1879/0.2331/0.2202             & 0.0761/0.0556/0.0556             & 1             \\ % \cline{2-11} 
                    & AdaLoRA                              & 0.0218/0.0000/0.0000                   & 0.0897/0.0364/0.0364             & 0.3383/0.4295/0.3920              & 0.1786/0.2150/0.1795              & 0.1221/\textbf{0.0833/0.0833}             & 0.1470/0.0066/0.0049              & 0.1773/0.1909/0.1770              & 0.1124/0.0556/0.0556             & 1             \\ % \cline{2-11} 
                    & BayesTune-LoRA                            & 0.0529/0.0042/0.0042             & 0.0970/\textbf{0.0557/0.0557}              & 0.3544/0.4257/0.3789             & 0.1891/0.2235/0.1860              & \textbf{0.1262}/0.0784/0.0784    & \textbf{0.1522}/0.0068/0.0068    & 0.1365/0.1603/0.1483             & 0.0895/0.0556/0.0556             & 1             \\ % \cline{2-11} 
                    & FedL2P                               & 0.0934/0.0042/0.0042             & 0.1138/0.0495/0.0495             & 0.3237/0.4366/0.4068             & 0.2438/0.2849/0.2451             & 0.0490/0.0000/0.0000                    & 0.1408/\textbf{0.0139/0.0139}             & 0.1946/0.2280/0.2130               & \textbf{0.1187}/0.0556/0.0556    & 1             \\ % \cline{2-11} 
                    & \method{}                                 & \textbf{0.1099/0.0132/0.0114}    & \textbf{0.1565}/0.0082/0.0082    & \textbf{0.4471/0.5240/0.4876}     & \textbf{0.2798}/0.2851/0.2475    & 0.0490/0.0000/0.0000                    & 0.1071/0.0069/0.0069             & \textbf{0.2160/0.2700/0.2559}       & 0.0862/0.0465/0.0465             & \textbf{3}    \\ \hline
\multirow{5}{*}{16} & LoRA                                   & 0.1193/0.0042/0.0042             & \textbf{0.1418/0.0587/0.0587}    & 0.3647/0.4240/0.3927              & 0.2939/\textbf{0.3164/0.2753}             & 0.0490/0.0000/0.0000                    & 0.0962/0.0089/0.0089             & 0.1916/0.2438/0.2320              & \textbf{0.1403/0.0556/0.0556}    & \textbf{3}    \\ % \cline{2-11} 
                    & AdaLoRA                              & 0.0079/0.0000/0.0000                   & 0.0848/0.0386/0.0361             & 0.3548/0.4216/0.3864             & 0.1814/0.2158/0.1750              & 0.1443/0.1333/0.1333             & 0.1408/\textbf{0.0115/0.0099}             & 0.1828/0.1892/0.1793             & 0.0957/0.0556/0.0556             & 1             \\ % \cline{2-11} 
                    & BayesTune-LoRA                            & 0.0674/0.0075/0.0075             & 0.0887/0.0420/0.0420               & 0.3508/0.4174/0.3788             & 0.2088/0.2442/0.2106             & 0.1681/\textbf{0.1905/0.1905}             & \textbf{0.1594}/0.0089/0.0089    & 0.1877/0.2362/0.2225             & 0.0972/0.0556/0.0556             & 1             \\ % \cline{2-11} 
                    & FedL2P                               & 0.1093/\textbf{0.0673/0.0673}             & 0.1171/0.0320/0.0296              & 0.3895/0.4515/0.4131             & 0.2629/0.2606/0.2150              & 0.1349/0.1026/0.1026             & 0.1335/0.0041/0.0020              & 0.2011/0.2149/0.2027             & 0.0645/0.0253/0.0253             & 1             \\ % \cline{2-11} 
                    & \method{}                                 & \textbf{0.1289}/0.0165/0.0147    & 0.1215/0.0166/0.0166             & \textbf{0.4048/0.4910/0.4520}      & \textbf{0.2968}/0.3065/0.2678    & \textbf{0.2887}/0.1667/0.1667    & 0.0960/0.0069/0.0069              & \textbf{0.2359/0.2863/0.2704}    & 0.0791/0.0222/0.0222             & 2             \\ \bottomrule
\end{tabular}
}
\end{scriptsize}
\vspace{-1.5em}
\end{table*}


\begin{figure*}
\centering
\begin{minipage}{.23\linewidth}
  \includegraphics[width=\linewidth]{figures/xnli_fedavg_out_0.5_seen_en_layerwiserank.png}
  \captionof{figure}{Language agnostic rank structure of mBERT in our XNLI setup where the \basemodel{} is trained with FedIFT full-finetuning ($r=16$).}
  \label{fig:xnli_fedavg_out_r16}
\end{minipage}
\hspace{.01\linewidth}
\begin{minipage}{.23\linewidth}
  \includegraphics[width=\linewidth]{figures/xnli_fedavg_out_0.0625_seen_en_layerwiserank.png}
  \captionof{figure}{Language agnostic rank structure of mBERT in our XNLI setup where the \basemodel{} is trained with FedIFT full-finetuning ($r=2$).}
  \label{fig:xnli_fedavg_out_r2}
\end{minipage}
\hspace{.01\linewidth}
\begin{minipage}{.23\linewidth}
  \includegraphics[width=\linewidth]{figures/xnli_dept_out_0.5_seen_en_layerwiserank.png}
  \captionof{figure}{Language agnostic rank structure of mBERT in our XNLI setup where the \basemodel{} is trained with DEPT(SPEC) ($r=16$).}
  \label{fig:xnli_dept_out_r16}
\end{minipage}
\hspace{.01\linewidth}
\begin{minipage}{.23\linewidth}
  \includegraphics[width=\linewidth]{figures/xnli_dept_out_0.0625_seen_en_layerwiserank.png}
  \captionof{figure}{Language agnostic rank structure of mBERT in our XNLI setup where the \basemodel{} is trained with DEPT(SPEC) ($r=2$).}
  \label{fig:xnli_dept_out_r2}
\end{minipage}
\end{figure*}

\begin{figure*}
\centering
\begin{minipage}{.24\linewidth}
  \includegraphics[width=\linewidth]{figures/masakha_out_0.5_seen_eng_layerwiserank.png}
  \captionof{figure}{Language agnostic rank structure of mBERT in our MasakhaNEWS setup where the \basemodel{} is trained with FedIFT full-finetuning ($r=16$).}
  \label{fig:masakha_fedavg_out_r16}
\end{minipage}
\hspace{.2\linewidth}
\begin{minipage}{.24\linewidth}
  \includegraphics[width=\linewidth]{figures/masakha_out_0.0625_seen_eng_layerwiserank.png}
  \captionof{figure}{Language agnostic rank structure of mBERT in our MasakhaNEWS setup where the \basemodel{} is trained with FedIFT full-finetuning ($r=2$).}
  \label{fig:masakha_fedavg_out_r2}
\end{minipage}
\end{figure*}


\begin{figure*}
\centering
\begin{minipage}{.23\linewidth}
  \includegraphics[width=\linewidth]{figures/mobilellama_out_0.5_seen_en_layerwiserank.png}
  \captionof{figure}{Language agnostic rank structure of MobileLLaMA-1.4B in our Fed-Aya setup where the \basemodel{} is trained with FedIFT LoRA ($r=16$). Zoom in for best results.}
  \label{fig:mobilellama_fedavg_out_r16}
\end{minipage}
\hspace{.01\linewidth}
\begin{minipage}{.23\linewidth}
  \includegraphics[width=\linewidth]{figures/mobilellama_out_0.0625_seen_en_layerwiserank.png}
  \captionof{figure}{Language agnostic rank structure of MobileLLaMA-1.4B in our Fed-Aya setup where the \basemodel{} is trained with FedIFT LoRA ($r=2$). Zoom in for best results.}
  \label{fig:mobilellama_fedavg_out_r2}
\end{minipage}
\hspace{.01\linewidth}
\begin{minipage}{.23\linewidth}
  \includegraphics[width=\linewidth]{figures/llama3_out_0.5_seen_en_layerwiserank.png}
  \captionof{figure}{Language agnostic rank structure of Llama-3.2-3B in our Fed-Aya setup where the \basemodel{} is an off-the-shelf instruction tuned Llama-3.2-3B-Instruct ($r=16$). Zoom in for best results.}
  \label{fig:llama3_fedavg_out_r16}
\end{minipage}
\hspace{.01\linewidth}
\begin{minipage}{.23\linewidth}
  \includegraphics[width=\linewidth]{figures/llama3_out_0.0625_seen_en_layerwiserank.png}
  \captionof{figure}{Language agnostic rank structure of Llama-3.2-3B in our Fed-Aya setup where the \basemodel{} is an off-the-shelf instruction tuned Llama-3.2-3B-Instruct ($r=2$). Zoom in for best results.}
  \label{fig:llama3_fedavg_out_r2}
\end{minipage}
\end{figure*}






\end{document}

