%%%%%%%% ICML 2025 EXAMPLE LATEX SUBMISSION FILE %%%%%%%%%%%%%%%%%

\documentclass{article}

% Recommended, but optional, packages for figures and better typesetting:
\usepackage{microtype}
\usepackage{graphicx}
\usepackage{subfigure}
\usepackage{booktabs} % for professional tables

% hyperref makes hyperlinks in the resulting PDF.
% If your build breaks (sometimes temporarily if a hyperlink spans a page)
% please comment out the following usepackage line and replace
% \usepackage{icml2025} with \usepackage[nohyperref]{icml2025} above.
\usepackage{hyperref}
\usepackage{url}

% Attempt to make hyperref and algorithmic work together better:
\newcommand{\theHalgorithm}{\arabic{algorithm}}

% Use the following line for the initial blind version submitted for review:
% \usepackage{icml2025}

% If accepted, instead use the following line for the camera-ready submission:
\usepackage[accepted]{icml2025}

% For theorems and such
\usepackage{amsmath}
\usepackage{amssymb}
\usepackage{mathtools}
\usepackage{amsthm}
\usepackage{multirow}
\usepackage{mathabx}
\usepackage{makecell}
\usepackage{xcolor}
\usepackage{colortbl}
\usepackage{wrapfig}
\usepackage{array}

\newcommand\Tstrut{\rule{0pt}{2.3ex}}         % = `top' strut
\newcommand\Bstrut{\rule[-0.9ex]{0pt}{0pt}}   % = `bottom' strut
\newcommand{\yicong}[1]{{\textcolor{orange}{#1}}}

% acronym of our method
\newcommand{\ours}{HTH}

% if you use cleveref..
\usepackage[capitalize,noabbrev]{cleveref}

%%%%%%%%%%%%%%%%%%%%%%%%%%%%%%%%
% THEOREMS
%%%%%%%%%%%%%%%%%%%%%%%%%%%%%%%%
\theoremstyle{plain}
\newtheorem{theorem}{Theorem}[section]
\newtheorem{proposition}[theorem]{Proposition}
\newtheorem{lemma}[theorem]{Lemma}
\newtheorem{corollary}[theorem]{Corollary}
\theoremstyle{definition}
\newtheorem{definition}[theorem]{Definition}
\newtheorem{assumption}[theorem]{Assumption}
\theoremstyle{remark}
\newtheorem{remark}[theorem]{Remark}

% Todonotes is useful during development; simply uncomment the next line
%    and comment out the line below the next line to turn off comments
%\usepackage[disable,textsize=tiny]{todonotes}
\usepackage[textsize=tiny]{todonotes}


\icmltitlerunning{Pushing the Boundaries of State Space Models for Image and Video Generation}

\begin{document}

\twocolumn[
\icmltitle{Pushing the Boundaries of State Space Models for Image and Video Generation}

\icmlsetsymbol{equal}{*}

\begin{icmlauthorlist}
\icmlauthor{Yicong Hong$^{1}$}{}
\icmlauthor{Long Mai$^{1}$}{}
\icmlauthor{Yuan Yao$^{1,2}$}{}
\icmlauthor{Feng Liu$^{1}$}{} \\ \vspace{5pt}
\icmlauthor{$^{1}$Adobe Research, $^{2}$University of Rochester}{} \\ \vspace{5pt}
\icmlauthor{\tt\small{Project Page:}\;\;\tt\small\url{https://yiconghong.me/HTH}}{}
\end{icmlauthorlist}

% \icmlaffiliation{yyy}{Adobe Research}
% \icmlaffiliation{comp}{Company Name, Location, Country}

% \icmlcorrespondingauthor{Firstname1 Lastname1}{first1.last1@xxx.edu}
% \icmlcorrespondingauthor{Firstname2 Lastname2}{first2.last2@www.uk}

% You may provide any keywords that you
% find helpful for describing your paper; these are used to populate
% the "keywords" metadata in the PDF but will not be shown in the document
\icmlkeywords{Machine Learning, ICML}

\vskip 0.3in
]

% this must go after the closing bracket ] following \twocolumn[ ...

% This command actually creates the footnote in the first column
% listing the affiliations and the copyright notice.
% The command takes one argument, which is text to display at the start of the footnote.
% The \icmlEqualContribution command is standard text for equal contribution.
% Remove it (just {}) if you do not need this facility.

\printAffiliationsAndNotice{}  % leave blank if no need to mention equal contribution
% \printAffiliationsAndNotice{\icmlEqualContribution} % otherwise use the standard text.

\begin{abstract}

While Transformers have become the dominant architecture for visual generation, linear attention models, such as the state-space models (SSM), are increasingly recognized for their efficiency in processing long visual sequences. However, the essential efficiency of these models comes from formulating a limited recurrent state, enforcing causality among tokens that are prone to inconsistent modeling of N-dimensional visual data, leaving questions on their capacity to generate long non-causal sequences. In this paper, we explore the boundary of SSM on image and video generation by building the largest-scale diffusion SSM-Transformer hybrid model to date (5B parameters) based on the sub-quadratic bi-directional Hydra and self-attention, and generate up to 2K images and 360p 8 seconds (16 FPS) videos. 
Our results demonstrate that the model can produce faithful results aligned with complex text prompts and temporal consistent videos with high dynamics, suggesting the great potential of using SSMs for visual generation tasks.
\end{abstract}

\section{Introduction}
\label{section:introduction}

% redirection is unique and important in VR
Virtual Reality (VR) systems enable users to embody virtual avatars by mirroring their physical movements and aligning their perspective with virtual avatars' in real time. 
As the head-mounted displays (HMDs) block direct visual access to the physical world, users primarily rely on visual feedback from the virtual environment and integrate it with proprioceptive cues to control the avatar’s movements and interact within the VR space.
Since human perception is heavily influenced by visual input~\cite{gibson1933adaptation}, 
VR systems have the unique capability to control users' perception of the virtual environment and avatars by manipulating the visual information presented to them.
Leveraging this, various redirection techniques have been proposed to enable novel VR interactions, 
such as redirecting users' walking paths~\cite{razzaque2005redirected, suma2012impossible, steinicke2009estimation},
modifying reaching movements~\cite{gonzalez2022model, azmandian2016haptic, cheng2017sparse, feick2021visuo},
and conveying haptic information through visual feedback to create pseudo-haptic effects~\cite{samad2019pseudo, dominjon2005influence, lecuyer2009simulating}.
Such redirection techniques enable these interactions by manipulating the alignment between users' physical movements and their virtual avatar's actions.

% % what is hand/arm redirection, motivation of study arm-offset
% \change{\yj{i don't understand the purpose of this paragraph}
% These illusion-based techniques provide users with unique experiences in virtual environments that differ from the physical world yet maintain an immersive experience. 
% A key example is hand redirection, which shifts the virtual hand’s position away from the real hand as the user moves to enhance ergonomics during interaction~\cite{feuchtner2018ownershift, wentzel2020improving} and improve interaction performance~\cite{montano2017erg, poupyrev1996go}. 
% To increase the realism of virtual movements and strengthen the user’s sense of embodiment, hand redirection techniques often incorporate a complete virtual arm or full body alongside the redirected virtual hand, using inverse kinematics~\cite{hartfill2021analysis, ponton2024stretch} or adjustments to the virtual arm's movement as well~\cite{li2022modeling, feick2024impact}.
% }

% noticeability, motivation of predicting a probability, not a classification
However, these redirection techniques are most effective when the manipulation remains undetected~\cite{gonzalez2017model, li2022modeling}. 
If the redirection becomes too large, the user may not mitigate the conflict between the visual sensory input (redirected virtual movement) and their proprioception (actual physical movement), potentially leading to a loss of embodiment with the virtual avatar and making it difficult for the user to accurately control virtual movements to complete interaction tasks~\cite{li2022modeling, wentzel2020improving, feuchtner2018ownershift}. 
While proprioception is not absolute, users only have a general sense of their physical movements and the likelihood that they notice the redirection is probabilistic. 
This probability of detecting the redirection is referred to as \textbf{noticeability}~\cite{li2022modeling, zenner2024beyond, zenner2023detectability} and is typically estimated based on the frequency with which users detect the manipulation across multiple trials.

% version B
% Prior research has explored factors influencing the noticeability of redirected motion, including the redirection's magnitude~\cite{wentzel2020improving, poupyrev1996go}, direction~\cite{li2022modeling, feuchtner2018ownershift}, and the visual characteristics of the virtual avatar~\cite{ogawa2020effect, feick2024impact}.
% While these factors focus on the avatars, the surrounding virtual environment can also influence the users' behavior and in turn affect the noticeability of redirection.
% One such prominent external influence is through the visual channel - the users' visual attention is constantly distracted by complex visual effects and events in practical VR scenarios.
% Although some prior studies have explored how to leverage user blindness caused by visual distractions to redirect users' virtual hand~\cite{zenner2023detectability}, there remains a gap in understanding how to quantify the noticeability of redirection under visual distractions.

% visual stimuli and gaze behavior
Prior research has explored factors influencing the noticeability of redirected motion, including the redirection's magnitude~\cite{wentzel2020improving, poupyrev1996go}, direction~\cite{li2022modeling, feuchtner2018ownershift}, and the visual characteristics of the virtual avatar~\cite{ogawa2020effect, feick2024impact}.
While these factors focus on the avatars, the surrounding virtual environment can also influence the users' behavior and in turn affect the noticeability of redirection.
This, however, remains underexplored.
One such prominent external influence is through the visual channel - the users' visual attention is constantly distracted by complex visual effects and events in practical VR scenarios.
We thus want to investigate how \textbf{visual stimuli in the virtual environment} affect the noticeability of redirection.
With this, we hope to complement existing works that focus on avatars by incorporating environmental visual influences to enable more accurate control over the noticeability of redirected motions in practical VR scenarios.
% However, in realistic VR applications, the virtual environment often contains complex visual effects beyond the virtual avatar itself. 
% We argue that these visual effects can \textbf{distract users’ visual attention and thus affect the noticeability of redirection offsets}, while current research has yet taken into account.
% For instance, in a VR boxing scenario, a user’s visual attention is likely focused on their opponent rather than on their virtual body, leading to a lower noticeability of redirection offsets on their virtual movements. 
% Conversely, when reaching for an object in the center of their field of view, the user’s attention is more concentrated on the virtual hand’s movement and position to ensure successful interaction, resulting in a higher noticeability of offsets.

Since each visual event is a complex choreography of many underlying factors (type of visual effect, location, duration, etc.), it is extremely difficult to quantify or parameterize visual stimuli.
Furthermore, individuals respond differently to even the same visual events.
Prior neuroscience studies revealed that factors like age, gender, and personality can influence how quickly someone reacts to visual events~\cite{gillon2024responses, gale1997human}. 
Therefore, aiming to model visual stimuli in a way that is generalizable and applicable to different stimuli and users, we propose to use users' \textbf{gaze behavior} as an indicator of how they respond to visual stimuli.
In this paper, we used various gaze behaviors, including gaze location, saccades~\cite{krejtz2018eye}, fixations~\cite{perkhofer2019using}, and the Index of Pupil Activity (IPA)~\cite{duchowski2018index}.
These behaviors indicate both where users are looking and their cognitive activity, as looking at something does not necessarily mean they are attending to it.
Our goal is to investigate how these gaze behaviors stimulated by various visual stimuli relate to the noticeability of redirection.
With this, we contribute a model that allows designers and content creators to adjust the redirection in real-time responding to dynamic visual events in VR.

To achieve this, we conducted user studies to collect users' noticeability of redirection under various visual stimuli.
To simulate realistic VR scenarios, we adopted a dual-task design in which the participants performed redirected movements while monitoring the visual stimuli.
Specifically, participants' primary task was to report if they noticed an offset between the avatar's movement and their own, while their secondary task was to monitor and report the visual stimuli.
As realistic virtual environments often contain complex visual effects, we started with simple and controlled visual stimulus to manage the influencing factors.

% first user study, confirmation study
% collect data under no visual stimuli, different basic visual stimuli
We first conducted a confirmation study (N=16) to test whether applying visual stimuli (opacity-based) actually affects their noticeability of redirection. 
The results showed that participants were significantly less likely to detect the redirection when visual stimuli was presented $(F_{(1,15)}=5.90,~p=0.03)$.
Furthermore, by analyzing the collected gaze data, results revealed a correlation between the proposed gaze behaviors and the noticeability results $(r=-0.43)$, confirming that the gaze behaviors could be leveraged to compute the noticeability.

% data collection study
We then conducted a data collection study to obtain more accurate noticeability results through repeated measurements to better model the relationship between visual stimuli-triggered gaze behaviors and noticeability of redirection.
With the collected data, we analyzed various numerical features from the gaze behaviors to identify the most effective ones. 
We tested combinations of these features to determine the most effective one for predicting noticeability under visual stimuli.
Using the selected features, our regression model achieved a mean squared error (MSE) of 0.011 through leave-one-user-out cross-validation. 
Furthermore, we developed both a binary and a three-class classification model to categorize noticeability, which achieved an accuracy of 91.74\% and 85.62\%, respectively.

% evaluation study
To evaluate the generalizability of the regression model, we conducted an evaluation study (N=24) to test whether the model could accurately predict noticeability with new visual stimuli (color- and scale-based animations).
Specifically, we evaluated whether the model's predictions aligned with participants' responses under these unseen stimuli.
The results showed that our model accurately estimated the noticeability, achieving mean squared errors (MSE) of 0.014 and 0.012 for the color- and scale-based visual stimili, respectively, compared to participants' responses.
Since the tested visual stimuli data were not included in the training, the results suggested that the extracted gaze behavior features capture a generalizable pattern and can effectively indicate the corresponding impact on the noticeability of redirection.

% application
Based on our model, we implemented an adaptive redirection technique and demonstrated it through two applications: adaptive VR action game and opportunistic rendering.
We conducted a proof-of-concept user study (N=8) to compare our adaptive redirection technique with a static redirection, evaluating the usability and benefits of our adaptive redirection technique.
The results indicated that participants experienced less physical demand and stronger sense of embodiment and agency when using the adaptive redirection technique. 
These results demonstrated the effectiveness and usability of our model.

In summary, we make the following contributions.
% 
\begin{itemize}
    \item 
    We propose to use users' gaze behavior as a medium to quantify how visual stimuli influences the noticebility of redirection. 
    Through two user studies, we confirm that visual stimuli significantly influences noticeability and identify key gaze behavior features that are closely related to this impact.
    \item 
    We build a regression model that takes the user's gaze behavioral data as input, then computes the noticeability of redirection.
    Through an evaluation study, we verify that our model can estimate the noticeability with new participants under unseen visual stimuli.
    These findings suggest that the extracted gaze behavior features effectively capture the influence of visual stimuli on noticeability and can generalize across different users and visual stimuli.
    \item 
    We develop an adaptive redirection technique based on our regression model and implement two applications with it.
    With a proof-of-concept study, we demonstrate the effectiveness and potential usability of our regression model on real-world use cases.

\end{itemize}

% \delete{
% Virtual Reality (VR) allows the user to embody a virtual avatar by mirroring their physical movements through the avatar.
% As the user's visual access to the physical world is blocked in tasks involving motion control, they heavily rely on the visual representation of the avatar's motions to guide their proprioception.
% Similar to real-world experiences, the user is able to resolve conflicts between different sensory inputs (e.g., vision and motor control) through multisensory integration, which is essential for mitigating the sensory noise that commonly arises.
% However, it also enables unique manipulations in VR, as the system can intentionally modify the avatar's movements in relation to the user's motions to achieve specific functional outcomes,
% for example, 
% % the manipulations on the avatar's movements can 
% enabling novel interaction techniques of redirected walking~\cite{razzaque2005redirected}, redirected reaching~\cite{gonzalez2022model}, and pseudo haptics~\cite{samad2019pseudo}.
% With small adjustments to the avatar's movements, the user can maintain their sense of embodiment, due to their ability to resolve the perceptual differences.
% % However, a large mismatch between the user and avatar's movements can result in the user losing their sense of embodiment, due to an inability to resolve the perceptual differences.
% }

% \delete{
% However, multisensory integration can break when the manipulation is so intense that the user is aware of the existence of the motion offset and no longer maintains the sense of embodiment.
% Prior research studied the intensity threshold of the offset applied on the avatar's hand, beyond which the embodiment will break~\cite{li2022modeling}. 
% Studies also investigated the user's sensitivity to the offsets over time~\cite{kohm2022sensitivity}.
% Based on the findings, we argue that one crucial factor that affects to what extent the user notices the offset (i.e., \textit{noticeability}) that remains under-explored is whether the user directs their visual attention towards or away from the virtual avatar.
% Related work (e.g., Mise-unseen~\cite{marwecki2019mise}) has showcased applications where adjustments in the environment can be made in an unnoticeable manner when they happen in the area out of the user's visual field.
% We hypothesize that directing the user's visual attention away from the avatar's body, while still partially keeping the avatar within the user's field-of-view, can reduce the noticeability of the offset.
% Therefore, we conduct two user studies and implement a regression model to systematically investigate this effect.
% }

% \delete{
% In the first user study (N = 16), we test whether drawing the user's visual attention away from their body impacts the possibility of them noticing an offset that we apply to their arm motion in VR.
% We adopt a dual-task design to enable the alteration of the user's visual attention and a yes/no paradigm to measure the noticeability of motion offset. 
% The primary task for the user is to perform an arm motion and report when they perceive an offset between the avatar's virtual arm and their real arm.
% In the secondary task, we randomly render a visual animation of a ball turning from transparent to red and becoming transparent again and ask them to monitor and report when it appears.
% We control the strength of the visual stimuli by changing the duration and location of the animation.
% % By changing the time duration and location of the visual animation, we control the strengths of attraction to the users.
% As a result, we found significant differences in the noticeability of the offsets $(F_{(1,15)}=5.90,~p=0.03)$ between conditions with and without visual stimuli.
% Based on further analysis, we also identified the behavioral patterns of the user's gaze (including pupil dilation, fixations, and saccades) to be correlated with the noticeability results $(r=-0.43)$ and they may potentially serve as indicators of noticeability.
% }

% \delete{
% To further investigate how visual attention influences the noticeability, we conduct a data collection study (N = 12) and build a regression model based on the data.
% The regression model is able to calculate the noticeability of the offset applied on the user's arm under various visual stimuli based on their gaze behaviors.
% Our leave-one-out cross-validation results show that the proposed method was able to achieve a mean-squared error (MSE) of 0.012 in the probability regression task.
% }

% \delete{
% To verify the feasibility and extendability of the regression model, we conduct an evaluation study where we test new visual animations based on adjustments on scale and color and invite 24 new participants to attend the study.
% Results show that the proposed method can accurately estimate the noticeability with an MSE of 0.014 and 0.012 in the conditions of the color- and scale-based visual effects.
% Since these animations were not included in the dataset that the regression model was built on, the study demonstrates that the gaze behavioral features we extracted from the data capture a generalizable pattern of the user's visual attention and can indicate the corresponding impact on the noticeability of the offset.
% }

% \delete{
% Finally, we demonstrate applications that can benefit from the noticeability prediction model, including adaptive motion offsets and opportunistic rendering, considering the user's visual attention. 
% We conclude with discussions of our work's limitations and future research directions.
% }

% \delete{
% In summary, we make the following contributions.
% }
% % 
% \begin{itemize}
%     \item 
%     \delete{
%     We quantify the effects of the user's visual attention directed away by stimuli on their noticeability of an offset applied to the avatar's arm motion with respect to the user's physical arm. 
%     Through two user studies, we identified gaze behavioral features that are indicative of the changes in noticeability.
%     }
%     \item 
%     \delete{We build a regression model that takes the user's gaze behavioral data and the offset applied to the arm motion as input, then computes the probability of the user noticing the offset.
%     Through an evaluation study, we verified that the model needs no information about the source attracting the user's visual attention and can be generalizable in different scenarios.
%     }
%     \item 
%     \delete{We demonstrate two applications that potentially benefit from the regression model, including adaptive motion offsets and opportunistic rendering.
%     }

% \end{itemize}

\begin{comment}
However, users will lose the sense of embodiment to the virtual avatars if they notice the offset between the virtual and physical movements.
To address this, researchers have been exploring the noticing threshold of offsets with various magnitudes and proposing various redirection techniques that maintain the sense of embodiment~\cite{}.

However, when users embody virtual avatars to explore virtual environments, they encounter various visual effects and content that can attract their attention~\cite{}.
During this, the user may notice an offset when he observes the virtual movement carefully while ignoring it when the virtual contents attract his attention from the movements.
Therefore, static offset thresholds are not appropriate in dynamic scenarios.

Past research has proposed dynamic mapping techniques that adapted to users' state, such as hand moving speed~\cite{frees2007prism} or ergonomically comfortable poses~\cite{montano2017erg}, but not considering the influence of virtual content.
More specifically, PRISM~\cite{frees2007prism} proposed adjusting the C/D ratio with a non-linear mapping according to users' hand moving speed, but it might not be optimal for various virtual scenarios.
While Erg-O~\cite{montano2017erg} redirected users' virtual hands according to the virtual target's relative position to reduce physical fatigue, neglecting the change of virtual environments. 

Therefore, how to design redirection techniques in various scenarios with different visual attractions remains unknown.
To address this, we investigate how visual attention affects the noticing probability of movement offsets.
Based on our experiments, we implement a computational model that automatically computes the noticing probability of offsets under certain visual attractions.
VR application designers and developers can easily leverage our model to design redirection techniques maintaining the sense of embodiment adapt to the user's visual attention.
We implement a dynamic redirection technique with our model and demonstrate that it effectively reduces the target reaching time without reducing the sense of embodiment compared to static redirection techniques.

% Need to be refined
This paper offers the following contributions.
\begin{itemize}
    \item We investigate how visual attractions affect the noticing probability of redirection offsets.
    \item We construct a computational model to predict the noticing probability of an offset with a given visual background.
    \item We implement a dynamic redirection technique adapting to the visual background. We evaluate the technique and develop three applications to demonstrate the benefits. 
\end{itemize}



First, we conducted a controlled experiment to understand how users perceived the movement offset while subjected to various distractions.
Since hand redirection is one of the most frequently used redirections in VR interactions, we focused on the dynamic arm movements and manually added angular offsets to the' elbow joint~\cite{li2022modeling, gonzalez2022model, zenner2019estimating}. 
We employed flashing spheres in the user's field of view as distractions to attract users' visual attention.
Participants were instructed to report the appearing location of the spheres while simultaneously performing the arm movements and reporting if they perceived an offset during the movement. 
(\zhipeng{Add the results of data collection. Analyze the influence of the distance between the gaze map and the offset.}
We measured the visual attraction's magnitude with the gaze distribution on it.
Results showed that stronger distractions made it harder for users to notice the offset.)
\zhipeng{Need to rewrite. Not sure to use gaze distribution or a metric obtained from the visual content.}
Secondly, we constructed a computational model to predict the noticing probability of offsets with given visual content.
We analyzed the data from the user studies to measure the influence of visual attractions on the noticing probability of offsets.
We built a statistical model to predict the offset's noticing probability with a given visual content.
Based on the model, we implement a dynamic redirection technique to adjust the redirection offset adapted to the user's current field of view.
We evaluated the technique in a target selection task compared to no hand redirection and static hand redirection.
\zhipeng{Add the results of the evaluation.}
Results showed that the dynamic hand redirection technique significantly reduced the target selection time with similar accuracy and a comparable sense of embodiment.
Finally, we implemented three applications to demonstrate the potential benefits of the visual attention adapted dynamic redirection technique.
\end{comment}

% This one modifies arm length, not redirection
% \citeauthor{mcintosh2020iteratively} proposed an adaptation method to iteratively change the virtual avatar arm's length based on the primary tasks' performance~\cite{mcintosh2020iteratively}.



% \zhipeng{TO ADD: what is redirection}
% Redirection enables novel interactions in Virtual Reality, including redirected walking, haptic redirection, and pseudo haptics by introducing an offset to users' movement.
% \zhipeng{TO ADD: extend this sentence}
% The price of this is that users' immersiveness and embodiment in VR can be compromised when they notice the offset and perceive the virtual movement not as theirs~\cite{}.
% \zhipeng{TO ADD: extend this sentence, elaborate how the virtual environment attracts users' attention}
% Meanwhile, the visual content in the virtual environment is abundant and consistently captures users' attention, making it harder to notice the offset~\cite{}.
% While previous studies explored the noticing threshold of the offsets and optimized the redirection techniques to maintain the sense of embodiment~\cite{}, the influence of visual content on the probability of perceiving offsets remains unknown.  
% Therefore, we propose to investigate how users perceive the redirection offset when they are facing various visual attractions.


% We conducted a user study to understand how users notice the shift with visual attractions.
% We used a color-changing ball to attract the user's attention while instructing users to perform different poses with their arms and observe it meanwhile.
% \zhipeng{(Which one should be the primary task? Observe the ball should be the primary one, but if the primary task is too simple, users might allocate more attention on the secondary task and this makes the secondary task primary.)}
% \zhipeng{(We need a good and reasonable dual-task design in which users care about both their pose and the visual content, at least in the evaluation study. And we need to be able to control the visual content's magnitude and saliency maybe?)}
% We controlled the shift magnitude and direction, the user's pose, the ball's size, and the color range.
% We set the ball's color-changing interval as the independent factor.
% We collect the user's response to each shift and the color-changing times.
% Based on the collected data, we constructed a statistical model to describe the influence of visual attraction on the noticing probability.
% \zhipeng{(Are we actually controlling the attention allocation? How do we measure the attracting effect? We need uniform metrics, otherwise it is also hard for others to use our knowledge.)}
% \zhipeng{(Try to use eye gaze? The eye gaze distribution in the last five seconds to decide the attention allocation? Basically constructing a model with eye gaze distribution and noticing probability. But the user's head is moving, so the eye gaze distribution is not aligned well with the current view.)}

% \zhipeng{Saliency and EMD}
% \zhipeng{Gaze is more than just a point: Rethinking visual attention
% analysis using peripheral vision-based gaze mapping}

% Evaluation study(ideal case): based on the visual content, adjusting the redirection magnitude dynamically.

% \zhipeng{(The risk is our model's effect is trivial.)}

% Applications:
% Playing Lego while watching demo videos, we can accelerate the reaching process of bricks, and forbid the redirection during the manipulation.

% Beat saber again: but not make a lot of sense? Difficult game has complicated visual effects, while allows larger shift, but do not need large shift with high difficulty



% \section{Related Work}
\label{sec:related}


% \subsection{Radiance Fields for Novel View Synthesis}
\noindent {\bf NeRF.}
Neural Radiance Fields (NeRF)\citep{mildenhall2020nerf} revolutionized novel view synthesis via differentiable volume rendering\citep{tulsiani2017mvsupervision, henzler2019platonicgan} and positional encoding~\citep{vaswani2017attentionisallyouneed, gehring2017convolutional}. NeRF models improved in efficiency~\citep{liu2020neural, Garbin_2021_ICCV, chen2024improving}, rendering quality~\citep{mipnerf, zhang2020nerf++,meuleman2023progressively}, handling dynamic scenes~\cite{liu2023robust}, and data efficiency~\citep{pixel.nerf, ibrnet, lin2024frugalnerf,su2024boostmvsnerfs}. Despite excelling at view synthesis, NeRF’s implicit representation complicates scene editing. Recent work on object manipulation~\citep{yang2021learning}, stylization~\citep{wang2023nerf, haque2023instruct}, and inpainting~\citep{nerf.in, spinnerf, mirzaei2023reference} struggles with 3D consistency and structural priors, especially in unbounded scenes.
% Neural Radiance Fields (NeRF)~\citep{original.nerf} revolutionized novel view synthesis, enabling photorealistic scene reconstruction via differentiable volume rendering~\citep{tulsiani2017mvsupervision, henzler2019platonicgan} and positional encoding~\citep{vaswani2017attentionisallyouneed, gehring2017convolutional}. NeRF-based models have since improved in efficiency~\citep{liu2020neural, Garbin_2021_ICCV, chen2024improving}, rendering quality~\citep{mipnerf, zhang2020nerf++,su2024boostmvsnerfs}, and data efficiency~\citep{pixel.nerf, ibrnet, lin2024frugalnerf}. While NeRF excels in view synthesis, its implicit volumetric representation complicates scene editing. Recent works on object manipulation~\citep{yang2021learning}, stylization~\citep{wang2023nerf, haque2023instruct}, and inpainting~\citep{nerf.in, spinnerf, mirzaei2023reference} face challenges in 3D inpainting in unbounded environments, as NeRF struggles with 3D consistency and leveraging explicit structural priors.


\vspace{3pt}
\noindent {\bf 3D Gaussian Splatting.}
3D Gaussian Splatting (3DGS)~\cite{kerbl20233d} efficiently represents scenes with explicit 3D Gaussians, enabling faster rendering, easier training, and flexible editing\citep{chen2024gaussianeditor}. Recent extensions like Scaffold-GS~\citep{scaffoldgs} enhance efficiency with dynamic anchors, while 2DGS~\citep{huang20242d} refines multi-view geometry. 3DGS has also expanded to dynamic scenes~\citep{yang2024deformable, luiten2023dynamic, Wu_2024_CVPR,fan2025spectromotion} and semantic representations~\citep{ye2023gaussian, qin2023langsplat}, supporting advanced editing and novel view synthesis~\citep{qiu-2024-featuresplatting, huang20242d}. Gaussian-based methods thus offer strong potential for explicit 3D inpainting.
% 3D Gaussian Splatting (3DGS)~\citep{kerbl20233d} is an efficient alternative to NeRF, representing scenes with explicit 3D Gaussians for faster rendering, easier training, and more flexible scene editing~\citep{chen2024gaussianeditor}. Recent extensions include Scaffold-GS~\citep{scaffoldgs}, which improves rendering efficiency with dynamic anchor points, and 2DGS~\citep{Huang2DGS2024}, which refines multi-view reconstructions for view-consistent geometry. 3DGS has also been extended to dynamic environments~\citep{yang2024deformable, luiten2023dynamic, Wu_2024_CVPR} and semantic-aware representations~\citep{ye2023gaussian, qin2023langsplat}, advancing scene manipulation and novel view synthesis~\citep{qiu-2024-featuresplatting, huang20242d}. These advancements highlight the potential of Gaussian-based representations for explicit scene editing, making them well-suited for 3D inpainting tasks.

% \subsection{2D Image Inpainting}
% \paragraph{Traditional methods.}
% Image inpainting has evolved from early PDE-based techniques \citep{bertalmio2000image} to exemplar-based methods \citep{criminisi2004region}. Texture synthesis \citep{efros1999texture} and patch-based approaches like PatchMatch \citep{barnes2009patchmatch} further advanced the field. Despite limitations with large missing regions and complex textures \citep{jam2021comprehensive,liu2018image}, these methods established principles now incorporated into learning-based approaches \citep{liu2018image,yu2019free}. Their computational efficiency remains valuable in resource-constrained scenarios \citep{jam2021comprehensive}.

% \vspace{-3mm}
% \paragraph{Deep learning-based methods.}
% Deep learning has revolutionized image inpainting, with CNNs like Context Encoders \citep{feature.learning.by.inpainting} pioneering the field. GANs \citep{gan} and models like DeepFillv2 \citep{yu2019free} further improved results. Large Mask Inpainting (LaMa) \citep{lama} addressed large missing regions. Recently, diffusion models\citep{NEURIPS2020_4c5bcfec}, particularly Stable Diffusion\citep{rombach2022high}, have demonstrated remarkable inpainting capabilities, leveraging complex data distributions\citep{dhariwal2021diffusion}. Beyond text-to-image generation, diffusion models are commonly used for image-to-image tasks, including image editing and inpainting. SDEdit\citep{meng2022sdedit} leverages diffusion models for semantic editing by injecting Gaussian noise into input images and performing iterative denoising, ensuring structural coherence while modifying visual content. To further improve image manipulation fidelity, Noise Inversion techniques such as DDIM Inversion~\citep{song2021denoising} enable precise latent code inference through deterministic reverse diffusion sampling. This approach retains finer details of the original image, making it particularly effective for manipulating real images within diffusion-based generative models. In the context of inpainting, models like SDXL-Inpainting have been developed by fine-tuning diffusion models specifically for inpainting tasks. While these methods have significantly improved image inpainting quality, Stable Diffusion-based inpainting often introduces scene-inconsistent artifacts within the inpainted regions. This challenge becomes even more pronounced when leveraging 2D diffusion priors for 3D inpainting, as it can lead to multi-view inconsistencies—a major limitation for 3D scene reconstruction~\citep{li2023diffusion}. The success of diffusion-based inpainting has inspired extensions to 3D inpainting tasks\citep{nerf.in, inpaint3d}, though adapting 2D approaches to 3D presents additional challenges, such as geometry misalignment, depth inconsistencies, and occlusion handling\citep{mirzaei2023reference}.



% \vspace{-3mm}
% \paragraph{Reference-based methods.}
% Reference-based inpainting methods \citep{zhao2022geofill} enhance traditional inpainting by incorporating visual context from reference images, improving content accuracy and consistency. LeftRefill \citep{tang2023realfill} uses a two-stage architecture with feature matching and refinement networks, enabling inpainting from different viewpoints based on reference information \citep{zhao2022geofill}. While these methods show promise in various applications \citep{jam2021comprehensive}, challenges remain in seamless integration and reference selection \citep{li2023diffusion}, particularly when views diverge significantly from the reference. The success of these approaches has also inspired 3D inpainting extensions \citep{nerf.in, inpaint3d}, though adapting to 3D introduces additional complexities \citep{mirzaei2023reference}.

% \vspace{-2mm}

% \subsection{Image Inpainting}
\vspace{3pt}
\noindent {\bf Traditional and learning-based image inpainting.}
Early image inpainting techniques, including PDE-based~\citep{bertalmio2000image}, exemplar-based~\citep{criminisi2004region}, and PatchMatch~\citep{barnes2009patchmatch}, were effective for small regions but struggled with complex textures and large gaps~\citep{jam2021comprehensive, liu2018image}. Deep learning advanced the field significantly, starting with Context Encoders~\citep{feature.learning.by.inpainting} and GAN-based methods like DeepFill~\citep{generative.inpainting, yu2019free}, improving content synthesis and coherence. Recent models such as LaMa~\citep{lama} use Fourier convolutional networks to address large masks. Diffusion models~\citep{NEURIPS2020_4c5bcfec}, notably Stable Diffusion~\citep{rombach2022high}, introduced iterative refinement capabilities, providing more flexible and structurally consistent inpainting compared to GANs~\citep{dhariwal2021diffusion}.
% Early image inpainting techniques, including PDE-based~\citep{bertalmio2000image}, exemplar-based~\citep{criminisi2004region}, and patch-based methods like PatchMatch~\citep{barnes2009patchmatch}, were effective for small missing regions but struggled with complex textures and large gaps~\citep{jam2021comprehensive, liu2018image}. Deep learning brought significant advancements, starting with CNN-based models like Context Encoders~\citep{feature.learning.by.inpainting} and GANs such as DeepFill~\citep{generative.inpainting, yu2019free}, which improved content synthesis and structural coherence. More recent models like Large Mask Inpainting (LaMa)~\citep{lama} further enhanced quality by using Fourier convolutional networks for large masked regions. The rise of diffusion models~\citep{NEURIPS2020_4c5bcfec}, notably Stable Diffusion~\citep{rombach2022high}, introduced powerful text-to-image and image-to-image capabilities, enabling more flexible and structurally consistent inpainting by iteratively refining missing regions, unlike GANs~\citep{dhariwal2021diffusion}.

% \vspace{-4mm}

\vspace{3pt}
\noindent {\bf Diffusion models for image editing and inpainting.}
Beyond direct inpainting, diffusion models are widely used for image editing. SDEdit~\citep{meng2022sdedit} injects Gaussian noise and iteratively denoises, enabling semantic edits while preserving global structure. Noise inversion techniques~\cite{mokady2022null, miyake2024negativepromptinversionfastimage}, such as DDIM Inversion~\citep{song2020denoising}, further improve editing fidelity by enabling precise latent inference through deterministic reverse diffusion.
Inpainting-specific diffusion models like SDXL-Inpainting~\cite{podell2023sdxlimprovinglatentdiffusion} enhance image reconstruction by fine-tuning Stable Diffusion. Reference-based methods~\cite{tang2023realfill}, such as LeftRefill~\citep{cao2024leftrefill}, use diffusion models for reference-guided synthesis but struggle in regions distant from reference views.
Despite advancements, Stable Diffusion-based inpainting~\cite{inpaint3d} still suffers from inconsistent artifacts in scene-dependent contexts, causing multi-view inconsistencies problematic for 3D scenes~\citep{li2023diffusion}. This motivates our use of SDEdit and DDIM Inversion to preserve structural information and ensure multi-view coherence.
% Beyond direct inpainting, diffusion models are widely used for image editing. SDEdit~\citep{meng2022sdedit} injects controlled Gaussian noise followed by iterative denoising, enabling semantic modifications while preserving global structure. To improve editing fidelity, Noise Inversion techniques~\cite{, mokady2022null, miyake2024negativepromptinversionfastimage} like DDIM Inversion~\citep{song2020denoising} allow precise latent code inference through deterministic reverse diffusion sampling. By inverting an image to a specific noise level and denoising it back, Noise-Inversion provides finer control over content preservation, making it highly effective for inpainting real images while minimizing distortion during denoising.
% % 
% Inpainting-specific diffusion models like SDXL-Inpainting~\cite{podell2023sdxlimprovinglatentdiffusion} enhance the process by fine-tuning Stable Diffusion models for image reconstruction. Reference-based method~\cite{tang2023realfill} such as LeftRefill~\citep{cao2024leftrefill} uses pre-trained diffusion models for reference-guided synthesis, stitching reference and target views to enable contextual inpainting, view synthesis, and image completion via task-specific prompt tuning. However, LeftRefill struggles in regions far from the reference view, where alignment becomes less reliable.
% % 
% Despite these advancements, Stable Diffusion-based inpainting~\cite{inpaint3d} often produces inconsistent artifacts, particularly in scene-dependent contexts. When applied to 3D inpainting, these artifacts lead to multi-view inconsistencies, a critical limitation for scene reconstruction and object removal~\citep{li2023diffusion}. This motivates our use of SDEdit and DDIM Inversion for 3D inpainting, ensuring that denoising preserves critical structural information while maintaining coherence across viewpoints.  

% Although diffusion-based inpainting has inspired 3D inpainting extensions\citep{nerf.in, inpaint3d}, adapting 2D methods to 3D introduces challenges like geometry misalignment, depth inconsistencies, and occlusion handling.



% \subsection{3D Scene Inpainting}
% \paragraph{Methods without multi-view background knowledge.}
% As 3D scene representation and reconstruction techniques have advanced, the need for 3D inpainting methods has grown. Early approaches to 3D scene inpainting often relied on single-view or limited-view information, attempting to extend 2D inpainting concepts into the 3D domain without leveraging extensive multi-view knowledge.
% One category of methods focuses on direct 3D shape completion. These approaches typically operate on point clouds or voxel representations. For instance, PCN (Point Completion Network) introduced by \citet{yuan2018pcn} uses an encoder-decoder architecture to complete partial point clouds. While effective for object-level completion, these methods often struggle with complex, large-scale scene inpainting.
% Another approach involves using 2.5D representations, where depth information is incorporated alongside RGB data. Depth-aware inpainting methods, such as the work by \citet{3d.photography}, extend 2D inpainting techniques by considering depth as an additional channel. These methods can produce more geometrically consistent results but are limited by their reliance on a single viewpoint.
% Some researchers have explored the use of generative models for 3D inpainting. 3D-GAN, proposed by \citet{wu2016learning}, generates 3D shapes from a probabilistic space, which can be adapted for inpainting tasks. However, these methods often struggle with fine details and scene-level consistency.
% In the context of neural rendering, early attempts at NeRF editing and inpainting also fall into this category. Methods like EditNeRF by \citet{liu2021editing} allow for object-level editing in NeRF scenes but are limited in their ability to handle large-scale scene modifications or inpainting of complex structures.
% Standalone NeRF inpainting methods, such as NeRF-In by \citet{nerf.in}, attempt to inpaint 3D scenes represented as Neural Radiance Fields. These approaches often rely on 2D inpainting results as supervision, projecting them back into the 3D space. While they can produce plausible results for small edits, they struggle with view consistency and large-scale modifications.
% A common limitation of these single-view or limited-view methods is their inability to fully leverage the 3D structure of the scene. They often produce results that are inconsistent across different viewpoints or fail to capture the true geometry of occluded regions \citep{spinnerf}. Additionally, these methods may struggle with understanding the global context of the scene, leading to inpainted content that doesn't align well with the overall scene structure \citep{wang2023inpaintnerf360}.
% Despite these limitations, these methods have laid important groundwork for 3D scene inpainting. They have highlighted the challenges specific to 3D inpainting, such as maintaining geometric consistency and handling occlusions, which have informed the development of more advanced, multi-view aware techniques \citep{mirzaei2023reference}.
% Existing 3D inpainting approaches~\cite{weder2022removing, spinner, nerfin} extended 2D concepts to 3D without extensive multi-view knowledge. These include direct 3D shape completion methods like PCN \citep{yuan2018pcn}, 2.5D representations \citep{3d.photography}, and generative models like 3D-GAN \citep{wu2016learning}. In the field of neural rendering, EditNeRF \citep{liu2021editing} and NeRF-In \citep{nerf.in} pioneered NeRF editing and inpainting. These methods often struggle with view consistency \citep{spinnerf} and global context \citep{wang2023inpaintnerf360}. Despite limitations, they laid groundwork for more advanced, multi-view aware techniques \citep{mirzaei2023reference}. 

% Existing 3D inpainting approaches用在NeRf上~\cite{weder2022removing, spinnerf, nerfin} 因為nerf implicit representation的特性,通常都是leverage 2d inpainter to 3d, 像spinnerf~\cite{spinnerf}用一個lpips loss來減緩muli-view inpainting 的inconcsitency. 而reference-based method, 為了要進一步解決multi-view inconsitency的問題,他們提出想要只使用少數的reference image來代表要inpaint的區域。然而他們通常只能能render的novel view角度很小, 局限在forward facing scene, 很能利用到unbounded 360 scene上。InNeRF360~\cite{wang2023inpaintnerf360}雖然可以使用在360場景,使用Hallucinating Density Removal來清除inconsistency造成的flaoters, 但卻一樣是利用將object先inpaint掉在拿去train nerf的方式, 沒辦法利用原有場景資訊。

% 而得天GaussianSplatting explict的特性, GaussianGrouping可以將semantic 資訊加到每一顆Gaussian上, InFusion [citation] approaches 3D Gaussian inpainting by leveraging depth completion and progressive reference view synthesis. While achieving efficient results, the method's limitations include manual view selection requirements and potential inaccuracies in depth completion for complex geometries, 而且他們的depth completion model需奧finetuning. GScream leverages 3D Gaussian Splatting for object removal by integrating monocular depth guidance and cross-attention feature propagation between visible and in-painted regions to achieve consistent geometry and textures. However, it's hard to extend to unbounded 360 scene 因為他一樣是reference-based method. 而我們的方法...


\vspace{3pt}
\noindent {\bf 3D scene inpainting.}
Existing 3D inpainting methods for NeRF~\cite{weder2022removing, spinnerf, shen2023nerfin, yin2023or} typically adapt 2D models to NeRF’s implicit representation. For instance, SPIn-NeRF~\cite{spinnerf} employs perceptual loss to improve multi-view consistency. Reference-based methods~\cite{mirzaei2023reference, mirzaei2024reffusionreferenceadapteddiffusion, wang2024gscream} enhance consistency using reference images but remain limited to small-angle view rendering, restricting their use in 360° scenes. NeRFiller~\cite{weber2023nerfiller} iteratively refines consistency with grid prior but struggles with fine-grained textures due to image downsampling. InNeRF360~\cite{wang2023inpaintnerf360} handles 360° scenes via density hallucination but has limited scene utilization.
% 
Gaussian Splatting-based methods like Gaussian Grouping~\cite{ye2023gaussian} inject semantic information, while InFusion~\cite{liu2024infusion} employs depth completion but requires manual view selection. GScream~\cite{scaffoldgs} integrates Scaffold-GS but faces difficulties in unbounded 360° scenes. Our method addresses these issues by enhancing multi-view consistency and depth-aware inpainting in 360° scenarios using Gaussian Splatting.
% Existing 3D inpainting approaches for NeRF~\cite{weder2022removing, spinnerf, shen2023nerfin, yin2023or} often extend 2D inpainting models into 3D due to NeRF’s implicit representation. For example, SPIn-NeRF~\cite{spinnerf} uses perceptual loss to reduce multi-view inpainting inconsistencies. Reference-based methods~\cite{mirzaei2023reference, mirzaei2024reffusionreferenceadapteddiffusion, wang2024gscream} aim to further reduce these inconsistencies by using a few reference images to represent the inpainting area. However, these methods are generally restricted to rendering novel views from small angles, making them less suitable for unbounded 360° environments. NeRFiller~\cite{weber2023nerfiller} improves multi-view consistency with Grid Prior, extends it to Joint Multi-View Inpainting, and refines missing regions iteratively via Dataset Update, ensuring 3D structural coherence without reference images or object masks. Yet, its reliance on image downsampling limits high-frequency detail reconstruction, reducing effectiveness for fine-grained textures. InNeRF360~\cite{wang2023inpaintnerf360} adapts to 360° scenes using Hallucinating Density Removal to address view inconsistencies but remains limited by its object inpainting approach before NeRF training, restricting full scene utilization.
% % 
% Gaussian Splatting enables precise inpainting, as seen in Gaussian Grouping~\cite{ye2023gaussian}, which injects semantic information into each Gaussian. InFusion~\cite{liu2024infusion} enhances 3D Gaussian inpainting with depth completion and progressive reference synthesis but is limited by manual view selection and fine-tuning. GScream integrates Scaffold-GS~\cite{scaffoldgs} for object removal, using monocular depth and cross-attention for consistency but struggles with 360° unbounded scenes due to fixed reference views. Our method addresses these challenges by improving multi-view consistency in 360° environments, leveraging Gaussian Splatting for explicit scene manipulation and depth-aware inpainting.




% Previous work such as SPIn-NeRF that integrates 2d inpainting model with perceptual loss and depth inpainting guidance to reconstruct Inpainted Neural Radiance Fields (NeRF). OR-NeRF Removing Objects From NeRF use confidence-based view selection automatically removes inconsistent views from the optimization preventing unwanted artefacts in the final result.MVIP-NeRF utilizes a multi-view approach to perform 3D inpainting on NeRF scenes by employing diffusion priors, where it jointly optimizes RGB and normal map completion through an iterative Score Distillation Sampling (SDS) process, ensuring consistent appearance and geometry alignment across multiple views while leveraging a multi-view scoring mechanism to distill generative priors from different perspectives. 然而這些方法並沒有透過leverage 已有場景資訊來進行inpainting. Reference-guided controllable inpainting, as presented by Mirzaei et al. (2023), leverages reference images and view-dependent effects to guide the inpainting process in 3D scenes, enabling consistent and visually coherent completions across multiple perspectives while handling challenges like disocclusions and geometric consistency. GScream introduces a robust framework for object removal in 3D scenes by optimizing Gaussian primitives' positions for geometric consistency and utilizing a cross-attention feature propagation mechanism to enhance texture coherence, effectively restoring both geometry and texture across visible and occluded areas. 
% Reference-guided inpainting in NeRF  

% 先nerf方法, 指出雖然..., 他們1需要非常精確的object mask來remov場景的objec, 但又inplcit-method, 所以沒辦法很好的透出場景資訊
% 在講道






% \vspace{-3mm}
% \paragraph{Methods leveraging multi-view information.}
% As the limitations of single-view 3D inpainting methods became apparent, researchers began to explore approaches that leverage multi-view information. These methods aim to produce more consistent and geometrically accurate results by utilizing the rich information available from multiple viewpoints of a scene.
% One of the pioneering works in this direction is SPIn-NeRF by \citet{spinnerf}. This method combines Neural Radiance Fields (NeRF) with multi-view image inpainting to remove objects from 3D scenes. SPIn-NeRF uses a two-stage approach: first, it inpaints each input view using a 2D inpainting method, then it optimizes a NeRF to fit these inpainted views. By leveraging multi-view consistency, SPIn-NeRF can produce more coherent results across different viewpoints than single-view methods.
% Another significant contribution in this area is the work by \citet{philip2018plane} on object removal for image-based rendering. Their method uses multi-view stereo to reconstruct the scene geometry and then performs inpainting in both color and depth spaces across multiple views. This approach demonstrates the importance of considering both appearance and geometry in multi-view 3D inpainting.
% Inpaint3D, proposed by \citet{inpaint3d}, takes a different approach by training a 3D-aware inpainting network on a large dataset of indoor scenes. This method can leverage the learned 3D priors to produce geometrically consistent inpaintings across multiple views, even for large missing regions.
% Recent advancements in NeRF-based representations have led to more sophisticated multi-view inpainting methods. For instance, InpaintNeRF360 by \citet{wang2023inpaintnerf360} extends inpainting capabilities to 360-degree scenes. This method uses a combination of 2D inpainting guidance and 3D consistency optimization to handle the challenges of inpainting in omnidirectional environments.
% Gaussian Grouping, introduced by \citet{ye2023gaussian}, presents a novel approach to 3D scene editing using 3D Gaussian Splatting. While not specifically designed for inpainting, this method demonstrates how multi-view information can be leveraged to segment and manipulate 3D scenes represented by Gaussians, opening new possibilities for 3D inpainting tasks.
% A common thread among these multi-view methods is their ability to maintain consistency across different viewpoints, a crucial aspect of 3D scene inpainting. By leveraging information from multiple views, these approaches can better understand the underlying 3D structure of the scene and produce inpainted results that are coherent with the global scene geometry \citep{mirzaei2023reference}.
% However, challenges remain. Many of these methods still struggle with large-scale occlusions or complex geometric structures \citep{weder2022removing}. The computational cost of processing multiple views can also be significant, especially for high-resolution or large-scale scenes \citep{barron2023zip}. Additionally, balancing the influence of different views and handling potential inconsistencies between them remains an active area of research \citep{yin2023or}.
% Despite these challenges, multi-view 3D inpainting methods have significantly advanced the state of the art, enabling more realistic and consistent scene editing and completion. As research progresses, we can expect to see further improvements in the quality and efficiency of these techniques, potentially leading to new applications in fields such as virtual reality, augmented reality, and digital twin technologies \citep{bommasani2021opportunities}.

% Multi-view 3D inpainting methods address the limitations of single-view approaches. SPIn-NeRF \citep{spinnerf} combines NeRF with multi-view image inpainting. \citet{philip2018plane} use multi-view stereo for object removal in image-based rendering. Inpaint3D \citep{inpaint3d} leverages learned 3D priors. InpaintNeRF360 \citep{wang2023inpaintnerf360} extends to 360-degree scenes, while Gaussian Grouping \citep{ye2023gaussian} uses 3D Gaussian Splatting. These methods maintain consistency across viewpoints \citep{mirzaei2023reference} but face challenges with large-scale occlusions \citep{weder2022removing}, computational costs \citep{barron2023zip}, and view inconsistencies \citep{yin2023or}. Despite challenges, they advance scene editing and completion, potentially leading to new applications \citep{bommasani2021opportunities}.










\section{Preliminaries}
\label{sec:preliminaries}

Let $P \subset \mathbb{R}^2$ be a set of points in the plane.
We denote the Euclidean distance between two points $u,v \in P$ by $d(u,v)$.
For a connected geometric graph $G = (P, E)$ with $E \subseteq {P \choose 2}$, we denote the Euclidean shortest path between two points $u,v \in P$ by $\pi_G(u,v)$ and its length by $|\pi_G(u,v)|$,
omitting $G$ if it is clear from context.
The dilation $\rho_G(u,v)$ between two points $u,v$ in $G$ is the ratio $\rho_G(u,v) := \frac{|\pi_G(u,v)|}{d(u,v)}$ between the shortest path length and the Euclidean distance.
The dilation $\rho(G)$ of the graph $G$ is defined as the maximum dilation between any two points in $P$,
i.e., $\rho(G) := \max \{ \rho_G(u,v) \mid u,v \in P, u \neq v\}.$

In the remainder of this work, the graph $G$ we consider is a triangulation, i.e., a maximal crossing-free graph on $P$.
Two edges $e_1 = (p_1, q_1), e_2 = (p_2, q_2)$ are said to \emph{cross} or \emph{intersect} iff the line segments they induce intersect in their interior.
Given a point set $P$, the Minimum Dilation Triangulation problem (MDT) asks to find a triangulation $T$ of $P$ minimizing $\rho(T)$.

A detailed overview of the proposed architecture that converts images and control commands
into trajectories is depicted in~\autoref{fig:monoforce}.
The model consists of several learnable modules that deeply interact with each other.
The \emph{terrain encoder} carefully transforms visual features from the input image
into the heightmap space using known camera geometry.
The resultant heightmap features are further refined into interpretable physical quantities
that capture properties of the terrain such as its shape, friction, stiffness, and damping.
Next, the \emph{physics engine} combines the terrain properties with the robot model,
robot state, and control commands and delivers reaction forces at points of robot-terrain contacts.
It then solves the equations of motion dynamics by integrating these forces
and delivers the trajectory of the robot.
Since the complete computational graph of the feedforward pass is retained,
the backpropagation from an arbitrary loss, constructed on top of delivered trajectories,
or any other intermediate outputs is at hand.

\subsection{Terrain Encoder}\label{subsec:terrain_encoder}

The part of the MonoForce architecture (\autoref{fig:monoforce})
that predicts terrain properties $\mathbf{m}$ from sensor measurements $\mathbf{z}$ is called \emph{terrain encoder}.
The proposed architecture starts by converting pixels from a 2D image plane into a heightmap with visual features.
Since the camera is calibrated, there is a substantial geometrical prior that connects heightmap cells with the pixels.
We incorporate the geometry through the Lift-Splat-Shoot architecture~\cite{philion2020lift}.
This architecture uses known camera intrinsic parameters to estimate rays corresponding to particular pixels~--
pixel rays, \autoref{fig:bevfusion}.
For each pixel ray, the convolutional network then predicts depth probabilities and visual features.
Visual features are vertically projected on a virtual heightmap for all possible depths along the corresponding ray.
The depth-weighted sum of visual features over each heightmap cell is computed,
and the resulting multichannel array is further refined by deep convolutional network
to estimate the terrain properties $\mathbf{m}$.

The terrain properties include the geometrical heightmap $\mathcal{H}_g$,
the heights of the terrain supporting layer hidden under the vegetation $\mathcal{H}_t = \mathcal{H}_g - \Delta\mathcal{H}$,
terrain friction $\mathcal{M}$, stiffness $\mathcal{K}$, and dampening $\mathcal{D}$.
The intuition behind the introduction of the $\Delta\mathcal{H}$ term is
that $\mathcal{H}_t$ models a partially flexible layer of terrain (e.g. mud) that is hidden under flexible vegetation,~\autoref{fig:monoforce_heightmaps}.


\subsection{Differentiable Physics Engine}\label{subsec:dphysics}
The differentiable physics engine solves the robot motion equation and estimates
the trajectory corresponding to the delivered forces.
The trajectory is defined as a sequence of robot states $\tau = \{s_0, s_1, \ldots, s_T\}$,
where $\mathbf{s}_t = [\mathbf{x}_t, \mathbf{v}_t, R_t, \boldsymbol{\omega}_t]$
is the robot state at time $t$,
$\mathbf{x}_t \in \mathbb{R}^3$ and $\mathbf{v}_t \in \mathbb{R}^3$ define the robot's position and velocity in the world frame,
$R_t \in \mathbb{R}^{3 \times 3}$ is the robot's orientation matrix, and $\boldsymbol{\omega}_t \in \mathbb{R}^3$ is the angular velocity.
To get the next state $\mathbf{s}_{t+1}$, in general, we need to solve the following ODE:
\begin{equation}
    \label{eq:state_propagation}
    \mathbf{\dot{s}}_{t+1} = f(\mathbf{s}_t, \mathbf{u}_t, \mathbf{z}_t)
\end{equation}
where $\mathbf{u}_t$ is the control input and $\mathbf{z}_t$ is the environment state.
In practice, however, it is not feasible to obtain the full environment state $\mathbf{z}_t$.
Instead, we utilize terrain properties $\mathbf{m}_t = [\mathcal{H}_t, \mathcal{K}_t, \mathcal{D}_t, \mathcal{M}_t]$
predicted by the terrain encoder.
In this case, the motion ODE~\eqref{eq:state_propagation} can be rewritten as:
\begin{equation}
    \label{eq:state_propagation_terrain}
    \mathbf{\dot{s}}_{t+1} = \hat{f}(\mathbf{s}_t, \mathbf{u}_t, \mathbf{m}_t)
\end{equation}

Let's now derive the equation describing the state propagation function $\hat{f}$.
The time index $t$ is omitted further for brevity.
We model the robot as a rigid body with total mass $m$ represented by a~set of mass points
$\mathcal{P} = \{(\mathbf{p}_i, m_i)\; | \; \mathbf{p}_i~\in~\mathbb{R}^3, m_i~\in~\mathbb{R}^+, i=1~\dots~N\}$,
where $\mathbf{p}_i$ denotes coordinates of the $i$-th 3D point in the robot's body frame.
We employ common 6DOF dynamics of a rigid body~\cite{contact_dynamics-2018} as follows:
\begin{equation}
  \begin{split}
    \dot{\mathbf{x}} &= \mathbf{v}\\
    \dot{\mathbf{v}} &= \frac{1}{m}\sum_i\mathbf{F}_i
  \end{split}
  \quad\quad
  \begin{split}
    \dot{R} &= \Omega R\\
    \dot{\boldsymbol{\omega}} &= \mathbf{J}^{-1}\sum_i \mathbf{p}_i\times\mathbf{F}_i
  \end{split}
  \label{eq:contact_dynamics}
\end{equation}
where $\Omega = [\boldsymbol{\omega}]_{\times}$ is the skew-symmetric matrix of $\boldsymbol{\omega}$.
We denote $\mathbf{F}_i\in\mathbb{R}^3$ a total external force acting on $i$-th robot's body point.
Total mass $m = \sum_i~m_i$ and moment of inertia $\mathbf{J}\in\mathbb{R}^{3\times 3}$ of the robot's rigid body are assumed to be known
static parameters since they can be identified independently in laboratory conditions.
Note that the proposed framework allows backpropagating the gradient with respect to these quantities, too,
which makes them jointly learnable with the rest of the architecture.
The trajectory of the rigid body is the iterative solution of differential equations~\eqref{eq:contact_dynamics},
that can be obtained by any ODE solver for given external forces and initial state (pose and velocities).

When the robot is moving over a terrain, two types of external forces are acting
on the point cloud $\mathcal{P}$ representing its model:
(i) gravitational forces and (ii) robot-terrain interaction forces.
The former is defined as $m_i\mathbf{g} = [0, 0, -m_ig]^\top$ and acts on
all the points of the robot at all times,
while the latter is the result of complex physical interactions that are not easy
to model explicitly and act only on the points of the robot that are in contact
with the terrain.
There are two types of robot-terrain interaction forces:
(i) normal terrain force that prevents the penetration of the terrain by the robot points,
(ii) tangential friction force that generates forward acceleration when the tracks are moving,
and prevents side slippage of the robot.

\textbf{Robot-terrain interaction forces}

\begin{figure}[t]
    \centering
    \includegraphics[width=0.7\columnwidth]{imgs/dphysics/spring_forces}
    \caption{\textbf{Terrain force model}: Simplified 2D sketch demonstrating
    normal reaction forces acting on a robot body consisting of two points $p_i$ and $p_j$ .}
    \label{fig:spring_terrain_model}
\end{figure}

\textit{Normal reaction forces}.

One extreme option is to predict the 3D force vectors $\mathbf{F}_i$ directly
by a neural network, but we decided to enforce additional prior assumptions to reduce the risk of overfitting.
These prior assumptions are based on common intuition from the contact dynamics of flexible objects.
In particular, we assume that the magnitude of the force that the terrain exerts on the point $\mathbf{p}_i\in \mathcal{P}$
increases proportionally to the deformation of the terrain.
Consequently, the network does not directly predict the force,
but rather predicts the height of the terrain $h\in\mathcal{H}_t$
at which the force begins to act on the robot body and the stiffness of the terrain $e\in\mathcal{K}$.
We understand the quantity $e$ as an equivalent of the spring constant from Hooke's spring model, \autoref{fig:spring_terrain_model}.
Given the stiffness of the terrain and the point of the robot that penetrated the terrain
by ${\Delta}h$, the reaction force is calculated as $e\cdot{\Delta}h$.
% \begin{figure}[t]
%     \centering
%     \includegraphics[width=0.4\columnwidth]{imgs/dphysics/robot-terrain_forces}
%     \caption{\textbf{Robot-terrain interaction forces} acting on the robot's body at its contact points
%     with the terrain.
%     The point cloud was sampled from the MARV (\autoref{fig:robot_platforms}(b)) robot's 3D model.}
%     \label{fig:interaction_forces}
% \end{figure}

Since such a force, without any additional damping, would lead to an eternal bumping
of the robot on the terrain, we also introduce a robot-terrain damping coefficient $d\in\mathcal{D}$,
which similarly reduces the force proportionally to the velocity of the point
that is in contact with the terrain.
The model applies reaction forces in the normal direction $\mathbf{n}_i$ of the terrain surface,
where the $i$-th point is in contact with the terrain.
\begin{equation}\label{eq:normal_force}
    \mathbf{N}_{i} = \begin{cases}
 (e_i\Delta h_i - d_i(\dot{\mathbf{p}}_{i}^\top\mathbf{n}_i))\mathbf{n}_i  & \text{if } \mathbf{p}_{zi}\leq h_i \\
\mathbf{0} & \text{if } \mathbf{p}_{zi}> h_i
\end{cases},
\end{equation}
where terrain penetration $\Delta h_i = (h_i-\mathbf{p}_{zi})\mathbf{n}_{zi}$ is
estimated by projecting the vertical distance on the normal direction.
For a better gradient propagation, we use the smooth approximation of the Heaviside step function:
\begin{equation}
    \label{eq:smooth_normal_force}
    \mathbf{N}_i = (e_i\Delta h_i - d_i(\dot{\mathbf{p}}_{i}^\top\mathbf{n}_i))\mathbf{n}_i \cdot \sigma(h_i - \mathbf{p}_{zi}),
\end{equation}
where $\sigma(x) = \frac{1}{1+e^{-kx}}$ is the sigmoid function with a steepness hyperparameter $k$.

\begin{figure}[t]
    \centering
    \includegraphics[width=\columnwidth]{imgs/dphysics/optimization}
    \caption{\textbf{Terrain computed by backpropagating through $\nabla$Physics:}
    Shape of the terrain (border of the area where terrain forces start to act) outlined by heightmap surface,
    its color represents the friction of the terrain.
    The optimized trajectory is in green, and the ground truth trajectory is in blue.}
    \label{fig:terrain_optim}
\end{figure}

\textit{Tangential friction forces}.

Our tracked robot navigates by moving the main tracks and 4 flippers (auxiliary tracks).
The flipper motion is purely kinematic in our model.
This means that in a given time instant, their pose is uniquely determined by a $4$-dimensional vector
of their rotations, and they are treated as a rigid part of the robot.
The motion of the main tracks is transformed into forces tangential to the terrain.
The friction force delivers forward acceleration of the robot when robot tracks
(either on flippers or on main tracks) are moving.
At the same time, it prevents the robot from sliding sideways.
When a robot point $\mathbf{p}_i$, which belongs to a track, is in contact with terrain with
friction coefficient $\mu\in\mathcal{M}$, the resulting friction force at a contact point is computed as follows,~\cite{yong2012vehicle}:
\begin{equation}\label{eq:friction_force}
    \mathbf{F}_{f, i} = \mu_i |\mathbf{N}_i| ((\mathbf{u}_i - \mathbf{\dot{p}}_i)^\top\boldsymbol{\tau}_i)\boldsymbol{\tau}_i,
\end{equation}
where $\mathbf{u}_i = [u, 0, 0]^\top$, $u$ is the velocity of a track, and $\mathbf{\dot{p}}_i$ is the velocity of the point $\mathbf{p}_i$
with respect to the terrain transformed into the robot coordinate frame,
$\boldsymbol{\tau}_i$ is the unit vector tangential to the terrain surface at the point $\mathbf{p}_i$.
This model can be understood as a simplified Pacejka's tire-road model~\cite{pacejka-book-2012}
that is popular for modeling tire-road interactions.

To summarize, the state-propagation ODE~\eqref{eq:state_propagation_terrain}
(state $\mathbf{s}~=~[\mathbf{x},~\mathbf{v},~R,~\boldsymbol{\omega}]$) for a mobile robot moving over a terrain
is described by the equations of motion~\eqref{eq:contact_dynamics} where the force applied at a robot's $i$-th body point is computed as follows:
\begin{equation}\label{eq:forces}
    \begin{split}
        \mathbf{F}_i &= m_i\mathbf{g} + \mathbf{N}_i + \mathbf{F}_{f, i}
    \end{split}
\end{equation}
The robot-terrain interaction forces at contact points $\mathbf{N}_i$ and $\mathbf{F}_{f, i}$
are defined by the equations~\eqref{eq:smooth_normal_force} and~\eqref{eq:friction_force} respectively.


\textbf{Implementation of the Differentiable ODE Solver}

We implement the robot-terrain interaction ODE~\eqref{eq:contact_dynamics} in PyTorch~\cite{Paszke-NIPS-2019}.
The \textit{Neural ODE} framework~\cite{neural-ode-2021} is used to solve the system of ODEs.
For efficiency reasons, we utilize the Euler integrator for the ODE integration.
The differentiable ODE solver~\cite{neural-ode-2021} estimates the gradient through the implicit function theorem.
Additionally, we implement the ODE~\eqref{eq:contact_dynamics} solver that
estimates gradient through \textit{auto-differentiation}~\cite{Paszke-NIPS-2019},
i.e. it builds and retains the full computational graph of the feedforward integration.


\subsection{Data-driven Trajectory Prediction}\label{subsec:data_driven_baseline}
Inspired by the work~\cite{pang2019aircraft}, we design a data-driven LSTM architecture (\autoref{fig:traj_lstm}) for our outdoor mobile robot's trajectory prediction.
We call the model TrajLSTM and use it as a baseline for our $\nabla$Physics engine.
\begin{figure}
    \centering
    \includegraphics[width=\columnwidth]{imgs/architectures/lstm}
    \caption{\textbf{TrajLSTM} architecture. The model takes as input: initial state $\mathbf{x}_0$, terrain $\mathcal{H}$, control sequence $\mathbf{u}_t, t \in \{0 \dots T\}$. It predicts the trajectory as a sequence of states $\mathbf{x}_t, t \in \{0 \dots T\}.$}
    \label{fig:traj_lstm}
\end{figure}
Given an initial robot's state $\mathbf{x}_0$ and a sequence of control inputs for a time horizon $T$, $\mathbf{u}_t, t \in \{0 \dots T\}$, the TrajLSTM model provides a sequence of states at control command time moments, $\mathbf{x}_t, t \in \{0 \dots T\}$.
As in outdoor scenarios the robot commonly traverses uneven terrain, we additionally include the terrain shape input to the model in the form of heightmap $\mathcal{H}=\mathcal{H}_0$ estimated at initial time moment $t=0$.
Each timestep's control input $\mathbf{u}_i$ is concatenated with the shared spatial features $\mathbf{x}_i$, as shown in \autoref{fig:traj_lstm}.
The combined features are passed through dense layers to prepare for temporal processing.
The LSTM unit~\cite{hochreiter1997long} processes the sequence of features (one for each timestep).
As in our experiments, the time horizon for trajectory prediction is reasonably small, $T=5 [\si{\sec}]$, and the robot's trajectories lie within the heightmap area, we use the shared heightmap input for all the LSTM units of the network.
So the heightmap is processed through the convolutional layers \textbf{once} and flattened, producing a fixed-size spatial feature vector.
This design choice (of not processing the heightmaps at different time moments) is also motivated by computational efficiency reason.
At each moment $t$, this heightmap vector is concatenated with the fused spatial-control features and processed by an LSTM unit.
The LSTM unit output for each timestep $t$ is passed through a fully connected (dense) layer to produce the next state $\mathbf{x}_{t+1}$.
The sequence of states form the predicted trajectory, $\{\mathbf{x}_0, \dots \mathbf{x}_T\}$.


\subsection{End-to-end Learning}\label{subsec:end2end_learning}
Self-supervised learning of the proposed architecture minimizes three different losses:

\textbf{Trajectory loss} that minimizes
the difference between SLAM-reconstructed trajectory $\tau^\star$ and predicted trajectory $\tau$:
\begin{equation}~\label{eq:traj_loss}
   \mathcal{L}_\tau = \|\tau-\tau^\star\|^2
\end{equation}

\textbf{Geometrical loss} that minimizes the difference between
ground truth lidar-reconstructed heightmap $\mathcal{H}_g^\star$
and predicted geometrical heightmap $\mathcal{H}_g$:
 \begin{equation}~\label{eq:geom_loss}
     \mathcal{L}_g = \|\mathbf{W}_g\circ(\mathcal{H}_g-\mathcal{H}_g^\star)\|^2
 \end{equation}
$\mathbf{W}_g$ denotes an array selecting the heightmap channel corresponding to the terrain shape.

\textbf{Terrain loss} that minimizes the difference between ground truth $\mathcal{H}_t^\star$
and predicted $\mathcal{H}_t$ supporting heightmaps containing rigid objects detected
with Microsoft's image segmentation model SEEM~\cite{zou2023segment},
that is derived from Segment Anything foundation model~\cite{li2023semantic}:
 \begin{equation}~\label{eq:terrain_loss}
     \mathcal{L}_t = \|\mathbf{W}_t\circ(\mathcal{H}_t-\mathcal{H}_t^\star)\|^2
 \end{equation}
$\mathbf{W}_t$ denotes the array selecting heightmap cells that are covered by rigid materials
(e.g. stones, walls, trunks), and $\circ$ is element-wise multiplication.

Since the architecture \autoref{fig:model_overview} is end-to-end differentiable,
we can directly learn to predict all intermediate outputs just using trajectory loss~\eqref{eq:traj_loss}.
An example of terrain learning with the trajectory loss is visualized in \autoref{fig:terrain_optim}.
To make the training more efficient and the learned model explainable, we employ the
geometrical loss~\eqref{eq:geom_loss} and terrain loss~\eqref{eq:terrain_loss} as regularization terms.
stat

\begin{figure*}
    \centering
    \includegraphics[width=\textwidth]{imgs/predictions/monoforce/qualitative_results_experiments}
    \caption{\textbf{MonoForce prediction examples}.
    \emph{Left}: The robot is moving through a narrow passage between a wall and tree logs.
    \emph{Right}: The robot is moving on a gravel road with rocks on the sides.
    It starts its motion from the position marked with a coordinate frame and the trajectory is predicted for $10~[\si{\sec}]$ using real control commands.
    The camera images are taken from the robot's initial position (\emph{top row}).
    The visualization includes predicted supporting terrain $\mathcal{H}_t$ (\emph{second row}).
    It is additionally shown in 3D and colored with predicted friction values (\emph{third row}).
    }
    \label{fig:monoforce_predictions}
\end{figure*}

The \autoref{fig:monoforce_predictions} show the prediction examples of the MonoForce model in diverse outdoor environments.
From the example on the left,
we can see that the model correctly predicts the robot's trajectory and the terrain shape suppressing traversable vegetation,
while the rigid obstacles (wall and tree logs) are correctly detected.
The example on the right demonstrates the model's ability to predict the robot's trajectory ($10~[\si{\sec}]$-long horizon)
with reasonable accuracy and to detect the rigid obstacles (stones) on the terrain.
It could also be noticed that the surfaces that provide the robot good traction (paved and gravel roads) are marked with a higher friction value,
while for the objects that might not give good contact with the robot's tracks (walls and tree logs) the friction value is lower.

We argue that the friction estimates are approximate and an interesting research direction could be
comparing them with real-world measurements or with the values provided by a high-fidelity physics engine (e.g. AGX Dynamics~\cite{Berglund2019agxTerrain}).
However, one of the benefits of our differentiable approach is that the model does not require ground-truth friction values for training.
The predicted heightmap's size is $12.8\times12.8\si{\meter}^2$ and the grid resolution is $0.1\si{\meter}$.
It has an upper bound of $1~[\si{\meter}]$ and a lower bound of $-1~[\si{\meter}]$.
This constraint was introduced based on the robot's size and taking into account hanging objects (tree branches)
that should not be considered as obstacles (\autoref{fig:nav_monoforce}).
Additionally, the terrain is predicted in the gravity-aligned frame.
That is made possible thanks to the inclusion of camera intrinsics and extrinsics as input to the model,
\autoref{fig:monoforce}.
It also allows correctly modeling the robot-terrain interaction forces (and thus modeling the robot's trajectory accurately)
for the scenarios with non-flat terrain, for example, going uphill or downhill.
This will not be possible if only camera images are used as input.
\section{Experiments}
In this section, we present the data collection and pre-processing, implementation details, and experimental results of the proposed approach. 

\subsection{Data Collection and Implementation Details}
\subsubsection{Dataset}
To the best of our knowledge, there are no existing datasets that demonstrate the stroke-by-stroke evolution of visual art from the initial sketch to the final painting. The closest available dataset for ordered stroke sequences is QuickDraw \cite{ha2017neural}, which consists of simple hand-drawn vector sketches of common objects, created in an online game where players are to draw specific objects within 20 seconds. However, these sketches do not align with our objective of generating the evolution of complex real-world artworks.

For our purposes, we curate a sample dataset from WikiArt \cite{saleh2015large} that includes a diverse range of artworks by renowned artists. In particular, we sampled 500 artworks from various artists to evaluate the effectiveness of the proposed method. Additionally, to investigate the proposed method in diverse settings, we harvested 90 sketch images and 70 RGB images, which include line art, face sketches, and natural images. We randomly sampled face sketches from FS2K-SDE Dataset \cite{dai2023sketch}, line art sketches from \cite{lineartweb}, and natural images from \cite{tong2021sketch}. 

\subsubsection{Implementation details}
To obtain sketches from paintings and natural images, we leverage the line drawing method \cite{chan2022learning} that trained on sampled COCO dataset using CLIP features. Further, to attain a vector image of the input sketch or image, we convert pixel image into vector curves through SVG conversion via the vectorizing tool \cite{vectwebsite}. We impose no restrictions on the dimensions of image inputs or the number of strokes within each image. And, the proximity distance is treated as a hyper-parameter. Here, the number of clusters and the number of strokes per cluster are determined based on this proximity distance. We found that setting proximity distance to approximately 
$\max(Input_{width}, Input_{height})/8$ yields compact clusters that provide a good balance between the number of clusters and the number of strokes per cluster.

\begin{figure}[!t]
    \centering
    \includegraphics[width=\textwidth]{Samples/Fig5_Evolution_WikiArt.pdf}
    \caption{Sketch \& Paint stroke evolution sequences on WikiArt samples.}
    \label{fig:wikiart_results}
\end{figure}

 
\subsection{Results}
We extensively test our algorithm on inputs with varying degrees of complexity and structure. Since there are no formal quantitative measures to gauge the valid stroke sequence evolution on input images, we principally assess the effectiveness of the proposed model qualitatively.

\subsubsection{Results on WikiArt}
 Figure \ref{fig:wikiart_results} shows some examples of stroke-by-stroke ordering on the WikiArt dataset. From this figure, we can observe that the proposed algorithm successfully composes stroke sequence evolution from sketch to paint. Additionally, it effectively handles images with varied resolutions, intricate details, numerous strokes, and diverse color palettes. 


\begin{figure}[!t]

    \centering
    \includegraphics[width=\textwidth]{Samples/Fig6_Evolution_DiverseData.pdf}   
    \caption{Demonstration of stroke evolution on various other input data types such as (A-B) Simple line art \cite{lineartweb} (C-D) Face sketches sampled from FS2K-SDE \cite{dai2023sketch}, (E-F) Natural images from \cite{tong2021sketch}.}
    \label{fig:aaai_results}
\end{figure}

To evaluate the robustness of the proposed method, we further apply it to other forms of data such as line art, face sketches, and natural images. Figure \ref{fig:aaai_results} presents sampled sequences from these diverse input images. The results demonstrate that the predicted stroke sequence order closely mirrors a pragmatic drawing process, regardless of the input type. Specifically, the algorithm produces plausible drawing sequences for less complex images like line art (Figure \ref{fig:aaai_results}, A-B) and face sketches (Figure \ref{fig:aaai_results}, C-D). These predicted sequences follow a logical and intuitive order, closely mirroring the natural progression an artist is likely to adopt. As seen in Figure \ref{fig:aaai_results} (E-F), our method can also effectively interpret natural images that are complex in terms of resolution, detail, and stroke count. From all the results presented above, we can infer that our algorithm can comprehend a variety of input images and produce a pragmatic drawing process. Dynamic examples of drawing evolution, from sketch to painting across various inputs, can be viewed at \cite{youtube_video}. 


\subsubsection{Comparison with other methods}
In this section, we analyze the effectiveness of our algorithm relative to other state-of-the-art methods. Figure \ref{fig:comparison} illustrates the comparative evolution of our method against prominent techniques \cite{tong2021sketch, liu2021paint}. 
For a fair assessment, we qualitatively compare only our image-to-sketch translation with VectorFlow’s \cite{tong2021sketch} image-to-pencil translation, and only our colored paint stroke sequence with Paint Transformer’s \cite{liu2021paint} paint sequence. 
In other words, to ensure fair comparability, we omit color sequencing for the VectorFlow comparison and sketch sequencing for the Paint Transformer comparison. From Figure \ref{fig:comparison}, we can infer that our proposed method can provide systematic sequencing rather than projecting strokes in random order as in \cite{liu2021paint,tong2021sketch}.

\begin{figure}[!t]
    \centering
    \includegraphics[width=\textwidth]{Samples/Fig7_Comparison_RelatedWork.pdf}
    \caption{Qualitative comparison of our method over Vector Flow \cite{tong2021sketch} and Paint Transformer \cite{liu2021paint} (We only include the relevant corresponding portions of generated sequence from our method).} 
    \label{fig:comparison}
\end{figure}

\begin{figure}[!t]
    \centering
    \includegraphics[width=\textwidth]{Samples/Fig8_UserSurvey.pdf}
    \caption{User-survey results for quality evaluation of our method, Paint \& Sketch, in comparison with  Vector Flow (A) \cite{tong2021sketch} and Paint Transformer (B). \cite{liu2021paint} }
    \label{fig:user_survey}
\end{figure}

Additionally, we conducted an initial user study with limited participants (5), each evaluating 5 image generations with VectorFlow \cite{tong2021sketch} and 3 image generations with PaintTransformer \cite{liu2021paint} along with corresponding generations through our method. Participants rated the systems on naturalness, user engagement, stroke quality, and overall experience, with scores ranging from 1 to 10. The survey asked the participants about how accurately the system emulated a pragmatic drawing process, how engaged they felt during the interaction, what the quality of stroke texture and tone was, and what their overall satisfaction was. The bar chart depicting the average user study results is shown in Figure \ref{fig:user_survey}. These results demonstrate that our method provides a more engaging and satisfying user experience than other related methods.

\subsubsection{Limitations} The proposed approach has some known limitations: (1) The method depends upon the quality of line drawing for sketch generation. Hence, it may fail to extract contour lines when the image is dominated by black-intensity regions. (2) There is no mechanism that identifies and learns from the generated sequences that are better aligned to human drawing processes. (3) The paint strokes are coarse and typically not in the form of strokes from any particular drawing medium such as painting brushes.
\section{Conclusion}
In this work we show that training high quality \slms with a very modest compute budget, is feasible. We give these main guidelines: (i) \textbf{Do not skimp on the model} - not all model families are born equal and the TWIST initialisation exaggerates this, thus it is worth selecting a stronger / bigger text-LM even if it means less tokens. we found Qwen$2.5$ to be a good choice; (ii) \textbf{Utilise synthetic training data} - pre-training on data generated with TTS helps a lot; (iii) \textbf{Go beyond next token prediction} - we found that DPO boosts performance notably even when using synthetic data, and as little as $30$ minutes training massively improves results; (iv) \textbf{Optimise hyper-parameters} - as researchers we often dis-regard this stage, yet we found that tuning learning rate schedulers and optimising code efficiency can improve results notably. We hope that these insights, and open source resources will be of use to the research community in furthering research into remaining open questions in \slms.

\bibliography{main_paper}
\bibliographystyle{icml2025}

\newpage
\appendix
\onecolumn
\newcommand{\gradientcellAA}[1]{\gradientcell{#1}{50}{76}}
\newcommand{\gradientcellBB}[1]{\gradientcell{#1}{36}{61}}
\newcommand{\gradientcellCC}[1]{\gradientcell{#1}{32}{47}}
\newcommand{\gradientcellDD}[1]{\gradientcell{#1}{24}{71}}
\newcommand{\gradientcellEE}[1]{\gradientcell{#1}{46}{83}}
\newcommand{\gradientcellFF}[1]{\gradientcell{#1}{49}{67}}

\begin{table*}[t]
\centering
\resizebox{0.9\textwidth}{!}{  
\begin{tabular}{cl|ccc|cccc}
 & & \multicolumn{3}{c|}{\textbf{Knowledge-Based}} & \multicolumn{3}{c|}{\textbf{Reasoning-Based}} \\

& \textbf{Option} & \textbf{MMLU} & \textbf{MMLU-Pro} & \textbf{GPQA} & \textbf{SQuAD} & \textbf{StrategyQA} & \textbf{MuSR} \\ \midrule
& Baseline                    & \gradientcellAA{72.7} \tiny{$\pm$ 0.8} & \gradientcellBB{57.5} \tiny{$\pm$ 0.7} & \gradientcellCC{45.2} \tiny{$\pm$ 2.0} & \gradientcellDD{69.5} \tiny{$\pm$ 0.8} & \gradientcellEE{76.6} \tiny{$\pm$ 2.3} & \gradientcellFF{63.2} \tiny{$\pm$ 1.2}  \\
& Baseline with \ac{CoT}      & \textbf{\gradientcellAA{75.5} \tiny{$\pm$ 0.8}} & \textbf{\gradientcellBB{60.8} \tiny{$\pm$ 1.0}} & \gradientcellCC{45.9} \tiny{$\pm$ 1.5} & \gradientcellDD{68.0} \tiny{$\pm$ 1.4} & \gradientcellEE{78.2} \tiny{$\pm$ 1.0} & \textbf{\gradientcellFF{66.8} \tiny{$\pm$ 1.6}} \\
\midrule
\raisebox{-1.4cm}[0pt][0pt]{\rotatebox{90}{\textbf{Voting}}} & Simple               & \gradientcellAA{75.5} \tiny{$\pm$ 1.3} & \gradientcellBB{56.5} \tiny{$\pm$ 5.4} & \gradientcellCC{45.7} \tiny{$\pm$ 0.3} & \gradientcellDD{69.5} \tiny{$\pm$ 0.5} & \gradientcellEE{81.2} \tiny{$\pm$ 1.4} & \gradientcellFF{59.3 }\tiny{$\pm$ 3.0}\\
& Ranked               & \gradientcellAA{73.3} \tiny{$\pm$ 1.9} & \gradientcellBB{54.3} \tiny{$\pm$ 0.9} & \gradientcellCC{44.2} \tiny{$\pm$ 2.1} & \textbf{\gradientcellDD{70.6} \tiny{$\pm$ 1.1}} & \gradientcellEE{80.3} \tiny{$\pm$ 1.0} & \gradientcellFF{59.5} \tiny{$\pm$ 0.5}\\
& Cumulative           & \gradientcellAA{72.5} \tiny{$\pm$ 1.9} & \gradientcellBB{53.0} \tiny{$\pm$ 0.4} & \gradientcellCC{43.9} \tiny{$\pm$ 3.1} & \gradientcellDD{69.7} \tiny{$\pm$ 1.1} & \gradientcellEE{80.0} \tiny{$\pm$ 0.8} & \gradientcellFF{60.0} \tiny{$\pm$ 1.7}\\
& Approval             & \gradientcellAA{50.2} \tiny{$\pm$ 2.1} & \gradientcellBB{36.3} \tiny{$\pm$ 3.7} & \gradientcellCC{33.0} \tiny{$\pm$ 3.4} & \gradientcellDD{24.3} \tiny{$\pm$ 5.0} & \gradientcellEE{46.4} \tiny{$\pm$ 13.9} & \gradientcellFF{49.1} \tiny{$\pm$ 12.4}  \\
& Average\footnotemark 
                            & \gradientcellAA{73.8} \tiny{$\pm$ 1.7} & \gradientcellBB{54.6} \tiny{$\pm$ 2.2} & \gradientcellCC{44.6} \tiny{$\pm$ 1.8} & \gradientcellDD{69.9} \tiny{$\pm$ 0.9} & \gradientcellEE{80.5} \tiny{$\pm$ 0.7} & \gradientcellFF{59.6} \tiny{$\pm$ 1.7}\\
\midrule
\raisebox{-1.4cm}[0pt][0pt]{\rotatebox{90}{\textbf{Consensus}}} &Majority          & \gradientcellAA{74.0} \tiny{$\pm$ 1.4} & \gradientcellBB{57.3} \tiny{$\pm$ 3.0} & \gradientcellCC{43.7} \tiny{$\pm$ 1.0} & \gradientcellDD{58.2} \tiny{$\pm$ 1.0} & \gradientcellEE{80.1} \tiny{$\pm$ 0.3} & \gradientcellFF{61.3} \tiny{$\pm$ 3.3} \\
& Supermajority        & \gradientcellAA{71.9} \tiny{$\pm$ 1.2} & \gradientcellBB{57.0} \tiny{$\pm$ 1.7} & \textbf{\gradientcellCC{46.6} \tiny{$\pm$ 0.8}} & \gradientcellDD{54.3} \tiny{$\pm$ 1.9} & \gradientcellEE{80.3} \tiny{$\pm$ 1.3} & \gradientcellFF{60.2} \tiny{$\pm$ 0.3} \\
& Unanimity             & \gradientcellAA{72.2} \tiny{$\pm$ 2.4} & \gradientcellBB{57.3} \tiny{$\pm$ 1.5} & \gradientcellCC{45.3} \tiny{$\pm$ 2.5} & \gradientcellDD{56.7} \tiny{$\pm$ 2.2} & \gradientcellEE{78.1} \tiny{$\pm$ 2.3} & \gradientcellFF{61.3} \tiny{$\pm$ 0.8}  \\
& Average                     & \gradientcellAA{72.7} \tiny{$\pm$ 1.7} & \gradientcellBB{57.2} \tiny{$\pm$ 2.1} & \gradientcellCC{45.2} \tiny{$\pm$ 1.4} & \gradientcellDD{56.4} \tiny{$\pm$ 1.7} & \gradientcellEE{79.5} \tiny{$\pm$ 1.3} & \gradientcellFF{60.9} \tiny{$\pm$ 1.5}\\
\midrule
& \textbf{Metric} & \textbf{Accuracy} & \textbf{Accuracy} & \textbf{Accuracy} & \textbf{F1 Score} & \textbf{Accuracy} & \textbf{Accuracy} \\
\end{tabular}}
\caption{Task performance{\tiny$\pm$std} for seven decision protocols (voting and consensus-based) on six tasks (knowledge and reasoning) based on agents with Llama 70B. \textbf{Bold} indicates the highest results per dataset. Standard deviation for three runs.}

\label{tab:results_70b}
\end{table*}
\footnotetext{Approval Voting is left out as it consistently fails to reach a voting decision as described in \Cref{sec:experiment1}.}

\section{Additional Results}

Additional results for the first experiment to provide further information.
\subsection{Task Performance with Llama 3 70B}
\label{app:decision_70b}

Compared to the results of the Llama 3 8B model, the larger Llama 3 70B model performs much better overall, as seen in \Cref{tab:results_70b}. Most of the results are a bit better than the baseline, but the multi-agent discussions are only in a few cases able to outperform the \ac{CoT} baseline. This model does not fail to follow the prompt for the MuSR baseline and consensus-based decision protocols. Therefore, the big performance gain from the smaller model cannot be observed here. SQuAD 2.0 and StrategyQA had the largest performance gains, even outperforming the \ac{CoT} baseline, similar to the results from the smaller model. This difference in task performance can have many reasons. As \citet{li_dawn_2024} showed, smaller models are more likely to hallucinate, which reduces task performance. This can be mitigated by using multiple agents because it is less likely that two agents hallucinate the same things. Larger models tend to hallucinate less, reducing this effect for the Llama 3 70B model \citep{li_dawn_2024}. In general, Llama 3 70B has a much higher baseline for task performance, making it more difficult to improve baseline results. Many of the improvements by the Llama 3 8B model are quite small, except for the ones where the Llama 3 70B model also outperforms the \ac{CoT} baseline. This can be taken as evidence that these multi-agent discussions require specific problem structures, or else the agents are just talking about the same results for multiple rounds and agreeing with each other. If these discussions continue too long, they can drift away from the original task, which reduces task performance. This has also been observed by \citet{becker_multi-agent_2024} and an example can be seen in \Cref{example:failed_discussion}. A positive example of how discussion can help task performance can be seen in \Cref{example:good_discussion}.

\subsection{Termination percent}
\label{app:termination_percent}
The data in \Cref{tab:termination_percentages} shows the number of turns that are needed for each decision protocol to reach a final decision for the MMLU dataset. Most of the voting decision protocols are able to vote for a final answer already in the first round in which they are allowed to vote. Simple voting has the highest agreement rate, but also cumulative and ranked voting only need in a few cases another round. In contrast, the approval decision protocol only achieves this in $\sim27\%$ of the cases. About $14\%$ need another round and the rest is canceled after the fifth round. This happens because these models like to agree with each other, and therefore they tend to vote for many of the answers, which often leads to a tie. Therefore, more restrictive voting decision protocols can reach a decision more easily, as a tie is less likely. 
The consensus decision protocols require only one to two rounds to reach consensus and still achieve a higher task performance because these results are based on the MMLU dataset. The decision protocols tend to behave similarly in terms of rounds needed to create a final answer, independent of task and model.

\begin{table*}[ht]
\centering
\resizebox{0.9\textwidth}{!}{  
\begin{tabular}{l|ccccc|c}
\toprule
\textbf{Group} & \textbf{Turn 1} & \textbf{Turn 2} & \textbf{Turn 3} & \textbf{Turn 4} & \textbf{Turn 5\footnotemark} & \textbf{Task Performance Score} \\
\midrule
Voting         & 0.00\%      & 0.00\%      & 99.33\%  & 0.50\%   & 0.17\%   & \gradientcellA{53.3} \tiny{$\pm$ 1.8} \\
Cumulative     & 0.00\%      & 0.00\%      & 94.00\%  & 5.50\%   & 0.50\%   & \gradientcellA{52.6} \tiny{$\pm$ 4.0} \\
Ranked         & 0.00\%      & 0.00\%      & 91.17\%  & 7.83\%   & 1.00\%   & \gradientcellA{49.2} \tiny{$\pm$ 1.5} \\
Approval       & 0.00\%      & 0.00\%      & 26.67\%  & 14.33\%  & 59.00\%  & \gradientcellA{43.0} \tiny{$\pm$ 2.1} \\
\midrule
Majority       & 80.00\%  & 13.67\%  & 4.83\%   & 1.00\%   & 0.50\%   & \gradientcellA{53.2} \tiny{$\pm$ 2.5} \\
Supermaj.  & 79.33\%  & 14.33\%  & 4.83\%   & 1.00\%   & 0.50\%   & \gradientcellA{54.6} \tiny{$\pm$ 3.6} \\
Unanimity      & 59.50\%  & 21.67\%  & 12.67\%  & 3.50\%   & 2.67\%   & \gradientcellA{54.2} \tiny{$\pm$ 1.0} \\
\bottomrule
\end{tabular}}
\caption{Number of rounds needed for each decision protocol to reach a final decision for the MMLU dataset.}
\label{tab:termination_percentages}
\end{table*}

\footnotetext{In this round the discussion is terminated.}


\section{Additional Details on Datasets}
\label{appendix:datasets}
\begin{table*}[ht]
    \footnotesize
    \centering
    \renewcommand{\arraystretch}{1.1} % Adjusts the row spacing
    \resizebox{16cm}{!} 
    { 
    \begin{tblr}{hline{1,2,Z} = 0.8pt, hline{3-Y} = 0.2pt,
                 colspec = {Q[l,m, 13em] Q[l,m, 6em] Q[c,m, 8em] Q[c,m, 5em] Q[l,m, 14em]},
                 colsep  = 4pt,
                 row{1}  = {0.4cm, font=\bfseries, bg=gray!30},
                 row{2-Z} = {0.2cm},
                 }
\textbf{Dataset}       & \textbf{Table Source} & \textbf{\# Tables / Statements} & \textbf{\# Words / Statement} & \textbf{Explicit Control}\\ 
\SetCell[c=5]{c} \textit{Single-sentence Table-to-Text}\\
ToTTo \cite{parikh2020tottocontrolledtabletotextgeneration}   & Wikipedia        & 83,141 / 83,141                  & 17.4                          & Table region      \\
LOGICNLG \cite{chen2020logicalnaturallanguagegeneration} & Wikipedia        & 7,392 / 36,960                  & 14.2                          & Table regions      \\ 
HiTab \cite{cheng-etal-2022-hitab}   & Statistics web   & 3,597 / 10,672                  & 16.4                          & Table regions \& reasoning operator \\ 
\SetCell[c=5]{c} \textit{Generic Table Summarization}\\
ROTOWIRE \cite{wiseman2017challengesdatatodocumentgeneration} & NBA games      & 4,953 / 4,953                   & 337.1                         & \textbf{\textit{X}}                   \\
SciGen \cite{moosavi2021scigen} & Sci-Paper      & 1,338 / 1,338                   & 116.0                         & \textbf{\textit{X}}                   \\
NumericNLG \cite{suadaa-etal-2021-towards} & Sci-Paper   & 1,355 / 1,355                   & 94.2                          & \textbf{\textit{X}}                    \\
\SetCell[c=5]{c} \textit{Table Question Answering}\\
FeTaQA \cite{nan2021fetaqafreeformtablequestion}     & Wikipedia      & 10,330 / 10,330                 & 18.9                          & Queries rewritten from ToTTo \\
\SetCell[c=5]{c} \textit{Query-Focused Table Summarization}\\
QTSumm \cite{zhao2023qtsummqueryfocusedsummarizationtabular}                        & Wikipedia      & 2,934 / 7,111                   & 68.0                          & Queries from real-world scenarios\\ 
\textbf{eC-Tab2Text} (\textit{ours})                           & e-Commerce products      & 1,452 / 3,354                   & 56.61                          & Queries from e-commerce products\\
    \end{tblr}
    }
\caption{Comparison between \textbf{eC-Tab2Text} (\textit{ours}) and existing table-to-text generation datasets. Statements and queries are used interchangeably. Our dataset specifically comprises tables from the e-commerce domain.}
\label{tab:datasets}
\end{table*}
The dataset selection is very important for this work. It needs to be tested whether decision protocols perform well in multiple domains and whether some protocols perform better with specific tasks than others. Therefore, we selected datasets from different domains and divided them into two groups:

\begin{itemize}
    \item \textbf{Knowledge-based Datasets}: MMLU, MMLU-Pro, and GPQA. These still require some reasoning and domain knowledge.
    \item \textbf{Reasoning-based Datasets}: StrategyQA, MuSR, and SQuAD 2.0. These emphasize multistep reasoning and textual comprehension.
\end{itemize}

An overview of all these datasets can be found in \Cref{tab:datasets} with a description and the number of samples used for evaluation.

\subsection{Sampling Strategy}
Because multi-agent discussions are expensive, we use a small subset of each dataset that still represents the dataset effectively. This follows approaches used by \citet{yin_exchange--thought_2023, chen_reconcile_2024, becker_multi-agent_2024} and ensures a 95\% confidence level with a 5\% margin of error:

\[
n_0 = \frac{Z^2 \cdot p \cdot (1 - p)}{d^2},
\]
where \(Z = 1.96\), \(p = 0.5\), and \(d = 0.05\) \citep{thompson_sampling_2012}. For finite datasets, a finite population correction is applied:
\[
n = \frac{n_0}{1 + \frac{n_0 - 1}{N}},
\]
where \(N\) is the total number of samples in each dataset \citep{cochran_sampling_1953}. The specific sample sizes reflecting this calculation are listed in \Cref{tab:datasets}.

\subsection{Repeatability}
Each dataset was tested three times to obtain a standard deviation of the results \citep{reimers_reporting_2017, chen_reconcile_2024, becker_multi-agent_2024}, ensuring reliable and robust performance estimates across multiple evaluations.



\section{Multi-Agent Framework}
\label{sec:mallm}
For our experiments, we use the \ac{MALLM} framework\footnote{Available here: \href{https://anonymous.4open.science/r/mallm}{anonymous.4open.science/r/mallm}}.

\subsection{Architecture Overview}
\label{sec:architecture}
To better understand the different modules, we take a closer look at each component and what role it plays in creating multi-agent discussions. An overview can be found in \Cref{fig:mallm_overview} as it provides an example workflow for the framework and how a discussion is created. The discussion starts with generating personas relevant to the given task and assigning them to the participating agents. The personas are generated using the same \ac{LLM} which is later used for the agents. After that the agents start to generate solutions and improve the suggestions from the other agents. The turn order of the agents is defined by the \textit{discussion paradigm}. This also defines which answers are visible to other agents and who can talk to whom. The \textit{response generator} defines how an agent receives the other answers and also the way it responds. After a certain number of rounds or when enough agents agree, a \textit{decision protocol} is used to select the best answer either via voting or just by looking for a certain consensus threshold. If the decision protocol fails, for example, due to a tied vote, the discussion continues for another round. In the framework a parameter can be defined to terminate discussions after a certain number of rounds to make sure they do not communicate forever.

\begin{figure*}
    \centering
    \includegraphics[width=0.95\linewidth]{pdf/MALLM-Overview.drawio.pdf}
    \caption{Example multi-agent discussion conducted in the \ac{MALLM} framework. It showcases the functionality of the four modules and how they work together to get an improved final solution.}
    \label{fig:mallm_overview}
\end{figure*}

\paragraph{Agent Personalities.}
The first step of the discussion is the generation of agent personas. Each of the agents participating in the discussion has a certain persona assigned to them. This can unlock more knowledge for the \ac{LLM} on a specific topic \citep{kim_persona_2024}. To get the best results, we want as diverse personas as possible while still maintaining them to be relevant to the task. The default setting for the framework is to prompt a \ac{LLM} and ask for a persona relevant to the given task \citep{wang_unleashing_2024}. After each generation it also provides the generated personas to avoid duplication. This way of generating personas provides a good starting point, but as this is built as a modular component, it can be swapped out with another function, which, for example, generates half of the agents with this method and initializes the other ones as neutral agents without a persona.
\paragraph{Response Generators.}
Another important part of multi-agent discussions is how each agent responds to the previous responses. Do we use \ac{CoT} to improve performance, or does this result in too long answers? By changing the way an agent is prompted, a lot of performance can be gained or lost. Therefore, it is key to make this as customizable as possible. The researcher has the possibility to change the default behavior (neutral answers), for example, by prompting the agent to be more critical or changing the way the discussion history is presented. The system prompt for the agent's persona can also be adjusted. \ac{MALLM} already has many different built-in response generators. The ones relevant for this work are the following.
\begin{itemize}
    \item \textbf{Free Text} is the most basic form of the agent prompt. Each agent gets a predefined number of discussion history rounds as memory. The prompt language is neutral, and the task is presented each round to mitigate the potential drift from the topic of the discussion \citep{becker_multi-agent_2024}. In addition, the agent is always asked to agree or disagree with the answer of the previous agent.
    \item \textbf{Simple} behaves very similar to the Free Text response generator, but the prompt is a bit simpler to make it easier to understand for the \ac{LLM} and reduce the context length.
    \item \textbf{Critical} forces the agent to respond very critically to the previous answer and try to find new solutions. Some studies have shown that \acp{LLM} can show some form of sycophancy, which is not helpful for a constructive discussion \citep{sharma_towards_2023}. Encouraging them to be more critical may reduce this.
    \item \textbf{Reasoning} doesn't allow the agents to communicate their final solution with the other agents. They can only share reasoning that can be used to find a final solution. In the end, each agent has to come up with its own solution without being directly influenced by other agents.
\end{itemize}
\paragraph{Discussion Paradigms.}
These paradigms define the discussion format for the entire task. They can control the order in which the agents communicate with each other, and which answers are visible only to certain agents. Currently, all the built-in discussion paradigms are static, meaning that the turn order is predefined and cannot be changed based on specific events during the discussion. However, due to the modular nature of \ac{MALLM}, a new discussion paradigm can be added, for example using an \ac{LLM} as a moderator to dynamically decide which agent should respond next. Current research by \citet{yin_exchange--thought_2023} and \citet{becker_multi-agent_2024} shows that discussion protocols have little impact on downstream task performance. \ac{MALLM} includes the following discussion paradigms, which are illustrated in \Cref{fig:discussion_paradigms}. The first four paragdims are inspired by the work of \citet{yin_exchange--thought_2023}, while the fifth was developed as part of this work.
\begin{figure}[H]
    \centering
    \includegraphics[width=\linewidth]{pdf/Memory.drawio-1.pdf}
    \caption{Illustration of Discussion Paradigms available for use in \ac{MALLM}}
    \label{fig:discussion_paradigms}
\end{figure}

\begin{itemize}
    \item \textbf{Memory} is the most basic discussion paradigm. The agents respond to the solution of the previous agents with feedback or an improved solution. All answers are visible to the other agents. 
    \item \textbf{Relay} behaves similarly to the memory paradigm. The turn order is the same, but each agent can only see the answer from the previous agent.
    \item \textbf{Report} introduces one agent as a moderator that can communicate with other agents. The other agents can only communicate with the moderator and have access to these messages only. Only the moderator can see all messages.
    \item \textbf{Debate} is similar to the report paradigm, as it also needs a moderator. Here, the other agents can communicate for a predefined number of rounds before they forward their reasoning to the moderator agent, which starts the next round of debate.
    \item \textbf{Collective Refinement} In this protocol, each agent first generates an answer independently. In each subsequent round, every agent receives the responses from all other agents at the same time. Using this shared information, each agent refines their own answer. This process continues throughout the rounds, helping agents gradually reach a shared and improved solution. There is no turn order, and all agents have the same level of knowledge in each round.
\end{itemize}

\paragraph{Decision Protocols.}
These are crucial for the framework as they decide which answer gets presented as the final answer to the problem. Multi-agent discussions produce multiple results for the same problem because each agent has its own reasoning and ideas on how to solve the problem. Therefore, some process is needed to decide which answer looks the most promising. We divide these decision protocols into three subtypes that we want to analyze. An overview of how each of these decision protocols works theoretically can be found in \Cref{fig:decision_protocols} and all prompts used for them can be found in \Cref{sec:app_prompts}.

\begin{figure*}[H]
    \centering
    \includegraphics[width=0.95\linewidth]{pdf/MALLM-Overview.drawio.pdf}
    \caption{Example multi-agent discussion conducted in the \ac{MALLM} framework. It showcases the functionality of the four modules and how they work together to get an improved final solution.}
    \label{fig:mallm_overview}
\end{figure*}

\subparagraph{Consensus Based Decision Protocols.} 
These are the simplest kinds of decision protocols. After each answer, the next agent has to agree or disagree with the previous statement. Depending on the response generator, this happens in the same message, and the agreement is extracted with a regular expression, or this is split into multiple answers. If enough agents agree in order, there is a consensus. The final answer is extracted by instructing the last agent to solve the given task with the information available in the latest messages. The prompt used for this can be found in \Cref{sec:app_final_answer_prompt}. There are several types of consensus decision protocols available in \ac{MALLM}. \textbf{Majority consensus} requires 50\% of the agents to agree. \textbf{Supermajority consensus} requires 66\% of the agents to agree, and \textbf{unanimity consensus} requires all agents to agree.

\subparagraph{Voting Based Decision Protocols.}
For voting based decision protocols, the process differs slightly compared to consensus-based decision protocols. The agents are forced to discuss for a predefined number of turns and afterward create a final solution. In the default setting, they have to discuss for three rounds, as current research such as \citet{du_improving_2023} shows that this allows for reasonable strong improvements considering computing resources. If there happens to be a tie in the voting, the agents have to discuss it for another round, and after that, they are asked to vote again. If they do not reach a final decision before exceeding the maximum number of rounds (defined in the discussion configuration), the solution of the first agent is used. To analyze the impact of the voting procedure, different processes similar to the work of \citet{yang_llm_2024} are implemented.
\begin{itemize}
    \item \textbf{Simple Voting} Each of the agents has only one vote. They can vote for any other agent or for themselves. The agent with the most votes wins.
    \item \textbf{Ranked Voting} The agents have to rank all final answers. The best solution is chosen by adding the ranking indices for a given agent and then selecting the answer with the best cumulative rank.
    \item \textbf{Cumulative Voting} Each agent has to distribute up to 25 points to all possible answers. They can also give fewer points and freely divide the points between all agents (even themselves). The winner is selected by adding all the points for a given agent and selecting the final answer with the most points.
    \item \textbf{Approval Voting} The agent has to provide a list of solutions that it approves. After that, the approvals from all agents are counted, and the answer with the most approvals wins the vote.
\end{itemize}
\clearpage




\section{Prompts}
\label{sec:app_prompts}
\subsection{Final Answer Extraction}
\label{sec:app_final_answer_prompt}
\medskip
\begin{figure}[H]
    \centering
    \begin{combinedprompt}
    \textbf{System Prompt:} \\
    \begingroup
    \colorbox{systemcolor}{\parbox{\dimexpr\linewidth-2\fboxsep\relax}{
    Your role: \texttt{<persona>} (\texttt{<persona description>})
    }}
    \endgroup

    \vspace{0.4em} %

    \textbf{User Prompt:} \\
    \begingroup
    \colorbox{usercolor}{\parbox{\dimexpr\linewidth-2\fboxsep\relax}{
    You are tasked with creating a final solution based on the given input and your previous response.\\
    Task: \texttt{<task>}\\
    Input: \texttt{<input sample>}\\
    Your previous response: \texttt{<previous answer>}\\
    Extract the final solution to the task from the provided text. Remove statements of agreement, disagreement, and explanations. Do not modify the text. Do not output any text besides the solution. If there is no solution provided, just copy the previous response.
    }}
    \endgroup
\end{combinedprompt}
    \caption{Prompt used to extract the final answer of a given agent from its previous response.}
    \label{fig:extract_prompt}
\end{figure}

\subsection{Voting Prompts}
\label{app:voting_prompts}

\begin{figure}[H]
    \centering
    \begin{combinedprompt}
    \textbf{System Prompt:} \\
    \begingroup
    \colorbox{systemcolor}{\parbox{\dimexpr\linewidth-2\fboxsep\relax}{
    Your role: \texttt{<persona>} (\texttt{<persona description>})
    }}
    \endgroup

    \vspace{0.4em} %

    \textbf{User Prompt:} \\
    \begingroup
    \colorbox{usercolor}{\parbox{\dimexpr\linewidth-2\fboxsep\relax}{
    You are tasked with voting for the best solution from the list provided below based on the given task.\\
    Task: \texttt{<task>}\\
    Question: \texttt{<input sample>}\\
    Here are the possible solutions:\\
    Solution 1: \texttt{<agent 1 final answer>}\\
    Solution 2: \texttt{<agent 2 final answer>}\\
    Solution 3: \texttt{<agent 3 final answer>}\\
    Based on the above solutions, please provide the number of the solution you are voting for. Answer only with the number.
    }}
    \endgroup
\end{combinedprompt}
    \caption{Prompt used to get a vote from each agent for the Simple Voting decision protocol.}
    \label{fig:simple_voting_prompt}
\end{figure}

\begin{figure}[H]
    \centering
    \begin{combinedprompt}
    \textbf{System Prompt:} \\
    \begingroup
    \colorbox{systemcolor}{\parbox{\dimexpr\linewidth-2\fboxsep\relax}{
    Your role: \texttt{<persona>} (\texttt{<persona description>})
    }}
    \endgroup

    \vspace{0.4em} %

    \textbf{User Prompt:} \\
    \begingroup
    \colorbox{usercolor}{\parbox{\dimexpr\linewidth-2\fboxsep\relax}{
    You are tasked with approving any number of solutions from the list provided below based on the given task.\\
    Task: \texttt{<task>}\\
    Question: \texttt{<input sample>}\\
    Here are the possible solutions:\\
    Solution 1: \texttt{<agent 1 final answer>}\\
    Solution 2: \texttt{<agent 2 final answer>}\\
    Solution 3: \texttt{<agent 3 final answer>}\\
    Based on the above solutions, please provide the numbers of the solutions you are approving, separated by commas. Answer only with the numbers.
    }}
    \endgroup
\end{combinedprompt}
    \caption{Prompt used to get a vote from each agent for the Approval Voting decision protocol.}
    \label{fig:approval_voting_prompt}
\end{figure}

\begin{figure}[H]
    \centering
    \begin{combinedprompt}
    \textbf{System Prompt:} \\
    \begingroup
    \colorbox{systemcolor}{\parbox{\dimexpr\linewidth-2\fboxsep\relax}{
    Your role: \texttt{<persona>} (\texttt{<persona description>})
    }}
    \endgroup

    \vspace{0.4em} %

    \textbf{User Prompt:} \\
    \begingroup
    \colorbox{usercolor}{\parbox{\dimexpr\linewidth-2\fboxsep\relax}{
    You are tasked with distributing 10 points among the provided solutions based on the given task.\\
    Task: \texttt{<task>}\\
    Question: \texttt{<input sample>}\\
    Here are the possible solutions:\\
    Solution 1: \texttt{<agent 1 final answer>}\\
    Solution 2: \texttt{<agent 2 final answer>}\\
    Solution 3: \texttt{<agent 3 final answer>}\\
    Based on the above solutions, please distribute 10 points among the solutions. Provide your points allocation as a JSON dictionary where keys are solution numbers (as int) and values are the points. The total points should sum up to 10. Answer only with the JSON dictionary.
    }}
    \endgroup
\end{combinedprompt}
    \caption{Prompt used to get a vote from each agent for the Cumulative Voting decision protocol.}
    \label{fig:cumulative_voting_prompt}
\end{figure}

\begin{figure}[H]
    \centering
    \begin{combinedprompt}
    \textbf{System Prompt:} \\
    \begingroup
    \colorbox{systemcolor}{\parbox{\dimexpr\linewidth-2\fboxsep\relax}{
    Your role: \texttt{<persona>} (\texttt{<persona description>})
    }}
    \endgroup

    \vspace{0.4em} %

    \textbf{User Prompt:} \\
    \begingroup
    \colorbox{usercolor}{\parbox{\dimexpr\linewidth-2\fboxsep\relax}{
    You are tasked with ranking the solutions from the most preferred to the least preferred based on the given task.\\
    Task: \texttt{<task>}\\
    Question: \texttt{<input sample>}\\
    Here are the possible solutions:\\
    Solution 1: \texttt{<agent 1 final answer>}\\
    Solution 2: \texttt{<agent 2 final answer>}\\
    Solution 3: \texttt{<agent 3 final answer>}\\
    Based on the above solutions, please provide the rankings of the solutions separated by spaces. Example: '0 2 1' if you prefer Solution 0 the most, then Solution 2, and finally Solution 1. Provide up to 5 rankings. Only answer with the rankings.
    }}
    \endgroup
\end{combinedprompt}
    \caption{Prompt used to get a vote from each agent for the Ranked Voting decision protocol.}
    \label{fig:ranked_voting_prompt}
\end{figure}







\subsection{Challenge Prompt}
\label{app:challenge_prompt}
\begin{figure}[H]
    \centering
    \begin{combinedprompt}
    \textbf{System Prompt:} \\
    \begingroup
    \colorbox{systemcolor}{\parbox{\dimexpr\linewidth-2\fboxsep\relax}{
    You are a participant in a group discussion. \\
    Your role: \texttt{<persona>} (\texttt{<persona description>})
    }}
    \endgroup

    \vspace{0.4em} %

    \textbf{User Prompt:} \\
    \begingroup
    \colorbox{usercolor}{\parbox{\dimexpr\linewidth-2\fboxsep\relax}{
    The task is: \texttt{<task>}. The question is: \texttt{<question>}. \\
    This is the final answer generated by the discussion: \texttt{<final\_answer>}. \\
    Please critically evaluate this answer. If you agree with the final answer, respond with the exact word 'AGREE' to confirm. If you do not agree, respond with the exact word 'DISAGREE' to challenge the answer.
    }}
    \endgroup
\end{combinedprompt}
    \caption{Prompt used to challenge the final answer.}
    \label{fig:challenge_prompt}
\end{figure}

\begin{figure}[H]
    \centering
    \begin{combinedprompt}
    \textbf{System Prompt:} \\
    \begingroup
    \colorbox{systemcolor}{\parbox{\dimexpr\linewidth-2\fboxsep\relax}{
    You are a participant in a group discussion. \\
    Your role: \texttt{<persona>} (\texttt{<persona description>})
    }}
    \endgroup

    \vspace{0.4em} %

    \textbf{User Prompt:} \\
    \begingroup
    \colorbox{usercolor}{\parbox{\dimexpr\linewidth-2\fboxsep\relax}{
    The task is: \texttt{<task>}. The question is: \texttt{<question>}. \\
    This is the final answer generated by the discussion: \texttt{<final\_answer>}. \\
    You dont agree with the final answer. Please provide a new answer to the question. Include the letter corresponding to your answer in the solution.
    }}
    \endgroup
\end{combinedprompt}
    \caption{Prompt used to generate a new answer in case the final answer got challenged.}
    \label{fig:challenge_new_prompt}
\end{figure}
\newpage
\onecolumn
\section{MALLM Setup}
\label{sec:mallm_setup_app}

    
\begin{configpython}[Default Parameters used for each experiment]{lst:config_default}
input_json_file_path: str = None
output_json_file_path: str = None
task_instruction_prompt: str = None
task_instruction_prompt_template: Optional[str] = None
endpoint_url: str = "https://api.openai.com/v1"
model_name: str = "gpt-3.5-turbo"
api_key: str = "-"
max_turns: int = 10
skip_decision_making: bool = False
discussion_paradigm: str = "memory"
response_generator: str = "simple"
decision_protocol: str = "hybrid_consensus"
visible_turns_in_memory: int = 2
debate_rounds: int = 2
concurrent_api_requests: int = 100
use_baseline: bool = False
use_chain_of_thought: bool = True
num_agents: int = 3
num_neutral_agents: int = 0
agent_generator: str = "expert"
agent_generators_list: list = []
trust_remote_code: bool = False
num_samples: Optional[int] = None
hf_dataset_split: Optional[str] = "test"
hf_token: Optional[str] = None
hf_dataset_version: Optional[str] = None
hf_dataset_input_column: Optional[str] = None
hf_dataset_reference_column: Optional[str] = None
hf_dataset_context_column: Optional[str] = None
use_ablation: bool = False
shuffle_input_samples: bool = False
all_agents_generate_first_draft: bool = False
all_agents_generate_draft: bool = False
policy: Optional[str] = None
voting_protocols_with_alterations: bool = False
calculate_persona_diversity: bool = False
\end{configpython}
\newpage
\section{Example Discussions}
\subsection{Successfull Voting Discussion}
All decision protocols are attached as an example to this discussion. The original discussion was created using simple voting.

\vspace{5mm}

\definecolor{lightyellow}{RGB}{255, 255, 204}

\begin{tcolorbox}[title=Overview, colframe=black!10, coltitle=black, fonttitle=\bfseries, boxrule=0.5mm, width=\textwidth]

\textbf{Task:} Answer the following question. If the question is not answerable with the provided information, write '[UNKNOWN]'.

\textbf{Input:} What choir from the area has sung at the Guildhall?

\textbf{Context:} Southampton has two large live music venues, the Mayflower Theatre (formerly the Gaumont Theatre) and the Guildhall. The Guildhall has seen concerts from a wide range of popular artists including Pink Floyd, David Bowie, Delirious?, Manic Street Preachers, The Killers, The Kaiser Chiefs, Amy Winehouse, Lostprophets, The Midnight Beast, Modestep, and All Time Low. It also hosts classical concerts presented by the Bournemouth Symphony Orchestra, City of Southampton Orchestra, Southampton Concert Orchestra, Southampton Philharmonic Choir and Southampton Choral Society.  

\begin{tcolorbox}[coltitle=black, fonttitle=\bfseries, boxrule=0.5mm, width=\textwidth]
\textbf{MALLM solution:} Southampton Philharmonic Choir \\
\textbf{Real solution:} Southampton Philharmonic Choir
\end{tcolorbox}
\end{tcolorbox}


\begin{tcolorbox}[breakable, enhanced jigsaw, title=Turn 1, colframe=black!10, coltitle=black, fonttitle=\bfseries, boxrule=0.5mm, width=\textwidth]

\begin{tcolorbox}[title=Music Connoisseur \hfill \tiny{\textnormal{A knowledgeable musician who is familiar with various music genres and has attended concerts at the Guildhall.}}, colback=lightyellow, colframe=orange!30, coltitle=black, fonttitle=\bfseries, boxrule=0.5mm, width=\textwidth]
\begin{minipage}{1.0\textwidth}
Based on the information provided, I propose the following solution: Southampton Philharmonic Choir. The text states: "It also hosts classical concerts presented by the Bournemouth Symphony Orchestra, City of Southampton Orchestra, Southampton Concert Orchestra, Southampton Philharmonic Choir and Southampton Choral Society." This suggests that the Southampton Philharmonic Choir is one of the choirs that has performed at the Guildhall, as it is mentioned alongside the other organizations that host classical concerts at the venue.
\end{minipage}
\end{tcolorbox}

\begin{tcolorbox}[title=Choir Conductor \hfill \tiny{\textnormal{A dedicated choral conductor with extensive experience in leading local choirs and orchestras, familiar with the performances at the Guildhall.}}, colback=lightyellow, colframe=orange!30, coltitle=black, fonttitle=\bfseries, boxrule=0.5mm, width=\textwidth]
\begin{minipage}{1.0\textwidth}
[AGREE] I agree with the current solution that the Southampton Philharmonic Choir has sung at the Guildhall. The text explicitly mentions it as one of the organizations that hosts classical concerts at the Guildhall, which indicates that the choir has performed at the venue.
\end{minipage}
\end{tcolorbox}

\begin{tcolorbox}[title=Music Journalist \hfill \tiny{\textnormal{A music enthusiast who has written articles and reviews about various concerts and events at the Guildhall, including performances by local choirs.}}, colback=lightyellow, colframe=orange!30, coltitle=black, fonttitle=\bfseries, boxrule=0.5mm, width=\textwidth]
\begin{minipage}{1.0\textwidth}
[AGREE] I completely agree with the current solution. The text explicitly mentions the Southampton Philharmonic Choir as one of the organizations that hosts classical concerts at the Guildhall, making it the most likely choice.
\end{minipage}
\end{tcolorbox}

\end{tcolorbox}

\begin{tcolorbox}[breakable, enhanced jigsaw, title=Turn 2, colframe=black!10, coltitle=black, fonttitle=\bfseries, boxrule=0.5mm, width=\textwidth]

\begin{tcolorbox}[title=Music Connoisseur \hfill \tiny{\textnormal{A knowledgeable musician who is familiar with various music genres and has attended concerts at the Guildhall.}}, colback=lightyellow, colframe=orange!30, coltitle=black, fonttitle=\bfseries, boxrule=0.5mm, width=\textwidth]
\begin{minipage}{1.0\textwidth}
[AGREE] I'm in agreement with the current solution, as the text explicitly mentions the Southampton Philharmonic Choir as one of the organizations that hosts classical concerts at the Guildhall. This makes it the most likely candidate to have sung at the venue.
\end{minipage}
\end{tcolorbox}

\begin{tcolorbox}[title=Choir Conductor \hfill \tiny{\textnormal{A dedicated choral conductor with extensive experience in leading local choirs and orchestras, familiar with the performances at the Guildhall.}}, colback=lightyellow, colframe=orange!30, coltitle=black, fonttitle=\bfseries, boxrule=0.5mm, width=\textwidth]
\begin{minipage}{1.0\textwidth}
[AGREE] I agree with the current solution that the Southampton Philharmonic Choir has sung at the Guildhall. The text explicitly mentions it as one of the organizations that hosts classical concerts at the Guildhall, which indicates that the choir has performed at the venue.
\end{minipage}
\end{tcolorbox}

\begin{tcolorbox}[title=Music Journalist \hfill \tiny{\textnormal{A music enthusiast who has written articles and reviews about various concerts and events at the Guildhall, including performances by local choirs.}}, colback=lightyellow, colframe=orange!30, coltitle=black, fonttitle=\bfseries, boxrule=0.5mm, width=\textwidth]
\begin{minipage}{1.0\textwidth}
[AGREE] I completely agree with the current solution. The text explicitly mentions the Southampton Philharmonic Choir as one of the organizations that hosts classical concerts at the Guildhall, making it the most likely choice.
\end{minipage}
\end{tcolorbox}

\end{tcolorbox}


\begin{tcolorbox}[breakable, enhanced jigsaw, title=Turn 3, colframe=black!10, coltitle=black, fonttitle=\bfseries, boxrule=0.5mm, width=\textwidth]

\begin{tcolorbox}[title=Music Connoisseur \hfill \tiny{\textnormal{A knowledgeable musician who is familiar with various music genres and has attended concerts at the Guildhall.}}, colback=lightyellow, colframe=orange!30, coltitle=black, fonttitle=\bfseries, boxrule=0.5mm, width=\textwidth]
\begin{minipage}{1.0\textwidth}
[AGREE] I agree with the current solution and the assessment that there is no need to improve it. The text provides explicit evidence that the Southampton Philharmonic Choir hosts classical concerts at the Guildhall, making it the most likely correct answer.
\end{minipage}
\end{tcolorbox}

\begin{tcolorbox}[title=Choir Conductor \hfill \tiny{\textnormal{A dedicated choral conductor with extensive experience in leading local choirs and orchestras, familiar with the performances at the Guildhall.}}, colback=lightyellow, colframe=orange!30, coltitle=black, fonttitle=\bfseries, boxrule=0.5mm, width=\textwidth]
\begin{minipage}{1.0\textwidth}
[AGREE] I agree with the current solution that the Southampton Philharmonic Choir has sung at the Guildhall. The text explicitly mentions it as one of the organizations that hosts classical concerts at the Guildhall, which indicates that the choir has performed at the venue.
\end{minipage}
\end{tcolorbox}

\begin{tcolorbox}[title=Music Journalist \hfill \tiny{\textnormal{A music enthusiast who has written articles and reviews about various concerts and events at the Guildhall, including performances by local choirs.}}, colback=lightyellow, colframe=orange!30, coltitle=black, fonttitle=\bfseries, boxrule=0.5mm, width=\textwidth]
\begin{minipage}{1.0\textwidth}
[AGREE] I completely agree with the current solution. The text explicitly mentions the Southampton Philharmonic Choir as one of the organizations that hosts classical concerts at the Guildhall, making it the most likely choice.
\end{minipage}
\end{tcolorbox}

\end{tcolorbox}


\begin{tcolorbox}[title=Result Simple Voting, colframe=black!10, coltitle=black, fonttitle=\bfseries, boxrule=0.5mm, width=\textwidth]
\textbf{Music Connoisseur final answer:} Southampton Philharmonic Choir

\textbf{Choir Conductor final answer:} Southampton Choral Society

\textbf{Music Journalist final answer:} Southampton Philharmonic Choir

\bigskip

\textbf{Music Connoisseur} voted for answer from \textbf{Music Journalist}

\textbf{Choir Conductor} voted for answer from \textbf{Music Journalist}

\textbf{Music Journalist} voted for answer from \textbf{Music Journalist}


\end{tcolorbox}

\begin{tcolorbox}[title=Result Approval Voting, colframe=black!10, coltitle=black, fonttitle=\bfseries, boxrule=0.5mm, width=\textwidth]
\textbf{Music Critic final answer:} Southampton Philharmonic Choir

\textbf{Local Music Event Coordinator final answer:} Southampton Choral Society and Southampton Philharmonic Choir

\textbf{Information Architect final answer:} Southampton Philharmonic Choir

Southampton Choral Society

\bigskip

\textbf{Music Critic} approved answers from \textbf{Information Architect}, \textbf{Music Critic}

\textbf{Local Music Event Coordinator} approved answers from \textbf{Information Architect}

\textbf{Information Architect} approved answers from \textbf{Information Architect}


\end{tcolorbox}


\begin{tcolorbox}[title=Result Cumulative Voting, colframe=black!10, coltitle=black, fonttitle=\bfseries, boxrule=0.5mm, width=\textwidth]
\textbf{Local Music Enthusiast final answer:} Southampton Choral Society

Southampton Philharmonic Choir

\textbf{Archivist final answer:} Southampton Philharmonic Choir

Southampton Choral Society

\textbf{Music Journalist final answer:} Southampton Philharmonic Choir and Southampton Choral Society

\bigskip

\textbf{Local Music Enthusiast} allocated points: \\
\textbf{Archivist}: 9\\
\textbf{Music Journalist}: 1
\bigskip

\textbf{Archivist} allocated points: \\
\textbf{Local Music Enthusiast}: 2\\
\textbf{Archivist}: 4\\
\textbf{Music Journalist}: 4
\bigskip

\textbf{Music Journalist} allocated points: \\
\textbf{Local Music Enthusiast}: 1\\
\textbf{Archivist}: 2\\
\textbf{Music Journalist}: 7


\end{tcolorbox}

\begin{tcolorbox}[title=Result Ranked Voting, colframe=black!10, coltitle=black, fonttitle=\bfseries, boxrule=0.5mm, width=\textwidth]
\textbf{Music Journalist final answer:} Southampton Philharmonic Choir and Southampton Choral Society.

\textbf{Local Music Expert final answer:} Southampton Philharmonic Choir and Southampton Choral Society

\textbf{Concert Promoter final answer:} Southampton Philharmonic Choir and Southampton Choral Society

\bigskip

\textbf{Music Journalist} ranked answers: \\
1. Music Journalist\\
2. Local Music Expert\\
3. Concert Promoter
\bigskip

\textbf{Local Music Expert} ranked answers:\\
1. Music Journalist\\
2. Local Music Expert\\
3. Concert Promoter
\bigskip

\textbf{Concert Promoter} ranked answers:\\
1. Music Journalist\\
2. Local Music Expert\\
3. Concert Promoter


\end{tcolorbox}















\label{example:good_discussion}



\newpage
\subsection{Agents Tricked Discussion}
In this discussion, the agents were tricked by information provided in the context.

\vspace{5mm}

\definecolor{lightyellow}{RGB}{255, 255, 204}


\begin{tcolorbox}[title=Overview, colframe=black!10, coltitle=black, fonttitle=\bfseries, boxrule=0.5mm, width=\textwidth]

\textbf{Task:} Answer the following question. If the question is not answerable with the provided information, write '[UNKNOWN]'.

\textbf{Input:} What provides critical support for drug discovery and the availability of economic resources?

\textbf{Context:} Biodiversity provides critical support for drug discovery and the availability of medicinal resources. A significant proportion of drugs are derived, directly or indirectly, from biological sources: at least 50\% of the pharmaceutical compounds on the US market are derived from plants, animals, and micro-organisms, while about 80\% of the world population depends on medicines from nature (used in either modern or traditional medical practice) for primary healthcare. Only a tiny fraction of wild species has been investigated for medical potential. Biodiversity has been critical to advances throughout the field of bionics. Evidence from  analysis and biodiversity science indicates that the decline in output from the pharmaceutical sector since the mid-1980s can be attributed to a move away from natural product exploration ("bioprospecting") in favor of genomics and synthetic chemistry, indeed claims about the value of undiscovered pharmaceuticals may not provide enough incentive for companies in free markets to search for them because of the high cost of development; meanwhile, natural products have a long history of supporting significant economic and health innovation. Marine ecosystems are particularly important, although inappropriate bioprospecting can increase biodiversity loss, as well as violating the laws of the communities and states from which the resources are taken.

\begin{tcolorbox}[coltitle=black, fonttitle=\bfseries, boxrule=0.5mm, width=\textwidth]
\textbf{MALLM solution:} Biodiversity provides critical support for drug discovery and the availability of economic resources. \\
\textbf{Real solution:} Not included in context!
\end{tcolorbox}

\end{tcolorbox}


\begin{tcolorbox}[breakable, enhanced jigsaw, title=Turn 1, colframe=black!10, coltitle=black, fonttitle=\bfseries, boxrule=0.5mm, width=\textwidth]

\begin{tcolorbox}[title=Economist \hfill \tiny{\textnormal{A researcher who studies the impact of biodiversity on economic systems, with expertise in market analysis and the economic benefits of natural resources.}}, colback=lightyellow, colframe=orange!30, coltitle=black, fonttitle=\bfseries, boxrule=0.5mm, width=\textwidth]
\begin{minipage}{1.0\textwidth}
As an economist, I'd like to propose a solution that addresses the decline in output from the pharmaceutical sector since the mid-1980s and leverages the untapped potential of biodiversity for drug discovery and economic innovation. Here's a multi-faceted approach: **Solution:** Establish a Public-Private Partnership (PPP) to incentivize the exploration and development of natural products for pharmaceuticals, while ensuring responsible and sustainable bioprospecting practices.
\end{minipage}
\end{tcolorbox}

\begin{tcolorbox}[title=Ethnobotanist \hfill \tiny{\textnormal{A scientist who studies the traditional uses of plants by indigenous communities and their potential for medicinal applications, with expertise in deciphering the cultural and historical context of natural resources.}}, colback=lightyellow, colframe=orange!30, coltitle=black, fonttitle=\bfseries, boxrule=0.5mm, width=\textwidth]
\begin{minipage}{1.0\textwidth}
As an ethnobotanist, I agree with the current solution in principle. The Public-Private Partnership (PPP) framework has the potential to bring together the necessary stakeholders and incentivize the exploration and development of natural products for pharmaceuticals while ensuring responsible and sustainable bioprospecting practices. The multistep approach outlined is comprehensive and addresses various aspects of the issue.
\end{minipage}
\end{tcolorbox}

\begin{tcolorbox}[title=Research Biologist \hfill \tiny{\textnormal{A scientist who studies the properties and potential applications of natural products from a wide range of living organisms, with expertise in the discovery and development of new medicines from biodiversity sources.}}, colback=lightyellow, colframe=orange!30, coltitle=black, fonttitle=\bfseries, boxrule=0.5mm, width=\textwidth]
\begin{minipage}{1.0\textwidth}
I, as a Research Biologist, AGREE with the current solution and the additional suggestions proposed by the Ethnobotanist. The Public-Private Partnership (PPP) framework is a well-rounded approach that addresses the decline in output from the pharmaceutical sector by providing a framework for responsible and sustainable bioprospecting, while promoting economic innovation and job creation. I particularly appreciate the emphasis on community engagement and empowerment, benefit-sharing agreements, traditional knowledge protection, and capacity building for ecosystem-based medicine. 
\end{minipage}
\end{tcolorbox}

\end{tcolorbox}


\begin{tcolorbox}[breakable, enhanced jigsaw, title=Turn 2, colframe=black!10, coltitle=black, fonttitle=\bfseries, boxrule=0.5mm, width=\textwidth]

\begin{tcolorbox}[title=Economist \hfill \tiny{\textnormal{A researcher who studies the impact of biodiversity on economic systems, with expertise in market analysis and the economic benefits of natural resources.}}, colback=lightyellow, colframe=orange!30, coltitle=black, fonttitle=\bfseries, boxrule=0.5mm, width=\textwidth]
\begin{minipage}{1.0\textwidth}
    The current solution has been improved with additional considerations from the ethnobotanist and research biologist. I notice that the additional suggestions enhance the comprehensive approach, addressing community engagement, benefit-sharing, and traditional knowledge protection.
\end{minipage}
\end{tcolorbox}

\begin{tcolorbox}[title=Ethnobotanist \hfill \tiny{\textnormal{A scientist who studies the traditional uses of plants by indigenous communities and their potential for medicinal applications, with expertise in deciphering the cultural and historical context of natural resources.}}, colback=lightyellow, colframe=orange!30, coltitle=black, fonttitle=\bfseries, boxrule=0.5mm, width=\textwidth]
\begin{minipage}{1.0\textwidth}
As an ethnobotanist, I agree with the current solution in principle. The Public-Private Partnership (PPP) framework has the potential to bring together the necessary stakeholders and incentivize the exploration and development of natural products for pharmaceuticals while ensuring responsible and sustainable bioprospecting practices. The multistep approach outlined is comprehensive and addresses various aspects of the issue.
\end{minipage}
\end{tcolorbox}

\begin{tcolorbox}[title=Research Biologist \hfill \tiny{\textnormal{A scientist who studies the properties and potential applications of natural products from a wide range of living organisms, with expertise in the discovery and development of new medicines from biodiversity sources.}}, colback=lightyellow, colframe=orange!30, coltitle=black, fonttitle=\bfseries, boxrule=0.5mm, width=\textwidth]
\begin{minipage}{1.0\textwidth}
I, as a Research Biologist, AGREE with the current solution and the additional suggestions proposed by the Ethnobotanist. The Public-Private Partnership (PPP) framework is a well-rounded approach that addresses the decline in output from the pharmaceutical sector by providing a framework for responsible and sustainable bioprospecting, while promoting economic innovation and job creation. I particularly appreciate the emphasis on community engagement and empowerment, benefit-sharing agreements, traditional knowledge protection, and capacity building for ecosystem-based medicine. These considerations will ensure that local communities are involved in the decision-making process, receive fair compensation for their involvement, and that their traditional knowledge and cultural heritage are respected.

\end{minipage}
\end{tcolorbox}

\end{tcolorbox}


\begin{tcolorbox}[breakable, enhanced jigsaw, title=Turn 3, colframe=black!10, coltitle=black, fonttitle=\bfseries, boxrule=0.5mm, width=\textwidth]

\begin{tcolorbox}[title=Economist \hfill \tiny{\textnormal{A researcher who studies the impact of biodiversity on economic systems, with expertise in market analysis and the economic benefits of natural resources.}}, colback=lightyellow, colframe=orange!30, coltitle=black, fonttitle=\bfseries, boxrule=0.5mm, width=\textwidth]
\begin{minipage}{1.0\textwidth}
[AGREE] The revised solution has effectively incorporated the suggestions from the ethnobotanist and research biologist, addressing community engagement, benefit-sharing, traditional knowledge protection, and other crucial aspects of bioprospecting. The emphasis on responsible and sustainable bioprospecting ensures that the development of novel pharmaceuticals benefits both humans and the environment. To further refine the solution, I agree with the research biologist's suggestion to develop a clear and measurable framework for assessing the impact of the Public-Private Partnership (PPP) on the pharmaceutical sector, local communities, and the environment. Establishing Key Performance Indicators (KPIs) and a monitoring and evaluation process will help ensure that the PPP is held accountable for its impact and remains a responsible and effective approach to bioprospecting.
\end{minipage}
\end{tcolorbox}

\begin{tcolorbox}[title=Ethnobotanist \hfill \tiny{\textnormal{A scientist who studies the traditional uses of plants by indigenous communities and their potential for medicinal applications, with expertise in deciphering the cultural and historical context of natural resources.}}, colback=lightyellow, colframe=orange!30, coltitle=black, fonttitle=\bfseries, boxrule=0.5mm, width=\textwidth]
\begin{minipage}{1.0\textwidth}
As an ethnobotanist, I agree with the current solution in principle. The Public-Private Partnership (PPP) framework has the potential to bring together the necessary stakeholders and incentivize the exploration and development of natural products for pharmaceuticals while ensuring responsible and sustainable bioprospecting practices. The multistep approach outlined is comprehensive and addresses various aspects of the issue.
\end{minipage}
\end{tcolorbox}

\begin{tcolorbox}[title=Research Biologist \hfill \tiny{\textnormal{A scientist who studies the properties and potential applications of natural products from a wide range of living organisms, with expertise in the discovery and development of new medicines from biodiversity sources.}}, colback=lightyellow, colframe=orange!30, coltitle=black, fonttitle=\bfseries, boxrule=0.5mm, width=\textwidth]
\begin{minipage}{1.0\textwidth}
I, as a Research Biologist, AGREE with the current solution and the additional suggestions proposed by the Ethnobotanist. The Public-Private Partnership (PPP) framework is a well-rounded approach that addresses the decline in output from the pharmaceutical sector by providing a framework for responsible and sustainable bioprospecting, while promoting economic innovation and job creation. I particularly appreciate the emphasis on community engagement and empowerment, benefit-sharing agreements, traditional knowledge protection, and capacity building for ecosystem-based medicine. These considerations will ensure that local communities are involved in the decision-making process, receive fair compensation for their involvement, and that their traditional knowledge and cultural heritage are respected.

\end{minipage}
\end{tcolorbox}



\end{tcolorbox}


\begin{tcolorbox}[title=Result, colframe=black!10, coltitle=black, fonttitle=\bfseries, boxrule=0.5mm, width=\textwidth]
\textbf{Economist final answer:} Biodiversity

\textbf{Ethnobotanist final answer:} Biodiversity provides critical support for drug discovery and the availability of economic resources.

\textbf{Research Biologist final answer:} Biodiversity

\bigskip

\textbf{Economist} voted for answer from \textbf{Ethnobotanist}

\textbf{Ethnobotanist} voted for answer from \textbf{Ethnobotanist}

\textbf{Research Biologist} voted for answer from \textbf{Ethnobotanist}


\end{tcolorbox}

\label{example:trick_discussion}

\newpage
\subsection{Bad Voting Discussion}
In this discussion, the agents were tricked by one agent who came up with a solution not provided in the context.

\vspace{5mm}

\definecolor{lightyellow}{RGB}{255, 255, 204}

\begin{tcolorbox}[title=Overview, colframe=black!10, coltitle=black, fonttitle=\bfseries, boxrule=0.5mm, width=\textwidth]

\textbf{Task:} Answer the following question. If the question is not answerable with the provided information, write '[UNKNOWN]'.

\textbf{Input:} Where must enclitics appear in an English sentence?

\textbf{Context:} Because Czech uses grammatical case to convey word function in a sentence (instead of relying on word order, as English does), its word order is flexible. As a pro-drop language, in Czech an intransitive sentence can consist of only a verb; information about its subject is encoded in the verb. Enclitics (primarily auxiliary verbs and pronouns) must appear in the second slot of a sentence, after the first stressed unit. The first slot must contain a subject and object, a main form of a verb, an adverb or a conjunction (except for the light conjunctions a, "and", i, "and even" or ale, "but").

\begin{tcolorbox}[coltitle=black, fonttitle=\bfseries, boxrule=0.5mm, width=\textwidth]
\textbf{MALLM solution:} C) In an English sentence, enclitics (auxiliary verbs and pronouns) tend to appear next to the main verb, usually before any other adverbs or modifiers, but their placement can vary depending on sentence structure and context. \\
\textbf{Real solution:} Not included in context!
\end{tcolorbox}

\end{tcolorbox}

\begin{tcolorbox}[breakable, enhanced jigsaw, title=Turn 1, colframe=black!10, coltitle=black, fonttitle=\bfseries, boxrule=0.5mm, width=\textwidth]

\begin{tcolorbox}[title=Linguist \hfill \tiny{\textnormal{A scholar specializing in Slavic languages, experienced in analyzing the grammatical structures of Czech and other languages.}}, colback=lightyellow, colframe=orange!30, coltitle=black, fonttitle=\bfseries, boxrule=0.5mm, width=\textwidth]
\begin{minipage}{1.0\textwidth}
As a linguist, I'd like to analyze the information given about Czech grammar. We know that in Czech, enclitics (auxiliary verbs and pronouns) must appear in the second syntactic slot of a sentence, after the first stressed unit. Now, we're trying to apply this rule to an English sentence. English does not have a similar grammatical case system, and its word order is more fixed than Czech. However, we can try to find a rough equivalent. So, my proposed solution is that in an English sentence, enclitics (auxiliary verbs and pronouns) tend to appear next to the main verb, and usually before any other adverbs or modifiers. What do you think? Would you like to refine this proposal or add any additional considerations?
\end{minipage}
\end{tcolorbox}

\begin{tcolorbox}[title=English Language Teacher \hfill \tiny{\textnormal{An expert in teaching English grammar and language structures, familiar with the challenges of translating and comparing languages like Czech.}}, colback=lightyellow, colframe=orange!30, coltitle=black, fonttitle=\bfseries, boxrule=0.5mm, width=\textwidth]
\begin{minipage}{1.0\textwidth}
[DISAGREE] While I understand the thought process behind the proposed solution, I think it's essential to consider the unique characteristics of the English language. The idea that enclitics in English tend to appear next to the main verb is accurate, but it's not as rigid as the rule in Czech. In Czech, enclitics must appear in the second syntactic slot, which implies a more fixed position, whereas in English, the placement of auxiliary verbs and pronouns can vary depending on sentence structure and context.
\end{minipage}
\end{tcolorbox}

\begin{tcolorbox}[title=Grammar Analyst \hfill \tiny{\textnormal{A meticulous researcher who has studied the intricacies of languages and their grammatical structures, with a focus on comparing and contrasting language features.}}, colback=lightyellow, colframe=orange!30, coltitle=black, fonttitle=\bfseries, boxrule=0.5mm, width=\textwidth]
\begin{minipage}{1.0\textwidth}
I agree with the English Language Teacher's revised solution. The initial proposal tried to apply the Czech rule to English, but the English language has its unique characteristics that require a more nuanced approach. The revised solution accurately captures the flexibility of English grammar, acknowledging that the placement of enclitics can vary depending on sentence structure and context.
\end{minipage}
\end{tcolorbox}

\end{tcolorbox}

\begin{tcolorbox}[breakable, enhanced jigsaw, title=Turn 2, colframe=black!10, coltitle=black, fonttitle=\bfseries, boxrule=0.5mm, width=\textwidth]

\begin{tcolorbox}[title=Linguist \hfill \tiny{\textnormal{A scholar specializing in Slavic languages, experienced in analyzing the grammatical structures of Czech and other languages.}}, colback=lightyellow, colframe=orange!30, coltitle=black, fonttitle=\bfseries, boxrule=0.5mm, width=\textwidth]
\begin{minipage}{1.0\textwidth}
I agree with the revised solution proposed by the English Language Teacher and the Grammar Analyst. The revised solution accurately captures the flexibility of English grammar and acknowledges that the placement of enclitics (auxiliary verbs and pronouns) can vary depending on sentence structure and context. * The placement of enclitics in English is not as rigid as in Czech, and their placement can vary depending on sentence structure and context. * English sentence structure is more complex, including indirect objects, passives, and compound verb forms, which can affect the placement of auxiliary verbs and pronouns.
\end{minipage}
\end{tcolorbox}

\begin{tcolorbox}[title=English Language Teacher \hfill \tiny{\textnormal{An expert in teaching English grammar and language structures, familiar with the challenges of translating and comparing languages like Czech.}}, colback=lightyellow, colframe=orange!30, coltitle=black, fonttitle=\bfseries, boxrule=0.5mm, width=\textwidth]
\begin{minipage}{1.0\textwidth}
While I understand the thought process behind the proposed solution, I think it's essential to consider the unique characteristics of the English language. The idea that enclitics in English tend to appear next to the main verb is accurate, but it's not as rigid as the rule in Czech. In Czech, enclitics must appear in the second syntactic slot, which implies a more fixed position, whereas in English, the placement of auxiliary verbs and pronouns can vary depending on sentence structure and context. Additionally, the provided examples, while correct in themselves, don't fully demonstrate the complexity of English sentence structure. For instance, consider sentences with indirect objects, passives, or compound verb forms, where the placement of auxiliary verbs and pronouns can be more nuanced.
\end{minipage}
\end{tcolorbox}

\begin{tcolorbox}[title=Grammar Analyst \hfill \tiny{\textnormal{A meticulous researcher who has studied the intricacies of languages and their grammatical structures, with a focus on comparing and contrasting language features.}}, colback=lightyellow, colframe=orange!30, coltitle=black, fonttitle=\bfseries, boxrule=0.5mm, width=\textwidth]
\begin{minipage}{1.0\textwidth}
I agree with the English Language Teacher's revised solution. The initial proposal tried to apply the Czech rule to English, but the English language has its unique characteristics that require a more nuanced approach. The revised solution accurately captures the flexibility of English grammar, acknowledging that the placement of enclitics can vary depending on sentence structure and context. C) In an English sentence, enclitics (auxiliary verbs and pronouns) tend to appear next to the main verb, usually before any other adverbs or modifiers, but their placement can vary depending on sentence structure and context. I believe this revised solution provides a more accurate and practical approximation of the English language's enclitic behavior, considering both the unique characteristics of English and the complexity of sentence structure.
\end{minipage}
\end{tcolorbox}

\end{tcolorbox}

\begin{tcolorbox}[breakable, enhanced jigsaw, title=Turn 3, colframe=black!10, coltitle=black, fonttitle=\bfseries, boxrule=0.5mm, width=\textwidth]

\begin{tcolorbox}[title=Linguist \hfill \tiny{\textnormal{A scholar specializing in Slavic languages, experienced in analyzing the grammatical structures of Czech and other languages.}}, colback=lightyellow, colframe=orange!30, coltitle=black, fonttitle=\bfseries, boxrule=0.5mm, width=\textwidth]
\begin{minipage}{1.0\textwidth}
The current solution accurately captures the flexibility of English grammar and acknowledges that the placement of enclitics (auxiliary verbs and pronouns) can vary depending on sentence structure and context. There's no need to improve it further. The revised solution: C) In an English sentence, enclitics (auxiliary verbs and pronouns) tend to appear next to the main verb, usually before any other adverbs or modifiers, but their placement can vary depending on sentence structure and context. is a clear and accurate representation of the English language's enclitic behavior.
\end{minipage}
\end{tcolorbox}

\begin{tcolorbox}[title=English Language Teacher \hfill \tiny{\textnormal{An expert in teaching English grammar and language structures, familiar with the challenges of translating and comparing languages like Czech.}}, colback=lightyellow, colframe=orange!30, coltitle=black, fonttitle=\bfseries, boxrule=0.5mm, width=\textwidth]
\begin{minipage}{1.0\textwidth}
While I understand the thought process behind the proposed solution, I think it's essential to consider the unique characteristics of the English language. The idea that enclitics in English tend to appear next to the main verb is accurate, but it's not as rigid as the rule in Czech. In Czech, enclitics must appear in the second syntactic slot, which implies a more fixed position, whereas in English, the placement of auxiliary verbs and pronouns can vary depending on sentence structure and context. Here's a revised solution: C) In an English sentence, enclitics (auxiliary verbs and pronouns) tend to appear next to the main verb, usually before any other adverbs or modifiers, but their placement can vary depending on sentence structure and context. What do you think?
\end{minipage}
\end{tcolorbox}

\begin{tcolorbox}[title=Grammar Analyst \hfill \tiny{\textnormal{A meticulous researcher who has studied the intricacies of languages and their grammatical structures, with a focus on comparing and contrasting language features.}}, colback=lightyellow, colframe=orange!30, coltitle=black, fonttitle=\bfseries, boxrule=0.5mm, width=\textwidth]
\begin{minipage}{1.0\textwidth}
I agree with the English Language Teacher's revised solution. The initial proposal tried to apply the Czech rule to English, but the English language has its unique characteristics that require a more nuanced approach. The revised solution accurately captures the flexibility of English grammar, acknowledging that the placement of enclitics can vary depending on sentence structure and context. The revised solution is an improvement over the initial proposal because it takes into account the complexity of English sentence structure, including indirect objects, passives, and compound verb forms, which can affect the placement of auxiliary verbs and pronouns. Additionally, the revised solution provides a more accurate description of the typical placement of enclitics in English sentences, namely next to the main verb, usually before any adverbs or modifiers.
\end{minipage}
\end{tcolorbox}

\end{tcolorbox}

\begin{tcolorbox}[title=Result, colframe=black!10, coltitle=black, fonttitle=\bfseries, boxrule=0.5mm, width=\textwidth]
\textbf{Linguist final answer:} C) In an English sentence, enclitics (auxiliary verbs and pronouns) tend to appear next to the main verb, usually before any other adverbs or modifiers, but their placement can vary depending on sentence structure and context.

\textbf{English Language Teacher final answer:} [UNKNOWN]

\textbf{Grammar Analyst final answer:} [UNKNOWN]

\bigskip

\textbf{Linguist} voted for answer from \textbf{Linguist}

\textbf{English Language Teacher} voted for answer from \textbf{Linguist}

\textbf{Grammar Analyst} voted for answer from \textbf{Linguist}


\end{tcolorbox}

\label{example:failed_discussion}

\newpage
\subsection{Majority Consensus Discussion}
In this discussion, the agents discussed only one round, as they already had a high enough agreement score.

\vspace{5mm}

\definecolor{lightyellow}{RGB}{255, 255, 204}


\begin{tcolorbox}[title=Overview, colframe=black!10, coltitle=black, fonttitle=\bfseries, boxrule=0.5mm, width=\textwidth]

\textbf{Task:} Answer the following question. If the question is not answerable with the provided information, write '[UNKNOWN]'.

\textbf{Input:} What choir from the area has sung at the Guildhall?

\textbf{Context:} Southampton has two large live music venues, the Mayflower Theatre (formerly the Gaumont Theatre) and the Guildhall. The Guildhall has seen concerts from a wide range of popular artists including Pink Floyd, David Bowie, Delirious?, Manic Street Preachers, The Killers, The Kaiser Chiefs, Amy Winehouse, Lostprophets, The Midnight Beast, Modestep, and All Time Low. It also hosts classical concerts presented by the Bournemouth Symphony Orchestra, City of Southampton Orchestra, Southampton Concert Orchestra, Southampton Philharmonic Choir and Southampton Choral Society.  

\begin{tcolorbox}[coltitle=black, fonttitle=\bfseries, boxrule=0.5mm, width=\textwidth]
\textbf{MALLM solution:} Southampton Choral Society \\
\textbf{Real solution:} Southampton Philharmonic Choir
\end{tcolorbox}
\end{tcolorbox}


\begin{tcolorbox}[breakable, enhanced jigsaw, title=Turn 1, colframe=black!10, coltitle=black, fonttitle=\bfseries, boxrule=0.5mm, width=\textwidth]

\begin{tcolorbox}[title=Music Connoisseur (SOLUTION) \hfill \tiny{\textnormal{A knowledgeable musician who is familiar with various music genres and has attended concerts at the Guildhall.}}, colback=lightyellow, colframe=orange!30, coltitle=black, fonttitle=\bfseries, boxrule=0.5mm, width=\textwidth]
\begin{minipage}{1.0\textwidth}
As a Music Historian, I've been through the archives and checked the event schedules of the Guildhall. According to my research, one of the choirs that has sung at the Guildhall is... Southampton Choral Society!
\end{minipage}
\end{tcolorbox}

\begin{tcolorbox}[title=Choir Conductor (AGREED) \hfill \tiny{\textnormal{A dedicated choral conductor with extensive experience in leading local choirs and orchestras, familiar with the performances at the Guildhall.}}, colback=lightyellow, colframe=orange!30, coltitle=black, fonttitle=\bfseries, boxrule=0.5mm, width=\textwidth]
\begin{minipage}{1.0\textwidth}
[AGREE] As a Choir Member of the Southampton Philharmonic Choir who has performed at the Guildhall, I can confirm that the Southampton Choral Society is indeed one of the choirs that have sung at the Guildhall. Our experiences have shown that we often share the stage with the Southampton Choral Society during joint performances with orchestras and other ensembles. So, I'm happy to confirm that Southampton Choral Society is a correct answer. 
\end{minipage}
\end{tcolorbox}
\end{tcolorbox}


\label{example:majority_discussion}

\newpage
\section{AI Usage Card}
{\sffamily
    \centering
    \tcbset{colback=white!10!white}
    \begin{tcolorbox}[
        title={\large \textbf{AI Usage Card} \hfill \makebox{\qrcode[height=1cm]{https://ai-cards.org}}},
        breakable,
        boxrule=0.7pt,
        width=\textwidth,
        center,
        before lower={\footnotesize{AI Usage Card v1.1 \hfill \url{https://ai-cards.org} \hfill \href{https://jpwahle.com/ai-cards-preprint}{PDF} | \href{https://jpwahle.com/cite/jcdl2023wahle.bib}{BibTeX}}},
        segmentation empty,
        halign lower=center,
        collower=black,
        coltitle=black, %
        colbacklower=gray!20, %
        colbacktitle=gray!20  %
        ]
        \vspace{-10pt}
        \footnotesize{
            \begin{longtable}{>
            {\raggedright\arraybackslash}p{.15\textwidth}>{\raggedright\arraybackslash}p{.25\textwidth}>{\raggedright\arraybackslash}p{.25\textwidth}>{\raggedright\arraybackslash}p{.25\textwidth}}
              
                {\color{LightBlue} \MakeUppercase{Project Details}} \newline 
                & {\color{LightBlue} \MakeUppercase{Project Name}} \newline Voting or Consensus? Decision-Making in Multi-Agent Debate
                & {\color{LightBlue} \MakeUppercase{Domain}} \newline Paper
                & {\color{LightBlue} \MakeUppercase{Key Application}} \newline Natural Language Processing
                
                \\
                
                {\color{LightBlue} \MakeUppercase{Contact(s)}}  
                & {\color{LightBlue} \MakeUppercase{Name(s)}} 
                & {\color{LightBlue} \MakeUppercase{Email(s)}} 
                & {\color{LightBlue} \MakeUppercase{Affiliation(s)}} 
                \\ & Lars Benedikt Kaesberg & larsbenedikt.kaesberg@uni-goettingen.de & University Göttingen
                
                \\
                
                {\color{LightBlue} \MakeUppercase{Model(s)}} 
                & {\color{LightBlue} \MakeUppercase{Model Name(s)}} 
                & {\color{LightBlue} \MakeUppercase{Version(s)}} 
                \\ & Llama & 3 70b, 3 8b
                \\ & ChatGPT & 4o, o1
                \\ & Github Copilot & latest

                \\
                \cmidrule{1-4}
                \\
                {\color{LightBlue} \MakeUppercase{Literature Review}} \newline 
                & {\color{LightBlue} \MakeUppercase{Finding literature}} \newline ChatGPT 
                & {\color{LightBlue} \MakeUppercase{Finding examples from known literature or adding literature for existing statements}} \newline ChatGPT  
                & {\color{gray} \MakeUppercase{Comparing literature}}  
                \\
                \cmidrule{2-4}
                \\        
                {\color{LightBlue} \MakeUppercase{Writing}} \newline    
                & {\color{LightBlue} \MakeUppercase{Generating new text based on instructions}} \newline Llama 
                & {\color{LightBlue} \MakeUppercase{Assisting in improving own content or Paraphrasing related work}} \newline ChatGPT 
                & {\color{gray} \MakeUppercase{Putting other works in perspective}}  
                \\
                \cmidrule{2-4}
                \\
                {\color{LightBlue} \MakeUppercase{Coding}} \newline 
                & {\color{LightBlue} \MakeUppercase{Generating new code based on descriptions or existing code}} \newline ChatGPT \newline Github Copilot 
                & {\color{LightBlue} \MakeUppercase{Refactoring and optimizing existing code}} \newline ChatGPT \newline Github Copilot 
                & {\color{gray} \MakeUppercase{Comparing aspects of existing code}}  
                \\
                \cmidrule{1-4}
                \\
        
                {\color{LightBlue} \MakeUppercase{Ethics}} \newline    
                & {\color{LightBlue} \MakeUppercase{Why did we use AI for this project?}} \newline Efficiency / Speed \newline Scalability \newline Expertise Access  
                & {\color{LightBlue} \MakeUppercase{What steps are we taking to mitigate errors of AI?}} \newline None
                & {\color{LightBlue} \MakeUppercase{What steps are we taking to minimize the chance of harm or inappropriate use of AI?}} \newline None

                \\
                \cmidrule{1-4}
                \\
            \end{longtable}
        
        

        \medskip
        
        \textbf{\color{LightBlue} \MakeUppercase{The corresponding authors verify and agree with the modifications or generations of their  used AI-generated content}}
        }
        
        \tcblower
    \end{tcolorbox}
}




\end{document}

