\usepackage{amsmath,amsfonts}
\usepackage{multirow}
\usepackage{multicol}
\usepackage{amsthm}
\usepackage{listings}
\usepackage{algorithm}
\usepackage{algpseudocode}
\usepackage{graphicx}
\usepackage{textcomp}
\usepackage{xcolor}
\usepackage{cleveref}
\usepackage{booktabs}
\usepackage{comment}
\usepackage{tabularx}
\usepackage{multirow}
\usepackage{bm}
\usepackage{url}
\usepackage{xspace}
\usepackage{pifont}
\usepackage{arydshln}
\usepackage{verbatim}
\usepackage[referable]{threeparttablex}
\usepackage{calc} 
%\usepackage{ulem}
\usepackage{color}
\usepackage{float}
\usepackage{stfloats}
\usepackage{subcaption}
% \usepackage[linesnumbered,ruled,vlined]{algorithm2e}
\usepackage[utf8]{inputenc}
\usepackage{booktabs} % For prettier tables
\usepackage{tabularx} % For tables with adjustable-width columns
% \usepackage[ruled,vlined]{algorithm2e}
% \usepackage[utf8]{inputenc}
% \usepackage[english]{babel}
%\usepackage{etoolbox}
%\appto\TPTnoteSettings{\footnotesize} % make table notes smaller
\usepackage{tikz}
\usetikzlibrary{shapes.geometric, arrows}
\thispagestyle{empty}

\tikzstyle{startstop} = [rectangle, rounded corners, minimum width=3cm, minimum height=1cm,text centered, draw=black]
\tikzstyle{io} = [trapezium, trapezium left angle=70, trapezium right angle=110, minimum width=3cm, minimum height=1cm, text centered, draw=black]
\tikzstyle{process} = [rectangle, minimum width=3cm, minimum height=1cm, text centered, draw=black]
\tikzstyle{arrow} = [thick,->,>=stealth]

\usepackage{filecontents}                                  % support to pgfplots
\usepackage{pgfplots}
\usepackage{pgfplotstable}
\usepackage{scalefnt}
\pgfplotsset{compat=newest}
\newcommand{\true}{\top}
\newcommand{\false}{\bot}
\newcommand{\Tr}{\mathit{Tr}}
\newcommand{\Prop}{\mathit{P}}
\newcommand{\tp}{\mathit{t_p}}
\newcommand{\tl}{\mathit{t_l}}
\newcommand{\Var}{{\mathit{Var}}}
\newcommand{\V}{{\mathit{Var}}}
\newcommand{\Vp}{{\mathit{Var'}}}
\newcommand{\Cnstr}{{\mathit{C}}}
\newcommand{\D}{{\mathit{D}}}
\newcommand{\I}{{\mathit{I}}}
%\newcommand{\U}{{\mathit{U}}}
\newcommand{\Up}{{\mathit{U'}}}
\newcommand{\rfmap}{{\mathit{r}}}
\newcommand{\Init}{\mathit{Init}}
\newcommand{\Trans}{\mathit{Tr}}
\newcommand{\Bad}{\mathit{Bad}}
\newcommand{\Expr}{\mathit{Expr}}
\newcommand{\Inv}{\mathit{Inv}}
\newcommand{\base}{\mathit{base}}
\newcommand{\tuple}[1]{\langle #1 \rangle}
\newcommand{\THeshold}{\mathit{Th}}
\newcommand{\tensor}[1]{\mathcal{#1}}      % define tensor command
\renewcommand{\vec}[1]{\boldsymbol{#1}}    % re-define vec command

\newcommand{\red}[1]{\textcolor{red}{#1}}
\newcommand{\blue}[1]{\textcolor{blue}{#1}}
\definecolor{USTgold}{RGB}{153,102,0}
\definecolor{USTyellow}{RGB}{204,153,0}
\definecolor{USTyellowlight}{RGB}{255,212,0}
\definecolor{USTorange}{RGB}{255,166,26}
\definecolor{USTpink}{RGB}{255,157,157}
\definecolor{USTblue}{RGB}{0,51,102}
\definecolor{USTmiddle}{RGB}{0,116,188}
\definecolor{USTlight}{RGB}{99,202,225}
\definecolor{USTgray}{RGB}{204,204,204}
\definecolor{USTred}{RGB}{237,27,47}
\definecolor{USTdarkred}{RGB}{124,35,72}

\definecolor{CUHKorange}{RGB}{244,106,18} %F47012
\definecolor{CUHKblue}{RGB}{0,111,190}    %006FBE
\definecolor{CUHKgreen}{RGB}{0,127,128}   %007F80
\definecolor{CUHKred}{RGB}{228,46,36}     %E42E24
\definecolor{CUHKyellow}{RGB}{198,148,34} %C69422
\definecolor{CUHKdark}{RGB}{114,44,114}   %722C72
\definecolor{CUHKmiddle}{RGB}{144,44,144} %902C90
\definecolor{CUHKlight}{RGB}{167,44,167} 


\newcommand{\deftitle}[0]{{\texttt{E-Syn}}\xspace}
\newcommand{\deftextEsyn}[0]{\mbox{\texttt{E-Syn}}\xspace}

\newtheorem{theorem}{Theorem}%[section]
\newtheorem{corollary}{Corollary}%[theorem]
\newtheorem{lemma}[theorem]{Lemma}

\iftrue
\def\BibTeX{{\rm B\kern-.05em{\sc i\kern-.025em b}\kern-.08em
    T\kern-.1667em\lower.7ex\hbox{E}\kern-.125emX}}

\setlength{\columnsep}{14pt}                               % set space between columns
\fi

\acmConference[DAC '24]{Design Automation Conference}{June 23--27, 2024}{San Francisco, CA}

\newcommand{\minisection}[1]{\vspace{.06in}\noindent{\textbf{#1}}}

% ==== page margin settings
\iftrue
\geometry{twoside=true, head=13pt,
	paperwidth=8.5in, paperheight=11in,
	includeheadfoot, columnsep=2pc,
	top=50pt, bottom=72pt, inner=54pt, outer=54pt,
	marginparwidth=2pc,heightrounded
}%
\fi

\iftrue
% === shrink page num
\usepackage{titlesec}
\titlespacing\section{2pt}{5pt plus 1pt minus 1pt}{0pt plus 1pt minus 1pt}
\titlespacing\subsection{2pt}{5pt plus 1pt minus 1pt}{0pt plus 1pt minus 1pt}
\titlespacing\subsubsection{2pt}{5pt plus 1pt minus 1pt}{2pt plus 1pt minus 1pt}
\usepackage[inline]{enumitem}
\setlist{leftmargin=5.08mm}
\fi

\iftrue
\setlength{\textfloatsep}{3pt plus 1pt minus 1pt}          % set space between float and text
\setlength{\floatsep}{3pt plus 1pt minus 1pt}              % set space between two floats
\setlength{\intextsep}{3pt plus 1pt minus 1pt}             % set space between text and float
\setlength{\columnsep}{16pt}                               % set space between columns
% ==== reduce space around equations
\setlength{\belowdisplayskip}{2pt} \setlength{\belowdisplayshortskip}{2pt}
\setlength{\abovedisplayskip}{2pt} \setlength{\abovedisplayshortskip}{2pt}
\fi

