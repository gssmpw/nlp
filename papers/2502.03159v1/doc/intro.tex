\section{Introduction}
\label{sec:intro}
Photonic Integrated Circuits (PICs) represent a groundbreaking advancement in chip design, harnessing the properties of light to enable faster data processing and greater energy efficiency. 
As the demand for high-performance computing and communication systems continues to rise, PICs have emerged as a critical solution to meet these requirements \cite{sun2015single,kitayama2019novel,lu2024empowering,xu2024large}. 
However, unlike the maturity of electronic design automation (EDA) tools, the development of photonic design automation (PDA) tools capable of supporting automated design pipelines for circuit simulation and layout remains at an early stage.
The design and layout of photonic circuits and components still heavily rely on manual input, which introduces significant inefficiencies. 
Photonic designs are inherently complex, often requiring repetitive, low-level coding for devices and connections. 
This process is time-intensive and prone to human error, particularly as the size and complexity of the designs increase.
Consequently, there is an urgent need for a comprehensive set of tools to fully automate photonic circuit design and layout processes. 
Advancements in large language models (LLMs) \cite{team2023gemini,TheC3,achiam2023gpt} offer a promising opportunity to address these challenges and accelerate the development of PDA solutions.

Recently, LLMs have demonstrated significant potential in automating code generation for hardware designs, which offers substantial support to engineers in designing and verifying these systems. 
RTLLM \cite{lu2024rtllm} introduced a benchmark framework comprising 30 designs spanning diverse complexities and scales for Verilog generation. 
Then VerilogEval \cite{liu2023verilogeval} proposed an extensive dataset of 156 problems and a robust testing procedure to facilitate the systematic evaluation of generated code. 
Beyond Verilog generation, SPICEPilot \cite{vungarala2024spicepilot} investigated the capabilities of LLMs in generating SPICE code. 
Another work ChatEDA \cite{wu2024chateda} demonstrated the ability to generate code for interacting with EDA tools using natural language instructions. 

Nevertheless, the application of LLM in photonic circuit design has been limited to a few works. 
Li \etal \cite{li2023english} utilized LLM generating FDTD code for simulating the photonic crystal surface emitting laser (PCSEL) structure and AI code for subsequent optimizations of the PCSEL model. 
However, their approach was not fully automated, as it relied heavily on human experts to iteratively specify requirements and debug errors.
Liu \etal \cite{liu2024towards} presented an automated framework that translates natural language prompts into Python code capable of generating GDSII files using an open-source library. 
However, their framework was tested on only seven simple device designs, leaving the performance and scalability of LLM-based solutions insufficiently evaluated. 
The absence of a reliable and automated testing framework, coupled with limited datasets and the lack of a standardized benchmark, significantly hinders both the development and fair evaluation of LLM solutions in PICs design. 
To address these challenges, a comprehensive benchmark is needed—one that encompasses a wide range of design problems, includes a reliable and automated evaluation framework to minimize testing variance, and clearly distinguishes the correctness and efficiency of solutions.

\begin{figure}[!tb]
    \centering
    \subfloat{%
      \includegraphics[width=0.44\textwidth]{fig/framework_ver.pdf}
      }
      \caption{PICBench framework that automated design generation and evaluation.}
      \label{fig:flow}
\end{figure}

% \begin{figure*}[!tb]
%     \centering
%     \subfloat{%
%       \includegraphics[width=0.98\textwidth]{fig/framework.pdf}
%       }
%       \caption{PICBench framework that automated design generation and evaluation. Users only need to provide their problem descriptions.}
%       \label{fig:flow}
% \end{figure*}

In this paper, we introduce PICBench, an open-source and comprehensive benchmark for PIC design using natural language to generate simulation-ready netlists.
The benchmark includes 24 meticulously crafted PIC design problems, covering a wide range of design complexities and scales. 
Each problem features clear descriptions and is accompanied by ground-truth designs created by human experts, serving as a golden result for evaluation.
Leveraging an open-source simulator SAX \cite{sax2023}, PICBench enables efficient and automated evaluation of any LLM-generated results.

Our contributions are summarized as follows:
\begin{itemize}
    \item We introduce PICBench, the first comprehensive open-source benchmark for PIC design using LLMs, comprising 24 carefully designed PIC design problems.
    \item We proposed a simple but efficient feedback-based method that further enhances the model’s proficiency in PIC design tasks.
    \item We exhaustively evaluated the state-of-the-art commercial LLMs with our benchmark on both syntax and functionality.
\end{itemize}
