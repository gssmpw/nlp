%%
%% This is file `sample-sigconf.tex',
%% generated with the docstrip utility.
%\documentclass[sigconf,natbib=true,anonymous=true,square,numbers]{acmart}

\documentclass[sigconf,natbib=true,square,numbers]{acmart}
% \documentclass[sigconf, anonymous]{acmart}
\usepackage{array}
\usepackage{xspace}
\usepackage{listings}
\usepackage{xcolor}
\usepackage{appendix}
\usepackage{booktabs}
\usepackage{amsmath}
\usepackage{url}
\newtheorem{definition}{Definition}
\newtheorem{theorem}{Theorem}
\newtheorem{observation}{Observation}
\newtheorem{lemma}{Lemma}
\lstdefinestyle{promptstyle}{
    backgroundcolor=\color{gray!10},
    basicstyle=\ttfamily\fontsize{7}{9}\selectfont,
    frame=single,
    breaklines=true
}


\clubpenalty = 10000
\widowpenalty = 10000



%\copyrightyear{2024} 
%\acmYear{2024} 
%\setcopyright{rightsretained} 
%\acmConference{Proceedings of the 2024 ACM SIGIR International Conference on the Theory of Information Retrieval}{July 13, 2024}{Washington DC, USA} \acmBooktitle{Proceedings of the 2024 ACM SIGIR International Conference on the Theory of Information Retrieval (ICTIR '24), July 13, 2024, Washington DC, USA} \acmDOI{10.1145/3664190.3672517} \acmISBN{979-8-4007-0681-3/24/07}


 \settopmatter{printacmref=true, printfolios=false}
%\settopmatter{printacmref=false}
\renewcommand\footnotetextcopyrightpermission[1]{}

\fancyhead{}


%
\setlength\unitlength{1mm}
\newcommand{\twodots}{\mathinner {\ldotp \ldotp}}
% bb font symbols
\newcommand{\Rho}{\mathrm{P}}
\newcommand{\Tau}{\mathrm{T}}

\newfont{\bbb}{msbm10 scaled 700}
\newcommand{\CCC}{\mbox{\bbb C}}

\newfont{\bb}{msbm10 scaled 1100}
\newcommand{\CC}{\mbox{\bb C}}
\newcommand{\PP}{\mbox{\bb P}}
\newcommand{\RR}{\mbox{\bb R}}
\newcommand{\QQ}{\mbox{\bb Q}}
\newcommand{\ZZ}{\mbox{\bb Z}}
\newcommand{\FF}{\mbox{\bb F}}
\newcommand{\GG}{\mbox{\bb G}}
\newcommand{\EE}{\mbox{\bb E}}
\newcommand{\NN}{\mbox{\bb N}}
\newcommand{\KK}{\mbox{\bb K}}
\newcommand{\HH}{\mbox{\bb H}}
\newcommand{\SSS}{\mbox{\bb S}}
\newcommand{\UU}{\mbox{\bb U}}
\newcommand{\VV}{\mbox{\bb V}}


\newcommand{\yy}{\mathbbm{y}}
\newcommand{\xx}{\mathbbm{x}}
\newcommand{\zz}{\mathbbm{z}}
\newcommand{\sss}{\mathbbm{s}}
\newcommand{\rr}{\mathbbm{r}}
\newcommand{\pp}{\mathbbm{p}}
\newcommand{\qq}{\mathbbm{q}}
\newcommand{\ww}{\mathbbm{w}}
\newcommand{\hh}{\mathbbm{h}}
\newcommand{\vvv}{\mathbbm{v}}

% Vectors

\newcommand{\av}{{\bf a}}
\newcommand{\bv}{{\bf b}}
\newcommand{\cv}{{\bf c}}
\newcommand{\dv}{{\bf d}}
\newcommand{\ev}{{\bf e}}
\newcommand{\fv}{{\bf f}}
\newcommand{\gv}{{\bf g}}
\newcommand{\hv}{{\bf h}}
\newcommand{\iv}{{\bf i}}
\newcommand{\jv}{{\bf j}}
\newcommand{\kv}{{\bf k}}
\newcommand{\lv}{{\bf l}}
\newcommand{\mv}{{\bf m}}
\newcommand{\nv}{{\bf n}}
\newcommand{\ov}{{\bf o}}
\newcommand{\pv}{{\bf p}}
\newcommand{\qv}{{\bf q}}
\newcommand{\rv}{{\bf r}}
\newcommand{\sv}{{\bf s}}
\newcommand{\tv}{{\bf t}}
\newcommand{\uv}{{\bf u}}
\newcommand{\wv}{{\bf w}}
\newcommand{\vv}{{\bf v}}
\newcommand{\xv}{{\bf x}}
\newcommand{\yv}{{\bf y}}
\newcommand{\zv}{{\bf z}}
\newcommand{\zerov}{{\bf 0}}
\newcommand{\onev}{{\bf 1}}

% Matrices

\newcommand{\Am}{{\bf A}}
\newcommand{\Bm}{{\bf B}}
\newcommand{\Cm}{{\bf C}}
\newcommand{\Dm}{{\bf D}}
\newcommand{\Em}{{\bf E}}
\newcommand{\Fm}{{\bf F}}
\newcommand{\Gm}{{\bf G}}
\newcommand{\Hm}{{\bf H}}
\newcommand{\Id}{{\bf I}}
\newcommand{\Jm}{{\bf J}}
\newcommand{\Km}{{\bf K}}
\newcommand{\Lm}{{\bf L}}
\newcommand{\Mm}{{\bf M}}
\newcommand{\Nm}{{\bf N}}
\newcommand{\Om}{{\bf O}}
\newcommand{\Pm}{{\bf P}}
\newcommand{\Qm}{{\bf Q}}
\newcommand{\Rm}{{\bf R}}
\newcommand{\Sm}{{\bf S}}
\newcommand{\Tm}{{\bf T}}
\newcommand{\Um}{{\bf U}}
\newcommand{\Wm}{{\bf W}}
\newcommand{\Vm}{{\bf V}}
\newcommand{\Xm}{{\bf X}}
\newcommand{\Ym}{{\bf Y}}
\newcommand{\Zm}{{\bf Z}}

% Calligraphic

\newcommand{\Ac}{{\cal A}}
\newcommand{\Bc}{{\cal B}}
\newcommand{\Cc}{{\cal C}}
\newcommand{\Dc}{{\cal D}}
\newcommand{\Ec}{{\cal E}}
\newcommand{\Fc}{{\cal F}}
\newcommand{\Gc}{{\cal G}}
\newcommand{\Hc}{{\cal H}}
\newcommand{\Ic}{{\cal I}}
\newcommand{\Jc}{{\cal J}}
\newcommand{\Kc}{{\cal K}}
\newcommand{\Lc}{{\cal L}}
\newcommand{\Mc}{{\cal M}}
\newcommand{\Nc}{{\cal N}}
\newcommand{\nc}{{\cal n}}
\newcommand{\Oc}{{\cal O}}
\newcommand{\Pc}{{\cal P}}
\newcommand{\Qc}{{\cal Q}}
\newcommand{\Rc}{{\cal R}}
\newcommand{\Sc}{{\cal S}}
\newcommand{\Tc}{{\cal T}}
\newcommand{\Uc}{{\cal U}}
\newcommand{\Wc}{{\cal W}}
\newcommand{\Vc}{{\cal V}}
\newcommand{\Xc}{{\cal X}}
\newcommand{\Yc}{{\cal Y}}
\newcommand{\Zc}{{\cal Z}}

% Bold greek letters

\newcommand{\alphav}{\hbox{\boldmath$\alpha$}}
\newcommand{\betav}{\hbox{\boldmath$\beta$}}
\newcommand{\gammav}{\hbox{\boldmath$\gamma$}}
\newcommand{\deltav}{\hbox{\boldmath$\delta$}}
\newcommand{\etav}{\hbox{\boldmath$\eta$}}
\newcommand{\lambdav}{\hbox{\boldmath$\lambda$}}
\newcommand{\epsilonv}{\hbox{\boldmath$\epsilon$}}
\newcommand{\nuv}{\hbox{\boldmath$\nu$}}
\newcommand{\muv}{\hbox{\boldmath$\mu$}}
\newcommand{\zetav}{\hbox{\boldmath$\zeta$}}
\newcommand{\phiv}{\hbox{\boldmath$\phi$}}
\newcommand{\psiv}{\hbox{\boldmath$\psi$}}
\newcommand{\thetav}{\hbox{\boldmath$\theta$}}
\newcommand{\tauv}{\hbox{\boldmath$\tau$}}
\newcommand{\omegav}{\hbox{\boldmath$\omega$}}
\newcommand{\xiv}{\hbox{\boldmath$\xi$}}
\newcommand{\sigmav}{\hbox{\boldmath$\sigma$}}
\newcommand{\piv}{\hbox{\boldmath$\pi$}}
\newcommand{\rhov}{\hbox{\boldmath$\rho$}}
\newcommand{\upsilonv}{\hbox{\boldmath$\upsilon$}}

\newcommand{\Gammam}{\hbox{\boldmath$\Gamma$}}
\newcommand{\Lambdam}{\hbox{\boldmath$\Lambda$}}
\newcommand{\Deltam}{\hbox{\boldmath$\Delta$}}
\newcommand{\Sigmam}{\hbox{\boldmath$\Sigma$}}
\newcommand{\Phim}{\hbox{\boldmath$\Phi$}}
\newcommand{\Pim}{\hbox{\boldmath$\Pi$}}
\newcommand{\Psim}{\hbox{\boldmath$\Psi$}}
\newcommand{\Thetam}{\hbox{\boldmath$\Theta$}}
\newcommand{\Omegam}{\hbox{\boldmath$\Omega$}}
\newcommand{\Xim}{\hbox{\boldmath$\Xi$}}


% Sans Serif small case

\newcommand{\Gsf}{{\sf G}}

\newcommand{\asf}{{\sf a}}
\newcommand{\bsf}{{\sf b}}
\newcommand{\csf}{{\sf c}}
\newcommand{\dsf}{{\sf d}}
\newcommand{\esf}{{\sf e}}
\newcommand{\fsf}{{\sf f}}
\newcommand{\gsf}{{\sf g}}
\newcommand{\hsf}{{\sf h}}
\newcommand{\isf}{{\sf i}}
\newcommand{\jsf}{{\sf j}}
\newcommand{\ksf}{{\sf k}}
\newcommand{\lsf}{{\sf l}}
\newcommand{\msf}{{\sf m}}
\newcommand{\nsf}{{\sf n}}
\newcommand{\osf}{{\sf o}}
\newcommand{\psf}{{\sf p}}
\newcommand{\qsf}{{\sf q}}
\newcommand{\rsf}{{\sf r}}
\newcommand{\ssf}{{\sf s}}
\newcommand{\tsf}{{\sf t}}
\newcommand{\usf}{{\sf u}}
\newcommand{\wsf}{{\sf w}}
\newcommand{\vsf}{{\sf v}}
\newcommand{\xsf}{{\sf x}}
\newcommand{\ysf}{{\sf y}}
\newcommand{\zsf}{{\sf z}}


% mixed symbols

\newcommand{\sinc}{{\hbox{sinc}}}
\newcommand{\diag}{{\hbox{diag}}}
\renewcommand{\det}{{\hbox{det}}}
\newcommand{\trace}{{\hbox{tr}}}
\newcommand{\sign}{{\hbox{sign}}}
\renewcommand{\arg}{{\hbox{arg}}}
\newcommand{\var}{{\hbox{var}}}
\newcommand{\cov}{{\hbox{cov}}}
\newcommand{\Ei}{{\rm E}_{\rm i}}
\renewcommand{\Re}{{\rm Re}}
\renewcommand{\Im}{{\rm Im}}
\newcommand{\eqdef}{\stackrel{\Delta}{=}}
\newcommand{\defines}{{\,\,\stackrel{\scriptscriptstyle \bigtriangleup}{=}\,\,}}
\newcommand{\<}{\left\langle}
\renewcommand{\>}{\right\rangle}
\newcommand{\herm}{{\sf H}}
\newcommand{\trasp}{{\sf T}}
\newcommand{\transp}{{\sf T}}
\renewcommand{\vec}{{\rm vec}}
\newcommand{\Psf}{{\sf P}}
\newcommand{\SINR}{{\sf SINR}}
\newcommand{\SNR}{{\sf SNR}}
\newcommand{\MMSE}{{\sf MMSE}}
\newcommand{\REF}{{\RED [REF]}}

% Markov chain
\usepackage{stmaryrd} % for \mkv 
\newcommand{\mkv}{-\!\!\!\!\minuso\!\!\!\!-}

% Colors

\newcommand{\RED}{\color[rgb]{1.00,0.10,0.10}}
\newcommand{\BLUE}{\color[rgb]{0,0,0.90}}
\newcommand{\GREEN}{\color[rgb]{0,0.80,0.20}}

%%%%%%%%%%%%%%%%%%%%%%%%%%%%%%%%%%%%%%%%%%
\usepackage{hyperref}
\hypersetup{
    bookmarks=true,         % show bookmarks bar?
    unicode=false,          % non-Latin characters in AcrobatÕs bookmarks
    pdftoolbar=true,        % show AcrobatÕs toolbar?
    pdfmenubar=true,        % show AcrobatÕs menu?
    pdffitwindow=false,     % window fit to page when opened
    pdfstartview={FitH},    % fits the width of the page to the window
%    pdftitle={My title},    % title
%    pdfauthor={Author},     % author
%    pdfsubject={Subject},   % subject of the document
%    pdfcreator={Creator},   % creator of the document
%    pdfproducer={Producer}, % producer of the document
%    pdfkeywords={keyword1} {key2} {key3}, % list of keywords
    pdfnewwindow=true,      % links in new window
    colorlinks=true,       % false: boxed links; true: colored links
    linkcolor=red,          % color of internal links (change box color with linkbordercolor)
    citecolor=green,        % color of links to bibliography
    filecolor=blue,      % color of file links
    urlcolor=blue           % color of external links
}
%%%%%%%%%%%%%%%%%%%%%%%%%%%%%%%%%%%%%%%%%%%


\AtBeginDocument{%
  \providecommand\BibTeX{{%
    \normalfont B\kern-0.5em{\scshape i\kern-0.25em b}\kern-0.8em\TeX}}}

% % \begin{CCSXML}
% <ccs2012>
% <concept>
% <concept_id>10002951</concept_id>
% <concept_desc>Information systems</concept_desc>
% <concept_significance>500</concept_significance>
% </concept>
% <concept>
% <concept_id>10002951.10003317</concept_id>
% <concept_desc>Information systems~Information retrieval</concept_desc>
% <concept_significance>500</concept_significance>
% </concept>
% <concept>
% <concept_id>10002951.10003317.10003338</concept_id>
% <concept_desc>Information systems~Retrieval models and ranking</concept_desc>
% <concept_significance>500</concept_significance>
% </concept>
% </ccs2012>
% \end{CCSXML}

% \ccsdesc[500]{Information systems}
% \ccsdesc[500]{Information systems~Information retrieval}
% \ccsdesc[500]{Information systems~Retrieval models and ranking}


% \keywords{competitive search, ranking-incentivized manipulations, adversarial retrieval}


\begin{CCSXML}
<ccs2012>
   <concept>
       <concept_id>10002951</concept_id>
       <concept_desc>Information systems</concept_desc>
       <concept_significance>500</concept_significance>
       </concept>
   <concept>
       <concept_id>10002951.10003317</concept_id>
       <concept_desc>Information systems~Information retrieval</concept_desc>
       <concept_significance>500</concept_significance>
       </concept>
   <concept>
       <concept_id>10002951.10003317.10003365</concept_id>
       <concept_desc>Information systems~Search engine architectures and scalability</concept_desc>
       <concept_significance>500</concept_significance>
       </concept>
   <concept>
       <concept_id>10002951.10003317.10003365.10010850</concept_id>
       <concept_desc>Information systems~Adversarial retrieval</concept_desc>
       <concept_significance>500</concept_significance>
       </concept>
 </ccs2012>
\end{CCSXML}

\ccsdesc[500]{Information systems}
\ccsdesc[500]{Information systems~Information retrieval}
\ccsdesc[500]{Information systems~Search engine architectures and scalability}
\ccsdesc[500]{Information systems~Adversarial retrieval}

\keywords{competitive search, LLM}

\makeatletter
\gdef\@copyrightpermission{
 \begin{minipage}{0.3\columnwidth}
 \href{https://creativecommons.org/licenses/by/4.0/}{\includegraphics[width=0.90\textwidth]{figs/4ACM-CC-by-88x31.eps}}
 \end{minipage}\hfill
 \begin{minipage}{0.7\columnwidth}
 \href{https://creativecommons.org/licenses/by/4.0/}{This work is licensed under a Creative Commons 
Attribution International 4.0 License.}
 \end{minipage}
 \vspace{5pt}
}
\makeatother

\begin{document}

%%
%% The "title" command has an optional parameter,
%% allowing the author to define a "short title" to be used in page headers.
%\ok{Just a try, please revise}
% \title{Prompt-Based Document Modifications In Ranking Competitions}
\title{White Hat Search Engine Optimization using Large Language Models}
\settopmatter{printacmref=false, printfolios=false}

%%
%% The "author" command and its associated commands are used to define
%% the authors and their affiliations.
%% Of note is the shared affiliation of the first two authors, and the
%% "authornote" and "authornotemark" commands
%% used to denote shared contribution to the research.


\author{Niv Bardas}
\email{niv.b@campus.technion.ac.il}
\affiliation{%
  \institution{Technion}
  \city{Haifa}
  \country{Israel}
}

\author{Tommy Mordo}
\email{tommymordo@technion.ac.il}
\affiliation{
  \institution{Technion}
  \city{Haifa}
  \country{Israel}
}

\author{Oren Kurland}
\email{kurland@technion.ac.il}
\affiliation{%
  \institution{Technion}
  \city{Haifa}
  \country{Israel}
}

\author{Moshe Tennenholtz}
\email{moshet@technion.ac.il}
\affiliation{%
  \institution{Technion}
  \city{Haifa}
  \country{Israel}
}

\author{Gal Zur}
\email{gal.zur@campus.technion.ac.il}
\affiliation{%
  \institution{Technion}
  \city{Haifa}
  \country{Israel}
}



%%
%% By default, the full list of authors will be used in the page
%% headers. Often, this list is too long, and will overlap
%% other information printed in the page headers. This command allows
%% the author to define a more concise list
%% of authors' names for this purpose.
\renewcommand{\shortauthors}{Bardas et al/.}

%%
%% The abstract is a short summary of the work to be presented in the
%% article.
\begin{abstract}  
Test time scaling is currently one of the most active research areas that shows promise after training time scaling has reached its limits.
Deep-thinking (DT) models are a class of recurrent models that can perform easy-to-hard generalization by assigning more compute to harder test samples.
However, due to their inability to determine the complexity of a test sample, DT models have to use a large amount of computation for both easy and hard test samples.
Excessive test time computation is wasteful and can cause the ``overthinking'' problem where more test time computation leads to worse results.
In this paper, we introduce a test time training method for determining the optimal amount of computation needed for each sample during test time.
We also propose Conv-LiGRU, a novel recurrent architecture for efficient and robust visual reasoning. 
Extensive experiments demonstrate that Conv-LiGRU is more stable than DT, effectively mitigates the ``overthinking'' phenomenon, and achieves superior accuracy.
\end{abstract}  
\maketitle




\section{Introduction}


\begin{figure}[t]
\centering
\includegraphics[width=0.6\columnwidth]{figures/evaluation_desiderata_V5.pdf}
\vspace{-0.5cm}
\caption{\systemName is a platform for conducting realistic evaluations of code LLMs, collecting human preferences of coding models with real users, real tasks, and in realistic environments, aimed at addressing the limitations of existing evaluations.
}
\label{fig:motivation}
\end{figure}

\begin{figure*}[t]
\centering
\includegraphics[width=\textwidth]{figures/system_design_v2.png}
\caption{We introduce \systemName, a VSCode extension to collect human preferences of code directly in a developer's IDE. \systemName enables developers to use code completions from various models. The system comprises a) the interface in the user's IDE which presents paired completions to users (left), b) a sampling strategy that picks model pairs to reduce latency (right, top), and c) a prompting scheme that allows diverse LLMs to perform code completions with high fidelity.
Users can select between the top completion (green box) using \texttt{tab} or the bottom completion (blue box) using \texttt{shift+tab}.}
\label{fig:overview}
\end{figure*}

As model capabilities improve, large language models (LLMs) are increasingly integrated into user environments and workflows.
For example, software developers code with AI in integrated developer environments (IDEs)~\citep{peng2023impact}, doctors rely on notes generated through ambient listening~\citep{oberst2024science}, and lawyers consider case evidence identified by electronic discovery systems~\citep{yang2024beyond}.
Increasing deployment of models in productivity tools demands evaluation that more closely reflects real-world circumstances~\citep{hutchinson2022evaluation, saxon2024benchmarks, kapoor2024ai}.
While newer benchmarks and live platforms incorporate human feedback to capture real-world usage, they almost exclusively focus on evaluating LLMs in chat conversations~\citep{zheng2023judging,dubois2023alpacafarm,chiang2024chatbot, kirk2024the}.
Model evaluation must move beyond chat-based interactions and into specialized user environments.



 

In this work, we focus on evaluating LLM-based coding assistants. 
Despite the popularity of these tools---millions of developers use Github Copilot~\citep{Copilot}---existing
evaluations of the coding capabilities of new models exhibit multiple limitations (Figure~\ref{fig:motivation}, bottom).
Traditional ML benchmarks evaluate LLM capabilities by measuring how well a model can complete static, interview-style coding tasks~\citep{chen2021evaluating,austin2021program,jain2024livecodebench, white2024livebench} and lack \emph{real users}. 
User studies recruit real users to evaluate the effectiveness of LLMs as coding assistants, but are often limited to simple programming tasks as opposed to \emph{real tasks}~\citep{vaithilingam2022expectation,ross2023programmer, mozannar2024realhumaneval}.
Recent efforts to collect human feedback such as Chatbot Arena~\citep{chiang2024chatbot} are still removed from a \emph{realistic environment}, resulting in users and data that deviate from typical software development processes.
We introduce \systemName to address these limitations (Figure~\ref{fig:motivation}, top), and we describe our three main contributions below.


\textbf{We deploy \systemName in-the-wild to collect human preferences on code.} 
\systemName is a Visual Studio Code extension, collecting preferences directly in a developer's IDE within their actual workflow (Figure~\ref{fig:overview}).
\systemName provides developers with code completions, akin to the type of support provided by Github Copilot~\citep{Copilot}. 
Over the past 3 months, \systemName has served over~\completions suggestions from 10 state-of-the-art LLMs, 
gathering \sampleCount~votes from \userCount~users.
To collect user preferences,
\systemName presents a novel interface that shows users paired code completions from two different LLMs, which are determined based on a sampling strategy that aims to 
mitigate latency while preserving coverage across model comparisons.
Additionally, we devise a prompting scheme that allows a diverse set of models to perform code completions with high fidelity.
See Section~\ref{sec:system} and Section~\ref{sec:deployment} for details about system design and deployment respectively.



\textbf{We construct a leaderboard of user preferences and find notable differences from existing static benchmarks and human preference leaderboards.}
In general, we observe that smaller models seem to overperform in static benchmarks compared to our leaderboard, while performance among larger models is mixed (Section~\ref{sec:leaderboard_calculation}).
We attribute these differences to the fact that \systemName is exposed to users and tasks that differ drastically from code evaluations in the past. 
Our data spans 103 programming languages and 24 natural languages as well as a variety of real-world applications and code structures, while static benchmarks tend to focus on a specific programming and natural language and task (e.g. coding competition problems).
Additionally, while all of \systemName interactions contain code contexts and the majority involve infilling tasks, a much smaller fraction of Chatbot Arena's coding tasks contain code context, with infilling tasks appearing even more rarely. 
We analyze our data in depth in Section~\ref{subsec:comparison}.



\textbf{We derive new insights into user preferences of code by analyzing \systemName's diverse and distinct data distribution.}
We compare user preferences across different stratifications of input data (e.g., common versus rare languages) and observe which affect observed preferences most (Section~\ref{sec:analysis}).
For example, while user preferences stay relatively consistent across various programming languages, they differ drastically between different task categories (e.g. frontend/backend versus algorithm design).
We also observe variations in user preference due to different features related to code structure 
(e.g., context length and completion patterns).
We open-source \systemName and release a curated subset of code contexts.
Altogether, our results highlight the necessity of model evaluation in realistic and domain-specific settings.







\section{Document modification methods}
\label{sec:bots}

%\shared \contextualized

We present methods utilizing a large language model (LLM) that
modify a document so as to have it highly ranked in the next ranking induced
for a query by an undisclosed ranking function. We assume that past rankings for
the query and other queries can be observed. We post two additional
goals for the modification: having the resultant document of high
quality (in terms of discourse) and as ``faithful'' as possible to the original document in terms of the
content it contains \cite{goren2020ranking}.

The modification methods are based on prompting an LLM. The prompts have two parts: (i) a general shared part which
describes the task (document modification for improved ranking) with a
directive to maintain similarity to the original document (i.e.,
faithfulness); and, (ii) context-specific part, specific to a prompt, which provides information about past rankings. We assume that the LLM produces high quality content in terms of discourse. The general shared part is:

%with information about rankings
%induced by the undisclosed ranking function. The prompts include a
%specific directive to keep the modified document faithful to
%$\doc$. We assume that LLM-based modification results in high quality
%content in terms of discourse (specifically, coherence).
%





%A repeated ranking competition between document authors for query
%$\query$ is executed for several rounds
%\cite{raifer2017information,goren2020ranking}. In each round, all
%document authors can observe previous rankings induced for $\query$%
%(and other queries) by the undicsclosed ranking function. Then they
%might modify their documents to have them highly ranked in the next
%round.

%Our document modification strategies use large language models (LLMs)
%with a three fold objective (cf. \cite{goren2020ranking}): having the
%modified document (i) attain high ranking in the next round, (ii) be
%``faithful'' (in terms of content) to the original document, and (iii)
%be of high quality (in terms of discourse and coherence). We use prompts composed of a ``system part'', shared by all prompts, %and a ``user part'' which is context-specific. The system part specifies the task and the need to maintain faithfulness to the original document\footnote{We assumed the LLM generates content of high quality in terms of discourse and coherence.}:



%in-context (implicit) learning of
%the undisclosed ranking function with LLMs. Specifically, we present
%prompting templates

%Our goal is to devise ranking-incentivized document modification
%methods. The methods take as input a document, a query $\query$, and
%past rankings induced for $\query$ or other queries. The output is a
%modified document formed to attain as high as possible ranking
%for $\query$. In addition to the ranking objective, we aim to
%have the modified document ``faithful'' (in terms of content) to the original document, and of high quality
%\cite{goren2020ranking}.

%The document modification methods we present use large language models (LLMs) with various
%prompts that differ by the type of context they provide; e.g., about
%rankings induced by the ranking function. All prompts share the following generic part:

%In this Section, we introduce our novel approach for content modification aimed to promote the ranking of documents in a ranking competition with a query {\query} and an undisclosed ranking function.
%We introduce a novel approach to that introduced by Goren et al. \cite{goren2020ranking}, which relies on replacing a passage to get a modified document. We use a prompt-driven approach, and provide examples of previous documents and their corresponding rankings. Each prompt defines an \bt{} that strategically modify his documents in order to promote his ranking. Similar to Goren et al. \cite{goren2020ranking}, our approach adheres to three desiderata for content modification: ranking promotion, quality preservation and faithfulness to the original content.

%To formalize our prompt-driven method, we define a structured family of prompts, each consisting of two key components. The first component (referred to as the \textbf{System prompt}) follows a consistent format across all agents. it includes: (i) the instruction given to the LLM about the ranking competition, (ii) the query associated with the game, and (iii) the current document of the agent at a specific round of the game. The second component (referred to as the \textbf{User prompt}) provides contextualized information tailored to past rankings of previous rounds and other games, and the corresponding documents. By incorporating such contextual information, the User prompt facilitates adaptive strategies for content modification.
%This prompt architecture enables us to systematically evaluate the impact of varying contextual information in the User prompt on multiple metrics aligned with the desiderata.


% Our approach to constructing the \bt s for generating texts in our "white-hat ranking-incentivized
% modifications" study was intentionally method-agnostic, meaning we kept the almost the entire prompt identical across methods, with the sole exception being the “context” element. This flexible design allowed us to focus on assessing the impact of contextual variations on text generation. Specifically, while each \bt{} employed a uniform prompt structure to maintain consistency across the study, we introduced distinct contextual content in targeted sections to explore how different contexts influenced rank promotion outcomes. This approach ensured that any observed effects could be attributed more confidently to the context structure rather than to external factors.

%\myparagraph{Prompt Template} The structure of the prompt provided to the LLM can be illustrated as:


\definecolor{titlecolor}{rgb}{0.9, 0.5, 0.1}
\definecolor{anscolor}{rgb}{0.2, 0.5, 0.8}
\definecolor{labelcolor}{HTML}{48a07e}
\begin{table*}[h]
	\centering
	
 % \vspace{-0.2cm}
	
	\begin{center}
		\begin{tikzpicture}[
				chatbox_inner/.style={rectangle, rounded corners, opacity=0, text opacity=1, font=\sffamily\scriptsize, text width=5in, text height=9pt, inner xsep=6pt, inner ysep=6pt},
				chatbox_prompt_inner/.style={chatbox_inner, align=flush left, xshift=0pt, text height=11pt},
				chatbox_user_inner/.style={chatbox_inner, align=flush left, xshift=0pt},
				chatbox_gpt_inner/.style={chatbox_inner, align=flush left, xshift=0pt},
				chatbox/.style={chatbox_inner, draw=black!25, fill=gray!7, opacity=1, text opacity=0},
				chatbox_prompt/.style={chatbox, align=flush left, fill=gray!1.5, draw=black!30, text height=10pt},
				chatbox_user/.style={chatbox, align=flush left},
				chatbox_gpt/.style={chatbox, align=flush left},
				chatbox2/.style={chatbox_gpt, fill=green!25},
				chatbox3/.style={chatbox_gpt, fill=red!20, draw=black!20},
				chatbox4/.style={chatbox_gpt, fill=yellow!30},
				labelbox/.style={rectangle, rounded corners, draw=black!50, font=\sffamily\scriptsize\bfseries, fill=gray!5, inner sep=3pt},
			]
											
			\node[chatbox_user] (q1) {
				\textbf{System prompt}
				\newline
				\newline
				You are a helpful and precise assistant for segmenting and labeling sentences. We would like to request your help on curating a dataset for entity-level hallucination detection.
				\newline \newline
                We will give you a machine generated biography and a list of checked facts about the biography. Each fact consists of a sentence and a label (True/False). Please do the following process. First, breaking down the biography into words. Second, by referring to the provided list of facts, merging some broken down words in the previous step to form meaningful entities. For example, ``strategic thinking'' should be one entity instead of two. Third, according to the labels in the list of facts, labeling each entity as True or False. Specifically, for facts that share a similar sentence structure (\eg, \textit{``He was born on Mach 9, 1941.''} (\texttt{True}) and \textit{``He was born in Ramos Mejia.''} (\texttt{False})), please first assign labels to entities that differ across atomic facts. For example, first labeling ``Mach 9, 1941'' (\texttt{True}) and ``Ramos Mejia'' (\texttt{False}) in the above case. For those entities that are the same across atomic facts (\eg, ``was born'') or are neutral (\eg, ``he,'' ``in,'' and ``on''), please label them as \texttt{True}. For the cases that there is no atomic fact that shares the same sentence structure, please identify the most informative entities in the sentence and label them with the same label as the atomic fact while treating the rest of the entities as \texttt{True}. In the end, output the entities and labels in the following format:
                \begin{itemize}[nosep]
                    \item Entity 1 (Label 1)
                    \item Entity 2 (Label 2)
                    \item ...
                    \item Entity N (Label N)
                \end{itemize}
                % \newline \newline
                Here are two examples:
                \newline\newline
                \textbf{[Example 1]}
                \newline
                [The start of the biography]
                \newline
                \textcolor{titlecolor}{Marianne McAndrew is an American actress and singer, born on November 21, 1942, in Cleveland, Ohio. She began her acting career in the late 1960s, appearing in various television shows and films.}
                \newline
                [The end of the biography]
                \newline \newline
                [The start of the list of checked facts]
                \newline
                \textcolor{anscolor}{[Marianne McAndrew is an American. (False); Marianne McAndrew is an actress. (True); Marianne McAndrew is a singer. (False); Marianne McAndrew was born on November 21, 1942. (False); Marianne McAndrew was born in Cleveland, Ohio. (False); She began her acting career in the late 1960s. (True); She has appeared in various television shows. (True); She has appeared in various films. (True)]}
                \newline
                [The end of the list of checked facts]
                \newline \newline
                [The start of the ideal output]
                \newline
                \textcolor{labelcolor}{[Marianne McAndrew (True); is (True); an (True); American (False); actress (True); and (True); singer (False); , (True); born (True); on (True); November 21, 1942 (False); , (True); in (True); Cleveland, Ohio (False); . (True); She (True); began (True); her (True); acting career (True); in (True); the late 1960s (True); , (True); appearing (True); in (True); various (True); television shows (True); and (True); films (True); . (True)]}
                \newline
                [The end of the ideal output]
				\newline \newline
                \textbf{[Example 2]}
                \newline
                [The start of the biography]
                \newline
                \textcolor{titlecolor}{Doug Sheehan is an American actor who was born on April 27, 1949, in Santa Monica, California. He is best known for his roles in soap operas, including his portrayal of Joe Kelly on ``General Hospital'' and Ben Gibson on ``Knots Landing.''}
                \newline
                [The end of the biography]
                \newline \newline
                [The start of the list of checked facts]
                \newline
                \textcolor{anscolor}{[Doug Sheehan is an American. (True); Doug Sheehan is an actor. (True); Doug Sheehan was born on April 27, 1949. (True); Doug Sheehan was born in Santa Monica, California. (False); He is best known for his roles in soap operas. (True); He portrayed Joe Kelly. (True); Joe Kelly was in General Hospital. (True); General Hospital is a soap opera. (True); He portrayed Ben Gibson. (True); Ben Gibson was in Knots Landing. (True); Knots Landing is a soap opera. (True)]}
                \newline
                [The end of the list of checked facts]
                \newline \newline
                [The start of the ideal output]
                \newline
                \textcolor{labelcolor}{[Doug Sheehan (True); is (True); an (True); American (True); actor (True); who (True); was born (True); on (True); April 27, 1949 (True); in (True); Santa Monica, California (False); . (True); He (True); is (True); best known (True); for (True); his roles in soap operas (True); , (True); including (True); in (True); his portrayal (True); of (True); Joe Kelly (True); on (True); ``General Hospital'' (True); and (True); Ben Gibson (True); on (True); ``Knots Landing.'' (True)]}
                \newline
                [The end of the ideal output]
				\newline \newline
				\textbf{User prompt}
				\newline
				\newline
				[The start of the biography]
				\newline
				\textcolor{magenta}{\texttt{\{BIOGRAPHY\}}}
				\newline
				[The ebd of the biography]
				\newline \newline
				[The start of the list of checked facts]
				\newline
				\textcolor{magenta}{\texttt{\{LIST OF CHECKED FACTS\}}}
				\newline
				[The end of the list of checked facts]
			};
			\node[chatbox_user_inner] (q1_text) at (q1) {
				\textbf{System prompt}
				\newline
				\newline
				You are a helpful and precise assistant for segmenting and labeling sentences. We would like to request your help on curating a dataset for entity-level hallucination detection.
				\newline \newline
                We will give you a machine generated biography and a list of checked facts about the biography. Each fact consists of a sentence and a label (True/False). Please do the following process. First, breaking down the biography into words. Second, by referring to the provided list of facts, merging some broken down words in the previous step to form meaningful entities. For example, ``strategic thinking'' should be one entity instead of two. Third, according to the labels in the list of facts, labeling each entity as True or False. Specifically, for facts that share a similar sentence structure (\eg, \textit{``He was born on Mach 9, 1941.''} (\texttt{True}) and \textit{``He was born in Ramos Mejia.''} (\texttt{False})), please first assign labels to entities that differ across atomic facts. For example, first labeling ``Mach 9, 1941'' (\texttt{True}) and ``Ramos Mejia'' (\texttt{False}) in the above case. For those entities that are the same across atomic facts (\eg, ``was born'') or are neutral (\eg, ``he,'' ``in,'' and ``on''), please label them as \texttt{True}. For the cases that there is no atomic fact that shares the same sentence structure, please identify the most informative entities in the sentence and label them with the same label as the atomic fact while treating the rest of the entities as \texttt{True}. In the end, output the entities and labels in the following format:
                \begin{itemize}[nosep]
                    \item Entity 1 (Label 1)
                    \item Entity 2 (Label 2)
                    \item ...
                    \item Entity N (Label N)
                \end{itemize}
                % \newline \newline
                Here are two examples:
                \newline\newline
                \textbf{[Example 1]}
                \newline
                [The start of the biography]
                \newline
                \textcolor{titlecolor}{Marianne McAndrew is an American actress and singer, born on November 21, 1942, in Cleveland, Ohio. She began her acting career in the late 1960s, appearing in various television shows and films.}
                \newline
                [The end of the biography]
                \newline \newline
                [The start of the list of checked facts]
                \newline
                \textcolor{anscolor}{[Marianne McAndrew is an American. (False); Marianne McAndrew is an actress. (True); Marianne McAndrew is a singer. (False); Marianne McAndrew was born on November 21, 1942. (False); Marianne McAndrew was born in Cleveland, Ohio. (False); She began her acting career in the late 1960s. (True); She has appeared in various television shows. (True); She has appeared in various films. (True)]}
                \newline
                [The end of the list of checked facts]
                \newline \newline
                [The start of the ideal output]
                \newline
                \textcolor{labelcolor}{[Marianne McAndrew (True); is (True); an (True); American (False); actress (True); and (True); singer (False); , (True); born (True); on (True); November 21, 1942 (False); , (True); in (True); Cleveland, Ohio (False); . (True); She (True); began (True); her (True); acting career (True); in (True); the late 1960s (True); , (True); appearing (True); in (True); various (True); television shows (True); and (True); films (True); . (True)]}
                \newline
                [The end of the ideal output]
				\newline \newline
                \textbf{[Example 2]}
                \newline
                [The start of the biography]
                \newline
                \textcolor{titlecolor}{Doug Sheehan is an American actor who was born on April 27, 1949, in Santa Monica, California. He is best known for his roles in soap operas, including his portrayal of Joe Kelly on ``General Hospital'' and Ben Gibson on ``Knots Landing.''}
                \newline
                [The end of the biography]
                \newline \newline
                [The start of the list of checked facts]
                \newline
                \textcolor{anscolor}{[Doug Sheehan is an American. (True); Doug Sheehan is an actor. (True); Doug Sheehan was born on April 27, 1949. (True); Doug Sheehan was born in Santa Monica, California. (False); He is best known for his roles in soap operas. (True); He portrayed Joe Kelly. (True); Joe Kelly was in General Hospital. (True); General Hospital is a soap opera. (True); He portrayed Ben Gibson. (True); Ben Gibson was in Knots Landing. (True); Knots Landing is a soap opera. (True)]}
                \newline
                [The end of the list of checked facts]
                \newline \newline
                [The start of the ideal output]
                \newline
                \textcolor{labelcolor}{[Doug Sheehan (True); is (True); an (True); American (True); actor (True); who (True); was born (True); on (True); April 27, 1949 (True); in (True); Santa Monica, California (False); . (True); He (True); is (True); best known (True); for (True); his roles in soap operas (True); , (True); including (True); in (True); his portrayal (True); of (True); Joe Kelly (True); on (True); ``General Hospital'' (True); and (True); Ben Gibson (True); on (True); ``Knots Landing.'' (True)]}
                \newline
                [The end of the ideal output]
				\newline \newline
				\textbf{User prompt}
				\newline
				\newline
				[The start of the biography]
				\newline
				\textcolor{magenta}{\texttt{\{BIOGRAPHY\}}}
				\newline
				[The ebd of the biography]
				\newline \newline
				[The start of the list of checked facts]
				\newline
				\textcolor{magenta}{\texttt{\{LIST OF CHECKED FACTS\}}}
				\newline
				[The end of the list of checked facts]
			};
		\end{tikzpicture}
        \caption{GPT-4o prompt for labeling hallucinated entities.}\label{tb:gpt-4-prompt}
	\end{center}
\vspace{-0cm}
\end{table*}

The placeholder: <<median length of context document>> is the median number of words in documents provided in the context-specific part of the prompt that follows the general part.
%This directive is intended to avoid document-length bias in the ranking competitions.
%\newline
%\textit{\textbf{User prompt:}\newline
%<contextual information according to the method used>}\newline


\myparagraph{Context-specific part} The goal of this part of the
prompt is to provide the LLM with ``hints'' about the undisclosed
ranking function; specifically, past rankings. We present four
types of context, most of which are inspired by work on using LLMs to
induce ranking \cite{nogueira2020document, qin2023large, ma2023zero}.
\begin{itemize}
\item \firstmention{Pointwise}: The context includes pairs of a query and a
  document which was the highest ranked in the past for the
  query. This context is inspired by pointwise approaches to LLM ranking which provide the LLM with a query and examples of relevant and non-relevant documents (e.g., \cite{Liang+al:22a}). Including the documents most highly ranked in the past for the query at hand is conceptually reminiscent of a prevalent document modification strategy in competitive search \cite{raifer2017information,goren_driving_2021}: mimicking content in the most highly ranked documents in the past.
\item \firstmention{Pairwise}: Inspired by work on using pairwise approaches to prompt an LLM to induce ranking \cite{qin2023large}, the context
  includes queries accompanied with pairs of documents and an indication about which was ranked higher. 
\item \firstmention{Listwise}: Inspired by listwise prompting
  approaches for LLM ranking \cite{ma2023zero}, the context includes queries and ranked document lists. 
\item \firstmention{Temporal}: The context includes a query and
  versions of the same document and their ranks in past rankings induced for the query. The goal is to provide the LLM with information about how modifications of a document affected its ranking along time.
  %  Pairs of documents are considered (top-ranked documents or random documents);
  %This method selects
  %top-ranking documents (from one or multiple games), inspired by the
  %"mimicking the winner" strategy which has been shown to be optimal
  %in competitive search settings \cite{raifer2017information};
  \end{itemize}

%To gear the LLM towards the objective of
%modifying the document for improved ranking, our modification
%strategies provide the LLM in the user part of the prompt some
%information about how the undisclosed ranking function ranks
%documents. The types of information we used are inspired by work on
%using LLMs to rank documents \cite{nogueira2020document, qin2023large,
%  ma2023zero}.



%We propose four distinct classes of methods for constructing \contextualized s, inspired by previous work \cite{nogueira2020document, qin2023large, ma2023zero}. These methods differ in the way past documents and their associated rankings are integrated into the \contextualized. These documents, along with their corresponding rankings are the only signal an agent has about the ranking function. This information is crucial for enabling the agent to modify its current document to achieve an higher ranking in the subsequent round(s).
% assume that past documents and their rankings tha adapted these concepts, originally crafted from a ranker perspective, to serve the requirements of a ranking-incentivized modification agent, effectively transforming ranking principles into tools used for rank promotion.

% In this study, we introduce four distinct context-based methods. Each method follows a "few-shot" approach  when generating text with any large language model (LLM), enabling the model to generalize from limited examples without extensive training.

% Importantly, we chose not to fine-tune the LLMs utilized in our study, opting instead to work solely with prompts. This decision was made to demonstrate that a basic level of proficiency with LLMs is sufficient for executing this type of ranking-incentivized modification procedure, making it accessible to users without specialized technical knowledge. Future research might explore more performance-driven strategies, including model fine-tuning or alternative methods, which could potentially yield better empirical results.
%The four context-based methods for constructing \contextualized s are
%as follows: (i) \textbf{Pointwise (POW)}: This method selects
%top-ranking documents (from one or multiple games), inspired by the
%"mimicking the winner" strategy which has been shown to be optimal in
%competitive search settings \cite{raifer2017information}; (ii)
%\textbf{Pairwise (PAW)} \cite{qin2023large}: Pairs of documents are
%considered (top-ranked documents or random documents); (iii)
%\textbf{Listwise (LIW)} \cite{ma2023zero}: Ranked list(s) of documents
%are considered, providing broader context to the LLM; (iv)
%\textbf{Dynamic (DYN)}: A novel approach we introduce, this method
%selects different documents authored by the same individual. The
%dynamic approach is intended to capture variability in an agent's
%document protfolio.
% \begin{itemize} 
%     \item \textbf{Pointwise (POW)} \cite{nogueira2020document}: This method focuses solely on top-ranking document examples inspired by the "mimicking the winner" strategy proved to by optimal in competitive search settings \cite{raifer2017information}.
    
    % , implementing a “mimicking the winner” strategy similarly to what Raifer et al. \cite{raifer2017information} mentions in their article. The idea behind this method is that by analyzing only the highest-performing documents, the model gains insight into successful content features. Inspired by Nogueira et al.  \cite{}.
    
    % \item \textbf{Pairwise (PAW)} \cite{qin2023large}: In this approach, pairs of documents are considered, either top ranked documents or random documents.
    
    % often contrasting a top-ranking document with a lower-ranking or "loser" counterpart. In other instances, we select two documents at random, facilitating a comparative analysis of differing content attributes. Inspired by Zhen et al. .
    
    % \item \textbf{Listwise (LIW)}: Generalizing the former method mentioned, this method involves an exhaustive comparison, incorporating all documents from the match set(s). The motivation for this method is that by leveraging a broader context, we can capture a holistic view of the content landscape. Inspired by Ma et al. 
   
%     \item \textbf{Dynamic (DYN)}: The dynamic method evaluates a sequence of documents from a particular publisher over time, either a top-ranked document or a random document. This method is novel to best of our knowledge. 
% \end{itemize}
    
    % including the publisher's historical rankings over several rounds. This may involve using the latest top document and its previous iterations, or alternatively, a randomly selected document and its history, to understand how content evolution affects rankings. This method is novel.

\myparagraph{Prompt configurations}
There are numerous configurations
%($225$ all together)
of the context-specific part that we
study. These are determined by the following factors and parameters:
(i) the number of queries used; (ii) the number of examples used per
query: number of the most highest ranked documents in the past (Pointwise),
number of pairs (Pairwise), number of ranked lists (Listwise) and number of
documents with their temporal changes and ranks (Temporal); (iii) whether
the rank of the document to be modified in the last ranking induced
for the query is included; (iv) whether the query at hand is included
in the prompt, or alternatively only other queries; (v) for the
pairwise approach (Pairwise) we consider a random selection of the pair of
documents or the highest ranked document and an additional randomly
selected document; and, (vi) for the Temporal approach (Temporal), we vary the number of past ranks of the example document.  
%
%To further distinguish these prompts as distinct methods, we establish a rigorous parametrization: (i) \textbf{Number of queries}: Number of games included in the prompt; (ii) \textbf{Number of batches}: The number of batches per query; tailored to the method\footnote{For instance, the value 1 represents: a single document for Pointwise; a pair of documents for Pairwise; a ranked list for Listwise; and one document for Dynamic.}; (iii) \textbf{Inclusion of the current rank}: Indicates whether the rank of the \cd{} is included; (iv) \textbf{Inclusion of the current query}: Indicates whether the query for which we apply modification is included; (v) \textbf{Example type}: Defines whether the documents are selected randomly or by selecting the top ranked documents\footnote{This option applies only to Pairwise or Dynamic prompts.}. (vi) \textbf{History length}: The number of past rounds are included in the context\footnote{This option applies only to Dynamic prompt.}. In total, 225 unique \contextualized s structures were created.

% \begin{itemize} 
%     \item \textbf{Unique Query Number}: Specifies the optional number of different queries whose matches are included in the context. Options are to include 1-3 unique queries.
    % \item \textbf{Unique Example Number per Query/Topic}: Defines the number of distinct examples per query or topic, tailored to each method. Specifically, one example is set to 1 for the pointwise method, 1 pair for the pairwise method, 1 list of documents for the listwise method, and 1 "strand" of documents which its number is based on the "history length" feature for the dynamic method. The potential values for this parameter are 1-3 examples, depending on the requirements of the specific approach.
    % \item \textbf{Inclusion of Candidate's Ranking}: A binary feature that controls whether the current rank position of the \cd is shown. This setting enables testing of anchoring effects by either displaying or omitting the candidate’s ranking, which helps evaluate the LLM’s ability to assess content independently of explicit rankings of the input.
    % \item \textbf{Inclusion of Candidate's Topic/Query}: Another binary feature indicating whether the same topic/query as the candidate is included in the context at all. This tests whether the presence of topical relevance enhances the LLM’s ability to generalize the ranking-incentivized modifications' strategy effectively.
    % \item \textbf{Example Type}: Applicable solely to the "pairwise" and "dynamic" methods, this feature can be set to "random," where the example selection is fully randomized, or "tops," where at least one example (or the only example, if only one is specified) must be the top-ranked document. This parameter influences the contextual input quality by controlling the selection criteria for examples.
%     \item \textbf{History Length}: Relevant only to the "dynamic" method, this feature specifies the length of the historical strand per example. Options of 2 or 3 allow the model to examine a sequence of documents, capturing longitudinal patterns.
% \end{itemize}

% Each of these parameters was selected to enable the parametrization of different prompt-driven \bt s, allowing us to explore the influence of varying contextual \contextualizeds on the agent performance.

% \newline
% \newline

% Our ultimate objective was to select a single top-performing \bt from each method —--pointwise, pairwise (tops/random), listwise, and dynamic (tops/random)--- yielding six in total. To identify the optimal parameters and ultimately the best representative for each method, we conducted a comprehensive grid search, calculating the "Scaled Rank Promotion" metric (detailed below) on the test data described by Goren et al. \cite{goren2020ranking}, namely the 7th round in Raifer et al.'s competition \cite{raifer2017information}. For this process, we utilized Meta's Llama-2-13b model \cite{touvron2023llama}.

% Although OpenAI's GPT-4o model \cite{achiam2023gpt} demonstrated superior performance, we opted to use Llama-2 due to its open-source availability, which supports a scalable and cost-effective approach. In contrast, GPT-4o \cite{achiam2023gpt} operates on a pay-per-use model, making it less feasible for generating the thousands of documents required in this phase. By leveraging Llama-2 \cite{touvron2023llama}, we were able to implement a scalable heuristic that approximates the behavior of our \bt s on other LLMs, including GPT-4o, thereby streamlining the selection process.

% Once the grid search was completed, and the best-performing \bt s were identified for each method based on their "Scaled Rank Promotion" scores, we proceeded to apply these six selected methods with GPT-4o to optimize our final results. This two-step approach allowed us to balance scalability with performance, using Llama-2 for initial exploration and GPT-4o to maximize the quality of our outcomes. The final results, based on this methodology, are presented in the following section, and the chosen hyper-parameters chosen for each method are listed in Table \ref{table:1}.

% \input{Tables/chosen_grid_run}


% -	BOT descriptions (top to bottom, best 2 explained first):
% o	DYN_1100T2 – dynamic approach (history based) using a single query which is not the candidate’s query and a single example thread with length two (current epoch and previous epoch only) of the top document in the current round with no knowledge of the candidate’s document position.
% o	POW_1300 – Pointwise approach using a single query which is not the candidate’s query and three example documents, one of which is the top document, with no knowledge of the candidate’s document position.
% o	PAW_1301R – Pairwise approach using a single query which is the candidate’s query and three example document pairs chosen at random (not necessarily showing the top document in any example), there is no knowledge of the candidate’s document position, and it is not part of any example.
% o	DYN_1201R2 – dynamic approach (history based) using a single query which is the candidate’s query and two example threads with length two (current epoch and previous epoch only) of the random position documents in the current round with no knowledge of the candidate’s document position and without including it in the examples.
% o	PAW_1210T – Pairwise approach using a single query which is not the candidate’s query and two example document pairs including the top document and another chosen at random (necessarily showing the top document in any example), there is knowledge of the candidate’s document position, and it is not part of any example having not used the candidate query.
% o	LIW_1201 – Listwise approach using a single query which is the candidate’s query and two example lists (one for every epoch, using current and previous epochs) the lists are given in a ranked order but do not disclose the position of the candidate document and it is essentially not part of the examples.


\section{Experimental settings}
\label{sec:experimental-setup}
%We use two types of evaluation: offline and online. For offline
%evaluation,
\myparagraph{Datasets}
We use datasets which are reports of ranking
competitions \cite{raifer2017information,Mordo+al:25a}. In these
competitions, students were assigned to queries and had to produce
documents that would be highly ranked. Before the first round the students were provided with an example of a document relevant to the query. In each of the following rounds, the students observed past rankings for their queries and could modify their documents to potentially improve their next round ranking.

The first dataset, \firstmention{\firstDataset}, is the result of
ranking competitions held for $31$ queries from the TREC9-TREC12 Web
tracks \cite{raifer2017information}. Five to six students competed for each query. The undisclosed ranking function
was LambdaMART \cite{burges2010lambdamart} applied with various hand-crafted features. Following
Goren et al. \cite{goren2020ranking}, whose document modification
approach, \firstmention{\sentReplace}, serves as a baseline\footnote{We found that using LambdaMART instead of SVMrank as originally proposed \cite{goren2020ranking} yields improved performance.}, we use
round 7 for evaluation \cite{raifer2017information}. \sentReplace is a state-of-the-art feature-based supervised method for ranking-incentivized document modification. It
replaces a sentence in the document with another sentence to
improve ranking and to maintain content quality and faithfulness to
the original document.

The second dataset, \firstmention{\secondDataset}, is a report of
ranking competitions \cite{Mordo+al:25a} where the undisclosed ranking
function was the cosine between the E5 embedding vectors \cite{Wang+al:24a} of a document
and a query\footnote{The intfloat/e5-large-unsupervised version from
  the Hugging Face repository
  (\url{https://huggingface.co/intfloat/e5-large-unsupervised}).}. The competitions were run for 7 rounds with $15$
queries from the Web tracks of TREC9-TREC12; 4 players were competing
for each query \cite{wang2022text}.
%Our best performing document modification
%strategies (prompts) were used as bots in some rounds of these
%competitions for online evalution. (See more details below.)
%For
%offline evaluation,
We used round $4$ for evaluation to allow the document
modification methods to have enough history of past rankings. 

For both datasets just described, we apply the different document
modification methods, henceforth referred to as \firstmention{bots},
upon each of the documents in the ranked list for a query in the
specified round (except for the highest ranked document). For each
selected document, we induce a ranking using the same ranker used in
the competitions over its modified version and the original next-round
versions of the other documents (of students) from the round. We use
the evaluation measures described below upon the resultant ranking. We
average the evaluation results across all documents we modified per
round and over queries.

\myparagraph{Evaluation measures}
%We analyzed the performance of a document modification method using various evaluation measures, categorized into three primary groups: ranking properties, faithfulness properties and Quality and Relevance properties. All measures were computed per player and her document for a given query. The results were averaged over queries and grouped by the player type (student, baseline\footnote{i.e. the method of replacing paragraphs \cite{goren2020ranking}.}, a static document or one of the \bt s). For the online evaluation, the measures were also averaged over rounds.
To evaluate rank promotion (demotion) of documents as a result of
modification, we follow Goren et al. \cite{goren2020ranking} and
report \firstmention{Scaled Promotion}: the increase (decrease) of rank in the next round with respect to the current round normalized by the maximum possible rank promotion (demotion).



\omt{
%\begin{block}{Candidate Faithfulness at 1}
$CF@1(d_{curr},d_{next})=\frac{1}{n} \cdot \Sigma_{i=1}^{n} \mathbf{1}{\{ TT(d{curr},d_{next_{i}}) \geq 0.5 \}}$
%\end{block}

%\begin{block}{Normalized Candidate Faithfulness at 1}
$NCF@1(d_{curr},d_{next})=\frac{CF@1(d_{curr},d_{next})}{CF@1(d_{curr},d_{curr})}$
%\end{block}


%\begin{block}{Environmental Faithfulness at 10}
$EF@10(d_{next})=\frac{1}{2 \cdot 10} \cdot \Sigma_{i=1}^{10} [\mathbf{1}{\{ TT(d{\text{top}i}, d{\text{next}}) \geq 0.5 \}} + \mathbf{1}{\{ TT(d{\text{next}}, d_{\text{top}_i}) \geq 0.5 \}}]$
%\end{block}

%\begin{block}{Normalized Environmental Faithfulness at 10}
$NEF@10(d_{curr},d_{next})=\frac{EF@10(d_{next})}{EF@10(d_{curr})}$
%\end{block}
}


To evaluate the faithfulness of a modified document ($\dn$) to its
original (current) version ($\dc$), we compare
the two documents using Gekhman's et al. \cite{gekhman2023trueteacher}
natural language inference (NLI) approach. Specifically, we estimate
whether one document (denoted {\em hypothesis}) is entailed from the other
document (denoted {\em premise}) while preserving factual consistency. The estimate is the \trueteacher{} (TrueTeacher)
measure: \trueteacher $(premise, hypothesis)$ is the
output of the model in the range [0,1]; higher scores indicate stronger factual alignment.

To apply the TrueTeacher model, we first compute the average number of sentences in the modified document that are entailed\footnote{Entailment is determined by a threshold of $0.5$ for the TT score \cite{gekhman2023trueteacher}.} by the current document, which we refer to as raw faithfulness (RF):
%\trueteacher{} score between
%the current document ($\dc$) and all ($n$) sentences in the modified
%document ($\dn$):
$RF(\dn,\dc) \definedas \frac{1}{n} \sum_{i=1}^n \delta[\trueteacher
  (\dc, \dni) \ge 0.5];$ $d^{i}$ is the i'th sentence in document $d$;
$\delta$ is Kronecker's indicator function. Since $RF(\dc,\dc)$ is not
necessarily $1$, we normalize the raw
faithfulness to yield our \firstmention{\normFaith} measure: $\frac
{RF(\dn,\dc)}{RF(\dc,\dc)}$. 

Using LLMs to modify documents raises a concern
about hallucinations \cite{shuster2021retrieval}. We hence measure the
extent to which the content in the modified document is ``faithful''
to that in the entire corpus\footnote{For a corpus we use all the
  documents in all rounds prior to the round on which evaluation is
  performed.}. To that end, we treat the current document as a query,
and retrieve the top-$k$\footnote{We set $k=10$ in our experiments.}
documents in the corpus; $\topRet$ denotes the retrieved set. Retrieval is based on using cosine to compare a query
embedding and the document embedding. We use two types of embeddings:
E5 \cite{Wang+al:24a} and TF.IDF.  We define raw corpus faithfulness
(RCF) as: $RCF (\dn) \definedas \frac {1}{2k} \sum_{d \in \topRet}
(RF(\dn,d) + RF (d,\dn))$. The normalized corpus faithfulness
measure we use is: $CF (\dn) \definedas
\frac{RCF(\dn)}{RCF(\dc)}$. Using the E5 and TF.IDF embeddings results
in the \firstmention{\normCorpFaithE} and
\firstmention{\normCorpFaithT} normalized corpus faithfulness
measures, respectively.

Statistically significant differences are determined using the two-tailed paired permutation test
  with 100,000 random permutations and $p < 0.05$.

\omt{
The Normalized Candidate Faithfulness $NCF@1(\dc, \dn)$: the
normalization of the Candidate Faithfulness $CF@1(\dc, \dn)$ by the
self-consistency score: $\frac{CF@1 (\dc, \dn)}{CF@1(\dc, \dc)}$;
(iii) Environmental Faithfulness at 10 $EF@10(\dn)$: This metric
measures how much the generated document ($\dn$) maintains contextual
consistency with the broader corpus. Specifically, it measures the
similarity of $\dn$ to the top 10 documents most similar to it in the
corpus.  The corpus includes all the documents (across all queries)
available up to the test round. Two approaches are employed to compute
the similarity. The first approach is based on the (unsupervised) E5
\cite{wang2022text} representation with the cosine similarity
metric. The second approach is based on the TF.IDF
\cite{sparck1972statistical, salton1975vector} representation with the
cosine similarity metric. This metric is then calculated as follows:
$\frac{1}{2*10} \sum_{i=1}^{10} (\trueteacher(\dn, d_{top_{i}}) +
\trueteacher(d_{top_i}, \dn))$. Where $d_{top_{i}}$ represents the
$i$-th document, while ordering the documents with respect to the
similarity to $\dc$. The two approaches yield two variants of this
metric: $EF@10$\_dense and $EF@10$\_sparse, for the E5 and TF.IDF
representations, respectively; (iv) The Normalized Environmental
Faithfulness at 10 $NEF@10(\dn)$: The normalization of EF@10 by the
EF@10 of the current document: $\frac{EF@10(\dn)}{EF@10(\dc)}$. These
measures collectively provide a comprehensive framework for assessing
faithfulness. They evaluate the consistency of the modified document
not only with respect to the current document but also in relation to
other documents in the corpus.



\myparagraph{Relevance and Quality scores} The third category of evaluation measures focuses on the relevance and quality of documents. Both quality and relevance scores are assigned by crowdsourcing annotators via the Connect platform on CloudResearch \cite{noauthor_introducing_2024}, assessing the document's content quality and its relevance to the query\footnote{These evaluations of relevance and quality are conducted exclusively in the online evaluation setting.}. A document's quality or relevance score is set to 1 if at least three out of five English-speaking annotators marked it as valid or relevant to the query; otherwise, the score is set to 0. We report the ratio of documents that received a quality or relevance score of 1.
}


\myparagraph{Instantiating bots} For LLM we use Chat-GPT 4o
\cite{achiam2023gpt}. As described in Section
\ref{sec:bots}, there are a few parameters affecting the
instantiation of specific prompts. The number of queries is set to a
value in $\{1, 2\}$.  The number of examples per query is selected
from $\{1, 2, 3\}$. The number of past ranks (i.e., rounds) in the
Temporal prompt is selected from $\{2,3\}$. Using these
parameter values, and the other binary decision factors that affect
instantiation (see Section \ref{sec:bots}), results in $192$
different bots (prompts). In addition, we set the LLM's temperature parameter which controls potential drift to values in $\{0, 0.5, 1, 1.5, 2\}$ \cite{peeperkorn_is_2024}. 

\myparagraph{Rank promotion performance of bots} In terms of Scaled
Promotion, we found\footnote{Actual numbers are omitted due to space
  considerations and as they convery no additional insight.} that the
Pairwise bots (with random selection of document pairs) and the
Listwise bots were the best performing for both the \firstDataset and
\secondDataset datasets; the same specific instantiation of each of these two bots was
always among the top-3 performing bots for both datasets. This finding attests to the
rank-promotion effectiveness of these types of bots (prompts) for
different rankers (LambdaMART and E5). The Temporal bots (prompts), which provide rank-changes information along rounds, were less
effective (in terms of Scaled Promotion) than the Pairwise and
Listwise bots, but were more effective than the Pointwise bots. 

In what follows, we present the evaluation of the two
bots which posted for both datasets Scaled Promotion among the best three:\footnote{These bots were also the best performing in the online evaluation presented below.} 
%For efficiency considerations, we use LLama-2 with $13$B parameters
%\cite{touvron2023llama} to select the best performing
%configurations. The selection is performed with the \firstDataset
%dataset based on the scaled promotion evaluation measure. The best
%performing configurations for which we will report performance over the evaluation datasets are:

\begin{itemize}
\item Pairwise, where only the given query is included, one
  random pair of documents for each of the three last rounds is
  provided as examples, the current rank of the document is not
  used, and the temperature is set to $0.5$.
\item Listwise, where only the given query is included, two previous rounds are used, the current rank is not used, and the temperature is set to $0$.
\end{itemize}
Appendix \ref{appendix_prompt} provides the prompts for these bots.
%The fact that the pairwise and listwise approaches are the most
%effective is conceptually consistent of findings in work on using LLMs
%to induce ranking where the merits of pairwise and listwise approaches
%have been demonstrated \cite{ma2023zero,qin2023large}. For evaluation
%over \firstDataset and \secondDataset we use




\omt{
%The goal of the first phase is to identify a representative prompt for each class of prompts--that is, the prompt whose resultant agents maximize a metric related to ranking promotion. We conduct a comprehensive grid search over the 225 configurations described in Section \ref{sec:bots}, using Dataset 1. For this phase, we utilized Llama-2 with 13B parameters due to its availability \cite{touvron2023llama}. From each configuration, we constructed five agents, each with a different temperature setting for the probability model of the LLM\footnote{All other parameters of the LLM were fixed.}. The temperatures used were \{0, 0.5, 1, 1.5, 2\} and were selected based on the work of Peeperkorn et al \cite{peeperkorn_is_2024}. The selected agents compete for round 7, as was the case in Goren et al. \cite{goren2020ranking}. We do not report the detailed results of this phase due to space limitations in the paper.
}


\myparagraph{Online evaluation} The evaluation performed over the
\firstDataset and \secondDataset datasets is offline and therefore
spans a single round: the students who competed in the competition did
not respond to rankings induced over the documents we modify here. We
therefore also performed online evaluation where our instantiated
prompts competed as bots against students. We organized a ranking
competition\footnote{The competition was approved by institution and international ethics committees.} similar to that of Mordo et al. \cite{Mordo+al:25a} using 15
queries from TREC9-TREC12\footnote{These are different queries than
  those used in the \secondDataset dataset: 21, 55, 61, 64, 74, 75, 83, 96, 124,
  144, 161, 164, 166, 170, 194.}. In contrast to Mordo et al.'s
competitions \cite{Mordo+al:25a}, each game included 5 players: two-three
students, one of the two bots discussed above (Pairwise or
Listwise), and one or two static documents were created using a procedure similar to the one in Raifer et al. \cite{raifer2017information}: first, we used the query in the English Wikipedia search engine and selected a highly ranked page. We then extracted a candidate paragraph from this page, with a length of up to 150 words. Three annotators assessed the relevance of the passages, and we repeated the extraction process for each query until at least two annotators judged a paragraph as relevant. The selected paragraph was then used as a static document for the query for all students.

The students were not aware that they were competing
against bots. We applied our bots in rounds 5\footnote{Due to
  technical issues, we could not run the bots at round 4 as in the offline evaluation.}, 6 and 7 and report the
average performance over these three rounds.

We had documents in the online
evaluation judged for relevance and quality using crowdsourcing
annotators on the Connect platform of CloudResearch
\cite{noauthor_introducing_2024}. Following past work on ranking
competitions \cite{raifer2017information,goren2020ranking}, a
document's quality grade is set to $1$ if at least three out of five
English-speaking annotators marked it as valid (as opposed to keyword
stuffed or useless) and to $0$ otherwise. The relevance grade was $1$ if the document was
marked relevant by at least three annotators and $0$ otherwise. 







\endinput


We adopt an evaluation approach similar to that of Goren et al. \cite{goren2020ranking}. Two evaluation settings are considered: (i) Offline evaluation, where we leveraged existing datasets from ranking competitions, and (ii) Online evaluation, where a set of \bt s, each with a specific \contextualized, participate as a player in an ongoing ranking competition. The offline evaluation is run only for a single round, since the students did not respond to rankings that included the documents produced by our \bt s. In the online setting, other players may modify their documents simultaneously while our agents make their own modifications. In this section, we begin by describing the datasets used in our experiments (Section \ref{sec_datasets}). We then present the setups for both offline and online evaluations (Section \ref{sec_exp_set}). Finally, we outline the evaluation measures employed to assess performance of the \bt s and compare their performances against other types of agents (Section \ref{sec:eval-measures}).







\subsection{Datasets}\label{sec_datasets}
To perform offline and online evaluation, we deployed our approach in three different ranking competitions: one competition with feature-based ranking function, and two other utilizing transformer-based ranking function. The datasets employed in our experiments are as follows:

\myparagraph{Dataset 1} The first competition utilized for offline evaluation and comparison with the baseline model proposed by Goren et al. \cite{goren2020ranking}. It was organized by Raifer et al. \cite{raifer2017information}. In this competition, students enrolled in a course served as authors of documents and were assigned to 31 queries from the TREC9-TREC12 Web tracks. Each query defined a repeated-ranking-game. Students were incentivized with course grade bonuses and were asked to modify their documents for 8 rounds so that their document will be highly ranked for the played query. We selected round 7 for the offline evaluation, following Goren et al. \cite{goren2020ranking}. A total of 31 repeated-games (one per query) were conducted. A LambdaMART ranking function was applied \cite{burges2010lambdamart}.

\myparagraph{Dataset 2} This dataset used for offline evaluation and hyper-parameter tuning for the online evaluation. It was sourced from a ranking competition conducted by Mordo et al. \cite{div}. It involved a competition with 15 queries\footnote{From the TREC9-TREC12 Web Track as well.}, 7 rounds, and 4 players per game. In contrast to the competition described by Raifer et al. \cite{raifer2017information}, a transformer based ranking function was applied: the (unsupervised) E5 \footnote{The intfloat/e5-large-unsupervised version from the Hugging Face repository was used
  (\url{https://huggingface.co/intfloat/e5-large-unsupervised}).} \cite{wang2022text}. We focus on round 4, as it is the first round where we can apply our \bt s (recall that our \bt s require the context of previous rounds to modify a document).

\myparagraph{Dataset 3} We organize a ranking competition using 15 queries\footnote{From TREC9-TREC12; Different queries comparing to those used in Dataset 2: [21, 55, 61, 64, 74, 75, 83, 96, 124, 144, 161, 164, 166, 170, 194].}. The setup of this competition was similar to that of Mordo et al \cite{div}, with the following key difference: each games included 5 players. From round 5 \footnote{We initially planned to introduce our \bt s in round 4 as in Dataset 2; however, due to experimental constraints, we began their application in round 5.} of the competition, the players in each group consisted of: one \bt, two or three students and one or two planted documents\footnote{The same document as the initial document every participant started with.}. From the perspective of the students, the inclusion of \bt s did not alter the structure or appearance of the competition, preserving the integrity of the evaluation.

\subsection{Experimental setting}\label{sec_exp_set}

Our document modification approach operates as follows: first, the ranking for a given query is observed. Next, the approach modifies a specific document with the aim that the resulting document will be ranked higher in the next round of the game. In dynamic (online) settings, other documents may also be modified simultaneously, influencing the subsequent ranking. We design two evaluation paradigms—online and offline—both of which simulate a dynamic setting. The approach introduced by Goren et al. \cite{goren2020ranking} serves as a baseline to our approach.

\myparagraph{Offline evaluation}
The offline evaluation is divided into three phases. In each phase, we evaluated an \bt{} using a similar approach employed by Goren et al. \cite{goren2020ranking}: (i) Select a round and a game (query); (ii) Modify a document using the tested modification method (baseline \cite{goren2020ranking} or \bt{} with a specific prompt), excluding the top-ranked document. The exclusion of top-ranked documents is attributed to previous findings that their authors tend to avoid modifying their documents \cite{raifer2017information}. (iii) Evaluate the performance of the agent with respect to all other documents in the ranked list. (iv) Iterate over all the documents in the selected round and query and average the computed measure. 

% The performance of each metric is computed as the average over queries.

The goal of the first phase is to identify a representative prompt for each class of prompts--that is, the prompt whose resultant agents maximize a metric related to ranking promotion. We conduct a comprehensive grid search over the 225 configurations described in Section \ref{sec:bots}, using Dataset 1. For this phase, we utilized Llama-2 with 13B parameters due to its availability \cite{touvron2023llama}. From each configuration, we constructed five agents, each with a different temperature setting for the probability model of the LLM\footnote{All other parameters of the LLM were fixed.}. The temperatures used were \{0, 0.5, 1, 1.5, 2\} and were selected based on the work of Peeperkorn et al \cite{peeperkorn_is_2024}. The selected agents compete for round 7, as was the case in Goren et al. \cite{goren2020ranking}. We do not report the detailed results of this phase due to space limitations in the paper.

%In the second phase, we evaluated the performance of each representative prompt (Identified in Phase 1) on the same dataset and specifically on round 7, incorporating two key modifications: (i) we replaced Llama-2 with Chat-GPT 4o \cite{achiam2023gpt} as the latter demonstrated superior performance across multiple benchmarks\footnote{\url{https://docsbot.ai/models/compare/gpt-4o/llama-2-chat-13b}}. (ii) We included a baseline model introduced by Goren et al. \cite{goren2020ranking}, which modifies documents by replacing passages. Our implementation of the baseline consist of a primary difference: we replaced RankSVM \cite{joachims2002ranksvm} with LambdaMART \cite{burges2010lambdamart} due to its superior performance on Dataset 1. Details regarding the reproducibility process are omitted.

In the third phase, we evaluated the performance of each representative prompt on Dataset 2, which contains data from a ranking competition with a transformer-based ranking function. The evaluation procedure mirrored that of the second phase. We focused on round 4, as the round for evaluation.

\myparagraph{Online evaluation}
We adopted the two best performing \bt s (in terms of ranking promotion metric) evaluated on the transformer-based competition (Dataset 2) and assigned them as players in a similar ranking competition with different queries (resulting in Dataset 3). These \bt s joined the competition in round 5 and competed for the highest ranking in rounds 5, 6 and 7. Recall that the decision to introduce the \bt s in round 5, rather than at the beginning, was based on their dependency on past rankings, which were integrated into the \contextualized s to guide document modifications. The selection of round 5 over round 4 was due to experimental constraints.

% This article addresses the challenge of white-hat ranking-incentivized modifications, building on the work of Goren et al. \cite{goren2020ranking}, who explored this topic in offline and online competition settings on a ranking competition dataset comprised by Raifer et al. \cite{raifer2017information}. In addition to closely mimicking their approach, which utilized the feature-based LambdaMART ranker \cite{burges2010ranknet} and an LTR-based baseline, referred as the "feature-based" setting in this article, we adopt a transformer-based ranker—specifically, the E5 model introduced by Wang et al. \cite{wang2024multilingual}, both for offline and online evaluation. The E5 settings are referred to as the offline and online "transformer-based" settings in this article.

% Our study investigates the effectiveness of ranking-incentivized modifications within a comparable framework while leveraging the advantages of transformer-based models. The primary goal of this research is to evaluate strategies for rank promotion, focusing on leveraging large language models (LLMs) to implement modifications using various few-shot \cite{brown2020language} contextual approaches. By incorporating LLM-based methodologies, we aim to assess their ability to generate high-quality, contextually relevant modifications that adhere to the principles of white-hat ranking practices.


% \paragraph*{Dataset Creation}
% For the feature-based setting, we utilized the dataset created by Raifer et al. \cite{raifer2017information}, similarly to Goren et al. \cite{goren2020ranking}.

% In the transformer-based settings, we implemented and evaluated our \bt s within a ranking environment inspired by the 'ranking competition' framework introduced by Raifer et al. \cite{raifer2017information}. Similar to their approach, our ranking competition focused on optimizing documents for a black-box ranker. However, we introduced several adjustments to align with our experimental objectives and constraints.

% First, we utilized a subset of 15 queries derived from the TREC ClueWeb09 dataset \cite{clueweb09}, whereas Raifer et al. \cite{raifer2017information} used a broader set. This choice enabled us to conduct multiple experiments in parallel while maintaining a manageable workload and scalability. Additionally, we adhered to the original competition's guidelines, requiring concise, 150-word plain English submissions without links, special characters, or HTML tags, to ensure methodological consistency.

% The competition was structured into "matches" and "rounds." A match refers to a grouping of four competitors who worked on a single query. In each match, participants edited their texts to achieve the highest possible ranking for the query. A round is one iteration of competition during which all matches were conducted simultaneously for a specific query set. Each round provided participants with feedback, allowing them to see their own rankings as well as those of their competitors.

% The competition was divided into two parts, with each part consisting of seven rounds. In the first part, participants worked with 15 queries \cite{partA2024}. In the second part, these queries were replaced with 15 different ones, also sourced from ClueWeb09 \cite{clueweb09} \cite{partB2024 (TBA)}. The grouping of competitors and conditions remained fixed across rounds. We used the first part of the competition for offline evaluation and the second part for online evaluation, enabling a thorough analysis of our methodology under both controlled and dynamic conditions.

% By tailoring the competition to our needs while preserving its core principles, we ensured both comparability to prior work and the validity of our findings in the context of scalable and rigorous experimentation.


% \paragraph*{Offline Evaluation}
% The feature-based setting we developed was heavily influenced by the offline setting described in detail by Goren et al.\ \cite{goren2020ranking}. The evaluation of this setting was carried out meticulously, adhering closely to the methodology outlined in Goren et al.'s \cite{goren2020ranking} offline evaluation section.

% To rigorously assess our LLM-based ranking-incentivized modification methodology in the offline transformer-based setting, we constructed an evaluation setting inspired by the offline setting introduced by Goren et al.\ \cite{goren2020ranking}. Our experiments were conducted on documents initially ranked 2nd, 3rd, and 4th in the fifth round of a competition designed to rank documents against a shared set of 15 queries. These ranks were chosen deliberately, as they represented non-top-performing documents, providing a meaningful opportunity to evaluate the potential for improvement when modifications were applied, as suggested by Goren et al.\ \cite{goren2020ranking}.

% In each round of evaluation, four documents were subjected to ranking. Three of these were unaltered, human-authored documents selected from previous rankings in the competition. The fourth document was a modified version, generated by applying our LLM-based methodology. By incorporating multiple initial ranking positions and diverse queries, we ensured that the evaluation was not overly influenced by specific document characteristics or query types. This approach enhanced the generalizability of our findings.


% \paragraph*{Online Evaluation}
% To closely mimic the online experiment described by Goren et al.\ \cite{goren2020ranking}, we conducted the second part of the ranking competition using the same participant pool but a different set of 15 queries sourced from ClueWeb09 \cite{clueweb09}.

% For the online evaluation, we selected the two \bt{} methods that achieved the highest scores in the offline evaluation during one of our early experiments, the results of which are not depicted in this paper. We used "scaled rank promotion" as the metric for determining their performance. These \bt{}s were introduced into the competition starting from the 4th round of the second part, as their methodology required contextual information derived from at least three prior rounds to function effectively. The bots competed against the same set of human participants, with groupings and conditions held constant across all seven rounds of this phase. This ensured consistency in the evaluation and allowed for a direct comparison between \bt{}- and human-authored rankings.

% From the perspective of the human participants, the inclusion of \bt s did not alter the structure or appearance of the competition. Each round followed the same format and task descriptions as in the first part of the competition. This design ensured that human participants approached their ranking tasks without being influenced by knowledge of the \bt s' involvement, preserving the integrity of the evaluation.

% \paragraph*{Ranker and Document Embeddings}
% In the feature-based setting, we employed the same methodology and features previously utilized by Raifer et al.\ \cite{raifer2017information} and Goren et al.\ \cite{goren2020ranking}. Specifically, we used the exact trained LambdaMART model that was trained and used by Goren et al.\ \cite{goren2020ranking}. The method for text embeddings for this setting replicated Goren et al.'s \cite{goren2020ranking} approach - incorporating 25 content-based features. These features were selected either from those used in Microsoft's learning-to-rank datasets \cite{mslr}, or as query-independent measures of document quality. Notably, these included stopword-based metrics and the entropy of term distribution within a document, both of which have been proven effective in web retrieval scenarios.

% For the transformer-based offline and online settings, we utilized E5 as the ranker and the E5 embedder for the document embeddings.


% \paragraph*{Baseline}
% To establish a robust baseline, we implemented a ranking passage pairs approach, closely mirroring the methodology described in Goren et al. \cite{goren2020ranking}, with the primary difference being the replacement of RankSVM \cite{joachims2002ranksvm} with LambdaMART \cite{burges2010lambdamart}. LambdaMART was selected based on its superior empirical performance in prior evaluations. The dataset, features, and labels remained identical to the original setup. Labels for the passage pairs were generated using a dual-objective harmonic mean approach introduced in Goren et al. \cite{goren2020ranking}, integrating rank-promotion and local coherence objectives, where rank-promotion labels ranged from 0 to 4 based on positional improvement in rankings, and coherence labels quantified semantic similarity to maintain content quality. The harmonic mean was computed with $\beta = 1$, assigning equal weight to both objectives.

% Training was conducted on 57 documents extracted from Round 6 of the original competition dataset introduced by Raifer et al. \cite{raifer2017information}. The model’s performance was subsequently evaluated on 124 experimental settings, derived from Round 7 of the ranking competition, spanning documents ranked 2–5 across 31 queries. The validation set was configured similarly to Goren et al.'s \cite{goren2020ranking} procedure. Both training and validation utilized NDCG@1 as the evaluation metric, contrasting with the NDCG@5 used by Goren et al. \cite{goren2020ranking}, to align with the goal of selecting the top sentence-swapped document.

% We implemented LambdaMART \cite{burges2010lambdamart} using its default parameter settings for several features, specifically: \begin{itemize} \item \textbf{Minimum leaf support:} Minimum number of samples each leaf must contain, set to 1 (default). \item \textbf{Number of threshold candidates for tree splitting:} Set to 256 (default). \item \textbf{Early stopping rounds:} Set to 100 (default). \end{itemize}

% We conducted a grid search to tune other hyper-parameters, exploring different configurations. This grid search included: \begin{itemize} 
%     \item \textbf{Number of Trees:} 50, 500, 1000, 1200. 
%     \item \textbf{Number of Leaves per Tree:} 10, 50, 100. 
%     \item \textbf{Learning Rate (Shrinkage Value):} 0.01, 0.1, 0.2. 
% \end{itemize}


\begin{table*}[t]
  \centering
  \begin{tabular}{lll}
    \hline
    \textbf{Output}           & \textbf{natbib command} & \textbf{ACL only command} \\
    \hline
    \citep{Gusfield:97}       & \verb|\citep|           &                           \\
    \citealp{Gusfield:97}     & \verb|\citealp|         &                           \\
    \citet{Gusfield:97}       & \verb|\citet|           &                           \\
    \citeyearpar{Gusfield:97} & \verb|\citeyearpar|     &                           \\
    \citeposs{Gusfield:97}    &                         & \verb|\citeposs|          \\
    \hline
  \end{tabular}
  \caption{\label{citation-guide}
    Experimental analysis.
  }
\end{table*}


\vspace{-0.2cm}
\section{Impact: Why Free Scientific Knowledge?}
\vspace{-0.1cm}

Historically, making knowledge widely available has driven transformative progress. Gutenberg’s printing press broke medieval monopolies on information, increasing literacy and contributing to the Renaissance and Scientific Revolution. In today's world, open source projects such as GNU/Linux and Wikipedia show that freely accessible and modifiable knowledge fosters innovation while ensuring creators are credited through copyleft licenses. These examples highlight a key idea: \textit{access to essential knowledge supports overall advancement.} 

This aligns with the arguments made by Prabhakaran et al. \cite{humanrightsbasedapproachresponsible}, who specifically highlight the \textbf{ human right to participate in scientific advancement} as enshrined in the Universal Declaration of Human Rights. They emphasize that this right underscores the importance of \textit{ equal access to the benefits of scientific progress for all}, a principle directly supported by our proposal for Knowledge Units. The UN Special Rapporteur on Cultural Rights further reinforces this, advocating for the expansion of copyright exceptions to broaden access to scientific knowledge as a crucial component of the right to science and culture \cite{scienceright}. 

However, current intellectual property regimes often create ``patently unfair" barriers to this knowledge, preventing innovation and access, especially in areas critical to human rights, as Hale compellingly argues \cite{patentlyunfair}. Finding a solution requires carefully balancing the imperative of open access with the legitimate rights of authors. As Austin and Ginsburg remind us, authors' rights are also human rights, necessitating robust protection \cite{authorhumanrights}. Shareable knowledge entities like Knowledge Units offer a potential mechanism to achieve this delicate balance in the scientific domain, enabling wider dissemination of research findings while respecting authors' fundamental rights.

\vspace{-0.2cm}
\subsection{Impact Across Sectors}

\textbf{Researchers:} Collaboration across different fields becomes easier when knowledge is shared openly. For instance, combining machine learning with biology or applying quantum principles to cryptography can lead to important breakthroughs. Removing copyright restrictions allows researchers to freely use data and methods, speeding up discoveries while respecting original contributions.

\textbf{Practitioners:} Professionals, especially in healthcare, benefit from immediate access to the latest research. Quick access to newer insights on the effectiveness of drugs, and alternative treatments speeds up adoption and awareness, potentially saving lives. Additionally, open knowledge helps developing countries gain access to health innovations.

\textbf{Education:} Education becomes more accessible when teachers use the latest research to create up-to-date curricula without prohibitive costs. Students can access high-quality research materials and use LM assistance to better understand complex topics, enhancing their learning experience and making high-quality education more accessible.

\textbf{Public Trust:} When information is transparent and accessible, the public can better understand and trust decision-making processes. Open access to government policies and industry practices allows people to review and verify information, helping to reduce misinformation. This transparency encourages critical thinking and builds trust in scientific and governmental institutions.

Overall, making scientific knowledge accessible supports global fairness. By viewing knowledge as a common resource rather than a product to be sold, we can speed up innovation, encourage critical thinking, and empower communities to address important challenges.

\vspace{-0.2cm}
\section{Open Problems}
\vspace{-0.1cm}

Moving forward, we identify key research directions to further exploit the potential of converting original texts into shareable knowledge entities such as demonstrated by the conversion into Knowledge Units in this work:


\textbf{1. Enhancing Factual Accuracy and Reliability:}  Refining KUs through cross-referencing with source texts and incorporating community-driven correction mechanisms, similar to Wikipedia, can minimize hallucinations and ensure the long-term accuracy of knowledge-based datasets at scale.

\textbf{2. Developing Applications for Education and Research:}  Using KU-based conversion for datasets to be employed in practical tools, such as search interfaces and learning platforms, can ensure rapid dissemination of any new knowledge into shareable downstream resources, significantly improving the accessibility, spread, and impact of KUs.

\textbf{3. Establishing Standards for Knowledge Interoperability and Reuse:}  Future research should focus on defining standardized formats for entities like KU and knowledge graph layouts \citep{lenat1990cyc}. These standards are essential to unlock seamless interoperability, facilitate reuse across diverse platforms, and foster a vibrant ecosystem of open scientific knowledge. 

\textbf{4. Interconnecting Shareable Knowledge for Scientific Workflow Assistance and Automation:} There might be further potential in constructing a semantic web that interconnects publicly shared knowledge, together with mechanisms that continually update and validate all shareable knowledge units. This can be starting point for a platform that uses all collected knowledge to assist scientific workflows, for instance by feeding such a semantic web into recently developed reasoning models equipped with retrieval augmented generation. Such assistance could assemble knowledge across multiple scientific papers, guiding scientists more efficiently through vast research landscapes. Given further progress in model capabilities, validation, self-repair and evolving new knowledge from already existing vast collection in the semantic web can lead to automation of scientific discovery, assuming that knowledge data in the semantic web can be freely shared.

We open-source our code and encourage collaboration to improve extraction pipelines, enhance Knowledge Unit capabilities, and expand coverage to additional fields.

\vspace{-0.2cm}
\section{Conclusion}
\vspace{-0.1cm}

In this paper, we highlight the potential of systematically separating factual scientific knowledge from protected artistic or stylistic expression. By representing scientific insights as structured facts and relationships, prototypes like Knowledge Units (KUs) offer a pathway to broaden access to scientific knowledge without infringing copyright, aligning with legal principles like German \S 24(1) UrhG and U.S. fair use standards. Extensive testing across a range of domains and models shows evidence that Knowledge Units (KUs) can feasibly retain core information. These findings offer a promising way forward for openly disseminating scientific information while respecting copyright constraints.

\section*{Author Contributions}

Christoph conceived the project and led organization. Christoph and Gollam led all the experiments. Nick and Huu led the legal aspects. Tawsif led the data collection. Ameya and Andreas led the manuscript writing. Ludwig, Sören, Robert, Jenia and Matthias provided feedback. advice and scientific supervision throughout the project. 

\section*{Acknowledgements}

The authors would like to thank (in alphabetical order): Sebastian Dziadzio, Kristof Meding, Tea Mustać, Shantanu Prabhat for insightful feedback and suggestions. Special thanks to Andrej Radonjic for help in scaling up data collection. GR and SA acknowledge financial support by the German Research Foundation (DFG) for the NFDI4DataScience Initiative (project number 460234259). AP and MB acknowledge financial support by the Federal Ministry of Education and Research (BMBF), FKZ: 011524085B and Open Philanthropy Foundation funded by the Good Ventures Foundation. AH acknowledges financial support by the Federal Ministry of Education and Research (BMBF), FKZ: 01IS24079A and the Carl Zeiss Foundation through the project "Certification and Foundations of Safe ML Systems" as well as the support from the International Max Planck Research School for Intelligent Systems (IMPRS-IS). JJ acknowledges funding by the Federal Ministry of Education and Research of Germany (BMBF) under grant no. 01IS22094B (WestAI - AI Service Center West), under grant no. 01IS24085C (OPENHAFM) and under the grant DE002571 (MINERVA), as well as co-funding by EU from EuroHPC Joint Undertaking programm under grant no. 101182737 (MINERVA) and from Digital Europe Programme under grant no. 101195233 (openEuroLLM) 

% % \smallskip
% \myparagraph{Acknowledgments} We thank the reviewers for their comments.
% The work by Moshe Tennenholtz was supported by funding from the
% European Research Council (ERC) under the European Union's Horizon
% 2020 research and innovation programme (grant agreement 740435).


\subsection{Lloyd-Max Algorithm}
\label{subsec:Lloyd-Max}
For a given quantization bitwidth $B$ and an operand $\bm{X}$, the Lloyd-Max algorithm finds $2^B$ quantization levels $\{\hat{x}_i\}_{i=1}^{2^B}$ such that quantizing $\bm{X}$ by rounding each scalar in $\bm{X}$ to the nearest quantization level minimizes the quantization MSE. 

The algorithm starts with an initial guess of quantization levels and then iteratively computes quantization thresholds $\{\tau_i\}_{i=1}^{2^B-1}$ and updates quantization levels $\{\hat{x}_i\}_{i=1}^{2^B}$. Specifically, at iteration $n$, thresholds are set to the midpoints of the previous iteration's levels:
\begin{align*}
    \tau_i^{(n)}=\frac{\hat{x}_i^{(n-1)}+\hat{x}_{i+1}^{(n-1)}}2 \text{ for } i=1\ldots 2^B-1
\end{align*}
Subsequently, the quantization levels are re-computed as conditional means of the data regions defined by the new thresholds:
\begin{align*}
    \hat{x}_i^{(n)}=\mathbb{E}\left[ \bm{X} \big| \bm{X}\in [\tau_{i-1}^{(n)},\tau_i^{(n)}] \right] \text{ for } i=1\ldots 2^B
\end{align*}
where to satisfy boundary conditions we have $\tau_0=-\infty$ and $\tau_{2^B}=\infty$. The algorithm iterates the above steps until convergence.

Figure \ref{fig:lm_quant} compares the quantization levels of a $7$-bit floating point (E3M3) quantizer (left) to a $7$-bit Lloyd-Max quantizer (right) when quantizing a layer of weights from the GPT3-126M model at a per-tensor granularity. As shown, the Lloyd-Max quantizer achieves substantially lower quantization MSE. Further, Table \ref{tab:FP7_vs_LM7} shows the superior perplexity achieved by Lloyd-Max quantizers for bitwidths of $7$, $6$ and $5$. The difference between the quantizers is clear at 5 bits, where per-tensor FP quantization incurs a drastic and unacceptable increase in perplexity, while Lloyd-Max quantization incurs a much smaller increase. Nevertheless, we note that even the optimal Lloyd-Max quantizer incurs a notable ($\sim 1.5$) increase in perplexity due to the coarse granularity of quantization. 

\begin{figure}[h]
  \centering
  \includegraphics[width=0.7\linewidth]{sections/figures/LM7_FP7.pdf}
  \caption{\small Quantization levels and the corresponding quantization MSE of Floating Point (left) vs Lloyd-Max (right) Quantizers for a layer of weights in the GPT3-126M model.}
  \label{fig:lm_quant}
\end{figure}

\begin{table}[h]\scriptsize
\begin{center}
\caption{\label{tab:FP7_vs_LM7} \small Comparing perplexity (lower is better) achieved by floating point quantizers and Lloyd-Max quantizers on a GPT3-126M model for the Wikitext-103 dataset.}
\begin{tabular}{c|cc|c}
\hline
 \multirow{2}{*}{\textbf{Bitwidth}} & \multicolumn{2}{|c|}{\textbf{Floating-Point Quantizer}} & \textbf{Lloyd-Max Quantizer} \\
 & Best Format & Wikitext-103 Perplexity & Wikitext-103 Perplexity \\
\hline
7 & E3M3 & 18.32 & 18.27 \\
6 & E3M2 & 19.07 & 18.51 \\
5 & E4M0 & 43.89 & 19.71 \\
\hline
\end{tabular}
\end{center}
\end{table}

\subsection{Proof of Local Optimality of LO-BCQ}
\label{subsec:lobcq_opt_proof}
For a given block $\bm{b}_j$, the quantization MSE during LO-BCQ can be empirically evaluated as $\frac{1}{L_b}\lVert \bm{b}_j- \bm{\hat{b}}_j\rVert^2_2$ where $\bm{\hat{b}}_j$ is computed from equation (\ref{eq:clustered_quantization_definition}) as $C_{f(\bm{b}_j)}(\bm{b}_j)$. Further, for a given block cluster $\mathcal{B}_i$, we compute the quantization MSE as $\frac{1}{|\mathcal{B}_{i}|}\sum_{\bm{b} \in \mathcal{B}_{i}} \frac{1}{L_b}\lVert \bm{b}- C_i^{(n)}(\bm{b})\rVert^2_2$. Therefore, at the end of iteration $n$, we evaluate the overall quantization MSE $J^{(n)}$ for a given operand $\bm{X}$ composed of $N_c$ block clusters as:
\begin{align*}
    \label{eq:mse_iter_n}
    J^{(n)} = \frac{1}{N_c} \sum_{i=1}^{N_c} \frac{1}{|\mathcal{B}_{i}^{(n)}|}\sum_{\bm{v} \in \mathcal{B}_{i}^{(n)}} \frac{1}{L_b}\lVert \bm{b}- B_i^{(n)}(\bm{b})\rVert^2_2
\end{align*}

At the end of iteration $n$, the codebooks are updated from $\mathcal{C}^{(n-1)}$ to $\mathcal{C}^{(n)}$. However, the mapping of a given vector $\bm{b}_j$ to quantizers $\mathcal{C}^{(n)}$ remains as  $f^{(n)}(\bm{b}_j)$. At the next iteration, during the vector clustering step, $f^{(n+1)}(\bm{b}_j)$ finds new mapping of $\bm{b}_j$ to updated codebooks $\mathcal{C}^{(n)}$ such that the quantization MSE over the candidate codebooks is minimized. Therefore, we obtain the following result for $\bm{b}_j$:
\begin{align*}
\frac{1}{L_b}\lVert \bm{b}_j - C_{f^{(n+1)}(\bm{b}_j)}^{(n)}(\bm{b}_j)\rVert^2_2 \le \frac{1}{L_b}\lVert \bm{b}_j - C_{f^{(n)}(\bm{b}_j)}^{(n)}(\bm{b}_j)\rVert^2_2
\end{align*}

That is, quantizing $\bm{b}_j$ at the end of the block clustering step of iteration $n+1$ results in lower quantization MSE compared to quantizing at the end of iteration $n$. Since this is true for all $\bm{b} \in \bm{X}$, we assert the following:
\begin{equation}
\begin{split}
\label{eq:mse_ineq_1}
    \tilde{J}^{(n+1)} &= \frac{1}{N_c} \sum_{i=1}^{N_c} \frac{1}{|\mathcal{B}_{i}^{(n+1)}|}\sum_{\bm{b} \in \mathcal{B}_{i}^{(n+1)}} \frac{1}{L_b}\lVert \bm{b} - C_i^{(n)}(b)\rVert^2_2 \le J^{(n)}
\end{split}
\end{equation}
where $\tilde{J}^{(n+1)}$ is the the quantization MSE after the vector clustering step at iteration $n+1$.

Next, during the codebook update step (\ref{eq:quantizers_update}) at iteration $n+1$, the per-cluster codebooks $\mathcal{C}^{(n)}$ are updated to $\mathcal{C}^{(n+1)}$ by invoking the Lloyd-Max algorithm \citep{Lloyd}. We know that for any given value distribution, the Lloyd-Max algorithm minimizes the quantization MSE. Therefore, for a given vector cluster $\mathcal{B}_i$ we obtain the following result:

\begin{equation}
    \frac{1}{|\mathcal{B}_{i}^{(n+1)}|}\sum_{\bm{b} \in \mathcal{B}_{i}^{(n+1)}} \frac{1}{L_b}\lVert \bm{b}- C_i^{(n+1)}(\bm{b})\rVert^2_2 \le \frac{1}{|\mathcal{B}_{i}^{(n+1)}|}\sum_{\bm{b} \in \mathcal{B}_{i}^{(n+1)}} \frac{1}{L_b}\lVert \bm{b}- C_i^{(n)}(\bm{b})\rVert^2_2
\end{equation}

The above equation states that quantizing the given block cluster $\mathcal{B}_i$ after updating the associated codebook from $C_i^{(n)}$ to $C_i^{(n+1)}$ results in lower quantization MSE. Since this is true for all the block clusters, we derive the following result: 
\begin{equation}
\begin{split}
\label{eq:mse_ineq_2}
     J^{(n+1)} &= \frac{1}{N_c} \sum_{i=1}^{N_c} \frac{1}{|\mathcal{B}_{i}^{(n+1)}|}\sum_{\bm{b} \in \mathcal{B}_{i}^{(n+1)}} \frac{1}{L_b}\lVert \bm{b}- C_i^{(n+1)}(\bm{b})\rVert^2_2  \le \tilde{J}^{(n+1)}   
\end{split}
\end{equation}

Following (\ref{eq:mse_ineq_1}) and (\ref{eq:mse_ineq_2}), we find that the quantization MSE is non-increasing for each iteration, that is, $J^{(1)} \ge J^{(2)} \ge J^{(3)} \ge \ldots \ge J^{(M)}$ where $M$ is the maximum number of iterations. 
%Therefore, we can say that if the algorithm converges, then it must be that it has converged to a local minimum. 
\hfill $\blacksquare$


\begin{figure}
    \begin{center}
    \includegraphics[width=0.5\textwidth]{sections//figures/mse_vs_iter.pdf}
    \end{center}
    \caption{\small NMSE vs iterations during LO-BCQ compared to other block quantization proposals}
    \label{fig:nmse_vs_iter}
\end{figure}

Figure \ref{fig:nmse_vs_iter} shows the empirical convergence of LO-BCQ across several block lengths and number of codebooks. Also, the MSE achieved by LO-BCQ is compared to baselines such as MXFP and VSQ. As shown, LO-BCQ converges to a lower MSE than the baselines. Further, we achieve better convergence for larger number of codebooks ($N_c$) and for a smaller block length ($L_b$), both of which increase the bitwidth of BCQ (see Eq \ref{eq:bitwidth_bcq}).


\subsection{Additional Accuracy Results}
%Table \ref{tab:lobcq_config} lists the various LOBCQ configurations and their corresponding bitwidths.
\begin{table}
\setlength{\tabcolsep}{4.75pt}
\begin{center}
\caption{\label{tab:lobcq_config} Various LO-BCQ configurations and their bitwidths.}
\begin{tabular}{|c||c|c|c|c||c|c||c|} 
\hline
 & \multicolumn{4}{|c||}{$L_b=8$} & \multicolumn{2}{|c||}{$L_b=4$} & $L_b=2$ \\
 \hline
 \backslashbox{$L_A$\kern-1em}{\kern-1em$N_c$} & 2 & 4 & 8 & 16 & 2 & 4 & 2 \\
 \hline
 64 & 4.25 & 4.375 & 4.5 & 4.625 & 4.375 & 4.625 & 4.625\\
 \hline
 32 & 4.375 & 4.5 & 4.625& 4.75 & 4.5 & 4.75 & 4.75 \\
 \hline
 16 & 4.625 & 4.75& 4.875 & 5 & 4.75 & 5 & 5 \\
 \hline
\end{tabular}
\end{center}
\end{table}

%\subsection{Perplexity achieved by various LO-BCQ configurations on Wikitext-103 dataset}

\begin{table} \centering
\begin{tabular}{|c||c|c|c|c||c|c||c|} 
\hline
 $L_b \rightarrow$& \multicolumn{4}{c||}{8} & \multicolumn{2}{c||}{4} & 2\\
 \hline
 \backslashbox{$L_A$\kern-1em}{\kern-1em$N_c$} & 2 & 4 & 8 & 16 & 2 & 4 & 2  \\
 %$N_c \rightarrow$ & 2 & 4 & 8 & 16 & 2 & 4 & 2 \\
 \hline
 \hline
 \multicolumn{8}{c}{GPT3-1.3B (FP32 PPL = 9.98)} \\ 
 \hline
 \hline
 64 & 10.40 & 10.23 & 10.17 & 10.15 &  10.28 & 10.18 & 10.19 \\
 \hline
 32 & 10.25 & 10.20 & 10.15 & 10.12 &  10.23 & 10.17 & 10.17 \\
 \hline
 16 & 10.22 & 10.16 & 10.10 & 10.09 &  10.21 & 10.14 & 10.16 \\
 \hline
  \hline
 \multicolumn{8}{c}{GPT3-8B (FP32 PPL = 7.38)} \\ 
 \hline
 \hline
 64 & 7.61 & 7.52 & 7.48 &  7.47 &  7.55 &  7.49 & 7.50 \\
 \hline
 32 & 7.52 & 7.50 & 7.46 &  7.45 &  7.52 &  7.48 & 7.48  \\
 \hline
 16 & 7.51 & 7.48 & 7.44 &  7.44 &  7.51 &  7.49 & 7.47  \\
 \hline
\end{tabular}
\caption{\label{tab:ppl_gpt3_abalation} Wikitext-103 perplexity across GPT3-1.3B and 8B models.}
\end{table}

\begin{table} \centering
\begin{tabular}{|c||c|c|c|c||} 
\hline
 $L_b \rightarrow$& \multicolumn{4}{c||}{8}\\
 \hline
 \backslashbox{$L_A$\kern-1em}{\kern-1em$N_c$} & 2 & 4 & 8 & 16 \\
 %$N_c \rightarrow$ & 2 & 4 & 8 & 16 & 2 & 4 & 2 \\
 \hline
 \hline
 \multicolumn{5}{|c|}{Llama2-7B (FP32 PPL = 5.06)} \\ 
 \hline
 \hline
 64 & 5.31 & 5.26 & 5.19 & 5.18  \\
 \hline
 32 & 5.23 & 5.25 & 5.18 & 5.15  \\
 \hline
 16 & 5.23 & 5.19 & 5.16 & 5.14  \\
 \hline
 \multicolumn{5}{|c|}{Nemotron4-15B (FP32 PPL = 5.87)} \\ 
 \hline
 \hline
 64  & 6.3 & 6.20 & 6.13 & 6.08  \\
 \hline
 32  & 6.24 & 6.12 & 6.07 & 6.03  \\
 \hline
 16  & 6.12 & 6.14 & 6.04 & 6.02  \\
 \hline
 \multicolumn{5}{|c|}{Nemotron4-340B (FP32 PPL = 3.48)} \\ 
 \hline
 \hline
 64 & 3.67 & 3.62 & 3.60 & 3.59 \\
 \hline
 32 & 3.63 & 3.61 & 3.59 & 3.56 \\
 \hline
 16 & 3.61 & 3.58 & 3.57 & 3.55 \\
 \hline
\end{tabular}
\caption{\label{tab:ppl_llama7B_nemo15B} Wikitext-103 perplexity compared to FP32 baseline in Llama2-7B and Nemotron4-15B, 340B models}
\end{table}

%\subsection{Perplexity achieved by various LO-BCQ configurations on MMLU dataset}


\begin{table} \centering
\begin{tabular}{|c||c|c|c|c||c|c|c|c|} 
\hline
 $L_b \rightarrow$& \multicolumn{4}{c||}{8} & \multicolumn{4}{c||}{8}\\
 \hline
 \backslashbox{$L_A$\kern-1em}{\kern-1em$N_c$} & 2 & 4 & 8 & 16 & 2 & 4 & 8 & 16  \\
 %$N_c \rightarrow$ & 2 & 4 & 8 & 16 & 2 & 4 & 2 \\
 \hline
 \hline
 \multicolumn{5}{|c|}{Llama2-7B (FP32 Accuracy = 45.8\%)} & \multicolumn{4}{|c|}{Llama2-70B (FP32 Accuracy = 69.12\%)} \\ 
 \hline
 \hline
 64 & 43.9 & 43.4 & 43.9 & 44.9 & 68.07 & 68.27 & 68.17 & 68.75 \\
 \hline
 32 & 44.5 & 43.8 & 44.9 & 44.5 & 68.37 & 68.51 & 68.35 & 68.27  \\
 \hline
 16 & 43.9 & 42.7 & 44.9 & 45 & 68.12 & 68.77 & 68.31 & 68.59  \\
 \hline
 \hline
 \multicolumn{5}{|c|}{GPT3-22B (FP32 Accuracy = 38.75\%)} & \multicolumn{4}{|c|}{Nemotron4-15B (FP32 Accuracy = 64.3\%)} \\ 
 \hline
 \hline
 64 & 36.71 & 38.85 & 38.13 & 38.92 & 63.17 & 62.36 & 63.72 & 64.09 \\
 \hline
 32 & 37.95 & 38.69 & 39.45 & 38.34 & 64.05 & 62.30 & 63.8 & 64.33  \\
 \hline
 16 & 38.88 & 38.80 & 38.31 & 38.92 & 63.22 & 63.51 & 63.93 & 64.43  \\
 \hline
\end{tabular}
\caption{\label{tab:mmlu_abalation} Accuracy on MMLU dataset across GPT3-22B, Llama2-7B, 70B and Nemotron4-15B models.}
\end{table}


%\subsection{Perplexity achieved by various LO-BCQ configurations on LM evaluation harness}

\begin{table} \centering
\begin{tabular}{|c||c|c|c|c||c|c|c|c|} 
\hline
 $L_b \rightarrow$& \multicolumn{4}{c||}{8} & \multicolumn{4}{c||}{8}\\
 \hline
 \backslashbox{$L_A$\kern-1em}{\kern-1em$N_c$} & 2 & 4 & 8 & 16 & 2 & 4 & 8 & 16  \\
 %$N_c \rightarrow$ & 2 & 4 & 8 & 16 & 2 & 4 & 2 \\
 \hline
 \hline
 \multicolumn{5}{|c|}{Race (FP32 Accuracy = 37.51\%)} & \multicolumn{4}{|c|}{Boolq (FP32 Accuracy = 64.62\%)} \\ 
 \hline
 \hline
 64 & 36.94 & 37.13 & 36.27 & 37.13 & 63.73 & 62.26 & 63.49 & 63.36 \\
 \hline
 32 & 37.03 & 36.36 & 36.08 & 37.03 & 62.54 & 63.51 & 63.49 & 63.55  \\
 \hline
 16 & 37.03 & 37.03 & 36.46 & 37.03 & 61.1 & 63.79 & 63.58 & 63.33  \\
 \hline
 \hline
 \multicolumn{5}{|c|}{Winogrande (FP32 Accuracy = 58.01\%)} & \multicolumn{4}{|c|}{Piqa (FP32 Accuracy = 74.21\%)} \\ 
 \hline
 \hline
 64 & 58.17 & 57.22 & 57.85 & 58.33 & 73.01 & 73.07 & 73.07 & 72.80 \\
 \hline
 32 & 59.12 & 58.09 & 57.85 & 58.41 & 73.01 & 73.94 & 72.74 & 73.18  \\
 \hline
 16 & 57.93 & 58.88 & 57.93 & 58.56 & 73.94 & 72.80 & 73.01 & 73.94  \\
 \hline
\end{tabular}
\caption{\label{tab:mmlu_abalation} Accuracy on LM evaluation harness tasks on GPT3-1.3B model.}
\end{table}

\begin{table} \centering
\begin{tabular}{|c||c|c|c|c||c|c|c|c|} 
\hline
 $L_b \rightarrow$& \multicolumn{4}{c||}{8} & \multicolumn{4}{c||}{8}\\
 \hline
 \backslashbox{$L_A$\kern-1em}{\kern-1em$N_c$} & 2 & 4 & 8 & 16 & 2 & 4 & 8 & 16  \\
 %$N_c \rightarrow$ & 2 & 4 & 8 & 16 & 2 & 4 & 2 \\
 \hline
 \hline
 \multicolumn{5}{|c|}{Race (FP32 Accuracy = 41.34\%)} & \multicolumn{4}{|c|}{Boolq (FP32 Accuracy = 68.32\%)} \\ 
 \hline
 \hline
 64 & 40.48 & 40.10 & 39.43 & 39.90 & 69.20 & 68.41 & 69.45 & 68.56 \\
 \hline
 32 & 39.52 & 39.52 & 40.77 & 39.62 & 68.32 & 67.43 & 68.17 & 69.30  \\
 \hline
 16 & 39.81 & 39.71 & 39.90 & 40.38 & 68.10 & 66.33 & 69.51 & 69.42  \\
 \hline
 \hline
 \multicolumn{5}{|c|}{Winogrande (FP32 Accuracy = 67.88\%)} & \multicolumn{4}{|c|}{Piqa (FP32 Accuracy = 78.78\%)} \\ 
 \hline
 \hline
 64 & 66.85 & 66.61 & 67.72 & 67.88 & 77.31 & 77.42 & 77.75 & 77.64 \\
 \hline
 32 & 67.25 & 67.72 & 67.72 & 67.00 & 77.31 & 77.04 & 77.80 & 77.37  \\
 \hline
 16 & 68.11 & 68.90 & 67.88 & 67.48 & 77.37 & 78.13 & 78.13 & 77.69  \\
 \hline
\end{tabular}
\caption{\label{tab:mmlu_abalation} Accuracy on LM evaluation harness tasks on GPT3-8B model.}
\end{table}

\begin{table} \centering
\begin{tabular}{|c||c|c|c|c||c|c|c|c|} 
\hline
 $L_b \rightarrow$& \multicolumn{4}{c||}{8} & \multicolumn{4}{c||}{8}\\
 \hline
 \backslashbox{$L_A$\kern-1em}{\kern-1em$N_c$} & 2 & 4 & 8 & 16 & 2 & 4 & 8 & 16  \\
 %$N_c \rightarrow$ & 2 & 4 & 8 & 16 & 2 & 4 & 2 \\
 \hline
 \hline
 \multicolumn{5}{|c|}{Race (FP32 Accuracy = 40.67\%)} & \multicolumn{4}{|c|}{Boolq (FP32 Accuracy = 76.54\%)} \\ 
 \hline
 \hline
 64 & 40.48 & 40.10 & 39.43 & 39.90 & 75.41 & 75.11 & 77.09 & 75.66 \\
 \hline
 32 & 39.52 & 39.52 & 40.77 & 39.62 & 76.02 & 76.02 & 75.96 & 75.35  \\
 \hline
 16 & 39.81 & 39.71 & 39.90 & 40.38 & 75.05 & 73.82 & 75.72 & 76.09  \\
 \hline
 \hline
 \multicolumn{5}{|c|}{Winogrande (FP32 Accuracy = 70.64\%)} & \multicolumn{4}{|c|}{Piqa (FP32 Accuracy = 79.16\%)} \\ 
 \hline
 \hline
 64 & 69.14 & 70.17 & 70.17 & 70.56 & 78.24 & 79.00 & 78.62 & 78.73 \\
 \hline
 32 & 70.96 & 69.69 & 71.27 & 69.30 & 78.56 & 79.49 & 79.16 & 78.89  \\
 \hline
 16 & 71.03 & 69.53 & 69.69 & 70.40 & 78.13 & 79.16 & 79.00 & 79.00  \\
 \hline
\end{tabular}
\caption{\label{tab:mmlu_abalation} Accuracy on LM evaluation harness tasks on GPT3-22B model.}
\end{table}

\begin{table} \centering
\begin{tabular}{|c||c|c|c|c||c|c|c|c|} 
\hline
 $L_b \rightarrow$& \multicolumn{4}{c||}{8} & \multicolumn{4}{c||}{8}\\
 \hline
 \backslashbox{$L_A$\kern-1em}{\kern-1em$N_c$} & 2 & 4 & 8 & 16 & 2 & 4 & 8 & 16  \\
 %$N_c \rightarrow$ & 2 & 4 & 8 & 16 & 2 & 4 & 2 \\
 \hline
 \hline
 \multicolumn{5}{|c|}{Race (FP32 Accuracy = 44.4\%)} & \multicolumn{4}{|c|}{Boolq (FP32 Accuracy = 79.29\%)} \\ 
 \hline
 \hline
 64 & 42.49 & 42.51 & 42.58 & 43.45 & 77.58 & 77.37 & 77.43 & 78.1 \\
 \hline
 32 & 43.35 & 42.49 & 43.64 & 43.73 & 77.86 & 75.32 & 77.28 & 77.86  \\
 \hline
 16 & 44.21 & 44.21 & 43.64 & 42.97 & 78.65 & 77 & 76.94 & 77.98  \\
 \hline
 \hline
 \multicolumn{5}{|c|}{Winogrande (FP32 Accuracy = 69.38\%)} & \multicolumn{4}{|c|}{Piqa (FP32 Accuracy = 78.07\%)} \\ 
 \hline
 \hline
 64 & 68.9 & 68.43 & 69.77 & 68.19 & 77.09 & 76.82 & 77.09 & 77.86 \\
 \hline
 32 & 69.38 & 68.51 & 68.82 & 68.90 & 78.07 & 76.71 & 78.07 & 77.86  \\
 \hline
 16 & 69.53 & 67.09 & 69.38 & 68.90 & 77.37 & 77.8 & 77.91 & 77.69  \\
 \hline
\end{tabular}
\caption{\label{tab:mmlu_abalation} Accuracy on LM evaluation harness tasks on Llama2-7B model.}
\end{table}

\begin{table} \centering
\begin{tabular}{|c||c|c|c|c||c|c|c|c|} 
\hline
 $L_b \rightarrow$& \multicolumn{4}{c||}{8} & \multicolumn{4}{c||}{8}\\
 \hline
 \backslashbox{$L_A$\kern-1em}{\kern-1em$N_c$} & 2 & 4 & 8 & 16 & 2 & 4 & 8 & 16  \\
 %$N_c \rightarrow$ & 2 & 4 & 8 & 16 & 2 & 4 & 2 \\
 \hline
 \hline
 \multicolumn{5}{|c|}{Race (FP32 Accuracy = 48.8\%)} & \multicolumn{4}{|c|}{Boolq (FP32 Accuracy = 85.23\%)} \\ 
 \hline
 \hline
 64 & 49.00 & 49.00 & 49.28 & 48.71 & 82.82 & 84.28 & 84.03 & 84.25 \\
 \hline
 32 & 49.57 & 48.52 & 48.33 & 49.28 & 83.85 & 84.46 & 84.31 & 84.93  \\
 \hline
 16 & 49.85 & 49.09 & 49.28 & 48.99 & 85.11 & 84.46 & 84.61 & 83.94  \\
 \hline
 \hline
 \multicolumn{5}{|c|}{Winogrande (FP32 Accuracy = 79.95\%)} & \multicolumn{4}{|c|}{Piqa (FP32 Accuracy = 81.56\%)} \\ 
 \hline
 \hline
 64 & 78.77 & 78.45 & 78.37 & 79.16 & 81.45 & 80.69 & 81.45 & 81.5 \\
 \hline
 32 & 78.45 & 79.01 & 78.69 & 80.66 & 81.56 & 80.58 & 81.18 & 81.34  \\
 \hline
 16 & 79.95 & 79.56 & 79.79 & 79.72 & 81.28 & 81.66 & 81.28 & 80.96  \\
 \hline
\end{tabular}
\caption{\label{tab:mmlu_abalation} Accuracy on LM evaluation harness tasks on Llama2-70B model.}
\end{table}

%\section{MSE Studies}
%\textcolor{red}{TODO}


\subsection{Number Formats and Quantization Method}
\label{subsec:numFormats_quantMethod}
\subsubsection{Integer Format}
An $n$-bit signed integer (INT) is typically represented with a 2s-complement format \citep{yao2022zeroquant,xiao2023smoothquant,dai2021vsq}, where the most significant bit denotes the sign.

\subsubsection{Floating Point Format}
An $n$-bit signed floating point (FP) number $x$ comprises of a 1-bit sign ($x_{\mathrm{sign}}$), $B_m$-bit mantissa ($x_{\mathrm{mant}}$) and $B_e$-bit exponent ($x_{\mathrm{exp}}$) such that $B_m+B_e=n-1$. The associated constant exponent bias ($E_{\mathrm{bias}}$) is computed as $(2^{{B_e}-1}-1)$. We denote this format as $E_{B_e}M_{B_m}$.  

\subsubsection{Quantization Scheme}
\label{subsec:quant_method}
A quantization scheme dictates how a given unquantized tensor is converted to its quantized representation. We consider FP formats for the purpose of illustration. Given an unquantized tensor $\bm{X}$ and an FP format $E_{B_e}M_{B_m}$, we first, we compute the quantization scale factor $s_X$ that maps the maximum absolute value of $\bm{X}$ to the maximum quantization level of the $E_{B_e}M_{B_m}$ format as follows:
\begin{align}
\label{eq:sf}
    s_X = \frac{\mathrm{max}(|\bm{X}|)}{\mathrm{max}(E_{B_e}M_{B_m})}
\end{align}
In the above equation, $|\cdot|$ denotes the absolute value function.

Next, we scale $\bm{X}$ by $s_X$ and quantize it to $\hat{\bm{X}}$ by rounding it to the nearest quantization level of $E_{B_e}M_{B_m}$ as:

\begin{align}
\label{eq:tensor_quant}
    \hat{\bm{X}} = \text{round-to-nearest}\left(\frac{\bm{X}}{s_X}, E_{B_e}M_{B_m}\right)
\end{align}

We perform dynamic max-scaled quantization \citep{wu2020integer}, where the scale factor $s$ for activations is dynamically computed during runtime.

\subsection{Vector Scaled Quantization}
\begin{wrapfigure}{r}{0.35\linewidth}
  \centering
  \includegraphics[width=\linewidth]{sections/figures/vsquant.jpg}
  \caption{\small Vectorwise decomposition for per-vector scaled quantization (VSQ \citep{dai2021vsq}).}
  \label{fig:vsquant}
\end{wrapfigure}
During VSQ \citep{dai2021vsq}, the operand tensors are decomposed into 1D vectors in a hardware friendly manner as shown in Figure \ref{fig:vsquant}. Since the decomposed tensors are used as operands in matrix multiplications during inference, it is beneficial to perform this decomposition along the reduction dimension of the multiplication. The vectorwise quantization is performed similar to tensorwise quantization described in Equations \ref{eq:sf} and \ref{eq:tensor_quant}, where a scale factor $s_v$ is required for each vector $\bm{v}$ that maps the maximum absolute value of that vector to the maximum quantization level. While smaller vector lengths can lead to larger accuracy gains, the associated memory and computational overheads due to the per-vector scale factors increases. To alleviate these overheads, VSQ \citep{dai2021vsq} proposed a second level quantization of the per-vector scale factors to unsigned integers, while MX \citep{rouhani2023shared} quantizes them to integer powers of 2 (denoted as $2^{INT}$).

\subsubsection{MX Format}
The MX format proposed in \citep{rouhani2023microscaling} introduces the concept of sub-block shifting. For every two scalar elements of $b$-bits each, there is a shared exponent bit. The value of this exponent bit is determined through an empirical analysis that targets minimizing quantization MSE. We note that the FP format $E_{1}M_{b}$ is strictly better than MX from an accuracy perspective since it allocates a dedicated exponent bit to each scalar as opposed to sharing it across two scalars. Therefore, we conservatively bound the accuracy of a $b+2$-bit signed MX format with that of a $E_{1}M_{b}$ format in our comparisons. For instance, we use E1M2 format as a proxy for MX4.

\begin{figure}
    \centering
    \includegraphics[width=1\linewidth]{sections//figures/BlockFormats.pdf}
    \caption{\small Comparing LO-BCQ to MX format.}
    \label{fig:block_formats}
\end{figure}

Figure \ref{fig:block_formats} compares our $4$-bit LO-BCQ block format to MX \citep{rouhani2023microscaling}. As shown, both LO-BCQ and MX decompose a given operand tensor into block arrays and each block array into blocks. Similar to MX, we find that per-block quantization ($L_b < L_A$) leads to better accuracy due to increased flexibility. While MX achieves this through per-block $1$-bit micro-scales, we associate a dedicated codebook to each block through a per-block codebook selector. Further, MX quantizes the per-block array scale-factor to E8M0 format without per-tensor scaling. In contrast during LO-BCQ, we find that per-tensor scaling combined with quantization of per-block array scale-factor to E4M3 format results in superior inference accuracy across models. 

\balance
\bibliographystyle{ACM-Reference-Format}
% \bibliography{retrieval-methods}
% This must be in the first 5 lines to tell arXiv to use pdfLaTeX, which is strongly recommended.
\pdfoutput=1
% In particular, the hyperref package requires pdfLaTeX in order to break URLs across lines.

\documentclass[11pt]{article}

% Change "review" to "final" to generate the final (sometimes called camera-ready) version.
% Change to "preprint" to generate a non-anonymous version with page numbers.
\usepackage{acl}

% Standard package includes
\usepackage{times}
\usepackage{latexsym}

% Draw tables
\usepackage{booktabs}
\usepackage{multirow}
\usepackage{xcolor}
\usepackage{colortbl}
\usepackage{array} 
\usepackage{amsmath}

\newcolumntype{C}{>{\centering\arraybackslash}p{0.07\textwidth}}
% For proper rendering and hyphenation of words containing Latin characters (including in bib files)
\usepackage[T1]{fontenc}
% For Vietnamese characters
% \usepackage[T5]{fontenc}
% See https://www.latex-project.org/help/documentation/encguide.pdf for other character sets
% This assumes your files are encoded as UTF8
\usepackage[utf8]{inputenc}

% This is not strictly necessary, and may be commented out,
% but it will improve the layout of the manuscript,
% and will typically save some space.
\usepackage{microtype}
\DeclareMathOperator*{\argmax}{arg\,max}
% This is also not strictly necessary, and may be commented out.
% However, it will improve the aesthetics of text in
% the typewriter font.
\usepackage{inconsolata}

%Including images in your LaTeX document requires adding
%additional package(s)
\usepackage{graphicx}
% If the title and author information does not fit in the area allocated, uncomment the following
%
%\setlength\titlebox{<dim>}
%
% and set <dim> to something 5cm or larger.

\title{Wi-Chat: Large Language Model Powered Wi-Fi Sensing}

% Author information can be set in various styles:
% For several authors from the same institution:
% \author{Author 1 \and ... \and Author n \\
%         Address line \\ ... \\ Address line}
% if the names do not fit well on one line use
%         Author 1 \\ {\bf Author 2} \\ ... \\ {\bf Author n} \\
% For authors from different institutions:
% \author{Author 1 \\ Address line \\  ... \\ Address line
%         \And  ... \And
%         Author n \\ Address line \\ ... \\ Address line}
% To start a separate ``row'' of authors use \AND, as in
% \author{Author 1 \\ Address line \\  ... \\ Address line
%         \AND
%         Author 2 \\ Address line \\ ... \\ Address line \And
%         Author 3 \\ Address line \\ ... \\ Address line}

% \author{First Author \\
%   Affiliation / Address line 1 \\
%   Affiliation / Address line 2 \\
%   Affiliation / Address line 3 \\
%   \texttt{email@domain} \\\And
%   Second Author \\
%   Affiliation / Address line 1 \\
%   Affiliation / Address line 2 \\
%   Affiliation / Address line 3 \\
%   \texttt{email@domain} \\}
% \author{Haohan Yuan \qquad Haopeng Zhang\thanks{corresponding author} \\ 
%   ALOHA Lab, University of Hawaii at Manoa \\
%   % Affiliation / Address line 2 \\
%   % Affiliation / Address line 3 \\
%   \texttt{\{haohany,haopengz\}@hawaii.edu}}
  
\author{
{Haopeng Zhang$\dag$\thanks{These authors contributed equally to this work.}, Yili Ren$\ddagger$\footnotemark[1], Haohan Yuan$\dag$, Jingzhe Zhang$\ddagger$, Yitong Shen$\ddagger$} \\
ALOHA Lab, University of Hawaii at Manoa$\dag$, University of South Florida$\ddagger$ \\
\{haopengz, haohany\}@hawaii.edu\\
\{yiliren, jingzhe, shen202\}@usf.edu\\}



  
%\author{
%  \textbf{First Author\textsuperscript{1}},
%  \textbf{Second Author\textsuperscript{1,2}},
%  \textbf{Third T. Author\textsuperscript{1}},
%  \textbf{Fourth Author\textsuperscript{1}},
%\\
%  \textbf{Fifth Author\textsuperscript{1,2}},
%  \textbf{Sixth Author\textsuperscript{1}},
%  \textbf{Seventh Author\textsuperscript{1}},
%  \textbf{Eighth Author \textsuperscript{1,2,3,4}},
%\\
%  \textbf{Ninth Author\textsuperscript{1}},
%  \textbf{Tenth Author\textsuperscript{1}},
%  \textbf{Eleventh E. Author\textsuperscript{1,2,3,4,5}},
%  \textbf{Twelfth Author\textsuperscript{1}},
%\\
%  \textbf{Thirteenth Author\textsuperscript{3}},
%  \textbf{Fourteenth F. Author\textsuperscript{2,4}},
%  \textbf{Fifteenth Author\textsuperscript{1}},
%  \textbf{Sixteenth Author\textsuperscript{1}},
%\\
%  \textbf{Seventeenth S. Author\textsuperscript{4,5}},
%  \textbf{Eighteenth Author\textsuperscript{3,4}},
%  \textbf{Nineteenth N. Author\textsuperscript{2,5}},
%  \textbf{Twentieth Author\textsuperscript{1}}
%\\
%\\
%  \textsuperscript{1}Affiliation 1,
%  \textsuperscript{2}Affiliation 2,
%  \textsuperscript{3}Affiliation 3,
%  \textsuperscript{4}Affiliation 4,
%  \textsuperscript{5}Affiliation 5
%\\
%  \small{
%    \textbf{Correspondence:} \href{mailto:email@domain}{email@domain}
%  }
%}

\begin{document}
\maketitle
\begin{abstract}
Recent advancements in Large Language Models (LLMs) have demonstrated remarkable capabilities across diverse tasks. However, their potential to integrate physical model knowledge for real-world signal interpretation remains largely unexplored. In this work, we introduce Wi-Chat, the first LLM-powered Wi-Fi-based human activity recognition system. We demonstrate that LLMs can process raw Wi-Fi signals and infer human activities by incorporating Wi-Fi sensing principles into prompts. Our approach leverages physical model insights to guide LLMs in interpreting Channel State Information (CSI) data without traditional signal processing techniques. Through experiments on real-world Wi-Fi datasets, we show that LLMs exhibit strong reasoning capabilities, achieving zero-shot activity recognition. These findings highlight a new paradigm for Wi-Fi sensing, expanding LLM applications beyond conventional language tasks and enhancing the accessibility of wireless sensing for real-world deployments.
\end{abstract}

\section{Introduction}

In today’s rapidly evolving digital landscape, the transformative power of web technologies has redefined not only how services are delivered but also how complex tasks are approached. Web-based systems have become increasingly prevalent in risk control across various domains. This widespread adoption is due their accessibility, scalability, and ability to remotely connect various types of users. For example, these systems are used for process safety management in industry~\cite{kannan2016web}, safety risk early warning in urban construction~\cite{ding2013development}, and safe monitoring of infrastructural systems~\cite{repetto2018web}. Within these web-based risk management systems, the source search problem presents a huge challenge. Source search refers to the task of identifying the origin of a risky event, such as a gas leak and the emission point of toxic substances. This source search capability is crucial for effective risk management and decision-making.

Traditional approaches to implementing source search capabilities into the web systems often rely on solely algorithmic solutions~\cite{ristic2016study}. These methods, while relatively straightforward to implement, often struggle to achieve acceptable performances due to algorithmic local optima and complex unknown environments~\cite{zhao2020searching}. More recently, web crowdsourcing has emerged as a promising alternative for tackling the source search problem by incorporating human efforts in these web systems on-the-fly~\cite{zhao2024user}. This approach outsources the task of addressing issues encountered during the source search process to human workers, leveraging their capabilities to enhance system performance.

These solutions often employ a human-AI collaborative way~\cite{zhao2023leveraging} where algorithms handle exploration-exploitation and report the encountered problems while human workers resolve complex decision-making bottlenecks to help the algorithms getting rid of local deadlocks~\cite{zhao2022crowd}. Although effective, this paradigm suffers from two inherent limitations: increased operational costs from continuous human intervention, and slow response times of human workers due to sequential decision-making. These challenges motivate our investigation into developing autonomous systems that preserve human-like reasoning capabilities while reducing dependency on massive crowdsourced labor.

Furthermore, recent advancements in large language models (LLMs)~\cite{chang2024survey} and multi-modal LLMs (MLLMs)~\cite{huang2023chatgpt} have unveiled promising avenues for addressing these challenges. One clear opportunity involves the seamless integration of visual understanding and linguistic reasoning for robust decision-making in search tasks. However, whether large models-assisted source search is really effective and efficient for improving the current source search algorithms~\cite{ji2022source} remains unknown. \textit{To address the research gap, we are particularly interested in answering the following two research questions in this work:}

\textbf{\textit{RQ1: }}How can source search capabilities be integrated into web-based systems to support decision-making in time-sensitive risk management scenarios? 
% \sq{I mention ``time-sensitive'' here because I feel like we shall say something about the response time -- LLM has to be faster than humans}

\textbf{\textit{RQ2: }}How can MLLMs and LLMs enhance the effectiveness and efficiency of existing source search algorithms? 

% \textit{\textbf{RQ2:}} To what extent does the performance of large models-assisted search align with or approach the effectiveness of human-AI collaborative search? 

To answer the research questions, we propose a novel framework called Auto-\
S$^2$earch (\textbf{Auto}nomous \textbf{S}ource \textbf{Search}) and implement a prototype system that leverages advanced web technologies to simulate real-world conditions for zero-shot source search. Unlike traditional methods that rely on pre-defined heuristics or extensive human intervention, AutoS$^2$earch employs a carefully designed prompt that encapsulates human rationales, thereby guiding the MLLM to generate coherent and accurate scene descriptions from visual inputs about four directional choices. Based on these language-based descriptions, the LLM is enabled to determine the optimal directional choice through chain-of-thought (CoT) reasoning. Comprehensive empirical validation demonstrates that AutoS$^2$-\ 
earch achieves a success rate of 95–98\%, closely approaching the performance of human-AI collaborative search across 20 benchmark scenarios~\cite{zhao2023leveraging}. 

Our work indicates that the role of humans in future web crowdsourcing tasks may evolve from executors to validators or supervisors. Furthermore, incorporating explanations of LLM decisions into web-based system interfaces has the potential to help humans enhance task performance in risk control.






\section{Related Work}
\label{sec:relatedworks}

% \begin{table*}[t]
% \centering 
% \renewcommand\arraystretch{0.98}
% \fontsize{8}{10}\selectfont \setlength{\tabcolsep}{0.4em}
% \begin{tabular}{@{}lc|cc|cc|cc@{}}
% \toprule
% \textbf{Methods}           & \begin{tabular}[c]{@{}c@{}}\textbf{Training}\\ \textbf{Paradigm}\end{tabular} & \begin{tabular}[c]{@{}c@{}}\textbf{$\#$ PT Data}\\ \textbf{(Tokens)}\end{tabular} & \begin{tabular}[c]{@{}c@{}}\textbf{$\#$ IFT Data}\\ \textbf{(Samples)}\end{tabular} & \textbf{Code}  & \begin{tabular}[c]{@{}c@{}}\textbf{Natural}\\ \textbf{Language}\end{tabular} & \begin{tabular}[c]{@{}c@{}}\textbf{Action}\\ \textbf{Trajectories}\end{tabular} & \begin{tabular}[c]{@{}c@{}}\textbf{API}\\ \textbf{Documentation}\end{tabular}\\ \midrule 
% NexusRaven~\citep{srinivasan2023nexusraven} & IFT & - & - & \textcolor{green}{\CheckmarkBold} & \textcolor{green}{\CheckmarkBold} &\textcolor{red}{\XSolidBrush}&\textcolor{red}{\XSolidBrush}\\
% AgentInstruct~\citep{zeng2023agenttuning} & IFT & - & 2k & \textcolor{green}{\CheckmarkBold} & \textcolor{green}{\CheckmarkBold} &\textcolor{red}{\XSolidBrush}&\textcolor{red}{\XSolidBrush} \\
% AgentEvol~\citep{xi2024agentgym} & IFT & - & 14.5k & \textcolor{green}{\CheckmarkBold} & \textcolor{green}{\CheckmarkBold} &\textcolor{green}{\CheckmarkBold}&\textcolor{red}{\XSolidBrush} \\
% Gorilla~\citep{patil2023gorilla}& IFT & - & 16k & \textcolor{green}{\CheckmarkBold} & \textcolor{green}{\CheckmarkBold} &\textcolor{red}{\XSolidBrush}&\textcolor{green}{\CheckmarkBold}\\
% OpenFunctions-v2~\citep{patil2023gorilla} & IFT & - & 65k & \textcolor{green}{\CheckmarkBold} & \textcolor{green}{\CheckmarkBold} &\textcolor{red}{\XSolidBrush}&\textcolor{green}{\CheckmarkBold}\\
% LAM~\citep{zhang2024agentohana} & IFT & - & 42.6k & \textcolor{green}{\CheckmarkBold} & \textcolor{green}{\CheckmarkBold} &\textcolor{green}{\CheckmarkBold}&\textcolor{red}{\XSolidBrush} \\
% xLAM~\citep{liu2024apigen} & IFT & - & 60k & \textcolor{green}{\CheckmarkBold} & \textcolor{green}{\CheckmarkBold} &\textcolor{green}{\CheckmarkBold}&\textcolor{red}{\XSolidBrush} \\\midrule
% LEMUR~\citep{xu2024lemur} & PT & 90B & 300k & \textcolor{green}{\CheckmarkBold} & \textcolor{green}{\CheckmarkBold} &\textcolor{green}{\CheckmarkBold}&\textcolor{red}{\XSolidBrush}\\
% \rowcolor{teal!12} \method & PT & 103B & 95k & \textcolor{green}{\CheckmarkBold} & \textcolor{green}{\CheckmarkBold} & \textcolor{green}{\CheckmarkBold} & \textcolor{green}{\CheckmarkBold} \\
% \bottomrule
% \end{tabular}
% \caption{Summary of existing tuning- and pretraining-based LLM agents with their training sample sizes. "PT" and "IFT" denote "Pre-Training" and "Instruction Fine-Tuning", respectively. }
% \label{tab:related}
% \end{table*}

\begin{table*}[ht]
\begin{threeparttable}
\centering 
\renewcommand\arraystretch{0.98}
\fontsize{7}{9}\selectfont \setlength{\tabcolsep}{0.2em}
\begin{tabular}{@{}l|c|c|ccc|cc|cc|cccc@{}}
\toprule
\textbf{Methods} & \textbf{Datasets}           & \begin{tabular}[c]{@{}c@{}}\textbf{Training}\\ \textbf{Paradigm}\end{tabular} & \begin{tabular}[c]{@{}c@{}}\textbf{\# PT Data}\\ \textbf{(Tokens)}\end{tabular} & \begin{tabular}[c]{@{}c@{}}\textbf{\# IFT Data}\\ \textbf{(Samples)}\end{tabular} & \textbf{\# APIs} & \textbf{Code}  & \begin{tabular}[c]{@{}c@{}}\textbf{Nat.}\\ \textbf{Lang.}\end{tabular} & \begin{tabular}[c]{@{}c@{}}\textbf{Action}\\ \textbf{Traj.}\end{tabular} & \begin{tabular}[c]{@{}c@{}}\textbf{API}\\ \textbf{Doc.}\end{tabular} & \begin{tabular}[c]{@{}c@{}}\textbf{Func.}\\ \textbf{Call}\end{tabular} & \begin{tabular}[c]{@{}c@{}}\textbf{Multi.}\\ \textbf{Step}\end{tabular}  & \begin{tabular}[c]{@{}c@{}}\textbf{Plan}\\ \textbf{Refine}\end{tabular}  & \begin{tabular}[c]{@{}c@{}}\textbf{Multi.}\\ \textbf{Turn}\end{tabular}\\ \midrule 
\multicolumn{13}{l}{\emph{Instruction Finetuning-based LLM Agents for Intrinsic Reasoning}}  \\ \midrule
FireAct~\cite{chen2023fireact} & FireAct & IFT & - & 2.1K & 10 & \textcolor{red}{\XSolidBrush} &\textcolor{green}{\CheckmarkBold} &\textcolor{green}{\CheckmarkBold}  & \textcolor{red}{\XSolidBrush} &\textcolor{green}{\CheckmarkBold} & \textcolor{red}{\XSolidBrush} &\textcolor{green}{\CheckmarkBold} & \textcolor{red}{\XSolidBrush} \\
ToolAlpaca~\cite{tang2023toolalpaca} & ToolAlpaca & IFT & - & 4.0K & 400 & \textcolor{red}{\XSolidBrush} &\textcolor{green}{\CheckmarkBold} &\textcolor{green}{\CheckmarkBold} & \textcolor{red}{\XSolidBrush} &\textcolor{green}{\CheckmarkBold} & \textcolor{red}{\XSolidBrush}  &\textcolor{green}{\CheckmarkBold} & \textcolor{red}{\XSolidBrush}  \\
ToolLLaMA~\cite{qin2023toolllm} & ToolBench & IFT & - & 12.7K & 16,464 & \textcolor{red}{\XSolidBrush} &\textcolor{green}{\CheckmarkBold} &\textcolor{green}{\CheckmarkBold} &\textcolor{red}{\XSolidBrush} &\textcolor{green}{\CheckmarkBold}&\textcolor{green}{\CheckmarkBold}&\textcolor{green}{\CheckmarkBold} &\textcolor{green}{\CheckmarkBold}\\
AgentEvol~\citep{xi2024agentgym} & AgentTraj-L & IFT & - & 14.5K & 24 &\textcolor{red}{\XSolidBrush} & \textcolor{green}{\CheckmarkBold} &\textcolor{green}{\CheckmarkBold}&\textcolor{red}{\XSolidBrush} &\textcolor{green}{\CheckmarkBold}&\textcolor{red}{\XSolidBrush} &\textcolor{red}{\XSolidBrush} &\textcolor{green}{\CheckmarkBold}\\
Lumos~\cite{yin2024agent} & Lumos & IFT  & - & 20.0K & 16 &\textcolor{red}{\XSolidBrush} & \textcolor{green}{\CheckmarkBold} & \textcolor{green}{\CheckmarkBold} &\textcolor{red}{\XSolidBrush} & \textcolor{green}{\CheckmarkBold} & \textcolor{green}{\CheckmarkBold} &\textcolor{red}{\XSolidBrush} & \textcolor{green}{\CheckmarkBold}\\
Agent-FLAN~\cite{chen2024agent} & Agent-FLAN & IFT & - & 24.7K & 20 &\textcolor{red}{\XSolidBrush} & \textcolor{green}{\CheckmarkBold} & \textcolor{green}{\CheckmarkBold} &\textcolor{red}{\XSolidBrush} & \textcolor{green}{\CheckmarkBold}& \textcolor{green}{\CheckmarkBold}&\textcolor{red}{\XSolidBrush} & \textcolor{green}{\CheckmarkBold}\\
AgentTuning~\citep{zeng2023agenttuning} & AgentInstruct & IFT & - & 35.0K & - &\textcolor{red}{\XSolidBrush} & \textcolor{green}{\CheckmarkBold} & \textcolor{green}{\CheckmarkBold} &\textcolor{red}{\XSolidBrush} & \textcolor{green}{\CheckmarkBold} &\textcolor{red}{\XSolidBrush} &\textcolor{red}{\XSolidBrush} & \textcolor{green}{\CheckmarkBold}\\\midrule
\multicolumn{13}{l}{\emph{Instruction Finetuning-based LLM Agents for Function Calling}} \\\midrule
NexusRaven~\citep{srinivasan2023nexusraven} & NexusRaven & IFT & - & - & 116 & \textcolor{green}{\CheckmarkBold} & \textcolor{green}{\CheckmarkBold}  & \textcolor{green}{\CheckmarkBold} &\textcolor{red}{\XSolidBrush} & \textcolor{green}{\CheckmarkBold} &\textcolor{red}{\XSolidBrush} &\textcolor{red}{\XSolidBrush}&\textcolor{red}{\XSolidBrush}\\
Gorilla~\citep{patil2023gorilla} & Gorilla & IFT & - & 16.0K & 1,645 & \textcolor{green}{\CheckmarkBold} &\textcolor{red}{\XSolidBrush} &\textcolor{red}{\XSolidBrush}&\textcolor{green}{\CheckmarkBold} &\textcolor{green}{\CheckmarkBold} &\textcolor{red}{\XSolidBrush} &\textcolor{red}{\XSolidBrush} &\textcolor{red}{\XSolidBrush}\\
OpenFunctions-v2~\citep{patil2023gorilla} & OpenFunctions-v2 & IFT & - & 65.0K & - & \textcolor{green}{\CheckmarkBold} & \textcolor{green}{\CheckmarkBold} &\textcolor{red}{\XSolidBrush} &\textcolor{green}{\CheckmarkBold} &\textcolor{green}{\CheckmarkBold} &\textcolor{red}{\XSolidBrush} &\textcolor{red}{\XSolidBrush} &\textcolor{red}{\XSolidBrush}\\
API Pack~\cite{guo2024api} & API Pack & IFT & - & 1.1M & 11,213 &\textcolor{green}{\CheckmarkBold} &\textcolor{red}{\XSolidBrush} &\textcolor{green}{\CheckmarkBold} &\textcolor{red}{\XSolidBrush} &\textcolor{green}{\CheckmarkBold} &\textcolor{red}{\XSolidBrush}&\textcolor{red}{\XSolidBrush}&\textcolor{red}{\XSolidBrush}\\ 
LAM~\citep{zhang2024agentohana} & AgentOhana & IFT & - & 42.6K & - & \textcolor{green}{\CheckmarkBold} & \textcolor{green}{\CheckmarkBold} &\textcolor{green}{\CheckmarkBold}&\textcolor{red}{\XSolidBrush} &\textcolor{green}{\CheckmarkBold}&\textcolor{red}{\XSolidBrush}&\textcolor{green}{\CheckmarkBold}&\textcolor{green}{\CheckmarkBold}\\
xLAM~\citep{liu2024apigen} & APIGen & IFT & - & 60.0K & 3,673 & \textcolor{green}{\CheckmarkBold} & \textcolor{green}{\CheckmarkBold} &\textcolor{green}{\CheckmarkBold}&\textcolor{red}{\XSolidBrush} &\textcolor{green}{\CheckmarkBold}&\textcolor{red}{\XSolidBrush}&\textcolor{green}{\CheckmarkBold}&\textcolor{green}{\CheckmarkBold}\\\midrule
\multicolumn{13}{l}{\emph{Pretraining-based LLM Agents}}  \\\midrule
% LEMUR~\citep{xu2024lemur} & PT & 90B & 300.0K & - & \textcolor{green}{\CheckmarkBold} & \textcolor{green}{\CheckmarkBold} &\textcolor{green}{\CheckmarkBold}&\textcolor{red}{\XSolidBrush} & \textcolor{red}{\XSolidBrush} &\textcolor{green}{\CheckmarkBold} &\textcolor{red}{\XSolidBrush}&\textcolor{red}{\XSolidBrush}\\
\rowcolor{teal!12} \method & \dataset & PT & 103B & 95.0K  & 76,537  & \textcolor{green}{\CheckmarkBold} & \textcolor{green}{\CheckmarkBold} & \textcolor{green}{\CheckmarkBold} & \textcolor{green}{\CheckmarkBold} & \textcolor{green}{\CheckmarkBold} & \textcolor{green}{\CheckmarkBold} & \textcolor{green}{\CheckmarkBold} & \textcolor{green}{\CheckmarkBold}\\
\bottomrule
\end{tabular}
% \begin{tablenotes}
%     \item $^*$ In addition, the StarCoder-API can offer 4.77M more APIs.
% \end{tablenotes}
\caption{Summary of existing instruction finetuning-based LLM agents for intrinsic reasoning and function calling, along with their training resources and sample sizes. "PT" and "IFT" denote "Pre-Training" and "Instruction Fine-Tuning", respectively.}
\vspace{-2ex}
\label{tab:related}
\end{threeparttable}
\end{table*}

\noindent \textbf{Prompting-based LLM Agents.} Due to the lack of agent-specific pre-training corpus, existing LLM agents rely on either prompt engineering~\cite{hsieh2023tool,lu2024chameleon,yao2022react,wang2023voyager} or instruction fine-tuning~\cite{chen2023fireact,zeng2023agenttuning} to understand human instructions, decompose high-level tasks, generate grounded plans, and execute multi-step actions. 
However, prompting-based methods mainly depend on the capabilities of backbone LLMs (usually commercial LLMs), failing to introduce new knowledge and struggling to generalize to unseen tasks~\cite{sun2024adaplanner,zhuang2023toolchain}. 

\noindent \textbf{Instruction Finetuning-based LLM Agents.} Considering the extensive diversity of APIs and the complexity of multi-tool instructions, tool learning inherently presents greater challenges than natural language tasks, such as text generation~\cite{qin2023toolllm}.
Post-training techniques focus more on instruction following and aligning output with specific formats~\cite{patil2023gorilla,hao2024toolkengpt,qin2023toolllm,schick2024toolformer}, rather than fundamentally improving model knowledge or capabilities. 
Moreover, heavy fine-tuning can hinder generalization or even degrade performance in non-agent use cases, potentially suppressing the original base model capabilities~\cite{ghosh2024a}.

\noindent \textbf{Pretraining-based LLM Agents.} While pre-training serves as an essential alternative, prior works~\cite{nijkamp2023codegen,roziere2023code,xu2024lemur,patil2023gorilla} have primarily focused on improving task-specific capabilities (\eg, code generation) instead of general-domain LLM agents, due to single-source, uni-type, small-scale, and poor-quality pre-training data. 
Existing tool documentation data for agent training either lacks diverse real-world APIs~\cite{patil2023gorilla, tang2023toolalpaca} or is constrained to single-tool or single-round tool execution. 
Furthermore, trajectory data mostly imitate expert behavior or follow function-calling rules with inferior planning and reasoning, failing to fully elicit LLMs' capabilities and handle complex instructions~\cite{qin2023toolllm}. 
Given a wide range of candidate API functions, each comprising various function names and parameters available at every planning step, identifying globally optimal solutions and generalizing across tasks remains highly challenging.



\section{Preliminaries}
\label{Preliminaries}
\begin{figure*}[t]
    \centering
    \includegraphics[width=0.95\linewidth]{fig/HealthGPT_Framework.png}
    \caption{The \ourmethod{} architecture integrates hierarchical visual perception and H-LoRA, employing a task-specific hard router to select visual features and H-LoRA plugins, ultimately generating outputs with an autoregressive manner.}
    \label{fig:architecture}
\end{figure*}
\noindent\textbf{Large Vision-Language Models.} 
The input to a LVLM typically consists of an image $x^{\text{img}}$ and a discrete text sequence $x^{\text{txt}}$. The visual encoder $\mathcal{E}^{\text{img}}$ converts the input image $x^{\text{img}}$ into a sequence of visual tokens $\mathcal{V} = [v_i]_{i=1}^{N_v}$, while the text sequence $x^{\text{txt}}$ is mapped into a sequence of text tokens $\mathcal{T} = [t_i]_{i=1}^{N_t}$ using an embedding function $\mathcal{E}^{\text{txt}}$. The LLM $\mathcal{M_\text{LLM}}(\cdot|\theta)$ models the joint probability of the token sequence $\mathcal{U} = \{\mathcal{V},\mathcal{T}\}$, which is expressed as:
\begin{equation}
    P_\theta(R | \mathcal{U}) = \prod_{i=1}^{N_r} P_\theta(r_i | \{\mathcal{U}, r_{<i}\}),
\end{equation}
where $R = [r_i]_{i=1}^{N_r}$ is the text response sequence. The LVLM iteratively generates the next token $r_i$ based on $r_{<i}$. The optimization objective is to minimize the cross-entropy loss of the response $\mathcal{R}$.
% \begin{equation}
%     \mathcal{L}_{\text{VLM}} = \mathbb{E}_{R|\mathcal{U}}\left[-\log P_\theta(R | \mathcal{U})\right]
% \end{equation}
It is worth noting that most LVLMs adopt a design paradigm based on ViT, alignment adapters, and pre-trained LLMs\cite{liu2023llava,liu2024improved}, enabling quick adaptation to downstream tasks.


\noindent\textbf{VQGAN.}
VQGAN~\cite{esser2021taming} employs latent space compression and indexing mechanisms to effectively learn a complete discrete representation of images. VQGAN first maps the input image $x^{\text{img}}$ to a latent representation $z = \mathcal{E}(x)$ through a encoder $\mathcal{E}$. Then, the latent representation is quantized using a codebook $\mathcal{Z} = \{z_k\}_{k=1}^K$, generating a discrete index sequence $\mathcal{I} = [i_m]_{m=1}^N$, where $i_m \in \mathcal{Z}$ represents the quantized code index:
\begin{equation}
    \mathcal{I} = \text{Quantize}(z|\mathcal{Z}) = \arg\min_{z_k \in \mathcal{Z}} \| z - z_k \|_2.
\end{equation}
In our approach, the discrete index sequence $\mathcal{I}$ serves as a supervisory signal for the generation task, enabling the model to predict the index sequence $\hat{\mathcal{I}}$ from input conditions such as text or other modality signals.  
Finally, the predicted index sequence $\hat{\mathcal{I}}$ is upsampled by the VQGAN decoder $G$, generating the high-quality image $\hat{x}^\text{img} = G(\hat{\mathcal{I}})$.



\noindent\textbf{Low Rank Adaptation.} 
LoRA\cite{hu2021lora} effectively captures the characteristics of downstream tasks by introducing low-rank adapters. The core idea is to decompose the bypass weight matrix $\Delta W\in\mathbb{R}^{d^{\text{in}} \times d^{\text{out}}}$ into two low-rank matrices $ \{A \in \mathbb{R}^{d^{\text{in}} \times r}, B \in \mathbb{R}^{r \times d^{\text{out}}} \}$, where $ r \ll \min\{d^{\text{in}}, d^{\text{out}}\} $, significantly reducing learnable parameters. The output with the LoRA adapter for the input $x$ is then given by:
\begin{equation}
    h = x W_0 + \alpha x \Delta W/r = x W_0 + \alpha xAB/r,
\end{equation}
where matrix $ A $ is initialized with a Gaussian distribution, while the matrix $ B $ is initialized as a zero matrix. The scaling factor $ \alpha/r $ controls the impact of $ \Delta W $ on the model.

\section{HealthGPT}
\label{Method}


\subsection{Unified Autoregressive Generation.}  
% As shown in Figure~\ref{fig:architecture}, 
\ourmethod{} (Figure~\ref{fig:architecture}) utilizes a discrete token representation that covers both text and visual outputs, unifying visual comprehension and generation as an autoregressive task. 
For comprehension, $\mathcal{M}_\text{llm}$ receives the input joint sequence $\mathcal{U}$ and outputs a series of text token $\mathcal{R} = [r_1, r_2, \dots, r_{N_r}]$, where $r_i \in \mathcal{V}_{\text{txt}}$, and $\mathcal{V}_{\text{txt}}$ represents the LLM's vocabulary:
\begin{equation}
    P_\theta(\mathcal{R} \mid \mathcal{U}) = \prod_{i=1}^{N_r} P_\theta(r_i \mid \mathcal{U}, r_{<i}).
\end{equation}
For generation, $\mathcal{M}_\text{llm}$ first receives a special start token $\langle \text{START\_IMG} \rangle$, then generates a series of tokens corresponding to the VQGAN indices $\mathcal{I} = [i_1, i_2, \dots, i_{N_i}]$, where $i_j \in \mathcal{V}_{\text{vq}}$, and $\mathcal{V}_{\text{vq}}$ represents the index range of VQGAN. Upon completion of generation, the LLM outputs an end token $\langle \text{END\_IMG} \rangle$:
\begin{equation}
    P_\theta(\mathcal{I} \mid \mathcal{U}) = \prod_{j=1}^{N_i} P_\theta(i_j \mid \mathcal{U}, i_{<j}).
\end{equation}
Finally, the generated index sequence $\mathcal{I}$ is fed into the decoder $G$, which reconstructs the target image $\hat{x}^{\text{img}} = G(\mathcal{I})$.

\subsection{Hierarchical Visual Perception}  
Given the differences in visual perception between comprehension and generation tasks—where the former focuses on abstract semantics and the latter emphasizes complete semantics—we employ ViT to compress the image into discrete visual tokens at multiple hierarchical levels.
Specifically, the image is converted into a series of features $\{f_1, f_2, \dots, f_L\}$ as it passes through $L$ ViT blocks.

To address the needs of various tasks, the hidden states are divided into two types: (i) \textit{Concrete-grained features} $\mathcal{F}^{\text{Con}} = \{f_1, f_2, \dots, f_k\}, k < L$, derived from the shallower layers of ViT, containing sufficient global features, suitable for generation tasks; 
(ii) \textit{Abstract-grained features} $\mathcal{F}^{\text{Abs}} = \{f_{k+1}, f_{k+2}, \dots, f_L\}$, derived from the deeper layers of ViT, which contain abstract semantic information closer to the text space, suitable for comprehension tasks.

The task type $T$ (comprehension or generation) determines which set of features is selected as the input for the downstream large language model:
\begin{equation}
    \mathcal{F}^{\text{img}}_T =
    \begin{cases}
        \mathcal{F}^{\text{Con}}, & \text{if } T = \text{generation task} \\
        \mathcal{F}^{\text{Abs}}, & \text{if } T = \text{comprehension task}
    \end{cases}
\end{equation}
We integrate the image features $\mathcal{F}^{\text{img}}_T$ and text features $\mathcal{T}$ into a joint sequence through simple concatenation, which is then fed into the LLM $\mathcal{M}_{\text{llm}}$ for autoregressive generation.
% :
% \begin{equation}
%     \mathcal{R} = \mathcal{M}_{\text{llm}}(\mathcal{U}|\theta), \quad \mathcal{U} = [\mathcal{F}^{\text{img}}_T; \mathcal{T}]
% \end{equation}
\subsection{Heterogeneous Knowledge Adaptation}
We devise H-LoRA, which stores heterogeneous knowledge from comprehension and generation tasks in separate modules and dynamically routes to extract task-relevant knowledge from these modules. 
At the task level, for each task type $ T $, we dynamically assign a dedicated H-LoRA submodule $ \theta^T $, which is expressed as:
\begin{equation}
    \mathcal{R} = \mathcal{M}_\text{LLM}(\mathcal{U}|\theta, \theta^T), \quad \theta^T = \{A^T, B^T, \mathcal{R}^T_\text{outer}\}.
\end{equation}
At the feature level for a single task, H-LoRA integrates the idea of Mixture of Experts (MoE)~\cite{masoudnia2014mixture} and designs an efficient matrix merging and routing weight allocation mechanism, thus avoiding the significant computational delay introduced by matrix splitting in existing MoELoRA~\cite{luo2024moelora}. Specifically, we first merge the low-rank matrices (rank = r) of $ k $ LoRA experts into a unified matrix:
\begin{equation}
    \mathbf{A}^{\text{merged}}, \mathbf{B}^{\text{merged}} = \text{Concat}(\{A_i\}_1^k), \text{Concat}(\{B_i\}_1^k),
\end{equation}
where $ \mathbf{A}^{\text{merged}} \in \mathbb{R}^{d^\text{in} \times rk} $ and $ \mathbf{B}^{\text{merged}} \in \mathbb{R}^{rk \times d^\text{out}} $. The $k$-dimension routing layer generates expert weights $ \mathcal{W} \in \mathbb{R}^{\text{token\_num} \times k} $ based on the input hidden state $ x $, and these are expanded to $ \mathbb{R}^{\text{token\_num} \times rk} $ as follows:
\begin{equation}
    \mathcal{W}^\text{expanded} = \alpha k \mathcal{W} / r \otimes \mathbf{1}_r,
\end{equation}
where $ \otimes $ denotes the replication operation.
The overall output of H-LoRA is computed as:
\begin{equation}
    \mathcal{O}^\text{H-LoRA} = (x \mathbf{A}^{\text{merged}} \odot \mathcal{W}^\text{expanded}) \mathbf{B}^{\text{merged}},
\end{equation}
where $ \odot $ represents element-wise multiplication. Finally, the output of H-LoRA is added to the frozen pre-trained weights to produce the final output:
\begin{equation}
    \mathcal{O} = x W_0 + \mathcal{O}^\text{H-LoRA}.
\end{equation}
% In summary, H-LoRA is a task-based dynamic PEFT method that achieves high efficiency in single-task fine-tuning.

\subsection{Training Pipeline}

\begin{figure}[t]
    \centering
    \hspace{-4mm}
    \includegraphics[width=0.94\linewidth]{fig/data.pdf}
    \caption{Data statistics of \texttt{VL-Health}. }
    \label{fig:data}
\end{figure}
\noindent \textbf{1st Stage: Multi-modal Alignment.} 
In the first stage, we design separate visual adapters and H-LoRA submodules for medical unified tasks. For the medical comprehension task, we train abstract-grained visual adapters using high-quality image-text pairs to align visual embeddings with textual embeddings, thereby enabling the model to accurately describe medical visual content. During this process, the pre-trained LLM and its corresponding H-LoRA submodules remain frozen. In contrast, the medical generation task requires training concrete-grained adapters and H-LoRA submodules while keeping the LLM frozen. Meanwhile, we extend the textual vocabulary to include multimodal tokens, enabling the support of additional VQGAN vector quantization indices. The model trains on image-VQ pairs, endowing the pre-trained LLM with the capability for image reconstruction. This design ensures pixel-level consistency of pre- and post-LVLM. The processes establish the initial alignment between the LLM’s outputs and the visual inputs.

\noindent \textbf{2nd Stage: Heterogeneous H-LoRA Plugin Adaptation.}  
The submodules of H-LoRA share the word embedding layer and output head but may encounter issues such as bias and scale inconsistencies during training across different tasks. To ensure that the multiple H-LoRA plugins seamlessly interface with the LLMs and form a unified base, we fine-tune the word embedding layer and output head using a small amount of mixed data to maintain consistency in the model weights. Specifically, during this stage, all H-LoRA submodules for different tasks are kept frozen, with only the word embedding layer and output head being optimized. Through this stage, the model accumulates foundational knowledge for unified tasks by adapting H-LoRA plugins.

\begin{table*}[!t]
\centering
\caption{Comparison of \ourmethod{} with other LVLMs and unified multi-modal models on medical visual comprehension tasks. \textbf{Bold} and \underline{underlined} text indicates the best performance and second-best performance, respectively.}
\resizebox{\textwidth}{!}{
\begin{tabular}{c|lcc|cccccccc|c}
\toprule
\rowcolor[HTML]{E9F3FE} &  &  &  & \multicolumn{2}{c}{\textbf{VQA-RAD \textuparrow}} & \multicolumn{2}{c}{\textbf{SLAKE \textuparrow}} & \multicolumn{2}{c}{\textbf{PathVQA \textuparrow}} &  &  &  \\ 
\cline{5-10}
\rowcolor[HTML]{E9F3FE}\multirow{-2}{*}{\textbf{Type}} & \multirow{-2}{*}{\textbf{Model}} & \multirow{-2}{*}{\textbf{\# Params}} & \multirow{-2}{*}{\makecell{\textbf{Medical} \\ \textbf{LVLM}}} & \textbf{close} & \textbf{all} & \textbf{close} & \textbf{all} & \textbf{close} & \textbf{all} & \multirow{-2}{*}{\makecell{\textbf{MMMU} \\ \textbf{-Med}}\textuparrow} & \multirow{-2}{*}{\textbf{OMVQA}\textuparrow} & \multirow{-2}{*}{\textbf{Avg. \textuparrow}} \\ 
\midrule \midrule
\multirow{9}{*}{\textbf{Comp. Only}} 
& Med-Flamingo & 8.3B & \Large \ding{51} & 58.6 & 43.0 & 47.0 & 25.5 & 61.9 & 31.3 & 28.7 & 34.9 & 41.4 \\
& LLaVA-Med & 7B & \Large \ding{51} & 60.2 & 48.1 & 58.4 & 44.8 & 62.3 & 35.7 & 30.0 & 41.3 & 47.6 \\
& HuatuoGPT-Vision & 7B & \Large \ding{51} & 66.9 & 53.0 & 59.8 & 49.1 & 52.9 & 32.0 & 42.0 & 50.0 & 50.7 \\
& BLIP-2 & 6.7B & \Large \ding{55} & 43.4 & 36.8 & 41.6 & 35.3 & 48.5 & 28.8 & 27.3 & 26.9 & 36.1 \\
& LLaVA-v1.5 & 7B & \Large \ding{55} & 51.8 & 42.8 & 37.1 & 37.7 & 53.5 & 31.4 & 32.7 & 44.7 & 41.5 \\
& InstructBLIP & 7B & \Large \ding{55} & 61.0 & 44.8 & 66.8 & 43.3 & 56.0 & 32.3 & 25.3 & 29.0 & 44.8 \\
& Yi-VL & 6B & \Large \ding{55} & 52.6 & 42.1 & 52.4 & 38.4 & 54.9 & 30.9 & 38.0 & 50.2 & 44.9 \\
& InternVL2 & 8B & \Large \ding{55} & 64.9 & 49.0 & 66.6 & 50.1 & 60.0 & 31.9 & \underline{43.3} & 54.5 & 52.5\\
& Llama-3.2 & 11B & \Large \ding{55} & 68.9 & 45.5 & 72.4 & 52.1 & 62.8 & 33.6 & 39.3 & 63.2 & 54.7 \\
\midrule
\multirow{5}{*}{\textbf{Comp. \& Gen.}} 
& Show-o & 1.3B & \Large \ding{55} & 50.6 & 33.9 & 31.5 & 17.9 & 52.9 & 28.2 & 22.7 & 45.7 & 42.6 \\
& Unified-IO 2 & 7B & \Large \ding{55} & 46.2 & 32.6 & 35.9 & 21.9 & 52.5 & 27.0 & 25.3 & 33.0 & 33.8 \\
& Janus & 1.3B & \Large \ding{55} & 70.9 & 52.8 & 34.7 & 26.9 & 51.9 & 27.9 & 30.0 & 26.8 & 33.5 \\
& \cellcolor[HTML]{DAE0FB}HealthGPT-M3 & \cellcolor[HTML]{DAE0FB}3.8B & \cellcolor[HTML]{DAE0FB}\Large \ding{51} & \cellcolor[HTML]{DAE0FB}\underline{73.7} & \cellcolor[HTML]{DAE0FB}\underline{55.9} & \cellcolor[HTML]{DAE0FB}\underline{74.6} & \cellcolor[HTML]{DAE0FB}\underline{56.4} & \cellcolor[HTML]{DAE0FB}\underline{78.7} & \cellcolor[HTML]{DAE0FB}\underline{39.7} & \cellcolor[HTML]{DAE0FB}\underline{43.3} & \cellcolor[HTML]{DAE0FB}\underline{68.5} & \cellcolor[HTML]{DAE0FB}\underline{61.3} \\
& \cellcolor[HTML]{DAE0FB}HealthGPT-L14 & \cellcolor[HTML]{DAE0FB}14B & \cellcolor[HTML]{DAE0FB}\Large \ding{51} & \cellcolor[HTML]{DAE0FB}\textbf{77.7} & \cellcolor[HTML]{DAE0FB}\textbf{58.3} & \cellcolor[HTML]{DAE0FB}\textbf{76.4} & \cellcolor[HTML]{DAE0FB}\textbf{64.5} & \cellcolor[HTML]{DAE0FB}\textbf{85.9} & \cellcolor[HTML]{DAE0FB}\textbf{44.4} & \cellcolor[HTML]{DAE0FB}\textbf{49.2} & \cellcolor[HTML]{DAE0FB}\textbf{74.4} & \cellcolor[HTML]{DAE0FB}\textbf{66.4} \\
\bottomrule
\end{tabular}
}
\label{tab:results}
\end{table*}
\begin{table*}[ht]
    \centering
    \caption{The experimental results for the four modality conversion tasks.}
    \resizebox{\textwidth}{!}{
    \begin{tabular}{l|ccc|ccc|ccc|ccc}
        \toprule
        \rowcolor[HTML]{E9F3FE} & \multicolumn{3}{c}{\textbf{CT to MRI (Brain)}} & \multicolumn{3}{c}{\textbf{CT to MRI (Pelvis)}} & \multicolumn{3}{c}{\textbf{MRI to CT (Brain)}} & \multicolumn{3}{c}{\textbf{MRI to CT (Pelvis)}} \\
        \cline{2-13}
        \rowcolor[HTML]{E9F3FE}\multirow{-2}{*}{\textbf{Model}}& \textbf{SSIM $\uparrow$} & \textbf{PSNR $\uparrow$} & \textbf{MSE $\downarrow$} & \textbf{SSIM $\uparrow$} & \textbf{PSNR $\uparrow$} & \textbf{MSE $\downarrow$} & \textbf{SSIM $\uparrow$} & \textbf{PSNR $\uparrow$} & \textbf{MSE $\downarrow$} & \textbf{SSIM $\uparrow$} & \textbf{PSNR $\uparrow$} & \textbf{MSE $\downarrow$} \\
        \midrule \midrule
        pix2pix & 71.09 & 32.65 & 36.85 & 59.17 & 31.02 & 51.91 & 78.79 & 33.85 & 28.33 & 72.31 & 32.98 & 36.19 \\
        CycleGAN & 54.76 & 32.23 & 40.56 & 54.54 & 30.77 & 55.00 & 63.75 & 31.02 & 52.78 & 50.54 & 29.89 & 67.78 \\
        BBDM & {71.69} & {32.91} & {34.44} & 57.37 & 31.37 & 48.06 & \textbf{86.40} & 34.12 & 26.61 & {79.26} & 33.15 & 33.60 \\
        Vmanba & 69.54 & 32.67 & 36.42 & {63.01} & {31.47} & {46.99} & 79.63 & 34.12 & 26.49 & 77.45 & 33.53 & 31.85 \\
        DiffMa & 71.47 & 32.74 & 35.77 & 62.56 & 31.43 & 47.38 & 79.00 & {34.13} & {26.45} & 78.53 & {33.68} & {30.51} \\
        \rowcolor[HTML]{DAE0FB}HealthGPT-M3 & \underline{79.38} & \underline{33.03} & \underline{33.48} & \underline{71.81} & \underline{31.83} & \underline{43.45} & {85.06} & \textbf{34.40} & \textbf{25.49} & \underline{84.23} & \textbf{34.29} & \textbf{27.99} \\
        \rowcolor[HTML]{DAE0FB}HealthGPT-L14 & \textbf{79.73} & \textbf{33.10} & \textbf{32.96} & \textbf{71.92} & \textbf{31.87} & \textbf{43.09} & \underline{85.31} & \underline{34.29} & \underline{26.20} & \textbf{84.96} & \underline{34.14} & \underline{28.13} \\
        \bottomrule
    \end{tabular}
    }
    \label{tab:conversion}
\end{table*}

\noindent \textbf{3rd Stage: Visual Instruction Fine-Tuning.}  
In the third stage, we introduce additional task-specific data to further optimize the model and enhance its adaptability to downstream tasks such as medical visual comprehension (e.g., medical QA, medical dialogues, and report generation) or generation tasks (e.g., super-resolution, denoising, and modality conversion). Notably, by this stage, the word embedding layer and output head have been fine-tuned, only the H-LoRA modules and adapter modules need to be trained. This strategy significantly improves the model's adaptability and flexibility across different tasks.


\section{Experiment}
\label{s:experiment}

\subsection{Data Description}
We evaluate our method on FI~\cite{you2016building}, Twitter\_LDL~\cite{yang2017learning} and Artphoto~\cite{machajdik2010affective}.
FI is a public dataset built from Flickr and Instagram, with 23,308 images and eight emotion categories, namely \textit{amusement}, \textit{anger}, \textit{awe},  \textit{contentment}, \textit{disgust}, \textit{excitement},  \textit{fear}, and \textit{sadness}. 
% Since images in FI are all copyrighted by law, some images are corrupted now, so we remove these samples and retain 21,828 images.
% T4SA contains images from Twitter, which are classified into three categories: \textit{positive}, \textit{neutral}, and \textit{negative}. In this paper, we adopt the base version of B-T4SA, which contains 470,586 images and provides text descriptions of the corresponding tweets.
Twitter\_LDL contains 10,045 images from Twitter, with the same eight categories as the FI dataset.
% 。
For these two datasets, they are randomly split into 80\%
training and 20\% testing set.
Artphoto contains 806 artistic photos from the DeviantArt website, which we use to further evaluate the zero-shot capability of our model.
% on the small-scale dataset.
% We construct and publicly release the first image sentiment analysis dataset containing metadata.
% 。

% Based on these datasets, we are the first to construct and publicly release metadata-enhanced image sentiment analysis datasets. These datasets include scenes, tags, descriptions, and corresponding confidence scores, and are available at this link for future research purposes.


% 
\begin{table}[t]
\centering
% \begin{center}
\caption{Overall performance of different models on FI and Twitter\_LDL datasets.}
\label{tab:cap1}
% \resizebox{\linewidth}{!}
{
\begin{tabular}{l|c|c|c|c}
\hline
\multirow{2}{*}{\textbf{Model}} & \multicolumn{2}{c|}{\textbf{FI}}  & \multicolumn{2}{c}{\textbf{Twitter\_LDL}} \\ \cline{2-5} 
  & \textbf{Accuracy} & \textbf{F1} & \textbf{Accuracy} & \textbf{F1}  \\ \hline
% (\rownumber)~AlexNet~\cite{krizhevsky2017imagenet}  & 58.13\% & 56.35\%  & 56.24\%& 55.02\%  \\ 
% (\rownumber)~VGG16~\cite{simonyan2014very}  & 63.75\%& 63.08\%  & 59.34\%& 59.02\%  \\ 
(\rownumber)~ResNet101~\cite{he2016deep} & 66.16\%& 65.56\%  & 62.02\% & 61.34\%  \\ 
(\rownumber)~CDA~\cite{han2023boosting} & 66.71\%& 65.37\%  & 64.14\% & 62.85\%  \\ 
(\rownumber)~CECCN~\cite{ruan2024color} & 67.96\%& 66.74\%  & 64.59\%& 64.72\% \\ 
(\rownumber)~EmoVIT~\cite{xie2024emovit} & 68.09\%& 67.45\%  & 63.12\% & 61.97\%  \\ 
(\rownumber)~ComLDL~\cite{zhang2022compound} & 68.83\%& 67.28\%  & 65.29\% & 63.12\%  \\ 
(\rownumber)~WSDEN~\cite{li2023weakly} & 69.78\%& 69.61\%  & 67.04\% & 65.49\% \\ 
(\rownumber)~ECWA~\cite{deng2021emotion} & 70.87\%& 69.08\%  & 67.81\% & 66.87\%  \\ 
(\rownumber)~EECon~\cite{yang2023exploiting} & 71.13\%& 68.34\%  & 64.27\%& 63.16\%  \\ 
(\rownumber)~MAM~\cite{zhang2024affective} & 71.44\%  & 70.83\% & 67.18\%  & 65.01\%\\ 
(\rownumber)~TGCA-PVT~\cite{chen2024tgca}   & 73.05\%  & 71.46\% & 69.87\%  & 68.32\% \\ 
(\rownumber)~OEAN~\cite{zhang2024object}   & 73.40\%  & 72.63\% & 70.52\%  & 69.47\% \\ \hline
(\rownumber)~\shortname  & \textbf{79.48\%} & \textbf{79.22\%} & \textbf{74.12\%} & \textbf{73.09\%} \\ \hline
\end{tabular}
}
\vspace{-6mm}
% \end{center}
\end{table}
% 

\subsection{Experiment Setting}
% \subsubsection{Model Setting.}
% 
\textbf{Model Setting:}
For feature representation, we set $k=10$ to select object tags, and adopt clip-vit-base-patch32 as the pre-trained model for unified feature representation.
Moreover, we empirically set $(d_e, d_h, d_k, d_s) = (512, 128, 16, 64)$, and set the classification class $L$ to 8.

% 

\textbf{Training Setting:}
To initialize the model, we set all weights such as $\boldsymbol{W}$ following the truncated normal distribution, and use AdamW optimizer with the learning rate of $1 \times 10^{-4}$.
% warmup scheduler of cosine, warmup steps of 2000.
Furthermore, we set the batch size to 32 and the epoch of the training process to 200.
During the implementation, we utilize \textit{PyTorch} to build our entire model.
% , and our project codes are publicly available at https://github.com/zzmyrep/MESN.
% Our project codes as well as data are all publicly available on GitHub\footnote{https://github.com/zzmyrep/KBCEN}.
% Code is available at \href{https://github.com/zzmyrep/KBCEN}{https://github.com/zzmyrep/KBCEN}.

\textbf{Evaluation Metrics:}
Following~\cite{zhang2024affective, chen2024tgca, zhang2024object}, we adopt \textit{accuracy} and \textit{F1} as our evaluation metrics to measure the performance of different methods for image sentiment analysis. 



\subsection{Experiment Result}
% We compare our model against the following baselines: AlexNet~\cite{krizhevsky2017imagenet}, VGG16~\cite{simonyan2014very}, ResNet101~\cite{he2016deep}, CECCN~\cite{ruan2024color}, EmoVIT~\cite{xie2024emovit}, WSCNet~\cite{yang2018weakly}, ECWA~\cite{deng2021emotion}, EECon~\cite{yang2023exploiting}, MAM~\cite{zhang2024affective} and TGCA-PVT~\cite{chen2024tgca}, and the overall results are summarized in Table~\ref{tab:cap1}.
We compare our model against several baselines, and the overall results are summarized in Table~\ref{tab:cap1}.
We observe that our model achieves the best performance in both accuracy and F1 metrics, significantly outperforming the previous models. 
This superior performance is mainly attributed to our effective utilization of metadata to enhance image sentiment analysis, as well as the exceptional capability of the unified sentiment transformer framework we developed. These results strongly demonstrate that our proposed method can bring encouraging performance for image sentiment analysis.

\setcounter{magicrownumbers}{0} 
\begin{table}[t]
\begin{center}
\caption{Ablation study of~\shortname~on FI dataset.} 
% \vspace{1mm}
\label{tab:cap2}
\resizebox{.9\linewidth}{!}
{
\begin{tabular}{lcc}
  \hline
  \textbf{Model} & \textbf{Accuracy} & \textbf{F1} \\
  \hline
  (\rownumber)~Ours (w/o vision) & 65.72\% & 64.54\% \\
  (\rownumber)~Ours (w/o text description) & 74.05\% & 72.58\% \\
  (\rownumber)~Ours (w/o object tag) & 77.45\% & 76.84\% \\
  (\rownumber)~Ours (w/o scene tag) & 78.47\% & 78.21\% \\
  \hline
  (\rownumber)~Ours (w/o unified embedding) & 76.41\% & 76.23\% \\
  (\rownumber)~Ours (w/o adaptive learning) & 76.83\% & 76.56\% \\
  (\rownumber)~Ours (w/o cross-modal fusion) & 76.85\% & 76.49\% \\
  \hline
  (\rownumber)~Ours  & \textbf{79.48\%} & \textbf{79.22\%} \\
  \hline
\end{tabular}
}
\end{center}
\vspace{-5mm}
\end{table}


\begin{figure}[t]
\centering
% \vspace{-2mm}
\includegraphics[width=0.42\textwidth]{fig/2dvisual-linux4-paper2.pdf}
\caption{Visualization of feature distribution on eight categories before (left) and after (right) model processing.}
% 
\label{fig:visualization}
\vspace{-5mm}
\end{figure}

\subsection{Ablation Performance}
In this subsection, we conduct an ablation study to examine which component is really important for performance improvement. The results are reported in Table~\ref{tab:cap2}.

For information utilization, we observe a significant decline in model performance when visual features are removed. Additionally, the performance of \shortname~decreases when different metadata are removed separately, which means that text description, object tag, and scene tag are all critical for image sentiment analysis.
Recalling the model architecture, we separately remove transformer layers of the unified representation module, the adaptive learning module, and the cross-modal fusion module, replacing them with MLPs of the same parameter scale.
In this way, we can observe varying degrees of decline in model performance, indicating that these modules are indispensable for our model to achieve better performance.

\subsection{Visualization}
% 


% % 开始使用minipage进行左右排列
% \begin{minipage}[t]{0.45\textwidth}  % 子图1宽度为45%
%     \centering
%     \includegraphics[width=\textwidth]{2dvisual.pdf}  % 插入图片
%     \captionof{figure}{Visualization of feature distribution.}  % 使用captionof添加图片标题
%     \label{fig:visualization}
% \end{minipage}


% \begin{figure}[t]
% \centering
% \vspace{-2mm}
% \includegraphics[width=0.45\textwidth]{fig/2dvisual.pdf}
% \caption{Visualization of feature distribution.}
% \label{fig:visualization}
% % \vspace{-4mm}
% \end{figure}

% \begin{figure}[t]
% \centering
% \vspace{-2mm}
% \includegraphics[width=0.45\textwidth]{fig/2dvisual-linux3-paper.pdf}
% \caption{Visualization of feature distribution.}
% \label{fig:visualization}
% % \vspace{-4mm}
% \end{figure}



\begin{figure}[tbp]   
\vspace{-4mm}
  \centering            
  \subfloat[Depth of adaptive learning layers]   
  {
    \label{fig:subfig1}\includegraphics[width=0.22\textwidth]{fig/fig_sensitivity-a5}
  }
  \subfloat[Depth of fusion layers]
  {
    % \label{fig:subfig2}\includegraphics[width=0.22\textwidth]{fig/fig_sensitivity-b2}
    \label{fig:subfig2}\includegraphics[width=0.22\textwidth]{fig/fig_sensitivity-b2-num.pdf}
  }
  \caption{Sensitivity study of \shortname~on different depth. }   
  \label{fig:fig_sensitivity}  
\vspace{-2mm}
\end{figure}

% \begin{figure}[htbp]
% \centerline{\includegraphics{2dvisual.pdf}}
% \caption{Visualization of feature distribution.}
% \label{fig:visualization}
% \end{figure}

% In Fig.~\ref{fig:visualization}, we use t-SNE~\cite{van2008visualizing} to reduce the dimension of data features for visualization, Figure in left represents the metadata features before model processing, the features are obtained by embedding through the CLIP model, and figure in right shows the features of the data after model processing, it can be observed that after the model processing, the data with different label categories fall in different regions in the space, therefore, we can conclude that the Therefore, we can conclude that the model can effectively utilize the information contained in the metadata and use it to guide the model for classification.

In Fig.~\ref{fig:visualization}, we use t-SNE~\cite{van2008visualizing} to reduce the dimension of data features for visualization.
The left figure shows metadata features before being processed by our model (\textit{i.e.}, embedded by CLIP), while the right shows the distribution of features after being processed by our model.
We can observe that after the model processing, data with the same label are closer to each other, while others are farther away.
Therefore, it shows that the model can effectively utilize the information contained in the metadata and use it to guide the classification process.

\subsection{Sensitivity Analysis}
% 
In this subsection, we conduct a sensitivity analysis to figure out the effect of different depth settings of adaptive learning layers and fusion layers. 
% In this subsection, we conduct a sensitivity analysis to figure out the effect of different depth settings on the model. 
% Fig.~\ref{fig:fig_sensitivity} presents the effect of different depth settings of adaptive learning layers and fusion layers. 
Taking Fig.~\ref{fig:fig_sensitivity} (a) as an example, the model performance improves with increasing depth, reaching the best performance at a depth of 4.
% Taking Fig.~\ref{fig:fig_sensitivity} (a) as an example, the performance of \shortname~improves with the increase of depth at first, reaching the best performance at a depth of 4.
When the depth continues to increase, the accuracy decreases to varying degrees.
Similar results can be observed in Fig.~\ref{fig:fig_sensitivity} (b).
Therefore, we set their depths to 4 and 6 respectively to achieve the best results.

% Through our experiments, we can observe that the effect of modifying these hyperparameters on the results of the experiments is very weak, and the surface model is not sensitive to the hyperparameters.


\subsection{Zero-shot Capability}
% 

% (1)~GCH~\cite{2010Analyzing} & 21.78\% & (5)~RA-DLNet~\cite{2020A} & 34.01\% \\ \hline
% (2)~WSCNet~\cite{2019WSCNet}  & 30.25\% & (6)~CECCN~\cite{ruan2024color} & 43.83\% \\ \hline
% (3)~PCNN~\cite{2015Robust} & 31.68\%  & (7)~EmoVIT~\cite{xie2024emovit} & 44.90\% \\ \hline
% (4)~AR~\cite{2018Visual} & 32.67\% & (8)~Ours (Zero-shot) & 47.83\% \\ \hline


\begin{table}[t]
\centering
\caption{Zero-shot capability of \shortname.}
\label{tab:cap3}
\resizebox{1\linewidth}{!}
{
\begin{tabular}{lc|lc}
\hline
\textbf{Model} & \textbf{Accuracy} & \textbf{Model} & \textbf{Accuracy} \\ \hline
(1)~WSCNet~\cite{2019WSCNet}  & 30.25\% & (5)~MAM~\cite{zhang2024affective} & 39.56\%  \\ \hline
(2)~AR~\cite{2018Visual} & 32.67\% & (6)~CECCN~\cite{ruan2024color} & 43.83\% \\ \hline
(3)~RA-DLNet~\cite{2020A} & 34.01\%  & (7)~EmoVIT~\cite{xie2024emovit} & 44.90\% \\ \hline
(4)~CDA~\cite{han2023boosting} & 38.64\% & (8)~Ours (Zero-shot) & 47.83\% \\ \hline
\end{tabular}
}
\vspace{-5mm}
\end{table}

% We use the model trained on the FI dataset to test on the artphoto dataset to verify the model's generalization ability as well as robustness to other distributed datasets.
% We can observe that the MESN model shows strong competitiveness in terms of accuracy when compared to other trained models, which suggests that the model has a good generalization ability in the OOD task.

To validate the model's generalization ability and robustness to other distributed datasets, we directly test the model trained on the FI dataset, without training on Artphoto. 
% As observed in Table 3, compared to other models trained on Artphoto, we achieve highly competitive zero-shot performance, indicating that the model has good generalization ability in out-of-distribution tasks.
From Table~\ref{tab:cap3}, we can observe that compared with other models trained on Artphoto, we achieve competitive zero-shot performance, which shows that the model has good generalization ability in out-of-distribution tasks.


%%%%%%%%%%%%
%  E2E     %
%%%%%%%%%%%%


\section{Conclusion}
In this paper, we introduced Wi-Chat, the first LLM-powered Wi-Fi-based human activity recognition system that integrates the reasoning capabilities of large language models with the sensing potential of wireless signals. Our experimental results on a self-collected Wi-Fi CSI dataset demonstrate the promising potential of LLMs in enabling zero-shot Wi-Fi sensing. These findings suggest a new paradigm for human activity recognition that does not rely on extensive labeled data. We hope future research will build upon this direction, further exploring the applications of LLMs in signal processing domains such as IoT, mobile sensing, and radar-based systems.

\section*{Limitations}
While our work represents the first attempt to leverage LLMs for processing Wi-Fi signals, it is a preliminary study focused on a relatively simple task: Wi-Fi-based human activity recognition. This choice allows us to explore the feasibility of LLMs in wireless sensing but also comes with certain limitations.

Our approach primarily evaluates zero-shot performance, which, while promising, may still lag behind traditional supervised learning methods in highly complex or fine-grained recognition tasks. Besides, our study is limited to a controlled environment with a self-collected dataset, and the generalizability of LLMs to diverse real-world scenarios with varying Wi-Fi conditions, environmental interference, and device heterogeneity remains an open question.

Additionally, we have yet to explore the full potential of LLMs in more advanced Wi-Fi sensing applications, such as fine-grained gesture recognition, occupancy detection, and passive health monitoring. Future work should investigate the scalability of LLM-based approaches, their robustness to domain shifts, and their integration with multimodal sensing techniques in broader IoT applications.


% Bibliography entries for the entire Anthology, followed by custom entries
%\bibliography{anthology,custom}
% Custom bibliography entries only
\bibliography{main}
\newpage
\appendix

\section{Experiment prompts}
\label{sec:prompt}
The prompts used in the LLM experiments are shown in the following Table~\ref{tab:prompts}.

\definecolor{titlecolor}{rgb}{0.9, 0.5, 0.1}
\definecolor{anscolor}{rgb}{0.2, 0.5, 0.8}
\definecolor{labelcolor}{HTML}{48a07e}
\begin{table*}[h]
	\centering
	
 % \vspace{-0.2cm}
	
	\begin{center}
		\begin{tikzpicture}[
				chatbox_inner/.style={rectangle, rounded corners, opacity=0, text opacity=1, font=\sffamily\scriptsize, text width=5in, text height=9pt, inner xsep=6pt, inner ysep=6pt},
				chatbox_prompt_inner/.style={chatbox_inner, align=flush left, xshift=0pt, text height=11pt},
				chatbox_user_inner/.style={chatbox_inner, align=flush left, xshift=0pt},
				chatbox_gpt_inner/.style={chatbox_inner, align=flush left, xshift=0pt},
				chatbox/.style={chatbox_inner, draw=black!25, fill=gray!7, opacity=1, text opacity=0},
				chatbox_prompt/.style={chatbox, align=flush left, fill=gray!1.5, draw=black!30, text height=10pt},
				chatbox_user/.style={chatbox, align=flush left},
				chatbox_gpt/.style={chatbox, align=flush left},
				chatbox2/.style={chatbox_gpt, fill=green!25},
				chatbox3/.style={chatbox_gpt, fill=red!20, draw=black!20},
				chatbox4/.style={chatbox_gpt, fill=yellow!30},
				labelbox/.style={rectangle, rounded corners, draw=black!50, font=\sffamily\scriptsize\bfseries, fill=gray!5, inner sep=3pt},
			]
											
			\node[chatbox_user] (q1) {
				\textbf{System prompt}
				\newline
				\newline
				You are a helpful and precise assistant for segmenting and labeling sentences. We would like to request your help on curating a dataset for entity-level hallucination detection.
				\newline \newline
                We will give you a machine generated biography and a list of checked facts about the biography. Each fact consists of a sentence and a label (True/False). Please do the following process. First, breaking down the biography into words. Second, by referring to the provided list of facts, merging some broken down words in the previous step to form meaningful entities. For example, ``strategic thinking'' should be one entity instead of two. Third, according to the labels in the list of facts, labeling each entity as True or False. Specifically, for facts that share a similar sentence structure (\eg, \textit{``He was born on Mach 9, 1941.''} (\texttt{True}) and \textit{``He was born in Ramos Mejia.''} (\texttt{False})), please first assign labels to entities that differ across atomic facts. For example, first labeling ``Mach 9, 1941'' (\texttt{True}) and ``Ramos Mejia'' (\texttt{False}) in the above case. For those entities that are the same across atomic facts (\eg, ``was born'') or are neutral (\eg, ``he,'' ``in,'' and ``on''), please label them as \texttt{True}. For the cases that there is no atomic fact that shares the same sentence structure, please identify the most informative entities in the sentence and label them with the same label as the atomic fact while treating the rest of the entities as \texttt{True}. In the end, output the entities and labels in the following format:
                \begin{itemize}[nosep]
                    \item Entity 1 (Label 1)
                    \item Entity 2 (Label 2)
                    \item ...
                    \item Entity N (Label N)
                \end{itemize}
                % \newline \newline
                Here are two examples:
                \newline\newline
                \textbf{[Example 1]}
                \newline
                [The start of the biography]
                \newline
                \textcolor{titlecolor}{Marianne McAndrew is an American actress and singer, born on November 21, 1942, in Cleveland, Ohio. She began her acting career in the late 1960s, appearing in various television shows and films.}
                \newline
                [The end of the biography]
                \newline \newline
                [The start of the list of checked facts]
                \newline
                \textcolor{anscolor}{[Marianne McAndrew is an American. (False); Marianne McAndrew is an actress. (True); Marianne McAndrew is a singer. (False); Marianne McAndrew was born on November 21, 1942. (False); Marianne McAndrew was born in Cleveland, Ohio. (False); She began her acting career in the late 1960s. (True); She has appeared in various television shows. (True); She has appeared in various films. (True)]}
                \newline
                [The end of the list of checked facts]
                \newline \newline
                [The start of the ideal output]
                \newline
                \textcolor{labelcolor}{[Marianne McAndrew (True); is (True); an (True); American (False); actress (True); and (True); singer (False); , (True); born (True); on (True); November 21, 1942 (False); , (True); in (True); Cleveland, Ohio (False); . (True); She (True); began (True); her (True); acting career (True); in (True); the late 1960s (True); , (True); appearing (True); in (True); various (True); television shows (True); and (True); films (True); . (True)]}
                \newline
                [The end of the ideal output]
				\newline \newline
                \textbf{[Example 2]}
                \newline
                [The start of the biography]
                \newline
                \textcolor{titlecolor}{Doug Sheehan is an American actor who was born on April 27, 1949, in Santa Monica, California. He is best known for his roles in soap operas, including his portrayal of Joe Kelly on ``General Hospital'' and Ben Gibson on ``Knots Landing.''}
                \newline
                [The end of the biography]
                \newline \newline
                [The start of the list of checked facts]
                \newline
                \textcolor{anscolor}{[Doug Sheehan is an American. (True); Doug Sheehan is an actor. (True); Doug Sheehan was born on April 27, 1949. (True); Doug Sheehan was born in Santa Monica, California. (False); He is best known for his roles in soap operas. (True); He portrayed Joe Kelly. (True); Joe Kelly was in General Hospital. (True); General Hospital is a soap opera. (True); He portrayed Ben Gibson. (True); Ben Gibson was in Knots Landing. (True); Knots Landing is a soap opera. (True)]}
                \newline
                [The end of the list of checked facts]
                \newline \newline
                [The start of the ideal output]
                \newline
                \textcolor{labelcolor}{[Doug Sheehan (True); is (True); an (True); American (True); actor (True); who (True); was born (True); on (True); April 27, 1949 (True); in (True); Santa Monica, California (False); . (True); He (True); is (True); best known (True); for (True); his roles in soap operas (True); , (True); including (True); in (True); his portrayal (True); of (True); Joe Kelly (True); on (True); ``General Hospital'' (True); and (True); Ben Gibson (True); on (True); ``Knots Landing.'' (True)]}
                \newline
                [The end of the ideal output]
				\newline \newline
				\textbf{User prompt}
				\newline
				\newline
				[The start of the biography]
				\newline
				\textcolor{magenta}{\texttt{\{BIOGRAPHY\}}}
				\newline
				[The ebd of the biography]
				\newline \newline
				[The start of the list of checked facts]
				\newline
				\textcolor{magenta}{\texttt{\{LIST OF CHECKED FACTS\}}}
				\newline
				[The end of the list of checked facts]
			};
			\node[chatbox_user_inner] (q1_text) at (q1) {
				\textbf{System prompt}
				\newline
				\newline
				You are a helpful and precise assistant for segmenting and labeling sentences. We would like to request your help on curating a dataset for entity-level hallucination detection.
				\newline \newline
                We will give you a machine generated biography and a list of checked facts about the biography. Each fact consists of a sentence and a label (True/False). Please do the following process. First, breaking down the biography into words. Second, by referring to the provided list of facts, merging some broken down words in the previous step to form meaningful entities. For example, ``strategic thinking'' should be one entity instead of two. Third, according to the labels in the list of facts, labeling each entity as True or False. Specifically, for facts that share a similar sentence structure (\eg, \textit{``He was born on Mach 9, 1941.''} (\texttt{True}) and \textit{``He was born in Ramos Mejia.''} (\texttt{False})), please first assign labels to entities that differ across atomic facts. For example, first labeling ``Mach 9, 1941'' (\texttt{True}) and ``Ramos Mejia'' (\texttt{False}) in the above case. For those entities that are the same across atomic facts (\eg, ``was born'') or are neutral (\eg, ``he,'' ``in,'' and ``on''), please label them as \texttt{True}. For the cases that there is no atomic fact that shares the same sentence structure, please identify the most informative entities in the sentence and label them with the same label as the atomic fact while treating the rest of the entities as \texttt{True}. In the end, output the entities and labels in the following format:
                \begin{itemize}[nosep]
                    \item Entity 1 (Label 1)
                    \item Entity 2 (Label 2)
                    \item ...
                    \item Entity N (Label N)
                \end{itemize}
                % \newline \newline
                Here are two examples:
                \newline\newline
                \textbf{[Example 1]}
                \newline
                [The start of the biography]
                \newline
                \textcolor{titlecolor}{Marianne McAndrew is an American actress and singer, born on November 21, 1942, in Cleveland, Ohio. She began her acting career in the late 1960s, appearing in various television shows and films.}
                \newline
                [The end of the biography]
                \newline \newline
                [The start of the list of checked facts]
                \newline
                \textcolor{anscolor}{[Marianne McAndrew is an American. (False); Marianne McAndrew is an actress. (True); Marianne McAndrew is a singer. (False); Marianne McAndrew was born on November 21, 1942. (False); Marianne McAndrew was born in Cleveland, Ohio. (False); She began her acting career in the late 1960s. (True); She has appeared in various television shows. (True); She has appeared in various films. (True)]}
                \newline
                [The end of the list of checked facts]
                \newline \newline
                [The start of the ideal output]
                \newline
                \textcolor{labelcolor}{[Marianne McAndrew (True); is (True); an (True); American (False); actress (True); and (True); singer (False); , (True); born (True); on (True); November 21, 1942 (False); , (True); in (True); Cleveland, Ohio (False); . (True); She (True); began (True); her (True); acting career (True); in (True); the late 1960s (True); , (True); appearing (True); in (True); various (True); television shows (True); and (True); films (True); . (True)]}
                \newline
                [The end of the ideal output]
				\newline \newline
                \textbf{[Example 2]}
                \newline
                [The start of the biography]
                \newline
                \textcolor{titlecolor}{Doug Sheehan is an American actor who was born on April 27, 1949, in Santa Monica, California. He is best known for his roles in soap operas, including his portrayal of Joe Kelly on ``General Hospital'' and Ben Gibson on ``Knots Landing.''}
                \newline
                [The end of the biography]
                \newline \newline
                [The start of the list of checked facts]
                \newline
                \textcolor{anscolor}{[Doug Sheehan is an American. (True); Doug Sheehan is an actor. (True); Doug Sheehan was born on April 27, 1949. (True); Doug Sheehan was born in Santa Monica, California. (False); He is best known for his roles in soap operas. (True); He portrayed Joe Kelly. (True); Joe Kelly was in General Hospital. (True); General Hospital is a soap opera. (True); He portrayed Ben Gibson. (True); Ben Gibson was in Knots Landing. (True); Knots Landing is a soap opera. (True)]}
                \newline
                [The end of the list of checked facts]
                \newline \newline
                [The start of the ideal output]
                \newline
                \textcolor{labelcolor}{[Doug Sheehan (True); is (True); an (True); American (True); actor (True); who (True); was born (True); on (True); April 27, 1949 (True); in (True); Santa Monica, California (False); . (True); He (True); is (True); best known (True); for (True); his roles in soap operas (True); , (True); including (True); in (True); his portrayal (True); of (True); Joe Kelly (True); on (True); ``General Hospital'' (True); and (True); Ben Gibson (True); on (True); ``Knots Landing.'' (True)]}
                \newline
                [The end of the ideal output]
				\newline \newline
				\textbf{User prompt}
				\newline
				\newline
				[The start of the biography]
				\newline
				\textcolor{magenta}{\texttt{\{BIOGRAPHY\}}}
				\newline
				[The ebd of the biography]
				\newline \newline
				[The start of the list of checked facts]
				\newline
				\textcolor{magenta}{\texttt{\{LIST OF CHECKED FACTS\}}}
				\newline
				[The end of the list of checked facts]
			};
		\end{tikzpicture}
        \caption{GPT-4o prompt for labeling hallucinated entities.}\label{tb:gpt-4-prompt}
	\end{center}
\vspace{-0cm}
\end{table*}
% \section{Full Experiment Results}
% \begin{table*}[th]
    \centering
    \small
    \caption{Classification Results}
    \begin{tabular}{lcccc}
        \toprule
        \textbf{Method} & \textbf{Accuracy} & \textbf{Precision} & \textbf{Recall} & \textbf{F1-score} \\
        \midrule
        \multicolumn{5}{c}{\textbf{Zero Shot}} \\
                Zero-shot E-eyes & 0.26 & 0.26 & 0.27 & 0.26 \\
        Zero-shot CARM & 0.24 & 0.24 & 0.24 & 0.24 \\
                Zero-shot SVM & 0.27 & 0.28 & 0.28 & 0.27 \\
        Zero-shot CNN & 0.23 & 0.24 & 0.23 & 0.23 \\
        Zero-shot RNN & 0.26 & 0.26 & 0.26 & 0.26 \\
DeepSeek-0shot & 0.54 & 0.61 & 0.54 & 0.52 \\
DeepSeek-0shot-COT & 0.33 & 0.24 & 0.33 & 0.23 \\
DeepSeek-0shot-Knowledge & 0.45 & 0.46 & 0.45 & 0.44 \\
Gemma2-0shot & 0.35 & 0.22 & 0.38 & 0.27 \\
Gemma2-0shot-COT & 0.36 & 0.22 & 0.36 & 0.27 \\
Gemma2-0shot-Knowledge & 0.32 & 0.18 & 0.34 & 0.20 \\
GPT-4o-mini-0shot & 0.48 & 0.53 & 0.48 & 0.41 \\
GPT-4o-mini-0shot-COT & 0.33 & 0.50 & 0.33 & 0.38 \\
GPT-4o-mini-0shot-Knowledge & 0.49 & 0.31 & 0.49 & 0.36 \\
GPT-4o-0shot & 0.62 & 0.62 & 0.47 & 0.42 \\
GPT-4o-0shot-COT & 0.29 & 0.45 & 0.29 & 0.21 \\
GPT-4o-0shot-Knowledge & 0.44 & 0.52 & 0.44 & 0.39 \\
LLaMA-0shot & 0.32 & 0.25 & 0.32 & 0.24 \\
LLaMA-0shot-COT & 0.12 & 0.25 & 0.12 & 0.09 \\
LLaMA-0shot-Knowledge & 0.32 & 0.25 & 0.32 & 0.28 \\
Mistral-0shot & 0.19 & 0.23 & 0.19 & 0.10 \\
Mistral-0shot-Knowledge & 0.21 & 0.40 & 0.21 & 0.11 \\
        \midrule
        \multicolumn{5}{c}{\textbf{4 Shot}} \\
GPT-4o-mini-4shot & 0.58 & 0.59 & 0.58 & 0.53 \\
GPT-4o-mini-4shot-COT & 0.57 & 0.53 & 0.57 & 0.50 \\
GPT-4o-mini-4shot-Knowledge & 0.56 & 0.51 & 0.56 & 0.47 \\
GPT-4o-4shot & 0.77 & 0.84 & 0.77 & 0.73 \\
GPT-4o-4shot-COT & 0.63 & 0.76 & 0.63 & 0.53 \\
GPT-4o-4shot-Knowledge & 0.72 & 0.82 & 0.71 & 0.66 \\
LLaMA-4shot & 0.29 & 0.24 & 0.29 & 0.21 \\
LLaMA-4shot-COT & 0.20 & 0.30 & 0.20 & 0.13 \\
LLaMA-4shot-Knowledge & 0.15 & 0.23 & 0.13 & 0.13 \\
Mistral-4shot & 0.02 & 0.02 & 0.02 & 0.02 \\
Mistral-4shot-Knowledge & 0.21 & 0.27 & 0.21 & 0.20 \\
        \midrule
        
        \multicolumn{5}{c}{\textbf{Suprevised}} \\
        SVM & 0.94 & 0.92 & 0.91 & 0.91 \\
        CNN & 0.98 & 0.98 & 0.97 & 0.97 \\
        RNN & 0.99 & 0.99 & 0.99 & 0.99 \\
        % \midrule
        % \multicolumn{5}{c}{\textbf{Conventional Wi-Fi-based Human Activity Recognition Systems}} \\
        E-eyes & 1.00 & 1.00 & 1.00 & 1.00 \\
        CARM & 0.98 & 0.98 & 0.98 & 0.98 \\
\midrule
 \multicolumn{5}{c}{\textbf{Vision Models}} \\
           Zero-shot SVM & 0.26 & 0.25 & 0.25 & 0.25 \\
        Zero-shot CNN & 0.26 & 0.25 & 0.26 & 0.26 \\
        Zero-shot RNN & 0.28 & 0.28 & 0.29 & 0.28 \\
        SVM & 0.99 & 0.99 & 0.99 & 0.99 \\
        CNN & 0.98 & 0.99 & 0.98 & 0.98 \\
        RNN & 0.98 & 0.99 & 0.98 & 0.98 \\
GPT-4o-mini-Vision & 0.84 & 0.85 & 0.84 & 0.84 \\
GPT-4o-mini-Vision-COT & 0.90 & 0.91 & 0.90 & 0.90 \\
GPT-4o-Vision & 0.74 & 0.82 & 0.74 & 0.73 \\
GPT-4o-Vision-COT & 0.70 & 0.83 & 0.70 & 0.68 \\
LLaMA-Vision & 0.20 & 0.23 & 0.20 & 0.09 \\
LLaMA-Vision-Knowledge & 0.22 & 0.05 & 0.22 & 0.08 \\

        \bottomrule
    \end{tabular}
    \label{full}
\end{table*}




\end{document}

\end{document}