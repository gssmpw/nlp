\documentclass[conference]{IEEEtran}
\IEEEoverridecommandlockouts
% \usepackage{cite}
\usepackage{amsmath,amssymb,amsfonts}
\usepackage{algorithmic}
\usepackage{graphicx}
\usepackage{textcomp}
\usepackage{xcolor}
\usepackage{multirow}
\usepackage[utf8]{inputenc}
% \usepackage{biblatex}
% \usepackage[backend=biber, style=ieee]{biblatex}  % biber 사용 (bibtex 아님)
\usepackage{kotex}
\usepackage{caption}
\usepackage{graphicx}
\usepackage{amsmath}
\usepackage{booktabs}
\usepackage{multirow}
\usepackage{makecell}
\usepackage{rotating}
\usepackage{hyperref}

\usepackage[backend=biber,style=ieee]{biblatex} % Using IEEE style with biblatex
\addbibresource{references.bib} % Ensure the .bib file name is correct


% \bibliographystyle{IEEEtran}  % IEEEtran 스타일 사용

% \addbibresource{references.bib} % Add the references.bib file
% \bibliography{conference_101719}  % .bbl 없이 .bib 파일명을 입력
% \bibliography{conference_101719}  % .bbl 없이 .bib 파일명만 입력


\def\BibTeX{{\rm B\kern-.05em{\sc i\kern-.025em b}\kern-.08em
    T\kern-.1667em\lower.7ex\hbox{E}\kern-.125emX}}
\begin{document}

% \documentclass[conference]{IEEEtran}
\IEEEoverridecommandlockouts
% The preceding line is only needed to identify funding in the first footnote. If that is unneeded, please comment it out.
\usepackage{cite}
\usepackage{amsmath,amssymb,amsfonts}
\usepackage{algorithmic}
\usepackage{graphicx}
\usepackage{textcomp}
\usepackage{xcolor}
\def\BibTeX{{\rm B\kern-.05em{\sc i\kern-.025em b}\kern-.08em
    T\kern-.1667em\lower.7ex\hbox{E}\kern-.125emX}}
\begin{document}

\title{Privacy Preserving Properties \\of \\Vision Classifiers\\
%{\footnotesize \textsuperscript{*}Note: Sub-titles are not captured in %should not be used}
%\thanks{Identify applicable funding agency here. If none, delete this.}
}

%\author{\IEEEauthorblockN{Anonymous Authors}}

\author{\IEEEauthorblockN{1\textsuperscript{st} Pirzada Suhail}
\IEEEauthorblockA{\textit{IIT Bombay} \\
Mumbai, India \\
psuhail@iitb.ac.in}
\and
\IEEEauthorblockN{2\textsuperscript{nd} Amit Sethi}
\IEEEauthorblockA{\textit{IIT Bombay} \\
Mumbai, India \\
asethi@iitb.ac.in}
}

\maketitle

\begin{abstract}

Vision classifiers are often trained on proprietary datasets containing sensitive information, yet the models themselves are frequently shared openly under the privacy-preserving assumption. Although these models are assumed to protect sensitive information in their training data, the extent to which this assumption holds for different architectures remains unexplored. This assumption is challenged by inversion attacks which attempt to reconstruct training data from model weights, exposing significant privacy vulnerabilities. In this study, we systematically evaluate the privacy-preserving properties of vision classifiers across diverse architectures, including Multi-Layer Perceptrons (MLPs), Convolutional Neural Networks (CNNs), and Vision Transformers (ViTs). Using network inversion-based reconstruction techniques, we assess the extent to which these architectures memorize and reveal training data, quantifying the relative ease of reconstruction across models. Our analysis highlights how architectural differences, such as input representation, feature extraction mechanisms, and weight structures, influence privacy risks. By comparing these architectures, we identify which are more resilient to inversion attacks and examine the trade-offs between model performance and privacy preservation, contributing to the development of secure and privacy-respecting machine learning models for sensitive applications. Our findings provide actionable insights into the design of secure and privacy-aware machine learning systems, emphasizing the importance of evaluating architectural decisions in sensitive applications involving proprietary or personal data.

\end{abstract}

\begin{IEEEkeywords}
Safety, Privacy, Network Inversion, Reconstructions
\end{IEEEkeywords}

\section{Introduction}

The advent of modern vision classifiers has revolutionized a wide range of applications, from autonomous vehicles and medical imaging to facial recognition. These models, often trained on proprietary datasets, have become a cornerstone of advancements in artificial intelligence (AI). However, the growing practice of sharing pre-trained models has raised critical concerns about the privacy of the training data. Many assume that the act of sharing a trained model inherently preserves the privacy of the underlying dataset, but this assumption is increasingly being challenged by research demonstrating the potential for data leakage. The question remains: to what extent can different model architectures safeguard sensitive information against reconstruction or inversion attacks?

Model inversion attacks, where an adversary attempts to reconstruct the training data from a model's weights or output, highlight the vulnerability of machine learning systems to privacy breaches. This becomes particularly alarming in domains like healthcare, where proprietary datasets often contain highly sensitive information, or in applications involving biometric data, where privacy is paramount. Vision classifiers, in particular, process inherently personal and identifiable information, such as faces or medical scans, making it imperative to evaluate their privacy-preserving properties comprehensively.

In this study, we systematically assess the privacy-preserving capabilities of three prominent architectures: Multi-Layer Perceptrons (MLPs), Convolutional Neural Networks (CNNs)\cite{dosovitskiy2021imageworth16x16words}, and Vision Transformers (ViTs)\cite{oshea2015introductionconvolutionalneuralnetworks}. These architectures differ significantly in their structure, feature extraction mechanisms, and input processing pipelines, which could influence their tendencies to memorize and inadvertently leak training data. For example, CNNs are designed to capture spatial hierarchies and local patterns, while ViTs, on the other hand, employ self-attention mechanisms that focus on global relationships between input elements, potentially resulting in different privacy implications. By comparing these architectures, we aim to understand their relative vulnerabilities and the factors that contribute to privacy risks.

To evaluate the privacy-preserving properties of these classifiers, we utilize network inversion-based reconstruction techniques as in \cite{suhail2024net}. These methods attempt to recover the training data entirely from the model weights, providing a quantifiable measure of privacy leakage. Such reconstruction attacks exploit the tendency of models to memorize specific details about their training data, particularly when the training set is small or when over-parameterized architectures are used. By systematically applying these techniques, we assess the relative ease of reconstruction across architectures and analyze the impact of architectural differences on privacy.

We conduct our analysis in the most extreme case of privacy risk: at the end of the training process, without any knowledge of the training process itself, without utilizing any unobvious prior information, without any auxiliary datasets and relying on a single trained model. This setup provides a stringent evaluation of privacy risks, focusing on the inherent vulnerabilities of the model architectures. Further the quality of the reconstructed samples is assessed by comparing them to the original training samples using a similarity metric. Reconstructions with higher Structural Similarity Index Measure (SSIM) values indicate a greater privacy risk, as they suggest more effective memorization of the training data by the model. 

Our findings highlight that architectural differences in processing input images, feature extraction, and weight structures contribute to varying degrees of privacy leakage. We apply our evaluation to multiple benchmark datasets, including MNIST, FashionMNIST, CIFAR-10, and SVHN that cover a wide range of complexities and image types, allowing us to study how different architectures behave under varying data conditions.

\section{Related Works}

Privacy concerns in machine learning have led to extensive research in Privacy-Preserving Machine Learning (PPML), particularly in mitigating risks related to membership inference, attribute inference, and model inversion attacks. A foundational study by \cite{8677282} provides an overview of privacy threats in ML, including model inversion, and explores defenses such as differential privacy, homomorphic encryption, and federated learning. Similarly, \cite{xu2021privacypreservingmachinelearningmethods} introduces the Phase, Guarantee, and Utility (PGU) triad, a framework to evaluate PPML techniques across different phases of the ML pipeline. These studies highlight the need for privacy-preserving methods but primarily focus on algorithmic-level defenses rather than evaluating inherent vulnerabilities in different model architectures. Unlike these approaches, our study investigates the privacy risks posed by architectural design choices by analyzing how MLPs, CNNs, and ViTs differ in their susceptibility to model inversion attacks.

Network inversion has emerged as a powerful technique to understand how neural networks encode and manipulate training data. Initially developed for interpretability \cite{KINDERMANN1990277,784232}, it has since been shown to reconstruct sensitive training samples, raising significant privacy concerns \cite{Wong2017NeuralNI,ad}. Early works on network inversion focused on fully connected networks (MLPs) \cite{KINDERMANN1990277,SAAD200778}, demonstrating that they tend to memorize training data, making them vulnerable to inversion attacks. Evolutionary inversion procedures \cite{784232} improved the ability to capture input-output relationships, providing deeper insights into model memorization behavior. More recent studies extended these inversion techniques to CNNs \cite{ad}, showing that hierarchical feature extraction does not necessarily prevent training data leakage. The introduction of ViTs has further complicated this issue, as their global self-attention mechanisms process data differently than CNNs, raising new questions about how they store training information and whether their memorization patterns lead to higher or lower inversion risks.

In adversarial settings, model inversion attacks aim to reconstruct sensitive data by exploiting a model's predictions, gradients, or weights \cite{ad,kumar2019modelinversionnetworksmodelbased}. These attacks have been shown to succeed even without direct access to the training process, as demonstrated by \cite{9833677}, where an adversary with auxiliary knowledge reconstructs sensitive samples. Gradient-based inversion attacks further exacerbate these risks by leaking sensitive training information through shared gradients in federated learning setups \cite{pmlr-v206-wang23g}.

To improve the stability of inversion processes, recent works have explored novel optimization techniques. For example, \cite{liu2022landscapelearningneuralnetwork} proposed learning a loss landscape to make gradient-based inversion faster and more stable. Alternative approaches, such as encoding networks into Conjunctive Normal Form (CNF) and solving them using SAT solvers, offer deterministic solutions for inversion, as introduced by \cite{suhail2024network}. Although computationally expensive, these methods ensure diversity in the reconstructed samples by avoiding shortcuts in the optimization process.

Model Inversion (MI) attacks have also been extended to scenarios involving ensemble techniques, where multiple models trained on shared subjects or entities are used to guide the reconstruction process. The concept of ensemble inversion, as proposed by \cite{wang2021reconstructingtrainingdatadiverse}, enhances the quality of reconstructed data by leveraging the diversity of perspectives provided by multiple models. By incorporating auxiliary datasets similar to the presumed training data, this approach achieves high-quality reconstructions with sharper predictions and higher activations. This work highlights the risks posed by adversaries exploiting shared data entities across models, emphasizing the importance of robust defense mechanisms.

Reconstruction methods for training data have evolved significantly, focusing on improving the efficiency and accuracy of recovering data from models. Traditional optimization-based approaches relied on iteratively refining input data to match a model’s outputs or activations \cite{Wong2017NeuralNI}. More recent advancements have leveraged generative models, such as GANs and autoencoders, to synthesize high-quality reconstructions. These techniques aim to approximate the distribution of training data while maintaining computational efficiency. In the context of privacy risks, works like \cite{haim2022reconstructingtrainingdatatrained} demonstrated that significant portions of training data could be reconstructed from neural network parameters in binary classification settings. This work was later extended to multi-class classification by \cite{buzaglo2023reconstructingtrainingdatamulticlass}, showing that higher-quality reconstructions are possible and revealing the impact of regularization techniques, such as weight decay, on memorization behavior.

The ability to reconstruct training data from model gradients also presents a critical privacy challenge. The study by \cite{wang2023reconstructingtrainingdatamodel} demonstrated that training samples could be fully reconstructed from a single gradient query, even without explicit training or memorization. Recent advancements like \cite{oz2024reconstructingtrainingdatareal}, adapt reconstruction schemes to operate in the embedding space of large pre-trained models, such as DINO-ViT and CLIP. This approach, which introduces clustering-based methods to identify high-quality reconstructions from numerous candidates, represents a significant improvement over earlier techniques that required access to the original dataset. While \cite{pmlr-v162-guo22c} extended differential privacy guarantees to training data reconstruction attacks.

In this paper, we build upon prior work on network inversion and training data reconstruction. Drawing from works like \cite{suhail2024networkcnn} and \cite{suhail2024networkinversionapplications}, we employ network inversion methods to understand the internal representations of neural networks and the patterns they memorize during training. Our study systematically compares the privacy-preserving properties of different vision classifier architectures, including MLPs, CNNs, and ViTs, using network inversion-based reconstruction techniques \cite{suhail2024net}. By evaluating these techniques across datasets such as MNIST, FashionMNIST, CIFAR-10, and SVHN, we explore the impact of architectural differences, input processing mechanisms, and weight structures on the susceptibility of models to inversion attacks.


\section{Methodology}

\subsection{Overview}
In this study, we investigate the ease of reconstruction and the extent of memorization in trained vision classifiers based on different architectures. Our primary focus is on analyzing the most extreme case of training data reconstruction, where the inversion process relies almost entirely on the input-output relationships of the trained model and its learned weights. Unlike prior reconstruction approaches that leverage pre-trained models, auxiliary datasets, gradient information from the training process, or other unobvious priors, our method seeks to reconstruct training data with minimal external dependencies. This approach allows us to systematically evaluate how different architectures—Multi-Layer Perceptrons (MLPs), Convolutional Neural Networks (CNNs), and Vision Transformers (ViTs)—differ in their ability to preserve or expose sensitive training data.

To perform network inversion and data reconstruction, we build upon the methodology introduced in \cite{suhail2024networkcnn, suhail2024networkinversionapplications}, which has primarily focused on CNN-based classifiers. We extend this approach to other architectures, particularly MLPs and ViTs, to assess their relative vulnerability to inversion attacks. Briefly, network inversion techniques aim to generate inputs that align with the learned decision boundaries of a classifier by training a conditioned generator to reconstruct data that maximally activates specific output neurons. In its standard form, this inversion process does not necessarily yield images resembling actual training samples but instead produces arbitrary inputs that satisfy the model’s learned function. However, by modifying the inversion procedure following \cite{suhail2024network, suhail2024net}, we incentivize the generator to reconstruct training-like data by leveraging key properties of the classifier with respect to its training data. These modifications allow us to better quantify the extent of memorization across different architectures and assess the associated privacy risks. The proposed approach to Network Inversion and subsequent training data reconstruction uses a carefully conditioned generator that learns the data distributions in the input space of the trained classifier.

\subsection{Classifier Architectures}
We perform inversion and reconstruction on classifiers based on three distinct architectures:

\begin{itemize}
    \item \textbf{Multi-Layer Perceptrons (MLPs)}: These are fully connected networks where each neuron in one layer is connected to every neuron in the next layer. MLPs process flattened input images, lacking any inherent spatial hierarchy. The inversion and subsequent reconstruction will be performed using the logits and penultimate fully connected layers.
    \item \textbf{Convolutional Neural Networks (CNNs)}: CNNs use convolutional layers to extract hierarchical spatial features from images, enabling them to effectively capture local patterns. In this case the features from the fully connected layers are used after flattening the output of convolutional layers along with the logits in the last layer to perform inversion.
    \item \textbf{Vision Transformers (ViTs)}: ViTs utilize self-attention mechanisms to capture global dependencies across an image. This architecture is particularly effective in modeling long-range interactions within an image. In ViTs we particularly look at the classification token embeddings and use it in the same way as above.
\end{itemize}

These architectures have inherently different memory capacities and generalization properties, affecting their susceptibility to reconstruction attacks.

\subsection{Vector-Matrix Conditioned Generator}
The generator in our approach is conditioned on vectors and matrices to ensure that it learns diverse representations of the data distribution. Unlike simple label conditioning, the vector-matrix conditioning mechanism encodes the label information more intricately, allowing the generator to better capture the input space of the classifier. 

The generator is initially conditioned using $N$-dimensional vectors for an $N$-class classification task. These vectors are derived from a normal distribution and are softmaxed to form a probability distribution. They implicitly encode the labels, promoting diversity in the generated images.

Further, a Hot Conditioning Matrix of size $N \times N$ is used for deeper conditioning. In this matrix, all elements in a specific row or column are set to $1$, corresponding to the encoded label, while the rest are $0$. This conditioning is applied during intermediate stages of the generation process to refine the diversity of the outputs.

\subsection{Training Data Properties}
The classifier exhibits specific properties when interacting with training data, which are exploited to facilitate the reconstruction of training-like samples:
\begin{figure*}[t]
\centering
\includegraphics[width=1\textwidth]{tldr.png} 
\caption{Schematic Approach to Training-Like Data Reconstruction using Network Inversion}
\label{fig:reconstruction}
\end{figure*}

\begin{itemize}
    \item \textbf{Model Confidence:} The classifier is more confident when predicting labels for training samples compared to random samples. Hence, in order to take this into account we condition the generation on one-hot vectors and then minimise its KL Divergence from the classifier's output enforcing generation of samples that are confidently classified buy the classifier. This can be expressed as:
    \begin{equation}
    P(y_{\text{in}} | x_{\text{in}}; \theta) \gg P(y_{\text{ood}} | x_{\text{ood}}; \theta)
    \end{equation}

    \item \textbf{Robustness to Perturbations:} During training the model gets to observe a diverse set of data including different variations of the images in the same class. Due to which the model is relatively robust to perturbations around training data compared to random inverted samples, meaning small changes do not significantly affect predictions. We take this into account in by perturbing the generated images and then ensuring that the perturbed images also produce similar output,
    \begin{equation}
    \frac{\partial f_{\theta}(x_{\text{in}})}{\partial x_{\text{in}}} \ll \frac{\partial f_{\theta}(x_{\text{ood}})}{\partial x_{\text{ood}}}
    \end{equation}

    \item \textbf{Gradient Behavior:} By virtue of training the model on a certain dataset, the gradient of the loss with respect to model weights is expected to be lower for training data compared to random inverted samples, as the model has already been optimized on it, Hence we add a penalty on the gradients that encourages the generation of samples with low gradients as:
    \begin{equation}
    \|\nabla_{\theta} L(f_{\theta}(x_{\text{in}}), y_{\text{in}})\| \ll \|\nabla_{\theta} L(f_{\theta}(x_{\text{ood}}), y_{\text{ood}})\|
    \end{equation}
\end{itemize}

\subsection{Training Data Reconstruction}
The reconstruction process utilizes the generator to produce training-like samples by taking into account the specific properties of the training data with respect to the classifier. The approach is schematically illustrated in Figure \ref{fig:reconstruction}. 


The primary loss function used for reconstruction, 
\begin{align*}
\mathcal{L}_{\text{Recon}} = & \; \alpha \cdot \mathcal{L}_{\text{KL}} 
+ \alpha' \cdot \mathcal{L}_{\text{KL}}^{\text{pert}}
+ \beta \cdot \mathcal{L}_{\text{CE}} 
+ \beta' \cdot \mathcal{L}_{\text{CE}}^{\text{pert}} \\
& + \gamma \cdot \mathcal{L}_{\text{Cosine}} 
+ \delta \cdot \mathcal{L}_{\text{Ortho}} \\
& + \eta_1 \cdot \mathcal{L}_{\text{Var}} 
+ \eta_2 \cdot \mathcal{L}_{\text{Pix}} 
+ \eta_3 \cdot \mathcal{L}_{\text{Grad}}
\end{align*}

where \( \mathcal{L}_{\text{KL}} \) is the KL Divergence loss, \( \mathcal{L}_{\text{CE}} \) is the Cross Entropy loss, \( \mathcal{L}_{\text{Cosine}} \) is the Cosine Similarity loss, and \( \mathcal{L}_{\text{Ortho}} \) is the Feature Orthogonality loss. The hyperparameters \( \alpha, \beta, \gamma, \delta \) control the contribution of each individual loss term defined as:
\[
\mathcal{L}_{\text{KL}} = D_{\text{KL}}(P \| Q) = \sum_{i} P(i) \log \frac{P(i)}{Q(i)}
\]
\[
\mathcal{L}_{\text{CE}} = -\sum_{i} y_{i} \log(\hat{y}_{i})
\]
\[
\mathcal{L}_{\text{Cosine}} = \frac{1}{N(N-1)} \sum_{i \neq j} \cos(\theta_{ij})
\]
\[
\mathcal{L}_{\text{Ortho}} = \frac{1}{N^2} \sum_{i, j} (G_{ij} - \delta_{ij})^2
\]
where \( D_{\text{KL}} \) represents the KL Divergence between the input distribution \( P \) and the output distribution \( Q \), \( y_{i} \) is the set encoded label, \( \hat{y}_{i} \) is the predicted label from the classifier, \( \cos(\theta_{ij}) \) represents the cosine similarity between features of generated images \( i \) and \( j \), \( G_{ij} \) is the element of the Gram matrix, and \( \delta_{ij} \) is the Kronecker delta function. \( N \) is the number of feature vectors in the batch.

Further to take reconstruction into account we also use \(\mathcal{L}_{\text{KL}}^{\text{pert}}\) and \(\mathcal{L}_{\text{CE}}^{\text{pert}}\) that represent the KL divergence and cross-entropy losses applied on perturbed images, weighted by \( \alpha'\) and  \(\beta' \)respectively while \(\mathcal{L}_{\text{Var}}\), \(\mathcal{L}_{\text{Pix}}\) and \(\mathcal{L}_{\text{Grad}}\) represent the variational loss, Pixel Loss and penalty on gradient norm each weighted by \( \eta_1\), \( \eta_2\), and \(\eta_3\) respectively and defined for an Image \(I\) as:
\begin{align*}
\mathcal{L}_{\text{Var}} = \frac{1}{N} \sum_{i=1}^{N} \Bigg( \sum_{h,w} & \Big( ( I_{i, h+1, w} - I_{i, h, w} )^2 \\
& + ( I_{i, h, w+1} - I_{i, h, w} )^2 \Big) \Bigg)
\end{align*}
\[
\mathcal{L}_{\text{Grad}} = \left\| \nabla_{\theta} L(f_{\theta}(I), y) \right\|
\]

\[
\mathcal{L}_{\text{Pix}} = \sum \max(0, -I) + \sum \max(0, I - 1)
\]

By integrating these loss components, the generator is trained to produce samples that closely resemble the training data, thus revealing the extent of memorization within the classifier.


In this section, we present the experimental results obtained by applying the reconstruction technique on the MNIST \cite{deng2012mnist}, FashionMNIST \cite{xiao2017fashionmnistnovelimagedataset}, SVHN, and CIFAR-10 \cite{cf} datasets. Our goal is to evaluate the ease of reconstruction and the extent of memorization in different vision classifier architectures by training a generator to produce images that resemble the training data. The classifier is first trained normally on a given dataset and then held in evaluation mode for the purpose of reconstruction. The conditioned generator, which takes as input latent vectors and conditioning information, is trained to generate images that the classifier maps to specific labels.

We evaluate three vision classifier architectures: Multi-Layer Perceptrons (MLPs), Convolutional Neural Networks (CNNs), and Vision Transformers (ViTs). We implement a 5-layer MLP with Batch Normalization, Leaky ReLU activations, and Dropout layers \cite{JMLR:v15:srivastava14a} to mitigate memorization tendencies. The CNN consists of three convolutional layers followed by batch normalization \cite{pmlr-v37-ioffe15}, dropout layers, and a fully connected layer for classification. The ViT classifier uses a transformer-based architecture with three self-attention layers, each containing four attention heads while the input images are divided into non-overlapping patches of size \(4 \times 4\).

The generator, instead of traditional label embeddings, follows a Vector-Matrix Conditioning approach to ensure diverse and structured image generation. The class labels are encoded into random softmaxed vectors concatenated with the latent vector, followed by multiple layers of transposed convolutions, batch normalization, and dropout layers to encourage diversity in generated images. Once the concatenated latent and conditioning vectors are upsampled to \(N \times N\) spatial dimensions for an \(N\)-class classification task, they are concatenated with a conditioning matrix for further generation up to the required image size (\(28 \times 28\) for MNIST and FashionMNIST, and \(32 \times 32\) for SVHN and CIFAR-10).

To assess the extent of memorization and the relative ease of reconstruction across different architectures, we use the Structural Similarity Index Measure (SSIM). The goal was to evaluate how different architectures handle privacy risks by comparing the quality of reconstructed images. SSIM quantifies the perceptual similarity between two images based on luminance, contrast, and structural similarity components. In this study, SSIM is used to compare the reconstructed images with their closest matches from the training dataset. Higher SSIM values indicate stronger resemblance to training samples, suggesting a greater privacy risk due to increased memorization by the classifier.

Table~\ref{tab:ssim_values} presents the SSIM scores between reconstructed samples and training data, averaged across all classes for each dataset and architecture. Higher SSIM values suggest greater memorization and weaker privacy preservation, as the reconstructed samples strongly resemble real training data.

\begin{table}[h]
\centering
\caption{SSIM values for reconstructed samples across different architectures and datasets.}
\label{tab:ssim_values}
\resizebox{\linewidth}{!}{
\begin{tabular}{|l|c|c|c|}
\hline
\textbf{Dataset}       & \textbf{MLP} & \textbf{ViT} & \textbf{CNN} \\ \hline
\textbf{MNIST}         & 0.83         & 0.78         & 0.73         \\ \hline
\textbf{FashionMNIST}  & 0.74         & 0.64         & 0.63         \\ \hline
\textbf{SVHN}          & 0.71         & 0.68         & 0.69         \\ \hline
\textbf{CIFAR-10}      & 0.65         & 0.62         & 0.58         \\ \hline
\end{tabular}
}
\end{table}

From Table~\ref{tab:ssim_values}, we observe that MLPs exhibit the highest SSIM scores across all datasets, suggesting that they retain more training data details than CNNs and ViTs. This is likely due to their fully connected nature and lack of spatial inductive biases, leading to higher memorization tendencies. Also individual pixels in the images have dedicated weights associated, making memorization easier.

ViTs generally show lower SSIM values than MLPs but remain slightly higher than CNNs, indicating that self-attention mechanisms contribute to retaining finer details in the reconstructed images. Unlike CNNs, ViTs lack pooling layers, which means that more spatial information is preserved, making inversion attacks more feasible. 

CNNs show the lowest SSIM values across all datasets, suggesting that they inherently discard more specific input details due to weight sharing, local receptive fields, and pooling operations. This abstraction reduces direct memorization and makes CNNs relatively more privacy-preserving compared to MLPs and ViTs.

\begin{figure}[ht]
\centering
\includegraphics[width=0.95\linewidth]{riz.png} 
\caption{Comparison of reconstructed samples across MLP, ViT, and CNN architectures. The first column represents actual training samples, while subsequent columns show corresponding reconstructed images.}
\label{fig:sam}
\end{figure}

Figure~\ref{fig:sam} presents a qualitative comparison of reconstructed samples across different classifier architectures. The first column depicts actual training samples, while the subsequent columns display reconstructed images generated from MLPs, ViTs, and CNNs. Notably, MLP-generated reconstructions appear most similar to the original samples, while CNNs yield more abstract representations, indicating lower memorization.

These results highlight the variation in privacy risks across architectures, with MLPs being the most susceptible to memorization, followed by ViTs, and CNNs exhibiting the lowest reconstruction fidelity. The findings emphasize the importance of architectural choices in designing privacy-aware models, where CNNs may be preferable for applications requiring privacy preservation.

\section{Conclusion and Future Work}
In this paper, we systematically evaluated the privacy-preserving properties of vision classifiers by analyzing the extent of memorization and the ease of training data reconstruction across different architectures. Our experimental results, quantified through SSIM scores, indicate that MLPs tend to memorize more information about training samples compared to CNNs and ViTs, making them more prone to reconstruction. These findings highlight the crucial role of architectural choices in mitigating privacy risks, emphasizing the need for privacy-aware model deployment, especially in sensitive applications.

Future research can extend this study by exploring additional factors that influence privacy leakage, such as model depth, dataset complexity, and the impact of regularization techniques. In the case of ViTs it would be also of interest to see how the reconstructions differ with varying patch size. Investigating the impact of differential privacy mechanisms in reducing reconstruction risks could provide valuable insights into enhancing privacy.

\bibliographystyle{IEEEtran}  % Or any other preferred style
\bibliography{references} 

\end{document}
  % .bbl 파일 직접 포함


\title{PixleepFlow: A Pixel-Based Lifelog Framework for Predicting Sleep Quality and Stress Level}

\author{\IEEEauthorblockN{Younghoon Na\textsuperscript{\dag}}
\IEEEauthorblockA{\textit{Seoul National University} \\
yh0728@snu.ac.kr}
\and
\IEEEauthorblockN{Seunghun Oh\textsuperscript{\dag}}
\IEEEauthorblockA{\textit{Hallym University}\\
gnsgus190@gmail.com}
\and
\IEEEauthorblockN{Seongji Ko\textsuperscript{\dag}}
\IEEEauthorblockA{\textit{Enssel Co., Ltd.}\\
sjko@gmail.com}
\and
\IEEEauthorblockN{Hyunkyung Lee\textsuperscript{*}}
\IEEEauthorblockA{\textit{Seoul National University} \\
hyunkyung913@snu.ac.kr}
\and

\thanks{\textsuperscript{\dag} These authors contributed equally to this work.}
\thanks{\textsuperscript{*} Corresponding author.}

}

\maketitle

\begin{abstract}
The analysis of lifelogs can yield valuable insights
into an individual's daily life, particularly with regard to their health and well-being.
The accurate assessment of quality of life is necessitated by the use of diverse sensors and precise synchronization.
% 
To rectify this issue, this study proposes the image-based sleep quality and stress level estimation flow (PixleepFlow). PixleepFlow employs a conversion methodology into composite image data to examine sleep patterns and their impact on overall health. Experiments were conducted using lifelog datasets to ascertain the optimal combination of data formats. In addition, we identified which sensor information has the greatest influence on the quality of life through Explainable Artificial Intelligence(XAI). As a result, PixleepFlow produced more significant results than various data formats. This study was part of a written-based competition, and the additional findings from the lifelog dataset are detailed in Section \ref{Section:Extended Comparative Analysis}. More information about PixleepFlow can be found at\href{https://github.com/seongjiko/Pixleep}{https://github.com/seongjiko/Pixleep}. 

\end{abstract}

\begin{IEEEkeywords}
Lifelog, Sleep, Sleep Quality, Stress Level, Emotion Level, Deep Learning, Explainable AI, Wearable Devices, Health Monitoring, Image-Based Sleep Research, Multi-View Learning
\end{IEEEkeywords}

\section{Introduction}
Lifelogs, such as activity, heart rate and sleep data, reflect the quality of life. Changes in vital signals during daily activities and sleep can provide insights into sleep quality, emotions, and stress levels, which may have implications for overall health and well-being \cite{scataglini2023wearable,zhou2022population}.

Advanced sensors and deep learning models facilitate the inference of overall physical health from lifelog data without hospital visits. For instance, TzuAn Song's automated mobile sleep staging model (SLAMSS) predicts deep sleep (N3) with 79\% accuracy using wrist-worn actigraphy data \cite{song2023ai}. Rohit Gupta predicts stress and emotional states from wrist and chest sensor signals, achieving 95.54\% accuracy \cite{gupta2023multimodal}.
Hang Yuan developed a self-supervised deep learning model for sleep stage classification using wrist-worn accelerometers, a three-class (Wake/REM/NREM) classification F1 score of 0.57, indicating that the difference between polysomnography and model classification in external validation was a fairly accurate prediction of 34.7 minutes of total sleep time \cite{yuan2024self}. 
%
These advancements demonstrate the potential of data from smartphones and wearable to provide valuable insights into daily experiences and health monitoring.

\begin{figure}[ht]
\centering
\includegraphics[width=0.5\textwidth]{figures/pixleepFlow_framework.jpg}
\caption{PixleepFlow framework}
\label{fig:PixleepFlow}
\end{figure}

In most existing studies, sleep quality and stress level are predicted separately to assess the quality of life. However, we propose a more comprehensive approach by applying multi-label classification that simultaneously considers both sleep quality and stress level. To achieve this, we developed image-based sleep quality and stress level estimation flow (PixleepFlow), a model that leverages image-based input from various sensor data to concurrently predict seven key metrics related to sleep quality and stress level with high accuracy.

A key characteristic of PixleepFlow is its implementation of image-based inputs. By converting long-duration sensor data into image-based images, PixleepFlow effectively reduces high-dimensional data into three-dimensional (RGB) components. This approach also allows the data to be downsampled to a lower resolution, preventing the model from being overly focused on fine details. Additionally, this method enhances the model's ability to intuitively identify patterns and anomalies in the data compared to traditional time-series representations. These advantages make PixleepFlow an explainable deep learning model that comprehensively captures the multifaceted nature of human behavior in multi-modal tasks.

PixleepFlow is designed to estimate sleep quality and stress level by converting synchronized sensor data into composite image data. The model evaluates these metrics using a total of seven indicators derived from daily survey records and sleep sensor data: overall sleep quality as perceived by the participant immediately after waking up (Q1), emotional state just before sleep (Q2), stress level experienced just before sleep (Q3), total sleep time (TST, S1), sleep efficiency (SE, S2), sleep onset latency (SOL, S3), and wake after sleep onset (WASO, S4). Moreover, to capitalize on the advantages of image translation, PixleepFlow incorporates Explainable Artificial Intelligence (XAI) techniques to provide visual interpretations of the results, enhancing the transparency and explainability of the model's predictions. The PixleepFlow framework is shown in Fig. \ref{fig:PixleepFlow}.

By utilizing PixleepFlow, we can more effectively assess both sleep quality and stress level while providing an intuitive, visual understanding of the data patterns and anomalies. This allows for a comprehensive analysis and deeper understanding of the complex factors associated with sleep.

Based on the predicting human lifelog metrics competition, we aim to discover the following:

1) Efficacious Methodology: Describe effective methodologies for achieving optimal performance in the competition.

2) Optimal Model Architectures: Identify which model architectures perform best on lifelog datasets by comparing deep learning approaches.

3) Data Format Suitability: Determine which data formats (raw signals, spectrograms, or images) most effectively represent lifelog datasets and yield the highest predictive F1 score.

4) Critical Channels: Analyze which specific sensor channels (e.g., acceleration, heart rate, Global Positioning System GPS) are most strongly correlated with stress level metrics such as sleep quality and emotional states. In this context, ``channels" refer to the distinct types of data streams collected by various sensors, each providing unique insights into physiological and environmental conditions.

By conducting this research, deriving these results using sensors that can be encountered in everyday life is thought to contribute significantly to human understanding in the future.

\renewcommand{\arraystretch}{1.5}
\setlength{\tabcolsep}{10pt}


\section{Methodology}
\label{Section:Methodology}

\begin{table}[h]
\centering
\caption{Overview of Smartphone and Smartwatch Sensor Data. The term 'ext.' refers to sensors used in extension experiments, while 'org.' refers to sensors used during the original model training.}
\label{tab:sensor_data}
\small
\resizebox{\columnwidth}{!}{%
\begin{tabular}{@{}l l c c c c c c@{}}
\toprule
\multirow{2}{*}{\textbf{Device}} & \multirow{2}{*}{\textbf{Sensor}} & \multirow{2}{*}{\textbf{Feature}} & \multicolumn{4}{c}{\textbf{Channels}} & \multirow{2}{*}{\textbf{Freq. (Hz)}} \\
\cmidrule(r){4-7}
 &  &  & \textbf{5} & \textbf{7} & \textbf{11} & \textbf{18} & \\
\midrule
\multirow{9}[+12]{*}{\centering\rotatebox[origin=c]{90}{\textbf{Smartphone}}}
 & \multirow{3}{*}{\makecell[l]{Accele-\\ration}} & x & \checkmark & \checkmark & \checkmark & \checkmark & \multirow{3}{*}{50} \\
\cmidrule{3-7}
 &  & y & \checkmark & \checkmark & \checkmark & \checkmark & \\
\cmidrule{3-7}
 &  & z & \checkmark & \checkmark & \checkmark & \checkmark & \\
\cmidrule{2-8}
 & Activity & activity & \checkmark & \checkmark & \checkmark & \checkmark & 1/60 \\
\cmidrule{2-8}
 & \multirow{4}{*}{\makecell[l]{GPS\\coord.}} & altitude &  &  & \checkmark & \checkmark & \multirow{4}{*}{1/5} \\
\cmidrule{3-7}
 &  & latitude &  &  & \checkmark & \checkmark & \\
\cmidrule{3-7}
 &  & longitude &  &  & \checkmark & \checkmark & \\
\cmidrule{3-7}
 &  & speed &  & \checkmark & \checkmark & \checkmark & \\
\cmidrule{2-8}
 & Ambient & ambience &  & \checkmark & ext. & \checkmark & 1/120 \\
\midrule
\multirow{9}[+12]{*}{\centering\rotatebox[origin=c]{90}{\textbf{Smartwatch}}} 
 & Light & m\_light &  &  & org. & \checkmark & 1/600 \\
\cmidrule{2-8}
 & \multirow{7}{*}{\makecell[l]{Step\\counts}} & burned\_cal. &  &  &  & \checkmark & \multirow{7}{*}{1/60} \\
\cmidrule{3-7}
 &  & steps &  &  & \checkmark & \checkmark & \\
\cmidrule{3-7}
 &  & distance &  &  &  & \checkmark & \\
\cmidrule{3-7}
 &  & running\_steps &  &  &  & \checkmark & \\
\cmidrule{3-7}
 &  & speed &  &  &  & \checkmark & \\
\cmidrule{3-7}
 &  & step\_freq. &  &  &  & \checkmark & \\
\cmidrule{3-7}
 &  & walking\_steps &  &  &  & \checkmark & \\
\cmidrule{2-8}
 & Heart rate & heart\_rate & \checkmark & \checkmark & \checkmark & \checkmark & 1/60 \\
\bottomrule
\end{tabular}
}
\end{table}

\subsection{Datasets}

The "Real-world Multimodal Lifelog Dataset" encompasses over 10,000 hours of data collected from Android smartphones and Empatica E4 smartwatch devices in 2020 \cite{chung2022real}.
%
Similarly, the 2023 data collection employed the same methodology as the 2020 study, with eight patients' data being gathered using Android smartphones and either the Galaxy Watch4 or Watch5 smartwatch device \cite{oh2024human}.
%
The 2023 collected dataset, comprising a total of 220 days from the lifelog data of 8 patients, is considered due to the differences between smartwatch devices. 
%
However, a significant limitation of this dataset is its relatively small size, with only 4 patients used for training and 4 patients for testing. The limited sample size increases the risk of overfitting, which restricts the generalizability of the findings and poses challenges in achieving robust model performance. 
%
The objective is to predict seven overall life quality labels (sleep quality, emotional state, stress strength) and sleep stats (total sleep time, sleep efficiency, SL, WASO) scores using F1 metrics. The detailed evaluation metrics are described by Oh et al \cite{oh2024human}.


\subsection{Analysis of Selected Sensor Signals}
\label{Section II, B}
5 or 11 of the 18 signals in the dataset were employed for model training, respectively. The 5 channels composite image incorporated the three-axis acceleration, activity, and heart rate data obtained from the smartphone and smartwatch, while the 11 channels composite image augmented the dataset with GPS coordinates, light intensity, and step count information.
%
Detailed signal information is described in TABLE \ref{tab:sensor_data}. We selectively utilized these signals for training based on the following criteria. (1) Three-axis acceleration: Measured continuously for 24 hours provides a rough estimation of movement information during the day and night, and is often employed in sleep studies \cite{yoshihi2021estimating}. The acceleration signal measured by the smartphone has a frequency of 50 Hz, which allows for the indirect determination of the sleep onset time and wake-up time.
%
(2) Activity: Behavioral classification is recorded once per minute (1/60 Hz), indicating whether the user is in motion. This classification can be combined with acceleration data to enhance the accuracy of activity detection.
%
(3) GPS Coordinates: These include altitude, latitude, longitude, and speed values, collected at a frequency of 1/5 Hz, which facilitates the analysis of the movement and bedtime location of users. (4) Light intensity: The state of the lights can be used to infer whether the user is sleeping. (5) Steps: Reflects the activity level, potentially impacting the total sleep time. (6) Heart rate: Indicates various body states, which decrease during sleep compared to wakefulness, and it is well-established that severe anxiety and stress cause an increase in heart rate.

Signals collected in real-life environments, where factors like room brightness and noise are not well controlled, tend to be noisy, leading to lower prediction accuracy. Therefore, to obtain more accurate results, we selectively used only signals highly related to sleep or stress.

\begin{figure}[ht]
\centering
\includegraphics[width=0.5\textwidth]{figures/dataset_sample2.png}
\caption{Sample image of the synchronized 11 channels dataset mentioned in TABLE \ref{tab:sensor_data}.
The x-axis consists of 86,400 data points, representing 1-second intervals, showing synchronized data.
}
\label{fig:data}
\end{figure}


\subsection{Data Pre-processing}
\label{Section:Data pre-processing}
It should be noted that the provided dataset does not contain data for all hours of the day. Consequently, there may be instances when data is absent for a given duration, which varies by sensor. To illustrate, the heart rate channel is unavailable from midnight until morning in Fig.\ref{fig:data}. In addition, the frequency of data collection (Hz) differs from sensor to sensor, as well as the time of day when the data was collected. Therefore, a synchronization process is required to ensure the effective utilization of multiple sensors. Our proposed PixleepFlow approach to this issue is as follows:

\subsubsection{Frequency resampling}
Due to the varying collection frequencies of different sensors, it was essential to normalize these frequencies to achieve synchronization. We opted to aggregate data on a per-second basis (1Hz), aligning the nearest data point to each second.


\subsubsection{Signal Synchronization}
The data were then synchronized into a single dataset spanning from 00:00:00 to 23:59:59, representing a total of 86,400 seconds (24 hours). Every second was recorded, and the synchronized data can be transformed into various data types. 

\subsubsection{Missing values interpolation} 
\label{subsection2:missing_values}
For data with missing values, we needed a method of interpolation. We used linear interpolation. Moreover, we leave the data as it is without interpolating where there is no data at the beginning and end of the data. This is to avoid generating too much spurious data.


\subsection{Various Transformations of Signal Data}
% As previously stated in Section \ref{Section:Methodology}, an inherent drawback is the need for interpolation when configuring the matrix to integrate various types of raw data.
Following the preprocessing methodologies stated in Section \ref{Section:Methodology}, the following approaches illustrate different ways to transform and utilize the signal data:

\subsubsection{Using Raw signal data}
The method of leveraging the raw signal data is to utilize the value vectors captured by the sensor in their original form. By inputting each time series dataset into the model in its unprocessed state and conducting analysis, it is possible to preserve the integrity of the original data to the greatest extent. This approach allows for the direct utilization of raw data, facilitating the discovery of correlations and temporal changes between sensors without requiring additional processing.


\subsubsection{Using Spectrogram data}
A spectrogram is a visual representation of the frequency changes of a signal over time, displaying the signal's spectrum as it varies along the time axis. This transformation permits a more precise examination of the data's characteristics by transforming each sensor's data into a spectrogram, which captures both the temporal and frequency characteristics, relying on the raw time-series data. Similar to \cite{neshitov2021wavelet}, we employed the Short-Time Fourier Transform (STFT) for this conversion.


\subsubsection{Using Imaging Data}
A methodology was employed to visualize raw time-series data by representing sensor values over time in a single image. Converting data into images adheres to the methodology delineated in Section \ref{Section:Data pre-processing}. Using the subplot function in Matplotlib, the visualizations for each sensor were stacked vertically. This approach comprehensively captures the temporal variations of each sensor, and image processing techniques were applied to these visual representations for model training.
%
The data were transformed into diverse image formats using both 5 channels and 11 channels data, with the 5 channels image data demonstrating the most optimal performance. This approach presents a novel method for effectively analyzing and processing complex time-series data, with the transformed data depicted in Fig \ref{fig:visualization_channel}.



\begin{figure}[ht]
\centering
\begin{minipage}{0.22\textwidth}
    \centering
    \includegraphics[width=\textwidth]{figures/5channel_image.png}
    \caption*{(a) 5-channel Image}
\end{minipage}\hfill
\begin{minipage}{0.22\textwidth}
    \centering
    \includegraphics[width=\textwidth]{figures/all_channel_image.png}
    \caption*{(b) 11-channel Image}
\end{minipage}\hfill
\caption{Data Visualization of 5 and 11 channels}
\label{fig:visualization_channel}
\end{figure}

\section{EXPERIMENTS}

\subsection{Experimental Setup}
To optimize the model, we used the AdamW optimizer and set the weight decay to 0.1. To build a model that is robust to different user data distributions, we used the K-fold ensemble technique with random sampling, with 200 epochs of training for each fold. For each fold, we selected the model with the highest score, and the final score was calculated by averaging these best scores across all folds. The macro F1-score was used as the model storage criterion. Each class has a binary label, and Binary Cross Entropy Loss was used. The initial learning rate was set to 5e-5, and the cosine annealing scheduler was set to T=200 to progressively decrease the learning rate throughout training. All implementations were done using the PyTorch library, and training was performed on an NVIDIA H100 GPU.


\subsection{Experimental Approach}
We explored the efficacy of various view experiments by converting data in diverse formats, extending beyond the conventional approach of simple image conversion techniques. To assess the model's performance, we utilized a range of data types, including raw sensor data, composite images with various channels, and spectrograms which are mentioned in Section \ref{Section:Methodology}.



\subsubsection{Raw-based}
In the raw signal data format, the signal is embedded using 1D-CNN layers with the ResNet mechanism applied. Multiple synchronized channels of sensor values, which have been transformed as described in Section \ref{Section:Data pre-processing}, are inputted in their original form, allowing the model to learn directly from the intuitive data without any additional transformations. This approach not only preserves the characteristics of the original data but also maintains temporal continuity, enabling the model to effectively capture time-series patterns. Additionally, by handling 1D signals, the model learns clear patterns such as specific frequency components or temporal sequences, all while utilizing a relatively simple model structure that requires less computational resources.


\subsubsection{Image-based (PixleepFlow)}
For PixleepFlow, we used the SEResNeXt101\_32x4d and ResNeXt101\_32x32d \cite{xie2017aggregated,hu2018squeeze} models, which were pre-trained on the ImageNet 21k dataset. 

ResNeXt is a highly modularized network architecture that employs a cardinality dimension, allowing for increased model capacity and diversity without significantly increasing computational complexity. It builds upon the ResNet architecture by utilizing grouped convolutions, which enable multiple transformations to be processed in parallel, leading to enhanced feature learning capabilities \cite{xie2017aggregated}.

SE-ResNeXt extends the ResNeXt architecture by integrating Squeeze-and-Excitation (SE) blocks, which adaptively recalibrate channel-wise feature responses. These SE blocks enhance the network's sensitivity to informative features by explicitly modeling the interdependencies between channels \cite{hu2018squeeze}. The combination of ResNeXt's cardinality and SE blocks results in a powerful and efficient architecture capable of capturing complex patterns in the data.

\subsubsection{Spectrogram-based}
Each sensor data converted into a spectrogram is individually extracted while passing through ResNet-based 1D-CNN layers. Features extracted from each sensor channel are combined into one integrated feature vector through the concatenate process. These combined feature vectors are input to the final classifier model to execute classification. Through this process, more accurate behavioral pattern recognition is possible by using temporal-frequency information of each sensor data.

\begin{figure}[ht]
\centering
\begin{minipage}{0.155\textwidth}
    \centering
    \includegraphics[width=\textwidth]{figures/xai_image1.png}
\end{minipage}\hfill
\begin{minipage}{0.155\textwidth}
    \centering
    \includegraphics[width=\textwidth]{figures/xai_image3.png}
\end{minipage}\hfill
\begin{minipage}{0.155\textwidth}
    \centering
    \includegraphics[width=\textwidth]{figures/xai_image4.png}
\end{minipage}
\caption{Full-CAM visualization of XAI (Explainable Artificial Intelligence) highlighting the features utilized for sleep-related activity detection.} 
\label{fig:xai}
\end{figure}

\subsection{Model Interpretation}
PixleepFlow employs Explainable Artificial Intelligence (XAI) methodologies to achieve optimal performance and elucidate the rationale behind the model's predictions in a manner that is intelligible to humans. XAI is a technique that renders the output and decision-making process of machine learning models transparent, thereby facilitating user comprehension and trust in the model's behavior. In this study, we employed the Full-CAM (Full-Gradient Representation for Neural Network Visualization) technique \cite{srinivas2019full}. Full-CAM offers a visual representation of the impact of each input image on a specific output class within the network, emphasizing the image regions essential for the model to make certain decisions. The results obtained via this technique are illustrated in Fig \ref{fig:xai}.

As illustrated in Fig. \ref{fig:xai} the Full-CAM reveals that the activity information is not a significant indicator. Conversely, accelerometer (ACC) channels, particularly the Y and Z axis sensors and heart rate were frequently utilized as pivotal features \cite{yoshihi2021estimating, radha2019sleep}. Additionally, to identify features related to sleep, the primary prediction category in this competition, models demonstrated a tendency to focus on movements occurring immediately before or following approximate time in bed (TIB). Even when the participant checked their smartphone during the sleep period, the models showed a tendency to prioritize these subtle movements, even though they were less pronounced compared to the noticeable movements observed during wakefulness as shown in the central image Fig. \ref{fig:xai}.

\subsection{Result}
\begin{table}[h]
\centering
\caption{Model Performance Based on Channel and Input Type}
\label{tab:model_performance}
\small
\resizebox{\columnwidth}{!}{
\begin{tabular}{@{}l c c c@{}}
\toprule
\multirow{2}{*}{\textbf{Channel}} & \multicolumn{3}{c}{\textbf{Input Type}} \\
\cmidrule(r){2-4}
 & \textbf{Image-based (pixleep)} & \textbf{Raw-based} & \textbf{Spectrogram-based} \\
\midrule
\textbf{5} & 0.680 & 0.636 & 0.613 \\
\textbf{11} & 0.746 & 0.699 & 0.637 \\
\bottomrule
\end{tabular}
}
\end{table}




\begin{table}[h]
\centering
\caption{Model Performance Based on Channel and Input Type. The meaning of labels Q1 through S4 can be referenced in Fig \ref{fig:PixleepFlow}.}
\label{tab:pixleep_metrics}
\small
\resizebox{\columnwidth}{!}{%
\begin{tabular}{@{}lccccccc@{}}
\toprule
\multirow{2}{*}{\textbf{Channel}} & \multicolumn{7}{c}{\textbf{Image-based (Pixleep)}} \\ 
\cmidrule(r){2-8}
 & \textbf{Q1} & \textbf{Q2} & \textbf{Q3} & \textbf{S1} & \textbf{S2} & \textbf{S3} & \textbf{S4} \\ 
\midrule
\textbf{5}  & 0.762 & 0.800 & 0.308 & 0.615 & 0.700 & 0.857 & 0.714 \\ 
\textbf{11} & 0.706 & 0.804 & 0.767 & 0.593 & 0.751 & 0.855 & 0.749 \\ 
\bottomrule
\end{tabular}%
}
\end{table}

In this study, we undertake a comparative analysis of the performance of data from 5 and 11 channels for the Image (PixleepFlow), raw, and spectrogram modalities. The results of the experiment are presented in TABLE \ref{tab:model_performance}. The image-based  (PixleepFlow) demonstrated the most optimal performance in both channel configurations, attaining the highest F1-score of 0.746 on 11 channels of data, which is notably higher than 0.680 on 5 channels of data. The raw-based method yielded an F1-score of 0.636 for the 5 channels model and 0.699 for the 11 channels model. In comparison, the spectrogram-based approach exhibited the lowest performance of the three methods, with an F1-score of 0.613 for the 5 channels model and 0.637 for the 11 channels model. 
%
These results indicate that our proposed PixleepFlow approach not only achieves superior performance with 11 channels but also demonstrates comparable effectiveness with just 5 channels, achieving a performance level close to that of the raw-based method with 11 channels.

As seen in TABLE \ref{tab:pixleep_metrics}, the improvement in stress level (Q3) prediction observed in the 11 channels configuration, is likely due to the additional sensor data providing more informative cues. While the F1-score for Q3 was very low in the 5 channels setup (0.308), it significantly improved in the 11 channels setup (0.767). This suggests that diverse sensor data play a crucial role in accurately predicting stress level. However, the F1-score for S1 (sleep duration) remains low (11 channels: 0.593), indicating that sleep duration is a complex feature that may be challenging to capture fully with single image data or individual sensor inputs.

Thus, PixleepFlow demonstrates strong performance in predicting certain labels, such as emotional state (Q2: 0.804) and sleep onset latency (S3: 0.855), but the prediction of stress level (Q3: 0.767) and sleep duration (S1: 0.593) shows significant variance depending on the channel configuration. This indicates that integrating various sensor data could potentially enhance predictive performance.

Overall, as the number of channels increased, the performance tended to improve across all input types. It is interpreted that more channels contributed to increasing the accuracy of the classification task by providing richer information to the model. 

\section{Extended Comparative Analysis}
\label{Section:Extended Comparative Analysis}
In this experiment, we conducted independent tests separate from the competition to achieve more precise results. We examined the differences in performance based on various optimizers, schedulers, and the number of channels (5, 7, 11, and 18 channels, as seen in TABLE \ref{tab:sensor_data}), as well as the impact of different frequencies (1 Hz and 1/60 Hz).

\subsection{Extended Experimental Setup}
The experimental setting is analogous to the existing model, yet it exhibits several discernible distinctions. The model is fixed with SERexNeXt101\_32x4d and uses a model pre-train with the ImageNet 21k dataset. Furthermore, if the loss was 0.8 or higher in epoch 100 or beyond, early stopping was conducted beyond the reference value, which may indicate an increased risk of overfitting.

\subsubsection{SAM optimizer}
One method for mitigating overfitting due to the limited dataset is to utilize a SAM optimizer \cite{foret2020sharpness}. The SAM optimizer, which stands for Sharpness-Aware Minimization, has been demonstrated to enhance a model's generalization performance. By flattening the surface of the model's loss function, SAM optimizers facilitate improved generalization to novel data, reducing the probability of overfitting to a specific dataset.

\subsubsection{SGDR Scheduler}
Another generalization method employed is the Stochastic Gradient Descent with Warm Restarts (SGDR) scheduler. The SGDR method involves the periodic resetting of the learning rate, to achieve enhanced optimization. 

\subsubsection{Add Ambience Data}
For channels 7, 11, and 18, we incorporated the most recent ambient sound data. This dataset includes the top 10 probability values for various sounds detected by the smartphone, such as "breathing," "snoring," and "snorting," all of which are closely associated with sleep. Ambient sound data typically showed higher values during sleep periods, which may enhance the model's ability to capture relevant sleep-related information, as demonstrated in Fig. \ref{fig:data}. We hypothesize that integrating ambient sound data can lead to more accurate predictions of sleep efficiency. For a comprehensive breakdown of the utilized sensor data, please refer to TABLE \ref{tab:sensor_data}.

\subsubsection{Frequency test}
In the preceding experiment, a total of 86,400 data points were recorded at a rate of 1 Hz. To confirm the impact of Hz on performance, an additional experiment was conducted, in which the data were converted to units per minute (1/60 Hz). While this may result in the aggregation of subtle differences in the data, it may also mitigate the phenomenon of the model overfitting to fine features.

\subsection{Experiment}
\begin{table}[h]
\centering
\caption{Model Performance Based on Channel and Frequency}
\label{tab:extend_result}
\small
\resizebox{\columnwidth}{!}{%
\begin{tabular}{@{}l l c c c c c c@{}}
\toprule
\multirow{2}{*}{\textbf{Metric}} & \multirow{2}{*}{\textbf{Frequency}} & \multicolumn{4}{c}{\textbf{Channels}} \\
\cmidrule(r){3-6}
 &  & \textbf{5} & \textbf{7} & \textbf{11} & \textbf{18} \\
\midrule
\multirow{2}{*}{\textbf{Image-based (Pixleep)}} & \textbf{1/60Hz} & 0.612 & 0.629 & 0.635 & 0.638 \\
\cmidrule(r){2-6}
 & \textbf{1Hz} & 0.659 & 0.521 & 0.634 & 0.632 \\
\bottomrule
\end{tabular}
}
\end{table}

The mean score of our K-fold (k=5) indicated that the 5 channels data at 1 Hz yielded the optimal result, with a score of 0.659. This is the same sensor utilized in the initial submission. Subsequently, the 11, 18, and 7 channels at 1 Hz yielded scores of 0.634, 0.632, and 0.521, respectively. Additionally, the 18 channels, 11 channels, 7 channels, and 5 channels at 1/60 Hz yielded scores of 0.638, 0.635, 0.629, and 0.612, respectively. The experimental results are summarized in TABLE \ref{tab:extend_result}.

In general, the most stable frequency was 1/60 Hz, with the highest value being the 5 channels data at 1 Hz. In light of the interchannel interactions, it is probable that the 5 channels dataset comprised solely the indispensable sensor information and exhibited optimal interaction. The higher frequency of 1 Hz provides more data, enabling fine-grained pattern learning; however, including an excessive number of channels can impede the model's capacity for generalization. The data collected at 1/60 Hz was less complex, allowing the model to learn with greater reliability, particularly in the case of the 18 channels data, which may be attributed to the effective integration of information from all channels. Additionally, certain sensors, such as the accelerometer and heart rate, played a pivotal role, and the 5 channels data may have demonstrated superior performance due to its effective inclusion of information from these crucial sensors \cite{radha2019sleep}.

\section{Conclusion}
In conclusion, this study presented a comprehensive approach to predicting sleep quality and stress level through the innovative Pixel Sleep Flow Framework (PixleepFlow). By converting multimodal lifelog sensor data into composite images, PixleepFlow enhances the analysis of sleep patterns and their impacts on health. This methodology not only improves the accuracy of predicting key metrics such as total sleep time, sleep efficiency, and stress level but also provides a visually interpretable model by leveraging Explainable Artificial Intelligence.

The findings underscore the significance of synchronization and transformation of sensor data, highlighting the efficacy of the image conversion approach in capturing the multifaceted nature of human behavior. The results from the Full-CAM visualization confirmed the importance of specific sensor channels, such as accelerometer and heart rate, in predicting sleep and stress-related metrics.
%
Additionally, by leveraging the XAI technique, we were able to mitigate concerns of overfitting due to the limited dataset size and ensure that the model was learning effectively from the available data. 

Future research can build upon this work by exploring additional sensor combinations and refining the image conversion techniques to improve prediction accuracy. The application of PixleepFlow in real-world settings holds the potential for personalized health monitoring and promoting overall well-being through more accurate and interpretable assessments of sleep and stress.

\printbibliography



% \bibliographystyle{IEEEtran}  % IEEEtran 스타일 사용
% \nocite{*}  % 강제로 모든 참고문헌 출력

% \addbibresource{references.bib} % Add the references.bib file
% \bibliography{conference_101719}  % .bbl 없이 .bib 파일명을 입력
%\bibliography{conference_101719}  % .bbl 없이 .bib 파일명만 입력

\vspace{12pt}

\end{document}
