% \subsection{Model Selection - Timm}
% \textcolor{red}{
% 여기 timm 그 사진으로 넣은 내용인데 안 쓰면 싹 바꿔야 되려나...
% 본 연구에서는 5개의 시그널을 통합한 하나의 이미지 데이터를 효과적으로 처리할 수 있는 딥러닝 모델을 선정하는 것이 중요했다. 이러한 복합적인 데이터 구조는 특성을 추출하는데 있어서 정교한 데이터 처리 기능이 필요했다. 데이터는 5의 다양한 시그널에서 파생된 특성들을 통합하여 하나의 이미지로 구성되며, 각 시그널은 다양한 정보의 복잡성을 이미지에 담아낸다. 이를 효과적으로 분석하기 위해서는 강력한 특성 인식 능력이 필수적이다. 이러한 요구를 충족시킬 수 있는 모델로 `seresnext101\_32x4d`를 선택했다. 선택 과정에서는 Timm library models들의 Gflops와 Loss를 중요한 평가 지표로 고려하였다. 3) Justification and conclusion of the decision 여기 내용까지..}

% \subsubsection{Model Selection Criteria and Processes}
% 본 연구의 모델 평가는 다양한 딥러닝 아키텍처의 성능을 비교 분석하는 것에서 시작되었다. Gflops 대비 Loss와 Accuracy를 측정하여 각 모델의 계산 효율성과 성능을 평가하였다. Fig \ref{fig:timm_model}에 따르면, 여러 모델 중 `seresnext101\_32x4d`는 높은 accuracy를 유지하면서도 상대적으로 낮은 Loss를 보여주었다. 이는 복잡한 이미지 데이터를 효과적으로 처리할 수 있는 능력을 의미한다.

% \subsection{Model Selection - Timm}
% In this study, selecting a deep learning model capable of effectively processing an image data integrated from 11 signals was critical. This complex data structure required sophisticated data processing capabilities for feature extraction. The data comprised features derived from 11 distinct signals, each contributing to the complexity of the information encapsulated in a single image. Effective analysis of this data necessitated a model with robust feature recognition capabilities. The `seresnext101\_32x4d` was chosen as the model that meets these requirements. During the selection process, GFLOPS and Loss metrics from Timm library models were considered as important evaluation criteria.

% \subsubsection{Model Selection Criteria and Processes}
% The model evaluation in this study began with a comparative analysis of various deep learning architectures. Each model's computational efficiency and performance were assessed by measuring Loss and Accuracy relative to GFLOPS. According to Figure (2), among the various models, `seresnext101\_32x4d` demonstrated high accuracy while maintaining a relatively low Loss. This indicates its capability to effectively process complex image data.


% \subsubsection{Technical Basis and Performance Comparison}
% The `seresnext101\_32x4d` model utilizes Squeeze-and-Excitation (SE) blocks to recalibrate the importance of each channel, enabling more refined feature extraction. This functionality highlights critical information and suppresses noise in multidimensional data. As shown in Figure (2), this model demonstrates a tendency to maintain accuracy while minimizing loss as GFLOPS increase, thus proving its computational efficiency and performance excellence.


% \subsubsection{Justification and conclusion of the decision}
% The selection of `seresnext101\_32x4d` was a decision made to effectively handle high-dimensional, complex datasets and to maximize the accuracy and efficiency of the research. This model provides a reliable foundation for integrative analysis of complex signal data. The choice of model in this study was based on thorough comparisons and performance analyses, with `seresnext101\_32x4d` emerging as the optimal choice as a result.
