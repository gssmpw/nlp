 \section{Bounded doubling dimension}

In this section we prove \Cref{thm:equivalences}. We start with the following lemma that encapsulates the key observation: in the presence of a bound on the doubling dimension, the distance-$r$ balanced separators in $G$ can be translated into balanced separators in the corresponding distance graph.

 \begin{lemma}\label{lem:bdtw}
     Let $G$ be a graph with doubling dimension at most $m$, for some $m\in \N$. Suppose that for some $r\in \N_{>0}$, the graph $G$ has distance-$r$ balanced separator number at most $k$. Then, for every inclusion-wise maximal distance-$r$ independent set $I$ in $G$, the distance graph $H(G,I,r)$ has treewidth at most  $3k\cdot 2^{6m}$.
 \end{lemma}
\begin{proof}
    Denote $H=H(G,I,r)$ for brevity. By \cref{lem:bsn}, it suffices to prove that the balanced separator number of $H$ is bounded by $k\cdot 2^{6m}$. 
    Consider any weight function $\mu_H\colon V(H)\to \R_{\geq 0}$ on $H$. Let $\mu_G\colon V(G) \to \R_{\geq 0}$ be the weight function on $G$ defined by setting $\mu_G(u)=\mu_H(u)$ for all $u\in I$, and $\mu_G(u)=0$ for all $u\notin I$. Since $G$ has distance-$r$ balanced separator number at most $k$, there exist radius-$r$ balls $B_1,B_2,\dots, B_{k}$ in $G$ such that denoting $X\coloneqq \bigcup_{i=1}^{k} B_i$, every connected component of $G-X$ has weight at most~$\frac{1}{2}\mu_G(V(G))$.

    Let $u_1, u_2, \dots, u_{k}$ be the centers of balls $B_1,B_2,\ldots,B_k$, respectively. Since $I$ is an inclusion-wise maximal distance-$r$ independent set, we can find vertices $v_1,\ldots,v_r\in I$ such that $\dist_G(u_i,v_i)\leq r$, for each $i\in \{1,\ldots,k\}$. Further, let $N_H^2[v_i]$ be the second neighborhood of $v_i$ in $H$: the set consisting of $v_i$, the neighbors of $v_i$, and the neighbors of those neighbors.
    We define
    \[X_H\coloneqq \bigcup_{i=1}^k N_H^2[v_i]\subseteq V(H)=I.\]
    By \cref{clm:bdddim_implies_bddegree}, $\Delta(H)<2^{3m}$. It follows that \[|N_H^2[v_i]|\leq 1+\Delta(H)+\Delta(H)\cdot (\Delta(H)-1)=1+(\Delta(H))^2\leq 2^{6m},\quad\textrm{for each }i\in \{1,\ldots,k\}.\]
    Therefore, we have $|X_H|\leq k\cdot 2^{6m}$.
    
    We will show that $X_H$ is a balanced separator for~$\mu_H$.
    Note that since $X$ is a balanced separator for $\mu_G$, it suffices to prove the following claim: every two vertices $u,v\in I\setminus X_H$ that are in different components of $G-X$, are also in different components of~$H-X_H$.

    Consider any path $Q$ connecting $u$ and $v$ in $H$.
    Let $P$ be a walk connecting $u$ and $v$ in $G$ obtained by replacing every edge of $Q$, say $xy$, with a path $R_{xy}$ in $G$ of length at most $3r$. Such a path exists by the construction of $H$ as it is the $(I,r)$-distance graph of $G$.
    Since $u$ and $v$ are in different components of $G-X$, the walk $P$ must intersect $X$. This means that for some edge $xy$ of $Q$ and $i\in \{1,\ldots,k\}$, there exists a vertex $w\in V(R_{xy})$ such that $\dist_G(w,u_i)\leq r$, implying that $\dist_G(w,v_i)\leq 2r$.
    Since the length of $R_{xy}$ is at most $3r$, we have $\dist_G(w,x)\leq \frac{3}{2}r$ or $\dist_G(w,y)\leq \frac{3}{2}r$, implying that $\dist_G(v_i,x)\leq \frac{7}{2}r$ or $\dist_G(v_i,y)\leq \frac{7}{2}r$. Without loss of generality assume the former. Then on a shortest path between $x$ and $v_i$ we may find a vertex $z$ such that $\dist_G(x,z)\leq 2r$ and $\dist_G(v_i,z)\leq 2r$. Since $I$ is an inclusion-wise maximal distance-$r$ independent set, there exists some $z'\in I$ such that $\dist_G(z,z')\leq r$. In particular we have $\dist_G(x,z')\leq 3r$ and $\dist(z',v_i)\leq 3r$, which means that $x,z'$ are equal or adjacent in $H$, and also $z',v_i$ are equal or adjacent in $H$. We conclude that $x\in N_H^2[v_i]$, so $x\in X_H$ and therefore $Q$ intersects $X_H$. As $Q$ was chosen arbitrarily, we conclude that $u$ and $v$ lie in different components of $H-X_H$ and we are~done. 
\end{proof}
    
    
    
%    By triangle inequality we have $B_i\subseteq \B_G(v_i,2r)$, hence
%    \[\bigcup_{i=1}^{k} B_i \subseteq \bigcup_{i=1}^{k} \B_{G}(v_i, 2r).\] 
%    Since So each component of $G - \bigcup_{i=1}^{k_4} \B_{(X,d)}(v_i, 2r)$ has weight at most $\frac{\mu(X)}{2}$. Therefore each component of $H-\bigcup_{i=1}^{k_4}\B_{(V(H), d_H)}(v_i, 1)$ has weight at most $\frac{\mu(V(H))}{2}$ (recall that $z_1z_2\notin E(H)$ implies $d(z_1, z_2) > 3r$). Thus $\bigcup_{i=1}^{k_4} \B_{(V(H, d_H)}(v_i, 1)$ is a balanced separator for $\mu_H$ of $H$ of size $k_4 \cdot 2^m$, as each vertex of $H$ has degree bounded by $2^m$ by Claim~\ref{clm:bdddim_implies_bddegree}. 

%    We showed that for each weight function $\mu_H\colon V(H)\to \mathbb{R}_{\geq 0}$ there exists a balanced separator of $H$ of size $2^mk_4$. Thus $H$ has treewidth equal $O(k_42^m)$.



%\begin{theorem}\label{thm:equivalences}
%    Let $\Cc$ be a class of graphs of doubling dimension bounded by $m$, for some $m\in \N$. Then the following conditions are equivalent for every $r\in \N_{>0}$:
%    \begin{enumerate}[label=(\arabic*),ref=(\arabic*)]
%        \item\label{pr:tpw} There exist $k_1,\Delta_1\in \N$ such that every member of $\Cc$ has a tree-partition of spread $r$, maximum degree at most $\Delta_1$, and with $(k_1,r)$-coverable bags.
%        \item\label{pr:tw} There exists $k_2\in \N$ such that every member of $\Cc$ has a tree decomposition with $(k_2,r)$-coverable~bags.
 %       \item\label{pr:bsn} There exists $k_3\in \N$ such that every member of $\Cc$ has distance-$r$ balanced separator number at most $k_3$.
 %       \item\label{pr:qi} There exist $k_4,\Delta_4\in \N$ such that every member of $\Cc$ is $(3,3r)$-quasi-isometric with an edge-weighted graph of tree-partition width at most $k_4$, maximum degree at most $\Delta_4$, and every edge of weight $3r$.
 %       \item\label{pr:qia} There exist $k_5,\Delta_5\in \N$ and $\alpha,\beta,\gamma\in \R_{>0}$ such that every member of $\Cc$ is $(\alpha, \beta r)$-quasi-isometric with an edge-weighted graph of tree-partition width at most $k_5$, maximum degree at most $\Delta_5$, and every edge of weight at least $\gamma r$.
 %   \end{enumerate}
%\end{theorem}
\thmequivalences*
\begin{proof}
    \noindent{\bf{\ref{pr:tpw}$\Rightarrow$ \ref{pr:tw}.}} We use the standard construction sketched in \cref{sec:prelim}. Consider any $G\in \Cc$ and let $(T,\bag)$ be a tree-partition of $G$ with $(k_1,r)$-coverable bags (we will not use the assumption about the spread or the maximum degree). We construct a new tree $T'$ by subdividing every edge of $T$, say $xy$, with a new vertex~$z_{xy}$. On $T'$ we define a new bag function $\bag'$ as follows: $\bag'(x)=\bag(x)$ for each $x\in V(T)$ and $\bag'(z_{xy})=\bag(x)\cup \bag(y)$ for each $xy\in E(T)$. It is straightforward to see that $(T',\bag')$ is a tree decomposition of $G$, and its bag are $(k_2,r)$-coverable, where $k_2\coloneqq 2k_1$.

    \bigskip

    \noindent{\bf{\ref{pr:tw}$\Rightarrow$ \ref{pr:bsn}.}}
     Again, we use the standard argument of finding a balanced bag in a tree decomposition. Consider a graph $G\in \Cc$ and let
     $(T,\bag)$ be a tree decomposition of $G$ with $(k_2,r)$-coverable bags.  
    Consider also any weight function $\mu\colon V(G)\to \R_{\geq 0}$. We orient the edges of $T$ according to $\mu$ as follows. Consider an edge $xy$ of $T$; then removal of $xy$ from $T$ disconnects $T$ into two subtrees, say $T_x$ containing $x$ and $T_y$ containing $y$. If $\mu(\bigcup_{x'\in V(T_x)} \bag(x)) < \mu(\bigcup_{y'\in T_y} \bag(y))$, then orient $xy$ towards~$y$, and if $\mu(\bigcup_{x'\in V(T_x)} \bag(x)) > \mu(\bigcup_{y'\in T_y} \bag(y))$, then orient $xy$ towards $x$; in case of a tie, orient $xy$ in any way.
    Since every orientation of a tree has a sink, there exists a node $x\in V(T)$ such that all edges incident to $x$ are oriented towards $x$.
    Then each component of $G-\bag(x)$ has $\mu$-weight at most~$\frac{1}{2}\mu(V(G))$, as otherwise the edge connecting $x$ with the tree of $T-x$ containing this component would need to be oriented outwards from $x$. Since $\bag(x)$ is $(k_2,r)$-coverable, we conclude that $\bag(x)$ is a $(k_2,r)$-coverable balanced separator for $\mu$. And as $\mu$ was chosen arbitrarily, $G$ has distance-$r$ separation number bounded by $k_3\coloneqq  k_2$.

    \bigskip

    \noindent{\bf{\ref{pr:bsn}$\Rightarrow$ \ref{pr:qi}.}}
    Consider a graph $G\in \Cc$.
    Let $I$ be an inclusion-maximal distance-$r$ independent set in $G$, and let $H\coloneqq H(G,I,r)$ be the $(I,r)$-distance graph of $G$. Then $G$ is $(3,3r)$-quasi-isometric with~$H$ (by \cref{lem:quasi_isometry_to_derived_graph}), and $H$ has maximum degree bounded by $\Delta_4\coloneqq 2^{3m}-1$ (by \cref{clm:bdddim_implies_bddegree}) and tree-partition-width bounded by $k_4\coloneqq \frac{35}{4}\Delta_4\cdot (3k_3\cdot 2^{6m}+1)$ (by \cref{thm:tw-tpw,lem:bdtw}).
    
    %there exists a $(3,3r)$-quasi-isometry $\varphi:X\to V(H)$ . We will show that there is a tree-partition of $H$ of width bounded by a function of $k_4$, $r$ and $m$.

    %By Lemma~\ref{lem:bdtw} $H$ has treewidth equal $O(2^mk)$.
    %Since $H$ has maximum degree bounded by $2^m$, there exists a tree-partition $\mathcal{T}=(T, \{\beta_x\}_{x\in V(T)})$ of $H$ of width $24(\tw(H)+1)\Delta(H)=O(k_4\cdot2^{m}) := k_1$ and maximum degree bounded by $\Delta_1 := 24(\tw(G) + 1)\Delta(G)^2 = O(k_4\cdot 2^{m+1})$.
%    \newline
%    \newline
%    $\mathbf{1\Rightarrow 2}$.
%    Let us just take $\alpha=\beta =3$, $k_2=k_1$ and $\Delta_2=\Delta_1$. Then the statement is clearly fulfilled.

    \bigskip

    \noindent{\bf{\ref{pr:qi}$\Rightarrow$ \ref{pr:qia}.}} Trivial, take $k_5\coloneqq k_4$, $\Delta_5\coloneqq \Delta_4$, $\alpha\coloneqq 3$, $\beta\coloneqq 3$, and $\gamma\coloneqq 3$.

    \bigskip
    
    \noindent{\bf{\ref{pr:qia}$\Rightarrow$ \ref{pr:tpw}.}} We may assume without loss of generality that $\alpha,\beta\geq 1$. Consider a graph $G\in \Cc$ and let 
    $H$ be an edge-weighted graph with $\tpw(H)\leq k_5$, $\Delta(H)\leq \Delta_5$, and every edge of weight at least $\gamma r$, such that there is an $(\alpha,\beta r)$-quasi-isometry $\varphi \colon V(G)  \to V(H)$.
    Let $\Tt=(T,\bag)$ be a tree-partition of $H$ of width at most $k_5$.
    Note that we may assume that $T$ has maximum degree at most $\Delta_5'\coloneqq k_5\cdot \Delta_5$, for the total number of neighbors of vertices contained in a single bag of $\Tt$ is at most $\Delta_5'$.

    Our goal is to construct a suitable tree-partition of $G$. We first define a new bag function $\bag'$ by naturally pulling $\bag$ through the quasi-isometry $\varphi$:
    \[\bag'(x)\coloneqq \{u\in V(G)\mid \varphi(u)\in \bag(x)\} \qquad \textrm{for each }x\in V(T).\]
        
    We claim that for each $x\in V(T)$, $\bag(x)$ is $(k_5',r)$-coverable, where $k_5'\coloneqq k_5\cdot 2^{m\lceil\log \alpha\beta\rceil}$. For this, observe that if we have any pair of vertices $u,v\in V(G)$ with $\varphi(u)=\varphi(v)$, then 
    \[\tfrac{1}{\alpha}\dist_G(u,v) - \beta r \leq \dist_H(\varphi(u), \varphi(v)) = 0,\qquad\textrm{implying}\qquad \dist_G(u, v)\leq \alpha\beta r.\]

    Therefore, for each $v\in \bag'(x)$ with non-empty $\varphi^{-1}(v)$ we can select any vertex $u_v\in \varphi^{-1}(v)$, and then $\varphi^{-1}(v)\subseteq \B_G(u_v, \alpha\beta r)$.
    Since $\bag(x)$ consists of at most $k_5$ elements, it follows that $\bag'(x)$ is $(k_5,\alpha\beta r)$-coverable. Since $G$ has doubling dimension at most $m$, each radius-$\alpha\beta r$ ball in $G$ can be covered by $2^{m\lceil\log \alpha\beta\rceil}$ radius-$r$ balls. It follows that $\bag'(x)$ is $(k_5',r)$-coverable, as claimed.

    It would be natural to consider $(T,\bag')$ as the sought tree-partition, but it is not clear that this is a tree-partition at all. Instead, we will construct a suitable ``coarsening'' of $(T,\bag')$ as follows; see also \cref{fig:tree}. Set
    \[p\coloneqq \lceil (\alpha+\beta)\gamma^{-1}\rceil.\]
    Root $T$ in an arbitrary node $z$ and let
    \[I\coloneqq \{y\in V(T)\mid \dist_T(y,z)\equiv 0 \bmod p\}.\]
    Here, the edges of $T$ are considered to be of unit length. Further, define
    \[C_z\coloneqq \{x\in V(T)\mid \dist_T(x,z)<2p\},\]
    and for each $y\in I\setminus \{z\}$, 
    \[C_y\coloneqq \{x\in V(T)\mid p\leq \dist_T(x,y)<2p\textrm{ and }\dist_T(x,z)>\dist_T(y,z)\}.\]
    Note that $\{C_y \mid  y \in I\}$ is a partition of $V(T)$. Since $T$ has maximum degree at most $\Delta_5'$, we have
    \begin{equation}\label{eq:wydra}|C_y|\leq 1+\Delta_5'+(\Delta_5')^2+\ldots+(\Delta_5')^{2p-1}\eqqcolon M\qquad\textrm{for each }y\in I.\end{equation}
    Next, define $T''$ to be the tree on the node set $I$ where $y,y'\in I$ are adjacent in $T''$ whenever \[\dist_T(y,y')=p\qquad\textrm{and}\qquad|\dist_T(y,z)-\dist_T(y',z)|=p.\]
    Then, from the construction we immediately obtain the following: 
    \begin{equation}\label{eq:bobr}
        \textrm{for all $x,x'\in V(T)$, if $x\in C_y$ and $x'\in C_{y'}$ with $y\neq y'$ and $yy'\notin E(T')$, then $\dist_T(x,x')>p$.}
    \end{equation}
    Note also that $T''$ has maximum degree bounded by $\Delta_1\coloneqq 1+(\Delta'_5)^p$.

\begin{figure}[t]
    \centering
    \includegraphics[scale=0.72, page=1]{tree.pdf}
    \includegraphics[scale=0.72, page=2]{tree.pdf}
    \caption{The construction of $T''$ in the proof of \cref{thm:equivalences}. \textbf{Left:} The definition of the sets $I$ (red), $C_z$, and $C_y$ for a vertex $y \in V(T)$ (both sets blue). \textbf{Right:} The vertices and edges of $T''$.}
    \label{fig:tree}
\end{figure}

    We endow $T''$ with bag function $\bag''$ defined as follows:
    \[\bag''(y)\coloneqq \bigcup_{x\in C_y} \bag'(x),\qquad\textrm{for all }y\in I=V(T'').\]
    Since $\bag'(x)$ is $(k_5',r)$-coverable for each $x\in V(T)$, from \eqref{eq:wydra} it follows that $\bag''(y)$ is $(k_1,r)$-coverable for each $y\in V(T)$, where $k_1\coloneqq M\cdot k_5'$.
    It remains to show that $\Tt''\coloneqq (T'',\bag'')$ is a tree-partition of $G$ of spread $r$.
    
    First, note that $\{\bag'(y) \mid y \in V(T'') \}$ is a partition of $V(G)$.
    Indeed, this follows from the fact that $\varphi(v) \in V(H)$ is  uniquely defined for every $v \in V(G)$ and $\{\bag(x) \mid x \in V(T) \}$ is a partition of $V(H)$.
    Now consider any $u,v\in V(G)$ with $u\in \bag''(y)$ and $v\in \bag''(y')$, where the nodes $y$ and $y'$ are non-equal and non-adjacent in $T''$.
    Let $x,x'\in V(T)$ be such that $u\in \bag'(x)$ and $v\in \bag'(x')$.
    By \eqref{eq:bobr}, we infer that $\dist_T(x,x')>p$.
    Since $u\in \bag'(x)$, we have that $\varphi(u)\in \bag(x)$, and similarly $\varphi(v)\in \bag(x')$.
    Since $\Tt$ is a tree-partition of $H$ and every edge of $H$ has weight at least $\gamma r$, the assertion $\dist_T(x,x')>p$ implies that
    \[\dist_H(\varphi(u),\varphi(v))>p\cdot \gamma r\geq (\alpha+\beta)r.\]
    Recall now that $H$ is an $(\alpha,\beta r)$-quasi-isometry. Hence,
    \[\dist_H(\varphi(u),\varphi(v))\leq \alpha\cdot \dist_G(u,v)+\beta r.\]
    By combining the two inequalities above we conclude that $\dist_G(u,v)>r$, as required.
\end{proof}

Let us stress that in the equivalences provided by \cref{thm:equivalences}, the constants $k_1$, $\Delta_1$, $k_2$, $k_3$, $k_4$, $\Delta_4$, $k_5$,$\Delta_5$, $\alpha$, $\beta$, $\gamma$ can be bounded in terms of each other and of $m$, but the distance parameter $r$ is not involved in those bounds. In other words, the equivalence holds for any choice of the ``scale'' $r\in \N_{>0}$. Also, we note that only implications \ref{pr:bsn}$\Rightarrow$ \ref{pr:qi} and \ref{pr:qia}$\Rightarrow$ \ref{pr:tpw} make use of the assumption that the doubling dimension is bounded.
    
    %and let $I$ be an inclusion-wise maximal distance-$\ell$ independent set in $T$. (We consider here the edges of $T$ to have unit weights.) For each $z\in I$, we let $D_z\subseteq V(T)$ be the set of those nodes $y$ of $T$ for which $x$ is the closest (in $T$) node belonging to $I$; ties are broken arbitrarily. Note that $\{D_z\colon z\in I\}$ is a partition of $V(T)$ and each $D_z$ induces a connected subtree of $T$ of radius at most $\ell$. Since $T$ has maximum degree at most $\Delta_5'$, we have
    %\[|D_z|\leq 1+\Delta_5'+(\Delta_5')^2+\ldots+(\Delta_5')^\ell\eqqcolon M.\]
    %Let now $T''$ be the tree obtained from $T$ by contracting $D_z$ onto $z$, for each $z\in I$. We endow $T''$ with a bag function $\bag''$ defined as follows:
    %\[\bag''(z)\coloneqq \bigcup_{x\in D_z} \bag'(x).\]
    %Since $\bag'(x)$ is $(k_5',r)$-coverable for each $x\in V(T)$, it follows that $\bag''(z)$ is $(k_1,r)$-coverable, where $k_1\coloneqq M\cdot k_5'$. Further, since $T$ has maximum degree at most $\Delta_5'$, $T''$ has maximum degree at most $\Delta_1\coloneqq \Delta_5'\cdot M$. It remains to show that $\Tt''\coloneqq (T'',\bag'')$ is a tree-partition of $G$ of spread $r$.
    
    
    %Thus each $\beta_x$ can be covered with $k_2 2^{m\log\alpha\beta}$ balls of radius~$r$. 
    %However, $\T'$ is not necessarily a dist-$r$ tree partition of $(X, d)$, as there may be two points which are close in $X$ but belong to distant bags of $T$. To address this, we choose an inclusion-maximal dist-$r(\alpha + \beta)$ independent set $I \subseteq V(T)$ and fix an arbitrary order on it. Then  we take a Voronoi diagram $D_V(V(T), I)$. Let us now consider $\T''=(T'', \{\beta_x\}_{x\in V(T'')})$, where:
    %\[V(T'')= I, \quad E(T'')=\{u_1u_2\mid \exists_{v_1\in R_V(u_1)}\exists_{v_2\in R_V(u_2)}\ v_1v_2\in E(T)\},\]
    %and for every $x\in V(T'')$ we have $\beta_x=\bigcup_{v\in R_V(x)} \beta_v$. 
    %Now each bag $\beta_x$ can be covered by \[k_2 2^{m\log\alpha\beta} \cdot \Delta_2^{r(\alpha+\beta)+1}:=k_3\] balls of radius $r$.

    %For this, consider any two vertices $u,v\in V(G)$ such that $\dist_G(u,v) \leq r$ and suppose for a contradiction that $\dist_{T''}(z,w)>1$, where $z,w\in V(T'')$ are such that $u\in \bag''(z)$ and $v\in \bag''(w)$. By construction it follows that if $x,y\in V(T)$ are such that $u\in \bag'(x)$ and $v\in \bag'(y)$, then $\dist_T(x,y)$
    
    %Suppose that $\varphi(u_1)$ and $\varphi(u_2)$ do not belong to the same node or to two neighboring nodes of $T''$.
    %Thus nodes of $T$ containing $\varphi(u_1)$ and $\varphi(u_2)$ must be at distance greater than $r(\alpha +\beta)$ in $T$ and therefore we have $d_H(\varphi(u_1), \varphi(u_2))>r(\alpha+\beta)$. % By the definition of the isometry we get
    %\[d(u_1,u_2) \geq \tfrac1\alpha(d_H(\varphi(u_1), \varphi(u_2))-\beta r) > \tfrac1\alpha\left(r(\alpha+\beta)-\beta r\right) = r,\] which contradicts the assumption $d(u_1,u_2) \leq r$. 

    %Therefore the constructed structure is indeed a dist-$r$ tree partition of $(X,d)$ of width $k_3$.

% \begin{theorem}
%    Let $(X,d)$ be a metric space, $G$ a graph of maximal degree bounded by $\Delta$  and $d_G$ the shortest path metric on $G$. Suppose that treewidth of $G$ is equal $t$ and there exist $(\alpha, \beta)$-quasi-isometry $\varphi\colon X\to G$. Then there exists dist-$r$ tree partition decomposition  of $(X,d)$ of width TODO.
% \end{theorem}
% \begin{proof}
   % Let $\mathcal{T}=(T, \{\beta_x\}_{x\in V(T)}\}$ be a tree-partition of $G$ of width TODO and degree TODO. Let us split $T$ into subtrees, each of radius $r$, and contract each of them into a single vertex, obtaining a new tree $T'$. 
% \end{proof}

%tutaj napisać że skupialiśmy się na tree partittion width, ale możemy przejść także na język treewidth (patrzy pierwszy dowód, że skoro mamy tree partition to i mamy treewidth a i że z tego wychodzą naturalnie separatory
%In the Theorem~\ref{thm:equivalencies} we focused on metric spaces with bounded doubling dimension and described dependencies based on the notion of tree-partition in graphs and its metric equivalent. In the following theorem we focus on treewidth.

%\begin{theorem}\label{thm:treewidth}
%    Let $(X,d)$ be a metric space with doubling dimension bounded by $m$ and let $r$ be a real number. Then the following conditions are equivalent:
%    \begin{enumerate}
%        \item there exists $k_1$ such that for every inclusion-maximal dist-$r$ independent set $I$ in $(X,d)$, the graph $H$ derived from the tuple $(X,d,r,I)$ has treewidth $k_1$
%        \item there exists $k_2$ such that there exists a tree decomposition of $(X,d)$ of $r$-width equal $k_2$
%        \item there exists $k_3$ such that $(X,d)$ has dist-$r$ separator number equal $k_3$
%    \end{enumerate}
%\end{theorem}

%\begin{proof} 
%    $\mathbb{3\Rightarrow 1}$ It follows drom the Lemma~\ref{lem:bdtw}.
    
%    $\mathbf{1\Rightarrow 2}$ Let $\T=(T,\{\beta_x\}_{x\in V(T)})$ be a tree decomposition of graph $H$. 
%    Let us fix any order on $V(H)$ and consider a Voronoi diagram on $(X,V(H))$. 
 %   Then let us consider
%    $\T'=(T,\{\beta_x'\}_{x\in V(T)})$, where $\beta_x'=\bigcup_{v\in\beta_x}R_V(v)$ for every $x\in V(T)$.
%    We show that $\T'$ is a tree decomposition of $(X,d)$. Suppose otherwise and let $A$ be a connected subset of $X$. Then the set of Voronoi regions which intersect $A$ is connected and therefore centers of these regions form a connected subgraph in graph $H$. Thus vertices of $T$ whose bags $\beta_x$ contain these centers also form a connected subtree by the definition of a tree decomposition of a graph. Hence, in the decomposition $\T'$, vertices whose bags intersect $A$ non-empty also form a connected subtree.

%    For every $v\in V(H)$ we have $R_V(v)\subseteq \B_{(X,d)}(v,r)$, as $I=V(H)$ is an inclusion-maximal dist-$r$ independent set. Thus each bag $\beta_x'$ for $x\in V(T)$ can be covered with $k_1$ balls of radius $r$ centered in~$V(H)\cap \beta_x'$. 

%    $\mathbf{2\Rightarrow 3}$ The proof of this implication follows exactly the proof of Theorem~\ref{thm:equivalencies} of the implication ${3\Rightarrow 4}$. We omit here detailed proofs for brevity.
%    \Jadwiga{Do we want details here?}
%\end{proof}