\section{Introduction}\label{sec:intro}

The main aim of the area of {\em{coarse graph theory}} is to study the metric structure in graphs. Recently, Georgakopoulous and Papasoglu~\cite{coarse2023} have formulated a programme of understanding coarse counterparts of the fundamental tools, techniques, and results from the classic structural graph theory, particularly the theory of Graph Minors. In the analogy between the classic and the coarse settings, it is typical that the requirement of disjointness of objects is replaced with {\em{farness}}, and the requirement of intersection is replaced with {\em{closeness}}.

Let us illustrate this principle on the example of tree decompositions and treewidth, which will also be the main objects of interest in this work. Classically, in a tree decomposition of a graph $G$ of width $k$ one requires all the bags to consists of at most $k+1$ vertices; thus, the bags are simply bounded in terms of size. A natural coarse counterpart of this condition is to require the following: every bag can be covered by at most $k$ balls of radius $r$ in $G$, for some distance parameter $r\in \N$ fixed beforehand. (We will call such vertex sets {\em{$(k,r)$-coverable}}.) Very recently, Nguyen, Scott, and Seymour~\cite{coarsetw2025}, and independently Hickingbotham~\cite{hickingbotham2025twquasiisom}, studied graphs admitting such tree decompositions and showed that they are quasi-isometric to graphs of bounded treewidth. Here, a {\em{quasi-isometry}} is a mapping between two graphs that roughly preserves distances, which serves as the basic notion of equivalence in the coarse theory; see~\cref{sec:prelim} for a precise definition. The works of Nguyen et al.~\cite{coarsetw2025} and of Hickingbotham~\cite{hickingbotham2025twquasiisom} extend the previously known result of Berger and Seymour~\cite{BergerS24} that admitting a tree decomposition where every bag has bounded diameter, i.e., is $(1,r)$-coverable for a constant $r$, is equivalent to being quasi-isometric to a tree.

In the classic theory, there are a number of notions that are equivalent to treewidth, either exactly or functionally. To name just a few, there is the bramble number, the tangle number, or the largest size of a grid minor; see the survey of Harvey and Wood for an extensive discussion~\cite{HarveyW17}. It is unclear if any of these notions has a suitable coarse counterpart that would be equivalent to ``coarse treewidth'';  the coarse analogue of the Grid Minor Theorem is at this point only a far-reaching conjecture~\cite{coarse2023}. The goal of this work is to explore whether the probably simplest connection between treewidth and another notion --- {\em{balanced separators}} --- can be lifted to the coarse setting.

We need a few definitions. Suppose $G$ is a graph and $\mu\colon V(G)\to \R_{\geq 0}$ is a weight function that assigns each vertex of $G$ a nonnegative weight. We say that a set $X$ of vertices of $G$ is a {\em{balanced separator}} for $\mu$ if for every connected component $C$ of $G-X$, the total weight of vertices within $C$ is at most half of the total weight of $G$. On one hand, it is not hard to see that if $G$ admits a tree decomposition $\Tt$ of width at most $k$, then there is a bag of $\Tt$ that is a balanced separator for $\mu$; hence any weight function $\mu$ admits a balanced separator of size at most $k+1$. On the other hand, using standard approaches to approximating treewidth (see e.g.~\cite[Section~7.6]{platypus}) one can argue that if any weight function $\mu$ on a graph $G$ admits a balanced separator of size $\ell$, then the treewidth of $G$ is at most $3\ell$. Thus, treewidth and the {\em{balanced separator number}} (the smallest $\ell$ such that every weight function admits a balanced separator of size at most $\ell$) are bounded by linear functions of each other; see \cref{lem:bsn}.

There is a natural coarse analogue of balanced separators of bounded size: these would be just separators that are coverable by a bounded number of bounded-radius balls.
Let us remark that the special case that the radius of each ball is 1, i.e., each separator can be covered by a bounded number of neighborhoods of vertices, has received significant attention due to its strong connections to the complexity of certain problems in induced-minor-closed classes of graphs~\cite{GartlandThesis,DBLP:conf/soda/ChudnovskyGHLS25,DBLP:conf/stoc/GartlandLMPPR24,DBLP:conf/focs/GartlandL20}.

As our main motivation, we postulate the following coarse analogue of the connection between treewidth and the balanced separator number.

\begin{conjecture}\label{con:main}
 For all $k,r\in \N$ there exist $\ell,d\in \N$ such that the following holds. Suppose $G$ is a graph such that every weight function $\mu\colon V(G)\to \R_{\geq 0}$ admits a balanced separator that is $(k,r)$-coverable. Then $G$ admits a tree decomposition whose every bag is $(\ell,d)$-coverable.
\end{conjecture}

The converse implication is easy: if $G$ admits a tree decomposition $\Tt$ whose bags are $(\ell,d)$-coverable and $\mu$ is a weight function on $G$, then again there is a bag of $\Tt$ that is a balanced separator for $\mu$, hence $\mu$ has an $(\ell,d)$-coverable balanced separator.

We do not resolve \cref{con:main} in this work; in fact, even the resolution of case $r=1$ would be very interesting. Our contribution consists of the following:
\begin{itemize}[nosep]
 \item We settle \cref{con:main} under the additional assumption that the graph has {\em{bounded doubling dimension}}: every ball of some radius can be covered by a bounded number of balls of twice smaller radius. This holds even in the following strong sense: $d=r$ and $\ell$ depends only on $k$ and on the doubling dimension.
 \item We prove two weaker statements where either the number of balls to cover every bag, or the radii of the balls, may moderately depend on $n$ --- the vertex count of the graph. Precisely, we prove that the existence of $(k,r)$-coverable balanced separators implies the existence of a tree decomposition with $(\Oh(k\log n),r)$-coverable bags (this is very easy), and also the existence of a tree decomposition with $(\Oh(k^2\log k),r(\log k+\log \log n+\Oh(1)))$-coverable bags (this is quite~involved).
\end{itemize}
We now discuss these statements in more details. In what follows, we say that $G$ has {\em{distance-$r$ balanced separator number}} at most $k$ if every weight function $\mu\colon V(G)\to \R_{\geq 0}$ admits a $(k,r)$-coverable balanced separator.

\paragraph*{Doubling dimension.} We say that a metric space $(X,\delta)$ has {\em{doubling dimension}} at most $m$ if for every $r\in \R_{>0}$, every ball of radius $r$ in $(X,\delta)$ can be covered by $2^m$ balls of radius $r/2$. This definition can be applied to (unweighted) graphs by considering the shortest-path distance metric. The assumption of the boundedness of doubling dimension is well-established in the area of approximation algorithms for metric problems. In a nutshell, it is an abstract property inspired by the setting of Euclidean spaces of fixed dimension, in which multiple natural decompositional techniques can be applied; see e.g. the fundamental work of Talwar~\cite{Talwar04}.
In the context of \cref{con:main}, we prove the following; see \cref{sec:prelim} for undefined terms.

\begin{restatable}{theorem}{thmequivalences}
%\begin{theorem}
\label{thm:equivalences}
    Let $\Cc$ be a class of graphs of doubling dimension bounded by $m$, for some $m\in \N$. Then the following conditions are equivalent for any $r\in \N_{>0}$:
    \begin{enumerate}[label=(\arabic*),ref=(\arabic*),nosep]
        \item\label{pr:tpw} There exist $k_1,\Delta_1\in \N$ such that every member of $\Cc$ has a tree-partition of spread $r$, maximum degree at most $\Delta_1$, and with $(k_1,r)$-coverable bags.
        \item\label{pr:tw} There exists $k_2\in \N$ such that every member of $\Cc$ has a tree decomposition with $(k_2,r)$-coverable~bags.
        \item\label{pr:bsn} There exists $k_3\in \N$ such that every member of $\Cc$ has distance-$r$ balanced separator number at most $k_3$.
        \item\label{pr:qi} There exist $k_4,\Delta_4\in \N$ such that every member of $\Cc$ is $(3,3r)$-quasi-isometric with an edge-weighted graph of tree-partition width at most $k_4$, maximum degree at most $\Delta_4$, and every edge of weight $3r$.
        \item\label{pr:qia} There exist $k_5,\Delta_5\in \N$ and $\alpha,\beta,\gamma\in \R_{>0}$ such that every member of $\Cc$ is $(\alpha, \beta r)$-quasi-isometric with an edge-weighted graph of tree-partition width at most $k_5$, maximum degree at most $\Delta_5$, and every edge of weight at least $\gamma r$.
    \end{enumerate}
%end{theorem}
\end{restatable}

Let us stress that in \cref{thm:equivalences}, the constants $k_1,k_2,k_3,k_4,k_5,\Delta_1,\Delta_4,\Delta_5, \alpha, \beta, \gamma$ can be bounded by functions of each other and of the doubling dimension $m$, but are independent of $r$. Thus, the equivalence holds at every possible choice of the ``scale'' $r$.

The key to the proof of \cref{thm:equivalences} lies in observing that a generic construction of a graph that is quasi-isometric to a given graph $G$ at a given ``scale'' $r$ yields a graph $H$ of degree bounded in terms of the doubling dimension of $G$, regardless of the choice of $r$. (This construction is discussed in \cref{sec:quasi-isometries}.) This allows us to essentially work on a graph of bounded degree, where separators can be conveniently ``fattened'' by including their neighborhoods, and treewidth is functionally equivalent to the parameter {\em{tree-partition width}}~\cite{tree-partitions}. Tree-partition width is defined similarly to treewidth, except that the underlying notion of decomposition --- called a {\em{tree-partition}} --- consists of a tree of bags that form a {\em{partition}} of the vertex set of the graph, and adjacent vertices must lie in the same bag or in adjacent bags. Importantly, tree-partitions have the following ``spreading'' property: if $u$ and $v$ belong to bags that are at least $d$ apart in a tree-partition, then $u$ and $v$ must be also at least $d$ apart in the graph. This property, notoriously lacking in classic tree decompositions, appears very useful in the coarse setting.

More generally, it seems that in coarse graph theory, the assumption of having bounded doubling dimension can be interpreted as the assumption of having bounded degree ``on every possible scale of distances''. Given that several fundamental statements in the theory of induced minors benefit from the assumption of having bounded degree, see~\cite{BonnetHKM23,GartlandKL23,HendreyNST24,Korhonen23}, one could be hopeful that doubling dimension might prove insightful in the coarse theory.

\paragraph*{General setting.} The following statements describe the relaxations of \cref{con:main} that we~prove.

\begin{restatable}{theorem}{thmdecompsimple}
%\begin{theorem}
\label{thm:tdecomp-simple}
    Let $G$ be an $n$-vertex graph ($n\geq 2$) whose distance-$r$ balanced separator number is at most $k\geq 2$, for some positive integer $r$. Then $G$ has a tree decomposition whose every bag is $(k (\log n+2), r)$-coverable.
%\end{theorem}
\end{restatable}

\vspace{-0.3cm}

\begin{restatable}{theorem}{thmdecomp}
%\begin{theorem}
\label{thm:tdecomp}
    There is a constant $c\in \N$ such that the following holds. Let $G$ be an $n$-vertex graph ($n\geq 2$) whose distance-$r$ balanced separator number is at most $k\geq 2$, for some positive integer $r$. Then $G$ has a tree decomposition whose every bag is $(c k^2\log k, r(\log k +\log\log n +c))$-coverable.
%\end{theorem}
\end{restatable}

As mentioned, \cref{thm:tdecomp-simple} is very simple: we decompose the graph in a recursive way, always breaking the set of remaining vertices using a balanced separator. This yields a recursion of depth bounded by $\log n$, and the bags of the resulting tree decomposition are obtained by as the union of the balanced separators accumulated along every branch of the recursion.

\cref{thm:tdecomp} is more involved. On a high level, we emulate a recursive algorithm from the classic proof of the connection between treewidth and balanced separators. In this algorithm, the part of a graph that is left to decompose is separated from the rest of the graph by a separator $S$, which classically is bounded in size. In our recursion, we keep the invariant that $S$ can be covered by $\Oh(k^2\log k)$ balls, but for technical reasons we need to allow the radii of those balls to slowly grow in consecutive calls. To control the growth of the balls, we apply careful bookkeeping using a potential function that grows expontially with the radius.



