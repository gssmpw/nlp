\section{General case}

In this section we prove \cref{thm:tdecomp-simple,thm:tdecomp}. In both cases, we will construct a suitable tree decomposition explicitly, using a recursive procedure similar to the one used in the classic algorithms for constructing tree decompositions of graphs, see e.g.~\cite[Section~7.6]{platypus}. \cref{thm:tdecomp-simple} is very simple: we just iteratively break the graph using balanced separators, accumulating them on the way throughout $\log n$ levels of recursion. The proof of \cref{thm:tdecomp} is much more intricate: the recursion keeps track of a separator that can be covered only by a constant number of balls, but the radii of those balls will grow (very slowly) during the recursion.

To facilitate the description of our recursive procedures, we need the following definition of a partial tree decomposition that encapsulates the task of decomposing a subgraph of the given graph. Let $G$ be a graph, $S$ be a subset of vertices of $G$, and $U$ be the vertex set of a connected component of $G - S$. Then a \emph{partial tree decomposition} of $(S, U)$ is a tree decomposition $\Tt$ of $G[S\cup U]$ with the following additional property: there is a bag of $\Tt$ that contains the whole $S$.

In the description we will use the Iverson notation: for a condition $\psi$, $[\psi]$ is equal to $1$ if $\psi$ is true, and $0$ otherwise.


\subsection{Superconstant ball count}

We first prove \cref{thm:tdecomp-simple}.
\thmdecompsimple*
%\prz[inline]{Is it worth introducing $c$ in the theorem? From the lemma we get that the number of balls is $0 + k(\lceil \log n \rceil +1) \leq k(\log n + 2)$. The current statement might wrongly suggest that some large constant is involved.}

The construction of the required tree decomposition is encapsulated in the following~lemma.

\begin{lemma}\label{lem:tdecomp_step_simple}
    Fix $r\in \N_{>0}$.
    Let $G$ be a graph with distance-$r$ balanced separator number at most $k$, let $S$ be a set of vertices of $G$, and let $U$ be the vertex set of a connected component of $G - S$. Suppose $S$ is $(\ell,r)$-coverable for some $\ell\in \N$, and $|U| \leq 2^m$ for some $m \in \N$. Then there exists a partial tree decomposition of $(S,U)$ whose every bag is $(\ell+k(m+1),r)$-coverable. 
\end{lemma}
\begin{proof}
    We proceed by induction on $|U|$. The case $|U|=0$ holds vacuously, as we assume $U$ to be a (non-empty) connected component.

    Let us move on to the induction step. Consider the following weight function $\mu\colon V(G)\to \R_{\geq 0}$: for $u\in V(G)$, we set $\mu(u) \coloneqq [u\in U]$. Since $G$ has distance-$r$ balanced separator number bounded by $k$, we can find a $(k,r)$-coverable set $Z\subseteq V(G)$ that is a balanced separator for $\mu$. This means that every connected component of $G-Z$ contains at most $|U|/2$ vertices of $U$.

    Let $\Aa$ be the family of the vertex sets of all connected components of $G[U]-Z$. Since every $A\in \Aa$ is entirely contained in one connected component of $G-Z$, we have $|A|\leq |U|/2\leq 2^{m-1}$. Noting that $S\cup Z$ is $(\ell+k,r)$-coverable, we may apply the induction assumption to the pair $(S\cup Z,A)$ for each $A\in \Aa$, thus obtaining a partial tree decomposition $\Tt_A$ of $(S\cup (Z\cap U),A)$ satisfying the following:
    \begin{itemize}[nosep]
        \item $\Tt_A$ has a node, say $x_A$, whose bag contains $S\cup (Z\cap U)$; and
        \item every bag of $\Tt_A$ is $((\ell+k)+km,r)$-coverable, hence $(\ell+k(m+1),r)$-coverable.
    \end{itemize}
    We may now combine the tree decompositions $\{\Tt_A \mid A\in \Aa\}$ into a single tree decomposition $\Tt$ by adding a new node $x$ with bag $S\cup (Z\cap U)$, and making $x$ adjacent to all the nodes $\{x_A \mid A\in \Aa\}$. It is straightforward to verify that $\Tt$ is a tree decomposition of $G[S\cup U]$. Also, the bag of $x$ contains $S$, so $\Tt$ is a partial tree decomposition of $(S,U)$. Finally, $S\cup (Z\cap U)$ is clearly $(\ell+k,r)$-coverable, hence every bag of $\Tt$ is $(\ell+k(m+1),r)$-coverable.
\end{proof}

Now, \cref{thm:tdecomp-simple} follows immediately by applying \cref{lem:tdecomp_step_simple} for $S=\emptyset$, $U=V(G)$ (assuming without loss of generality that $G$ is connected), $\ell=0$, and $m=\lceil \log n\rceil$.





\subsection{Superconstant radii}

We now proceed to the proof of \cref{thm:tdecomp}. We will need a few extra definitions. Recall that in the context of \cref{thm:tdecomp}, we assume the existence of balanced separators consisting of $k$ balls of radius~$r$, but to cover the bags of the constructed decomposition, we allow balls of varying radii. We will only use balls of radii being multiples of $r$, so call a set $\Bb$ of balls {\em{round}} if for every ball $B\in \Bb$, the radius of $B$, denoted $\rad(B)$, is a positive integer multiple of $r$. We will use the following {\em{potential}} of $\Bb$ to keep track of the growth of radii: \[\Phi(\Bb) \coloneqq \sum_{B \in \Bb} 2^{\rad(B)/r}.\]
Note that these definitions depend on the radius parameter $r\in \N_{>0}$, fixed in the context.

Also, we set
\[\Gamma\coloneqq 2000\cdot k^2 \log k.\]
This will be the bound on the number of balls needed to cover every separator of the constructed tree decomposition. Again, this definition depends on the parameter $k\in \N_{>0}$ fixed in the context.

With these definitions in place, our recursive procedure can be captured by the lemma below.

\begin{lemma}\label{lem:tdecomp_step}
    Fix $r\in \N_{>0}$.
    Let $G$ be a graph with distance-$r$ balanced separator number at most $k$, let $S$ be a set of vertices of $G$, and let $U$ be the vertex set of a connected component of $G - S$. Let $\Bb$ be a round set of at most $\Gamma$ balls whose union contains $S$. Suppose $|U| \leq 2^m$ for some $m \in \N$. Then there exists a partial tree decomposition $\Tt$ of $(S, U)$ such that each bag of $\Tt$
    can be covered with a round set of balls of size at most $\Gamma+2k$ and potential at most $\Phi(\Bb) + 4k(m+1)$.
\end{lemma}
\begin{proof}
    The proof is by induction on the size of $U$. There is no base of induction needed: the case $|U|=0$ cannot happen, for $U$ is the vertex set of a (non-empty) connected component of $G-S$.

    
    %Assume $|I \cap U| = 0$. We create a partial decomposition consisting of a single bag containing whole $U \cup S$. Let $\mathcal{B}'$ be the set of all balls of $\mathcal{B}$ with radii increased by $r$. The potential of $\mathcal{B}'$ is exactly $2^r \cdot \Phi(\mathcal{B})$. Now, we argue that this set indeed covers whole $U$. Assume otherwise, i.e., that there exists a point $x \in U$ not contained in any $B \in \mathcal{B}'$. We argue that in this case the set $I \cup \{x\}$ is a distance-$r$ independent set, contradicting that $I$ is inclusion-maximal.
    
    %Assume otherwise, i.e., that there exists a point $y \in I$ such that $\dist(x, y) \leq r$, hence there exists a curve $\gamma$ between $y$ and $x$ of length at most $r$. Since $y \not\in U$, the curve $\gamma$ intersects $S$, and hence some ball $B \in \mathcal{B}$. Let $o$ be the center of $B$ and let $z$ be any point in $\gamma \cap B$. We have
    %$$
    %\dist(x, o) \leq \dist(x, z) + \dist(z, o) \leq \dist(x, y) + \dist(z, o) \leq r + \rad(B).
    %$$
    %This means that $x$ is covered by $B$ with radius increased by $r$, which is a contradiction. Therefore $x$ is at distance more than $r$ from every point in $I$, and therefore $I$ is not inclusion-wise maximal. This proves the base case for induction.
    
    We proceed to the induction step. First, we need to massage the ball set $\Bb$ in order to achieve a certain ``sparseness'' property that will be useful later.

    Fix \[\alpha \coloneqq 2 + \ceil{\log 2k}.\] For $\ell \in \N_{>0}$, we shall say that a vertex $x \in V(G)$  is {\em{$\ell$-crowded}} with respect to $\Bb$ if there exist at least $2^{\alpha}$ balls in $\Bb$ of radius exactly $\ell r$ whose centers are at distance at most $\alpha r$ from $x$. Let $\Bb'$ denote the set obtained from $\Bb$ by replacing all such balls with a single ball of radius $r(\ell + \alpha)$ with center at~$x$. Clearly, we have
    \[|\Bb'|<|\Bb|\qquad\textrm{and}\qquad \Phi(\Bb')\leq \Phi(\Bb)-2^\alpha\cdot 2^\ell+2^{\ell+\alpha}=\Phi(\Bb).\]
    Moreover, by triangle inequality, every ball removed from $\Bb$ is entirely covered by the ball added, hence
    \[\bigcup \Bb'\supseteq \bigcup \Bb.\]
    By applying this operation repeatedly as long as there exists an $\ell$-crowded vertex for some $\ell$, we arrive at a round set of balls $\Bb'$ such that 
    \begin{itemize}[nosep]
        \item $|\Bb'| \leq |\Bb|$,
        \item $\Phi(\Bb') \leq \Phi(\Bb)$,
        \item $\bigcup \Bb'\supseteq \bigcup \Bb$, and
        \item no vertex $u\in V(G)$ is $\ell$-crowded with respect to $\Bb'$, for any $\ell\in \N_{>0}$.
    \end{itemize}
    Also, without loss of generality we may assume that no ball in $\Bb'$ is entirely contained in another ball from $\Bb'$, for the smaller ball can be just removed from $\Bb'$ without breaking any of the properties above.

    We observe the following consequence of the construction of $\Bb'$.

    \begin{claim}\label{clm:balls_no_bound}
        For every vertex $u \in V(G)$, there are at most $2\alpha \cdot 2^{\alpha}$ balls in $\Bb'$ whose centers lie at distance at most $\alpha r$ from $u$.
    \end{claim}
    \begin{claimproof}
        Let $\Bb'_u$ be the set of balls from $\Bb'$ whose centers are at distance at most $\alpha r$ from $u$. Pick any two distinct balls $B_1, B_2 \in \Bb'_u$, say with centers $o_1, o_2$  and radii $\ell_1 r, \ell_2 r$, respectively. Suppose for a moment that $\ell_1 \geq \ell_2 + 2 \alpha$. Then for every vertex $v \in B_2$, we have
        \[
            \dist(v, o_1) \leq \dist(v, o_2) + \dist(o_2, u) + \dist(u, o_1) \leq
            (\ell_2 + 2 \alpha) r \leq \ell_1 r,
        \]
        implying $B_2 \subseteq B_1$, which is a contradiction with the construction of $\Bb'$. Similarly, supposition $\ell_2\geq \ell_1+2\alpha$ also leads to a contradiction. Therefore we have $|\ell_1 - \ell_2| < 2 \alpha$. Since $B_1,B_2$ were chosen arbitrarily, we conclude that there are at most $2\alpha$ different radii among the balls of $\Bb'_u$. Since $u$ is not $\ell$-crowded with respect to $\Bb'$ for any $\ell \in \N$, every fixed radius gives rise to at most $2^{\alpha}$ balls of $\Bb'_u$. We conclude that $|\Bb'_u| \leq 2\alpha \cdot 2^{\alpha}$, as claimed.
    \end{claimproof}

    With $\Bb'$ constructed, we proceed with the proof.
    Since the balls of $\Bb'$ are not contained one in another, they have pairwise different centers. Let then $O$ be the set of centers of balls from $\Bb'$.
    We define the following weight functions: for $u\in V(G)$, we set \[\mu_U(u) \coloneqq [u\in U]\qquad\textrm{and}\qquad \mu_O(u) \coloneqq [u \in O].\]
    Since $G$ has distance-$r$ balanced separator number bounded by $k$, we may find sets  $\Dd_U$ and $\Dd_O$ of radius-$r$ balls with $|\Dd_U|,|\Dd_O|\leq k$ such that $\bigcup \Dd_U$ is a balanced separator for $\mu_U$ and $\bigcup \Dd_O$ is a balanced separator for $\mu_O$. Define
    \[\Dd\coloneqq \Dd_U\cup \Dd_O\qquad\textrm{and}\qquad Z\coloneqq \bigcup \Dd.\]
    Observe that $|\Dd|\leq 2k$ and $Z$ is a balanced separator both for $\mu_U$ and for $\mu_O$. 
    In what follows we will define a number of sets of balls, see \cref{fig:balls}.
%    \Jadwiga{Shall we move this reference after we define the sets?}
%    \mipilin{No, the reader should be made aware that looking at the figure while parsing definitions is helpful.}
    
    Let $O_\Dd$ be the set of centers of balls from $\Dd$. Further, let $\wh{\Dd}$ be the set of those balls from $\Bb'$ whose centers lie at distance at most $\alpha r$ from any vertex of~$O_\Dd$. By \cref{clm:balls_no_bound}, we have \[|\wh{\Dd}| \leq 2\alpha \cdot 2^{\alpha} \cdot |O_\Dd| \leq 4\alpha k \cdot 2^{\alpha}.\]

    %Recall that $G[U]$ is a connected component of $G-S$, hence $G[U]-Z=G[U]-(\bigcup \Dd\cup \bigcup \wh{\Dd})$ is broken into several connected components; let $\Ww$ be the family of the vertex sets of those components.
    Let $W$ be the vertex set of any connected component of $G-Z$. Since $Z$ is a balanced separator for $\mu_U$ and for $\mu_O$, we have \[|W \cap U| \leq |U|/2 \leq 2^{m - 1}\qquad\textrm{and}\qquad |W \cap O| \leq |O|/2\leq \Gamma / 2.\] Consider all balls of $\Bb' - \wh{\Dd}$ with centers outside of $W$ and let $R_W$ denote the largest radius among them. We set $R_W \coloneqq (\alpha - 1)r$ in case there are no such balls or all such balls have radius less than $(\alpha - 1)r$. Let $\Dd_W$ be the set of balls obtained from the set $\Dd$ by changing the radius of every ball to $R_W - (\alpha - 2)r$ (note that this value is at least $r$). 
    Next,
let $\Bb_W'$ be those balls from $\Bb'$ whose centers lie in $W$. We~set \[\Bb_W \coloneqq \wh{\Dd} \cup \Dd_W \cup \Bb'_W.\]

\begin{figure}[t]
    \centering
    \includegraphics[width=0.7\linewidth]{balls}
    \caption{Various sets of balls in the proof of \cref{thm:tdecomp}. Blue disks indicate balls from $\Dd$ and their centers, i.e., the set $O_{\Dd}$, are shown by blue dots.
    Balls  $\Dd_W$ are obtained by (possibly) enlarging balls from $\Dd$, while keeping the same centers; this is not shown in the picture for the sake of clarity.
    Red disks depict balls from $\Bb'$. Balls filled with diagonal lines have their centers in $W$, i.e., they belong to~$\Bb'_W$. Filled red disks indicate the set $\wh{\Dd}$, i.e., their centers are close to the vertices from $O_{\Dd}$.    
    }
    \label{fig:balls}
\end{figure}

 Note that
    \[
    |\mathcal{B}_W| \leq |\wh{\Dd}| + |\Dd_W| + |\mathcal{B}'_W| \leq
    4\alpha k \cdot 2^{\alpha} + 2k + \Gamma/2 \leq \Gamma,
    \]
    where the last inequality can be argued by substituting $\alpha=2+\ceil{\log 2k}$ and $\Gamma=2000\cdot k^2\log k$ followed by direct estimations.

Now let us bound the potential of $\Bb_W$. We consider two cases: either there exists a ball $B \in \Bb' - \wh{\Dd}$ with center outside of $W$ and radius at least $(\alpha - 1) r$, or not. If not, then $\Dd_W=\Dd$ and all the balls of $\Dd_W$ have radius $r$, hence 
    \[
    \Phi(\mathcal{B}_W) \leq \Phi(\wh{\Dd} \cup \mathcal{B}'_W) + \Phi(\Dd_W) \leq \Phi(\Bb') + |\Dd_W|\cdot 2\leq \Phi(\Bb) + 4k.
    \]
    If yes, then since $B \not\in \wh{\Dd} \cup \Bb'_W$, we have
    \[
    \Phi(\Bb_W) \leq \Phi(\wh{\Dd} \cup \Bb'_W) + \Phi(\Dd_W) \leq
    \Phi(\Bb' - \{B\}) + \Phi(\Dd_W) \leq
    \Phi(\Bb') - 2^{R_W/r} + 2k \cdot 2^{R_W/r - (\alpha - 2)}.
    \]
    We have $\alpha - 2 = \ceil{\log 2k}$, so
    \[
    \Phi(\Bb_W) \leq \Phi(\Bb') - 2^{R_W/r} + 2^{R_W/r} \cdot (2k \cdot 2^{-\log 2k}) = \Phi(\Bb')\leq \Phi(\Bb).
    \]
   

    The intuition of the remainder of the proof is as follows. Recall that $G[U]$ is a connected component of $G-S$, hence the removal of $Z$ breaks $G[U]$ into several smaller components. Each such component $G[A]$ is contained in some component of $G-Z$, say $G[W]$. We would like to apply induction for all components $G[A]$ as above, keeping $\Bb_W$ as the cover for a suitable set $S_A$ separating $A$ from the rest of the graph. However, the construction of $\Bb_W$ used only a subset of the balls from $\Bb'$, so we need to make sure that the part of $S$ contained in $S_A$ is still covered by $\Bb_W$.
    %\Jadwiga{Well, we never write that we discard something, so this statement is a bit misleading.}
    %\mipilin{Rephrased.}
    %\tara{Maybe at the end say something slightly more precise, like, ``so we need to make sure that the part of $S$ contained in $S_A$ is still covered by $B_W$.'' ?}
    We do this in the following claim.

    
    %Intuitively, we will construct a partial decomposition of $(S, U)$ by ``gluing'' decompositions obtained by induction for all connected components $W$ using the separator $Z$. The set $\Bb_W$ will be used as the cover for the root bag of such smaller decomposition, hence we must argue that no balls from outside of $\mathcal{B}_W$ will ``leak'' into $W$. More formally, we need the following claim.
    
    \begin{claim}\label{clm:no_leaks}
        Let $W$ be the vertex set of a connected component of $G-Z$. Then every ball $B \in \Bb' - \wh{\Dd}$ whose center lies outside of $W$ is disjoint from $W - \bigcup \Dd_W$.
    \end{claim}

    \begin{claimproof}
        Pick any ball $B\in \Bb'-\wh{\Dd}$, say of radius $r' \leq R_W$, and let $o$ be its center; assume $o\notin W$. Pick any vertex $x \in W$ with $\dist(x, o) \leq r'$ and let $P$ be a shortest path connecting $o$ and $x$. Since $o \not\in W$ and $x\in W$, there exists a ball $B' \in \Dd$ which intersects $P$; recall that $B'$ has radius $r$.
        Let $o'$ denote the center of $B'$ and let $z$ be any vertex on $P$ that belongs to $B'$.
        As $P$ is a shortest path, we have that $\dist(o,x) = \dist(o,z) + \dist(z,x)$, and since $z \in B'$, we have $\dist(o',z)\leq r$.
        Since $B \not\in \wh{\Dd}$, we have
        $
        \dist(o, z) \geq \dist(o, o') - \dist(o', z) \geq (\alpha - 1) r,
        $
        hence
        \begin{align*}
        \dist(x, o') \leq \ &  \dist(o', z) + \dist(z, x) = \dist(o', z) + \dist(o, x) - \dist(o,z)\\
        \leq \ &  r + r' - \dist(o, z) \leq  r + R_W - (\alpha - 1)r = R_W - (\alpha - 2)r.
        \end{align*}
        In particular, the ball of radius $R_W - (\alpha - 2)r$ with center at $o'$ both contains $x$ and belongs to $\Dd_W$. So $x\in \bigcup \Dd_W$ are the proof is complete.
    \end{claimproof}

    Let $\Aa$ comprise the vertex sets of all the connected components of $G[U]-Z$.
    For $A\in \Aa$, let $G[W_A]$ be the connected component of $G-Z$ such that $A\subseteq W_A$. We define
    \[S_A \coloneqq (Z \cap (U \cup S)) \cup (S \cap W_A).\]
    Let us first verify that that the ball set $\Bb_{W_A}$ covers $S_A$.
    
    \begin{claim}\label{cl:coverage}
        For every $A\in \Aa$, we have $S_A\subseteq \bigcup \Bb_{W_A}$.
    \end{claim}

    \begin{claimproof}
        Pick any $x \in S_A$. If $x \in Z$, then $x$ is covered by some ball in $\Dd_{W_A}\subseteq \Bb_{W_A}$ (recall that every ball of $\Dd_{W_A}$ is obtained from a ball of $\Dd$ by possibly increasing the radius). 
        %\tara{I think these $W$ should be $W_A$?} 
        Now, assume that $x \in S \cap W_A$, so in particular there exists a ball $B \in \Bb'$ that contains $x$. If the center of $B$ lies in $W_A$, then $B\in \Bb'_{W_A}\subseteq \Bb_{W_A}$ and consequently $x\in \bigcup \Bb_{W_A}$. If $B\in \wh{\Dd}$, then $x\in \bigcup \wh{\Dd}\subseteq \bigcup \Bb_{W_A}$. And if $B\in \Bb'-\wh{\Dd}$ and the center of $B$ lies outside of $W$, then by \cref{clm:no_leaks} we have $x\in \bigcup \Dd_{W_A}\subseteq \bigcup \Bb_{W_A}$.
    \end{claimproof}

    Observe that since $|W\cap U|\leq |U|/2$ for each component $W$ of $G-Z$, we also have $|A|\leq |U|/2\leq 2^{m-1}$ for each $A\in \Aa$. Also, by \cref{cl:coverage} we have that $\Bb_{W_A}$ covers $S_A$, and recall that $|\Bb_{W_A}|\leq \Gamma$ and $\Phi(\Bb_{W_A})\leq \Phi(\Bb)+4k$.
    Therefore, we may apply induction to the pair $(S_A,A)$ for each $A\in \Aa$, thus obtaining a partial tree decomposition $\Tt_A$ of $(S_A,A)$ with the following properties:
    \begin{itemize}[nosep]
        \item $\Tt_A$ has a node, say $x_A$, whose bag contains the whole $S_A$; and
        \item every bag of $\Tt_A$ can be covered by a round set of at most $\Gamma+2k$ balls with potential bounded by
        \[\Phi(\Bb_{W_A})+4km\leq \Phi(\Bb)+4k+4km=\Phi(\Bb)+4k(m+1).\]
    \end{itemize}
    We now combine the decompositions $\{\Tt_A \mid A\in \Aa\}$ into a tree decomposition $\Tt$ of $G[S\cup U]$ by creating a fresh node $x$ with bag $S\cup (Z\cap U)$ and making it adjacent to all the nodes $x_A$ for $A\in \Aa$. It is straightforward to verify that $\Tt$ is a tree decomposition of $G[S\cup U]$; we leave the verification to the reader. Since $S$ is contained in the bag of $x$, $\Tt$ is a partial  tree decomposition of $(S,U)$.

    It only remains to check whether the bag of $x$ --- namely $S\cup (Z\cap U)$ --- can be covered by a round set of at most $\Gamma+2k$ balls with potential bounded by $\Phi(\Bb)+4k(m+1)$. For this, we take $\Bb\cup \Dd$. Note that
    \[\bigcup (\Bb\cup \Dd)=\bigcup \Bb\cup \bigcup \Dd\supseteq S\cup Z\supseteq S\cup (Z\cap U).\]
    Finally, we have
    \[|\Bb\cup \Dd|\leq |\Bb|+|\Dd|\leq \Gamma+2k\quad \textrm{and}\quad \Phi(\Bb\cup \Dd)\leq \Phi(\Bb)+\Phi(\Dd)\leq \Phi(\Bb)+4k\leq \Phi(\Bb)+4k(m+1),\]
    where the pre-last inequality follows from each ball of $\Dd$ having potential $2$.
\end{proof}


    

    %to obtain a partial decomposition $(T_W, \beta^{(W)}_x)$. The root bag of this decomposition contains $S_W$ and every bag can be covered with at most $\Gamma$ balls of potential at most
    %$$
    %2^r \cdot \Phi(\mathcal{B}_W) + 2k(m - 1) \cdot 4^r \leq
    %2^r (\Phi(\mathcal{B}) + 2k \cdot 2^r) + 2k(m - 1) \cdot 4^r =
    %2^r \cdot \Phi(\mathcal{B}) + 2km \cdot 4^r.
    %$$
    %We repeat this for every component $W$ of $X - Z$ and construct the partial decomposition $(T, \beta_x)$ by creating a root bag containing $S \cup (Z \cap U)$ and appending as a children the root of trees $T_W$ for all $W$. Let $r_T$ denote the root vertex of $T$ and let $r_W$ denote the root vertex of $T_W$.

    %For this to work we must prove that such choice of $S_W, U_W, \mathcal{B}_W$ satisfies the assumptions of induction, i.e., that the following claim holds.

    
    %Now, we want to show that $(T, \beta_x)$ is indeed a valid partial decomposition. Clearly, the union of all bags is exactly $U \cup S$, hence it remains to show the following.
    
    %\begin{claim}
    %    Fix any connected $A \subseteq X$ and let $A_T := \{t\in V(T) \mid \beta_t \cap A \neq \varnothing\}$. Then, $A_T$ induces a connected subtree of $V(T)$.
    %\end{claim}

    %\begin{proof}
    %    By induction, $A_T \cap V(T_W)$ is connected for every $W$, hence it suffices to prove that either:
    %    \begin{itemize}
    %        \item $A_T \subseteq V(T_W)$ for some $W$ or
    %        \item $A_T$ contains $r_T$ and every $r_W$ such that $A_T \cap V(T_W) \neq \emptyset$.
    %    \end{itemize}
        
    %    First, consider a case where $A$ intersects the root bag of $T$. If $A$ intersects $Z \cap U$, then it also intersects the root bags of $T_W$ for all $W$. Otherwise, it must intersect $S$. Pick any $W$ and a bag $\beta_t \in V(T_W)$ such that $A$ intersects $\beta_t$. Any curve between a point from $S_W \cup U_W$ and a point from $S$ must intersect $S_W$, hence $r_W \in A_T$. This shows that $A_T$ satisfies the second condition.
        
    %    Now, consider a case where $A$ is disjoint from the root bag of $T$. Then either $A$ is disjoint from $S \cup U$ in which case $A_T = \emptyset$ or is contained in $W \cap U$ for some connected component $W$ of $X - Z$. By our construction, for every other component $W'$, all bags assigned to $V(T_{W'})$ are disjoint from $W$, hence $A_T \subseteq V(T_W)$.
    %\end{proof}
    
    %This proves that $(T, \beta_x)$ is indeed a valid partial tree decomposition satisfying lemma statement.

Now \cref{thm:tdecomp} follows from an easy application of \cref{lem:tdecomp_step}.

\thmdecomp*
\begin{proof}
    Assuming without loss of generality that $G$ is connected, we apply \cref{lem:tdecomp_step} to $S=\emptyset$ and $U=V(G)$. Thus, we obtain a tree decomposition $\Tt$ of $G$ whose every bag can be covered by a round set of $\Gamma+2k\leq 2002 k^2\log k$ balls whose potential is at most $4k \cdot (\ceil{\log n}+1)$. Let $\Bb$ be any such set and denote $R \coloneqq \max_{B \in \Bb} \rad(B)$. We have
    \[
    2^{R/r} \leq \Phi(\mathcal{B}) \leq 4k \cdot (\ceil{\log n}+1)\leq 12k\log n.
    \]
    Taking a logarithm, we get $R \leq r\cdot (\log k + \log \log n +\log 12)$, which completes the proof.
\end{proof}