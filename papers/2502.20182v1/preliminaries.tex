\section{Preliminaries}\label{sec:prelim}

By $\N,\N_{>0},\R_{\geq 0},\R_{>0}$ we denote the sets of nonnegative integers, positive integers, nonnegative reals, and positive reals, respectively. All logarithms are base-$2$.

\paragraph*{Graphs.} We use standard graph notation. All graphs considered in this paper are finite, simple, and unweighted, unless explicitly stated.
The set of vertices of graph $G$ is denoted as $V(G)$ and the set of edges as $E(G)$. The maximum degree of a graph $G$ is denoted as $\Delta(G)$.

 The distance metric of a graph $G$ is denoted by $\dist_G(\cdot,\cdot)$: for two vertices $u,v\in V(G)$, their distance, denoted by $\dist_G(u,v)$, is the length (i.e., the number of edges) of a shortest path connecting $u$ and $v$, or $+\infty$ if no such path exists. For a vertex $u$ and a nonnegative real $r$, the {\em{radius-$r$ ball}} around $u$ is the set $\B_G(u,r)\coloneqq \{v\in V(G)\mid \dist_G(u,v)\leq r\}$. We say that a set of vertices $A$ in a graph $G$  is {\em{$(k,r)$-coverable}} if one can choose $k$ radius-$r$ balls in $G$ whose union contains $A$. We emphasize that the distances here are measured in $G$ and the balls do not have to be contained in $A$.
 A {\em{distance-$r$ independent set}} in $G$ is a set of vertices $I\subseteq V(G)$ such that for all distinct $u,v\in I$, we have $\dist_G(u,v)>r$.

An {\em{edge-weighted graph}} is a graph $G$ equipped with a weight function $\weight_G\colon E(G)\to \mathbb{R}_{\geq 0}$ on edges. We can naturally lift the notation for distances, balls, etc. to edge-weighted graphs by considering the length of a path to be the sum of the weights of its edges. 

In all of the notation above, we may omit the graph in the subscript if it is clear from the context.


%By $N_G(v)$ we denote the set of neighbors of vertex $v$ in graph $G$ (the so called \emph{open neighborhood of $v$}). If $G$ is known from the context, it may be omitted.

\paragraph*{Graph decompositions.} A \emph{tree decomposition} of a graph $G$ is a pair $\mathcal{T}=(T,\bag)$, where $T$ is a tree and $\bag$ is a function assigning every node $x$ a subset $\bag(x)$  of vertices of $G$ (called the {\it bag} of $x$) such that following conditions are fulfilled:
\begin{itemize}[nosep]
    \item for each edge $vu\in E(G)$, there is a node $x$ of $T$ such that $\bag(x)$ contains both $u$ and $v$; and 
    \item for each vertex $u$ of $G$, the set of nodes of $T$ whose bags contain $u$  induces a connected non-empty subtree of $T$.
\end{itemize}
The \emph{width} of a tree decomposition is equal to $\max_{x\in V(T)}|\bag(x)|-1$. The minimum width over all tree decompositions of a graph $G$ is called the \emph{treewidth} of $G$ and is denoted as $\tw(G)$.

A \emph{tree-partition} of a graph is a similar notion to a tree decomposition, with the main difference that here the bags should form a partition of the vertex set. A \emph{tree-partition} of a graph $G$ is again a pair $\mathcal{T}=(T,\bag)$, where $T$ is a tree and $\bag$ is a function mapping nodes of $T$ to subsets of vertices of $G$ (called {\em{bags}}). This time, we require the following properties:
\begin{itemize}[nosep]
    \item each vertex $u$ of $G$ belongs to exactly one set $\bag(x)$ for some $x\in V(T)$; and
    \item for every edge $uv\in E(G)$ either there exists $x\in V(T)$ such that $u,v\in \bag(x)$ or there exists an edge $xy\in E(T)$ such that $u\in \bag(x)$ and $v\in\bag(y)$.
\end{itemize}
The \emph{width} of a tree-partition decomposition is the size of the largest bag. The \emph{tree-partition width} of a graph $G$ is the minimum width over all tree-partitions of $G$ and is denoted as $\tpw(G)$.

It is straightforward to see that $\tw(G)\leq 2\tpw(G)-1$ for every graph $G$: given a tree-partition decomposition $(T,\bag)$ of $G$ of width $k$, we may obtain a tree decomposition of width at most $2k-1$ by subdividing every edge of $T$ once and assigning the vertex subdividing any edge, say $xy$, the bag $\bag(x)\cup \bag(y)$. While there is no relation between the two parameters in the other direction (consider a long path with a universal vertex added), they turn out to be functionally equivalent assuming the graph in question has bounded maximum degree. The following result with a worse bound was observed by an anonymous reviewer and reported in the work of Ding and Oporowski~\cite{tree-partitions}; the improved bound is due to Wood~\cite{Wood09}.

\begin{theorem}[\cite{tree-partitions,Wood09}]\label{thm:tw-tpw}
    For every graph $G$, it holds that $\tpw(G)\leq \frac{35}{4}\Delta(G)(\tw(G)+1)$.
\end{theorem}

By the maximum degree of a tree-partition $(T,\bag)$ we mean the maximum degree of $T$.
In the context of coarse graph theory, the following parameter of a tree-partition also seems insightful: For $r\in \N_{>0}$, we say that a tree-partition $(T,\bag)$ of a graph $G$ has {\em{spread}} $r$ if for every pair of vertices $u,v$ width $\dist_G(u,v)\leq r$, there is either a node $x\in V(T)$ with $u,v\in \bag(x)$ or an edge $xy\in E(T)$ with $u\in \bag(x)$ and $v\in \bag(y)$. Thus, the standard condition in the definition of tree-partitions is equivalent to having spread $1$, but intuitively, tree-partitions with larger spread break the graph into pieces that are further apart. The tree-partitions that we will construct within the proof of \cref{thm:equivalences} (the formalization of \cref{thm:equivalences}) will have spread $r$, rather than $1$. We note that in a follow-up work~\cite{HatzelP25}, Hatzel and the fifth author investigate some algorithmic aspects of tree-partitions with spread $r$ and $(k,r)$-coverable bags.

\paragraph*{Balanced separators.} Let $G$ be a graph and $\mu\colon V(G)\to \R_{\geq 0}$ be a weight function on the vertices of $G$. For a subset of vertices $A$ we denote $\mu(A)\coloneqq \sum_{u\in A} \mu(u)$, and for a subgraph $H$ we write $\mu(H)\coloneqq \mu(V(H))$.

We say that a vertex set $X\subseteq V(G)$ is a {\em{balanced separator}} for $\mu$ if for every connected component $C$ of $G-X$, we have $\mu(C)\leq \frac{1}{2}\mu(V(G)$. The {\em{balanced separator number}} of $G$, denoted $\bsn(G)$, is the smallest $k$ such that for every weight function $\mu$ there is a balanced separator of size at most $k$. We have the following standard lemma that connects the treewidth with the balanced separator number. 

%The proof can be easily extracted from the standard approaches to approximating treewidth, see e.g~\cite[Section 7.6]{platypus} or~\cite[Section 10.5]{KleinbergTardosBook}.

\begin{lemma}[{see e.g.~\cite[Section~5]{HarveyW17}}]\label{lem:bsn}
 For any graph $G$, we have
 \[\bsn(G)-1 \leq \tw(G)\leq 3\bsn(G).\]
\end{lemma}

%\mipilin{Citation for the above}
%\przin{is this true (with no additive constant?)}
%\mipilin{I think so. The $+1$ is in the right direction.}
%\mipilin{I could not find a good citation. I propose to leave it like this, I really don't feel like writing this proof again and again...}


As mentioned in \cref{sec:intro}, we will work with the following coarse variant of the balanced separator number. For $r\in \R_{\geq 0}$, the {\em{distance-$r$ balanced separator number}} of $G$ is the smallest $k$ such that for every weight function $\mu\colon V(G)\to \R_{\geq 0}$, there is balanced separator for $\mu$ that is $(k,r)$-coverable.

%\Jadwiga{here we quote theorem stating the dependence between max degree tpw and tw}

%A metric space is a pair $(X,d)$ where $X$ is a (not necessarily finite) set of {\em{points}} and $d\colon X\times X\to \mathbb{R}_{\geq 0}\cup\{+\infty\}$ is a symmetric function satisfying the triangle inequality: $d(x,y)+d(y,z)\geq d(x,z)$, for all points $x,y,z\in X$. With every (possibly edge-weighted) graph $G$ we associate the metric space $(V(G),\dist_G)$, and whenever applying the terminology of metric spaces to graphs, we mean this metric space. As in graphs, the {\em{$r$-ball}} around a point $x\in X$ is the set $\B_{(X,d)}(u,r)\coloneqq \{y\in X \colon d(x,y)\leq r\}$, and a {\em{distance-$r$ independent set}} is a set $I\subseteq X$ such that $d(x,y)>r$ for all distinct $x,y\in I$.


%By $(X,d)$ we denote a metric space with set $X$ and metric $d$.
%We say that a metric space $(X,d)$ is \emph{geodesic} if for each $x,y\in X$ there exists a curve $\gamma\colon x\to y$ of length $d(x,y)$.

%For $r \geq 0$, a set $I$ of $X$ is called a \emph{dist-$r$ independent set} if there are no $x,y\in I$ such that $d(x,y) \leq r$. 

%A \emph{ball} of radius $r$ and centre $x$ in metric space $(X,d)$ is a set of points $\{y\mid d(x,y)< r\}$ and is denoted as $\B_{(X,d)}(x,r)$. 

%Let $(X,d)$ be a metric space and let $\mathcal{F}\subseteq X$. Let $\leq$ be a linear order on $\F$. For each $v\in \F$ we define its \emph{Voronoi cell} 
%\begin{align}
%    R_V(v) \coloneqq &\left\{x \in X \mid d(x, v) < \min_{u \in \F \setminus \{v\}} d(x, u)\right\} \ \cup \\
%    &\left\{x \in X \mid d(x, v) = \min_{u \in \F \setminus \{v\}} d(x, u) \ \text{ and } \forall_{u\in\F\setminus\{v\}} \ d(x, v) = d(x, u) \Rightarrow x > v \right\}.
%\end{align} 
%Informally, we assign each element of $X$ to its closest point in $\F$, where $\leq$ is used for tie-breaking.
%A \emph{Voronoi diagram} $D_V(X,\F)$ is the collection of all Voronoi regions for all elements of $\F$. 



\paragraph*{Doubling dimension.} 
The \emph{doubling dimension} of a graph $G$ is the smallest $m$ such that for every nonnegative real $r$, every radius-$2r$ ball in $G$ can be covered by $2^m$ radius-$r$ balls. Note that by applying this condition to $r=\frac{1}{2}$ we may conclude that if a graph $G$ has doubling dimension $m$, then $\Delta(G)<2^m$.


%equal to $m$ if each ball of radius $2r$ can be covered by at most $2^m$ balls of radius $r$. It is denoted as $\ddim(X,d)$. 

%We say that a metric space $(X,d)$ has \emph{dist-$r$ separator number} equal to $k$ if for each weight function $\mu\colon X\to \mathbb{R}_{\geq 0}$ of finite support there exist at most $k$ balls $B_1, B_2, \dots, B_k$ of radius at most $r$ such that each component of $X-\bigcup_{i=1}^{k} B_i$ has measure at most $\frac{\mu(X)}{2}$.
%\newline
%\newline



\paragraph*{Quasi-isometries.} 
Let $G$ and $H$ be (possibly edge-weighted) graphs and $\alpha\geq 1,\beta\geq 0$ be reals. A mapping $\varphi\colon V(G)\to V(H)$ is called an \emph{$(\alpha, \beta)
$-quasi-isometry} if it satisfies the following properties: 
\begin{itemize}[nosep]
    \item for every pair of vertices $u, v \in V(G)$, it holds that \[\frac{1}{\alpha}\cdot \dist_G(u,v) - \beta \leq \dist_H(\varphi(u), \varphi(v))\leq \alpha \cdot \dist_G(u,v) + \beta; \ \text{ and} \]
    \item for every $w \in V(H)$ there is $u \in V(G)$ such that $\dist_H(w, \varphi(u)) \leq \beta$.
\end{itemize}
If such a mapping exists, we will also say that $G$ is {\em{$(\alpha,\beta)$-quasi-isometric}} with $H$.

%Roughly speaking, a quasi-isometry between two metric spaces is a mapping that preserves distances up to some bounded factor. In this context $\alpha$ can be thought of as the multiplicative factor and $\beta$ as the additive factor.
%\newline
%\newline
%A \emph{dist-$r$ tree-partition decomposition} of a metric space $(X,d)$ is a pair $\mathcal{T}=(T, \{\beta_x\}_{x\in V(T)})$, where $T$ is a tree whose every vertex $x$ is assigned a subset $\beta_x\subseteq X$ (called a {\it bag}) and the following conditions are fulfilled:
%\begin{itemize}
%    \item sets $\{\beta_x\}_{x\in V(T)}$ form a partition of $X$,
%    \item for each $u,v\in X$ such that $d(x,y)\leq r$ either there exists a $\beta_z$ containing both $u$ and $v$ or there exist $\beta_x$ and $\beta_y$ such that $xy\in E(T)$ and $u\in \beta_x, v\in \beta_y$.
%\end{itemize}
%The \emph{width} of $\mathcal{T}=(T, \{\beta_x\}_{x\in V(T)})$ is the smallest number $k$ such that each bag can be covered by at most $k$ balls of radius $r$.

%This notion for $r=1$ and graphs with the shortest path metric corresponds with the notion of tree-partitions of graphs.
%\newline\newline


%A \emph{tree decomposition} of a metric space $(X,d)$ is a pair $\T=(t,\{\beta_x\}_{x\in V(T)}$, where $T$ is a tree whose every vertex $x$ is assigned a subset $\beta_x$ (called a {\it bag}) and the following condition is fulfilled:
%\begin{itemize}
%    \item for each connected subset $A\subseteq X$ the set $\{x\in V(T)\mid \beta_x\cap A\neq \varnothing\}$ induces a connected subtree in $T$.
%\end{itemize}
%We say that $\T$ has \emph{$r$-width} equal $k$ if for each $x\in V(T)$ the bag $\beta_x$ can be covered with $k$ balls of radius $r$. 
