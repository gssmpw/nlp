\section{Distance graphs}\label{sec:quasi-isometries}
 In this section we describe a generic construction of a quasi-isometry between a graph and its ``coarsening'' with respect to some magnitude of distances. The construction can be considered folklore, see e.g.~\cite[Observation 2.1]{coarse2023} and further references mentioned there, but as we will later use its specific properties, we describe it in details. %We also choose to describe it in the setting of metric spaces for the sake of generality.

 Let $G$ be a graph and $r\in \N_{>0}$. Suppose $I$ is an inclusion-maximal distance-$r$ independent set in~$G$. The {\em{$(I,r)$-distance graph}} of $G$ is the edge-weighted graph $H=H(G,I,r)$ defined as follows:
 \begin{itemize}[nosep]
 \item the vertex set of $H$ is $I$; and
 \item for every two distinct vertices $u,v\in I$ satisfying $\dist(u,v)\leq 3r$, in $H$ we add an edge $uv$ of weight $3r$.
 \end{itemize}
 Thus, all edges of $H$ have weight $3r$. %Note that a priori $H$ may be an infinite graph in case $I$ is infinite, but we will apply the construction only in the context of $(X,d)$ being the distance metric of a finite~graph.
 
 %, we define a graph $H$ with vertex set $I$ and with an edge between two vertices $u, v$ of $I$ if $d(u, v) \leq 3r$. 
 %Each edge of $H$ has weight $3r$.
 %Let $d_H$ be a shortest path metric on graph $H$. Then $(V(H), d_H)$ is a metric space. We will say that $H$ and $(V(H), d_H)$ are \emph{derived} from the tuple $(X,d,I,r)$. In the following we will study the properties of $(V(H), d_H)$. 

%In the following we will slightly abuse the notation, sometimes referring to vertices of $H$ as to elements of $X$. 
%We will also consider the  distance between them according to the metric $d$ or $d_H$.

Let us first note that any distance graph derived from a graph of bounded doubling dimension has bounded maximum degree.

\begin{lemma}\label{clm:bdddim_implies_bddegree}
    Let $G$ be a graph of doubling dimension $m$, $I$ be an inclusion-wise maximal distance-$r$ independent set in $G$ for some $r\in \N_{>0}$, and $H$ be the $(I,r)$-distance graph of $G$. Then $\Delta(H)<2^{3m}$.
\end{lemma}
\begin{proof}
    Consider a vertex $u \in I$ and let $N_H[u]$ be the closed neighborhood of $u$ in $H$, i.e., the set comprising $u$ and all its neighbors. Since two vertices $x,y \in I$ are adjacent in $H$ only if $\dist_G(x,y) \leq 3r$, it follows that $N_H[u] \subseteq \B_G(u, 3r)$. Since $G$ has doubling dimension at most $m$, there are $2^{3m}$ balls $B_1, \hdots, B_{2^{3m}}$ of $X$, each of radius $3r/8$, such that $\B_G(u, 3r) \subseteq \bigcup_{i=1}^{2^{3m}} B_i$. Observe that for each $B_i$, the maximum distance between any two vertices in $B_i$ is at most $6r/8$. Since the vertex set of $H$ is a distance-$r$ independent set in $G$, we have $\dist_G(x,y) > r$ for any distinct $x,y \in N_H[u]$, so each ball $B_i$ contains at most one vertex of $N_H[u]$. As their union contains every vertex of $N_H[u]$, we conclude that $|N_H[u]| \leq 2^{3m}$, hence the degree of $u$ is strictly smaller than $2^{3m}$.
\end{proof}

We now show that every graph is suitably quasi-isometric to any its distance graph.

\begin{lemma}\label{lem:quasi_isometry_to_derived_graph}
    Suppose that $G$ is a graph, $I$ is an inclusion-wise maximal distance-$r$ independent set in $G$ for some $r\in \N_{>0}$, and $H$ is the $(I,r)$-distance graph of $G$. For every $u\in V(G)$, let $\varphi(u)$ be an arbitrary vertex of $I$ such that $\dist_G(u,\varphi(u))\leq r$ (such a vertex exists by the maximality of $I$). Then $\varphi$ is a $(3,3r)$-quasi-isometry from $G$ to $H$.
\end{lemma}
\begin{proof}
    For the second condition of quasi-isometry, observe that for every $w\in V(H)=I$, we have $\dist_G(w,\varphi(w))\leq 3r$. So it remains to  show that the first property is also satisfied. We may assume without loss of generality that $G$ is connected.

    %Let us fix an arbitrary order on $I$ and consider the Voronoi partition of $X$ with respect to the set $I$. Define a mapping $\varphi$ such that every point $x$ of $X$ is mapped to the unique vertex $f$ of $I$ such that $x \in R_V(f)$. Now $\varphi$ is indeed a mapping $\varphi: X \mapsto V(H)$. We show that $\varphi$ is a $(3, 3r)$-quasi-isometry. Since $f \in R_V(f)$ for all $f \in V(H)$, the mapping $\varphi$ is surjective and therefore satisfies the second property. We next show that the first property is also satisfied.


    % \tara{Since $I$ is an inclusion-maximal dist-$r$ independent set of $X$, it follows that $d(v, \varphi(v)) \leq r$ for every $v \in X$. Fix $u, v \in X$. Let $P = d_X(u, v)$ and $Q = d_Y(\varphi(u), \varphi(v))$. Let $q_1 \dd q_2 \dd \hdots \dd q_Q$ be a shortest path in $H$ from $q_1 = \varphi(u)$ to $q_Q = \varphi(v)$. Since two vertices are adjacent in $H$ only if their distance is at most $3r$ in $X$, it follows that there is a curve of length at most $r + 3r \cdot Q + r$ from $u$ to $v$ in $X$. Therefore, $P/3r - 2/3 \leq Q$. %$P \leq 3r\cdot Q + 2r$. 
    % %Next, let $p_1 \dd \hdots \dd p_S$ be the image of a shortest curve in $X$ from $u$ to $v$ under $\varphi$, so $p_1 = \varphi(u)$ and $p_S = \varphi(v)$. 
    % }

    %each edge of H has weight 3r and this is exactly the reason - we want to get rid of r here in the multiplicative factor

    
    %Since $I$ is an inclusion-maximal distance-$r$ independent set, every element of $X$ is at distance at most $r$ from an element of $I$, so $d(v,\varphi(v))\leq r$ for all $x \in X$.

    Consider any $u,v\in V(G)$ and let $P$ be a shortest path connecting $u$ and $v$. Along $P$ we may choose vertices $x_0, x_1, \dots, x_k, x_{k+1}$ in order so that $x_0=u$, $x_{k+1}=v$ and
    \begin{itemize}[nosep]
        \item for each $i\in \{0,1\dots, k-1\}$ we have $\dist_G(x_i, x_{i+1})=r$, and
        \item $\dist_G(x_k, x_{k+1})\leq r$.
    \end{itemize}
    Note that $\sum_{i=0}^k\ \dist_G(x_i, x_{i+1}) = \dist_G(u,v)$ and we have $kr<\dist_G(u,v) \leq (k+1)r$.

    For each $i\in\{0,1,\dots, k+1\}$, let $v_i=\varphi(x_i)\in I$. Note that $\dist_G(x_i,v_i)\leq r$ by construction. 
    We claim that for every $i\in \{0,1,\ldots,k\}$, the vertices $v_i$ and $v_{i+1}$ are adjacent in the graph $H$. 
    Indeed, $\dist_G(v_i, v_{i+1})\leq \dist_G(v_i, x_i) + \dist_G(x_i, x_{i+1}) + \dist_G(x_{i+1}, v_{i+1}) \leq 3r$ by the triangle inequality, so we have $v_iv_{i+1}\in E(H)$. 
    Consequently, $v_0,\ldots,v_{k+1}$ is a walk that connects $\phi(u) = v_0$ and $\phi(v)=v_{k+1}$ in $H$.
    Recalling that each edge in $H$ has weight $3r$, we may now estimate $\dist_H(\varphi(u), \varphi(v))$ as follows:
    %\[\dist_H(\varphi(u), \varphi(v))=\dist_H(x_0, x_{k+1})\leq \sum_{i=0}^k\ \dist_H(x_i, x_{i+1}) \leq 3r\cdot(k+1)\leq 3\cdot \dist_G(u, v) +3r.\]
    \[\dist_H(\varphi(u), \varphi(v))=\dist_H(v_0, v_{k+1})\leq \sum_{i=0}^k\ \dist_H(v_i, v_{i+1}) \leq 3r\cdot(k+1)\leq 3\cdot \dist_G(u, v) +3r.\]
    
    Next, since $\dist_G(x, y) \leq 3r$ for each edge $xy \in E(H)$ and every edge of $H$ has weight $3r$, it follows that $\dist_G(x,y) \leq \dist_H(x,y)$ for all $x,y \in I$. Now, we can estimate $\dist_G(u,v)$:
    \[\dist_G(u,v) \leq \dist_G(u, \varphi(u)) + \dist_G(\varphi(u), \varphi(v)) + \dist_G(\varphi(v), u) \leq r + \dist_H(\varphi(u), \varphi(v)) + r.\]
    Thus, we get the following inequalities:
    \[\dist_G(u,v) - 2r \leq \dist_H(\varphi(u), \varphi(v)) \leq 3\cdot \dist_G(u, v) + 3r,\]
    which means that also the following is true
    \[\tfrac{1}{3}\cdot \dist_G(u,v) - 3r \leq \dist_H(\varphi(u), \varphi(v)) \leq 3\cdot \dist_G(u, v) + 3r.\]
    All in all, we conclude that $\varphi$ is indeed a $(3,3r)$-quasi-isometry.
\end{proof}

