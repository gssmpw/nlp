\section{Experiment}
Given the meticulously built SeaExam and SeaBench, we then conduct experiments to quantitatively demonstrate how our benchmarks could better evaluate models' abilities on multilingual applications from: 1) how our datasets align more closely with the daily usage of regional languages (Section~\ref{subsection is_more_aligned}), and 2) how it effectively distinguishing differences in model performance across various languages (Section~\ref{finding1}) and distinguishing performance variations within the same model across different languages ((Section~\ref{finding2}) and (Section~\ref{finding3})). Through our fine-grained analysis using SeaBench, we have uncovered significant deficiencies in LLMs' response safety across multilingual usage scenarios. Consequently, we advocate for enhanced safety measures in models for multilingual contexts to better adapt to actual usage realities (Section~\ref{finding4})).


\begin{figure*}[!ht]
    \centering
    % Subfigure 1
    \begin{subfigure}[b]{0.48\linewidth}
        \centering
        \includegraphics[width=\textwidth]{latex/figs/cluster_distance/cluster_distance_SeaExam.pdf}
        \caption{}
        \label{subfig:seaexam_example}
    \end{subfigure}
    % \hfill
    \begin{subfigure}[b]{0.48\linewidth}
        \centering
        \includegraphics[width=\textwidth]{latex/figs/cluster_distance/cluster_distance_SeaBench.pdf}
        \caption{}
        \label{subfig:seabench_example}
    \end{subfigure}
    \caption{Cluster distance between each benchmark and Wild Queries. (a) Cluster distance of entity embeddings between each exam dataset and Wild Queries. (b) Cluster distance of sentence embeddings between each multi-turn dataset and Wild Queries.  A smaller value means more similar to Wild Queries.}
    \label{fig:cluster_distance}
\end{figure*}


\subsection{Are the Contructed SeaExam and SeaBench More Aligned with Actual Local Usage?}\label{subsection is_more_aligned}

Despite utilizing local exams and engaging native language experts specifically to tailor questions to the local context, the critical question remains unsolved: How do these questions more accurately reflect the actual local usage compared to those derived from translations? 
To evaluate the alignment of our benchmarks with actual local usage, we conduct a quantitative comparison between SeaExam and SeaBench and real-world user queries. As the first step, we construct the real-world user queries dataset ``Wild Queries'' as follows:

\textbf{Wild Queries} is constructed based on LMSYS-Chat-1M~\cite{zheng_lmsys-chat-1m_2023} and WildChat-1M~\cite{zhao2024wildchat,deng2024wildvisopensourcevisualizer}, which are databases of real-world human queries with millions of conversations across various application scenarios. Using these conversation data, we conducted a meticulous post-filtering process to obtain high-quality queries in SEA languages. First, we conducted 1) \textbf{Language Filter} for the corresponding SEA language using the original language labels and further refined our selection using the Google Translate API to confirm the query language. Given corresponding SEA queries, we have 2) \textbf{Data Balance Control} --- removing overly long conversations, limiting the data to extracting user inputs up to five rounds per conversation, to ensure data balance across different usage scenarios. Finally, we employ a capable multilingual model, GPT-4o, to process 3) \textbf{LLM-Based Heuristic Filter} to further filter out questions that are not queries or instructions.
After these three steps, we get a total of 4,658 queries real-world user queries in SEA languages. The statistic result is shown in Table~\ref{tab:stats_wild_queries} in the appendix.


Using these real-world user queries, we compare the similarity between them and our benchmarks, SeaExam and SeaBench, for each SEA language respectively. Specifically, we utilize the cluster distance (C-Dist) of sentence embeddings derived from the bge-multilingual-gemma2 model~\cite{bge-m3} to measure similarity. We also deploy translated MMLU (MMLU-SEA) and MT-bench (MT-bench-SEA) on SEA languages as baselines (more details on the datasets and the embedding calculation are shown in Appendix \ref{app:exp_details}).


As shown in Figure~\ref{fig:cluster_distance}, \textbf{SeaExam and SeaBench have a more similar distribution with Wild Queries than translated benchmarks}, with a smaller cluster distance by an average of 6 units. This demonstrates that our benchmarks could better evaluate model performance in real-world multilingual application scenarios.


\subsection{Can SeaExam and SeaBench better distinguish models across SEA language?}~\label{subsection better distinguish models}


We have quantitatively demonstrated that the constructed SeaExam and SeaBench benchmarks are more aligned with actual local usage questions (Section~\ref{subsection is_more_aligned}). However, does this greater alignment also improve our ability to distinguish between different models? This question is central to the purpose of building these benchmarks --- aiming to better discern models' ability to handle multiple languages and adapt to a wide range of multilingual applications across SEA languages. To answer the question, we evaluate nine LLMs, a detailed experiment setting as follows:

\textbf{Models:}
We consider multiple factors when selecting nine models for evaluation. First, instruction-following capability is a key requirement, as SeaBench necessitates models that can effectively adhere to given instructions. Second, we select only those with parameters ranging from 7B to 9B, as they offer a good balance between performance and inference speed. Based on these criteria, we select models from three groups: (1) the most popular open-source models, including Meta-Llama-3.1-8B-Instruct (Llama-3.1-8B)\cite{dubey_llama_2024}, Gemma-2-9b-it (Gemma-2-9B)\cite{gemma_team_gemma_2024}, Mistral-7B-Instruct-v0.3 (Mistral-7B)\cite{jiang_mistral_2023}, and Qwen2-7B-Instruct (Qwen2-7B)\cite{yang_qwen2_2024}; (2) models optimized for multilingual capabilities, including glm-4-9b-chat (glm-4-9b)\cite{glm_chatglm_2024} and Aya-23-8B\cite{aryabumi_aya_2024}; and (3) models specifically optimized for Southeast Asian languages, including SeaLLMs-v3-7B-Chat (SeaLLMs-v3-7B)\cite{zhang_seallms_2024}, llama3-8b-cpt-sealionv2-instruct (sealionv2)\cite{sea_lion_2024}, and Sailor-7B-Chat (Sailor-7B)~\cite{dou_sailor_2024}.

\textbf{Metrics and Setups:} 
For SeaExam, we conduct evaluation in 3-shot and use accuracy (\%) as the evaluation metric. For SeaBench, we employ LLMs-as-a-Judge~\cite{zheng_judging_2023, bai2023benchmarkingfoundationmodelslanguagemodelasanexaminer,ying2024automatingdatasetupdatesreliable}, setting GPT-4o as the judge model to evaluate LLM's responses based on the reference answers (construction details in Section~\ref{subsection SeaBench}). Considering that different categories of questions focus on assessing different aspects of model performance, we have designed a list of priority evaluation aspects for each category to facilitate a comprehensive judgment. We prompt GPT-4o to rate each response on a scale from 1 to 10.
These evaluation aspects are detailed in Table~\ref{tab:priority_aspects} and the evaluation prompt is shown in Figure~\ref{fig:prompt_template_turn1} and Figure~\ref{fig:prompt_template_turn2} in the appendix. More experimental and model setups is shown in Appendix \ref{app:eval_setup}.


Following this experimental setup, we conduct tests using SeaExam and SeaBench, with results presented in Table~\ref{tab:main_results}. Upon analyzing these results, we identify several interesting findings as follows:




\section{Appendix: Experimental results}
In this section, we present the results of the comprehensive experimental evaluation, which contains various combinations of query graphs, problem formulations, and classical and quantum optimizers. For each method proposed in this work, we evaluated the technique against five common query graph types: clique, star, chain, cycle, and tree. Each query graph is labeled using the format 'Graph name - number of nodes'. For each graph type and graph size, we randomly selected 20 query graph instances with cardinalities and selectivities. The cardinalities are randomly sampled from the range 10 to 50 and selectivity from interval $(0, 1]$. The costs are summed over 20 runs, describing a realistic cumulative error, and scaled with respect to the cost returned from the dynamic programming algorithm \textit{with} cross-products, which is the optimal left-deep plan. The anonymized code for this experimental evaluation can be found on GitHub \cite{anonymous2024qjoin}. Since we have used 20 query graph instances for five different graph types of sizes 3 to 60 and solved them with four different quantum and classical solvers, the total number of evaluated query graphs is in the thousands.

%After fixing one of the three proposed methods and a query graph instance, we have constructed the corresponding HUBO optimization problem. Then, we have submitted the problem for each selected solver. Depending on the solver's requirements, we might need to translate the HUBO problem into the equivalent QUBO problem. The available solvers include two quantum computing (D-wave quantum annealer, D-wave Hybrid quantum annealer) and two classical approaches (exact poly solver, Gurobi optimizer).

We have decided to focus on the quality of solutions instead of optimization time. Although time is crucial in real-life cases, integrating quantum computational systems with classical systems still brings an unavoidable overhead. Quantum computers work at the time scale of nano and milliseconds, but the classical pre-and post-processing makes the total computation time relatively long in practice. Concretely, these pre-and post-processing phases are demonstrated by such steps as encoding problems in HUBO/QUBO format, submitting them to a quantum computer over a possibly slow connection, and even waiting in line for the quantum computer to become available from the other users. After executing the workload, we need to translate the obtained results back into a format that allows us to interpret them in the light of the original problem.

\textbf{Summary of proposed methods.} We have proposed three algorithms to solve the join order selection problem with a higher-order unconstrained binary optimization model. Table \ref{table:methods} introduces names for these methods, which are used in this section.

\begin{table}[!ht]
\centering
\resizebox{0.5\columnwidth}{!}{
\begin{tabular}{|c|c|c|c|}
\hline
\textbf{Method name} & \textbf{Cost function} & \textbf{Validity constraint} \\ \hline
precise 1 & precise cost function & cost function dependent \\ \hline
precise 2 & precise cost function & cost function independent \\ \hline
heuristic & heuristic cost function & cost function dependent \\ \hline
\end{tabular}
}
\caption{Summary of proposed algorithms}
\label{table:methods}
\end{table}

\subsection{Evaluating \textrm{Precise 1} formulation}

First, we evaluate \textrm{Precise 1} formulation, which combines precise cost function and cost-dependent validity constraints. Fig.~\ref{fig:poly_solver_precise_1} shows the results of optimizing join order selection using the D-Wave's exact poly solver. Following the bounds given by Theorem \ref{thm:dp_bound}, the HUBO model consistently generates a plan that matches the quality of the plan produced by the dynamic programming algorithm without the cross products. We also see that the returned plans are only at most $0.7\%$ bigger than the optimal plan from dynamic programming with the cross products. We have excluded some results where the HUBO model produced the exact optimal plan: Clique-3, Cycle-3, Star-4, and Star-5.

\begin{figure*}[tbh]
    \centering
    \includegraphics[width=\textwidth]{results/precise_1_exact_poly_solver.png}
    \caption{Precise 1 results using the D-Wave's exact poly solver}
    \label{fig:poly_solver_precise_1}
    \Description[Other results]{}
\end{figure*}

% Graph Graph with 3 nodes and 3 edges is not included in the chart.
% costvalues: {'HUBO model': 1.0, 'Graph-DP': 1.0, 'Graph-Greedy': 1.0}
% graph type: clique
% Graph Graph with 3 nodes and 3 edges is not included in the chart.
% costvalues: {'HUBO model': 1.0, 'Graph-DP': 1.0, 'Graph-Greedy': 1.0}
% graph type: cycle
% Graph Graph with 4 nodes and 3 edges is not included in the chart.
% costvalues: {'HUBO model': 1.0000000000000002, 'Graph-DP': 1.0, 'Graph-Greedy': 1.0000000000000002}
% graph type: star
% Graph Graph with 5 nodes and 4 edges is not included in the chart.
% costvalues: {'HUBO model': 1.0000040588404762, 'Graph-DP': 1.000004058840476, 'Graph-Greedy': 1.0000040588404762}
% graph type: star
% Graph Graph with 5 nodes and 4 edges is not included in the chart.
% costvalues: {'HUBO model': 1.0, 'Graph-DP': 1.0, 'Graph-Greedy': 1.0}
% graph type: random

Second, we solved the same HUBO formulations using a classical Gurobi solver after the HUBO problem was translated into the equivalent QUBO problem. The results are presented in Fig.~\ref{fig:gurobi_precise_1}. The HUBO to QUBO translation does not decrease the algorithm's quality, and Gurobi finds the correct plans. The results stay very close to the optimal join tree, always being as good as a dynamic programming algorithm without cross products.

\begin{figure*}[tbh]
    \centering
    \includegraphics[width=\textwidth]{results/precise_1_gurobi.png}
    \caption{Precise 1 results using Gurobi solver}
    \label{fig:gurobi_precise_1}
    \Description[Other results]{}
\end{figure*}

% Graph Graph with 3 nodes and 3 edges is not included in the chart.
% costvalues: {'HUBO model': 1.0, 'Graph-DP': 1.0, 'Graph-Greedy': 1.0}
% graph type: clique
% Graph Graph with 3 nodes and 3 edges is not included in the chart.
% costvalues: {'HUBO model': 1.0, 'Graph-DP': 1.0, 'Graph-Greedy': 1.0}
% graph type: cycle
% Graph Graph with 4 nodes and 3 edges is not included in the chart.
% costvalues: {'HUBO model': 1.0000000000000002, 'Graph-DP': 1.0, 'Graph-Greedy': 1.0000000000000002}
% graph type: star
% Graph Graph with 5 nodes and 4 edges is not included in the chart.
% costvalues: {'HUBO model': 1.0000040588404762, 'Graph-DP': 1.000004058840476, 'Graph-Greedy': 1.0000040588404762}
% graph type: star
% Graph Graph with 4 nodes and 5 edges is not included in the chart.
% costvalues: {'HUBO model': 1.0, 'Graph-DP': 1.0, 'Graph-Greedy': 1.0}
% graph type: random

Third, we solved the same problems using D-wave's Leap Hybrid solver, a quantum-classical optimization platform in the cloud. In this case, the results are consistently as good as those from the dynamic program algorithm without the cross products, with some exceptions due to the heuristic nature of the quantum computer: Cycle-6, Chain-7, and Tree-6.

\begin{figure*}[tbh]
    \centering
    \includegraphics[width=\textwidth]{results/precise_1_dwave_LeapHybridSampler.png}
    \caption{Precise 1 results using D-Wave's Leap Hybrid solver}
    \label{fig:leap_precise_1}
    \Description[Other results]{}
\end{figure*}

% Graph Graph with 3 nodes and 3 edges is not included in the chart.
% costvalues: {'HUBO model': 1.0, 'Graph-DP': 1.0, 'Graph-Greedy': 1.0}
% graph type: clique
% Graph Graph with 3 nodes and 3 edges is not included in the chart.
% costvalues: {'HUBO model': 1.0, 'Graph-DP': 1.0, 'Graph-Greedy': 1.0}
% graph type: cycle
% Graph Graph with 4 nodes and 3 edges is not included in the chart.
% costvalues: {'HUBO model': 1.0000000000000002, 'Graph-DP': 1.0, 'Graph-Greedy': 1.0000000000000002}
% graph type: star
% Graph Graph with 5 nodes and 4 edges is not included in the chart.
% costvalues: {'HUBO model': 1.0000177876393395, 'Graph-DP': 1.000004058840476, 'Graph-Greedy': 1.0000040588404762}
% graph type: star

Finally, Fig.~\ref{fig:dwave_precise_1} shows the results from D-Wave's quantum annealer, which does not utilize hybrid features to increase solution quality. This resulted in performance that did not match the performance of the previous solvers, and this performance decrease was already identified in \cite{Schonberger_Scherzinger_Mauerer}. While quality was not as good as the previous solutions, the valid plans were still usable, mainly only a few percent from the global optimal. %Noise on the machine decreased the quality, and we could not embed many previous problems in the hardware. 

\begin{figure*}[tbh]
    \centering
    \includegraphics[width=\textwidth]{results/precise_1_dwave_DWaveSampler.png}
    \caption{Precise 1 results using D-Wave's standard solver}
    \label{fig:dwave_precise_1}
    \Description[Other results]{Everything is included in these results}
\end{figure*}
\subsection{Evaluating \textrm{Precise 2} formulation}

The key idea behind the Precise 2 formulation is to tackle larger join order optimization cases because the validity constraints are more efficient regarding the number of higher-order terms. We include the exact poly solver results to demonstrate that this formulation encodes precisely the correct plans. For the other solvers, we only show results that optimized larger queries compared to the previous Presice 1 method.

First, the results from the exact poly solver in Fig.~\ref{fig:precise_2_exact_poly_solver} demonstrate that this algorithm follows the bounds of Theorem \ref{thm:dp_bound}. In practice, the returned plans are again very close to the optimal plans. We can also see that compared to the Precise 1 method, the different sets of validity constraints work equally well.

\begin{figure*}[tbh]
    \centering
    \includegraphics[width=\textwidth]{results/precise_2_exact_poly_solver.png}
    \caption{Precise 2 results using the D-Wave's exact poly solver}
    \label{fig:precise_2_exact_poly_solver}
    \Description[Precise 2 results using the D-Wave's exact poly solver]{}
\end{figure*}

% Graph Graph with 3 nodes and 3 edges is not included in the chart.
% costvalues: {'Q-Join: Presice 2': 1.0, 'Dynamic programming w/o cross-products': 1.0, 'Greedy w/o cross-products': 1.0}
% graph type: clique
% Graph Graph with 3 nodes and 3 edges is not included in the chart.
% costvalues: {'Q-Join: Presice 2': 1.0, 'Dynamic programming w/o cross-products': 1.0, 'Greedy w/o cross-products': 1.0}
% graph type: cycle
% Graph Graph with 4 nodes and 3 edges is not included in the chart.
% costvalues: {'Q-Join: Presice 2': 1.0000000000000002, 'Dynamic programming w/o cross-products': 1.0, 'Greedy w/o cross-products': 1.0000000000000002}
% graph type: star
% Graph Graph with 5 nodes and 4 edges is not included in the chart.
% costvalues: {'Q-Join: Presice 2': 1.0000040588404762, 'Dynamic programming w/o cross-products': 1.000004058840476, 'Greedy w/o cross-products': 1.0000040588404762}
% graph type: star

Second, to evaluate the Gurobi solver, we scaled up the problem sizes remarkably from the Precise 1 method, although the experiments were performed on a standard laptop. The results are presented in Fig.~\ref{fig:precise_2_gurobi_1} and Fig.~\ref{fig:precise_2_gurobi_2}. We can see that finding the point that minimizes both cost and validity constraints becomes harder when the problem sizes increase. %The results remain close to optimal, with cost summed over multiple runs. However, even one suboptimal round can significantly impact the total cost and accuracy, so the results demonstrate solid performance considering the demanding evaluation setup.

\begin{figure*}[tbh]
    \centering
    \includegraphics[width=\textwidth]{results/precise_2_gurobi_1.png}
    \caption{Precise 2 results using Gurobi solver}
    \label{fig:precise_2_gurobi_1}
    \Description[Precise 2 results using Gurobi solver]{}
\end{figure*}

% Graph Graph with 3 nodes and 3 edges is not included in the chart.
% costvalues: {'Q-Join: Presice 2': 1.0, 'Dynamic programming w/o cross-products': 1.0, 'Greedy w/o cross-products': 1.0}
% graph type: clique
% Graph Graph with 3 nodes and 3 edges is not included in the chart.
% costvalues: {'Q-Join: Presice 2': 1.0, 'Dynamic programming w/o cross-products': 1.0, 'Greedy w/o cross-products': 1.0}
% graph type: clique
% Graph Graph with 3 nodes and 3 edges is not included in the chart.
% costvalues: {'Q-Join: Presice 2': 1.0, 'Dynamic programming w/o cross-products': 1.0, 'Greedy w/o cross-products': 1.0}
% graph type: cycle
% Graph Graph with 3 nodes and 3 edges is not included in the chart.
% costvalues: {'Q-Join: Presice 2': 1.0, 'Dynamic programming w/o cross-products': 1.0, 'Greedy w/o cross-products': 1.0}
% graph type: cycle
% Graph Graph with 4 nodes and 3 edges is not included in the chart.
% costvalues: {'Q-Join: Presice 2': 1.0000000000000002, 'Dynamic programming w/o cross-products': 1.0, 'Greedy w/o cross-products': 1.0000000000000002}
% graph type: star
% Graph Graph with 5 nodes and 4 edges is not included in the chart.
% costvalues: {'Q-Join: Presice 2': 1.0000040588404762, 'Dynamic programming w/o cross-products': 1.000004058840476, 'Greedy w/o cross-products': 1.0000040588404762}
% graph type: star
% Graph Graph with 6 nodes and 5 edges is not included in the chart.
% costvalues: {'Q-Join: Presice 2': 1.0000021564508312, 'Dynamic programming w/o cross-products': 1.0000011116483463, 'Greedy w/o cross-products': 1.0000011116483465}
% graph type: star
% Graph Graph with 4 nodes and 3 edges is not included in the chart.
% costvalues: {'Q-Join: Presice 2': 1.0000000000000002, 'Dynamic programming w/o cross-products': 1.0, 'Greedy w/o cross-products': 1.0000000000000002}
% graph type: star
% Graph Graph with 5 nodes and 4 edges is not included in the chart.
% costvalues: {'Q-Join: Presice 2': 1.0000040588404762, 'Dynamic programming w/o cross-products': 1.000004058840476, 'Greedy w/o cross-products': 1.0000040588404762}
% graph type: star

\begin{figure*}[tbh]
    \centering
    \includegraphics[width=\textwidth]{results/precise_2_gurobi_2.png}
    \caption{Precise 2 results using Gurobi solver}
    \label{fig:precise_2_gurobi_2}
    \Description[Precise 2 results using Gurobi solver]{}
\end{figure*}

Slightly unexpectedly, the Leap Hybrid solver did not perform as well as we expected, as shown in Fig.~\ref{fig:precise_2_dwave_LeapHybridSampler}. The solver does not have tunable hyperparameters, which we would be able to adjust to obtain better results. On the other hand, we used the developer access to the solver, which is limited to only one minute of quantum computing access per month. Finally, we did not include the results from the D-wave quantum solver due to space limitations since the solver did not scale to these cases. %It provided new results only for the Star-6 query graph with $7.07\%$ larger plans over 20 query graphs than the optimal ones.

\begin{figure*}[tbh]
    \centering
    \includegraphics[width=\textwidth]{results/precise_2_dwave_LeapHybridSampler.png}
    \caption{Precise 2 results using D-Wave's Leap Hybrid solver}
    \label{fig:precise_2_dwave_LeapHybridSampler}
    \Description[Precise 2 results using D-Wave's Leap Hybrid solver]{}
\end{figure*}

% Graph Graph with 3 nodes and 3 edges is not included in the chart.
% costvalues: {'Q-Join: Presice 2': 1.0, 'Dynamic programming w/o cross-products': 1.0, 'Greedy w/o cross-products': 1.0}
% graph type: clique
% Graph Graph with 3 nodes and 3 edges is not included in the chart.
% costvalues: {'Q-Join: Presice 2': 1.0, 'Dynamic programming w/o cross-products': 1.0, 'Greedy w/o cross-products': 1.0}
% graph type: clique
% Graph Graph with 3 nodes and 3 edges is not included in the chart.
% costvalues: {'Q-Join: Presice 2': 1.0, 'Dynamic programming w/o cross-products': 1.0, 'Greedy w/o cross-products': 1.0}
% graph type: cycle
% Graph Graph with 3 nodes and 3 edges is not included in the chart.
% costvalues: {'Q-Join: Presice 2': 1.0, 'Dynamic programming w/o cross-products': 1.0, 'Greedy w/o cross-products': 1.0}
% graph type: cycle
% Graph Graph with 4 nodes and 3 edges is not included in the chart.
% costvalues: {'Q-Join: Presice 2': 1.0000000000000002, 'Dynamic programming w/o cross-products': 1.0, 'Greedy w/o cross-products': 1.0000000000000002}
% graph type: star
% Graph Graph with 5 nodes and 4 edges is not included in the chart.
% costvalues: {'Q-Join: Presice 2': 1.0000141364642996, 'Dynamic programming w/o cross-products': 1.000004058840476, 'Greedy w/o cross-products': 1.0000040588404762}
% graph type: star
% Graph Graph with 6 nodes and 5 edges is not included in the chart.
% costvalues: {'Q-Join: Presice 2': 1.0000247692741169, 'Dynamic programming w/o cross-products': 1.0000011116483463, 'Greedy w/o cross-products': 1.0000011116483465}
% graph type: star
% Graph Graph with 4 nodes and 3 edges is not included in the chart.
% costvalues: {'Q-Join: Presice 2': 1.0000000000000002, 'Dynamic programming w/o cross-products': 1.0, 'Greedy w/o cross-products': 1.0000000000000002}
% graph type: star
% Graph Graph with 5 nodes and 4 edges is not included in the chart.
% costvalues: {'Q-Join: Presice 2': 1.0000177876393395, 'Dynamic programming w/o cross-products': 1.000004058840476, 'Greedy w/o cross-products': 1.0000040588404762}
% graph type: star
\subsection{Evaluating heuristic formulation}

The key motivation behind the heuristic formulation is to tackle even larger query graphs. Our main goal is to demonstrate that this algorithm reaches acceptable results with superior scalability compared to the previous Precise 1 and 2 formulations. The results also indicate that Theorem \ref{thm:greedy_bound} is respected in practice. The optimal results are computed with dynamic programming without cross products. Due to space limitations, we only included the results from the Gurobi solver, which we consider the most demonstrative, and we had unlimited access to it since it runs locally. 

The results are presented so that we have computed and scaled the difference between each pair of methods. A value that differs from 0 indicates that the two methods gave different join trees with different costs. Since one of the methods is near-optimal (DP without cross products), it is clear which method produced the suboptimal result. This way, we can compare all three methods at the same time. In all cases, we can see that the heuristic algorithm respects Theorem \ref{thm:greedy_bound} very well in practice, so the difference between quantum and greedy is always $0$.

Fig.~\ref{fig:clique_accuracies} shows the results of applying the heuristic method to clique query graphs. Although these results are good, the scalability in this hard case is modest. On the other hand, we are unaware of any quantum computing research that would have outperformed this scalability in the case of clique graphs. For example, the most scalable method \cite{10.14778/3632093.3632112} excluded clique graphs from their results.

\begin{figure}
    \centering
    \includegraphics[width=\linewidth]{results/clique_accuracies.png}
    \caption{Heuristic results for clique query graphs using Gurobi solver}
    \label{fig:clique_accuracies}
\end{figure}

The results for the tree (Fig.~\ref{fig:tree_accuracies}), chain (Fig.~\ref{fig:chain_accuracies}), cycle (Fig.~\ref{fig:cycle_accuracies}), and star graphs demonstrate the best scalability. We computed the results up to 60 tables to demonstrate advantageous scalability over the most scalable method in the previous research \cite{10.14778/3632093.3632112} where they considered queries up to 50 relations.

\begin{figure}
    \centering
    \includegraphics[width=\linewidth]{results/tree_accuracies.png}
    \caption{Heuristic results for tree query graphs using Gurobi solver}
    \label{fig:tree_accuracies}
\end{figure}

\begin{figure}
    \centering
    \includegraphics[width=\linewidth]{results/chain_accuracies.png}
    \caption{Heuristic results for chain query graphs using Gurobi solver}
    \label{fig:chain_accuracies}
\end{figure}
    
\begin{figure}
    \centering
    \includegraphics[width=\linewidth]{results/cycle_accuracies.png}
    \caption{Heuristic results for cycle query graphs using Gurobi solver}
    \label{fig:cycle_accuracies}
\end{figure}

We exclude the results for the star query graphs because, in this case, all three methods performed identically across up to 60 graphs and over 20 iterations, with no difference observed (a relative scaled difference of 0). These results may be because star graphs typically do not benefit from cross products \cite{10.14778/3632093.3632112}, and our method, which excludes them, performs better with such types of queries.


\subsubsection{Finding 1: SeaExam and SeaBench can better distinguish different models} \label{finding1}
\begin{figure}[htb]
    \centering
    % Subfigure 1
    \begin{subfigure}[b]{0.48\linewidth}
        \centering
        \includegraphics[width=\textwidth]{latex/figs/std_benchmarks/std_seaexam_mmlu_v2.pdf}
        \caption{}
        \label{subfig:seaexam_example}
    \end{subfigure}
    % \hfill
    \begin{subfigure}[b]{0.48\linewidth}
        \centering
        \includegraphics[width=\textwidth]{latex/figs/std_benchmarks/std_seabench_mtbench_v2.pdf}
        \caption{}
        \label{subfig:seabench_example}
    \end{subfigure}
    \caption{(a) Accuracy standard deviation across the nine models for each language on SeaExam and MMLU-SEA. (b) Score standard deviation across the nine models for each language  on SeaBench and MT-bench-SEA.}
    \label{fig:finding1}
\end{figure}


We compare the performance of tested models between SeaExam and MMLU-SEA, examining the standard deviation of model performances across three SEA languages. Results, as shown in Figure~\ref{fig:finding1}, indicate that the variances in SeaExam are significantly higher than those in MMLU-SEA by 9.3\%.
A similar phenomenon was observed when comparing SeaBench with MT-bench-SEA by 8.7\%.
This consistency suggests that, compared to direct translations, our benchmarks more effectively discern the capabilities of models in real-world application scenarios.


In Figure~\ref{fig:finding1}, we find the abnormal phenomenon that SeaExam has no distinct advantage in differentiating among models for the Indonesian language. 
This may be due to the poor performance across the models on Indonesian, each showing a decline of more than 4.5\% compared to MMLU-SEA, resulting in a lower standard deviation in differentiation.
This observation prompts us to explore further whether the ability to effectively separate models extends to aiding in a more nuanced analysis across different languages. 

% \input{latex/tables/corr_SeaExam_SeaBench}

\subsubsection{Finding 2: SeaBench can better distinguish performance variations within the same model across different languages} \label{finding2}



\begin{figure}[htb]
    \centering
    % Subfigure 1
    \begin{subfigure}[b]{0.48\linewidth}
        \centering
        \includegraphics[width=\textwidth]{latex/figs/std_benchmarks/std_lang_seaexam_mmlu_v2.pdf}
        \caption{}
        \label{subfig:seaexam_example}
    \end{subfigure}
    % \hfill
    \begin{subfigure}[b]{0.48\linewidth}
        \centering
        \includegraphics[width=\textwidth]{latex/figs/std_benchmarks/std_lang_seabench_mtbench.pdf}
        \caption{}
        \label{subfig:seabench_example}
    \end{subfigure}
    \caption{(a) Accuracy standard deviation across three SEA languages for the nine models on SeaExam and MMLU-SEA. (b) Score standard deviation across three SEA languages for the nine models on SeaBench and MT-bench-SEA.}
    \label{fig:finding2}
\end{figure}


We conduct a comparison of nine models' performance standard deviations on SeaExam across three SEA languages and compared these with performances on MMLU-SEA. As shown in Figure~\ref{fig:finding2}, SeaExam does not demonstrate a significant advantage in distinguishing language differences. In contrast, a notable distinction emerges when comparing SeaBench to MT-Bench. Specifically, the performance gaps across the three languages in SeaBench are significantly larger than those in the translated MT-bench-SEA, by 6.7\% on average, indicating that SeaBench more effectively highlights the performance variations within the same model across different languages. Additionally, we identified a few models, such as Sailor-7B, SeaLLMs-v3-7B, and Sealionv2, that exhibited more balanced performances across SEA languages in SeaBench. This is because these models were specifically trained with a focus on SEA daily scenarios, which resulted in a more balanced performance on SEA language tests.



Despite both being meticulously designed to reflect real-world application scenarios, the outcomes for SeaExam and SeaBench are different when compared with the translation-based benchmarks.
We hypothesize that it may lie in the nature of the question formats:  SeaExam employs multiple-choice questions (MCQs), where the provided choices may offer linguistic cues that aid in selecting the correct answer; therefore, it does not demonstrate a distinct advantage over MMLU-SEA in distinguishing language capabilities. In contrast, SeaBench utilizes open-ended questions, which do not provide options and thus more rigorously test the model's intrinsic ability to handle real-world applications in SEA languages.
To further validate our hypothesis, we conducted an in-depth analysis, which led to our third finding.

\subsubsection{Finding 3: Open-Ended Question Formats More Effectively Distinguish Model Capabilities} \label{finding3}

\begin{figure}[htb]
    \centering
    % Subfigure 1
    \begin{subfigure}[b]{0.48\linewidth}
        \centering
        \includegraphics[width=\textwidth]{latex/figs/std_benchmarks/std_seaexam_seabench.pdf}
        \caption{}
    \end{subfigure}
    % \hfill
    \begin{subfigure}[b]{0.48\linewidth}
        \centering
        \includegraphics[width=\textwidth]{latex/figs/std_benchmarks/std_lang_seaexam_seabench.pdf}
        \caption{}
    \end{subfigure}
    \caption{(a) Accuracy standard deviation across the models for each language on SeaExam and SeaBench. (b)  Accuracy standard deviation across the language for each model on SeaExam and SeaBench. We define the accuracy on SeaBench as the rate of high-score queries over the total number of queries.}
    \label{fig:find3}
\end{figure}


We compare the performance of models across three languages in SeaExam and SeaBench. Since SeaExam employs accuracy (\%) as its metric and SeaBench uses scores from a judge model, the scoring methods are not directly comparable. To standardize the evaluation, we converted the latter’s scores to accuracy rates and full mark rates (where a response is considered correct only if it achieves full marks on all aspects). The results, depicted in Figure~\ref{fig:find3}, reveal that the deviations among the nine models across the three languages are greater in SeaBench compared to SeaExam by 1.37 times. This observation supports our earlier hypothesis that open-ended question formats, requiring more extensive language use, better highlight differences in model capabilities. 


\subsubsection{Finding 4: LLMs Perform Poorly on Safety Questions} \label{finding4}
Through extensive experimental analysis, we have demonstrated that our benchmarks more effectively evaluate models' abilities in real-world multilingual applications. Building on this, we conduct a fine-grained analysis, with the results for SeaBench shown in Figure~\ref{fig:seabench_mean_cate}. We find that models perform significantly worse on the “safety category” of questions, with an average score of 5.02, which is 20\% lower than the highest-performing “STEM category”. These questions assess the model’s ability to avoid generating harmful responses. This finding highlights a notable deficiency in the models' safety performance in relevant usage scenarios. We speculate that most alignment efforts are conducted using data on the models' primary languages and overlooking other multilingual application contexts. Consequently, \textbf{we advocate for enhanced safety measures in models for multilingual contexts to better adapt to actual usage}.


\begin{figure}[htb]
    \centering
    \includegraphics[width=0.8\linewidth]{latex/figs/std_benchmarks/seabench_mean_cate_v2.pdf}
    \caption{The average scores of the nine LLMs on 8 categories of SeaBench. The models performs poorly on the safety questions.}
    \label{fig:seabench_mean_cate}
\end{figure}


\section{Human Evaluation}

\begin{table*}[!ht]
    \centering
    \small
    \begin{tabular}{llllllllll}
        \toprule
         \multirow{2}{*}{Judge model}& \multicolumn{4}{c}{With tie votes (R = 33.3\%)} & & \multicolumn{4}{c}{Without tie votes (R = 50\%)} \\ 
         \cmidrule{2-5} \cmidrule{7-10}
        & id & th & vi & avg & & id & th & vi & avg \\ 
        \midrule
        gpt-4o & \textbf{67.3\%} & \textbf{68.7\%} & 58.7\% & \textbf{64.9\%} & &91.3\% & 95.8\% & 86.7\% & 91.3\% \\ 
        claude-3.5-sonnet & 64.2\% & 67.1\% & \textbf{58.8\%} & 63.4\% & &\textbf{92.3\%} & 95.8\% & \textbf{88.4\%} & \textbf{92.2\%} \\ 
        gemini-pro-1.5 & 59.2\% & 64.6\% & 55.0\% & 59.6\% & &87.1\% & 94.0\% & 87.9\% & 89.7\% \\ 
        gpt-4o-mini & 59.8\% & 64.8\% & 56.5\% & 60.4\% & &91.3\% & \textbf{96.2\%} & 86.6\% & 91.4\% \\ 
        claude-3-haiku & 50.8\% & 53.3\% & 47.5\% & 50.6\% & &89.3\% & 94.0\% & 82.2\% & 88.5\% \\ 
        gemini-flash-1.5 & 60.5\% & 62.5\% & 60.0\% & 61.0\% & &91.4\% & 95.2\% & 86.3\% & 91.0\% \\ 
        \midrule
        Ensemble & 66.2\% & 70.6\% & 60.3\% & 65.7\% &  & 91.8\% & 96.5\% & 90.9\% & 93.1\% \\ 
        \bottomrule
    \end{tabular}
    \caption{Agreement between human evaluators and six judge models on SeaBench. The agreement between two random judges in each setup is denoted as “R=”. For the judge models, a tie is recorded if two scores differ by 1 or less.}
    \label{tab:seabench_agreement_thres_1}
\end{table*}
% \vspace{-3mm}

For both constructed benchmarks, SeaExam and SeaBench, each question and its corresponding reference answer are meticulously crafted by engaged native linguists, ensuring high quality. To further validate the reliability of our experimental results—particularly the evaluation scores assigned by GPT-4o for SeaBench—we conduct a human agreement evaluation. For each question, we randomly sample three distinct model pairs, ensuring that no model combination is repeated. Since SeaBench consists of 100 questions per language, each linguist evaluates 300 model pairs. As each language involves two turns, this approach results in a total of 600 votes per language. 


Annotators judge which of the two models produces a better response. If both responses are equally good, the result is marked as a tie. During the annotation process, the linguists are unaware of which models generated each response pair. The instructions for the human judges are provided in Figure \ref{fig:human_instruction} in the appendix. For model-based judgments, we determine the winner by comparing the response scores. To ensure a more balanced distribution of labels, we treat responses as ties if their scores differ by 1 point or less, as the model scores range from 1 to 10. Finally, we compare the human-generated votes with the model-derived votes to assess the level of agreement between them.


Results in Table~\ref{tab:seabench_agreement_thres_1} show that GPT-4o has a high agreement with human evaluations—64.9\% on average (with tie votes) and 91.3\% (without tie votes). In comparison, \citet{zheng_judging_2023} report 65\% agreement for human evaluators on MT-bench when including tie votes and 81.5\% when excluding them. This suggests that GPT-4o’s judgments align well with human preferences on SeaBench, confirming the reliability of our findings.


In addition to evaluating the results using GPT-4o as the judge in our experiment (more details in Section~\ref{subsection better distinguish models}), we expand our evaluation to include more judges, including GPT-4o-mini, Claude-3.5-Sonnet, Claude-3-Haiku, Gemini-Pro-1.5, and Gemini-Flash-1.5 and assess their results. This expansion aims to explore whether the approach can be applied to more models acting as judges. Considering that relying solely on GPT-4o might introduce biases, such as self-preference, especially when employing the LLMs-as-a-Judge approach, using different models helps mitigate the bias associated with exclusively using one judge~\cite{bai2023benchmarkingfoundationmodelslanguagemodelasanexaminer, ying2024automatingdatasetupdatesreliable, zhao2024auto}.
The result is shown in Table~\ref{tab:seabench_agreement_thres_1}. More details on the experimental setup and results are discussed in Appendix \ref{app:agreement}. 
