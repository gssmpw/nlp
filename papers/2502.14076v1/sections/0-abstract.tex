%The proliferation of latency-critical and compute-intensive edge-native applications is driving an increase in computing demands at the edge. Consequently, this surge in demand is expected to lead to high carbon emissions from edge computing. Until recently, regional carbon intensity data at the city or zone level were opaque, hindering efforts to reduce edge carbon emissions. Upon analyzing the new carbon intensity data within small geographical regions, we confirm the feasibility of reducing carbon emissions at the edge by distributing workloads across geographically dispersed edge regions (spatial shifting). Building on this finding, we introduce a carbon-aware control plane for edge computing, which optimizes carbon efficiency while meeting latency and resource demands. Finally, we showcase our control plane for task offloading using a three-tier orchestration platform. Empirical results on a real testbed, demonstrate that our carbon-aware placement approach, can reduce carbon emission by 1518 \emissionunit, while only adding 6.6 more latency. 
 
%%This surge is expected to result in high carbon emissions in edge computing. 
%%To address this challenge, we first 
%% Our analysis confirms that the carbon benefits of distributing workloads across geographically dispersed edge sites outweigh the associated latency overhead.  
%%Expanding on this finding,  we propose \proposedsystem, a carbon-aware control plane for edge computing. 

%%This system enhances carbon awareness across edge sites utilizing the fine-grain carbon intensity traces and minimizes carbon emissions through optimized workload placements. 


The proliferation of latency-critical and compute-intensive edge applications is driving increases in computing demand and carbon emissions at the edge. To better understand carbon emissions at the edge, we analyze granular carbon intensity traces at intermediate "mesoscales," such as within a single US state or among neighboring countries in Europe, and observe significant variations in carbon intensity at these spatial scales. Importantly, our analysis shows that carbon intensity variations, which are known to occur at large continental scales (e.g., cloud regions), also occur at much finer spatial scales, making it feasible to exploit geographic workload shifting in the edge computing context.
Motivated by these findings, we propose \proposedsystem, a carbon-aware framework for edge computing that optimizes the placement of edge workloads across mesoscale edge data centers to reduce carbon emissions while meeting latency SLOs. We implement \proposedsystem and evaluate it on a real edge computing testbed and through large-scale simulations for multiple edge workloads and settings. Our experimental results on a real testbed demonstrate that \proposedsystem can reduce emissions by up to 78.7\% for a regional edge deployment in central Europe. Moreover, our CDN-scale experiments show potential savings of 49.5\% and 67.8\% in the US and Europe, respectively, while limiting the one-way latency increase to less than 5.5 ms.

%when considering edge centers within the Central EU region. In addition, our CDN-scale experiments across Europe show that \proposedsystem reduces carbon emissions by 75\% while only incurring a 17.2 ms increase in round-trip latency.

%Our results demonstrate that \proposedsystem yields 1) 98.4\% carbon savings over a \textit{carbon-agnostic} placement policy  2) 78\% carbon savings over an \textit{energy-aware} placement policy, and 3) 75\% carbon savings over a state-of-the-art \textit{carbon-aware} policy.
%\todo{Fill with new results.}

% We evaluate \proposedsystem on real testbeds and large-scale simulations using various edge applications in real-world settings. 
% Our results demonstrate that \proposedsystem can simultaneously achieve i) * carbon reductions compared to carbon-agnostic policy with only * additional latency, and ii) * improvement over a state-of-the-art carbon-aware policy, all while meeting all latency constraints. 

% Until recently, the lack of fine-grain carbon intensity data in small-spatial scale areas has made it challenging to effectively address and mitigate these emissions. 

%reduce edge carbon emissions. 
% However, with the availability of more granular data, the analysis of carbon intensity within small geographical regions reveals the feasibility of reducing carbon emissions at the edge by distributing workloads across geographically dispersed sites, enabling an important opportunity for carbon-aware edge computing.


% Capitalizing on this opportunity, we propose \proposedsystem, a carbon-aware orchestration framework for deploying edge workloads across geographically dispersed edge sites. 
% \proposedsystem extends carbon-aware policies to accommodate different edge workloads and heterogeneous edge servers, aiming to minimize regional carbon emissions and latency. 
% We evaluate \proposedsystem on real testbeds and large-scale simulations using various edge applications in real-world settings. 
% Our results demonstrate that \proposedsystem can simultaneously achieve i) * carbon reductions compared to carbon-agnostic policy with only * additional latency, and ii) * improvement over a state-of-the-art carbon-aware policy, all while meeting all latency constraints. \lilly{Need to improve after getting all results}


%However, with more granular data available along with an analysis of the carbon intensity data within small geographical regions reveals the feasibility of reducing carbon emissions at the edge by distributing workloads across geographically dispersed edge sites. 
%Upon analyzing the fine-grain carbon intensity data within small geographical regions, we confirm the feasibility of reducing carbon emissions at the edge by distributing workloads across geographically dispersed edge sites. 
% Building on this finding, we propose \proposedsystem,  a carbon-aware edge-orchestration framework for placing edge workloads across geographically dispersed edge sites. \proposedsystem extends the carbon-aware policy to different edge workloads and heterogeneous edge servers, minimizing regional carbon emissions and latency. We implement a \proposedsystem prototype on top of an open-sourced edge-orchestration framework, named Sinfonia. We then evaluated \proposedsystem on real testbeds and large-scale simulations using multiple edge applications. The results show \proposedsystem can yield 1) * carbon reduction over carbon-agnostic execution while adding * more latency; 2) * over a state-of-the-art carbon-aware policy.
%We implement a \proposedsystem prototype on top of an open-sourced edge-orchestration framework, named Sinfonia. We then evaluated \proposedsystem on real testbeds and large-scale simulations using multiple edge applications. The results show \proposedsystem can yield 1) * carbon reduction over carbon-agnostic execution while adding * more latency; 2) * over a state-of-the-art carbon-aware policy.
%



% we utilize the Sinfonia edge-orchestration framework to implement the carbon-aware policy and showcase carbon reductions for task offloading scenarios. The evaluation on a real testbed and year-long simulations demonstrate that compared to a location-aware baseline, our carbon-aware policy can reduce daily carbon emission by 1518 \emissionunit while only adding 6.6ms more latency.






