Data centers consumed more than 460 terawatt-hours (TWh) of energy in 2022, and are expected to consume more than 1000 TWh by 2026~\cite{iea2024electricity}. As a result, data centers are already generating roughly 1\% of global carbon emissions and could emit more than 2.5 billion metric tons of $\text{CO}_2$ by the end of the decade. The sustainable growth of data center capacity has emerged as a critical challenge in our society's transition to a low-carbon future, especially with the accelerating build-out of data center capacity to satisfy the growing demand for AI workloads. Historically, cloud operators have addressed data center sustainability issues by optimizing their energy efficiency (i.e., their computational work done per unit of energy consumed).  However, optimizing energy-efficiency alone will likely not be sufficient to satisfy cloud platforms' carbon emissions targets~\cite{Bashir2022:HotAir}.  In particular, since data center energy-efficiency is already highly optimized after years of research, further optimizations are expected to yield diminishing marginal improvements moving forward. 

%However, recent research has highlighted optimizing energy-efficiency is not sufficient to cope with the high increases in cloud workloads and the sustainability goals of cloud operators.

As a result, researchers have recently focused on several alternative approaches for reducing carbon footprint of cloud data centers and improving their carbon efficiency (i.e., their computational work done per unit of carbon emitted). Specifically, to optimize hyperscale data centers' carbon efficiency, cloud operators have deployed both supply- and demand-side approaches.  On the supply-side, cloud operators have procured green energy (e.g., wind or solar) through long-term contracts to power their data center operations \cite{google_offshore_wind_2024}, while on the demand-side, researchers have explored techniques for modulating data center workload demand and its resulting carbon emissions to optimize their carbon footprint \cite{wait-awhile}. Since the electric grid in different regions uses different mixes of generation sources, grids with a higher penetration of low-carbon sources, such as hydro, solar, or wind, tend to produce lower-carbon electricity. Spatial workload shifting approaches exploit these regional differences in energy's carbon intensity by proactively shifting workloads to data center locations with lower-carbon energy, thereby performing the same computation while incurring fewer emissions. Recent research has shown that cloud workloads, such as machine learning training and batch processing, are amenable to such spatial shifting optimizations and can yield significant reductions in applications' carbon footprint~\cite{sukprasert2024limitations, cloudcarbon, Gsteiger2024:Caribou, Murillo2024:CDNShifter}. However, these optimizations typically incur large network delays to migrate workloads over long distances to a different data centers, and thus can increase user latency for interactive workloads that are latency-sensitive.

Since spatial differences in the grid's carbon intensity are clearly evident over large geographical distances, spatial shifting has largely been studied for cloud workloads at continental scales, i.e., across entire continents or between continents.  For example, shifting workloads from eastern to western North America, or shifting workloads from North America or Asia to Europe. The differences in grid carbon intensity at these scales are due to the vastly different generation mixes at distant locations. { \em As a result, conventional wisdom has held that spatial workload shifting is unsuitable for edge data centers, since moving edge workloads over such long distances to distant edge data centers would result in unacceptable increases in the latency of edge applications.} Consequently, edge data centers thus far have not leveraged this key carbon optimization technique. 

In this paper, we challenge this conventional wisdom and show that spatial workload shifting is a feasible carbon-optimization approach for edge data centers deployed in many, although not all, parts of the world. Our key insight is that spatial differences in grid carbon intensity do frequently occur even at ``mesoscales'' (i.e., smaller distances of tens to a few hundred kilometers), especially as the penetration of wind and solar renewables continues to grow.  While, on average, variations in carbon intensity are certainly larger at longer distances, there are meaningful differences in energy's carbon intensity at short distances in many parts of the world.  Such mesoscale differences open up new opportunities for spatial workload shifting across nearby edge data centers, enabling edge workloads to optimize their carbon footprint with limited performance impact on latency-sensitive applications. In contrast, as mentioned above, prior work has focused primarily on exploiting spatial differences in energy's carbon intensity at large continental scales, i.e., across a thousand kilometers or more, where variations in energy's carbon intensity arise from large environmental differences.  For example, at continental scales, it may be daytime in one location with plentiful solar generation and nighttime in another with zero solar generation. Instead, mesoscale differences in carbon intensity generally arise from differences in a location's specific mix of various generation sources, e.g., hydro, coal, natural gas, oil, solar, wind, nuclear, etc., and types of generators.  For example, a municipal utility that serves a small town may have its own low-carbon hydro-generating plant, while nearby towns are served by a private utility that generates most of its power from high-carbon fossil fuels. Such differences give rise to variations in energy's carbon intensity, even at relatively short distances. 

%%\todo{Mention the fancy things that David said :D}

%\todo{What are the challenges and the trade-offs? We have carbon-latency trade-offs and carbon-energy trade-offs (only pops up when we use heterogeneity)}

Motivated by these observations, this paper presents \proposedsystem, a carbon-aware orchestration framework for distributed edge data centers that supports spatial workload shifting at mesoscales. \proposedsystem optimizes workload placement to significantly reduce carbon emissions of edge applications within a mesoscale region while satisfying latency constraints. 
%As a result, \proposedsystem balances the benefit of shifting workload further away to utilize lower carbon energy with the cost of an edge application's increased latency. 
Importantly, \proposedsystem considers the diversity of edge applications and resource heterogeneity when determining workload placement, which affects how applications consume energy at specific locations. This aspect is crucial because the carbon emissions of applications depend on both their energy consumption and the carbon intensity of the energy used. We hypothesize that 
small spatial-scale variations in carbon intensity can enable 
\proposedsystem to reduce the operational carbon footprint of edge applications without significantly impacting their low-latency benefits.
%can reduce the carbon emissions of latency-sensitive edge applications by leveraging mesoscale variations in energy's carbon intensity while satisfying latency SLOs. 
In designing, implementing, and evaluating \proposedsystem, we make the following contributions. 

\begin{enumerate}[leftmargin=*]
    \item \textbf{Mesoscale Carbon Analysis.} Our analysis is the first to demonstrate significant variations in grid carbon intensity at mesoscale distances, thereby making it feasible to deploy workload shifting optimizations in edge computing platforms.
    %To the best of our knowledge, we are the first to identify significant variations in grid carbon intensity over mesoscale distances, indicating the feasibility of using spatial workload shifting in edge data centers without violating the strict latency SLOs. 
    We present a detailed empirical analysis of the granular carbon intensity data and latency traces of 148 regions in the world (\autoref{sec:carbon_analysis}).

    % Using real-world grid carbon intensity data from 148 regions across the world, we conduct a detailed empirical analysis to show that there can be significant variations in grid carbon intensity at shorter mesoscale distances of tens to hundreds of kilometers.  We combine these traces with empirical latency data to show that shifting workload at such mesoscales results in a small latency overhead, which highlights the potential for exploiting mesoscale differences for edge computing workloads placement (\autoref{sec:carbon_analysis}).
    
    % We analyze real-world fine-grain carbon intensity and latency trances across small geographical zones of regions, including small regions with 5 zones and a global region with 173  zones. The analysis highlights the potential of placing workloads across dispersed edge sites with substantial carbon savings and negligible latency overheads (\autoref{sec:carbon_analysis}). 

   \item \textbf{\proposedsystem Design and Implementation.} Based on our findings, we propose \proposedsystem, a carbon-aware placement framework to reduce carbon emissions from edge data centers at mesoscales. \proposedsystem integrates carbon intensity variations across edge data center locations and accounts for energy-efficiency differences among heterogeneous resources to intelligently distribute edge workloads to minimize carbon emissions (\autoref{sec:design}). Additionally, we implement a full prototype of a carbon-aware edge orchestration framework on top of Sinfonia, a Kubernetes-based framework for edge data centers, and plan to release it as open source (\autoref{sec:implementation}). 
   
   % We propose a carbon-aware edge placement framework, called \proposedsystem, that minimizes carbon emissions from geographically distributed edge data centers while satisfying edge applications' latency SLOs. \proposedsystem enables carbon awareness at the edge and includes a new emissions-optimal policy to minimize emissions in complex edge environments with heterogeneous resources and diverse workloads (\autoref{sec:design}). 
   
    \item \textbf{Experimental Evaluation.} We evaluate \proposedsystem in both edge testbeds and large-scale simulations, using real-world traces, edge workloads, and diverse edge settings. Our experimental results on real testbed demonstrate that \proposedsystem can reduce emissions by up to 78.7\% in mesoscale regional edge deployments. Furthermore, our CDN-scale simulations indicate that \proposedsystem yields 49.5\% and 67.8\% savings in the US and Europe, respectively, while limiting the one-way latency increase to less than 5.5 ms %Notably, \proposedsystem outperforms state-of-the-art methods by at least 63\% in savings in heterogeneous edge settings
    (\autoref{sec:evaluation}).
    
    % 49.5\% and 67.8\% savings in the US and Europe, respectively, while incurring a round-trip latency increase of less than 11 ms (\autoref{sec:implementation} and \autoref{sec:evaluation}).
    
  %  across Europe show that \proposedsystem reduces carbon emissions by 75\% while only incurring a 17.2 ms increase in round-trip latency (\autoref{sec:implementation} and \autoref{sec:evaluation}).
    %i) 98.4\% compared to a carbon-agnostic baseline, ii) 78\% compared to energy-aware placement, and iii) 75\% compared to a state-of-the-art carbon-aware policy (\autoref{sec:evaluation}). \todo{Fill with new results}
\end{enumerate}



% Edge computing and edge clouds have emerged as extensions of traditional cloud computing, enabling the deployment of applications with tight latency and massive bandwidth requirements~\cite{Satya17_emergence, Satya09_Cloudlets}. 
% Edge sites, often hosting small server clusters and strategically positioned closer to users, enable cloud-like services with reduced network costs. This makes edge computing a well-suited paradigm for latency-sensitive applications,  such as multi-user cloud gaming, machine learning (ML), and immersive 360-degree VR applications,  which require very low response time ($<$100ms)~\cite{Zhang2019_HeteroEdge}. Edge computing is expected to see a staggering compound annual growth rate (CAGR) of 36.9\% from 2024 to 2030~\cite{edgemarket}, raising concerns about the sustainability of edge applications and the growing carbon emissions from edge systems. 
% %Such a rapid surge in computing demand has raised concerns about rising carbon emissions. In response, there is a growing emphasis on developing sustainable strategies to ensure that the advantages of edge computing do not come at the cost of environmental degradation.



% %To optimize computing's carbon efficiency (cycles/\emissionunit), researchers focus on optimizing their systems' energy efficiency (cycles/kWh) or their energy's carbon efficiency (kWh/\emissionunit). 
% To address the sustainability concerns, researchers have focused on directly reducing carbon emissions by optimizing their systems' energy efficiency and scheduling techniques that leverages the spatial and temporal variations in carbon intensity. To maximize energy efficiency, researchers have considered resource consolidation techniques ~\cite{Li2018_energy_placement, Goudarzi21Placement}

% address sustainability concerns, researchers have shifted their focus to directly reducing carbon emissions from computing through \emph{supply-side} and \emph{demand-side} adaptations~\cite{bashir2021enabling}. Supply-side adaptations aim to decrease the carbon intensity of the energy supply -- measured in \carbonunit -- by incorporating renewable energy sources like solar and wind.
% However, completely eliminating carbon emissions using supply-side techniques alone is challenging and can be costly~\cite{acun2023carbon,Cole:2021}. 
% In contrast, demand-side adaptations take advantage of the spatial and temporal differences in energy's carbon intensity within and across different geographical regions. For instance, regions heavily reliant on fossil fuels, such as coal and gas, tend to have high carbon intensity and, consequently, greater carbon emissions. Conversely, regions that utilize renewable energy sources generally exhibit low carbon intensity and fewer emissions.
% Moreover, the intermittent nature of renewables, such as solar and wind, introduce fluctuations in the energy mix throughout the day and across regions. This variability highlights  opportunities for users to reduce carbon emissions by shifting workloads both spatially~\cite{maji_hotcarbon23, carboncast, igsc2023-casper, sukprasert2023quantifying} and temporally~\cite{ecovisor, acun2023carbon, Wiesner2022:Cucumber, hanafy2023carbonscaler,  hanafy23-war, wait-awhile, hanafy2024gaia}.
% %In this paper, we focus on reducing carbon emissions of latency-sensitive edge applications by leveraging the variability in energy's carbon intensity across small geographical locations.

% %For example, regions that mostly rely on fossil fuels, e.g., coal and gas, to cover their customers' energy demand often have higher carbon intensity than regions that incorporate renewables into their supply mix~\cite{sukprasert2023quantifying}. In addition, the carbon intensity varies as the local demand and the supply mixture change throughout the day and across regions. These characteristics underscore the variety of demand-side opportunities users can leverage to shift their compute demand temporally~\cite{ecovisor, acun2023carbon, Wiesner2022:Cucumber, hanafy2023carbonscaler, hanafy2024gaia, hanafy23-war, wait-awhile, sukprasert2023quantifying} and spatially~\cite{maji_hotcarbon23, carboncast, igsc2023-casper, sukprasert2023quantifying} to reduce the carbon footprint. 

% % \abel{WE NEED A TRANSITION PARAGRAPH HERE. For example: \\
% In light of these opportunities, researchers have increasingly focused on carbon-aware strategies for placing cloud workloads. In contrast to earlier work that concentrated on performance- and energy-aware approaches~\cite{Li2018_energy_placement, Goudarzi21Placement}, recent studies have explored carbon-aware methods~\cite{igsc2023-casper, maji_hotcarbon23, sukprasert2023quantifying, cloudcarbon}. Although these methods can achieve significant carbon savings, they often involve performance trade-offs. For instance, the benifits of temporal shifting depend on how long the user is willing to wait~\cite{wait-awhile, hanafy2024gaia}, where the longer you wait, the greater the likelihood of finding a time slot with lower carbon intensity. Additionally, spatial shifting depends on the flexibility of latency requirements, allowing users to place their applications in more distant but greener locations.  For instance, \citet{sukprasert2023quantifying} shows that achieving a 52\% reduction in carbon emissions requires a 50ms increase in the response time.  However, unlike cloud applications with flexible latency requirements, edge workloads have stricter performance demands.  Thus, to reduce carbon emissions in edge computing,  carbon-aware approaches must meticulously evaluate both latency requirements and potential carbon savings. 

% % Thus, edge computing presents an critical trade-off: reducing carbon emissions typically results in longer response times. In this context, edge orchestrators must meticulously evaluate the latency requirements and resource availability when performing edge application placement.
% %In this case, carbon-aware edge application placement must meticulously consider the latency requirements and resource availability.

% % However, near real-time interactive workloads, such as video object detection, have strict performance requirements, and spatially migrating requests may increase the perceived performance of these workloads. 

% %In addition, modern distributed workloads have various levels of temporal, spatial, and performance flexibility. These workloads are geo-distributed across various sites, and prior work has looked at various mechanisms to meet their performance constraints while minimizing resource usage. 
% %Old Comment%%For example, prior work has proposed optimizations to enable interactive workloads such as AI inference to take advantage of the spatial flexibility to optimize their carbon emissions by shifting jobs to regions where the energy carbon intensity is low. 
% %While state-of-the-art techniques consider resource provisioning trade-offs, they often overlook the performance implications of carbon-aware optimizations. Specifically, carbon-aware scheduling uses the spatial flexibility of interactive jobs to shift execution to regions with low carbon intensity. However, near real-time interactive workloads, such as video object detection, have strict performance requirements, and migrating requests spatially can increase the perceived latency times of these jobs. Thus, distributed carbon-aware scheduling presents a carbon-performance trade-off: reducing carbon emissions generally leads to longer response times. %}

% % \lilly{The following is not very interesting, need to improve!}

% To explore these considerations, this paper demonstrates the feasibility of reducing carbon emissions while maintaining minimal latency overheads across small-spatial scale areas, which represent individual edge sites. We analyze fine-grain carbon intensity and latency traces across these small geographical areas in regions, such as cities in a single U.S. state and neighboring countries in Europe. The results reveal that the spatial variations of carbon intensity across small-spatial scale areas can be up to XXX, with additional XXX ms latency.  This finding underscores the potential for distributing edge workloads across geographically dispersed edge sites, where the carbon benefits can outweigh the associated latency overheads. 

% % In this paper, we focus on carbon-aware edge application placement by leveraging the variability in energy's carbon intensity while considering the tight latency requirements of edge applications. We base our results on the \emph{fine-grain} visibility into the carbon intensity from providers like Electricity Maps~\cite{electricity-map} across nearby edge sites which introduces an opportunity for edge offloading, where carbon savings can outweigh the latency overheads.
% % Our analysis across the US and Europe shows that the spatial variations of carbon intensity due to different energy profiles across small edge regions can be up to $10.8\times$, for a 20ms latency overhead\walid{Validate this numbers}.
% %This result shows an attractive trade-off between carbon saving and overhead for small regions. 

% Expanding on our findings, we introduce a carbon-aware control plane, named \proposedsystem, to reduce carbon emissions from edge computing. 
% \proposedsystem enhances the carbon awareness of geographically dispersed edge sites with fine-grain carbon intensity and optimizes workload placements to reduce emissions.  In particular, we propose a new carbon-aware policy to minimize emissions from complex edge environments with heterogeneous edge sites and mixed workloads. Lastly, we implement \proposedsystem on top of a cross-tier edge orchestration named Sinfonia, and evaluate its performance on a real testbed and large-scale simulations, using real-world carbon intensity traces and various edge resources and workloads.  We summarize our main contribution as follows: 

% \begin{enumerate}[leftmargin=*]
%     \item \textbf{Regional  Analysis.} We analyze real-world fine-grain carbon intensity and latency trances across small geographical zones of regions, including small regions with 5 zones and a global region with 173  zones. The analysis highlights the potential of placing workloads across dispersed edge sites with substantial carbon savings and negligible latency overheads (\autoref{sec:carbon_analysis}). 

%     % \item \textbf{\proposedsystem.} We present the design and implementation of \proposedsystem, a carbon-aware edge control plane, based on Sinfonia, an open-sourced cross-tier orchestrator for edge-native applications 
%     % ~\cite{satyanarayanan2022sinfonia}(\autoref{sec:design}).

%    \item \textbf{\proposedsystem.} We propose a carbon-aware control plane, \proposedsystem, for reducing emissions from geographically distributed edge sites. \proposedsystem enables carbon awareness at the edge  and includes a new emission-optimal policy to minimize emissions in complex edge environments with heterogeneous resources and diverse workloads (\autoref{sec:design}). 

%    % \item %, and introduce its optimal placement policy. 
%    % \proposedsystem employs a multiple-objective \walid{??} optimization policy to minimize both carbon emissions and latency while placing offloading edge workloads across dispersed edge sites. Further, we implement \proposedsystem on top of Sinfonia, an open-sourced cross-tier orchestrator for edge-native applications~\cite{satyanarayanan2022sinfonia} (\autoref{sec:design}). 
   
%     \item \textbf{Evaluation. } We implemented and evaluated \proposedsystem against multiple baselines using numerous real-world traces, edge workloads, and heterogenous settings. Our experimental results on a real testbed and large-scale simulations show that \proposedsystem can i) reduce carbon emissions by 98.4\% compared to a carbon agnostic baseline and ii) 78\% compared to energy-aware placement iii) 75\% compared to state-of-the-art carbon-aware policy (\autoref{sec:evaluation}).
%     %can yield up to 1) reduces carbon emission by 1518 \emissionunit while adding only 6.6ms overhead compared to the carbon-agnostic execution, 2) * over the state-of-the-art carbon-aware policy ). 
    
%     % using real testbeds and large-scale simulations. Empirical results demonstrate that our carbon-aware offloading policy reduces carbon emission by 1518 \emissionunit while adding only 6.6ms overhead compared to the location-aware policy (\autoref{sec:evaluation}).
% \end{enumerate}





%In addition, advancements in hardware accelerators, such as Apple’s Neural Engine, many applications have transitioned from solely routing requests to distant servers to processing requests directly on user devices. This approach eliminates the need for a network connection, thereby avoiding network latency while also handling issues with privacy and security, at the cost of higher usage of local resources such as battery. 
%The Edge is particularly beneficial for lightweight requests, such as image preprocessing and inference requests on compressed neural networks, where the impact of on-device performance degradation compared to remote servers is less significant, allowing applications to benefit from zero network costs.


