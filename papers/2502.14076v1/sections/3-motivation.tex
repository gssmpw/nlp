

This section presents an empirical study of grid carbon intensity differences that occur over mesoscale geographic distances of tens to hundreds of kilometers. We also analyze the increases in network latency at these scales. Our empirical study seeks to answer two key questions. 

\begin{enumerate}[leftmargin=*]
    \item {\em How much does energy's carbon intensity vary within mesoscale regions that span tens to hundreds of kilometers, and are these differences large enough to warrant the use of spatial workload shifting in distributed edge data centers?}
    
    % How do these variations compare to those at large continental scales or cloud regions?\walid{We do not answer this question.}
    \item {\em How prevalent are these types of mesoscale variations in different parts of the world? Are they sufficiently common to warrant the broad deployment of carbon optimization techniques in edge data centers across the world?}
\end{enumerate}


% This section analyzes the benefits of {\em mesoscale} carbon intensity information. Then, we highlight the latency benefits of mesoscale-level spatial shifting.



%%%%%%%%%%%%%%%%%%%%%%%%%%%%%%%%%%%%%%
%%% Yearly average
%%%%%%%%%%%%%%%%%%%%%%%%%%%%%%%%%%%%%%
\begin{figure}[t]
  \centering%
  %   \hfill
  % \begin{subfigure}{0.24\linewidth}%
  % \centering
  %      \includegraphics[width=\linewidth]{figures/yearly_fl.pdf}%
  %      \caption{Florida}
  %      \label{fig:cv_yearly_r1}
  %   \end{subfigure}%
% \hfill%
 \begin{subfigure}{0.45\linewidth}%
 \centering
       \includegraphics[width=\linewidth]{figures/yearly_cross_us.pdf}%
       \caption{West US}
       \label{fig:cv_yearly_r2}
    \end{subfigure}%
    % \hfill%
%  \begin{subfigure}{0.2\linewidth}%
%  \centering
%        \includegraphics[width=\linewidth]{figures/yearly_cross_us_nm.pdf}%
%        \caption{R3: New Mexico}
%        \label{fig:cv_yearly_r3}
%     \end{subfigure}%
% \hfill%
% \begin{subfigure}{0.24\linewidth}%
%     \centering
%        \includegraphics[width=\linewidth]{figures/yearly_it.pdf}%
%        \caption{Italy}
%        \label{fig:cv_yearly_r4}
%     \end{subfigure}%
\hfill%
\begin{subfigure}{0.45\linewidth}%
        \centering
       \includegraphics[width=\linewidth]{figures/yearly_cross_eu.pdf}%
       \caption{Central EU}
       \label{fig:cv_yearly_r3}
    \end{subfigure}%
    % \hfill
    % \hfill
    \caption{Yearly carbon intensity of two mesoscale regions.}% highlighting differences of up to 10.8$\times$.
    \label{fig:cv_yearly}%
\end{figure}

%%%%%%%%%%%%%%%%%%%%%%%%%%%%%%%%%%%%%%
%%% Two day time series
%%%%%%%%%%%%%%%%%%%%%%%%%%%%%%%%%%%%%%
\begin{figure}[t]
    \centering%
    \includegraphics[width=0.9\linewidth]{figures/monthly_us_legend.pdf}\\
    % \hfill
    \begin{subfigure}{0.45\linewidth}% 
    \centering%
    \captionsetup{justification=centering}
    \includegraphics[width=\linewidth]{figures/daily_cross_us.pdf}
    \caption{Two-day (Dec 25-27)}
    \label{fig:carbon_intensity_temporal_days_WUS}
    \end{subfigure}
    \quad
    \begin{subfigure}{0.45\linewidth}% 
    \centering%  
    \captionsetup{justification=centering}
    \includegraphics[width=\linewidth]{figures/monthly_us.pdf}
    \caption{Year-long (2023)} 
    \label{fig:carbon_intensity_temporal_months_WUS}
    \end{subfigure}
    % \hfill%
    % \begin{subfigure}{0.22\linewidth}% 
    % \centering%  
    % \captionsetup{justification=centering}
    % \includegraphics[width=\linewidth]{figures/daily_cross_eu.pdf}
    % \caption{Central EU \\ (July 17-19) } %Variations at different times of day between July 17-19.
    % \label{fig:carbon_intensity_temporal_days_CEU}
    % \end{subfigure}
    % \hfill%
    % \begin{subfigure}{0.22\linewidth}% 
    % \centering%  
    % \captionsetup{justification=centering}
    % \includegraphics[width=\linewidth]{figures/monthly_europe.pdf}
    % \caption{Central EU \\ (2023)} % Variations across different months throughout 2023.
    % \label{fig:carbon_intensity_temporal_months_CEU}
    % \end{subfigure}
    % \hfill
    % \hfill
    \caption{Spatial-temporal variations in carbon intensity over two days and 12 months in 2023 in West US.}
    \label{fig:carbon_intensity_temporal}
\end{figure}



\begin{figure*}[tb]
    \centering
    \subfloat[\centering D = 200 km ]{\includegraphics[width=0.22\linewidth]{figures/carbon_saving_200km_text.pdf}%
    \label{fig:carbon_saving_200km}%
    }%
    \hfill
    \subfloat[\centering D = 500 km ]{\includegraphics[width=0.22\linewidth]{figures/carbon_saving_500km_text.pdf}%
    \label{fig:carbon_saving_500km}%
    }%
    \hfill
    \subfloat[\centering D = 1000 km ]{\includegraphics[width=0.22\linewidth]{figures/carbon_saving_1000km_text.pdf}
    \label{fig:carbon_saving_1000km}
    }%
    \hfill
    \subfloat[\centering Radius-Latency ]{\includegraphics[width=0.22\linewidth]{figures/distance_latency.pdf}%
    \label{fig:mesoscale_round_trip_latency}%
    }%
    % \vspace{-2mm}
    \caption{Carbon savings with search radii of 200 km, 500 km, and 1000 km. (d) One-way latency across pairwise distances.}
    \label{fig:carbon_saving_distance}
    \vspace{-.4cm}
\end{figure*}




\subsection{Carbon Intensity Analysis at Mesoscales}

To understand the differences in grid carbon intensity that are seen at mesoscales, we conducted a measurement study where we collected carbon intensity traces for 148 carbon zones worldwide for an entire year (2023). For the purpose of our study, a carbon zone, or simply a \textit{zone}, is a geographic area whose grid operator provides carbon intensity data. The geographic size of a carbon zone depends on the area served by the grid operator and can vary from a city to an entire state or even a small country. %This dataset is described in detail in \autoref{sec:real_world_traces}. 
Further, we also collected round-trip latency traces from the WonderNetwork~\cite{wonder-proxy-2020}, which provides ping traces (in milliseconds) to cities across the world. We describe our data sources in \autoref{sec:real_world_traces}. 

To illustrate the carbon intensity differences at the mesoscale, we first select four specific mesoscale regions, each comprising five carbon zones, across the United States and Europe. \autoref{fig:cv_snapshot} depicts a heat-map of the carbon intensity variations within each mesoscale region for a single hour in 2023, with darker colors representing higher carbon intensity values. We assume that each of the five carbon zones within a mesoscale region has an edge data center. 
The Florida region, for example, consists of five cities, each hosting an edge data center, that is a few hundred kilometers apart from one another. 
The figure shows significant differences in carbon intensity values even at this scale, with inter-zone variations of 2.5$\times$ in Florida, 7.9$\times$ in the west US, 2.2$\times$ in Italy, and 19.5$\times$ in Central Europe. 

~\autoref{fig:cv_yearly} then plots the mean carbon intensity over the entire year for two regions. The figure confirms that the differences in carbon intensity persist across the year. Furthermore, the average difference between the maximum and minimum carbon intensity across zones in a region is %1.9$\times$ in Florida, 
2.7$\times$ in the west US, %2.2$\times$ in Italy, 
and 10.8$\times$ in central Europe. Importantly, these differences compare favorably to those reported across global cloud regions. For instance, a recent study of spatial differences in carbon intensity across Amazon's cloud regions reported an order of magnitude difference across AWS cloud regions in Europe and Asia~\cite{sukprasert2024limitations}.
% The primary reason for these differences at smaller geographic scales is the diverse range of generation sources and different amounts of renewable sources in local electric grids. 
% For instance, Region 4 (Central Europe), which spans five countries, is powered by a diverse range of providers and energy mixtures. Nuclear energy represents 63.7\% of France's energy supply, making its carbon intensity low and stable, while Germany heavily relies on fossil fuels, representing 33.7\% of its energy supply~\cite{electricity-map}. 
Moreover, since the relative mix of energy sources changes over time, \autoref{fig:carbon_intensity_temporal} shows temporal fluctuations in the carbon intensity of edge data centers within each mesoscale region within a day (\autoref{fig:carbon_intensity_temporal_days_WUS}) %and \autoref{fig:carbon_intensity_temporal_days_CEU}) 
and across seasons (\autoref{fig:carbon_intensity_temporal_months_WUS}). %and \autoref{fig:carbon_intensity_temporal_months_CEU}).
For instance, Flagstaff, AZ (see \autoref{fig:carbon_intensity_temporal_days_WUS}) exhibits a daily difference of $\sim$300\carbonunit.
%Moreover, ~\autoref{fig:carbon_intensity_temporal} also indicates how the carbon intensity ascending order can shift across zones, dictating adjustments run-time placement adjustment. For instance, the carbon intensity of Munich, DE, can be higher or lower than that of Milan, IT, by 204.9 and 68.1 \carbonunit, respectively. In addition to carbon variations across times of day, carbon intensities are highly affected by seasonal changes due to changes in weather and demand patterns. 
~\autoref{fig:carbon_intensity_temporal_months_WUS} shows how monthly average carbon intensity changes.
For example, Kingman, AZ, exhibits a $\sim$200 \carbonunit change between March and November due to its reliance on solar energy. %In contrast, Lyon, FR, and Bern, CH have low and stable carbon intensity due to the high penetration of nuclear energy. 
%%%%
% Latency
%%%%
\begin{table}[t]
  \centering
    \caption{One-way network latency (ms).  %Shifting workload from Jacksonville to Miami in Florida achieves a 47\% carbon reduction while incurring only a 7.3 ms delay.  between edge locations in two mesoscale regions
    } 
  \label{tab:latency}
  \subfloat[Florida]{%
  \resizebox{0.53\linewidth}{!}{
  \begin{tabular}{|c|c|c|c|c|}
    \hline
    Location &  Miami &  Orlando &  Tampa & Tallah. \\
    \hline
    Jacksonville & 3.64 & 5.32 & 6.86 & 3.42\\ \hline
    Miami &   - & 4.5  & 3.37 & 7.2 \\ \hline
    Orlando &   & - &  1.86 & 4.35\\ \hline
    Tampa &   &  & - & 4.14\\ \hline
    Tallahassee  &  & &  & -\\ \hline
  \end{tabular}
  \label{tab:latency_r1} 
  }
  }
  %\\
  % \quad
  \subfloat[Central EU]{%
  \resizebox{0.47\linewidth}{!}{
    \begin{tabular}{|c|c|c|c|c|c|}
    \hline
    Location &  Graz &  Lyon &  Milan &  Munich  \\
    \hline
    Bern, CH & 8.78 & 6.28 & 6.45 &3.985\\ \hline
    Graz, AT  & - & 16.22  & 11.98 & 8.36\\ \hline
    Lyon, FR  &  & - &  9.34 & 8.82\\ \hline
    Milan, IT   &  &  & - & 8.65\\ 
    \hline
    Munich, DE   &  &  &  & -\\ 
    \hline
  \end{tabular}
    % \label{tab:latency_r2}
  }
  } 
\end{table}




Finally, ~\autoref{tab:latency} shows the pairwise one-way network latency between edge data centers, within two mesoscale regions. The table shows that, unsurprisingly, the latency grows with geographic distance. However, the increase in latency due to shifting workload from one edge location to another ranges from a few milliseconds to $\sim$16 ms, depending on the distance and the network topology between locations. 



%\lilly{Extend this and show the best case.}

%\walid{What do I learn from Fig 4?}
%, often due to fluctuations in electricity-producing intermittent renewable sources.  

% \subsection{Carbon Analysis}
% We begin by showing {\em zones} carbon intensity diversity within a mesoscale region. We utilize four exemplary {\em mesoscale-regions}, with areas ranging from 807km$\times$712km to 1238km$\times$1335km, encompassing 15 sites across the US and Europe. 
% \autoref{fig:cv_snapshot} depicts a snapshot of the average carbon intensity across the four mesoscale regions and their and zones. The figure shows color-coded carbon intensity variations where dark brown depicts zones where energy's carbon intensity is high (i.e., the electricity grid is sourced from fossil fuels) and green represents zones where energy's carbon intensity is low (i.e., high renewable penetration). We can see carbon intensity distinctions across zones with a region, with variations such as 2.5$\times$ in Florida, 7.8$\times$ in West US, and 19.5$\times$ in Central Europe. \todo{add italy}. 

% \autoref{fig:cv_yearly} generalizes the differences and shows the yearly average carbon intensity across regions, where the error bars represent the standard deviations. As shown, there are consistent differences in energy's carbon intensity across all the selected regions and zones. Even within a single US state, such as between Miami and Jacksonville spanning 525 kilometers (see \autoref{fig:cv_yearly_r1}), there is a 1.9$\times$ difference,  representing a 270.8 \carbonunit difference. Furthermore, the difference can be substantial, up to 10.8$\times$, between Lyon, France, and Munich, Germany,  spanning 575 kilometers (see \autoref{fig:cv_yearly_r3}), representing a 298.8 \carbonunit difference.


% The main reason for such variability is that different zones, although adjacent, utilize different energy sources and, hence, have different carbon intensity profiles. For instance, Region 1 (Florida) is powered by 8 electricity providers, while Region 2 (Central Europe), which spans five countries, is powered by a diverse range of providers and energy mixtures. In particular, Electricity Maps~\cite{electricity-map} reports that for 2023, 
% nuclear energy represents 63.7\% of France's energy supply, making its carbon intensity low and stable. At the same time, Germany heavily relies on fossil fuels, where coal and gas represent 23.7\% and 10.1\% of its energy supply (see ~\autoref{fig:cv_snapshot_r3} and ~\autoref{fig:cv_yearly_r3}). 
% Another example is West US, where states like California have a high penetration of renewable energy, representing 35\% of its energy supply, leading to lower carbon intensity than southern Arizona, where fossil fuel represents 54.4\% of its energy supply (see ~\autoref{fig:cv_snapshot_r2} and ~\autoref{fig:cv_yearly_r2}). 










% \todo{Merge labels for a-b and c-d and increase font.}
% Energy profiles not only affect average carbon intensity but also affect how carbon intensity varies across times of day and across seasons. \autoref{fig:carbon_intensity_temporal} illustrates how carbon intensity varies temporally across days and months within neighboring zones. 
% The plots emphasize the carbon intensity fluctuations within zones across days (\autoref{fig:carbon_intensity_temporal_days_CEU} and  \autoref{fig:carbon_intensity_temporal_days_WUS}) and across month (\autoref{fig:carbon_intensity_temporal_months_CEU} and  \autoref{fig:carbon_intensity_temporal_months_WUS}). For instance, Munich, DE (see \autoref{fig:carbon_intensity_temporal_days_CEU}) exhibits a daily difference of 330.6\carbonunit.
% Moreover, ~\autoref{fig:carbon_intensity_temporal} also indicates how the carbon intensity ascending order can shift across zones, dictating adjustments run-time placement adjustment. For instance, the carbon intensity of Munich, DE, can be higher or lower than that of Milan, IT, by 204.9 and 68.1 \carbonunit, respectively. In addition to carbon variations across times of day, carbon intensities are highly affected by seasonal changes due to changes in weather and demand patterns. ~\autoref{fig:carbon_intensity_temporal_months_CEU} shows how monthly average carbon intensity changes. For example, Graz, AT exhibits a 171.9 \carbonunit change between February and May due to its reliance on solar energy. In contrast, Lyon, FR, and Bern, CH have low and stable carbon intensity due to the high penetration of nuclear energy. 



% \subsection{Latency Analysis.}
% Although carbon intensity differences demonstrated earlier highlight the possible carbon savings of shifting workloads from high carbon zones to low carbon zones, they do not convey the whole story of the associated overheads of such actions. \autoref{tab:latency} lists the round-trip latency (ms) between different zones in the same region.\footnote{We underline missing measurements, estimated using a linear regression model of latency and distance using the known latency between zones within the same region.}
% As expected, the latency between different zones is very small due to their proximity. For example, the latency between Orlando and Tampa can be as small as 3.738ms. Finally, although the latency between zones is highly correlated with the distance, it also reflects the network connectivity between zones. 
% For instance, the distance between Jacksonville and Miami (524.8km) in the Florida region is 1.57$\times$ larger than the distance between Bern and Milan (204.6km) in the Central Europe region. However, the latency is 43.5\% lower.
% %For instance, although the distance between Jacksonville and Miami (524.8km) in Region 1 is larger than the distance between Bern and Milan (204.6km) in Region 5 by 157\%, the latency is lower by 43.5\%. 

% Combining carbon intensity variations in ~\autoref{fig:cv_yearly} and round-trip latency in Table~\ref{tab:latency}, we observe up to a 1.9$\times$ carbon intensity reduction with adding only 7.295ms latency overhead when shifting workloads from Jacksonville to Miami, Florida, and up to a 10.8$\times$ reduction with a 17.65ms latency overhead when shifting workloads from Munich, Germany to Lyon, France. These findings highlight the potential for significant carbon reductions with minimal latency overhead. 

% We next extend our analysis to large regions encompassing more zones and sites.



% \subsection{Large Scale Analysis}
% \label{subsec:large_scale_analysis}

% Whether the pattern extensively exists in other small regions? 

% We extend our analysis to a large scale, examining 128 sites with both latency and carbon intensity traces across the US and Europe. 

% \lilly{Re-do the large-scale analysis;  carbon variations (same as Fig.2) vs number of sites; latency overhead vs number of sites; region size vs carbon variations; region size vs latency overhead. --> to arbitrarily define small region? or use all sites to define small region with latency limit. --> Mesoscale, it is about distance / but edge computing is about latency; carbon intensity is geographical concept; --> mesoscale edge computing's network latency is in a range of 5 ms to 50 ms.}

% To understand the potential to save carbon emissions across edge sites and edge sites only interact with their neighboring sites to meet the low-latency demands of edge applications, we define micro-regions with latency limit as the boundary. For each site, we define a micro-region that includes nearby sites that are reachable within a given latency limit, creating a maximum of 128 micro-regions. For instance, for an edge site located in Boston with a latency limit of 10 ms, its micro-region includes all surrounding edge sites reachable within that 10 ms boundary. 


% First,  we examine whether each site has at least one neighboring site within its micro-region with lower carbon intensity, indicating the potential for carbon emission reduction. ~\autoref{fig:large_scale_percentage} illustrates the percentage 

% Figure~\ref{fig:large_scale_percentage} illustrates the percentage \textit{SLO-Satisfied} sites (red line) and \textit{SLO \& Greener} sites (green line), with latency limits ranging from 10 ms to 50 ms. The results show that even with a 10 ms latency limit, 39.06\% of sites have a greener neighboring site. Additionally, the figure indicates that as the latency limit increases, the proportion of \textit{SLO-Satisfied} and \textit{SLO \& Greener} sites increase. However, the rate of increase diminishes after a 30 ms latency limit, with 85.16\% of sites having access to greener zones without violating latency requirements.


% ~\autoref{fig:large_scale_percentage} shows the total number of micro-regions with different latency limits. In addition, we examine the number of micro-regions 

% based on a given latency limit. 

% In practice, edge sites are geographically dispersed across large regions such as the US and Europe. In this section, we extend our findings to a region encompassing 128 sites: 64 sites distributed across 24 zones in the US and 64 sites across 30 zones in Europe. Given the latency-sensitive nature of edge workloads, each site interacts only with neighboring sites within a specified latency range. Consequently, our analysis focuses on sites that fall within a defined latency limit for each site.



% First,  we examine whether a site can have at least one neighboring site that meets a specific latency limit (labeled as \textit{SLO-Satisfied}) and has a lower carbon intensity (labeled as \textit{SLO \& Greener}). Figure~\ref{fig:large_scale_percentage} illustrates the percentage \textit{SLO-Satisfied} sites (red line) and \textit{SLO \& Greener} sites (green line), with latency limits ranging from 10 ms to 50 ms. The results show that even with a 10 ms latency limit, 39.06\% of sites have a greener neighboring site. Additionally, the figure indicates that as the latency limit increases, the proportion of \textit{SLO-Satisfied} and \textit{SLO \& Greener} sites increase. However, the rate of increase diminishes after a 30 ms latency limit, with 85.16\% of sites having access to greener zones without violating latency requirements. 

% Next, we analyze the number of greener neighboring sites for each site and estimate the potential carbon savings that can be achieved. 


% As illustrated in \autoref{fig:large_scale_green_zones} and ~\autoref{fig:large_scale_carbon_reduction}, most sites have greener neighboring sites, yielding average reductions of 46\% with a 10 ms latency limit. As the latency limit increases, both the number of greener neighboring sites and potential carbon reductions increase.  For instance, with a 50 ms latency limit, each site has an average of 22 greener sites, achieving an average carbon reduction of 65.7\% and a maximum of 99.7\%. However, we observe that increases in latency limits result in diminishing returns in terms of the number of greener neighboring sites and the carbon reduction ratios. This diminishing return occurs because as the latency limit grows, the marginal benefit of additional greener sites and further reductions in carbon intensity becomes smaller.



% %%%%%%%%%%%%%%%%%%%%%%%%%%%%%%%%%%%%%%
% %%% Large-scale carbon and latency analysis
% %%%%%%%%%%%%%%%%%%%%%%%%%%%%%%%%%%%%%%
% \begin{figure}[t]
%     \centering%
%     \begin{subfigure}{0.3\linewidth}% 
%     \centering%
%     \captionsetup{justification=centering}
%      \includegraphics[width=\linewidth]{figures/carbon_latency_percentage.pdf}
%     \caption{}
%     \label{fig:large_scale_percentage}
%     \end{subfigure}
%     \hfill%
%     \begin{subfigure}{0.3\linewidth}% 
%     \centering%
%     \captionsetup{justification=centering}
%     \includegraphics[width=\linewidth]{figures/greener_zones_boxplot.pdf}
%     \caption{}
%     \label{fig:large_scale_green_zones}
%     \end{subfigure}
%     \hfill%
%     \begin{subfigure}{0.3\linewidth}% 
%     \centering%  
%     \captionsetup{justification=centering}
%     \includegraphics[width=\linewidth]{figures/carbon_saving_times_boxplot.pdf}
%     \caption{}
%     \label{fig:large_scale_carbon_reduction}
%     \end{subfigure}
%     % \hfill%
%     % \begin{subfigure}{0.24\linewidth}% 
%     % \centering%  
%     % \captionsetup{justification=centering}
%     % \includegraphics[width=\linewidth]{figures/distance_latency.pdf}
%     % \caption{Distance.}
%     % \label{fig:large_scale_distance}
%     % \end{subfigure}

%     % \vspace{-2mm}
%     \caption{ The potential for having greener sites under varying latency limits and the statistics of these sites. the percentage of sites with \textit{SLO-Satisfied} and \textit{SLO-Greener} neighboring sites (a); the number of \textit{SLO-Greener} neighboring sites (b); and (c) the carbon intensity difference ($\times$) between sites and their greenest \textit{SLO-Greener} neighboring sites.}
%     % \vspace{-5mm}
%     \label{fig:large_scale_analysis}
% \end{figure}


% %%%%%%%%%%%%%%%%%%%%%%%%%%%%%%%%%%%%%%
% %%% Network latency = 20ms
% %%%%%%%%%%%%%%%%%%%%%%%%%%%%%%%%%%%%%%
% \begin{figure}[t]
%     \centering%
%     \hfill
%     \begin{subfigure}{0.22\linewidth}% 
%     \centering%  
%     \captionsetup{justification=centering}
%     \includegraphics[width=\linewidth]{figures/his_ci_10.pdf}
%     \caption{CI (10 ms)}
%     \label{fig:latency_limit_hist_10}
%     \end{subfigure}
%     \hfill
%     \begin{subfigure}{0.22\linewidth}% 
%     \centering%  
%     \captionsetup{justification=centering}
%     \includegraphics[width=\linewidth]{figures/his_ci_40.pdf}
%     \caption{CI (40 ms)}
%     \label{fig:latency_limit_hist_40}
%     \end{subfigure}
%     \hfill
%     \begin{subfigure}{0.22\linewidth}% 
%     \centering%  
%     \captionsetup{justification=centering}
%     \includegraphics[width=\linewidth]{figures/his_latency_10.pdf}
%     \caption{Latency (10 ms)}
%     \label{fig:latency_limit_hist_10}
%     \end{subfigure}
%     \hfill
%     \begin{subfigure}{0.22\linewidth}% 
%     \centering%  
%     \captionsetup{justification=centering}
%     \includegraphics[width=\linewidth]{figures/his_latency_40.pdf}
%     \caption{Latency (40 ms)}
%     \label{fig:latency_limit_hist_40}
%     \end{subfigure}
%     \hfill%
%     \caption{Carbon difference and network latency distributions of the greenest sites with different latency limits : (a) Carbon difference (\carbonunit) at 10 ms; (b) Carbon difference (\carbonunit) at 40 ms; (c) Network latency at 10 ms; (d) Network latency at 40 ms.}
%     \label{fig:latency_limit_hist}
% \end{figure}



% After confirming the potential of finding greener neighboring sites with substantial carbon reductions across a large set of micro-regions, we further investigate the empirical distribution of carbon intensity differences (measured in \carbonunit) and the associated latency overhead. 

% ~\autoref{fig:latency_limit_hist} reveals intriguing patterns in carbon intensity differences and latency distribution for latency limits of 10 ms and 40 ms. The carbon intensity difference can reach up to 711.9 \carbonunit, with most values clustering around 291.3 \carbonunit. As the latency limit increases to 40 ms,  more sites experience higher differences, with a notable peak at 210.8 \carbonunit. In contrast,  with a 10 ms latency limit, the majority of sites exhibit a difference of 140.96 \carbonunit. ~\autoref{fig:latency_limit_hist_10} highlights that with a 10 ms limit, the actual latency overhead of saving carbon footprint can be as low as 1.63 ms,  with an average of 6.78 ms.  Meanwhile, ~\autoref{fig:latency_limit_hist_40} demonstrates that under a 40 ms limit,  50\% of the sites experience a 22.54 ms latency overhead. These findings underscore the delicate balance between reducing carbon emissions and maintaining acceptable latency levels. Tighter latency limits generally correlate with lower carbon intensity differences but can also potentially increase latency overhead for greener operations.


\noindent \textit{\textbf{Key Takeaways.} Our results show significant differences in the carbon intensity of electricity at mesoscale distances, similar to those reported at continental scales between cloud regions. These mesoscale variations demonstrate the feasibility of using spatial workload-shifting optimizations for edge data centers.}

\subsection{Mesoscale Analysis across Continents}
Having shown that there can be significant differences in carbon intensity at the mesoscale, a key question is whether such differences are commonplace in different parts of the world or confined to a few specific locations. To answer this question, we conduct an analysis of carbon intensity traces across 496 Akamai edge data centers in the United States and Europe. For each edge data center, we find the location with the highest carbon intensity difference within a threshold radius distance $D$ and compute the percentage difference in carbon intensity between the two locations. ~\autoref{fig:carbon_saving_distance} plots a CDF of the observed pairwise differences for different values of threshold radius $D$ (from $D$ = 200 km to $D$ = 1000 km).

For a radius of 200 km, ~\autoref{fig:carbon_saving_200km} shows that 32\% of the edge data centers have at least data center with a carbon intensity difference of more than 20\%, and 12\% of locations have a data center with a carbon intensity difference of more than 40\%. At the same time, 68\% of the edge data centers do not have any location with a significant spatial carbon intensity difference (i.e., more than 20\%). As the radius increases, the chances of finding an edge location with significant carbon intensity differences grows.
%carbon intensity lowest by 20\%. In contrast, 76\% cannot find another data center with a significant carbon intensity difference (i.e., more than 20\%), or is the greenest among its peers. To generalize our findings, we extend the search radius to 500km (see ~\autoref{fig:carbon_saving_500km}) and 1000km (see~\autoref{fig:carbon_saving_1000km}). 
As shown in \autoref{fig:carbon_saving_500km} and \autoref{fig:carbon_saving_1000km}, increasing the radius to 500 km and 1000 km allows 57\% and 78\% of edge data centers to reduce their emissions by more than 20\%. In addition, this increase enables 27\% and 45\% of edge data centers to {\em significantly} reduce their carbon emissions by more than 40\% for the 500 km and 1000 km radius, respectively. The fraction of edge locations without any significant carbon intensity differences within its radius falls to 22\% for $D=1000$ km. Lastly, \autoref{fig:mesoscale_round_trip_latency} shows that the median increase in latency ranges from 5.3 ms for $D=200$ km to 14.3 ms for $D=1000$ km.
%Lastly, we note that most of the regions that do not get considerable savings (i.e., $\geq 20\%$) are powered by green energy, where XXX\% of them are below 100\emissionunit.

\noindent \textit{\textbf{Key Takeaways.} More than 78\% of the edge locations in Europe and North America see carbon intensity differences exceeding 20\% within a radius of 1000 km, indicating that mesoscale carbon intensity variations are prevalent in many regions of the world.\footnote{Our analysis could not be extended to other continents (e.g., Asia, Australia) due to the unavailability of  fine-grain spatial carbon intensity data, but we anticipate similar trends will persist as the adoption of renewables continues to grow globally.}}

%, where a 500 km search radius allows more than 24\% edge data centers to find a peer with significantly lower carbon intensity.


% \textit{Carbon intensity at mesoscale poses substantial variations. These variations present valuable opportunities to reduce carbon emissions by strategically distributing workloads across geographically dispersed edge data centers, all while ensuring low latency performance.}


% Substantial spatial variations in carbon intensity with minimal latency increase across neighboring zones present opportunities for distributing latency-sensitive workloads across edge sites with carbon savings.
% Edge orchestrators can leverage these variations in their placement decisions to reduce carbon emissions while adhering to strict latency requirements.
% In addition, edge providers can incorporate this variability in planning decisions, such as building and increasing the capacity of edge infrastructures in zones with the lowest carbon footprint. }

% \end{visionbox}
%\end{mdframed}


% \lilly{Need to add sentences to conclude the carbon and latency analysis and transit to macro-region analysis.}

%%%%%%%%%%%%%%%%%%%%%%%%%%%%%%%%%%%%%%
%%% Zones in Global Analysis
%%%%%%%%%%%%%%%%%%%%%%%%%%%%%%%%%%%%%%
% Please add the following required packages to your document preamble:
% \usepackage{graphicx}
% \begin{table*}[t]
% \caption{Zones in global analysis.}
% \label{tab:global_regions}
% \resizebox{0.3\linewidth}{!}{%
% \begin{tabular}{|c|c|}
% \hline
% \textbf{Continent} & \textbf{Number of Zones}  \\ \hline
% North America & 66  \\ \hline
% Europe & 68 \\ \hline
% Asia & 21 \\ \hline
% Oceania & 9  \\ \hline
% South America & 6  \\ \hline
% Africa  & 2  \\ \hline
% In Total  & 173  \\ \hline
% \end{tabular}%
% }
% \end{table*}


%%%
% Carbon varions 
%%%
% \begin{figure}
%   \centering%
%   \begin{subfigure}[b]{0.48\linewidth}%
%   \centering
%        \includegraphics[width=\linewidth]{figures/yearly_fl.pdf}%
%        \caption{Region 1: Florida}
%        \label{fig:motivation_florida_carbon}
%     \end{subfigure}%
%     \hfill
%   \begin{subfigure}[b]{0.48\linewidth}%
%   \centering
%        \includegraphics[width=\linewidth]{figures/yearly_cross_eu.pdf}%
%        \caption{Region 4: Central Europe}
%        \label{fig:motivation_EU_carbon}
%     \end{subfigure}%
%     \caption{Average carbon intensity in 2023 across zones.}%
%     \label{fig:motivation_florida}%
% \end{figure}


% such as 2.5$\times$ in Florida, covering an area of 807km$\times$712km, 7.8$\times$ in West US, covering an area of 963km$\times$890km, 4.7$\times$ in New Mexico, covering an area of 1444km$\times$1335km, 2.3$\times$ in Italy, covering an area of 1349km$\times$1335km, 19.5$\times$ in Central Europe, covering an area of 1238km$\times$1335km.

% Even across 600 kilometers, there are 1.9$\times$ between Miami and Jacksonville within a single US state (see \autoref{fig:cv_yearly_r1}) and up to 10.8$\times$ between Lyon, France, and Munich, Germany (see \autoref{fig:cv_yearly_r3}), 


% In the rest of the paper, we focus on carbon-aware edge placement in multiple scenarios and demonstrate the carbon savings potential.


%In addition, the control plane of edge sites can distribute edge-native applications to greener regions while fulfilling the low-latency demands. 
% users can migrate their interactive workloads to greener regions without a high latency overhead. In addition, edge providers can embrace such variability in their planning decisions, building and increasing the capacity of their edge infrastructures where the energy is greenest. In the rest of this paper, we focus on the carbon-aware placement of latency-sensitive workloads and demonstrate the possible savings and carbon-performance trade-offs of carbon-aware task offloading.


%\autoref{fig:latency_limit_20_carbon_difference} shows the histogram of the carbon differences between zones and their greenest nearby zones that meet the 20ms latency limit. The carbon difference is [8.1, 580.2] \carbonunit. 69.3\% of these zones can migrate to zones that are at least 100 \carbonunit greener. 24.5\% of these zones can migrate to zones that are at least 200 \carbonunit greener. Figure~\ref{fig:latency_limit_20_network_latency} shows the histogram of the network latency between zones and their greenest nearby zones that meet the 20ms latency limit. The minimum network overhead between a zone and its greenest zone can be as small as 3.52ms. 23.7\% of the greenest zones are reachable within 10ms; 59.6\% of the greenest zones are reachable within 15ms. 


% \begin{table}[h]
%   \caption{Location-to-location latency}
%   \label{tab:acm_table}
%   \begin{tabular}{|c|c|c|c|c|}
%     \hline
%     City & Jacksonville  & Orlando & Tampa & Miami \\
%     \hline
%     Jacksonville & - & 10.656 & 13.729 & 7.295 \\ \hline
%     Orlando & 10.768  & - & 3.738  & 9.109 \\ \hline
%     Tampa & 13.897 & 3.666 & - &  6.721\\ \hline
%     Miami & 7.354 & 9.009 & 6.755 & - \\ 
%     \hline
%   \end{tabular}
% \end{table}


% \todo{stacked bars for different latency limits; --> Ploted it but it is difficult to understand}

%%%%%%%%%%%%%%%%%%%%%%%%%%%%%%%%%%%%%%
%%% Network latency = 20ms
%%%%%%%%%%%%%%%%%%%%%%%%%%%%%%%%%%%%%%
% \begin{figure}[t]
%     \centering%
%     \hfill
%     \begin{subfigure}{0.3\linewidth}% 
%     \centering%  
%     \captionsetup{justification=centering}
%     \includegraphics[width=\linewidth]{figures/his_latency_10.pdf}
%     \caption{10ms }
%     \label{fig:latency_limit_hist_10}
%     \end{subfigure}
%     \hfill
%     \begin{subfigure}{0.3\linewidth}% 
%     \centering%  
%     \captionsetup{justification=centering}
%     \includegraphics[width=\linewidth]{figures/his_latency_20.pdf}
%     \caption{20ms}
%     \label{fig:latency_limit_hist_20}
%     \end{subfigure}
%     \hfill
%     \begin{subfigure}{0.3\linewidth}% 
%     \centering%  
%     \captionsetup{justification=centering}
%     \includegraphics[width=\linewidth]{figures/his_latency_40.pdf}
%     \caption{40ms}
%     \label{fig:latency_limit_hist_40}
%     \end{subfigure}
%     \hfill%
%     \hfill%
%     \caption{Network latency distributions of greener zones across different latency limits.}
%     \label{fig:latency_limit_hist}
% \end{figure}





%%%%%%%%%%%%%%%%%%%%%%%%%%%%%%%%%%%%%%
%%% Two day time series
%%%%%%%%%%%%%%%%%%%%%%%%%%%%%%%%%%%%%%
% \begin{figure}[t]
%     \centering%
%     % \begin{subfigure}{0.19\linewidth}% 
%     % \centering%
%     % \captionsetup{justification=centering}
%     %  \includegraphics[width=\linewidth]{figures/daily_fl.pdf}
%     % \caption{R1: Florida\\(Mar 9-11).}
%     % \label{fig:carbon_intensity_timeseries_r1}
%     % \end{subfigure}
%     % \hfill%
%     \hfill
%     \begin{subfigure}{0.4\linewidth}% 
%     \centering%
%     \captionsetup{justification=centering}
%     \includegraphics[width=\linewidth]{figures/monthly_us.pdf}
%     \caption{R2: West US Monthly.}
%     \label{fig:carbon_intensity_timeseries_r2}
%     \end{subfigure}
%     % \hfill%
%     % \begin{subfigure}{0.19\linewidth}% 
%     % \centering%
%     % \captionsetup{justification=centering}
%     % \includegraphics[width=\linewidth]{figures/daily_cross_us_nw.pdf}
%     % \caption{R3: New Mexico\\(July 04-06).}
%     % \label{fig:carbon_intensity_timeseries_r3}
%     % \end{subfigure}
%     % \hfill%
%     % \begin{subfigure}{0.19\linewidth}% 
%     % \centering    
%     % \captionsetup{justification=centering}
%     % \includegraphics[width=\linewidth]{figures/daily_it.pdf}
%     % \caption{R4: Italy\\(May 9-11).}
%     % \label{fig:carbon_intensity_timeseries_r4}
%     % \end{subfigure}
%     % \hfill%
%     % \begin{subfigure}{0.4\linewidth}% 
%     % \centering%  
%     % \captionsetup{justification=centering}
%     % \includegraphics[width=\linewidth]{figures/monthly_europe.pdf}
%     % \caption{R3: Central Europe Monthly.}
%     % \label{fig:carbon_intensity_timeseries_r3}
%     % \end{subfigure}
%     % \hfill
%     % \hfill
%     \vspace{-2mm}
%     \caption{Carbon intensity variations for 2 days in 2023.}
%     \vspace{-5mm}
%     \label{fig:carbon_intensity_timeseries}
% \end{figure}



% In our analysis, we map a site (defined by coordinators) to a carbon zone (represented by a multi-polygon) by determining which polygon the coordinators fall within. 

% Each location in WonderNetwork is a point with a longitude and a latitude.  Carbon zones in the Electricity Map are defined as MultiPolygon. We identify the carbon zone of each location by checking whether its coordinator is within a MultiPolygon. 

% In the USA, it has 67 locations, covering 36 states and 26 carbon zones, among them 24 zones (65 locations in 35 states) have carbon intensity traces.  XXX \% of the locations  have the same carbon zones  and XXX \% of locations have unique carbon zones.  The area of the 24 zones ranges from XXX $km^2$  to XXX $km^2$
% The latency of these locations ranges from XXX ms to XXX ms.

% In Europe, it has 76 cities, covering 34 countries and 30 carbon zones. 

% Since carbon intensity traces covers ISOs and RTOs, while  WonderNetwork data covers specific cities,  we mapped the cities into carbon zones according their coordinates, and limit our analysis to zones within the WonderNetwork traces. As a result, we get 173 zones for our analysis, including 66 in North America, 68 in Europe, 21 in Asia, 9 in Oceania, 6 in South America, and 2 in Africa. 

% USA: 65 cities; in 35 states; --> 24 zones
% Europe: 76 cities; in 34 countries --> 30 zones
% Worldwide: 246 cities

% %%%%%%%%%%%%%%%%%%%%%%%%%%%%%%%%%%%%%
% %% Table of Zones and Regions
% %%%%%%%%%%%%%%%%%%%%%%%%%%%%%%%%%%%%%
% % Please add the following required packages to your document preamble:
% % \usepackage{graphicx}
% \begin{table*}[t]
% \caption{Traces used in this paper.}
% \vspace{-2mm}
% \label{tab:regions}
% \resizebox{\linewidth}{!}{%
% \begin{tabular}{|l|l|l|l|l|l|l|l|l|l|l|l|l|}
% \hline
% \textbf{Regions} & \multicolumn{3}{c|}{\textbf{Carbon Traces}}  & \multicolumn{4}{c|}{\textbf{Latency Traces}} & \multicolumn{4}{c|}{\textbf{Akamai CDN Traces}}  \\ \hline
%  & States&  Zones &  Area ($km^2$) & Locations &States &  Zones & Latency (ms) & Sites  &  States &  Zones & Distance (km)\\ \hline
% USA & 50 &  54 & 123.73~\sim1106,425.96  &  64  &  34 & 24 & 0.93~\sim184.67 & 390  & 48  & 40 & 2.06~\sim299.58 \\ \hline
% Europe & 31 &  45 & 560.98~\sim539,571.85  &  64 & 26  & 30 & 1.12~\sim156.74 &  106  & 27  & 34 & 1.33~\sim915.23\\ \hline
% %R3 & New Mexico &  Phoenix, AZ &  Salt Lake City, UT &  Colorado Springs, CO &  Akbuqueque, NM & Las Cruces, NM\\ \hline
% %R4 & Italy &  Milan &  Rome &  Arezzo &  Palermo & Cagliari \\ \hline
% % R3 & Central Europe & Bern, CH &  Graz, AT &  Lyon, FR &  Milan, IT &  Munich, DE  \\ \hline
% \end{tabular}%
% }
% \end{table*}

% Google Cloud has 7 data centers in the US and 6 in Europe
% AWS has 4 data centers in the US and 7 in Europe

% USA: 47 edge sites have carbon and latency data
% Europe: 32 Akamai edge sites have carbon and latency data


% The distance between an US edge site to its closest site ranges from 2.06 km to 299.58 km, with an average of 54.82 km, while the distance in European sites ranges from 1.33 km to 915.23 km, with an average of 99.14 km. 

% Each trace reports energy’s average carbon intensity, measured in grams of carbon dioxide equivalent per kilowatthour (g·CO2eq/kWh), in hourly granularity. The hourly granularity is the highest granularity for average carbon-intensity data currently available from public sources.  Since grid energy’s carbon intensity rarely varies significantly within a 2-3 hour period, higher granularity data would likely not change the results of our analysis.  

% The 148 zones include our entire carbon trace dataset and encompasses ** known edge data center locations in Akamai traces.  --> Compare to the cloud data center. 

% USA: 50 states; 54 zones; [123.73 $km^2$  to 1106425.96 $km^2$]

% Europe: 31 countries; 45 zones; [from 560.98 $km^2$  to 539571.85 $km^2$]


% We start exploring the benefits with small-scale use cases across zones within small geographic regions, ranging from a single US state to neighboring countries in Europe. Then, we generalize our findings by exploring global carbon intensity and latency traces. We base our analysis on carbon intensity traces from ElectricityMaps~\cite{electricity-map} and incorporate the latency between zones from WonderNetwork~\cite{wonder-proxy-2020}, which reports the average ping time (in milliseconds) between major cities. 


% This section demonstrates the applicability of leveraging variations of carbon intensity at small spatial scales for edge offloading. We utilize carbon intensity and round-trip latency across zones within small geographic regions, ranging from a single US state to neighboring countries in Europe. In this analysis, we assume each zone is represented by a major city with an edge site.
