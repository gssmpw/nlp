%Researchers have shown the potential of carbon-aware spatial shifting as we detail below.
%workload optimization in reducing computing carbon footprint in multiple contexts.

% \noindent \textbf{Temporal Shifting.}
% Researchers have exploited the temporal variations in carbon intensity and shown how that temporal shifting can significantly reduce carbon emissions of delay-tolerant batch workloads~\cite{hanafy2023carbonscaler, acun2023carbon, Radovanovic2021CarbonAwareCF, Perotin2023:Risk,wait-awhile}. For instance, the authors of \cite{hanafy2023carbonscaler} presented a carbon-aware scheduling algorithm that alters the time and rate of execution according to temporal variations of carbon intensity. Moreover, in \cite{Radovanovic2021CarbonAwareCF}, the researchers proposed a carbon-aware virtual capacity limit, which limits usable capacity at high-carbon periods, forcing delay-tolerant jobs into low-carbon periods.

%\noindent \textbf{Spatial Shifting.} 
Researchers have analyzed the potential of spatial shifting for batch workloads~\cite{cloudcarbon, sukprasert2024limitations, Zheng2020:Curtailment, Lin2023:Adapting}. For instance, \cite{cloudcarbon} have shown how spatial shifting can reduce the carbon emissions of machine learning training workloads, while in \cite{Zheng2020:Curtailment}, researchers have analyzed how spatial shifting can utilize curtailed energy, which increases the utility of renewable energy.
In addition to batch workloads, researchers have shown how interactive workloads can benefit from spatial shifting~\cite{sukprasert2024limitations, igsc2023-casper, Baolin2023:Clover, Chadha2023:GreenCourier, maji_hotcarbon23, Gsteiger2024:Caribou, Gao-2012-being-green, Murillo2024:CDNShifter}. For instance, in \cite{sukprasert2024limitations, igsc2023-casper}, the authors demonstrated that spatial shifting across geographically dispersed cloud data centers can reduce the carbon emissions of web requests. Moreover,~\cite{Baolin2023:Clover} analyzed how spatial shifting can be used for machine learning inference, where the authors showed how combining spatial shifting with model selection can reduce carbon emission further. In this paper, we underscore the potential of geospatial workload shifting across mesoscale edge data centers, where the benefits of carbon savings outweigh the cost of latency increases. Moreover, we propose a carbon-aware framework for optimizing edge application placements to reduce emissions at the edge. 
%\noindent \textbf{Trade-offs in Carbon-aware optimizations.}
Lastly, many researchers have highlighted many prevalent trade-offs in carbon-aware optimizations to include carbon-performance trade-offs \cite{wait-awhile, igsc2023-casper}, carbon-energy trade-offs~\cite{Baolin2023:Clover, Gupta2022:Chasing, Jiang204:EcoLife}, carbon-accuracy\cite{Baolin2023:Clover}, carbon-cost trade-offs~\cite{Gao-2012-being-green, Murillo2024:CDNShifter}.
This paper considers the carbon-performance and carbon-energy trade-offs that are more prevalent in edge computing.



% The resource limitations of mobile devices have led to the wide adoption of edge computing~\cite{Satya09_Cloudlets, Satya17_emergence}. Recent research has analyzed the impact of the offloading and placement decisions and the related trade-offs ~\cite{Liu2023_Offloading, Ren2023_OffloadingPower, Wang2022_OffloadingDelay, liu2019delay, islam2021survey,  Balazs2021_placementsurvey, Goudarzi21Placement}. 
% For instance,  \citet{Wang2022_OffloadingDelay} and \citet{liu2019delay} focused on latency-aware offloading, while \citet{Ren2023_OffloadingPower, Goudarzi21Placement, Li2018_energy_placement} focused on energy-aware offloading and placement. 
% \citet{Li2018_energy_placement} and \citet{Goudarzi21Placement} focused on energy-aware placement, while \citet{Bahreini17_multicomponent} and \citet{Fan2019Cost} considered cost-aware placement as well as the latency-cost trade-offs.
%Although researchers addressed the sustainability of cloud interactive workloads and the benefits of continental-scale spatial shifting, ~\cite{igsc2023-casper, Gao-2012-being-green, sukprasert2024limitations, maji_hotcarbon23}, the carbon efficiency of edge applications has seen less attention. 
% For instance, researchers often assume that applications’ latency can be extended by tens or hundreds of milliseconds, which is not always feasible. In addition, they focus on the cloud context, where servers are homogenous, and resources are plentiful. Lastly, they have not addressed the prevalent carbon-performance and carbon-energy trade-off, highlighted in earlier work~\cite{hanafy23-war, hanafy2024gaia}.  
% In contrast, we underscore scenarios where the benefits of carbon-aware computing outweigh the performance overheads by uncovering opportunities in mesoscale regions while considering the resource limitations and heterogeneity of edge sites.

%In this paper, we point out several scenarios where spatial shifting reduces significant carbon emissions with much less latency overheads. 

%Recent research has shown that carbon-aware temporal and spatial shifting comes with performance ~\cite{sukprasert2023quantifying, Gao-2012-being-green, hanafy2023carbonscaler, igsc2023-casper, cloudcarbon} and cost ~\cite{Gao-2012-being-green, hanafy23-war, hanafy2024gaia, hanafy2023carbonscaler} overheads.  
%In contrast, we underscore scenarios where the benefits of carbon-aware computing outweigh the performance overheads.

%\noindent{\textbf{Edge Placement.}}
%The resource limitations of mobile devices have led to the wide adoption of edge computing~\cite{Satya09_Cloudlets, Satya17_emergence}. Recent research has analyzed the impact of the placement decisions and the related trade-offs ~\cite{Li2018_energy_placement, Balazs2021_placementsurvey, Shiqiang2017OnlinePlacement, Bahreini17_multicomponent, Fan2019Cost, Goudarzi21Placement}. 
%For instance,  \citet{Li2018_energy_placement} and \citet{Goudarzi21Placement} focused on energy-aware placement, while \citet{Bahreini17_multicomponent} and \citet{Fan2019Cost} considered cost-aware placement as well as the latency-cost trade-offs.
%For instance,  \citet{Li2018_energy_placement} designed a particle swarm optimization-based placement algorithm that minimized the total energy consumption under latency constraints. \citet{Goudarzi21Placement} generalized this objective and considered a multi-user, multi-application in heterogeneous edge server with a cloud connection. Researchers have also considered the operations costs of their applications where \citet{Bahreini17_multicomponent} optimized the total execution and relocation costs of edge workloads while employing a latency threshold. In addition, \citet{Fan2019Cost} considered the trade-offs between the two by using a trade-off factor that balances the cost and latency overheads.

%\noindent{\textbf{Carbon-Aware Spatial Shifting.}}
%In addition to cost and energy-aware placement,  researchers have proposed carbon-aware spatial shifting methods and systems by considering the carbon intensity across multiple cloud regions~\cite{igsc2023-casper, Gao-2012-being-green, sukprasert2023quantifying, maji_hotcarbon23, cloudcarbon}. 
%In this case, the objective is to minimize the total carbon emissions by placing services at greener locations. However, most of the research in this area does not consider the tight latency constraints of edge applications, assuming that applications’ latency can be extended by tens or hundreds of milliseconds, which is not always feasible. 
%In this paper, we point out several scenarios where spatial shifting reduces significant carbon emissions with much less latency overheads. 

% applications can reduce their carbon emissions without compromising the requirements of edge applications.


%\noindent{\textbf{Trade-offs Carbon-Aware Computing.}} 
%Recent research has shown that carbon-aware temporal and spatial shifting comes with performance ~\cite{sukprasert2023quantifying, Gao-2012-being-green, hanafy2023carbonscaler, igsc2023-casper, cloudcarbon} and cost ~\cite{Gao-2012-being-green, hanafy23-war, hanafy2024gaia, hanafy2023carbonscaler} overheads.  
%In contrast, we underscore scenarios where the benefits of carbon-aware computing outweigh the performance overheads.

%can yield considerable benefits with no or negligible performance overheads.

% The carbon benifits do not outway the overheads (everywhere).