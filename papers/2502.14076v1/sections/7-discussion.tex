We have shown the potential of mesoscale carbon intensity information in decarbonizing edge computing. In this section, we reflect on the adoption of \proposedsystem for futuristic edge infrastructures and the limitations.

\noindent {\bf Adopting \proposedsystem.}
Although the carbon intensity in many mesoscale regions is still opaque, we have shown that mesoscale spatial shifting can effectively reduce carbon emissions. In addition, the advances in carbon accounting and the continuing adoptions of local renewable energy will enable fine-grained carbon information at a region and even at the edge data center level. In this case, as highlighted in earlier research~\cite{acun2023carbon, Zheng2020:Curtailment, Gsteiger2024:Caribou}, carbon-aware spatial shifting becomes more crucial in the decarbonization of computing.  


\noindent {\bf Limitations.} Currently, \proposedsystem does not automatically redeploy applications to adapt to dynamic workloads and changes in carbon intensity, ensuring low computational overhead and service continuity. Additionally, while carbon-aware spatial shifting has potential and applicability for edge applications, our evaluations were limited to edge data centers with available data from the Akamai CDN traces and the WonderNetwork traces\cite{wonder-proxy-2020}.

% Despite the potential of carbon-aware spatial shifting and its applicability for edge applications, our analyses were limited to edge data centers with available information in the Akamai CDN traces and the WonderNetwork traces\cite{wonder-proxy-2020}. %For example, we only utilized XXX\% and YYY\% of the edge data centers in the US and Europe, respectively. 

% Furthermore, our analyses assumed a homogeneous load among edge data centers, which is reasonable as resource provisioning is typically proportional to load. However, our results did not account for the natural bias in computing demand, which is a function of the population density. 

\noindent {\bf Holistic Emissions Reduction.}
In this paper, we only focused on the {\em operational emissions} of edge workloads from energy consumption, while the {\em embodied emissions} from manufacturing servers is beyond the scope of this paper\cite{Eeckhout2024:FOCAL, Gupta2022:ACT, Li2023:Toward, Gupta2022:Chasing}. Nonetheless, we note that \proposedsystem does not require increases in the number of servers, and earlier research has shown how spatial shifting can help extend the hardware lifespan, amortizing its embodied emissions~\cite{Liu2024:Relocation, Switzer2023:Junkyard}.

% % \subsection{What-If Scenarios}\label{sec:discussion}

% \proposedsystem reduces carbon emissions in complex edge environments with small latency overhead. We explore the other potentials of applying \proposedsystem to edge computing. 


% \noindent\textbf{Traces limitations}
% Limited dataset. not all states have fine-grain carbon intensity data and latency data is missing 

% Run Europe with 105 sites.

% Some zones, like US-MIDW-MISO at 1106,425.96 $km^2$ cover multiple central US states, and FR at 539571.85 $km^2$ covers all of France, are quite large. However,  these larger zones do not impact our analysis, as we focus on regions with different carbon intensity traces. 
