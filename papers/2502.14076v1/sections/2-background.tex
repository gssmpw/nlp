This section provides background on grid energy's carbon intensity, carbon-aware workload optimization, and edge data centers. 

\subsection{Electric Grid Carbon Intensity}

The electricity supplied by the electric grid at a given location comes from a mix of generation sources, such as natural gas, coal, hydro, solar, and wind. The relative proportion of generation from each source varies from one region to another, depending on the types of generation sources present in each region. For example, as shown in ~\autoref{fig:cloud_energy_source}, in the Ontario region of Canada, most energy comes from nuclear and hydroelectric energy sources. At the same time, eastern European countries such as Poland have a more significant proportion of coal and natural gas. The carbon intensity of electricity is defined as the total $\text{CO}_2$ emissions per unit of electricity generation and is measured in \carbonunit. For each location, it is computed as the weighted average of the carbon intensity of the source energy mix at that location. \autoref{fig:cloud_ci} shows the carbon intensity of the energy supply in four different countries and regions ---  Ontario region of Canada, California and New York in the US, and Poland in Europe --- and shows that there are significant differences in the carbon intensity of electricity at the spatial granularity of countries or large geographic regions. 

As grid operators have begun to report their real-time energy generation mixes, third-party carbon information services, such as Electricity Maps~\cite{electricity-map} and WattTime~\cite{watttime}, have begun exposing this carbon intensity data to data center operators and applications via real-time APIs and forecasting services. Our paper assumes the availability of such carbon intensity data for carbon optimizations in edge data centers. 




\begin{figure}[t]
    \centering
    \subfloat[\centering Energy Mix]{{
        \includegraphics[width=0.45\linewidth]{figures/cloud_energy_source.pdf}}
        \label{fig:cloud_energy_source}
    }%
    \quad
    \subfloat[\centering Carbon Intensity ]{{
        \includegraphics[width=0.45\linewidth]{figures/cloud_ci.pdf}}
        \label{fig:cloud_ci}
    }%
    \caption{Energy mix and carbon intensity of four regions.}
    \label{fig:cloud_example}
\end{figure}



%%%%%%%%%%%%%%%%%%%%%%%%%%%%%%%%%%%%%%
%%% Snap Shots
%%%%%%%%%%%%%%%%%%%%%%%%%%%%%%%%%%%%%%
\begin{figure*}[t]
    \centering%
    \hfill
    \begin{subfigure}{0.225\linewidth}%
        \centering
        \includegraphics[width=\linewidth]{figures/fl_ci_geo_km.pdf}%
        \caption{Florida}
        \label{fig:cv_snapshot_r1}
    \end{subfigure}%
    \hfill
    \begin{subfigure}{0.21\linewidth}%
        \centering
        \includegraphics[width=\linewidth]{figures/cross_us_ci_geo_km.pdf}%
        \caption{West US}
        \label{fig:cv_snapshot_r2}
    \end{subfigure}%
    \hfill
    \begin{subfigure}{0.19\linewidth}%
        \centering
        \includegraphics[width=\linewidth]{figures/it_ci_geo_km.pdf}%
        \caption{Italy}
        \label{fig:cv_snapshot_r3}
    \end{subfigure}%
    \hfill
    \begin{subfigure}{0.175\linewidth}%
        \centering
        \includegraphics[width=\linewidth]{figures/cross_eu_ci_geo_km.pdf}%
        \caption{Central EU}
        \label{fig:cv_snapshot_r4}
    \end{subfigure}%
    \includegraphics[width=0.085\linewidth]{figures/CI_label.pdf}
    \hfill
    \hfill 
    \caption{Carbon intensity snapshots of four mesoscale regions, highlighting variations across zones.}%
    \label{fig:cv_snapshot}%
    \vspace{-.4cm}
\end{figure*}

\subsection{Carbon-aware Workload Optimizations}

The availability of real-time carbon intensity data has motivated cloud providers and applications to schedule workloads based on variations in the carbon intensity of electricity. The resulting carbon-aware scheduling approaches can be broadly viewed as workload shifting, which exploits temporal and spatial variations in the carbon intensity of electricity. Temporal workload shifting exploits temporal fluctuations in carbon intensity at a given location by scheduling (or delaying) jobs to periods of low carbon intensity. Such techniques are well suited for batch workloads, which have temporal flexibility and can tolerate delays in their completion times.  There have been numerous recent works that leverage temporal workload shifting to reduce the carbon emissions of batch workloads in the cloud~\cite{acun2023carbon, wait-awhile, sukprasert2024limitations, ecovisor, hanafy2023carbonscaler}.  %Note that temporal workload shifting is not applicable to latency-sensitive edge applications, since latency-sensitive requests can generally not be delayed to a lower-carbon time period. 

In contrast, spatial workload shifting exploits spatial variations in carbon intensity across locations by moving jobs or requests to data center locations with lower carbon intensity supply. 
Such spatial shifting optimizations have been studied in the cloud context by moving cloud workloads across data center locations that span large geographic regions, countries, or even continents \cite{cloudcarbon,sukprasert2024limitations, Gao-2012-being-green, Gsteiger2024:Caribou, Murillo2024:CDNShifter}. 
For example, as depicted in \autoref{fig:cloud_ci}, cloud workloads can be moved from the New York region of a public cloud to the Ontario region, whose electricity supply has a lower carbon intensity. %or can be moved from the central Europe region to the Sweden region. 
In theory, spatial shifting can be implemented for both interactive workloads, such as web services, as well as batch workloads. In practice, however, migrating interactive requests to distant data centers increases their user-perceived latency, and hence, spatial shifting has been primarily utilized for batch applications, such as machine learning training~\cite{cloudcarbon}.  Prior work has also shown that spatial shifting generally has much more potential for reducing carbon compared to temporal shifting~\cite{sukprasert2024limitations}.  This insight derives from the fact that there tend to be much larger differences in carbon between locations than within any one location over time. 


%Further, since the temporal variations in the carbon intensity of a given location are limited while spatial variations across locations can be significant, prior work has shown more substantial carbon footprint reduction from spatial shifting (e.g., by over XXX ) than temporal shifting. \todo{this sentence is out of order.}


\subsection{Edge Data Centers}

Edge computing involves deploying computing resources in the form of small server clusters at the network's edge close to end users. Edge computing is well-suited for low latency services since it avoids network delays incurred by traversing to more distant cloud data centers. For example, 
regional edge clusters, deployed by edge or even cloud providers, have been used to host latency-sensitive applications such as mobile offloading, augmented reality, and deep learning inference \cite{Satya17_emergence,Satya09_Cloudlets}.  
Another example is a content delivery network that operates large geo-distributed edge clusters and serves web and multimedia content to users from proximate edge locations.

While edge data centers can optimize their carbon footprint via temporal workload shifting, such methods are ill-suited for interactive or latency-sensitive applications that are prevalent at the edge, since such workloads cannot be delayed or time-shifted. In contrast, spatial shifting optimizations have traditionally been performed at larger continental or global scales, i.e., across entire continents or across multiple continents, to exploit carbon intensity variations present at that scale~\cite{cloudcarbon, Gsteiger2024:Caribou, sukprasert2024limitations}. While these methods work well for cloud-based batch workloads, spatial shifting of interactive edge workloads at such large scales results in large latency increases. Hence, spatial shifting has not been considered for edge applications in prior work. We argue that spatial shifting is feasible even in the edge context by exploiting mesoscale variations in grid carbon intensity that are beginning to appear in today's energy grids. 





% Below, we provide background on edge computing and applications, grid carbon intensity, and carbon-aware scheduling.
% % \textbf{Edge-native Applications}


% \subsection{Edge Computing and Applications}
% Edge computing involves deploying computing and storage resources at the network's edge --- geographically closer to where user data is generated or needed --- to provide users with low-latency access and processing power. 
% Edge sites are often shared among multiple users with varying latency and resource requirements. However, unlike cloud platforms, edge sites are more resource- and energy-constrained require careful placement decisions~\cite{Balazs2021_placementsurvey, xu2020:edge-intelligence}.
% To harness the benefits of edge computing, emerging edge native applications, such as cloud gaming and mobile AR, are applications that require low-latency processing to ensure high user responsiveness, imposing latency constraints on edge processing. 
% In addition, advances in AI accelerators, made edge applications often rely on AI models to ensure higher quality experiences. The low latency, high bandwidth, and high resource requirements, combined with the resource limitations of the edge, made edge application placement a highly challenging task~\cite{xu2020:edge-intelligence}. For example, ~\cite{Goudarzi21Placement, Li2018_energy_placement} propose energy-aware placement techniques for edge applications,  while ~\cite{Liu2023_Offloading, Zhang2019_HeteroEdge, Liang2020:AIEdge} focus on implications of resource and applications heterogeneity in edge systems. %However, the proliferation of edge applications has raised concerns regarding their increasing carbon footprint.


% % made edge applications  

% % %In addition, many AI applications use machine learning models, often in the form of Deep Neural Networks (DNNs), to process data at the edge. 
% % %Since many edge applications are inherently latency-sensitive, it is important to minimize the performance impacts of user applications by properly selecting where requests are processed. 

% % This can be achieved through scheduling mechanisms that respect certain Service Level Objectives (SLOs), when possible, and by carefully maximizing the utilization and sharing of each server across tenant applications to increase resource efficiency.

% % %Our work focuses on emerging edge native applications such as cloud gaming and mobile AR, where applications require low-latency processing to ensure high user responsiveness, imposing latency constraints on edge processing. 
% % In addition, many AI applications use machine learning models, often in the form of Deep Neural Networks (DNNs), to process data at the edge. 
% % Recent advances have produced a number of sophisticated and highly effective DNNs for common processing tasks, such as classification, object detection, and segmentation. 
% % These advances have led to a set of libraries with pre-trained DNNs, such as ResNet~\cite{resnet}, Inception~\cite{inception}, MobileNet~\cite{mobilenet}, and Yolo~\cite{yolo}, among others. 
% % An edge cloud developer can easily integrate one of these pre-trained DNN models within their application for inference tasks like object detection or recognition over images. 
% % Alternatively, developers can train custom DNNs for their applications using user-friendly ML frameworks.

% \subsection{Electricity Grid Carbon Intensity}
% The carbon intensity of electricity, measured in \carbonunit, represents the weighted average of the energy sources. For example, locations that highly utilize fossil fuel, e.g., coal and gas, to cover their customers’ energy demand often have higher carbon intensity than regions where renewable, e.g., solar and wind. The intermittent nature of renewables frequently alters the supply mixture throughout the day and, hence, the carbon intensity. 
% Moreover, the variations of energy sources across \textit{locations} result in unique carbon intensity. 

% However, until recently, carbon intensity information was only known at the country or state level, greatly reducing the benefits of carbon-aware resource management, especially in edge-setting, which latency is tightly bound. Fortunately, carbon information services such as ElectricityMap~\cite{electricity-map} and WattTime~\cite{watttime} were able to utilize the energy market to estimate this \textit{mesoscale} carbon intensity information. In ~\autoref{sec:carbon_analysis}, we analyze the benefits of the newly available information and how it can be leveraged to minimize carbon emissions of edge services.

% % Although until recently,  and \textit{temporal} carbon intensity data was opaque, c. In section, 

% % used ISO-level information to estimate the real-time carbon intensity of the grid.
% % For instance, the US is divided into ten Independent System Operators (ISOs), each containing one or more states, each with different energy and carbon intensity profiles. For instance,  
% % . 

% % have developed real-time carbon intensity APIs and forecasting services. 

% % Using the recently available carbon intensity data, we initially confirm the feasibility of spatial shifting across geographically nearby edge sites for carbon-saving (See Section~\ref{sec:carbon_analysis}). The analysis shows that distributing the workload inside a region or across neighboring regions can reduce at least $2\times$ carbon emissions with a small amount of network delay \walid{verify numbers}.

% %generation measures the amount of carbon dioxide emissions produced per unit of electricity generated.  Regions with a high proportion of fossil fuel-based generation, such as coal and natural gas, generally have higher carbon intensities compared to regions with more renewable energy or nuclear power.  Each region’s mix of generators differs based on its unique climate and access to natural resources. For example, while some regions have abundant hydropower due to the presence of large rivers, such as in the northwest U.S., others have abundant solar power, such as in the southwest U.S.  Additionally, renewable energy, like wind and solar, has more significant fluctuations in generation due to the changes in weather conditions. Thus, regions with a higher share of renewable energy, like wind and solar, tend to have more carbon-intensity variations over time. 

% % Until recently, the grid energy carbon intensity was opaque to consumers. The balancing authorities have made information about active generators and their real-time energy output accessible via web APIs. Consumers now have greater visibility into the carbon intensity of the electricity they consume.  Furthermore, carbon information services, like Electricity Map and WattTime leverage real-time generation data from the grid, and combine it with models that incorporate each generator’s characteristics such as fuel type and emissions profile.  By analyzing this information, these services can infer the real-time or future carbon intensity of grid energy in each region. These services provide users with easy access to carbon intensity via web APIs, allowing for integration into various applications, platforms, and services. This enables consumers to make more environmentally conscious decisions about their energy usage in real-time. 



% \subsection{Carbon-Aware Optimizations}
% The variability in energy's carbon intensity has motivated users and cloud providers to schedule their workloads based on the availability of low-carbon energy. Carbon-aware scheduling often relies on \emph{temporal} and \emph{spatial} flexibility of the workloads. For example, temporal-shifting methods utilize completion time flexibility of non-interactive batch jobs to defer them to times when carbon intensity is low~\cite{wait-awhile, hanafy2023carbonscaler, hanafy23-war, hanafy2024gaia, sukprasert2023quantifying}. 

% In contrast to temporal shifting, which only works for non-interactive applications, spatial shifting utilizes spatial flexibility by running workloads in locations where energy is greener and works for both batch and latency-sensitive applications. For instance, researchers have carbon-aware placement and load-balancing methods to reduce the carbon footprint of AI workloads and cloud applications~\cite{cloudcarbon, igsc2023-casper, maji_hotcarbon23, sukprasert2023quantifying}. However, previous work did not address opportunities and challenges of carbon-aware placement of edge workloads. For instance, \citet{maji_hotcarbon23} did not consider capacity constraints, while ~\citet{igsc2023-casper} did not consider resource heterogeneity. Moreover, these works relied on large-scale carbon intensity variations (e.g., country or state level) by shifting for 10s of ms, overlooking the applicability and benefits under edge applications requirements. 

% % \subsection{Motivation}


% % \begin{figure}[tb]
% %     \centering
% %     \includegraphics[width=0.8\linewidth]{figures/wa_ci_geo.pdf}
% %     \caption{Fine-grained carbon intensity for regions: up to 10 zones inside Washington State are with carbon intensity information. }
% %     \label{fig:fine_grained_ci}
% %     \vspace{-5mm}
% % \end{figure}

% % \begin{figure}[tb]
% %     \centering
% %     \includegraphics[width=0.8\linewidth]{figures/ca_ci_geo.pdf}
% %     \caption{Fine-grained carbon intensity for regions: up to 6 zones inside California State are with carbon intensity information. }
% %     \label{fig:fine_grained_ci}
% %     \vspace{-5mm}
% % \end{figure}






% % \subsection{Regional Carbon Analysis}

% % Carbon information provider (ElectricityMap) provides fine-grained --> blog


% % Region-level spatial-shifting

% % inter-region
% % inter-region

% % show the variations of the carbon intensity of regions:


% % The trade-off between carbon saving and latency: 

% % carbon emission: 
% % latency: 







