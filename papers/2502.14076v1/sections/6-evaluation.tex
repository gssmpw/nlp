
In this section, we evaluate the performance of \proposedsystem using real experiments and large-scale simulations. We start with evaluating \proposedsystem in mesoscale edge deployments, showcasing its efficiency in reducing carbon emissions on a regional scale. Next, we extend our analysis to continental-scale edge data centers (e.g., a Content Delivery Network, CDN), highlighting that the benefits of saving carbon emissions with granular carbon intensity are commonplace. In doing so, we address the following questions:

\begin{enumerate}[leftmargin=*]
    \item {\em What are the potential carbon savings of spatial shifting for an edge provider with multiple regional edge data center locations?}
    \item {\em How can a CDN exploit mesoscale variations for carbon-aware edge hosting across a large network of edge locations? How do latency limits affect potential carbon savings?}
    \item {\em How do seasonal variations, demand, and capacity affect carbon savings and placement decisions?}
    \item {\em How does the heterogeneity in edge resources impact savings? What are the carbon-energy trade-offs in these settings?}
\end{enumerate}

Next, we detail our real-world datasets, experimental settings, baselines, and evaluation metrics.

\begin{figure}[t]
    \centering
    \includegraphics[width=0.6\linewidth]{figures/7_energy_consumption_models_legend.pdf}\\
    \subfloat[\centering Energy]{\includegraphics[width=0.3\linewidth]{figures/7_energy_consumption_models.pdf}%
    \label{fig:models_profile_energy}%
    }%
    % \quad
    % \subfloat[\centering SLO-App]{
    % \includegraphics[width=0.22\linewidth]{figures/ru_gpu_mem_latency.pdf}
    % \label{fig:comparion_regions_real_latency}
    % }%
    % \quad
    \subfloat[\centering GPU Memory]{
    \includegraphics[width=0.3\linewidth]{figures/7_gpu_memory_models.pdf}
    \label{fig:models_profile_memory}
    }%
    % \quad
    \subfloat[\centering Inference Time]{
    \includegraphics[width=0.3\linewidth]{figures/7_inference_time_models.pdf}
    \label{fig:models_profile_latency}
    }%
    \caption{Energy consumption, memory usage, and inference time of ML workloads across devices.}
    \label{fig:models_profile}
\end{figure}

\subsection{Experimental Methodology}
\subsubsection{Real World Traces.}\label{sec:real_world_traces}\hfill\\
This section describes our real-world traces and how we combined them in our evaluations.

\noindent{\bf Carbon Intensity Traces.}
We utilize the carbon intensity traces for 2023 from Electricity Maps ~\cite{electricity-map}. The trace contains the hourly carbon intensity, measured in \carbonunit, for 148 carbon zones worldwide, including 54 and 45 in the US and Europe, respectively. Electricity Maps define each zone according to the structure of the regional electricity grid. For example, the results show that the area of carbon zones can be as small as 123.73 $km^2$ (Tallahassee, Florida).% and down to 560.98 $km^2$ in Europe (e.g., DK-BHM in Denmark), providing fine-grain visibility of carbon intensity at the mesoscale. \walid{Electricty maps didn't come up with these regions, they follow the energy grid also, US-FLA-TAL is not an official naming scheme.}

\noindent{\bf Latency Traces.} To incorporate realistic network latencies between edge data centers. 
We used the round-trip latency traces from WonderNetwork~\cite{wonder-proxy-2020}, which provides ping times (in milliseconds) between 246 major cities worldwide. The data covers 64 cities in the US and 64 cities in Europe. Each city is associated with longitude and latitude coordinates. The data highlights that in the US, latency can range between 0.93 ms to 184.67 ms, with an average of 43.17 ms, while in Europe, it ranges from 1.12 ms to 156.74 ms, with an average of 36.94 ms. %\todo{better do a small plot of how the ranges differ between the US and Europe in both carbon and latency} \lilly{Are these details necessary? }

\noindent{\bf Edge Workloads.}
We utilize two types of compute-intensive edge workloads: a CPU-based edge application that emulates edge sensor data processing and a GPU-based model-serving application that emulates edge AI inference. The CPU-based application is implemented using Python and numpy v1.26, while the model-serving application uses TensorRTv10.2 and CUDA 12.1.
~\autoref{fig:models_profile} depicts the three selected models that cover different tasks and resource requirements: EfficientNetB0~\cite{efficient-net}, ResNet50~\cite{resnet}, and YOLOv4~\cite{bochkovskiy2020yolov4}. ~\autoref{fig:models_profile_energy} highlights the diversity of our workloads, where energy consumption can reach 45$\times$  across models in the same device, and 2$\times$ across devices for the same model. Similarly, ~\autoref{fig:models_profile_memory} shows that memory also differs across models and devices. We evaluate the effect of heterogeneity in ~\autoref{sec:eval_hetero}. Unless mentioned otherwise, we assume a round-trip network latency constraint of 20 ms ($\sim$500km) in our experiments. 

%\lilly{Edge workload is not traces. Need to move it to 6.1.2} \walid{I agree this may not be the best place, but we only use them in simulations as profiles.}


\noindent{\bf Edge Data Centers.}
To emulate real-world edge deployment, we utilize Akamai CDN traces, which include the location of edge data centers globally identifiable by their coordinates. 
% We apply even distributed demand and capacity across edge data centers and examine their impact on carbon savings using different distributions in ~\autoref{sec:eval_demand_capacity}.

% \noindent{\bf Edge Services Traces.}
% To emulate real-world edge deployment, we utilize Akamai CDN traces, which include the location of 2691 edge sites globally highlighted by their GPS coordinates. 
% \todo{Double check and rethink how the data is used.}

\noindent{\bf Integrating Traces.}
Since the availability of granular data differs between traces, we integrate the traces using the following steps:
\begin{enumerate}[leftmargin=*]
    % \item We do not inject extra network latency between the end-device and the edge data centers in the same zone and let the device only be exposed to the processing latency. \lilly{This is for experiment setup. Should be mentioned in 6.1.2, Mesoscale edge providers.}\walid{We don't add any latency, but its ok to move.}
    \item We map each data center in the Akamai trace to its corresponding carbon intensity zone using its coordinates.
    \item We compute the cross-data center latency by mapping each data center to the nearest city. We assume that users exhibit the same latency as their original edge data center.
    \item We limit our evaluations to Akamai CDN edge data centers where the carbon intensity and latency traces are available. 
    \item In the case of multiple data centers in the same city, we combine them into a single data center.
    % \item Lastly, we only consider the average carbon intensity, which we directly compute from the trace. This is a reasonable assumption as our planning decisions are coarse-grained, where carbon intensity and weather forecasts are predictable.
\end{enumerate}

\subsubsection{Experimental Setup} \hfill\\
We evaluate our \proposedsystem under two deployment scenarios:
%\lilly{evaluate \proposedsystem or carbon-aware policy?}%Both are fine

\noindent {\bf Mesoscale Regional Edge Deployment.} We first evaluate the performance of \proposedsystem in mesoscale edge deployments. In our experiments, we use eleven servers to emulate a mesoscale edge network, which comprises five edge data centers distributed across five cities. Each data center is represented by a server and associated with an end device, also represented by a server, for issuing application placement and service requests. \proposedsystem operates on a separate server to prevent any interference. The eleven servers are Dell PowerEdge R630, each equipped with a 40-core Xeon E5-2660v3 CPU, 256GB of memory, and a 1Gb/s network connection. Additionally, each edge server contains an NVIDIA A2 GPU that has 1280 CUDA cores, 16GB of memory, and 60W of maximum power consumption, enabling us to evaluate \proposedsystem on a GPU cluster. Lastly, we used a workload generator based on Locust\footnote{Locust: \url{https://locust.io/}} and used the Linux traffic control tool ($tc$\footnote{ \url{https://man7.org/linux/man-pages/man8/tc.8.html}}) to emulate network latency across edge data centers.


\noindent {\bf Continental-scale CDN Edge Deployment.}
In addition to mesoscale evaluations, we show how mesoscale variations can help decarbonize CDN edge deployment that spans an entire continent (e.g., Akamai CDN and AWS Local Zones). In this case, we utilize trace-driven simulations to evaluate the year-long global behavior of the CDN across the US and Europe.
In ~\autoref{sec:eval_hetero}, we show the impact of heterogeneity across data centers, where profiled the ML workloads mentioned above on Nivida A2 (1280 CUDA cores, 16GB memory, 60W), NVIDIA Jetson Nano (1024 CUDA cores, 8GB memory, and 15W ), and NVIDIA GTX 1080 (2560 CUDA cores, 8GB memory, 180W).   %, the number of servers per edge data center is 3, and number of applications is 6 per edge data center, each with 10 requests/sec. 



% Finally, we compute the carbon emissions and associated overheads by repeating the experiment over multiple runs, assuming that applications arrive in batches.



\subsubsection{Baselines} \hfill\\
We evaluate \proposedsystem against multiple baselines.
% inspired on state-of-the-art approaches.

\begin{enumerate}[leftmargin=*]
\item {\bf \latencyaware}: This policy allocates workloads to the nearest edge data centers to minimize latency overhead, a strategy commonly employed in edge computing~\cite{yi2017:Lavea}.

\item {\bf \energyaware}: This policy distributes workloads to energy-efficient edge data centers to decrease energy consumption~\cite{Goudarzi21Placement, Li2018_energy_placement}. We implemented this policy by minimizing energy usage while adhering to latency and resource constraints. 
 
\item {\bf \intensityaware}: This policy greedily assigns workloads to the greenest edge data centers with the lowest carbon intensity values while respecting the latency and resource constraints. 

% greedily allocating workloads on the greenest edge data centers that respect latency and resource constraints.  

% This policy represents earlier work ~\cite{igsc2023-casper} that do not consider energy proportionality and differences in energy efficiency between workloads and servers. In this case, the objective is maximizing workloads in low-carbon regions and not the total carbon emissions. 
%We note that in homogeneous settings, this policy maps to the proposed \proposedsystem policy. \lilly{this is not true. We also consider base power}  
\end{enumerate}  



\subsubsection{Evaluation Metrics} \hfill\\
We evaluate \proposedsystem with three key metrics: Carbon Emissions, Response Time, and Energy Consumption, where we report absolute values as well as carbon savings (\%), {\em round-trip} latency increases (ms), and energy consumption compared to the \latencyaware baseline. 


% \begin{figure}[t]
%     \centering
%     \includegraphics[width=0.7\linewidth]{figures/fl_labels_large.pdf}\\
    
%     \subfloat[\centering Carbon Intensity]{{
%         \includegraphics[width=0.28\linewidth]{figures/ce_carbon_intensity.pdf}}
%         \label{fig:carbon_emission_ts_intensity}
%     }%
%     \quad
%     \subfloat[\centering Latency-aware]{{
%         \includegraphics[width=0.28\linewidth]{figures/ce_carbon_agnostic.pdf}}
%         \label{fig:carbon_emission_ts_agnostic}
%     }%
%     \quad
%     \subfloat[\centering CarbonEdge]{{
%         \includegraphics[width=0.28\linewidth]{figures/ce_carbon_aware.pdf}}
%         \label{fig:carbon_emission_ts_aware}
%     }%
%     \caption{Carbon intensity and emissions across edge data centers in Florida.}
%     \label{fig:carbon_emission_ts}
% \end{figure}

% \begin{figure}[t]
%     \centering
%     \includegraphics[width=0.85\linewidth]{figures/response_time_horizonal.pdf}
%     \caption{End-to-end response times of applications across edge data centers in Florida.}
%     \label{fig:response_time_ts}
% \end{figure}

\begin{figure}[t]
    \centering
    \includegraphics[width=0.9\linewidth]{figures/ce_carbon_aware_legend.pdf}\\
    
    \subfloat[\centering Carbon Intensity]{{
        \includegraphics[width=0.3\linewidth]{figures/ce_carbon_intensity.pdf}}
        \label{fig:carbon_emission_ts_intensity}
    }%
    % \quad
    \subfloat[\centering Latency-aware]{{
        \includegraphics[width=0.3\linewidth]{figures/ce_carbon_agnostic.pdf}}
        \label{fig:carbon_emission_ts_agnostic}
    }%
    % \quad
    \subfloat[\centering CarbonEdge]{{
        \includegraphics[width=0.3\linewidth]{figures/ce_carbon_aware.pdf}}
        \label{fig:carbon_emission_ts_aware}
    }%
    \caption{Carbon intensity and emissions across edge data centers in Florida.}
    \label{fig:carbon_emission_ts}
\end{figure}

\begin{figure}[t]
    \centering
    \includegraphics[width=1\linewidth]{figures/response_time_horizonal.pdf}
    \caption{End-to-end response times of applications across edge data centers in Florida.}
    \label{fig:response_time_ts}
\end{figure}



 % Note that carbon savings and latency increases are the same across applications.

\subsection{Mesoscale Evaluation}\label{sec:eval_mesoscale}
We start by evaluating the performance of the \proposedsystem prototype for two regional deployments (in Florida and Central Europe) over a 24-hour period. We compare the performance of \proposedsystem to the \latencyaware baseline. \autoref{fig:carbon_emission_ts} illustrate the carbon intensity and emissions of the CPU-based application across five zones within the Florida region. 
As explained in ~\autoref{sec:carbon_analysis}, zones in Florida exhibit high variations (see ~\autoref{fig:carbon_emission_ts_intensity}). ~\autoref{fig:carbon_emission_ts_agnostic} shows the carbon emissions of the \latencyaware policy, which highly resemble each zone's carbon intensity in \autoref{fig:carbon_emission_ts_intensity}. In contrast, \autoref{fig:carbon_emission_ts_aware} shows how \proposedsystem places all applications in the greenest zone (Miami), resulting in an equivalent 20-23 \emissionunit of emissions for all applications.
\autoref{fig:response_time_ts} lists how the response time changes between \latencyaware and \proposedsystem across different data centers. As expected, increases in response time are limited due to the proximity of different data centers, where the response time increases are <10.1 ms, with an average increase of 6.61 ms.  %\lilly{keep the concepts consistent, e.g., zones, data centers, response time.}

~\autoref{fig:comparion_regions_real} depicts the aggregate emissions and latency overheads across regions for the CPU-based and GPU-based applications (ResNet50). As shown in ~\autoref{fig:comparion_regions_real_carbon}, carbon emissions vary significantly across regions and applications. For instance, in Central Europe, total carbon emissions are reduced by up to 2.6$\times$ and 10.3$\times$ for the \latencyaware and \proposedsystem policy compared to those in Florida, where total emission is a function of the average carbon intensity, as highlighted in earlier research~\cite{hanafy2023carbonscaler}.
The figure also highlights how power consumption impacts total carbon emissions. For instance, the GPU-based application emits 54.7\% less carbon, which is proportional to the differences in power consumption between the CPU-based application and the GPU-based application. However, since the proposed system implements the same placement decisions apart from the application requirements, the carbon savings and latency increases remain consistent. Overall, \proposedsystem lowers carbon emissions by 39.4\% in Florida and 78.7\% in Central EU. Meanwhile, response time increased by 6.6 ms for Florida and 10.5 ms for Central EU (shown in ~\autoref{fig:comparion_regions_real_latency}).

% ~\autoref{fig:comparion_regions_real_saving} and ~\autoref{fig:comparion_regions_real_latency} show 
% the carbon savings and latency increases across regions for either application. As illustrated, 



\noindent \textit{\textbf{Key Takeaways.} 
In mesoscale edge settings, \proposedsystem can highly optimize the carbon emissions resulting in 39.4\% and 78.7\% carbon savings for Florida and Central EU, respectively. 
}

\begin{figure}[tb]
    \subfloat[\centering Carbon Emissions ]{\includegraphics[width=0.55\linewidth]{figures/carbon_emission_comparison.pdf}%
    \label{fig:comparion_regions_real_carbon}%
   }
   % \hfill
   %  \subfloat[\centering Carbon Savings ]{\includegraphics[width=0.24\linewidth]{figures/carbon_saving.pdf}%
   %  \label{fig:comparion_regions_real_saving}%
   %  }% 
    % \hfill
    \subfloat[\centering Latency Increases]{
    \includegraphics[width=0.4\linewidth]{figures/increased_latency.pdf}
    \label{fig:comparion_regions_real_latency}
    }%
    % \quad
    % \subfloat[\centering Carbon Saving (\emissionunit)]{\includegraphics[width=0.3\linewidth]{figures/sci_rnet50.pdf}%
    % \label{fig:comparion_app}%
    % }%
    \caption{Performance of \proposedsystem across applications, policies, and locations.}
\label{fig:comparion_regions_real}
\end{figure}



% \begin{figure}[tb]
%     \subfloat[\centering Online and Offine comparison ]{
%     \includegraphics[width=0.25\linewidth]{figures/1_resnet_carbon_saving_online_offline.pdf}
%     \label{fig:large_scale_diff_carbon}
%     }%

%     \caption{Incremental online compared to offline in the US and Europe.}
%     \lilly{Newly added}
%     \label{fig:large_scale_diff}
% \end{figure}


\subsection{Mesoscale Evaluation for a CDN}\label{sec:eval_global}
In this section, we evaluate \proposedsystem for deploying edge applications in a continental scale CDN using a year-long simulation. %\walid{remove number of data centers}
We focus our evaluation on US %(44 edge data centers)
and European CDN edge data centers only %(33 edge data centers) 
since fine-granular carbon intensity information for other continents was unavailable.  
In a CDN, edge applications arrive at edge data centers over time. Each edge application comes from a specific zone, and its placement is limited to a subset of edge data centers within a certain radius (or latency) of that site.

% , with different applications arriving at different CDN edge sites over time.

%As noted earlier, we leveraged the daily average carbon intensity to make placement decisions and repeated the placement process across five days per month.

\subsubsection{Year-long Performance Evaluation}\label{sec:eval_understanding}\hfill\\
\autoref{fig:large_scale} presents the year-long performance of \proposedsystem in terms of carbon savings and latency overheads when considering a latency limit of 20 ms ($\sim$500 km). As shown, \proposedsystem reduces carbon emissions by 49.5\% in the US and 67.8\% in Europe. Meanwhile, the average latency increases by 10.8 ms in the US and 10.5 ms in Europe. We note that Europe experiences higher carbon savings as the European data centers reside in greener zones, and the carbon intensity variations across these data centers are larger than those in the US. In addition, ~\autoref{fig:large_scale_cdf} illustrates the workload shifting with \latencyaware and \proposedsystem in the US and Europe. The results highlight that \proposedsystem shifts workloads toward low-carbon edge locations. For instance, compared to the \latencyaware baselines, \proposedsystem increases application execution at 200\emissionunit by 40\% and 33.9\% for the US and Europe, respectively. Moreover, the figure highlights examples where edge data centers do not have any of their load shifted as they are far away from other greener regions. For instance, in the US, the edge data center in Salt Lake City, Utah, does not offload any of its load.

% Real workload
% Carbon savings: Euorpe: 65.7195535985085  US: 47.7144397057836
% Latency increases: Europe: 10.552984848484847 US: 11.380166666666668

% , which follows the average carbon intensity across the US and Europe. The figure also highlights how the \proposedsystem changes the demand allocation by shifting the demand towards low carbon zones. 
% Note that although the load distribution has shifted highly towards low carbon, some regions do not experience any change. For instance, compared to the \latencyaware baselines, \proposedsystem increase application executing at 200\emissionunit by 40\% and 33.9\% for the US and Europe, respectively. Moreover, the graph highlights examples, where edge data centers do not have any of their load shifted as they are far away from other greener regions. For instance, in the US, the edge data center in Salt Lake City, Utah, does not offload any of its load.

\noindent \textit{\textbf{Key Takeaways.} 
By shifting the demand towards low carbon zones, \proposedsystem decreases carbon emissions by 49.5\% and 67.8\% for the US and Europe, respectively, while increasing the round-trip latency by less than 11 ms.
}

\begin{figure}[tb]
    \subfloat[\centering Carbon  ]{
    \includegraphics[width=0.26\linewidth]{figures/1_resnet_carbon_saving.pdf}
    \label{fig:large_scale_carbon}
    }%
    % \hfill
    \subfloat[\centering Latency  ]{
    \includegraphics[width=0.26\linewidth]{figures/1_resnet_increased_latency.pdf}
    \label{fig:large_scale_latency}
    }%
    % \hfill
    \subfloat[\centering Load Distribution]{
    \includegraphics[width=0.42\linewidth]{figures/1_global_placements.pdf}
    \label{fig:large_scale_cdf}
    }%
    % \subfloat[\centering Carbon (real-workload) ]{
    % \includegraphics[width=0.15\linewidth]{figures/1_resnet_carbon_saving_real_workload.pdf}
    % \label{fig:large_scale_carbon_real_workload}
    % }%
    % % \hfill
    % \subfloat[\centering Latency (real-workload) ]{
    % \includegraphics[width=0.15\linewidth]{figures/1_resnet_increased_latency_real_workload.pdf}
    % \label{fig:large_scale_latency_real_workload}
    % }%

    \caption{Carbon savings, latency increases, load distribution in the US and Europe.}
    %\lilly{Online version now}
    \label{fig:large_scale}
\end{figure}

% \begin{figure}[tb]
%     \centering
%     \subfloat[\centering Carbon Saving ]{\includegraphics[width=0.28\linewidth]{figures/5_SLOs_carbon_saving.pdf}%
%     \label{fig:latency_effect_carbon}%
%     }%
%     \quad
%     \subfloat[\centering Latency Overhead ]{
%     \includegraphics[width=0.28\linewidth]{figures/5_SLOs_increased_latency.pdf}
%     \label{fig:latency_effect_overhead}
%     }%
%     \quad
%     % \subfloat[\centering Server Activation ]{
%     % \includegraphics[width=0.25\linewidth]{figures/5_SLOs_servers_usa.pdf}
%     % \label{fig:5_slo_severs_usa}
%     % }%
%     \subfloat[\centering Placements ]{
%     \includegraphics[width=0.33\linewidth]{figures/5_SLOs_apps_usa.pdf}
%     \label{fig:latency_effect_placement}
%     }%
%     % \quad
%     \caption{Effect of latency tolerance on carbon savings and latency overhead.}
%     \todo{Make (a) box plots -- we need to comment on the minimum, make (b) carbon saving per ms, and make(c) a CDF for placement with six lines (latency-aware, 10, 20, 30, 40, 50)} \lilly{Updated in below figure.}
%     \label{fig:latency_effect}
% \end{figure}





% \begin{figure}[tb]
%     \centering
%     \subfloat[\centering Carbon Saving ]{\includegraphics[width=0.25\linewidth]{figures/5_SLOs_carbon_saving.pdf}%
%     \label{fig:latency_effect_carbon}%
%     }%
%     \quad
%     \subfloat[\centering Latency Overhead ]{
%     \includegraphics[width=0.25\linewidth]{figures/5_SLOs_carbon_per_ms_incremental.pdf}
%     \label{fig:latency_effect_overhead}
%     }%
%     \quad
%     % \subfloat[\centering Server Activation ]{
%     % \includegraphics[width=0.25\linewidth]{figures/5_SLOs_servers_usa.pdf}
%     % \label{fig:5_slo_severs_usa}
%     % }%
%     \subfloat[\centering Placements in the USA ]{
%     \includegraphics[width=0.4\linewidth]{figures/5_SLOs_placements.pdf}
%     \label{fig:latency_effect_placement}
%     }%
%     % \quad
%     \caption{Effect of latency tolerance on carbon savings and latency overhead.}
%     \label{fig:latency_effect}
% \end{figure}

%     Region  Latency Limit          Metric      Value  Standard Deviation
% 0   Europe              5  Carbon Savings   1.062535           11.313773
% 1   Europe             10  Carbon Savings  45.502631            7.402397
% 2   Europe             15  Carbon Savings  58.781957            5.843898
% 3   Europe             20  Carbon Savings  69.023864            6.119543
% 4   Europe             25  Carbon Savings  71.778622            5.505896
% 5   Europe             30  Carbon Savings  75.348484            4.714934
% 6   Europe             35  Carbon Savings  86.271432            3.274339
% 7   Europe             40  Carbon Savings  92.311505            2.301611
% 8   Europe             45  Carbon Savings  95.622927            1.825207
% 9   Europe             50  Carbon Savings  97.862387            1.064726
% 10     USA              5  Carbon Savings  12.307002            7.286185
% 11     USA             10  Carbon Savings  29.990326            6.035541
% 12     USA             15  Carbon Savings  44.576483            4.935454
% 13     USA             20  Carbon Savings  53.373752            4.599282
% 14     USA             25  Carbon Savings  61.226620            3.842407
% 15     USA             30  Carbon Savings  66.024072            3.612312
% 16     USA             35  Carbon Savings  70.166792            3.139480
% 17     USA             40  Carbon Savings  74.150786            2.553690
% 18     USA             45  Carbon Savings  76.973776            2.427873
% 19     USA             50  Carbon Savings  78.263891            2.112501



\subsubsection{Impact of Latency Tolerance}\label{sec:eval_latency}\hfill\\
Carbon savings are a function of placement flexibility, where applications with no latency requirements can be placed in locations with zero carbon intensity~\cite{sukprasert2024limitations}. However, in practice, edge applications have tight latency requirements. ~\autoref{fig:latency_effect} depicts the carbon savings and latency overheads across different round-trip latency limits in the US and Europe. As shown in ~\autoref{fig:latency_effect_carbon}, allowing a round-trip latency tolerance of 10 ms can yield 28\% and 44.8\% carbon savings in the US and Europe, respectively, while raising the latency limit to 20 ms can increase these carbon savings by 23\% and 23.4\%. These increasing carbon savings come from placing more workload in greener regions that meet the latency limits. 
Moreover, the figure emphasizes that increasing latency limits leads to diminishing returns. For example, in Europe, increasing the latency limit from 5 ms to 10 ms increases savings by 43.8\%, whereas increasing the limit from 25 ms to 30 ms only yields an extra 4\% savings. ~\autoref{fig:latency_effect_overhead} shows that performance overheads increase linearly with rising latency limits. Importantly, the results indicate that the benefits consistently outweigh the overheads. For instance, the figure demonstrates that the \proposedsystem reduces carbon emissions by 74.7\% while incurring only a 17.2 ms increase in round-trip latency. 

% Lastly, ~\autoref{fig:latency_effect_placement} illustrates how increased latency limits influence placement decisions. For instance, at a latency limit of 10 ms, 43\% of applications operate in regions with carbon intensity $\leq300$~\emissionunit, yielding 13.5\% more savings than \latencyaware. Furthermore, at a latency limit of 30 ms, 27\% more applications run in these greener regions.


% 10 300 41.786616161616166
% 20 300 46.91287878787879
% 30 300 60.83964646464647
% latency-aware 300 29.545454545454547



% Approach  Region  Latency Limit  Increased Latency  Carbon Savings  \
% 0         Europe              5           0.126727        0.837259   
% 1         Europe             10           3.652924       40.203283   
% 2         Europe             15           6.153970       55.373929   
% 3         Europe             20           9.312182       61.766799   
% 4         Europe             25          13.729833       68.558757   
% 5         Europe             30          16.808182       73.340174   
% 6            USA              5           0.567742       11.587107   
% 7            USA             10           2.766985       26.147058   
% 8            USA             15           7.187788       41.079595   
% 9            USA             20          11.018409       48.361293   
% 10           USA             25          13.746621       56.105412   
% 11           USA             30          17.289455       61.101597   

% Approach  carbon_latency  
% 0               6.606775  
% 1              11.005781  
% 2               8.998083  
% 3               6.632903  
% 4               4.993415  
% 5               4.363362  
% 6              20.409092  
% 7               9.449657  
% 8               5.715193  
% 9               4.389136  
% 10              4.081397  
% 11              3.534038  



\noindent \textit{\textbf{Key Takeaways.} 
For a 10 ms increase in latency, \proposedsystem derives 28\% and 44.8\% carbon savings in the US and Europe, respectively. Notably, the results demonstrate that the benefits constantly outweigh the overheads, where \proposedsystem reduces carbon emissions by 74.7\% while incurring only a 17.2 ms increase in round-trip latency.
%Our analysis shows that there exists a latency limit that constitutes a carbon-latency balance point, where \proposedsystem can save up to 11\% for each ms increase in latency.
}

\begin{figure}[tb]
    \centering
    \subfloat[\centering Carbon Savings ]{\includegraphics[width=0.4\linewidth]{figures/5_SLOs_carbon_saving.pdf}%
    \label{fig:latency_effect_carbon}%
    }%
    \quad
    \subfloat[\centering Latency Increases ]{
    \includegraphics[width=0.4\linewidth]{figures/5_SLOs_increased_latency.pdf}
    \label{fig:latency_effect_overhead}
    }%
    % \hfill
    % % \subfloat[\centering Server Activation ]{
    % % \includegraphics[width=0.25\linewidth]{figures/5_SLOs_servers_usa.pdf}
    % % \label{fig:5_slo_severs_usa}
    % % }%
    % \subfloat[\centering Placements (US) ]{
    % \includegraphics[width=0.3\linewidth]{figures/5_SLOs_placements.pdf}
    % \label{fig:latency_effect_placement}
    % }%
    % \quad
    \caption{Effect of latency tolerance on carbon savings and latency increases.} 
    \label{fig:latency_effect}
\end{figure}


\begin{figure*}[t]
    \raggedleft
    \quad\includegraphics[width=0.44\linewidth]{figures/4_ci_legend.pdf}\quad \quad \quad \\
    \centering
    \subfloat[\centering Carbon Savings]{\includegraphics[width=0.22\linewidth]{figures/4_carbon_saving_year_long_lineplot.pdf}%
    \label{fig:global_monthly_carbon}%
    }%
    \quad
    \subfloat[\centering Latency Increases]{
    \includegraphics[width=0.22\linewidth]{figures/4_increased_latency_year_long_lineplot.pdf}
    \label{fig:global_monthly_latency}
    }%
    \quad
    \subfloat[\centering Carbon Intensity]{
    \includegraphics[width=0.22\linewidth]{figures/4_ci.pdf}
    \label{fig:monthly_changes_intensity}
    }%
    \quad
    \subfloat[\centering Placements]{
    \includegraphics[width=0.22\linewidth]{figures/4_workloads.pdf}
    \label{fig:monthly_changes_placement}
    }%
    \caption{Effect of seasonality on carbon savings and latency overhead across the US and Europe.} 
    \label{fig:global_monthly}
    \vspace{-.3cm}
\end{figure*}



% \begin{figure}[t]
%     \centering
%     \subfloat[\centering Carbon Savings (\%)]{\includegraphics[width=0.45\linewidth]{figures/4_carbon_saving_year_long_boxplot.pdf}%
%     \label{fig:global_monthly_carbon}%
%     }%
%     \quad
%     \quad
%     \subfloat[\centering Monthly Latency Increase (ms)]{
%     \includegraphics[width=0.45\linewidth]{figures/4_increased_latency_year_long_boxplot.pdf}
%     \label{fig:global_monthly_latency}
%     }%
%     \caption{Effect of seasonality on carbon savings and latency overhead across the US and Europe.}
%     \todo{Make line plot - 5 days is fine.} \lilly{Updated! Check the below figure}
%     \label{fig:global_monthly}
% \end{figure}



% \begin{figure}[t]
%     \centering
%     \includegraphics[width=0.65\linewidth]{figures/4_workloads_labels.pdf}\\
%     \hfill
%     \subfloat[\centering Carbon intensity of four zones ]{
%     \includegraphics[width=0.45\linewidth]{figures/4_ci.pdf}
%     \label{fig:monthly_changes_intensity}
%     }%
%     \hfill
%     \subfloat[\centering Placements of four zones  ]{
%     \includegraphics[width=0.45\linewidth]{figures/4_workloads.pdf}
%     \label{fig:monthly_changes_placement}
%     }%
%     \hfill \hfill
%     \caption{Illustrating how changes in carbon intensity affect placement decisions in Europe. } \lilly{Online version changes the placement. Need to redo this plot}
%     \label{fig:monthly_changes}
% \end{figure}
\subsubsection{Impact of Seasonality}\label{sec:eval_seasonality}\hfill\\
To better understand the effect of seasonality on spatial decisions, ~\autoref{fig:global_monthly} illustrates the fluctuations in carbon savings and latency increases over 12 months in the US and Europe. ~\autoref{fig:global_monthly_carbon} shows that carbon savings in the US exhibit minimal changes, with a maximum difference of 3.3\% in carbon savings (i.e., between July and April). In contrast, carbon savings vary significantly in Europe, resulting in a 9.9\% difference between February and June. Furthermore, ~\autoref{fig:global_monthly_latency} indicates that latency overheads see only slight changes, with only 1.2 ms differences in both the US and Europe.


~\autoref{fig:monthly_changes_intensity} and \autoref{fig:monthly_changes_placement} further detail how seasonal variations in carbon intensity impact placement decisions in \proposedsystem, demonstrating the need for migrating long-running applications across regions. As shown in ~\autoref{fig:monthly_changes_intensity}, the changes in carbon intensity vary significantly between locations. For example, Zagreb, HR exhibits a 102 \carbonunit difference between April and May, whereas Oslo, NO only sees a 2.4 \carbonunit change. Reflecting these changes in carbon intensity, the number of applications assigned to these data centers fluctuates greatly. As indicated in \autoref{fig:monthly_changes_placement}, the number of applications assigned to Paris varies by 1.3$\times$ between July and August, while in Oslo, it changes by 3$\times$ between December and November. Additionally, the figure indicates that variations in demand  might be reflected in the carbon intensity of nearby areas rather than in the region itself. For example, Oslo, which exhibits the most significant fluctuation in application assignments, demonstrates little change in carbon intensity. Conversely, Vienna, with the largest shifts in carbon intensity, sees only minor changes in the volume of assigned applications.
 
 % , such as Oslo and Vienna. 
 
 % 






\noindent \textit{\textbf{Key Takeaways.} 
The seasons' changes in carbon intensity highly affect the carbon savings that change by up to 10\% across months. The intertwined relations between regions change across seasons resulting in up to 3$\times$ change in resource allocation.
}



% \begin{figure}[tb]
%     % \raggedleft
%     % \quad\includegraphics[width=0.3\linewidth]{figures/8_distribution_legend.pdf}\quad \quad \quad \\
%     \centering
%     \subfloat[\centering Carbon Savings]{\includegraphics[width=0.32\linewidth]{figures/8_workload_capacity.pdf}
%     \label{fig:8_carbon}%
%     }%
%     % \quad
%     \subfloat[\centering Latency Increases]{
%     \includegraphics[width=0.32\linewidth]{figures/8_workload_increased_latency.pdf}
%     \label{fig:8_latency}
%     }%
%     % \quad
%     \subfloat[\centering Distributions (US)]{
%     \includegraphics[width=0.32\linewidth]{figures/8_distribution_app.pdf}
%     \label{fig:8_distribution}
%     }%
%     \caption{Effect of workload and capacity distribution on carbon savings and latency increases in the US and Europe.}
%     % \lilly{Online version now}
%     \label{fig:8_workload_capacity}
% \end{figure}

\begin{figure}[tb]
    \centering
    \includegraphics[width=0.5\linewidth]{figures/8_workload_capacity_legend.pdf} \\
    \subfloat[\centering Carbon Savings]{\includegraphics[width=0.4\linewidth]{figures/8_workload_capacity.pdf}%
    \label{fig:workload_capacity}%
    }%
    \quad
    \subfloat[\centering Increased Latency]{
    \includegraphics[width=0.4\linewidth]{figures/8_workload_increased_latency.pdf}
    \label{fig:workload_increased_latency}
    }%
    % \subfloat[\centering Carbon Savings (\%)]{\includegraphics[width=0.2\linewidth]{figures/1_resnet_carbon_saving_real_capacity.pdf}%
    % \label{fig:resnet_carbon_saving_real_capacity}%
    % }%
    % % \hfill
    % \subfloat[\centering Increased Latency (ms)]{
    % \includegraphics[width=0.2\linewidth]{figures/1_resnet_increased_latency_real_capacity.pdf}
    % \label{fig:resnet_increased_latency_real_capacity}
    % }%
    \caption{Effect of demand and capacity. }
    \label{fig:demand_capacity}
\end{figure}

\subsubsection{Impact of Demand and Capacity}\label{sec:eval_demand_capacity}\hfill\\
Carbon savings from edge placements are affected by regional demand and resource capacity. This section evaluates the impact of demand and capacity using population data as a proxy for such differences. Our intuition behind this is that locations with high populations typically have high demand. Similarly, edge providers tend to increase their capacities near them.
%, where locations with higher populations tend to have more demand and capacity. 
%To understand the impact, we independently vary demand or capacity (while fixing the other) across edge data centers. 
~\autoref{fig:demand_capacity} demonstrates the impact of demand and capacity variations on carbon savings. The demand represents the case where the workload across data centers is proportional to the population across cities while keeping the capacity fixed, while the capacity scenario changes the capacity distribution according to the population density while keeping the demand fixed. Lastly, for reference, we add the homogeneous scenario (labeled as Homo), where the demand and capacity are constant across data centers.  
As shown, in the US, changes in demand and capacity can limit the flexibility to do spatial shifting and decrease carbon savings. For instance, changes in capacity per the population reduce carbon emissions by 6\%. This is because high carbon intensity locations (e.g., FL) have no nearby green regions to shit workloads. In contrast, in Europe, the population is more evenly distributed within the data centers we utilize, where carbon savings changes are <1.6\% and latency changes by <0.6 ms. 


\noindent \textit{\textbf{Key Takeaways.} 
Changes in demand and capacity can impact carbon savings based on the carbon intensity of their origin. %\proposedsystem achieves carbon savings comparable to homogeneous setups in Europe ($\Delta \leq 1.6\%$) while reducing US emissions by 1.75 \emissiontunit by prioritizing high-capacity, low carbon-intensity edge locations (e.g., New York, Los Angeles). 
}

% using four distributions: Akamai, Population, Even, and Popularity. The Akamai distribution is based on Akamai traces; the population distribution derives from city population statistics. The even distribution, used in other experiments, deploys the same demand and capacity at all data centers. Lastly, we design a carbon-aware demand and capacity distribution using a Zipf popularity distribution, which deploys more demand and capacity to green regions. 

% ~\autoref{fig:8_carbon} and ~\autoref{fig:8_latency} present the carbon savings and latency increases of \proposedsystem with the four demand and capacity distributions in the US and Europe, and ~\autoref{fig:8_distribution} exemplifies the histogram of the distributions using the US sites. With the Akamai distribution, \proposedsystem saves less carbon emissions, particularly in the US, compared to the other distribution. This is because Akamai traces have a big demand and resource capacity bias in the US. In our experiments, 8 out of 44 sites have demands and resource capacity, and their carbon intensity variations are small, as shown in ~\autoref{fig:8_distribution}. With the population-based distribution, \proposedsystem performs similarly to the even distribution. Importantly, the results highlight that \proposedsystem achieves higher relative carbon savings with smaller latency increases when the distribution is carbon-aware. 




% using four distributions: Akamai, Population, Even, and Popularity. The Akamai distribution is based on Akamai traces; the population distribution derives from city population statistics. The even distribution, used in other experiments, deploys the same demand and capacity at all data centers. Lastly, we design a carbon-aware demand and capacity distribution using a Zipf popularity distribution, which deploys more demand and capacity to green regions. 

% ~\autoref{fig:8_carbon} and ~\autoref{fig:8_latency} present the carbon savings and latency increases of \proposedsystem with the four demand and capacity distributions in the US and Europe, and ~\autoref{fig:8_distribution} exemplifies the histogram of the distributions using the US sites. With the Akamai distribution, \proposedsystem saves less carbon emissions, particularly in the US, compared to the other distribution. This is because Akamai traces have a big demand and resource capacity bias in the US. In our experiments, 8 out of 44 sites have demands and resource capacity, and their carbon intensity variations are small, as shown in ~\autoref{fig:8_distribution}. With the population-based distribution, \proposedsystem performs similarly to the even distribution. Importantly, the results highlight that \proposedsystem achieves higher relative carbon savings with smaller latency increases when the distribution is carbon-aware. 

% \noindent \textit{\textbf{Key Takeaways.} The demand and capacity of edge data centers significantly impact carbon savings, with edge locations across which there are small variations in carbon intensity showing less savings. However, carbon-aware resource provisioning can further decrease carbon emissions and latency overheads.}




% \subsection{Heterogeneous Edge Settings }\label{sec:hetero}
% In edge computing, heterogeneity is an inherent property that arises in both applications and systems~\cite{HeteroEdge}. In this section, we evaluate the performance of \proposedsystem with diverse edge applications and heterogeneous edge servers. In addition, we show the potential carbon-energy trade-off with \proposedsystem. Similar to the earlier section, we simulate the behavior of \proposedsystem in CDN-scale settings while considering a latency limit of 20 ms ($\sim$500 km).

% In this case, a key question is: What is the impact of application and resource heterogeneity in carbon-aware placement? To answer this question, we evaluate the performance of  \proposedsystem when deploying diverse applications on different hardware. Similar to the earlier section, we simulate the behavior of \proposedsystem across a wide range of workloads and resources in CDN-scale settings while considering a latency limit of 20 ms ($\sim$500 km).

% \begin{figure}[tb]
%     \centering
%     \subfloat[\centering Carbon Emissions (t) ]{\includegraphics[width=0.45\linewidth]{figures/1_carbon_emission_europe.pdf}
%     \label{fig:application_het_carbon}%
%     }%
%     % \quad
%     \subfloat[\centering Energy Consumption  ]{
%     \includegraphics[width=0.45\linewidth]{figures/1_energy_consumption_europe.pdf}
%     \label{fig:application_het_energy}
%     }%
%     % \quad
%     % \subfloat[\centering Increased Latency (ms) ]{
%     % \includegraphics[width=0.22\linewidth]{figures/1_increased_latency_models.pdf}
%     % \label{fig:1_increased_latency_models}
%     % }%
%     \caption{Carbon emissions and energy consumption when placing diverse applications on Nvidia A2 homogeneous edge data centers.}
%     % \lilly{Online version now}
%     \label{fig:application_het}
% \end{figure}

% \subsubsection{Impact of Heterogeneous Application Requirements}\hfill\\
% In this section, we evaluate the performance of \proposedsystem for different workloads with different characters (see ~\autoref{fig:models_profile}). To make results comparable, we fixed the number of applications despite their resource requirement differences. ~\autoref{fig:application_het} depicts the total carbon emissions and energy across
% three machine learning applications: EfficientNetB0, ResNet50, YOLOv4, and a mixture of the three when considering homogeneous resources (Nvidia A2). ~\autoref{fig:application_het_carbon} shows the carbon emissions for four baselines: \latencyaware, \energyaware, \intensityaware, and \proposedsystem. As shown, by reducing the total energy consumption, the \energyaware policy reduces the carbon emissions by  2.24\emissiontunit (43.3\%) and 2\emissiontunit (8.3\%) for the EfficientNetB0 and YOlOv4, respectively. 
% In contrast, the figure shows that utilizing variations in carbon intensity results in much higher carbon savings. For instance, the \intensityaware can provide up to 69\% and 45.6\% carbon savings with respect to the \latencyaware and \energyaware policies. Finally, the figure shows that the \proposedsystem outperforms all other baselines across all applications, where it can reduce carbon emissions by 71\% and 49\% compared to the \latencyaware and \energyaware policies. %\lilly{the result is very similar to the Intensity-aware policy. 0.06\% compared to the \latencyaware and the closest runner-up (\intensityaware) policies. }

% ~\autoref{fig:application_het_energy} highlights total energy consumption across different applications. As expected, the EfficientNetB0, the application with the lowest energy consumption (see ~\autoref{fig:models_profile_energy}), has the lowest energy consumption, while the YOLOv4 has the highest. In addition, the figure shows that since 
% all policies consolidate the number of used servers, all policies reduce the total energy consumption.  For instance, the \intensityaware policy reduces energy by 44\%, while the \energyaware reduces the energy by 54\%. The figure also highlights the carbon-energy trade-off, where carbon-aware placement increases the total energy consumption. For instance, the \intensityaware and \proposedsystem policies increase the total energy consumption by up to 22.8\% and 7\% compared to the \energyaware baselines. %\lilly{It is not very high here due to homogeneous settings} \walid{22.8 and 7 is wrong}.
% Finally, we note the round-trip latency overheads were typically comparable across energy- and carbon-aware baselines, in a range of 9 $\sim$ 11 ms. 

% \noindent \textit{\textbf{Key Takeaways.} 
% Despite the benefits of energy-efficient placement, carbon-aware placement is much more effective in reducing carbon emissions. Our results highlight that exploiting the variability in carbon intensity across regions results in up to 49\% more reductions than energy-efficient placement.}






\subsubsection{Impact of Heterogeneity}\label{sec:eval_hetero}\hfill\\
Heterogeneity is an inherent property of edge computing that appears in applications and systems~\cite{HeteroEdge}. In this section, we evaluate the performance of \proposedsystem with diverse edge applications and heterogeneous edge servers and compare it to three baselines: \latencyaware, \energyaware, and \intensityaware.

~\autoref{fig:server_heteroginity} illustrates the carbon emissions and energy consumption of a mix of applications, including EfficientNetB0, ResNet50, and YOLOv4, across three different resources (Orion Nano, A2, and GTX 1080) and a mix of them (labeled as Hetero.). ~\autoref{fig:server_heteroginity_carbon} shows \energyaware, \intensityaware, and \proposedsystem can reduce carbon emissions compared to \latencyaware. The figure highlights the energy efficiency of different hardware, showing that serving the same load using Orion Nano uses 95.6\% less energy than GTX 1080. However, \proposedsystem achieves 53\% and 62\% carbon reductions for Orion Nano and GTX 1080, respectively.  This is because, although GTX 1080 has higher energy consumption, its low inference time (see~\autoref{fig:models_profile_latency})  can enlarge the potential of spatial shifting, allowing requests to be offloaded to low-carbon locations.  Importantly, when considering heterogeneous resources, \proposedsystem can further reduce carbon emissions by interplaying the differences in energy efficiency, carbon intensity, and processing speed, decreasing carbon emissions by 98.4\%, 79\%, and 63\% reductions compared to the \latencyaware, \intensityaware, and \energyaware baselines, respectively. ~\autoref{fig:server_heteroginity_energy} 
highlights the carbon-energy trade-off, where carbon-aware placement increases the total energy consumption.  Compared to the energy-aware placement, \intensityaware and \proposedsystem can use 12$\times$ and 5.5$\times$ more energy. 


\noindent \textit{\textbf{Key Takeaways.} 
By interplaying the differences in energy efficiency, carbon intensity, and processing speed,  \proposedsystem can reduce carbon emissions by 98\%, 79\%, and 63\% compared to the \latencyaware, \intensityaware, and \energyaware baselines, respectively. 
}

\begin{figure}[tb]
    \centering
    \includegraphics[width=1.0\linewidth]{figures/7_energy_consumption_europe_legend.pdf} \\ 
    \subfloat[\centering Carbon Emissions (log t)]{\includegraphics[width=0.485\linewidth]{figures/7_carbon_emission_europe.pdf}%
    \label{fig:server_heteroginity_carbon}%
    }%
    \hfill
    \subfloat[\centering Energy Consumption]{
    \includegraphics[width=0.485\linewidth]{figures/7_energy_consumption_europe.pdf}
    \label{fig:server_heteroginity_energy}
    }%
    \caption{Carbon emissions and energy consumption across workloads on heterogeneous resources. }
    \label{fig:server_heteroginity}
\end{figure}


\subsection{Navigating the Carbon-Energy Trade-off}\label{sec:eval_carbon_energy}%\hfill\\
Despite the importance of carbon emissions from edge computing, it is crucial not to ignore the trade-off between lowering carbon and energy consumption, as energy typically incurs a monetary cost. 
To understand the breadth of the carbon-energy trade-off, we augment the optimization objective in \autoref{eq:objective} as follows:
{\small 
\begin{align}
    \label{eq:carbon_energy_tradeoff}
    & \min_{\substack{x_{ij}^*, y_j^*}} \quad \alpha \cdot p + (1-\alpha) \cdot f
\end{align}
}
% {\small 
% \begin{equation}
%     \text{min} \quad \alpha \cdot p + (1-\alpha) \cdot f
% \end{equation}
% }

where $p$ is the total energy consumption, $f$ is the total carbon footprint and $\alpha$ is a weighting factor, where $\alpha=0$ results in the vanilla \proposedsystem policy, while $\alpha=1$ is the \energyaware policy.
Note that to navigate the trade-off seamlessly, we normalize the carbon intensity and energy consumption between $ [0, 1]$ using min-max normalization.


~\autoref{fig:carbon_energy} depicts how changing $\alpha$ can affect carbon emissions and energy consumption within two scenarios: low and high resource utilization. As shown, as the utilization increases, the magnitude of carbon emissions and energy highly increases, where carbon emissions and energy consumption increase by 15.5$\times$ and 20$\times$ from the low to the high utilization scenarios. 
Moreover, in both cases, \proposedsystem significantly reduces carbon emissions, where it reduces carbon emissions by 98.4\% and 90.5\% for the low and high utilization compared to the \latencyaware, respectively. Nonetheless, in both cases, carbon-efficient placement ($\alpha=0$) increases energy consumption compared to energy-efficient placement ($\alpha=1$) by 9.1$\times$ and 1.84$\times$ for the low and high utilization scenarios, respectively.
Lastly, the figure highlights that, in both cases, a balance point exists where carbon reductions come at a lower energy cost. For instance, in the low utilization scenario (\autoref{fig:carbon_energy_low}), using $\alpha=0.1$ allows \proposedsystem to retain 97.5\% of its carbon savings while decreasing energy consumption by 67\%. In contrast, in the high utilization scenario (~\autoref{fig:carbon_energy_high}), the trade-off is more prominent, where using $\alpha=0.5$ allows \proposedsystem to retain 83.7\% of its carbon savings while increasing energy by 15\%.

\begin{figure}[tb]
    \centering
    \subfloat[\centering Low Utilization ]{\includegraphics[width=0.48\linewidth]{figures/7_carbon_energy_6_absolute.pdf}%
    \label{fig:carbon_energy_low}%
    }%
    % \quad
    % \subfloat[\centering Median Utilization ]{
    % \includegraphics[width=0.3\linewidth]{figures/7_carbon_energy_12.pdf}
    % \label{fig:carbon_energy_median}
    % }%
    \hfill
    \subfloat[\centering High Utilization ]{
    \includegraphics[width=0.48\linewidth]{figures/7_carbon_energy_18_absolute.pdf}
    \label{fig:carbon_energy_high}
    }%
    \caption{\proposedsystem with carbon-energy trade-offs.} 
    \label{fig:carbon_energy}
    \vspace{-.3cm}
\end{figure}



\noindent \textit{\textbf{Key Takeaways.} 
The inherent carbon-energy trade-off is  pronounced in heterogeneous edge settings. By augmenting the objective function with energy-awareness, \proposedsystem can maintain 97.5\% of its carbon savings while decreasing energy consumption by 67\%.
}

\subsection{System Overhead}\label{sec:eval_overhead}
We evaluate the runtime performance of \proposedsystem in the mesoscale regional edge deployments. When a workload arrives, \proposedsystem requires approximately 3.3 $ms$ to determine the placement and  1.01 $s$ to initiate the application deployment. We further analyze the overhead of our incremental placement algorithm, which directly affects system performance. ~\autoref{fig:9_scalability} shows how the runtime and memory usage of our algorithm scale with two key parameters: the number of servers and the number of applications. By isolating these parameters (varying one while fixing the other), we demonstrate that our incremental placement algorithm scales efficiently to 400 servers and 140 applications, completing computations within 3 seconds while consuming less than 200 MB of memory. %Notably, the number of edge data center locations does not affect the performance of our algorithm since we reduce this dimension in our problem formulation.

% \begin{table}[t]
%   \centering
%     \caption{System Overhead} 
%   \resizebox{1.0 \linewidth}{!}{
%   \begin{tabular}{|c|c|c|c|c|}
%     \hline
%     Process & Get configurations &  Decision making &  Deployment &  End-to-end \\\hline
%     Time (ms) & 3.33 & 0.02 & 1012.3 & 1015.83\\ \hline
%   \end{tabular}
%   \label{tab:overhead} 
%   }
% \end{table}



\begin{figure}[t]
    \subfloat[\centering Number of Servers]{
    \includegraphics[width=0.24\linewidth]{figures/9_server_time.pdf}
    \label{fig:server_time}
    \includegraphics[width=0.24\linewidth]{figures/9_server_mem.pdf}
    \label{fig:server_mem}
    }%
    \subfloat[\centering Number of Applications]{\includegraphics[width=0.24\linewidth]{figures/9_app_time.pdf}%
    \label{fig:app_time}%
    \includegraphics[width=0.24\linewidth]{figures/9_app_mem.pdf}
    \label{fig:app_mem}
    }%
    % \quad
    % \subfloat[\centering Memory (App)]{
    % \includegraphics[width=0.24\linewidth]{figures/9_app_mem.pdf}
    % \label{fig:app_mem}
    % }%
    % \quad
    % \subfloat[\centering Memory (Server)]{
    % \includegraphics[width=0.24\linewidth]{figures/9_server_mem.pdf}
    % \label{fig:server_mem}
    % }%
    \caption{Scalability of \proposedsystem to input parameters.} 
    \label{fig:9_scalability}
\end{figure}









