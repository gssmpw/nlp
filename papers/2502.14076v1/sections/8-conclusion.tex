In this paper, we analyzed fine-grained carbon intensity traces at intermediate ``mesoscales,'' such as within a single U.S. state or neighboring countries in Europe, and showed that intelligently distributing workload at these mesoscales can reduce carbon without violating latency SLOs.
To build upon this observation, we presented \proposedsystem, a carbon-aware placement framework for edge applications, which optimizes workload placement and power management decisions across edge data centers within a mesoscale region to minimize carbon emissions while satisfying latency SLOs. 
Our evaluation highlights that our \proposedsystem can exploit the mesoscale carbon intensity variations and present carbon savings that outweigh the latency overhead. 
%\proposedsystem on a real testbed shows that \proposedsystem can reduce emissions by up to 78.7\% when considering edge centers within the Central EU region. In addition, our CDN-scale experiments across Europe show that \proposedsystem reduces carbon emissions by 73\% while only incurring a 16.8ms increase in latency.
In future work, we will consider the problem of carbon-aware resource provisioning and the role of local renewables in decarbonizing edge computing systems.


%The proliferation of latency-critical and compute-intensive edge applications is driving an increase in computing demand and carbon emissions at the edge.  To better understand edge carbon emissions, we analyze fine-grained carbon intensity traces at intermediate ``mesoscales,'' such as within a single U.S. state or neighboring countries in Europe, and observe that intelligently distributing workload at these mesoscales has the potential to reduce carbon without violating latency SLOs.  Importantly, our analysis shows that geographic workload distribution for reducing carbon emissions is applicable to edge computing, which cannot leverage carbon variations across large continental scales observed in prior work due to their strict latency requirements. To address the problem, we design \proposedsystem, a carbon-aware control plane for edge computing, which optimizes workload placement across edge sites within a mesoscale region to minimize carbon emissions while satisfying latency SLOs. We implement \proposedsystem and evaluate it on a real edge computing testbed and using large-scale simulations for multiple edge workloads.  Our results demonstrate that \proposedsystem yields 1) 98.4\% carbon savings over a \textit{carbon-agnostic} placement policy  2) 78\% carbon savings over an \textit{energy-aware} placement policy, and 3) 75\% carbon savings over a state-of-the-art \textit{carbon-aware} policy.
