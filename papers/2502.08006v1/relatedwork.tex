\section{Related Works}
Training-free methods for diffusion/flow models is an active area of research with many different techniques proposed with differing strengths.
Diffusion Posterior Sampling (DPS)~\citep{chung2023diffusion} is a guidance method that uses Tweedie's formula~\citep{stein1981estimation} to estimate the gradient of some guidance function defined in the output state \wrt the noisy state, \ie, $\ex[\bfx_1|\bfx_t]$. 
Likewise the work of~\citet{bansal2023universal,wang2023zeroshot,yu2023freedom} explore similar concepts by employing Tweedie's formula for diffusion models, with FreeDoM including an additional ``time-travel strategy''~\citep{wang2023zeroshot} to help improve the visual fidelity of the generated images.
More relevant to our work,~\citet{blasingame2024greedydim} explore guidance using posterior sampling from a greedy perspective; however, their analysis is limited as their work is only focused on a particular experimental application.
Specifically, they only consider a first-order numerical scheme for the diffusion ODE solver and they do not connect the greedy strategy to other guided generation methods.
\citet{liu2023flowgrad} provide guidance by adding an additional time-dependent control term to the vector field that is optimized using a type of discretize-then-optimize
method by finding an efficient non-uniform Euler discretization of the sampling trajectory and perform backpropagation through this discretization to update the control signal.


Several recent works have explored the use of continuous adjoint equations for diffusion/flow models.
\citet{nie2022diffpure} was the first to explore this topic, solving the continuous adjoint equations for adversarial purification with diffusion SDEs.
Later work by~\cite{pan2024adjointdpm,pan2023adjointsymplectic} explore special solvers for the continuous adjoint equations of VP-type diffusion ODEs.
\citet{blasingame2024adjointdeis} extends these works by developing bespoke solvers for VP-type diffusion ODEs and SDEs.
\citet{marion2024implicit} explore using the continuous adjoint equations as a part of a larger bi-level optimization scheme for guided generation.
The work of \citet{ben-hamu2024dflow} extends the analysis of continuous adjoint equations for diffusion models to flow-based models and provides an alternative perspective to the analyis performed by the earlier works.
\citet{wallace2023end} use EDICT~\citep{wallace2023edict}, an invertible formulation of diffusion models, to perform backpropagation through the diffusion model; this can be viewed as a specific discretization scheme of continuous adjoint equations.
Recent work by~\citet{wang2024training} explores an extension of~\citet{ben-hamu2024dflow} to Riemannian manifolds which incorporates a control signal to the vector field and optimizes both the solution state and \textit{co-state}, they call their approach OC-Flow.