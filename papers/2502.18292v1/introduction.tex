\section{Introduction} \label{sec: introduction}
\begin{figure}[t]
\centering
\includegraphics[width=1.0\linewidth]{fig/example.pdf}
\caption{Example of the special text relevance in LCR. Left: the query and candidate legal cases in LCR. Right: two applicable law articles of the legal cases.}
\label{fig: LCR_example}
\end{figure}
Legal case retrieval (LCR) aims to search for similar precedents for a query case, which is highly valuable in diverse judicial systems.
Because the principle of comparable judgments for similar cases
\footnote{This principle means that if different cases have the same illegal conduct, they should be sentenced to the same applicable laws, crimes, and forms of punishment (e.g., fines, detention, imprisonment, death penalty, etc). Furthermore, if the degree of illegality involved in the cases is similar, then the degree of punishment (e.g., the amount of fines, the duration of detention and imprisonment, etc.) they were sentenced should be roughly the same.} 
upholds the broad applicability and fairness of the judicial system, comparable precedents not only influence the incentives of decision-makers in the common law system but also provide the basis for legal reasoning in the civil law system.
In particular, in recent years, some countries with a civil law system (e.g., China) have promulgated official documents\footnote{https://www.court.gov.cn/fabu/xiangqing/334151.html} that institutionally incorporate the process of LCR.
This significant development has directed scholarly attention towards the challenges of LCR within the civil law system~\cite{ma2021LeCaRD, yu2022Explainable, sun2023law}.
This paper primarily investigates the LCR problem under the civil law system.

% {\color{blue}[Modify and focus on the civil law system] Under the Anglo-American legal system (e.g., Britain, America, and Canada), comparable precedents are the primary foundation of judicial trials, directly involved in the trial process.
% In the civil law system (e.g., China, France, and Germany), although prior cases cannot serve as direct arguments for judicial trial, the concept of comparable decisions for similar instances still upholds the fairness of the legal systems.
% In recent years, as intelligent judicature has grown, more and more academics have focused on information retrieval (IR) in the legal field~\cite{oard2013TREC,bhattacharya2019AILA,rabelo2019COLIEE,ma2021LeCaRD,yu2022Explainable}.}

In earlier studies, researchers viewed LCR as an application of similar text retrieval models that seeks to find a precedent with a similar semantic context given a query case~\cite{xiao2021Lawformer,bhattacharya2020LCR_survey,saravanan2009LCR_1,zeng2005LCR_2,zhaochun2024answer}.
However, researchers discovered challenges that set LCR apart from general text retrievals in further research~\cite {shao2020Bert_PLI,hu2022Bert_LF}.
These challenges can be summarized as follows: (1) Excessively long legal case documents; (2) Special definition of text relevance in LCR, extending beyond the semantic similarity of conventional text retrieval; (3) Insufficient professional-labeling data for model training.
Due to the continuous release of open-source data~\cite{ma2021LeCaRD,yu2022Explainable,xiao2018CAIL2018} during the digitization of legal information and the extensive research on long-text models~\cite{Yang@Transformer_XL, NEURIPS@XL_Net, arxiv@Longformer} by NLP experts, challenges (1) and (3) have been partially overcome. 
However, challenge (2) persists, perplexing researchers, and impeding the effectiveness of current semantic-based models.
To illustrate the impact of challenge (2) on semantic-based models, we present an example related to Chinese criminal cases in Fig.~\ref{fig: LCR_example}, where we showcase two query candidates and indicate the relevant law articles\footnote{Law articles, which are usually enacted by the administration of justice (e.g., Criminal Law in China,  Act of Congress in America, Penal Code in India, and so on), are the foundation of the civil law systems. They are also termed ‘statutes’ in some countries.} on the right side. % \markred{[explain the source of the legal material used]}
\begin{example}{
\emph{Candidate A} and \emph{Query} have a lot of similar text in the fact description (black font), but they apply different laws (PRC Criminal Law\footnote{https://flk.npc.gov.cn/detail2.html?ZmY4MDgxODE3OTZhNjM2YTAxNzk4MjJhMTk2NDBjOTI}, \emph{Article 128} and \emph{Article 348}) because they have different legal characteristics (green and blue font). This results in the expert-annotated labels of Query and Candidate A being "mismatch" although the existing language models calculate that they have a relatively high semantic similarity.
As for \emph{Candidate B}, it has the same applicable law articles (\emph{Article 348}) and similar legal characteristics (green font) as Query, so its label is still a "match" even though it has a lot of different contexts than Query.
}
\end{example}

To solve the challenge (2), the Chinese law system’s official guideline document\footnote{\url{https://www.court.gov.cn/fabu/xiangqing/243981.html}} highlights that a relevant case must be assessed based on three aspects, including the applicable law, the central point of the dispute, and the essential facts.
This suggests that the pertinent cases in LCR exhibit not only semantic similarities but also some level of legal congruity.
Thus, several researchers have implemented some techniques of capturing jurisprudential relevance between legal cases, which involve incorporating manual labels or elements such as legal events labels~\cite{yao2022LEVEN}, legal-rational labels~\cite{yu2022Explainable}, legal entities~\cite{shao2020Bert_PLI}, and other related factors.
Nevertheless, manually labeling these attributes can be an arduous and time-consuming process that heavily depends on experts. 
Furthermore, these labeling-based approaches tend to be highly specific to a particular legal system, consequently restricting the broad applicability of existing methodologies across diverse judicial systems.
Thus, in practical application, we desire a general and effective method to capture the jurisprudential relevance between cases.
The most relevant existing work that meets this requirement is the approach taken by Sun et al~\cite{sun2023law}, who draw inspiration from guideline documents to utilize applicable law articles to reconstruct case representations and thereby improve the overall performance of the legal case matching (LCM) task.
Their method exhibits strong generalization capabilities across criminal and civil cases, with easily accessible law articles.
Yet, despite its efficacy, the practical implementation of this approach can be challenging due to its unrealistic assumption that all query cases have access to applicable law article references. 
Actually, in the judicial practice of the civil law system, judges often use similar historical precedents to support their judgment conclusions in cases.
This means that, from the perspective of real-world judicial procedures, similar case retrieval occurs before determining the applicable law articles for cases.
Thus, using applicable law articles as input to evaluate similar cases is difficult to apply to practical scenarios.
% Indeed, the ability to identify applicable law articles for query cases is often the ultimate objective of the LCR task, rather than simply considering them as inputs.

To address the challenge of capturing jurisprudential relevance in LCR and LCM tasks under a more realistic situation assumption, in this paper, we propose an end-to-end framework named the \textbf{L}egal \textbf{C}ase \textbf{M}atching Network with \textbf{L}aw \textbf{A}rticle \textbf{I}ntervention (\textbf{LCM-LAI}). 
Regarding each law article as a text expression of jurisprudential information by professionals, LCM-LAI leverages the complete set of law articles as valuable side information.
More specifically, LCM-LAI tackles the challenges (2) mentioned above from two key aspects.
On the one hand, we treat the legal-rational information in legal case fact descriptions as intermediate variables and propose the multi-task learning framework of LCM-LAI after theoretical analysis and formula derivation (cf. Sec.~\ref{sec:analysis}).
This framework includes a relevant law article prediction sub-task to effectively capture the legal-rational information from the fact descriptions of legal cases.
Such a multi-task learning framework eliminates the strong constraint that requires applicable law articles as input like Sun et al.~\cite{sun2023law} and reasonably solves the LCR and LCM tasks with the captured legal-rational information.
On the other hand, to effectively model the jurisprudential interaction between two legal cases, LCM-LAI proposes an innovative article-aware attention mechanism that not only captures critical legal-rational information, but also measures the legal-rational correlation between two across-case sentences to enhance the model’s predictive capabilities.
Instead of directly calculating the similarity between sentence embeddings like general semantic correlation measurement, this attention mechanism composes a vector of the attention scores between sentences and each law, forming the law distribution vector. 
Subsequently, we compute the similarity between the law distribution vector of sentences as their jurisprudential interaction score.
This approach allows for a more effective measurement of the jurisprudential correlation between two cross-case sentences as it considers the distribution of legal-rational information within relevant law articles. 
Combining semantic and jurisprudential interaction knowledge, LCM-LAI outperforms existing baselines in both LCR and LCM tasks.

We summarize the contributions of our work as follows:
\begin{enumerate}
    % \item {Grounded in rigorous theoretical analysis and formulaic derivation, we propose a novel end-to-end dependent multi-task learning framework termed LCM-LAI\footnote{Our source codes are available at \url{https://github.com/prometheusXN/LCM-LAI}.}. This framework adeptly incorporates a sub-task of law article prediction, effectively capturing and integrating legal-rational information across cases.}
    
    \item {We introduce LCM-LAI\footnote{Our source codes are available at \url{https://github.com/prometheusXN/LCM-LAI}.}, a novel end-to-end dependent multi-task learning framework, to solve LCR and LCM tasks. Relying on a sub-task of law article prediction, it effectively evaluates the correlation between cases from both semantic and legal aspects.}
    
    \item {We propose a novel article-aware attention mechanism to effectively measure the jurisprudential correlation between two across-case sentences by considering the distribution of relevant laws instead of directly calculating the similarity between sentence embeddings.
    % Besides, it efficiently solves the challenging multi-label law article prediction subtask.
    }
    
    \item {We conducted extensive experiments on four real-world datasets for both LCR and LCM tasks. 
    Comprehensive experiments illustrate that our model not only outperforms all existing state-of-the-art methods with a maximum of $7.02\%$ relative improvement, but also meets the time efficiency of practical applications.}
    
\end{enumerate}

We organize the rest of this paper as follows. 
Sec.~\ref{sec:related_work} summarizes related works.
Sec.~\ref{sec:methodology} formulates the problem, carries on the rigorous theoretical analysis, and obtains the framework design of the LCM-LAI.
Sec.~\ref{sec:methods} presents our method LCM-LAI in detail.
The performance evaluation and testing results are in
Sec.~\ref{sec:experiments}.
Some discussion and conclusion remarks then follow.  