\section{Discussion 2: Generalizability and Limitations}
\noindent {\bf Generalizability of LCM-LAI in civil law systems.}
It should be emphasized that compared with the general document retrieval model, the proposed LCM-LAI only differs in the requirement to take the text definition of law articles as inputs and the law article prediction sub-task to capture the legal-rational correlation between cases. The applicability of LCM-LAI is extensive within civil law systems (e.g., China, France, Japan, and so on), primarily due to the explicit codification of law articles by the relevant judicial authorities. There are two reasons why this paper only conducts experiments on Chinese legal data: 1) The Chinese legal system can effectively represent the civil law system; 2) For authors, the accessibility of Chinese legal data is the least difficult, especially the detailed text definition of relevant law articles.


\noindent {\bf Limitations of LCM-LAI in common law systems.}
Due to the lack of written law articles, directly applying the proposed LCM-LAI model in the common law system (e.g., Britain, America, Canada, etc.) is limited.
Here, we also provide an envisioned feasible strategy to apply LCM-LAI to the common law system.
As mentioned in Sec.~\ref{sec:analysis}, LCM-LAI treats the law articles as the summary of the legal-rational information for the corresponding class of cases, an intuitive way to make LCM-LAI work is to summarize the law substitutes from a large number of legal cases.
Fortunately, in both common law systems and civil law systems, it is natural to cluster historical legal cases according to the labels of the convictions.
Next, by applying text summarization, topic modeling, and even ChatGPT technology, we can easily derive a text summarization for each set of legal cases, which can substitute for the law articles in our LCM-LAI.
Of course, the quality of this summary generation will affect the effect of the subsequent use of LCM-LAI.
As for the specific effects of the above strategy, we will further explore it in the future.


% \section{Discussion: how to apply on the Anglo-American Law System} \label{sec:discussion}
% {\color{blue} [delete this section] In the Anglo-American law system (e.g., Britain, America, Canada, etc.), the lack of written law articles may make it difficult to apply the LCM-LAI model.
% In this section, we discuss an alternative to our LCM-LAI model in the absence of written law articles.

% As mentioned in Sec.~\ref{sec:analysis}, LCM-LAI treats the law articles as the summary of the legal-rational information for the corresponding class of cases, an intuitive way to make LCM-LAI work is to summarize the substitutes of the law from a large number of legal cases.
% Fortunately, in both Anglo-American law systems and civil law systems, it is natural to cluster historical legal cases according to the labels of the convictions.
% Next, with the application of text summarization, topic modeling, and even ChatGPT technology, we can easily derive a text summarization for each set of legal cases, which can be the substitutes for the law articles in our LCM-LAI.
% Of course, the quality of this summary generation will affect the effect of the subsequent use of LCM-LAI. 
% This is out of the scope of this work, and we only discuss feasible solutions here.}
% % Referring to a lot of previous research~\cite{zhao2011Topic_model, yurochkin2019OT_LDA}, the topic is a good choice.
