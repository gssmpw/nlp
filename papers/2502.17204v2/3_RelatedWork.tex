\subsection{Complex Instruction Following}
% 从通用的指令遵循到多约束的指令遵循
Riding on the wave of the large language model, the instruction following has attracted increasing attention for it is easy to be perceived by the users~\cite{zhou2023instructionfollowing, lou2024large}. Practical instructions are complex, usually incorporated with multiple constraints of different types~\cite{zhou2023instruction, he2024can}. A lot of evaluation benchmarks have found that multi-constraint instruction following is nontrivial for the LLMs~\cite{jiang2023followbench, wen2024benchmarking, qin2024infobench}. Consequently, several works propose to improve the LLM's complex instruction following capacity by introducing additional instruction fine-tuning~\cite{sun2024conifer, cheng2024spar, zhang2024divideverifyrefine}. 

Different from these works, we focus on the inference stage of the LLMs instead of model training. Especially, we aim to investigate the position bias problem brought by the constraint order, which poses an essential impact on the model performance.




\subsection{Position Bias in the LLM}
% 一些工作已经发现了指令中条件位置的改变会影响模型的推理能力 引一些数学和逻辑的work
% 但是 没有工作研究 这个指令中约束位置对于模型指令遵循的影响 最相关的一篇工作是SIFO 提出了一个benchm ark,the model must follow multiple instructions step by step to reach the final desired outcome, thus mitigating the positional bias influence. However, 作者初步发现了约束顺序可能会影响模型的指令遵循表现,但是没有做出系统性的分析the author did not make a systematical investigation about the positona bias in LLM
The position bias problem is common in the various LLM tasks~\cite{liu2024lost, zheng2023judging, zeng2023evaluating}. Researchers fisrt find that the LLM's performance degrades dramatically by merely changing the order of relevant information in the long-context question answering. A lot of works have studied the position bias problem in the field of logical reasoning~\cite{chenpremise, liu2023concise, berglund2023reversal}. They find the LLM is sensitive to the order of premises, although such ordering actually does not alter the reasoning task~\cite{chenpremise, liu2023concise}.  

Despite so, none of these works has studied the position bias problem in the field of instruction following, especially multi-constraint instruction following. SIFo~\cite{chen2024sifo} is the most related work to ours. They manually differentiate the constraints based on the context length they will influence and conduct an empirical study to verify whether the model performance will be affected by the constraint order. However, Their investigation of position bias is fairly qualitative. Different from them, we are the first to make a systematical and thorough investigation on the position bias of constraints in multi-constraint instruction following.