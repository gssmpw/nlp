% This must be in the first 5 lines to tell arXiv to use pdfLaTeX, which is strongly recommended.
\pdfoutput=1
% In particular, the hyperref package requires pdfLaTeX in order to break URLs across lines.

\documentclass[11pt]{article}

% Change "review" to "final" to generate the final (sometimes called camera-ready) version.
% Change to "preprint" to generate a non-anonymous version with page numbers.
% \usepackage[review]{acl}
\usepackage[preprint]{acl}

% Standard package includes
\usepackage{times}
\usepackage{latexsym}

% For proper rendering and hyphenation of words containing Latin characters (including in bib files)
\usepackage[T1]{fontenc}
% For Vietnamese characters
% \usepackage[T5]{fontenc}
% See https://www.latex-project.org/help/documentation/encguide.pdf for other character sets

% This assumes your files are encoded as UTF8
\usepackage[utf8]{inputenc}

% This is not strictly necessary, and may be commented out,
% but it will improve the layout of the manuscript,
% and will typically save some space.
\usepackage{microtype}

% This is also not strictly necessary, and may be commented out.
% However, it will improve the aesthetics of text in
% the typewriter font.
\usepackage{inconsolata}

%Including images in your LaTeX document requires adding
%additional package(s)
\usepackage{graphicx}

% If the title and author information does not fit in the area allocated, uncomment the following
%
%\setlength\titlebox{<dim>}
%
% and set <dim> to something 5cm or larger.

\usepackage{color, xcolor}
\usepackage{booktabs}
\usepackage{multicol}
\usepackage{multirow, makecell, caption}
\usepackage{colortbl}
\usepackage{tikz}
\usepackage{arydshln}
\usepackage{pgfplots}
\usepackage{amsmath}
\usepackage{amssymb}

\usepackage{tabularx}
\usepackage{enumerate}

\makeatletter
\def\adl@drawiv#1#2#3{%
        \hskip.5\tabcolsep
        \xleaders#3{#2.5\@tempdimb #1{1}#2.5\@tempdimb}%
                #2\z@ plus1fil minus1fil\relax
        \hskip.5\tabcolsep}
\newcommand{\cdashlinelr}[1]{%
  \noalign{\vskip\aboverulesep
           \global\let\@dashdrawstore\adl@draw
           \global\let\adl@draw\adl@drawiv}
  \cdashline{#1}
  \noalign{\global\let\adl@draw\@dashdrawstore
           \vskip\belowrulesep}}
\makeatother

\title{Order Matters: Investigate the Position Bias in Multi-constraint Instruction Following}

% Author information can be set in various styles:
% For several authors from the same institution:
% \author{Author 1 \and ... \and Author n \\
%         Address line \\ ... \\ Address line}
% if the names do not fit well on one line use
%         Author 1 \\ {\bf Author 2} \\ ... \\ {\bf Author n} \\
% For authors from different institutions:
% \author{Author 1 \\ Address line \\  ... \\ Address line
%         \And  ... \And
%         Author n \\ Address line \\ ... \\ Address line}
% To start a separate ``row'' of authors use \AND, as in
% \author{Author 1 \\ Address line \\  ... \\ Address line
%         \AND
%         Author 2 \\ Address line \\ ... \\ Address line \And
%         Author 3 \\ Address line \\ ... \\ Address line}

\author{
    Jie Zeng\textsuperscript{1}, Qianyu He\textsuperscript{1}, Qingyu Ren\textsuperscript{1,3}, Jiaqing Liang\textsuperscript{2\textdagger}, Yanghua Xiao\textsuperscript{1\textdagger}\\\textbf{Weikang Zhou\textsuperscript{3}, Zeye Sun\textsuperscript{3}, Fei Yu\textsuperscript{3}}\\
    \\
    \textsuperscript{1}Shanghai Key Laboratory of Data Science, School of Computer Science, Fudan University \\
    \textsuperscript{2}School of Data Science, Fudan University  \textsuperscript{3}Ant Group\\
    \{jzeng23, qyhe21, qyren24\}@m.fudan.edu.cn, \{liangjiaqing, shawyh\}@fudan.edu.cn\\
}

%\author{
%  \textbf{First Author\textsuperscript{1}},
%  \textbf{Second Author\textsuperscript{1,2}},
%  \textbf{Third T. Author\textsuperscript{1}},
%  \textbf{Fourth Author\textsuperscript{1}},
%\\
%  \textbf{Fifth Author\textsuperscript{1,2}},
%  \textbf{Sixth Author\textsuperscript{1}},
%  \textbf{Seventh Author\textsuperscript{1}},
%  \textbf{Eighth Author \textsuperscript{1,2,3,4}},
%\\
%  \textbf{Ninth Author\textsuperscript{1}},
%  \textbf{Tenth Author\textsuperscript{1}},
%  \textbf{Eleventh E. Author\textsuperscript{1,2,3,4,5}},
%  \textbf{Twelfth Author\textsuperscript{1}},
%\\
%  \textbf{Thirteenth Author\textsuperscript{3}},
%  \textbf{Fourteenth F. Author\textsuperscript{2,4}},
%  \textbf{Fifteenth Author\textsuperscript{1}},
%  \textbf{Sixteenth Author\textsuperscript{1}},
%\\
%  \textbf{Seventeenth S. Author\textsuperscript{4,5}},
%  \textbf{Eighteenth Author\textsuperscript{3,4}},
%  \textbf{Nineteenth N. Author\textsuperscript{2,5}},
%  \textbf{Twentieth Author\textsuperscript{1}}
%\\
%\\
%  \textsuperscript{1}Affiliation 1,
%  \textsuperscript{2}Affiliation 2,
%  \textsuperscript{3}Affiliation 3,
%  \textsuperscript{4}Affiliation 4,
%  \textsuperscript{5}Affiliation 5
%\\
%  \small{
%    \textbf{Correspondence:} \href{mailto:email@domain}{email@domain}
%  }
%}

\begin{document}
\maketitle

\begin{abstract}

\begin{abstract}
% We present a learning-based method for multiple human rendering from sparse sets of multi-view images. Most current works are focused on single human settings that deliver accurate geometry and appearance using implicit neural representations. However, extending these methods for estimating multiple humans from sparse images remains challenging due to additional occlusion and clutter of multiple humans and the limited number of input views. We propose a neural implicit reconstruction method that addresses the inherent challenges. First, we propose to use geometry constraints by exploiting pre-computed meshes using a human body model (SMPL). Specifically, we regularize the signed distances using the SMPL mesh and leverage bounding boxes for improved rendering. Second, we propose a patch-based ray regularization to minimize rendering inconsistencies and a saturation regularization for robust optimization in variable illumination. Extensive experiments on both real-world and synthetic datasets demonstrate the benefits of our approach and show state-of-the-art performance against existing neural reconstruction methods. 

We present a method for recovering the shape and radiance of a scene consisting of multiple people given solely a few images. 
Multi-human scenes are complex due to additional occlusion and clutter. For single-human settings, existing approaches using implicit neural representations have achieved impressive results that deliver accurate geometry and appearance. 
However, it remains challenging to extend these methods for estimating multiple humans from sparse views. 
We propose a neural implicit reconstruction method that addresses the inherent challenges of this task through the following contributions: First, we propose to use geometry constraints by exploiting pre-computed meshes using a human body model (SMPL). Specifically, we regularize the signed distances using the SMPL mesh and leverage bounding boxes for improved rendering. Second, we propose a ray regularization scheme to minimize rendering inconsistencies, and a saturation regularization for robust optimization in variable illumination.  Extensive experiments on both real and synthetic datasets demonstrate the benefits of our approach and show state-of-the-art performance against existing neural reconstruction methods. 

% Finally, we demonstrate how our framework can be used for editing applications. 
\end{abstract}

% We present a neural learning method for multiple human reconstruction from a sparse set of camera views. Recent neural surface reconstruction methods could generate both 3D geometry and appearance, while usually fail to reconstruct consistent surfaces due to lacking of explicit multi-view geometry constraints, especially for complex scene(\eg multiple human). {\bf Previous works demonstrated that using geometry priors for simple objects significantly enhance the quality of the surfaces. In this work, we propose to extend this idea to using human body model (\eg SMPL) priors, represented by signed distance functions (SDF).} However, since human SMPL usually lacks sufficient details (\eg hair, cloth), we estimate SDF together with its' uncertainty for explicit smooth geometry regularization. In addition, we propose a patch based ray regularization to minimizes potential reconstruction inconsistency and saturation regularization for robust optimization in variable illuminations.  Our evaluations on both real human dataset (CMU Panoptic Dataset\cite{Simon_2017_CVPR,Joo_2017_TPAMI}) and synthetic data (THUman2.0 Dataset and MultiHuman-Dataset \cite{tao2021function4d,zheng2021deepmulticap} demonstrate state-of-the-art performance quantitatively and qualitatively. Extensive experiments show our methods enable a variety editing applications in 3D space without additional assistance of depth, masks, or segmentation.

%Existing multi-person reconstruction is often based on human models, which could reconstruct complex geometry but lose rendering quality due to lacking hair and clothing detail. Recently, volumetric rendering (\eg NeRF\cite{mildenhall2020nerf}) demonstrate promising rendering quality on novel views synthesis from dense input views, while reconstructing fidelity surfaces remains a challenge. Moreover, surface presentation methods \cite{yariv2021volume,wang2021neus} explore surface reconstruction during volume rendering, but it's still hard to handle occlusions and complex geometry (\eg multi-human) from sparse input views. Our methods take the advantages of the human model(\eg SMPL), recent volumetric rendering and scene representation methods. Specifically,we define our multi-human SMPL's surfaces as a zero-level set of a signed distance function (SDF) and train a neural SDF representation. We show how to incorporate this representation in point sampling, neural rendering and reconstruction, and propose a joint optimization for non-rigid shapes (\egclothed humans). Finally,

% We present a learning-based method for reconstructing multiple humans from a sparse set of camera views. Current surface reconstruction methods can generate geometry and appearance simultaneously, but suffer to reconstruct 3D consistent surfaces due to a lack of explicit multi-view geometric constraints, especially for complex scenes (e.g. multiple humans). Previous works show that the use of geometric priors for simple objects (or single scenes) significantly improves the fidelity of reconstructed surfaces. Leveraging that, we propose the use of utilizing human body model (e.g. SMPL) priors, represented by signed distance functions (SDF). However, human SMPL often lacks sufficient details (e.g. hair, cloth), and thus, we estimate the SDF along with its uncertainty to obtain smooth geometry regularization. In addition, we propose a patch-based method for ray regularization to potentially minimize inconsistencies in the reconstruction and to enable saturation regularization to increase the robustness during the optimization of variable illuminations. Our evaluations on both real human datasets (CMU Panoptic Dataset [32, 63]) and synthetic data (THUman2.0 Dataset and MultiHuman Dataset [83, 94] demonstrate state-of-the-art performance quantitatively and qualitatively. Extensive experiments show that our method enables a variety of editing applications in 3D space without additional assistance of depth, masks, or segmentation.



\end{abstract}

\section{Introduction}\label{sec:intro}

\section{Introduction}
\label{sec:intro}

Human reconstruction from single images \cite{choutas2020monocular, kanazawa2018end,liu2022recent}, multiple images \cite{guo2019relightables, collet2015high}, RGB videos \cite{alldieck2018detailed,Kocabas20} or RGB-D data \cite{yu2017bodyfusion,yu2018doublefusion} has received a lot of attention, much less explored is the task of \emph{multiple} human scenario, which is essential for scene understanding, behavior modeling, collaborative augmented reality, and sports analysis.  
%
The multi-human setting introduces additional challenges, as there is now a higher level of occlusion and clutter 
which hinders matching and reconstruction. 
Although in principle one could approach this by first detecting and then independently processing each person, 
simultaneous reconstruction of multiple humans can help to globally reason about occlusion at the level of the scene~\cite{jiang2020coherent, sun2022putting}, 
%which has been shown to produce better results~\cite{jiang2020coherent, sun2022putting}, 
and can potentially recover coherent 3D spatial relations among the people.

Several recent works have attempted to recover multiple humans from a single view \cite{choi2022learning, sun2022putting, sun2021monocular, zanfir2018deep, zanfir2018monocular, fieraru2020three, jiang2020coherent, zhang2021body, ugrinovic2021body,mustafa2021multi}. However, the majority of these are based on regressing the parameters of a human body model --typically SMPL \cite{loper2015smpl}--
which provides coarse reconstructions that %cannot handle 
lack hair, clothing, and geometric details. 
Multi-view settings can help resolve some of the occlusions as well as depth ambiguities, but require a dense array of RGB cameras to achieve a detailed reconstruction \cite{collet2015high, joo2015panoptic,vlasic2009dynamic}.
% which is not easily accessible. % available. 
A more convenient capture system %that could in principle still deliver optimal results 
is the \emph{sparse} multi-view setting, where only a handful of cameras is required.
%, where the number of cameras is limited to 2-15. 
However, due to the decreased number of views and increased level of occlusion, 
existing methods require segmentation masks and a pre-scanned template mesh \cite{liu2011markerless, wu2013set}, rely on a coarse body model \cite{zhang2021lightweight, huang2021dynamic}, or require temporal information \cite{zheng2021deepmulticap, huang2021dynamic}.

A parallel line of work simultaneously tackles the novel-view-synthesis and geometry-reconstruction problems by combining neural coordinate-based representations, \eg implicit signed distance functions (SDFs) \cite{park2019deepsdf}, with differentiable rendering \cite{yariv2021volume,wang2021neus,yariv2020multiview,mildenhall2020nerf}. 
This approach has the advantage of producing, along with geometry, renderings from novel viewpoints that can capture complex surface/light interactions, increasing the scope of applications. 
NeRF~\cite{mildenhall2020nerf}, for example, uses volumetric rendering to produce impressive images under novel views, albeit at the cost of sub-optimal geometries due to the unconstrained volumetric representation. 
SDF-based methods \cite{yariv2021volume,wang2021neus,yariv2020multiview}, while delivering images of slightly lower quality, have been shown to produce 3D surfaces that are competitive with classical approaches. 
For humans, this has been leveraged to obtain geometry and appearance from monocular video \cite{jiang2022selfrecon,chen2021animatable}, RGB-D video \cite{dong2022pina}, and sparse multi-view video \cite{wang2022arah, liu2021neural, peng2021neural, zheng2021deepmulticap, kwon2021neural, peng2021animatable, xu2021h, weng2022humannerf}. 
%% NOTE: \cite{xu2021h} also has experiments for static sparse MV image
However, none of these works, with the exception of \cite{zheng2021deepmulticap,zhang2021editable}, were designed to handle the increased geometric complexity and occlusion of the multi-human case. 
%%%%%%%%%%%%%%%%%%%%%%%%%%%%%%%%%%%%%%%%%%%%%not the only work, STnerf, SIgsia2022 and deep multicap
Current works \cite{zheng2021deepmulticap,zhang2021editable} address the multi-human setting, but both require a set of videos, which effectively becomes a dense array of views as long as deformations are modeled correctly.
%DeepMultiCap \cite{zheng2021deepmulticap} is the only work that addresss the multi-human setting, but the method requires segmentation and was focused on reconstruction from videos, which effectively becomes a dense array of views as long as deformations are modeled correctly.

In this paper, we address the problem of multiple 3D human surfaces and volume rendering from sparse static multi-view images. Our key insight is that human-specific geometric constraints can be leveraged to tackle the challenging sparse-view setting.

Specifically, we first obtain a SMPL body model from the input data and use it to initialize the implicit SDF network, where we define the surface of a multi-human scene as the zero-level set of the SDF. 
Then, the geometry network is optimized with multi-view images by leveraging surface and volume rendering ~\cite{wang2021neus} along with uncertainty estimation methods \cite{deng2022depth,roessle2022dense}, where the SMPL meshes are treated as noisy estimations.  
% [to keep the method general for novel scenes, we do not rely on features pre-trained on a large dataset]
To achieve higher rendering quality from sparse training views, we additionally propose a patch-based regularization loss that guarantees consistency across different rays and a saturation regularization that ensures consistency for variable image illuminations within the same scene.% \va{I feel this is a problem that was not pointed out before, and it would be nice to elaborate better}.

We evaluate our method quantitatively and qualitatively on real multi-human (CMU Panoptic~\cite{Simon_2017_CVPR,Joo_2017_TPAMI}) and synthetic (MultiHuman~\cite{zheng2021deepmulticap}) datasets. We demonstrate results on 5,10,15 and 20 training views, where we achieve state-of-the-art performance in terms of surface reconstruction and novel view quality. 

%In summary, our contributions are: 
%\begin{itemize}
%\item We propose the first neural implicit surface and volume rendering for multiple humans from sparse static images; 
%\item We propose a novel geometric initialization and regularization based on human SMPL, which allows for multi-human rendering simultaneously; %To address the problem of occlusion, we propose the use of SMPL for geometric regularization; 
%\item We propose a patch-based ray consistency regularization and an image saturation regularization that ensures illumination consistency across views;
%\item Our method achieves state-of-the-art performance and the code will be public online. 
%\end{itemize}

%%%%%%%%%%%%%%%%%%%%%%
%% MOTIVATION
% Motivations for multi-human from the multi-human, single-view domain:
%       - A lot of occlusions, if only a crop of the person is used then SOTA methods might fail (\cite{choi2022learning})
%       - Simultanesouly considering multiple person has been empirically shown to work better \cte{all the works on mult-human SMPL}
%       - The reconstructions can be made coherent in 3D space, i.e. with correct relative depth among the people. Plausible 3D spatial relations.
% The problem with SMPL based methods is that they cannot reconstruct clothes, hair, etc

% Applications in behavior analysis, automatic video analysis of sport events, or collaborative augmented reality applications
% human-computer interaction, human behavioral modeling, assisted therapy, monitoring sports performances, protection and security, special effects, modeling and indexing archival footage, or self-driving cars. (this is for monocular)

% We take one step beyond SMPL and use implicit representations for more accurate reconstructions

% "The straightforward solution consists in regarding different people as independent instances and estimating the body shapes and poses one by one using a single-person approach. This strategy, however, may result in inconsistent spatial arrangements and erroneous poses of the reconstructed people." (re-phrase)

% *** Why we cannot do cut one person then reconstruct it --> there should be experiments, else be careful with claims

%%%%%%%%%%%%%%%%%%%%%%
%% STORY
%   - Reconstructing geometry from a scene with multiple humans is hard because of occlusions
%   - Multiple views can help disambiguate, but it's stil hard.
%       => most are single human. For multi-human one can detect and reconstruct, but this can fail - and matching detections among views might not be straight-forward (?) Also, computational time scales (linearly) with the number of people
%       => there are a few multi-human, but ?
%       => people resort to SMPL to make the problem more tractable
%           *** classic matching methods: problem with multi-humans??
%   - SMPL doesn't have clothes or hair so geometry sucks
%   - We consider in particular the scenario where only a sparse number of cameras is available; dense rigs are not easily accessible. Dense => expensive and sophisticated hardware setup and low run-time efficiency (re-phrase)
%   - To address this, we present a method for multi-human 3D reconstruction from multiple views based on neural rendering. 
%   - To handle the occlussions and sparse views, we first fit a SMPL model. Following the recent line of works that make use of geometric quantities to improve reconstruction, we propose to to initialize the reconstruction by building an SDF from the acquired SMPL models.
%   - 

%%%%%%%%%%%%%%%%%%%%%%
% Multi-view capture with dense views and/or temporal data demands a significant cost in time and money, with increased complexity of device requirements, synchronisation, data storage and transfer times. Consequently, 3D reconstruction from a sparse set of images has become an increasingly popular problem~\cite{niemeyer2022regnerf, kim2022infonerf,long2022sparseneus,dong2021shape}, more so after the recent progress in deep neural rendering and reconstruction approaches~\cite{mildenhall2020nerf, park2019deepsdf}. % However, it remains a difficult problem due to the sparsity of 3D cues, requiring carefully designed regularization losses~\cite{kim2022infonerf,niemeyer2022regnerf} or knowledge transfer from pre-trained networks~\cite{yu2021pixelnerf}. This cue scarcity problem becomes even worse when reconstructing cluttered scenes, such as those containing a multitude of people., due to increased occlusions.
% % Multi-view capture with dense views and/or temporal data entails significant capture device requirements, synchronisation, data storage and transfer challenges. Hence, being able to capture scenes from a mere set of static sparse images has become a more than ever popular problem \cite{niemeyer2022regnerf, kim2022infonerf,long2022sparseneus,dong2021shape} especially in the wake of deep learning. 
% Yet, the scarcity of 3D cues in such setups requires carefully designed regularization or knowledge transfer. This scarcity is exacerbated further with cluttered scenes, such as scenes containing a multitude of people.   

% 3D reconstruction of humans from single images \cite{choutas2020monocular, kanazawa2018end,liu2022recent}, multiple images \cite{guo2019relightables, collet2015high}, RGB videos \cite{alldieck2018detailed,Kocabas20} or RGB-D data \cite{yu2017bodyfusion,yu2018doublefusion} has received a lot of attention, much less explored is the task of \emph{multiple} human reconstruction, which is essential for scene understanding, behavior modeling, collaborative augmented reality, and sports analysis.  
% %
% The multi-human setting introduces additional challenges, as there is now a higher level of occlusion and clutter 
% which hinders matching and reconstruction. 
% Although in principle one could approach this by first detecting and then independently processing each person, 
% simultaneous reconstruction of multiple humans can help to globally reason about occlusion at the level of the scene~\cite{jiang2020coherent, sun2022putting}, 
% %which has been shown to produce better results~\cite{jiang2020coherent, sun2022putting}, 
% and can potentially recover coherent 3D spatial relations among the people. %, taking a step towards better scene understanding.
% % simultaneous reconstruction of multiple humans has several advantages. 
% % First, this strategy  allows to further recover coherent 3D spatial relations among the people, taking a step towards better scene understanding.
% % Second, the presence of multiple people can imply a high level of occlusion. While some works explicitly handle this case for single humans \cite{khirodkar2022occluded, zhang2020object}, globally reasoning about occlusion at the level of the scene has been shown to produce better results \cite{jiang2020coherent, sun2022putting}.  
% % Finally, the detect-and-reconstruct approach increases (linearly) in computational time with each new person. 

% % While 
% Several recent works have attempted to recover multiple humans from a single view \cite{choi2022learning, sun2022putting, sun2021monocular, zanfir2018deep, zanfir2018monocular, fieraru2020three, jiang2020coherent, zhang2021body, ugrinovic2021body,mustafa2021multi}. However, the majority of these are based on regressing the parameters of a human body model --typically SMPL \cite{loper2015smpl}--
% which provides coarse reconstructions that %cannot handle 
% lack hair, clothing, and geometric details. 
% Multi-view settings can help resolve some of the occlusions as well as depth ambiguities, but require a dense array of RGB cameras to achieve a detailed reconstruction \cite{collet2015high, joo2015panoptic,vlasic2009dynamic} 
% % which is not easily accessible. % available. 
% A more convenient capture system %that could in principle still deliver optimal results 
% is the \emph{sparse} multi-view setting, where only a handful of cameras is required.
% %, where the number of cameras is limited to 2-15. 
% However, due to the decreased number of views and increased level of occlusion, 
% existing methods require segmentation masks and a pre-scanned template mesh \cite{liu2011markerless, wu2013set}, rely on a coarse body model \cite{zhang2021lightweight, huang2021dynamic}, or require temporal information \cite{zheng2021deepmulticap, huang2021dynamic}.

% %%%\todo{pifu should go somewhere here... pifu is single human though. Goes into the recent line of work: pre-train on a large dataset and then at test time use sparse views: \cite{zheng2021deepmulicap}}


% % --> send the message that MV with SMPL is kind of solved, but we need to take the next step. 
% %but as shown by several works \cite{...}, this is not sufficient to solve the problem. In fact, % in effect, in practice, at bottom
% % "The first sparse for multi-humans" -> \cite{zhang2021lightweight}, 

% % In parallel, there has been development in neural 3D geometry reconstruction by using SDFs \cite{..} or volume-rendered SDFs \cite{...}. These are great but require many views. For the sparse view case, infonerf, regnerf, dietnerf. They show improved results, but as demonstrated here these are not sufficient for the multi-human reconstruction problem.




% %%%%\todo{improving geometry to improve rendering?}

% % Inspired by the success of NeRF~\cite{mildenhall2020nerf}, 
% A parallel line of work simultaneously tackles the novel-view-synthesis and geometry-reconstruction problems by combining neural coordinate-based representations, \eg implicit signed distance functions (SDFs) \cite{park2019deepsdf}, with differentiable rendering \cite{yariv2021volume,wang2021neus,yariv2020multiview,mildenhall2020nerf}. 
% This approach has the advantage of producing, along with geometry, renderings from novel viewpoints that can capture complex surface/light interactions, increasing the scope of applications. 
% NeRF~\cite{mildenhall2020nerf}, for example, uses volumetric rendering to produce impressive images under novel views, albeit at the cost of sub-optimal geometries due to the unconstrained volumetric representation. 
% SDF-based methods \cite{yariv2021volume,wang2021neus,yariv2020multiview}, while delivering images of slightly lower quality, have been shown to produce 3D surfaces that are competitive with classical approaches. 
% For humans, this has been leveraged to obtain geometry and appearance from monocular video \cite{jiang2022selfrecon,chen2021animatable}, RGB-D video \cite{dong2022pina}, and sparse multi-view video \cite{wang2022arah, liu2021neural, peng2021neural, zheng2021deepmulticap, kwon2021neural, peng2021animatable, xu2021h, weng2022humannerf}. 
% %% NOTE: \cite{xu2021h} also has experiments for static sparse MV images (single human)
% %% Monocular video: also "SelfNeRF: Fast Training NeRF for Human from Monocular Self-rotating Video". But this isn't published yet
% %
% However, 
% none of these works, with the exception of \cite{zheng2021deepmulticap}, were designed to handle the increased geometric complexity and occlusion of the multi-human case. DeepMultiCap \cite{zheng2021deepmulticap} is the only %multi-view neural reconstruction 
% work that addresss the multi-human setting, but the method requires a set of videos, which effectively becomes a dense array of views as long as deformations are modeled correctly.

% In this paper we address, for the first time, 
% the problem of reconstructing multiple 3D humans from a \emph{static} and \emph{sparse} set of cameras using neural implicit surfaces. 
% Our key insight is that human-specific geometric constraints can be leveraged to tackle the challenging sparse-view setting.
% %by first fitting  a SMPL body model to the input, building on top of the works 
% % building on the set of works that can faithfully disambiguate pose and shape for multiple people, but only deliver coarse reconstructions, \eg~\cite{huang2017towards,zhang2021lightweight}. %\va{This is a bit risky, since the way we get smpl is a bit shady (in practice, from dense MV cameras)}. 
% Specifically, we first obtain a SMPL body model from the input data, and use this to train a geometry-only implicit SDF network, where we define the surface of a multi-human scene as the zero-level set of the SDF. 
% In a second step, the geometry network is fine-tuned using multi-view images by leveraging the recently-proposed NeuS~\cite{wang2021neus} along with uncertainty-based rendering \cite{deng2022depth,roessle2022dense}, where the SMPL meshes are treated as noisy estimations.  
% % [to keep the method general for novel scenes, we do not rely on features pre-trained on a large dataset]
% To achieve higher rendering quality from sparse training views, we additionally propose a patch-based regularization loss 
% that guarantees consistency across different rays, and a saturation regularization that ensure consistency for variable image illuminations within a same scene.% \va{I feel this is a problem that was not pointed out before, and it would be nice to elaborate better}.

% We evaluate our method quantitatively and qualitatively on real multi-human (CMU Panoptic~\cite{Simon_2017_CVPR,Joo_2017_TPAMI}) and synthetic (MultiHuman~\cite{zheng2021deepmulticap}) datasets. We demonstrate results on 5,10,15 and 20 training views, where we achieve state-of-the-art performance in terms of surface reconstruction and image quality. %Another advantage is that the proposed geometry initialization enables more efficient learning.

% In summary, our contributions are: 
% % 1. We demonstrate an approach to use multiple human bodies as geometric regularization. We propose to use SDF along with estimated uncertainty as explicit geometry constraints, allowing the  learning of details. \\
% (1) We propose the first neural 
% implicit geometry and appearance reconstruction method for multiple humans using a sparse set of static views; 
% (2) To address the problem of occlusion, we propose the use SMPL for geometric regularization; 
% (3) To handle sparse views under occlusion, we propose a patch-based ray consistency regularization, and an image saturation regularization that ensures illumination consistency across views.

%  Code and models will be made available.
% 3.  \todo{I will re-phrase this after finishing the method. This needs to come before in the introduction, if it stays} We combine box rendering with editing, which enables editing multi-humans in 3D space during inference without using masks and segmentation, while preserving the detail and quality of rendered novel multi-human views.   \\



%%%%%%%%%%%%%%%%%%%%%%
% --------------------------------


% Multiple human reconstruction has been a popular and significant topic in computer vision and computer graphics, which enables various applications, such as Virtual Reality (VR) and Augmented Reality (AR), films, teleconferences, and so on. Researches for human reconstruction mainly consist of model-based and model-free, both of them has achieved tremendous progress in recent years. 

% Traditional model-based approaches utilize parametric human bodies (\eg human SMPL \cite{loper2015smpl}) or truncated signed distance fields (TSDFs) for human geometry reconstruction \cite{loper2015smpl,bhatnagar2020combining,liu2021neural,yu2017bodyfusion,yu2018doublefusion,peng2021neural}. %Though human SMPL could provide solid geometry notes, 
% While relying on the human model(\eg SMPL skinning weights of a naked body) may risk losing hair and clothing details. Model-free methods learns from a set of multi-view images or videos with loosing detail constraints \cite{peng2021neural,dong2022pina,peng2021neural,dong2022pina, liu2021neural}, allowing more realistic reconstructed details. With the recent progress of implicit neural representations\cite{sitzmann2019deepvoxels,xu2022point,wiles2020synsin, mildenhall2020nerf,zhang2020NeRF++}, many model-free methods utilize NeRF to achieve photo-realistic rendering. However, NeRF\cite{mildenhall2020nerf} based methods do not sufficiently constrain the 3D geometry, making them hard to reconstruct high-fidelity surfaces or accurate geometry. Further, most existing researches focus on single person reconstruction from videos \cite{peng2021neural,dong2022pina} or images\cite{peng2021neural,dong2022pina, liu2021neural}. For multiple human reconstructions from a single frame, the task becomes even harder to recover geometry and appearance with sufficient details due to multiple human scenes usually containing more complex geometry and occlusions. In this paper, we address those challenges by taking advantage of recent neural implicit surface reconstruction methods\cite{yariv2021volume,wang2021neus,yariv2020multiview} and human bodies models(\eg SMPL\cite{loper2015smpl}), reconstructing multi-human 3D geometry and appearance simultaneously. 

% Recent neural implicit surface reconstruction methods\cite{yariv2021volume,wang2021neus,yariv2020multiview} propose to use a signed distance function (SDF) to present the surface and combine SDF-based density function with volumetric rendering. They could learn an implicit SDF representation and reconstruct geometry and appearance simultaneously. However, the optimization of those methods still depends on direct color field and lack explicit geometry constraints, making them hard to reconstruct consistent geometry and appearance for unseen regions(\eg sparse input view) and occlusion areas (\eg multiple human scene). To alleviate those limitations, we propose to use a multiple human bodies models(SMPL) as geometry initialization, as well as to provide explicit geometry constraints.

% %\cite{yariv2021volume} uses volume density function as Laplace’s cumulative distribution function (CDF) for geometry representation, 
% %Neus\cite{wang2021neus} and Volsdf\cite{yariv2021volume} are among scene representation methodsuse signed distance functions (SDF) for surface representation and introduce the SDF-induced density function to enable the volume rendering to

% %More recently, with the rapidly progress of implicit neural representations\cite{sitzmann2019deepvoxels,xu2022point,wiles2020synsin, mildenhall2020nerf,zhang2020NeRF++}, many researches combine differentiable neural rendering with human bodies model \cite{liu2021neural,peng2021neural,dong2022pina}. Among those,  \cite{liu2021neural} combines human SMPL with implicit learning, and uses it as a proxy to unwarp the surrounding 3D space into a canonical pose, \cite{peng2021neural} could render detailed novel views (\eg with hair,cloths) of the human body from a set sparse input video. Thanks to neural radiance fields (NeRF\cite{mildenhall2020nerf}), those approaches could achieve photo-realistic rendering. However, NeRF\cite{mildenhall2020nerf} does not sufficiently constrain the 3D geometry, making those methods hard to reconstruct high-fidelity surfaces or accurate geometry.

% %Recent scene representation methods combines surface representation with neural volume rendering \cite{yariv2021volume,wang2021neus,yariv2020multiview}. % In order to achieve high quality surface reconstruction while retaining rendering quality, 
% %Among those, \cite{oechsle2021unisurf} proposes a unified framework to reconstruct solid objects from 2D image inputs, 

% Specifically, inspired by Neus\cite{wang2021neus} and Volsdf \cite{yariv2021volume}, we define the surface of multiple human SMPL as a zero-level set of a signed distance function (SDF)  and use it to train a neural SDF presentation as a geometry prior. However, relying on SDF sampled from SMPL directly may risk losing hair and clothing details. Thus, we learn SDF together with uncertainty and combine them together for explicit geometry regularization, which enables the learning of multiple person's clothing and hair details. In addition,
% to achieve higher rendering quality from sparse training views, we propose a patch-based regularization to guarantee consistency across different rays and saturation regularization to ensure image illumination consistency. 
% %retaining multi-view consistency remains a challenge for NeuS-related methods, especially for complex thin structures and large smooth regions. To tackle this challenge, we propose a patch-based ray regularization for photo-metric consistency and saturation regularization for image illumination consistency. 

% %Assuming this SDF presentation storing the multi-human geometry information, we incorporate this by re-initializing the follow-up neural network with the learned weights, utilizing the SDF value for the weights of hierarchical sampling \cite{mildenhall2020nerf} and volume rendering. Inspired by \cite{ortiz2022isdf, dong2022pina}, instead of using the SDF value sampled from SMPL to supervise the neural network learning directly(which consumes lots of time and lacks details), we propose a point-based SDF regularization allowing for the learning of multi-human clothing and hair details. Moreover, we present an approach for human editing during rendering, without additional training or information(\eg depth, mask or segmentation).   \\
% We evaluate our method on both real multi-human datasets (CMU Panoptic Dataset\cite{Simon_2017_CVPR,Joo_2017_TPAMI}) and synthetic data (THUman2.0 Dataset
% and MultiHuman-Dataset \cite{tao2021function4d,zheng2021deepmulticap}) both qualitatively and quantitatively. Specifically,  we demonstrate testing results on 10,15,20 training views,respectively, and achieve state-of-the-art performance on both real datasets and synthetic datasets. %Another advantage is that the proposed geometry initialization enables more efficient learning. 
% In summary, our contributions include:\\
% 1. We demonstrate an approach to use multiple human bodies as geometric initialization. We propose to use SDF along with estimated uncertainty as explicit geometry constraints, allowing the  learning of details. \\
% 2. We propose a patch-based KL regularization to ensure consistency across different rays and image saturation regularization for illumination consistency. \\
% 3. We combine box rendering with editing, which enables editing multi-humans in 3D space during inference without using masks and segmentation, while preserving the detail and quality of rendered novel multi-human views.   \\


\section{Related Work}

\section{Background and Related Work}
\label{sec: RelatedWork}
This section provides an overview of relevant background and previous work that underpins this study. We begin by discussing the advances in LLMs, move on to research related to slang detection and identification, and conclude by exploring the application of LLMs in evolutionary game theory and social simulations.

\subsection{Large Language Models}
The advent of LLMs has revolutionized the field of natural language processing (NLP), with models like  GPT-4~\cite{openai2024gpt4technicalreport} and LLaMA~\cite{touvron2023llama} showcasing state-of-the-art performance in various linguistic tasks. These models, built upon the Transformer architecture~\cite{vaswani2017attention}, leverage self-attention mechanisms to handle sequential data efficiently, enabling them to capture complex linguistic patterns, such as syntax and semantics, across vast corpora of text.

% , PaLM~\cite{chowdhery2022palm}, and Bard

The training of these models is based on large-scale datasets, which allows them to generalize across diverse linguistic contexts, including different languages, genres, and registers. A noteworthy aspect of LLMs is their ability to exhibit zero-shot and few-shot learning, which empowers them to perform well on tasks they have not been explicitly trained on~\cite{li2024exploring,Zhao2023ASO,Wang2023ASO,10.1145/3686803}. Additionally, techniques like Reinforcement Learning from Human Feedback (RLHF)~\cite{instructGPT} enhance their capability to align with human ethical norms, improving both the quality and appropriateness of generated content. As a result, these models have been deployed in various real-world applications, ranging from content creation to decision-making in social contexts.

\subsection{Slang Detection and Identification}
The detection and identification of slang have long been significant challenges in NLP due to the constantly evolving nature of informal language. Early study relied on traditional rule-based approaches and static slang dictionaries to identify non-standard expressions in text~\cite{Wang_icceasia23}. These methods, while effective in detecting known slang, often struggled to keep pace with rapidly changing linguistic trends, especially in online communities where new slang emerges frequently.

More recent approaches have incorporated machine learning models, such as Naive Bayes and Support Vector Machines (SVMs), to detect informal language~\cite{10308036,9961254}. While these models offered more flexibility, they still faced limitations when confronted with novel or context-dependent slang terms. In response to this challenge, cognitive approaches to slang prediction have been developed, such as the work by~\cite{sun2019slang}, which explores the use of categorization models to predict the emergence of slang based on the selection of new vocabulary. This method emphasizes the role of cognitive processes in slang generation and demonstrates superior performance over random guessing.

Further refinement came with the introduction of frameworks like the Semantically Informed Slang Interpretation (SSI) model~\cite{sun2022semantically}, which leverages semantic and cognitive theories to better understand how slang evolves within specific contexts. This approach not only improves the interpretation of slang but also sheds light on the mechanisms underlying its evolution, providing a more dynamic view of language change in informal settings. However, these studies primarily focus on detecting and predicting existing slang, while the generation and adaptation of new slang remain relatively unexplored—a gap that this study seeks to address to some extent.

\subsection{LLMs in Evolutionary Game Theory and Social Simulation}
The intersection of LLMs with evolutionary game theory and social simulation has opened new avenues for studying complex interactions in controlled environments. Research has demonstrated that LLMs can simulate sophisticated strategies in negotiation-based games, as evidenced by~\cite{fu2023improving}, where models refine their bargaining strategies through iterative self-play. This iterative process mirrors the real-world adaptation of communication strategies, showing the potential of LLMs to autonomously improve their decision-making capabilities.

LLMs have also shown promise in social deduction games, such as Werewolf, where they analyze historical communication patterns to develop effective game strategies~\cite{xu2023exploring}. This study highlights the models’ ability to evolve their behaviors and responses based on the context and previous interactions. Additionally, combining LLMs with reinforcement learning, as discussed by~\cite{xu2023language}, has enabled the development of agents that make competitive decisions while maintaining linguistic coherence. Such advancements illustrate the growing role of LLMs in not only simulating language but also in evolving strategic behavior in complex scenarios.

Beyond game theory, LLMs have been applied to broader social simulations, including modeling historical and social dynamics. In~\cite{Park2023GenerativeAI}, LLM-driven agents were used to simulate interactions in a Wild West-style setting, demonstrating how these models can autonomously generate diverse behaviors without relying on real-world data. Similarly, the S3 framework~\cite{gao2023s3} simulates social media interactions by predicting user demographics and behaviors, providing a realistic model of social network dynamics. 
In \cite{tang2024gensim}, a general-purpose, error-correcting social simulation platform based on large model agents is proposed. This platform supports large-scale simulations involving up to 100k participants and has been tested in various scenarios, including labor market simulations and network user behavior simulations, with a discussion of its effectiveness.
Similarly, \cite{yang2024oasis} takes this further by introducing OASIS, a scalable and extensible social media simulator. OASIS extends the number of simulated users to the million level.
LLM-based simulations have also been used to reconstruct historical events, as seen in~\cite{hua2023war}, where multi-agent systems were employed to simulate military confrontations and decision-making processes in historical contexts.

These diverse applications highlight the versatility of LLMs in simulating social interactions. However, existing study has primarily focused on open or historically constrained scenarios, whereas we examine the trade-offs users make between expression needs and platform moderation in regulated environments. Understanding this dynamic is crucial for optimizing moderation strategies and balancing freedom of expression with compliance requirements. Our goal is to uncover the evolutionary mechanisms of language strategies in such environments and provide feasible simulation approaches.


%\paragraph{Differences from Prior Research}
%Our previous study \cite{DBLP:conf/cec/CaiLZLWT24} shares the same overall framework as \method{}; however, significant differences exist in the design of the Reflection Module. Specifically, the prior research did not incorporate the concept of genetic algorithms. Instead, both Constraint Strategies and Expression Strategies were directly managed by a LLM. During each reflection cycle, the LLM would generate several new strategies based on violation logs or dialogue history and replace the existing strategies entirely. Through experiments and further investigation, we identified a notable issue with this approach: when the LLM updates strategies by capturing new information from the context each time, it struggles to balance global information with local details. For example, when maintaining Constraint Strategies, the LLM might focus more on frequently occurring errors in the violation logs, thereby neglecting certain highly valuable individual records.

%In contrast, \method{} significantly enhances the strategy optimization process within the Reflection Module by introducing genetic algorithms. Specifically, genetic algorithms enable the LLM to generate new strategies targeting small batches of violation records during the Mutation phase, allowing it to concentrate more effectively on local information features. Concurrently, the Selection process, based on fitness evaluation, assesses each strategy's contribution to the overall dialogue performance, ensuring that the selection process accounts for global effectiveness. Through this approach, \method{} not only preserves strategies that perform well at the local level within the strategy pool but also maintains overall coordination and optimization of strategies. 

\section{Method}





%\section{When Physics-Informed Neural Networks Meet State Space Models}

%\subsection{Continuous-Discrete Mismatch}

%\subsection{Sampling Continuous Dynamics in Sequence}

%PDE to infinite-dimensional ODE

%\subsection{Modeling Sequence with State Space Models}

%infinite-dimensional ODE to SSMs

%\subsection{Linear Time Variant}





%\clearpage

\vspace{-1mm}

\section{Combating Failure Mode with State-Space Model and Sub-sequential Alignment}
\label{sec:ssmsub}


    To address the problems in Section~\ref{sec:fail}, we propose (1) a discrete state-space-based encoder that models the sequences of individual collection points in continuous dynamics, to match with \textit{Continuous-Discrete Mismatch}, and propagates the information from the initial condition to subsequent times (Section~\ref{sec:ssm}).  and (2) a sub-sequence contrastive alignment mechanism that aligns different outputs of the same collection point in different sub-sequences, to form an agreement that eliminates simplicity bias (Section~\ref{sec:subseq}).
    
\vspace{-2mm}


\subsection{Continuous Time Propagation of Initial Condition Information with State Space Model}
\vspace{-1mm}
\label{sec:ssm}
As we discussed in Section~\ref{sec:fail}, the \textit{Continuous-Discrete Mismatch} of PINNs raises the intrinsic difficulty of modeling, since the time dependency in a dynamic PDE system is not captured spontaneously by discrete sampling. 
    We argue that such a dynamic time dependency can be modeled by SSM. 
To this end, we first consider the PDE as a spatially infinite-dimensional ODE to simplify the problem. We view the solution $u_\theta(x,t)$ in a function space that, if we let:
\begin{equation}
    U(t) := u_\theta(\cdot,t),
\end{equation}
be a function $x \to u_\theta(x,t)$, by $M$-point spatial sampling:
\begin{equation}
    U_i(t) := u(x_i,t),
\end{equation}
\begin{equation}
    \mathbf {u}(t) = \left[U_1(t),U_2(t),\cdots,U_M(t) \right]^\top .
    \label{}
\end{equation}

\textbf{Sequential Modeling Continuity with SSM.}
In continuous time, we now model the function $\mathbf {u}(t)$ to the dynamic system described by SSM as in Eq.~\ref{equ:hiddenssm} and~\ref{equ:outputssm}. Here we let $\mathbf x(t) = \text{Embed}(x,t)$, where $\text{Embed}(\cdot)$ is the Spatio-Temporal Embedding in Fig~\ref{fig:main}. After temporal discretization $\mathbf u_k=\mathbf u(k\Delta t),\mathbf h_k=\mathbf h(k\Delta t)$ and $\mathbf x_k=\mathbf x(k\Delta t)$, we get: 
\begin{equation}
    \mathbf u_k=C\bar A^k \mathbf h_0 + C\sum_{i=0}^k\bar A^{k-i}\bar B\mathbf x_i.
    \label{equ:ssmu}
\end{equation}
Reversibly, by the inverse of the discretization rule defined by Eq.~\ref{equ:disc1},~\ref{equ:disc2}, we can restore this temporal dependency to continuous time. This kind of restoration can help achieve PINN's generalization to any moment in $[0, T]$. 

\textbf{Pattern Propagation by Joint Optimization.}
Combine Eq.~\ref{equ:lossequ} with~\ref{equ:ssmu}, in a sequence start with $t=0$, the sum of loss of collection points at time $k\Delta t$, would be: 
\begin{align}
  &\sum_{i=1}^M \mathcal L_\mathcal F(u(x_i,k\Delta t)) =  \frac{1}{M}\mathcal \|\mathcal F( \mathbf 1_M\cdot \mathbf{u}_k)\Arrowvert^2\nonumber\\&=\frac{1}{M}\|\mathcal F\left(\mathbf1_M\cdot(C\bar A^k \mathbf h_0 + C\sum_{i=0}^k\bar A^{k-i}\bar B\mathbf x_i)\right)\Arrowvert^2,
  \label{equ:timeloss}
\end{align}
where $1_M=[1,1,\cdots,1] \in \mathbb R^M$. In Eq.~\ref{equ:timeloss}, we notice that the $\mathbf h_0$ should satisfy both the initial condition and the equation by jointly optimizing the losses:
\begin{align}\label{equ:loss0equ}
   \mathcal L_\mathcal F(\mathbf u_0) =  \frac{1}{M}\|\mathcal F\left(\mathbf1_M\cdot(C \mathbf h_0 )\right)\Arrowvert^2;\\
   \mathcal L_\mathcal I(\mathbf u_0) =  \frac{1}{M}\|\mathcal I\left(\mathbf1_M\cdot(C \mathbf h_0 )\right)\Arrowvert^2 .
   \label{equ:loss0init}
\end{align}
Thereby, for each collection point, the numerical value of its solution should be jointly optimized by Eq.~\ref{equ:timeloss},~\ref{equ:loss0equ}, and~\ref{equ:loss0init}, thus receiving the pattern defined by the initial conditions.


\textbf{Uniformed Derivatives Scale.} Another benefit that can be got from SSM is, by parameterizing differential state matrix $A$ in Eq.\ref{equ:hiddenssm} with HiPPO matrix~\cite{gu2020hippo} which contains the derivative information,  we can align the derivatives of the system with respect to time on a uniform scale. This uniform scale will help to reduce the problem of ruggedness on the loss landscape due to gradient vanishing or exploding.

\textbf{Time-Varying SSM.} In practice, we use the time-varying Selective SSM~\cite{gu2023mamba}, instead of the function defined by Eq.~\ref{equ:ssmu} being the SSM on a linear time-invariant system. The time-varying SSM has two advantages, one is that such input-dependent models typically have stronger representational capabilities~\cite{xu2024infinite}, while the other is that it will make diverse predictions that help to eliminate simplicity bias in the model, as we will discuss in section~\ref{sec:subseq}. This time-variance will make $(\bar A,\bar B, C)$ time-dependent, and therefore, Eq.~\ref{equ:ssmu} and \ref{equ:timeloss} need minor adjustments. These adjustments won't impact the initial condition propagation, and we will discuss them in Appendix~\ref{apx:LTI}.





\subsection{Eliminating Simplicity Bias of Models with Sub-Sequence Contrastive Alignment}
\label{sec:subseq}


\begin{figure}[t!]
    \centering
    \includegraphics[width=\linewidth]{_fig/fig4}
    \vspace{-5mm}
    \caption{Comparison of Sequence Granularity}
    \label{fig4}
    \vspace{-6mm}
  %  \vspace{-1mm}
\end{figure}





Although SSM can make the information about the initial conditions propagate in time coordinates, it still cannot mitigate the simplicity bias of neural networks. 
    The model is still prone to falling into an over-smoothed local optimum. 
        There are two key points to address this over-smoothness caused by simplicity bias: (1) appropriate sequence granularity to guarantee a smooth optimization process. (2) Mitigating the effect of simplicity bias through the diversity of model prediction paradigms~\cite{pagliardiniagree}. 
        
        \textbf{Sequence Granularity.} A proper sequence granularity ensures smooth propagation of the initial conditions while making the model easier to optimize. As shown in Fig.~\ref{fig4}, there are three ways to define sequence, which are pseudo sequence~\cite{zhao2024pinnsformer}, long sequence~\cite{nguyen2024sequence}, and the proposed sub-sequence.
        We propose to use a sub-sequence with medium granularity overlapping. The sub-sequential modeling can avoid: (1) the difficulty of crossing the loss barrier that makes the model trapping in the over-smooth local optimum, which is caused by the huge inertia of long sequence; (2) the difficulty of broadcasting information globally on the time coordinate, that caused by construct on small neighborhoods of a collection point in pseudo sequence. Sub-sequence takes only the first output in the sequence as the output value of the current collection point. Its successors' values will pass information crossing the time coordinate through subsequences alignment and form diverse predictions to eliminate simplicity bias.

\textbf{Contrastive Alignment for Information Propagation.} As shown in Fig.~\ref{fig4}, we construct a sub-sequence for each collection point together with its finite successors, which form overlapping collection points. By aligning the predictions of these collection points with a contrastive loss, each collection point becomes a soft relay of the pattern. Thus, it forms the propagation of patterns in the whole time domain.%$[0,T]$.
%In this way, we can realize a filtered propagation of pattern across sub-sequences by recursive alignment, 


\textbf{Eliminating the Simplicity Bias.} Previous work~\cite{teney2022evading,pagliardiniagree} has pointed out that the agreement obtained from diverse predictions is the key to eliminating the effects of simplicity bias. We argue that this agreement from diverse predictions is naturally obtained in the sub-sequence alignment. This is because the fact that,
    since the SSM we constructed in section~\ref{sec:ssm} is time-varying and a collection point will be at different time coordinates in different sub-sequences, the predictions for this collection point are naturally diverse. And we force these diverse predictions to arrive at a consensus by contrastive alignment.


      %  If we model a PDE with an overly long sequence, when trapped in an oversmoothed local optimum, decreasing the loss of a point at one moment may increase the loss of all other points on the sequence, which causes the model to have a huge inertia when optimizing.
%The large inertia caused by long sequences makes it difficult for the model to cross the loss barrier. At the other extreme, a sequence that is too small may make it difficult to propagate the time dependency. The pseudo sequence can only reflects temporal information on small neighborhoods of a single collection point, instead of broadcasting the information globally on time coordinate.


\section{PINNMamba}

In conjunction with the high-level ideas described in Section~\ref{sec:ssmsub}, in this section, we present PINNMamba, a novel physics-informed learning framework that effectively combats the failure modes in the PINNs.


\begin{figure*}[t!]
    \centering
    \includegraphics[width=\textwidth]{_fig/conv}
    \vspace{-8mm}
    \caption{The ground truth solution, prediction (top), and absolute error (bottom) on convection equations.}
    \label{fig:conv}
    \vspace{-5mm}
  %  \vspace{-1mm}
\end{figure*}



\textbf{Sub-Sequential I/O.} As shown in Fig.~\ref{fig:main}, PINNMamba first samples the grid of collection points over the entire spatio-temporal domain bounded by the PDE. We assume that the grid picks $M$ spatial coordinates and $N$ temporal coordinates, and denote the temporal sampling interval as $\Delta t = T/(N-1)$. For a collection point $(x,t)$, we construct a sequence $X(x,t)$ with its $k-1$ temporal successors:  
\begin{equation}
    X(x,t) = \{(x,t),(x,t+\Delta t),\cdots,(x,t+(k-1)\Delta t)\}.
\end{equation}
PINNMamba takes such $M\times N$ sequences as the input of models. 
    For each sequence $X(x,t)$, PINNMamba computes a sub-sequence prediction $\{\bar u_\theta^t (x,t),\bar u_\theta^t (x,t+\Delta t),\cdots,\bar u_\theta^t (x,t+(k-1)\Delta t)\}$ corresponding to every collection point in the sequence, where $\bar u_\theta^t (x,t+i\Delta t)$ denote the tentative prediction of collection point $(x,t+i\Delta t)$ in a sequence start with time $t$. The $\bar u_\theta^t (x,t)$ will be taken as the output of collection point $(x,t)$ and the rest of the sequence will be used to construct the sub-sequence contrastive alignment loss we will discuss later in Section~\ref{sec:subseq}. The residual losses of the model w.r.t the sub-sequence will be:
    
\vspace{-5mm}

    \small{
    \begin{equation}
    \mathcal L_{\mathcal F}^\text{seq}(u_\theta)= \frac{1}{k|\chi|}\sum_{(x_i,t_i)\in \chi}\sum_{j=0}^{k-1}\|\mathcal F(u_\theta^{t_i}(x_i,t_i+j\Delta t)\|^2;
        %\mathcal L_{\mathcal F}(u_\theta)= \frac{1}{k|\chi|}\sum_{i=1}^{|\chi|}\sum_{j=0}^{k-1}\|\mathcal F(u_\theta^{t_i}(x_i,t_i+k\Delta t)\|^2;
    \label{equ:lossequseq}
\end{equation}
\begin{equation}
    \mathcal L_{\mathcal I}^\text{seq}(u_\theta)= \frac{1}{k|\chi_0|}\sum_{(x_i,t_i)\in \chi_0}\sum_{j=0}^{k-1}\|\mathcal I(u_\theta^{t_i}(x_i,t_i+j\Delta t)\|^2;
    \label{equ:lossinitseq}
\end{equation}
\begin{equation}
    \mathcal L_{\mathcal B}^\text{seq}(u_\theta)= \frac{1}{k|\partial\chi|}\sum_{(x_i,t_i)\in \partial\chi}\sum_{j=0}^{k-1}\|\mathcal B(u_\theta^{t_i}(x_i,t_i+j\Delta t)\|^2.
    \label{equ:lossboundseq}
\end{equation}
}
%As shown in Fig.~\ref{fig:main}, PINNMamba 
\normalsize
\vspace{-5mm}
%\textbf{Sub-Sequence of Collection Points.}

\textbf{Model Architecture.} As shown in Fig.~\ref{fig:main}, PINNMamba employs an encoder-only architecture, which encodes fixed-size input sub-sequence into a sub-sequence prediction with the same length. First, for each token in the sequence, an MLP-based Spatio-Temporal Embedding layer first embeds the $(x,t)$ coordinates into high-dimensional representation. The embeddings will be sent to a Mamba-based encoder, which consists of several PINNMamba blocks. 

The PINNMamba block employed here consists of two branches: (1) the first is a stack of a linear projection layer, a 1d-convolution layer, a Wavelet activation~\cite{zhao2024pinnsformer}, and an SSM layer with parallel scan~\cite{gu2023mamba}; (2) the second is a stake of a linear projection layer and a Wavelet activation. The two branches are then connected with an element-wise multiplication, followed by another linear projection and residual connection. With input $X^l$,
the PINNMamba block can be formulated as: 
\begin{equation}
    X_1^l = \text{SSM}(\sigma(\text{Conv}(W_aX^l)));
\end{equation}
\vspace{-5mm}
\begin{equation}
    X_2^l = \sigma(W_bX^l);
\end{equation}
\vspace{-5mm}
\begin{equation}
    X^{l+1} = X^l+W_c(X_1^l\otimes X_2^l),
\end{equation}
where $\sigma(x)=\omega_1\sin(x)+\omega_2\cos(x)$ is Wavelet activation function~\cite{zhao2024pinnsformer}, in which $\omega_1,\omega_2$ are learnable. $\otimes$ denotes an element-wise multiplication. 

\textbf{Sub-Sequence Contrastive Alignment.} PINNMamba predicts the same collection multiple times in different subsequences. For example, the collection point $(x,k+\Delta t)$ appears on sequences from $X(x,t+\Delta t)$ to $X(x,t+k\Delta t)$. We align the predictions on these subsequences to make the information defined by the initial conditions propagate over time. To do this, for each subsequence, we design a contrastive loss with the last subsequence for alignment:
\begin{align}
        \mathcal L_\text{alig}(u_\theta)= \frac{1}{(k-1)|\chi|}&\sum_{(x_i,t_i)\in \chi} \sum_{j=1}^{k-1} \Big[u_\theta^{t_i}(x_i,t_i+j\Delta t)\nonumber\\&-u_\theta^{t_i+\Delta t}(x_i,t_i+j\Delta t)\Big]^2.
\end{align}
\normalsize

Thus, the empirical loss for PINNMamba is defined as:
\small
\begin{equation}
     \mathcal L(u_\theta)=\lambda_{\mathcal F}\mathcal L_{\mathcal F}^\text{seq}(u_\theta)+\lambda_{\mathcal I}\mathcal L_{\mathcal I}^\text{seq}(u_\theta)+\lambda_{\mathcal B}\mathcal L_{\mathcal B}^\text{seq}(u_\theta)+\lambda_\text{alig}\mathcal L_\text{alig}(u_\theta).
\end{equation}
\vspace{-8mm}
    

\normalsize

\section{Empirical Study}

\begin{figure*}[t] 
    \centering
        \includegraphics[width=0.9\textwidth, height=5cm]{model_performance_single.pdf}
    % \captionsetup{font={small}} 
    \caption{The performance of different LLMs in the single-round inference. The left and right figures show the results with the number of constraints $n$ set to 7 and 9, respectively. With the increase of the CDDI, the constraint order changes from ``easy-to-hard'' to ``hard-to-easy''.}
    \label{fig:7cons}
\end{figure*}

\begin{figure*}[t] 
    \centering
        \includegraphics[width=0.9\textwidth,height=5cm]{model_performance_multi.pdf}
    % \captionsetup{font={small}} 
    \caption{The performance of different LLMs in the multi-round inference. The left and right figures show the results with the number of constraints $n$ set to 7 and 9, respectively. With the increase of the CDDI, the constraint order changes from ``easy-to-hard'' to ``hard-to-easy''.}
    \label{fig:multi_7cons}
\end{figure*}

\begin{table*}[t]
\newcolumntype{g}{>{\columncolor{green!4}}r}
%\setlength\tabcolsep{7pt}
\newcolumntype{b}{>{\columncolor{blue!4}}r}
% \setlength\tabcolsep{8pt}
\renewcommand{\arraystretch}{0.9} % 增加行间距到1.5倍
% 在表格环境之前设置全局字体为罗马字体
\renewcommand{\familydefault}{\rmdefault}
\resizebox{\textwidth}{!}{
\begin{tabular}{cccccccccgb}
\toprule
\textbf{CDDI} & \textbf{Length} & \textbf{Language} & \textbf{Punctuation} & \textbf{Format} & \textbf{Keywords} & \textbf{ChangeCase} & \textbf{Startend} & \textbf{Content} & \textbf{C\_level} & \textbf{I\_level} \\ \midrule
\rowcolor[gray]{0.95}\multicolumn{11}{c}{\textit{Single-round Inference}} \\ 
\textbf{-1}  & 27.50 & 28.20  & 23.30 & 71.14 & 68.58 & 49.57 & \underline{62.92} & \textbf{81.22} & 53.30 & 1.95  \\ 
\textbf{-0.8} & 28.23 & 30.70  & 23.90 & 73.64 & 68.46 & 49.00 & \textbf{63.50} & 77.78 & 53.60 & 1.80  \\ 
\textbf{-0.6} & 28.53 & 31.10  & 26.60 & 71.23 & 69.25 & 49.79 & 60.92 & \underline{78.22} & 53.56 & 1.95 \\ 
\textbf{-0.4} & 28.53 & 36.10  & 30.70 & 72.41 & 71.58 & 51.64 & 62.33 & 77.83 & 55.05 & 2.10  \\ 
\textbf{-0.2} & 29.33 & 39.30  & 35.30 & 73.82 & 72.08 & 50.07 & 60.75 & 77.44 & 55.74 & 2.40  \\ 
\textbf{-0.05} & \textbf{30.27} & 42.90 & 36.80 & 74.95 & 73.46 & 52.14 & 60.50 & 77.50 & 56.91 & 2.90  \\ \cdashlinelr{1-11} 
\textbf{0.05}  & 29.17 & 46.70  & 38.00 & 72.68 & 74.75 & 50.79 & 61.50 & 75.33 & 56.57 & 2.75  \\ 
\textbf{0.2}   & 28.17 & 50.50  & 43.30 & 76.05 & 75.92 & 52.07 & 61.42 & 72.94 & 57.55 & 2.75  \\ 
\textbf{0.4}   & 30.23 & 54.50  & 46.40 & 76.64 & 76.29 & 54.07 & 62.83 & 74.17 & 59.14 & 2.65  \\ 
\textbf{0.6}   & 29.83 & 59.20  & 49.70 & \textbf{79.09} & \underline{77.42} & 56.71 & 58.58 & 74.33 & 60.12 & 3.00  \\ 
\textbf{0.8}   & 29.40 & \underline{60.50}  & \underline{51.70} & 77.91 & \textbf{77.63} & \underline{58.07} & 58.25 & 73.89 & \underline{60.16} & \underline{3.05}  \\ 
\textbf{1}     & \underline{30.03} & \textbf{67.10} & \textbf{53.10} & \underline{78.00} & 77.21 & \textbf{59.21} & 57.42 & 74.61 & \textbf{60.95} & \textbf{3.50}  \\
\rowcolor[gray]{0.95}\multicolumn{11}{c}{\textit{Multi-round Inference}} \\ 
\textbf{-1}  & \textbf{62.60} & 64.00 & 54.20 & 21.59 & 57.79 & 62.50 & 10.83 & 16.61 & 44.47 & 0.75  \\ 
\textbf{-0.8} & \underline{59.63} & 64.90 & 61.50 & 22.27 & 62.46 & 61.93 & 13.17 & 17.39 & 45.57 & 0.75  \\ 
\textbf{-0.6} & 54.65 & 67.87 & 65.67 & 25.74 & 67.83 & 59.47 & 22.75 & 20.08 & 47.40 & 0.65  \\ 
\textbf{-0.4} & 52.77 & 68.74 & 64.46 & 32.30 & 69.57 & 61.44 & 30.78 & 26.21 & 49.98 & 1.05  \\ 
\textbf{-0.2} & 48.73 & 68.74 & 62.42 & 38.67 & 74.67 & 59.16 & 39.00 & 32.02 & 52.07 & 1.25  \\ 
\textbf{-0.05} & 46.48 & 69.97 & 67.17 & 46.38 & 76.04 & 60.97 & 48.79 & 46.58 & 56.35 & 1.40  \\ \cdashlinelr{1-11} 
\textbf{0.05}  & 45.32 & 70.08 & 68.84 & 51.19 & 76.62 & 62.18 & 52.68 & 50.47 & 58.04 & 1.80  \\ 
\textbf{0.2}   & 44.81 & 69.91 & 66.73 & 58.35 & 80.20 & 60.34 & 62.41 & 63.22 & 61.79 & 3.41  \\ 
\textbf{0.4}   & 44.30 & \textbf{72.50} & \underline{69.10} & 64.50 & 81.75 & \underline{63.50} & 68.00 & 73.56 & 65.39 & \underline{5.60}  \\ 
\textbf{0.6}   & 43.71 & 68.87 & 68.20 & 71.71 & 83.87 & 59.83 & 71.98 & 81.77 & 67.47 & 5.05  \\ 
\textbf{0.8}   & 44.35 & 68.37 & 68.00 & \underline{75.94} & \underline{84.49} & 61.54 & \underline{70.31} & \underline{84.88} & \underline{68.76} & \textbf{6.00}  \\ 
\textbf{1}     & 44.07 & \underline{70.90} & \textbf{69.60} & \textbf{81.41} & \textbf{85.08} & \textbf{64.43} & \textbf{72.58} & \textbf{87.22} & \textbf{70.74} & 4.00  \\
\bottomrule
\end{tabular}
}
  \caption{
  The overall performance of LLaMA3-8B-Instruct on multi-constraint instructions with different CDDI values. From left to right, we sort the constraint types from the hardest to the easiest.
  }
  \label{tab:main}
\end{table*}


\subsection{Experiment Setup}
\paragraph*{Models} For our probing task, to ensure the generalizability of our study, we conduct experiments on both closed and open-source LLMs with varying architectures and parameter sizes. Specifically, we introduce the following models: (1) LLaMA3-8B-Instruct and LLaMA3-70B-Instruct~\cite{dubey2024llama}. (2) LLaMA2-13B-Chat~\cite{touvron2023llama}. (3) Mistral-7B-Instruct~\cite{jiang2023mistral}.\footnote{We use the latest v0.3 version.} (4) Qwen2.5-7B-Instruct~\cite{yang2024qwen2}. (5) GPT4o-mini~\cite{achiam2023gpt}.

% \footnote{We use the default version, i.e., the gpt-4o-mini-2024-07-18}

\paragraph*{Datasets} We construct various multi-constraint instructions with different constraint orders (Sec.\ref{method}). We empirically set the number of constraints ${n}$ to 7. To ensure the diversity and complexity, we set the number of constraint combinations $n_{cc}$ to 10 and the number of difficulty distributions $n_{dd}$ to 12, finally obtaining $200\times10\times12=24\text{K}$ samples. To verify the influence of constraint number, we also conduct experiments on the setting when ${n=9}$. The statistic of the data for the probing task is provided in Fig.~\ref{fig:statistic}.







\subsection{Results}
\paragraph*{LLMs prefer to “hard-to-easy” constraint distribution.} As shown in Fig.~\ref{fig:7cons}, most of the LLMs exhibit a dramatic performance fluctuation on instructions with varying constraint distributions. When the constraint number is set to 7, the LLaMA3-8B-Instruct and Qwen2.5-7B-Instruct show approximately 7$\%$ and 5$\%$ performance disparity in extreme situations. This indicates the vulnerability of existing LLMs to the position bias brought by the constraint order. Also, the LLMs tend to be more performant to instructions with higher CDDI values. Even the LLaMA3-70B-Instruct exposes a clear preference for higher CDDI value as the number of constraints increases to 9, demonstrating that ``hard-to-easy'' is a superior constraint distribution for existing LLMs.

\paragraph*{Multi-round inference exhibits more severe position bias compared with the single-round inference.} The LLMs' performance in multi-round inference is presented in the Fig.~\ref{fig:multi_7cons}. Compared with the results in the single-round inference, the performance gap becomes more prominent. All the LLMs gain approximately 10$\%$ improvement on C$\_$level accuracy. Surprisingly, the  LLaMA3-8B-Instruct and LLaMA3-70B-Instruct achieve approximately 25$\%$ performance improvement by changing the constraint distribution from ``easy-to-hard'' (CDDI=-1) to ``hard-to-easy'' (CDDI=1). This indicates that the LLMs are more sensitive to the position bias problem in a multi-round scenario.

\paragraph*{LLMs perform better in multi-round inference when provided with the instructions in appropriate constraint order} Comparing the results in single-round (Fig.~\ref{fig:7cons}) and multi-round inference (Fig.~\ref{fig:multi_7cons}), we observe that the LLMs reach better performance if the incorporated constraints are arranged in an appropriate order. Specifically, when the CDDI value is negative, the performance of LLMs in multi-round inference lags behind that in single-round inference. Nevertheless, with the increase of the CDDI value, the LLMs can achieve superior performance in multi-round inference and reach their best performance in CDDI=1. An exception is the Mistral-7B-Instruct-v0.3. We attribute this to its inferiority in processing multi-round information~\cite{chen2024sifo}.

\paragraph*{Position bias varies in different types of constraints.} We present the performance of the LLaMA3-8B-Instruct across different types of constraints in Tab.~\ref{tab:main}. As observed, with the increase of the CDDI value, the model's performance across most constraint types shows an upward trend except for Startend and Content, indicating that not all the constraints can benefit from the ``hard-to-easy'' constraint distribution in single-round inference. We make a more comprehensive explanation study in Sec.~\ref{sec:experiment2} for further investigation. Regarding the multi-round inference, the model's performance only exhibits a drop tendency in the Length type as the CDDI value increases, indicating that the LLMs struggle to generate a length-controlled final response when the length constraint is applied early in the multi-round inference~\cite{yuan2024following}.








% 放附录 稳定性实验 选一个组合(tau=-0.05)跑3次证明表现差不多

\begin{tikzpicture}
    \begin{axis}[
        width=\linewidth,
        ylabel style={font=\scriptsize,yshift=-0.6em},
        y tick label style={font=\scriptsize},
        x tick label style={font=\scriptsize},
        ybar,
        %axis lines=left,  
        ymajorgrids,
        symbolic x coords={XGBoost, gMLP, PedCA-FT},
        %xtick={XGBoost, LightGBM, {ours}},
        ylabel={Sensitivity},
        ymin=0,
        ymax=55,
        bar shift=0pt,
        %bar width=0.5cm,
        nodes near coords, 
        nodes near coords style={font=\scriptsize}, 
        %enlargelimits=0.10,
    ]
        \addplot[
            fill=Set2-A,
            ybar,
            error bars/.cd,
            y dir=both,
            y explicit,
        ] coordinates {
            (XGBoost, 39.04) += (0, 8.1) -= (0, 7.53)
        };
        \addplot[
            fill=Set2-B,
            ybar,
            error bars/.cd,
            y dir=both,
            y explicit,
        ] coordinates {
            (gMLP, 8.22) += (0, 5.6) -= (0, 3.46)
        };
        \addplot[
            fill=Set2-C,
            ybar,
            error bars/.cd,
            y dir=both,
            y explicit,
        ] coordinates {
            (PedCA-FT, 42.47) += (0, 8.11) -= (0, 4.73)
        };
    \end{axis}
\end{tikzpicture}

\subsection{Robustness of CDDI}
Since the CDDI is calculated by comparing the concordant and discordant pairs of two different constraint orders, there are usually multiple constraint orders sharing the same CDDI value. Therefore, we conduct a testing experiment to assess whether the LLM exhibits significant fluctuations across different constraint orders with the same CDDI value. Specifically, we set the CDDI to -0.05, a value that includes the most constraint orders in our setting, and conduct single-round inference for 3 times. The experiment results are shown in Tab~\ref{tab:sensitivity}. We calculate the P-value of the data, finding that the P-value is much larger than 0.05. This indicates that the fluctuation of LLM's performance is negligible among different constraint orders in the same CDDI value.




\section{Explanation Study}

\begin{figure*}[t] 
    \centering
        \includegraphics[width=1\textwidth]{position_scores.pdf}
    % \captionsetup{font={small}} 
    \caption{(a) The importance weights assigned by the LLM when handling constraints in different positions. (b) The total importance weights which designated to the constraint part in the multi-constraint instructions among three different constraint distributions.}
    \label{fig:position_score}
\end{figure*}


\begin{figure}[t] 
    \centering
        \includegraphics[width=0.48\textwidth]{types_scores.pdf}
    % \captionsetup{font={small}} 
    \caption{The importance weights across different types of constraint in three different constraint distributions.}
    \label{fig:type_score}
\end{figure}







\subsection{Explanation Metric}
To make an explanation for the influence brought by the constraints of different orders, we make an explanation study on where the LLMs mainly focus when handling multi-constraint instructions via a feature attribution-based explanation method~\cite{li2016visualizing, wu2020perturbed}. Specifically, we leverage the importance of the input tokens to measure the LLMs' attention to them. To obtain the importance of a specific instruction token $t_x$ to a response token $t_y$, we calculate the confidence change after the removal of the $t_x$, as formulated below:
\begin{equation}
    \label{eq6}
    I_{t_x,t_y}=p(t_y|Z_y)-p(t_y|Z_{y,/t_x}),
\end{equation}
where $p(\cdot|\cdot)$ is the conditional probability produced by the LLM $f$, $Z_y$ is the tokens before the $t_y$ and $Z_{y,/t_x}$ is the tokens of $Z_y$ after removing the token $t_x$. To reduce the computation, we approximate the $I_{t_x,t_y}$ with the first-order gradient $\frac{\partial f\left(t_y \mid Z_y\right)}{\partial \mathbf{E}\left[t_x\right]}$ ~\cite{wu2023language}, where $\mathbf{E}\left[t_x\right]$ is the token embedding of $t_x$. We normalize the importance $I_{t_x,t_y}$ and obtain the standard importance $S_{t_x,t_y}$ with the formula:
\begin{equation}
    \label{eq7}
    S_{t_x,t_y}= \frac{L\times I_{t_x,t_y}}{{\max_{i=1}^{N_{X}}}I_{t_i,t_y}},
\end{equation}
where $N_X$ is the number of instruction tokens and $L$ is a hyper-parameter which helps to filter the noise brought by the first-order approximation. To visualize the LLMs' attention to different constraints, we calculate the importance weight of a specific constraint $C_x$ to the final response $Y$ with the formula:
\begin{equation}
    \label{eq8}
    S_{C_x,Y}=\frac{1}{N_Y}\sum_{t_y\in Y}\sum_{t_x\in C_x}S_{t_x,t_y},
\end{equation}
where $N_Y$ is the number of response tokens.







\subsection{Experiment Set-up}
We conduct our explanation study on the LLaMA3-8B-Instruct model. We set the hyper-parameter $L$ to 10 in Eq.(\ref{eq7}) and select three most typical difficulty distributions: hard-to-easy (indicated by CDDI=1), easy-to-hard (indicated by CDDI=-1) and random (indicated by CDDI=-0.05) to conduct our experiments. We randomly sample 200 instances from the corresponding data which fall in the required CDDI value in the probing task to serve as the dataset.







\subsection{Results}
\paragraph*{Hard-to-easy constraint order induces the LLM to pay more attention to the constraint part in the multi-constraint instructions.} We visualize the importance weights of the model on the constraints in different positions. As shown in Fig.~\ref{fig:position_score} (a), in the multi-constraint instruction following, the model's attention on different positions varies with changes in the constraint orders. Specifically, when the constraints are randomly distributed across different positions (represented by CDDI=-0.05), the model assigns similar attention to all positions. As the constraint order becomes more structured (represented by CDDI=-1 and CDDI=1), the model's attention neither exhibits the “lost in the middle” phenomenon observed in long-context processing~\cite{liu2024lost}, nor a simply sequential distribution, but follows an iterative, laddered order. Then, in Fig.~\ref{fig:position_score} (b), we present the total importance weight the model assigns to the constraint part. We observe that the “hard-to-easy” constraint order attracts the most attention from the model towards the constraint part, which provides an explanation for the superiority of this constraint order.

\paragraph*{The LLM's performance on various constraints is strongly correlated with its attention patterns.} The importance weights of the model on different types of constraints are presented in Fig.~\ref{fig:type_score}. Among the three distinct difficulty distributions, the “hard-to-easy” (represented by CDDI = 1) assigns the highest importance weights to various types of constraints except for the Content and Startend. It is worth noting that this is exactly in accord with quantitative results in Tab.~\ref{tab:main}, i.e., as the CDDI value increases, the model's performance on the Content and Startend constraints shows a decreasing trend instead. Overall, the results show that the model's accuracy in following a specific type of constraint is strongly correlated with the attention assigned to it by the model. \label{sec:experiment2}

\section{Conclusion}

\section{Limitations and Future Work}
The proposed OpenFly platform incorporates various rendering engines/techniques to provide high-quality scenes. Specifically, this is the first attempt to use 3D GS reconstructed scenes to support real-to-sim training and testing, while in the reconstruction of large-scale areas, a few visual artifacts are inevitably present. Future work will focus on exploring more effective reconstruction methods to enhance realism in large-scale scenes. Besides, the proposed OpenFly-Agent is built upon the large VLN model architecture, which is not practical for real-time deployment on UAVs. To address this, future research should focus on developing more efficient architectures and effective quantization techniques. 


\section{Conclusion}
In this work, we present OpenFly, a platform designed for large-scale data collection in aerial Vision-and-Language Navigation (VLN). OpenFly integrates multiple rendering engines and advanced real-to-sim techniques for data generation, enabling efficient collection of diverse, high-quality aerial VLN data. The resulting large-scale dataset comprises 100k trajectories across 18 distinct scenes, spanning a wide range of altitudes and difficulty levels, which is significantly superior than existing ones. Furthermore, we propose OpenFly-Agent, a keyframe-aware aerial navigation model capable of directly predicting flight actions based on observations and language instructions. Extensive experiments validate the effectiveness of the proposed method, and establishing a comprehensive benchmark for future advancements in aerial navigation. 
%The toolchain, dataset, and code will be publicly released, providing a valuable resource for future research in this field.

\section{Limitations}

Our work mainly focuses on the position bias problem in the multi-constraint instruction following. We make a quantitative analysis of the influence brought by different constraint orders in the instructions. However, there are still some limitations. The constraints in our work are usually parallel to each other, which means the order change will not affect the semantic meaning of the instructions. The position bias problem for for those sequential constraints need to be further explored. Moreover, we only investigate the phenomenon of position bias in existing LLM without offering a solution. In further work, we will conduct a further probing task in sequential constraints to improve the generalization of our findings.

% \section*{Acknowledgments}

% This document has been adapted
% by St

% Bibliography entries for the entire Anthology, followed by custom entries
%\bibliography{anthology,custom}
% Custom bibliography entries only
\bibliography{acl_latex}

\appendix

\section{Appendix}

\newpage
\appendix
\section{Appendix}
\subsection{Metric Optimization}  \label{app:pg}
We utilize the REINFORCE algorithm to optimize the performance metric. The detailed optimization process is proved in the following equations:
    \begin{equation}
        \small
        \begin{aligned}
            &\nabla_{\Lambda}\hat{l}(\Lambda)
            =\nabla_{\Lambda} \mathbb{E}_{s\sim \pi{(\mathcal{B},\cdot;{\Lambda})}} \mathcal{R}(\hat{\mathcal{D}}, f(\Theta^*(\Lambda)))\\
            &=\nabla_{\Lambda}\sum_{s\in[0,1]^{|\mathcal{B}|}} \mathcal{R}(\hat{\mathcal{D}}, f(\Theta^*(\Lambda))) \cdot \pi(\mathcal{B},s;{\Lambda}) \\
            &=\sum_{s\in[0,1]^{|\mathcal{B}|}} \mathcal{R}(\hat{\mathcal{D}}, f(\Theta^*(\Lambda))) \cdot 
            \frac{\nabla_{\Lambda}\pi(\mathcal{B},s;{\Lambda})}{\pi(\mathcal{B},s;{\Lambda})}\cdot \pi(\mathcal{B},s;{\Lambda})\\
            &= \sum_{s\in[0,1]^{|\mathcal{B}|}} \mathcal{R}(\hat{\mathcal{D}}, f(\Theta^*(\Lambda))) \cdot \nabla_{\Lambda}log(\pi(\mathcal{B},s;{\Lambda}))\cdot \pi(\mathcal{B},s;{\Lambda})\\
            &=\mathbb{E}_{s\sim \pi(\mathcal{B},\cdot;{\Lambda})}[\mathcal{R}(\hat{\mathcal{D}}, f(\Theta^*(\Lambda)))\cdot \nabla_{\Lambda}log(\pi(\mathcal{B},s;{\Lambda}))],
        \end{aligned}
    \end{equation}

\subsection{Learning Algorithm}  \label{app:learning_algorithm}
The detailed optimization steps of the proposed framework are given in Algorithm \ref{al:method}.

\subsection{Detail of Studied Methods} \label{app:studied method}
To show the compatibility of our method, we apply the DVR framework on four recommendation backbones, i.e., BRPMF~\cite{koren2009matrix}, NeuMF~\cite{he2017neural}, MGCF~\cite{wang2019neural}, and LightGCN~\cite{he2020lightgcn}. We select BPRMF due to its widespread adoption in recommendation systems and proven effectiveness in practical applications. NeuMF, an MLP-based approach, extends the capabilities of BPRMF by capturing intricate user-item relationships. We leverage GNN-based models, such as MGCF and LightGCN, known for their state-of-the-art performance and competitive outcomes across various techniques, to serve as the recommendation backbone. 

Based on these backbones, different versions of the DVR model are tailored to optimize diverse metrics. For simplicity, we designate models optimized for ranking accuracy as DVR-Loss, DVR-Recall, and DVR-NDCG. Likewise, models focused on diversity and fairness metrics are labeled as DVR-CC, DVR-ILD, and DVR-Gini. 


We compare our framework with various data valuation methods for recommendations. BPR~\cite{10.5555/1795114.1795167} uniformly samples negative items and treats all training data equally in constructing the training objective. TCE-BPR and RCE-BPR are extensions of the TCE and RCE techniques \cite{10.1145/3437963.3441800}, aimed at dynamically filtering out noisy positive interactions during training based on loss values. In our implementation, we replace the original point-wise loss with a pair-wise ranking loss objective to ensure a fair comparison with these methods. AOBPR \cite{10.1145/2556195.2556248} enhances the BPR algorithm by incorporating adaptive sampling techniques that prioritize popular negative items. WBPR \cite{gantner2012personalized} assumes that unexplored popular items by a user are more likely to be true negatives. PRIS \cite{10.1145/3366423.3380187} assigns higher weights to informative negative samples using importance sampling. TIL-UI and TIL-MI \cite{wu2022adapting} learn the data value of training triplets through two aggregation strategies by optimizing the BPR loss within the training batch.

\subsection{Implementation Details} \label{app:implenmentation}
We optimize all models using the Adam optimizer with Xavier initialization \cite{glorot2010understanding} and maintain a fixed embedding size of 64 across all methods. When constructing the ranking loss objective, every positive item is associated with one sampled negative item for an efficient computation. Grid search is applied to choose learning rate and weight decay from $\left\{1e^{-4}, 1e^{-3}, 1e^{-2}, 1e^{-1}\right\}$ and $\left\{1e^{-6}, 1e^{-5}, 1e^{-4}, 1e^{-3}\right\}$. The backbone models NeuMF, MGCCF, and LightGCN utilize the provided implementations, with MGCF and LightGCN featuring two graph convolution layers. The total number of training epochs is set to 2000 for all models with an early stopping design. Given the initial training stages' limited information, we pre-train the recommendation model without data valuator for 1000 epochs to get meaningful embeddings. We set the number of the pre-training epochs to 1000. All experiments are conducted on GPU machines (NVIDIA GeForce RTX 3090).

\begin{algorithm}[H]
    \caption{The Proposed Method}
    \label{al:method}
    
    \textbf{Input:} Learning rates $\alpha$ and $\beta$, outer mini-batch size $B_1$, inner mini-batch size $B_2$, outer iteration count $T_1$, inner iteration count $T_2$, moving average window $W$, training pairs $\mathcal{D}_{1}=\{(u,i)\}_{k=1}^{L_1}$, validation pairs $\mathcal{D}_{2}=\{(u,i)\}_{k=1}^{L_2}$
    
    \textbf{Initialize:} parameters $\Theta$ and $\Lambda$, moving average $\delta=0$
    
    \begin{algorithmic}[1]
    \FOR{outer iteration $t_1=1,2,...,T_1$}
        \STATE Sample a mini-batch from the entire training dataset: $\mathcal{\hat{B}}=(u,i)_{k=1}^{B_1}\sim \mathcal{D}_1$
        \STATE Uniformly sample negative items $(j)_{k=1}^{B_1}$ for training pairs $(u,i)_{k=1}^{B_1}$ to get training data $\mathcal{B}=(u,i,j)_{k=1}^{B_1}$ for the recommendation model
        \FOR{$(u,i,j) \in \mathcal{B}$}
        \STATE Calculate the Shapley value by Eq. (\ref{eq:svcal}) and assign it to $w_{uij}$
        \ENDFOR
        \STATE Normalize the Shapley value $w_{uij}$ within the batch $\mathcal{B}$ as $\hat{w}_{uij}$ 
        \STATE Compute sample selection vector $s_{uij}=\text{Ber}(\hat{w}_{uij})$ from Bernoulli distribution
        \STATE Update the data valuator model by \ref{eq:mse}
        \FOR{inner iteration $t_2=1,2,...,T_2$}
        \STATE Sample a mini-batch $(u,i,j)_{m=1}^{B_2}\sim \mathcal{B}$
        \STATE Update the recommendation model:
    $$\Theta \leftarrow \Theta-\frac{\alpha}{B_2} \sum_{m=1}^{B_2} s_{uij} \cdot \nabla_\Theta \mathcal{L}_{\text{BPR}}(u,i,j;\Theta)$$
        \ENDFOR
    \STATE Update the data valuator:
    $$ \begin{array}{r}
    \Lambda \longleftarrow \Lambda - \beta [\mathcal{R}(\mathcal{D}_2, f(\Theta^*(\Lambda)))-\delta ]\\ \cdot \nabla_{\Lambda}log(\pi(\mathcal{B},(s_{uij})_{k=1}^{B_1};{\Lambda})
    \end{array}
    $$
    \STATE Update the moving average reward:
    $$
    \delta \leftarrow \frac{W-1}{W} \delta+\frac{1}{W} \mathcal{R}(\mathcal{D}_2, f(\Theta^*(\Lambda)))
    $$                               
    \ENDFOR
    \end{algorithmic}
\end{algorithm}







\end{document}
