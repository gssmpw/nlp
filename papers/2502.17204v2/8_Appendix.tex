\subsection{Implementation Details}
We utilize 8 NVIDIA A800 80GB GPUs to conduct all the experiments. We employ the vLLM framework~\cite{kwon2023efficient} to accelerate the model inference. For reproducibility, we employ the greed search in the whole inference (i.e., setting the ``do$\_$sample'' to false.).


\subsection{More details for Comstraint Sampling} \label{appx:cons_tax}
In this work, We categorize the constraints into 8 different groups. The categorization is shown in the Tab.~\ref{tab:ifeval}. For each group, there are multiple types of constraints. Specifically, the constraints are designated to: (1) Keyword constraints. These constraints focus on controlling the inclusion or exclusion of specific words or phrases within the response. (2) Language constraints. Language constraints govern the linguistic properties of the response, including the language in which the response is written (e.g., English). (3) Length constraints. These constraints focus on controlling the overall length of the response, including the number of paragraphs, words, and sentences. (4) Content Constraints. Content-related constraints define additional rules to ensure the response contains specific elements. (5) Format constraints. Formatting constraints focus on how the response is structured and styled. For example. (6) ChangeCase Constraints. These constraints focus on adjusting the case of words in the response. They may require the entire response to be in uppercase letters (e.g., ALL CAPS), or entirely in lowercase letters (e.g., all lowercase). (7) StartEnd constraints. These constraints limit the very beginning or ending of the model outputs. (8) Punctuation constraints. These constraints limit the appearance of specific commas.

\renewcommand{\arraystretch}{1.15}
\begin{table*}
\centering
\begin{tabular}{>{\raggedright}m{3cm}|m{4cm}|m{9cm}}
\hline
\textbf{Constraint Group} & \textbf{Constraint} & \textbf{Description Example} \\
\hline
\multirow{4}{*}{Keyword} & Include Keywords & Include keywords \texttt{[keyword1], [keyword2]} in your response. \\
\cline{2-3}
 & Exclude Keywords & Do not include keywords \texttt{[forbidden words]} in the response. \\
\cline{2-3}
 & Keyword Frequency & In your response, the word should appear \texttt{N} times. \\
\cline{2-3}
 & Letter Frequency & In your response, the letter \texttt{[letter]} should appear \texttt{[N]} times. \\
\hline
Language & Response Language & Your ENTIRE response should be in \texttt{[language]}, no other language is allowed. \\
\hline
\multirow{4}{*}{Length} & Number Paragraphs & Your response should contain \texttt{[N]} paragraphs. You separate paragraphs using the markdown divider \texttt{***}. \\
\cline{2-3}
 & Number Words & Answer with at least/around/at most \texttt{[N]} words. \\
\cline{2-3}
& Number Sentences & Answer with at least/around/at most \texttt{[N]} sentences. \\
\cline{2-3}
& Number Paragraphs + First Word in i-th Paragraph & There should be \texttt{[N]} paragraphs. Paragraphs and only paragraphs are separated with each other by two line breaks. The \texttt{[i]}-th paragraph must start with \texttt{[first\_word]}. \\
\hline
\multirow{2}{*}{Content} & Postscript & At the end of your response, please add a postscript starting with \texttt{[postscript marker]}. \\
\cline{2-3}
 & Number Placeholder & The response must contain at least \texttt{[N]} placeholders representing the word space brackets, such as \texttt{[address]}. \\
\hline
\multirow{6}{*}{Format} & Number Bullets & Your response must contain exactly \texttt{[N]} bullet points. Use the markdown bullet points such as: * This is a pont. \\
\cline{2-3}
& Title & Your answer must contain a title, wrapped in double angular brackets, such as \texttt{<<option of joy>>}. \\
\cline{2-3}
 & Choose From & Your response should contain one of the following options: \texttt{[options]}. \\
\cline{2-3}
 & Minimum Number Highlighted Section & Highlight at least \texttt{[N]} sections in your answer with markdown, i.e.  *highlighted section*. \\
\cline{2-3}
 & Multiple Sections & Your response must have \texttt{[N]} sections. Mark the beginning of each section with \texttt{[section\_splitter]} X. \\
\cline{2-3}
 & JSON Format & Entire output should be wrapped in JSON format. \\
\hline
\multirow{3}{*}{ChangeCase} & All Uppercase & Your entire response should be in English, capital letters only. \\
\cline{2-3}
 & All Lowercase & Your response should be in English, and in all lowercase letters. No capital letters are allowed. \\
\cline{2-3}
& Frequency of All-capital Words & In your response, words with all capital letters should appear at least \texttt{[N]} times. \\
\hline
\multirow{2}{*}{StartEnd} & End Checker & Your response must finish with this phrase: \texttt{<end\_phrase>}.  \\
\cline{2-3}
& Quotation & Wrap uour entire response with double marks. \\
\hline
Punctuation & No Commas & In your entire response, refrain from the use of any commas. \\
\hline
\end{tabular}
\caption{The categorization for different constraints.}
\label{tab:ifeval}
\end{table*}




Considering the LLM is vulnerable to different descriptions of the constraints~\cite{yan2024contrastive}, we employ the GPT4o-mini to generate different descriptions of the same constraints. Specifically, given a description example, we leverage the prompt shown in the Tab.~\ref{tab:100diverse} to seven distinct variants. Overall, we obtain 8 distinct descriptions for a specific type of constraint.

\begin{table*}[t]
% \resizebox{0.95\textwidth}{!}{
\small
    \begin{tabularx}{\linewidth}{X}
    \toprule
    \color{gray}{/* \textit{Task prompt} */}\\
    You are provided with a <constraint> in an instruction. As a prompt engineer, your task is to rephrase the provided <constraint> to make it more diverse. You ought to provide five more variants of the <constraint>. Make sure your revision does not change the meaning of the original <constraint>. \\
    \color{gray}{/* \textit{Example} */}\\
    ---INPUT--- \\
    <constraint>:\\
    Your response should contain at least 3 sentences.\\
    ---OUTPUT---\\
    variants:\\
    1. Respond with at least three sentences\\
    2. Use at least 3 sentences in your reply\\
    3. Your entire response should include at least three sentences\\
    4. Organize your entire response in at least 3 sentences\\
    5. Please make sure the response is at least 3 sentences long\\
    \color{gray}{/* \textit{Input} */}\\
    ---INPUT---\\
    <constraint>:\\
    \{\textbf{Given\_constraint}\}\\
    ---OUTPUT---\\
    variants:\\
    \bottomrule
    \end{tabularx}
    % }
  \caption{
    The prompts for diversifying the descriptions of a given constraint. We utilize one-shot in-context learning to enhance the performance. The information that requires manual input is highlighted.
  }
  \label{tab:100diverse}
\end{table*}

