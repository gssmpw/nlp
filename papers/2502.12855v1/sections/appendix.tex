\appendix
\clearpage

\section{Arithmetic in Reasoning Context}
\label{app:gsm8k_arith_details}
We evaluate the frequency of numerical computation errors made by models within a reasoning context. For this evaluation, we use the \gsm{} test set and identify arithmetic computations based on the provided calculation annotations (enclosed within \texttt{<<>>}). Given a question and its solution up to an annotation, models are prompted to generate the next few tokens. As illustrated in Figure~\ref{fig:gsm_arith_examples}, each model generates five tokens per input. We extract the numerical values at the beginning of the generated text and compare them to the expected output to measure accuracy.

\section{Prompt for Math Word Problems}
\label{app:math_prompt}
We use four exemplars from the prompt from \citet{wei2022chain} to evaluate the models on math word problem datasets except AQuA. The prompt is shown in Listing~\ref{fig:mwp_prompt}. The prompt for AQuA is shown in Listing~\ref{fig:aqua_prompt}.

\section{MAWPS Detailed Results}
\label{app:mawps_detailed_results}
Table~\ref{tab:mawps_detailed_results} shows the accuracy (\%) achieved by the instruction-tuned models on different datasets within MAWPS.

\section{Robustness with Greedy Decoding}
\label{app:robustness_greedy}
We also evaluate the robustness of post-trained models with greedy decoding. Our results show that the model fine-tuned on the \tulu{} mixture and the arithmetic dataset is more robust to numerical changes than the model fine-tuned only on the \tulu{} mixture. The results on GSM-Plus and GSM-Symbolic are illustrated in Figure~\ref{fig:gsm_plus_accuracy_greedy} and Table~\ref{tab:gsm_symbolic_results_greedy}, respectively.

\begin{figure}[t]
    \centering
    \resizebox{\linewidth}{!}{
        \Large\ttfamily
        \includesvg[]{img/gsm_plus_category_wise_greedy.svg}
    }
    \caption{Performance of the pre-trained and instruction-tuned GPT2-Large models on GSM-Plus for different perturbation types using greedy decoding. The model fine-tuned on \tulu{} mixture and the arithmetic dataset performs better across different perturbation types. The percentages above the bars represent the performance drop relative to the original GSM8k dataset, as shown in Table~\ref{tab:inst_tuning_results}.}
    \label{fig:gsm_plus_accuracy_greedy}
\end{figure}

\begin{table}[h]
    \centering
    \begin{small}
    \begin{tabular}{ccrrr}
    \toprule
    \tulun{} & \multirow{2}{*}{Arith.} & GSM8k & GSM-Symb. & $\Delta$ \\
    SFT & {} & Acc. (\%) & Acc. (\%) & (\%) \\
    \midrule
    \cmark & \xmark & 10.0 & 6.8 $\pm$ 1.9 & -32.0 \\
    \cmark & \cmark & 17.0 & 13.3 $\pm$ 2.6 & \textbf{-21.8} \\
    \bottomrule
    \end{tabular}
    \end{small}
    \caption{Performance of the post-trained models using greedy decoding when evaluated on the original GSM8k problems vs the same problems in GSM-Symbolic. The performance drop is indicated by $\Delta$.}
    \label{tab:gsm_symbolic_results_greedy}
\end{table}


\section{Dataset Statistics}
Table~\ref{tab:dataset_stats} shows the statistics of the datasets used in this work.
\sisetup{table-text-alignment=center,table-format=6.0}
\begin{tabular}{lSSSSrrS}
\toprule
{\textbf{Attribute}} & {\textbf{Declarations}} & {\textbf{Users}} & {\textbf{Inactive Users}} & {\textbf{Ambiguous Users}} & \multicolumn{2}{c}{\textbf{Class Proportion}} & {\textbf{Subreddits}} \\
\midrule
Year of Birth & 420803 & 401390 & 1630 & 17341 & Old: 56.19\% & Young: 43.81\% & 9806 \\
Gender & 424330 & 403428 & 1634 & 18337 & Male: 50.89\% & Female: 49.11\% & 9809 \\
Partisan Affiliation & 6369 & 6118 & 4 & 251 & Dem.: 54.55\% & Rep.: 45.45\% & 9137 \\
\bottomrule
\end{tabular}


\begin{figure*}
    \centering
    \begin{figure*}
    \centering
    \begin{figure*}
    \centering
    \input{img/gsm8k_arith_example}
    \caption{An example of how models are evaluated for arithmetic errors in reasoning contexts. This example has two arithmetic computations, each resulting in a test example for the GSM8k arithmetic evaluation.}
    \label{fig:gsm_arith_examples}
\end{figure*}
    \caption{An example of how models are evaluated for arithmetic errors in reasoning contexts. This example has two arithmetic computations, each resulting in a test example for the GSM8k arithmetic evaluation.}
    \label{fig:gsm_arith_examples}
\end{figure*}
    \caption{An example of how models are evaluated for arithmetic errors in reasoning contexts. This example has two arithmetic computations, each resulting in a test example for the GSM8k arithmetic evaluation.}
    \label{fig:gsm_arith_examples}
\end{figure*}
\begin{table*}
    \centering
    \begin{small}
    \begin{tabular}{ccrrrrr}
    \toprule
    \textbf{\tulu{}} & \textbf{Arith.} & \textbf{MultiArith} & \textbf{AddSub} & \textbf{SingleOp} & \textbf{SingleEq} & \textbf{SimulEq} \\
    \midrule
    \multicolumn{7}{c}{\it Greedy Decoding} \\
    \midrule
    \xmark & \xmark & 2.8 & 2.8 & 4.4 & 1.8 & 1.4 \\
    \cmark & \xmark & 39.4 & 11.9 & 45.3 & 43.1 & \textbf{6.8} \\
    \cmark & \cmark & \textbf{50.0} & \textbf{36.7} & \textbf{64.2} & \textbf{59.6} & 4.8 \\
    \midrule
    \multicolumn{7}{c}{\it Self-Consistency Decoding} \\
    \midrule
    \xmark & \xmark & 3.7 & 1.2 & 2.7 & 0.6 & 2.5 \\
    \cmark & \xmark & 56.1 & 18.7 & 51.4 & 47.7 & 5.3 \\
    \cmark & \cmark & \textbf{65.9} & \textbf{41.3} & \textbf{68.3} & \textbf{65.7} & \textbf{5.5} \\
    \bottomrule
    \end{tabular}
    \end{small}
    \caption{Accuracy (\%) achieved by the instruction-tuned GPT2-Large models on datasets in MAWPS. The first rows under both greedy and self-consistency decoding denote the pre-trained model.}
    \label{tab:mawps_detailed_results}
\end{table*}

\begin{figure*}[t]
    \lstinputlisting[
        breaklines,
        basicstyle=\footnotesize\ttfamily,
        columns=flexible,
        captionpos=b,
        caption=Prompt for math word problem datasets except for AQuA.,
        backgroundcolor=\color{Gray!10},
        frame=tlbr,
        framesep=5pt,
        % numbers=left,
        label=fig:mwp_prompt,
    ]{txt/gsm8k_4shot.txt}
\end{figure*}
\begin{figure*}[t]
    \lstinputlisting[
        breaklines,
        basicstyle=\footnotesize\ttfamily,
        columns=flexible,
        captionpos=b,
        caption=Prompt for AQuA.,
        backgroundcolor=\color{Gray!10},
        frame=tlbr,
        framesep=5pt,
        % numbers=left,
        label=fig:aqua_prompt,
    ]{txt/aqua_4shot.txt}
\end{figure*}