\documentclass[12pt,twoside]{report}



\newcommand{\reporttitle}{LLMs to Support a Domain Specific Knowledge Assistant}
\newcommand{\reportauthor}{Maria-Flavia Lovin}
\newcommand{\supervisor}{Dr. Pedro Baiz Villafranca}
\newcommand{\secondmarker}{Dr. Tolga Birdal}
\newcommand{\degreetype}{MSc Artificial Intelligence}


\usepackage{wrapfig}

% If you use the following package, be sure to comment out \usepackate{xcolor}

\usepackage{multirow,mathtools } \usepackage{algorithm,algpseudocode}

% Very useful for various special symbols
\usepackage{pifont}
% For \rowcolor
\usepackage{color, colortbl}

\usepackage{blindtext}
\usepackage{lipsum}

\usepackage{multirow}
\usepackage{graphicx}
\usepackage{listings}

\usepackage{bbm}
\usepackage{dsfont}

\usepackage[most]{tcolorbox}

\newcommand{\JH}[1]{\textcolor{blue}{JH: #1}}
\newcommand{\SL}[1]{\textcolor{purple}{SL: #1}}
\newcommand{\JC}[1]{\textcolor{cyan}{JC: #1}}
\newcommand{\SM}[1]{\textcolor{red}{SM: #1}}
\newcommand{\CF}[1]{\textcolor{orange}{CF: #1}}
\newcommand{\RZ}[1]{\textcolor{red}{RZ: #1}}
\newcommand{\LL}[1]{\textcolor{red}{LL: #1}}
\newcommand{\YY}[1]{\textcolor{red}{YY: #1}}



% !TeX root = main.tex 


\newcommand{\lnote}{\textcolor[rgb]{1,0,0}{Lydia: }\textcolor[rgb]{0,0,1}}
\newcommand{\todo}{\textcolor[rgb]{1,0,0.5}{To do: }\textcolor[rgb]{0.5,0,1}}


\newcommand{\state}{S}
\newcommand{\meas}{M}
\newcommand{\out}{\mathrm{out}}
\newcommand{\piv}{\mathrm{piv}}
\newcommand{\pivotal}{\mathrm{pivotal}}
\newcommand{\isnot}{\mathrm{not}}
\newcommand{\pred}{^\mathrm{predict}}
\newcommand{\act}{^\mathrm{act}}
\newcommand{\pre}{^\mathrm{pre}}
\newcommand{\post}{^\mathrm{post}}
\newcommand{\calM}{\mathcal{M}}

\newcommand{\game}{\mathbf{V}}
\newcommand{\strategyspace}{S}
\newcommand{\payoff}[1]{V^{#1}}
\newcommand{\eff}[1]{E^{#1}}
\newcommand{\p}{\vect{p}}
\newcommand{\simplex}[1]{\Delta^{#1}}

\newcommand{\recdec}[1]{\bar{D}(\hat{Y}_{#1})}





\newcommand{\sphereone}{\calS^1}
\newcommand{\samplen}{S^n}
\newcommand{\wA}{w}%{w_{\mathfrak{a}}}
\newcommand{\Awa}{A_{\wA}}
\newcommand{\Ytil}{\widetilde{Y}}
\newcommand{\Xtil}{\widetilde{X}}
\newcommand{\wst}{w_*}
\newcommand{\wls}{\widehat{w}_{\mathrm{LS}}}
\newcommand{\dec}{^\mathrm{dec}}
\newcommand{\sub}{^\mathrm{sub}}

\newcommand{\calP}{\mathcal{P}}
\newcommand{\totspace}{\calZ}
\newcommand{\clspace}{\calX}
\newcommand{\attspace}{\calA}

\newcommand{\Ftil}{\widetilde{\calF}}

\newcommand{\totx}{Z}
\newcommand{\classx}{X}
\newcommand{\attx}{A}
\newcommand{\calL}{\mathcal{L}}



\newcommand{\defeq}{\mathrel{\mathop:}=}
\newcommand{\vect}[1]{\ensuremath{\mathbf{#1}}}
\newcommand{\mat}[1]{\ensuremath{\mathbf{#1}}}
\newcommand{\dd}{\mathrm{d}}
\newcommand{\grad}{\nabla}
\newcommand{\hess}{\nabla^2}
\newcommand{\argmin}{\mathop{\rm argmin}}
\newcommand{\argmax}{\mathop{\rm argmax}}
\newcommand{\Ind}[1]{\mathbf{1}\{#1\}}

\newcommand{\norm}[1]{\left\|{#1}\right\|}
\newcommand{\fnorm}[1]{\|{#1}\|_{\text{F}}}
\newcommand{\spnorm}[2]{\left\| {#1} \right\|_{\text{S}({#2})}}
\newcommand{\sigmin}{\sigma_{\min}}
\newcommand{\tr}{\text{tr}}
\renewcommand{\det}{\text{det}}
\newcommand{\rank}{\text{rank}}
\newcommand{\logdet}{\text{logdet}}
\newcommand{\trans}{^{\top}}
\newcommand{\poly}{\text{poly}}
\newcommand{\polylog}{\text{polylog}}
\newcommand{\st}{\text{s.t.~}}
\newcommand{\proj}{\mathcal{P}}
\newcommand{\projII}{\mathcal{P}_{\parallel}}
\newcommand{\projT}{\mathcal{P}_{\perp}}
\newcommand{\projX}{\mathcal{P}_{\mathcal{X}^\star}}
\newcommand{\inner}[1]{\langle #1 \rangle}

\renewcommand{\Pr}{\mathbb{P}}
\newcommand{\Z}{\mathbb{Z}}
\newcommand{\N}{\mathbb{N}}
\newcommand{\R}{\mathbb{R}}
\newcommand{\E}{\mathbb{E}}
\newcommand{\F}{\mathcal{F}}
\newcommand{\var}{\mathrm{var}}
\newcommand{\cov}{\mathrm{cov}}


\newcommand{\calN}{\mathcal{N}}

\newcommand{\jccomment}{\textcolor[rgb]{1,0,0}{C: }\textcolor[rgb]{1,0,1}}
\newcommand{\fracpar}[2]{\frac{\partial #1}{\partial  #2}}

\newcommand{\A}{\mathcal{A}}
\newcommand{\B}{\mat{B}}
%\newcommand{\C}{\mat{C}}

\newcommand{\I}{\mat{I}}
\newcommand{\M}{\mat{M}}
\newcommand{\D}{\mat{D}}
%\newcommand{\U}{\mat{U}}
\newcommand{\V}{\mat{V}}
\newcommand{\W}{\mat{W}}
\newcommand{\X}{\mat{X}}
\newcommand{\Y}{\mat{Y}}
\newcommand{\mSigma}{\mat{\Sigma}}
\newcommand{\mLambda}{\mat{\Lambda}}
\newcommand{\e}{\vect{e}}
\newcommand{\g}{\vect{g}}
\renewcommand{\u}{\vect{u}}
\newcommand{\w}{\vect{w}}
\newcommand{\x}{\vect{x}}
\newcommand{\y}{\vect{y}}
\newcommand{\z}{\vect{z}}
\newcommand{\fI}{\mathfrak{I}}
\newcommand{\fS}{\mathfrak{S}}
\newcommand{\fE}{\mathfrak{E}}
\newcommand{\fF}{\mathfrak{F}}

\newcommand{\Risk}{\mathcal{R}}

\renewcommand{\L}{\mathcal{L}}
\renewcommand{\H}{\mathcal{H}}

\newcommand{\cn}{\kappa}
\newcommand{\nn}{\nonumber}


\newcommand{\Hess}{\nabla^2}
\newcommand{\tlO}{\tilde{O}}
\newcommand{\tlOmega}{\tilde{\Omega}}

\newcommand{\calF}{\mathcal{F}}
\newcommand{\fhat}{\widehat{f}}
\newcommand{\calS}{\mathcal{S}}

\newcommand{\calX}{\mathcal{X}}
\newcommand{\calY}{\mathcal{Y}}
\newcommand{\calD}{\mathcal{D}}
\newcommand{\calZ}{\mathcal{Z}}
\newcommand{\calA}{\mathcal{A}}
\newcommand{\fbayes}{f^B}
\newcommand{\func}{f^U}


\newcommand{\bayscore}{\text{calibrated Bayes score}}
\newcommand{\bayrisk}{\text{calibrated Bayes risk}}

\newtheorem{example}{Example}[section]
\newtheorem{exc}{Exercise}[section]
%\newtheorem{rem}{Remark}[section]

\newtheorem{theorem}{Theorem}[section]
\newtheorem{definition}{Definition}
\newtheorem{proposition}[theorem]{Proposition}
\newtheorem{corollary}[theorem]{Corollary}

\newtheorem{remark}{Remark}[section]
\newtheorem{lemma}[theorem]{Lemma}
\newtheorem{claim}[theorem]{Claim}
\newtheorem{fact}[theorem]{Fact}
\newtheorem{assumption}{Assumption}

\newcommand{\iidsim}{\overset{\mathrm{i.i.d.}}{\sim}}
\newcommand{\unifsim}{\overset{\mathrm{unif}}{\sim}}
\newcommand{\sign}{\mathrm{sign}}
\newcommand{\wbar}{\overline{w}}
\newcommand{\what}{\widehat{w}}
\newcommand{\KL}{\mathrm{KL}}
\newcommand{\Bern}{\mathrm{Bernoulli}}
\newcommand{\ihat}{\widehat{i}}
\newcommand{\Dwst}{\calD^{w_*}}
\newcommand{\fls}{\widehat{f}_{n}}


\newcommand{\brpi}{\pi^{br}}
\newcommand{\brtheta}{\theta^{br}}

% \newcommand{\M}{\mat{M}}
% \newcommand\Mmh{\mat{M}^{-1/2}}
% \newcommand{\A}{\mat{A}}
% \newcommand{\B}{\mat{B}}
% \newcommand{\C}{\mat{C}}
% \newcommand{\Et}[1][t]{\mat{E_{#1}}}
% \newcommand{\Etp}{\Et[t+1]}
% \newcommand{\Errt}[1][t]{\mat{\bigtriangleup_{#1}}}
% \newcommand\cnM{\kappa}
% \newcommand{\cn}[1]{\kappa\left(#1\right)}
% \newcommand\X{\mat{X}}
% \newcommand\fstar{f_*}
% \newcommand\Xt[1][t]{\mat{X_{#1}}}
% \newcommand\ut[1][t]{{u_{#1}}}
% \newcommand\Xtinv{\inv{\Xt}}
% \newcommand\Xtp{\mat{X_{t+1}}}
% \newcommand\Xtpinv{\inv{\left(\mat{X_{t+1}}\right)}}
% \newcommand\U{\mat{U}}
% \newcommand\UTr{\trans{\mat{U}}}
% \newcommand{\Ut}[1][t]{\mat{U_{#1}}}
% \newcommand{\Utinv}{\inv{\Ut}}
% \newcommand{\UtTr}[1][t]{\trans{\mat{U_{#1}}}}
% \newcommand\Utp{\mat{U_{t+1}}}
% \newcommand\UtpTr{\trans{\mat{U}_{t+1}}}
% \newcommand\Utptild{\mat{\widetilde{U}_{t+1}}}
% \newcommand\Us{\mat{U^*}}
% \newcommand\UsTr{\trans{\mat{U^*}}}
% \newcommand{\Sigs}{\mat{\Sigma}}
% \newcommand{\Sigsmh}{\Sigs^{-1/2}}
% \newcommand{\eye}{\mat{I}}
% \newcommand{\twonormbound}{\left(4+\DPhi{\M}{\Xt[0]}\right)\twonorm{\M}}
% \newcommand{\lamj}{\lambda_j}

% \renewcommand\u{\vect{u}}
% \newcommand\uTr{\trans{\vect{u}}}
% \renewcommand\v{\vect{v}}
% \newcommand\vTr{\trans{\vect{v}}}
% \newcommand\w{\vect{w}}
% \newcommand\wTr{\trans{\vect{w}}}
% \newcommand\wperp{\vect{w}_{\perp}}
% \newcommand\wperpTr{\trans{\vect{w}_{\perp}}}
% \newcommand\wj{\vect{w_j}}
% \newcommand\vj{\vect{v_j}}
% \newcommand\wjTr{\trans{\vect{w_j}}}
% \newcommand\vjTr{\trans{\vect{v_j}}}

% \newcommand{\DPhi}[2]{\ensuremath{D_{\Phi}\left(#1,#2\right)}}
% \newcommand\matmult{{\omega}}


\date{September 2024}

\begin{document}


\begin{titlepage}

\newcommand{\HRule}{\rule{\linewidth}{0.5mm}} 


\includegraphics[width = 4cm]{./figures/imperial}\\[0.5cm] 

\center 

\textsc{\Large Imperial College London}\\[0.5cm] 
\textsc{\large Department of Computing}\\[0.5cm] 

\HRule \\[0.4cm]
{ \huge \bfseries \reporttitle}\\ 
\HRule \\[1.5cm]

\begin{minipage}{0.4\textwidth}
\begin{flushleft} \large
\emph{Author:}\\
\reportauthor 
\end{flushleft}
\end{minipage}
~
\begin{minipage}{0.4\textwidth}
\begin{flushright} \large
\emph{Supervisor:} \\
\supervisor \\ 
\emph{Second marker:}\\
\secondmarker
\end{flushright}
\end{minipage}\\[4cm]

\vfill
Submitted in partial fulfillment of the requirements for the MSc degree in
\degreetype~of Imperial College London\\[0.5cm]

\makeatletter
\@date 
\makeatother


\end{titlepage}


\pagenumbering{roman}
\setcounter{page}{1}
\pagestyle{fancy}

\begin{abstract}
This work presents a custom approach to developing a domain-specific knowledge assistant for sustainability reporting using the International Financial Reporting Standards (IFRS). In this domain, there is no publicly available question-answer dataset, which has impeded the development of a high-quality chatbot to support companies with IFRS reporting. The two key contributions of this project therefore are:\\

(1) A high-quality synthetic question-answer dataset based on IFRS sustainability standards, created using a novel generation and evaluation pipeline leveraging Large Language Models (LLMs). This dataset comprises 1,063 diverse question-answer pairs that address a wide spectrum of potential user queries in sustainability reporting. Various LLM-based techniques are employed to create the dataset, including chain-of-thought reasoning and few-shot prompting. A custom evaluation framework is developed to assess question and answer quality across multiple dimensions, including faithfulness, relevance, and domain specificity. The resulting dataset averages a score range of 8.16 out of 10 on these metrics.\\

(2) Two custom architectures tailored for question-answering in the sustainability reporting domain - a retrieval augmented generation (RAG) pipeline and a fully LLM-based pipeline. The architectures are developed by experimenting, fine-tuning, and training on the question-answer dataset. The final pipelines feature an LLM fine-tuned on domain-specific data and an industry classification component to improve the handling of complex, multi-industry queries. The RAG architecture achieves an accuracy of 85.32\% on single-industry and 72.15\% on cross-industry multiple-choice questions, outperforming the baseline approach by 4.67 and 19.21 percentage points, respectively. The LLM-based pipeline achieves an accuracy of 93.45\% on single-industry and 80.30\% on cross-industry multiple-choice questions, an improvement of 12.80 and 27.36 percentage points over the baseline, respectively.  \\

\end{abstract}

\section*{Acknowledgments}

I extend my deepest appreciation to Dr. Pedro Baiz Villafranca for his invaluable guidance and support throughout this project. His encouragement and constructive feedback from the project's inception have been instrumental in its success.\\

I also wish to thank Dr. Tolga Birdal for his co-supervision and for providing clarity on a number of questions during the early stages of this project.\\

Lastly, I want to express my gratitude to my family and close friends for their continuous love and support. Without them, this work would not have been possible.

\tableofcontents 

\newpage

\pagenumbering{arabic}
\setcounter{page}{1}

\section{Introduction}
\label{sec:introduction}
The business processes of organizations are experiencing ever-increasing complexity due to the large amount of data, high number of users, and high-tech devices involved \cite{martin2021pmopportunitieschallenges, beerepoot2023biggestbpmproblems}. This complexity may cause business processes to deviate from normal control flow due to unforeseen and disruptive anomalies \cite{adams2023proceddsriftdetection}. These control-flow anomalies manifest as unknown, skipped, and wrongly-ordered activities in the traces of event logs monitored from the execution of business processes \cite{ko2023adsystematicreview}. For the sake of clarity, let us consider an illustrative example of such anomalies. Figure \ref{FP_ANOMALIES} shows a so-called event log footprint, which captures the control flow relations of four activities of a hypothetical event log. In particular, this footprint captures the control-flow relations between activities \texttt{a}, \texttt{b}, \texttt{c} and \texttt{d}. These are the causal ($\rightarrow$) relation, concurrent ($\parallel$) relation, and other ($\#$) relations such as exclusivity or non-local dependency \cite{aalst2022pmhandbook}. In addition, on the right are six traces, of which five exhibit skipped, wrongly-ordered and unknown control-flow anomalies. For example, $\langle$\texttt{a b d}$\rangle$ has a skipped activity, which is \texttt{c}. Because of this skipped activity, the control-flow relation \texttt{b}$\,\#\,$\texttt{d} is violated, since \texttt{d} directly follows \texttt{b} in the anomalous trace.
\begin{figure}[!t]
\centering
\includegraphics[width=0.9\columnwidth]{images/FP_ANOMALIES.png}
\caption{An example event log footprint with six traces, of which five exhibit control-flow anomalies.}
\label{FP_ANOMALIES}
\end{figure}

\subsection{Control-flow anomaly detection}
Control-flow anomaly detection techniques aim to characterize the normal control flow from event logs and verify whether these deviations occur in new event logs \cite{ko2023adsystematicreview}. To develop control-flow anomaly detection techniques, \revision{process mining} has seen widespread adoption owing to process discovery and \revision{conformance checking}. On the one hand, process discovery is a set of algorithms that encode control-flow relations as a set of model elements and constraints according to a given modeling formalism \cite{aalst2022pmhandbook}; hereafter, we refer to the Petri net, a widespread modeling formalism. On the other hand, \revision{conformance checking} is an explainable set of algorithms that allows linking any deviations with the reference Petri net and providing the fitness measure, namely a measure of how much the Petri net fits the new event log \cite{aalst2022pmhandbook}. Many control-flow anomaly detection techniques based on \revision{conformance checking} (hereafter, \revision{conformance checking}-based techniques) use the fitness measure to determine whether an event log is anomalous \cite{bezerra2009pmad, bezerra2013adlogspais, myers2018icsadpm, pecchia2020applicationfailuresanalysispm}. 

The scientific literature also includes many \revision{conformance checking}-independent techniques for control-flow anomaly detection that combine specific types of trace encodings with machine/deep learning \cite{ko2023adsystematicreview, tavares2023pmtraceencoding}. Whereas these techniques are very effective, their explainability is challenging due to both the type of trace encoding employed and the machine/deep learning model used \cite{rawal2022trustworthyaiadvances,li2023explainablead}. Hence, in the following, we focus on the shortcomings of \revision{conformance checking}-based techniques to investigate whether it is possible to support the development of competitive control-flow anomaly detection techniques while maintaining the explainable nature of \revision{conformance checking}.
\begin{figure}[!t]
\centering
\includegraphics[width=\columnwidth]{images/HIGH_LEVEL_VIEW.png}
\caption{A high-level view of the proposed framework for combining \revision{process mining}-based feature extraction with dimensionality reduction for control-flow anomaly detection.}
\label{HIGH_LEVEL_VIEW}
\end{figure}

\subsection{Shortcomings of \revision{conformance checking}-based techniques}
Unfortunately, the detection effectiveness of \revision{conformance checking}-based techniques is affected by noisy data and low-quality Petri nets, which may be due to human errors in the modeling process or representational bias of process discovery algorithms \cite{bezerra2013adlogspais, pecchia2020applicationfailuresanalysispm, aalst2016pm}. Specifically, on the one hand, noisy data may introduce infrequent and deceptive control-flow relations that may result in inconsistent fitness measures, whereas, on the other hand, checking event logs against a low-quality Petri net could lead to an unreliable distribution of fitness measures. Nonetheless, such Petri nets can still be used as references to obtain insightful information for \revision{process mining}-based feature extraction, supporting the development of competitive and explainable \revision{conformance checking}-based techniques for control-flow anomaly detection despite the problems above. For example, a few works outline that token-based \revision{conformance checking} can be used for \revision{process mining}-based feature extraction to build tabular data and develop effective \revision{conformance checking}-based techniques for control-flow anomaly detection \cite{singh2022lapmsh, debenedictis2023dtadiiot}. However, to the best of our knowledge, the scientific literature lacks a structured proposal for \revision{process mining}-based feature extraction using the state-of-the-art \revision{conformance checking} variant, namely alignment-based \revision{conformance checking}.

\subsection{Contributions}
We propose a novel \revision{process mining}-based feature extraction approach with alignment-based \revision{conformance checking}. This variant aligns the deviating control flow with a reference Petri net; the resulting alignment can be inspected to extract additional statistics such as the number of times a given activity caused mismatches \cite{aalst2022pmhandbook}. We integrate this approach into a flexible and explainable framework for developing techniques for control-flow anomaly detection. The framework combines \revision{process mining}-based feature extraction and dimensionality reduction to handle high-dimensional feature sets, achieve detection effectiveness, and support explainability. Notably, in addition to our proposed \revision{process mining}-based feature extraction approach, the framework allows employing other approaches, enabling a fair comparison of multiple \revision{conformance checking}-based and \revision{conformance checking}-independent techniques for control-flow anomaly detection. Figure \ref{HIGH_LEVEL_VIEW} shows a high-level view of the framework. Business processes are monitored, and event logs obtained from the database of information systems. Subsequently, \revision{process mining}-based feature extraction is applied to these event logs and tabular data input to dimensionality reduction to identify control-flow anomalies. We apply several \revision{conformance checking}-based and \revision{conformance checking}-independent framework techniques to publicly available datasets, simulated data of a case study from railways, and real-world data of a case study from healthcare. We show that the framework techniques implementing our approach outperform the baseline \revision{conformance checking}-based techniques while maintaining the explainable nature of \revision{conformance checking}.

In summary, the contributions of this paper are as follows.
\begin{itemize}
    \item{
        A novel \revision{process mining}-based feature extraction approach to support the development of competitive and explainable \revision{conformance checking}-based techniques for control-flow anomaly detection.
    }
    \item{
        A flexible and explainable framework for developing techniques for control-flow anomaly detection using \revision{process mining}-based feature extraction and dimensionality reduction.
    }
    \item{
        Application to synthetic and real-world datasets of several \revision{conformance checking}-based and \revision{conformance checking}-independent framework techniques, evaluating their detection effectiveness and explainability.
    }
\end{itemize}

The rest of the paper is organized as follows.
\begin{itemize}
    \item Section \ref{sec:related_work} reviews the existing techniques for control-flow anomaly detection, categorizing them into \revision{conformance checking}-based and \revision{conformance checking}-independent techniques.
    \item Section \ref{sec:abccfe} provides the preliminaries of \revision{process mining} to establish the notation used throughout the paper, and delves into the details of the proposed \revision{process mining}-based feature extraction approach with alignment-based \revision{conformance checking}.
    \item Section \ref{sec:framework} describes the framework for developing \revision{conformance checking}-based and \revision{conformance checking}-independent techniques for control-flow anomaly detection that combine \revision{process mining}-based feature extraction and dimensionality reduction.
    \item Section \ref{sec:evaluation} presents the experiments conducted with multiple framework and baseline techniques using data from publicly available datasets and case studies.
    \item Section \ref{sec:conclusions} draws the conclusions and presents future work.
\end{itemize}

\section{Background}\label{sec:backgrnd}

\subsection{Cold Start Latency and Mitigation Techniques}

Traditional FaaS platforms mitigate cold starts through snapshotting, lightweight virtualization, and warm-state management. Snapshot-based methods like \textbf{REAP} and \textbf{Catalyzer} reduce initialization time by preloading or restoring container states but require significant memory and I/O resources, limiting scalability~\cite{dong_catalyzer_2020, ustiugov_benchmarking_2021}. Lightweight virtualization solutions, such as \textbf{Firecracker} microVMs, achieve fast startup times with strong isolation but depend on robust infrastructure, making them less adaptable to fluctuating workloads~\cite{agache_firecracker_2020}. Warm-state management techniques like \textbf{Faa\$T}~\cite{romero_faa_2021} and \textbf{Kraken}~\cite{vivek_kraken_2021} keep frequently invoked containers ready, balancing readiness and cost efficiency under predictable workloads but incurring overhead when demand is erratic~\cite{romero_faa_2021, vivek_kraken_2021}. While these methods perform well in resource-rich cloud environments, their resource intensity challenges applicability in edge settings.

\subsubsection{Edge FaaS Perspective}

In edge environments, cold start mitigation emphasizes lightweight designs, resource sharing, and hybrid task distribution. Lightweight execution environments like unikernels~\cite{edward_sock_2018} and \textbf{Firecracker}~\cite{agache_firecracker_2020}, as used by \textbf{TinyFaaS}~\cite{pfandzelter_tinyfaas_2020}, minimize resource usage and initialization delays but require careful orchestration to avoid resource contention. Function co-location, demonstrated by \textbf{Photons}~\cite{v_dukic_photons_2020}, reduces redundant initializations by sharing runtime resources among related functions, though this complicates isolation in multi-tenant setups~\cite{v_dukic_photons_2020}. Hybrid offloading frameworks like \textbf{GeoFaaS}~\cite{malekabbasi_geofaas_2024} balance edge-cloud workloads by offloading latency-tolerant tasks to the cloud and reserving edge resources for real-time operations, requiring reliable connectivity and efficient task management. These edge-specific strategies address cold starts effectively but introduce challenges in scalability and orchestration.

\subsection{Predictive Scaling and Caching Techniques}

Efficient resource allocation is vital for maintaining low latency and high availability in serverless platforms. Predictive scaling and caching techniques dynamically provision resources and reduce cold start latency by leveraging workload prediction and state retention.
Traditional FaaS platforms use predictive scaling and caching to optimize resources, employing techniques (OFC, FaasCache) to reduce cold starts. However, these methods rely on centralized orchestration and workload predictability, limiting their effectiveness in dynamic, resource-constrained edge environments.



\subsubsection{Edge FaaS Perspective}

Edge FaaS platforms adapt predictive scaling and caching techniques to constrain resources and heterogeneous environments. \textbf{EDGE-Cache}~\cite{kim_delay-aware_2022} uses traffic profiling to selectively retain high-priority functions, reducing memory overhead while maintaining readiness for frequent requests. Hybrid frameworks like \textbf{GeoFaaS}~\cite{malekabbasi_geofaas_2024} implement distributed caching to balance resources between edge and cloud nodes, enabling low-latency processing for critical tasks while offloading less critical workloads. Machine learning methods, such as clustering-based workload predictors~\cite{gao_machine_2020} and GRU-based models~\cite{guo_applying_2018}, enhance resource provisioning in edge systems by efficiently forecasting workload spikes. These innovations effectively address cold start challenges in edge environments, though their dependency on accurate predictions and robust orchestration poses scalability challenges.

\subsection{Decentralized Orchestration, Function Placement, and Scheduling}

Efficient orchestration in serverless platforms involves workload distribution, resource optimization, and performance assurance. While traditional FaaS platforms rely on centralized control, edge environments require decentralized and adaptive strategies to address unique challenges such as resource constraints and heterogeneous hardware.



\subsubsection{Edge FaaS Perspective}

Edge FaaS platforms adopt decentralized and adaptive orchestration frameworks to meet the demands of resource-constrained environments. Systems like \textbf{Wukong} distribute scheduling across edge nodes, enhancing data locality and scalability while reducing network latency. Lightweight frameworks such as \textbf{OpenWhisk Lite}~\cite{kravchenko_kpavelopenwhisk-light_2024} optimize resource allocation by decentralizing scheduling policies, minimizing cold starts and latency in edge setups~\cite{benjamin_wukong_2020}. Hybrid solutions like \textbf{OpenFaaS}~\cite{noauthor_openfaasfaas_2024} and \textbf{EdgeMatrix}~\cite{shen_edgematrix_2023} combine edge-cloud orchestration to balance resource utilization, retaining latency-sensitive functions at the edge while offloading non-critical workloads to the cloud. While these approaches improve flexibility, they face challenges in maintaining coordination and ensuring consistent performance across distributed nodes.



\chapter{Synthetic Question-Answer Dataset }

This chapter presents the data and methods used to arrive at the final question-answer (QA) generation pipeline that is used to create a synthetic dataset of 1,063 QA pairs for evaluation of the RAG-based knowledge assistant. We start by introducing the domain and presenting the data collection and preprocessing pipeline. Next, we describe the methodologies adopted to generate and filter questions. First, we outline the requirements of our synthetic QA dataset and present the dataset structure created to meet these requirements. Then, we motivate our QA generation approach based on existing techniques, and describe the iterative development of the LLM architectures used, leveraging prompt engineering techniques like Chain of Thought (CoT) reasoning and Few-shot learning. Subsequently, we present a custom evaluation framework for QA quality, using both standard and novel metrics, and introduce three post-processing functions to improve the quality of the dataset. Finally, we present the results of our different question generation methods on the evaluation framework, define the final QA generation pipeline, and present a new dataset of QA pairs to be used for RAG training and evaluation in the sustainability reporting domain.




\section{Data collection and preparation}\label{data_collection}

This work adopts Sustainability Reporting as the knowledge domain, using data published by the International Financial Reporting Standards (IFRS) \cite{ifrsorgc9:online}, which provides international sustainability-related standards used by over 29,000 listed companies globally \cite{ifrs_companies}. These standards undergo periodic review and have recently been modified from previous versions in an effort to unify global reporting standards. The standards comprise 72 documents over 975 pages as described below:

\begin{enumerate}
    \item \textit{IFRS S1 General Requirements for Disclosure of Sustainability-related Financial Information (with an additional `accompanying guidance' document)}. This is the baseline financial-related reporting standard with the objective to ``require an entity to disclose information about its sustainability-related risks and opportunities that is useful to primary users of general purpose financial reports in making decisions relating to providing resources to the entity" \cite{ifrs_s1}. 
    \item \textit{IFRS S2 Climate-related Disclosures (with an additional `accompanying guidance' document)}. This is the baseline climate-related reporting standard with the objective to ``require an entity to disclose information about its climate-related risks and opportunities that is useful to primary users of general purpose financial reports in making decisions relating to providing resources to the entity" \cite{ifrs_s2}.
    \item \textit{Appendix B—Industry-based Disclosure Requirements}. This is the appendix to IFRS S2, comprising 68 separate industry-specific documents grouped into 11 different industry categories. These documents ``set out the [draft] requirements for identifying, measuring and disclosing information related to significant climate-related risks and opportunities associated with particular industries" \cite{ifrs_appendix}. It is important to note that the industry classification of companies is not always clear - IFRS states that entities can be ``associated with an industry through [their] business model, economic activities, production processes or other actions" \cite{ifrs_appendix}.
\end{enumerate}

The documents were downloaded in PDF format from the IFRS website at \cite{ifrs_s1,ifrs_s2,ifrs_appendix}, and the reports were manually examined to understand their structures and content. It was noted that, while S1 and S2 have custom structures, all Appendix reports follow the same structure. They start with identical title, disclaimer, and introduction pages, which are deleted from all but one report to avoid repetition. The material content in each report starts with an `Industry Description' section, followed by a `Sustainability Disclosure Topics \& Metrics' section. The latter contains one or two tables: all reports contain `Table 1. Sustainability Disclosure Topics \& Metrics', and most, though not all, contain `Table 2. Activity Metrics'. The columns of these tables are standardised across all reports, displaying the topics, with respective metrics, that must be reported for the particular industry, as well as as metric codes, categories, and units of measurement. The rest of the report contains detailed information on the topics and metrics outlined in the tables, structured as one section per topic, with subsections pertaining to each topic metric. \\

Finally, it is important to note that, having recently undergone modifications from previous versions, the standards contain many sections of crossed-out text that have been removed from prior versions, often followed by sections of underlined text that have been added. Crossed-out and newly added text is present in both tables as well as free-text sections. As outlined in Section \ref{data_processing}, crossed-out text is removes from the data, and underlined text is kept.

\subsubsection{Data preprocessing} \label{data_processing}

Leveraging the structured nature of the IFRS reports, a custom multi-modal pipeline (Figure \ref{fig:pdf_parse}) was designed to parse the text and table content into markdown format, preserving the hierarchy of report sections. A multi-modal model was chosen for parsing PDFs as it can accurately differentiate between crossed-out and underlined text chunks in the reports, and only extract the underlined ones. Other PDF parsing packages and OCR techniques were tested and were unsuccessful at accurately extracting only the underlined text and not the crossed-out text. The prompt used for text and table extraction is displayed in Appendix \ref{lst:pdf_parse_prompt}. The prompt is designed with particular attention towards crossed-out text, as well as to preserve the hierarchical structure of the report sections. \\

The multi-modal parsing pipeline follows four steps:

\begin{enumerate}
    \item PDF to image conversion: Each page of the PDF is converted into an image using a zoom factor of 2.5 to increase the resolution of the images for more accurate content parsing. The PyMuPDF package is used for this step.
    \item Table presence check: Each image is passed through a multi-modal model which checks if a table is present on the page. The model used is Claude 3.5 Sonnet.
    \item Conditional processing: Pages without tables are inserted into a text extraction prompt which, combined with a JSON schema for structuring the output, is passed through a multi-modal model to extract the text content and associated page number into a JSON file. Pages with tables present are inserted into a table and text extraction prompt, along with a more complex JSON schema, to extract tabular data with custom table columns (alongside any other text content and the page number). 
    \item JSON to markdown conversion: The JSONs for each page are combined into a structured markdown file that maintains section headers and hierarchies, and isolates the tables from the free-text content.
\end{enumerate}

The resulting 72 markdowns are saved to be used in QA generation as well as in the RAG system development described in Chapter 4.


\begin{figure}[H]
    \centering
    \includegraphics[width = 0.8\textwidth]{figures/pdf_parse.pdf}
    \caption{Multi-modal PDF to markdown parsing pipeline.}
    \label{fig:pdf_parse}
\end{figure}




\newpage
\section{Methods}



\subsection{QA design}\label{qa_design}

The ultimate aim of the knowledge assistant is to answer user queries correctly based on the source data. To ensure this, the RAG system must be evaluated on questions that are (1) representative of real user queries and (2) useful for testing the capabilities of the RAG system. The QA dataset is designed based on these two principles.\\

To design questions that are representative of real-life queries, several sources were used. First, the contents of the IFRS reports were manually analysed to identify key topics that are often repeated or heavily emphasised, as well as smaller but critical details that a person likely wants to know. Second, a range of real corporate sustainability reports from SASB (now IFRS) reporting companies were downloaded from \cite{sasb_reports} and studied to identify the ways in which companies use IFRS information to prepare sustainability reports. In most cases, companies prepare tables that explicitly provide the metrics required by IFRS, meaning they are very likely to ask detailed questions about metrics. Third, inspiration for typical user queries was taken from FinQA \cite{chen2022finqa} and Financebench \cite{islam2023financebench} - two prominent expert-annotated QA datasets that are based on corporate reporting. Both the questions themselves as well as their preparation methods were analysed. From the actual Financebench and FinQA question datasets, 20 questions were selected and adapted to the Sustainability Reporting domain. From the Financebench preparation methodology, the idea of preparing ``domain-relevant questions" that are ``generically relevant" to the user's task was adopted. In the case of Financebench, the user's task is defined as the financial analysis of a publicly traded company \cite{islam2023financebench}, whereas the user's task defined for this project is the use of IFRS standards to prepare corporate sustainability reports. It is important to note that these methods were used by experts to manually create questions, whereas this project adapts these methods to LLMs using prompt engineering. \\

Based on these sources, a list of 10-18 user-representative sample questions was created for each question category - Multiple-choice (MCQ) Local, MCQ Cross-industry, Free-text Local, Free-text Cross-industry - to be used for few-shot prompting techniques in the question generation pipeline as discussed in Section \ref{QA_gen}. Six examples are displayed in Listing \ref{lst:qa_structures_example} below, and the full list of sample questions can be found in Appendix \ref{lst:qa_structures}.

\begin{lstlisting}[language=JSON,firstnumber=1,label={lst:qa_structures_example},caption={Select user-representative question structures}]
Local single-hop:
"Can you provide for me the unit of measure I should use for the xxx metric in the xxx industry?",
"On what page can I find details about xxx for the xxx industry?"

Local multi-hop:
"Can you provide for me the unit of measure I should use for each of the metrics for 'xxx' topic(s) in the xxx industry?",
"What is the category of xxx metric in the xxx industry and how should this metric be calculated?"

Cross-industry single-hop:
"What topics should I report on for xxx and xxx (and xxx...) industries?",
"Give me all the metrics from Table 1 that are in the category 'Discussion and Analysis' for xxx and xxx (and xxx...) industries."

Cross-industry multi-hop:
"Give me the codes for all the 'Quantitative' metrics in Table 1 for xxx and xxx (and xxx...) industries.",
"How does the calculation of xxx-related metrics differ between xxx and xxx (and xxx...) companies?"
\end{lstlisting}


To design questions that are useful for testing the RAG system, questions were generated on three dimensions: question span, question complexity, and answer structure.\\

\textbf{1. Question span:} testing the RAG's ability to retrieve information from and reason over multiple source documents. The options are:
\begin{itemize}
    \item Single industry (`Local'): Questions that cover only one industry-related source document. They are designed to test retrieval abilities from one document.
    \item Two industries (`Cross-industry'): Questions that cover a pair of industries, in a way that compares/contrasts or gathers information from both industries. Industry pairs were chosen by an LLM prompted to pair up industries that would realistically be compared based on the industry descriptions. These form a key part of the evaluation dataset, as many IFRS reporting companies span different industries, and are designed to test the RAG's ability to identify the correct documents and source the relevant chunks from both.
\end{itemize}

\textbf{2. Question complexity:} testing the RAG's ability to reason through a question and retrieve necessary data. The options are:
\begin{itemize}
    \item Single-hop: Questions where the answer can be extracted directly from the document. They are designed to evaluate the RAG system's ability to answer factual questions based on small snippets of the source document.
    \item Multi-hop: Questions that require information from multiple sections of one or more reports, or involve multiple logical steps, to be answered. They are designed to evaluate the RAG system's ability to reason in multiple steps and extract multiple information chunks to answer a question.
\end{itemize}

\textbf{3. Answer structure:} testing the RAG's knowledge as well as its ability to answer a question correctly and coherently. The options are:
\begin{itemize}
    \item Multiple-choice: Questions with five answer options and only one correct answer option. They are designed to test the accuracy of the RAG system in an objective way that does not rely on NLP metrics.
    \item Free-text: Questions that have a free-text answer. They are designed to test the accuracy and coherence of the RAG system using traditional NLP metrics such as BLEU and ROUGE.
\end{itemize}

These categories are combined into eight different question configurations: local single-hop multiple-choice questions (MCQ), local multi-hop MCQ, local single-hop free-text questions (FTQ), local multi-hop FTQ, cross-industry single-hop MCQ, cross-industry multi-hop MCQ, cross-industry single-hop FTQ, and cross-industry multi-hop FTQ. 


\subsection{QA Generation}\label{QA_gen}

The overall aim of this project is to generate questions that are as specific as possible to minimise the chance of hallucination, and then generalise the phrasing of the questions in post-processing. As discussed later in this chapter, LLMs tend to generate vague questions based on a given context. As such, steps are taken to guide the LLM to create questions and answers that are highly specific to the domain to maximise their faithfulness and relevance to the text. Post-processing functions are then designed to enable the adjustment of highly specific questions to better mimic natural human questions while maintaining their underlying materiality.\\

The question generation and evaluation pipeline is defined in Algorithm \ref{alg:final-question-generation}. The steps of the pipeline are as follows:
\begin{enumerate}
    \item For each question setting (or type) described in Section \ref{qa_design}, a certain number of questions are generated using LLM prompting. Three techniques are tested for this step: a baseline prompt, a Chain-of-Thought (CoT) prompt, and a prompt that uses both CoT and provides few-shot examples of sample questions.
    \item Each question is then evaluated using a number of custom metric implementations. The evaluation metrics designed and employed are faithfulness, relevance, and specificity. Free-text questions are also evaluated using BLEU and ROUGE-L.
    \item Questions that fail to pass certain evaluation thresholds may be improved in post-processing using a quality improvement function.
    \item Questions that are ``too" specific may be passed through a generalisation function to make them sound more vague while maintaining their factfulness.
    \item A similarity filter is used to exclude question duplicates.
    \item MCQs are checked to ensure they have a `single best answer' (SBA). Questions that fail this test are discarded.
\end{enumerate}
\begin{algorithm}[H]
\caption{Final Question Generation Pipeline}
\label{alg:final-question-generation}
\begin{algorithmic}[1]
\Require Set of industry markdowns $M$, Number of questions $N$, Question types $T$ (free-text, MCQ)
\Ensure Set of quality question-answer pairs $Q$
\State $Q \gets \emptyset$
\For{each question type $t \in T$}
    \For{$i = 1$ to $N$}
        \State Select $k \geq 1$ industry contexts $C = \{c_1, \ldots, c_k\}$ from $M$
        \State $q \gets \Call{GenerateQAPair}{C, t}$ 
        \Comment{using Claude 3.5 Sonnet}
        \State $metrics \gets \Call{EvaluateMetrics}{q}$
        \If{$\exists m \in metrics : m < \theta_m$} \Comment{$\theta_m$: metric threshold}
            \State $q \gets \Call{QualityImprovement}{q, metrics}$
        \EndIf
        \If{$q$ contains industry name}
            \State $q \gets \Call{QuestionGeneralization}{q}$
        \EndIf
        \State $is\_similar \gets \Call{SimilarityFilter}{q}$
        \If{$\lnot is\_similar$ and $\forall m \in metrics : m \geq \theta_m$}
            \If{$t$ is multiple-choice}
                \State $has\_single\_best\_answer \gets \Call{SingleBestAnswer}{q}$
                \If{$has\_single\_best\_answer$}
                    \State $Q \gets Q \cup \{q\}$
                \EndIf
            \Else
                \State $Q \gets Q \cup \{q\}$
            \EndIf
        \EndIf
    \EndFor
\EndFor
\State \Return $Q$
\end{algorithmic}
\end{algorithm}
To design the prompt architecture for QA generation, three approaches were tried. Firstly, a baseline prompt was used to generate questions based on the markdown content as a baseline. To reduce the chance of hallucination, a second approach was adopted using a more sophisticated prompt with Chain-of-Thought (CoT) reasoning. Finally, to ensure that the questions are representative of real user queries, the last prompt-based approach incorporates few-shot learning into the prompt. \\

All questions are structured to have the question, the answer, the reference text that the question and answer is based on, and the pages where the reference text is sourced from. Multiple-choice questions also have answer options A through E, where only one option is the correct answer. 

\subsubsection{Baseline Prompting}

In the baseline method, question-answer pairs are generated with an LLM function call using a prompt that gives the LLM context of the task at hand - generating questions based on the markdown from the perspective of a company preparing sustainability reports. However, it is given no guidance on how these questions should be generated. The prompt takes as inputs the markdown and the question complexity: single-hop or multi-hop. For questions comparing different industries, the prompt asks the LLM to generate questions based on all markdowns given. The prompt used is shown in Listing \ref{lst:naive_prompt}.\\
\begin{lstlisting}[language=JSON,firstnumber=1,label={lst:naive_prompt},caption={Baseline prompt for `local' multiple-choice question generation}]
system = "You are a sustainability reporting expert that helps companies draft their corporate sustainability reports using the IFRS reporting standards. You are preparing some questions that a company might ask while preparing its sustainability report, for which the answer can be taken from the context given in the markdown below."

user =  f"""
Here is the markdown content:
{markdown_content}

Based on the markdown content, generate {n} multiple-choice questions of type {qa_type}.

Generate {n} QA pairs for the specified type and return them using the provided schema."""
\end{lstlisting}

LLM `tool use' is used to ensure the output is in the required format containing the question, answer, reference text, and pages (and answer options for MCQs). A JSON schema is used as the tool to structure the output, and this is shown in Appendix \ref{lst:mcq_schema}.



\subsubsection{Chain-of-Thought Prompting}

In the second approach, several additions are made to the prompt to reduce the possibility of hallucination and improve the style of questions generated. The main addition is CoT reasoning to guide the LLM on the steps to take to generate questions, adapting traditional question-generation methods. Specifically, the LLM is prompted to:

\begin{enumerate}
    \item Select a list of one or more sentences/snippets of the markdown content that can be used to form an answer to a question. This will form the reference text.
    \item Write a question that requires the reader to understand the content of the selected text to answer correctly. The question should be based only on the selected text and should not require any additional information.
    \item Write five answer options, one of which is correct and the other four are incorrect. The correct answer should be complete and taken verbatim from the selected section(s) of the markdown content.
\end{enumerate}


This process adapts the traditional question-generation methods to LLMs by first guiding it to select the text that serves as the response before generating the QA pair. This adaptation has the advantage that the LLM has full context of what the text is about and is thus able to jointly select pieces of reference text that are not only most relevant to the specific domain but also would make the ``best" questions. Furthermore, it can do so with a reinforced understanding of its role as a human that wants information specifically for a company preparing sustainability reports. \\

Further additions to the prompt for MCQs include making explicit that the questions should have a single best answer, which should require very specific information from the context to be answered, as well as ensuring that the other answer options are not so obviously wrong that they can easily be discarded. The prompt used to achieve this (for local multiple-choice questions) is shown in Listing \ref{lst:cot_prompt}. \\

\begin{lstlisting}[language=JSON,firstnumber=1,label={lst:cot_prompt},caption={CoT prompt for `local' multiple-choice question generation}]
system = [unchanged]

user = f"""
Here is the markdown content:
{markdown_content}

Based on the markdown content, generate {n} 'Single Best Answer' (SBA) questions that have only one correct answer out of five options. The correct answer should not be obvious and *should really require specific information from the source document to be able to be answered*. The incorrect answer options should not be so ridiculous or extreme that they are obviously wrong. The questions must be of the type: {qa_type_str}

The questions should be ones that could occur when a human wants information from the chatbot. They should be directly relevant to companies preparing their sustainability reports and reflect real-world scenarios that reporting teams might encounter.

To generate questions, follow these steps:
1. Select a list of one or more sentences/snippets of the markdown content that can be used to form an answer to a question. This will form the reference text. Remember this should be relevant to the human for drafting sustainability reports.
2. Write a question that requires the reader to understand the content of the selected text to answer correctly. The question should be based only on the selected text and should not require any additional information. Remember this should be the type of question a human would ask when drafting sustainability reports.
3. Write five answer options, one of which is correct and the other four are incorrect. The correct answer should be complete and taken verbatim from the selected section(s) of the markdown content.

Generate {n} unique and diverse QA pairs for the specified type and return them using the provided schema.
"""
\end{lstlisting}

\subsubsection{CoT + Few-shot Prompting}

The third approach introduces few-shot examples to the prompt using the sample questions curated from different sources as described in Section \ref{qa_design} (the list of sample questions can be found in Appendix \ref{lst:qa_structures}). The LLM is given 10-12 question styles and asked to follow the CoT question generation steps to create questions in the same style. It is asked to select the question styles at random and adapt any that do not directly make sense in the context of the provided markdown(s).

\begin{lstlisting}[language=JSON,firstnumber=1,label={lst:memprompt},caption={CoT + few-shot prompt for `local' multiple-choice question generation}]
system = [unchanged]

user = f"""
[unchanged]

Some example {qa_type} question structures are shown below. Please choose {n} structures at random but do not limit yourself to these types only. If some of these structures do not make sense given the content of the document, adapt them to the context as appropriate or choose ones that you think are appropriate.

Here are some question structures:
{question_structures_str}  
    
Generate {n} unique and diverse QA pairs for the specified type and return them using the provided schema.
"""
\end{lstlisting}


All of the prompting techniques were experimented with using four different temperature settings - 0.0, 0.2, 0.5, 1.0 - to test the impact of varying LLM creativity on the questions generated. 




\subsection{QA Evaluation}

\subsubsection{Faithfulness and Relevance}

After the QA pairs are generated, they are evaluated on three LLM-based metrics - faithfulness and relevance, as well as a custom `domain specificity' metric. Faithfulness is a measure of how accurately LLM-generated text is grounded in the provided context(s) \cite{es2023ragas}. A faithful question must be directly answerable from the given information without any need for external knowledge or inference, while a faithful answer is true based on the context. Relevance is a measure of how focused LLM-generated text is on the given content, containing as little irrelevant information as possible \cite{es2023ragas}. A relevant question precisely aligns with the specific context taken from the report, while a relevant answer addresses the actual question that was given. \\

Implementing QA evaluation is a more complex task than a simple evaluation of LLM-generated text since questions and answers that are not simply factual typically deviate in phrasing and/or semantics from the source document - this can be seen in the drastic drop of BLEU and ROUGE-L scores as questions span more industries (see Table \ref{tab:perindustry_results}). Typical LLM-based checks of faithfulness and relevance are designed to evaluate LLM-generated responses to queries, rather than the quality of a whole question-answer pair relative to a complete source document, making them unsuitable for a direct application to the QA evaluation task. This was tested using two popular open-source evaluation frameworks - LlamaIndex \cite{llamaindex_eval} and DeepEval \cite{deepeval} which use various prompting techniques, most notably chain of thought, to measure metrics on a binary (LlamaIndex) or continuous (DeepEval) scale. They were manually tested on a subset of questions and did not prove to be effective at identifying bad question quality (see Appendix \ref{lst:opensource_eval} for an example). The tools perform particularly poorly on evaluating the relevance of a question, as they rely on generic prompts that do not provide context on how to measure relevance as defined from the perspective of a company using the context to produce sustainability reports. Nevertheless, these tools are effective at evaluating simple (non-question) statements relative to source context \cite{deepeval_paper}. \\

This project combines open-source evaluation tools with a custom `question quality' evaluator into an evaluation pipeline shown in Figure \ref{fig:metrics_arch}. With this, the resulting tools are (1) able to evaluate full question-answer pairs and (2) targeted to the sustainability reporting domain. DeepEval (chosen as it provides scores on a continuous scale) is used to evaluate the faithfulness and relevance of the reference text relative to the entire report markdown, as well as of the answer relative to the reference text. A standard evaluation tool is suitable for these tasks because it evaluates text taken straight from the document without the need for additional context.\\

A custom evaluator is used to check the quality of the question itself relative to the reference text. This evaluator uses domain context and few-shot prompting to guide the LLM on grading the faithfulness and relevancy of the question. The custom evaluator used is shown in Listing \ref{lst:fewshot_prompt}. Note that, for relevance, the LLM is also asked to check whether the question relates to all the industries it is set to cover (e.g. if it is asked to create a question comparing two industries, but the question asks about only one of those industries, it is considered irrelevant). \\

A design choice is made to automatically exclude any QA pairs that have a reference text faithfulness or relevance below 0.7, to avoid falsely inflating the scores of the question and answer that are based on the reference text. This threshold was chosen from manual inspection of question quality to define what is generally deemed acceptable. The average faithfulness/relevancy score for the reference text, question, and answer is then taken. Note that the custom question evaluator works on a scale from 1 to 10 as, experimentally, this has helped the LLM be more nuanced in its scoring than using a 0-1 scale. Therefore, the DeepEval scores are re-based on a 1-10 scale before taking the average. \\


\begin{figure}[H]
    \centering
    \includegraphics[width = 0.9\textwidth]{figures/metrics_arch.pdf}
    \caption{Faithfulness and relevance QA evaluation pipeline}
    \label{fig:metrics_arch}
\end{figure}

\begin{lstlisting}[language=JSON,firstnumber=1,label={lst:fewshot_prompt},caption={Prompt for question faithfulness/relevance evaluation}]
system = "You are an expert question evaluator. Given a question, you assess its faithfulness and relevance relative to the reference text.

user = f"""
Critically evaluate the following question based on the provided context for the relevant {'industry' if len(industries) == 1 else 'industries'}. Be extremely rigorous and unforgiving in your assessment.

Faithfulness: Measure how accurately the question is grounded in the provided context{'s' if len(industries) > 1 else ''}. A faithful question must be directly answerable from the given information without any need for external knowledge or inference. Be extremely critical - even minor discrepancies or omissions should significantly impact the score.

Relevancy: Assess how precisely the question aligns with {'the' if len(industries) == 1 else 'each'} industry's specific context, challenges, and goals. A highly relevant question should directly address key aspects, metrics, or challenges unique to the industry. Be very strict - even slight deviations from industry-specific concerns should result in lower scores. ALSO: if the question does not cover all the industries required, it must be considered irrelevant and given a score of 1.

Score both metrics on a scale of 1 to 10, where:
1-2: Completely irrelevant or unfaithful
3-4: Major flaws in relevancy or faithfulness
5-6: Moderate issues, but still lacking
7-8: Generally good, with minor issues
9-10: Excellent, near-perfect alignment

Example Context:
"The Apparel, Accessories & Footwear industry faces significant sustainability challenges, particularly in raw materials sourcing. Key metrics include:
1. Percentage of raw materials third-party certified to environmental and/or social sustainability standards.
2. Priority raw materials: Description of environmental and social risks and/or hazards associated with priority raw materials used for products.
3. Environmental impacts in the supply chain: Percentage of (1) Tier 1 supplier facilities and (2) supplier facilities beyond Tier 1 that have completed the Sustainable Apparel Coalition's Higg Facility Environmental Module (Higg FEM) assessment or an equivalent environmental data assessment."

Examples:
Good Question  (Faithfulness: 10, Relevancy: 9):
"What percentage of raw materials in the Apparel, Accessories & Footwear industry should be third-party certified to environmental or social sustainability standards, according to the context?"
Explanation: This question directly addresses a specific metric mentioned in the industry context and can be answered solely based on the provided information.

Very Bad Question (Faithfulness: 1, Relevancy: 2):
"What is the average salary of a fashion designer in New York City?"
Explanation: This question is neither relevant to the industry's sustainability metrics nor answerable from the given context.
"""

\end{lstlisting}

\subsubsection{Domain Specificity}

Domain specificity is a newly introduced metric in this project to measure how specific the QA pair is to the sustainability reporting domain through the use of key concepts or trends found in the report. The question "What does Table 1 contain?" is generic, whereas "Can you provide me the unit of measure for the \{xxx\} metric in Table 1?" is more specific, as it gives more context for what the question should focus on from a domain perspective. Maintaining high question specificity ensures that the generated questions are representative of the domain and reduces the possibility of hallucination by guiding the LLM to the exact details it should be using from the source document to generate the question and answer. Furthermore, specific questions are less likely to have obvious answers that can be determined without reference to the source document (i.e. without RAG), so they are more useful for testing the system's true domain knowledge. \\

Domain specificity is measured on the question (plus answer options if it is an MCQ) relative to the whole markdown report (rather than just the reference text). This gives the LLM full context on what the report is about rather than a single text snippet, enabling it to identify keywords or ideas that are important in the report and therefore should be mentioned in the question. A scale of 1-10 is used, and the few-shot prompt is shown in Listing \ref{lst:qa_eval_prompt}. 


\begin{lstlisting}[language=JSON,firstnumber=1,label={lst:qa_eval_prompt},caption={Few-shot prompt for domain specificity}]
system: = ""You are an expert question evaluator."

user = f"""
    Full content:
    {full_context}

    Evaluate the specificity of the following question based on the provided context and compared to the highly specific question examples provided.
    Score the specificity on a scale of 1 to 10, where 1 is extremely broad or general and 10 is very specific. 
    
    {"Consider both the question itself and the answer options given." if is_multiple_choice else "Consider the question and the expected level of detail in the answer."}
    
    If a question requires a very specific answer directly from a specific sentence/part of the document, it is considered more specific. If a question can be answered in multiple ways or is broad, it is considered less specific.

    To help you, here are some example questions below:

    Highest specificity questions that score 10:
    "What is the unit of measure for the 'Percentage of raw materials third-party certified to an environmental and/or social sustainability standard, by standard' metric in the Apparel, Accessories & Footwear industry (as listed in the relevant table)?"

    High specificity questions that score 8:
    "What topics are covered in the 'Raw Materials Sourcing' section of the Apparel, Accessories & Footwear document, and what are the key takeaways for a company writing its sustainability report in this industry?"

    Medium specificity questions that score 6:
    "A company in the Household & Personal Products industry is facing water scarcity issues in multiple manufacturing locations. What is the most comprehensive approach to address this challenge in their sustainability report?"

    Low specificity questions that score 3:
    "How might the increasing focus on energy efficiency certifications in the appliance industry influence future regulatory trends and consumer behaviour?"

    Lowest specificity questions that score 1:
    "What broader implications does the industry's focus on energy management have for environmental sustainability?"

    Now the question to be evaluated:
    Question: {question_data['question']}
    """
\end{lstlisting}

\subsubsection{BLEU and ROUGE-L}

Finally, for free-text questions, BLEU and ROUGE-L are calculated for the answer relative to the reference text. ROUGE-L is chosen (over ROUGE-N) as it is designed to evaluate the semantic similarity and content coverage of text regardless of word order. ROUGE-N on the other hand simply evaluates the grammatical correctness and fluency of the generated text, which is not a requirement of the answers - succinct and factually accurate answers are preferred.\\

MCQ answers are not evaluated using BLEU and ROUGE-L. This is because low BLEU and/or ROUGE-L scores are not indicative of a wrong answer. For example, the correct answer option for an MCQ may be ``All of the above", which is not a phrase found in the text, and so will score 0 on both metrics despite being correct. An extension of this problem is that, if all answer options were to be evaluated instead of just the single correct one, multiple answer options may score highly as the options are designed to be similar enough that the correct answer is not obvious. It then becomes difficult to discern between correct and incorrect options from metric scores.

\subsection{Post-processing}

A selection of post-processing functions was developed to make the questions as high quality as possible (i.e. free of hallucination, relevant, and domain-specific) while making them sound human-like. For this project, human-like questions are defined as having some degree of vagueness, particularly with regards to the industry being asked about. To achieve this aim, two functions were introduced - quality improvement, and question generalisation. Additionally, MCQs are verified to ensure they have a single best answer (SBA) among the answer options. Finally, a similarity filter is introduced to remove identical question copies (in case these arise).

\subsubsection{Quality Improvement Function}


The first post-processing step involves passing questions that fail to meet a threshold on any of the three quality metrics through a quality improvement function that is designed to address the identified shortcomings. The flagged questions are passed along with their relevant industry markdown(s) through a prompt that instructs the LLM to slightly adjust the question on the required aspect. For example, if a question scored low on faithfulness, the prompt instructs the model to reformulate the question to more closely align with the factual information from the report. Similarly, for questions with low relevance scores, the prompt guides the model to focus more tightly on the specific sustainability reporting context from the report. This approach targets improvements to address each question's weaknesses while preserving its original intent and structure. The prompt used for this function is shown in Appendix \ref{lst:quality_improvement_prompt}.\\


To avoid any chances of hallucination by repeated prompting (as the LLM "forgets" the initial goal), this function is applied only once, and the question metrics are checked again. If the improved question still fails to meet the threshold, or meets the threshold for the original weak metric but now fails on another metric, it is discarded. Additionally, if the original question fails to meet the threshold on more than one metric, it is discarded, as it would not provide the LLM enough ``starting material" for the improvement function to reformulate the question while preserving its initial intent. \\

There is no generally accepted choice of threshold for faithfulness and relevance, as thresholds are designed to be adapted to the aim of the LLM application. In principle, any score above 0.5 would suggest that the LLM is better than random and may be deemed acceptable. For QA evaluation, the thresholds are determined based on the results of the experiments shown in Table \ref{tab:qa_evals_methods} of Section \ref{qa_experiment_results}. 


\subsubsection{Question Generalisation Function}

The second post-processing function generalises questions that are ``too specific to be human". It alters any mentions of very specific industry names into more generic industry ``areas", mimicking the more natural conversational-style queries that a user would ask. For example, a question that originally asked ``What does the IFRS require companies to disclose regarding energy management for the Processed Foods industry?" might be rephrased to ``What does the IFRS require food companies to disclose regarding energy management?". This generalisation increases the difficulty of the questions, requiring the RAG system to have a deeper understanding of the content rather than relying on keyword matching. The efficacy of this function is tested qualitatively. The prompt used for this function is shown in Appendix \ref{lst:generalisation_prompt}.


\subsubsection{Single Best Answer Verification}

The questions are filtered to remove any MCQs that have more than one correct answer (i.e. that do not pass the SBA test). This is checked by giving a strong LLM (Claude 3.5 Sonnet) the full markdown context and asking it to choose all correct answer options based on it. If more than one option is chosen, the question fails the SBA test and is discarded. Finally, a similarity filter is created to remove identical questions. The prompt used for this function is shown in Appendix \ref{lst:sba_prompt}.

\subsubsection{Similarity Filter}

Finally, to avoid question repetition, the questions are passed through a similarity filter to remove identical questions. The nature of the question generation pipeline, which uses few-shot prompting to create very specific questions, means that many questions are likely to be very similar in their structures. As such, the filter is set to a threshold of 0.99\% similarity, and questions above this are discarded.

\section{Results}



\subsection{Question Embeddings}

The embeddings for a sample of local free-text questions from six industry categories are mapped using t-SNE (t-distributed Stochastic Neighbour Embedding), a dimensionality reduction technique for visualisation of high-dimensional data. For each industry category, questions are randomly selected from each industry. The questions used are generated using the CoT + few-shot method. The results are shown in Figure \ref{fig:question_embeddings}.

\begin{figure}[H]
    \centering
    \includegraphics[width = \textwidth]{figures/question_embeddings_tsne_sample.pdf}
    \caption{Sample t-SNE visualisation of question embeddings, coloured by industry group}
    \label{fig:question_embeddings}
\end{figure}




\subsection{Experimental Results}\label{qa_experiment_results}

The three question generation methods outlined in Section \ref{QA_gen} are used to generate a sample of questions for quality evaluation experiments. For each method, the questions are generated according to the settings described in Section \ref{qa_design}:
\begin{itemize}
    \item Question span: For local questions, eight industries are chosen at random, while for cross-industry questions, eight industry pairs are chosen out of the LLM-generated industry pairs (generation of these pairs is described in Section \ref{qa_design}). (Note that a subset of industry (pairs) is used for experimentation due to cost reasons. The full QA dataset will be generated on all industries and industry pairs.)
    \item Question complexity: Single- and multi-hop questions are generated.
    \item Answer structure: MCQ and free-text questions are generated.
    \item Temperature: All settings are tried with four temperatures: 0.0, 0.2, 0.5, 1.0.
\end{itemize}

To conduct experiments (before running the final question generation pipeline), 32 questions of each type and temperature are created and evaluated. Tables \ref{tab:qa_evals_methods}, \ref{tab:qa_evals_methods_free}, and \ref{tab:temperature-comparison} below show the average scores of question evaluations across different methods and temperatures.


\begin{table}[H]
\centering
\begin{tabular}{lcccccc}
\hline
              & \multicolumn{3}{c}{MCQ Local}                                                & \multicolumn{3}{c}{MCQ Cross-industry}                                       \\ \cline{2-7} 
              & Baseline & CoT     & \begin{tabular}[c]{@{}c@{}}CoT +\\ Few-shot\end{tabular} & Baseline & CoT     & \begin{tabular}[c]{@{}c@{}}CoT +\\ Few-shot\end{tabular} \\ \cline{2-7} 
No. Questions & 64       & 64      & 64                                                       & 64       & 64      & 64                                                       \\
Faithfulness  & 6.17     & 7.25    & \textbf{9.45}                                                     & 5.58     & 6.06    & \textbf{7.69}                                                     \\
Relevance     & 6.25     & 7.78    & \textbf{9.68}                                                     & 3.46     & 5.93    & \textbf{7.86}                                                     \\
Specificity   & 5.61     & 7.47    & \textbf{9.24}                                                     & 5.04     & 6.31    & \textbf{7.69}                                                     \\
SBA (\%)      & 67.19\%  & 81.25\% & \textbf{92.19\%}                                                  & 68.76\%  & 75.01\% & \textbf{92.19\%}                                                  \\ \hline
\end{tabular}
\caption{Experimental results per method, averaged over single- and multi-hop MCQs. The experiment was done with questions generated on a selection of 8 industries (for local) and 8 industry pairs (for cross-industry) with a temperature of 0.5.}
\label{tab:qa_evals_methods}
\end{table}


\begin{table}[H]
\centering
\begin{tabular}{lcccccc}
\hline
              & \multicolumn{3}{c}{Free-text Local}                                       & \multicolumn{3}{c}{Free-text Cross-industry}                              \\ \cline{2-7} 
              & Baseline & CoT  & \begin{tabular}[c]{@{}c@{}}CoT +\\ Few-shot\end{tabular} & Baseline & CoT  & \begin{tabular}[c]{@{}c@{}}CoT +\\ Few-shot\end{tabular} \\ \cline{2-7} 
No. Questions & 64       & 64   & 64                                                       & 64       & 64   & 64                                                       \\
Faithfulness  & 5.68     & 7.12 & \textbf{8.75}                                                    & 5.09     & 5.89 & \textbf{7.13}                                                     \\
Relevance     & 5.93     & 7.30  & \textbf{9.56}                                                     & 3.34     & 5.82 & \textbf{7.32}                                                     \\
Specificity   & 5.04     & 7.01 & \textbf{9.06}                                                     & 5.07     & 6.25 & \textbf{7.44}                                                     \\
BLEU          & 0.31     & 0.68 & \textbf{0.95}                                                     & 0.12     & 0.28 & \textbf{0.36}                                                     \\
ROUGE-L       & 0.35     & 0.71 & \textbf{0.95}                                                     & 0.29     & 0.41 & \textbf{0.49}                                                     \\ \hline
\end{tabular}
\caption{Experimental results per method, averaged over single- and multi-hop free-text questions. Experiment done with questions generated on a selection of 8 industries (for local) and 8 industry pairs (for cross-industry) with a temperature of 0.5.}
\label{tab:qa_evals_methods_free}
\end{table}


Tables \ref{tab:qa_evals_methods} and \ref{tab:qa_evals_methods_free} show significant improvements in question quality across the three methods for both MCQs and free-text questions - there is a steady increase in scores for faithfulness, relevance, and consistency, as well as, for MCQs, the percentage of questions that have a single best answer. On average (across faithfulness, relevance, and specificity), local and cross-industry MCQs generated by the CoT + few-shot method are 57.6\% and 72.5\% better than the baseline method, respectively. For free-text questions, these respective improvements are 65.0\% and 68.7\%. \\

The CoT + few-shot method yields the best questions for both local and cross-industry spans, though cross-industry questions consistently score lower than local questions, not scoring above 8 on average. Cross-industry free-text questions also have much lower BLEU and ROUGE-L scores on average than local questions. Free-text questions score slightly lower than MCQs in general, and the extent of the difference in quality varies - free-text local questions, for example, score 8.75 on faithfulness, which is significantly less than the 9.45 scored by MCQs, though the difference between their relevance and specificity is much smaller.


\begin{table}[H]
\centering
\small
\begin{tabular}{lcccccccc}
\hline
\multirow{2}{*}{} & \multicolumn{4}{c}{MCQ}              & \multicolumn{4}{c}{Free-text} \\ \cline{2-9} 
                  & 0.0     & 0.2     & 0.5     & 1.0     & 0.0    & 0.2   & 0.5   & 1.0   \\ \cline{2-9} 
Faithfulness      & \textbf{8.83}    & 8.58    & 8.57    & 8.04    & 7.21   & 6.93  & \textbf{7.94}  & 7.63  \\
Relevance         & \textbf{9.10}    & 8.92    & 8.77    & 8.21    & 7.95   & 7.68  & \textbf{8.44}  & 8.38  \\
Specificity       & \textbf{8.50}    & 8.36    & 8.47    & \textbf{8.50}    & 8.08   & 8.13  & \textbf{8.25}  & 8.00  \\
BLEU              & --      & --      & --      & --      & 0.49   & 0.42  & \textbf{0.66}  & \textbf{0.66} \\
ROUGE-L           & --      & --      & --      & --      & 0.60   & 0.53  & 0.72  & \textbf{0.75}  \\
SBA (\%)          & 92.19\% & \textbf{92.97\%} & 92.19\% & 87.50\% & -      & -     & -     & -     \\ \hline
\end{tabular}
\caption{Temperature sensitivity for free-text and MCQ generated using the CoT + Few-shot method on a selection of 8 industries (for local questions) and 8 industry pairs (for cross-industry questions). Results averaged across single-/multi-hop and local/cross-industry questions. SBA (\%) stands for the percentage of questions that have a `single best answer'.}
\label{tab:temperature-comparison}
\end{table}

Table \ref{tab:temperature-comparison} shows that question quality varies little with LLM temperature, though small trends are observed. MCQs tend to score better with lower temperatures, though this is not the case for specificity and SBA percentages, which show no real trend. Free-text questions seem to score best with higher temperatures - faithfulness, relevance, and specificity are highest with a temperature of 0.5, while ROUGE-L is highest with a temperature of 1.

\subsection{Post-processing}

The efficacy of the quality improvement function is tested on the questions generated using the CoT + few-shot method. There are no standard accepted thresholds for faithfulness and relevance within LLM text generation literature. Generally, any score above 0.5 can be deemed to be acceptable, as more than half the generated text is truthful and related to the context. For the quality improvement function, we therefore base our threshold selection on the experimental results for CoT + few-shot shown in Tables \ref{tab:qa_evals_methods} and \ref{tab:qa_evals_methods_free}. We set a conditional threshold - local questions must score at least 9 on all metrics, while cross-industry questions must score at least 7 (both metrics are out of 10). \\

The post-processing functions are cost-intensive operations. In particular, the quality improvement function involves passing the entire question and source document(s) through the quality improvement function as well as re-evaluating the question on all metrics. Therefore, the application of the quality improvement and generalisation functions to the entire dataset is left as future work. These functions were tested qualitatively, with two examples shown below.
\newpage
\begin{lstlisting}[language=JSON,firstnumber=1,label={lst:quality_improvement},caption={Application of quality improvement post-processing on a cross-industry MCQ}]
Original question:
"question": "What are the metrics for Coal Operations and Apparel, Accessories & Footwear industries.",
    "optionA": "Production of thermal coal, Production of metallurgical coal, Number of Tier 1 suppliers and suppliers beyond Tier 1",
    "optionB": "Production of thermal coal, Number of Tier 1 suppliers and suppliers beyond Tier 1",
    "optionC": "Production of metallurgical coal, Number of Tier 1 suppliers and suppliers beyond Tier 1",
    "optionD": "Production of thermal coal, Production of metallurgical coal",
    "optionE": "Number of Tier 1 suppliers and suppliers beyond Tier 1",
"answer": "A"
Original specificity score: 5

Improved question:
"question": "List the activity metrics for Coal Operations and Apparel, Accessories & Footwear industries.",
    "optionA": "Production of thermal coal, Production of metallurgical coal, Number of Tier 1 suppliers and suppliers beyond Tier 1",
    "optionB": "Production of thermal coal, Number of Tier 1 suppliers and suppliers beyond Tier 1",
    "optionC": "Production of metallurgical coal, Number of Tier 1 suppliers and suppliers beyond Tier 1",
    "optionD": "Production of thermal coal, Production of metallurgical coal",
    "optionE": "Number of Tier 1 suppliers and suppliers beyond Tier 1",
"answer": "A"
New specificity score: 9
\end{lstlisting}

\begin{lstlisting}[language=JSON,firstnumber=1,label={lst:generalisation_example},caption={Application of generalisation post-processing on a cross-industry MCQ}]
"original question": "What is the code for the 'Gross global Scope 1 emissions, percentage covered under emissions-limiting regulations' metric in the Coal Operations industry?",
    "optionA": "EM-CO-110a.2",
    "optionB": "EM-CO-140a.1",
    "optionC": "EM-CO-110a.1",
    "optionD": "EM-CO-420a.3",
    "optionE": "EM-CO-000.A",
"answer": "C",

"refined question": "What is the code for the 'Gross global Scope 1 emissions, percentage covered under emissions-limiting regulations' metric in the fossil fuel operations industry?"
\end{lstlisting}

\subsection{Final Question Evaluation}


The final questions are generated using the CoT + few-shot method using all industries and industry pairs. 1,179 questions are generated using 68 industries (for local questions) and 44 industry pairs (for cross-industry questions) with question number per type shown in Table \ref{tab:final_qa_numbers}. The average quality results of these questions is shown in Tables \ref{tab:perindustry_results} and \ref{tab:per_hop}.

\begin{table}[H]
\centering
\begin{tabular}{lcccc}
\hline
           & \multicolumn{2}{c}{MCQ}                               & \multicolumn{2}{c}{Free-text}                         \\ \hline
           & \multicolumn{1}{l}{Local} & \multicolumn{1}{l}{Cross-industry} & \multicolumn{1}{l}{Local} & \multicolumn{1}{l}{Cross-industry} \\ \cline{2-5} 
Single-hop & 272                       & 93                        & 136                       & 88                        \\
Multi-hop  & 272                       & 94                        & 136                       & 88                        \\ \hline
\end{tabular}
\caption{Number of questions generated, per type, using the CoT + few-shot method.}
\label{tab:final_qa_numbers}
\end{table}


\begin{table}[H]
\centering
\begin{tabular}{lcccc}
\hline
              & \multicolumn{2}{c}{MCQ}  & \multicolumn{2}{c}{Free-text} \\ \cline{2-5} 
              & Local   & Cross-industry & Local     & Cross-industry    \\ \cline{2-5} 
No. Questions & 544     & 187            & 272       & 176               \\
Faithfulness  & \textbf{9.40}    & 7.64           & 8.32      & 6.42              \\
Relevance     & \textbf{9.68}    & 7.65           & 8.78      & 6.88              \\
Specificity   & \textbf{9.06}    & 7.15           & 8.80      & 8.22              \\
BLEU          & -       & -              & \textbf{0.93}      & 0.34              \\
ROUGE-L       & -       & -              & \textbf{0.95}      & 0.46              \\
SBA (\%)      & \textbf{92.47\%} & 91.45\%        & -         & -                 \\ \hline
\end{tabular}
\caption{Results per question span. Results are averaged across single- and multi-hop. Questions are generated using a temperature of 0.5. SBA (\%) stands for the percentage of questions that have a `single best answer'.}
\label{tab:perindustry_results}
\end{table}

\begin{table}[H]
\centering
\begin{tabular}{lcccc}
\hline
              & \multicolumn{2}{c}{MCQ} & \multicolumn{2}{c}{Free-text} \\ \cline{2-5} 
              & Single-hop  & Multi-hop & Single-hop     & Multi-hop    \\ \cline{2-5} 
No. Questions & 365         & 366       & 224            & 224          \\
Faithfulness  & \textbf{9.06}        & 7.98      & 7.52           & 7.23         \\
Relevance     & \textbf{8.84}        & 8.49      & 7.79           & 7.86         \\
Specificity   & 8.20         & 8.00         & \textbf{8.65}           & 8.35         \\
BLEU          & -           & -         & \textbf{0.76}           & 0.51         \\
ROUGE-L       & -           & -         & \textbf{0.82}           & 0.60          \\
SBA (\%)      & \textbf{93.43\%}     & 90.44\%   & -              & -            \\ \hline
\end{tabular}
\caption{Results per question complexity. Results are averaged across local and cross-industry. Questions are generated using a temperature of 0.5. SBA (\%) stands for the percentage of questions that have a `single best answer'.}
\label{tab:per_hop}
\end{table}

The average quality score (across faithfulness, relevance, and specificity) across all question types is 8.16. Local questions score 9 on average, while cross-industry ones score 7.32. Multiple-choice questions score 8.42 on average, while free-text questions score 7.9. Single-hop questions score 8.34 while multi-hop score 8 on average. \\

Based on these results, two general trends are observed: (1) `simpler' questions (i.e. local and single-hop) score better than more complex questions (i.e. cross-industry and multi-hop), and (2) MCQs score better than free-text questions. Among both MCQs and free-text questions, there is a bigger difference in quality between local and cross-industry questions than between single- and multi-hop questions. On average, single-hop MCQs score 6.7\% better than multi-hop MCQs, while local MCQs score 25.4\% better than cross-industry MCQs. For free-text questions, these differences are 2.2\% and 21.4\%, respectively. 


\section{Discussion}

\subsection{Question Embeddings}

Figure \ref{fig:question_embeddings} shows two key trends - the question embeddings across different industry groups are generally distinct, while the embeddings within industry groups are similar. This means that, while it is easy to distinguish between the industry groups a query relates to, it is more difficult to distinguish between industries within the group based on the query.



\subsection{Experimental Results} \label{qa_experiments_discussion}


Each question generation method offered additional advantages for question quality. Table \ref{tab:qa_ablation} below displays the incremental improvements attributed to each method relative to the previous one. Note that the percentages for `Few-shot vs CoT' are calculated on a larger baseline than the percentages for `CoT vs baseline', so although the contributions of each additional method may seem similar, the absolute contributions of Few-shot are larger. 

\begin{table}[H]
\centering
\begin{tabular}{lcccc}
\hline
                     & \multicolumn{2}{c}{Local}                                                                                              & \multicolumn{2}{c}{Cross-industry}                                                                                    \\ \cline{2-5} 
                     & \begin{tabular}[c]{@{}c@{}}CoT \\ vs baseline\end{tabular} & \begin{tabular}[c]{@{}c@{}}Few-shot\\ vs CoT\end{tabular} & \begin{tabular}[c]{@{}c@{}}CoT\\ vs baseline\end{tabular} & \begin{tabular}[c]{@{}c@{}}Few-shot\\ vs CoT\end{tabular} \\ \cline{2-5} 
\multicolumn{1}{c}{} & \multicolumn{4}{c}{MCQ}                                                                                                                                                                                                                        \\ \cline{2-5} 
Faithfulness         & 17.50\%                                                    & 30.34\%                                                   & 8.60\%                                                    & 26.90\%                                                   \\
Relevance            & 24.48\%                                                    & 24.42\%                                                   & 71.39\%                                                   & 32.55\%                                                   \\
Specificity          & 33.16\%                                                    & 23.69\%                                                   & 25.20\%                                                   & 21.87\%                                                   \\
\textbf{Average}     & \textbf{25.05\%}                                           & \textbf{26.15\%}                                          & \textbf{35.06\%}                                          & \textbf{27.10\%}                                          \\ \cline{2-5} 
\multicolumn{1}{c}{} & \multicolumn{4}{c}{Free-text}                                                                                                                                                                                                                  \\ \cline{2-5} 
Faithfulness         & 25.35\%                                                    & 22.89\%                                                   & 15.72\%                                                   & 21.05\%                                                   \\
Relevance            & 23.10\%                                                    & 30.96\%                                                   & 74.25\%                                                   & 25.77\%                                                   \\
Specificity          & 39.09\%                                                    & 29.24\%                                                   & 23.27\%                                                   & 19.04\%                                                   \\
\textbf{Average}     & \textbf{29.18\%}                                           & \textbf{27.70\%}                                          & \textbf{37.75\%}                                          & \textbf{21.96\%}                                          \\ \hline
\end{tabular}
\caption{Percentage improvements in question quality across different methods. Calculated based on data in Tables \ref{tab:qa_evals_methods} and \ref{tab:qa_evals_methods_free}}
\label{tab:qa_ablation}
\end{table}


The percentage contributions of each additional method to question quality are mixed. Integrating CoT reasoning into the prompt yields average quality improvements in the range of 25\% to 37.8\%, with cross-industry questions benefiting most from this method. This suggests that guiding the LLM through a three-step process for creating questions helps it to navigate through multiple documents and isolate relevant chunks from each of them to form a question. \\

Few-shot prompting yields additional benefits ranging from 22\% to 27.7\%, and these benefits are mixed across local and cross-industry questions. However, in absolute terms, Tables \ref{tab:qa_evals_methods} and \ref{tab:qa_evals_methods_free} show that local questions are higher quality than cross-industry ones, scoring an average of 9.29 (across both MCQ and free-text) versus 7.52. A similar observation is made when comparing MCQs with free-text questions - though they both benefit equally from the advanced methods, MCQs are (albeit to a lesser extent) higher quality in absolute terms, scoring an average of 8.60 versus 8.21. \\

These observations suggest that there exists a limit to LLMs' capabilities of generating questions using prompt methods. It may struggle with using multiple documents to create questions, as well as ensuring that free-text questions are as accurate as multiple-choice ones. \\

In the following subsections, we discuss the quality issues encountered with each method, providing examples that have contributed to their respective quality scores.

\subsubsection{Baseline Prompting} 

Several issues are encountered in the baseline question generation method, as it does not guide the LLM on how to create questions and does not provide any example structures it should aim for. Four examples are shown below to highlight the biggest issues with the baseline approach (in addition to standard faithfulness problems).

\begin{lstlisting}[language=JSON,firstnumber=1,label={lst:memprompt},caption={Question Example 1: Local single-hop free-text}]
"question": "How many incidents of non-compliance with water quality permits, standards, and regulations have occurred for coal operations?",
"answer": "The entity shall disclose the total number of instances of non-compliance, including violations of a technology-based standard and exceedances of quantity and/or quality-based standards.",
\end{lstlisting}

\textit{Issue \#1: Questions are not related to sustainability reporting.} This is depicted in the example above, which displays phrasing that is not in line with user queries about how to report on the metrics. This in turn makes the answer, which is in fact correct based on the source, irrelevant to the question. A more relevant phrasing for this question would be ``How should I disclose incidents of non-compliance with water quality permits, standards, and regulations in the Coal Operations industry?". This issue is captured and measured by the relevance metric.

\begin{lstlisting}[language=JSON,firstnumber=1,label={lst:memprompt},caption={Question Example 2: Local single-hop MCQ}]
"question": "According to the topic summary, how can companies in this industry be better positioned to meet consumer demand and reduce future liabilities?",
"optionA": "By focusing solely on product innovation",
"optionB": "By reducing production costs at all costs",
"optionC": "By addressing environmental concerns and enhancing product take-back",
"optionD": "By increasing marketing budgets",
"optionE": "By outsourcing all manufacturing processes",
"answer": "C. By addressing environmental concerns and enhancing product take-back",
\end{lstlisting}

\textit{Issue \#2: Questions are vague, making the correct answer obvious.} In the example above, there is no specific reference to what piece of the document, and what industry, the question refers to. Nevertheless, the question is easily answerable even without knowing the context, as options A, B, D, and E are very obviously wrong. This issue is measured by the domain specificity metric, which shows the lowest scores of all metrics for the baseline method, indicating that this is a particular issue with this method.


\begin{lstlisting}[language=JSON,firstnumber=1,label={lst:memprompt},caption={Question Example 3: Cross-industry single-hop MCQ}]
"question": "Which of the following is a key strategy home builders can use to mitigate environmental legal risks?",
"optionA": "Ensuring all homes are certified to a third-party green building standard",
"optionB": "Implementing corrective actions in response to any legal proceedings related to environmental regulations",
"optionC": "Focusing development only in regions with low baseline water stress",
"optionD": "Both B and C",
"optionE": "All of the above",
"answer": "B",
"industries": ["b8-construction-materials","b35-home-builders"]
\end{lstlisting}

\textit{Issue \#3: Cross-industry questions do not cover all required industries.} In the example above, not all the required industries are covered by the question. This issue is captured and measured by the relevance metric, contributing to its low score.

\begin{lstlisting}[language=JSON,firstnumber=1,label={lst:memprompt},caption={Question Example 4: Cross-industry single-hop MCQ}]
"question": "Which of the following is a key risk for home builders related to climate change adaptation?",
    "optionA": "Increased costs associated with flood insurance for homes in 100-year flood zones",
    "optionB": "Difficulty obtaining permits for developments in water-stressed regions",
    "optionC": "Reduced long-term demand for homes in volatile climate regions",
    "optionD": "Need for more robust construction materials to withstand extreme weather events",
    "optionE": "Higher energy costs for cooling homes in regions experiencing rising temperatures",
"answer": "A"
"true correct answers": A, B, C
\end{lstlisting}


\textit{Issue \#4: Questions do not have a single best answer.} The example question above has three answer options that are correct, though the LLM has selected only one. Multiple-choice questions often have multiple correct answers listed in the options. This is driven by questions being vague, inviting multiple ways of answering them. The LLM simply chooses one of its generated answer options to be the `correct' one, however several of the other given options can also be deemed correct based on the source information. Having multiple correct answer options artificially deflates the accuracy of RAG systems tested on the MCQs. The lack of a single best answer affects one in three questions generated by the baseline method (see Table \ref{tab:qa_evals_methods}).


\subsubsection{CoT Prompting}

The CoT approach addresses some of the issues encountered in the baseline approach, and the highest quality improvements are seen for cross-industry questions, where the quality improved by an average of 36.3\% across all metrics (the average improvement for local questions is 27\%). Most notably, cross-industry questions become 72.8\% more relevant and 24.2\% more specific. Nevertheless, in absolute terms, these questions do not achieve a satisfactory level, with scores being only slightly above average. Some sample questions are displayed below to highlight persistent issues with question quality.

\begin{lstlisting}[language=JSON,firstnumber=1,label={lst:cot_example1},caption={Question Example 1: Local single-hop free-text}]
"question": "What is the percentage of the company's Scope 1 emissions that are covered under emissions-limiting regulations?",

"answer": "The percentage shall be calculated as the total amount of gross global Scope 1 GHG emissions (CO2-e) that are covered under emissions-limiting regulations divided by the total amount of gross global Scope 1 GHG emissions (CO2-e).",
\end{lstlisting}

\textit{Issue \#1: Questions are vague.} This question has no reference to what type of industry it is referring to and is therefore applicable to all industries. It would be nearly impossible for any RAG system to get this question correct with no further context.


\begin{lstlisting}[language=JSON,firstnumber=1,label={lst:cot_example2},caption={Question Example 2: Cross-industry single-hop MCQ}]
"question": "Which industry discusses the percentage of products by revenue that contain IEC 62474 declarable substances?",
    "optionA": "Metals & Mining",
    "optionB": "Electrical & Electronic Equipment",
    "optionC": "Both industries",
    "optionD": "Neither industry",
    "optionE": "Not specified",
    "answer": "B"
    "industries": ["b10-metals-and-mining","b49-electrical-and-electronic-equipment"]
\end{lstlisting}

\textit{Issues \#2 and \#3: Questions do not cover all industries required, and are not representative of real-life user queries  for sustainability reporting.} This question is supposed to be comparative between two different industries. Instead, it lists industries as multiple-choice options, a format that is not user-typical. The question itself doesn't refer to both industries it is given. These issues contribute to lower relevancy and specificity. 




\subsubsection{CoT + Few-shot Prompting}

This is the final selected method for question generation, which displays the best quality metric scores and the highest percentage of questions that pass the SBA test out of all the methods. These improvements can be attributed to the precise wording and structuring of the few-shot examples provided to the LLM, as described in Section \ref{qa_design}. \\

Below are some sample questions that depict high-quality questions that are representative of real-life user queries.

\begin{lstlisting}[language=JSON,firstnumber=1,label={lst:few_example2},caption={Question Example 1: Local multi-hop MCQ}]
{
"question": "What is the category of the metric 'Total landfill gas generated' in the Waste Management industry and how should this metric be calculated?",
      "optionA": "Quantitative; calculated in millions of British Thermal Units (MMBtu)",
      "optionB": "Discussion and Analysis; calculated based on engineering estimates",
      "optionC": "Quantitative; calculated in metric tons (t)",
      "optionD": "Qualitative; based on expert interviews",
      "optionE": "Quantitative; calculated in gigajoules (GJ)",
"answer": "A",
"reference_text": [
    "1 The entity shall disclose (1) the total amount, in millions of British Thermal Units (MMBtu), of landfill gas generated from its owned or operated facilities.",
    "1.1 Landfill gas is defined as gas produced as a result of anaerobic decomposition of waste materials in the landfill."
      ],
"pages": ["401"]
\end{lstlisting}

\begin{lstlisting}[language=JSON,firstnumber=1,label={lst:few_example1},caption={Question Example 2: Cross-industry multi-hop MCQ}]
"question": "Give me the codes for all the 'Quantitative' metrics in Table 1 for Insurance and Real Estate Services industries.",
      "optionA": "FN-IN-410b.1, FN-IN-450a.1, FN-IN-450a.2, IF-RS-410a.1, IF-RS-410a.2, IF-RS-410a.3",
      "optionB": "FN-IN-410b.1, FN-IN-450a.1, FN-IN-450a.2, IF-RS-410a.1",
      "optionC": "FN-IN-410b.1, FN-IN-450a.1, IF-RS-410a.1, IF-RS-410a.2",
      "optionD": "FN-IN-450a.1, FN-IN-450a.2, IF-RS-410a.1, IF-RS-410a.3",
      "optionE": "FN-IN-410b.1, FN-IN-450a.1, IF-RS-410a.2, IF-RS-410a.3",
"answer": "A",
"reference_text": [
    "| Policies Designed to Incentivize Responsible Behavior | Net premiums written related to energy efficiency and low carbon technology | Quantitative | Reporting currency | FN-IN-410b.1 |",
    "| Environmental Physical Risk Exposure | Probable Maximum Loss (PML) of insured products from weather-related natural catastrophes 21 | Quantitative | Reporting currency | FN-IN-450a.1 |",
    "| Environmental Physical Risk Exposure | Total amount of monetary losses attributable to insurance payouts from (1) modeled natural catastrophes and (2) non-modeled natural catastrophes, by type of event and geographic segment (net and gross of reinsurance) 22 | Quantitative | Reporting currency | FN-IN-450a.2 |",
    "| Sustainability Services | Revenue from energy and sustainability services 55 | Quantitative | Reporting currency | IF-RS-410a.1 |",
    "| Sustainability Services | (1) Floor area and (2) number of buildings under management provided with energy and sustainability services | Quantitative | Square feet (ft\u00b2), Number | IF-RS-410a.2 |",
    "| Sustainability Services | (1) Floor area and (2) number of buildings under management that obtained an energy rating | Quantitative | Square feet (ft\u00b2), Number | IF-RS-410a.3 |"],
"pages": ["Page 156","Page 391"],
\end{lstlisting}




Nevertheless, these questions are not perfect, and there are limits to what types of questions can be generated using advanced prompting methods. One such limit was encountered when trying to generate questions spanning more than two industries, where the LLM was not able to capture information from all industries to create a question. Instead, it selected one or two of the given industries and made a question based on that. This is not a significant problem for our application, as such complicated questions may be less likely to occur in our domain, but presents some room for potential future work on question-generation methods.


\subsubsection{Temperature}

Table \ref{tab:temperature-comparison} shows the evaluation of questions generated using the CoT + few-shot method with different LLM temperature settings. The differences in evaluation scores across temperature values are small, showing that LLM temperature does not have a strong impact on question quality. This is because the LLM is guided to follow certain question styles through few-shot prompting, removing much scope for LLM creativity (which is what temperature dictates). Given the small variance, and to ensure all questions are created with consistent settings, the decision is made to use a temperature of 0.5 for all question generation. 

\subsection{Final Question Evaluation}


The results in Tables \ref{tab:perindustry_results} and \ref{tab:per_hop} show that the CoT + few-shot generation method produces high-quality questions across all formats. Nevertheless, table \ref{tab:final_ablation} shows that question span is still a significant driver of question quality, as is question complexity, although to a much lesser extent. This highlights that LLMs possess some ability to reason in multiple steps but struggle more with utilising data from multiple source documents. A manual evaluation of a subset of the questions show that they are high quality (as depicted in the examples shown in Section \ref{qa_experiments_discussion}), though conducting human evaluation on the entire dataset is left as future work.

\begin{table}[H]
\centering
\begin{tabular}{lcc}
\hline
             & Question Span           & Question Complexity     \\ \hline
             & Local vs Cross-industry & Single-hop vs Multi-hop \\ \cline{2-3} 
             & \multicolumn{2}{c}{MCQ}                           \\ \cline{2-3} 
Faithfulness & 23.04\%                 & 13.53\%                 \\
Relevance    & 26.54\%                 & 4.12\%                  \\
Specificity  & 26.71\%                 & 2.50\%                  \\
\textbf{Average}      & \textbf{25.43\% }                &\textbf{ 6.72\%}                  \\ \cline{2-3} 
             & \multicolumn{2}{c}{Free-text}                     \\ \cline{2-3} 
Faithfulness & 29.60\%                 & 4.01\%                  \\
Relevance    & 27.62\%                 & -0.89\%                 \\
Specificity  & 7.06\%                  & 3.59\%                  \\
\textbf{Average}      & \textbf{21.42\% }                & \textbf{2.24\%}                  \\ \hline
\end{tabular}
\caption{Percentage differences in question quality of simpler questions relative to more difficult questions, based on question span and question complexity. Calculated based on data in Tables \ref{tab:perindustry_results} and \ref{tab:per_hop}.}
\label{tab:final_ablation}
\end{table}

For the final dataset, 116 questions are removed as they do not have a single best answer, leaving 1,063 questions to be used for evaluating RAG systems.



\chapter{Domain-Specific Assistant}


This chapter presents the techniques implemented to design a useful domain-specific knowledge assistant. We start by defining the knowledge scope of the chatbot. Then, we describe the methods investigated to design such a chatbot, including RAG, fine-tuning, and classification techniques, and propose two novel architectures for the chatbot - a RAG-based pipeline and an LLM-based pipeline. The dataset developed in Chapter 3 is used along with the IFRS markdown texts to train the classification models, fine-tune LLMs, and conduct experiments to evaluate all techniques. We present the results of the experiments as well as the final performance of the proposed architectures on the QA dataset.


\section{Knowledge Scope}

The knowledge scope for the assistant is defined by the IFRS Sustainability Reporting standards detailed in Section \ref{data_collection}. To be `useful' for answering user queries based on this knowledge, the assistant must be able to retrieve, comprehend, and reason over this data to answer questions, while being constrained from answering out-of-scope queries.

\section{Methods} 

First, an initial investigation is conducted into different RAG techniques by conducting experiments using the QA dataset. Then, an LLM is fine-tuned and tested on the QA dataset. Next, several methods are explored to design an industry classifier, including training an NLP model, two machine learning (ML) methods, a multi-layer perceptron (MLP), and a prompt-based LLM classifier. Finally, two chatbot architectures are proposed - a RAG pipeline and an LLM-based pipeline.

\subsection{Retrieval Augmented Generation (RAG)}

This section explores a selection of RAG techniques for the domain-specific assistant. The steps of a basic RAG pipeline (shown in Figure \ref{fig:basic_rag} of Chapter 2) are outlined below, alongside the alternative methods tested for each step, which are described in detail in the following sections.

\begin{enumerate}
    \item \textbf{Indexing:} text is extracted from the documents and broken down into smaller chunks, which are embedded into vector representation using an embedding model and stored in a vector DB. \\
    \textit{Chunking methods tested:} fixed-size chunks of 256, 512, and 1024 tokens, page, rolling window, semantic chunking, and custom markdown chunking.
    \item \textbf{Retrieval:} When a user asks a question, the query is embedded (using the same model as the data) into a vector and is compared to the stored vectors to find the most similar matches. In the basic case, the query is embedded as is (with no transforms applied), and KNN is used as the retrieval technique.\\
    \textit{Retrieval techniques tested:} KNN and hybrid.\\
    \textit{Query Transformation techniques tested:} HyDE, multi-query.
    \item \textbf{Generation: }This is the final step, where the user's question and the best-matching information are combined and given to an LLM to create an answer.
    \textit{LLMs tested:} Google Gemma 2B, Meta’s Llama 2 13B, Llama 3.1 8B, Llama 3 70B, Mixtral 8x7B, and fine-tuned Llama 3.1 8B.
\end{enumerate}


\subsubsection{Models}


To assess the impact of LLM size and capabilities on RAG performance, a selection of models is tested. The models included are smaller parameter models like Google's Gemma 2B and Meta's Llama 3.1 8B, mid-sized models such as Llama 2 13B, and larger models like Llama 3 70B and Mixtral 8x7B. The models differ in their training data, architecture, and parameter count, which can influence their reasoning capabilities. For instance, the impact of model scale on performance can be examined using the Llama family models (2 13B, 3.1 8B, and 3 70B) share a similar architecture but vary in size. Mixtral 8x7B, on the other hand, utilises a mixture-of-experts architecture \cite{shazeer2017outrageouslylargeneuralnetworks} and gives insight into how this compares to the other models. \\

The fine-tuned Llama 3.1 8B model described in Section \ref{finetuning} is also tested to evaluate the impact of domain-specific fine-tuning on performance.

\subsubsection{Data Chunking} \label{data_chunking_methods}

One limitation of LLMs is their restricted context window which limits their comprehension abilities for extensive documents \cite{liu2024lost, lmsys}. To prepare the markdown data extracted from the PDF reports for retrieval, it must be broken down into chunks - smaller, more focused segments that the LLM can process with greater precision. Data chunking can impact the performance of RAG systems by directly impacting the quality and relevance of retrieved context being inputted into the LLM for answer generation \cite{liu2024lost}. This project explores seven chunking strategies to determine the optimal approach for the Sustainability Reporting domain. The chunking strategies explored are fixed-size chunks of 256, 512, and 1024 tokens, page-based chunking, a rolling window approach, semantic chunking, and custom markdown chunking. \\

Fixed-size chunks of 512 tokens are a common baseline in RAG systems, balancing granularity with computational efficiency. To investigate the impact of chunk size on performance, 256- and 1024-token chunks are also evaluated. These larger chunks may provide less/more comprehensive context for complex queries but could potentially reduce retrieval precision by missing information or confusing the LLM.\\



One issue with fixed-length chunking strategies is that relevant information might be split across chunk boundaries. To address this, a rolling window approach with 512-token chunks and 10\% overlap is implemented to improve retrieval robustness by ensuring that key information is not inadvertently separated. \\

A page-based chunking strategy is implemented to preserve the original document structure and may be particularly helpful for queries referencing specific pages. Semantic chunking is also explored to create chunks based on topical coherence rather than arbitrary token counts. This method aims to preserve the semantic integrity of the content, potentially improving the relevance of retrieved chunks for complex queries. \\

Finally, this project leverages the markdown structure of the reports to implement a custom chunking strategy. Markdown chunking splits the reports by their headings at a chosen level of hierarchy (e.g. at the section/subsection level). Additionally, it chunks tables as a whole, separately from free-text content. Maintaining the integrity of tables reduces the risk of the retriever missing or misinterpreting relevant rows or columns of data. This method is akin to that introduced by Yepes et al \cite{yepes2024financial} for the finance domain. \\

The chunked data is embedded using the OpenAI embedding model text-embedding-3-small and stored in Pinecone, a vector database. Metadata is appended to each chunk containing the page number the text is taken from, the report name, the industry, and the type of information (free text or table) contained in the chunk. Chunking strategies are systematically evaluated on the diverse synthetic QA dataset to determine the most effective method for this domain. Results are shown in Section \ref{chunking_results}.



\subsubsection{Retrieval Techniques}

Retrieval techniques define how the most relevant context chunks for a query are determined, and choice of technique can significantly influence the effectiveness of RAG systems \cite{gao2024retrievalaugmented}. This study examines three widely used retrieval techniques: KNN and hybrid.\\

K-Nearest Neighbors (KNN) retrieves the K most similar chunks based on embedding cosine similarity. This is a simple and efficient method but may struggle with capturing complex relationships between queries and documents. Hybrid approaches combine embedding-based similarity with traditional text search techniques such as BM25 (Best Match), a search algorithm that selects documents according to their query relevance, using Term Frequency (TF), Inverse Document Frequency (IDF), and Document Length to compute a relevance score \cite{sawarkar2024blended}. This can be particularly effective for queries that contain both conceptual elements and key terms or phrases. These retrieval techniques are systematically evaluated to identify the most effective method for the sustainability reporting domain.  \\

Three further popular retrievers - SVM, linear regression, and Maximum Marginal Relevance (MMR) - are also tested, with results shown in Appendix \ref{retriever_appx}. 
 

\subsubsection{Query Transforms}


The study explores two query transformation techniques: Hypothetical Document Embeddings (HyDE), and multi-query expansion. Each method is an approach to modifying or expanding the original query to potentially improve retrieval performance. The baseline approach involves using the original query without any modifications. \\

Hypothetical Document Embeddings (HyDE) \cite{gao2022precise} is a query transformation technique that leverages LLMs to generate a hypothetical document that could answer the query, and then uses the embedding of this document for retrieval. The process involves inputting the original query to an LLM, which then generates a hypothetical document that could potentially answer the query. This hypothetical document is then embedded and used, instead of the original query, to retrieve real documents. HyDE aims to bridge the semantic gap between queries and documents by creating a more comprehensive representation of the user's information needs. Gao et al. \cite{gao2022precise} demonstrated that HyDE can improve retrieval performance across various domains, particularly for complex or ambiguous queries. However, the hypothetical document is fake and therefore may contain false details, introducing hallucinations into the retrieval process. \\

Multi-query expansion \cite{medium_rag_not_working} involves generating multiple variations or reformulations of the original query to capture different aspects of the information needed. This technique analyses the original query to identify key terms or concepts and formulates alternative formulations of the query, often using synonyms, related terms, or different phrasings. These multiple queries are then used for retrieval using parallel vector searches and intelligent reranking to aggregate results. The goal of multi-query expansion is to improve recall by addressing vocabulary mismatch problems and capturing a broader range of relevant documents.


\subsection{Fine-tuning}\label{finetuning}

Another method for integrating domain-specific data into an LLM is fine-tuning (as discussed in Chapter 2). The Llama 3.1 8B model was fine-tuned using the final MCQ QA pairs to enhance its performance on multiple-choice questions without RAG, by learning the specific information and thus reducing the hallucination rate. LoRA finetuning was implemented using Together.AI. The performance of the fine-tuned Llama 3.1 8B was compared to both the original Llama 3.1 8B and a larger Llama 3.1 40B model on local and cross-industry questions. 

 
\subsection{Industry Classification} 
Six models were implemented for multi-label industry classification: a natural language processing (NLP) approach, a machine learning (ML) approach, a Multi-Layer Perceptron (MLP) model, and two prompting-based approaches using GPT-4o and GPT-4o Mini. Each model was designed to output multiple industry labels per input question, accommodating up to 5 industries per question.

\subsubsection{NLP Model (DistilBERT)} 

The NLP model is based on DistilBERT \cite{sanh2020distilbertdistilledversionbert}, a smaller and more efficient version of BERT tailored for sequence classification. The model includes a pre-classifier layer and a final classifier with sigmoid activation, allowing for multi-label classification by outputting probabilities for each industry. Input text was tokenized using the distilbert-base-uncased tokenizer.

\subsubsection{ML Models (Random Forest and XGBoost)} 

The machine learning approach involved two models: a Random Forest classifier and an XGBoost model. Both models were trained on text embeddings generated using OpenAI's text-embedding-3-small model. The Random Forest model was composed of 200 decision trees, while the XGBoost model used 100 estimators. Each model was tuned to predict multiple industry labels per input, with a custom threshold applied to the output probabilities to determine the final industry predictions.

\subsubsection{MLP Classifier}

The Multi-Layer Perceptron (MLP) classifier was structured with two hidden layers and trained using the same embeddings as the ML models, using a sigmoid activation in the output layer to generate probabilities for each industry, allowing for multi-label classification. This model was selected to learn non-linear patterns in the input data.

\subsubsection{LLM Classifier} \label{llm_classifier}

The prompting-based approach used GPT-4o and GPT-4o Mini to classify industries using a custom prompt containing industry descriptions. The returned predictions were structured using a Pydantic model, allowing for variable industry outputs depending on the input query.


\subsubsection{Evaluation Metrics} 

The performance of each model was evaluated using the following metrics: \begin{itemize} \item Macro F1 Score: This metric calculates the harmonic mean of precision and recall for each industry label, treating each label equally, regardless of its frequency in the dataset. It provides a balanced measure of performance across all industry categories. \item Precision: Precision measures the proportion of correctly predicted industry labels out of all labels predicted by the model, indicating how accurate the model is when it makes a positive prediction. \item Recall: Recall calculates the proportion of correctly predicted industry labels out of all true industry labels, evaluating how well the model identifies all relevant industries. \item Hamming Loss: This metric quantifies the fraction of labels that are incorrectly predicted. It accounts for both false positives and false negatives, providing insight into how often the model makes incorrect label assignments. \end{itemize}


\subsection{Proposed Architecture}

For the final question-answering system, two architectures are proposed. The first one leverages the RAG techniques studied along with the GPT-4o Mini industry classifier. The second one uses LLMs throughout the entire pipeline. Both pipelines integrate a pre-processing step to stop the chatbot from answering queries that are not related to IFRS sustainability reporting standards. This is done by passing the user query through an LLM call to verify that it is relevant before proceeding. 

\subsubsection{Custom RAG Pipeline}

A custom RAG pipeline is designed that combines the strengths of the RAG methods tested. In particular, it integrates two novel elements - the prompt-based industry classifier and the fine-tuned Llama 3.1 8B model - with established RAG techniques to create a more targeted system for answering queries related to sustainability reporting standards. The pipeline is shown in Figure \ref{fig:rag_pipeline}.\\

User queries are checked for domain relevancy and passed into the LLM industry classifier described in Section \ref{llm_classifier}, which identifies the most likely industries the query is referring to. The chunks stored in the vector database, which were chunked by the custom markdown method, are then filtered using their metadata (since each chunk is labelled by industry as described in Section \ref{data_chunking_methods}) to constrain the database to only chunks from those industries. The top 5 most relevant chunks are then retrieved from the filtered database using KNN. These are passed to the fine-tuned Llama 3.1 8B model which uses them to generate a response.

\begin{figure}
    \centering
    \includegraphics[width=0.7\linewidth]{figures/rag_pipeline.pdf}
    \caption{Custom RAG pipeline.}
    \label{fig:rag_pipeline}
\end{figure}

\subsubsection{Fully LLM-based Pipeline}

The second question-answering pipeline, shown in Figure \ref{fig:llm_rag}, is designed using LLMs throughout. After the query has been deemed relevant to the domain, the information retrieval and answer generation steps are as follows:
\begin{enumerate}
    \item The LLM classifier is used to output the most likely industries related to the query.
    \item The markdown report is selected for each industry. 
    \item For each industry separately, the associated industry markdown report is passed along with the user query to an LLM, which retrieves the chunks of context that are best suited to answer the question. The LLM used for this step is GPT-4o Mini as it has a long enough context length.
    \item The chunks for each industry are labelled with the industry, and all industry chunks are combined.
    \item The combined chunks are passed along with the user query to the fine-tuned Llama 3.1 8B model to generate an answer.
\end{enumerate}


This architecture displays two key differences relative to the RAG pipeline. Firstly, the system operates by passing inputs directly into LLMs, without embedding data or the query. This bypasses the need for similarity search methods for retrieving relevant chunks and instead leverages the reasoning abilities of the LLM to gauge relevance. Secondly, the data is not broken down into chunks beforehand. Instead, an LLM call takes as input the entire industry markdown to select the most relevant chunks based on the user query. This provides flexibility to tailor the number and content of the retrieved chunks to each query, rather than relying on a fixed chunking strategy and retrieval number. \\


\begin{figure}[H]
    \centering
    \includegraphics[width=0.75\linewidth]{figures/llm_pipeline.pdf}
    \caption{Custom LLM pipeline.}
    \label{fig:llm_rag}
\end{figure}


\section{Results}

All methods are evaluated on both MCQ and free-text questions. Accuracy is used on MCQs, and BLEU and ROUGE-L are used for evaluation on free-text questions.

\subsection{RAG}

\subsubsection{Models}



\begin{table}[H]
\centering
\begin{tabular}{lcccccc}
\hline
\multirow{3}{*}{Model} & \multicolumn{2}{c}{MCQ} & \multicolumn{4}{c}{Free text} \\ \cline{2-7} 
 & Local & Cross-Industry & \multicolumn{2}{c}{Local} & \multicolumn{2}{c}{Cross-Industry} \\  
 & Accuracy & Accuracy & BLEU & ROUGE & BLEU & ROUGE \\ \hline
Google Gemma 2B         & 12.90 & 17.64 & 0.18 & 0.36 & 0.05 & 0.14 \\
Llama 2 13B             & 38.71 & 17.64 & 0.17 & 0.34 & 0.01 & 0.03 \\
Llama 3.1 8B            & 80.65 & 52.94 & 0.41 & 0.60 & 0.06 & 0.23 \\
\begin{tabular}[c]{@{}l@{}}Llama 3.1 8B \\ Fine-tuned\end{tabular} & \textbf{83.87} & 53.16 & \textbf{0.49} & \textbf{0.65} & \textbf{0.07} & \textbf{0.24} \\
Llama 3 70B             & 80.65 &\textbf{ 60.72} & 0.42 & 0.62 & 0.04 & 0.19 \\
Mixtral 8x7B            & 80.65 & 37.25 & 0.18 & 0.38 & 0.05 & 0.02 \\ \hline
\end{tabular}
\caption{RAG results with different LLM and baseline settings: 512-token chunking, KNN retrieval with top K=5 chunks, no query transform, and LLM temperature of 0.5.}
\label{tab:models}
\end{table}


Table \ref{tab:models} shows that Llama 3.1 models generally outperform others on both local and cross-industry questions, and little variance is observed across model size between Llama 3.1 8B and 40B. However, a large gap exists between Llama 3.1 8B and Llama 2 13B, which is a much less capable model despite being larger. Llama 2's performance is more comparable to Google Gemma 2B, showing the same performance in MCQ accuracy (17.64\%). \\

The fine-tuned Llama 3.1 8B model shows the best performance across all metrics for local questions, as measured by both MCQ accuracy and free-text BLEU and ROUGE. In particular, it shows better performance than the significantly larger model Llama 3 70B. However, the fine-tuned model's outperformance over the standard Llama 3.1 8B model is smaller for cross-industry questions, performing worse on these than the larger 70B model. This suggests that fine-tuning (when coupled with RAG) improves the model's ability to handle highly specific `factual' queries, though provides limited benefit for more complex queries, where larger models can leverage their enhanced reasoning power.\\

All models show a noticeable drop in performance for cross-industry questions compared to local ones. Nevertheless, one notable observation is that, while cross-industry performance is low, it is still above average since a random answer selection strategy would result in a 20\% accuracy for MCQs with 5 answer options.\\

Finally, it is of note that BLEU and ROUGE are generally very low, particularly for cross-industry questions. This is discussed in Section \ref{models_discussion}.
 

\subsubsection{Data Chunking}\label{chunking_results}


\begin{table}[H]
\centering
\begin{tabular}{lcccccc}
\hline
\multirow{3}{*}{Chunking} & \multicolumn{2}{c}{MCQ} & \multicolumn{4}{c}{Free text} \\ \cline{2-7} 
 & Local & Cross-Industry & \multicolumn{2}{c}{Local} & \multicolumn{2}{c}{Cross-Industry} \\   
 & Accuracy & Accuracy & BLEU & ROUGE & BLEU & ROUGE \\ \hline
Sentence 256    & 80.65 & 41.26 & 0.25 & 0.48 & \textbf{0.09} & \textbf{0.24} \\
Sentence 512    & 80.65 & 52.94 & \textbf{0.41} & \textbf{0.60} & 0.06 & 0.23 \\
Sentence 1024   & 38.71 & 31.09 & 0.28 & 0.49 & 0.06 & 0.22 \\
Sentence window & 81.32 & 54.55 & 0.04 & 0.19 & 0.05 & 0.15 \\
Page            & 65.12 & 37.45 & 0.03 & 0.16 & 0.06 & 0.17 \\
Semantic        & 83.30 & 63.90 & 0.33 & 0.55 & 0.06 & 0.23 \\
Custom Markdown & \textbf{84.52} & \textbf{69.10} & 0.04 & 0.19 & 0.04 & 0.15 \\ \hline
\end{tabular}
\caption{RAG results with different chunking methods and baseline settings: Llama 3.1 8B, KNN retrieval with top K=5 chunks, no query transform, and LLM temperature of 0.5.}
\label{tab:chunking}
\end{table}


Table \ref{tab:chunking} shows the performance of the Llama 3.1 8B model across different chunking strategies. For MCQ accuracy, the custom markdown chunking method performs best, achieving 84.52\% for local and 69.10\% for cross-industry questions. This is followed closely by semantic chunking at 83.30\% and 63.90\% respectively.\\

In the free text evaluation, sentence-based chunking strategies with 256 and 512 tokens generally outperform other methods, particularly for local scenarios. Sentence 512 chunking achieves the highest BLEU (0.41) and ROUGE (0.60) scores for local free-text responses. Performance drops significantly for cross-industry tasks across all chunking methods. The page-level chunking shows the poorest performance overall, particularly for free text tasks. \\

A notable result is that Sentence 1024 shows very poor performance for MCQs, but reasonable results on free-text questions. Reasons for this are discussed in Section \ref{chunking_discussion}. 


 

\subsubsection{Retrieval Techniques}
\begin{table}[H]
\centering
\begin{tabular}{lcccccc}
\hline
\multirow{3}{*}{Retriever} & \multicolumn{2}{c}{MCQ} & \multicolumn{4}{c}{Free text} \\ \cline{2-7} 
 & Local & Cross-Industry & \multicolumn{2}{c}{Local} & \multicolumn{2}{c}{Cross-Industry} \\  
 & Accuracy & Accuracy & BLEU & ROUGE & BLEU & ROUGE \\ \hline
KNN     & \textbf{80.65} & \textbf{52.94} & \textbf{0.41} & \textbf{0.60} & \textbf{0.06} & \textbf{0.23} \\
Hybrid            & \textbf{80.65} & 47.05 & 0.37 & 0.58 & \textbf{0.06} & \textbf{0.23} \\ \hline
\end{tabular}
\caption{RAG results with different chunking methods and baseline settings: Llama 3.1 8B, retrieval of top K=5 chunks, no query transform, and LLM temperature of 0.5.}
\label{tab:retriever-comparison}
\end{table}


Table \ref{tab:retriever-comparison} presents the performance of the two retrieval techniques tested. The retrievers perform equally well on local MCQs, achieving 80.65\% accuracy, though KNN performs better on Cross-industry MCQs, achieving 52.94\% versus 47.05\%. KNN also displays the strongest performance on free-text questions, obtaining BLEU and ROUGE scores of 0.41 and 0.60 for local free text responses. Both retrievers exhibit a significant drop in performance when moving from local to cross-industry tasks. 


\subsubsection{Query Transforms}

\begin{table}[H]
\centering
\begin{tabular}{lcccccc}
\hline
\multirow{3}{*}{Query Transform} & \multicolumn{2}{c}{MCQ} & \multicolumn{4}{c}{Free text} \\ \cline{2-7} 
 & Local & Cross-Industry & \multicolumn{2}{c}{Local} & \multicolumn{2}{c}{Cross-Industry} \\  
 & Accuracy & Accuracy & BLEU & ROUGE & BLEU & ROUGE \\ \hline
None        & 80.65 & 52.94 & \textbf{0.41} & \textbf{0.60} & 0.06 & 0.23 \\
HyDE        & \textbf{83.87} & 48.38 & 0.26 & 0.46 & 0.07 & 0.22 \\
Multi-Query & 48.39 & \textbf{54.83} & 0.18 & 0.35 & \textbf{0.11} & \textbf{0.25} \\ \hline
\end{tabular}
\caption{RAG results with different chunking methods and baseline settings: Llama 3.1 8B, KNN retrieval with top K=5 chunks, and LLM temperature of 0.5.}
\label{tab:query-transform-comparison}
\end{table}

Table \ref{tab:query-transform-comparison} illustrates the impact of different query transforms on the performance of the Llama 3.1 8B model across local and cross-industry tasks. The ``None" transform, representing no query modification, serves as a baseline, achieving 80.65\% accuracy for local MCQs and 52.94\% for cross-industry MCQs. It also shows relatively strong performance on local free text questions with BLEU and ROUGE scores of 0.41 and 0.60 respectively.\\

HyDE (Hypothetical Document Embeddings) demonstrates the best performance for local MCQs, improving accuracy to 83.87\%. However, it shows a slight decrease in cross-industry MCQ accuracy (48.38\%) compared to the baseline. HyDE's impact on free text metrics is mixed, with lower scores for local tasks but a marginal improvement in cross-industry BLEU.\\

The Multi-Query transform significantly underperforms in local MCQs (48.39\%), despite achieving the highest accuracy for cross-industry MCQs at 54.83\%. This suggests that the Multi-Query approach may be particularly effective for more complex queries requiring information synthesis across multiple domains (this is discussed further in Section \ref{query_transform_discussion}. Its free text performance is generally lower than the baseline, except for a slight improvement in cross-industry BLEU and ROUGE scores.


\subsection{Fine-tuning}

\begin{table}[H]
\centering
\begin{tabular}{lcc}
\hline
& \begin{tabular}[c]{@{}c@{}}MCQ\ Local\end{tabular} & \begin{tabular}[c]{@{}c@{}}MCQ\ Cross-Industry\end{tabular} \\ \hline
Llama 3.1 8B  & 0.46  & 0.48\\
Llama 3.1 40B & \textbf{0.58} & 0.50   \\
Llama 3.1 8B fine-tuned & 0.49  & \textbf{0.52}\\ \hline
\end{tabular}
\caption{Accuracy of the Llama 3.1 8B, Llama 3.1 40B, and fine-tuned Llama 3.1 8B models on MCQs, without RAG.}
\label{tab:fine-tuning}
\end{table}

Table \ref{tab:fine-tuning} presents the accuracy of three Llama 3.1 models on local and cross-industry multiple-choice questions (MCQs) without using RAG. The Llama 3.1 40B model demonstrates the highest accuracy (0.58) on local MCQs, while the fine-tuned Llama 3.1 8B model performs best on cross-industry MCQs with an accuracy of 0.52. Notably, the fine-tuned 8B model outperforms the base 8B model on both local and cross-industry questions, suggesting that domain-specific fine-tuning can enhance model performance. The 40B model shows superior performance on local questions, indicating that it is able to use its reasoning capabilities to deduce the correct answer. Nevertheless, this result is much lower than the performance of Llama 3.1 8B with RAG.


\subsection{Industry Classification}

\begin{table}[H]
\centering
\begin{tabular}{lcccc}
\hline
Model                    & Recall & Precision & Macro F1 score & Hamming Loss \\ \hline
RAG                      & 0.29   & 0.28      & 0.44     & 0.06         \\
Random Forest Classifier & \textbf{0.84}   & 0.45      & 0.59     & 0.02         \\
XGBoost                  & 0.62   & 0.77      & 0.69     & \textbf{0.01}         \\
MLP (Neural Network)     & 0.96   & 0.65      & 0.78     & \textbf{0.01}         \\
gpt-4o-mini              & 0.83   & \textbf{0.93}      & 0.86     & \textbf{0.01}         \\
gpt-4o                   & \textbf{0.84}   & \textbf{0.93}      & \textbf{0.87}     & \textbf{0.01}    \\ \hline     
\end{tabular}
\caption{Performance comparison of cross-industry classifiers.}
\label{tab:industry-classifiers}
\end{table}

Table \ref{tab:industry-classifiers} compares the performance of various industry classification models. The evaluation metrics include recall, precision, macro F1 score, and Hamming loss. The gpt-4o model achieves the highest overall performance with a macro F1 score of 0.87 and a Hamming loss of 0.01. The MLP (Neural Network) model demonstrates the highest recall (0.96), while gpt-4o and gpt-4o-mini show the highest precision (0.93). The RAG-based approach exhibits the lowest performance across all metrics, indicating that specialised classification models are more effective for this task. Among the machine learning models, XGBoost shows a balanced performance with high precision (0.77) and a low Hamming loss (0.01). These results suggest that both pre-trained language models and specialised machine learning approaches can be effective for industry classification, although LLM approaches are superior.

\subsection{Proposed Architecture}

\begin{table}[H]
\centering
\begin{tabular}{lcccccc}
\hline
\multirow{2}{*}{Pipeline} & \multicolumn{2}{c}{MCQ} & \multicolumn{4}{c}{Free text} \\ \cline{2-7} 
 & Local & Cross-Industry & \multicolumn{2}{c}{Local} & \multicolumn{2}{c}{Cross-Industry} \\  
 & Accuracy & Accuracy & BLEU & ROUGE & BLEU & ROUGE \\ \hline
Baseline        & 80.65 & 52.94 & 0.41 & 0.60 & 0.06 & 0.23 \\
Custom RAG & 85.32 & 72.15 & 0.45 & 0.61 & 0.13 & 0.26 \\
Fully LLM-based & \textbf{93.45} & \textbf{80.30} & \textbf{0.47} & \textbf{0.66} & \textbf{0.15} & \textbf{0.30} \\ \hline
\end{tabular}
\caption{Performance comparison of the custom RAG and LLM-based pipelines on MCQ and free-text questions. The baseline approach uses Llama 3.1 8B, and KNN retrieval with top K=5.}
\label{tab:proposed-pipelines}
\end{table}

The table \ref{tab:proposed-pipelines} provides a comparative analysis of the performance metrics for two different pipelines: the custom RAG pipeline and the fully LLM-based pipeline.\\ 

For MCQs, the custom RAG pipeline achieves an accuracy of 85.32\% for local questions and 72.15\% for cross-industry questions. The fully LLM-based pipeline demonstrates higher accuracy, with 93.45\% for local questions and 80.30\% for cross-industry questions. Most notably, both approaches achieve much higher performance on cross-industry questions than the baseline, and the LLM-based pipeline achieves an accuracy on cross-industry MCQs that is similar to the baseline performance on local questions.\\

On free-text questions, the custom RAG pipeline records BLEU and ROUGE scores of 0.45 and 0.61 for local queries, and 0.13 and 0.26 for cross-industry queries. The fully LLM-based pipeline achieves BLEU and ROUGE scores of 0.47 and 0.66 for local queries, and 0.15 and 0.30 for cross-industry queries. BLEU and ROUGE scores of the custom pipelines show a lower improvement than accuracy on MCQs, though this is a result of the metrics themselves, as discussed in Section \ref{models_discussion}. \\

Overall, the fully LLM-based pipeline achieves the best performance across all types of questions.

\section{Discussion}

\subsection{RAG}
\subsubsection{Models}\label{models_discussion}

Model size and capability have a strong impact on RAG performance. The smallest model, Google Gemma 2B, is most often unable to answer the question posed, and when it does provide an answer it is usually wrong. Llama 2 13B is an older, less capable model, and although it is bigger than the newer 8B models, it struggles to answer questions directly and recites the source content instead, showing its weak reasoning capabilities. These issues are particularly pronounced for more complicated cross-industry questions.\\

Listing \ref{lst:models_qa_example} below shows a local multi-hop question example with the RAG responses given by each model. The correct answer must contain two parts: the first part requires extracting information from a Table, while the second is verbatim text from the source document. Most models (except Gemma 2B) got the second part correct. However, only the fine-tuned model gets the first part of the answer correct, showcasing that certain knowledge has been embedded into the fine-tuned model and is being used successfully when combined with RAG methods.\\

Also notable is that all models contain slightly different phrasings of the same answer, containing additional words/details to various extents. Even though multiple phrasings can be correct, BLEU and ROUGE metrics are sensitive to exact phrasing, resulting in low BLEU and ROUGE scores, particularly for more complicated cross-industry questions. Local questions, on the other hand, can more successfully be evaluated using BLEU and ROUGE as they are often shorter and more factual, so a correct answer will score above 0.9 on both metrics.

\begin{lstlisting}[language=json,firstnumber=1,label={lst:models_qa_example},caption={Example local multi-hop question with RAG answers using different models}]
Question: "What is the category of the 'Payload fuel economy' metric in the Health Care Distributors industry and how should this metric be calculated?"

Correct answer: "The category of the 'Payload fuel economy' metric is Quantitative. Payload fuel economy shall be calculated as: total gallons of fuel consumed / revenue tons miles (RTM), where revenue ton miles (RTM) = total weight of paid tonnage transported (payload) x total distance in miles goods were transported."

Google Gemma 2B answer: "The category of the 'Payload fuel economy' metric in the Health Care Distributors industry is **Transportation**.\n\nThe metric should be calculated as:\n\n```\nPayload fuel economy = total gallons of fuel consumed / revenue tons miles (RTM)\n```\n\nwhere:\n\n* **total gallons of fuel consumed** is the total amount of fuel consumed by the entity's transportation fleet.\n* **revenue tons miles (RTM)** is the total weight of paid tonnage transported (payload) multiplied by the total distance in miles goods were transported."
"bleu_score": 0.36
"rouge_l_score": 0.65

Llama 2 13B answer: "Based on the context information provided, the 'Payload fuel economy' metric in the Health Care Distributors industry falls under the category of 'Fleet Fuel Management' and should be calculated as the total gallons of fuel consumed divided by the revenue tons miles (RTM) of the delivery fleet, where RTM is the weight of paid tonnage transported (payload) multiplied by the total distance in miles goods were transported. This calculation should be aggregated for types of transportation, including air, marine, rail, and road."
"bleu_score": 0.30
"rouge_l_score": 0.50

Llama 3.1 8B answer: "The category of the 'Payload fuel economy' metric is HC-DI-110a.1. This metric should be calculated as total gallons of fuel consumed divided by revenue ton miles (RTM), where revenue ton miles (RTM) is the total weight of paid tonnage transported multiplied by the total distance in miles goods were transported."
"bleu_score": 0.60
"rouge_l_score": 0.75

Llama 3.1 8B fine-tuned answer: "The category of the 'Payload fuel economy' metric is Quantitative. This metric should be calculated as total gallons of fuel consumed divided by revenue ton miles (RTM), where revenue ton miles (RTM) is the total weight of paid tonnage transported multiplied by the total distance in miles goods were transported."
"bleu_score": 0.65
"rouge_l_score": 0.77

Llama 3 70B answer: "The category of the 'Payload fuel economy' metric in the Health Care Distributors industry is HC-DI-110a.1. This metric should be calculated as total gallons of fuel consumed divided by revenue tons miles (RTM), where revenue ton miles (RTM) equals total weight of paid tonnage transported (payload) multiplied by total distance in miles goods were transported."
"bleu_score": 0.61
"rouge_l_score": 0.74

Mixtral 8x7B answer: "The 'Payload fuel economy' metric in the Health Care Distributors industry falls under the category of Fleet Fuel Management. This metric should be calculated as the total gallons of fuel consumed divided by revenue tons miles (RTM), where RTM is equal to the total weight of paid tonnage transported (payload) multiplied by the total distance in miles goods were transported. Payload includes the weight of paid tonnage and excludes the vehicle weight. The entity should aggregate payload fuel economy for different types of transportation, including air, marine, rail, and road transportation."
"bleu_score": 0.31
"rouge_l_score": 0.56
\end{lstlisting}



\subsubsection{Data Chunking} \label{chunking_discussion}

A notable observation from Table \ref{tab:chunking} is that combining large fixed-sized chunks (Sentence 1024) with top-5 retrieval hurts performance on MCQs, but not on free-text questions. This suggests that the retrieved context confuses the LLM when it has to choose between five similar answer options, but not when the LLM writes its own response to the question in free text format. Interestingly, this is not a problem for semantic/markdown chunking, which also contains large information chunks. This is because, even though the chunks are large, they are semantically coherent, whereas fixed-size chunks do not contain complete information as text is simply cut off at fixed points. \\

Semantic and custom markdown chunking methods yield the best results, as the context given to the LLM is coherent. The custom markdown adds the additional benefit of isolating tables, so more specific context can be retrieved for questions that ask for specific information from tables.



\subsubsection{Retrieval Techniques}

While both KNN and hybrid retrievers yield similar performance, there is a notable drop in accuracy on cross-industry MCQs using the hybrid retriever. This could suggest that incorporating text-based keyword search does not provide a benefit for the sustainability reporting domain, and may indeed detract from the power of embedding similarity.


\subsubsection{Query Transforms}\label{query_transform_discussion}



Hyde exhibits 4\% better performance on local MCQ accuracy than the baseline, but 8.6\% lower performance on cross-industry performance. This suggests that the hypothetical generated document is often accurate for factual questions, but struggles for more complex queries based on multiple source documents. This is evident in Listing \ref{lst:query_transforms_example}, where HyDE's answer to a cross-industry question contains hallucinated content such as the ``FTSE Nareit Classification Structure". \\


Multi-query exhibits a 40\% reduction in accuracy on local answers and a 3.6\% improvement in cross-industry accuracy relative to baseline. Its superior ability to handle complex questions is shown in its highly accurate answer in Listing \ref{lst:query_transforms_example}. However, the process of rephrasing questions in multiple ways harms the response to specific factual questions that are sensitive to detail and phrasing. This is shown in Listing \ref{lst:query_transforms_local}, where the baseline query gets the correct answer, but rephrasing the query multiple times induces hallucination in the last part of the answer. The fall in quality of simple factual responses is too great to justify the advantages of multi-query transforms for answering complex questions.
\newpage
\begin{lstlisting}[language=json,firstnumber=1,label={lst:query_transforms_local},caption={Example local single-hop question with RAG answers using different query transforms}]
Question: "What are some examples of efforts to reduce the environmental impact of logistics for the Health Care Distributors industry?"

Correct answer: "Relevant efforts to describe include, but are not limited to, upgrades to fleet (fuel efficiency), usage of alternative and/or renewable fuels, optimized logistics routes, and idling reduction programs.",

Baseline answer: "Efforts to reduce the environmental impact of logistics for the Health Care Distributors industry may include upgrading to more fuel-efficient fleets, using alternative and/or renewable fuels, optimizing logistics routes, and implementing idling reduction programs."
"bleu_score": 0.34
"rouge_l_score": 0.57

Multi-query answer: "Efforts to reduce the environmental impact of logistics for the Health Care Distributors industry may include upgrading to more fuel-efficient fleets, using alternative and/or renewable fuels, optimizing logistics routes, implementing idling reduction programs, partnering with energy-efficient suppliers, and investing in smart logistics technologies to enhance route efficiency and decrease fuel consumption.",
"bleu_score": 0.13
"rouge_l_score": 0.28

HyDE answer: "Health care distributors can implement various strategies to reduce the environmental impact of their logistics operations. For instance, they can optimize their transportation routes and modes to minimize fuel consumption and emissions. This might involve using more fuel-efficient vehicles, such as electric or hybrid trucks, or switching to rail or sea transportation for longer distances. Additionally, distributors can work with suppliers to reduce packaging materials and implement sustainable packaging practices, such as using biodegradable or recyclable materials. They can also explore alternative fuel options, like biofuels or hydrogen fuel cells, to power their fleets. Furthermore, distributors can invest in route optimization software to reduce idling time and lower emissions. By implementing these strategies, health care distributors can not only reduce their environmental footprint but also improve their bottom line by reducing fuel costs and increasing efficiency.",
"bleu_score": 0.004
"rouge_l_score": 0.115
\end{lstlisting}

\begin{lstlisting}[language=json,firstnumber=1,label={lst:query_transforms_example},caption={Example cross-industry single-hop question with RAG answers using different query transforms}]
"question": "Which industry has more 'Quantitative' metrics to disclose in Table 1 between Real Estate Services and Agricultural Products industries?"

"correct_answer": "The Agricultural Products industry has more 'Quantitative' metrics to disclose, with seven metrics in Table 1 compared to the Real Estate Services industry, with three metrics."

No transform answer: "Based on the provided context, it appears that the Agricultural Products industry has more 'Quantitative' metrics to disclose in Table 1. This is evident from the detailed metrics listed under the \"Metrics\" section, which include specific calculations and thresholds for disclosure, such as FB-AG-440a.1 and FB-AG-110a.1. In contrast, there is no mention of specific metrics or disclosure requirements for the Real Estate Services industry in the provided context."

Multi-query answer: "The Agricultural Products industry has more 'Quantitative' metrics to disclose in Table 1 compared to the Real Estate Services industry."

HyDE answer: "The Real Estate industry has 4 'Quantitative' metrics to disclose in Table 1, specifically Leasable floor area, by property subsector, Percentage of indirectly managed assets, by property subsector, and Average occupancy rate, by property subsector. These metrics are aligned with the FTSE Nareit Classification Structure and are disclosed separately for each portion of the entity's portfolio where properties are classified into subsectors."
\end{lstlisting}

\subsubsection{Industry Retrieval}

A consistent theme observed in the RAG results across all methods is that performance on cross-industry questions is lower than on local ones. This phenomenon was manually investigated by examining the chunks retrieved for cross-industry queries. It was found that the retriever gets chunks from multiple industries that could be related to the query, thus confusing the LLM that generates the answer. One such example question is shown in Listing \ref{lst:industry_query_example} below along with the retrieved industries using the baseline RAG method, as well as those classified by the LLM industry classifier. The baseline RAG selects five chunks from four different related industries, where the target is two industries. The LLM classifier identifies the correct industries as being the highest probability ones intended by the user.


\begin{lstlisting}[language=json,firstnumber=1,label={lst:industry_query_example},caption={Example cross-industry single-hop question}]
"question": "What are the activity metrics for the car transport and air transport industries?"
"target industries": "b61-airlines", "b63-automobiles"
"industries retrieved by baseline RAG": "b62-auto-parts" (x2), "b63-automobiles", "b60-air-freight-and-logistics", "b61-airlines"
"industries output by the LLM classifier": "b63-automobiles", "b61-airlines"
\end{lstlisting}


\subsection{Proposed Architecture}

Both custom pipelines demonstrate significant improvements in accuracy on cross-industry questions, and this can be largely attributed to the integration of the LLM industry classifier, which constrains the domain of knowledge from which information is retrieved for a query, thus increasing the likelihood of retrieving the correct chunks. \\

Only the LLM-based pipeline, however, reaches performance on cross-industry questions that is comparable to performance on local ones, despite both architectures using the same industry classifier. This is explained by the difference in chunking and retrieval methods adopted in the two approaches. The RAG pipeline still utilises information that has been `generically' segmented, rather than chunks that are customised to the query itself. Furthermore, top-5 retrieval is used rather than allowing an LLM to decide on the appropriate number of chunks based on their context. While the top-5 retrieval hurdle can be tackled using more sophisticated techniques, such as by setting a similarity threshold and selecting all chunks above this, selecting this threshold is complex, requiring thorough experimentation, and there is unlikely to be a single threshold that is appropriate for all query types. Finally, the RAG-based pipeline relies on embedding similarity to select chunks, which was the best-performing retriever tested.\\

The LLM-based pipeline retrieves the most customised context for answering the specific query, lessening the burden on the generator LLM of utilising its own reasoning capabilities to understand the information and generate an answer. It also minimises the chances of irrelevant context being retrieved (which is a problem when a specific number of chunks must be retrieved at each query). Nevertheless, this architecture is not without weaknesses. Most notably, the increased number of LLM calls may increase the chance of hallucination of the retrieved content. Furthermore, there seems to be a limit to its performance on complex questions still, as the accuracy on cross-industry questions is 80.30\% rather than being closer to the 93.25\% accuracy on local MCQs. This suggests there exist limits to the reasoning capabilities of the generator LLM, despite being provided with improved context.





\section{Conclusion}
In this work, we propose a simple yet effective approach, called SMILE, for graph few-shot learning with fewer tasks. Specifically, we introduce a novel dual-level mixup strategy, including within-task and across-task mixup, for enriching the diversity of nodes within each task and the diversity of tasks. Also, we incorporate the degree-based prior information to learn expressive node embeddings. Theoretically, we prove that SMILE effectively enhances the model's generalization performance. Empirically, we conduct extensive experiments on multiple benchmarks and the results suggest that SMILE significantly outperforms other baselines, including both in-domain and cross-domain few-shot settings.


\subsection{Lloyd-Max Algorithm}
\label{subsec:Lloyd-Max}
For a given quantization bitwidth $B$ and an operand $\bm{X}$, the Lloyd-Max algorithm finds $2^B$ quantization levels $\{\hat{x}_i\}_{i=1}^{2^B}$ such that quantizing $\bm{X}$ by rounding each scalar in $\bm{X}$ to the nearest quantization level minimizes the quantization MSE. 

The algorithm starts with an initial guess of quantization levels and then iteratively computes quantization thresholds $\{\tau_i\}_{i=1}^{2^B-1}$ and updates quantization levels $\{\hat{x}_i\}_{i=1}^{2^B}$. Specifically, at iteration $n$, thresholds are set to the midpoints of the previous iteration's levels:
\begin{align*}
    \tau_i^{(n)}=\frac{\hat{x}_i^{(n-1)}+\hat{x}_{i+1}^{(n-1)}}2 \text{ for } i=1\ldots 2^B-1
\end{align*}
Subsequently, the quantization levels are re-computed as conditional means of the data regions defined by the new thresholds:
\begin{align*}
    \hat{x}_i^{(n)}=\mathbb{E}\left[ \bm{X} \big| \bm{X}\in [\tau_{i-1}^{(n)},\tau_i^{(n)}] \right] \text{ for } i=1\ldots 2^B
\end{align*}
where to satisfy boundary conditions we have $\tau_0=-\infty$ and $\tau_{2^B}=\infty$. The algorithm iterates the above steps until convergence.

Figure \ref{fig:lm_quant} compares the quantization levels of a $7$-bit floating point (E3M3) quantizer (left) to a $7$-bit Lloyd-Max quantizer (right) when quantizing a layer of weights from the GPT3-126M model at a per-tensor granularity. As shown, the Lloyd-Max quantizer achieves substantially lower quantization MSE. Further, Table \ref{tab:FP7_vs_LM7} shows the superior perplexity achieved by Lloyd-Max quantizers for bitwidths of $7$, $6$ and $5$. The difference between the quantizers is clear at 5 bits, where per-tensor FP quantization incurs a drastic and unacceptable increase in perplexity, while Lloyd-Max quantization incurs a much smaller increase. Nevertheless, we note that even the optimal Lloyd-Max quantizer incurs a notable ($\sim 1.5$) increase in perplexity due to the coarse granularity of quantization. 

\begin{figure}[h]
  \centering
  \includegraphics[width=0.7\linewidth]{sections/figures/LM7_FP7.pdf}
  \caption{\small Quantization levels and the corresponding quantization MSE of Floating Point (left) vs Lloyd-Max (right) Quantizers for a layer of weights in the GPT3-126M model.}
  \label{fig:lm_quant}
\end{figure}

\begin{table}[h]\scriptsize
\begin{center}
\caption{\label{tab:FP7_vs_LM7} \small Comparing perplexity (lower is better) achieved by floating point quantizers and Lloyd-Max quantizers on a GPT3-126M model for the Wikitext-103 dataset.}
\begin{tabular}{c|cc|c}
\hline
 \multirow{2}{*}{\textbf{Bitwidth}} & \multicolumn{2}{|c|}{\textbf{Floating-Point Quantizer}} & \textbf{Lloyd-Max Quantizer} \\
 & Best Format & Wikitext-103 Perplexity & Wikitext-103 Perplexity \\
\hline
7 & E3M3 & 18.32 & 18.27 \\
6 & E3M2 & 19.07 & 18.51 \\
5 & E4M0 & 43.89 & 19.71 \\
\hline
\end{tabular}
\end{center}
\end{table}

\subsection{Proof of Local Optimality of LO-BCQ}
\label{subsec:lobcq_opt_proof}
For a given block $\bm{b}_j$, the quantization MSE during LO-BCQ can be empirically evaluated as $\frac{1}{L_b}\lVert \bm{b}_j- \bm{\hat{b}}_j\rVert^2_2$ where $\bm{\hat{b}}_j$ is computed from equation (\ref{eq:clustered_quantization_definition}) as $C_{f(\bm{b}_j)}(\bm{b}_j)$. Further, for a given block cluster $\mathcal{B}_i$, we compute the quantization MSE as $\frac{1}{|\mathcal{B}_{i}|}\sum_{\bm{b} \in \mathcal{B}_{i}} \frac{1}{L_b}\lVert \bm{b}- C_i^{(n)}(\bm{b})\rVert^2_2$. Therefore, at the end of iteration $n$, we evaluate the overall quantization MSE $J^{(n)}$ for a given operand $\bm{X}$ composed of $N_c$ block clusters as:
\begin{align*}
    \label{eq:mse_iter_n}
    J^{(n)} = \frac{1}{N_c} \sum_{i=1}^{N_c} \frac{1}{|\mathcal{B}_{i}^{(n)}|}\sum_{\bm{v} \in \mathcal{B}_{i}^{(n)}} \frac{1}{L_b}\lVert \bm{b}- B_i^{(n)}(\bm{b})\rVert^2_2
\end{align*}

At the end of iteration $n$, the codebooks are updated from $\mathcal{C}^{(n-1)}$ to $\mathcal{C}^{(n)}$. However, the mapping of a given vector $\bm{b}_j$ to quantizers $\mathcal{C}^{(n)}$ remains as  $f^{(n)}(\bm{b}_j)$. At the next iteration, during the vector clustering step, $f^{(n+1)}(\bm{b}_j)$ finds new mapping of $\bm{b}_j$ to updated codebooks $\mathcal{C}^{(n)}$ such that the quantization MSE over the candidate codebooks is minimized. Therefore, we obtain the following result for $\bm{b}_j$:
\begin{align*}
\frac{1}{L_b}\lVert \bm{b}_j - C_{f^{(n+1)}(\bm{b}_j)}^{(n)}(\bm{b}_j)\rVert^2_2 \le \frac{1}{L_b}\lVert \bm{b}_j - C_{f^{(n)}(\bm{b}_j)}^{(n)}(\bm{b}_j)\rVert^2_2
\end{align*}

That is, quantizing $\bm{b}_j$ at the end of the block clustering step of iteration $n+1$ results in lower quantization MSE compared to quantizing at the end of iteration $n$. Since this is true for all $\bm{b} \in \bm{X}$, we assert the following:
\begin{equation}
\begin{split}
\label{eq:mse_ineq_1}
    \tilde{J}^{(n+1)} &= \frac{1}{N_c} \sum_{i=1}^{N_c} \frac{1}{|\mathcal{B}_{i}^{(n+1)}|}\sum_{\bm{b} \in \mathcal{B}_{i}^{(n+1)}} \frac{1}{L_b}\lVert \bm{b} - C_i^{(n)}(b)\rVert^2_2 \le J^{(n)}
\end{split}
\end{equation}
where $\tilde{J}^{(n+1)}$ is the the quantization MSE after the vector clustering step at iteration $n+1$.

Next, during the codebook update step (\ref{eq:quantizers_update}) at iteration $n+1$, the per-cluster codebooks $\mathcal{C}^{(n)}$ are updated to $\mathcal{C}^{(n+1)}$ by invoking the Lloyd-Max algorithm \citep{Lloyd}. We know that for any given value distribution, the Lloyd-Max algorithm minimizes the quantization MSE. Therefore, for a given vector cluster $\mathcal{B}_i$ we obtain the following result:

\begin{equation}
    \frac{1}{|\mathcal{B}_{i}^{(n+1)}|}\sum_{\bm{b} \in \mathcal{B}_{i}^{(n+1)}} \frac{1}{L_b}\lVert \bm{b}- C_i^{(n+1)}(\bm{b})\rVert^2_2 \le \frac{1}{|\mathcal{B}_{i}^{(n+1)}|}\sum_{\bm{b} \in \mathcal{B}_{i}^{(n+1)}} \frac{1}{L_b}\lVert \bm{b}- C_i^{(n)}(\bm{b})\rVert^2_2
\end{equation}

The above equation states that quantizing the given block cluster $\mathcal{B}_i$ after updating the associated codebook from $C_i^{(n)}$ to $C_i^{(n+1)}$ results in lower quantization MSE. Since this is true for all the block clusters, we derive the following result: 
\begin{equation}
\begin{split}
\label{eq:mse_ineq_2}
     J^{(n+1)} &= \frac{1}{N_c} \sum_{i=1}^{N_c} \frac{1}{|\mathcal{B}_{i}^{(n+1)}|}\sum_{\bm{b} \in \mathcal{B}_{i}^{(n+1)}} \frac{1}{L_b}\lVert \bm{b}- C_i^{(n+1)}(\bm{b})\rVert^2_2  \le \tilde{J}^{(n+1)}   
\end{split}
\end{equation}

Following (\ref{eq:mse_ineq_1}) and (\ref{eq:mse_ineq_2}), we find that the quantization MSE is non-increasing for each iteration, that is, $J^{(1)} \ge J^{(2)} \ge J^{(3)} \ge \ldots \ge J^{(M)}$ where $M$ is the maximum number of iterations. 
%Therefore, we can say that if the algorithm converges, then it must be that it has converged to a local minimum. 
\hfill $\blacksquare$


\begin{figure}
    \begin{center}
    \includegraphics[width=0.5\textwidth]{sections//figures/mse_vs_iter.pdf}
    \end{center}
    \caption{\small NMSE vs iterations during LO-BCQ compared to other block quantization proposals}
    \label{fig:nmse_vs_iter}
\end{figure}

Figure \ref{fig:nmse_vs_iter} shows the empirical convergence of LO-BCQ across several block lengths and number of codebooks. Also, the MSE achieved by LO-BCQ is compared to baselines such as MXFP and VSQ. As shown, LO-BCQ converges to a lower MSE than the baselines. Further, we achieve better convergence for larger number of codebooks ($N_c$) and for a smaller block length ($L_b$), both of which increase the bitwidth of BCQ (see Eq \ref{eq:bitwidth_bcq}).


\subsection{Additional Accuracy Results}
%Table \ref{tab:lobcq_config} lists the various LOBCQ configurations and their corresponding bitwidths.
\begin{table}
\setlength{\tabcolsep}{4.75pt}
\begin{center}
\caption{\label{tab:lobcq_config} Various LO-BCQ configurations and their bitwidths.}
\begin{tabular}{|c||c|c|c|c||c|c||c|} 
\hline
 & \multicolumn{4}{|c||}{$L_b=8$} & \multicolumn{2}{|c||}{$L_b=4$} & $L_b=2$ \\
 \hline
 \backslashbox{$L_A$\kern-1em}{\kern-1em$N_c$} & 2 & 4 & 8 & 16 & 2 & 4 & 2 \\
 \hline
 64 & 4.25 & 4.375 & 4.5 & 4.625 & 4.375 & 4.625 & 4.625\\
 \hline
 32 & 4.375 & 4.5 & 4.625& 4.75 & 4.5 & 4.75 & 4.75 \\
 \hline
 16 & 4.625 & 4.75& 4.875 & 5 & 4.75 & 5 & 5 \\
 \hline
\end{tabular}
\end{center}
\end{table}

%\subsection{Perplexity achieved by various LO-BCQ configurations on Wikitext-103 dataset}

\begin{table} \centering
\begin{tabular}{|c||c|c|c|c||c|c||c|} 
\hline
 $L_b \rightarrow$& \multicolumn{4}{c||}{8} & \multicolumn{2}{c||}{4} & 2\\
 \hline
 \backslashbox{$L_A$\kern-1em}{\kern-1em$N_c$} & 2 & 4 & 8 & 16 & 2 & 4 & 2  \\
 %$N_c \rightarrow$ & 2 & 4 & 8 & 16 & 2 & 4 & 2 \\
 \hline
 \hline
 \multicolumn{8}{c}{GPT3-1.3B (FP32 PPL = 9.98)} \\ 
 \hline
 \hline
 64 & 10.40 & 10.23 & 10.17 & 10.15 &  10.28 & 10.18 & 10.19 \\
 \hline
 32 & 10.25 & 10.20 & 10.15 & 10.12 &  10.23 & 10.17 & 10.17 \\
 \hline
 16 & 10.22 & 10.16 & 10.10 & 10.09 &  10.21 & 10.14 & 10.16 \\
 \hline
  \hline
 \multicolumn{8}{c}{GPT3-8B (FP32 PPL = 7.38)} \\ 
 \hline
 \hline
 64 & 7.61 & 7.52 & 7.48 &  7.47 &  7.55 &  7.49 & 7.50 \\
 \hline
 32 & 7.52 & 7.50 & 7.46 &  7.45 &  7.52 &  7.48 & 7.48  \\
 \hline
 16 & 7.51 & 7.48 & 7.44 &  7.44 &  7.51 &  7.49 & 7.47  \\
 \hline
\end{tabular}
\caption{\label{tab:ppl_gpt3_abalation} Wikitext-103 perplexity across GPT3-1.3B and 8B models.}
\end{table}

\begin{table} \centering
\begin{tabular}{|c||c|c|c|c||} 
\hline
 $L_b \rightarrow$& \multicolumn{4}{c||}{8}\\
 \hline
 \backslashbox{$L_A$\kern-1em}{\kern-1em$N_c$} & 2 & 4 & 8 & 16 \\
 %$N_c \rightarrow$ & 2 & 4 & 8 & 16 & 2 & 4 & 2 \\
 \hline
 \hline
 \multicolumn{5}{|c|}{Llama2-7B (FP32 PPL = 5.06)} \\ 
 \hline
 \hline
 64 & 5.31 & 5.26 & 5.19 & 5.18  \\
 \hline
 32 & 5.23 & 5.25 & 5.18 & 5.15  \\
 \hline
 16 & 5.23 & 5.19 & 5.16 & 5.14  \\
 \hline
 \multicolumn{5}{|c|}{Nemotron4-15B (FP32 PPL = 5.87)} \\ 
 \hline
 \hline
 64  & 6.3 & 6.20 & 6.13 & 6.08  \\
 \hline
 32  & 6.24 & 6.12 & 6.07 & 6.03  \\
 \hline
 16  & 6.12 & 6.14 & 6.04 & 6.02  \\
 \hline
 \multicolumn{5}{|c|}{Nemotron4-340B (FP32 PPL = 3.48)} \\ 
 \hline
 \hline
 64 & 3.67 & 3.62 & 3.60 & 3.59 \\
 \hline
 32 & 3.63 & 3.61 & 3.59 & 3.56 \\
 \hline
 16 & 3.61 & 3.58 & 3.57 & 3.55 \\
 \hline
\end{tabular}
\caption{\label{tab:ppl_llama7B_nemo15B} Wikitext-103 perplexity compared to FP32 baseline in Llama2-7B and Nemotron4-15B, 340B models}
\end{table}

%\subsection{Perplexity achieved by various LO-BCQ configurations on MMLU dataset}


\begin{table} \centering
\begin{tabular}{|c||c|c|c|c||c|c|c|c|} 
\hline
 $L_b \rightarrow$& \multicolumn{4}{c||}{8} & \multicolumn{4}{c||}{8}\\
 \hline
 \backslashbox{$L_A$\kern-1em}{\kern-1em$N_c$} & 2 & 4 & 8 & 16 & 2 & 4 & 8 & 16  \\
 %$N_c \rightarrow$ & 2 & 4 & 8 & 16 & 2 & 4 & 2 \\
 \hline
 \hline
 \multicolumn{5}{|c|}{Llama2-7B (FP32 Accuracy = 45.8\%)} & \multicolumn{4}{|c|}{Llama2-70B (FP32 Accuracy = 69.12\%)} \\ 
 \hline
 \hline
 64 & 43.9 & 43.4 & 43.9 & 44.9 & 68.07 & 68.27 & 68.17 & 68.75 \\
 \hline
 32 & 44.5 & 43.8 & 44.9 & 44.5 & 68.37 & 68.51 & 68.35 & 68.27  \\
 \hline
 16 & 43.9 & 42.7 & 44.9 & 45 & 68.12 & 68.77 & 68.31 & 68.59  \\
 \hline
 \hline
 \multicolumn{5}{|c|}{GPT3-22B (FP32 Accuracy = 38.75\%)} & \multicolumn{4}{|c|}{Nemotron4-15B (FP32 Accuracy = 64.3\%)} \\ 
 \hline
 \hline
 64 & 36.71 & 38.85 & 38.13 & 38.92 & 63.17 & 62.36 & 63.72 & 64.09 \\
 \hline
 32 & 37.95 & 38.69 & 39.45 & 38.34 & 64.05 & 62.30 & 63.8 & 64.33  \\
 \hline
 16 & 38.88 & 38.80 & 38.31 & 38.92 & 63.22 & 63.51 & 63.93 & 64.43  \\
 \hline
\end{tabular}
\caption{\label{tab:mmlu_abalation} Accuracy on MMLU dataset across GPT3-22B, Llama2-7B, 70B and Nemotron4-15B models.}
\end{table}


%\subsection{Perplexity achieved by various LO-BCQ configurations on LM evaluation harness}

\begin{table} \centering
\begin{tabular}{|c||c|c|c|c||c|c|c|c|} 
\hline
 $L_b \rightarrow$& \multicolumn{4}{c||}{8} & \multicolumn{4}{c||}{8}\\
 \hline
 \backslashbox{$L_A$\kern-1em}{\kern-1em$N_c$} & 2 & 4 & 8 & 16 & 2 & 4 & 8 & 16  \\
 %$N_c \rightarrow$ & 2 & 4 & 8 & 16 & 2 & 4 & 2 \\
 \hline
 \hline
 \multicolumn{5}{|c|}{Race (FP32 Accuracy = 37.51\%)} & \multicolumn{4}{|c|}{Boolq (FP32 Accuracy = 64.62\%)} \\ 
 \hline
 \hline
 64 & 36.94 & 37.13 & 36.27 & 37.13 & 63.73 & 62.26 & 63.49 & 63.36 \\
 \hline
 32 & 37.03 & 36.36 & 36.08 & 37.03 & 62.54 & 63.51 & 63.49 & 63.55  \\
 \hline
 16 & 37.03 & 37.03 & 36.46 & 37.03 & 61.1 & 63.79 & 63.58 & 63.33  \\
 \hline
 \hline
 \multicolumn{5}{|c|}{Winogrande (FP32 Accuracy = 58.01\%)} & \multicolumn{4}{|c|}{Piqa (FP32 Accuracy = 74.21\%)} \\ 
 \hline
 \hline
 64 & 58.17 & 57.22 & 57.85 & 58.33 & 73.01 & 73.07 & 73.07 & 72.80 \\
 \hline
 32 & 59.12 & 58.09 & 57.85 & 58.41 & 73.01 & 73.94 & 72.74 & 73.18  \\
 \hline
 16 & 57.93 & 58.88 & 57.93 & 58.56 & 73.94 & 72.80 & 73.01 & 73.94  \\
 \hline
\end{tabular}
\caption{\label{tab:mmlu_abalation} Accuracy on LM evaluation harness tasks on GPT3-1.3B model.}
\end{table}

\begin{table} \centering
\begin{tabular}{|c||c|c|c|c||c|c|c|c|} 
\hline
 $L_b \rightarrow$& \multicolumn{4}{c||}{8} & \multicolumn{4}{c||}{8}\\
 \hline
 \backslashbox{$L_A$\kern-1em}{\kern-1em$N_c$} & 2 & 4 & 8 & 16 & 2 & 4 & 8 & 16  \\
 %$N_c \rightarrow$ & 2 & 4 & 8 & 16 & 2 & 4 & 2 \\
 \hline
 \hline
 \multicolumn{5}{|c|}{Race (FP32 Accuracy = 41.34\%)} & \multicolumn{4}{|c|}{Boolq (FP32 Accuracy = 68.32\%)} \\ 
 \hline
 \hline
 64 & 40.48 & 40.10 & 39.43 & 39.90 & 69.20 & 68.41 & 69.45 & 68.56 \\
 \hline
 32 & 39.52 & 39.52 & 40.77 & 39.62 & 68.32 & 67.43 & 68.17 & 69.30  \\
 \hline
 16 & 39.81 & 39.71 & 39.90 & 40.38 & 68.10 & 66.33 & 69.51 & 69.42  \\
 \hline
 \hline
 \multicolumn{5}{|c|}{Winogrande (FP32 Accuracy = 67.88\%)} & \multicolumn{4}{|c|}{Piqa (FP32 Accuracy = 78.78\%)} \\ 
 \hline
 \hline
 64 & 66.85 & 66.61 & 67.72 & 67.88 & 77.31 & 77.42 & 77.75 & 77.64 \\
 \hline
 32 & 67.25 & 67.72 & 67.72 & 67.00 & 77.31 & 77.04 & 77.80 & 77.37  \\
 \hline
 16 & 68.11 & 68.90 & 67.88 & 67.48 & 77.37 & 78.13 & 78.13 & 77.69  \\
 \hline
\end{tabular}
\caption{\label{tab:mmlu_abalation} Accuracy on LM evaluation harness tasks on GPT3-8B model.}
\end{table}

\begin{table} \centering
\begin{tabular}{|c||c|c|c|c||c|c|c|c|} 
\hline
 $L_b \rightarrow$& \multicolumn{4}{c||}{8} & \multicolumn{4}{c||}{8}\\
 \hline
 \backslashbox{$L_A$\kern-1em}{\kern-1em$N_c$} & 2 & 4 & 8 & 16 & 2 & 4 & 8 & 16  \\
 %$N_c \rightarrow$ & 2 & 4 & 8 & 16 & 2 & 4 & 2 \\
 \hline
 \hline
 \multicolumn{5}{|c|}{Race (FP32 Accuracy = 40.67\%)} & \multicolumn{4}{|c|}{Boolq (FP32 Accuracy = 76.54\%)} \\ 
 \hline
 \hline
 64 & 40.48 & 40.10 & 39.43 & 39.90 & 75.41 & 75.11 & 77.09 & 75.66 \\
 \hline
 32 & 39.52 & 39.52 & 40.77 & 39.62 & 76.02 & 76.02 & 75.96 & 75.35  \\
 \hline
 16 & 39.81 & 39.71 & 39.90 & 40.38 & 75.05 & 73.82 & 75.72 & 76.09  \\
 \hline
 \hline
 \multicolumn{5}{|c|}{Winogrande (FP32 Accuracy = 70.64\%)} & \multicolumn{4}{|c|}{Piqa (FP32 Accuracy = 79.16\%)} \\ 
 \hline
 \hline
 64 & 69.14 & 70.17 & 70.17 & 70.56 & 78.24 & 79.00 & 78.62 & 78.73 \\
 \hline
 32 & 70.96 & 69.69 & 71.27 & 69.30 & 78.56 & 79.49 & 79.16 & 78.89  \\
 \hline
 16 & 71.03 & 69.53 & 69.69 & 70.40 & 78.13 & 79.16 & 79.00 & 79.00  \\
 \hline
\end{tabular}
\caption{\label{tab:mmlu_abalation} Accuracy on LM evaluation harness tasks on GPT3-22B model.}
\end{table}

\begin{table} \centering
\begin{tabular}{|c||c|c|c|c||c|c|c|c|} 
\hline
 $L_b \rightarrow$& \multicolumn{4}{c||}{8} & \multicolumn{4}{c||}{8}\\
 \hline
 \backslashbox{$L_A$\kern-1em}{\kern-1em$N_c$} & 2 & 4 & 8 & 16 & 2 & 4 & 8 & 16  \\
 %$N_c \rightarrow$ & 2 & 4 & 8 & 16 & 2 & 4 & 2 \\
 \hline
 \hline
 \multicolumn{5}{|c|}{Race (FP32 Accuracy = 44.4\%)} & \multicolumn{4}{|c|}{Boolq (FP32 Accuracy = 79.29\%)} \\ 
 \hline
 \hline
 64 & 42.49 & 42.51 & 42.58 & 43.45 & 77.58 & 77.37 & 77.43 & 78.1 \\
 \hline
 32 & 43.35 & 42.49 & 43.64 & 43.73 & 77.86 & 75.32 & 77.28 & 77.86  \\
 \hline
 16 & 44.21 & 44.21 & 43.64 & 42.97 & 78.65 & 77 & 76.94 & 77.98  \\
 \hline
 \hline
 \multicolumn{5}{|c|}{Winogrande (FP32 Accuracy = 69.38\%)} & \multicolumn{4}{|c|}{Piqa (FP32 Accuracy = 78.07\%)} \\ 
 \hline
 \hline
 64 & 68.9 & 68.43 & 69.77 & 68.19 & 77.09 & 76.82 & 77.09 & 77.86 \\
 \hline
 32 & 69.38 & 68.51 & 68.82 & 68.90 & 78.07 & 76.71 & 78.07 & 77.86  \\
 \hline
 16 & 69.53 & 67.09 & 69.38 & 68.90 & 77.37 & 77.8 & 77.91 & 77.69  \\
 \hline
\end{tabular}
\caption{\label{tab:mmlu_abalation} Accuracy on LM evaluation harness tasks on Llama2-7B model.}
\end{table}

\begin{table} \centering
\begin{tabular}{|c||c|c|c|c||c|c|c|c|} 
\hline
 $L_b \rightarrow$& \multicolumn{4}{c||}{8} & \multicolumn{4}{c||}{8}\\
 \hline
 \backslashbox{$L_A$\kern-1em}{\kern-1em$N_c$} & 2 & 4 & 8 & 16 & 2 & 4 & 8 & 16  \\
 %$N_c \rightarrow$ & 2 & 4 & 8 & 16 & 2 & 4 & 2 \\
 \hline
 \hline
 \multicolumn{5}{|c|}{Race (FP32 Accuracy = 48.8\%)} & \multicolumn{4}{|c|}{Boolq (FP32 Accuracy = 85.23\%)} \\ 
 \hline
 \hline
 64 & 49.00 & 49.00 & 49.28 & 48.71 & 82.82 & 84.28 & 84.03 & 84.25 \\
 \hline
 32 & 49.57 & 48.52 & 48.33 & 49.28 & 83.85 & 84.46 & 84.31 & 84.93  \\
 \hline
 16 & 49.85 & 49.09 & 49.28 & 48.99 & 85.11 & 84.46 & 84.61 & 83.94  \\
 \hline
 \hline
 \multicolumn{5}{|c|}{Winogrande (FP32 Accuracy = 79.95\%)} & \multicolumn{4}{|c|}{Piqa (FP32 Accuracy = 81.56\%)} \\ 
 \hline
 \hline
 64 & 78.77 & 78.45 & 78.37 & 79.16 & 81.45 & 80.69 & 81.45 & 81.5 \\
 \hline
 32 & 78.45 & 79.01 & 78.69 & 80.66 & 81.56 & 80.58 & 81.18 & 81.34  \\
 \hline
 16 & 79.95 & 79.56 & 79.79 & 79.72 & 81.28 & 81.66 & 81.28 & 80.96  \\
 \hline
\end{tabular}
\caption{\label{tab:mmlu_abalation} Accuracy on LM evaluation harness tasks on Llama2-70B model.}
\end{table}

%\section{MSE Studies}
%\textcolor{red}{TODO}


\subsection{Number Formats and Quantization Method}
\label{subsec:numFormats_quantMethod}
\subsubsection{Integer Format}
An $n$-bit signed integer (INT) is typically represented with a 2s-complement format \citep{yao2022zeroquant,xiao2023smoothquant,dai2021vsq}, where the most significant bit denotes the sign.

\subsubsection{Floating Point Format}
An $n$-bit signed floating point (FP) number $x$ comprises of a 1-bit sign ($x_{\mathrm{sign}}$), $B_m$-bit mantissa ($x_{\mathrm{mant}}$) and $B_e$-bit exponent ($x_{\mathrm{exp}}$) such that $B_m+B_e=n-1$. The associated constant exponent bias ($E_{\mathrm{bias}}$) is computed as $(2^{{B_e}-1}-1)$. We denote this format as $E_{B_e}M_{B_m}$.  

\subsubsection{Quantization Scheme}
\label{subsec:quant_method}
A quantization scheme dictates how a given unquantized tensor is converted to its quantized representation. We consider FP formats for the purpose of illustration. Given an unquantized tensor $\bm{X}$ and an FP format $E_{B_e}M_{B_m}$, we first, we compute the quantization scale factor $s_X$ that maps the maximum absolute value of $\bm{X}$ to the maximum quantization level of the $E_{B_e}M_{B_m}$ format as follows:
\begin{align}
\label{eq:sf}
    s_X = \frac{\mathrm{max}(|\bm{X}|)}{\mathrm{max}(E_{B_e}M_{B_m})}
\end{align}
In the above equation, $|\cdot|$ denotes the absolute value function.

Next, we scale $\bm{X}$ by $s_X$ and quantize it to $\hat{\bm{X}}$ by rounding it to the nearest quantization level of $E_{B_e}M_{B_m}$ as:

\begin{align}
\label{eq:tensor_quant}
    \hat{\bm{X}} = \text{round-to-nearest}\left(\frac{\bm{X}}{s_X}, E_{B_e}M_{B_m}\right)
\end{align}

We perform dynamic max-scaled quantization \citep{wu2020integer}, where the scale factor $s$ for activations is dynamically computed during runtime.

\subsection{Vector Scaled Quantization}
\begin{wrapfigure}{r}{0.35\linewidth}
  \centering
  \includegraphics[width=\linewidth]{sections/figures/vsquant.jpg}
  \caption{\small Vectorwise decomposition for per-vector scaled quantization (VSQ \citep{dai2021vsq}).}
  \label{fig:vsquant}
\end{wrapfigure}
During VSQ \citep{dai2021vsq}, the operand tensors are decomposed into 1D vectors in a hardware friendly manner as shown in Figure \ref{fig:vsquant}. Since the decomposed tensors are used as operands in matrix multiplications during inference, it is beneficial to perform this decomposition along the reduction dimension of the multiplication. The vectorwise quantization is performed similar to tensorwise quantization described in Equations \ref{eq:sf} and \ref{eq:tensor_quant}, where a scale factor $s_v$ is required for each vector $\bm{v}$ that maps the maximum absolute value of that vector to the maximum quantization level. While smaller vector lengths can lead to larger accuracy gains, the associated memory and computational overheads due to the per-vector scale factors increases. To alleviate these overheads, VSQ \citep{dai2021vsq} proposed a second level quantization of the per-vector scale factors to unsigned integers, while MX \citep{rouhani2023shared} quantizes them to integer powers of 2 (denoted as $2^{INT}$).

\subsubsection{MX Format}
The MX format proposed in \citep{rouhani2023microscaling} introduces the concept of sub-block shifting. For every two scalar elements of $b$-bits each, there is a shared exponent bit. The value of this exponent bit is determined through an empirical analysis that targets minimizing quantization MSE. We note that the FP format $E_{1}M_{b}$ is strictly better than MX from an accuracy perspective since it allocates a dedicated exponent bit to each scalar as opposed to sharing it across two scalars. Therefore, we conservatively bound the accuracy of a $b+2$-bit signed MX format with that of a $E_{1}M_{b}$ format in our comparisons. For instance, we use E1M2 format as a proxy for MX4.

\begin{figure}
    \centering
    \includegraphics[width=1\linewidth]{sections//figures/BlockFormats.pdf}
    \caption{\small Comparing LO-BCQ to MX format.}
    \label{fig:block_formats}
\end{figure}

Figure \ref{fig:block_formats} compares our $4$-bit LO-BCQ block format to MX \citep{rouhani2023microscaling}. As shown, both LO-BCQ and MX decompose a given operand tensor into block arrays and each block array into blocks. Similar to MX, we find that per-block quantization ($L_b < L_A$) leads to better accuracy due to increased flexibility. While MX achieves this through per-block $1$-bit micro-scales, we associate a dedicated codebook to each block through a per-block codebook selector. Further, MX quantizes the per-block array scale-factor to E8M0 format without per-tensor scaling. In contrast during LO-BCQ, we find that per-tensor scaling combined with quantization of per-block array scale-factor to E4M3 format results in superior inference accuracy across models. 


\bibliographystyle{unsrtnat}
\bibliography{refs}

\end{document}
