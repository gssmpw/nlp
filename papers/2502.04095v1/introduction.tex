
\chapter{Introduction}


\section{Motivation}\label{motivations}

In recent years, businesses worldwide have undergone a shift toward sustainable practices, driven by increasing environmental concerns, regulatory pressures, and stakeholder expectations \cite{ey}. In response to this, the International Financial Reporting Standards (IFRS) Foundation introduced sustainability-related disclosure standards that provide a comprehensive framework for companies to report on their environmental, social, and governance (ESG) impacts \citep{ifrsorgc9:online}. However, the complexity and evolving nature of these standards present challenges for organizations that strive to produce compliant sustainability reports. In particular, the preparation of such reports requires deep domain knowledge, the ability to interpret and apply complex guidelines, and an effort to keep up to date on changes to standards over time. This can present a barrier for companies that lack the resources or expertise to navigate these requirements effectively, which, in turn, can hinder compliance and impede the broader goal of promoting sustainable business practices.\\

This setting presents a much-needed application for AI to assist companies with knowledge of IFRS reporting standards. In fact, a growing interest is expressed in developing intelligent assistants for a wide range of corporate applications \cite{nicolescu2022human, kim2024ragqaragintegratinggenerative}. However, no work has been published to date developing such a system in IFRS sustainability reporting, despite this being a domain that is relevant to thousands of companies worldwide. This has resulted in a two-fold research gap: there are no published, high-quality question-answering systems based on this information, and there is no publicly available dataset that can be utilised to train and evaluate such systems in this domain. The lack of a dataset presents a hurdle towards implementing AI question-answering systems as they cannot be trained and evaluated appropriately. Simultaneously, the creation of such a dataset, whether by humans or by Large Language Models (LLMs), is a challenging task. The problem, therefore, becomes self-perpetuating.

\section{Objectives}
The aim of this project is to design, implement, and evaluate a question-answering system within the IFRS sustainability reporting domain. Specifically, we first strive to leverage LLMs to make a question-answer (QA) dataset that we can then use to develop an architecture for answering user queries on this topic. The objectives of this project are to: 

\begin{enumerate}
    \item Define the knowledge scope of the assistant and the typical queries a user would ask within this domain.
    \item Design and implement a QA generation and evaluation pipeline using state-of-the-art LLMs.
    \item Produce a synthetic QA dataset based on the IFRS sustainability reporting standards.
    \item Leverage LLMs and retrieval augmented generation (RAG) systems to develop a robust architecture for question-answering on the sustainability reporting domain.
    \item Evaluate and optimise the system's performance on a diverse range of user queries.
\end{enumerate}

\section{Contributions}

This work presents a two-fold solution to the problem defined in Section \ref{motivations}. We present a first-of-its-kind test suite for evaluating LLMs on the question-answering task within the sustainability reporting domain and design an assistant that provides accurate guidance on sustainability reporting by leveraging advancements in LLMs and RAG. This project presents several contributions:
\begin{enumerate}
    \item A novel dataset of diverse, synthetically generated question-answer (QA) pairs, in both multiple-choice and free-text format, based on IFRS Sustainability Reporting Standards.
    \item A custom QA generation and evaluation pipeline.
    \item Custom LLM-based evaluation metrics for quality control of synthetically-generated QA pairs.
    \item A systematic evaluation of language models and RAG methods using the synthesised QA dataset.
    \item Two custom architectures for sustainability reporting question-answering - a RAG-based pipeline and a fully LLM-based pipeline. Both of these integrate an industry classifier and an LLM fine-tuned on the QA dataset.
\end{enumerate}





