% % % % % % % % % % Document class % % % % % % % % % %
\documentclass[pdflatex, sn-apa]{sn-jnl}

% % % % % % % % % % Packages % % % % % % % % % %
\usepackage{amsmath}
\usepackage{anyfontsize}
\usepackage[utf8]{inputenc}
\usepackage[style = apa, backend = biber]{biblatex}
\usepackage[T1]{fontenc}
\usepackage{lmodern}
\usepackage{microtype}

% % % % % % % % % % Commands % % % % % % % % % %
\addbibresource{bibliography.bib}

% % % % % % % % % % Document % % % % % % % % % %
\begin{document}
%
\thispagestyle{empty}
\begin{center}
    %
	{
        \large
        Information sheet
    }%
    \\ \medskip%
	%
	{
        \LARGE
        \textbf{A comparative analysis of rank aggregation methods for the partial label ranking problem}
    }%
    \\ \medskip%
	%
	Jiayi Wang, Juan C. Alfaro and Viktor Bengs
	 
\end{center}

\paragraph{What is the main claim of the paper?}
\paragraph{Why is this an important contribution to the machine learning/data mining literature?} 
%
%
This paper aims to reintroduce various rank aggregation methods to the research community and systematically assess their suitability for partial label ranking problems.
%
The range of methods investigated spans from simple, \textit{scoring-based} approaches to more advanced, \textit{probabilistic-based} methods.
%
A key contribution of this work is adapting certain ranking aggregation methods to address partial label ranking scenarios~effectively.
%
%

This contribution is significant as the  variety of aggregation methods employed in partial label ranking remains limited, primarily relying on the \textit{optimal bucket order} problem.
%
The optimal bucket order problem is a generalization of the \textit{Kemeny consensus}, widely used in label ranking problems and other subfields of preference learning. 
%
However, both are NP-hard problems, often requiring complex approximation techniques, which influences the complexity of the learning and inference processes of all methods built around these rank aggregation approaches.
%
%

We find that, surprisingly, the simple scoring-based variants already have at least as good, and in some cases even better, predictive accuracy than the complex variants.
%
This is good news for the effective use of resources, as the simple methods allow for more cost-effective training and inference processes. 
%
Since the extensions of suggested simple aggregation methods have a critical hyperparameter, we also propose a technique for obtaining suitable hyperparameters for the latter based on the dataset's metadata.
%
%

\paragraph{What is the evidence provided to support claims?}
%
%
We carried out extensive experiments in which we
%
\begin{itemize}
%
    \item used standard benchmarks,
    %
    \item applied training and testing procedures according to the usual scientific standards,
    %
    \item employed the well-established metric of the rank correlation coefficient to measure the prediction quality, 
    %
    \item and utilized the Friedman test and Holm's procedure to compare the methods statistically.
%
\end{itemize}
%
This rigorous scientific setup allows us to conclude the above claims of the paper.
%
%

\paragraph{Report 3-5 most closely related contributions in the past 7 years (authored by researchers outside the authors’ research group) and briefly state the relation of the submission to them.}
%
The following contributions are the most recent works within the field of the label ranking problem, which is a special case of our setting and, therefore, quite closely~related.
%
\begin{itemize}
    %
    \item \textcite{zhou_heuristic_2024} considers heuristic search methods for rank aggregation within the label ranking problem.
    %
    \item \textcite{adam_inferring_2024} extends the Plackett–Luce model by using imprecise probabilities as the mathematical tool and leveraging this (imprecise) statistical ranking model for the label ranking problem. 
    %
    \item \textcite{albano_weighted_2023} enhances the state-of-the-art decision tree ensemble model for label ranking by leveraging similarities and individual label importance. 
    %
\end{itemize}
%
It is worth mentioning that the partial label ranking is a quite recent generalization of the established label ranking problem and \textbf{all} contributions are made by one of the authors of this manuscript.
%
%

\paragraph{Specify 5 general keywords and 5 specific keywords describing the main research activity presented in the manuscript.}
%
%
General keywords:
%
\begin{itemize}
    %
    \item Supervised learning
    %
    \item Preference learning
    %
    \item Label ranking problem 
    %
    \item Partial label ranking problem
    %
    \item Rank aggregation problem
    %
\end{itemize}
%
Specific keywords:
%
\begin{itemize}
    %
    \item Copeland ranking
    %
    \item Borda ranking
    %
    \item Maximal lottery ranking
    %
    \item Markov chain ranking
    %
    \item Optimal bucket order problem 
    %
\end{itemize}

\paragraph{Most appropriate reviewers.}
%
%
To ensure a thorough and insightful analysis of this contribution, we recommend the following reviewers as the most appropriate:
%
\begin{itemize}
    %
	\item Sébastien Destercke, Université de Technologie de Compiegne  (\url{https://www.hds.utc.fr/~sdesterc/dokuwiki/doku.php}), who has recently contributed to the field of label ranking, a special case of our setting and thus quite closely related.
    %
    \item Weiwei Cheng, Senior Principal Scientist at Zalando (\url{https://www.weiweicheng.com/research/}), who made significant contributions to the subfield of label ranking.
    %
    \item Krzysztof Dembczynski, Poznań University of Technology, (\url{http://www.cs.put.poznan.pl/kdembczynski/}), a member of the \textit{Machine Learning} Journal Editorial Board, who has also made early contributions to label ranking within the preference learning domain.
    %
	\item Paolo Viappiani, Université Paris Dauphine-PSL. (\url{https://www.lamsade.dauphine.fr/~pviappiani/Paolo%20VIAPPIANI.html}), an expert in the broader field of preference learning.
%	
\end{itemize}
%
%

\paragraph{Competing interests.}
%
%
The following researchers have competing or conflicting interests due to recent collaborations with one of the authors:
%
\begin{itemize}
%
	\item Robert Busa-Fekete, Google Research;
    %
    \item Eyke Hüllermeier, Ludwig-Maximilians-Universität;
    %
    \item Willem Waegeman, Universiteit Gent;
    %
    \item José A. Gámez, Universidad de Castilla-La Mancha;
    %
    \item Juan A. Aledo, Universidad de Castilla-La Mancha;
    %
    \item José M. Puerta, Universidad de Castilla-La Mancha;
    %
    \item Pablo Torrijos, Universidad de Castilla-La Mancha;
%
\end{itemize}
%
%

\paragraph{Conflicting email domains.}
%
%
\begin{itemize}
    %
	\item Viktor Bengs: @uni-marburg.de; @upb.de; @uni-paderborn.de; @lmu.de
    %
    \item Juan Carlos Alfaro: @uclm.es
    %
\end{itemize}
%
%

% % % % % % % % % % Bibliography % % % % % % % % % %
\printbibliography

\end{document}