\section{Related Work}
\paragraph{LLM Agent with Memory.}
Memory storing user-agent interactions provides valuable insights for LLM agents in solving real-word applications, making it an essential component of LLM agents \cite{survey_agent_memory}. 
However, while equipping LLM agents with memory improves performance, it also introduces privacy risks. 
For instance, healthcare agents \cite{EHRAgent_shi_emnlp24,chatdoctor_li_23} store sensitive information about patients, web application agents \cite{RAP_kagaya_24} record user preferences, and autonomous driving agents \cite{AgentDriver_mao_colm24,DiLu_autodriving_wen_iclr24} accumulate past driving scenarios. 
As these memory modules inherently store highly sensitive user data, a systematic investigation into the risks of memory leakage is crucial for revealing and mitigating potential threats. 



\vspace{-6pt}
\paragraph{Privacy Risk in RAG.}
Recent works in RAG have extensively explored the privacy issues associated with external data. 
\citet{RAG-privacy_zeng_ACL24} first revealed that the private data integrated into RAG systems is vulnerable to manually crafted adversarial prompts, while \citet{RAG_privacy_qi_24} conducted a more comprehensive investigation across multiple RAG configurations. 
To automate extraction, \citet{RAG_thief_jiang_24} developed an agent-based attack, and \citet{RAG_privacy_Maio_24} proposed an adaptive strategy to progressively extract the private knowledge. 
These works suggest that similar privacy threats can arise in LLM agents, owing to the similar data retrieval mechanisms employed by both systems.