% This must be in the first 5 lines to tell arXiv to use pdfLaTeX, which is strongly recommended.
\pdfoutput=1
% In particular, the hyperref package requires pdfLaTeX in order to break URLs across lines.

\documentclass[11pt]{article}

% Change "review" to "final" to generate the final (sometimes called camera-ready) version.
% Change to "preprint" to generate a non-anonymous version with page numbers.
\usepackage[]{acl}
\definecolor{darkred}{rgb}{0.8, 0.0, 0.0}
% Standard package includes
\usepackage{times}
\usepackage{latexsym}
\usepackage{hyperref}
\usepackage{booktabs}
\usepackage{multirow}
\usepackage{amsmath}
\usepackage{seqsplit}
\usepackage{amsmath}
\usepackage{amssymb}
\usepackage{amsfonts}
\usepackage{graphicx}
\usepackage{subcaption}
\usepackage{longtable}
\usepackage{float}

% For proper rendering and hyphenation of words containing Latin characters (including in bib files)
\usepackage[T1]{fontenc}
% For Vietnamese characters
% \usepackage[T5]{fontenc}
% See https://www.latex-project.org/help/documentation/encguide.pdf for other character sets

% This assumes your files are encoded as UTF8
\usepackage[utf8]{inputenc}

% This is not strictly necessary, and may be commented out,
% but it will improve the layout of the manuscript,
% and will typically save some space.
\usepackage{microtype}

% This is also not strictly necessary, and may be commented out.
% However, it will improve the aesthetics of text in
% the typewriter font.
\usepackage{inconsolata}

%Including images in your LaTeX document requires adding
%additional package(s)
\usepackage{graphicx}
\usepackage{color}
\usepackage{xcolor, colortbl}
\usepackage[most]{tcolorbox}
\usepackage{soul}
\newcommand{\fix}{\marginpar{FIX}}
\newcommand{\new}{\marginpar{NEW}}
\newcommand{\lz}[1]{\textcolor{blue}{{#1}}}
\newcommand{\green}[1]{\textcolor{green}{{#1}}}
\newcommand{\red}[1]{\textcolor{red}{{#1}}}
% \newcommand{\wanlong}[1]{{ \color{teal} [Wanlong says ``#1'']}}



\newcommand{\mycolorbox}[2]{%
  \colorbox{#1}{\raisebox{0pt}[8pt][0pt]{#2}}%
}

\definecolor{lightpink}{rgb}{0.945, 0.816, 0.804}
\definecolor{lightgreen}{rgb}{0.851, 0.906, 0.839}
\definecolor{lightblue}{rgb}{0.8, 0.9, 1}
\definecolor{lightyellow}{rgb}{0.992, 0.949, 0.816}

\newtcolorbox[list inside=prompt,auto counter,number within=section]{prompt}[1][]{
    colbacktitle=black!60,
    coltitle=white,
    fontupper=\footnotesize,
    boxsep=5pt,
    breakable,
    enhanced,
    left=0pt,
    right=0pt,
    top=0pt,
    bottom=0pt,
    boxrule=1pt,
    #1   
}
% If the title and author information does not fit in the area allocated, uncomment the following
%
%\setlength\titlebox{<dim>}
%
% and set <dim> to something 5cm or larger.

\title{Large Language Models Penetration in Scholarly Writing and Peer Review}

\newcommand{\cuhksz}{$^1$}
\newcommand{\uestc}{$^2$}
\newcommand{\whut}{$^3$}
\newcommand{\ku}{$^4$}
\newcommand{\sribd}{$^5$}



% Li, Ruijie, Xunlian, Daniel, Haizhou

% Author information can be set in various styles:
% For several authors from the same institution:
% \author{Author 1 \and ... \and Author n \\
%         Address line \\ ... \\ Address line}
% if the names do not fit well on one line use
%         Author 1 \\ {\bf Author 2} \\ ... \\ {\bf Author n} \\
% For authors from different institutions:
% \author{Author 1 \\ Address line \\  ... \\ Address line
%         \And  ... \And
%         Author n \\ Address line \\ ... \\ Address line}
% To start a separate ``row'' of authors use \AND, as in
% \author{Author 1 \\ Address line \\  ... \\ Address line
%         \AND
%         Author 2 \\ Address line \\ ... \\ Address line \And
%         Author 3 \\ Address line \\ ... \\ Address line}

% \author{First Author \\
%   Affiliation / Address line 1 \\
%   Affiliation / Address line 2 \\
%   Affiliation / Address line 3 \\
%   \texttt{email@domain} \\\And
%   Second Author \\
%   Affiliation / Address line 1 \\
%   Affiliation / Address line 2 \\
%   Affiliation / Address line 3 \\
%   \texttt{email@domain} \\}
% Li Zhou\cuhksz, Ruijie Zhang\uestc, Xunlian Dai\whut, Daniel Hershcovich\ku, Haizhou Li\cuhksz$^,$\sribd
\author{
 \textbf{Li Zhou\cuhksz},
 \textbf{Ruijie Zhang\uestc},
 \textbf{Xunlian Dai\whut},
 \textbf{Daniel Hershcovich\ku},
 \textbf{Haizhou Li\cuhksz$^,$\sribd}
\\
{\cuhksz}The Chinese University of Hong Kong, Shenzhen \\ 
{\uestc}University of Electronic Science and Technology of China \\ 
{\whut}Wuhan University of Technology {\ku}University of Copenhagen \\
{\sribd}Shenzhen Research Institute of Big Data\\
 \small{
   % \textbf{Correspondence:} 
   \texttt{lizhou21@cuhk.edu.cn}
 }
}

\begin{document}
\maketitle
\begin{abstract}

While the widespread use of Large Language Models (LLMs) brings convenience, it also raises concerns about the credibility of academic research and scholarly processes. To better understand these dynamics, we evaluate the penetration of LLMs across academic workflows from multiple perspectives and dimensions, providing compelling evidence of their growing influence. We propose a framework with two components: \texttt{ScholarLens}, a curated dataset of human-written and LLM-generated content across scholarly writing and peer review for multi-perspective evaluation, and \texttt{LLMetrica}, a tool for assessing LLM penetration using rule-based metrics and model-based detectors for multi-dimensional evaluation. Our experiments demonstrate the effectiveness of \texttt{LLMetrica}, revealing the increasing role of LLMs in scholarly processes. These findings emphasize the need for transparency, accountability, and ethical practices in LLM usage to maintain academic credibility.

% We propose a framework with two components: \texttt{ScholarLens}, a curated dataset for technical measurement and development, includes both human-written and LLM-generated content across scholarly writing and peer review, enabling multi-perspective evaluation. and \texttt{LLMetrica}, a tool to assess LLM penetration, combines rule-based metrics to capture linguistic and semantic features with model-based detectors to identify LLM-generated content, supporting multi-dimensional evaluation.
% \texttt{ScholarLens} includes both human-written and LLM-generated content across scholarly writing and peer review, enabling multi-perspective evaluation. 
% \texttt{LLMetrica} combines rule-based metrics to capture linguistic and semantic features with model-based detectors to identify LLM-generated content, supporting multi-dimensional evaluation.


% Specifically, We propose a framework with two key components: \texttt{ScholarLens}, a curated dataset for developing technical measurement methods, and \texttt{LLMetrica} (\S\ref{sec: LLMetria}), a set of metrics to assess LLM penetration in scholarly workflows.
% \texttt{ScholarLens} includes both human-written and LLM-generated content, considering both scholarly writing and peer review, providing a 
% foundation for multiple perspectives evaluation; 
% \texttt{LLMetrica} combines rule-based metrics to capture linguistic and semantic feature preference of LLM-generated text, along with training scholar-specific model-based detectors to better identify whether content is LLM-generated. 

% While the widespread use of Large Language Models (LLMs) brings convenience, it also raises concerns about the credibility of academic research and scholarly processes.
% In this paper, we quantify the increasing penetration of LLMs across various academic workflows, examining their influence from multiple perspectives.
% 区分LLM 辅助角色的重要性


% The rapid advancements in Large Language Models (LLMs) have significantly impacted scholars, influencing activities ranging from research authorship to peer review. This has raised concerns among key figures, including journal editors, conference chairs, and authors.
% Understanding these dynamics is crucial for addressing risks to academic integrity and developing strategies for the ethical and effective use of LLMs. 
% In this paper, we quantify recent trends in LLM penetration across various aspects of academia, exploring their impact from multiple dimensions.
% Specifically, we construct a comprehensive dataset \texttt{ScholarLens} that includes both human- and LLM-written content, sourced from both author and reviewer perspectives.
% Additionally, we propose a set of quantifiable metrics \texttt{LLMetrica} for evaluating LLM penetration, incorporating interpretable measures alongside trained LLM detection models.
% By leveraging \texttt{ScholarLens} and \texttt{LLMetrica}, we present a structured framework for evaluating LLM penetration,  revealing notable trends in the increasing adoption of LLMs, highlighting areas of concern and offering actionable strategies for managing their integration. 


% The emergence of large language models (LLMs), particularly ChatGPT, has significantly influenced academic writing, necessitating an exploration of their impact on scholarly communication.
% we focus on three aspects about academic domain: title generation, abstract polish, and meta-review creation.
% We develop \lz{a comprehensive evaluation suite} and an \lz{AIGC detection tool} to analyze academic content produced before and after the introduction of ChatGPT. 
% \lz{We find that ……}
% \lz{These insights underscore……}
% These insights underscore the profound significance of establishing frameworks for editors and chairs to monitor AI engagement, ensuring the integrity and quality of academic discourse in an increasingly automated landscape.

\end{abstract}



% humans are sensitive to the way information is presented.

% introduce framing as the way we address framing. say something about political views and how information is represented.

% in this paper we explore if models show similar sensitivity.

% why is it important/interesting.



% thought - it would be interesting to test it on real world data, but it would be hard to test humans because they come already biased about real world stuff, so we tested artificial.


% LLMs have recently been shown to mimic cognitive biases, typically associated with human behavior~\citep{ malberg2024comprehensive, itzhak-etal-2024-instructed}. This resemblance has significant implications for how we perceive these models and what we can expect from them in real-world interactions and decisionmaking~\citep{eigner2024determinants, echterhoff-etal-2024-cognitive}.

The \textit{framing effect} is a well-known cognitive phenomenon, where different presentations of the same underlying facts affect human perception towards them~\citep{tversky1981framing}.
For example, presenting an economic policy as only creating 50,000 new jobs, versus also reporting that it would cost 2B USD, can dramatically shift public opinion~\cite{sniderman2004structure}. 
%%%%%%%% 图1:  %%%%%%%%%%%%%%%%
\begin{figure}[t]
    \centering
    \includegraphics[width=\columnwidth]{Figs/01.pdf}
    \caption{Performance comparison (Top-1 Acc (\%)) under various open-vocabulary evaluation settings where the video learners except for CLIP are tuned on Kinetics-400~\cite{k400} with frozen text encoders. The satisfying in-context generalizability on UCF101~\cite{UCF101} (a) can be severely affected by static bias when evaluating on out-of-context SCUBA-UCF101~\cite{li2023mitigating} (b) by replacing the video background with other images.}
    \label{fig:teaser}
\end{figure}


Previous research has shown that LLMs exhibit various cognitive biases, including the framing effect~\cite{lore2024strategic,shaikh2024cbeval,malberg2024comprehensive,echterhoff-etal-2024-cognitive}. However, these either rely on synthetic datasets or evaluate LLMs on different data from what humans were tested on. In addition, comparisons between models and humans typically treat human performance as a baseline rather than comparing patterns in human behavior. 
% \gabis{looks good! what do we mean by ``most studies'' or ``rarely'' can we remove those? or we want to say that we don't know of previous work doing both at the same time?}\gili{yeah the main point is that some work has done each separated, but not all of it together. how about now?}

In this work, we evaluate LLMs on real-world data. Rather than measuring model performance in terms of accuracy, we analyze how closely their responses align with human annotations. Furthermore, while previous studies have examined the effect of framing on decision making, we extend this analysis to sentiment analysis, as sentiment perception plays a key explanatory role in decision-making \cite{lerner2015emotion}. 
%Based on this, we argue that examining sentiment shifts in response to reframing can provide deeper insights into the framing effect. \gabis{I don't understand this last claim. Maybe remove and just say we extend to sentiment analysis?}

% Understanding how LLMs respond to framing is crucial, as they are increasingly integrated into real-world applications~\citep{gan2024application, hurlin2024fairness}.
% In some applications, e.g., in virtual companions, framing can be harnessed to produce human-like behavior leading to better engagement.
% In contrast, in other applications, such as financial or legal advice, mitigating the effect of framing can lead to less biased decisions.
% In both cases, a better understanding of the framing effect on LLMs can help develop strategies to mitigate its negative impacts,
% while utilizing its positive aspects. \gabis{$\leftarrow$ reading this again, maybe this isn't the right place for this paragraph. Consider putting in the conclusion? I think that after we said that people have worked on it, we don't necessarily need this here and will shorten the long intro}


% If framing can influence their outputs, this could have significant societal effects,
% from spreading biases in automated decision-making~\citep{ghasemaghaei2024understanding} to reducing public trust in AI-generated content~\citep{afroogh2024trust}. 
% However, framing is not inherently negative -- understanding how it affects LLM outputs can offer valuable insights into both human and machine cognition.
% By systematically investigating the framing effect,


%It is therefore crucial to systematically investigate the framing effect, to better understand and mitigate its impact. \gabis{This paragraph is important - I think that right now it's saying that we don't want models to be influenced by framing (since we want to mitigate its impact, right?) When we talked I think we had a more nuanced position?}




To better understand the framing effect in LLMs in comparison to human behavior,
we introduce the \name{} dataset (Section~\ref{sec:data}), comprising 1,000 statements, constructed through a three-step process, as shown in Figure~\ref{fig:fig1}.
First, we collect a set of real-world statements that express a clear negative or positive sentiment (e.g., ``I won the highest prize'').
%as exemplified in Figure~\ref{fig:fig1} -- ``I won the highest prize'' positive base statement. (2) next,
Second, we \emph{reframe} the text by adding a prefix or suffix with an opposite sentiment (e.g., ``I won the highest prize, \emph{although I lost all my friends on the way}'').
Finally, we collect human annotations by asking different participants
if they consider the reframed statement to be overall positive or negative.
% \gabist{This allows us to quantify the extent of \textit{sentiment shifts}, which is defined as labeling the sentiment aligning with the opposite framing, rather then the base sentiment -- e.g., voting ``negative'' for the statement ``I won the highest prize, although I lost all my friends on the way'', as it aligns with the opposite framing sentiment.}
We choose to annotate Amazon reviews, where sentiment is more robust, compared to e.g., the news domain which introduces confounding variables such as prior political leaning~\cite{druckman2004political}.


%While the implications of framing on sensitive and controversial topics like politics or economics are highly relevant to real-world applications, testing these subjects in a controlled setting is challenging. Such topics can introduce confounding variables, as annotators might rely on their personal beliefs or emotions rather than focusing solely on the framing, particularly when the content is emotionally charged~\cite{druckman2004political}. To balance real-world relevance with experimental reliability, we chose to focus on statements derived from Amazon reviews. These are naturally occurring, sentiment-rich texts that are less likely to trigger strong preexisting biases or emotional reactions. For instance, a review like ``The book was engaging'' can be framed negatively without invoking specific cultural or political associations. 

 In Section~\ref{sec:results}, we evaluate eight state-of-the-art LLMs
 % including \gpt{}~\cite{openai2024gpt4osystemcard}, \llama{}~\cite{dubey2024llama}, \mistral{}~\cite{jiang2023mistral}, \mixtral{}~\cite{mistral2023mixtral}, and \gemma{}~\cite{team2024gemma}, 
on the \name{} dataset and compare them against human annotations. We find  that LLMs are influenced by framing, somewhat similar to human behavior. All models show a \emph{strong} correlation ($r>0.57$) with human behavior.
%All models show a correlation with human responses of more than $0.55$ in Pearson's $r$ \gabis{@Gili check how people report this?}.
Moreover, we find that both humans and LLMs are more influenced by positive reframing rather than negative reframing. We also find that larger models tend to be more correlated with human behavior. Interestingly, \gpt{} shows the lowest correlation with human behavior. This raises questions about how architectural or training differences might influence susceptibility to framing. 
%\gabis{this last finding about \gpt{} stands in opposition to the start of the statement, right? Even though it's probably one of the largest models, it doesn't correlate with humans? If so, better to state this explicitly}

This work contributes to understanding the parallels between LLM and human cognition, offering insights into how cognitive mechanisms such as the framing effect emerge in LLMs.\footnote{\name{} data available at \url{https://huggingface.co/datasets/gililior/WildFrame}\\Code: ~\url{https://github.com/SLAB-NLP/WildFrame-Eval}}

%\gabist{It also raises fundamental philosophical and practical questions -- should LLMs aim to emulate human-like behavior, even when such behavior is susceptible to harmful cognitive biases? or should they strive to deviate from human tendencies to avoid reproducing these pitfalls?}\gabis{$\leftarrow$ also following Itay's comment, maybe this is better in the dicsussion, since we don't address these questions in the paper.} %\gabis{This last statement brings the nuance back, so I think it contradicts the previous parapgraph where we talked about ``mitigating'' the effect of framing. Also, I think it would be nice to discuss this a bit more in depth, maybe in the discussion section.}








\section{Related Work}

% \lz{Li is writing this section}



Recent discussions in the scientific community have focused on improving the peer review process~\cite{gurevych_et_al:DagRep.14.1.130, kuznetsov2024natural} to address issues like misalignment between reviewers and paper topics, as well as social~\cite{huber2022nobel, tomkins2017reviewer, manzoor2021uncovering} and cognitive biases~\cite{lee2015commensuration, stelmakh2021prior}. Proposed solutions include enhancing structural incentives for reviewers~\cite{rogers-augenstein-2020-improve}, using natural language processing for intelligent support~\cite{kuznetsov-etal-2022-revise, zyska-etal-2023-care, dycke-etal-2023-nlpeer, guo-etal-2023-automatic, kumar-etal-2023-reviewers}, and other policy recommendations~\cite{dycke-etal-2022-yes}. Furthermore, some studies focus on the collection and analysis of review data~\cite{kennard-etal-2022-disapere, staudinger-etal-2024-analysis, darcy-etal-2024-aries}.\footnote{\href{https://arr-data.aclweb.org/}{ACL Rolling Review Data Collection (ARR-DC)}.} 
However, these efforts largely focus on human reviewers: what if instead the reviewers are LLMs~\cite{weber2024other, gao2024reviewer2, hossain2025llmsmetareviewersassistantscase}?

Previous research has demonstrated that human-written and LLM-generated texts exhibit distinct linguistic characteristics~\cite{cheng2024beyond, song-etal-2025-assessing}.
For instance, LLM-generated texts often display the recurrent use of specific syntactic templates~\cite{shaib-etal-2024-detection}, which largely reflect patterns learned from the training data and highlight the model's memorization capacity~\cite{karamolegkou-etal-2023-copyright, zeng-etal-2024-exploring, zhu-etal-2024-beyond}.
Furthermore, some studies develop detection models to identify LLM-generated text~\cite{antoun-etal-2024-text, cheng2024beyond, xu-etal-2024-detecting, kumar-etal-2024-quis, abassy-etal-2024-llm}.
However, \citet{cheng2024beyond} points out that supervised detectors exhibit poor cross-domain generalizability.

Unlike previous work, we simulate LLM usage in scholarly writing and peer review, creating a comparison between LLM-generated and human-written texts.
We also develop a robust framework to assess the distinctive tendencies of LLM-generated content and identify it within the scholarly domain. This framework offers a comprehensive approach to evaluating LLM penetration from multiple perspectives and dimensions.

% Unlike previous works, we create the \texttt{ScholarLens} dataset, which includes both human-written and LLM-generated text, taking into account the roles of both authors and (meta-)reviewers. Our research focuses on developing a methodological framework, \texttt{LLMetrica}, to assess the distinctive characteristics of LLM-generated text and accurately identify scholar-domain LLM-generated content. This framework facilitates the evaluation and prediction of LLM penetration in scholarly writing from multiple perspectives.

% [31][32][34][35][38][43][45][46]
% For instance, LLM-generated text often repeats specific syntactic patterns from the training data~\cite{shaib-etal-2024-detection}, reflecting the model's memorization capacity~\cite{karamolegkou-etal-2023-copyright, zeng-etal-2024-exploring, zhu-etal-2024-beyond}.


% The recurrent use of specific syntactic templates in LLM-generated text largely reflects patterns from the training data~\cite{shaib-etal-2024-detection}, 






% AgentReview~\cite{jin2024agentreview}

% BUST: Benchmark for the evaluation of detectors of LLM-Generated Text~\cite{cornelius-etal-2024-bust}


% % \subsection{Large Language Models Penetration}


% \lz{\cite{wei2024understanding}: this paper performs the first analysis of the impact that AIGC has had on the social media ecosystem, through the lens of Pixiv.}

% With the rapid development of large language models (LLMs), the penetration of LLM technology into various fields has deepened,and especially academia, such as science journal writing~\cite{jiang2024llm}, Academic information seeking~\cite{wang2024solution} by using a SOAY tool to enhances LLMs' academic paper retrieval, and peer review generation tasks~\cite{su-etal-2023-reviewriter},
% which provides a tool for writing peer reviews in German.
% Simultaneously, some related studies have proposed methods utilizing large language models (LLMs) to facilitate the work of paper reviewers.
% ~\cite{weber2024other} provide a fine-tuned model used to automate peer reviews which is named RoboReviewe,
% and ~\cite{gao2024reviewer2optimizingreviewgeneration} purpose propose an efficient two-stage review generation framework for helping reviewer to find the deficiency of the sketched review.
% Meanwhile, ~\cite{santu2024promptingllmscomposemetareview} Introduced a method to compose meta-review drafts from peer-review narratives.

% Although the use of LLM-generated review/meta-review has facilitated the paper review process,
% the widespread application of these technologies raises concerns about the level of engagement and
% effort put in by reviewers. 
% ~\cite{jin-etal-2024-agentreview} reported that 15.8\% of ICLR 2024 reviews were AI-assisted, with AI-assisted reviews consistently assigning higher scores than their human counterparts and AI-assisted reviews contributed to a 4.9 percentage point increase in acceptance rates for borderline submissions.
% ~\cite{du-etal-2024-llms} collaborated with 40 experts in LLM-related reviewing to conduct a fine-grained evaluation of reviews generated by LLMs.
% Moreover, ~\cite{zhou-etal-2024-llm} shows that LLM generate review prefers excessively positive word than human.
% both of them reveal a signal of the importance to investigate the involvement of AI in the this process.In this work, we make a distinction from previous studies by focusing on the participation rate of AI-generated text in the paper review process.

% LLMs have developed a strong capability in text generation tasks. Specifically in the academic domain, \cite{birhane2023science} have highlighted the convenience of LLMs in peer review and abstract writing. During the peer review process, some reviewers utilize LLMs for summary tasks to aid in reading, and even instruct LLMs to generate review sketches\cite{song2024assessing}. Additionally, some meta-reviewers will employ LLMs to summarize reviewers' suggestions for creating meta-reviews\cite{santu2024prompting}. However, the summary results produced by LLMs may not fully capture the research information of a paper. Reviewers who overly rely on these summary results may compromise the objectivity and authenticity of the review process.

% On the other hand, in abstract writing tasks, LLMs are often used for polishing and text completion\cite{khalifa2024using}, such as generating the Background for the abstract. This, however, poses the risk of making published outputs more uniform, potentially reducing the distinctive voice and style of human-authored works. In above, We believe that studying the penetration of LLMs from these two aspects can effectively reflect the extent of LLM penetration in academia.






% \lz{Summarize current research on LLM penetration across multiple domains, particularly highlighting advancements and impacts within the academic domain.}

% \lz{Perhaps we can summarize an analytical perspective of the existing work and compare it}





% \lz{@Li Zhou: including Detection Model, Linguistic Feature metric, and so on}

% candidate:
% automatic metric: BLEU, ROUGE, ……
% Linguistic Features: ……


% \lz{look through: start}

\section{\texttt{ScholarLens} Curation}
\label{sec: ScholarLens}
% To investigate the penetration rate of LLMs in these stages, 

In this section, we detail the process of curating \texttt{ScholarLens}, including the consideration of data types and the setup for collecting both human-written and LLM-generated text.

% the collection setup of human-written and LLM-generated text.



% criteria for paper selection, the collection of human-written and LLM-generated reviews, the annotation procedure, and the measures taken to ensure data quality.

% \subsection{What Aspects of LLMs in Academia Are Considered?}




% \subsection{How to Construct LLM-Generated Data for Comparison?}

% \begin{table}[]
% \centering
% \scalebox{0.85}{
% \begin{tabular}{@{}l|rrrrrr@{}}
% \toprule
% \textbf{Source} & \textbf{$\leq 2019$} & \textbf{$2020$} & \textbf{$2021$} & \textbf{$2022$} & \textbf{$2023$} & \textbf{$2024$} \\ \midrule
% \textbf{ICLR}   & 2831           & 2213          & 2594          & 2619          & 3797          & 5780          \\
% \textbf{ACL}    &                &               &               &               &               &               \\
% \textbf{EMNLP}  &                &               &               &               &               &               \\
% \textbf{}       &                &               &               &               &               &               \\ \bottomrule
% \end{tabular}}
% \end{table}

\subsection{Data Types}

% We develope a dataset, \texttt{ScholarLens}, including both human from two distinct perspectives: ``refined'' and ``synthesized''. The ``refined'' perspective focuses on enhancing and polishing existing drafts, while the ``synthesized'' perspective involves producing content based on provided texts.  In this section, we describe the process of curating \texttt{ScholarLens}, including considerations of academic aspects and the collection of LLM-generated text details.

We formalize a research paper as $\mathrm{P}=\left\{ \mathrm{T},\mathrm{A},\mathrm{C},\mathrm{R},\mathrm{MR} \right\}$ where $\mathrm{T}$, $\mathrm{A}$, and $\mathrm{C}$ represent the title, abstract, and main content, respectively, and $\mathrm{R}=\left\{ r_i \right\}$ denotes individual reviews, with $\mathrm{MR}$ representing the meta-review summarizing feedback from multiple reviewers.
Since the process of creating a research paper involves both author drafting and peer review stages, our dataset includes content from both the author and (meta-)reviewer roles. 
When creating LLM-generated text, we consider two perspectives: `refined', which enhances existing drafts, and `synthesized', which summarizes and generates content from provided texts.


% \paragraph{Author Role}
For the author role, we focus on abstract writing, as it is a key element for summarizing the paper and is easily accessible, making it ideal for this study. Specifically, we adopt the `refined' approach, where the original human-written abstract is input into the LLM to generate a refined version.
For the (meta-)reviewer roles, we focus on their comment content. To simulate human-written reviews and meta-reviews, the LLM-generated version primarily adopts the `synthesized' perspective. 
% Specifically, we provide a generation template as a guideline for the LLM.
The review process requires the full text of the paper as input, while the meta-review process includes all associated reviews of the paper.
All prompts used to create LLM-generated content are provided in the Appendix~\ref{app: prompts}.


% For author roles, our primary focus is on the integration of large language models (LLMs) in refining abstract content. The abstract serves as the core element of a paper, offering a concise summary that is both easily accessible and reliably extracted from public sources. Specifically, we employ an LLM as a `polisher', where the original human-written abstract is input into the model, which then generates a refined version. To ensure diversity and robustness in the refinement process, we design a prompt set, consisting of five distinct prompts. One prompt is randomly selected during each refinement cycle. 



% \paragraph{(Meta-)Reviewer Roles}


% A technical reviewer provides detailed feedback on the paper, evaluating aspects such as research methods, writing structure, experimental results, and novelty. In contrast, a meta-reviewer synthesizes the comments from multiple reviewers, offering a comprehensive summary that directly influences the final recommendation regarding the paper's acceptance.
% To simulate the process of human-written reviews and meta-reviews, the LLM-generated version primarily adopts the `synthesized' perspective. Specifically, we provide a generation template format as a guideline for the LLM. The review process requires the full text of the paper as input, while the meta-review process necessitates the inclusion of all the reviews of the paper. Detials of the generation tempalte are also shown in the Appendix~\ref{app: prompts}.


\subsection{Data Collection Setup}




Considering the challenges associated with parsing full-text papers, typically in PDF format, and the high computational cost of generating LLM-based reviews from lengthy input data, the \texttt{ScholarLens} collection integrates pre-existing review data with self-constructed abstracts and meta-reviews.

For the self-constructed data, we first collect raw data from all main conference papers in ICLR up to 2019, totaling 2,831 papers, through the OpenReview website. 
This selection is motivated by two factors: first, ICLR's peer review process provides comprehensive and detailed (meta-)review data; second, by focusing on papers before 2019, we can assume that the source data remains entirely human-written, as it predates the release of ChatGPT. For each paper, we generate two types of LLM-generated content: LLM-refined abstracts and LLM-synthesized meta-reviews. Both types are created using three advanced closed-source LLMs: GPT-4o, Gemini-1.5~\cite{team2024gemini}, and Claude-3 Opus~\cite{claude3modelcard}. This ensures that for every human-written version of the content, there is a corresponding LLM-generated version from each of the three models.
For the pre-existing review data, we directly leverage the review data from ReviewCritique~\cite{du-etal-2024-llms}\footnote{Note that ReviewCritique includes LLM-generated reviews for only 20 papers.}, which incorporates the same three LLMs.
 Details of \texttt{ScholarLens} are in Appendix~\ref{app:dataset}, including statistics and LLM settings.

% For each paper, we create two types of LLM-generated content: LLM-refined abstracts and LLM-synthesized meta-reviews. Both are created using three of the most advanced closed-source LLMs: GPT-4o, Gemini-1.5~\cite{team2024gemini}, and Claude-3 Opus~\cite{claude3modelcard}.
% This ensures that, for every human-written version of the content, there is a corresponding version generated by each of the three LLMs.
% Furthermore, due to the challenges associated with parsing full-text papers, typically in PDF format, and the high computational cost of constructing LLM-generated reviews from long input data, we do not generate reviews directly. Instead, we leverage review data from ReviewCritique~\cite{du-etal-2024-llms}\footnote{ReviewCritique only includes LLM-generated reviews for 20 papers.}, which incorporates the three same LLMs. \lz{The statistics of \texttt{ScholarLens} are shown in Table~\ref{tab: AcademicLens-statistics} in the Appendix~\ref{app:dataset}.}

% This choice is driven by two considerations: firstly, ICLR's peer review process offers the most comprehensive and detailed data available; secondly, selecting papers published before 2019 ensures that the source data remains purely human-written, as it predates the release of ChatGPT.
% For each paper, we generate three types of LLM-generated content: coarse-grained refined abstracts, fine-grained synthesized abstracts, and LLM-synthesized meta-reviews. The fine-grained synthesized abstracts are produced using GPT-4o, while the other two types of content are generated with three of the most advanced closed-source LLMs: GPT-4o, Gemini-1.5, and Claude-3 Opus. Furthermore, due to the challenges of parsing the full-body text of papers, which are typically in PDF format, and the high computational cost associated with constructing LLM-generated reviews from lengthy input data, we do not generate them directly. Instead, we utilize review data from ReviewCritique, which incorporates the same three LLMs.



% This choice is motivated by the comprehensive nature of ICLR's peer review process and the desire to ensure the data is purely human-written, as it predates the release of ChatGPT.


  
% In this section, we describe the three focal aspects of the academia domain. For each aspect, we provide a detailed definition and explain the construction of the corresponding dataset, which serves as the foundation for our experiments.


% \subsubsection{Raw Data}
% The existing peer review platform called OpenReview is dedicated to advancing science through improved peer review, offering a configurable system that supports various levels of openness in the review process, including well-known conferences such as ICLR, ACL, and others.
% In our study,we base on the official OpenReview API,  collecting submission and review data from ICLR, a leading machine learning conference, spanning the years 2018 to 2024. And the collection includes a total of 19,847 main conference submissions, each with its corresponding reviews and meta-reviews.

% With the widespread use and growing popularity of AI models such as ChatGPT, and as former research points out, there is growing suspicion that academic reviewers may have begun utilizing AI tools in their review processes in recent years. Therefore, we set the year 2020, the release year of GPT-3, as a cutoff point, presupposing that the data before 2020 had negligible or no AI involvement, and that the data from 2020 to 2024 is likely to have AI involvement, separating data into two sub-datasets, named non-AI data and prob-AI data, where the non-AI data includes 2,844 out of 19,847 abstracts, corresponding to 2,844 meta-reviews and 8,592 reviews, while the prob-AI data includes (XXXX). \textbf{Our framework aims to generate an AI context based on the non-AI data and evaluate the effectiveness of the metrics we established below. So that we could assess AI penetration in the prob-AI data, thereby revealing the extent of AI involvement in academic peer review in recent years.}

% \subsubsection{Meta-review}

% \input{figure/data_pair_extraction_and_split}

% % \begin{table}[ht]
% \centering
% \footnotesize
% \begin{tabular}{@{}p{0.8\textwidth}@{}}
% \toprule
% \textbf{Prompts} \\ \midrule
% Can you help me revise the abstract? Please response directly with the revised abstract: {abstract} \\
% please revise the abstract, and response directly with the revised abstract: {abstract} \\
% Can you check if the flow of the abstract makes sense? Please response directly with the revised abstract: {abstract} \\
% Please revise the abstract to make it more logical, response it directly with the revised abstract: {abstract} \\
% Please revise the abstract to make it more formal and academic, response it directly with the revised abstract: {abstract}\\
% \bottomrule
% \end{tabular}
% \caption{Polishing Prompts}
% \label{tab:polishing_prompts}
% \end{table}



% We collected the submission and review data from ICLR, a major conference in machine learning. The data was gathered through the official OpenReview API, consisting of submissions and reviews from conferences held between 2018 and 2024. The collection includes a total of 19,847 main conference submissions, each with its corresponding reviews.

% Next, we selected data from 2017 to 2019 for cleaning, extracting 2,844 abstracts, 8,592 official reviews, and 2,844 meta-reviews.

% To generate AI meta-reviews, we designed a simplified prompt:\\

% \textit{the prompt we use to generate meta-review}\\

% The prompt adapts based on the review content and abstract, guiding ChatGPT-4o model to generate a comprehensive meta-review.

% During the analysis, we observed that some meta-reviews followed a specific structure, such as listing strengths (Pros) followed by weaknesses (Cons). To accurately simulate this structure, we measured the occurrence frequency of this specific format in the collection and appropriately introduced a template into the generation prompts based on these proportions.

% We used the official review and abstract for each submission as inputs. Using our prompts, the ChatGPT-4o model generated AI meta-reviews that matched the original word count.

% To simulate the process of a meta-reviewer’s feedback, we combine the review context with the paper abstract to guide LLMs in producing the meta-review where we use the gpt-4o-2024-08-06(hereinafter referred to as GPT-4o) model here.Additionally, since the review criteria for ICLR change annually, the review template is typically determined by these evolving criteria, which vary across years. To capture this diversity in reviews, we analyze the frequency of review templates used from 2018 to 2024. This allows us to refactor the reviews according to the frequency distribution, ensuring that the generated meta-reviews reflect the shifting review criteria over time. Based on this analysis, we design the following prompt to guide the meta-review generation:\\
% \textit{Please write a meta review of the given reviewers' response around \{word\_count\} words. Do not include any section titles or headings. Do not reference individual reviewers by name or number. Instead, focus on synthesizing collective feedback and overall opinion.}

% \textit{Please include the given format in your meta review: \{selected\_format\}}

% \textit{\#\#\# }

% \textit{Abstract: \{abstract\}}

% \textit{\#\#\#}

% \textit{Reviewers' feedback:}

% \textit{\{review\_text\}}




% In total, we generated 2,844 AI-generated meta-reviews. We then labeled the AI-generated meta-reviews as "1" and the original meta-reviews as "0," forming a dataset of 5,686 entries.

% In this study, we use the Longformer model and tokenizer, optimized for a binary classification task. In our cleaned dataset, each entry includes human-written meta-reviews from ICLR and AI-generated meta-review by ChatGPT-4o.

% We first extract these pairs, then shuffle them using a fixed random seed to ensure consistency across runs.
% Then we divide the data pairs into three sets—training, validation, and test—in a 7:2:1 ratio. Finally, we format the data for model input. The “result” texts are labeled as 1 and “meta-review” texts as 0. Figure~\ref{fig:data_pair_extraction_and_split} presents a specific example.

% Key training settings, like learning rate, batch size, and training epochs, are shown in the table.~\ref{tab:training_parameters}





% \textbf{Coarse-grained}: We consider coarese-grained aspect and use the abstract data above, 2,844 abstracts from ICLR papers. The method we obtain the AI-generated abstracts is also created by GPT-4o, and we randomly select one of the prompts for each generation, the prompts could be seen in % \ref{tab:polishing_prompts}
% , At the end, we get 2,844 abstracts polished by GPT-4o, intent to finding the discrepancy of the original abstracts and the AI-generated revisions.
% \textbf{Fine-grained}, In more detail, we consider the complexity of AI involvement in abstract writing: polishing, generation, and even hybrid approaches (polishing + generation). To explore this, we use the GPT-4 model for a text completion task. Usually, a paper abstract consists of several parts: Background, Motivation, Problem Statement, Approach, Results/Key Findings, Conclusion, and Implications. 

% First, we use GPT-4 to pre-label the parts of the abstract and manually annotate them for proofreading. Then, we apply the following operations to each abstract with probabilities of 0.4, 0.4, and 0.2, respectively: delete Background, delete Implications, and delete both. Based on these results, we provide three different prompts to instruct the LLM to complete the context, yielding 2,844 results labeled as 'human + generate.' At the same time, we perform the same operations on polished abstracts, which are labeled 5as 'human + polishing + generate.'

% Above all, in the fine-grained dataset, each entry consists of four different types of abstracts: 'human', 'polishing', 'human + generate', and 'human + polishing + generate'.

% \subsubsection{Review}
% As the amount of reviews reach over 8000 pairs, 
% We will also focus on AI involvement in peer review. In the ICLR dataset, each paper is associated with 3 to 5 peer reviews. For simplicity, we select 100 papers mentioned in {}. Each of these papers carries 3 to 5 peer reviews, and 20 of them additionally have 3 peer reviews generated by GPT-4o, Claude-3 (Opus, 2024-02-29), and Gemini-1.5-Pro (version 002), respectively. This dataset not only allows us to contrast the differences between human and AI writing, but also helps indirectly assess the robustness of the ICLR dataset we are using.
% 为什么选取其他数据集,选取了怎么构造,有什么意义--侧面印证我们的metric
% As the number of reviews exceeds 8,000 pairs, generating AI reviews for such a large volume becomes increasingly cumbersome. For simplicity, we select the 100 high-quality papers mentioned in {}. Each of these papers has 3 to 5 peer reviews, and 20 of them are further supplemented with 3 additional peer reviews generated by GPT-4o, Claude-3 (Opus, 2024-02-29), and Gemini-1.5-Pro (version 002), respectively. This curated dataset not only allows us to directly compare the differences between human and AI-generated writing, but also provides an indirect means of evaluating the robustness and reliability of the metrics on ICLR dataset in this work.



% \subsubsection{Title}




% \subsection{Linguistic Feature Comparison} 
% \begin{table}[ht]

% \setlength{\tabcolsep}{2.5pt} % Adjust column padding to fit content better
\centering
\footnotesize 
\begin{tabular}{@{}l|r@{\hskip 2mm}r@{\hskip 2mm}r@{\hskip 2mm}r@{}}
\toprule
\textbf{Feature}         & \textbf{Human}         & \textbf{AI}           \\ \midrule
\textbf{Avg. Word Count}          & 106.96±84.94 & 114.03±69.60 \\
\textbf{Avg. Sent. Count}       & 5.33±4.19   & 6.78±3.83  \\
\textbf{Senti. Polar. Score} & 0.13±0.13    & 0.12±0.09    \\
\textbf{Gram. Errors} & 18.22±20.60   & 8.28±11.42    \\
\textbf{Synt. Diversity} & 1.12±0.77   & 0.65±0.37    \\
\textbf{Voca. Richness}   & 0.77±0.11     & 0.81±0.09    \\
\textbf{Read. Score}     & 18.49±4.05    & 22.33±2.20    \\
\bottomrule
\end{tabular}
\caption{Feature differences between meta-review. The value in the corresponding cell indicates the mean ± standard deviation.}
\label{tab:feature_comparison}
\end{table}
% To systematically compare the linguistic feature differences between human-written meta-review and AI-generated ones, we introduce seven linguistic feature metrics. Table~\ref{tab:feature_comparison} presents a comparison of these linguistic features. Average word and sentence count measure the number of words and sentences in a news article. Sentiment polarity score represents the emotional tone of a text, ranging from -1 to 1, with higher values indicating more positive sentiment, and lower values reflecting more negative sentiment. Grammatical errors measure the number of grammatical mistakes that occur per 1,000 words. Syntactic diversity measure the structural complexity by analyzing clause patterns. Vocabulary richness measures lexical diversity, ranging from 0 to 1, with higher values indicating greater lexical variation. Readability score measures the complexity of a text, with higher values indicating greater reading difficulty. 

% As shown in Table~\ref{tab:feature_comparison}, we observe that the lengths and number of sentences between human-written and AI-generated meta-reviews are roughly similar due to the prompt settings during generation. Human-written meta-reviews exhibit a higher frequency of grammatical errors, employ a broader range of sentence structures, and demonstrate lower readability.

% In contrast, LLM-polished and LLM-extended news, incorporating more human inputs, are significantly richer and more comprehensive. The various types of news exhibit trivial differences in their sentiment polarity scores. From other linguistic features, human writing shows greater lexical and syntactic variation with lower reading difficulty, whereas LLM writing is more standardized, featuring fewer grammatical errors and minimal use of informal writing styles. 

\section{\texttt{LLMetrica} Framework}
\label{sec: LLMetria}
In this section, we introduce the \texttt{LLMetrica} framework, designed to evaluate the penetration rate of LLM-generated content in scholarly writing and peer review. 
The framework includes rule-based metrics for assessing linguistic features and semantic similarity, as well as model-based detectors fine-tuned specifically to identify LLM-generated content within the scholarly domain.

% which includes rule-based metrics for evaluating linguistic features and semantic similarity, along with model-based detectors fine-tuned specifically to identify LLM-generated content within the scholarly domain.


% \subsection{Linguistic and Semantic Metrics}  
\subsection{Rule-Based Metrics: Preference}  
\label{sec:metric}
Rule-Based Metrics define a metric function $m$ to measure the feature value $v$ of an input text $x$, i.e., $v=m(x)$, enabling the comparison and evaluation of feature preferences in LLM-generated text. 
Specifically, we use 10 general linguistic feature metrics and design 4 specialized semantic feature metrics to capture both linguistic and semantic characteristics.


% Linguistic and semantic metrics are proposed to analyze and compare the structural, syntactic, and meaning-based features of texts, helping to identify LLM preferences and infer trends in its generated text.

\subsubsection{General Linguistic Features}


General linguistic features are applicable to all types of text and can be categorized into word-level, sentence-level, and other related metrics.
Specifically, word-level metrics include Average Word Length (AWL), Long Word Ratio (LWR), Stopword Ratio (SWR), and Type Token Ratio (TTR). Given the nature of scholarly writing, the threshold for `long word' is set at 10. 
For sentence-level metrics, we include Average Sentence Length (ASL), Dependency Relation Variety (DRV), and Subordinate Clause Density (SCD).
DRV quantifies the diversity of dependency relations within the text using Shannon entropy~\cite{lin1991divergence}, while SCD focuses on dependency relations such as `advcl', `ccomp', `xcomp', `relcl', and `acl'~\cite{nivre-etal-2017-universal}.
In addition, we incorporate Flesch Reading Ease (FRE)~\cite{farr1951simplification} to evaluate the overall readability of the text, Sentiment Polarity Score (PS, range: [-1, 1], negative→positive) to assess the sentiment, and Sentiment Subjectivity Score (SS, range: [0, 1], objective → subjective) to measure the degree of subjectivity or objectivity in the text. The implementation details of these metrics are provided in the Appendix~\ref{app:general}.



\subsubsection{Specific Semantic Features}
% applicable to all types of text and can be categorized into word-level, sentence-level, and other related metrics.
Inspired by \citet{du-etal-2024-llms}, which shows that human-written reviews have greater diversity and segment-level specificity than LLM-generated ones, we design four semantic metrics to analyze meta-reviews and reviews, focusing on overall semantic similarity and sentence-level specificity.



% \citet{du-etal-2024-llms} demonstrate that human-written reviews typically exhibit greater diversity and higher segment-level specificity compared to those generated by LLMs. Building on this insight, we conduct a quantitative analysis of MR and R, focusing specifically on two aspects: (i) the overall semantic similarity to the reference text, and (ii) sentence-level specificity at a finer granularity.



\paragraph{Overall Semantic Similarity}

We propose two semantic similarity metrics:
(i) \textbf{MRSim}: measures the similarity between the MR and its reference set R, defined as the average semantic similarity between MR and each review $r_i \in R$.
(ii) \textbf{RSim}: measures the similarity among reviews within R, defined as the maximum similarity among all pairs of reviews in R. The formulas are:
% The mathematical expressions are as follows:
\begin{equation}
\small{\mathrm{MRSim}=\frac{1}{\left| \mathrm{R} \right|}\sum_{r_i\in \mathrm{R}}{\mathrm{sim} \left( \mathrm{MR}, r_i \right)} }
\end{equation}
\begin{equation}
\small{\mathrm{RSim}=\underset{r_i,r_j\in \mathrm{R},r_i\ne r_j}{\max}\mathrm{sim} \left( r_i, r_j \right) }
\end{equation}
Using maximum similarity for RSim accounts for the fact that not all reviews in R are LLM-generated, as averaging could obscure key differences. Focusing on the maximum similarity highlights the strongest alignment, offering a more accurate measure of overall similarity.

% The use of maximum similarity for RSim considers that not all reviews in R are LLM-generated, so averaging could obscure key differences, while focusing on the maximum similarity captures the strongest alignment, providing a more accurate measure of overall similarity.



\begin{figure*}[ht]
    \centering
    % \vspace{1cm}
    \begin{minipage}{1\linewidth}
        \centering
        \includegraphics[width=1.0\linewidth]{fig/feature.pdf}

        \subcaption{Comparison of Features for ALL Data Types}
    \end{minipage}

    \begin{minipage}{1\linewidth}
        \centering
        \scalebox{0.9}{
        \begin{tabular}{@{}l|lll@{}}
        \toprule
          Data Type  & ↑                        & ↓                          & →                 \\ \midrule
        \textbf{Abstract}    & 3 (\textbf{AWL}, \textbf{LWR}. TTR)         & 5 (\textbf{SWR}, ASL, DRV, SCD, \textbf{FRE}) & 2 (PS,SS)           \\
        \textbf{Meta-Review} & 3 (\textbf{AWL}, \textbf{LWR}. TTR)         & 3 (\textbf{SWR}, DRV, \textbf{FRE})           & 4 (ASL, SCD, PS, SS) \\
        \textbf{Review}      & 5 (\textbf{AWL}, \textbf{LWR}, ASL, PS, SS) & 3 (\textbf{SWR}, TTR, \textbf{FRE})           & 2 (DRV, SCD)         \\ \bottomrule
        \end{tabular}}

        \subcaption{Feature preference of LLM-generated text: ↑ indicates an increase across all LLMs, ↓ indicates a decrease, → indicates inconsistency. \textbf{Bold} denotes consistent trends across all data types.}
    \end{minipage}


    \caption{Comparison of Human-Written and LLM-Generated Text Based on \textbf{General Features} in \texttt{ScholarLens}}
    \label{fig:compare_general}
\end{figure*}


% \begin{figure*}[ht]
%     \centering
%     \includegraphics[width=1.0\linewidth]{fig/feature.pdf}
%     \caption{Caption}
%     \label{fig:enter-label}
% \end{figure*}






\paragraph{Sentence-Level Specificity}
Building on ITF-IDF~\cite{du-etal-2024-llms}\footnote{Unlike ITF-IDF~\cite{du-etal-2024-llms}, we only measure the SF-IRF within a single paper, considering the meta-review and review levels.} and the classic TF-IDF framework, we introduce the \textbf{SF-IRF} (\textbf{S}entence \textbf{F}requency-\textbf{I}nverse \textbf{R}everence \textbf{F}requency) metric to quantify the significance of sentences within a (meta-)review. 
Specifically, for a given target (meta)-review $r$ consisting of $n$ sentences, $\mathrm{SF}\text{-}\mathrm{IRF}\left( s,r,\mathrm{R}_{\mathrm{ref}} \right)$ captures the importance of a sentence $s$ in $r$ by considering: (i) its frequency of occurrence within $r$ (SF), and (ii) its rarity across the reference reviews $\mathrm{R}_\mathrm{ref}$ (IRF). The metric is formally defined as:
\begin{equation}
\footnotesize{
    \begin{aligned}
        \mathrm{SF}\text{-}\mathrm{IRF}\left( s,r,\mathrm{R}_{\mathrm{ref}} \right) &= \mathrm{SF}\left( s,r \right) \cdot \mathrm{IRF}\left( s, \mathrm{R}_{\mathrm{ref}} \right) \\
        &= \frac{O_{s}^{r}}{n} \cdot \log \left( \frac{m}{Q_{s}^{\mathrm{R}_{\mathrm{ref}}}} \right)
    \end{aligned}}
\end{equation}
Here, if $r$ represents a review, then $\mathrm{R}_{\mathrm{ref}}=\mathrm{R}-r$; if $r$ is a meta-review, then $\mathrm{R}_{\mathrm{ref}}=\mathrm{R}$. $O_{s}^{r}$ quantifies the ``soft'' occurrence of sentence $s$ within the target review $r$, while $Q_{s}^{\mathrm{R}_{\mathrm{ref}}}$ represents the ``soft'' count of reviews in $\mathrm{R}_{\mathrm{ref}}$ that contain the sentence $s$. Additionally, $m$ denotes the total number of reviews in $\mathrm{R}_{\mathrm{ref}}$. $O_{s}^{r}$ and $Q_{s}^{\mathrm{R}_{\mathrm{ref}}}$ are computed as follows:
\begin{equation}
\footnotesize{
    O_{s}^{r} = \sum\limits_{\tilde{s} \in r} \mathbb{I} \left( \mathop {\mathrm{sim}} \left( s, \tilde{s} \right) \geqslant t \right) \cdot \mathop {\mathrm{sim}}  \left( s, \tilde{s} \right)}
\end{equation}
\begin{equation}
\footnotesize{
    Q_{s}^{R_{\mathrm{ref}}} = \sum\limits_{\tilde{r} \in  R_{\mathrm{ref}}} \mathbb{I} \left( \mathop {\max \mathrm{sim}} \limits_{\tilde{s} \in \tilde{r}} \left( s, \tilde{s} \right) \geqslant t \right) \cdot \mathop {\max \mathrm{sim}} \limits_{\tilde{s} \in \tilde{r}} \left( s, \tilde{s} \right)}
\end{equation}
A segment $s$ is counted when its similarity exceeds the threshold $t$, with the corresponding similarity score.
% Following \citet{du-etal-2024-llms}, 
We use SentenceBERT~\cite{reimers-gurevych-2019-sentence} to calculate the all similarities, with $t$ set to 0.5.


% \subsection{Detection Models}
\subsection{Model-Based Detectors: Distinction}
\label{sec:detectors}
Model-based detectors are designed to train scholar-specific detection models 
$f$, capable of accurately identifying whether a scholarly input text $x$ is human-written or LLM-generated. We use our curated \texttt{ScholarLens} dataset to train these models, collectively referred to as ScholarDetect.




Specifically, we split abstracts and meta-reviews data within \texttt{ScholarLens} into training and test sets in a 7:3 ratio, based on the human-written versions. 
The corresponding LLM-generated content is partitioned accordingly, ensuring that each piece of LLM-generated text is paired with its human-written counterpart. 
All reviews data in \texttt{ScholarLens} are incorporated into the test set, ensuring a comprehensive evaluation.\footnote{Data statistics after splitting for model training and evaluation are in Table~\ref{tab:data_split}.}
All test sets
% , which include LLM-generated text from multiple LLM resources, 
serve as benchmarks to assess the performance of both baseline models and the trained ScholarDetect models.
We create three types of detection models based on the training data: one using only abstracts, one using only meta-reviews, and one using a hybrid of both.
To maintain class balance in the training data, we ensure a 1:1 ratio between human-written and LLM-generated version. We employ two strategies: one using a single LLM (GPT-4o, Gemini, or Claude), and another using a mixed-LLM approach, where each human-written piece is paired with LLM-generated content from a randomly selected model.
As a result, the ScholarDetect framework involves a total of 12 distinct detection models.




\section{Experiments}
\label{sec:Experiments}
We first demonstrate the effectiveness of rule-based metrics (\S\ref{sec: Features Comparison}) and ScholarDetect models (\S\ref{sec:ScholarDetect Evaluation}) in \texttt{LLMetrica}, then apply these methods to real-world conference data to assess and predict LLM penetration trends (\S\ref{sec:Temporal Analysis}). Finally, case studies are used to explore the specific differences between human-written and LLM-generated content (\S\ref{sec: Case Study}).

% case studies provide deeper insights into the distinguishing features of human-written vs. LLM-generated content (\S\ref{sec: Case Study}).

% we conduct case studies to gain deeper insights into the distinctive characteristics of human-written and LLM-generated content (\S\ref{sec: Case Study}).





\subsection{Features Comparison: Human vs LLM}
\label{sec: Features Comparison}
We apply the proposed rule-based metrics (\S\ref{sec:metric}) to  \texttt{ScholarLens} to compare the features of human-written and LLM-generated texts, and find that the feature preferences of LLM-generated texts can be effectively compared and evaluated.

% Results are shown in Figure~\ref{fig:compare_general} (general features) and Figure~\ref{fig:compare_specific} (specific features).


% Please add the following required packages to your document preamble:
% \usepackage{multirow}
% \begin{table*}[]
% \begin{tabular}{c|c|ccc|ccc|ccc}
% \hline
% \multirow{2}{*}{}         & \multirow{2}{*}{Model} & \multicolumn{3}{c|}{Abstract} & \multicolumn{3}{c|}{Meta-Reivew} & \multicolumn{3}{c}{Reiview}                  \\ \cline{3-11} 
%                           &                        & Human   & LLM     & Overall   & Human    & LLM      & Overall    & Human & LLM   & \multicolumn{1}{c|}{Overall} \\ \cline{2-11} 
% \multirow{3}{*}{Baseline} & MAGE                   & 64.46   & 37.07   & 54.58     & 65.80    & 28.34    & 53.70      & 92.98 & 57.60 & \multicolumn{1}{c|}{87.95}   \\
%                           & RaiDetector            & 62.04   & 71.21   & 67.25     & 57.39    & 70.25    & 64.96      & 24.48 & 26.84 & \multicolumn{1}{c|}{25.68}   \\
%                           & HNDCDetector           & 54.59   & 85.85   & 78.43     & 62.63    & 76.74    & 71.33      & 86.39 & 54.90 & \multicolumn{1}{c|}{79.09}   \\ \hline
% \end{tabular}
% \caption{the F1 score performance of our fine-tuning model compare with two of current detect model}
% \label{tab:fine-detection}
% \end{table*}


% Please add the following required packages to your document preamble:
% \usepackage{booktabs}
% \usepackage{multirow}
\begin{table*}[t]
\centering
\scalebox{0.64}{
\begin{tabular}{@{}l|p{1.3cm}|rrr|rrr|rrr|r@{}}
\toprule
\multirow{2}{*}{\textbf{Model}}                                  & \multirow{2}{*}{\parbox{1.3cm}{\textbf{LLM Source}}} & \multicolumn{3}{c|}{\textbf{Abstract}}                          & \multicolumn{3}{c|}{\textbf{Meta-Review}}                       & \multicolumn{3}{c|}{\textbf{Review}}                            & \multicolumn{1}{l}{\multirow{2}{*}{\textbf{Avg.}}} \\ \cmidrule(lr){3-11}
                                                                 &   \multirow{1}{*}                     & Human               & LLM                 & Overall             & Human               & LLM                 & Overall             & Human               & LLM                 & Overall             & \multicolumn{1}{l}{}                                  \\ \midrule
\textbf{MAGE}                                                    & \multirow{3}{*}{-}                 & 40.62               & 35.14               & 38.00               & 40.58               & 33.43               & 37.21               & 92.98               & 57.60                & 87.95               & 54.39                                                 \\
\textbf{RAIDetect}                                               &                                    & 49.51               & 78.36               & 69.71               & 38.42               & 73.50                & 62.94               & 24.48               & 26.85               & 25.68               & 52.78                                                 \\
\textbf{HNDCDetect}                                              &                                    & 54.59               & 85.85               & 78.43               & 57.93               & 83.88               & 76.69               & 86.39               & 54.90                & 79.09               & 78.07                                                 \\ \midrule
\multirow{4}{*}{\textbf{$\text{ScholarDetect}_{\text{Abs}}$}}    & GPT-4o                             & 97.02\small ±0.50          & 99.02\small ±0.15          & 98.53\small ±0.23          & 87.65\small ±3.01          & 95.06\small ±1.48          & 92.95\small ±2.00          & 97.44\small ±0.20          & 79.97\small ±1.96          & 95.45\small ±0.37          & 95.64                                                 \\
                                                                 & Gemini                             & 84.40\small ±3.11          & 93.48\small ±1.59          & 90.80\small ±2.13          & 90.62\small ±1.98          & 95.06\small ±1.21          & 92.93\small ±1.63          & 96.41\small ±0.64          & 68.68\small ±7.26          & 93.56\small ±1.19          & 92.43                                                 \\
                                                                 & Claude                             & 91.61\small ±1.09          & 96.90\small ±0.46          & 95.47\small ±0.65          & 83.79\small ±4.74          & 93.02\small ±2.60          & 90.25\small ±3.40          & 95.69\small ±0.89          & 58.84\small ±12.55         & 92.20\small ±1.68          & 91.97                                                 \\
                                                                 & Mix                                & \underline{97.94\small ±0.16}          & \underline{99.32\small ±0.05}          & \underline{98.98\small ±0.07}          & 93.84\small ±0.23          & 97.81\small ±0.09          & 96.76\small ±0.13          & 97.56\small ±0.20          & 81.16\small ±1.92          & 95.68\small ±0.37          & 97.14                                                 \\ \midrule
\multirow{4}{*}{\textbf{$\text{ScholarDetect}_{\text{Meta}}$}}   & GPT-4o                             & 78.59\small ±1.64          & 90.47\small ±1.19          & 86.81\small ±1.45          & 99.53\small ±0.13          & 99.84\small ±0.04          & 99.76\small ±0.06          & 97.98\small ±0.21          & 84.95\small ±1.89          & 96.44\small ±0.39          & 94.34                                                 \\
                                                                 & Gemini                             & 80.13\small ±2.71          & 91.45\small ±1.56          & 88.05\small ±2.02          & 97.70\small ±1.86          & 99.19\small ±0.67          & 98.80\small ±0.98          & 97.65\small ±0.30          & 81.90\small ±2.69          & 95.83\small ±0.54          & 94.23                                                 \\
                                                                 & Claude                             & 62.42\small ±13.17         & 69.60\small ±16.38         & 66.93\small ±15.79         & 86.72\small ±7.94          & 94.15\small ±3.67          & 91.88\small ±5.02          & 97.82\small ±0.62          & 83.23\small ±5.56          & 96.14\small ±1.11          & 84.98                                                 \\
                                                                 & Mix                                & 84.20\small ±3.59          & 94.12\small ±2.24          & 91.47\small ±2.89          & \underline{99.84\small ±0.10}          & \underline{99.95\small ±0.03}          & \underline{99.92\small ±0.05}          & \textbf{99.52\small ±0.16} & \textbf{96.86\small ±1.08} & \textbf{99.17\small ±0.28} & 96.85                                                 \\ \midrule
\multirow{4}{*}{\textbf{$\text{ScholarDetect}_{\text{Hybrid}}$}} & GPT-4o                             & 97.69\small ±0.35          & 99.23\small ±0.13          & 98.84\small ±0.19          & 99.33\small ±0.17          & 99.78\small ±0.06          & 99.67\small ±0.08          & 97.73\small ±0.16          & 82.72\small ±1.42          & 95.98\small ±0.28          & \underline{98.16}                                                 \\
                                                                 & Gemini                             & 85.88\small ±0.19          & 94.32\small ±0.10          & 91.90\small ±0.13          & 97.84\small ±1.19          & 99.25\small ±0.43          & 98.89\small ±0.63          & 97.69\small ±0.31          & 82.29\small ±2.83          & 95.91\small ±0.56          & 95.56                                                 \\
                                                                 & Claude                             & 85.61\small ±1.67          & 94.11\small ±0.84          & 91.64\small ±1.13          & 96.49\small ±1.22          & 98.77\small ±0.44          & 98.18\small ±0.65          & 97.48\small ±0.26          & 80.35\small ±2.44          & 95.53\small ±0.47          & 95.13                                                 \\
                                                                 & Mix                                & \textbf{98.11\small ±0.35} & \textbf{99.37\small ±0.11} & \textbf{99.06\small ±0.17} & \textbf{99.88\small ±0.00} & \textbf{99.96\small ±0.00} & \textbf{99.94\small ±0.00} & \underline{98.06\small ±0.18}          & \underline{85.69\small ±1.53}          & \underline{96.59\small ±0.32}          & \textbf{98.53}                                        \\ \bottomrule
\end{tabular}}
\caption{Detection performance comparison of baseline models and ScholarDetect. \textbf{Bold} denotes the best performance, and \underline{underlined} denotes the second-best. ``Avg.'' shows the average overall score across the three test data types.}
\label{tab:fine-detection}
\end{table*}




\paragraph{General Linguistic Features}

Figure~\ref{fig:compare_general} shows trends in the characteristics of LLM-generated texts, with slight variations across different data types. Each metric reflects the consistency of features across texts generated by the three LLMs in at least one data type, demonstrating the `comparability' effectiveness of the chosen metrics.
Moreover, regardless of the data type or LLM used, LLM-generated texts consistently show higher values for Average Word Length (AWL) and Long Word Ratio (LWR), and lower values for Stopword Ratio (SWR) and Readability (FRE). This suggests that LLM-generated texts tend to use longer words, avoid excessive stopwords, and have lower readability.
For shorter text types, such as abstracts and meta-reviews, the observed increase in Type Token Ratio (TTR) reflects greater lexical diversity in LLM-generated texts.
This may be due to the conciseness inherent in short-form LLM-generated content. In contrast, for longer reviews, TTR decreases, potentially highlighting the limitations of LLMs in producing long-form content~\cite{wang-etal-2024-m4, wu2025survey}. Longer reviews may lack specificity~\cite{du-etal-2024-llms}, resulting in redundancy and repetitive segments.
Additionally, LLM-generated reviews tend to be more positive and subjective, suggesting a more favorable tone and less neutral objectivity. This aligns with \citet{jin-etal-2024-agentreview}, who found that LLM-generated reviews generally assign higher scores and show a higher acceptance rate.

\paragraph{Specific Semantic Features}
Figure~\ref{fig:compare_specific} shows the preferences of human-written and LLM-generated texts in both meta-reviews and reviews, based on four specific semantic features.
Notably, Since each LLM generates only one review per paper in \texttt{ScholarLens}, while each paper usually has multiple reviews, we combine reviews from all three LLMs into a unified set, so comparisons do not distinguish between them.
Comparative results show that LLM-generated meta-reviews exhibit higher semantic similarity to the referenced reviews, with lower sentence specificity.
This suggests that sentences within LLM-generated meta-reviews are more semantically similar to each other (prone to redundancy) and tend to mirror the content of the referenced reviews. 
A similar trend is observed for reviews, where the two specific features also show consistent patterns.
It is important to note that for the reviews, we assume all are LLM-generated in this experiment, which may amplify the differences in these semantic features. In reality, having more than two LLM-generated reviews per paper may be uncommon, which would likely reduce the observed disparity.

% Figure~\ref{fig:compare_specific} shows that LLM-generated meta-reviews exhibit higher semantic similarity to the referenced reviews. Additionally, the specificity of segments is lower, indicating that segments within LLM-generated meta-reviews are more semantically similar to each other (prone to redundancy) and tend to mirror the content of the referenced reviews. Similarly, for reviews, the trends observed in the two specific features are also consistent.
% It is important to note that for the reviews, we assume all are LLM-generated in this experiment, which may amplify the differences in the two semantic features. In reality, having more than two LLM-generated reviews per paper is rare, which would likely reduce the observed disparity.


\begin{figure}[t]
    \centering
    \begin{minipage}{0.55\linewidth}
        \centering
        \includegraphics[width=\linewidth]{fig/meta_specific_compare.pdf}
        \captionsetup{font=footnotesize}
        \subcaption{Meta-Review}
    \end{minipage}
    \hfill
    \begin{minipage}{0.4\linewidth}
        \centering
        \includegraphics[width=\linewidth]{fig/review_specific_compare.pdf}
        \captionsetup{font=footnotesize}
        \subcaption{Review}
    \end{minipage}
    \hfill
    \caption{Comparison of Human-Written and LLM-Generated Text Based on \textbf{Specific features} for Review and Meta-Review.}
    \label{fig:compare_specific}
\end{figure}
% AgentReview~\cite{jin-etal-2024-agentreview}.

\subsection{ScholarDetect Evaluation: Detectability}
\label{sec:ScholarDetect Evaluation}
We evaluate the trained model-based detectors, ScholarDetect (\S\ref{sec:detectors}), on the \texttt{ScholarLens} test sets and find that scholarly LLM-generated texts can be effectively identified.




\paragraph{Experimental Setup}
\textbf{(i)} \textbf{Training Setup}: 
We adopt Longformer~\cite{Beltagy2020Longformer} as the base model for training our ScholarDetect detection models, as it has shown competitive performance among pretrained language models~\cite{li-etal-2024-mage, cheng2024beyond}. Specifically, we train for five epochs in each configuration of the training set, using a learning rate of 2-e5.
\textbf{(ii)} \textbf{Metric}: 
For evaluation metrics, we report the F1 score for each class (human-written and LLM-generated), as well as the overall weighted F1 score to account for class imbalance. 
Each experimental setup (training data type and LLM-generated text source) is evaluated through three random trials, and we report the average performance along with the standard deviation. 
\textbf{(iii)} \textbf{Baselines}:
We compare the performance of three advanced detection model baselines: MAGE~\cite{li-etal-2024-mage}, RAIDetect~\cite{dugan-etal-2024-raid}, and HNDCDetect~\cite{cheng2024beyond}.
\textbf{(iv)} \textbf{Test Sets}:
All models are evaluated on test sets from three data types: abstract, meta-review, and review, with the first two being shorter texts and reviews being long-form. 
The LLM-generated data includes tasks such as refinement (for abstracts) and summarization (for meta-reviews and reviews).

\paragraph{Experimental Results}
The detection performance comparison results are presented in Table~\ref{tab:fine-detection}. Our trained ScholarDetect models consistently outperform the existing advanced baseline models, underscoring the importance of developing detection systems specifically tailored for the scholarly domain.
The training approach that combines mixed LLM sources and hybrid data types yields the best overall performance, demonstrating robustness across various LLM sources and data types.
Interestingly, the model trained on meta-reviews performs best when tests on reviews, likely because both data types share a similar comment-based focus and offer a ``synthesized'' perspective in LLM-generated text. 
This is further supported by ScholarDetect$_\text{Abs}$, which struggles to identify LLM-generated reviews when trained only on abstracts (F1: LLM < Human).
Additionally, when trained on a single LLM source, the GPT-4o-based detectors show the strongest generalization, especially on the abstract test set. 
Most ScholarDetect models outperform human-written text in detecting LLM-generated content on the meta-review and review test sets, but the reverse is true for the review test set.


% Here we present the results of training longformer. The loss rate of the trained model is shown in the table~\ref{tab:longformer_training_loss}, and the model is named according to the timestamp at which it was trained.

% We used our trained models to detect the metareview data from ICLR 2020-2024. We found that the AI generation rate was low from 2020 to 2023, but there was a sharp increase in 2024. Although the training was based on metareview data, we also applied our models to analyze abstracts, where a similar sharp rise was observed in 2024. Please see figure~\ref{fig:AI-generated metareview percentage} and figure~\ref{fig:AI-processed abstract percentage} for details.



\begin{figure}[t]
    \centering
    \begin{minipage}{0.98\linewidth}
        \centering
        \includegraphics[width=\linewidth]{fig/feature_trend_Abstract.png}
        \captionsetup{font=footnotesize}
        \subcaption{Abstract}
    \end{minipage}
    \hfill
    \begin{minipage}{0.98\linewidth}
        \centering
        \includegraphics[width=\linewidth]{fig/feature_trend_Meta-Review.png}
        \captionsetup{font=footnotesize}
        \subcaption{Meta-Review}
    \end{minipage}
    \hfill
    \begin{minipage}{0.98\linewidth}
        \centering
        \includegraphics[width=\linewidth]{fig/feature_trend_Review.png}
        \captionsetup{font=footnotesize}
        \subcaption{Review}
    \end{minipage}
    \hfill

    \caption{Temporal trends based on \textbf{four robust general linguistic metrics}.
    % \red{$\boldsymbol{\uparrow \downarrow \rightarrow}$} represents the feature preference of LLM-generated text, as identified in the comparisons in Figures~\ref{fig:compare_general} and~\ref{fig:compare_specific}. \green{$\boldsymbol{\checkmark}$} indicates alignment between the feature trend and preference, signifying increased LLM penetration.
    }
    \label{fig:trend_general}
\end{figure}

% \begin{figure}[t]
%     \centering
%     \begin{minipage}{1\linewidth}
%         \centering
%         \includegraphics[width=\linewidth]{fig/specific_trend_meta.PNG}
%         \captionsetup{font=footnotesize}
%         \subcaption{Meta-Review}
%     \end{minipage}
%     \hfill
%     \begin{minipage}{1\linewidth}
%         \centering
%         \includegraphics[width=\linewidth]{fig/specific_trend_review.PNG}
%         \captionsetup{font=footnotesize}
%         \subcaption{Review}
%     \end{minipage}
%     \hfill
%     \caption{Temporal trends based on \textbf{specific semantic metrics}.}
%     \label{fig:trend_specific}
% \end{figure}

\begin{figure}[t]
    \centering
    \begin{minipage}{0.49\linewidth}
        \centering
        \includegraphics[width=\linewidth]{fig/specific_trend_meta.png}
        \captionsetup{font=footnotesize}
        \subcaption{Meta-Review}
    \end{minipage}
    \hfill
    \begin{minipage}{0.49\linewidth}
        \centering
        \includegraphics[width=\linewidth]{fig/specific_trend_review.png}
        \captionsetup{font=footnotesize}
        \subcaption{Review}
    \end{minipage}
    \hfill
    \caption{Temporal trends based on \textbf{specific semantic metrics}.}
    \label{fig:trend_specific}
\end{figure}

\begin{figure}[t]
    \centering
    \includegraphics[width=1\linewidth]{fig/detection_trend_conference.png}
    \caption{Abstarct: Trend based on detection model.}
    \label{fig:trend_detect_abs}
\end{figure}

\begin{figure}[t]
    \centering
    \includegraphics[width=1\linewidth]{fig/detection_trend_review.png}
    \caption{Abstarct: Trend based on detection model.}
    \label{fig:trend_detect_review}
\end{figure}


\subsection{LLM Penetration: Temporal Analysis}
\label{sec:Temporal Analysis}
We apply the proposed rule-based metrics and model-based detectors to assess and detect LLM penetration in recent scholarly texts (up to 2024), including abstracts, meta-reviews, and reviews. 

% We apply the proposed rule-based metrics and model-based detectors to assess and detect LLM penetration in recent scholarly texts (up to 2024), including abstracts, meta-reviews, and reviews.
% specifically, for rule-based metrics, we only adopt the four most robust general linguistic metrics across three data types.

\paragraph{Trend in Rule-based Evaluation}
For the general linguistic metrics, we use only the four most robust (AWL, LWR, SWR, FRE), which show consistent preferences across the three data types, and we adopt all four specific semantic metrics.
Figure~\ref{fig:trend_general} illustrates the trend in general linguistic features across three data types in ICLR, while Figure~\ref{fig:trend_specific} shows the trend in specific semantic features for meta-reviews and reviews. 
Almost all the metrics show consistent LLM preference trends across their associated data types, with an overall year-on-year increase, supporting the rising trend of LLM penetration in scholarly writing.
Interestingly, among these metrics used to evaluate reviews, four show anomalous trend changes in 2023, highlighting the difficulties of using rule-based metrics to track LLM penetration in the complex and varied nature of review data.



% Additionally, six features in review data show anomalous trend changes in 2023, highlighting the challenges of using rule-based metrics to track LLM penetration.
% For the specific metrics, except for the SFIRF feature in reviews, which shows the same anomalous trend changes in 2023, the remaining three metrics also demonstrate a clear increasing trend in LLM penetration.



% These trends partially demonstrate the increasing LLM penetration but also highlight the limitations and challenges of relying on simple rule-based metrics.
% Specifically, regarding the general metrics, the four features (AWL, LWR, SWR, FRE) with consistent preferences in Figure~\ref{fig:compare_general}(b) demonstrate increasing LLM penetration across all data types, except for an outlier in the FRE feature for reviews in 2023. This underscores the robustness of these metrics. 
% These general metrics capture LLM penetration most effectively for abstracts, but fewer features exhibit this trend for meta-reviews and reviews, likely due to the more standardized nature of abstracts compared to the complexity of comment-type data. 
% Additionally, six features in review data show anomalous trend changes in 2023, highlighting the challenges of using rule-based metrics to track LLM penetration.
% For the specific metrics, except for the SFIRF feature in reviews, which shows the same anomalous trend changes in 2023, the remaining three metrics also demonstrate a clear increasing trend in LLM penetration.

% Specifically, the red arrows next to the title of each sub-figure represent the feature preference of LLM-generated text, as identified in the feature comparisons presented in Figures~\ref{fig:compare_general} and~\ref{fig:compare_specific}. The green checkmarks above the x-axis indicate that the feature trend for each corresponding metric aligns with the feature preference, signifying an increase in LLM penetration.

% \red{$\boldsymbol{\uparrow \downarrow \rightarrow}$} represent the feature preference of LLM-generated text, as identified in the feature comparisons presented in Figures~\ref{fig:compare_general} and~\ref{fig:compare_specific}. \green{$\boldsymbol{\checkmark}$} indicates that the feature trend for each corresponding metric aligns with the feature preference, signifying an increase in LLM penetration.

\paragraph{Trend in Model-based Detection}
Based on the performance shown in Table~\ref{tab:fine-detection} and the available evaluation data, we select ScholarDetect$_\text{Hybrid}$, which performs best on abstracts, to detect instances of LLM-assisted writing in seven conference abstracts.
% Based on the performance presented in Table~\ref{tab:fine-detection} and the available evaluation data sources, we adopt ScholarDetect$_\text{Hybrid}$, which performs best on abstracts, to probe seven conference abstracts for identifying instances of LLM-assisted writing.
Additionally, we utilize three variants of ScholarDetect (Abs, Meta, Hybrid) to analyze all ICLR meta-reviews and reviews. The detected LLM penetration rates (i.e., the proportion of text predicted to be LLM-generated) are presented in Figures~\ref{fig:trend_detect_abs} and~\ref{fig:trend_detect_review}.
In the abstract evaluation data, the LLM penetration rate across all involved conferences increases starting in 2023 and continues to rise in 2024, likely driven by ChatGPT’s initial release in November 2022 and its subsequent updates.
In contrast, a noticeable increase appears in 2024 for comment-based data, particularly in reviews, although the overall rate remains lower than in abstracts.
This may be attributed to the 2023 update of ChatGPT\footnote{\href{https://help.openai.com/en/articles/6825453-chatgpt-release-notes}{ChatGPT — Release Notes}}, which enabled PDF uploads and content analysis, as well as the higher standards required for LLM-generated content in reviews, which limit the penetration rate.
Specifically, ScholarDetect$_\text{Hybrid}$ predicts the highest LLM penetration rate for two comment-based data types in 2024.  For shorter meta-review texts, ScholarDetect$_\text{Abs}$'s rate is close to ScholarDetect$_\text{Hybrid}$ but higher than ScholarDetect$_\text{meta}$. We hypothesize this is due to the greater role of LLMs in refining these texts. Based on insights from ~\citet{cheng2024beyond}, we propose a fine-grained LLM-generated text detection approach using three-class role recognition (human-written, LLM-synthesized, LLM-refined) for meta-reviews. Our results show that the LLM-refined role plays a more dominant part in LLM penetration.\footnote{Experimental details of the three-class LLM role recognition are in Appendix~\ref{app:Fine-Grained}.}



\begin{table}[t]
    \centering
    \scalebox{0.8}{
    \begin{tabular}{@{}l|p{8cm}}
    \toprule
    \textbf{POS} & \textbf{ Top 10 GPT-4o Preferred Words in Meta-Reviews} \\ \midrule
    \textbf{\small NOUN}   & \small \underline{\textbf{refinement}}, \underline{\textbf{advancements}}, \underline{\textbf{methodologies}}, \underline{\textbf{articulation}}, 
    \textbf{highlights}, \underline{reliance}, \underline{\textbf{enhancement}}, \underline{\textbf{underpinnings}}, \underline{\textbf{enhancements}}, \underline{\textbf{transparency}}  \\ \midrule
    \textbf{\small VERB}   & \small \underline{enhance}, \underline{enhancing}, deemed, \underline{\textbf{showcasing}}, express, \underline{offering}, \underline{enhances}, \underline{\textbf{recognizing}}, commend, praised                  \\ \midrule
    \textbf{\small ADJ}  & \small \underline{\textbf{innovative}}, \underline{\textbf{collective}}, \underline{enhanced}, \underline{\textbf{established}}, \underline{notable}, \underline{outdated}, varied, \underline{undefined}, \underline{\textbf{comparative}}, \underline{\textbf{noteworthy}}
            \\ \midrule
    \textbf{\small ADV}  & \small \underline{\textbf{collectively}}, \underline{\textbf{inadequately}}, \underline{\textbf{reportedly}}, \underline{\textbf{comprehensively}}, \underline{robustly}, \underline{\textbf{occasionally}}, \underline{\textbf{predominantly}}, \underline{notably}, \underline{\textbf{innovatively}}, \underline{\textbf{effectively}}
          \\ 
  \bottomrule
  \toprule
    \textbf{POS} & \textbf{ Top 10 GPT-4o Preferred Words in Abstracts} \\ \midrule
    \textbf{\small NOUN}   & \small abstract, \underline{\textbf{advancements}}, realm, \underline{\textbf{alterations}}, aligns, \underline{\textbf{methodologies}}, \underline{clarity}, \underline{\textbf{adaptability}}, \underline{surpasses}, \underline{\textbf{examination}}
  \\ \midrule
    \textbf{\small VERB}   & \small \underline{enhancing}, \underline{\textbf{necessitates}}, \underline{\textbf{necessitating}}, \underline{featuring}, \underline{revised}, \underline{\textbf{influenced}}, \underline{\textbf{encompassing}}, \underline{enhances}, \underline{\textbf{showcasing}}, \underline{surpasses}
                  \\ \midrule
    \textbf{\small ADJ}  & \small \underline{\textbf{innovative}}, \underline{\textbf{exceptional}}, \underline{pertinent}, \underline{intricate}, \underline{pivotal}, \underline{\textbf{necessitate}}, \underline{\textbf{distinctive}}, \underline{enhanced}, akin, potent
            \\ \midrule
    \textbf{\small ADV}  & \small \underline{\textbf{inadequately}}, \underline{\textbf{predominantly}}, \underline{\textbf{meticulously}}, \underline{\textbf{strategically}}, \underline{notably}, abstract, swiftly, \underline{\textbf{additionally}}, \underline{adeptly}, \underline{thereby}
          \\ 
  \bottomrule
    \end{tabular}
    }
    \caption{Top-10 LLM-preferred words in GPT-4o-generated vs. human-written meta-reviews and abstracts. \textbf{Bold} denotes \textit{long words}, and \underline{underlined} denotes \textit{complex-syllabled words}.}
    \label{tab:word_case}
\end{table}

\subsection{Case Study}
\label{sec: Case Study}
To investigate the specific differences between LLM-generated and human-written text, we focus on GPT-4o, conducting case studies at both the word and pattern levels. 
(i) At the word level, we design a Two-Sample t-test based on word proportions~\cite{cressie1986use, 10.1093/biomet/34.1-2.28}\footnote{\url{https://www.statology.org/two-sample-t-test/}. Method details of case studies and additional results are in the Appendix~\ref{app: cs-details}.} to identify the LLM-preferred words.
Table~\ref{tab:word_case} shows the top 10 preferred words in four key part-of-speech (POS)
% \footnote{\lz{We have tried various POS tagging tools such as NLTK, spaCy, Stanford CoreNLP and others. All the tools that we have tried have the problem of mislabeling the POS of minor words.}} 
categories from GPT-4o-generated abstracts and meta-reviews, compared to those in human-written versions. We find that LLMs tend to generate \textit{long words} ($\geq$ 10 letters) and \textit{complex-syllabled words} ($\geq$ 3 syllables)~\cite{gunning1952technique}. This further supports the reliability of the four general linguistic metrics for assessing LLM penetration.
Moreover, GPT-4o shows a strong preference for the word `enhance' in scholarly writing and peer reviews, with its variants appearing in the top 10 list.
(ii) At the pattern level, manual inspection of paired data samples from comment-based data\footnote{Conducted by one of the authors on 100 paired meta-reviews and 20 paired reviews.}, followed by automated evaluation of the full dataset, reveals that human-written (meta-)reviews exhibit: 
\textit{personability}, frequently using the first person to express opinions; \textit{interactivity}, often incorporating questions; and \textit{attention to detail}, citing relevant literature to support arguments. 


%\subsubsection{Word-level}
% Considering the validity of word-level preference, we introduce a two-sample statistical hypothesis-testing framework~\cite{cressie1986use, 10.1093/biomet/34.1-2.28}\footnote{\url{https://www.statology.org/two-sample-t-test/}} to explore LLMs' preferred words. The details are presented in the \ref{subsubsec:hypothesis_calculation_judgment} section of the appendix.

% A manual inspection of a selected word set reveals some typical features of words preferred by LLMs:
% (1) \textbf{Rich and precise in meaning}: Words that convey a complex and exact sense. (e.g., "Applicability"(noun) specifically refers to "the applicability of a theory)
% (2) \textbf{Strong sense of academic formality}: Words used to maintain a serious and formal tone. (e.g., “Enhance” implicitly conveys the precise meaning of "systematic improvement", adding a serious and formal style to texts)

%Table~\ref{tab:gpt4o_preferred_meta_t5} presents the top 5 most preferred words for GPT4o, categorized by POS in the meta-review. These words reflect the two features mentioned before.

%For instance: 
%"Comprehensive"(adjective) precisely means "thorough and without omission". 
%"Applicability"(noun) specifically refers to "the applicability of a theory" .
%"Enhance" (verb) is more formal than "improve" and implicitly conveys the precise meaning of "systematic improvement".
%"particularly"(adverb) precisely means "especially", adding a serious and formal style to texts.



% \begin{table}[h]
%    \centering
%    \scalebox{0.9}{
%        \begin{tabular}{@{}l|p{2cm}|c|p{2cm}@{}}
%            \toprule
%            \textbf{POS} & \begin{tabular}[c]{@{}c@{}}\textbf{word}\end{tabular} & \textbf{POS} & \begin{tabular}[c]{@{}c@{}}\textbf{word}\end{tabular} \\
%            \midrule
%            \multirow{5}{*}{\textbf{Noun}} 
%            & clarity & \multirow{5}{*}{\textbf{Adj}} 
%            & innovative \\
%            & applicability &  & potential \\
%            & comparisons &  & comprehensive \\
%            & validation &  & theoretical \\
%            & improvements &  & insufficient \\
%            \midrule
%            \multirow{5}{*}{\textbf{Verb}} 
%            & existing & \multirow{5}{*}{\textbf{Adv}} 
%            & particularly \\
%            & introduces &  & however \\
%            & enhance &  & effectively \\
%            & presents &  & collectively \\
%            & lacks &  & especially \\
%            \bottomrule
%        \end{tabular}
%    }
%    \caption{Top5 preferred words for GPT4o categorized by POS(meta-review)}
%    \label{tab:gpt4o_preferred_meta_t5}
% \end{table}

%\subsubsection{Sentence and paper Level}
%Further more, we conducted an analysis of the differences between human-written and llm-generated texts at the sentence and paper level, focusing on meta-reviews and reviews.
%Furthermore, we proposed three major categories of features - \textbf{language style features}, \textbf{structural pattern features}, and \textbf{content functional features} - to analyse the differences between human-written and LLM-generated texts at the sentence and paper levels, focusing on meta-reviews and reviews.

%\paragraph{meta-review:} In meta-reviews, LLMs have features as follows:
%\textbf{language style features:}
%(1) \textbf{Question Usage Style:} LLMs don't like to use questions. Instead, they prefer to state the information directly.
%(2) \textbf{First-Person Term Frequency:} First-person expressions such as “we...” and “I...” are rare in LLM-generated meta-reviews.
%\textbf{structural pattern features:}
%(3) \textbf{No additional Citations:} LLMs do not cite additional materials, such as citations like "[1] https://arxiv.org/pdf/2112.06905.pdf" in review.
%(4) \textbf{Preferred Opening Phrases:} Different LLMs have their own favored starting patterns. E.g., the pattern “(The) Reviewers...” is preferred by Gemini and Claude.
%\textbf{content functional features:}
%(5) \textbf{Content Focus Difference:} LLMs mainly focus on providing summaries in meta-reviews. In contrast, human offer more detailed suggestions or different opinions.% or express the reviewers' attitudes towards the paper.

%The specific examples and data used for this analysis are included in the appendix for reference.

%\paragraph{review:} %The review consists of five parts: 1. Summary of the Paper; 2. Strengths and Weaknesses; 3. Clarity, Quality, Novelty, and Reproducibility; 4. Summary of the Review. 
%In reviews, LLMs exhibit the same features in terms of (1) \textbf{Question Usage Style} and (2) \textbf{First-Person Term Frequency} as in meta - reviews. (3) \textbf{No additional Citations:}
%LLMs also exhibit some more specific features
%(4) \textbf{Content Telling Style:} In the "Summary of the Paper" part, LLMs directly state the paper's content, while human often mention the authors' thoughts lhor thinks...", which rarely appears in LLM - generated paper summaries.
%(5) \textbf{Preferred Opening Phrases:} In the "Summary of the Review" section of review, GPT4o, Gemini and Claude all prefer to "This/The paper...".
%(6) \textbf{Cliche-filled summary:} In the "Summary or the review" part, LLMs often make broad and empty statements and exaggerate by overusing general phrases like "significant contribution".%, and this situation is more obvious in reviews than in meta-reviews.
%(6) \textbf{No additional Citations:} LLMs do not cite additional materials, such as citations like "[1] https://arxiv.org/pdf/2112.06905.pdf" in review.

%The specific examples and data used for this analysis are included in the appendix for reference.

\section{Conclusion and Suggestions}

Our work, including the creation of \texttt{ScholarLens} and the proposal of \texttt{LLMetrica}, provides methods for assessing LLM penetration in scholarly writing and peer review. By incorporating diverse data types and a range of evaluation techniques, we consistently observe the growing influence of LLMs across various scholarly processes, raising concerns about the credibility of academic research. As LLMs become more integrated into scholarly workflows, it is crucial to establish strategies that ensure their responsible and ethical use, addressing both content creation and the peer review process. 

Despite existing guidelines restricting LLM-generated content in scholarly writing and peer review,\footnote{\href{https://aclrollingreview.org/acguidelines\#-task-3-checking-review-quality-and-chasing-missing-reviewers}{Area Chair} \&  \href{https://aclrollingreview.org/reviewerguidelines\#q-can-i-use-generative-ai}{Reviewer} \& \href{https://www.aclweb.org/adminwiki/index.php/ACL_Policy_on_Publication_Ethics\#Guidelines_for_Generative_Assistance_in_Authorship}{Author} guidelines.} challenges still remain. 
To address these, we propose the following based on our work and findings: 
(i) \textbf{Increase transparency in LLM usage within scholarly processes} by incorporating LLM assistance into review checklists, encouraging explicit acknowledgment of LLM support in paper acknowledgments, and 
reporting LLM usage patterns across diverse demographic groups;
% reporting LLM penetration based on social demographic features;
(ii) \textbf{Adopt policies to prevent irresponsible LLM reviewers} by establishing feedback channels for authors on LLM-generated reviews and developing fine-grained LLM detection models~\cite{abassy-etal-2024-llm, cheng2024beyond, artemova2025beemobenchmarkexperteditedmachinegenerated} to distinguish acceptable LLM roles (e.g., language improvement vs. content creation);
(iii) \textbf{Promote data-driven research in scholarly processes} by supporting the collection of review data for further robust analysis~\cite{dycke-etal-2022-yes}.\footnote{\url{https://arr-data.aclweb.org/}}

% make LLM usage transparent in scholarly processes: such as incorporating LLM usage into review checklists, encouraging explicit acknowledgment of LLM assistance in paper acknowledgments, and reporting LLM penetration based on social demographic features; (ii) Adopt policies to prevent irresponsible LLM reviewers: such as providing authors feedback on LLM-assisted reviews, and developing fine-grained LLM detection models~\cite{cheng2024beyond} to distinguish acceptable LLM roles (e.g., language improvement vs. content creation); (iii) Encourage data-driven research in scholarly processes: such as supporting review data collection for further research.

 


\section*{Limitations}
While this study provides valuable insights into the penetration of LLMs in scholarly writing and peer review, it may not fully represent the complexities of the real-world scenario. 
On one hand, the analysis focuses on peer review data from the ICLR conference, where the process is fully transparent and the quality of reviews is generally well-maintained. However, in many journals and conferences where peer review remains closed, the penetration of LLMs could be even more pronounced. 
On the other hand, the data simulation may not fully capture the intricate dynamics of LLM-human collaboration in real-world settings, making it difficult to distinguish between acceptable and unacceptable levels of LLM involvement, and potentially leading to a reduced ability of the model to detect LLM-generated text, which in turn lowers the assessment of LLM penetration.
Therefore, the penetration of large language models in scholarly writing and peer review may be more significant in real-world scenarios than what is presented in this study.
% \section*{Acknowledgments}



% Bibliography entries for the entire Anthology, followed by custom entries
%\bibliography{anthology,custom}
% Custom bibliography entries only
\bibliography{custom, anthology}

\appendix

\section{Prompts for Data Construction}
\label{app: prompts}


\begin{table*}[t]
\centering
\scalebox{0.85}{
\begin{tabular}{c|p{18cm}@{}}
\toprule
\textbf{Id} & \textbf{Prompt} \\  \midrule
1 & Can you help me revise the abstract? Please response directly with the revised abstract: \{abstract\} \\ 

2 & Please revise the abstract, and response directly with the revised abstract: \{abstract\} \\ 

3 & Can you check if the flow of the abstract makes sense? Please response directly with the revised abstract: \{abstract\} \\ 

4 & Please revise the abstract to make it more logical, response it directly with the revised abstract: \{abstract\} \\ 

5 & Please revise the abstract to make it more formal and academic, response it directly with the revised abstract: \{abstract\} \\ 
        \bottomrule

\end{tabular}}
\caption{Five distinct prompts used to refine human-written abstracts.}
\label{tab:prompt diversity}
\end{table*}




\begin{table*}[ht]
\centering
\scalebox{0.85}{
\begin{tabular}{@{}p{18.5cm}@{}}
\toprule

\multicolumn{1}{l}{\textbf{Basic Prompt Guideline}} \\ \cmidrule(r){1-1}
You are an AI assistant tasked with generating meta-reviews from multiple reviewers' feedback.\\
Please write a meta review of the given reviewers' response around \{$n$\} words.\\
Do not include any section titles or headings. 
Do not reference individual reviewers by name or number. 
Instead, focus on synthesizing collective feedback and overall opinion.

\\
\#\#\# Abstract: \{abstract\}

\#\#\# Reviewers' feedback:\{review\_text\}
\\ \midrule

\multicolumn{1}{l}{\textbf{Formatted Prompt Guideline 1}} \\ \cmidrule(r){1-1}
You are an AI assistant tasked with generating meta-reviews from multiple reviewers' feedback.\\
Please write a meta review of the given reviewers' response around \{$n$\} words. \\
Do not include any section titles or headings. Do not reference individual reviewers by name or number. Instead, focus on synthesizing collective feedback and overall opinion.

\

Please include the given format in your meta review:

Give a concise summary here.\\
Strength: [List the strengths of the paper in points based on reviews.]\\
Weakness: [List the weaknesses of the paper in points based on reviews.]

\\

\#\#\# Abstract: \{abstract\} \\
\#\#\# Reviewers' feedback:\{review\_text\}
\\ \midrule

\multicolumn{1}{l}{\textbf{Formatted Prompt Guideline 2}} \\ \midrule
You are an AI assistant tasked with generating meta-reviews from multiple reviewers' feedback.

Please write a meta review of the given reviewers' response around \{$n$\} words. 


Do not include any section titles or headings. Do not reference individual reviewers by name or number.
Instead, focus on synthesizing collective feedback and overall opinion.

\

Please include the given format in your meta review:

Give a concise summary here.\\
Pros: [List the strengths of the paper in points based on reviews.]\\
Cons: [List the weaknesses of the paper in points based on reviews.]

\\

\#\#\# Abstract: \{abstract\}
\#\#\# Reviewers' feedback:\{review\_text\}
\\ \bottomrule

\end{tabular}}
\caption{Three prompts as guidelines for constructing LLM-generated meta-reviews. Here, $n$ represents the approximate word length for the generated content.}
\label{tab:prompt meta}
\end{table*}


\begin{table*}[t]
    \centering

    \begin{tabular}{>{\centering\arraybackslash}m{5cm}>{\centering\arraybackslash}m{3cm}>{\centering\arraybackslash}m{3cm}>{\centering\arraybackslash}m{3cm}}
        \toprule
        Word Count  & Basic Prompt & Formatted 1 & Formatted 2 \\
        \midrule
        $n \leq 50$ & 1.000 & 0.000 & 0.000 \\
        $50 < n \leq 110$ & 0.800 & 0.100 & 0.100 \\
        $110 < n \leq 160$ & 0.400 & 0.300 & 0.300 \\
        $160 < n \leq 220$ & 0.550 & 0.225 & 0.225 \\
        $> 220$ & 0.250 & 0.375 & 0.375 \\
        \bottomrule
    \end{tabular}
    \caption{The statistical distribution of word lengths observed for each involved meta-review format.}
    \label{tab:prompt frequency}
\end{table*}



\subsection{Prompts for abstract}
To ensure diversity in the refinement process, we design five different prompts for polishing the abstract, as shown in Table \ref{tab:prompt diversity}. Each human-written abstract is randomly assigned one prompt to generate the refined content.



\subsection{Prompts for meta-review}
To ensure that LLM-generated meta-reviews closely mirror the writing style of human-written meta-reviews and maintain authenticity, we analyze the characteristics of human-written meta-reviews. Based on this analysis, we provide three generation templates as guidelines for constructing LLM-generated meta-reviews, as shown in Table~\ref{tab:prompt meta}. The basic prompt does not include any formalized structures, while the other two prompts define more distinct meta-review formats.
Specifically, we conduct a detailed statistical analysis of the frequencies of these two paradigms and selected the corresponding prompts based on these frequencies. The probabilities we use are shown in Table \ref{tab:prompt frequency}.
% Specifically, $n$ represents the approximate word length of the generated content, which is randomly selected according to the statistical distribution of word lengths observed for each format (see Table \ref{tab:prompt frequency}).

% Specifically, we provide three generation template as a guideline for the LLM.
% Purpose on generating the meta-review on a synthetical perspective, we design the prompts based on the abstract and the few reviews of the papers. For make the generated meta-reviews consisted with the actual situation, we limited the generated length by word count and added two common paradigms to extended our prompt diversity, which respectively shown as the \textcolor{blue}{blue-marked} and the 
% \textcolor{red}{red-marked}.

% Each meta-review was
% corresponding one of the MetaReview template and feed to LLMs, triggered through probability determined by word counts, detailing in Table \ref{tab:prompt frequency}.

% However, in our preliminary attempts, we found that the LLMs prefer to summarize the comments of each reviewer separately (such as Reviewer 1: ..., Reviewer 2: ...), while the human-written reviews occurred only a few number of cases like this(only 462 reviews in 19,834 from 2017 to 2024 have occurred). Since we didn't provide the specific paper and reviewer numbers to the LLM, to avoid the LLM generating content with irrelevant identifications, we set the conditions in the \textcolor{orange}{orange-marked} part of the prompt (Table~\ref{tab:prompt meta}). 

% Thus we 

% The table \ref{tab:prompt meta} shows the three prompts

% \subsubsection{Basic-MetaReview Prompt}
% We define the basic prompt at first. In this prompt, we define the role of an AI assistant and its task of generating meta-reviews based on reviews and abstracts at the beginning.

% To ensure that the word count distribution of meta-reviews in the constructed dataset is consistent with the actual situation, we have added a word count limit, which is the \textcolor{blue}{blue-marked} part of prompt in the table \ref{tab:prompt meta}.

% In our preliminary attempts, we found that large language models (LLMs) are very likely to summarize the comments of each reviewer separately (such as Reviewer 1: ..., Reviewer 2: ...). However, in the human data, only a small number of cases are like this (only 462 out of 19,834 cases from 2017 to 2024 have this situation). Therefore, we set the conditions in the \textcolor{orange}{orange-marked} part of prompt in table \ref{tab:prompt meta} to make the generated meta-reviews more in line with the actual paradigm.

% We provide the abstract and reviews to the LLM at the end of prompt.

% \subsubsection{Formatted-MetaReview 1\&2 Prompt}
% Through statistical analysis of the data from 2017 to 2019, we have identified two common paradigms. To make the generated meta-reviews more consistent with the paradigms in actual situation, we have incorporated these two paradigms into the basic prompts, resulting in "Formatted-MetaReview 1" and "Formatted-MetaReview 2". The \textcolor{red}{red-marked} parts in their prompts are the relevant paradigm contents. 

% We conduct a detailed statistical analysis of the frequencies of these two paradigms and selected the corresponding prompts based on these frequencies. The probabilities we use are shown in Table \ref{tab:prompt frequency}.

% \begin{table*}[ht]
\centering
\footnotesize 
\begin{tabular}{@{}l|p{0.4\textwidth}|p{0.4\textwidth}@{}}
\toprule
\textbf{Structural Features}         & \textbf{Description} & \textbf{Computing Method} \\ \midrule
\textbf{Average Word Length}          & Measures the average number of characters in each word in the text. &  \\ 
\textbf{Average Sentence Length}       & Measures the average number of words in each sentence in the text. &  \\ 
\textbf{Average Number of Words} & Counts the average total words in the text. &  \\ 
\textbf{Average Number of Sentences}     & Counts the average total sentences in the text. &  \\ \midrule




\textbf{Lexical Features}          \\ \midrule
\textbf{Percent of Long Words}   & Measures the percentage of “long words” in the text. We define long words as words with more than 6 letters. & $\text{Percent of Long Words} = \left(\frac{\text{Long Words}}{\text{Total Words}}\right) \times 100\%$ \\ 
\textbf{Percent of Stopwords}   & Measures the percentage of common words in the text, like "the," "and," and "is." A high percentage means the text has more filler words. & $\text{Percent of Stopwords} = \left(\frac{\text{Stopwords}}{\text{Total Words}}\right) \times 100\%$ \\ 
\textbf{Vocabulary Richness}   & Measures how many unique words are used compared to the total number of words. "Unique words" are the different words that appear in a text, without counting repeats. 

For example, in “apple orange apple,” there are 3 words, but only 2 unique words: "apple" and "orange."

A higher ratio means the text has more varied vocabulary. & $\text{Vocabulary Richness} = \frac{\text{Unique Words}}{\text{Total Words}}$  \\ \midrule




\textbf{Linguistic Quality Features}         \\ \midrule
\textbf{Syntactic Diversity} & Syntactic diversity assesses structural complexity by examining clause patterns. &  \\ 
\textbf{Readability} & Readability indices measure how easily a reader can understand a written text. We use the following metrics to measure readability:

-Flesch-Kincaid Scores: Measures how easy the text is to read, based on sentence length and word syllables.

-Flesch Scores: Rates how easy or hard the text is to read for different age levels.

-Gunning Fog Scores: Shows the education level needed to understand the text.

-Coleman-Liau Scores: Uses word and sentence length to tell what grade level can read the text.

-Dale-Chall Scores: Rates readability based on the use of familiar or common words.

-ARI Scores: Estimates readability using the number of characters and words.

-Linsear Write Scores: Measures readability by looking at sentence structure and word difficulty.
&

We use the \text{textstat} library directly to calculate these seven metrics. The specific calculation formulas are as follows:

\text{Flesch-Kincaid Grade Level} = 

$0.39 \times \frac{\text{Total Words}}{\text{Total Sentences}} + 11.8 \times \frac{\text{Total Syllables}}{\text{Total Words}} - 15.59 $
\medskip

\text{Flesch Reading Ease} = 

$206.835 - 1.015 \times \frac{\text{Total Words}}{\text{Total Sentences}} - 84.6 \times \frac{\text{Total Syllables}}{\text{Total Words}}$ 
\medskip

\text{Gunning Fog Index} = 

$0.4 \times \left( \frac{\text{Total Words}}{\text{Total Sentences}} + 100 \times \frac{\text{Complex Words}}{\text{Total Words}} \right)$ 
\medskip

\text{Coleman-Liau Index} = 

$0.0588 \times \text{Average Number of Letters per 100 Words} - 0.296 \times \text{Total Sentences} - 15.8$ 
\medskip

\text{Dale-Chall Readability Score} = 

$0.1579 \times \frac{\text{Difficult Words}}{\text{Total Words}} + 0.0496 \times \frac{\text{Total Words}}{\text{Total Sentences}} + 3.6365 $
\medskip

\text{Automated Readability Index (ARI)} =

$4.71 \times \frac{\text{Total Characters}}{\text{Total Words}} + 0.5 \times \frac{\text{Total Words}}{\text{Total Sentences}} - 21.43$ 
\medskip

\text{Linsear Write Formula} = 
$\frac{\left( \text{Easy Words} + 3 \times \text{Difficult Words} \right)}{\text{Total Sentences}}$




\\ 
\textbf{Fluency} & Fluency is the quality of a text based on how well it is written and whether it is grammatically correct. & this may be abandoned \\ 




\bottomrule
\end{tabular}
\caption{Direct Linguistic Features with brief descriptions}
\label{tab:Direct_Linguistic_Features_description}
\end{table*}

% \begin{table*}[ht]
\centering
\footnotesize 
\begin{tabular}{@{}l|p{0.4\textwidth}|p{0.4\textwidth}@{}}
\toprule
\textbf{Relatedness}         & \textbf{Description} & \textbf{Computing Method} \\ \midrule
\textbf{bert score}      
 &BERT score is a metric that evaluates the alignment between two texts by comparing their contextual embeddings. It compares words in the candidate text with words in the reference text one by one, accumulating word vector similarities to get an overall similarity score. Precision, Recall, and F1 scores are calculated separately. We compute the BERT score between the meta-review and the three reviews, then compute the average and variance to measure their similarity.  &The BERT score between two texts is typically calculated using the cosine similarity of their contextual embeddings. Given a reference text R(here is review) and a candidate text C(here is meta-review), BERT score evaluates Precision, Recall, and F1 metrics as follows:  
 
\text{Precision (P)}

$P = \frac{1}{|C|} \sum_{c \in C} \max_{r \in R} \text{cos\_sim}(c, r)$ 
\medskip

\text{Recall (R)}

$R = \frac{1}{|R|} \sum_{r \in R} \max_{c \in C} \text{cos\_sim}(r, c)$ 
\medskip

\text{F1 Score}

$F1 = \frac{2 \cdot P \cdot R}{P + R}$ 
 \\ 
\textbf{sentence bert}       & Sentence-BERTScore is a metric that evaluates the alignment between two texts by comparing their contextual embeddings, optimized for sentence-level similarity. It encodes each sentence or short text into a single vector for the whole sentence, then calculates the similarity between two sentence vectors (with cosine similarity) to measure how similar the sentences are. We compute the Sentence-BERTScore between the meta-review and each of the three reviews, then calculate the average and variance to measure their overall similarity. &  
$\mathbf{E}_{\text{meta-review}}$ represents the sentence embedding of the meta-review.
$\mathbf{E}_{\text{review}}$ represents the sentence embedding of a single review.

\text{Similarity} = 
$\frac{\mathbf{E}_{\text{meta-review}} \cdot \mathbf{E}_{\text{review}}}{\|\mathbf{E}_{\text{meta-review}}\| \|\mathbf{E}_{\text{review}}\|}$

\\ 
\midrule

\textbf{Writers' attitude}          & &  \\ \midrule
\textbf{Emotions}   &Measure the distribution of emotions in the text. There are various types of emotions, such as disappointment, sadness, annoyance, joy, anger, caring, optimism, curiosity, gratitude, and more. These emotions cover a range from positive to negative, helping us to understand the emotional tone and intent of the writing. & We use $SamLowe/roberta-base-go\_emotions$,a model trained on the Go-Emotions dataset to classify over 28 emotions. \\

\textbf{Persuasiveness}   &Binary classification (Non-persuasive, Persuasive)
 &To finish the task, we use the $paragon-analytics/roberta\_persuade model$, a binary classifier, to finish this task.\\ 
\textbf{Convincingness}   &Binary classification (Non-Convincingness, Convincingness) &To finish the task, we use $jakub014/bert-base-uncased-IBM-argQ-30k-finetuned-convincingness-IBM$, a model trained on the assumption that convincingness is linked to
high quality arguments which are typically clear, relevant, and with high impact.\\ \midrule


\bottomrule
\end{tabular}
\caption{Indirect Features with brief descriptions}
\label{tab:Indirect_Linguistic_Features_description}
\end{table*}


% \section{Feature Description/definition}
% \ref{tab:Direct_Linguistic_Features_description}
% \ref{tab:Indirect_Linguistic_Features_description}


% \begin{figure*}[ht!]
% \centering

% \begin{prompt}[title={Useless Doc ($r_0$)}] \label{prompt_r0}
% <Documents> \newline
% [1] \texttt{\{<Document 1>\}}\newline
% </Documents> \newline

% Your task is to generate an English question q* and a corresponding response a* based on the provided <Documents>. Please note that the question q* can take various forms, not limited to questions with a question mark, but also including statements, instructions, and other formats. You need to follow the requirements below to generate the q* and a* \textcolor{blue}{(RAG Paradigms)}:  

% \textcolor{blue}{1. q* should be related to the <Documents>, but the <Documents> can not provide any useful information for answering q*. \newline
% 2. a* should be able to answer q*, ensuring that the response a* is accurate, detailed, and comprehensive.} 
% \newline

% Additionally, to ensure diversity, richness, and high quality in the question q* you generate, we will randomly provide a question for you to emulate. In other words, while satisfying the requirements above, make q* similar in task requirement and expression to the \textcolor{darkred}{<Simulated Instruction>} below:  

% \textcolor{darkred}{
% <Simulated Instruction> \newline
% \texttt{\{<Simulated Instruction>\}} \newline
% </Simulated Instruction> \newline
% }

% Please directly generate the question-answer pair (q*, a*) following all the rules above in the format of \{"q*": ..., "a*": ...\}. Ensure the quality of the generated (q*, a*).
% \end{prompt}
% \caption{The prompt for synthesizing Useless Doc ($r_0$) data.}
% \label{tab:r_0}
% \end{figure*}










\section{Dataset Details}
\label{app:dataset}
Table~\ref{tab: AcademicLens-statistics} shows the statistics of \texttt{ScholarLens}. 
The LLM versions used for data construction are: GPT-4o (\texttt{gpt-4o-2024-08-06}), Gemini-1.5 (\texttt{gemini-1.5-pro-002}), and Claude-3-Opus (\texttt{claude-3-opus-20240229}).

% GPT-4o (\texttt{gpt-4o-2024-08-06}), Gemini-1.5 (\texttt{gemini-1.5-pro-002}), Claude-3-Opus (\texttt{claude-3-opus-20240229}).



\begin{table*}[]
\scalebox{0.85}{
\begin{tabular}{@{}l|r|r|r@{}}
\toprule
\textbf{Academic Aspects} & \textbf{Human (Size)} & \textbf{LLM Source(Size)}                                & \textbf{Data Source}  \\ \midrule
\textbf{Abstract}         & \multirow{2}{*}{2831} & GPT-4o (2831) / Gemini-1.5 (2831) / Claude-3 Opus (2831) & \multirow{2}{*}{Ours} \\ \cmidrule(r){1-1} \cmidrule(lr){3-3}
\textbf{Meta-Review}      &                       & GPT-4o (2831) / Gemini-1.5 (2831) / Claude-3 Opus (2831) &                       \\ \midrule
\textbf{Review}           & 20 × |R|              & GPT-4 (20) / Gemini-1.5 (20) / Claude-3 Opus (20)        & ReviewCritique        \\ \bottomrule
\end{tabular}}
\caption{Statistics of \texttt{ScholarLens}: 20 × |R| represents 20 papers, each with |R| reviews, where |R| varies by paper.}
\label{tab: AcademicLens-statistics}
\end{table*}


% Please add the following required packages to your document preamble:
% \usepackage{booktabs}
% \usepackage{multirow}
\begin{table}[]
\centering
\begin{tabular}{@{}l|l|c|r@{}}
\toprule
\textbf{Data Type}                    & \textbf{Data Source} & \textbf{Train}        & \textbf{Test} \\ \midrule
\multirow{2}{*}{\textbf{Abstract}}    & Human                & \multirow{4}{*}{1981} & 850           \\
                                      & LLM                  &                       & 2550          \\ \cmidrule(r){1-2} \cmidrule(l){4-4} 
\multirow{2}{*}{\textbf{Meta-Review}} & Human                &                       & 850           \\
                                      & LLM                  &                       & 2550          \\ \midrule
\multirow{2}{*}{\textbf{Review}}      & Human                & \multirow{2}{*}{-}    & 20            \\
                                      & LLM                  &                       & 60            \\ \bottomrule
\end{tabular}
\caption{Dataset split for training detection models.}
\label{tab:data_split}
\end{table}




\section{General Linguistic Metrics Implementation}
\label{app:general}
% We have 10 general linguistic metric: Average Word Length (AWL), Long Word Ratio (LWR), Stopword Ratio (SWR), Type Token Ratio (TTR), Average Sentence Length (ASL), Dependency Relation Variety (DRV), Subordinate Clause Density (SCD), Flesch Reading Ease (FRE), Sentiment Polarity Score (PS), and Sentiment Subjectivity Score (SS).
For word-level metrics, \texttt{NLTK} tokenization is used, and only alphabetic words are considered.
For sentence-level metrics, \texttt{spaCy} is used to process the text and extract features such as sentence length and the dependency relation label of each word. 
Additionally, Sentiment Polarity Score (PS) and Sentiment Subjectivity Score (SS) are evaluated using \texttt{TextBlob}, while FRE is calculated using \texttt{Textstat}.


% \lz{wait}


\section{Fine-Grained Detection Model}
\label{app:Fine-Grained}
Using the existing meta-review data from \texttt{ScholarLens} (including both human-written and LLM-synthesized versions), we apply the LLM-refined abstract construction method to generate an LLM-refined version for each human-written meta-review. We utilize GPT-4o as the single LLM source and train a fine-grained three-class detector on the meta-reviews using the same data split. This trained detection model is then used to predict the 2024 ICLR meta-reviews, with approximately 35.32\% predicted as LLM-refined and 1.39\% as LLM-synthesized, which  show that the LLM-refined role plays a more dominant part in LLM penetration.

\section{Case Study Details}
\label{app: cs-details}
\subsection{Word-Level Algorithm and Experiments}
\subsubsection{Hypothesis Testing Algorithm}
\label{app:TestingAlgorithm}
Building on the Two-Sample t-test, we propose a word-proportion-based method to identify word-level LLM preferences.
Specifically, given a set of pairs of human-written and LLM-generated texts $\mathcal{D} =\left\{ \left( x_{i}^{h}, x_{i}^{l} \right) \right\} $, where where $x_i^{h}$ represents the human-written case and $x_i^{l}$ represents the corresponding LLM-generated version, our goal is to determine whether a word $w$ is preferentially generated by the LLM.

\paragraph{(i) Word Proportion} 
We define the proportion of word $w$ appearing in the human-written set $\left\{ x_{i}^{h} \right\} $ and the LLM-generated set $\left\{ x_{i}^{l} \right\} $ as $\hat{p}_h\left( w \right) $ and $\hat{p}_l\left( w \right) $, respectively, representing the fraction of texts in which $w$ occurs:
\begin{equation}
    \hat{p}_h\left(w\right) ={\small{\frac{\mathrm{cnt}_h\left( w \right) +\epsilon}{\left| \mathcal{D} \right|}}}
\end{equation}
\begin{equation}
    \hat{p}_l(w) ={\small{\frac{\mathrm{cnt}_h\left( w \right) +\epsilon}{\left| \mathcal{D} \right|}}}
\end{equation}
where $\mathrm{cnt}\left( w \right) =\sum\nolimits_i^{}{\mathbb{I} \left( w\in x_{i}^{} \right)}$ counts the number of texts in the set ${x_i}$ where the word $x$ appears, with $\mathbb{I}(w \in x_i)$ being an indicator function that returns 1 if $w$ appears in $x_i$, and $\epsilon=1$ as a smoothing constant to account for words that do not appear in a given text.




\paragraph{(ii) Hypothesis Setting}
Then, we define the following two hypotheses:
\begin{itemize}
    \item \textbf{Null hypothesis} ($H_0$): $\hat{p}_h\left( w \right) \geqslant \hat{p}_l\left( w \right) $, suggesting that LLMs do \textbf{not} preferentially generate the word $w$.
    % suggests that the word $x$ appears more frequently, or equally, in human-written text compared to LLM-generated text, suggesting that LLMs do not preferentially generate the word $x$.
    % This implies that the proportion of the word's occurrence in LLM-generated texts is not higher than that in human-written texts.(i.e., the word is preferred by LLMs)
    \item \textbf{Alternative hypothesis} ($H_1$): $\hat{p}_h\left( w \right) < \hat{p}_l\left( w \right) $, suggesting that LLMs preferentially generate the word $w$.
    % which suggests that the word $x$ appears more frequently in LLM-generated text than in human-written text, implying a preference by LLMs.
    
    % This indicates that the proportion of the word's occurrence in LLM-generated texts is significantly higher than that in human-written texts.(i.e., the word is preferred by LLMs)
\end{itemize}



% Here, $\hat{p}_1$ and $\hat{p}_2$ represent the proportions of the word's occurrence in human-written texts and LLM-generated texts.

\paragraph{(iii) Hypothesis Testing}

Considering that the variance of word proportion may differ between the two text groups, we adopt Welch's t-test to quantify the difference. Specifically, the test statistic and degrees of freedom are computed as follows:

\begin{equation}
t(w)=\frac{\hat{p}_l\left( w \right) -\hat{p}_h\left( w \right)}{\sqrt{\small{\frac{s_{h}^{2}+s_{l}^{2}}{\left| \mathcal{D} \right|}}}}
\end{equation}
\begin{equation}
    df(w)=\frac{\left( \frac{s_{h}^{2}+s_{l}^{2}}{\left| \mathcal{D} \right|} \right) ^2}{\frac{(s_{h}^{2}/\left| \mathcal{D} \right|)^2+(s_{l}^{2}/\left| \mathcal{D} \right|)^2}{\left| \mathcal{D} \right|-1}}
\end{equation}

where $s_h$ and $s_l$ represent the standard deviations of the corresponding word proportions, calculated as follows:
\begin{equation}
    s=\sqrt{\frac{\hat{p}(1-\hat{p})}{\left| \mathcal{D} \right|}}
\end{equation}


% Here, $SE$ represents the standard error of the difference between the two proportions.


% \subparagraph{Standard Error Calculation}
% The standard error $SE_i$ of the sample proportion in the $i$-th type of text is:
% \begin{equation}
% SE_i=\sqrt{\frac{\hat{p}_i(1 - \hat{p}_i)}{n_i+\alphUI_{smooth}\cdot|V|}}
% \end{equation}
% The standard error $SE$ of the difference between the two proportions is:
% \begin{equation}
% SE=\sqrt{SE_1^2 + SE_2^2}
% \end{equation}

% \subparagraph{Degrees of Freedom Calculation}
% In general, for two independent samples with variances $s_1^2$ and $s_2^2$ and sample sizes $n_1$ and $n_2$, the original formula for the degrees of freedom $df$ in Welch's $t$-test is:
% \begin{equation}
% df = \frac{\left(\frac{s_1^2}{n_1}+\frac{s_2^2}{n_2}\right)^2}{\frac{(s_1^2 / n_1)^2}{n_1 - 1}+\frac{(s_2^2 / n_2)^2}{n_2 - 1}}
% \end{equation}
% In our context, with $s_i^2 = SE_i^2$ and considering the adjusted sample sizes $n_i'=n_i+\alphUI_{smooth}\cdot|V|$, the simplified formula for degrees of freedom is:
% \begin{equation}
% df=\frac{(SE_1^2 + SE_2^2)^2}{\frac{SE_1^4}{n_1+\alphUI_{smooth}\cdot|V| - 1}+\frac{SE_2^4}{n_2+\alphUI_{smooth}\cdot|V| - 1}}
% \end{equation}

\paragraph{(iv) Hypothesis Decision}
We define the critical t-value, $t_c$, as the threshold for rejecting or accepting the null hypothesis. 
It is calculated using the inverse of the cumulative distribution function (CDF) of the t-distribution:
\begin{equation}
    t_c(w) = t_{\alpha, \, df(w)}^{-1}
\end{equation}
where $\alpha=0.05$ is the significance level.
If $t(w) > t_c(w)$, we reject the null hypothesis and conclude that the word occurs significantly more often in LLM-generated texts than in human-written texts. In this way, we can identify the words favored by LLMs.


\subsubsection{Experimental Setup}

% \lz{@Ruijie, write: how we filter these words}
% 例如需要描写到的:考虑到一些词,在上下文可能以名词出现、也可能以动词出现,所以我们没有直接以word本身作为最小单位去filter,而是使用(word, pos)为最小单位……我们使用什么词性分析;我们不考虑停用词;我们不考虑大小写……我们使用什么作为排序的准则……

To address part-of-speech variability of the same word (e.g., `record' functioning as both a noun and a verb in a sentence), we adopt (word, POS) pairs as the fundamental unit for analysis rather than isolated words. We use SpaCy for POS tagging, and stopwords are excluded from consideration. 
Using the proposed Hypothesis Testing Algorithm (\S\ref{app:TestingAlgorithm}), we filter the LLM-preferred word set and rank these words based on their Word Usage Increase Ratio (WUIR), defined as follows:
\begin{equation}
    \mathrm{WUIR}\left(w\right) ={
    \small{\frac{\mathrm{cnt}_l\left( w \right) -\mathrm{cnt}_h\left(w\right)}{\mathrm{cnt}_h\left(w\right) +\epsilon}}}
\end{equation}


% \paragraph{(i) Preprocessing Rule}
% Before filtering, we implement three preprocessing rule on texts:
% \textbf{Stopword Removal}: Exclude tokens matching NLTK's English stopword list.
% \textbf{Case Insensitivity}: Convert all words to lowercase before processing.

% \paragraph{(ii) Word-POS Pair Unit}
% To address lexical ambiguity (e.g., words like "record" functioning as both nouns and verbs), we adopt (word, POS) pairs as the fundamental unit for analysis rather than isolated words. We use Spacy for POS tagging.

% \paragraph{(iii) Hypothesis Testing}
% For words that share the same POS(such as when focusing on adjectives (ADJ), we only consider words of this specific POS), we implement the Hypothesis Testing Algorithm. Through this process, we are able to identify the set of words that are preferred by LLMs.

% \paragraph{(iv) Ranking}
% The final word set is sorted by the metric 
% \begin{equation}
% \frac{\left( \sum_i \mathbb{I}(w \in x_i^l) + \epsilon \right) - \left( \sum_i \mathbb{I}(w \in x_i^h) + \epsilon \right)}{\sum_i \mathbb{I}(w \in x_i^h) + 1}
% \end{equation}
% in descending order.


\subsubsection{Results: LLM-Preferred words}
% \lz{@Ruijie, show the words result and briefly describe them }% 先展示GPT-4o的,分别简要描述在abstract和meta-review的倾好词,最后简要提及一下Claude和Gemini的结果,如表所示,不需要额外描述多的。
%For GPT4o, in meta-reviews, \textit{long words} account for 39.96\% and \textit{complex-syllabled words} account for 67.82\%. In abstracts, the proportions are 40.48\% and 73.45\% respectively.

%For GPT4o, in the meta - review, we obtained a set of preferred words consisting of 600 NOUN, 371 VERB, 275 ADJ, and 43 ADV.
%For its abstract, the set of preferred words we counted is composed of 201 NOUN, 422 VERB, 127 ADJ, and 65 ADV.
%The Top-30 LLM-preferred words of GPT4o in the meta-review and abstract are shown in Table~\ref{tab:4m}~\ref{tab:4a}.

\begin{table*}[t]
    \centering
    \resizebox{\textwidth}{!}{
        \begin{tabular}{@{}c|lrlrlr@{}}
            \toprule
            \textbf{POS} & \textbf{Word} & \textbf{WUIR} & \textbf{Word} & \textbf{WUIR} & \textbf{Word} & \textbf{WUIR}\\
            \midrule
            \multirow{10}{*}{\textbf{NOUN}} 
            & abstract & 40.00 & advancements & 28.50 & realm & 24.00 \\
            & alterations & 20.00 & aligns & 19.00 & methodologies & 18.80 \\
            & clarity & 17.00 & adaptability & 13.00 & surpasses & 13.00 \\
            & examination & 10.00 & competitiveness & 9.00 & aids & 9.00 \\
            & reliance & 8.75 & necessitating & 8.00 & assurances & 8.00 \\
            & necessitates & 8.00 & assertions & 8.00 & threats & 8.00 \\
            & assessments & 7.33 & advancement & 7.25 & enhancements & 7.25 \\
            & demands & 7.00 & findings & 6.91 & standpoint & 6.00 \\
            & oversight & 6.00 & study & 5.42 & exhibit & 5.33 \\
            & enhancement & 5.00 & adjustments & 5.00 & capitalizes & 5.00 \\
            \midrule
            \multirow{10}{*}{\textbf{VERB}} 
            & enhancing & 73.00 & necessitates & 58.00 & necessitating & 50.00 \\
            & featuring & 47.00 & revised & 46.00 & influenced & 40.00 \\
            & encompassing & 31.00 & enhances & 30.10 & showcasing & 29.00 \\
            & surpasses & 19.50 & underscoring & 19.00 & facilitating & 18.67 \\
            & necessitate & 18.00 & managing & 18.00 & concerning & 15.33 \\
            & garnered & 14.50 & employing & 14.06 & surpassing & 14.00 \\
            & adhere & 14.00 & neglecting & 14.00 & comprehend & 14.00 \\
            & underscore & 13.00 & discern & 13.00 & examines & 12.00 \\
            & accommodates & 11.00 & detail & 11.00 & utilizing & 10.87 \\
            & enhance & 10.82 & begins & 10.50 & integrating & 10.21 \\
            \midrule
            \multirow{10}{*}{\textbf{ADJ}} 
            & innovative & 34.17 & exceptional & 24.00 & pertinent & 22.00 \\
            & intricate & 16.33 & pivotal & 16.00 & necessitate & 12.00 \\
            & distinctive & 11.00 & enhanced & 10.80 & akin & 10.40 \\
            & potent & 10.00 & adaptable & 9.67 & unfamiliar & 9.00 \\
            & straightforward & 8.42 & accessible & 8.00 & versatile & 7.13 \\
            & adept & 7.00 & devoid & 7.00 & advantageous & 6.80 \\
            & extended & 6.67 & prevalent & 6.00 & underexplored & 6.00 \\
            & commendable & 6.00 & contingent & 6.00 & foundational & 5.75 \\
            & comprehensive & 5.09 & strategic & 5.00 & renowned & 5.00 \\
            & attributable & 5.00 & unidentified & 5.00 & numerous & 4.82 \\
            \midrule
            \multirow{10}{*}{\textbf{ADV}} 
            & inadequately & 17.00 & predominantly & 16.67 & meticulously & 16.00 \\
            & strategically & 14.00 & notably & 12.60 & abstract & 12.00 \\
            & swiftly & 12.00 & additionally & 9.08 & adeptly & 8.00 \\
            & thereby & 7.52 & conversely & 7.40 & traditionally & 7.33 \\
            & initially & 7.10 & innovatively & 7.00 & subsequently & 6.06 \\
            & unexpectedly & 6.00 & excessively & 5.00 & historically & 5.00 \\
            & seamlessly & 4.50 & nonetheless & 4.25 & primarily & 4.00 \\
            & markedly & 4.00 & short & 4.00 & infrequently & 4.00 \\
            & effectively & 3.97 & solely & 3.80 & consequently & 3.78 \\
            & inherently & 3.40 & concurrently & 3.33 & particularly & 3.09 \\


            \bottomrule
        \end{tabular}
    }
    \caption{Top-30 LLM-preferred Words in \textbf{GPT-4o}-generated vs. human-written \textbf{abstracts}, with long words making up 40.48\%, and complex-syllabled words 73.45\%.}
    \label{tab:4a}
\end{table*}
\begin{table*}[t]
    \centering
    \resizebox{\textwidth}{!}{
        \begin{tabular}{@{}c|lrlrlr@{}}
            \toprule
            \textbf{POS} & \textbf{Word} & \textbf{WUIR} & \textbf{Word} & \textbf{WUIR} & \textbf{Word} & \textbf{WUIR}\\
            \midrule
            \multirow{10}{*}{\textbf{NOUN}} 
            & refinement & 284.00 & advancements & 88.00 & methodologies & 58.50 \\
            & articulation & 50.00 & highlights & 41.00 & reliance & 39.50 \\
            & enhancement & 39.00 & underpinnings & 37.00 & enhancements & 28.33 \\
            & transparency & 26.00 & complexities & 25.00 & skepticism & 24.67 \\
            & adaptability & 24.00 & narrative & 24.00 & integration & 23.43 \\
            & persist & 22.00 & acknowledgment & 22.00 & differentiation & 21.67 \\
            & advancement & 21.33 & contextualization & 21.00 & foundation & 20.50 \\
            & inconsistencies & 20.00 & reception & 20.00 & demands & 20.00 \\
            & backing & 19.50 & sections & 18.71 & refinements & 17.50 \\
            & credibility & 16.50 & benchmarking & 16.00 & reliability & 15.50 \\
            \midrule
            \multirow{10}{*}{\textbf{VERB}} 
            & enhance & 239.00 & enhancing & 120.50 & deemed & 79.33 \\
            & showcasing & 66.00 & express & 52.17 & offering & 50.50 \\
            & enhances & 45.00 & recognizing & 41.00 & commend & 37.25 \\
            & praised & 37.20 & integrating & 36.25 & criticized & 32.00 \\
            & hindering & 32.00 & utilizes & 31.00 & highlights & 28.00 \\
            & surpass & 27.00 & bolster & 26.00 & emphasizing & 25.33 \\
            & substantiate & 25.00 & integrates & 24.50 & solidify & 23.00 \\
            & questioning & 22.67 & arise & 21.17 & expanding & 20.67 \\
            & faces & 20.67 & weakens & 20.00 & recognized & 19.71 \\
            & criticize & 19.00 & illustrating & 19.00 & critique & 19.00 \\
            \midrule
            \multirow{10}{*}{\textbf{ADJ}} 
            & innovative & 114.50 & collective & 99.00 & enhanced & 45.00 \\
            & established & 30.00 & notable & 28.00 & outdated & 22.00 \\
            & varied & 16.00 & undefined & 15.00 & comparative & 14.70 \\
            & noteworthy & 14.50 & broader & 14.10 & comprehensive & 13.67 \\
            & clearer & 13.61 & intriguing & 13.50 & foundational & 13.00 \\
            & organizational & 13.00 & typographical & 12.50 & contextual & 12.00 \\
            & traditional & 11.83 & advanced & 11.00 & inadequate & 10.89 \\
            & diverse & 9.84 & insightful & 9.56 & prevalent & 9.50 \\
            & spatiotemporal & 9.00 & engaging & 9.00 & adaptable & 9.00 \\
            & illustrative & 8.50 & robotic & 8.50 & commendable & 8.33 \\
            \midrule
            \multirow{10}{*}{\textbf{ADV}} 
            & collectively & 223.00 & inadequately & 38.00 & reportedly & 18.00 \\
            & comprehensively & 13.00 & robustly & 12.00 & occasionally & 10.00 \\
            & predominantly & 9.00 & notably & 8.38 & innovatively & 8.00 \\
            & effectively & 7.81 & insufficiently & 6.71 & additionally & 6.59 \\
            & particularly & 5.09 & creatively & 5.00 & distinctly & 5.00 \\
            & positively & 4.71 & overall & 4.30 & primarily & 3.32 \\
            & convincingly & 3.07 & elegantly & 3.00 & marginally & 2.93 \\
            & selectively & 2.50 & conclusively & 2.33 & especially & 2.26 \\
            & favorably & 2.20 & universally & 2.00 & theoretically & 1.71 \\
            & overly & 1.71 & consistently & 1.65 & potentially & 1.59 \\
            \bottomrule
        \end{tabular}
    }
    \caption{Top-30 LLM-preferred Words in \textbf{GPT-4o}-generated vs. human-written \textbf{meta-reviews}, with long words making up 39.96\% and complex-syllabled words 67.82\%.}
    \label{tab:4m}
\end{table*}

Tables~\ref{tab:4a} and~\ref{tab:4m} display the top-30 preferred words across four key part-of-speech (POS) categories in GPT-4o-generated abstracts and meta-reviews, with long words accounting for 40.48\% and 39.96\%, and complex-syllabled words for 73.45\% and 67.82\%. Furthermore, Tables~\ref{tab:ga} and~\ref{tab:gm} show the top-30 preferred words in four key POS categories for Gemini-generated abstracts and meta-reviews, while Tables~\ref{tab:ca} and~\ref{tab:cm} display the same for Claude-generated abstracts and meta-reviews. All show a high proportion of long words and complex-syllabled words.



\begin{table*}[t]
    \centering
    \resizebox{\textwidth}{!}{
        \begin{tabular}{@{}c|lrlrlr@{}}
            \toprule
            \textbf{POS} & \textbf{Word} & \textbf{WUIR} & \textbf{Word} & \textbf{WUIR} & \textbf{Word} & \textbf{WUIR}\\
            \midrule
            \multirow{10}{*}{\textbf{NOUN}} 
            & abstract & 40.00 & advancements & 28.50 & realm & 24.00 \\
            & alterations & 20.00 & aligns & 19.00 & methodologies & 18.80 \\
            & clarity & 17.00 & adaptability & 13.00 & surpasses & 13.00 \\
            & examination & 10.00 & competitiveness & 9.00 & aids & 9.00 \\
            & reliance & 8.75 & necessitating & 8.00 & assurances & 8.00 \\
            & necessitates & 8.00 & assertions & 8.00 & threats & 8.00 \\
            & assessments & 7.33 & advancement & 7.25 & enhancements & 7.25 \\
            & demands & 7.00 & findings & 6.91 & standpoint & 6.00 \\
            & oversight & 6.00 & study & 5.42 & exhibit & 5.33 \\
            & enhancement & 5.00 & adjustments & 5.00 & capitalizes & 5.00 \\
            \midrule
            \multirow{10}{*}{\textbf{VERB}} 
            & enhancing & 73.00 & necessitates & 58.00 & necessitating & 50.00 \\
            & featuring & 47.00 & revised & 46.00 & influenced & 40.00 \\
            & encompassing & 31.00 & enhances & 30.10 & showcasing & 29.00 \\
            & surpasses & 19.50 & underscoring & 19.00 & facilitating & 18.67 \\
            & necessitate & 18.00 & managing & 18.00 & concerning & 15.33 \\
            & garnered & 14.50 & employing & 14.06 & surpassing & 14.00 \\
            & adhere & 14.00 & neglecting & 14.00 & comprehend & 14.00 \\
            & underscore & 13.00 & discern & 13.00 & examines & 12.00 \\
            & accommodates & 11.00 & detail & 11.00 & utilizing & 10.87 \\
            & enhance & 10.82 & begins & 10.50 & integrating & 10.21 \\
            \midrule
            \multirow{10}{*}{\textbf{ADJ}} 
            & innovative & 34.17 & exceptional & 24.00 & pertinent & 22.00 \\
            & intricate & 16.33 & pivotal & 16.00 & necessitate & 12.00 \\
            & distinctive & 11.00 & enhanced & 10.80 & akin & 10.40 \\
            & potent & 10.00 & adaptable & 9.67 & unfamiliar & 9.00 \\
            & straightforward & 8.42 & accessible & 8.00 & versatile & 7.13 \\
            & adept & 7.00 & devoid & 7.00 & advantageous & 6.80 \\
            & extended & 6.67 & prevalent & 6.00 & underexplored & 6.00 \\
            & commendable & 6.00 & contingent & 6.00 & foundational & 5.75 \\
            & comprehensive & 5.09 & strategic & 5.00 & renowned & 5.00 \\
            & attributable & 5.00 & unidentified & 5.00 & numerous & 4.82 \\
            \midrule
            \multirow{10}{*}{\textbf{ADV}} 
            & inadequately & 17.00 & predominantly & 16.67 & meticulously & 16.00 \\
            & strategically & 14.00 & notably & 12.60 & abstract & 12.00 \\
            & swiftly & 12.00 & additionally & 9.08 & adeptly & 8.00 \\
            & thereby & 7.52 & conversely & 7.40 & traditionally & 7.33 \\
            & initially & 7.10 & innovatively & 7.00 & subsequently & 6.06 \\
            & unexpectedly & 6.00 & excessively & 5.00 & historically & 5.00 \\
            & seamlessly & 4.50 & nonetheless & 4.25 & primarily & 4.00 \\
            & markedly & 4.00 & short & 4.00 & infrequently & 4.00 \\
            & effectively & 3.97 & solely & 3.80 & consequently & 3.78 \\
            & inherently & 3.40 & concurrently & 3.33 & particularly & 3.09 \\
            \bottomrule
        \end{tabular}
    }
    \caption{Top-30 LLM-preferred Words in \textbf{Gemini}-generated vs. human-written \textbf{abstracts}, with long words making up 44.35\%, and complex-syllabled words 74.80\%}
    \label{tab:ga}
\end{table*}
\begin{table*}[t]
    \centering
    \resizebox{\textwidth}{!}{
        \begin{tabular}{@{}c|lrlrlr@{}}
            \toprule
            \textbf{POS} & \textbf{Word} & \textbf{WUIR} & \textbf{Word} & \textbf{WUIR} & \textbf{Word} & \textbf{WUIR}\\
            \midrule
            \multirow{10}{*}{\textbf{NOUN}} 
            & refinement & 59.00 & leans & 55.00 & reliance & 45.50 \\
            & generalizability & 37.00 & handling & 35.00 & advancements & 28.00 \\
            & achieves & 24.00 & explores & 23.00 & inconsistencies & 20.33 \\
            & underpinnings & 17.00 & core & 16.38 & clarification & 16.29 \\
            & hinder & 16.00 & contingent & 16.00 & implications & 14.42 \\
            & articulation & 14.00 & calculations & 14.00 & typos & 13.86 \\
            & quantification & 13.00 & testing & 12.71 & availability & 12.50 \\
            & efficacy & 12.33 & referencing & 12.00 & mitigation & 12.00 \\
            & contextualization & 12.00 & duration & 12.00 & investigation & 11.97 \\
            & practicality & 11.43 & grounding & 11.25 & overfitting & 11.00 \\
            \midrule
            \multirow{10}{*}{\textbf{VERB}} 
            & deemed & 153.67 & solidify & 94.00 & hindering & 66.00 \\
            & criticized & 63.67 & praised & 52.20 & weakens & 42.00 \\
            & exceeding & 41.00 & praising & 40.33 & questioned & 37.68 \\
            & leans & 32.57 & recognizing & 32.00 & drew & 31.00 \\
            & leaned & 29.50 & enhance & 28.00 & offering & 27.25 \\
            & mitigating & 27.00 & weakened & 26.50 & utilizes & 25.00 \\
            & showcasing & 23.00 & hinders & 23.00 & arose & 21.67 \\
            & expanding & 20.67 & recurring & 19.00 & challenged & 17.00 \\
            & leverages & 16.75 & desired & 15.46 & raising & 15.33 \\
            & promoting & 15.00 & termed & 15.00 & differing & 15.00 \\
            \midrule
            \multirow{10}{*}{\textbf{ADJ}} 
            & core & 65.25 & established & 29.00 & undefined & 17.00 \\
            & nuanced & 16.00 & cautious & 16.00 & presentational & 15.00 \\
            & absent & 13.00 & combined & 12.00 & innovative & 11.62 \\
            & outdated & 11.00 & simplified & 9.25 & robotic & 9.00 \\
            & benchmark & 8.56 & inconsistent & 8.50 & illustrative & 8.50 \\
            & diverse & 8.26 & adaptable & 8.00 & dataset & 7.44 \\
            & insightful & 7.33 & grammatical & 7.29 & observed & 7.20 \\
            & comprehensive & 7.03 & spatiotemporal & 7.00 & repetitive & 7.00 \\
            & certain & 6.98 & rigorous & 6.86 & deeper & 6.64 \\
            & superior & 6.55 & compact & 6.50 & unconvincing & 6.30 \\
            \midrule
            \multirow{10}{*}{\textbf{ADV}} 
            & definitively & 9.00 & solely & 8.43 & reportedly & 8.00 \\
            & particularly & 7.35 & purportedly & 7.00 & primarily & 6.89 \\
            & positively & 5.43 & generally & 5.02 & straightforward & 5.00 \\
            & adaptively & 5.00 & specifically & 4.96 & favorably & 4.80 \\
            & demonstrably & 4.00 & locally & 3.50 & consistently & 3.41 \\
            & incrementally & 3.25 & potentially & 3.21 & visually & 3.00 \\
            & furthermore & 2.93 & theoretically & 2.58 & overhead & 2.50 \\
            & effectively & 2.35 & publicly & 2.29 & additionally & 2.15 \\
            & fine & 1.88 & especially & 1.76 & computationally & 1.71 \\
            & finally & 1.62 & insufficiently & 1.50 & overly & 1.47 \\
            \bottomrule
        \end{tabular}
    }
    \caption{Top-30 LLM-preferred Words in \textbf{Gemini}-generated vs. human-written \textbf{meta-reviews}, with long words making up 34.58\%, and complex-syllabled words 62.07\%}
    \label{tab:gm}
\end{table*}

\begin{table*}[t]
    \centering
    \resizebox{\textwidth}{!}{
        \begin{tabular}{@{}c|lrlrlr@{}}
            \toprule
            \textbf{POS} & \textbf{Word} & \textbf{WUIR} & \textbf{Word} & \textbf{WUIR} & \textbf{Word} & \textbf{WUIR}\\
            \midrule
            \multirow{10}{*}{\textbf{NOUN}} 
            & abstract & 40.00 & advancements & 28.50 & realm & 24.00 \\
            & alterations & 20.00 & aligns & 19.00 & methodologies & 18.80 \\
            & clarity & 17.00 & adaptability & 13.00 & surpasses & 13.00 \\
            & examination & 10.00 & competitiveness & 9.00 & aids & 9.00 \\
            & reliance & 8.75 & necessitating & 8.00 & assurances & 8.00 \\
            & necessitates & 8.00 & assertions & 8.00 & threats & 8.00 \\
            & assessments & 7.33 & advancement & 7.25 & enhancements & 7.25 \\
            & demands & 7.00 & findings & 6.91 & standpoint & 6.00 \\
            & oversight & 6.00 & study & 5.42 & exhibit & 5.33 \\
            & enhancement & 5.00 & adjustments & 5.00 & capitalizes & 5.00 \\
            \midrule
            \multirow{10}{*}{\textbf{VERB}} 
            & enhancing & 73.00 & necessitates & 58.00 & necessitating & 50.00 \\
            & featuring & 47.00 & revised & 46.00 & influenced & 40.00 \\
            & encompassing & 31.00 & enhances & 30.10 & showcasing & 29.00 \\
            & surpasses & 19.50 & underscoring & 19.00 & facilitating & 18.67 \\
            & necessitate & 18.00 & managing & 18.00 & concerning & 15.33 \\
            & garnered & 14.50 & employing & 14.06 & surpassing & 14.00 \\
            & adhere & 14.00 & neglecting & 14.00 & comprehend & 14.00 \\
            & underscore & 13.00 & discern & 13.00 & examines & 12.00 \\
            & accommodates & 11.00 & detail & 11.00 & utilizing & 10.87 \\
            & enhance & 10.82 & begins & 10.50 & integrating & 10.21 \\
            \midrule
            \multirow{10}{*}{\textbf{ADJ}} 
            & innovative & 34.17 & exceptional & 24.00 & pertinent & 22.00 \\
            & intricate & 16.33 & pivotal & 16.00 & necessitate & 12.00 \\
            & distinctive & 11.00 & enhanced & 10.80 & akin & 10.40 \\
            & potent & 10.00 & adaptable & 9.67 & unfamiliar & 9.00 \\
            & straightforward & 8.42 & accessible & 8.00 & versatile & 7.13 \\
            & adept & 7.00 & devoid & 7.00 & advantageous & 6.80 \\
            & extended & 6.67 & prevalent & 6.00 & underexplored & 6.00 \\
            & commendable & 6.00 & contingent & 6.00 & foundational & 5.75 \\
            & comprehensive & 5.09 & strategic & 5.00 & renowned & 5.00 \\
            & attributable & 5.00 & unidentified & 5.00 & numerous & 4.82 \\
            \midrule
            \multirow{10}{*}{\textbf{ADV}} 
            & inadequately & 17.00 & predominantly & 16.67 & meticulously & 16.00 \\
            & strategically & 14.00 & notably & 12.60 & abstract & 12.00 \\
            & swiftly & 12.00 & additionally & 9.08 & adeptly & 8.00 \\
            & thereby & 7.52 & conversely & 7.40 & traditionally & 7.33 \\
            & initially & 7.10 & innovatively & 7.00 & subsequently & 6.06 \\
            & unexpectedly & 6.00 & excessively & 5.00 & historically & 5.00 \\
            & seamlessly & 4.50 & nonetheless & 4.25 & primarily & 4.00 \\
            & markedly & 4.00 & short & 4.00 & infrequently & 4.00 \\
            & effectively & 3.97 & solely & 3.80 & consequently & 3.78 \\
            & inherently & 3.40 & concurrently & 3.33 & particularly & 3.09 \\
            \bottomrule
        \end{tabular}
    }
    \caption{Top-30 LLM-preferred Words in \textbf{Claude}-generated vs. human-written \textbf{abstracts}, with long words making up 42.81\%, and complex-syllabled words 74.80\%}
    \label{tab:ca}
\end{table*}

\begin{table*}[t]
    \centering
    \resizebox{\textwidth}{!}{
        \begin{tabular}{@{}c|lrlrlr@{}}
            \toprule
            \textbf{POS} & \textbf{Word} & \textbf{WUIR} & \textbf{Word} & \textbf{WUIR} & \textbf{Word} & \textbf{WUIR}\\
            \midrule
            \multirow{10}{*}{\textbf{NOUN}} 
            & refinement & 604.00 & center & 151.00 & demonstrates & 108.00 \\
            & foundations & 99.00 & articulation & 70.00 & relies & 62.00 \\
            & generalizability & 60.57 & tier & 38.00 & reservations & 37.43 \\
            & sentiments & 37.00 & critiques & 34.00 & explores & 31.00 \\
            & vulnerabilities & 29.00 & transparency & 27.00 & achieves & 27.00 \\
            & promise & 25.67 & advancement & 25.67 & substantiation & 25.00 \\
            & introduces & 24.00 & benchmarking & 23.00 & leans & 23.00 \\
            & underpinnings & 22.00 & capabilities & 21.00 & skepticism & 21.00 \\
            & shows & 20.00 & challenges & 19.53 & ambiguities & 19.00 \\
            & narrative & 19.00 & highlights & 19.00 & differentiation & 18.67 \\
            \midrule
            \multirow{10}{*}{\textbf{VERB}} 
            & synthesizes & 984.00 & recognizing & 153.00 & view & 131.33 \\
            & critique & 92.00 & express & 91.83 & revealing & 81.33 \\
            & appreciating & 73.67 & praising & 56.67 & expanding & 52.67 \\
            & highlighting & 47.53 & offering & 42.50 & acknowledging & 39.92 \\
            & center & 37.50 & mitigating & 36.00 & substantiate & 32.67 \\
            & persist & 27.00 & enhance & 26.50 & deemed & 26.00 \\
            & conducting & 24.00 & reveals & 24.00 & graph & 23.00 \\
            & handling & 20.67 & contexts & 20.00 & emerge & 19.75 \\
            & recognize & 19.43  & generative & 17.50 
            & represents & 17.10\\ & praised & 16.80 & offers & 16.66 & bridging&16.00 \\
            \midrule
            \multirow{10}{*}{\textbf{ADJ}} 
            & collective & 1049.00 & nuanced & 292.00 & innovative & 109.50 \\
            & cautious & 46.00 & comprehensive & 40.22 & comparative & 36.30 \\
            & revolutionary & 33.00 & definitive & 31.00 & meta & 27.84 \\
            & methodological & 22.28 & rigorous & 22.24 & transformative & 21.00 \\
            & noteworthy & 18.50 & substantive & 18.00 & presentational & 18.00 \\
            & notable & 17.75 & clearer & 17.06 & scholarly & 17.00 \\
            & academic & 16.00 & undefined & 15.00 
            & intriguing & 14.00\\ & addresses & 14.00 & robotic & 14.00 
            & core & 13.38 \\& diverse & 12.11 & adaptable & 12.00
            & primary & 11.26\\ & scientific & 10.69 & deeper & 10.64 &broader&10.17\\
            \midrule
            \multirow{10}{*}{\textbf{ADV}} 
            & collectively & 1112.00 & definitively & 54.00 & comprehensively & 37.00 \\
            & positively & 25.14 & conclusively & 16.00 & critically & 16.00 \\
            & consistently & 15.24 & scientifically & 12.00 & marginally & 8.57 \\
            & predominantly & 7.00 & cautiously & 6.50 & unanimously & 5.94 \\
            & robustly & 5.00 & adaptively & 5.00 & particularly & 4.59 \\
            & incrementally & 4.50 & meaningfully & 4.33 & generally & 4.11 \\
            & fully & 3.95 & primarily & 3.61 & overhead & 3.50 \\
            & short & 3.38 & potentially & 3.37 & genuinely & 3.33 \\
            & dynamically & 2.78 & fundamentally & 2.71 & technically & 2.54 \\
            & methodologically & 2.50 & semantically & 2.33 & dramatically & 2.33 \\
            \bottomrule
        \end{tabular}
    }
    \caption{Top-30 LLM-preferred Words in \textbf{Claude}-generated vs. human-written \textbf{meta-reviews}, with long words making up 43.93\%, and complex-syllabled words 70.49\%}
    \label{tab:cm}
\end{table*}




% The Top-30 LLM-preferred words of GPT4o, Claude and Gemini are presented in the Table~\ref{tab:4m}, Table~\ref{tab:4a}, Table~\ref{tab:gm}, Table~\ref{tab:ga}, Table~\ref{tab:cm} and Table~\ref{tab:ca}.

% Furthermore, the Statistical results of \textit{long words} and \textit{complex-syllabled words} are presented in Table~\ref{tab:word-level-long} and Table~\ref{tab:word-level-complex}. Take GPT4o for example, in meta-review, \textit{long words} account for 39.96\% and \textit{complex-syllabled words} account for 67.82\%. In abstract, the proportions are 40.48\% and 73.45\% respectively. The results reflect LLMs' preference for \textit{long words} and \textit{complex-syllabled words}.


% Moreover, we use the preferred word set to validate the correctness of our Fine-Grained Detection Model \S\ref{app:Fine-Grained} on ICLR 2024 meta-reviews. The model categorizes the meta-reviews into three groups: human-written, LLM-synthesized, and LLM-refined. To simplify the analysis, the latter two groups were combined into an LLM-mixed group, which represents all content involving LLMs. We calculate the percentage of words in each meta-review that appeared in our preferred word set. After that, we separately compute the average of these percentages for meta-reviews in Human group and LLM-mixed group. The ratios for Human and LLMs-mixed group are 35.15\% and 44.95\% respectively. The significantly higher matching ratio of LLM-mixed group further validates the correctness of our model.

% \begin{table}[]
%     \centering
%     \begin{tabular}{l|l|r}
%         \hline
%         {Category}&role & ratio\\
%         \hline
%         \multirow{3}{*}{Meta-review} 
%             & GPT4o &39.96\\
%             & Gemini &34.58\\
%             & Claude &43.93 \\
%         \hline
%         \multirow{3}{*}{Abstract} 
%             & GPT4o &40.48\\
%             & Gemini &44.35\\
%             & Claude &42.81\\
%         \hline
%         \multirow{1}{*}{ICLR 2024} 
%             & LLMs &36.92\\
%         \hline
%     \end{tabular}
%     \caption{long words percentage for LLM preferred words}
%     \label{tab:word-level-long}
% \end{table}

% \begin{table}[]
%     \centering
%     \begin{tabular}{l|l|r}
%         \hline
%         {Category}&role &ratio\\
%         \hline
%         \multirow{3}{*}{Meta-review} 
%             & GPT4o &67.82\\
%             & Gemini &62.07\\
%             & Claude &70.49 \\
%         \hline
%         \multirow{3}{*}{Abstract} 
%             & GPT4o &73.45\\
%             & Gemini &74.80\\
%             & Claude &75.00\\
%         \hline
%         \multirow{1}{*}{ICLR 2024} 
%             & LLMs &64.18\\
%         \hline
%     \end{tabular}
%     \caption{complex-syllabled words percentage for LLM preferred words}
%     \label{tab:word-level-complex}
% \end{table}








\subsection{Pattern-Level Feature Statistics}
We identify the following pattern-level features in human-written (meta-)reviews: \textit{personability}, characterized by frequent use of the first person to express opinions; \textit{interactivity}, marked by the inclusion of questions; and \textit{attention to detail}, demonstrated by citing relevant literature to support arguments. 
To compare these pattern-level features between human-written and LLM-generated content, we calculate two metrics for each pattern in both meta-reviews and reviews within the ScholarLens dataset: \textbf{Feature Proportion (FP)} and \textbf{Feature Intensity (FI)}. FP is defined as the proportion of instances exhibiting the feature within the target data group, while FI is the average number of occurrences of the feature within instances that exhibit it. We report the FP and FI values for each pattern in meta-reviews and reviews across different data types—Human-written, GPT-4-generated, Gemini-generated, and Claude-generated— as shown in Table~\ref{tab:pattern_comparison}.
% (\textit{Personability} Pattern Comparison), Table~\ref{tab:Interactivity_c} (\textit{Interactivity} Pattern Comparison), and Table~\ref{tab:Attention_c} (\textit{Attention to Detail} Pattern Comparison).
The results show that, in both meta-reviews and reviews, the FP and FI values for each pattern feature in the Human-written data type are significantly higher than those in the LLM-generated versions.
% These suggest that humans have a higher breadth of coverage and intensity of usage for first person sentences, question sentences, and URL links.


% \lz{@Ruijie, description: how to filter? what's the result?} % 统计信息论证case study中提及的pattern features
% \subsubsection{Experimental Setup}




% To evaluate \textit{personability},\textit{interactivity} and \textit{attention to detail},we first define two general metrics: Coverage Rate ($CR_f$) and Usage Intensity ($UI_f$) for $f \in \{fp, q, l\}$. Here, $fp$ means first person sentence, $q$ means question sentence, and $l$ means URL link.

% $CR_f$ represents the proportion of texts in the entire sample that contain at least one instance of $f$, reflecting the breadth of coverage of $f$.
% $UI_f$ represents the average number of occurrences of $f$ within texts that already contain it, reflecting the intensity of usage of $f$.
% \begin{equation}
% CR_f = \frac{\sum_{i=1}^{N} \mathbb{I}(f \in x_i)}{N}
% \end{equation}
% \begin{equation}
% UI_f = \frac{\sum_{i=1}^{N} c_f(x_i)}{\sum_{i=1}^{N} \mathbb{I}(f \in x_i)}
% \end{equation}

% where \( \{x_i\} \) denotes a text set. \( N \) is the total number of samples. $\sum_{i=1}^{N} \mathbb{I}(f \in x_i)$ counts the number of texts in the set ${x_i}$ where $f$ appears. $\sum_{i=1}^{N} c_f(x_i)$ counts the total number of $f$.

% Based on the above general definitions, the specific metrics corresponding to \textit{interactivity}, \textit{personability}, and \textit{attention to details} are as follows: First Person Coverage Rate ($CR_{fp}$) and First Person Usage Intensity ($UI_{fp}$), Question Coverage Rate ($CR_{q}$) and Question Usage Intensity ($UI_{q}$), Link Coverage Rate ($CR_{l}$) and Link Usage Intensity ($UI_{l}$). 

%\paragraph{(i) Preprocessing Rule}
%Before filtering, we first implement two preprocessing rule on the texts:

%\textbf{Case Insensitivity}: Convert all words to lowercase.
%\textbf{Regular expression cleaning}: \verb|re.sub(r'[\r\n*\\]+', ' ', text)|

%\paragraph{(a) Personability Filtering}
%Using NLTK to tokenize a text into sentences and check if a sentence contains \"i\" or \"we\", we calculate First-Person Presence Ratio ($CR_{fp}$) and Average First-Person Quantity ($UI_{fp}$), defined as follows:

%\begin{equation}  
%CR_{fp} = \frac{\sum_{i=1}^{N} \mathbb{I}(fp \in x_i)}{N}  
%\end{equation}  
%where $ \mathbb{I}(fp \in x_i) $ is an indicator function that returns 1 if $ x_i $ contains at least one first-person sentence, and $ N $ is the total number of meta-reviews (or reviews).  

%\begin{equation}  
%UI_{fp} = \frac{\sum_{i=1}^{N} c_{fp}(x_i)}{\sum_{i=1}^{N} \mathbb{I}(fp \in x_i)}  
%\end{equation}  
%where $ c_{fp}(x_i) $ is the count of first-person sentences in $ x_i $.  



%\paragraph{(b) Interactivity Filtering}
%Using NLTK to tokenize a text into sentences and check if a sentence end with a question mark, we calculate Question Presence Ratio ($CR_{q}$) and Average Question Quantity ($UI_{q}$), defined as follows: 

%\begin{equation}  
%CR_q = \frac{\sum_{i=1}^{N} \mathbb{I}(q \in x_i)}{N}  
%\end{equation}  
%where $ \mathbb{I}(q \in x_i) $ is an indicator function that returns 1 if $ x_i $ contains at least one question.  

%\begin{equation}  
%UI_q = \frac{\sum_{i=1}^{N} c_q(x_i)}{\sum_{i=1}^{N} \mathbb{I}(q \in x_i)}  
%\end{equation}  
%where $ c_q(x_i) $ is the count of questions in $ x_i $.  

%\paragraph{(c) Attention to Detail Filtering}
%Matching the links in texts, we calculate Link Presence Ratio ($CR_{l}$) and Average Link Quantity ($UI_{l}$), defined as follows: 
%\begin{equation}  
%CR_l = \frac{\sum_{i = 1}^{N} \mathbb{I}(l \in x_i)}{N}  
%\end{equation}  
%where $ \mathbb{I}(l \in x_i) $ is an indicator function that returns 1 if $ x_i $ contains at least one link.

%\begin{equation}  
%UI_l = \frac{\sum_{i = 1}^{N} c_l(x_i)}{\sum_{i = 1}^{N} \mathbb{I}(l \in x_i)}  
%\end{equation}  
%where $ c_l(x_i) $ is the count of links in $ x_i $.

% \subsubsection{Results}
% Table~\ref{tab:personability_c}, ~\ref{tab:Interactivity_c} and ~\ref{tab:Attention_c} separately display the results of $CR_{fp}$, $UI_{fp}$, $CR_{q}$, $UI_{q}$, $CR_{l}$ and $UI_{l}$ metrics on the comment-based data. In both meta-review and review, the values of these six metrics in humans are significantly higher than those in LLMs. These suggest that humans have a higher breadth of coverage and intensity of usage for first person sentences, question sentences, and URL links.
%These suggest a higher frequency of first person, question, and relevant literature in humans.

%First, we conducted an automated evaluation on the comment-based data. 


%\subparagraph{(a) Personability Filtering result}
%As shown in Table~\ref{tab:personability_c}, in the comment-based data, Human exhibits much higher $CR_{fp}$ and $UI_{fp}$ values than LLMs, indicating that Human uses the first person more frequently.

%As shown in Table~\ref{tab:personability_f}, in ICLR 2024, the values of $CR_{fp}$ and $UI_{fp}$ for LLM-synthesized are significantly lower than those for LLM-refined, LLM-mixed, and Human. Although the values for LLM-refined and the LLM-mixed are higher than those for LLM-synthesized, they are still lower than those for Human.

%\subparagraph{(b) Interactivity Filtering result}
%As shown in Table~\ref{tab:personability_c}, in the comment-based data, the $CR_{q}$ and $UI_{q}$ values for LLMs are all 0. Humans exhibit significantly higher scores for the two values than LLMs, indicating that humans interact more frequently(incorporate questions more often). 

%As shown in Table~\ref{tab:personability_f}, in ICLR 2024, LLM-synthesized's $CR_{q}$ and $UI_{q}$ are still 0, significantly lower than those for LLM-refined, LLM-mixed, and Human. The $CR_{q}$ and $UI_{q}$ values for LLM-refined and LLM-mixed are lower than those for Human.

%\subparagraph{(c) Attention to Detail Filtering result}
%As shown in Table~\ref{tab:Attention_c}, in the comment-based data, the $CR_{l}$ and $UI_{l}$ values for LLMs are all 0. Humans exhibit significantly higher scores for the two values than LLMs, indicating that humans pay more attention to details(cite more relevant literature). 

%As shown in Table~\ref{tab:Attention_f}, in ICLR 2024, the $CR_{l}$ and $UI_{l}$ values for LLMs are still 0. The two values for LLM-refined and LLM-mixed are lower than those for Human.




% Under a given significance level $\alpha$ (we set $\alpha = 0.0001$), we calculate the critical value $t_{critical}$ of the $t$-distribution:
% \begin{equation}
% t_{critical}=t_{1 - \alpha, df}
% \end{equation}

% If the calculated $t$-score is greater than the critical value $t_{critical}$, i.e., $t > t_{critical}$, we reject the null hypothesis and conclude that the word occurs significantly more often in LLM-generated texts than in human-written texts. In this way, we can identify the words favored by LLMs, which appear significantly more frequently in LLM-generated texts.


% \paragraph{Detailed Calculation Process of $t$}
% \subparagraph{Sample Proportion Calculation}\label{subsubsec:hypothesis_calculation_judgment}
% For the $i$-th type of text ($i = 1$ for human-written texts and $i = 2$ for LLM-generated texts), the sample proportion $\hat{p}_i$ of the word is calculated as:
% \begin{equation}
% \hat{p}_i=\frac{x_i+\alphUI_{smooth}}{n_i+\alphUI_{smooth}\cdot|V|}
% \end{equation}
% where $x_i$ is the frequency of the word's occurrence in the $i$-th type of text, $n_i$ is the total number of words in the $i$-th type of text, $\alphUI_{smooth}$ is the smoothing parameter used to avoid zero counts, and $|V|$ is the number of different words that appear in the entire text data.


% \subsubsection{Part-of-Speech(POS) Filtering}
% To focus on semantically meaningful word categories, 
% we use \textbf{NLTK}'s \textbf{word\_tokenize} to tokenize the text and \textbf{pos\_tag} to assign fine-grained part-of-speech(POS) tags. Then, we map these fine-grained tags to coarse-grained ones, retaining only nouns, verbs, adjectives and adverbs. The mapping rules are provided in table~\ref{tab:POS}.

% We conducted the hypothesis-testing separately for each of the POS categories in the meta-reviews generated and abstracts polished by three models: GPT4o, Claude, and Gemini. In total, we processed 4×2×3 groups of data, through which we identified the preferences of different LLMs for words of various parts of speech.

% %To be more specific, take the adjective "innovative" in the meta-reviews generated by GPT4o as an example. In human-written meta-reviews, this word only appeared 7 times. However, in the texts generated by GPT4o (llm-generated meta-reviews), it appeared 973 times. The $t$-score calculated for this word was 30.887, which is much higher than the critical value of 3.719 when $\alpha = 0.0001$. Thus, we reject the null hypothesis, indicating that GPT4o shows a clear preference for the word "innovative" during the process of generating meta-reviews.




% \subsubsection{word preferences}
% The \textbf{t-score} represents the statistical significance. \textbf{h\_cnt} indicates the number of occurrences of the word in human-written meta-review, and \textbf{llm\_cnt} represents the number of occurrences in llm-generated meta-review. ($\alpha$) is set to 0.0001, and the corresponding $t_{critical}$ is 3.7192040400651254.

\subsection{Validation of Detection Model Reliability}
We use our filtered full LLM-preferred word set and the identified pattern-level features to validate the reliability of our detection models.
Specifically, we classify all meta-reviews from ICLR 2024 into two groups based on the fine-grained detection results in Appendix~\S\ref{app:Fine-Grained}: human-written and LLM-generated (including LLM-refined and LLM-synthesized prediction). For each meta-review, we calculate the proportion of words that belong to the full GPT-4-preferred word set, defined as the ratio of matching words to the total number of words in the set. The average ratios for the human-written and LLM-generated groups are 35.15\% and 44.95\%, respectively. 
We then compute the FR and FI values for each pattern feature in each group, with results shown in Table~\ref{tab:pattern_vali}. The FR and FI values for predicted LLM-generated text are lower than those for predicted human-written text.
% These results  demonstrate the reliability of our model's detection capabilities.
These results provide evidence of the reliability of our model's detection capabilities.
% Then we compute the FR and FI value of each pattern feature in each group, the results are shown in Table~\ref{tab:pattern_vali}, 预测为LLM-generated text呈现的FR和FI value 低于预测为human-written的text


% \clearpage


\begin{figure}[h]
    \centering
    \begin{minipage}{1\linewidth}
        \centering
        \scalebox{0.95}{
        \begin{tabular}{l|l|r|r}
        \toprule
        {\textbf{Data Type}}&\textbf{Resource }&\textbf{FR (\%)}&\textbf{FI}\\
        \midrule
        \multirow{5}{*}{Meta-review} 

            & Human & 32.00& 1.62\\
            & GPT4o & 0.07 & 1.00\\
            & Gemini & 0.11& 1.33\\
            & Claude & 0.07& 1.00\\
        \midrule
        \multirow{4}{*}{Review} 
            & Human & 76.32 & 2.07\\
            & GPT4o &0.00&0.00\\
            & Gemini &0.00&0.00\\
            & Claude & 0.00&0.00\\
        \bottomrule
    \end{tabular}}
        % \captionsetup{font=footnotesize}
        \subcaption{\textit{Personability}}
    \end{minipage}
    \hfill
    \begin{minipage}{1\linewidth}
        \centering
        \scalebox{0.95}{
        \begin{tabular}{l|l|r|r}
        \hline
        {\textbf{Data Type}}&\textbf{Resource }&\textbf{FR (\%)}&\textbf{FI}\\
        \toprule
        \multirow{5}{*}{Meta-review} 

            & Human& 2.01& 1.54 \\
            & GPT4o& 0.00&0.00 \\
            & Gemini& 0.00&0.00 \\
            & Claude& 0.00&0.00 \\
        \midrule
        \multirow{4}{*}{Review} 
            & Human& 17.11& 1.85 \\
            & GPT4o& 0.00&0.00 \\
            & Gemini& 0.00&0.00 \\
            & Claude& 0.00&0.00 \\
        \bottomrule
    \end{tabular}}
        % \captionsetup{font=footnotesize}
        \subcaption{\textit{Interactivity}}
    \end{minipage}
    \hfill
    \begin{minipage}{1\linewidth}
        \centering
        \scalebox{0.95}{
        \begin{tabular}{l|l|r|r}
        \toprule
        {\textbf{Data Type}}&\textbf{Resource }&\textbf{FR (\%)}&\textbf{FI}\\
        \midrule
        \multirow{5}{*}{Meta-review} 
            & Human & 1.48&1.60\\
            & GPT4o & 0.00&0.00\\
            & Gemini & 0.00&0.00\\
            & Claude & 0.00&0.00\\
%            & meta\_overall & 0/519\\
%            & meta\_overall\_except & 392/17003 \\
        \bottomrule
        \multirow{4}{*}{Review} 
            & Human & 7.89&3.17\\
            & GPT4o & 0.00&0.00\\
            & Gemini & 0.00&0.00\\
            & Claude & 0.00&0.00\\
        \hline
    \end{tabular}}
        % \captionsetup{font=footnotesize}
        \subcaption{\textit{Attention to Detail}}
    \end{minipage}
    \hfill

    \caption{Comparison of Pattern Features of Meta-Reviews and Reviews in \texttt{ScholarLens}.}
    \label{tab:pattern_comparison}
\end{figure}



\begin{figure}[ht]
    \centering
    \begin{minipage}{1\linewidth}
        \centering
        \begin{tabular}{l|l|r|r}
        \toprule
\textbf{Role} &\textbf{FR (\%)}&\textbf{FI}\\
        \midrule

LLM-synthesized&6.12&1.00\\
LLM-refined&34.32&1.53\\
LLM-generated&32.61&1.53\\
human-written &39.50&1.77\\
%            & meta\_meta & 10.77&1.18\\
%            & meta\_meta\_comp & 40.11&1.72 \\
        \bottomrule
    \end{tabular}
        % \captionsetup{font=footnotesize}
        \subcaption{\textit{Personability}}
    \end{minipage}
    \hfill
    \begin{minipage}{1\linewidth}
        \centering
        \begin{tabular}{l|r|r}
        \toprule
\textbf{Role} & \textbf{FR (\%)}&\textbf{FI}\\
        \midrule

LLM-synthesized &0.00&0.00\\
LLM-refined &2.02 &1.53\\
LLM-generated &1.89 &1.53\\
human-written &4.87 &1.88\\
%            & meta\_meta & 2/520&1.00\\
%            & meta\_meta\_comp & 229/5260&1.84 \\
        \bottomrule
    \end{tabular}
        % \captionsetup{font=footnotesize}
        \subcaption{\textit{Interactivity}}
    \end{minipage}
    \hfill
    \begin{minipage}{1\linewidth}
        \centering
        \begin{tabular}{l|r|r}
        \toprule
        \textbf{Role} & \textbf{FR (\%)}&\textbf{FI}\\
        \midrule
            LLM-synthesized &0.00&0.00\\
            LLM-refined &1.57&1.13\\
            LLM-generated &1.47&1.06\\
            human-written &3.77&2.25\\
%            & meta\_meta & 1/520\\
%            & meta\_meta\_comp & 109/5260 \\
        \bottomrule
    \end{tabular}
        % \captionsetup{font=footnotesize}
        \subcaption{\textit{Attention to Detail}}
    \end{minipage}
    \hfill

    \caption{Comparison of Pattern Features in Each Prediction Data Group for ICLR 2024 Meta-Reviews.}
    \label{tab:pattern_vali}
\end{figure}



% Moreover, We use six metrics mentioned earlier to validate the correctness of our Fine-Grained Detection Model \S\ref{app:Fine-Grained} on ICLR 2024 meta-reviews.
%Then, we further verified our results on the ICLR 2024 Meta-reviews (three-class role recognition). We take the union of the LLM-synthesized and LLM-refined as LLM-mixed.
% Table~\ref{tab:personability_f}, ~\ref{tab:Interactivity_f} and ~\ref{tab:Attention_f} separately display the results of the six metrics. The values of the six metrics all show that LLM-synthesized < LLM-refined $\approx$ LLM-mixed < human-written. These further validate the correctness of our model.

% \subsection{Sentence and paper level}

% \clearpage
% \begin{minipage}{0.48\textwidth}
% \begin{table}[H]
%     \centering
%     \begin{tabular}{l|l|r|r}
%         \toprule
%         {\textbf{Data Type}}&\textbf{Resource }&\textbf{FR (\%)}&\textbf{FI}\\
%         \midrule
%         \multirow{5}{*}{Meta-review} 

%             & Human & 32.00& 1.62\\
%             & GPT4o & 0.07 & 1.00\\
%             & Gemini & 0.11& 1.33\\
%             & Claude & 0.07& 1.00\\
%         \midrule
%         \multirow{4}{*}{Review} 
%             & Human & 76.32 & 2.07\\
%             & GPT4o &0.00&0.00\\
%             & Gemini &0.00&0.00\\
%             & Claude & 0.00&0.00\\
%         \bottomrule
%     \end{tabular}
%     \caption{\textit{Personability} Pattern Comparison.}
%     \label{tab:personability_c}
% \end{table}


% \begin{table}[H]
%     \centering
%     \begin{tabular}{l|l|r|r}
%         \hline
%         {\textbf{Data Type}}&\textbf{Resource }&\textbf{FR (\%)}&\textbf{FI}\\
%         \toprule
%         \multirow{5}{*}{Meta-review} 

%             & Human& 2.01& 1.54 \\
%             & GPT4o& 0.00&0.00 \\
%             & Gemini& 0.00&0.00 \\
%             & Claude& 0.00&0.00 \\
%         \midrule
%         \multirow{4}{*}{Review} 
%             & Human& 17.11& 1.85 \\
%             & GPT4o& 0.00&0.00 \\
%             & Gemini& 0.00&0.00 \\
%             & Claude& 0.00&0.00 \\
%         \bottomrule
%     \end{tabular}
%     \caption{\textit{Interactivity} Pattern Comparison.}
%     \label{tab:Interactivity_c}
% \end{table}

% \begin{table}[H]
%     \centering
%     \begin{tabular}{l|l|r|r}
%         \toprule
%         {\textbf{Data Type}}&\textbf{Resource }&\textbf{FR (\%)}&\textbf{FI}\\
%         \midrule
%         \multirow{5}{*}{Meta-review} 
%             & Human & 1.48&1.60\\
%             & GPT4o & 0.00&0.00\\
%             & Gemini & 0.00&0.00\\
%             & Claude & 0.00&0.00\\
% %            & meta\_overall & 0/519\\
% %            & meta\_overall\_except & 392/17003 \\
%         \bottomrule
%         \multirow{4}{*}{Review} 
%             & Human & 7.89&3.17\\
%             & GPT4o & 0.00&0.00\\
%             & Gemini & 0.00&0.00\\
%             & Claude & 0.00&0.00\\
%         \hline
%     \end{tabular}
%     \caption{\textit{Attention to Detail} Pattern Comparison.}
%     \label{tab:Attention_c}
% \end{table}
% \end{minipage}
% \hfill

% \begin{minipage}{0.48\textwidth}
% \begin{table}[H]
%     \centering
%     \begin{tabular}{l|l|r|r}
%         \hline
% \textbf{Role} &\textbf{FR (\%)}&\textbf{FI}\\
%         \hline

% LLM-synthesized&6.12&1.00\\
% LLM-refined&34.32&1.53\\
% LLM-generated&32.61&1.53\\
% human-written &39.50&1.77\\
% %            & meta\_meta & 10.77&1.18\\
% %            & meta\_meta\_comp & 40.11&1.72 \\
%         \hline
%     \end{tabular}
%     \caption{\textit{Personability} Pattern Comparison in ICLR 2024 Meta-reviews (detected by Fine-Grained Detection Model)}
%     \label{tab:personability_f}
% \end{table}

% \begin{table}[H]
%     \centering
%     \begin{tabular}{l|r|r}
%         \hline
% \textbf{Role} & \textbf{FR (\%)}&\textbf{FI}\\
%         \hline

% LLM-synthesized &0.00&0.00\\
% LLM-refined &2.02 &1.53\\
% LLM-generated &1.89 &1.53\\
% human-written &4.87 &1.88\\
% %            & meta\_meta & 2/520&1.00\\
% %            & meta\_meta\_comp & 229/5260&1.84 \\
%         \hline
%     \end{tabular}
%     \caption{\textit{Interactivity} Pattern Comparison in ICLR 2024 Meta-reviews (detected by Fine-Grained Detection Model)}    \label{tab:Interactivity_f}
% \end{table}


% \begin{table}[H]
%     \centering
%     \begin{tabular}{l|r|r}
%         \hline
%         \textbf{Role} & \textbf{FR (\%)}&\textbf{FI}\\
%         \hline
%             LLM-synthesized &0.00&0.00\\
%             LLM-refined &1.57&1.13\\
%             LLM-generated &1.47&1.06\\
%             human-written &3.77&2.25\\
% %            & meta\_meta & 1/520\\
% %            & meta\_meta\_comp & 109/5260 \\
%         \hline
%     \end{tabular}
%     \caption{\textit{Attention to Detail} Pattern Comparison in ICLR 2024 Meta-reviews (detected by Fine-Grained Detection Model)}
%     \label{tab:Attention_f}
% \end{table}
% \end{minipage}































\end{document}
