\begin{table*}[ht]
\centering
\footnotesize 
\begin{tabular}{@{}l|p{0.4\textwidth}|p{0.4\textwidth}@{}}
\toprule
\textbf{Structural Features}         & \textbf{Description} & \textbf{Computing Method} \\ \midrule
\textbf{Average Word Length}          & Measures the average number of characters in each word in the text. &  \\ 
\textbf{Average Sentence Length}       & Measures the average number of words in each sentence in the text. &  \\ 
\textbf{Average Number of Words} & Counts the average total words in the text. &  \\ 
\textbf{Average Number of Sentences}     & Counts the average total sentences in the text. &  \\ \midrule




\textbf{Lexical Features}          \\ \midrule
\textbf{Percent of Long Words}   & Measures the percentage of “long words” in the text. We define long words as words with more than 6 letters. & $\text{Percent of Long Words} = \left(\frac{\text{Long Words}}{\text{Total Words}}\right) \times 100\%$ \\ 
\textbf{Percent of Stopwords}   & Measures the percentage of common words in the text, like "the," "and," and "is." A high percentage means the text has more filler words. & $\text{Percent of Stopwords} = \left(\frac{\text{Stopwords}}{\text{Total Words}}\right) \times 100\%$ \\ 
\textbf{Vocabulary Richness}   & Measures how many unique words are used compared to the total number of words. "Unique words" are the different words that appear in a text, without counting repeats. 

For example, in “apple orange apple,” there are 3 words, but only 2 unique words: "apple" and "orange."

A higher ratio means the text has more varied vocabulary. & $\text{Vocabulary Richness} = \frac{\text{Unique Words}}{\text{Total Words}}$  \\ \midrule




\textbf{Linguistic Quality Features}         \\ \midrule
\textbf{Syntactic Diversity} & Syntactic diversity assesses structural complexity by examining clause patterns. &  \\ 
\textbf{Readability} & Readability indices measure how easily a reader can understand a written text. We use the following metrics to measure readability:

-Flesch-Kincaid Scores: Measures how easy the text is to read, based on sentence length and word syllables.

-Flesch Scores: Rates how easy or hard the text is to read for different age levels.

-Gunning Fog Scores: Shows the education level needed to understand the text.

-Coleman-Liau Scores: Uses word and sentence length to tell what grade level can read the text.

-Dale-Chall Scores: Rates readability based on the use of familiar or common words.

-ARI Scores: Estimates readability using the number of characters and words.

-Linsear Write Scores: Measures readability by looking at sentence structure and word difficulty.
&

We use the \text{textstat} library directly to calculate these seven metrics. The specific calculation formulas are as follows:

\text{Flesch-Kincaid Grade Level} = 

$0.39 \times \frac{\text{Total Words}}{\text{Total Sentences}} + 11.8 \times \frac{\text{Total Syllables}}{\text{Total Words}} - 15.59 $
\medskip

\text{Flesch Reading Ease} = 

$206.835 - 1.015 \times \frac{\text{Total Words}}{\text{Total Sentences}} - 84.6 \times \frac{\text{Total Syllables}}{\text{Total Words}}$ 
\medskip

\text{Gunning Fog Index} = 

$0.4 \times \left( \frac{\text{Total Words}}{\text{Total Sentences}} + 100 \times \frac{\text{Complex Words}}{\text{Total Words}} \right)$ 
\medskip

\text{Coleman-Liau Index} = 

$0.0588 \times \text{Average Number of Letters per 100 Words} - 0.296 \times \text{Total Sentences} - 15.8$ 
\medskip

\text{Dale-Chall Readability Score} = 

$0.1579 \times \frac{\text{Difficult Words}}{\text{Total Words}} + 0.0496 \times \frac{\text{Total Words}}{\text{Total Sentences}} + 3.6365 $
\medskip

\text{Automated Readability Index (ARI)} =

$4.71 \times \frac{\text{Total Characters}}{\text{Total Words}} + 0.5 \times \frac{\text{Total Words}}{\text{Total Sentences}} - 21.43$ 
\medskip

\text{Linsear Write Formula} = 
$\frac{\left( \text{Easy Words} + 3 \times \text{Difficult Words} \right)}{\text{Total Sentences}}$




\\ 
\textbf{Fluency} & Fluency is the quality of a text based on how well it is written and whether it is grammatically correct. & this may be abandoned \\ 




\bottomrule
\end{tabular}
\caption{Direct Linguistic Features with brief descriptions}
\label{tab:Direct_Linguistic_Features_description}
\end{table*}
