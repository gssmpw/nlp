% !TEX root = ../main.tex

\section{Threats to Validity}
\label{sec:threats}

\textbf{Construct Validity} threats pertain to the relationship between theory and observation, primarily concerning the measurements used to answer our research questions. In this context, the main threat in our study lies in the selection of metrics to evaluate the quality of the generated summaries. As outlined in \secref{sub:analysis}, we utilized well-established metrics for assessing code summary quality \cite{haque2022semantic}, along with newer metrics such as SIDE \cite{mastropaolo2024evaluating}.


\textbf{Internal Validity} threats relate to factors internal to our study that could affect the achieved results. One possible threat can be the selected models to conduct the analysis. We partially mitigate this issue by considering models from the state-of-the-art that have been used in several prior studies~\cite{virk2024enhancing,zhu2024effectiveness,yang2024synthesizing,li2023structured}. 

%To further address potential threats related to model selection, we conducted Full Fine-Tuning (FFT) and QLoRA Fine-Tuning (QTF) on Phi-3-mini, a 3.8 billion parameter model obtaining results that closely align with those observed for CodeLlama and DeepSeek-Coder. This, further validating the consistency of our findings. Detailed information about these experiments, including setups and results, is provided in our replication package \cite{replication}.

Another threat is the selection of the QLoRA hyperparameters. In this regard, we resorted to the literature and particularly followed the recommendations provided in the original paper that introduced QLoRA \cite{dettmers2024qlora}. %However, we do not exclude that a different configuration of hyperparameters could bring different outcomes.

A further potential limitation concerns memory measurement, which was conducted based on a single run within a specific environment. Factors such as background processes or system load variations could impact memory usage metrics. To mitigate this, we maintained a controlled and consistent environment across experiments. However, we acknowledge that conducting multiple runs under varying conditions could offer additional insights.

\textbf{Conclusion Validity} threats involve the link between the experimental process and its outcomes. As detailed in \secref{sub:analysis}, where appropriate, our conclusions are supported by suitable statistical methods.


\textbf{External Validity} threats concern the generalizability of our findings.  In this regard, our study focuses solely on code summarization as a representative bi-modal task. Furthermore, although we conducted experiments using Python and Java— widely studied programming languages in software engineering automation \cite{hou2023large,deepseek,codellama}—we recognize that results may differ for other programming languages.
