\documentclass[conference]{IEEEtran}
%\IEEEoverridecommandlockouts
% The preceding line is only needed to identify funding in the first footnote. If that is unneeded, please comment it out.
%Template version as of 6/27/2024

\usepackage{algorithmic}
\usepackage{graphicx}
\usepackage{textcomp}
\usepackage[dvipsnames]{xcolor}
\usepackage{xspace}
\usepackage{fontawesome}
\usepackage{multirow}
\usepackage{colortbl}
\usepackage{rotating}
\usepackage{tikz}
\usepackage{soul}
\usepackage{makecell}
\usepackage{url}
\usepackage{enumitem}
\usepackage{fancybox}
\usepackage{booktabs}
\usepackage{fontawesome}
\usepackage{setspace}               % for LINE SPACING
\usepackage{pifont}
%%\usepackage{amssymb} 
%\usepackage[colorlinks=true, allcolors=black]{hyperref}
\usepackage[many]{tcolorbox}    	% for COLORED BOXES (tikz and xcolor included)
\usepackage[caption=false]{subfig}
%\usepackage[table]{xcolor}
\usepackage[export]{adjustbox}
\usepackage{amsmath,amsfonts}
\usepackage{MnSymbol}
\usepackage{color,colortbl,booktabs}
%\usepackage[table,xcdraw]{xcolor}
\usepackage[noadjust]{cite}
\usepackage{balance}
%\usepackage{parskip}

\definecolor{main}{HTML}{cccccc}    % setting main color to be used
\definecolor{sub}{HTML}{000000}     % setting sub color to be used



\newtcolorbox{boxM}{
	fontupper = \color{black},
	rounded corners,
	arc = 6pt,
	colback = main!80, 
	colframe = main, 
	boxrule = 0pt, 
	bottomrule = 4.5pt,
	enhanced,
	fuzzy shadow = {0pt}{-3pt}{-0.5pt}{0.5pt}{black!35}
}



\newcommand{\re}{\textcolor{red}{\textbf{[REF]}\xspace}}
\newcommand{\cmark}{\ding{51}}%
\newcommand{\xmark}{\ding{55}}
\newcommand{\ie}{\emph{i.e.,}\xspace}
\newcommand{\eg}{\emph{e.g.,}\xspace}
\newcommand{\etc}{etc.\xspace}
\newcommand{\etal}{\emph{et~al.}\xspace}
\newcommand{\secref}[1]{Section~\ref{#1}\xspace}
\newcommand{\chapref}[1]{Chapter~\ref{#1}\xspace}
\newcommand{\appref}[1]{Appendix~\ref{#1}\xspace}
\newcommand{\figref}[1]{Fig.~\ref{#1}\xspace}
\newcommand{\listref}[1]{Listing~\ref{#1}\xspace}
\newcommand{\tabref}[1]{Table~\ref{#1}\xspace}
\newcommand{\greenAI}{\emph{\textcolor{ForestGreen}{Green} AI}\xspace}
\newcommand*\circled[1]{\tikz[baseline=(char.base)]{
		\node[shape=circle,fill,inner sep=0.8pt] (char) {\textcolor{white}{#1}};}}
%% CUSTOM COMMANDS
\newboolean{showcomments}
\setboolean{showcomments}{true}

\ifthenelse{\boolean{showcomments}}
{\newcommand{\nb}[2]{
		\fbox{\bfseries\sffamily\scriptsize#1}
		{\sf\small$\blacktriangleright$\textit{#2}$\blacktriangleleft$}
	}
	\newcommand{\cvsversion}{\emph{\scriptsize$-$Id: macro.tex,v 1.9 2005/12/09 22:38:33 giulio Exp $}}
}
{\newcommand{\nb}[2]{}
	\newcommand{\cvsversion}{}
}

% Merge citations into single brackets with commas
\renewcommand{\citepunct}{,\penalty\citepunctpenalty\,}
\renewcommand{\citedash}{--}% optionally


\newcommand{\verylightgray}[1]{\cellcolor{gray!10}{#1}}
\newcommand{\lightgray}[1]{\cellcolor{gray!22}{#1}}
\newcommand{\gray}[1]{\cellcolor{gray!33}{#1}}
\newcommand{\darkgray}[1]{\cellcolor{gray!45}{#1}}

\newcommand\SAIMA[1]{\textcolor{purple}{\nb{SAIMA}{#1}}}
\newcommand\ANTONIO[1]{\textcolor{red}{\nb{ANTONIO}{#1}}}
\newcommand\os[1]{\textcolor{blue}{\nb{OSCAR}{#1}}}
\newcommand\KHAI[1]{\textcolor{red}{\nb{KHAI}{#1}}}
\newcommand\JOSEPH[1]{\textcolor{red}{\nb{JOSEPH}{#1}}}

% Define a new command for "et al."
% \newcommand{\etal}{\textit{et al.}}

% Other preamble content

%%
%% \BibTeX command to typeset BibTeX logo in the docs
\AtBeginDocument{%
	\providecommand\BibTeX{{%
			Bib\TeX}}}
\def\BibTeX{{\rm B\kern-.05em{\sc i\kern-.025em b}\kern-.08em
    T\kern-.1667em\lower.7ex\hbox{E}\kern-.125emX}}
\begin{document}

\title{Resource-Efficient \& Effective Code Summarization}

\author{
	\IEEEauthorblockN{Saima Afrin\IEEEauthorrefmark{1}, Joseph Call\IEEEauthorrefmark{2}, Khai-Nguyen Nguyen\IEEEauthorrefmark{3}, Oscar Chaparro\IEEEauthorrefmark{4}, Antonio Mastropaolo\IEEEauthorrefmark{5}}
	\IEEEauthorblockA{
		\textit{William \& Mary}, \textit{Department of Computer Science} \\
		\textit{Williamsburg, Virginia, USA} \\
		\IEEEauthorrefmark{1}safrin@wm.edu,
		\IEEEauthorrefmark{2}jbcall@wm.edu,
		\IEEEauthorrefmark{3}knguyen07@wm.edu,
		\IEEEauthorrefmark{4}oscarch@wm.edu,
		\IEEEauthorrefmark{5}amastropaolo@wm.edu}
}

%\author{\IEEEauthorblockN{Saima Afrin}
%\IEEEauthorblockA{
%	%\textit{Department of Computer Science} \\
%\textit{William \& Mary} \\
%\textit{Dept. of Computer Science} \\
%\textit{Williamsburg, VA, USA} \\
%safrin@wm.edu}
%\and
%
%\IEEEauthorblockN{Joseph Call}
%\IEEEauthorblockA{
%\textit{William \& Mary} \\
%\textit{Dept. of Computer Science} \\
%\textit{Williamsburg, VA, USA} \\
%jbcall@wm.edu}
%
%\and
%\IEEEauthorblockN{Khai-Nguyen Nguyen}
%\IEEEauthorblockA{
%\textit{William \& Mary} \\
%\textit{Dept. of Computer Science} \\
%\textit{Williamsburg, VA, USA} \\
%knguyen07@wm.edu}
%\and
%
%
%
%\IEEEauthorblockN{Oscar Chaparro}
%\IEEEauthorblockA{%\textit{Department of Computer Science} \\
%\textit{William \& Mary} \\
%\textit{Dept. of Computer Science} \\
%\textit{Williamsburg, VA, USA} \\
%oscarch@wm.edu} 
%
%
%\and
%
%\IEEEauthorblockN{Antonio Mastropaolo}
%\IEEEauthorblockA{%\textit{Department of Computer Science} \\
%\textit{William \& Mary} \\
%\textit{Dept. of Computer Science} \\
%\textit{Williamsburg, VA, USA} \\
%amastropaolo@wm.edu}
%\and
%\IEEEauthorblockN{6\textsuperscript{th} Given Name Surname}
%\IEEEauthorblockA{\textit{dept. name of organization (of Aff.)} \\
%\textit{name of organization (of Aff.)}\\
%City, Country \\
%email address or ORCID}


\maketitle

\begin{abstract}
Code Language Models (CLMs) have  demonstrated high effectiveness in automating software engineering tasks such as bug fixing, code generation, and code documentation. This progress has been driven by the scaling of large models, ranging from millions to trillions of parameters (\eg GPT-4).
However, as models grow in scale, sustainability concerns emerge, as they are extremely resource-intensive, highlighting the need for efficient, environmentally conscious solutions. \textcolor{ForestGreen}{GreenAI} techniques, such as QLoRA (Quantized Low-Rank Adaptation), offer a promising path for dealing with large models' sustainability as they enable resource-efficient model fine-tuning.
Previous research has shown the effectiveness of QLoRA in code-related tasks, particularly those involving natural language inputs and code as the target output (NL-to-Code), such as code generation. However, no studies have explored its application to tasks that are fundamentally similar to NL-to-Code (natural language to code) but operate in the opposite direction, such as code summarization. This leaves a gap in understanding how well QLoRA can generalize to Code-to-NL tasks, which are equally important for supporting developers in understanding and maintaining code.
To address this gap, we investigate the extent to which QLoRA's capabilities in NL-to-Code tasks can be leveraged and transferred to code summarization, one representative Code-to-NL task.
Our study evaluates two state-of-the-art CLMs (CodeLlama and DeepSeek-Coder) across two programming languages: Python and Java. Each model was tasked with generating a meaningful description for Python and Java code methods. The findings of our research confirm previous patterns that emerged when applying QLoRA to source code generation. Notably, we observe that QLoRA not only allows efficient fine-tuning of CLMs for code summarization but also achieves the best results with minimal parameter adjustment compared to full model fine-tuning, which requires expensive recalibration of all model parameters in the traditional fine-tuning process.
\looseness=-1



\end{abstract}

\begin{IEEEkeywords}
Code Summarization, PEFT, Quantization, QLoRA, Code Language Models
\end{IEEEkeywords}

  
\section{Introduction}
\label{sec:intro}

\begin{figure*}[tb]
    \centering
    \includegraphics[width=0.848\linewidth]{figs/circuitnn.pdf} 
    \caption{Illustration of differentiable CircuitNN. CircuitNN is designed based on differentiable NAND gates. After DAS is guided by PI and PO pairs of the truth table, CircuitNN can get the precise circuit architecture logic equivalent to the truth table.}
    \label{fig:circuitnn}
\end{figure*}

% 1. Describe the importance of logic synthesis
% 2. Existing Problems
% (a) Neural Architecture Search: Unstable, Predefined Setting, etc.
% (b) Circuit Generation: Probabilistic Model, Logic Equivalence

With the rapid advancement of technology, the scale of integrated circuits (ICs) has expanded exponentially. 
This expansion has introduced significant challenges in chip manufacturing, particularly concerning power and area metrics.
A primary objective in IC design is achieving the same circuit function with fewer transistors, thereby reducing power usage and area occupancy.

Logic synthesis~\cite{hachtel2005logicsynth}, a critical step in electronic design automation (EDA), transforms behavioral-level circuit designs into optimized gate-level circuits, ultimately yielding the final IC layout. 
The primary goal of logic synthesis is to identify the physical implementation with the fewest gates for a given circuit function. 
This task constitutes a challenging NP-hard combinatorial optimization problem. 
Current logic synthesis tools~\cite{brayton2010abc, wolf2013yosys} rely on human-designed heuristics, often leading to sub-optimal outcomes.

Differentiable architecture search (DAS) techniques~\cite{liu2018darts, chu2020darts} offer novel perspectives on addressing challenges in this problem.
Circuit functions can be represented through truth tables, which map binary inputs to their corresponding outputs. 
Truth tables provide a precise representation of input-output relationships, ensuring the design of functionally equivalent circuits.
Inspired by this, researchers~\cite{deepmind2024ai4sys, wang2024tnet} have begun exploring the application of DAS to synthesize circuits directly from truth tables.
Specifically, \citet{deepmind2024ai4sys} proposed CircuitNN, a framework that learns differentiable connection structures with logic gates, enabling the automatic generation of logic circuits from truth tables.
This approach significantly reduces the complexity of traditional circuit generation. 
Building on this, \citet{wang2024tnet} introduced T-Net, a triangle-shaped variant of CircuitNN, incorporating regularization techniques to enhance the efficiency of DAS.

Despite these advancements, several challenges remain. 
The computational complexity of DAS grows quadratically with the number of gates, posing scalability issues.
Although triangle-shaped architecture~\cite{wang2024tnet} partially mitigates this problem, redundancy persists. 
%Additionally, DAS is susceptible to converging to local optima, limiting the ability to search architectures that satisfy the given truth tables~\cite{liu2018darts}. 
%Furthermore, hyperparameters (network depth and layer width) require extensive searches, introducing complexity and prolonging the synthesis process. 
Additionally, DAS is susceptible to converging to local optima~\cite{liu2018darts} and hyperparameters (network depth and layer width) require extensive searches. 
The challenges arise from the vast search space in DAS. 
% Even with predefined settings for CircuitNN, finding a configuration that meets the truth table requires extensive trial and error during the DAS process. 
Intuitively, limiting the search space through predefined parameters (network depth, gates per layer, and connection probabilities) can significantly reduce the complexity.

Recent advances~\cite{openai2023gpt4, abramson2024alphafold3, esser2024sd3, li2024mar} in conditional generative models have demonstrated remarkable performance across language, vision, and graph generation tasks. 
Motivated by these developments, we propose a novel approach to circuit generation that generates preliminary circuit structures to guide DAS in generating refined circuits matching specified truth tables. 
Firstly, we introduce CircuitVQ, a tokenizer with a discrete codebook for circuit tokenization. 
Built upon our Circuit AutoEncoder framework~\cite{hou2022graphmae,li2023maskgae,wu2025mgvga}, CircuitVQ is trained through a circuit reconstruction task. 
Specifically, the CircuitVQ encoder encodes input circuits into discrete tokens using a learnable codebook, while the decoder reconstructs the circuit adjacency matrix based on these tokens.
Subsequently, the CircuitVQ encoder serves as a circuit tokenizer for CircuitAR pretraining, which employs a masked autoregressive modeling paradigm~\cite{chang2022maskgit, li2023mage}. 
In this process, the discrete codes function as supervision signals. 
After training, CircuitAR can generate discrete tokens progressively, which can be decoded into initial circuit structures by the decoder of the CircuitVQ. 
These prior insights can guide DAS in producing refined circuits that match the target truth tables precisely.

Our key contributions can be summarized as follows:
\begin{itemize}
\item We introduce CircuitVQ, a circuit tokenizer that facilitates graph autoregressive modeling for circuit generation, based on our Circuit AutoEncoder framework;
\item Develop CircuitAR, a model trained using masked autoregressive modeling, which generates initial circuit structures conditioned on given truth tables;
\item Propose a refinement framework that integrates differentiable architecture search to produce functionally equivalent circuits guided by target truth tables;
\item Comprehensive experiments demonstrating the scalability and capability emergence of our CircuitAR and the superior performance of the proposed circuit generation approach.
\end{itemize}

% Motivation
% (a) Diffusion (Vision, Graph), Autoregressive (Language, Vision)
% (b) Circuit Generation for Predefined Setting
% (c) Neural Architecture Search for Strict Logic Equivalence

% Contribution
% (a) Circuit Tokenizer (new transformer arch, training strategy)
% (b) CircuitAR (train and gen strategies, post-ar strategy)
% (c) Extensive Evaluation including BitD (Bit Distance) for Scalability


\section{Related Work} \label{sec:related}

% \textbf{Adversarial Attack}
\textbf{Attacks on SLAM.} 
%With the rise of machine learning, 
The robustness of computer vision systems is being actively investigated. With the emergence of adversarial images in the digital domain by adding optimized noise directly to images~\cite{szegedy2013intriguing,carlini2017towards}, researchers find that such attacks also exist physically in the real world \cite{eykholt2018robust,song2018physical,zhao2019seeing}. To fill the gap between attacks in the digital and physical worlds, recent studies have demonstrated that attacks on real-world computer vision systems are practical \cite{eykholt2018robust,li2019adversarial,man2020ghostimage,sharif2016accessorize,zhao2019seeing,zhou2018invisible}. However, attacks on traditional computer vision methods such as SLAM are relatively less explored. \cite{yoshida2022adversarial} proposes an attack against the scan matching algorithm in LiDAR-based SLAM, while most SLAMs in AR/VR devices rely on different sensors like RGB/depth cameras and IMUs. \cite{ikram2022perceptual} and \cite{chen2024adversary} mislead visual SLAM by poisoning the images with special patterns, and \cite{wang2021can} causes the camera to fail using infrared light. In our work, we demonstrate attacks on Visual-Inertial SLAM (VI-SLAM) by perturbing the IMU readings, rather than cameras, and showing its impact on XR user experience. 

\textbf{Acoustic Injection Attacks.} Among various physical attacks, acoustic injection attacks are attractive due to their low cost. Son~\etal~\cite{son2015rocking} were the first to introduce acoustic attacks on MEMS gyroscopes, demonstrating how these attacks could lead to sensor denial-of-service and result in drone crashes. WALNUT~\cite{trippel2017walnut} expanded on this by developing output biasing and control attacks that enable precise manipulation of MEMS accelerometer outputs using modulated sound waves. Wang et al.~\cite{wang2017sonic} demonstrated a sonic gun, showcasing the vulnerability of various smart devices (\eg drones and self-balancing vehicles) to acoustic attacks. Tu et al. \cite{tu2018injected} designed side-swing and switching attacks to alter the outputs of MEMS gyroscopes and accelerometers. Furthermore, Ji et al. \cite{ji2021poltergeist} fool the object detectors by applying acoustic attack to the image stabilizers commonly used in modern cameras. However, none of the existing works study the relationship between the acoustic injections and SLAM outputs on recent XR devices. 

% \zijian{Do we need one session about security in AR/VR?}
% \yicheng{TODO}
%\jiasi{cite the AIVR paper (UMass Amherst?) paper is we have not already. They add IMU perturbation but w/o SLAM, iirc} \yicheng{Cited}

\textbf{XR Security and Privacy.} 
%Security and privacy concerns in XR systems have gained significant attention. 
For single-user XR systems, researchers have demonstrated various side-channel attacks to extract sensitive information (\eg keystrokes) through video feeds~\cite{ling2019know}, head movements~\cite{nair2023unique, slocum2023going}, architectural hints~\cite{zhang2023its,shang2020arspy}, power usage~\cite{li2024dangers}, and EM side-channel leakages~\cite{al2021vr}. In multi-user XR systems, Su et al.~\cite{su2024remote} use avatar motion data to infer keystrokes in shared VR environments. Slocum et al.~\cite{slocum2024doesn} reveal vulnerabilities in the shared state frameworks of multi-user AR. Similarly, Lebeck et al.~\cite{lebeck2017securing} highlight risks like deceptive virtual objects and emphasize access control for managing shared physical and virtual spaces. Ruth et al.~\cite{ruth2019secure} further propose a secure multi-user AR framework focusing on content sharing and permissions.
Chandio et al.~\cite{chandio2024stealthy} %introduced a multi-modal spatiotemporal attack that 
simultaneously manipulated visual and inertial sensors to disrupt XR pose estimation. However, their study evaluated the attack using offline datasets and assumed the attacker's capability to manipulate IMU data streams through acoustic means, without real experiments. Ours is the first to demonstrate acoustic injection attacks on recent XR devices, like the Hololens 2, in the real world.
 


\section{Design}\label{sec:design}

%%%%%%%%%%%%%%%%%%%%%%%%%%%%%%


\begin{figure*}[t]
    \centering
    \includegraphics[trim = 15 530 15 15, width=1\textwidth]{Algorithm_drawio.pdf}
    \caption{Overview of KiSS}
    \label{fig:overview}
\end{figure*}


The results we gleaned from the previous section (see Section~\ref{sec:work_anly}) helped in developing our policy: KiSS. The KiSS or \textbf{Keep it Separated Serverless} policy aims to address critical challenges in Function-as-a-Service (FaaS) platforms, particularly in edge computing environments, by achieving the following objectives:

\begin{itemize}
    \item \textbf{Reduced Cold Start Latency:} Prioritizes high-frequency functions to minimize delays in real-time applications.
    \item \textbf{Improved Resource Efficiency:} Optimizes memory and compute usage while avoiding unnecessary overhead from static warm states.
    \item \textbf{Minimized Inter-Function Interference:} Enhances throughput and scalability through modular resource partitioning.
    \item \textbf{Improved Function Service Rate:} Adopts resource-aware policies to reduce dropped requests and maximize system reliability.
\end{itemize}


\subsection{KiSS Policy Overview}

KiSS introduces a modular, data-driven orchestration strategy designed to optimize serverless execution in resource-constrained environments, particularly at the edge. By leveraging our workload analysis (refer Section 2.5), our policy segments functions based on key metrics—memory footprint, invocation frequency, and execution time—to optimize performance across diverse workloads.

The edge computing context introduces unique challenges like limited memory, heterogeneous resources, and dynamic workloads. Generalized cloud strategies often fail to adapt to such constraints. KiSS addresses this gap by analyzing workload characteristics and implementing a resource-efficient, modular strategy that aligns with edge-specific demands.

\subsection{Components of KiSS Policy Design}
Figure~\ref{fig:overview} shows the overall architecture of KiSS. 
The incoming \textit{FaaS traffic} will include both small and large functions. 
The \textit{request handler} accepts the incoming functions and shares the function information to the workload analyzer. 
The \textit{workload analyser} processes the function information to profile the incoming function traffic information and generate data such as invocation frequency, memory footprint etc.
The \textit{KiSS policy} uses this data to estimate where this function will be placed between the two different warm pool partitions.

The \textit{load balancer} implements a partitioning logic where functions are allocated to distinct warm pools using (\textit{invoker 1} and \textit{invoker 2}) based on profiling thresholds:

(i)~Small Functions Pool: Dedicated to high-frequency, low-memory functions to ensure low latency, and (ii)~Large Functions Pool: Allocated for low-frequency, memory-intensive functions, minimizing contention with smaller containers.
Each warm pool operates autonomously achieving Policy Independence.
The \textit{Warm Pool Replacement Policy} for each warm container pool can independently implement different workload-specific strategies to reduce contention and enhance temporal locality.


These factors form the foundation of KiSS’s multi-tiered warm pool framework, allowing it to effectively manage serverless resources and enhance performance in edge computing. By addressing these challenges, KiSS positions itself as a practical and scalable solution for FaaS platforms in environments with diverse and demanding resource constraints.


\subsection{Innovations of KiSS Policy}

One of the most innovative features of KiSS is its multi-level warm pool partitioning, which isolates high- and low-frequency functions into separate pools. This design eliminates inefficiencies inherent in monolithic resource strategies by ensuring that small, frequently invoked functions are always ready to execute, while larger, less frequent functions remain accessible without competing for resources. This adaptability extends to the ability to add more pools as workload patterns evolve, making KiSS a flexible and future-proof solution. Moreover, its modular architecture supports diverse deployment scenarios, from centralized clouds to resource-constrained edge environments. Integration with traffic-aware schedulers ensures that KiSS maintains scalability and responsiveness even under fluctuating workloads.


\subsubsection{Advantages of KiSS}

The advantages of KiSS are particularly pronounced in edge environments. By keeping frequently accessed containers in warm states, it drastically reduces cold start latency, which is critical for real-time applications such as IoT and AI analytics. Static warm pool partitioning, based on workload analysis, optimizes memory usage by eliminating unnecessary overhead, ensuring that resources are used efficiently even in environments with stringent memory constraints. This strategy not only enhances performance but also reduces operational costs by consolidating memory usage and minimizing cold starts. KiSS’s platform-agnostic design further enhances its versatility, enabling seamless deployment across various serverless frameworks.


\begin{table}[ht!]
\centering
\caption{\textbf{Super Resolution Performance Results.} Our proposed WGAN EEG Spatial Upsampling method significantly outperforms a baseline of Bicubic Interpolation commonly used in EEG upsampling pipelines.}
\label{tab:results}
\resizebox{0.8\linewidth}{!}{%
\begin{tabular}{@{}cccccc@{}}
\toprule
\multirow{2}{*}{\textbf{Dataset}} & \multirow{2}{*}{\textbf{Scale}} & \multicolumn{2}{c}{\textbf{Bicubic}} & \multicolumn{2}{c}{\textbf{WGAN}} \\ \cmidrule(l){3-6} 
                      &   & \textbf{MSE} & \textbf{MAE} & \textbf{MSE}    & \textbf{MAE}   \\
\toprule
\multirow{2}{*}{Val}  & 2 & 3.71E7       & 3.89E3       & \textbf{2.01E3} & \textbf{24.38} \\
                      & 4 & 7.23E7       & 6.42E3       & \textbf{8.53E3} & \textbf{63.83} \\
\midrule
\multirow{2}{*}{Test} & 2 & 3.75E7       & 3.91E3       & \textbf{2.06E3} & \textbf{24.66} \\
                      & 4 & 7.30E7       & 6.45E3       & \textbf{8.68E3} & \textbf{64.39} \\
\bottomrule
\end{tabular}%
}
\end{table}
% implications for future systems and applications

A scalable training framework such as AxoNN and access to large supercomputers
such as Frontier and Alps can enable studying properties of LLMs at scales that
were impossible before. Below, we present a study on the behavior of
memorization by large language models.

% \begin{table}[h]
% 	\centering\tiny
% 	\begin{tabular}{llll}
% 		\toprule
% 		Epoch Count & Experiment & Avg. Exact Match \% (min, max) & With Goldfish Loss \\
% 		\midrule
% 		0 Ep & 1B TinyLlama & 0.0\% (0.0\%, 0.0\%) & 0.0\% (0.0\%, 0.0\%) \\
% 		0 Ep & 7B Llama 2 & 0.0\% (0.0\%, 0.0\%) & 0.0\% (0.0\%, 0.0\%) \\
%		0 Ep & 8B Llama 3.1 & 0.0\% (0.0\%, 0.0\%) & 0.5\% (0.5\%, 0.5\%) \\
%		0 Ep & 13B Llama 2 & 0.0\% (0.0\%, 0.0\%) & 0.0\% (0.0\%, 0.0\%) \\
%		0 Ep & 70B Llama 2 & 0.5\% (0.5\%, 0.5\%) & 0.5\% (0.5\%, 0.5\%) \\
%		0 Ep & 70B Llama 3.1 & 0.3\% (0.0\%, 0.5\%) & 0.5\% (0.5\%, 0.5\%) \\
%		0 Ep & 405B Llama 3.1 & 11.0\% (11.0\%, 11.0\%) & 11.0\% (11.0\%, 11.0\%) \\
% 		1 Ep & 1B TinyLlama & 0.2\% (0.0\%, 0.5\%) & 0.2\% (0.0\%, 0.5\%) \\
% 		1 Ep & 7B Llama 2 & 0.0\% (0.0\%, 0.0\%) & 0.0\% (0.0\%, 0.0\%) \\
% 		1 Ep & 8B Llama 3.1 & 0.0\% (0.0\%, 0.0\%) & 0.0\% (0.0\%, 0.0\%) \\
% 		1 Ep & 13B Llama 2 & 0.2\% (0.0\%, 0.5\%) & 0.1\% (0.0\%, 0.5\%) \\
%		1 Ep & 70B Llama 2 & 1.8\% (1.5\%, 2.0\%) & 0.5\% (0.0\%, 1.5\%) \\
%		1 Ep & 70B Llama 3.1 & 5.2\% (4.5\%, 6.0\%) & 0.5\% (0.0\%, 1.5\%) \\
%		1 Ep & 405B Llama 3.1 & 7.0\% (7.0\%, 7.0\%) & 10.5\% (10.5\%, 10.5\%) \\
% 		4 Ep & 1B TinyLlama & 0.6\% (0.0\%, 1.0\%) & 0.6\% (0.0\%, 1.0\%) \\
% 		4 Ep & 7B Llama 2 & 0.6\% (0.0\%, 1.0\%) & 0.9\% (0.0\%, 1.5\%) \\
% 		4 Ep & 8B Llama 3.1 & 0.3\% (0.0\%, 0.5\%) & 0.3\% (0.0\%, 0.5\%) \\
% 		4 Ep & 13B Llama 2 & 0.9\% (0.0\%, 1.5\%) & 1.1\% (0.0\%, 2.0\%) \\
%		4 Ep & 70B Llama 2 & 12.5\% (10.0\%, 14.0\%) & 1.5\% (0.5\%, 2.5\%) \\
%		4 Ep & 70B Llama 3.1 & 32.2\% (29.5\%, 36.0\%) & 0.2\% (0.0\%, 0.5\%) \\
%		4 Ep & 405B Llama 3.1 & 9.5\% (9.5\%, 9.5\%) & 12.5\% (12.5\%, 12.5\%) \\
% 		6 Ep & 1B TinyLlama & 0.2\% (0.0\%, 0.5\%) & 0.2\% (0.0\%, 0.5\%) \\
% 		6 Ep & 7B Llama 2 & 0.2\% (0.0\%, 0.5\%) & 0.2\% (0.0\%, 0.5\%) \\
% 		6 Ep & 8B Llama 3.1 & 0.0\% (0.0\%, 0.0\%) & 0.0\% (0.0\%, 0.0\%) \\
% 		6 Ep & 13B Llama 2 & 0.2\% (0.0\%, 0.5\%) & 0.2\% (0.0\%, 0.5\%) \\
%		6 Ep & 70B Llama 2 & 47.0\% (44.5\%, 51.5\%) & 0.5\% (0.5\%, 0.5\%) \\
%		6 Ep & 70B Llama 3.1 & 67.8\% (67.5\%, 68.5\%) & 1.3\% (1.0\%, 1.5\%) \\
%		6 Ep & 405B Llama 3.1 & 34.0\% (34.0\%, 34.0\%) & 16.0\% (16.0\%, 16.0\%) \\
% 		\bottomrule
% 	\end{tabular}
% 	\caption{With and without GL}
% \end{table}

\subsection{Memorization of Training Data by Large Language Models}

A growing body of work has shown that language models memorize a portion of
their training data and can reproduce this training data at inference time
\cite{carlini2023quantifying}. The ability of LLMs to reproduce training data
has become a flashpoint for the AI community, as it poses major privacy and
legal risks for commercial models \cite{grynbaum2023times,
carlini2021extracting,carlini2023quantifying}.
% This effect grows rapidly as a function of the number of times a sequence is
% repeated in the data but findings also suggest that, memorization rates also
% climb with parameter count.

It is thought that memorization is largely due to training data repetition, and
it may be mitigated by dataset deduplication. Other factors such as data
structure and model size may play a factor, but the issue is not well
understood because public experiments have been constrained to smaller models
(e.g.~the popular Llama-2 7 billion parameter model~\cite{touvron2023llama})
with limited capacity and correspondingly small rates of memorization
\cite{carlini2023quantifying,biderman2023pythia}. As we observe below, the
ability to memorize entire documents emerges only for large model
sizes. Further, we hypothesize that models above a certain size threshold
may exhibit \textit{catastrophic memorization}, in which documents are
memorized immediately in one single pass. When training a model above this size
limit, even perfectly deduplicated datasets may still result in privacy and
copyright leaks.

By creating scalable, user-friendly and portable access to model parallelism,
AxoNN unlocks the potential for training and fine-tuning much larger models
under commodity computing constraints using sequential LLM training codebases.
This creates a scientific laboratory where large-model phenomena
such as memorization can be publicly reproduced and studied. It also raises the
ability of many practitioners to fine-tune large models on domain-specific
data, expanding the need to understand memorization risks.

\subsection{Experimental Setup: Training Llama models on Wikipedia}

We design a targeted set of continued pre-training experiments to quantify the
relationship between model size and memorization. We consider the Llama family
of LLMs with publicly available pre-trained weights, and use the AxoNN infused
LitGPT framework (introduced in Section~\ref{sec:setup-desc}) to parallelize
the models. Our experiments start with pre-trained checkpoints for the
TinyLlama-1B model~\cite{zhang2024tinyllama}, the 7B, 13B, and 70B parameter
models in the Llama 2 family~\cite{touvron2023llama} and the 8B, 70B, and 405B
parameter models from the recent Llama 3.1 release~\cite{dubey2024llama}. We
train on English text data from Wikipedia with varying levels of repetition to quantify how
memorization depends on model scale.

We train on English Wikipedia pages with $2048$ tokens or more. The articles
are randomly placed into one of four disjoint ``buckets,'' each with 200 articles. During
training, the first three buckets are repeated for 1, 4, or 6 ``epochs'' (one
pass over every page in the bucket) respectively. The fourth bucket is a
control group to measure baseline preexisting memorization from pre-training,
and we do not perform any further training on the pages in the fourth bucket. After training is
complete, we prompt the model with the beginning of each training sequence, and
let the model write the last 50 tokens. We consider a sequence memorized if the
model perfectly reproduces the correct 50 tokens.

%For each model size, we train on three buckets of Wikipedia pages for $1$, $4$
%and $6$ epochs respectively\footnote{The actual epoch counts are $1$, $\sim4.4$
%	and $\sim6.6$ epochs due to the stochastic nature of our batch sampling code.}.
We train the $1$B, $7$B, and $8$B models on eight GCDs of Frontier using
$8$-way $Z$-tensor parallelism (i.e. $G_{z}=8$), the $13$B model using $16$
GCDs, the $70$B models using $64$ GCDs, and the $405$B model using $128$ GCDs,
each with a corresponding level of $Z$-tensor parallelism. The total batch size
is fixed at $128$ samples for all model sizes. In the case of smaller models,
lower level of tensor parallelism is needed, so data parallelism is used to
utilize the remaining GPUs.  We warm up each model for $50$ steps, increasing
the learning rate to $3\times10^{-4}$ on the non-bucketed Wikipedia pages, and
then inject the three buckets of target data over the next $50$ steps of
training while decaying the learning rate to $3\times10^{-5}$. We report
memorization for each bucket separately, and also for the held-out (``$0$ Ep'')
control bucket.

% trim = left, bottom, right, top
\begin{figure*}[t]
	\centering
	\includegraphics[width=0.49\textwidth]{figs/goldfish/update_mem_em_v_epochs_grouped_bars_small_models.pdf}
	\includegraphics[width=0.49\textwidth]{figs/goldfish/update_mem_em_v_epochs_grouped_bars_large_models.pdf}
    \caption{Memorization as a function of parameter count and epochs
(repetitions of the training data). For each model size, we show the ``Exact
Match'' rate at which the model correctly reproduces the last $50$ tokens of
articles after being trained on them for various numbers of epochs.
\textbf{(Left)}  Memorization is difficult to observe for small models.
\textbf{(Right)} The ability to efficiently memorize emerges at larger models
scales.  We see that a 70B model is even capable of {\em catastrophic
memorization}, as it memorized entire documents after seeing them just once.
For models with parameter counts in the
$1$B-$13$B range, we report the average over five trials, for $70$B, we report
the average over three trials, and for $405$B we report a single trial. Error
bars depict the min and max observed scores.}
    \label{fig:mem-results}
\end{figure*}

\subsection{Results: Catastrophic Memorization as a Function of Model Size}

% Chart data is here if you want to cite more specific numbers:
% figs/data_mem_raw_update/gordon_bell_update_2024-08-08_17-59-04.csv
% non_member=0Ep,bucket3=1Ep,bucket4=4Ep,bucket5=6Ep

Figure~\ref{fig:mem-results} shows the impact of parameter count and number of
epochs on exact memorization under otherwise identical conditions. At the
$1$B-$13$B scale (left plot), training for up to six epochs causes memorization of less
than $1\%$ of the $200$ documents on average. However, we observe that the $70$B
models and the 405B model are capable of significant memorization (right plot). After just six passes over the
data, the $70$B Llama 2 and $70$B Llama 3.1 models memorize $\mathbf{47\%}$ and
$\mathbf{67\%}$ of documents on average respectively. Furthermore, we observe
catastrophic memorization behavior starting at the $70$B scale; roughly 5\% of
documents are memorized in just one single pass.

Moving to the $405$B scale, we make several surprising observations. This model
had already memorized over 10\% of the control documents (see the bars labeled
``$0$ Ep'') before our experiment even began, showing that the ability to
memorize and retain documents during pre-training has emerged at this scale.
While Wikipedia pages were certainly included in the training corpus of the
Llama 3.1 series of models, only this largest model in the family exhibits such
non-trivial levels of memorization without further continued training.
Counterintuitively, we note that the rate of memorization of the $405$B model
during continued training was slower than that of the $70$B model. This is
likely because we used one set of hyperparameters for all models, and extreme
scales likely require different hyperparameters for optimal learning.

% \fix{405B caveats}
% @tomg other caveats not described above

% memorization rate appears to go down after we train for 1 and 4 epochs versus the number at 0.
% because we dont track the "diff" we cant tell the difference between the 405 forgetting some sequences versus learning some new ones.
% We couldn't run the model in the same precision setting. It is bf16-true, all other models are bf16-mixed. could matter more for very large deep models with loss of gradient precision when trying to fit tokens perfectly in a few passes.

\subsection{Results: Goldfish Loss Stops Memorization in its Tracks}

% Tom will rewrite this in his way I am sure :]
Observing extreme levels of memorization for models at the $70$B parameter
scale and above, we deploy a recently proposed technique for mitigating
memorization in large language models. Language model training
minimizes the expected cross-entropy between the language model's next-token
distribution and the true tokens as they appear in the training corpus. The
\textit{Goldfish Loss}~\cite{hans2024goldfish} technique introduces a mask such that some
tokens in any given training sequence are randomly omitted from the loss
computation. The model cannot memorize the masked tokens, and must ``guess''
them when trying to reproduce a training sequence at inference time, making it
very unlikely that long sequences can be exactly reproduced.

\begin{figure}[h]
	\centering
	\includegraphics[width=\columnwidth]{figs/goldfish/update_mem_em_v_epochs_grouped_bars_w_tld_large_models.pdf}
    \caption{The impact of applying Goldfish Loss during training to mitigate
memorization in large models. The Exact Match rate reduces to levels comparable
to the control data.}
    \label{fig:goldfish}
\end{figure}

Figure~\ref{fig:goldfish} shows the results of re-running our training
experiments with Goldfish Loss activated (using Goldfish parameters k=2, h=13).
Even after continued training, memorization now reduces to levels comparable to the
control data (0 Ep). We do observe a small increase in memorization as the 405B
model trains, likely because the model has already memorized the masked tokens
from when it was pre-trained on Wikipedia. However, as we can see, the reduction in memorization when using the Goldfish Loss is significant, both for the 70B models and the 405B model.


\section{Threats to Validity}
\label{threats}

\textbf{Construct Validity:} Primary threats involve our operationalization of key concepts. Our classification of major Ethereum events involves subjectivity, though mitigated by selecting widely acknowledged events with cross-domain impact. Using commit activity and issue resolution time as development metrics may not capture all contributions or team dynamics, while network analysis through shared comments might miss other collaboration forms. We address these limitations through multiple metrics and cross-repository validation. Repository selection bias is minimized by using objective criteria (activity levels, ecosystem roles) to ensure complete coverage of the Ethereum ecosystem.
To ensure broad impact, we define major events as those spanning at least two categories (infrastructure, market, or development trajectory). This criterion avoids overestimating the influence of isolated events. Our classification is based on historical records beyond our dataset, including Ethereum Foundation communications, network upgrades, market shifts, and security reports from 2014–2024. While a constructed definition, it provides consistency in assessing event significance. Future work could refine this approach by exploring alternative classification methods.


\textbf{Internal Validity:} Several factors could affect the relationship between events and observed developer activity. First, concurrent events or changes not included in our analysis might influence our results. We addressed this through our 90-day window analysis and by validating findings against random time periods.
Repository sizes vary significantly (from 678 to 24617 commits). However, our key findings and primary statistical inferences are drawn from the major repositories ($>15K$ commits): Go-ethereum, MetaMask, and Solidity. While we analyze medium ($5-15K$) and smaller ($<5K$) repositories for completeness, their results primarily serve to complement our main conclusions. This approach ensures our statistical inferences remain robust despite the size variations across the dataset.\\
\textbf{External Validity:} Our study focuses specifically on the Ethereum ecosystem during 2014–2024, examining how different types of events impact development patterns across its diverse repositories. While our findings provide insights into Ethereum's development dynamics, we acknowledge that these patterns are shaped by Ethereum's unique characteristics as a leading smart contract platform. Factors such as its decentralized governance, token incentives, and community-driven protocol upgrades differentiate Ethereum from traditional OSS projects.
Our repository selection spans multiple layers of the Ethereum ecosystem—from core infrastructure (Go-ethereum) to development tools (Hardhat, Truffle) and user-facing applications (MetaMask)—providing a broad view of development patterns within this blockchain platform. This diversity strengthens our findings regarding how different components of the Ethereum ecosystem respond to various types of events. However, we recognize that event-response patterns may differ in non-Ethereum projects, where governance models, incentive structures, and development workflows vary. 
Future research could explore whether similar trends hold across other blockchain and non-blockchain OSS ecosystems, particularly in projects with distinct governance mechanisms or without direct market exposure.\\ 
\textbf{Conclusion Validity:} We ensured statistical reliability by selecting appropriate tests based on data distributions, calculating effect sizes, and conducting multiple complementary analyses. Our survival and network analyses are based on assumptions of censoring independence and comment co-occurrence as a proxy for collaboration. To address multiple comparisons, we applied the Benjamini-Hochberg procedure to control the false discovery rate (FDR).  
Data completeness may be influenced by GitHub API limitations, particularly for older events. However, to support validation and reproducibility, we provide a complete replication package containing all data, code, and analysis scripts \cite{Vaccargiu2025}.



\section*{Conclusion}
This paper aims to enhance our understanding of the computational complexity of computing various Shapley value variants. We found that for various ML models --- including decision trees, regression tree ensembles, weighted automata, and linear regression --- both local and global interventional and baseline SHAP can be computed in polynomial time under HMM modeled distributions. This extends popular algorithms, such as TreeSHAP, beyond their empirical distributional scope. We also establish strict complexity gaps between the various SHAP variants (baseline, interventional, and conditional) and prove the intractability of computing SHAP for tree ensembles and neural networks in simplified scenarios. Overall, we present SHAP as a versatile framework whose complexity depends on four key factors: \begin{inparaenum}[(i)] \item model type, \item SHAP variant, \item distribution modeling approach, \item and local vs. global explanations\end{inparaenum}. We believe this perspective provides deeper insight into the computational complexity of SHAP, paving the way for future work.




%We believe that our framework provides a more intricate understanding of SHAP computation complexity across different models, distributions, and variants, paving the way for further research.

Our work opens promising directions for future research. First, expanding our computational analysis to other SHAP-related metrics, such as asymmetric SHAP~\citep{frye20} and SAGE~\citep{covert2020understanding}, would be valuable. Additionally, we aim to explore more expressive distribution classes and relaxed assumptions beyond those in Section \ref{sec:tractable} while maintaining tractable SHAP computation. Finally, when exact computation is intractable (Section \ref{sec:intractable}), investigating the approximability of SHAP metrics through approximation and parameterized complexity theory~\citep{downey2012parameterized} is an important direction.

%Our work opens several promising avenues for future research on the computational properties of explainable AI methods, with a particular focus on SHAP. First, it would be interesting to broaden the computational analysis conducted in this work to include other popular SHAP-related metrics in the literature, such as asymmetric SHAP \cite{frye20} and SAGE \cite{covert2020understanding}. Also, in the future, we aim to explore more expressive distribution classes and relaxed distributional assumptions—extending beyond those examined in Section \ref{sec:tractable} —that still yield tractable SHAP computation. Finally, when exact computation proves intractable (Section \ref{sec:intractable}), it is worthwhile to theoretically investigate the question of the approximability of computing the SHAP metrics across various configurations, through the lens of approximation and parametrized complexity theory \cite{arora2009computational}.

%This paper aims to deepen our understanding of the computational complexity involved in obtaining different Shapley value variants. We found that for a variety of ML models, including decision trees, tree ensembles for regression, weighted automata, and linear regression models — computing both local and global interventional and baseline SHAP can be done in polynomial time when distributions are modeled by HMMs. This extends the distributional scope of popular algorithms like TreeSHAP, which is limited to empirical distributions. Additionally, we demonstrate a strict complexity gap between SHAP variants, showing that interventional and baseline SHAP can be strictly easier to compute than conditional SHAP. Despite these positive results, we uncovered intractability for various SHAP variants in neural networks and tree ensembles. Finally, we provided generalized complexity relations across SHAP variants. We believe that our framework offers a deeper understanding of the complexity involved in computing SHAP across various variants, models, distributions, as well as in both local and global computations, laying the groundwork for future research.


%\os{make sure to have the format of the references normalized, either remove conference acronyms or keep them, and if we keep them, have a consistent format (e.g., ICSE'24 vs ICSE)}

\balance
\bibliographystyle{IEEEtran}
\bibliography{main}


\end{document}
