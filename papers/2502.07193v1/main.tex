\pdfoutput=1
\documentclass[11pt]{article}
\usepackage{fullpage}
%!TEX root=main.tex
\newif\ifspacehack
%\spacehacktrue
\usepackage{natbib}
\hypersetup{
    colorlinks = blue,
    breaklinks,
    linkcolor = blue,
    citecolor = blue,
    urlcolor  = blue,
}
\usepackage{url} 
\usepackage{graphicx}
\usepackage{mathtools}
\usepackage{footnote}
\usepackage{float}
\usepackage{xspace}
\usepackage{multirow}
\usepackage{xcolor}
\usepackage{wrapfig}
\usepackage{framed}
\usepackage{bbm}
\usepackage[most]{tcolorbox}

\usepackage{footnote}
\usepackage{nicefrac}
\usepackage{makecell}
\usepackage[ruled,vlined]{algorithm2e}
\usepackage{amssymb}
\usepackage{bm}
\makesavenoteenv{tabular}
\makesavenoteenv{table}

\newcommand{\hl}[1]{{\color{red}[HL: #1]}}
\newcommand{\mznote}[1]{{\color{blue}[MZ: #1]}}

% macros@Peng
\newcommand\innerp[2]{\langle #1, #2 \rangle}
\renewcommand{\tilde}{\widetilde}
\renewcommand{\hat}{\widehat}


\newcommand{\TVD}[1]{\norm{#1}_\text{TV}}
\newcommand{\corral}{\textsc{Corral}\xspace}
\newcommand{\expthree}{\textsc{Exp3}\xspace}
\newcommand{\expfour}{\ensuremath{\mathsf{Exp4}}\xspace}
\newcommand{\expthreeP}{\textsc{Exp3.P}\xspace}
\newcommand{\scrible}{\textsc{SCRiBLe}\xspace}

\def \R {\mathbb{R}}
\newcommand{\eps}{\epsilon}
\newcommand{\vecc}{\mathrm{vec}}
\newcommand{\LS}{\mathrm{LS}}
\newcommand{\FG}{\mathrm{FG}}
\newcommand{\DL}{\Delta \ellhat}
\newcommand{\calA}{{\mathcal{A}}}
\newcommand{\smax}{{\mathrm{smax}}}
\newcommand{\calB}{{\mathcal{B}}}
\newcommand{\calX}{{\mathcal{X}}}
\newcommand{\calS}{{\mathcal{S}}}
\newcommand{\calF}{{\mathcal{F}}}
\newcommand{\calI}{{\mathcal{I}}}
\newcommand{\calJ}{{\mathcal{J}}}
\newcommand{\calK}{{\mathcal{K}}}
\newcommand{\calH}{{\mathcal{H}}}
\newcommand{\calD}{{\mathcal{D}}}
\newcommand{\calE}{{\mathcal{E}}}
\newcommand{\calG}{{\mathcal{G}}}
\newcommand{\calU}{{\mathcal{U}}}
\newcommand{\calR}{{\mathcal{R}}}
\newcommand{\calT}{{\mathcal{T}}}
\newcommand{\calP}{{\mathcal{P}}}
\newcommand{\calQ}{{\mathcal{Q}}}
\newcommand{\calZ}{{\mathcal{Z}}}
\newcommand{\calM}{{\mathcal{M}}}
\newcommand{\calN}{{\mathcal{N}}}
\newcommand{\calW}{{\mathcal{W}}}
\newcommand{\calY}{{\mathcal{Y}}}
\newcommand{\cD}{{\mathcal{D}_{\mathcal{X}}}}
\newcommand{\mcD}{{\mathcal{D}}}
\newcommand{\cF}{{\mathcal{F}}}
\newcommand{\cA}{{\mathcal{A}}}
\newcommand{\cX}{{\mathcal{X}}}
\newcommand{\cE}{{\mathcal{E}}}
\newcommand{\cV}{{\mathcal{V}}}
\newcommand{\cR}{{\mathcal{R}}}
\newcommand{\wcR}{\widehat{\mathcal{R}}}
\newcommand{\Reg}{{\mathrm{Reg}}}
\newcommand{\Alg}{{\mathsf{Alg}}}
\newcommand{\wReg}{\widehat{\mathrm{Reg}}}
\newcommand{\cB}{\mathcal{B}}
\newcommand{\cP}{\mathcal{P}}
\newcommand{\nctx}{\text{n-ctx}}
\newcommand{\ctx}{\text{ctx}}
\newcommand{\E}{{\mathbb{E}}}
\newcommand{\V}{\mathbb{V}}
\newcommand{\Prob}{\mathbb{P}}
\newcommand{\1}{\mathbb{I}}
\newcommand{\N}{\mathbb{N}}
\newcommand{\tup}[1]{t^{(#1)}}
\newcommand{\gup}[1]{g^{(#1)}}
\newcommand{\hatfm}{\widehat{f}_m}
\newcommand{\haty}{\widehat{y}}
\newcommand{\hatx}{\widehat{x}}
\newcommand{\yhat}{\widehat{y}}
\newcommand{\xhat}{\widehat{x}}
\newcommand{\fhat}{\widehat{f}}
\newcommand{\ghat}{\widehat{g}}

\newcommand{\inner}[1]{ \left\langle {#1} \right\rangle }
\newcommand{\ind}{\mathbb{I}}
\newcommand{\diag}{\textrm{diag}}
\newcommand{\Nout}{N_{\textrm{out}}}
\newcommand{\nout}{N_{\textrm{out}}}
\newcommand{\Nin}{{\textrm{Nin}}}
\newcommand{\nin}{{\textrm{Nin}}}
\newcommand{\order}{\mathcal{O}}


\newcommand{\Acal}{\mathcal{A}}
\newcommand{\Bcal}{\mathcal{B}}
\newcommand{\Ccal}{\mathcal{C}}
\newcommand{\Dcal}{\mathcal{D}}
\newcommand{\Ecal}{\mathcal{E}}
\newcommand{\Fcal}{\mathcal{F}}
\newcommand{\Gcal}{\mathcal{G}}
\newcommand{\Hcal}{\mathcal{H}}
\newcommand{\Ical}{\mathcal{I}}
\newcommand{\Jcal}{\mathcal{J}}
\newcommand{\Kcal}{\mathcal{K}}
\newcommand{\Lcal}{\mathcal{L}}
\newcommand{\Mcal}{\mathcal{M}}
\newcommand{\Ncal}{\mathcal{N}}
\newcommand{\Ocal}{\mathcal{O}}
\newcommand{\Pcal}{\mathcal{P}}
\newcommand{\Qcal}{\mathcal{Q}}
\newcommand{\Rcal}{\mathcal{R}}
\newcommand{\Scal}{\mathcal{S}}
\newcommand{\Tcal}{\mathcal{T}}
\newcommand{\Ucal}{\mathcal{U}}
\newcommand{\Vcal}{\mathcal{V}}
\newcommand{\Wcal}{\mathcal{W}}
\newcommand{\Xcal}{\mathcal{X}}
\newcommand{\Ycal}{\mathcal{Y}}
\newcommand{\Zcal}{\mathcal{Z}}
\newcommand{\wkdn}{d}


\newcommand{\avgR}{\wh{\cal{R}}}
%\newcommand{\ips}{\wh{r}}
\newcommand{\whpi}{\wh{\pi}}
\newcommand{\whE}{\wh{\E}}
\newcommand{\whV}{\wh{V}}

\newcommand{\whReg}{\wh{\text{\rm Reg}}}
\newcommand{\flg}{\text{\rm flag}}
\newcommand{\one}{\boldsymbol{1}}
\newcommand{\var}{\Delta}
\newcommand{\Var}{\mathrm{Var}}
\newcommand{\bvar}{\bar{\Delta}}
\newcommand{\p}{\prime}
\newcommand{\evt}{\textsc{Event}}
\newcommand{\unif}{\text{\rm Unif}}
\newcommand{\KL}{\text{\rm KL}}
\newcommand{\Lstar}{{L^\star}}
\newcommand{\istar}{{i^\star}}
\newcommand{\dynreg}{\text{Dyn-Reg}}
\newcommand{\tildedynreg}{\widetilde{\text{Dyn-Reg}}}
\newcommand{\Bstar}{{B^\star}}
\newcommand{\Ustar}{\rho}
\newcommand{\Aconst}{a}
\newcommand{\dplus}[1]{\bm{#1}}
\newcommand{\lambdamax}{\lambda_\text{\rm max}}
\newcommand{\biasone}{\textsc{Deviation}\xspace}
\newcommand{\bias}{\textsc{Bias-1}\xspace}
\newcommand{\biastwo}{\textsc{Bias-2}\xspace}
\newcommand{\biasthree}{\textsc{Bias-3}\xspace}
\newcommand{\term}[1]{\texttt{Term} ~(#1)\xspace}
\newcommand{\x}{\mathbf{x}}
\newcommand{\errorterm}{\textsc{Error}\xspace}
\newcommand{\Err}[1]{\textsc{Err-Term}(#1)\xspace}
\newcommand{\regnctx}{\textsc{Reg-NCTX}\xspace}
\newcommand{\regterm}{\textsc{Reg-Term}\xspace}
\newcommand{\LTtilde}{\wt{L}_T}
\newcommand{\Bomega}{B_{\Omega}}
\newcommand{\UOB}{UOB-REPS\xspace}
\newcommand{\Holder}{{H{\"o}lder}\xspace}
\newcommand{\dpg}{\dplus{g}}
\newcommand{\dpM}{\dplus{M}}
\newcommand{\dpf}{\dplus{f}}
\newcommand{\dpX}{\dplus{\calX}}
\newcommand{\dpw}{\dplus{w}}
\newcommand{\dpF}{\dplus{F}}
\newcommand{\dpu}{\dplus{u}}
\newcommand{\dpwtilde}{\dplus{\wtilde}}
\newcommand{\dps}{\dplus{s}}
\newcommand{\dpe}{\dplus{e}}
\newcommand{\dpx}{\dplus{x}}
\newcommand{\dpy}{\dplus{y}}
\newcommand{\dpH}{\dplus{H}}
\newcommand{\dpOmega}{\dplus{\Omega}}
\newcommand{\dpellhat}{\dplus{\ellhat}}
\newcommand{\dpell}{\dplus{\ell}}
\newcommand{\dpr}{\dplus{r}}
\newcommand{\dpxi}{\dplus{\xi}}
\newcommand{\dpv}{\dplus{v}}
\newcommand{\dpI}{\dplus{I}}
\newcommand{\dpA}{\dplus{A}}
\newcommand{\dph}{\dplus{h}}
\newcommand{\cprob}{6}
\newcommand{\sigmoid}{\ensuremath{\mathsf{Sigmoid}}\xspace}
\newcommand{\relu}{\ensuremath{\mathsf{ReLU}}\xspace}

\DeclareMathOperator*{\argmin}{argmin}
\DeclareMathOperator*{\argmax}{argmax}
\DeclareMathOperator*{\argsmax}{argsmax}
%\DeclareMathOperator*{\range}{range}
%\DeclareMathOperator*{\mydet}{det_{+}}
%\DeclarePairedDelimiter\abs{\lvert}{\rvert}
%\DeclarePairedDelimiter\bigabs{\big\lvert}{\big\rvert}
\DeclarePairedDelimiter\ceil{\lceil}{\rceil}
%\DeclarePairedDelimiter\floor{\lfloor}{\rfloor}
%\DeclarePairedDelimiter\bigceil{\big\lceil}{\big\rceil}
%\DeclarePairedDelimiter\bigfloor{\big\lfloor}{\big\rfloor}

\newcommand{\field}[1]{\mathbb{#1}}
\newcommand{\fY}{\field{Y}}
\newcommand{\fX}{\field{X}}
\newcommand{\fH}{\field{H}}
\newcommand{\fR}{\field{R}}
\newcommand{\fN}{\field{N}}
\newcommand{\fS}{\field{S}}
\newcommand{\UCB}{{\operatorname{UCB}}}
\newcommand{\LCB}{{\operatorname{LCB}}}
\newcommand{\testblock}{\textsc{EndofBlockTest}\xspace}
\newcommand{\testreplay}{\textsc{EndofReplayTest}\xspace}

\newcommand{\theset}[2]{ \left\{ {#1} \,:\, {#2} \right\} }
% \newcommand{\inner}[1]{ \langle {#1} \rangle }
\newcommand{\inn}[1]{ \langle {#1} \rangle }
\newcommand{\Ind}[1]{ \field{I}_{\{{#1}\}} }
\newcommand{\eye}[1]{ \boldsymbol{I}_{#1} }
\newcommand{\norm}[1]{\left\|{#1}\right\|}
%\newcommand{\trace}[1]{\text{tr}\left({#1}\right)}
\newcommand{\trace}[1]{\textsc{tr}({#1})}


\newcommand{\defeq}{\stackrel{\rm def}{=}}
\newcommand{\sgn}{\mbox{\sc sgn}}
\newcommand{\scI}{\mathcal{I}}
\newcommand{\scO}{\mathcal{O}}
\newcommand{\scN}{\mathcal{N}}
\newcommand{\msmwc}{\textsc{MsMwC}}

\newcommand{\dt}{\displaystyle}
\renewcommand{\ss}{\subseteq}
\newcommand{\wh}{\widehat}
\newcommand{\wt}{\widetilde}
\newcommand{\wb}{\overline}
\newcommand{\ve}{\varepsilon}
\newcommand{\hlambda}{\wh{\lambda}}

\newcommand{\Jd}{J}
\newcommand{\ellhat}{\wh{\ell}}
\newcommand{\rhat}{\wh{r}}
\newcommand{\elltilde}{\wt{\ell}}
\newcommand{\wtilde}{\wt{w}}
\newcommand{\what}{\wh{w}}

\DeclareMathOperator{\conv}{conv}
\newcommand{\ellprime}{\ellhat^\prime}

\newcommand{\upconf}{\phi}

%\newcommand{\Ltilde}{\wt{L}}

\newcommand{\hDelta}{\wh{\Delta}}
\newcommand{\hdelta}{\wh{\delta}}
\newcommand{\spin}{\{-1,+1\}}

\newcommand{\ep}[1]{\E\!\left[#1\right]}
\newcommand{\LT}{L_T}
\newcommand{\LTbar}{\overline{L}_T}
\newcommand{\LTbarep}{\mathring{L}_T}
\newcommand{\circxhat}{\mathring{\xh}}
\newcommand{\circx}{\mathring{x}}
\newcommand{\circu}{\mathring{u}}
\newcommand{\circcalX}{\mathring{\calX}}
\newcommand{\circg}{\mathring{g}}
\newcommand{\Lubar}{\overline{L}_{u}}
%\newcommand{\Lustarbar}{\overline{L}_{u^\star}}

\newcommand{\Lyr}{J}
\newcommand{\QQ}{{w}}
\newcommand{\Qt}{{\QQ_t}}
\newcommand{\Qstar}{{u}}
\newcommand{\Qpistar}{{\Qstar^{\star}}}
\newcommand{\Qhat}{\wh{\QQ}}
\newcommand{\Ut}{{\upconf_t}}
\newcommand{\intO}{\mathrm{int}(\Omega)}
\newcommand{\intK}{\mathrm{int}(K)}

\newcommand{\squareCB}{\ensuremath{\mathsf{SquareCB}}\xspace}
\newcommand{\feelgood}{\ensuremath{\mathsf{FGTS}}\xspace}
\newcommand{\graphCB}{\ensuremath{\mathsf{SquareCB.G}}\xspace}
\newcommand{\squareCBAuc}{\ensuremath{\mathsf{SquareCB.A}}\xspace}
\newcommand{\AlgSq}{\ensuremath{\mathsf{AlgSq}}\xspace}
\newcommand{\AlgLog}{\ensuremath{\mathsf{AlgLog}}\xspace}
\newcommand{\ips}{\ensuremath{\mathsf{(IPS)}}\xspace}
\newcommand{\optsq}{\ensuremath{\mathsf{(OptSq)}}\xspace}
\newcommand{\sq}{\ensuremath{\mathsf{(Sq)}}\xspace}
\newcommand{\dec}{\ensuremath{\mathsf{dec}_\gamma}\xspace}
\newcommand{\dectwo}{\ensuremath{\mathsf{dec}_{\gamma_1,\gamma_2}}\xspace}
%\newcommand{\theHalgorithm}{\arabic{algorithm}}
\newtheorem{cor}[theorem]{Corollary}
\newcommand{\context}{\text{ctx}}
\newcommand{\noncontext}{\text{n-ctx}}
%\newtheorem{remark}{Remark}
%\newtheorem{prop}{Proposition}
%\newtheorem{definition}{Definition}
%\newtheorem{assumption}{Assumption}
\newtheorem{event}{Event}
%\newtheorem*{main}{Main Result}
%\newtheorem{fact}[theorem]{Fact}

\newcommand{\paren}[1]{\left({#1}\right)}
\newcommand{\brackets}[1]{\left[{#1}\right]}
\newcommand{\braces}[1]{\left\{{#1}\right\}}

\newcommand{\normt}[1]{\norm{#1}_{t}}
\newcommand{\dualnormt}[1]{\norm{#1}_{t,*}}

\newcommand{\otil}{\ensuremath{\tilde{\mathcal{O}}}}

\newcommand{\dist}{\calP}

%%%%  brackets
\newcommand{\rbr}[1]{\left(#1\right)}
\newcommand{\sbr}[1]{\left[#1\right]}
\newcommand{\cbr}[1]{\left\{#1\right\}}
\newcommand{\nbr}[1]{\left\|#1\right\|}
\newcommand{\abr}[1]{\left|#1\right|}

\usepackage{lipsum,booktabs}
\usepackage{amsmath,mathrsfs,amssymb,amsfonts,bm,enumitem}
\usepackage{rotating}
\usepackage{pdflscape}
\usepackage{hyperref,url}
\hypersetup{
    colorlinks,
    breaklinks,
    linkcolor = blue,
    citecolor = blue,
    urlcolor  = blue,
}
\allowdisplaybreaks
\usepackage{appendix}
\usepackage{multirow,makecell}

%\usepackage{algorithmic,algorithm}
%\renewcommand{\algorithmicrequire}{ \textbf{Input:}}
%\renewcommand{\algorithmicensure}{ \textbf{Output:}}

\renewcommand{\tilde}{\widetilde}
\renewcommand{\hat}{\widehat}
\newcommand{\obs}{O}
\newcommand{\unobs}{E}
\newcommand{\unbiasSize}{c}
\newcommand{\unbias}{C}
\newcommand{\cnt}{k}

% define some macros
\def \A {\mathcal{A}}

\def \B {\mathbb{B}}
\def \B {\mathcal{B}}
\def \C {\mathcal{C}}
\def \D {\mathcal{D}}
\def \E {\mathbb{E}}
\def \F {\mathcal{F}}
\def \G {\mathcal{G}}
\def \H {\mathcal{H}}
\def \I {\mathcal{I}}
\def \J {\mathcal{J}}
\def \K {\mathcal{K}}
\def \L {\mathcal{L}}
\def \M {\mathcal{M}}
\def \N {\mathcal{N}}
\def \O {\mathcal{O}}
\def \P {\mathcal{P}}
\def \Q {\mathcal{Q}}
\def \R {\mathbb{R}}
\def \S {\mathcal{S}}
% \def \T {\mathrm{T}}
\def \T {\top}
\def \U {\mathcal{U}}
\def \V {\mathcal{V}}
\def \W {\mathcal{W}}
\def \X {\mathcal{X}}
\def \Y {\mathcal{Y}}
\def \Z {\mathcal{Z}}

\def \a {\mathbf{a}}
\def \b {\mathbf{b}}
\def \c {\mathbf{c}}
\def \d {\mathbf{d}}
\def \e {\mathbf{e}}
\def \f {\mathbf{f}}
\def \g {\mathbf{g}}
\def \h {\mathbf{h}}
\def \m {\mathbf{m}}
\def \p {\mathbf{p}}
\def \q {\mathbf{q}}
\def \u {\mathbf{u}}
\def \w {\mathbf{w}}
\def \s {\mathbf{s}}
\def \t {\mathbf{t}}
\def \v {\mathbf{v}}
\def \x {\mathbf{x}}
\def \y {y}
\def \z {\mathbf{z}}

\def \ph {\hat{p}}

\def \fh {\hat{f}}
\def \fb {\bar{f}}
\def \ft{\tilde{f}}

\def \gh {\hat{\g}}
\def \gb {\bar{\g}}
\def \gt {\tilde{g}}

\def \uh {\hat{\u}}
\def \ub {\bar{\u}}
\def \ut{\tilde{\u}}

\def \vh {\hat{\v}}
\def \vb {\bar{\v}}
\def \vt{\tilde{\v}}

\def \xh {\hat{x}}
\def \xb {\bar{\x}}
\def \xt {\tilde{\x}}

\def \zh {\hat{\z}}
\def \zb {\bar{\z}}
\def \zt {\tilde{\z}}

\def \Ecal {\mathcal{E}}
\def \Rcal {\mathcal{R}}
\def \Ot {\tilde{\O}}
\def \indicator {\mathds{1}}
\def \regret {\mbox{Regret}}
\def \proj {\mbox{Proj}}
\def \Pr {\mathsf{Pr}}
\def \ellb {\boldsymbol{\ell}}
\def \thetah {\hat{\theta}}

\newcommand{\RegSq}{\ensuremath{\mathrm{\mathbf{Reg}}_{\mathsf{Sq}}}\xspace}
\newcommand{\RegCB}{\ensuremath{\mathrm{\mathbf{Reg}}_{\mathsf{CB}}}\xspace}
\newcommand{\RegDyn}{\ensuremath{\mathrm{\mathbf{Reg}}_{\mathsf{Dyn}}}\xspace}
\usepackage{mathtools}
\let\oldnorm\norm   % <-- Store original \norm as \oldnorm
\let\norm\undefined % <-- "Undefine" \norm
\DeclarePairedDelimiter\norm{\lVert}{\rVert}
\DeclarePairedDelimiter\abs{\lvert}{\rvert}
%\newcommand\inner[2]{\langle #1, #2 \rangle}
\newcommand*\diff{\mathop{}\!\mathrm{d}}
\newcommand*\Diff[1]{\mathop{}\!\mathrm{d^#1}}

%\DeclareMathOperator*{\Reg}{Regret}
\DeclareMathOperator*{\AReg}{A-Regret}
\DeclareMathOperator*{\WAReg}{WA-Regret}
\DeclareMathOperator*{\SAReg}{SA-Regret}
\DeclareMathOperator*{\DReg}{\mbox{D-Regret}}
\DeclareMathOperator*{\poly}{poly}
%\DeclareMathOperator*{\argmax}{arg\,max}
%\DeclareMathOperator*{\argmin}{arg\,min}

% define new theorem environments
% \let\proof\relax
% \let\endproof\relax
% \newenvironment{proof}{\par\noindent{\bf Proof\ }}{\hfill\BlackBox\\[2mm]}
% \renewcommand\qedsymbol{$\blacksquare$}
\newtheorem{myThm}{Theorem}
\newtheorem{myFact}{Fact}
\newtheorem{myClaim}{Claim}
\newtheorem{myLemma}[myThm]{Lemma}
\newtheorem{myObservation}{Observation}
\newtheorem{myProp}[myThm]{Proposition}
\newtheorem{myProperty}{Property}

% Define a custom environment for prompts
\newtcolorbox{promptbox}[1][]{
  colback=blue!5!white, colframe=blue!75!black,
  fonttitle=\bfseries, title=Prompt,
  left=2mm, right=2mm, top=2mm, bottom=2mm,
  boxrule=0.5mm,  % Thickness of the frame
  coltitle=black, % Color of the title text
  colbacktitle=blue!15!white, % Background color of the title
  breakable,      % Allows the box to break across pages
  #1
}
\newtheorem{myAssum}{Assumption}
\newtheorem{myConj}{Conjecture}
\newtheorem{myCor}{Corollary}
\newtheorem{myDef}{Definition}
\newtheorem{myExample}{Example}
\newtheorem{myNote}{Note}
\newtheorem{myProblem}{Problem}

\newtheorem{myRemark}{Remark}

% add comments
\usepackage{graphicx,color} % more modern
\newcommand{\red}{\color{red}}
\newcommand{\blue}{\color{blue}}
\definecolor{wine_red}{RGB}{228,48,64}
\definecolor{DSgray}{cmyk}{0,1,0,0}
%\newcommand{\Authornote}[2]{{\small\textcolor{NavyBlue}{\sf$<<<${  #1: #2 }$>>>$}}}
% \newcommand{\Authormarginnote}[2]{\marginpar{\parbox{2cm}{\raggedright\tiny \textcolor{DSgray}{#1: #2}}}}
% \newcommand{\pnote}[1]{{\Authornote{Peng}{#1}}}
% \newcommand{\pmarginnote}[1]{{\Authormarginnote{Peng}{#1}}}

\usepackage{prettyref}
\newcommand{\pref}[1]{\prettyref{#1}}
\newcommand{\pfref}[1]{Proof of \prettyref{#1}}
\newcommand{\savehyperref}[2]{\texorpdfstring{\hyperref[#1]{#2}}{#2}}
\newrefformat{eq}{\savehyperref{#1}{Eq. \textup{(\ref*{#1})}}}
\newrefformat{eqn}{\savehyperref{#1}{Eq.~(\ref*{#1})}}
\newrefformat{lem}{\savehyperref{#1}{Lemma~\ref*{#1}}}
\newrefformat{event}{\savehyperref{#1}{Event~\ref*{#1}}}
\newrefformat{def}{\savehyperref{#1}{Definition~\ref*{#1}}}
\newrefformat{line}{\savehyperref{#1}{Line~\ref*{#1}}}
\newrefformat{thm}{\savehyperref{#1}{Theorem~\ref*{#1}}}
\newrefformat{tab}{\savehyperref{#1}{Table~\ref*{#1}}}
\newrefformat{corr}{\savehyperref{#1}{Corollary~\ref*{#1}}}
\newrefformat{cor}{\savehyperref{#1}{Corollary~\ref*{#1}}}
\newrefformat{sec}{\savehyperref{#1}{Section~\ref*{#1}}}
\newrefformat{app}{\savehyperref{#1}{Appendix~\ref*{#1}}}
\newrefformat{assum}{\savehyperref{#1}{Assumption~\ref*{#1}}}
\newrefformat{asm}{\savehyperref{#1}{Assumption~\ref*{#1}}}
\newrefformat{ex}{\savehyperref{#1}{Example~\ref*{#1}}}
\newrefformat{fig}{\savehyperref{#1}{Figure~\ref*{#1}}}
\newrefformat{alg}{\savehyperref{#1}{Algorithm~\ref*{#1}}}
\newrefformat{rem}{\savehyperref{#1}{Remark~\ref*{#1}}}
\newrefformat{conj}{\savehyperref{#1}{Conjecture~\ref*{#1}}}
\newrefformat{prop}{\savehyperref{#1}{Proposition~\ref*{#1}}}
\newrefformat{proto}{\savehyperref{#1}{Protocol~\ref*{#1}}}
\newrefformat{prob}{\savehyperref{#1}{Problem~\ref*{#1}}}
\newrefformat{claim}{\savehyperref{#1}{Claim~\ref*{#1}}}
\newrefformat{que}{\savehyperref{#1}{Question~\ref*{#1}}}
\newrefformat{op}{\savehyperref{#1}{Open Problem~\ref*{#1}}}
\newrefformat{fn}{\savehyperref{#1}{Footnote~\ref*{#1}}}

\def \p {\boldsymbol{p}}
\def \s {\boldsymbol{s}}
\def \m {\boldsymbol{m}}
\def \epsilon {\varepsilon}

% \def \base {\mathtt{base}\mbox{-}\mathtt{regret}}
% \def \meta {\mathtt{meta}\mbox{-}\mathtt{regret}}
\def \base {\textsc{base-regret}}
\def \meta {\textsc{meta-regret}}
\def \xref {\x_{\text{ref}}}
\def \fb {\bar{f}}
\def \interior {\text{int}}
\def \yh {\hat{\y}}
\def \RegLog {\Reg_{\log}^G}
\newcommand{\bra}[1]{\left[#1\right]}
\newcommand{\pa}[1]{\left(#1\right)}
\newcommand{\hhat}{\wh{h}}
\newcommand{\epsn}{\epsilon_N}
\newcommand{\rad}{\mathsf{rad}}
\newcommand{\hatr}{\wh{r}}
\newcommand{\fl}{\underline{f}^\star}
\hypersetup{
    colorlinks,
    breaklinks,
    urlcolor = black,
    linkcolor = blue,
    citecolor = blue,
}
\usepackage{authblk}
\usepackage{graphicx}
\usepackage{subfigure}
\usepackage{footmisc}
\usepackage{caption}

\begin{document}

\title{Provably Efficient RLHF Pipeline: \\ A Unified View from Contextual Bandits}

\author{Long-Fei Li$^*$, Yu-Yang Qian$^*$, Peng Zhao, Zhi-Hua Zhou}
\affil[]{
  National Key Laboratory for Novel Software Technology, Nanjing University, China\\
  School of Artificial Intelligence, Nanjing University, China,\\
  \texttt{\{lilf, qianyy, zhaop, zhouzh\}@lamda.nju.edu.cn}
  }
\date{}

\renewcommand{\thefootnote}{\fnsymbol{footnote}}
\footnotetext[1]{Equal contribution.}
\renewcommand{\thefootnote}{\arabic{footnote}} 
\setcounter{footnote}{0}

\maketitle

\begin{abstract}
    Reinforcement Learning from Human Feedback (RLHF) is a widely used approach for aligning Large Language Models (LLMs) with human preferences. While recent advancements have provided valuable insights into various stages and settings of RLHF, a comprehensive theoretical understanding of the entire RLHF pipeline remains lacking. Towards this end, we propose a \emph{unified} framework for the RLHF pipeline from the view of \emph{contextual bandits} and provide provable efficiency guarantees. In particular, we decompose the RLHF process into two distinct stages: \mbox{(post-)training} and deployment, exploring both passive and active data collection strategies during the training phase. By employing the Bradley-Terry preference model with a linearly parameterized reward function, we reformulate RLHF as a contextual preference bandit problem. We then develop novel algorithms for each stage, demonstrating significant improvements over existing approaches in both statistical and computational efficiency. Finally, we apply our method to train and deploy \texttt{Llama-3-8B-Instruct} on the \mbox{Ultrafeedback-binarized} dataset, and empirical results confirm the effectiveness of our approach.
\end{abstract}

\section{Introduction}


\begin{figure}[t]
\centering
\includegraphics[width=0.6\columnwidth]{figures/evaluation_desiderata_V5.pdf}
\vspace{-0.5cm}
\caption{\systemName is a platform for conducting realistic evaluations of code LLMs, collecting human preferences of coding models with real users, real tasks, and in realistic environments, aimed at addressing the limitations of existing evaluations.
}
\label{fig:motivation}
\end{figure}

\begin{figure*}[t]
\centering
\includegraphics[width=\textwidth]{figures/system_design_v2.png}
\caption{We introduce \systemName, a VSCode extension to collect human preferences of code directly in a developer's IDE. \systemName enables developers to use code completions from various models. The system comprises a) the interface in the user's IDE which presents paired completions to users (left), b) a sampling strategy that picks model pairs to reduce latency (right, top), and c) a prompting scheme that allows diverse LLMs to perform code completions with high fidelity.
Users can select between the top completion (green box) using \texttt{tab} or the bottom completion (blue box) using \texttt{shift+tab}.}
\label{fig:overview}
\end{figure*}

As model capabilities improve, large language models (LLMs) are increasingly integrated into user environments and workflows.
For example, software developers code with AI in integrated developer environments (IDEs)~\citep{peng2023impact}, doctors rely on notes generated through ambient listening~\citep{oberst2024science}, and lawyers consider case evidence identified by electronic discovery systems~\citep{yang2024beyond}.
Increasing deployment of models in productivity tools demands evaluation that more closely reflects real-world circumstances~\citep{hutchinson2022evaluation, saxon2024benchmarks, kapoor2024ai}.
While newer benchmarks and live platforms incorporate human feedback to capture real-world usage, they almost exclusively focus on evaluating LLMs in chat conversations~\citep{zheng2023judging,dubois2023alpacafarm,chiang2024chatbot, kirk2024the}.
Model evaluation must move beyond chat-based interactions and into specialized user environments.



 

In this work, we focus on evaluating LLM-based coding assistants. 
Despite the popularity of these tools---millions of developers use Github Copilot~\citep{Copilot}---existing
evaluations of the coding capabilities of new models exhibit multiple limitations (Figure~\ref{fig:motivation}, bottom).
Traditional ML benchmarks evaluate LLM capabilities by measuring how well a model can complete static, interview-style coding tasks~\citep{chen2021evaluating,austin2021program,jain2024livecodebench, white2024livebench} and lack \emph{real users}. 
User studies recruit real users to evaluate the effectiveness of LLMs as coding assistants, but are often limited to simple programming tasks as opposed to \emph{real tasks}~\citep{vaithilingam2022expectation,ross2023programmer, mozannar2024realhumaneval}.
Recent efforts to collect human feedback such as Chatbot Arena~\citep{chiang2024chatbot} are still removed from a \emph{realistic environment}, resulting in users and data that deviate from typical software development processes.
We introduce \systemName to address these limitations (Figure~\ref{fig:motivation}, top), and we describe our three main contributions below.


\textbf{We deploy \systemName in-the-wild to collect human preferences on code.} 
\systemName is a Visual Studio Code extension, collecting preferences directly in a developer's IDE within their actual workflow (Figure~\ref{fig:overview}).
\systemName provides developers with code completions, akin to the type of support provided by Github Copilot~\citep{Copilot}. 
Over the past 3 months, \systemName has served over~\completions suggestions from 10 state-of-the-art LLMs, 
gathering \sampleCount~votes from \userCount~users.
To collect user preferences,
\systemName presents a novel interface that shows users paired code completions from two different LLMs, which are determined based on a sampling strategy that aims to 
mitigate latency while preserving coverage across model comparisons.
Additionally, we devise a prompting scheme that allows a diverse set of models to perform code completions with high fidelity.
See Section~\ref{sec:system} and Section~\ref{sec:deployment} for details about system design and deployment respectively.



\textbf{We construct a leaderboard of user preferences and find notable differences from existing static benchmarks and human preference leaderboards.}
In general, we observe that smaller models seem to overperform in static benchmarks compared to our leaderboard, while performance among larger models is mixed (Section~\ref{sec:leaderboard_calculation}).
We attribute these differences to the fact that \systemName is exposed to users and tasks that differ drastically from code evaluations in the past. 
Our data spans 103 programming languages and 24 natural languages as well as a variety of real-world applications and code structures, while static benchmarks tend to focus on a specific programming and natural language and task (e.g. coding competition problems).
Additionally, while all of \systemName interactions contain code contexts and the majority involve infilling tasks, a much smaller fraction of Chatbot Arena's coding tasks contain code context, with infilling tasks appearing even more rarely. 
We analyze our data in depth in Section~\ref{subsec:comparison}.



\textbf{We derive new insights into user preferences of code by analyzing \systemName's diverse and distinct data distribution.}
We compare user preferences across different stratifications of input data (e.g., common versus rare languages) and observe which affect observed preferences most (Section~\ref{sec:analysis}).
For example, while user preferences stay relatively consistent across various programming languages, they differ drastically between different task categories (e.g. frontend/backend versus algorithm design).
We also observe variations in user preference due to different features related to code structure 
(e.g., context length and completion patterns).
We open-source \systemName and release a curated subset of code contexts.
Altogether, our results highlight the necessity of model evaluation in realistic and domain-specific settings.





\section{RELATED WORK}
\label{sec:relatedwork}
In this section, we describe the previous works related to our proposal, which are divided into two parts. In Section~\ref{sec:relatedwork_exoplanet}, we present a review of approaches based on machine learning techniques for the detection of planetary transit signals. Section~\ref{sec:relatedwork_attention} provides an account of the approaches based on attention mechanisms applied in Astronomy.\par

\subsection{Exoplanet detection}
\label{sec:relatedwork_exoplanet}
Machine learning methods have achieved great performance for the automatic selection of exoplanet transit signals. One of the earliest applications of machine learning is a model named Autovetter \citep{MCcauliff}, which is a random forest (RF) model based on characteristics derived from Kepler pipeline statistics to classify exoplanet and false positive signals. Then, other studies emerged that also used supervised learning. \cite{mislis2016sidra} also used a RF, but unlike the work by \citet{MCcauliff}, they used simulated light curves and a box least square \citep[BLS;][]{kovacs2002box}-based periodogram to search for transiting exoplanets. \citet{thompson2015machine} proposed a k-nearest neighbors model for Kepler data to determine if a given signal has similarity to known transits. Unsupervised learning techniques were also applied, such as self-organizing maps (SOM), proposed \citet{armstrong2016transit}; which implements an architecture to segment similar light curves. In the same way, \citet{armstrong2018automatic} developed a combination of supervised and unsupervised learning, including RF and SOM models. In general, these approaches require a previous phase of feature engineering for each light curve. \par

%DL is a modern data-driven technology that automatically extracts characteristics, and that has been successful in classification problems from a variety of application domains. The architecture relies on several layers of NNs of simple interconnected units and uses layers to build increasingly complex and useful features by means of linear and non-linear transformation. This family of models is capable of generating increasingly high-level representations \citep{lecun2015deep}.

The application of DL for exoplanetary signal detection has evolved rapidly in recent years and has become very popular in planetary science.  \citet{pearson2018} and \citet{zucker2018shallow} developed CNN-based algorithms that learn from synthetic data to search for exoplanets. Perhaps one of the most successful applications of the DL models in transit detection was that of \citet{Shallue_2018}; who, in collaboration with Google, proposed a CNN named AstroNet that recognizes exoplanet signals in real data from Kepler. AstroNet uses the training set of labelled TCEs from the Autovetter planet candidate catalog of Q1–Q17 data release 24 (DR24) of the Kepler mission \citep{catanzarite2015autovetter}. AstroNet analyses the data in two views: a ``global view'', and ``local view'' \citep{Shallue_2018}. \par


% The global view shows the characteristics of the light curve over an orbital period, and a local view shows the moment at occurring the transit in detail

%different = space-based

Based on AstroNet, researchers have modified the original AstroNet model to rank candidates from different surveys, specifically for Kepler and TESS missions. \citet{ansdell2018scientific} developed a CNN trained on Kepler data, and included for the first time the information on the centroids, showing that the model improves performance considerably. Then, \citet{osborn2020rapid} and \citet{yu2019identifying} also included the centroids information, but in addition, \citet{osborn2020rapid} included information of the stellar and transit parameters. Finally, \citet{rao2021nigraha} proposed a pipeline that includes a new ``half-phase'' view of the transit signal. This half-phase view represents a transit view with a different time and phase. The purpose of this view is to recover any possible secondary eclipse (the object hiding behind the disk of the primary star).


%last pipeline applies a procedure after the prediction of the model to obtain new candidates, this process is carried out through a series of steps that include the evaluation with Discovery and Validation of Exoplanets (DAVE) \citet{kostov2019discovery} that was adapted for the TESS telescope.\par
%



\subsection{Attention mechanisms in astronomy}
\label{sec:relatedwork_attention}
Despite the remarkable success of attention mechanisms in sequential data, few papers have exploited their advantages in astronomy. In particular, there are no models based on attention mechanisms for detecting planets. Below we present a summary of the main applications of this modeling approach to astronomy, based on two points of view; performance and interpretability of the model.\par
%Attention mechanisms have not yet been explored in all sub-areas of astronomy. However, recent works show a successful application of the mechanism.
%performance

The application of attention mechanisms has shown improvements in the performance of some regression and classification tasks compared to previous approaches. One of the first implementations of the attention mechanism was to find gravitational lenses proposed by \citet{thuruthipilly2021finding}. They designed 21 self-attention-based encoder models, where each model was trained separately with 18,000 simulated images, demonstrating that the model based on the Transformer has a better performance and uses fewer trainable parameters compared to CNN. A novel application was proposed by \citet{lin2021galaxy} for the morphological classification of galaxies, who used an architecture derived from the Transformer, named Vision Transformer (VIT) \citep{dosovitskiy2020image}. \citet{lin2021galaxy} demonstrated competitive results compared to CNNs. Another application with successful results was proposed by \citet{zerveas2021transformer}; which first proposed a transformer-based framework for learning unsupervised representations of multivariate time series. Their methodology takes advantage of unlabeled data to train an encoder and extract dense vector representations of time series. Subsequently, they evaluate the model for regression and classification tasks, demonstrating better performance than other state-of-the-art supervised methods, even with data sets with limited samples.

%interpretation
Regarding the interpretability of the model, a recent contribution that analyses the attention maps was presented by \citet{bowles20212}, which explored the use of group-equivariant self-attention for radio astronomy classification. Compared to other approaches, this model analysed the attention maps of the predictions and showed that the mechanism extracts the brightest spots and jets of the radio source more clearly. This indicates that attention maps for prediction interpretation could help experts see patterns that the human eye often misses. \par

In the field of variable stars, \citet{allam2021paying} employed the mechanism for classifying multivariate time series in variable stars. And additionally, \citet{allam2021paying} showed that the activation weights are accommodated according to the variation in brightness of the star, achieving a more interpretable model. And finally, related to the TESS telescope, \citet{morvan2022don} proposed a model that removes the noise from the light curves through the distribution of attention weights. \citet{morvan2022don} showed that the use of the attention mechanism is excellent for removing noise and outliers in time series datasets compared with other approaches. In addition, the use of attention maps allowed them to show the representations learned from the model. \par

Recent attention mechanism approaches in astronomy demonstrate comparable results with earlier approaches, such as CNNs. At the same time, they offer interpretability of their results, which allows a post-prediction analysis. \par


\newcommand{\tabincell}[2]{\begin{tabular}{@{}#1@{}}#2\end{tabular}}
\newcommand{\rowstyle}[1]{\gdef\currentrowstyle{#1}%
	#1\ignorespaces
}

\newcommand{\className}[1]{\textbf{\sf #1}}
\newcommand{\functionName}[1]{\textbf{\sf #1}}
\newcommand{\objectName}[1]{\textbf{\sf #1}}
\newcommand{\annotation}[1]{\textbf{#1}}
\newcommand{\todo}[1]{\textcolor{blue}{\textbf{[[TODO: #1]]}}}
\newcommand{\change}[1]{\textcolor{blue}{#1}}
\newcommand{\fetch}[1]{\textbf{\em #1}}
\newcommand{\phead}[1]{\vspace{1mm} \noindent {\bf #1}}
\newcommand{\wei}[1]{\textcolor{blue}{{\it [Wei says: #1]}}}
\newcommand{\peter}[1]{\textcolor{red}{{\it [Peter says: #1]}}}
\newcommand{\zhenhao}[1]{\textcolor{dkblue}{{\it [Zhenhao says: #1]}}}
\newcommand{\feng}[1]{\textcolor{magenta}{{\it [Feng says: #1]}}}
\newcommand{\jinqiu}[1]{\textcolor{red}{{\it [Jinqiu says: #1]}}}
\newcommand{\shouvick}[1]{\textcolor{violet(ryb)}{{\it [Shouvick says: #1]}}}
\newcommand{\pattern}[1]{\emph{#1}}
%\newcommand{\tool}{{{DectGUILag}}\xspace}
\newcommand{\tool}{{{GUIWatcher}}\xspace}


\newcommand{\guo}[1]{\textcolor{yellow}{{\it [Linqiang says: #1]}}}

\newcommand{\rqbox}[1]{\begin{tcolorbox}[left=4pt,right=4pt,top=4pt,bottom=4pt,colback=gray!5,colframe=gray!40!black,before skip=2pt,after skip=2pt]#1\end{tcolorbox}}






\begin{table*}[!htbp]
 \caption{
 Detailed configuration for each training stage of the \OurMethod model.
 }
 % \vspace{-15pt}
 % \footnotesize
 \begin{center}
 \begin{adjustbox}{max width=0.99\textwidth}
 \begin{tabular}{c|c|c|c|c|c}
 \toprule
 
 
 
 \multicolumn{2}{c|}{} & Stage 1 & Stage 2 & Stage 3 & Stage 4\\

\midrule
\multicolumn{2}{c|}{Sequence Length} & 32K & 16K & 128K & 1M \\
\midrule
\multicolumn{2}{c|}{Batch Size} & 528 & 528 & 64 & 8 \\

\midrule
\multicolumn{2}{c|}{Training Iterations} & 1,000 & 5,000 & 1,000 & 500 \\
\midrule
\multicolumn{2}{c|}{Training Tokens} & 16B & 40B & 8B & 4B \\
\midrule
\multicolumn{2}{c|}{Sequence Packing} & \checkmarknew & \checkmarknew & \checkmarknew & \checkmarknew \\


\midrule
\multirow{4}{*}{Parallelism}
& \multicolumn{1}{c|}{Tensor} & 8 & 8 & 8 & 8 \\
& \multicolumn{1}{c|}{Pipeline} & 1 & 1 & 1 & 1 \\
& \multicolumn{1}{c|}{Context} & 1 & 1 & 2 & 8 \\
& \multicolumn{1}{c|}{Data} & 8 & 8 & 4 & 1 \\


\midrule
\multirow{2}{*}{Image} & Resolution & \multicolumn{4}{c}{$448 + 448 \times \{1\times2, 2\times1, \dots, 3\times4, 4\times3\} $} \\
\cmidrule{2-6}
&Max Number & 128 & 64 & 512 & 4,096 \\
\midrule
\multirow{3}{*}{Video} & Resolution & \multicolumn{4}{c}{$448$} \\
\cmidrule{2-6}
&FPS & \multicolumn{4}{c}{$1$} \\
\cmidrule{2-6}
&Max Frames & 128 & 64 & 512 & 4,096 \\
\midrule




\multirow{3}{*}{Learing Rate} & Vision & $0.0$ & $1.0 \times 10^{-6}$ & $1.0 \times 10^{-6}$ & $1.0 \times 10^{-6}$ \\
\cmidrule{2-6}
& Projector & $1.0 \times 10^{-3}$ & $1.0 \times 10^{-5}$ & $1.0 \times 10^{-5}$ & $1.0 \times 10^{-5}$ \\
\cmidrule{2-6}
& LLM & $0.0$ & $1.0 \times 10^{-5}$ & $1.0 \times 10^{-5}$ & $1.0 \times 10^{-5}$ \\
\midrule
\multirow{1}{*}{Learing Rate Decay} & Vision & $0.0$ & \multicolumn{3}{c}{$0.9$} \\
\midrule
\multicolumn{2}{c|}{\multirow{1}{*}{Learing Rate scheduler}} & \multicolumn{4}{c}{Cosine} \\
\midrule
\multicolumn{2}{c|}{Weight Decay} & \multicolumn{4}{c}{$0.0$} \\
\midrule
\multicolumn{2}{c|}{Gradient Clip} & \multicolumn{4}{c}{$1.0$} \\
\midrule
\multicolumn{2}{c|}{Rotary Base} & \multicolumn{4}{c}{$1,000,000$} \\
\midrule
\multicolumn{2}{c|}{Adam Beta1} & \multicolumn{4}{c}{$0.9$} \\
\midrule
\multicolumn{2}{c|}{Adam Beta2} & \multicolumn{4}{c}{$0.999$} \\



 \bottomrule%==============================================================================================================
 \end{tabular}
 \end{adjustbox}
 \end{center}
 \label{table_train}
 % \vspace{-20pt}
\end{table*}
\section{Deployment Stage}
\label{sec:deploy}
In the deployment stage, the learner uses the trained model to generate responses to user inputs and simultaneously gathering informative feedback to improve the model. This contrasts with the active data collection phase during training, which is focused solely on optimizing the final policy. In deployment, however, the learner is faced with a dual objective: selecting actions that maximize rewards to ensure a positive user experience, while also choosing actions that provide informative feedback for further model improvement.

\paragraph{Performance Measure.} To this end, we consider the cumulative regret compared to the optimal policy, which is defined as 
\begin{align*}
  {\Reg}_T = \sum_{t=1}^T \left(r\left(x_t, \pi^*(x_t)\right)- \frac{r\left(x_t, a_t\right) + r\left(x_t, a'_t\right)}{2}\right),
\end{align*}
where $\pi^*$ is the optimal policy, $a_t$ and $a'_t$ are the actions selected by the learner at round $t$. This measure is stronger than the measure defined in~\citet{arXiv'24:Ji-RLHF-active}, which considers the gap between the optimal action and a single selected action. In the deployment phase, our proposed measure is more appropriate, as both actions must be sufficiently effective to ensure a positive user experience.

While our measure is more suitable for deployment, it also introduces significant challenges. To optimize the measure of \citet{arXiv'24:Ji-RLHF-active}, the learner can exploit one action with high reward for exploitation while using the other action for exploration. In contrast, our setting requires both actions to be good enough. To address this, we propose a novel algorithm that balances exploration and exploitation for deployment. At a high level, our algorithm consists of two key components: estimator construction and query selection. We describe each component in detail below.

\paragraph{Estimator Construction.} The estimator construction is the same as the process used in the training stage. For each round $t\in[T]$, given $\left\{\left(x_i, a_i, a_i^{\prime}, y_i\right)\right\}_{i=1}^{t-1}$, we construct the estimator $\thetat_t$ by OMD in an one-pass manner as in Eq.~\eqref{eq:omd}. Specifically, the estimator is updated by
\begin{align}
  \label{eq:omd-deploy}
  \thetat_{t+1}=\argmin_{\theta \in \Theta} \Big\{\big\langle g_t(\thetat_t), \theta \big\rangle +\frac{1}{2 \eta}\big\|\theta-\thetat_t\big\|_{\Ht_t}^2\Big\}.
\end{align}

\paragraph{Query Selection.} Unlike in the training stage, during deployment, the learner must balance exploration and exploitation to minimize regret. Specifically, the learner must select queries that are both informative and likely to yield high rewards. Focusing solely on high-reward queries may cause the learner to miss valuable feedback for model improvement, while prioritizing only informative queries could prevent reward maximization. To address this, we propose a novel strategy inspired by \citet{NeurIPS'21:Saha-Preference-bandits}, which accounts for both reward maximization and uncertainty maximization.

Based on the estimator $\theta_t$, we first define the promising set $\A_t$ that only contains the actions with potential to be the best action. Specifically, we define the promising set as
\begin{align}
    \A_t = \big \{ a \in \A \mid (\phi(x_t, a) - \phi(x_t, a'))^{\top} \thetat_t + \betat_t \norm{\phi(x_t, a) - \phi(x_t, a')}_{\H_t^{-1}} \geq 0, \forall a' \in \A \big\}.\label{eq:promising-set}
\end{align}
Then, we select the action pairs with the maximum uncertainty in the promising set, i.e.,
\begin{align}
    \label{eq:query-selection-deploy}
    (a_t, a_t') = \argmax_{(a, a') \in \A_t \times \A_t} \left\|\phi(x_t, a)-\phi(x_t, a')\right\|_{\H_t^{-1}}.
\end{align}
The overall algorithm is summarized in Algorithm~\ref{alg:deploy}. We show it enjoys the following regret bound.


\begin{algorithm}[!t]
    \caption{Deployment Stage}
    \label{alg:deploy}
    \begin{algorithmic}[1]
        \REQUIRE Regularization $\lambda>0$, failure probability $\delta \in(0,1]$
        \STATE Initialize $\thetat_1=0$, $\H_1 = \lambda I_d$.
        \FOR {$t=1,2, \ldots, T$}
        \STATE Observes context $x_t$.
        \STATE Compute the promising set by Eq.~\eqref{eq:promising-set}
        \STATE Choose duel actions by Eq.~\eqref{eq:query-selection-deploy}.
        \STATE Observe preference feedback $y_t$
        \STATE Update $\Ht_t = \H_t + \eta H_t(\thetat_t)$
        \STATE Compute $\thetat_{t+1}$ by Eq.~\eqref{eq:omd-deploy}
        \STATE Update $\H_{t+1} = \H_{t} + H_t(\thetat_{t+1})$
        \ENDFOR
    \end{algorithmic}
\end{algorithm}

\begin{myThm}
    \label{thm:deploy}
    Let $\delta \in(0,1]$, set parameters as in Lemma~\ref{lem:confidence_set}, with probability at least $1-\delta$, Algorithm~\ref{alg:deploy} enjoys the following regret guarantee:
    \begin{align*}
      {\Reg}_T \leq \Ot \big(d\sqrt{{\kappa}{T}}\big).
    \end{align*}
\end{myThm}

\paragraph{Comparison with~\citet{NeurIPS'21:Saha-Preference-bandits}.} Our result improves upon theirs in both computational and statistical efficiency. For statistical efficiency, our algorithm achieves a regret of $\Ot \big(d\sqrt{{\kappa}{T}}\big)$, which is tighter by a factor of $\sqrt{\kappa}$ compared to their $\Ot \big(d{\kappa}\sqrt{{T}}\big)$ result. For computational efficiency, our algorithm updates in an online manner, with $\O(1)$ storage and $\O(T)$ time complexity for the total $T$ rounds. In contrast, their algorithm requires storing all the data and solving an optimization problem each round, leading to $\O(T)$ storage and $\O(T^2 \log T)$ time complexity. As $T$ is large or even infinite during deployment, our algorithm has a significant advantage in computational efficiency. 

\section{Experiments}
\label{sec:exp}
Following the settings in Section \ref{sec:existing}, we evaluate \textit{NovelSum}'s correlation with the fine-tuned model performance across 53 IT datasets and compare it with previous diversity metrics. Additionally, we conduct a correlation analysis using Qwen-2.5-7B \cite{yang2024qwen2} as the backbone model, alongside previous LLaMA-3-8B experiments, to further demonstrate the metric's effectiveness across different scenarios. Qwen is used for both instruction tuning and deriving semantic embeddings. Due to resource constraints, we run each strategy on Qwen for two rounds, resulting in 25 datasets. 

\subsection{Main Results}

\begin{table*}[!t]
    \centering
    \resizebox{\linewidth}{!}{
    \begin{tabular}{lcccccccccc}
    \toprule
    \multirow{3}*{\textbf{Diversity Metrics}} & \multicolumn{10}{c}{\textbf{Data Selection Strategies}} \\
    \cmidrule(lr){2-11}
    & \multirow{2}*{\textbf{K-means}} & \multirow{2}*{\vtop{\hbox{\textbf{K-Center}}\vspace{1mm}\hbox{\textbf{-Greedy}}}}  & \multirow{2}*{\textbf{QDIT}} & \multirow{2}*{\vtop{\hbox{\textbf{Repr}}\vspace{1mm}\hbox{\textbf{Filter}}}} & \multicolumn{5}{c}{\textbf{Random}} & \multirow{2}{*}{\textbf{Duplicate}} \\ 
    \cmidrule(lr){6-10}
    & & & & & \textbf{$\mathcal{X}^{all}$} & ShareGPT & WizardLM & Alpaca & Dolly &  \\
    \midrule
    \rowcolor{gray!15} \multicolumn{11}{c}{\textit{LLaMA-3-8B}} \\
    Facility Loc. $_{\times10^5}$ & \cellcolor{BLUE!40} 2.99 & \cellcolor{ORANGE!10} 2.73 & \cellcolor{BLUE!40} 2.99 & \cellcolor{BLUE!20} 2.86 & \cellcolor{BLUE!40} 2.99 & \cellcolor{BLUE!0} 2.83 & \cellcolor{BLUE!30} 2.88 & \cellcolor{BLUE!0} 2.83 & \cellcolor{ORANGE!20} 2.59 & \cellcolor{ORANGE!30} 2.52 \\    
    DistSum$_{cosine}$  & \cellcolor{BLUE!30} 0.648 & \cellcolor{BLUE!60} 0.746 & \cellcolor{BLUE!0} 0.629 & \cellcolor{BLUE!50} 0.703 & \cellcolor{BLUE!10} 0.634 & \cellcolor{BLUE!40} 0.656 & \cellcolor{ORANGE!30} 0.578 & \cellcolor{ORANGE!10} 0.605 & \cellcolor{ORANGE!20} 0.603 & \cellcolor{BLUE!10} 0.634 \\
    Vendi Score $_{\times10^7}$ & \cellcolor{BLUE!30} 1.70 & \cellcolor{BLUE!60} 2.53 & \cellcolor{BLUE!10} 1.59 & \cellcolor{BLUE!50} 2.23 & \cellcolor{BLUE!20} 1.61 & \cellcolor{BLUE!30} 1.70 & \cellcolor{ORANGE!10} 1.44 & \cellcolor{ORANGE!20} 1.32 & \cellcolor{ORANGE!10} 1.44 & \cellcolor{ORANGE!30} 0.05 \\
    \textbf{NovelSum (Ours)} & \cellcolor{BLUE!60} 0.693 & \cellcolor{BLUE!50} 0.687 & \cellcolor{BLUE!30} 0.673 & \cellcolor{BLUE!20} 0.671 & \cellcolor{BLUE!40} 0.675 & \cellcolor{BLUE!10} 0.628 & \cellcolor{BLUE!0} 0.591 & \cellcolor{ORANGE!10} 0.572 & \cellcolor{ORANGE!20} 0.50 & \cellcolor{ORANGE!30} 0.461 \\
    \midrule    
    \textbf{Model Performance} & \cellcolor{BLUE!60}1.32 & \cellcolor{BLUE!50}1.31 & \cellcolor{BLUE!40}1.25 & \cellcolor{BLUE!30}1.05 & \cellcolor{BLUE!20}1.20 & \cellcolor{BLUE!10}0.83 & \cellcolor{BLUE!0}0.72 & \cellcolor{ORANGE!10}0.07 & \cellcolor{ORANGE!20}-0.14 & \cellcolor{ORANGE!30}-1.35 \\
    \midrule
    \midrule
    \rowcolor{gray!15} \multicolumn{11}{c}{\textit{Qwen-2.5-7B}} \\
    Facility Loc. $_{\times10^5}$ & \cellcolor{BLUE!40} 3.54 & \cellcolor{ORANGE!30} 3.42 & \cellcolor{BLUE!40} 3.54 & \cellcolor{ORANGE!20} 3.46 & \cellcolor{BLUE!40} 3.54 & \cellcolor{BLUE!30} 3.51 & \cellcolor{BLUE!10} 3.50 & \cellcolor{BLUE!10} 3.50 & \cellcolor{ORANGE!20} 3.46 & \cellcolor{BLUE!0} 3.48 \\ 
    DistSum$_{cosine}$ & \cellcolor{BLUE!30} 0.260 & \cellcolor{BLUE!60} 0.440 & \cellcolor{BLUE!0} 0.223 & \cellcolor{BLUE!50} 0.421 & \cellcolor{BLUE!10} 0.230 & \cellcolor{BLUE!40} 0.285 & \cellcolor{ORANGE!20} 0.211 & \cellcolor{ORANGE!30} 0.189 & \cellcolor{ORANGE!10} 0.221 & \cellcolor{BLUE!20} 0.243 \\
    Vendi Score $_{\times10^6}$ & \cellcolor{ORANGE!10} 1.60 & \cellcolor{BLUE!40} 3.09 & \cellcolor{BLUE!10} 2.60 & \cellcolor{BLUE!60} 7.15 & \cellcolor{ORANGE!20} 1.41 & \cellcolor{BLUE!50} 3.36 & \cellcolor{BLUE!20} 2.65 & \cellcolor{BLUE!0} 1.89 & \cellcolor{BLUE!30} 3.04 & \cellcolor{ORANGE!30} 0.20 \\
    \textbf{NovelSum (Ours)}  & \cellcolor{BLUE!40} 0.440 & \cellcolor{BLUE!60} 0.505 & \cellcolor{BLUE!20} 0.403 & \cellcolor{BLUE!50} 0.495 & \cellcolor{BLUE!30} 0.408 & \cellcolor{BLUE!10} 0.392 & \cellcolor{BLUE!0} 0.349 & \cellcolor{ORANGE!10} 0.336 & \cellcolor{ORANGE!20} 0.320 & \cellcolor{ORANGE!30} 0.309 \\
    \midrule
    \textbf{Model Performance} & \cellcolor{BLUE!30} 1.06 & \cellcolor{BLUE!60} 1.45 & \cellcolor{BLUE!40} 1.23 & \cellcolor{BLUE!50} 1.35 & \cellcolor{BLUE!20} 0.87 & \cellcolor{BLUE!10} 0.07 & \cellcolor{BLUE!0} -0.08 & \cellcolor{ORANGE!10} -0.38 & \cellcolor{ORANGE!30} -0.49 & \cellcolor{ORANGE!20} -0.43 \\
    \bottomrule
    \end{tabular}
    }
    \caption{Measuring the diversity of datasets selected by different strategies using \textit{NovelSum} and baseline metrics. Fine-tuned model performances (Eq. \ref{eq:perf}), based on MT-bench and AlpacaEval, are also included for cross reference. Darker \colorbox{BLUE!60}{blue} shades indicate higher values for each metric, while darker \colorbox{ORANGE!30}{orange} shades indicate lower values. While data selection strategies vary in performance on LLaMA-3-8B and Qwen-2.5-7B, \textit{NovelSum} consistently shows a stronger correlation with model performance than other metrics. More results are provided in Appendix \ref{app:results}.}
    \label{tbl:main}
    \vspace{-4mm}
\end{table*}


\begin{table}[t!]
\centering
\resizebox{\linewidth}{!}{
\begin{tabular}{lcccc}
\toprule
\multirow{2}*{\textbf{Diversity Metrics}} & \multicolumn{3}{c}{\textbf{LLaMA}} & \textbf{Qwen}\\
\cmidrule(lr){2-4} \cmidrule(lr){5-5} 
& \textbf{Pearson} & \textbf{Spearman} & \textbf{Avg.} & \textbf{Avg.} \\
\midrule
TTR & -0.38 & -0.16 & -0.27 & -0.30 \\
vocd-D & -0.43 & -0.17 & -0.30 & -0.31 \\
\midrule
Facility Loc. & 0.86 & 0.69 & 0.77 & 0.08 \\
Entropy & 0.93 & 0.80 & 0.86 & 0.63 \\
\midrule
LDD & 0.61 & 0.75 & 0.68 & 0.60 \\
KNN Distance & 0.59 & 0.80 & 0.70 & 0.67 \\
DistSum$_{cosine}$ & 0.85 & 0.67 & 0.76 & 0.51 \\
Vendi Score & 0.70 & 0.85 & 0.78 & 0.60 \\
DistSum$_{L2}$ & 0.86 & 0.76 & 0.81 & 0.51 \\
Cluster Inertia & 0.81 & 0.85 & 0.83 & 0.76 \\
Radius & 0.87 & 0.81 & 0.84 & 0.48 \\
\midrule
NovelSum & \textbf{0.98} & \textbf{0.95} & \textbf{0.97} & \textbf{0.90} \\
\bottomrule
\end{tabular}
}
\caption{Correlations between different metrics and model performance on LLaMA-3-8B and Qwen-2.5-7B.  “Avg.” denotes the average correlation (Eq. \ref{eq:cor}).}
\label{tbl:correlations}
\vspace{-2mm}
\end{table}

\paragraph{\textit{NovelSum} consistently achieves state-of-the-art correlation with model performance across various data selection strategies, backbone LLMs, and correlation measures.}
Table \ref{tbl:main} presents diversity measurement results on datasets constructed by mainstream data selection methods (based on $\mathcal{X}^{all}$), random selection from various sources, and duplicated samples (with only $m=100$ unique samples). 
Results from multiple runs are averaged for each strategy.
Although these strategies yield varying performance rankings across base models, \textit{NovelSum} consistently tracks changes in IT performance by accurately measuring dataset diversity. For instance, K-means achieves the best performance on LLaMA with the highest NovelSum score, while K-Center-Greedy excels on Qwen, also correlating with the highest NovelSum. Table \ref{tbl:correlations} shows the correlation coefficients between various metrics and model performance for both LLaMA and Qwen experiments, where \textit{NovelSum} achieves state-of-the-art correlation across different models and measures.

\paragraph{\textit{NovelSum} can provide valuable guidance for data engineering practices.}
As a reliable indicator of data diversity, \textit{NovelSum} can assess diversity at both the dataset and sample levels, directly guiding data selection and construction decisions. For example, Table \ref{tbl:main} shows that the combined data source $\mathcal{X}^{all}$ is a better choice for sampling diverse IT data than other sources. Moreover, \textit{NovelSum} can offer insights through comparative analyses, such as: (1) ShareGPT, which collects data from real internet users, exhibits greater diversity than Dolly, which relies on company employees, suggesting that IT samples from diverse sources enhance dataset diversity \cite{wang2024diversity-logD}; (2) In LLaMA experiments, random selection can outperform some mainstream strategies, aligning with prior work \cite{xia2024rethinking,diddee2024chasing}, highlighting gaps in current data selection methods for optimizing diversity.



\subsection{Ablation Study}


\textit{NovelSum} involves several flexible hyperparameters and variations. In our main experiments, \textit{NovelSum} uses cosine distance to compute $d(x_i, x_j)$ in Eq. \ref{eq:dad}. We set $\alpha = 1$, $\beta = 0.5$, and $K = 10$ nearest neighbors in Eq. \ref{eq:pws} and \ref{eq:dad}. Here, we conduct an ablation study to investigate the impact of these settings based on LLaMA-3-8B.

\begin{table}[ht!]
\centering
\resizebox{\linewidth}{!}{
\begin{tabular}{lccc}
\toprule
\textbf{Variants} & \textbf{Pearson} & \textbf{Spearman} & \textbf{Avg.} \\
\midrule
NovelSum & 0.98 & 0.96 & 0.97 \\
\midrule
\hspace{0.10cm} - Use $L2$ distance & 0.97 & 0.83 & 0.90\textsubscript{↓ 0.08} \\
\hspace{0.10cm} - $K=20$ & 0.98 & 0.96 & 0.97\textsubscript{↓ 0.00} \\
\hspace{0.10cm} - $\alpha=0$ (w/o proximity) & 0.79 & 0.31 & 0.55\textsubscript{↓ 0.42} \\
\hspace{0.10cm} - $\alpha=2$ & 0.73 & 0.88 & 0.81\textsubscript{↓ 0.16} \\
\hspace{0.10cm} - $\beta=0$ (w/o density) & 0.92 & 0.89 & 0.91\textsubscript{↓ 0.07} \\
\hspace{0.10cm} - $\beta=1$ & 0.90 & 0.62 & 0.76\textsubscript{↓ 0.21} \\
\bottomrule
\end{tabular}
}
\caption{Ablation Study for \textit{NovelSum}.}
\label{tbl:ablation}
\vspace{-2mm}
\end{table}

In Table \ref{tbl:ablation}, $\alpha=0$ removes the proximity weights, and $\beta=0$ eliminates the density multiplier. We observe that both $\alpha=0$ and $\beta=0$ significantly weaken the correlation, validating the benefits of the proximity-weighted sum and density-aware distance. Additionally, improper values for $\alpha$ and $\beta$ greatly reduce the metric's reliability, highlighting that \textit{NovelSum} strikes a delicate balance between distances and distribution. Replacing cosine distance with Euclidean distance and using more neighbors for density approximation have minimal impact, particularly on Pearson's correlation, demonstrating \textit{NovelSum}'s robustness to different distance measures.






\section{Conclusion}
In this work, we propose a simple yet effective approach, called SMILE, for graph few-shot learning with fewer tasks. Specifically, we introduce a novel dual-level mixup strategy, including within-task and across-task mixup, for enriching the diversity of nodes within each task and the diversity of tasks. Also, we incorporate the degree-based prior information to learn expressive node embeddings. Theoretically, we prove that SMILE effectively enhances the model's generalization performance. Empirically, we conduct extensive experiments on multiple benchmarks and the results suggest that SMILE significantly outperforms other baselines, including both in-domain and cross-domain few-shot settings.


% \newpage
\bibliographystyle{plainnat}
\bibliography{rlhf}

\newpage
\appendix
\subsection{Lloyd-Max Algorithm}
\label{subsec:Lloyd-Max}
For a given quantization bitwidth $B$ and an operand $\bm{X}$, the Lloyd-Max algorithm finds $2^B$ quantization levels $\{\hat{x}_i\}_{i=1}^{2^B}$ such that quantizing $\bm{X}$ by rounding each scalar in $\bm{X}$ to the nearest quantization level minimizes the quantization MSE. 

The algorithm starts with an initial guess of quantization levels and then iteratively computes quantization thresholds $\{\tau_i\}_{i=1}^{2^B-1}$ and updates quantization levels $\{\hat{x}_i\}_{i=1}^{2^B}$. Specifically, at iteration $n$, thresholds are set to the midpoints of the previous iteration's levels:
\begin{align*}
    \tau_i^{(n)}=\frac{\hat{x}_i^{(n-1)}+\hat{x}_{i+1}^{(n-1)}}2 \text{ for } i=1\ldots 2^B-1
\end{align*}
Subsequently, the quantization levels are re-computed as conditional means of the data regions defined by the new thresholds:
\begin{align*}
    \hat{x}_i^{(n)}=\mathbb{E}\left[ \bm{X} \big| \bm{X}\in [\tau_{i-1}^{(n)},\tau_i^{(n)}] \right] \text{ for } i=1\ldots 2^B
\end{align*}
where to satisfy boundary conditions we have $\tau_0=-\infty$ and $\tau_{2^B}=\infty$. The algorithm iterates the above steps until convergence.

Figure \ref{fig:lm_quant} compares the quantization levels of a $7$-bit floating point (E3M3) quantizer (left) to a $7$-bit Lloyd-Max quantizer (right) when quantizing a layer of weights from the GPT3-126M model at a per-tensor granularity. As shown, the Lloyd-Max quantizer achieves substantially lower quantization MSE. Further, Table \ref{tab:FP7_vs_LM7} shows the superior perplexity achieved by Lloyd-Max quantizers for bitwidths of $7$, $6$ and $5$. The difference between the quantizers is clear at 5 bits, where per-tensor FP quantization incurs a drastic and unacceptable increase in perplexity, while Lloyd-Max quantization incurs a much smaller increase. Nevertheless, we note that even the optimal Lloyd-Max quantizer incurs a notable ($\sim 1.5$) increase in perplexity due to the coarse granularity of quantization. 

\begin{figure}[h]
  \centering
  \includegraphics[width=0.7\linewidth]{sections/figures/LM7_FP7.pdf}
  \caption{\small Quantization levels and the corresponding quantization MSE of Floating Point (left) vs Lloyd-Max (right) Quantizers for a layer of weights in the GPT3-126M model.}
  \label{fig:lm_quant}
\end{figure}

\begin{table}[h]\scriptsize
\begin{center}
\caption{\label{tab:FP7_vs_LM7} \small Comparing perplexity (lower is better) achieved by floating point quantizers and Lloyd-Max quantizers on a GPT3-126M model for the Wikitext-103 dataset.}
\begin{tabular}{c|cc|c}
\hline
 \multirow{2}{*}{\textbf{Bitwidth}} & \multicolumn{2}{|c|}{\textbf{Floating-Point Quantizer}} & \textbf{Lloyd-Max Quantizer} \\
 & Best Format & Wikitext-103 Perplexity & Wikitext-103 Perplexity \\
\hline
7 & E3M3 & 18.32 & 18.27 \\
6 & E3M2 & 19.07 & 18.51 \\
5 & E4M0 & 43.89 & 19.71 \\
\hline
\end{tabular}
\end{center}
\end{table}

\subsection{Proof of Local Optimality of LO-BCQ}
\label{subsec:lobcq_opt_proof}
For a given block $\bm{b}_j$, the quantization MSE during LO-BCQ can be empirically evaluated as $\frac{1}{L_b}\lVert \bm{b}_j- \bm{\hat{b}}_j\rVert^2_2$ where $\bm{\hat{b}}_j$ is computed from equation (\ref{eq:clustered_quantization_definition}) as $C_{f(\bm{b}_j)}(\bm{b}_j)$. Further, for a given block cluster $\mathcal{B}_i$, we compute the quantization MSE as $\frac{1}{|\mathcal{B}_{i}|}\sum_{\bm{b} \in \mathcal{B}_{i}} \frac{1}{L_b}\lVert \bm{b}- C_i^{(n)}(\bm{b})\rVert^2_2$. Therefore, at the end of iteration $n$, we evaluate the overall quantization MSE $J^{(n)}$ for a given operand $\bm{X}$ composed of $N_c$ block clusters as:
\begin{align*}
    \label{eq:mse_iter_n}
    J^{(n)} = \frac{1}{N_c} \sum_{i=1}^{N_c} \frac{1}{|\mathcal{B}_{i}^{(n)}|}\sum_{\bm{v} \in \mathcal{B}_{i}^{(n)}} \frac{1}{L_b}\lVert \bm{b}- B_i^{(n)}(\bm{b})\rVert^2_2
\end{align*}

At the end of iteration $n$, the codebooks are updated from $\mathcal{C}^{(n-1)}$ to $\mathcal{C}^{(n)}$. However, the mapping of a given vector $\bm{b}_j$ to quantizers $\mathcal{C}^{(n)}$ remains as  $f^{(n)}(\bm{b}_j)$. At the next iteration, during the vector clustering step, $f^{(n+1)}(\bm{b}_j)$ finds new mapping of $\bm{b}_j$ to updated codebooks $\mathcal{C}^{(n)}$ such that the quantization MSE over the candidate codebooks is minimized. Therefore, we obtain the following result for $\bm{b}_j$:
\begin{align*}
\frac{1}{L_b}\lVert \bm{b}_j - C_{f^{(n+1)}(\bm{b}_j)}^{(n)}(\bm{b}_j)\rVert^2_2 \le \frac{1}{L_b}\lVert \bm{b}_j - C_{f^{(n)}(\bm{b}_j)}^{(n)}(\bm{b}_j)\rVert^2_2
\end{align*}

That is, quantizing $\bm{b}_j$ at the end of the block clustering step of iteration $n+1$ results in lower quantization MSE compared to quantizing at the end of iteration $n$. Since this is true for all $\bm{b} \in \bm{X}$, we assert the following:
\begin{equation}
\begin{split}
\label{eq:mse_ineq_1}
    \tilde{J}^{(n+1)} &= \frac{1}{N_c} \sum_{i=1}^{N_c} \frac{1}{|\mathcal{B}_{i}^{(n+1)}|}\sum_{\bm{b} \in \mathcal{B}_{i}^{(n+1)}} \frac{1}{L_b}\lVert \bm{b} - C_i^{(n)}(b)\rVert^2_2 \le J^{(n)}
\end{split}
\end{equation}
where $\tilde{J}^{(n+1)}$ is the the quantization MSE after the vector clustering step at iteration $n+1$.

Next, during the codebook update step (\ref{eq:quantizers_update}) at iteration $n+1$, the per-cluster codebooks $\mathcal{C}^{(n)}$ are updated to $\mathcal{C}^{(n+1)}$ by invoking the Lloyd-Max algorithm \citep{Lloyd}. We know that for any given value distribution, the Lloyd-Max algorithm minimizes the quantization MSE. Therefore, for a given vector cluster $\mathcal{B}_i$ we obtain the following result:

\begin{equation}
    \frac{1}{|\mathcal{B}_{i}^{(n+1)}|}\sum_{\bm{b} \in \mathcal{B}_{i}^{(n+1)}} \frac{1}{L_b}\lVert \bm{b}- C_i^{(n+1)}(\bm{b})\rVert^2_2 \le \frac{1}{|\mathcal{B}_{i}^{(n+1)}|}\sum_{\bm{b} \in \mathcal{B}_{i}^{(n+1)}} \frac{1}{L_b}\lVert \bm{b}- C_i^{(n)}(\bm{b})\rVert^2_2
\end{equation}

The above equation states that quantizing the given block cluster $\mathcal{B}_i$ after updating the associated codebook from $C_i^{(n)}$ to $C_i^{(n+1)}$ results in lower quantization MSE. Since this is true for all the block clusters, we derive the following result: 
\begin{equation}
\begin{split}
\label{eq:mse_ineq_2}
     J^{(n+1)} &= \frac{1}{N_c} \sum_{i=1}^{N_c} \frac{1}{|\mathcal{B}_{i}^{(n+1)}|}\sum_{\bm{b} \in \mathcal{B}_{i}^{(n+1)}} \frac{1}{L_b}\lVert \bm{b}- C_i^{(n+1)}(\bm{b})\rVert^2_2  \le \tilde{J}^{(n+1)}   
\end{split}
\end{equation}

Following (\ref{eq:mse_ineq_1}) and (\ref{eq:mse_ineq_2}), we find that the quantization MSE is non-increasing for each iteration, that is, $J^{(1)} \ge J^{(2)} \ge J^{(3)} \ge \ldots \ge J^{(M)}$ where $M$ is the maximum number of iterations. 
%Therefore, we can say that if the algorithm converges, then it must be that it has converged to a local minimum. 
\hfill $\blacksquare$


\begin{figure}
    \begin{center}
    \includegraphics[width=0.5\textwidth]{sections//figures/mse_vs_iter.pdf}
    \end{center}
    \caption{\small NMSE vs iterations during LO-BCQ compared to other block quantization proposals}
    \label{fig:nmse_vs_iter}
\end{figure}

Figure \ref{fig:nmse_vs_iter} shows the empirical convergence of LO-BCQ across several block lengths and number of codebooks. Also, the MSE achieved by LO-BCQ is compared to baselines such as MXFP and VSQ. As shown, LO-BCQ converges to a lower MSE than the baselines. Further, we achieve better convergence for larger number of codebooks ($N_c$) and for a smaller block length ($L_b$), both of which increase the bitwidth of BCQ (see Eq \ref{eq:bitwidth_bcq}).


\subsection{Additional Accuracy Results}
%Table \ref{tab:lobcq_config} lists the various LOBCQ configurations and their corresponding bitwidths.
\begin{table}
\setlength{\tabcolsep}{4.75pt}
\begin{center}
\caption{\label{tab:lobcq_config} Various LO-BCQ configurations and their bitwidths.}
\begin{tabular}{|c||c|c|c|c||c|c||c|} 
\hline
 & \multicolumn{4}{|c||}{$L_b=8$} & \multicolumn{2}{|c||}{$L_b=4$} & $L_b=2$ \\
 \hline
 \backslashbox{$L_A$\kern-1em}{\kern-1em$N_c$} & 2 & 4 & 8 & 16 & 2 & 4 & 2 \\
 \hline
 64 & 4.25 & 4.375 & 4.5 & 4.625 & 4.375 & 4.625 & 4.625\\
 \hline
 32 & 4.375 & 4.5 & 4.625& 4.75 & 4.5 & 4.75 & 4.75 \\
 \hline
 16 & 4.625 & 4.75& 4.875 & 5 & 4.75 & 5 & 5 \\
 \hline
\end{tabular}
\end{center}
\end{table}

%\subsection{Perplexity achieved by various LO-BCQ configurations on Wikitext-103 dataset}

\begin{table} \centering
\begin{tabular}{|c||c|c|c|c||c|c||c|} 
\hline
 $L_b \rightarrow$& \multicolumn{4}{c||}{8} & \multicolumn{2}{c||}{4} & 2\\
 \hline
 \backslashbox{$L_A$\kern-1em}{\kern-1em$N_c$} & 2 & 4 & 8 & 16 & 2 & 4 & 2  \\
 %$N_c \rightarrow$ & 2 & 4 & 8 & 16 & 2 & 4 & 2 \\
 \hline
 \hline
 \multicolumn{8}{c}{GPT3-1.3B (FP32 PPL = 9.98)} \\ 
 \hline
 \hline
 64 & 10.40 & 10.23 & 10.17 & 10.15 &  10.28 & 10.18 & 10.19 \\
 \hline
 32 & 10.25 & 10.20 & 10.15 & 10.12 &  10.23 & 10.17 & 10.17 \\
 \hline
 16 & 10.22 & 10.16 & 10.10 & 10.09 &  10.21 & 10.14 & 10.16 \\
 \hline
  \hline
 \multicolumn{8}{c}{GPT3-8B (FP32 PPL = 7.38)} \\ 
 \hline
 \hline
 64 & 7.61 & 7.52 & 7.48 &  7.47 &  7.55 &  7.49 & 7.50 \\
 \hline
 32 & 7.52 & 7.50 & 7.46 &  7.45 &  7.52 &  7.48 & 7.48  \\
 \hline
 16 & 7.51 & 7.48 & 7.44 &  7.44 &  7.51 &  7.49 & 7.47  \\
 \hline
\end{tabular}
\caption{\label{tab:ppl_gpt3_abalation} Wikitext-103 perplexity across GPT3-1.3B and 8B models.}
\end{table}

\begin{table} \centering
\begin{tabular}{|c||c|c|c|c||} 
\hline
 $L_b \rightarrow$& \multicolumn{4}{c||}{8}\\
 \hline
 \backslashbox{$L_A$\kern-1em}{\kern-1em$N_c$} & 2 & 4 & 8 & 16 \\
 %$N_c \rightarrow$ & 2 & 4 & 8 & 16 & 2 & 4 & 2 \\
 \hline
 \hline
 \multicolumn{5}{|c|}{Llama2-7B (FP32 PPL = 5.06)} \\ 
 \hline
 \hline
 64 & 5.31 & 5.26 & 5.19 & 5.18  \\
 \hline
 32 & 5.23 & 5.25 & 5.18 & 5.15  \\
 \hline
 16 & 5.23 & 5.19 & 5.16 & 5.14  \\
 \hline
 \multicolumn{5}{|c|}{Nemotron4-15B (FP32 PPL = 5.87)} \\ 
 \hline
 \hline
 64  & 6.3 & 6.20 & 6.13 & 6.08  \\
 \hline
 32  & 6.24 & 6.12 & 6.07 & 6.03  \\
 \hline
 16  & 6.12 & 6.14 & 6.04 & 6.02  \\
 \hline
 \multicolumn{5}{|c|}{Nemotron4-340B (FP32 PPL = 3.48)} \\ 
 \hline
 \hline
 64 & 3.67 & 3.62 & 3.60 & 3.59 \\
 \hline
 32 & 3.63 & 3.61 & 3.59 & 3.56 \\
 \hline
 16 & 3.61 & 3.58 & 3.57 & 3.55 \\
 \hline
\end{tabular}
\caption{\label{tab:ppl_llama7B_nemo15B} Wikitext-103 perplexity compared to FP32 baseline in Llama2-7B and Nemotron4-15B, 340B models}
\end{table}

%\subsection{Perplexity achieved by various LO-BCQ configurations on MMLU dataset}


\begin{table} \centering
\begin{tabular}{|c||c|c|c|c||c|c|c|c|} 
\hline
 $L_b \rightarrow$& \multicolumn{4}{c||}{8} & \multicolumn{4}{c||}{8}\\
 \hline
 \backslashbox{$L_A$\kern-1em}{\kern-1em$N_c$} & 2 & 4 & 8 & 16 & 2 & 4 & 8 & 16  \\
 %$N_c \rightarrow$ & 2 & 4 & 8 & 16 & 2 & 4 & 2 \\
 \hline
 \hline
 \multicolumn{5}{|c|}{Llama2-7B (FP32 Accuracy = 45.8\%)} & \multicolumn{4}{|c|}{Llama2-70B (FP32 Accuracy = 69.12\%)} \\ 
 \hline
 \hline
 64 & 43.9 & 43.4 & 43.9 & 44.9 & 68.07 & 68.27 & 68.17 & 68.75 \\
 \hline
 32 & 44.5 & 43.8 & 44.9 & 44.5 & 68.37 & 68.51 & 68.35 & 68.27  \\
 \hline
 16 & 43.9 & 42.7 & 44.9 & 45 & 68.12 & 68.77 & 68.31 & 68.59  \\
 \hline
 \hline
 \multicolumn{5}{|c|}{GPT3-22B (FP32 Accuracy = 38.75\%)} & \multicolumn{4}{|c|}{Nemotron4-15B (FP32 Accuracy = 64.3\%)} \\ 
 \hline
 \hline
 64 & 36.71 & 38.85 & 38.13 & 38.92 & 63.17 & 62.36 & 63.72 & 64.09 \\
 \hline
 32 & 37.95 & 38.69 & 39.45 & 38.34 & 64.05 & 62.30 & 63.8 & 64.33  \\
 \hline
 16 & 38.88 & 38.80 & 38.31 & 38.92 & 63.22 & 63.51 & 63.93 & 64.43  \\
 \hline
\end{tabular}
\caption{\label{tab:mmlu_abalation} Accuracy on MMLU dataset across GPT3-22B, Llama2-7B, 70B and Nemotron4-15B models.}
\end{table}


%\subsection{Perplexity achieved by various LO-BCQ configurations on LM evaluation harness}

\begin{table} \centering
\begin{tabular}{|c||c|c|c|c||c|c|c|c|} 
\hline
 $L_b \rightarrow$& \multicolumn{4}{c||}{8} & \multicolumn{4}{c||}{8}\\
 \hline
 \backslashbox{$L_A$\kern-1em}{\kern-1em$N_c$} & 2 & 4 & 8 & 16 & 2 & 4 & 8 & 16  \\
 %$N_c \rightarrow$ & 2 & 4 & 8 & 16 & 2 & 4 & 2 \\
 \hline
 \hline
 \multicolumn{5}{|c|}{Race (FP32 Accuracy = 37.51\%)} & \multicolumn{4}{|c|}{Boolq (FP32 Accuracy = 64.62\%)} \\ 
 \hline
 \hline
 64 & 36.94 & 37.13 & 36.27 & 37.13 & 63.73 & 62.26 & 63.49 & 63.36 \\
 \hline
 32 & 37.03 & 36.36 & 36.08 & 37.03 & 62.54 & 63.51 & 63.49 & 63.55  \\
 \hline
 16 & 37.03 & 37.03 & 36.46 & 37.03 & 61.1 & 63.79 & 63.58 & 63.33  \\
 \hline
 \hline
 \multicolumn{5}{|c|}{Winogrande (FP32 Accuracy = 58.01\%)} & \multicolumn{4}{|c|}{Piqa (FP32 Accuracy = 74.21\%)} \\ 
 \hline
 \hline
 64 & 58.17 & 57.22 & 57.85 & 58.33 & 73.01 & 73.07 & 73.07 & 72.80 \\
 \hline
 32 & 59.12 & 58.09 & 57.85 & 58.41 & 73.01 & 73.94 & 72.74 & 73.18  \\
 \hline
 16 & 57.93 & 58.88 & 57.93 & 58.56 & 73.94 & 72.80 & 73.01 & 73.94  \\
 \hline
\end{tabular}
\caption{\label{tab:mmlu_abalation} Accuracy on LM evaluation harness tasks on GPT3-1.3B model.}
\end{table}

\begin{table} \centering
\begin{tabular}{|c||c|c|c|c||c|c|c|c|} 
\hline
 $L_b \rightarrow$& \multicolumn{4}{c||}{8} & \multicolumn{4}{c||}{8}\\
 \hline
 \backslashbox{$L_A$\kern-1em}{\kern-1em$N_c$} & 2 & 4 & 8 & 16 & 2 & 4 & 8 & 16  \\
 %$N_c \rightarrow$ & 2 & 4 & 8 & 16 & 2 & 4 & 2 \\
 \hline
 \hline
 \multicolumn{5}{|c|}{Race (FP32 Accuracy = 41.34\%)} & \multicolumn{4}{|c|}{Boolq (FP32 Accuracy = 68.32\%)} \\ 
 \hline
 \hline
 64 & 40.48 & 40.10 & 39.43 & 39.90 & 69.20 & 68.41 & 69.45 & 68.56 \\
 \hline
 32 & 39.52 & 39.52 & 40.77 & 39.62 & 68.32 & 67.43 & 68.17 & 69.30  \\
 \hline
 16 & 39.81 & 39.71 & 39.90 & 40.38 & 68.10 & 66.33 & 69.51 & 69.42  \\
 \hline
 \hline
 \multicolumn{5}{|c|}{Winogrande (FP32 Accuracy = 67.88\%)} & \multicolumn{4}{|c|}{Piqa (FP32 Accuracy = 78.78\%)} \\ 
 \hline
 \hline
 64 & 66.85 & 66.61 & 67.72 & 67.88 & 77.31 & 77.42 & 77.75 & 77.64 \\
 \hline
 32 & 67.25 & 67.72 & 67.72 & 67.00 & 77.31 & 77.04 & 77.80 & 77.37  \\
 \hline
 16 & 68.11 & 68.90 & 67.88 & 67.48 & 77.37 & 78.13 & 78.13 & 77.69  \\
 \hline
\end{tabular}
\caption{\label{tab:mmlu_abalation} Accuracy on LM evaluation harness tasks on GPT3-8B model.}
\end{table}

\begin{table} \centering
\begin{tabular}{|c||c|c|c|c||c|c|c|c|} 
\hline
 $L_b \rightarrow$& \multicolumn{4}{c||}{8} & \multicolumn{4}{c||}{8}\\
 \hline
 \backslashbox{$L_A$\kern-1em}{\kern-1em$N_c$} & 2 & 4 & 8 & 16 & 2 & 4 & 8 & 16  \\
 %$N_c \rightarrow$ & 2 & 4 & 8 & 16 & 2 & 4 & 2 \\
 \hline
 \hline
 \multicolumn{5}{|c|}{Race (FP32 Accuracy = 40.67\%)} & \multicolumn{4}{|c|}{Boolq (FP32 Accuracy = 76.54\%)} \\ 
 \hline
 \hline
 64 & 40.48 & 40.10 & 39.43 & 39.90 & 75.41 & 75.11 & 77.09 & 75.66 \\
 \hline
 32 & 39.52 & 39.52 & 40.77 & 39.62 & 76.02 & 76.02 & 75.96 & 75.35  \\
 \hline
 16 & 39.81 & 39.71 & 39.90 & 40.38 & 75.05 & 73.82 & 75.72 & 76.09  \\
 \hline
 \hline
 \multicolumn{5}{|c|}{Winogrande (FP32 Accuracy = 70.64\%)} & \multicolumn{4}{|c|}{Piqa (FP32 Accuracy = 79.16\%)} \\ 
 \hline
 \hline
 64 & 69.14 & 70.17 & 70.17 & 70.56 & 78.24 & 79.00 & 78.62 & 78.73 \\
 \hline
 32 & 70.96 & 69.69 & 71.27 & 69.30 & 78.56 & 79.49 & 79.16 & 78.89  \\
 \hline
 16 & 71.03 & 69.53 & 69.69 & 70.40 & 78.13 & 79.16 & 79.00 & 79.00  \\
 \hline
\end{tabular}
\caption{\label{tab:mmlu_abalation} Accuracy on LM evaluation harness tasks on GPT3-22B model.}
\end{table}

\begin{table} \centering
\begin{tabular}{|c||c|c|c|c||c|c|c|c|} 
\hline
 $L_b \rightarrow$& \multicolumn{4}{c||}{8} & \multicolumn{4}{c||}{8}\\
 \hline
 \backslashbox{$L_A$\kern-1em}{\kern-1em$N_c$} & 2 & 4 & 8 & 16 & 2 & 4 & 8 & 16  \\
 %$N_c \rightarrow$ & 2 & 4 & 8 & 16 & 2 & 4 & 2 \\
 \hline
 \hline
 \multicolumn{5}{|c|}{Race (FP32 Accuracy = 44.4\%)} & \multicolumn{4}{|c|}{Boolq (FP32 Accuracy = 79.29\%)} \\ 
 \hline
 \hline
 64 & 42.49 & 42.51 & 42.58 & 43.45 & 77.58 & 77.37 & 77.43 & 78.1 \\
 \hline
 32 & 43.35 & 42.49 & 43.64 & 43.73 & 77.86 & 75.32 & 77.28 & 77.86  \\
 \hline
 16 & 44.21 & 44.21 & 43.64 & 42.97 & 78.65 & 77 & 76.94 & 77.98  \\
 \hline
 \hline
 \multicolumn{5}{|c|}{Winogrande (FP32 Accuracy = 69.38\%)} & \multicolumn{4}{|c|}{Piqa (FP32 Accuracy = 78.07\%)} \\ 
 \hline
 \hline
 64 & 68.9 & 68.43 & 69.77 & 68.19 & 77.09 & 76.82 & 77.09 & 77.86 \\
 \hline
 32 & 69.38 & 68.51 & 68.82 & 68.90 & 78.07 & 76.71 & 78.07 & 77.86  \\
 \hline
 16 & 69.53 & 67.09 & 69.38 & 68.90 & 77.37 & 77.8 & 77.91 & 77.69  \\
 \hline
\end{tabular}
\caption{\label{tab:mmlu_abalation} Accuracy on LM evaluation harness tasks on Llama2-7B model.}
\end{table}

\begin{table} \centering
\begin{tabular}{|c||c|c|c|c||c|c|c|c|} 
\hline
 $L_b \rightarrow$& \multicolumn{4}{c||}{8} & \multicolumn{4}{c||}{8}\\
 \hline
 \backslashbox{$L_A$\kern-1em}{\kern-1em$N_c$} & 2 & 4 & 8 & 16 & 2 & 4 & 8 & 16  \\
 %$N_c \rightarrow$ & 2 & 4 & 8 & 16 & 2 & 4 & 2 \\
 \hline
 \hline
 \multicolumn{5}{|c|}{Race (FP32 Accuracy = 48.8\%)} & \multicolumn{4}{|c|}{Boolq (FP32 Accuracy = 85.23\%)} \\ 
 \hline
 \hline
 64 & 49.00 & 49.00 & 49.28 & 48.71 & 82.82 & 84.28 & 84.03 & 84.25 \\
 \hline
 32 & 49.57 & 48.52 & 48.33 & 49.28 & 83.85 & 84.46 & 84.31 & 84.93  \\
 \hline
 16 & 49.85 & 49.09 & 49.28 & 48.99 & 85.11 & 84.46 & 84.61 & 83.94  \\
 \hline
 \hline
 \multicolumn{5}{|c|}{Winogrande (FP32 Accuracy = 79.95\%)} & \multicolumn{4}{|c|}{Piqa (FP32 Accuracy = 81.56\%)} \\ 
 \hline
 \hline
 64 & 78.77 & 78.45 & 78.37 & 79.16 & 81.45 & 80.69 & 81.45 & 81.5 \\
 \hline
 32 & 78.45 & 79.01 & 78.69 & 80.66 & 81.56 & 80.58 & 81.18 & 81.34  \\
 \hline
 16 & 79.95 & 79.56 & 79.79 & 79.72 & 81.28 & 81.66 & 81.28 & 80.96  \\
 \hline
\end{tabular}
\caption{\label{tab:mmlu_abalation} Accuracy on LM evaluation harness tasks on Llama2-70B model.}
\end{table}

%\section{MSE Studies}
%\textcolor{red}{TODO}


\subsection{Number Formats and Quantization Method}
\label{subsec:numFormats_quantMethod}
\subsubsection{Integer Format}
An $n$-bit signed integer (INT) is typically represented with a 2s-complement format \citep{yao2022zeroquant,xiao2023smoothquant,dai2021vsq}, where the most significant bit denotes the sign.

\subsubsection{Floating Point Format}
An $n$-bit signed floating point (FP) number $x$ comprises of a 1-bit sign ($x_{\mathrm{sign}}$), $B_m$-bit mantissa ($x_{\mathrm{mant}}$) and $B_e$-bit exponent ($x_{\mathrm{exp}}$) such that $B_m+B_e=n-1$. The associated constant exponent bias ($E_{\mathrm{bias}}$) is computed as $(2^{{B_e}-1}-1)$. We denote this format as $E_{B_e}M_{B_m}$.  

\subsubsection{Quantization Scheme}
\label{subsec:quant_method}
A quantization scheme dictates how a given unquantized tensor is converted to its quantized representation. We consider FP formats for the purpose of illustration. Given an unquantized tensor $\bm{X}$ and an FP format $E_{B_e}M_{B_m}$, we first, we compute the quantization scale factor $s_X$ that maps the maximum absolute value of $\bm{X}$ to the maximum quantization level of the $E_{B_e}M_{B_m}$ format as follows:
\begin{align}
\label{eq:sf}
    s_X = \frac{\mathrm{max}(|\bm{X}|)}{\mathrm{max}(E_{B_e}M_{B_m})}
\end{align}
In the above equation, $|\cdot|$ denotes the absolute value function.

Next, we scale $\bm{X}$ by $s_X$ and quantize it to $\hat{\bm{X}}$ by rounding it to the nearest quantization level of $E_{B_e}M_{B_m}$ as:

\begin{align}
\label{eq:tensor_quant}
    \hat{\bm{X}} = \text{round-to-nearest}\left(\frac{\bm{X}}{s_X}, E_{B_e}M_{B_m}\right)
\end{align}

We perform dynamic max-scaled quantization \citep{wu2020integer}, where the scale factor $s$ for activations is dynamically computed during runtime.

\subsection{Vector Scaled Quantization}
\begin{wrapfigure}{r}{0.35\linewidth}
  \centering
  \includegraphics[width=\linewidth]{sections/figures/vsquant.jpg}
  \caption{\small Vectorwise decomposition for per-vector scaled quantization (VSQ \citep{dai2021vsq}).}
  \label{fig:vsquant}
\end{wrapfigure}
During VSQ \citep{dai2021vsq}, the operand tensors are decomposed into 1D vectors in a hardware friendly manner as shown in Figure \ref{fig:vsquant}. Since the decomposed tensors are used as operands in matrix multiplications during inference, it is beneficial to perform this decomposition along the reduction dimension of the multiplication. The vectorwise quantization is performed similar to tensorwise quantization described in Equations \ref{eq:sf} and \ref{eq:tensor_quant}, where a scale factor $s_v$ is required for each vector $\bm{v}$ that maps the maximum absolute value of that vector to the maximum quantization level. While smaller vector lengths can lead to larger accuracy gains, the associated memory and computational overheads due to the per-vector scale factors increases. To alleviate these overheads, VSQ \citep{dai2021vsq} proposed a second level quantization of the per-vector scale factors to unsigned integers, while MX \citep{rouhani2023shared} quantizes them to integer powers of 2 (denoted as $2^{INT}$).

\subsubsection{MX Format}
The MX format proposed in \citep{rouhani2023microscaling} introduces the concept of sub-block shifting. For every two scalar elements of $b$-bits each, there is a shared exponent bit. The value of this exponent bit is determined through an empirical analysis that targets minimizing quantization MSE. We note that the FP format $E_{1}M_{b}$ is strictly better than MX from an accuracy perspective since it allocates a dedicated exponent bit to each scalar as opposed to sharing it across two scalars. Therefore, we conservatively bound the accuracy of a $b+2$-bit signed MX format with that of a $E_{1}M_{b}$ format in our comparisons. For instance, we use E1M2 format as a proxy for MX4.

\begin{figure}
    \centering
    \includegraphics[width=1\linewidth]{sections//figures/BlockFormats.pdf}
    \caption{\small Comparing LO-BCQ to MX format.}
    \label{fig:block_formats}
\end{figure}

Figure \ref{fig:block_formats} compares our $4$-bit LO-BCQ block format to MX \citep{rouhani2023microscaling}. As shown, both LO-BCQ and MX decompose a given operand tensor into block arrays and each block array into blocks. Similar to MX, we find that per-block quantization ($L_b < L_A$) leads to better accuracy due to increased flexibility. While MX achieves this through per-block $1$-bit micro-scales, we associate a dedicated codebook to each block through a per-block codebook selector. Further, MX quantizes the per-block array scale-factor to E8M0 format without per-tensor scaling. In contrast during LO-BCQ, we find that per-tensor scaling combined with quantization of per-block array scale-factor to E4M3 format results in superior inference accuracy across models. 

\section{Details of Experiments}
\label{sec:detail_exp}

In this section, we provide the omitted details of the experiment details and additional results.

\subsection{Implementation Details}
\label{subsec:detail_exp:impl}
% \vspace{2mm}
\noindent \textbf{Datasets.~~}
We use the UltraFeedback-binarized dataset~\citep{NeurIPS'23:DPO} for the experiments. This dataset is derived from the original UltraFeedback dataset, which comprises 64, 000 prompts sourced from diverse datasets including UltraChat, ShareGPT, Evol-Instruct, TruthfulQA, FalseQA, and FLAN. For each prompt, four model completions were generated using various open-source and proprietary language models, with GPT-4 providing comprehensive evaluations across multiple criteria including helpfulness, honesty, and truthfulness. The binarized version was constructed by selecting the completion with the highest overall score as the "chosen" response and randomly selecting one of the remaining completions as the "rejected" response, creating clear preference pairs suitable for reward modeling and direct preference optimization. This dataset structure aligns well with our experimental setup, providing a robust foundation for evaluating different preference learning approaches. The dataset's diverse prompt sources and evaluation criteria make it particularly valuable for training and evaluating reward models in a real-world context. To further tailor the dataset to our experimental setup, we organize the dataset as follows for the training and deployment stages:
\begin{itemize}
    \item Training stage with \emph{passive data collection}: We randomly sample $T=30, 000$ data points from the UltraFeedback-binarized dataset's \texttt{train\_prefs} split for training. Each data point consists of a prompt and two responses with a binary preference label indicating the preferred response. We use the \texttt{test\_prefs} split for evaluation.
    \item Training stage with \emph{active data collection}: We allow the method to actively select 6,400 samples from the \texttt{train\_prefs} split according to different selection strategies. The global batch size is set to 8 for training. The selection is performed iteratively, where in each iteration, the method selects the most informative samples based on its selection criterion.
    \item Deployment stage: We use a pre-processed online variant of the UltraFeedback-binarized dataset from the \texttt{test\_gen} split. The dataset is divided into 20 sequential chunks to simulate an online deployment scenario. For each chunk, we generate responses using the current policy (the foundation model of policy model is chosen to be \texttt{meta-llama / Llama-3.2-1B}), evaluate them using both the learned reward model and an oracle reward model. We choose \texttt{NCSOFT/Llama-3-OffsetBias-RM-8B}~\citep{arXiv'24:Park-OffsetBias} as the oracle reward model. After each chunk, we use the policy model to randomly generate 64 responses using different seeds. We then apply various strategies (\emph{Random}, \emph{Best-Two}, etc.) to select responses and construct new preference pairs, which are then used to update the reward model and the policy model.
\end{itemize}

\begin{algorithm}[!t]
    \caption{Efficient Update using Hessian-Vector Product with Conjugate Gradient}
    \label{alg:hvp-cg}
    \begin{algorithmic}[1]
        \REQUIRE Current parameter $\thetat_t$, gradient $g_t(\thetat_t)$, learning rate $\eta$, max CG steps $K$, base damping $\lambda_0$, error tolerance $\epsilon$
        \STATE Initialize $v_0 = 0$, $r_0 = g_t(\thetat_t)$, $p_0 = r_0$
        \STATE Compute damping $\lambda_t = \lambda_0 \cdot \min\{1, f(t/T)\}$
        \FOR {$k=0,1,\ldots,K-1$}
        \STATE Compute HVP: $\Ht_t p_k = \nabla_\theta(\nabla_\theta \mathcal{L}(\theta)^\top p_k)|_{\theta=\thetat_t} + \lambda_t p_k$
        \STATE $\alpha_k = \frac{r_k^\top r_k}{p_k^\top \Ht_t p_k}$, $v_{k+1} = v_k + \alpha_k p_k$, $r_{k+1} = r_k - \alpha_k \Ht_t p_k$,
        \STATE  $\beta_{k+1} = \frac{r_{k+1}^\top r_{k+1}}{r_k^\top r_k}$, $p_{k+1} = r_{k+1} + \beta_{k+1} p_k$
        \IF{$\|r_{k+1}\| \le \epsilon$}
        \STATE \textbf{break}
        \ENDIF
        \ENDFOR
        \STATE Update parameter: $\thetat_{t+1} = \thetat_t - \eta v_{K}$
        \ENSURE Updated parameter $\thetat_{t+1}$
    \end{algorithmic}
\end{algorithm}

\noindent \textbf{Practical Implementation.~~}
To efficiently implement the OMD update in Eq.~\eqref{eq:omd} without the costly computing and storing the full Hessian matrix, we utilize the Hessian-vector product (HVP) method combined with conjugate gradient descent. The key insight is that the update can be reformulated as solving a linear system.
For the OMD update:
\begin{align*}
    \thetat_{t+1} & = \argmin_{\theta \in \Theta} \Big\{\big\langle g_t(\thetat_t), \theta \big\rangle +\frac{1}{2 \eta}\big\|\theta-\thetat_t\big\|_{\Ht_t}^2\Big\}  \approx \thetat_t - \eta \Ht_t^{-1}g_t(\thetat_t),
\end{align*}
where we ignore the projection operation.
The key challenge is computing \(\Ht_t^{-1}g_t(\thetat_t)\) efficiently. Instead of explicitly computing \(\Ht_t^{-1}\), we solve the linear system $\Ht_t v = g_t(\thetat_t)$
\begin{equation*}
    \Ht_t v = g_t(\thetat_t),
\end{equation*}
where \(v\) is the solution we seek. We use the conjugate gradient method with a damped Hessian to solve this system iteratively.
The HVP operation \(\Ht_t v\) for any vector \(v\) can be computed efficiently using automatic differentiation $\Ht_t v = \nabla_\theta \big(\nabla_\theta \mathcal{L}(\theta)^\top v\big)\big|_{\theta=\thetat_t} + \lambda v$, where \(\lambda\) is a damping coefficient that ensures numerical stability and \(\mathcal{L}(\theta)\) is the loss function. The full algorithm is summarized in Algorithm~\ref{alg:hvp-cg}. To ensure stability and convergence, we employ an adaptive damping strategy as $\lambda_t = \lambda_0 \cdot \min\{1, f(t/T)\}$, where $f(\cdot)$ is a damping growth function that can be linear, logarithmic, or cosine. In practice, we set $K = 3$ and $\lambda_0 = 0.8$ for all experiments.


\subsection{Validating the Magnitude of $\kappa$}  
\label{subsec:detail_exp:kappa}  

We validate the magnitude of $\kappa$ in Eq.~\eqref{eq:kappa} by computing its value during the training process. The results show that $\kappa = 171.62 \pm 85.49$ during our training process, which is relatively large.


\subsection{Combined with Adam Optimizer}
\label{subsec:detail_exp:adam}

\begin{figure*}[!t]
    \begin{minipage}[t]{0.99\textwidth}
        \centering
        \subfigure[training loss]{\includegraphics[width=0.235\columnwidth]{sections/figs/offline_all_train_loss_adam.pdf}
            \label{fig:passive-train-loss-adam}}
        \hfill
        \subfigure[training accuracy]{\includegraphics[width=0.235\columnwidth]{sections/figs/offline_all_train_acc_adam.pdf}
            \label{fig:passive-train-acc-adam}}
        \hfill
        \subfigure[evaluation loss]{\includegraphics[width=0.235\columnwidth]{sections/figs/offline_all_eval_loss_adam.pdf}
            \label{fig:passive-eval-loss-adam}}
        \hfill
        \subfigure[evaluation accuracy]{\includegraphics[width=0.235\columnwidth]{sections/figs/offline_all_eval_acc_adam.pdf}
            \label{fig:passive-eval-acc-adam}}
    \end{minipage}
    \caption{For the training stage with \emph{passive data collection}, we compare our proposed method and MLE~\citep{ICML'23:Zhu-Principled} in with passive data collection combined with \emph{Adam}. We report the average accuracy and loss of the reward model during the training process.}
    \label{fig:training_passive_adam}
\end{figure*}

In previous experiments, we used SGD to update model parameters. In this section, we integrate the methods with the \emph{Adam optimizer}~\citep{ICLR'15:Adam}, i.e., adding the first and second momentum terms to the model updates. The results, shown in Figure~\ref{fig:training_passive_adam}, indicate that the Adam optimizer further enhances the performance of our method by leveraging the momentum term to accelerate convergence. With the momentum term, our method remains superior to the MLE-based method; however, the performance gap is reduced. We think that this is because the Adam optimizer incorporates second-order information for optimization, diminishing the advantage of our method compared to the SGD cases.



\subsection{Comparison with DPO}
\label{subsec:detail_exp:dpo}

\begin{wrapfigure}{r}{0.43\textwidth}
    \vspace{-6.5mm}
    \begin{center}
        \includegraphics[width=0.4\textwidth]{sections/figs/deployment_cumulative_rewards_regular_with_dpo.pdf}
    \end{center}
    \vspace{-6mm}
    \caption{Comparison with DPO in the deployment stage.}
    \label{fig:dpo-compare}
    \vspace{-2mm}
\end{wrapfigure}
We also compare with DPO (Direct Preference Optimization)~\citep{NeurIPS'23:DPO} in the deployment stage, as a reward-free method, DPO optimizes the policy directly using preference feedback without explicit reward modeling. 
To ensure a fair comparison, we initialize the policy with 400 samples and use the same dataset settings as PPO to iteratively update the policy model using the DPO algorithm.
The results are illustrated in Figure~\ref{fig:dpo-compare}.
While DPO outperforms the random baseline (Rand-MLE), it achieves lower cumulative rewards than the PromisingSet-based method. This result suggests that DPO's online learning capability remains a challenge. In contrast, the reward model learned by our selection strategy effectively learned streaming data and continuously updates the policy as new data arrive, indicating that in our deployment stage, a reward model with PPO may be a more suitable choice for sequentially learning from new data.


% This performance gap can be attributed to two main factors: First, our method explicitly models the reward function, which provides more informative learning signals for policy optimization. Second, our active data collection strategy enables more efficient learning by selecting the most informative samples, while DPO relies on passive preference data.

% The experimental results align with theoretical insights from previous work~\citep{NeurIPS'23:DPO, ICML'23:Zhu-Principled}, which suggest that reward modeling can provide more structured guidance for policy optimization compared to direct preference optimization. Specifically, our method achieves approximately 8\% higher cumulative rewards than DPO after 20 iterations, demonstrating the benefits of combining reward modeling with active learning strategies.



\end{document}

