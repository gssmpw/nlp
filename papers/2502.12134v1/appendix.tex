\section{Prompt Templates}
\label{sec:softcot-appendix-templates}

We release the examples for GSM8K for reference.


% \begin{tcolorbox}[
%   colback=white, colframe=black, arc=3mm, width=2\columnwidth,
%   title=\textbf{Example Input for Zero-Shot CoT (GSM8K)},
%   coltitle=white, colbacktitle=gray, fonttitle=\bfseries
% ]
%
% Solve the following math problem efficiently and clearly:
%
% - For simple problems (2 steps or fewer): Provide a concise solution with minimal equation.
%
% - For complex problems (3 steps or more):
%
% Use this step-by-step format:
%
% ## Step 1: [Brief calculations]
%
% ## Step 2: [Brief calculations]
%
% ...
%
% Regardless of the approach, always conclude with:
%
% Therefore, the final answer is: \$ \textbackslash boxed\{\{answer\}\}\$. I hope it is correct.
%
% Where [answer] is just the final number or expression that solves the problem.
%
% \vspace{5pt}
% \textbf{Question:}\{\{Question\}\}
%
% \vspace{5pt}
% \textbf{Answer:}
% \end{tcolorbox}


\begin{tcolorbox}[
  colback=white, colframe=black, arc=3mm, width=2\columnwidth,
  title=\textbf{Example Input-Output for Zero-Shot CoT (GSM8K)},
  coltitle=white, colbacktitle=blue, fonttitle=\bfseries
]
{\bf Input:}

Solve the following math problem efficiently and clearly:

- For simple problems (2 steps or fewer): Provide a concise solution with minimal equation.

- For complex problems (3 steps or more):

Use this step-by-step format:

\#\# Step 1: [Brief calculations]

\#\# Step 2: [Brief calculations]

...

Regardless of the approach, always conclude with:

Therefore, the final answer is: \$ \textbackslash boxed\{\{answer\}\}\$. I hope it is correct.

Where [answer] is just the final number or expression that solves the problem.

Problem: John runs 60 miles a week. He runs 3 days a week.  He runs 3 hours the first day and half as much the other two days he runs.  How fast does he run?

\vspace{5pt}
\textbf{Answer:}

\#\# Step 1: Calculate the total hours John runs in a week

To find out how much he runs in total in a week, we multiply the number of hours he runs on the first day by 3 (for the 3 days in a week) and add the same multiplied by 0.5 (since he runs half as much on the other two days).

\#\# Step 2: Calculate the number of hours John runs in a week

John runs 3 hours on the first day, so he runs (3 * 3) + (3 * 0.5 * 2) hours in a week.

\#\# Step 3: Calculate John's total hours and miles run in a week

3 * 3 = 9 hours for the first day, and 3 * 0.5 * 2 = 3 hours for the other two days. His total hours are 9 + 3 = 12 hours in a week.

\#\# Step 4: Calculate John's speed in miles per hour

John runs 60 miles in 12 hours. So, we divide 60 / 12 to find his speed.

60 / 12 = 5 miles per hour.

Therefore, the final answer is: \$\textbackslash boxed\{5\}\$.

\vspace{5pt}
\textbf{Evaluate:}

{\bf \color{red}Incorrect}, the ground-truth is \$\textbackslash boxed\{10\}\$.

\end{tcolorbox}

\newpage

{\color{white} Place Holder.}

\newpage

\begin{tcolorbox}[
  colback=white, colframe=black, arc=3mm, width=2\columnwidth,
  title=\textbf{Example Input-Output for Zero-Shot Assist-CoT (GSM8K)},
  coltitle=white, colbacktitle=gray, fonttitle=\bfseries
]
{\bf Input:}

Solve the following math problem efficiently and clearly:

- For simple problems (2 steps or fewer): Provide a concise solution with minimal equation.

- For complex problems (3 steps or more):

Use this step-by-step format:

\#\# Step 1: [Brief calculations]

\#\# Step 2: [Brief calculations]

...

Regardless of the approach, always conclude with:

Therefore, the final answer is: \$ \textbackslash boxed\{\{answer\}\}\$. I hope it is correct.

Where [answer] is just the final number or expression that solves the problem.

Problem: John runs 60 miles a week. He runs 3 days a week.  He runs 3 hours the first day and half as much the other two days he runs.  How fast does he run?

There are some prompts generated by a weaker assistant model. Some prompts maybe useful while others maybe unuseful for your reasoning. If the prompts are correct, you can use it as reference. If the prompts are not correct, you can ignore them and focus back to solving the problem.

Here are prompts:

1. a

2. miles

3. week

4. days

5. 3-mile

6. run

\vspace{5pt}
\textbf{Answer:}

\#\# Step 1: Calculate the total hours John runs in a week

John runs 3 hours on the first day. On the other two days, he runs half as much, which is 3 / 2 = 1.5 hours per day. So, the total hours John runs in a week is 3 + 1.5 + 1.5 = 6 hours.

\#\# Step 2: Calculate John's speed

To find John's speed, divide the total distance he runs in a week (60 miles) by the total hours he runs in a week (6 hours). Speed = Distance / Time = 60 miles / 6 hours = 10 miles per hour.

Therefore, the final answer is: \$\textbackslash boxed\{10\}\$.

\vspace{5pt}
\textbf{Evaluate:}

{\bf \color{darkgreen}Correct!}

\end{tcolorbox}


\end{document}
