% \begin{figure*}[!t]
%     \centering
%     \subfigure[POPE]{
%         \includegraphics[width=0.28\linewidth]{latex/figure/latency_vs_performance_POPE.pdf}
%     }
%     % \hfill
%     % \subfigure[MME]{
%     %     \includegraphics[width=0.23\linewidth]{latex/figure/latency_vs_performance_POPE.pdf}
%     % }
%     \subfigure[MME]{
%         \includegraphics[width=0.28\linewidth]{latex/figure/latency_vs_performance_MME.pdf}
%     }
%     \subfigure[MMBench]{
%         \includegraphics[width=0.28\linewidth]{latex/figure/latency_vs_performance_MMBench.pdf}
%     }
%     \vspace{-4mm}
%     \caption{...}
%     \vspace{-4mm}
%     \label{fig:latency_vs_performance}
% \end{figure*}


\begin{figure*}[!ht]
    \centering
    \includegraphics[width=1.02\linewidth]{latex/figure/latency_vs_performance/latency_vs_performance_all_v2.pdf}
    \vspace{-4mm}
    \caption{\textbf{Performance-Latency trade-off comparisons} across different datasets on LLaVA-Next-7B. \algname consistently achieves better performance under varying latency constraints compared to other approaches.}
    \vspace{-4mm}
    \label{fig:latency_vs_performance}
\end{figure*}

% \textbf{Performance-Latency trade-off comparisons} across different datasets on LLaVA-Next-7B: (a) POPE, (b) MME, (c) MMBench, and (d) VizWiz. Our proposed method consistently achieves better performance under varying latency constraints compared to other approaches, including Random, FastV, SparseVLM, and MustDrop.