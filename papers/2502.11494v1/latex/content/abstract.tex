\begin{abstract}
Vision tokens in multimodal large language models often dominate huge computational overhead due to their excessive length compared to linguistic modality.
Abundant recent methods aim to solve this problem with token pruning, which first defines an importance criterion for tokens and then prunes the unimportant vision tokens during inference. 
However, in this paper, we show that the importance is not an ideal indicator to decide whether a token should be pruned. Surprisingly, it usually results in inferior performance than random token pruning and leading to incompatibility to efficient attention computation operators.
Instead, we propose \textbf{DART} (\textbf{D}uplication-\textbf{A}ware \textbf{R}eduction of \textbf{T}okens), which prunes tokens based on its duplication with other tokens, leading to significant and training-free acceleration.
Concretely, DART selects a small subset of pivot tokens and then retains the tokens with low duplication to the pivots, ensuring minimal information loss during token pruning. 
Experiments demonstrate that DART can prune \textbf{88.9\%} vision tokens while maintaining comparable performance, leading to a \textbf{1.99$\times$} and \textbf{2.99$\times$} speed-up in total time and prefilling stage, respectively, with good compatibility to efficient attention operators.
% Experiments demonstrate that DART can prune \textbf{88.9\%} vision tokens while maintaining comparable performance, leading to \textbf{1.99$\times$} speed-up with good compatibility to efficient attention operators. 
% \emph{Codes are available in the supplementary material and will be released in Github.}

\end{abstract}