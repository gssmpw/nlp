\section{Related Work}
\subsection{Large Language Models}
Large Language Models (LLMs) have significantly evolved over recent years, demonstrating remarkable capabilities across a variety of natural language processing tasks. Notable models such as DeepSeek V3____, GPT-4____, Llama ____, and more recently, Mistral 7B____, have set new benchmarks in terms of language understanding and generation capabilities. Mistral 7B, for example, presents a sophisticated architecture that balances model complexity with computational efficiency, supporting more nuanced text generation and comprehension tasks. These models are pretrained on extensive datasets, allowing them to generalize well across diverse linguistic contexts and effectively handle tasks such as translation, summarization, and question answering.

\subsection{Large Language Models in Financial Domain}
The application of large language models within the financial domain is an emerging area of interest, driven by the potential to transform traditional financial analysis and decision-making processes. LLMs are increasingly being used to enhance financial forecasting, sentiment analysis, and risk management by leveraging their ability to process and analyze large volumes of textual data from financial reports, news articles, and social media. For instance, FinBERT, a specialized variant of BERT, has been fine-tuned for finance-specific tasks and shows improved performance in sentiment analysis of news headlines and economic reports. XuanYuan ____.

Recent studies have explored the potential of using LLMs for predictive analytics in finance, such as predicting stock movements and credit scoring. These models are designed to discern financial insights from unstructured data, complementing traditional quantitative models. Moreover, LLMs offer an advantage in terms of scalability and adaptability, making them suitable for real-time analysis and decision support in fluctuating market conditions.

Furthermore, the integration of LLMs in financial services raises questions of ethical AI and fairness, particularly concerning biases inherent in training data that could affect financial decision-making. Therefore, ongoing research focuses on developing techniques for bias mitigation and ethical deployment of LLMs to ensure fair, transparent, and accountable use in financial applications.