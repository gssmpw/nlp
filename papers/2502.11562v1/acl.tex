\pdfoutput=1

\documentclass[11pt]{article}

% \usepackage[review]{acl}
\usepackage[final]{acl}


% Standard package includes
\usepackage{times}
\usepackage{latexsym}
\usepackage{multirow}
\usepackage{booktabs}
\usepackage{amsmath}
\usepackage{amssymb}
\usepackage{adjustbox}
\usepackage{amsfonts}

\usepackage{cleveref}
\usepackage{graphicx}
\usepackage{epstopdf}
\usepackage{hyperref}
\usepackage{colortbl}
\usepackage{float}
\usepackage{pifont}
\usepackage{xcolor}
\usepackage{mdframed}

\usepackage{tabularx}

\definecolor{my_green}{RGB}{51,102,0}
\definecolor{my_red}{RGB}{204, 0, 0}
\renewcommand{\checkmark}{\textcolor{my_green}{\ding{51}}} % 
\newcommand{\crossmark}{\textcolor{my_red}{\ding{55}}} % 


\renewcommand{\sfdefault}{qag} % TeX Gyre Adventor has small caps

\makeatletter
\let\scshape\relax % to avoid a warning
\DeclareRobustCommand\scshape{%
  \not@math@alphabet\scshape\relax
  \ifnum\pdf@strcmp{\f@family}{\familydefault}=\z@
    \fontfamily{qbk}%
  \fi
  \fontshape\scdefault\selectfont}
\makeatother


\usepackage[T1]{fontenc}
\usepackage[utf8]{inputenc}
\usepackage{microtype}
% This is also not strictly necessary, and may be commented out.
% However, it will improve the aesthetics of text in
% the typewriter font.
\usepackage{inconsolata}
\usepackage{graphicx}




% \title{A Self-Boosting Framework For Domain Adaptation Of Dense Retrieval} 
% \title{Reinforced-IR: A Self-Boosting Framework For Cross-Domain Retrieval}  
\title{Reinforced Information Retrieval}   


\author{
    Chaofan Li$^{1,2}$\thanks{~ The two authors contribute equally.}, \ Zheng Liu$^{1,4*}$, \ Jianlyu Chen$^{1,3}$, 
    \ \textbf{Defu Lian}$^{3}$, \ \textbf{Yingxia Shao}$^{2}$  \\
    $^1$  BAAI,  \ \ \ \ $^2$ BUPT, \ \ \ \ $^3$  USTC, \ \ \ \ $^4$ HKPU  \\
    \texttt{zhengliu1026@gmail.com}  \quad \texttt{\{cli,yxshao\}@bupt.edu.cn}
}


\begin{document}
\maketitle
\begin{abstract}
% While retrieval tools are widely used in practice, it still faces severe challenges in cross-domain scenarios. Recently, generation-augmented methods have emerged as a promising solution to address these issues. These methods enhance raw queries by incorporating extra information from a LLM-based generator, thereby enabling more direct retrieval of relevant documents. However, the existing methods struggle to handle highly specialized situations that require extensive domain expertise. Moreover, the reliance on proprietary LLM APIs makes these methods prohibitively expensive for practical use and raises the concerns of data privacy.  

% In this paper, we present \textbf{Reinforced-IR}, a novel approach that jointly adapts a generally pre-trained retriever and a local lightweight LLM for precise cross-domain retrieval. A key innovation of Reinforced-IR is its \textbf{Self-Boosting} framework, which enables the retriever and LLM to learn from each other's feedback. Specifically, the LLM is reinforced to generate query augmentations that enhance the retriever's performance, while the retriever is trained to better discriminate the relevant documents identified by the LLM. This iterative process allows the end-to-end retrieval performance to be progressively optimized using an unlabeled corpus from the target domain. Our extensive evaluations demonstrate that Reinforced-IR outperforms existing domain adaptation methods by large margins, leading to substantial improvements in retrieval quality across a diverse set of datasets and model configurations.  

While retrieval techniques are widely used in practice, they still face significant challenges in cross-domain scenarios. Recently, generation-augmented methods have emerged as a promising solution to this problem. These methods enhance raw queries by incorporating additional information from an LLM-based generator, facilitating more direct retrieval of relevant documents.  However, existing methods struggle with highly specialized situations that require extensive domain expertise. To address this problem, we present \textbf{Reinforced-IR}, a novel approach that jointly adapts a pre-trained retriever and generator for precise cross-domain retrieval. A key innovation of Reinforced-IR is its \textbf{Self-Boosting} framework, which enables retriever and generator to learn from each other's feedback. Specifically, the generator is reinforced to generate query augmentations that enhance the retriever's performance, while the retriever is trained to better discriminate the relevant documents identified by the generator. This iterative process allows the end-to-end retrieval performance to be progressively optimized using an unlabeled corpus from the target domain. In our experiment, Reinforced-IR outperforms existing domain adaptation methods by a large margin, leading to substantial improvements in retrieval quality across a wide range of application scenarios.   

\end{abstract} 

\section{Introduction} 
With the rapid advancement of large language models (LLMs), AI copilots have become deeply integrated into a wide variety of activities, such as addressing knowledge-intensive problems, analyzing professional documents, developing computer programs, and providing personal assistance \cite{achiam2023gpt,team2023gemini,anthropic2024claude}. To produce reliable and trustworthy results in these tasks, it is essential to incorporate useful knowledge from external databases, a process known as retrieval-augmented generation of LLMs, i.e., RAG \cite{lewis2020retrieval}. Because of the advantages in broad applicability and simplicity, dense retrieval emerges as a popular form of retriever in such applications. It employs an embedder to map the data into a vector space, enabling the retrieval of relevant information based on vector similarity \cite{zhao2024dense}. Recently, numerous open-source models and API services have been made publicly available \cite{izacard2021unsupervised,xiao2024c,neelakantan2022text}, which significantly facilitate the utilization of corresponding techniques.   

Given the diverse range of applications, it's important to adapt general retrievers to new working scenarios beyond their original training domains. To this end, a variety of domain adaptation methods have been proposed in recent years. A notable breakthrough was made by the development of \textit{HyDE}-style methods (hypothetical document embedding) \cite{gao2022precise,wang2023query2doc}, or more broadly, the \textit{GAR} techniques (generation-augmented retrieval) \cite{mao2020generation}. These methods leverage LLMs, like ChatGPT, to enrich the query with extra information, thus enabling relevant documents to be identified in a straightforward way. \textit{However, the existing methods mainly rely on LLMs trained on general domains, which may lack necessary knowledge needed by a highly specialized domain, such as medical or legal retrieval.} Besides, the heavy reliance on proprietary LLMs often results in prohibitively high costs, which limits their applicability in many situations.  

\begin{figure*}[tb]
    \centering
    \includegraphics[width=0.95\textwidth]{figures/irstudio_1.pdf}
    \vspace{-0.3cm}
    \caption{\textbf{Reinforced-IR} jointly adapts retriever and generator with an unlabeled domain corpus via self-boosting. The well-adapted generator augments raw query with hypothetical docs, which enables relevant docs to be retrieved.} 
    \vspace{-0.5cm}
    \label{fig:frame}
\end{figure*}  


To address these challenges, we propose a novel domain adaptation framework called \textbf{Reinforced-IR}, \textit{which jointly adapts the {retriever} and {LLM-based generator} using a {unlabeled corpus}}. Our method is distinguished for its design of \textbf{self-boosting} algorithm. It starts with a list of pseudo questions generated from the target domain's unlabeled corpus. On one hand, the LLM-based generator is reinforced to perform high-quality query augmentation using the retriever's feedback, such that relevant documents can be optimally retrieved for downstream tasks. This step is referred as the \textit{Reinforcement Learning of generator with Retriever's Feedback} (\textbf{RLRF}). On the other hand, the retriever is reinforced to discriminate the relevant documents preferred by the LLM-based generator. This step is called the \textit{Reinforcement Learning of retriever with Generator's Feedback} (\textbf{RLGF}). With the alternating execution of these two operations, the end-to-end retrieval performance can be progressively enhanced for the target domain. 


% We perform comprehensive experimental studies in this paper, which reflect the effectiveness of our method from the following perspectives. 1) DR-Studio substantially enhances the cross-domain performance of pre-trained retrievers and achieves notable advantages over the existing domain adaptation baselines. 2) DR-Studio well maintains its superior performance across a variety of datasets from BEIR \cite{thakur2021beir} and AIR-Bench \cite{chen2024air}, as well as under different configurations of pre-trained retrievers and LLM backbones. These two points demonstrate the general effectiveness of 

% 3) The empirical advantages are especially pronounced on low-resource datasets that differ substantially from the original domains of the retrievers and LLMs 


We perform a comprehensive evaluation based on a variety of domain-specific datasets from BEIR \cite{thakur2021beir} and AIR-Bench \cite{chen2024air}. We also include various retrievers and LLMs in our evaluation. According to the experiment result, Reinforced-IR substantially enhances the cross-domain performance of pre-trained retrievers and demonstrates notable advantages over the existing domain adaptation baselines. Additionally, the performance gains are especially pronounced on low-resource datasets that differ substantially from the original domains of the retrievers and LLMs, which further highlights the effectiveness of our approach for domain adaptation. Our model and source code will be shared with the public to advance the future research in this field.   

In summary, the contributions of this paper are presented as follows:
\begin{itemize}
    \item We introduce Reinforced-IR, a novel framework for cross-domain retrieval. To the best of our knowledge, this is the first work that jointly adapts retriever and generator to optimize end-to-end retrieval performance.      
    \item We design the \textbf{RLRF} and \textbf{RLGF} algorithms, enabling retriever and generator to mutually enhance each other’s performance based on an unlabeled corpus from the target domain.  
    \item We conduct comprehensive experimental studies, which verify our significant advantage over existing cross-domain retrieval methods. 
\end{itemize}

% \section{Related Works} 
% dense retrieval and rag
%   - what is dense retrieval
%   - dense retrieval based on transformer and bert
%   - dense retrieval based on hard negative and distillation
%   - open-source models and API
%   - applications in the age of llms 
% domain adaptation of dense retrieval 
%   - two classes: generally pre-trained or domain specific retrievers 
%   - limited performance for a specialized domain 
%   - domain adaptation by continual fine-tuning
%       - labeled relevance
%       - synthetic relevance (inpairs)
%       - contrastive learning (Q Gen) or distillation (GPL)
%   - domain adaptation by generation 
%       - hypothetical document (hyde, query2doc)
%       - doc2query 
%       - generation augmented IR (llms know how to bridge the connection) 
% OK for popular scenarios
% but lack of domain expertise for highly specialized areas
% expecially for ecomonical-scale llms 


\begin{figure*}[tb]
    \centering
    \includegraphics[width=0.90\textwidth]{figures/irstudio_2.pdf}
    \vspace{-0.3cm}
    \caption{\textbf{Self-Boosting workflow}. 1) \textbf{RLRF}: the generator is reinforced to produce the retriever's preferred query augmentation (marked by thumb-up) through DPO. 2) \textbf{RLGF}: the retriever is reinforced to discriminate the generator's preferred documents (measured by preference score $w_{-}$) in the form of knowledge distillation.} 
    \vspace{-0.3cm}
    \label{fig:method}
\end{figure*}  


\section{Method}
In this section, we will first introduce the workflow of generation-augmented retrieval and formulate the problem. Then, we will elaborate the self-boosting algorithm, which optimizes the end-to-end retrieval performance using unlabeled data. 

\subsection{Generation-Augmented Retrieval} 
As a popular IR paradigm, dense retrieval identifies a query's relevant documents based on embedding similarity. Given an embedding model $enc(\cdot)$, the query $q$ and document $d$ are transformed into latent vectors: $\boldsymbol{v}_q \leftarrow enc(q)$, $\boldsymbol{v}_d \leftarrow enc(d)$. On top of such results, the relevance score is calculated as the following inner product: $\sigma_{q,d} \leftarrow \boldsymbol{v}_q^T \boldsymbol{v}_d$. It is expected that the most relevant document ($d^*$) can produce the highest relevant score compared to the rest of documents, i.e., $d^*: \max \{ \boldsymbol{v}_q^T \boldsymbol{v}_d \}_{d \in D}$. 

When applied to a new scenario, the model needs to handle different relevance patterns between query and document from its original domain. To bridge this gap, the query is augmented with extra information (Figure \ref{fig:frame}), like hypothetical docs in HyDE \cite{gao2022precise}. Despite possible incomplete or inaccurate details, the generation-augmented retrieval (GAR) facilitates query and relevant docs to be matched in a more straightforward way. Nowadays, the query augmentation is often performed by a LLM-based generator $gen(\cdot)$, which are directly prompted to generate a list of hypothetical docs ($H_q$) for the query: 
\begin{equation}
    H_q: \{ h_i \leftarrow gen(q, ~prompt) \}_{i=1,...,L}.
\end{equation}
Here, $h_i$ is one of the sampled generation results. The system prompt is defined w.r.t. each concrete scenario, e.g., ``\textit{\{Query\} (symptoms of some disease). Generate the treatment for the described disease}'' for a medical retrieval problem. Following the proposed workflow in HyDE, the augmented query embedding ($\boldsymbol{v}'_{q}$) is calculated as the linear combination of raw query embedding ($\boldsymbol{v}_q$) and each of the hypothetical document embeddings: 
\begin{equation}\label{eq:2}
    \boldsymbol{v}'_{q} \leftarrow \alpha_0*\boldsymbol{v}_q + \sum\nolimits_{1...L} \alpha_i * \boldsymbol{v}_{h_i}, 
\end{equation} 
where $\boldsymbol{v}_{h_i} \leftarrow enc(h_i)$, $\alpha_i>0$, and $\sum_{0...L}\alpha_i=1$. Ultimately, the augmented query embedding $\boldsymbol{v}'_{q}$ is used for the retrieval of relevant documents. 

With the above definition, our problem is formulated as the joint optimization of embedding and generation model: $enc(\cdot)$, $gen(\cdot)$, such that the relevant documents in the target domain can be identified using the augmented query embedding, i.e., $d^*: \max \{ \boldsymbol{v}'^T_q \boldsymbol{v}_d \}_{d \in D}$. 

% problem is defined as the jointly adaptation of encoding model $enc(\cdot)$ and generation model $gen(\cdot)$ on the target domain, such that the relevant documents can be identified on top of the augmented query embedding, i.e., $d^*: \max \{ \langle \boldsymbol{v}'_q, \boldsymbol{v}_d \rangle\}_{d \in D}$. 


\subsection{Self-Boosting} 
The optimization process begins with a unlabeled corpus ($D$) from the target domain. Following established practices \cite{ma2020zero,thakur2021beir}, we prompt the LLM to generate a set of synthetic queries: $Q \leftarrow \{q: QGen(d^*) \}_{D^*}$ for sampled documents $D^*$. The resulting pairs $\{(q, d^*)\}_Q$ serve the training source for domain adaptation. Building on this foundation, we introduce the \textbf{Self-Boosting} algorithm, which consists of two dual steps: {generator optimization} by \textit{reinforcement learning from retriever's feedback} (\textbf{RLRF}), and {retriever optimization} by \textit{reinforcement learning from generator's feedback} (\textbf{RLGF}). The two steps are iteratively performed, enabling progressive improvement in end-to-end retrieval performance. 

\subsubsection{Generator optimization by RLRF} 
The generator is prompted to produce a group of candidate hypothetical documents for each training query $q$: $H_q \leftarrow \{h_i \leftarrow gen(q, prompt)\}_{i=1,...,K}$, where $K$ is the predefined sample size. Given this sampling result, the generator is reinforced to produce the best candidate which optimizes the retriever's performance. Particularly, we simplify the calculation of augmented query embedding in Eq. \ref{eq:2} as the case with one single hypothetical document: $\boldsymbol{v}'_q \leftarrow \alpha *\boldsymbol{v}_q + (1-\alpha)*\boldsymbol{v}_{h_i}$, where we compute the retriever's preference score as: 
\begin{equation}\label{eq:4}
    s_{q,h_i} \leftarrow \boldsymbol{v}'^T_q \boldsymbol{v}_{d^*} . 
\end{equation}  
By applying the above computation to every augmented query, we get $S_q \leftarrow \{s_{q,h_i}\}_{H_q}$ as the retriever's feedback to the whole hypothetical documents. We further conduct direct preference optimization (DPO) to reinforce the generation of the retriever's preferred query augmentation \cite{rafailov2024direct}. For the simplicity of training, we only consider the hypothetical documents of the highest and lowest scores, and leverage them as the wining and losing candidates: $h_w$, $h_l$. To screen out low-quality samples, we introduce the following filtering rules to the candidate documents: 
\begin{equation}
    1. ~ s_{q,h_w} > s_{q}, \ \, 2. ~ s_{q,h_w} > \gamma * s_{q,h_l},
\end{equation}
where $\gamma$ is a scaling factor: $\gamma>1$, $s_{q}$ indicates the preference score without using hypothetical document: $s_{q} = \boldsymbol{v}_q^T \boldsymbol{v}_{d^*}$. The first rule regularizes that the winning candidate must positively contribute to the retrieval result, while the second one guarantees the significance of winning candidate's contribution. Finally, we apply the following loss for DPO: 
\begin{equation}
    \mathcal{L}^{dpo} = - \log \sigma\big(\beta\log\frac{\pi(h_w|q)}{\pi'(h_w|q)} - \beta\log\frac{\pi(h_l|q)}{\pi'(h_l|q)} \big), 
\end{equation} 
where $\pi$ and $\pi'$ are the conditional likelihood from the adapted generator and the original generator respectively, and $\sigma(\cdot)$ is the sigmoid function.  

\subsubsection{Retriever Optimization by RLGF}
The retriever needs to make effective use of the augmented query. To this end, we maximize the relevance score between $\boldsymbol{v}'_q$ and $\boldsymbol{v}_{d}$: $\boldsymbol{v}'^T_q \boldsymbol{v}_{d}$. As $\boldsymbol{v}'_q$ is a linear combination of multiple embeddings (Eq. \ref{eq:2}), the following decomposition is made: $\alpha_0 * \boldsymbol{v}_q^T \boldsymbol{v}_{d} + \sum_{L} \alpha_i * \boldsymbol{v}_{h_i}^T \boldsymbol{v}_{d}$. However, optimizing this objective involves two different capabilities from the retriever: 1) \textit{query-to-doc} matching as required by $\boldsymbol{v}_q^T \boldsymbol{v}_{d}$, 2) \textit{doc-to-doc} matching as needed by $\boldsymbol{v}^T_{h_i} \boldsymbol{v}_{d}$. Thus, the direct optimization process is challenging, as it must realize two distinct goals simultaneously. To address this problem, we propose the \textit{proximity objective} ($\rho_{q,d}$) as an alternative: 
\begin{equation}\label{eq:6}
    \rho_{q,d} = \alpha_0 * \boldsymbol{v}_q^T \boldsymbol{v}_{d} + \sum\nolimits_{L} \alpha_i * \boldsymbol{v}_q^T \boldsymbol{v}_{h_i}.
\end{equation} 
The proximity objective maximizes the embedding similarity between the query and hypothetical documents, i.e., $\boldsymbol{v}_q^T \boldsymbol{v}_{h_i}$. Consequently, the retriever focuses solely on the \textit{query-to-doc} matching capability, which makes it easier to optimize. In addition, the above objective leverages $\boldsymbol{v}_q$ as an anchor where both $\boldsymbol{v}_{d}$ and $\boldsymbol{v}_{h_i}$ are moved close to it. Therefore, the similarity between $\boldsymbol{v}_{d}$ and $\boldsymbol{v}_{h_i}$ can also be improved from its optimization. Based on the above definition, we initially formulate the contrastive loss for retriever's training: 
\begin{equation}
    \mathcal{L}^{ctr} = -  \sum\nolimits_{D^*_q} \log \frac{ \exp( \boldsymbol{v}_q^T \boldsymbol{v}_{d} )}{\sum\nolimits_{D'_q} \exp( \boldsymbol{v}_q^T \boldsymbol{v}_{d'} ) },
\end{equation} 
where $D^*_q$ is the entire collection of positive documents to $q$, including $d^*$ and $H_q$, $D'_q$ comprises $d$ $(d^*$ or $H_q)$ and the negative documents to $q$. 

% Knowing that the LLM-based generator can provide precise assessment of relevance due to its inherent re-ranking capability \cite{sun2023chatgpt}, we leverage its feedback for fine-grained training of retriever. Particularly, we apply the following template $\mathcal{T}$: ``\textit{Query: \{q\}. Doc: \{d\}. Whether the doc can fully solve the given query?}'', and make use of the logit of ``\textit{Yes}'' token as the preference to the document: $w_{q,d} \leftarrow \pi^*(\textit{Yes}|\mathcal{T}, q, d)$. Based on this reward,  we introduce the following loss function: 
% \begin{equation}
%     \mathcal{L}^{dst} = - \sum_{D_q} \frac{\exp(w_{q,d})}{\sum\limits_{D_q} \exp(w_{q,d'}) } \log \frac{ \exp( \boldsymbol{v}_q^T \boldsymbol{v}_{d} )}{\sum\limits_{D_q} \exp( \boldsymbol{v}_q^T \boldsymbol{v}_{d'} ) }, 
% \end{equation} 
% where the retriever is reinforced to discriminate the preferred documents via  knowledge distillation ($D_q$: the collection of positive and negative docs).

Knowing that the LLM-based generator can provide precise assessment of relevance due to its inherent re-ranking capability \cite{sun2023chatgpt}, we leverage its feedback for fine-grained training of retriever. Particularly, we apply the following template $\mathcal{T}$: ``\textit{Query: \{q\}. Doc [1]: \{d\_1\}, Doc [2]: \{d\_2\}, \ldots Rank these documents based on their relevance to the query.}'', and obtain the generator's ranking list ``$D_q = d_1, d_2, \dots, d_N$''. Based on this feedback,  we define the following loss function: 
\begin{align}
% \begin{split}
    \mathcal{L}^{dst} = - \frac{1}{\left | D_q \right |} \sum_{d_k = d_1}^{d_N}  \log \frac{ \exp( \boldsymbol{v}_q^T \boldsymbol{v}_{d_k} )}{\sum\limits_{D'_{q,k}} \exp( \boldsymbol{v}_q^T \boldsymbol{v}_{d'} ) }, 
    % \\
% \end{split}
\end{align}
which follows a variational form of knowledge distillation. Here, $D'_{q,k} = d_{k}, ..., d_N$ indicates $d_k$ and the lower ranked documents of $d_k$. By minimizing the above loss, the retriever is reinforced to discriminate the documents preferred by the generator. 


% To simply the computation 

% The objective of self-boosting 
% - improve performance on target domain
% - using unlabeled data 
% using llm to generate synthetic query and fine-tune the model 
% our key novelty: self-boosting 
% RLRF
% - the motivation
% - the process
% RLGF
% - the motivation
% - the process 
% iterative running
% - query generation
% - learn gen fix ret
% - learn rat fix gen  

\begin{table*}[ht]
    \centering
    \resizebox{\linewidth}{!}{%
    \begin{tabular}{l|cccccccc|c|ccccc|c}
        \toprule
         & \multicolumn{9}{c|}{BEIR} & \multicolumn{6}{c}{AIR-Bench} \\
        \cmidrule(lr){2-10} \cmidrule(lr){11-16}
         & FiQA & Scidocs & Fever & Arguana & Scifact & T-Covid & Touche & DBPedia & \textbf{AVG} & Law & News & Health & Finance & ArXiv & \textbf{AVG} \\
        \midrule 
        % \hline
        % \hline         
        \multicolumn{16}{l}{Contriever} \\
        \midrule
        % BM25 & 23.6 & 15.8 & 75.3 & 31.5 & 66.5 & 65.6 & 36.7 & 31.3 & 43.3 & 18.3 & 46.8 & 38.5 & 41.1 & 31.5 & 35.2 & 40.2 \\
        Contriever & 24.5 & 14.9 & 68.2 & 37.9 & 64.9 & 27.3 & 16.7 & 29.2 & 35.4 & 13.2 & 36.2 & 34.3 & 36.2 & 23.0 & 28.6 \\
        % Contriever-ft & 32.9 & 16.5 & 75.8 & 44.6 & 67.7 & 59.6 & 20.4 & 41.3 & 44.9 & 9.1 & 36.2 & 41.3 & 37.3 & 27.1 & 30.2 & 39.2 \\
        \midrule
        HyDE & 26.2 & 12.4 & 69.5 & 40.9 & 66.5 & 59.7 & 15.8 & 33.9 & 40.6 & 11.0 & 31.4 & 33.9 & 27.7 & 21.8 & 25.2 \\
        Doc2query & 25.5 & 15.3 & 69.4 & 39.5 & 65.9 & 29.2 & 12.6 & 29.9 & 35.9 & 12.7 & 37.8 & 34.3 & 37.4 & 23.1 & 29.1 \\
        QGen & 31.6 & 17.8 & 66.9 & 50.1 & 69.0 & 63.3 & 19.0 & 34.3 & 44.0 & 28.0 & 44.9 & 42.2 & 43.6 & 33.4 & 38.4 \\
        GPL & 33.2 & 17.7 & 77.5 & 44.5 & 58.4 & 67.1 & 21.6 & 42.7 & 45.3 & 24.7 & 44.9 & 44.6 & 41.7 & 36.5 & 38.5 \\
        HyDE+QGen & 34.5 & 17.0 & 70.9 & 49.9 & 68.2 & 69.3 & 21.2 & 38.1 & 46.1 & 21.9 & 40.9 & 31.0 & 36.0 & 29.0 & 31.7 \\
        HyDE+GPL & 35.0 & 17.3 & 76.6 & 42.3 & 57.3 & 74.9 & 25.1 & 42.3 & 46.4 & 18.7 & 40.9 & 40.4 & 32.1 & 32.5 & 32.9 \\
        Reinforced-IR & \textbf{36.8} & \textbf{19.2} & \textbf{81.3} & \textbf{52.6} & \textbf{70.9} & \textbf{78.6} & \textbf{31.1} & \textbf{47.5} & \textbf{52.3} & \textbf{28.4} & \textbf{47.6} & \textbf{45.3} & \textbf{46.1} & \textbf{38.4} & \textbf{41.2} \\
        \midrule
        % \hline
        % \hline        
        \multicolumn{16}{l}{BGE-M3} \\
        \midrule
        BGE-M3 & 41.1 & 16.4 & 81.0 & 54.1 & 64.2 & 54.7 & 22.3 & 39.8 & 46.7 & 25.6 & 50.8 & 49.1 & 46.0 & 37.4 & 41.8 \\
        \midrule
        HyDE & 39.2 & 16.9 & 75.8 & 53.2 & 67.2 & 71.7 & 19.6 & 42.5 & 48.3 & 20.7 & 45.6 & 44.1 & 42.7 & 31.6 & 36.9 \\
        Doc2query & 37.7 & 16.8 & 74.8 & 56.0 & 64.3 & 39.4 & 14.0 & 39.5 & 42.8 & 23.8 & 48.4 & 44.7 & 45.7 & 37.8 & 40.1 \\
        QGen & 41.8 & 18.3 & 80.3 & 65.6 & 67.8 & 70.1 & 22.1 & 41.4 & 50.9 & 32.2 & 50.7 & 45.3 & 48.3 & 38.2 & 42.9 \\
        GPL & 43.2 & 18.7 & 79.1 & 65.5 & 65.3 & 74.6 & 24.1 & 41.5 & 51.5 & 27.5 & 50.6 & 51.0 & 45.6 & 37.7 & 42.5 \\
        HyDE+QGen & 42.0 & 19.2 & 75.7 & 60.4 & 68.6 & 78.6 & 22.1 & 42.4 & 51.1 & 27.8 & 46.2 & 40.7 & 46.5 & 31.5 & 38.5 \\
        HyDE+GPL & 40.8 & 19.4 & 75.6 & 58.0 & 67.4 & 78.3 & 23.4 & 42.4 & 50.7 & 20.5 & 44.9 & 44.7 & 41.3 & 30.5 & 36.4 \\
        Reinforced-IR & \textbf{45.8} & \textbf{19.2} & \textbf{84.7} & \textbf{65.1} & \textbf{68.2} & \textbf{83.9} & \textbf{32.4} & \textbf{45.5} & \textbf{55.6} & \textbf{32.5} & \textbf{52.6} & \textbf{51.5} & \textbf{48.9} & \textbf{39.7} & \textbf{45.0} \\
        \bottomrule
    \end{tabular}%
    }
    \vspace{-0.2cm}
    \caption{Overall evaluation (nDCG@10 [\%]) based on BEIR and AIR-Bench datasets.}
    \vspace{-0.3cm} 
    \label{tab:1}
\end{table*} 


\section{Experiment}
% basic setting: contriever & llama3
% - improvements are significant 
% - outperform strong baselines 
% - improvements everywhere  

The experiments are performed for the following research problems. \textbf{RQ 1}. Can Reinforced-IR effectively improve the cross-domain performance over the base retriever? \textbf{RQ 2}. Can Reinforced-IR outperform existing domain-adaptation methods? \textbf{RQ 3}. Whether Reinforced-IR is generally effective with different datasets and model options? \textbf{RQ 4}. Whether the proposed technical designs substantially contribute to the ultimate performance? 

Following the settings in HyDE, We adopt Contriever \cite{izacard2021unsupervised} as our default retriever. Because Contriever is a pre-trained model from unlabeled data, it provides an ideal option to analyze the domain-adaptation effect \cite{gao2022precise}. We also consider Contriever-ft and RetroMAE \cite{xiao2022retromae}, which are fine-tuned from MSMARCO \cite{bajaj2016ms}, as well as BGE M3 \cite{chen2024bge}, GTE \cite{li2023towards}, Stella \cite{zhang2024jasper}, which are fine-tuned from various labeled datasets. 
% \cite{xiao2022retromae} and BGE M3 \cite{chen2024bge}, which are two fine-tuned retrievers from MSMARCO \cite{bajaj2016ms} and a variety of question-answering datasets, respectively.
We leverage Llama-3-8B as our default generator, which is one of the strongest sub 10B LLMs at the time of this paper \cite{dubey2024llama}. We perform extended analysis using both similarly sized LLMs, like Mistral-7B \cite{jiang2023mistral} and Qwen-2.5-7B \cite{hui2024qwen2}, as well as larger and stronger models, including Qwen-2.5-72B, Llama-3-70B, and GPT-4o-mini\footnote{Contriever and Llama-3-8B are set as the default combination of retriever and generator unless specific declaration.}. 

% In self-boosting, we perform three rounds of iteration, with training data increasing progressively by 3k, 8k, and 15k. For all baseline models, the same retriever and generator are employed, while for models requiring training, we utilize the entire 26k of training data.

We evaluate the experiment result with two dataset sources. The first one comprises eight low-resource datasets from BEIR \cite{thakur2021beir}. These datasets have not been fine-tuned by any of the retrievers used in the experiments, making them suitable for assessing cross-domain retrieval performance \cite{gao2022precise}. The second source includes five domain-specific datasets from Air-Bench development sets \cite{chen2024air}. As these datasets were recently produced, they have not been used for training either the retrievers or the generators involved in the experiments. 


\begin{table*}[ht]
    \centering
    \resizebox{\linewidth}{!}{%
    \begin{tabular}{l|cccccccc|c|ccccc|c}
        \toprule
         & \multicolumn{9}{c|}{BEIR} & \multicolumn{6}{c}{AIR-Bench} \\
        \cmidrule(lr){2-10} \cmidrule(lr){11-16}
         & FiQA & Scidocs & Fever & Argu. & Scifact & T-Covid & Touche & DBPedia & \textbf{AVG} & Law & News & Health & Fin. & ArXiv & \textbf{AVG} \\ 
        \midrule
        Contriever & 24.5 & 14.9 & 68.2 & 37.9 & 64.9 & 27.3 & 16.7 & 29.2 & 35.5 & 13.2 & 36.2 & 34.3 & 36.2 & 23.0 & 28.6 \\
        HyDE (Mist, Ctrv) & 25.3 & 12.4 & 69.5 & 38.6 & 69.2 & 55.0 & 19.7 & 35.8 & 40.7 & 11.3 & 33.4 & 32.5 & 28.6 & 21.6 & 25.5 \\
        HyDE (Qwen, Ctrv) & 24.6 & 13.2 & 63.1 & 39.0 & 68.2 & 52.1 & 17.8 & 31.6 & 38.7 & 11.0 & 32.2 & 33.8 & 29.2 & 22.8 & 25.8 \\
        HyDE (Llama, Ctrv) & 26.2 & 12.4 & 69.5 & 40.9 & 66.5 & 59.7 & 15.8 & 33.9 & 40.6 & 11.0 & 31.4 & 33.9 & 27.7 & 21.8 & 25.2 \\
        Ours (Mist, Ctrv) & \textbf{38.0} & \textbf{19.5} & 81.0 & 52.4 & 70.5 & 78.3 & 30.6 & \textbf{47.5} & 52.2 & 29.7 & \textbf{48.2} & \textbf{46.3} & \textbf{46.8} & 38.3 & \textbf{41.9} \\
        Ours (Qwen, Ctrv) & 36.7 & 19.4 & 78.2 & 52.0 & \textbf{71.1} & 75.7 & \textbf{31.2} & 46.2 & 51.3 & \textbf{30.1} & 46.9 & 45.5 & 46.1 & 37.8 & 41.3 \\
        Ours (Llama, Ctrv) & 36.8 & 19.2 & \textbf{81.3} & \textbf{52.6} & 70.9 & \textbf{78.6} & 31.1 & \textbf{47.5} & \textbf{52.3} & 28.4 & 47.6 & 45.3 & 46.1 & \textbf{38.4} & 41.2 \\
        \midrule
        % SimLM & 17.7 & 11.9 & 31.7 & 34.3 & 49.2 & 40.7 & 9.8 & 33.1 & 28.6 & 1.8 & 15.3 & 41.0 & 32.7 & 17.8 & 21.7 \\
        % HyDE (Llama, Sim) & 20.5 & 12.9 & 39.5 & 34.4 & 52.3 & 58.9 & 16.4 & 39.1 & 34.2 & 8.3 & 18.1 & 39.0 & 30.6 & 17.1 & 21.1 \\
        % Ours (Llama, Sim) & \textbf{35.0} & \textbf{18.3} & \textbf{77.8} & \textbf{58.7} & \textbf{66.3} & \textbf{74.3} & \textbf{29.7} & \textbf{43.1} & \textbf{50.4} & \textbf{25.3} & \textbf{43.7} & \textbf{47.7} & \textbf{41.6} & \textbf{34.6} & \textbf{38.6} \\
        % \midrule
        RetroMAE & 31.6 & 15.0 & 77.3 & 43.4 & 65.3 & 77.2 & 23.7 & 39.0 & 46.6 & 14.5 & 45.7 & 44.7 & 41.0 & 34.5 & 36.1 \\
        HyDE (Llama, Ret) & 26.4 & 13.7 & 75.9 & 37.4 & 62.8 & 72.0 & 25.4 & 38.1 & 44.0 & 11.8 & 44.2 & 29.2 & 37.4 & 26.5 & 29.8 \\
        Ours (Llama, Ret) & \textbf{37.5} & \textbf{17.9} & \textbf{86.2} & \textbf{56.2} & \textbf{70.0} & \textbf{82.6} & \textbf{32.7} & \textbf{46.2} & \textbf{53.7} & \textbf{29.7} & \textbf{52.0} & \textbf{46.1} & \textbf{47.3} & \textbf{38.2} & \textbf{42.6} \\
        \midrule
        Contriever-ft & 32.9 & 16.5 & 75.8 & 44.6 & 67.7 & 59.6 & 20.4 & 41.3 & 44.9 & 13.3 & 46.3 & 45.3 & 43.0 & 32.8 & 36.1 \\
        HyDE (Llama, C-ft) & 31.2 & 15.8 & 77.4 & 40.6 & 67.7 & 72.9 & 26.6 & 41.8 & 46.8 & 12.4 & 45.3 & 39.1 & 40.3 & 29.4 & 33.3 \\
        Ours (Llama, C-ft) & \textbf{38.7} & \textbf{18.7} & \textbf{84.4} & \textbf{52.1} & \textbf{70.4} & \textbf{78.5} & \textbf{34.3} & \textbf{46.4} & \textbf{52.9} & \textbf{30.4} & \textbf{48.3} & \textbf{49.6} & \textbf{48.3} & \textbf{37.4} & \textbf{42.8} \\
        \midrule
        GTE-large & 44.6 & 23.4 & \textbf{84.5} & 57.3 & \textbf{74.3} & 70.2 & 25.5 & 42.4 & 52.8 & 16.1 & 46.0 & 51.5 & 43.0 & 36.7 & 38.9 \\
        HyDE (Llama, gte) & 43.5 & \textbf{23.6} & 81.0 & 53.9 & 75.4 & 75.5 & 22.5 & 44.8 & 52.5 & 13.6 & 44.5 & 47.6 & 40.5 & 34.3 & 36.1 \\
        Ours (Llama, gte) & \textbf{46.1} & 23.2 & 84.1 & \textbf{66.8} & 73.8 & \textbf{84.8} & \textbf{31.7} & \textbf{47.3} & \textbf{57.2} & \textbf{30.3} & \textbf{52.8} & \textbf{56.4} & \textbf{48.3} & \textbf{42.7} & \textbf{46.1} \\
        \midrule
        Stella-base-en-v2 & 38.6 & 18.6 & 79.1 & 60.7 & 72.5 & 64.7 & 21.9 & 39.7 & 49.5 & 15.9 & 42.7 & 50.0 & 40.7 & 30.4 & 36.0 \\
        HyDE (Llama, stella) & 37.8 & 21.2 & 71.5 & 55.1 & 73.6 & 80.4 & 25.3 & 42.2 & 50.9 & 13.3 & 43.3 & 46.4 & 39.2 & 30.8 & 34.6 \\
        Ours (Llama, stella) & \textbf{43.7} & \textbf{22.3} & \textbf{84.2} & \textbf{64.7} & \textbf{74.1} & \textbf{84.2} & \textbf{30.1} & \textbf{45.4} & \textbf{56.1} & \textbf{27.3} & \textbf{49.4} & \textbf{53.2} & \textbf{47.1} & \textbf{37.8} & \textbf{43.0} \\
        \midrule
        BGE-M3 & 41.1 & 16.4 & 81.0 & 54.1 & 64.2 & 54.7 & 22.3 & 39.8 & 46.7 & 25.6 & 50.8 & 49.1 & 46.0 & 37.4 & 41.8 \\
        HyDE (Llama, M3) & 39.2 & 16.9 & 75.8 & 53.2 & 67.2 & 71.7 & 19.6 & 42.5 & 48.3 & 20.7 & 45.6 & 44.1 & 42.7 & 31.6 & 36.9 \\
        Ours (Llama, M3) & \textbf{45.8} & \textbf{19.2} & \textbf{84.7} & \textbf{65.1} & \textbf{68.2} & \textbf{83.9} & \textbf{32.4} & \textbf{45.5} & \textbf{55.6} & \textbf{32.5} & \textbf{52.6} & \textbf{51.5} & \textbf{48.9} & \textbf{39.7} & \textbf{45.0} \\
        \bottomrule
    \end{tabular}%
    }
    \vspace{-0.2cm}
    \caption{Extended evaluation based on additional generators and retrievers.} 
    \vspace{-0.3cm}
    \label{tab:exp-2}
\end{table*}


\begin{table*}[ht]
    \centering
    \resizebox{\linewidth}{!}{%
    \begin{tabular}{l|cccccccc|c|ccccc|c}
        \toprule
         & \multicolumn{9}{c|}{BEIR} & \multicolumn{6}{c}{AIR-Bench} \\
        \cmidrule(lr){2-10} \cmidrule(lr){11-16}
         & FiQA & Scidocs & Fever & Argu. & Scifact & T-Covid & Touche & DBPedia & \textbf{AVG} & Law & News & Health & Fin. & ArXiv & \textbf{AVG} \\
        \midrule
        \multicolumn{16}{l}{Contriever} \\
        \midrule
        \multicolumn{16}{l}{\textit{HyDE}} \\
        \quad Llama3-70B & 28.1 & 14.6 & 74.4 & 40.6 & 69.6 & 51.1 & 19.7 & 36.0 & 41.8 & 12.8 & 36.5 & 35.8 & 32.2 & 24.6 & 28.4 \\
        \quad Qwen2.5-72B & 25.5 & 14.1 & 77.7 & 46.6 & 70.1 & 56.8 & 18.3 & 35.1 & 43.0 & 11.6 & 31.2 & 32.3 & 28.3 & 23.2 & 25.3 \\
        \quad GPT-4o-mini & 26.1 & 13.2 & 76.7 & 44.4 & 68.2 & 57.3 & 18.8 & 33.4 & 42.3 & 12.0 & 34.1 & 34.6 & 31.7 & 24.0 & 27.3 \\
        \multicolumn{16}{l}{\textit{HyDE+QGen}} \\
        \quad Llama3-70B & 35.9 & 18.3 & 75.9 & \textbf{51.1} & \textbf{72.8} & 69.8 & 24.1 & 40.5 & 48.6 & 24.0 & 44.0 & 37.9 & 40.0 & 32.4 & 35.7 \\
        \quad Qwen2.5-72B & 35.7 & 18.0 & 79.3 & 55.1 & 72.4 & 71.0 & 22.6 & 38.4 & 49.1  & 22.4 & 37.7 & 33.7 & 34.8 & 30.2 & 31.8 \\
        \quad GPT-4o-mini & 35.8 & 17.5 & 78.4 & 53.4 & 71.2 & 70.3 & 23.8 & 36.7 & 48.4  & 24.5 & 43.0 & 36.0 & 39.4 & 30.9 & 34.8 \\
        \multicolumn{16}{l}{\textit{HyDE+GPL}} \\
        \quad Llama3-70B & 36.3 & 18.4 & 80.9 & 42.9 & 62.9 & 73.7 & 29.0 & 44.1 & 48.5 & 21.1 & 41.9 & 42.4 & 34.7 & 34.4 & 34.9 \\
        \quad Qwen2.5-72B & 35.9 & 18.5 & \textbf{83.2} & 47.1 & 63.9 & 73.0 & 25.6 & 41.3 & 48.6  & 19.7 & 35.8 & 38.3 & 31.2 & 32.8 & 31.6 \\
        \quad GPT-4o-mini & 35.2 & 17.5 & \textbf{83.2} & 44.9 & 63.3 & 73.7 & 27.6 & 40.9 & 48.3 & 20.9 & 41.1 & 41.6 & 34.7 & 34.2 & 34.5 \\
        \midrule
        Reinforced-IR & \textbf{36.8} & \textbf{19.2} & 81.3 & 52.6 & 70.9 & \textbf{78.6} & \textbf{31.1} & \textbf{47.5} & \textbf{52.3} & \textbf{28.4} & \textbf{47.6} & \textbf{45.3} & \textbf{46.1} & \textbf{38.4} & \textbf{41.2} \\
        \midrule
        \multicolumn{16}{l}{BGE-M3} \\
        \midrule
        \multicolumn{16}{l}{\textit{HyDE}} \\
        \quad Llama3-70B & 40.9 & 17.5 & 80.8 & 54.4 & 70.2 & 68.9 & 22.5 & 43.6 & 49.9 & 22.1 & 46.2 & 44.2 & 40.7 & 32.8 & 37.2 \\
        \quad Qwen2.5-72B & 40.6 & 16.9 & 83.9 & 55.1 & 71.2 & 68.6 & 20.8 & 42.9 & 50.0 & 20.9 & 39.5 & 41.9 & 36.2 & 30.4 & 33.8 \\
        \quad GPT-4o-mini & 40.3 & 16.9 & 83.9 & 52.7 & 69.5 & 72.4 & 21.7 & 42.8 & 50.0 & 22.9 & 44.7 & 44.8 & 41.1 & 31.9 & 37.1 \\
        \multicolumn{16}{l}{\textit{HyDE+QGen}} \\
        \quad Llama3-70B & 43.8 & 19.8 & 79.6 & 62.9 & \textbf{71.9} & 75.6 & 26.4 & 43.5 & 52.9 & 28.0 & 45.8 & 42.2 & 44.4 & 33.3 & 38.7 \\
        \quad Qwen2.5-72B & 42.3 & 20.0 & 82.9 & 60.4 & 71.1 & 74.5 & 25.9 & 43.0 & 52.5 & 21.0  & 39.3 & 42.0  & 36.8 & 31.1 & 34.0 \\
        \quad GPT-4o-mini & 42.7 & 19.5 & 83.1 & 57.5 & 71.1 & 81.0 & 28.4 & 42.4 & 53.2 & 22.7 & 44.6 & 45.2 & 40.0 & 31.8 & 36.9 \\
        \multicolumn{16}{l}{\textit{HyDE+GPL}} \\
        \quad Llama3-70B & 42.8 & \textbf{20.4} & 79.9 & 60.4 & 70.3 & 77.7 & 28.7 & 44.0 & 53.0 & 22.4 & 45.1 & 45.0 & 40.8 & 32.4 & 37.1 \\
        \quad Qwen2.5-72B & 39.2 & 20.0 & 75.6 & 60.3 & 63.8 & 69.0 & 25.4 & 42.5 & 49.5 & 21.3 & 39.7 & 41.6 & 36.0 & 31.2 & 34.0 \\
        \quad GPT-4o-mini & 40.2 & 19.6 & 74.3 & 57.2 & 68.1 & 74.3 & 27.2 & 42.6 & 50.4 & 22.6 & 44.9 & 44.7 & 40.3 & 32.0 & 36.9 \\
        \midrule
        Reinforced-IR & \textbf{45.8} & 19.2 & \textbf{84.7} & \textbf{65.1} & 68.2 & \textbf{83.9} & \textbf{32.4} & \textbf{45.5} & \textbf{55.6} & \textbf{32.5} & \textbf{52.6} & \textbf{51.5} & \textbf{48.9} & \textbf{39.7} & \textbf{45.0} \\ 
        \bottomrule
    \end{tabular}%
    }
    \vspace{-0.2cm}
    \caption{Extended evaluation based on larger LLMs.} 
    \vspace{-0.3cm}
    \label{tab:exp-3}
\end{table*}


% \begin{table*}[ht]
%     \centering
%     \resizebox{\linewidth}{!}{%
%     \begin{tabular}{l|cccccccc|c|ccccc|c}
%         \toprule
%          & \multicolumn{9}{c|}{BEIR} & \multicolumn{6}{c}{AIR-Bench} \\
%         \cmidrule(lr){2-10} \cmidrule(lr){11-16}
%          & FiQA & Scidocs & Fever & Argu. & Scifact & T-Covid & Touche & DBPedia & {AVG} & Law & News & Health & Fin. & ArXiv & {AVG} \\
%         \midrule
%         Base model & 24.5 & 14.9 & 68.2 & 37.9 & 64.9 & 27.3 & 16.7 & 29.2 & 35.4 & 2.3 & 16.9 & 27.1 & 25.3 & 11.3 & 16.6 \\
%         \midrule
%         RLRF only & 28.5 & 15.4 & 74.6 & 40.2 & 66.7 & 41.4 & 19.3 & 34.9 & 40.1 & 2.7 & 17.7 & 30.7 & 24.2 & 12.6 & 17.6 \\
%         RLGF only & 35.6 & 19.1 & 75.7 & 52.7 & 70.6 & 74.7 & 27.9 & 45.7 & 50.3 & 25.1 & 43.6 & 42.5 & 39.9 & 32.3 & 36.7 \\
%         RLRF \& RLGF & 36.8 & 19.2 & 81.3 & 52.6 & 70.9 & 78.6 & 31.1 & 47.5 & 52.3 & 28.3 & 43.8 & 43.2 & 40.7 & 33.4 & 37.9 \\ 
%         \bottomrule
%     \end{tabular}%
%     }
%     \vspace{-0.1cm}
%     \caption{.}
%     \vspace{-0.1cm}
%     \label{tab:exp-5}
% \end{table*}



\subsection{Experiment Analysis} 
The experiment results are analyzed in comparison with two classes of baselines. The first class relies on generative augmentation, including HyDE \cite{gao2022precise}, which augments the query with hypothetical documents, and Doc2query \cite{nogueira2019document}, which augments the document with pseudo queries. The second class leverages continual fine-tuning, including QGen \cite{ma2020zero,thakur2021beir}, which fine-tunes the retriever based on synthetic queries obtained from the target domain by contrastive learning, and GPL, which performs knowledge distillation for fine-grained training. For the sake of fair comparison, all methods (baselines and Reinforced-IR) are applied to the same set of synthetic queries and the same generator and retriever backbones. 


\subsubsection{Overall Evaluation} 
The overall evaluation is demonstrated in Table \ref{tab:1}, where the following analysis is made.  

$\bullet$ \textit{Improvement over base retrievers}. Reinforced-IR substantially improves the base retrievers' cross-domain retrieval performances across all datasets. This effect is particularly evident with Contriever, a pre-trained model from massive unlabeled data. Specifically, it enables the average performance to be improved from 35.4 to 52.3 on BEIR. Moreover, it achieves even larger improvements on AIR-Bench, with the average performance raised from 28.6 to 41.2. Although another base retriever, BGE M3, has been broadly fine-tuned with various question answering datasets, Reinforced-IR still contributes to its performance, increasing its average performance from 46.7 to 55.6 on BEIR, and from 41.8 to 45.0 on AIR-Bench, respectively. The improvement on BGE M3 is remarkable, considering that a broadly fine-tuned retriever has already gained a good command of necessary knowledge on the target domain, where traditional domain adaptation methods struggle to make further improvements \cite{gao2022precise}. 

$\bullet$ \textit{Improvements over domain-adaptation baselines}. Reinforced-IR also demonstrates significant advantages over existing domain-adaptation methods. Notably, it outperforms both generative augmentation methods (HyDE, Doc2Query) and continual fine-tuning methods (QGen, GPL) individually, as well as the combination of the two methods (HyDE+QGen, HyDE+GPL). A closer analysis of the experimental results reveals that the baseline methods struggle to deliver consistent improvements across different datasets and base retrievers. For instance, while HyDE enhances Contriever’s performance on BEIR, it contributes little to BGE M3, as the latter has already undergone extensive fine-tuning. Besides, the benefits of more advanced fine-tuning operations, such as those employed in GPL, are significantly diminished when applied to BGE M3. Additionally, the native combination of HyDE and fine-tuning based methods yields minimal benefit, probably due to the discrepancy between the two strategies. These observations highlight the the limitations of existing domain-adaptation methods, particularly regarding their applicability and significance. In contrast, Reinforced-IR addresses these challenges through its self-boosting mechanism, which effectively mitigates such issues and drives substantial progress. 


\begin{table*}[h]
    \centering
    \resizebox{\linewidth}{!}{%
    \begin{tabular}{l|cccccccc|c|ccccc|c}
        \toprule
        & \multicolumn{9}{c|}{BEIR} & \multicolumn{6}{c}{AIR-Bench} \\
        \cmidrule(lr){2-10} \cmidrule(lr){11-16} 
        & FiQA & Scidocs & Fever & Argu. & Scifact & T-Covid & Touche & DBPedia & {AVG} & Law & News & Health & Fin. & ArXiv & {AVG} \\
        \midrule
        Base model & 24.5 & 14.9 & 68.2 & 37.9 & 64.9 & 27.3 & 16.7 & 29.2 & 35.4 & 13.2  & 36.2 & 34.3 & 36.2 & 23.0 & 28.6 \\
        Ret: 0, Gen: 1 & 28.9 & 15.6 & 74.0 & 40.3 & 67.1 & 42.9 & 20.3 & 34.9 & 40.5 & 13.6 & 39.3 & 36.8 & 34.2 & 24.9 & 29.8 \\
        Ret: 1, Gen: 1 & 29.8 & 17.6 & 72.2 & 43.2 & 69.8 & 64.6 & 25.3 & 35.1 & 44.7 & 18.1 & 36.6 & 38.4 & 40.7 & 28.7 & 32.5 \\
        Ret: 1, Gen: 2 & 30.8 & 17.7 & 74.9 & 42.6 & 69.9 & 69.4 & 25.2 & 35.7 & 45.8 & 18.5 & 38.0 & 38.3 & 41.2 & 29.6 & 33.1 \\
        Ret: 2, Gen: 2 & 33.5 & 18.8 & 78.2 & 49.1 & 70.4 & 74.8 & 30.4 & 43.9 & 49.9 & 25.6 & 45.0 & 43.3 & 44.8 & 37.1 & 39.2 \\
        Ret: 2, Gen: 3 & 33.6 & 19.1 & 79.1 & 49.7 & 70.5 & 73.6 & 31.0 & 45.2 & 50.2 & 26.2 & 45.8 & 43.6 & 44.8 & 37.4 & 39.6 \\
        Ret: 3, Gen: 3 & 36.8 & 19.2 & 81.3 & 52.6 & 70.9 & 78.8 & 31.1 & 47.5 & 52.3 & 28.4 & 47.6 & 45.3 & 46.1 & 38.4 & 41.2 \\
        \bottomrule
    \end{tabular}%
    }
    \vspace{-0.2cm}
    \caption{Impact from iterative optimization of generator (Gen) and retriever (Ret).}
    \vspace{-0.3cm}
    \label{tab:exp-4}
\end{table*}  


\subsection{Extended Evaluation}
We conduct the extended experiments to explore Reinforced-IR's effectiveness under more situations. 

$\bullet$ \textit{Analysis of different backbones}. We study the impact from using different generators and retrievers, as demonstrated in Table \ref{tab:exp-2}. Our experiment compares three LLMs of similar sizes, including Llama-3 8B (Llama), Qwen-2.5 7B (Qwen), and Mistral 7B (Mist). Besides, we also consider the following types of retrievers: 1) pre-trained model: Contriever, 2) models fine-tuned only with MS MARCO: Contriever-ft, RetroMAE, and 3) broadly fine-tuned models with various datasets: BGE M3, GTE (large), and Stella-v2. From these evaluations, we derive several key observations. 

First, Reinforced-IR consistently outperforms both the base retriever and HyDE baseline when working with different LLM backbones. Despite some underlying differences, e.g., LLama-3 and Mistral are pre-trained more comprehensively than Qwen-2.5, and Qwen-2.5 is more of a bi-lingual LLM compared to the other two models, all methods converge to a superior performance through Reinforced-IR. 
In contrast, none of the HyDE alternatives surpasses Contriever on AIR-Bench, underscoring the incapability of existing generative augmentation methods in dealing with new tasks. 

Second, Reinforced-IR achieves a significant advantage over the baselines when applied to pre-trained and MS MARCO-finetuned retrievers. This result highlights Reinforced-IR's effect in enhancing cross-domain retrieval performance. Moreover, Reinforced-IR also makes substantial contributions to the broadly finetuned retrievers, particularly on AIR-Bench datasets, which demonstrates its generally applicability across diverse application scenarios. 

$\bullet$ \textit{Analysis of larger LLMs}. We incorporate three powerful LLMs to the experiment (Table \ref{tab:exp-4}): Llama3-70B and Qwen2.5-72B, and GPT-4o-mini. This allows us to explore the optimal effect of the existing generative augmentation methods. Our evaluation includes both HyDE and its combinations with other approaches. 

Our experimental results reveal that the use of powerful LLMs can enhance baseline performance in certain scenarios, such as Contriever and BGE M3's retrieval performance on BEIR. However, these improvements are inconsistent across different datasets. Besides, there remains a large performance gap between these methods and Reinforced-IR in most cases. These results highlight that 
it's not enough to simply count on the increased capacity of LLMs. Instead, they underscore the necessity of jointly adapting LLMs and retrievers to optimize the cross-domain retrieval performance. 

$\bullet$ \textit{Analysis of self-boosting}. 
We evaluate the impact of self-boosting by analyzing Reinforced-IR's performance growth throughout the training process (Table \ref{tab:exp-4}). Specifically, the complete set of training queries is divided into three subsets. For each subset, we conduct one self-boosting iteration, consisting of a round of generator optimization via RLRF (Gen-$i$), followed by a round of retriever optimization through RLGF (Ret-$i$). 

The experimental results demonstrate that both self-boosting operations contributes substantially to the improvement of retrieval performance. In each iteration, the optimization of the generator enables the production of more effective query augmentations for the current retriever, improving the performance from ``\textit{Ret}-$i$, \textit{Gen}-$i$'' to ``\textit{Ret}-$i$, \textit{Gen}-($i$+1)''. While the optimization of retriever allows it to make better use of the augmented queries from the current generator, which further improves the performance from ``\textit{Ret}-$i$, \textit{Gen}-($i$+1)'' to ``\textit{Ret}-($i$+1), \textit{Gen}-($i$+1)''. This iterative refinement ultimately results in Reinforced-IR's superior performance across the entire training dataset. 

\subsubsection{Ablation Study} 
We make detailed analysis for Reinforced-IR's technical factors with the ablation study in Table \ref{tab:ablation}. 

$\bullet$ \textit{Training methods}. In our experiment, we replace the original DPO with supervised fine-tuning for the generator's training, using the winning candidate as the supervision label (w/o DPO). Additionally, we substituted basic contrastive learning for knowledge distillation during retriever's training (w/o Distillation). The experiment result shows that both modifications lead to significant decline of empirical performance on the two evaluation benchmarks. This decline is attributed to the alternative methods' inability to incorporate the fine-grained feedback from the generator and retriever, specifically, the usability of augmented queries and document relevance, thereby hindering the effective utilization of training data.  

$\bullet$ \textit{Impact of proximity objective}. We further replace the proximity objective in Eq. \ref{eq:6} with the basic objective: $\alpha_0 * \boldsymbol{v}_q^T \boldsymbol{v}_{d} + \sum_{L} \alpha_i * \boldsymbol{v}_{h_i}^T \boldsymbol{v}_{d}$. As discussed, this alternative form requires the model to accomplish both query-to-doc and doc-to-doc matching, thus increasing the training difficulty. Our experiment result verifies proximity objective's overall effectiveness in general scenarios, as the alternative method (w/o Proximity) significantly reduces the performance on AIR-Bench, which solely comprises question-answering style tasks, while leading to a neural impact on BEIR, which constitutes miscellaneous tasks. 

$\bullet$ \textit{Candidate filtering}. We disable the filtering rules in Eq. \ref{eq:4} and make direct use of the unfiltered candidates (w/o Filtering rule-1, w/o Filtering rule-2). The experiment result highlights the significance of both rules on BEIR's performance. This can be attributed to BEIR's diverse retrieval tasks, which increases the likelihood of generating unsuitable query augmentation from the generator. As such, the filtering operations are essential to optimizing the performance. In contrast, AIR-Bench focuses solely on question-answering tasks, allowing for more reliable query augmentation and diminishing the need for candidate filtering.  

% \begin{table}[t]
%     \centering
%     \resizebox{\linewidth}{!}{
%     % \small
%     \begin{tabular}{l|cc|cc}
%         \toprule
%          % & \multicolumn{6}{c|}{AIR-Bench} \\
%         % \cmidrule(lr){2-7}
%          & BEIR & $\Delta$ & AIR-Bench & $\Delta$ \\
%         \midrule
%         Reinforced-IR (default) & 52.3 & - & 38.0 & - \\
%         \midrule
%         \quad w/o DPO & 50.6 & -1.7 & 36.1 & -1.9 \\
%         \quad w/o Distillation & 49.3 & -3.0 & 34.8 & -3.2 \\         
%         \quad w/o Proximity & 52.3 & -0.0 & 35.0 & -3.0 \\
%         \quad w/o Filtering rule-1 & 49.0 & -3.3 & 38.3 & +0.3 \\
%         \quad w/o Filtering rule-2 & 49.7 & -2.6 & 38.5 & +0.5 \\
%         % \quad w/o linear combination & 52.8 & +0.5 & 35.2 & -1.7 \\
%         \bottomrule
%     \end{tabular}%
%     }
%     \vspace{-0.2cm}
%     \caption{Ablation studies.} 
%     \vspace{-0.2cm} 
%     \label{tab:ablation}
% \end{table} 

\begin{table}[t]
    \centering
    \resizebox{\linewidth}{!}{
    % \small
    \begin{tabular}{l|cc|cc}
        \toprule
         % & \multicolumn{6}{c|}{AIR-Bench} \\
        % \cmidrule(lr){2-7}
         & BEIR & $\Delta$ & AIR-Bench & $\Delta$ \\
        \midrule
        Reinforced-IR (default) & 52.3 & - & 41.2 & - \\
        \midrule
        \quad w/o DPO & 50.6 & -1.7 & 40.6 & -0.6 \\
        \quad w/o Distillation & 49.3 & -3.0 & 38.9 & -2.3 \\         
        \quad w/o Proximity & 52.3 & -0.0 & 40.2 & -1.0 \\
        \quad w/o Filtering rule-1 & 49.0 & -3.3 & 41.1 & -0.1 \\
        \quad w/o Filtering rule-2 & 49.7 & -2.6 & 40.9 & -0.3 \\
        \bottomrule
    \end{tabular}%
    }
    \vspace{-0.2cm}
    \caption{Ablation studies.} 
    \vspace{-0.6cm} 
    \label{tab:ablation}
\end{table} 


\section{Related Work} 
% \vspace{-0.1cm}
% The related works are discussed from two aspects: dense retrieval and domain adaptation techniques for cross-domain retrieval.  
% Given the fast-growing demands in practice, it becomes imperative for retrieval models to handle diversified applications beyond their training scenarios. To examine the models' cross-domain retrieval performance, a series of evaluation benchmarks were introduced by the community. For example, one pioneering progress was made by BEIR \cite{thakur2021beir}, where comprehensive evaluation datasets are collected, covering various types of retrieval tasks (e.g., question-answering, document retrieval) and source data (such as Wikipedia, medical, science). In addition, MICRAL incorporates multi-lingual datasets to evaluate the model's retrieval performance across different languages, while AIR-Bench leverages LLM-generated data to facilitate the cross-domain evaluations for people's interested domains \cite{chen2024air}. According to the evaluation results on these benchmarks, the pre-trained retrievers are prone to inferior performances when directly applied for new working scenarios. 
% With the rapidly growing demands in practice, it becomes imperative for retrieval models to handle diverse applications beyond their training scenarios. To evaluate models' cross-domain retrieval performance, the research community has introduced a series of evaluation benchmarks. For instance, a pioneering contribution was made by BEIR \cite{thakur2021beir}, which compiled comprehensive datasets, including various retrieval tasks (e.g., question-answering, document retrieval) and diverse data sources (e.g., Wikipedia, medical, and scientific domains). Additionally, MICRAL incorporates multilingual datasets to assess retrieval performance across different languages \cite{zhang2023miracl}, while AIR-Bench utilizes LLM-generated data to facilitate cross-domain evaluations for arbitrary domains \cite{chen2024air}. Evaluation results on these benchmarks reveal that pre-trained retrievers often exhibit inferior performance when applied directly to new scenarios. 
Cross-domain retrieval is an important but challenging problem for existing techniques. As demonstrated by the popular benchmarks in this field \cite{thakur2021beir}, the pre-trained retrievers are prone to inferior performances when they are applied directly for a new working scenario. To tackle this challenge, one common strategy is to perform multi-task training, where a pre-trained retriever is broadly fine-tuned using extensive labeled datasets. By learning from diverse tasks during training, the retriever develops a stronger ability to handle new tasks during testing \cite{wang2022text,xiao2024c,su2022one}. However, multi-task retrievers often make trade-offs between individual tasks to optimize overall performance, leading to significant performance gaps when compared to specialized retrievers in target domains. 

Another line of research focuses on the continual fine-tuning of pre-trained retrievers using synthetic data generated from a target domain \cite{ma2020zero,thakur2021beir,wang2021gpl}. These approaches leverage generators to produce synthetic queries for unlabeled documents, which creates training samples to fine-tune the pre-trained models. Thanks to the popularity of language models, it's made possible to produce synthetic queries at scale, \cite{bonifacio2022inpars,jeronymo2023inpars,wang2023improving}, enabling the corresponding approaches to be easily conducted in practice. However, these approaches could deliver limited performance gains due to the potential mismatch between the synthetic data and actual scenarios. 

Different from the fine-tuning methods, generation augmented retrieval (GAR) makes direct use of generation models to address the cross-domain problems \cite{mao2020generation}. These methods enrich query and document with extra information, enabling relevant data to identified in a straightforward way. 
Nowadays, large language models are widely adopted as the backbone generator \cite{gao2022precise,wang2023query2doc}, which contributes to the performance and applicability of corresponding methods. Although GAR is widely perceived as a promising strategy, it's not enough solely rely on general LLMs, as they still lack necessary knowledge required to generate effective query augmentations for highly specialized problems.  

\section{Conclusion} 
\vspace{-0.2cm}
In this paper, we introduce Reinforced-IR, a novel self-boosting framework for cross-domain retrieval. Our method employs two advanced learning algorithms: RLRF and RLGF. These algorithms enable the generator and retriever to mutually reinforce each other through feedback, leading to a progressive enhancement of retrieval performance. The effectiveness of Reinforced-IR is thoroughly validated, as it outperforms existing domain adaptation methods by a huge advantage, delivering superior performance across various application scenarios. 

% \section*{Limitations}
% While Reinforced-IR achieves a substantial improvement over existing cross-domain retrieval methods, it can still be improved from several perspectives. First, the current approach assumes the presence of a unlabeled corpus from the target domain. Although this is a very mild requirement which can be easily satisfied in practice, a corpus-free extension will further improve the applicability of this approach. Second, the introduction of additional costs is a common problem for all existing GAR methods. As a result, the future improvement will target on query augmentations of not only high retrieval utility but also low generation costs. 

% Our work presents two limitations. Firstly, it is currently limited to English-centric datasets, highlighting the need for future exploration in multilingual scenarios. Secondly, while Reinforced-IR introduces minimal computational overhead during inference, the training process incurs additional computational overhead, which requires optimization in future research. 

% \begin{table*}[ht]
%     \centering
%     \resizebox{\linewidth}{!}{%
%     \begin{tabular}{l|ccccccccc|cccccc|c}
%         \toprule
%          & \multicolumn{9}{c|}{BEIR} & \multicolumn{6}{c|}{AIR-Bench} & \multirow{2}{*}{\textbf{AVG}} \\
%         \cmidrule(lr){2-10} \cmidrule(lr){11-16}
%          & FiQA & Scidocs & FEVER & Arguana & Scifact & TREC-COVID & Touche & DBPedia & \textbf{avg} & law & news & healthcare & finance & arxiv & \textbf{avg} & \\
%         \midrule
%         Contriever & 24.5 & 14.9 & 68.2 & 37.9 & 64.9 & 27.3 & 16.7 & 29.2 & 35.4 & 2.3 & 16.9 & 27.1 & 25.3 & 11.3 & 16.6 & 28.2 \\
%         \midrule
%         w/ RLRF & 28.5 & 15.4 & 74.6 & 40.2 & 66.7 & 41.4 & 19.3 & 34.9 & 40.1 & 2.7 & 17.7 & 30.7 & 24.2 & 12.6 & 17.6 & 31.5 \\
%         w/ RLGF & 35.6 & 19.1 & 75.7 & 52.7 & 70.6 & 74.7 & 27.9 & 45.7 & 50.3 & 25.1 & 43.6 & 42.5 & 39.9 & 32.3 & 36.7 & 45.0 \\
%         w/ RLRF \& RLGF & 36.8 & 19.2 & 81.3 & 52.6 & 70.9 & 78.6 & 31.1 & 47.5 & 52.3 & 28.3 & 43.8 & 43.2 & 40.7 & 33.4 & 37.9 & 46.7 \\
%         \bottomrule
%     \end{tabular}%
%     }
%     \vspace{0.1cm}
%     \caption{The performance of each technology on the BEIR and AIR-Bench benchmarks.}
%     \label{technology}
% \end{table*}


% - presents new self-boosting framework for cross-domain retrieval
% - enables the improvement of retrieval performance using unlabeled corpus from target domain 
% - formulates the joint optimization framework, where the generator and retriever can be mutually reinforced from each other's feedback 
% leading to progressive improvement of retrieval performance 
% experiment study validates the effectiveness 
% leading to notable improvements for a variety of popular retrievers  
% outperforming the existing domain-adaptation methods with significant advantages 



% \clearpage
% \newpage
% \putsec{related}{Related Work}

\noindent \textbf{Efficient Radiance Field Rendering.}
%
The introduction of Neural Radiance Fields (NeRF)~\cite{mil:sri20} has
generated significant interest in efficient 3D scene representation and
rendering for radiance fields.
%
Over the past years, there has been a large amount of research aimed at
accelerating NeRFs through algorithmic or software
optimizations~\cite{mul:eva22,fri:yu22,che:fun23,sun:sun22}, and the
development of hardware
accelerators~\cite{lee:cho23,li:li23,son:wen23,mub:kan23,fen:liu24}.
%
The state-of-the-art method, 3D Gaussian splatting~\cite{ker:kop23}, has
further fueled interest in accelerating radiance field
rendering~\cite{rad:ste24,lee:lee24,nie:stu24,lee:rho24,ham:mel24} as it
employs rasterization primitives that can be rendered much faster than NeRFs.
%
However, previous research focused on software graphics rendering on
programmable cores or building dedicated hardware accelerators. In contrast,
\name{} investigates the potential of efficient radiance field rendering while
utilizing fixed-function units in graphics hardware.
%
To our knowledge, this is the first work that assesses the performance
implications of rendering Gaussian-based radiance fields on the hardware
graphics pipeline with software and hardware optimizations.

%%%%%%%%%%%%%%%%%%%%%%%%%%%%%%%%%%%%%%%%%%%%%%%%%%%%%%%%%%%%%%%%%%%%%%%%%%
\myparagraph{Enhancing Graphics Rendering Hardware.}
%
The performance advantage of executing graphics rendering on either
programmable shader cores or fixed-function units varies depending on the
rendering methods and hardware designs.
%
Previous studies have explored the performance implication of graphics hardware
design by developing simulation infrastructures for graphics
workloads~\cite{bar:gon06,gub:aam19,tin:sax23,arn:par13}.
%
Additionally, several studies have aimed to improve the performance of
special-purpose hardware such as ray tracing units in graphics
hardware~\cite{cho:now23,liu:cha21} and proposed hardware accelerators for
graphics applications~\cite{lu:hua17,ram:gri09}.
%
In contrast to these works, which primarily evaluate traditional graphics
workloads, our work focuses on improving the performance of volume rendering
workloads, such as Gaussian splatting, which require blending a huge number of
fragments per pixel.

%%%%%%%%%%%%%%%%%%%%%%%%%%%%%%%%%%%%%%%%%%%%%%%%%%%%%%%%%%%%%%%%%%%%%%%%%%
%
In the context of multi-sample anti-aliasing, prior work proposed reducing the
amount of redundant shading by merging fragments from adjacent triangles in a
mesh at the quad granularity~\cite{fat:bou10}.
%
While both our work and quad-fragment merging (QFM)~\cite{fat:bou10} aim to
reduce operations by merging quads, our proposed technique differs from QFM in
many aspects.
%
Our method aims to blend \emph{overlapping primitives} along the depth
direction and applies to quads from any primitive. In contrast, QFM merges quad
fragments from small (e.g., pixel-sized) triangles that \emph{share} an edge
(i.e., \emph{connected}, \emph{non-overlapping} triangles).
%
As such, QFM is not applicable to the scenes consisting of a number of
unconnected transparent triangles, such as those in 3D Gaussian splatting.
%
In addition, our method computes the \emph{exact} color for each pixel by
offloading blending operations from ROPs to shader units, whereas QFM
\emph{approximates} pixel colors by using the color from one triangle when
multiple triangles are merged into a single quad.



% \newpage
% The experimental results are in exp.tex
% 
%%%%%%%%%%%%%%%%%%%%%%%%%%%%%%%%%%%%%%%%%%%%%%%%%%%%%%%%%%%%%%%%%%%%%%%%%%%%%%%%%%%%%%%%%%%%%%%%%%%%%%

%%%%%%%%%%%%%%%%%%%%%%%%%%%%%%%%%%%%%%%%%%%%%%
\begin{table*}[t]
\setlength{\tabcolsep}{3pt}
\centering
\renewcommand{\arraystretch}{1.1}
\tabcolsep=0.2cm
\begin{adjustbox}{max width=\textwidth}  % Set the maximum width to text width
\begin{tabular}{c| cccc ||  c| cc cc}
\toprule
General & \multicolumn{3}{c}{Preference} & Accuracy & Supervised & \multicolumn{3}{c}{Preference} & Accuracy \\ 
LLMs & PrefHit & PrefRecall & Reward & BLEU & Alignment & PrefHit & PrefRecall & Reward & BLEU \\ 
\midrule
GPT-J & 0.2572 & 0.6268 & 0.2410 & 0.0923 & Llama2-7B & 0.2029 & 0.803 & 0.0933 & 0.0947 \\
Pythia-2.8B & 0.3370 & 0.6449 & 0.1716 & 0.1355 & SFT & 0.2428 & 0.8125 & 0.1738 & 0.1364 \\
Qwen2-7B & 0.2790 & 0.8179 & 0.1593 & 0.2530 & Slic & 0.2464 & 0.6171 & 0.1700 & 0.1400 \\
Qwen2-57B & 0.3086 & 0.6481 & 0.6854 & 0.2568 & RRHF & 0.3297 & 0.8234 & 0.2263 & 0.1504 \\
Qwen2-72B & 0.3212 & 0.5555 & 0.6901 & 0.2286 & DPO-BT & 0.2500 & 0.8125 & 0.1728 & 0.1363 \\ 
StarCoder2-15B & 0.2464 & 0.6292 & 0.2962 & 0.1159 & DPO-PT & 0.2572 & 0.8067 & 0.1700 & 0.1348 \\
ChatGLM4-9B & 0.2246 & 0.6099 & 0.1686 & 0.1529 & PRO & 0.3025 & 0.6605 & 0.1802 & 0.1197 \\ 
Llama3-8B & 0.2826 & 0.6425 & 0.2458 & 0.1723 & \textbf{\shortname}* & \textbf{0.3659} & \textbf{0.8279} & \textbf{0.2301} & \textbf{0.1412} \\ 
\bottomrule
\end{tabular}
\end{adjustbox}
\caption{Main results on the StaCoCoQA. The left shows the performance of general LLMs, while the right presents the performance of the fine-tuned LLaMA2-7B across various strong benchmarks for preference alignment. Our method SeAdpra is highlighted in \textbf{bold}.}
\label{main}
\vspace{-0.2cm}
\end{table*}
%%%%%%%%%%%%%%%%%%%%%%%%%%%%%%%%%%%%%%%%%%%%%%%%%%%%%%%%%%%%%%%%%%%%%%%%%%%%%%%%%%%%%%%%%%%%%%%%%%%%
\begin{table}[h]
\centering
\renewcommand{\arraystretch}{1.02}
% \tabcolsep=0.1cm
\begin{adjustbox}{width=0.48\textwidth} % Adjust table width
\begin{tabularx}{0.495\textwidth}{p{1.2cm} p{0.7cm} p{0.95cm}p{0.95cm}p{0.7cm}p{0.7cm}}
     \toprule
    \multirow{2}{*}{\small \textbf{Dataset}} & \multirow{2}{*}{\small Model} & \multicolumn{2}{c}{\small Preference} & \multicolumn{2}{c}{\small Acc } \\ 
    & & \small \textit{PrefHit} & \small \textit{PrefRec} & \small \textit{Reward} & \small \textit{Rouge} \\ 
    \midrule
    \multirow{2}{*}{\small \textbf{Academia}}   & \small PRO & 33.78 & 59.56 & 69.94 & 9.84 \\ 
                                & \small \textbf{Ours} & 36.44 & 60.89 & 70.17 & 10.69 \\ 
    \midrule
    \multirow{2}{*}{\small \textbf{Chemistry}}  & \small PRO & 36.31 & 63.39 & 69.15 & 11.16 \\ 
                                & \small \textbf{Ours} & 38.69 & 64.68 & 69.31 & 12.27 \\ 
    \midrule
    \multirow{2}{*}{\small \textbf{Cooking}}    & \small PRO & 35.29 & 58.32 & 69.87 & 12.13 \\ 
                                & \small \textbf{Ours} & 38.50 & 60.01 & 69.93 & 13.73 \\ 
    \midrule
    \multirow{2}{*}{\small \textbf{Math}}       & \small PRO & 30.00 & 56.50 & 69.06 & 13.50 \\ 
                                & \small \textbf{Ours} & 32.00 & 58.54 & 69.21 & 14.45 \\ 
    \midrule
    \multirow{2}{*}{\small \textbf{Music}}      & \small PRO & 34.33 & 60.22 & 70.29 & 13.05 \\ 
                                & \small \textbf{Ours} & 37.00 & 60.61 & 70.84 & 13.82 \\ 
    \midrule
    \multirow{2}{*}{\small \textbf{Politics}}   & \small PRO & 41.77 & 66.10 & 69.52 & 9.31 \\ 
                                & \small \textbf{Ours} & 42.19 & 66.03 & 69.74 & 9.38 \\ 
    \midrule
    \multirow{2}{*}{\small \textbf{Code}} & \small PRO & 26.00 & 51.13 & 69.17 & 12.44 \\ 
                                & \small \textbf{Ours} & 27.00 & 51.77 & 69.46 & 13.33 \\ 
    \midrule
    \multirow{2}{*}{\small \textbf{Security}}   & \small PRO & 23.62 & 49.23 & 70.13 & 10.63 \\ 
                                & \small \textbf{Ours} & 25.20 & 49.24 & 70.92 & 10.98 \\ 
    \midrule
    \multirow{2}{*}{\small \textbf{Mean}}       & \small PRO & 32.64 & 58.05 & 69.64 & 11.51 \\ 
                                & \small \textbf{Ours} & \textbf{34.25} & \textbf{58.98} & \textbf{69.88} & \textbf{12.33} \\ 
    \bottomrule
\end{tabularx}
\end{adjustbox}
\caption{Main results (\%) on eight publicly available and popular CoQA datasets, comparing the strong list-wise benchmark PRO and \textbf{ours with bold}.}
\label{public}
\end{table}



%%%%%%%%%%%%%%%%%%%%%%%%%%%%%%%%%%%%%%%%%%%%%%%%%%%%%
\begin{table}[h]
\centering
\renewcommand{\arraystretch}{1.02}
\begin{tabularx}{0.48\textwidth}{p{1.45cm} p{0.56cm} p{0.6cm} p{0.6cm} p{0.50cm} p{0.45cm} X}
\toprule
\multirow{2}{*}{Method} & \multicolumn{3}{c}{Preference \((\uparrow)\)} & \multicolumn{3}{c}{Accuracy \((\uparrow)\)} \\ \cmidrule{2-4} \cmidrule{5-7}
& \small PrefHit & \small PrefRec & \small Reward & \small CoSim & \small BLEU & \small Rouge \\ \midrule
\small{SeAdpra} & \textbf{34.8} & \textbf{82.5} & \textbf{22.3} & \textbf{69.1} & \textbf{17.4} & \textbf{21.8} \\ 
\small{-w/o PerAl} & \underline{30.4} & 83.0 & 18.7 & 68.8 & \underline{12.6} & 21.0 \\
\small{-w/o PerCo} & 32.6 & 82.3 & \underline{24.2} & 69.3 & 16.4 & 21.0 \\
\small{-w/o \(\Delta_{Se}\)} & 31.2 & 82.8 & 18.6 & 68.3 & \underline{12.4} & 20.9 \\
\small{-w/o \(\Delta_{Po}\)} & \underline{29.4} & 82.2 & 22.1 & 69.0 & 16.6 & 21.4 \\
\small{\(PerCo_{Se}\)} & 30.9 & 83.5 & 15.6 & 67.6 & \underline{9.9} & 19.6 \\
\small{\(PerCo_{Po}\)} & \underline{30.3} & 82.7 & 20.5 & 68.9 & 14.4 & 20.1 \\ 
\bottomrule
\end{tabularx}
\caption{Ablation Results (\%). \(PerCo_{Se}\) or \(PerCo_{Po}\) only employs Single-APDF in Perceptual Comparison, replacing \(\Delta_{M}\) with \(\Delta_{Se}\) or \(\Delta_{Po}\). The bold represents the overall effect. The underlining highlights the most significant metric for each component's impact.}
\label{ablation}
% \vspace{-0.2cm}
\end{table}

\subsection{Dataset}

% These CoQA datasets contain questions and answers from the Stack Overflow data dump\footnote{https://archive.org/details/stackexchange}, intended for training preference models. 

Due to the additional challenges that programming QA presents for LLMs and the lack of high-quality, authentic multi-answer code preference datasets, we turned to StackExchange \footnote{https://archive.org/details/stackexchange}, a platform with forums that are accompanied by rich question-answering metadata. Based on this, we constructed a large-scale programming QA dataset in real-time (as of May 2024), called StaCoCoQA. It contains over 60,738 programming directories, as shown in Table~\ref{tab:stacocoqa_tags}, and 9,978,474 entries, with partial data statistics displayed in Figure~\ref{fig:dataset}. The data format of StaCoCoQA is presented in Table~\ref{fig::stacocoqa}.

The initial dataset \(D_I\) contains 24,101,803 entries, and is processed by the following steps:
(1) Select entries with "Questioner-picked answer" pairs to represent the preferences of the questioners, resulting in 12,260,106 entries in the \(D_Q\).
(2) Select data where the question includes at least one code block to focus on specific-domain programming QA, resulting in 9,978,474 entries in the dataset \(D_C\).
(3) All HTML tags were cleaned using BeautifulSoup \footnote{https://beautiful-soup-4.readthedocs.io/en/latest/} to ensure that the model is not affected by overly complex and meaningless content.
(4) Control the quality of the dataset by considering factors such as the time the question was posted, the size of the response pool, the difference between the highest and lowest votes within a pool, the votes for each response, the token-level length of the question and the answers, which yields varying sizes: 3K, 8K, 18K, 29K, and 64K. 
The controlled creation time variable and the data details after each processing step are shown in Table~\ref{tab:statistics}.

To further validate the effectiveness of SeAdpra, we also select eight popular topic CoQA datasets\footnote{https://huggingface.co/datasets/HuggingFaceH4/stack-exchange-preferences}, which have been filtered to meet specific criteria for preference models \cite{askell2021general}. Their detailed data information is provided in Table~\ref{domain}.
% Examples of some control variables are shown in Table~\ref{tab:statistics}.
% \noindent\textbf{Baselines}. 
% Following the DPO \cite{rafailov2024direct}, we evaluated several existing approaches aligned with human preference, including GPT-J \cite{gpt-j} and Pythia-2.8B \cite{biderman2023pythia}.  
% Next, we assessed StarCoder2 \cite{lozhkov2024starcoder}, which has demonstrated strong performance in code generation, alongside several general-purpose LLMs: Qwen2 \cite{qwen2}, ChatGLM4 \cite{wang2023cogvlm, glm2024chatglm} and LLaMA serials \cite{touvron2023llama,llama3modelcard}.
% Finally, we fine-tuned LLaMA2-7B on the StaCoCoQA and compared its performance with other strong baselines for supervised learning in preference alignment, including SFT, RRHF \cite{yuan2024rrhf}, Silc \cite{zhao2023slic}, DPO, and PRO \cite{song2024preference}.
%%%%%%%%%%%%%%%%%%%%%%%%%%%%%%%%%%%%%%%%%%%%%%%%%%%%%%%%%%%%%%%%%%%%%%%%%%%%%%%%%%%%%%%%%%%%%%%%%%%%%%%%%%%%%%%%%%%%%%%%%%%%%%%%%%

% For preference evaluation, traditional win-rate assessments are costly and not scalable. For instance, when an existing model \(M_A\) is evaluated against comparison methods \((M_B, M_C, M_D)\) in terms of win rates, upgrading model \(M_A\) would necessitate a reevaluation of its win rates against other models. Furthermore, if a new comparison method \(M_E\) is introduced, the win rates of model \(M_A\) against \(M_E\) would also need to be reassessed. Whether AI or humans are employed as evaluation mediators, binary preference between preferred and non-preferred choices or to score the inference results of the modified model, the costs of this process are substantial. 
% Therefore, from an economic perspective, we propose a novel list preference evaluation method. We utilize manually ranking results as the gold standard for assessing human preferences, to calculate the Hit and Recall, referred to as PrefHit and PrefRecall, respectively. Regardless of whether improving model \(M_A\) or expanding comparison method \(M_E\), only the calculation of PrefHit and PrefRecall for the modified model is required, eliminating the need for human evaluation. 
% We also employ a professional reward model\footnote{https://huggingface.co/OpenAssistant/reward-model-deberta-v3-large}
% for evaluation, denoted as the Reward metric.

% \subsection{Baseline} 
% Following the DPO \cite{rafailov2024direct}, we evaluated several existing approaches aligned with human preference, including GPT-J \cite{gpt-j} and Pythia-2.8B \cite{biderman2023pythia}.  
% Next, we assessed StarCoder2 \cite{lozhkov2024starcoder}, which has demonstrated strong performance in code generation, alongside several general-purpose LLMs: Qwen2 \cite{qwen2}, ChatGLM4 \cite{wang2023cogvlm, glm2024chatglm} and LLaMA serials \cite{touvron2023llama,llama3modelcard}.
% Finally, we fine-tuned LLaMA2-7B on the StaCoCoQA and compared its performance with other strong baselines for supervised learning in preference alignment, including SFT, RRHF \cite{yuan2024rrhf}, Silc \cite{zhao2023slic}, DPO, and PRO \cite{song2024preference}.
\subsection{Evaluation Metrics}
\label{sec: metric}
For preference evaluation, we design PrefHit and PrefRecall, adhering to the "CSTC" criterion outlined in Appendix \ref{sec::cstc}, which overcome the limitations of existing evaluation methods, as detailed in Appendix \ref{metric::mot}.
In addition, we demonstrate the effectiveness of thees new evaluation from two main aspects: 1) consistency with traditional metrics, and 2) applicability in different application scenarios in Appendix \ref{metric::ana}.
Following the previous \cite{song2024preference}, we also employ a professional reward.
% Following the previous \cite{song2024preference}, we also employ a professional reward model\footnote{https://huggingface.co/OpenAssistant/reward-model-deberta-v3-large} \cite{song2024preference}, denoted as the Reward.

For accuracy evaluation, we alternately employ BLEU \cite{papineni2002bleu}, RougeL \cite{lin2004rouge}, and CoSim. Similar to codebertscore \cite{zhou2023codebertscore}, CoSim not only focuses on the semantics of the code but also considers structural matching.
Additionally, the implementation details of SeAdpra are described in detail in the Appendix \ref{sec::imp}.
\subsection{Main Results}
We compared the performance of \shortname with general LLMs and strong preference alignment benchmarks on the StaCoCoQA dataset, as shown in Table~\ref{main}. Additionally, we compared SeAdpra with the strongly supervised alignment model PRO \cite{song2024preference} on eight publicly available CoQA datasets, as presented in Table~\ref{public} and Figure~\ref{fig::public}.

\textbf{Larger Model Parameters, Higher Preference.}
Firstly, the Qwen2 series has adopted DPO \cite{rafailov2024direct} in post-training, resulting in a significant enhancement in Reward.
In a horizontal comparison, the performance of Qwen2-7B and LLaMA2-7B in terms of PrefHit is comparable.
Gradually increasing the parameter size of Qwen2 \cite{qwen2} and LLaMA leads to higher PrefHit and Reward.
Additionally, general LLMs continue to demonstrate strong capabilities of programming understanding and generation preference datasets, contributing to high BLEU scores.
These findings indicate that increasing parameter size can significantly improve alignment.

\textbf{List-wise Ranking Outperforms Pair-wise Comparison.}
Intuitively, list-wise DPO-PT surpasses pair-wise DPO-{BT} on PrefHit. Other list-wise methods, such as RRHF, PRO, and our \shortname, also undoubtedly surpass the pair-wise Slic.

\textbf{Both Parameter Size and Alignment Strategies are Effective.}
Compared to other models, Pythia-2.8B achieved impressive results with significantly fewer parameters .
Effective alignment strategies can balance the performance differences brought by parameter size. For example, LLaMA2-7B with PRO achieves results close to Qwen2-57B in PrefHit. Moreover, LLaMA2-7B combined with our method SeAdpra has already far exceeded the PrefHit of Qwen2-57B.

\textbf{Rather not Higher Reward, Higher PrefHit.}
It is evident that Reward and PrefHit are not always positively correlated, indicating that models do not always accurately learn human preferences and cannot fully replace real human evaluation. Therefore, relying solely on a single public reward model is not sufficiently comprehensive when assessing preference alignment.

% In conclusion, during ensuring precise alignment, SeAdpra will focuse on PrefHit@1, even though the trade-off between PrefHit and PrefRecall is a common issue and increasing recall may sometimes lead to a decrease in hit rate. The positive correlation between Reward and BLEU, indicates that improving the quality of the generated text typically enhances the Reward. 
% Most importantly, evaluating preferences solely based on reward is clearly insufficient, as a high reward does not necessarily correspond to a high PrefHit or PrefRecall.
%%%%%%%%%%%%%%%%%%%%%%%%%%%%%%%%%%%%%%%%%%%
%%%%%%%%%%%%
\begin{figure}
  \centering
  \begin{subfigure}{0.49\linewidth}
    \includegraphics[width=\linewidth]{latex/pic/hit.png}
    \caption{The PrefHit}
    \label{scale:hit}
  \end{subfigure}
  \begin{subfigure}{0.49\linewidth}
    \includegraphics[width=\linewidth]{latex/pic/Recall.png}
    \caption{The PrefRecall}
    \label{scale:recall}
  \end{subfigure}
  \medskip
  \begin{subfigure}{0.48\linewidth}
    \includegraphics[width=\linewidth]{latex/pic/reward.png}
    \caption{The Reward}
    \label{scale:reward}
  \end{subfigure}
  \begin{subfigure}{0.48\linewidth}
    \includegraphics[width=\linewidth]{latex/pic/bleu.png}
    \caption{The BLEU}
    \label{scale:bleu}
  \end{subfigure}
  \caption{The performance with Confidence Interval (CI) of our SeAdpra and PRO at different data scales.}
  \label{fig:scale}
  % \vspace{-0.2cm}
\end{figure}
%%%%%%%%%%%%%%%%%%%%%%%%%%%%%%%%%%%%%%%%%%%%%%%%%%%%%%%%%%%%%%%%%%%%%%%%%%%%%%%%%%%%%%%%%%%%%%%%%%%%%%%%%%%%%%%%

\subsection{Ablation Study}

In this section, we discuss the effectiveness of each component of SeAdpra and its impact on various metrics. The results are presented in Table \ref{ablation}.

\textbf{Perceptual Comparison} aims to prevent the model from relying solely on linguistic probability ordering while neglecting the significance of APDF. Removing this Reward will significantly increase the margin, but PrefHit will decrease, which may hinder the model's ability to compare and learn the preference differences between responses.

\textbf{Perceptual Alignment} seeks to align with the optimal responses; removing it will lead to a significant decrease in PrefHit, while the Reward and accuracy metrics like CoSim will significantly increase, as it tends to favor preference over accuracy.

\textbf{Semantic Perceptual Distance} plays a crucial role in maintaining semantic accuracy in alignment learning. Removing it leads to a significant decrease in BLEU and Rouge. Since sacrificing accuracy recalls more possibilities, PrefHit decreases while PrefRecall increases. Moreover, eliminating both Semantic Perceptual Distance and Perceptual Alignment in \(PerCo_{Po}\) further increases PrefRecall, while the other metrics decline again, consistent with previous observations.


\textbf{Popularity Perceptual Distance} is most closely associated with PrefHit. Eliminating it causes PrefHit to drop to its lowest value, indicating that the popularity attribute is an extremely important factor in code communities.

% In summary, each module has a varying impact on preference and accuracy, but all outperform their respective foundation models and other baselines, as shown in Table \ref{main}, proving their effectiveness.


\subsection{Analysis and Discussion}

\textbf{SeAdpra adept at high-quality data rather than large-scale data.}
In StaCoCoQA, we tested PRO and SeAdpra across different data scales, and the results are shown in Figure~\ref{fig:scale}.
Since we rely on the popularity and clarity of questions and answers to filter data, a larger data scale often results in more pronounced deterioration in data quality. In Figure~\ref{scale:hit}, SeAdpra is highly sensitive to data quality in PrefHit, whereas PRO demonstrates improved performance with larger-scale data. Their performance on Prefrecall is consistent. In the native reward model of PRO, as depicted in Figure~\ref{scale:reward}, the reward fluctuations are minimal, while SeAdpra shows remarkable improvement.

\textbf{SeAdpra is relatively insensitive to ranking length.} 
We assessed SeAdpra's performance on different ranking lengths, as shown in Figure 6a. Unlike PRO, which varied with increasing ranking length, SeAdpra shows no significant differences across different lengths. There is a slight increase in performance on PrefHit and PrefRecall. Additionally, SeAdpra performs better at odd lengths compared to even lengths, which is an interesting phenomenon warranting further investigation.


\textbf{Balance Preference and Accuracy.} 
We analyzed the effect of control weights for Perceptual Comparisons in the optimization objective on preference and accuracy, with the findings presented in Figure~\ref{para:weight}.
When \( \alpha \) is greater than 0.05, the trends in PrefHit and BLEU are consistent, indicating that preference and accuracy can be optimized in tandem. However, when \( \alpha \) is 0.01, PrefHit is highest, but BLEU drops sharply.
Additionally, as \( \alpha \) changes, the variations in PrefHit and Reward, which are related to preference, are consistent with each other, reflecting their unified relationship in the optimization. Similarly, the variations in Recall and BLEU, which are related to accuracy, are also consistent, indicating a strong correlation between generation quality and comprehensiveness. 

%%%%%%%%%%%%%%%%%%%%%%%%%%%%%%%%%%%%%%%%%%%%%%%%%%%%%%%%%%%%%%%%%%%%%%%%%%%%%%%%%
\begin{figure}
  \centering
  \begin{subfigure}{0.475\linewidth}
    \includegraphics[width=\linewidth]{latex/pic/Rank1.png}
    \caption{Ranking length}
    \label{para:rank}
  \end{subfigure}
  \begin{subfigure}{0.475\linewidth}
    \includegraphics[width=\linewidth]{latex/pic/weights1.png}
    \caption{The \(\alpha\) in \(Loss\)}
    \label{para:weight}
  \end{subfigure}
  \caption{Parameters Analysis. Results of experiments on different ranking lengths and the weight \(\alpha\) in \(Loss\).}
  \label{fig:para}
  % \vspace{-0.2cm}
\end{figure}
%%%%%%%%%%%%%%%%%%%%%%%%%%%%%%%%%%%%%%%%%%%%
\begin{figure*}
  \centering
  \begin{subfigure}{0.305\linewidth}
    \includegraphics[width=\linewidth]{latex/pic/se2.pdf}
    \caption{The \(\Delta_{Se}\)}
    \label{visual:se}
  \end{subfigure}
  \begin{subfigure}{0.305\linewidth}
    \includegraphics[width=\linewidth]{latex/pic/po2.pdf}
    \caption{The \(\Delta_{Po}\)}
    \label{visual:po}
  \end{subfigure}
  \begin{subfigure}{0.305\linewidth}
    \includegraphics[width=\linewidth]{latex/pic/sv2.pdf}
    \caption{The \(\Delta_{M}\)}
    \label{visual:sv}
  \end{subfigure}
  \caption{The Visualization of Attribute-Perceptual Distance Factors (APDF) matrix of five responses. The blue represents the response with the highest APDF, and SeAdpra aligns with the fifth response corresponding to the maximum Multi-APDF in (c). The green represents the second response that is next best to the red one.}
  \label{visual}
  % \vspace{-0.2cm}
\end{figure*}
%%%%%%%%%%%%%%%%%%%%%%%%%%%%%%%%%%%%%%%%%
\textbf{Single-APDF Matrix Cannot Predict the Optimal Response.} We randomly selected a pair with a golden label and visualized its specific iteration in Figure~\ref{visual}.
It can be observed that the optimal response in a Single-APDF matrix is not necessarily the same as that in the Multi-APDF matrix.
Specifically, the optimal response in the Semantic Perceptual Factor matrix \(\Delta_{Se}\) is the fifth response in Figure~\ref{visual:se}, while in the Popularity Perceptual Factor matrix \(\Delta_{Po}\) (Figure~\ref{visual:po}), it is the third response. Ultimately, in the Multiple Perceptual Distance Factor matrix \(\Delta_{M}\), the third response is slightly inferior to the fifth response (0.037 vs. 0.038) in Figure~\ref{visual:sv}, and this result aligns with the golden label.
More key findings regarding the ADPF are described in Figure \ref{fig::hot1} and Figure \ref{fig::hot2}.

% \clearpage
\bibliography{custom}
\bibliographystyle{acl_natbib}

\subsection{Lloyd-Max Algorithm}
\label{subsec:Lloyd-Max}
For a given quantization bitwidth $B$ and an operand $\bm{X}$, the Lloyd-Max algorithm finds $2^B$ quantization levels $\{\hat{x}_i\}_{i=1}^{2^B}$ such that quantizing $\bm{X}$ by rounding each scalar in $\bm{X}$ to the nearest quantization level minimizes the quantization MSE. 

The algorithm starts with an initial guess of quantization levels and then iteratively computes quantization thresholds $\{\tau_i\}_{i=1}^{2^B-1}$ and updates quantization levels $\{\hat{x}_i\}_{i=1}^{2^B}$. Specifically, at iteration $n$, thresholds are set to the midpoints of the previous iteration's levels:
\begin{align*}
    \tau_i^{(n)}=\frac{\hat{x}_i^{(n-1)}+\hat{x}_{i+1}^{(n-1)}}2 \text{ for } i=1\ldots 2^B-1
\end{align*}
Subsequently, the quantization levels are re-computed as conditional means of the data regions defined by the new thresholds:
\begin{align*}
    \hat{x}_i^{(n)}=\mathbb{E}\left[ \bm{X} \big| \bm{X}\in [\tau_{i-1}^{(n)},\tau_i^{(n)}] \right] \text{ for } i=1\ldots 2^B
\end{align*}
where to satisfy boundary conditions we have $\tau_0=-\infty$ and $\tau_{2^B}=\infty$. The algorithm iterates the above steps until convergence.

Figure \ref{fig:lm_quant} compares the quantization levels of a $7$-bit floating point (E3M3) quantizer (left) to a $7$-bit Lloyd-Max quantizer (right) when quantizing a layer of weights from the GPT3-126M model at a per-tensor granularity. As shown, the Lloyd-Max quantizer achieves substantially lower quantization MSE. Further, Table \ref{tab:FP7_vs_LM7} shows the superior perplexity achieved by Lloyd-Max quantizers for bitwidths of $7$, $6$ and $5$. The difference between the quantizers is clear at 5 bits, where per-tensor FP quantization incurs a drastic and unacceptable increase in perplexity, while Lloyd-Max quantization incurs a much smaller increase. Nevertheless, we note that even the optimal Lloyd-Max quantizer incurs a notable ($\sim 1.5$) increase in perplexity due to the coarse granularity of quantization. 

\begin{figure}[h]
  \centering
  \includegraphics[width=0.7\linewidth]{sections/figures/LM7_FP7.pdf}
  \caption{\small Quantization levels and the corresponding quantization MSE of Floating Point (left) vs Lloyd-Max (right) Quantizers for a layer of weights in the GPT3-126M model.}
  \label{fig:lm_quant}
\end{figure}

\begin{table}[h]\scriptsize
\begin{center}
\caption{\label{tab:FP7_vs_LM7} \small Comparing perplexity (lower is better) achieved by floating point quantizers and Lloyd-Max quantizers on a GPT3-126M model for the Wikitext-103 dataset.}
\begin{tabular}{c|cc|c}
\hline
 \multirow{2}{*}{\textbf{Bitwidth}} & \multicolumn{2}{|c|}{\textbf{Floating-Point Quantizer}} & \textbf{Lloyd-Max Quantizer} \\
 & Best Format & Wikitext-103 Perplexity & Wikitext-103 Perplexity \\
\hline
7 & E3M3 & 18.32 & 18.27 \\
6 & E3M2 & 19.07 & 18.51 \\
5 & E4M0 & 43.89 & 19.71 \\
\hline
\end{tabular}
\end{center}
\end{table}

\subsection{Proof of Local Optimality of LO-BCQ}
\label{subsec:lobcq_opt_proof}
For a given block $\bm{b}_j$, the quantization MSE during LO-BCQ can be empirically evaluated as $\frac{1}{L_b}\lVert \bm{b}_j- \bm{\hat{b}}_j\rVert^2_2$ where $\bm{\hat{b}}_j$ is computed from equation (\ref{eq:clustered_quantization_definition}) as $C_{f(\bm{b}_j)}(\bm{b}_j)$. Further, for a given block cluster $\mathcal{B}_i$, we compute the quantization MSE as $\frac{1}{|\mathcal{B}_{i}|}\sum_{\bm{b} \in \mathcal{B}_{i}} \frac{1}{L_b}\lVert \bm{b}- C_i^{(n)}(\bm{b})\rVert^2_2$. Therefore, at the end of iteration $n$, we evaluate the overall quantization MSE $J^{(n)}$ for a given operand $\bm{X}$ composed of $N_c$ block clusters as:
\begin{align*}
    \label{eq:mse_iter_n}
    J^{(n)} = \frac{1}{N_c} \sum_{i=1}^{N_c} \frac{1}{|\mathcal{B}_{i}^{(n)}|}\sum_{\bm{v} \in \mathcal{B}_{i}^{(n)}} \frac{1}{L_b}\lVert \bm{b}- B_i^{(n)}(\bm{b})\rVert^2_2
\end{align*}

At the end of iteration $n$, the codebooks are updated from $\mathcal{C}^{(n-1)}$ to $\mathcal{C}^{(n)}$. However, the mapping of a given vector $\bm{b}_j$ to quantizers $\mathcal{C}^{(n)}$ remains as  $f^{(n)}(\bm{b}_j)$. At the next iteration, during the vector clustering step, $f^{(n+1)}(\bm{b}_j)$ finds new mapping of $\bm{b}_j$ to updated codebooks $\mathcal{C}^{(n)}$ such that the quantization MSE over the candidate codebooks is minimized. Therefore, we obtain the following result for $\bm{b}_j$:
\begin{align*}
\frac{1}{L_b}\lVert \bm{b}_j - C_{f^{(n+1)}(\bm{b}_j)}^{(n)}(\bm{b}_j)\rVert^2_2 \le \frac{1}{L_b}\lVert \bm{b}_j - C_{f^{(n)}(\bm{b}_j)}^{(n)}(\bm{b}_j)\rVert^2_2
\end{align*}

That is, quantizing $\bm{b}_j$ at the end of the block clustering step of iteration $n+1$ results in lower quantization MSE compared to quantizing at the end of iteration $n$. Since this is true for all $\bm{b} \in \bm{X}$, we assert the following:
\begin{equation}
\begin{split}
\label{eq:mse_ineq_1}
    \tilde{J}^{(n+1)} &= \frac{1}{N_c} \sum_{i=1}^{N_c} \frac{1}{|\mathcal{B}_{i}^{(n+1)}|}\sum_{\bm{b} \in \mathcal{B}_{i}^{(n+1)}} \frac{1}{L_b}\lVert \bm{b} - C_i^{(n)}(b)\rVert^2_2 \le J^{(n)}
\end{split}
\end{equation}
where $\tilde{J}^{(n+1)}$ is the the quantization MSE after the vector clustering step at iteration $n+1$.

Next, during the codebook update step (\ref{eq:quantizers_update}) at iteration $n+1$, the per-cluster codebooks $\mathcal{C}^{(n)}$ are updated to $\mathcal{C}^{(n+1)}$ by invoking the Lloyd-Max algorithm \citep{Lloyd}. We know that for any given value distribution, the Lloyd-Max algorithm minimizes the quantization MSE. Therefore, for a given vector cluster $\mathcal{B}_i$ we obtain the following result:

\begin{equation}
    \frac{1}{|\mathcal{B}_{i}^{(n+1)}|}\sum_{\bm{b} \in \mathcal{B}_{i}^{(n+1)}} \frac{1}{L_b}\lVert \bm{b}- C_i^{(n+1)}(\bm{b})\rVert^2_2 \le \frac{1}{|\mathcal{B}_{i}^{(n+1)}|}\sum_{\bm{b} \in \mathcal{B}_{i}^{(n+1)}} \frac{1}{L_b}\lVert \bm{b}- C_i^{(n)}(\bm{b})\rVert^2_2
\end{equation}

The above equation states that quantizing the given block cluster $\mathcal{B}_i$ after updating the associated codebook from $C_i^{(n)}$ to $C_i^{(n+1)}$ results in lower quantization MSE. Since this is true for all the block clusters, we derive the following result: 
\begin{equation}
\begin{split}
\label{eq:mse_ineq_2}
     J^{(n+1)} &= \frac{1}{N_c} \sum_{i=1}^{N_c} \frac{1}{|\mathcal{B}_{i}^{(n+1)}|}\sum_{\bm{b} \in \mathcal{B}_{i}^{(n+1)}} \frac{1}{L_b}\lVert \bm{b}- C_i^{(n+1)}(\bm{b})\rVert^2_2  \le \tilde{J}^{(n+1)}   
\end{split}
\end{equation}

Following (\ref{eq:mse_ineq_1}) and (\ref{eq:mse_ineq_2}), we find that the quantization MSE is non-increasing for each iteration, that is, $J^{(1)} \ge J^{(2)} \ge J^{(3)} \ge \ldots \ge J^{(M)}$ where $M$ is the maximum number of iterations. 
%Therefore, we can say that if the algorithm converges, then it must be that it has converged to a local minimum. 
\hfill $\blacksquare$


\begin{figure}
    \begin{center}
    \includegraphics[width=0.5\textwidth]{sections//figures/mse_vs_iter.pdf}
    \end{center}
    \caption{\small NMSE vs iterations during LO-BCQ compared to other block quantization proposals}
    \label{fig:nmse_vs_iter}
\end{figure}

Figure \ref{fig:nmse_vs_iter} shows the empirical convergence of LO-BCQ across several block lengths and number of codebooks. Also, the MSE achieved by LO-BCQ is compared to baselines such as MXFP and VSQ. As shown, LO-BCQ converges to a lower MSE than the baselines. Further, we achieve better convergence for larger number of codebooks ($N_c$) and for a smaller block length ($L_b$), both of which increase the bitwidth of BCQ (see Eq \ref{eq:bitwidth_bcq}).


\subsection{Additional Accuracy Results}
%Table \ref{tab:lobcq_config} lists the various LOBCQ configurations and their corresponding bitwidths.
\begin{table}
\setlength{\tabcolsep}{4.75pt}
\begin{center}
\caption{\label{tab:lobcq_config} Various LO-BCQ configurations and their bitwidths.}
\begin{tabular}{|c||c|c|c|c||c|c||c|} 
\hline
 & \multicolumn{4}{|c||}{$L_b=8$} & \multicolumn{2}{|c||}{$L_b=4$} & $L_b=2$ \\
 \hline
 \backslashbox{$L_A$\kern-1em}{\kern-1em$N_c$} & 2 & 4 & 8 & 16 & 2 & 4 & 2 \\
 \hline
 64 & 4.25 & 4.375 & 4.5 & 4.625 & 4.375 & 4.625 & 4.625\\
 \hline
 32 & 4.375 & 4.5 & 4.625& 4.75 & 4.5 & 4.75 & 4.75 \\
 \hline
 16 & 4.625 & 4.75& 4.875 & 5 & 4.75 & 5 & 5 \\
 \hline
\end{tabular}
\end{center}
\end{table}

%\subsection{Perplexity achieved by various LO-BCQ configurations on Wikitext-103 dataset}

\begin{table} \centering
\begin{tabular}{|c||c|c|c|c||c|c||c|} 
\hline
 $L_b \rightarrow$& \multicolumn{4}{c||}{8} & \multicolumn{2}{c||}{4} & 2\\
 \hline
 \backslashbox{$L_A$\kern-1em}{\kern-1em$N_c$} & 2 & 4 & 8 & 16 & 2 & 4 & 2  \\
 %$N_c \rightarrow$ & 2 & 4 & 8 & 16 & 2 & 4 & 2 \\
 \hline
 \hline
 \multicolumn{8}{c}{GPT3-1.3B (FP32 PPL = 9.98)} \\ 
 \hline
 \hline
 64 & 10.40 & 10.23 & 10.17 & 10.15 &  10.28 & 10.18 & 10.19 \\
 \hline
 32 & 10.25 & 10.20 & 10.15 & 10.12 &  10.23 & 10.17 & 10.17 \\
 \hline
 16 & 10.22 & 10.16 & 10.10 & 10.09 &  10.21 & 10.14 & 10.16 \\
 \hline
  \hline
 \multicolumn{8}{c}{GPT3-8B (FP32 PPL = 7.38)} \\ 
 \hline
 \hline
 64 & 7.61 & 7.52 & 7.48 &  7.47 &  7.55 &  7.49 & 7.50 \\
 \hline
 32 & 7.52 & 7.50 & 7.46 &  7.45 &  7.52 &  7.48 & 7.48  \\
 \hline
 16 & 7.51 & 7.48 & 7.44 &  7.44 &  7.51 &  7.49 & 7.47  \\
 \hline
\end{tabular}
\caption{\label{tab:ppl_gpt3_abalation} Wikitext-103 perplexity across GPT3-1.3B and 8B models.}
\end{table}

\begin{table} \centering
\begin{tabular}{|c||c|c|c|c||} 
\hline
 $L_b \rightarrow$& \multicolumn{4}{c||}{8}\\
 \hline
 \backslashbox{$L_A$\kern-1em}{\kern-1em$N_c$} & 2 & 4 & 8 & 16 \\
 %$N_c \rightarrow$ & 2 & 4 & 8 & 16 & 2 & 4 & 2 \\
 \hline
 \hline
 \multicolumn{5}{|c|}{Llama2-7B (FP32 PPL = 5.06)} \\ 
 \hline
 \hline
 64 & 5.31 & 5.26 & 5.19 & 5.18  \\
 \hline
 32 & 5.23 & 5.25 & 5.18 & 5.15  \\
 \hline
 16 & 5.23 & 5.19 & 5.16 & 5.14  \\
 \hline
 \multicolumn{5}{|c|}{Nemotron4-15B (FP32 PPL = 5.87)} \\ 
 \hline
 \hline
 64  & 6.3 & 6.20 & 6.13 & 6.08  \\
 \hline
 32  & 6.24 & 6.12 & 6.07 & 6.03  \\
 \hline
 16  & 6.12 & 6.14 & 6.04 & 6.02  \\
 \hline
 \multicolumn{5}{|c|}{Nemotron4-340B (FP32 PPL = 3.48)} \\ 
 \hline
 \hline
 64 & 3.67 & 3.62 & 3.60 & 3.59 \\
 \hline
 32 & 3.63 & 3.61 & 3.59 & 3.56 \\
 \hline
 16 & 3.61 & 3.58 & 3.57 & 3.55 \\
 \hline
\end{tabular}
\caption{\label{tab:ppl_llama7B_nemo15B} Wikitext-103 perplexity compared to FP32 baseline in Llama2-7B and Nemotron4-15B, 340B models}
\end{table}

%\subsection{Perplexity achieved by various LO-BCQ configurations on MMLU dataset}


\begin{table} \centering
\begin{tabular}{|c||c|c|c|c||c|c|c|c|} 
\hline
 $L_b \rightarrow$& \multicolumn{4}{c||}{8} & \multicolumn{4}{c||}{8}\\
 \hline
 \backslashbox{$L_A$\kern-1em}{\kern-1em$N_c$} & 2 & 4 & 8 & 16 & 2 & 4 & 8 & 16  \\
 %$N_c \rightarrow$ & 2 & 4 & 8 & 16 & 2 & 4 & 2 \\
 \hline
 \hline
 \multicolumn{5}{|c|}{Llama2-7B (FP32 Accuracy = 45.8\%)} & \multicolumn{4}{|c|}{Llama2-70B (FP32 Accuracy = 69.12\%)} \\ 
 \hline
 \hline
 64 & 43.9 & 43.4 & 43.9 & 44.9 & 68.07 & 68.27 & 68.17 & 68.75 \\
 \hline
 32 & 44.5 & 43.8 & 44.9 & 44.5 & 68.37 & 68.51 & 68.35 & 68.27  \\
 \hline
 16 & 43.9 & 42.7 & 44.9 & 45 & 68.12 & 68.77 & 68.31 & 68.59  \\
 \hline
 \hline
 \multicolumn{5}{|c|}{GPT3-22B (FP32 Accuracy = 38.75\%)} & \multicolumn{4}{|c|}{Nemotron4-15B (FP32 Accuracy = 64.3\%)} \\ 
 \hline
 \hline
 64 & 36.71 & 38.85 & 38.13 & 38.92 & 63.17 & 62.36 & 63.72 & 64.09 \\
 \hline
 32 & 37.95 & 38.69 & 39.45 & 38.34 & 64.05 & 62.30 & 63.8 & 64.33  \\
 \hline
 16 & 38.88 & 38.80 & 38.31 & 38.92 & 63.22 & 63.51 & 63.93 & 64.43  \\
 \hline
\end{tabular}
\caption{\label{tab:mmlu_abalation} Accuracy on MMLU dataset across GPT3-22B, Llama2-7B, 70B and Nemotron4-15B models.}
\end{table}


%\subsection{Perplexity achieved by various LO-BCQ configurations on LM evaluation harness}

\begin{table} \centering
\begin{tabular}{|c||c|c|c|c||c|c|c|c|} 
\hline
 $L_b \rightarrow$& \multicolumn{4}{c||}{8} & \multicolumn{4}{c||}{8}\\
 \hline
 \backslashbox{$L_A$\kern-1em}{\kern-1em$N_c$} & 2 & 4 & 8 & 16 & 2 & 4 & 8 & 16  \\
 %$N_c \rightarrow$ & 2 & 4 & 8 & 16 & 2 & 4 & 2 \\
 \hline
 \hline
 \multicolumn{5}{|c|}{Race (FP32 Accuracy = 37.51\%)} & \multicolumn{4}{|c|}{Boolq (FP32 Accuracy = 64.62\%)} \\ 
 \hline
 \hline
 64 & 36.94 & 37.13 & 36.27 & 37.13 & 63.73 & 62.26 & 63.49 & 63.36 \\
 \hline
 32 & 37.03 & 36.36 & 36.08 & 37.03 & 62.54 & 63.51 & 63.49 & 63.55  \\
 \hline
 16 & 37.03 & 37.03 & 36.46 & 37.03 & 61.1 & 63.79 & 63.58 & 63.33  \\
 \hline
 \hline
 \multicolumn{5}{|c|}{Winogrande (FP32 Accuracy = 58.01\%)} & \multicolumn{4}{|c|}{Piqa (FP32 Accuracy = 74.21\%)} \\ 
 \hline
 \hline
 64 & 58.17 & 57.22 & 57.85 & 58.33 & 73.01 & 73.07 & 73.07 & 72.80 \\
 \hline
 32 & 59.12 & 58.09 & 57.85 & 58.41 & 73.01 & 73.94 & 72.74 & 73.18  \\
 \hline
 16 & 57.93 & 58.88 & 57.93 & 58.56 & 73.94 & 72.80 & 73.01 & 73.94  \\
 \hline
\end{tabular}
\caption{\label{tab:mmlu_abalation} Accuracy on LM evaluation harness tasks on GPT3-1.3B model.}
\end{table}

\begin{table} \centering
\begin{tabular}{|c||c|c|c|c||c|c|c|c|} 
\hline
 $L_b \rightarrow$& \multicolumn{4}{c||}{8} & \multicolumn{4}{c||}{8}\\
 \hline
 \backslashbox{$L_A$\kern-1em}{\kern-1em$N_c$} & 2 & 4 & 8 & 16 & 2 & 4 & 8 & 16  \\
 %$N_c \rightarrow$ & 2 & 4 & 8 & 16 & 2 & 4 & 2 \\
 \hline
 \hline
 \multicolumn{5}{|c|}{Race (FP32 Accuracy = 41.34\%)} & \multicolumn{4}{|c|}{Boolq (FP32 Accuracy = 68.32\%)} \\ 
 \hline
 \hline
 64 & 40.48 & 40.10 & 39.43 & 39.90 & 69.20 & 68.41 & 69.45 & 68.56 \\
 \hline
 32 & 39.52 & 39.52 & 40.77 & 39.62 & 68.32 & 67.43 & 68.17 & 69.30  \\
 \hline
 16 & 39.81 & 39.71 & 39.90 & 40.38 & 68.10 & 66.33 & 69.51 & 69.42  \\
 \hline
 \hline
 \multicolumn{5}{|c|}{Winogrande (FP32 Accuracy = 67.88\%)} & \multicolumn{4}{|c|}{Piqa (FP32 Accuracy = 78.78\%)} \\ 
 \hline
 \hline
 64 & 66.85 & 66.61 & 67.72 & 67.88 & 77.31 & 77.42 & 77.75 & 77.64 \\
 \hline
 32 & 67.25 & 67.72 & 67.72 & 67.00 & 77.31 & 77.04 & 77.80 & 77.37  \\
 \hline
 16 & 68.11 & 68.90 & 67.88 & 67.48 & 77.37 & 78.13 & 78.13 & 77.69  \\
 \hline
\end{tabular}
\caption{\label{tab:mmlu_abalation} Accuracy on LM evaluation harness tasks on GPT3-8B model.}
\end{table}

\begin{table} \centering
\begin{tabular}{|c||c|c|c|c||c|c|c|c|} 
\hline
 $L_b \rightarrow$& \multicolumn{4}{c||}{8} & \multicolumn{4}{c||}{8}\\
 \hline
 \backslashbox{$L_A$\kern-1em}{\kern-1em$N_c$} & 2 & 4 & 8 & 16 & 2 & 4 & 8 & 16  \\
 %$N_c \rightarrow$ & 2 & 4 & 8 & 16 & 2 & 4 & 2 \\
 \hline
 \hline
 \multicolumn{5}{|c|}{Race (FP32 Accuracy = 40.67\%)} & \multicolumn{4}{|c|}{Boolq (FP32 Accuracy = 76.54\%)} \\ 
 \hline
 \hline
 64 & 40.48 & 40.10 & 39.43 & 39.90 & 75.41 & 75.11 & 77.09 & 75.66 \\
 \hline
 32 & 39.52 & 39.52 & 40.77 & 39.62 & 76.02 & 76.02 & 75.96 & 75.35  \\
 \hline
 16 & 39.81 & 39.71 & 39.90 & 40.38 & 75.05 & 73.82 & 75.72 & 76.09  \\
 \hline
 \hline
 \multicolumn{5}{|c|}{Winogrande (FP32 Accuracy = 70.64\%)} & \multicolumn{4}{|c|}{Piqa (FP32 Accuracy = 79.16\%)} \\ 
 \hline
 \hline
 64 & 69.14 & 70.17 & 70.17 & 70.56 & 78.24 & 79.00 & 78.62 & 78.73 \\
 \hline
 32 & 70.96 & 69.69 & 71.27 & 69.30 & 78.56 & 79.49 & 79.16 & 78.89  \\
 \hline
 16 & 71.03 & 69.53 & 69.69 & 70.40 & 78.13 & 79.16 & 79.00 & 79.00  \\
 \hline
\end{tabular}
\caption{\label{tab:mmlu_abalation} Accuracy on LM evaluation harness tasks on GPT3-22B model.}
\end{table}

\begin{table} \centering
\begin{tabular}{|c||c|c|c|c||c|c|c|c|} 
\hline
 $L_b \rightarrow$& \multicolumn{4}{c||}{8} & \multicolumn{4}{c||}{8}\\
 \hline
 \backslashbox{$L_A$\kern-1em}{\kern-1em$N_c$} & 2 & 4 & 8 & 16 & 2 & 4 & 8 & 16  \\
 %$N_c \rightarrow$ & 2 & 4 & 8 & 16 & 2 & 4 & 2 \\
 \hline
 \hline
 \multicolumn{5}{|c|}{Race (FP32 Accuracy = 44.4\%)} & \multicolumn{4}{|c|}{Boolq (FP32 Accuracy = 79.29\%)} \\ 
 \hline
 \hline
 64 & 42.49 & 42.51 & 42.58 & 43.45 & 77.58 & 77.37 & 77.43 & 78.1 \\
 \hline
 32 & 43.35 & 42.49 & 43.64 & 43.73 & 77.86 & 75.32 & 77.28 & 77.86  \\
 \hline
 16 & 44.21 & 44.21 & 43.64 & 42.97 & 78.65 & 77 & 76.94 & 77.98  \\
 \hline
 \hline
 \multicolumn{5}{|c|}{Winogrande (FP32 Accuracy = 69.38\%)} & \multicolumn{4}{|c|}{Piqa (FP32 Accuracy = 78.07\%)} \\ 
 \hline
 \hline
 64 & 68.9 & 68.43 & 69.77 & 68.19 & 77.09 & 76.82 & 77.09 & 77.86 \\
 \hline
 32 & 69.38 & 68.51 & 68.82 & 68.90 & 78.07 & 76.71 & 78.07 & 77.86  \\
 \hline
 16 & 69.53 & 67.09 & 69.38 & 68.90 & 77.37 & 77.8 & 77.91 & 77.69  \\
 \hline
\end{tabular}
\caption{\label{tab:mmlu_abalation} Accuracy on LM evaluation harness tasks on Llama2-7B model.}
\end{table}

\begin{table} \centering
\begin{tabular}{|c||c|c|c|c||c|c|c|c|} 
\hline
 $L_b \rightarrow$& \multicolumn{4}{c||}{8} & \multicolumn{4}{c||}{8}\\
 \hline
 \backslashbox{$L_A$\kern-1em}{\kern-1em$N_c$} & 2 & 4 & 8 & 16 & 2 & 4 & 8 & 16  \\
 %$N_c \rightarrow$ & 2 & 4 & 8 & 16 & 2 & 4 & 2 \\
 \hline
 \hline
 \multicolumn{5}{|c|}{Race (FP32 Accuracy = 48.8\%)} & \multicolumn{4}{|c|}{Boolq (FP32 Accuracy = 85.23\%)} \\ 
 \hline
 \hline
 64 & 49.00 & 49.00 & 49.28 & 48.71 & 82.82 & 84.28 & 84.03 & 84.25 \\
 \hline
 32 & 49.57 & 48.52 & 48.33 & 49.28 & 83.85 & 84.46 & 84.31 & 84.93  \\
 \hline
 16 & 49.85 & 49.09 & 49.28 & 48.99 & 85.11 & 84.46 & 84.61 & 83.94  \\
 \hline
 \hline
 \multicolumn{5}{|c|}{Winogrande (FP32 Accuracy = 79.95\%)} & \multicolumn{4}{|c|}{Piqa (FP32 Accuracy = 81.56\%)} \\ 
 \hline
 \hline
 64 & 78.77 & 78.45 & 78.37 & 79.16 & 81.45 & 80.69 & 81.45 & 81.5 \\
 \hline
 32 & 78.45 & 79.01 & 78.69 & 80.66 & 81.56 & 80.58 & 81.18 & 81.34  \\
 \hline
 16 & 79.95 & 79.56 & 79.79 & 79.72 & 81.28 & 81.66 & 81.28 & 80.96  \\
 \hline
\end{tabular}
\caption{\label{tab:mmlu_abalation} Accuracy on LM evaluation harness tasks on Llama2-70B model.}
\end{table}

%\section{MSE Studies}
%\textcolor{red}{TODO}


\subsection{Number Formats and Quantization Method}
\label{subsec:numFormats_quantMethod}
\subsubsection{Integer Format}
An $n$-bit signed integer (INT) is typically represented with a 2s-complement format \citep{yao2022zeroquant,xiao2023smoothquant,dai2021vsq}, where the most significant bit denotes the sign.

\subsubsection{Floating Point Format}
An $n$-bit signed floating point (FP) number $x$ comprises of a 1-bit sign ($x_{\mathrm{sign}}$), $B_m$-bit mantissa ($x_{\mathrm{mant}}$) and $B_e$-bit exponent ($x_{\mathrm{exp}}$) such that $B_m+B_e=n-1$. The associated constant exponent bias ($E_{\mathrm{bias}}$) is computed as $(2^{{B_e}-1}-1)$. We denote this format as $E_{B_e}M_{B_m}$.  

\subsubsection{Quantization Scheme}
\label{subsec:quant_method}
A quantization scheme dictates how a given unquantized tensor is converted to its quantized representation. We consider FP formats for the purpose of illustration. Given an unquantized tensor $\bm{X}$ and an FP format $E_{B_e}M_{B_m}$, we first, we compute the quantization scale factor $s_X$ that maps the maximum absolute value of $\bm{X}$ to the maximum quantization level of the $E_{B_e}M_{B_m}$ format as follows:
\begin{align}
\label{eq:sf}
    s_X = \frac{\mathrm{max}(|\bm{X}|)}{\mathrm{max}(E_{B_e}M_{B_m})}
\end{align}
In the above equation, $|\cdot|$ denotes the absolute value function.

Next, we scale $\bm{X}$ by $s_X$ and quantize it to $\hat{\bm{X}}$ by rounding it to the nearest quantization level of $E_{B_e}M_{B_m}$ as:

\begin{align}
\label{eq:tensor_quant}
    \hat{\bm{X}} = \text{round-to-nearest}\left(\frac{\bm{X}}{s_X}, E_{B_e}M_{B_m}\right)
\end{align}

We perform dynamic max-scaled quantization \citep{wu2020integer}, where the scale factor $s$ for activations is dynamically computed during runtime.

\subsection{Vector Scaled Quantization}
\begin{wrapfigure}{r}{0.35\linewidth}
  \centering
  \includegraphics[width=\linewidth]{sections/figures/vsquant.jpg}
  \caption{\small Vectorwise decomposition for per-vector scaled quantization (VSQ \citep{dai2021vsq}).}
  \label{fig:vsquant}
\end{wrapfigure}
During VSQ \citep{dai2021vsq}, the operand tensors are decomposed into 1D vectors in a hardware friendly manner as shown in Figure \ref{fig:vsquant}. Since the decomposed tensors are used as operands in matrix multiplications during inference, it is beneficial to perform this decomposition along the reduction dimension of the multiplication. The vectorwise quantization is performed similar to tensorwise quantization described in Equations \ref{eq:sf} and \ref{eq:tensor_quant}, where a scale factor $s_v$ is required for each vector $\bm{v}$ that maps the maximum absolute value of that vector to the maximum quantization level. While smaller vector lengths can lead to larger accuracy gains, the associated memory and computational overheads due to the per-vector scale factors increases. To alleviate these overheads, VSQ \citep{dai2021vsq} proposed a second level quantization of the per-vector scale factors to unsigned integers, while MX \citep{rouhani2023shared} quantizes them to integer powers of 2 (denoted as $2^{INT}$).

\subsubsection{MX Format}
The MX format proposed in \citep{rouhani2023microscaling} introduces the concept of sub-block shifting. For every two scalar elements of $b$-bits each, there is a shared exponent bit. The value of this exponent bit is determined through an empirical analysis that targets minimizing quantization MSE. We note that the FP format $E_{1}M_{b}$ is strictly better than MX from an accuracy perspective since it allocates a dedicated exponent bit to each scalar as opposed to sharing it across two scalars. Therefore, we conservatively bound the accuracy of a $b+2$-bit signed MX format with that of a $E_{1}M_{b}$ format in our comparisons. For instance, we use E1M2 format as a proxy for MX4.

\begin{figure}
    \centering
    \includegraphics[width=1\linewidth]{sections//figures/BlockFormats.pdf}
    \caption{\small Comparing LO-BCQ to MX format.}
    \label{fig:block_formats}
\end{figure}

Figure \ref{fig:block_formats} compares our $4$-bit LO-BCQ block format to MX \citep{rouhani2023microscaling}. As shown, both LO-BCQ and MX decompose a given operand tensor into block arrays and each block array into blocks. Similar to MX, we find that per-block quantization ($L_b < L_A$) leads to better accuracy due to increased flexibility. While MX achieves this through per-block $1$-bit micro-scales, we associate a dedicated codebook to each block through a per-block codebook selector. Further, MX quantizes the per-block array scale-factor to E8M0 format without per-tensor scaling. In contrast during LO-BCQ, we find that per-tensor scaling combined with quantization of per-block array scale-factor to E4M3 format results in superior inference accuracy across models. 
 

\end{document}
