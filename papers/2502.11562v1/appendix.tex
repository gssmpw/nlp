\clearpage
\newpage
\appendix

\section{Implementation Details}
When get query embeddings, we set $\alpha$ to 0.8. In RLRF, $\gamma$ is sequentially set to 1.05, 1.08, and 1.1 over three iterations. Meanwhile, in RLGF, both $\alpha_0$ and $\alpha_i$ are defined as 0.5.

For LLM, we use the following version:

1) Meta-Llama-3-8B-Instruct, 

2) Qwen-2.5-7B-Instruct, 

3) Mistral-7B-Instruct-v0.3, 

4) Meta-Llama-3-70B-Instruct, 

5) Qwen-2.5-72B-Instruct, 

6) gpt-4o-mini-2024-07-18

\section{Query Generation}
To enhance the effectiveness of query generation, we have implemented refinements to QGen, which encompasses two distinct phases: query generation and quality control.
During the query generation phase, we provide the LLM with relevant documents and tasks, prompting it to produce corresponding queries. In the quality control phase, the tasks, documents, and generated queries are presented to the LLM again, enabling it to evaluate whether the generated queries are relevant to the documents. Queries deemed irrelevant are then filtered out.

Taking Trec-covid dataset as an example, for query generation, we use the following prompt:
\begin{mdframed}[backgroundcolor=gray!20, linecolor=gray]
\small
Here is a retrieval task (Task) and a document (Passage):\\
\\
Task: Given a query on COVID-19, retrieve documents that answer the query.\\
\\
Passage: \{passage\}\\
\\
Given the retrieval task and the document, your mission is:\\
- Generate a query on COVID-19 that the document can answer.\\
\\
Note:\\
- The generated query should not contain the pronouns such as "this", "that", "it", "there", "here", etc.\\
- The generated query should be clear and 5 to 10 words.\\
- The generated query should be common and formal in terms of language style.\\
\\
Your output should be a string of the generated query. Remember do not explain your output.\\
\\
Your output:
\end{mdframed}

For quality control, we use the following prompt:
\begin{mdframed}[backgroundcolor=gray!20, linecolor=gray]
\small
Given a retrieval task (Task), a query (Query), and a document (Passage), your mission is Judge whether the document can answer the query..\\
\\
Task: Given a query on COVID-19, retrieve documents that answer the query.\\
\\
Query: \{query\}\\
Passage: \{passage\}\\
\\
Your output must be one of the following:\\
- 0: No, the document cannot answer the query.\\
- 1: Yes, the document can answer the query.\\
\\
Do not explain your answer in the output. Your output must be a single number.\\
\\
Your output:
\end{mdframed}

\section{Hypothetical Document}
For hypothetical document, we use the following prompt:

\begin{mdframed}[backgroundcolor=gray!20, linecolor=gray]
\small
Given a retrieval task and a query, your mission is to generate a brief document for the query in the context of the retrieval task.\\
\\
Task: Given a query on COVID-19, retrieve documents that answer the query.\\
\\
Query: \{query\}\\
\\
Your output:
\end{mdframed}