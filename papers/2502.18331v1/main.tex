% This must be in the first 5 lines to tell arXiv to use pdfLaTeX, which is strongly recommended.
\pdfoutput=1
% In particular, the hyperref package requires pdfLaTeX in order to break URLs across lines.

\documentclass[11pt]{article}

% Change "review" to "final" to generate the final (sometimes called camera-ready) version.
% Change to "preprint" to generate a non-anonymous version with page numbers.
\usepackage[preprint]{acl}

% Standard package includes
\usepackage{times}
\usepackage{latexsym}

% For proper rendering and hyphenation of words containing Latin characters (including in bib files)
\usepackage[T1]{fontenc}
% For Vietnamese characters
% \usepackage[T5]{fontenc}
% See https://www.latex-project.org/help/documentation/encguide.pdf for other character sets

% This assumes your files are encoded as UTF8
\usepackage[utf8]{inputenc}

% This is not strictly necessary, and may be commented out,
% but it will improve the layout of the manuscript,
% and will typically save some space.
\usepackage{microtype}

% This is also not strictly necessary, and may be commented out.
% However, it will improve the aesthetics of text in
% the typewriter font.
\usepackage{inconsolata}

%Including images in your LaTeX document requires adding
%additional package(s)
\usepackage{graphicx}
\usepackage{amsmath}
\usepackage{mathtools}
\usepackage{booktabs}
\usepackage{enumitem}
\usepackage{hhline}
\usepackage{multirow}
\usepackage{bbm}
\usepackage{colortbl}
\usepackage{listings}
\usepackage{mdframed}
\usepackage{adjustbox}
\usepackage[most]{tcolorbox}

\definecolor{lightgray}{RGB}{240,240,240}
\definecolor{Gray}{gray}{0.7}
\definecolor{Green}{rgb}{0.05, 0.5, 0.06}
\definecolor{Purple}{rgb}{0.56, 0.0, 1.0}
\definecolor{Orange}{rgb}{1.0, 0.55, 0.0}\definecolor{Blue}{rgb}{0.0, 0.2, 0.4}

% If the title and author information does not fit in the area allocated, uncomment the following
%
%\setlength\titlebox{<dim>}
%
% and set <dim> to something 5cm or larger.

\title{\textsc{BottleHumor}: Self-Informed Humor Explanation using the Information Bottleneck Principle}
% Unsupervised Discovery of Useful Information\\for Multimodal Humor Understanding}

% Author information can be set in various styles:
% For several authors from the same institution:
% \author{Author 1 \and ... \and Author n \\
%         Address line \\ ... \\ Address line}
% if the names do not fit well on one line use
%         Author 1 \\ {\bf Author 2} \\ ... \\ {\bf Author n} \\
% For authors from different institutions:
% \author{Author 1 \\ Address line \\  ... \\ Address line
%         \And  ... \And
%         Author n \\ Address line \\ ... \\ Address line}
% To start a separate ``row'' of authors use \AND, as in
% \author{Author 1 \\ Address line \\  ... \\ Address line
%         \AND
%         Author 2 \\ Address line \\ ... \\ Address line \And
%         Author 3 \\ Address line \\ ... \\ Address line}
\author{EunJeong Hwang$^{1,2}$, Peter West$^{1}$, and Vered Shwartz$^{1,2}$ \\
$^1$ University of British Columbia~~~$^2$ Vector Institute for AI\\
{\tt \{ejhwang,pwest,vshwartz\}@cs.ubc.ca}}

% \author{EunJeong Hwang \\
%   Affiliation / Address line 1 \\
%   Affiliation / Address line 2 \\
%   Affiliation / Address line 3 \\
%   \texttt{email@domain} \\\And
%   Second Author \\
%   Affiliation / Address line 1 \\
%   Affiliation / Address line 2 \\
%   Affiliation / Address line 3 \\
%   \texttt{email@domain} \\}

%\author{
%  \textbf{First Author\textsuperscript{1}},
%  \textbf{Second Author\textsuperscript{1,2}},
%  \textbf{Third T. Author\textsuperscript{1}},
%  \textbf{Fourth Author\textsuperscript{1}},
%\\
%  \textbf{Fifth Author\textsuperscript{1,2}},
%  \textbf{Sixth Author\textsuperscript{1}},
%  \textbf{Seventh Author\textsuperscript{1}},
%  \textbf{Eighth Author \textsuperscript{1,2,3,4}},
%\\
%  \textbf{Ninth Author\textsuperscript{1}},
%  \textbf{Tenth Author\textsuperscript{1}},
%  \textbf{Eleventh E. Author\textsuperscript{1,2,3,4,5}},
%  \textbf{Twelfth Author\textsuperscript{1}},
%\\
%  \textbf{Thirteenth Author\textsuperscript{3}},
%  \textbf{Fourteenth F. Author\textsuperscript{2,4}},
%  \textbf{Fifteenth Author\textsuperscript{1}},
%  \textbf{Sixteenth Author\textsuperscript{1}},
%\\
%  \textbf{Seventeenth S. Author\textsuperscript{4,5}},
%  \textbf{Eighteenth Author\textsuperscript{3,4}},
%  \textbf{Nineteenth N. Author\textsuperscript{2,5}},
%  \textbf{Twentieth Author\textsuperscript{1}}
%\\
%\\
%  \textsuperscript{1}Affiliation 1,
%  \textsuperscript{2}Affiliation 2,
%  \textsuperscript{3}Affiliation 3,
%  \textsuperscript{4}Affiliation 4,
%  \textsuperscript{5}Affiliation 5
%\\
%  \small{
%    \textbf{Correspondence:} \href{mailto:email@domain}{email@domain}
%  }
%}

\newcommand{\method}{\textsc{BottleHumor}}
\newcommand{\base}{\textsc{ZS}}
\newcommand{\chain}{\textsc{CoT}}
\newcommand{\critic}{\textsc{SR}}
\newcommand{\nocritic}{\textsc{SR-noC}}

\begin{document}
\maketitle
\begin{abstract}
% Humor plays a crucial role in understanding ourselves, society, and the world, shaped by cultural, social, and personal factors, and expressed through images, text, and audio. 
Humor is prevalent in online communications and it often relies on more than one modality (e.g., cartoons and memes).  
Interpreting humor in multimodal settings requires drawing on diverse types of knowledge, including metaphorical, sociocultural, and commonsense knowledge. However, identifying the most useful knowledge remains an open question. 
We introduce \method{}, a method inspired by the information bottleneck principle that elicits relevant world knowledge from vision and language models which is iteratively refined for generating an explanation of the humor in an unsupervised manner. Our experiments on three datasets confirm the advantage of our method over a range of baselines. 
Our method can further be adapted in the future for additional tasks that can benefit from eliciting and conditioning on relevant world knowledge and open new research avenues in this direction.

%We provide a detailed analysis of the IB components and the impact of implications on final explanations.


\end{abstract}

\section{Introduction}
\label{sec:intro}
%!TEX root = gcn.tex
\section{Introduction}
Graphs, representing structural data and topology, are widely used across various domains, such as social networks and merchandising transactions.
Graph convolutional networks (GCN)~\cite{iclr/KipfW17} have significantly enhanced model training on these interconnected nodes.
However, these graphs often contain sensitive information that should not be leaked to untrusted parties.
For example, companies may analyze sensitive demographic and behavioral data about users for applications ranging from targeted advertising to personalized medicine.
Given the data-centric nature and analytical power of GCN training, addressing these privacy concerns is imperative.

Secure multi-party computation (MPC)~\cite{crypto/ChaumDG87,crypto/ChenC06,eurocrypt/CiampiRSW22} is a critical tool for privacy-preserving machine learning, enabling mutually distrustful parties to collaboratively train models with privacy protection over inputs and (intermediate) computations.
While research advances (\eg,~\cite{ccs/RatheeRKCGRS20,uss/NgC21,sp21/TanKTW,uss/WatsonWP22,icml/Keller022,ccs/ABY318,folkerts2023redsec}) support secure training on convolutional neural networks (CNNs) efficiently, private GCN training with MPC over graphs remains challenging.

Graph convolutional layers in GCNs involve multiplications with a (normalized) adjacency matrix containing $\numedge$ non-zero values in a $\numnode \times \numnode$ matrix for a graph with $\numnode$ nodes and $\numedge$ edges.
The graphs are typically sparse but large.
One could use the standard Beaver-triple-based protocol to securely perform these sparse matrix multiplications by treating graph convolution as ordinary dense matrix multiplication.
However, this approach incurs $O(\numnode^2)$ communication and memory costs due to computations on irrelevant nodes.
%
Integrating existing cryptographic advances, the initial effort of SecGNN~\cite{tsc/WangZJ23,nips/RanXLWQW23} requires heavy communication or computational overhead.
Recently, CoGNN~\cite{ccs/ZouLSLXX24} optimizes the overhead in terms of  horizontal data partitioning, proposing a semi-honest secure framework.
Research for secure GCN over vertical data  remains nascent.

Current MPC studies, for GCN or not, have primarily targeted settings where participants own different data samples, \ie, horizontally partitioned data~\cite{ccs/ZouLSLXX24}.
MPC specialized for scenarios where parties hold different types of features~\cite{tkde/LiuKZPHYOZY24,icml/CastigliaZ0KBP23,nips/Wang0ZLWL23} is rare.
This paper studies $2$-party secure GCN training for these vertical partition cases, where one party holds private graph topology (\eg, edges) while the other owns private node features.
For instance, LinkedIn holds private social relationships between users, while banks own users' private bank statements.
Such real-world graph structures underpin the relevance of our focus.
To our knowledge, no prior work tackles secure GCN training in this context, which is crucial for cross-silo collaboration.


To realize secure GCN over vertically split data, we tailor MPC protocols for sparse graph convolution, which fundamentally involves sparse (adjacency) matrix multiplication.
Recent studies have begun exploring MPC protocols for sparse matrix multiplication (SMM).
ROOM~\cite{ccs/SchoppmannG0P19}, a seminal work on SMM, requires foreknowledge of sparsity types: whether the input matrices are row-sparse or column-sparse.
Unfortunately, GCN typically trains on graphs with arbitrary sparsity, where nodes have varying degrees and no specific sparsity constraints.
Moreover, the adjacency matrix in GCN often contains a self-loop operation represented by adding the identity matrix, which is neither row- nor column-sparse.
Araki~\etal~\cite{ccs/Araki0OPRT21} avoid this limitation in their scalable, secure graph analysis work, yet it does not cover vertical partition.

% and related primitives
To bridge this gap, we propose a secure sparse matrix multiplication protocol, \osmm, achieving \emph{accurate, efficient, and secure GCN training over vertical data} for the first time.

\subsection{New Techniques for Sparse Matrices}
The cost of evaluating a GCN layer is dominated by SMM in the form of $\adjmat\feamat$, where $\adjmat$ is a sparse adjacency matrix of a (directed) graph $\graph$ and $\feamat$ is a dense matrix of node features.
For unrelated nodes, which often constitute a substantial portion, the element-wise products $0\cdot x$ are always zero.
Our efficient MPC design 
avoids unnecessary secure computation over unrelated nodes by focusing on computing non-zero results while concealing the sparse topology.
We achieve this~by:
1) decomposing the sparse matrix $\adjmat$ into a product of matrices (\S\ref{sec::sgc}), including permutation and binary diagonal matrices, that can \emph{faithfully} represent the original graph topology;
2) devising specialized protocols (\S\ref{sec::smm_protocol}) for efficiently multiplying the structured matrices while hiding sparsity topology.


 
\subsubsection{Sparse Matrix Decomposition}
We decompose adjacency matrix $\adjmat$ of $\graph$ into two bipartite graphs: one represented by sparse matrix $\adjout$, linking the out-degree nodes to edges, the other 
by sparse matrix $\adjin$,
linking edges to in-degree nodes.

%\ie, we decompose $\adjmat$ into $\adjout \adjin$, where $\adjout$ and $\adjin$ are sparse matrices representing these connections.
%linking out-degree nodes to edges and edges to in-degree nodes of $\graph$, respectively.

We then permute the columns of $\adjout$ and the rows of $\adjin$ so that the permuted matrices $\adjout'$ and $\adjin'$ have non-zero positions with \emph{monotonically non-decreasing} row and column indices.
A permutation $\sigma$ is used to preserve the edge topology, leading to an initial decomposition of $\adjmat = \adjout'\sigma \adjin'$.
This is further refined into a sequence of \emph{linear transformations}, 
which can be efficiently computed by our MPC protocols for 
\emph{oblivious permutation}
%($\Pi_{\ssp}$) 
and \emph{oblivious selection-multiplication}.
% ($\Pi_\SM$)
\iffalse
Our approach leverages bipartite graph representation and the monotonicity of non-zero positions to decompose a general sparse matrix into linear transformations, enhancing the efficiency of our MPC protocols.
\fi
Our decomposition approach is not limited to GCNs but also general~SMM 
by 
%simply 
treating them 
as adjacency matrices.
%of a graph.
%Since any sparse matrix can be viewed 

%allowing the same technique to be applied.

 
\subsubsection{New Protocols for Linear Transformations}
\emph{Oblivious permutation} (OP) is a two-party protocol taking a private permutation $\sigma$ and a private vector $\xvec$ from the two parties, respectively, and generating a secret share $\l\sigma \xvec\r$ between them.
Our OP protocol employs correlated randomnesses generated in an input-independent offline phase to mask $\sigma$ and $\xvec$ for secure computations on intermediate results, requiring only $1$ round in the online phase (\cf, $\ge 2$ in previous works~\cite{ccs/AsharovHIKNPTT22, ccs/Araki0OPRT21}).

Another crucial two-party protocol in our work is \emph{oblivious selection-multiplication} (OSM).
It takes a private bit~$s$ from a party and secret share $\l x\r$ of an arithmetic number~$x$ owned by the two parties as input and generates secret share $\l sx\r$.
%between them.
%Like our OP protocol, o
Our $1$-round OSM protocol also uses pre-computed randomnesses to mask $s$ and $x$.
%for secure computations.
Compared to the Beaver-triple-based~\cite{crypto/Beaver91a} and oblivious-transfer (OT)-based approaches~\cite{pkc/Tzeng02}, our protocol saves ${\sim}50\%$ of online communication while having the same offline communication and round complexities.

By decomposing the sparse matrix into linear transformations and applying our specialized protocols, our \osmm protocol
%($\prosmm$) 
reduces the complexity of evaluating $\numnode \times \numnode$ sparse matrices with $\numedge$ non-zero values from $O(\numnode^2)$ to $O(\numedge)$.

%(\S\ref{sec::secgcn})
\subsection{\cgnn: Secure GCN made Efficient}
Supported by our new sparsity techniques, we build \cgnn, 
a two-party computation (2PC) framework for GCN inference and training over vertical
%ly split
data.
Our contributions include:

1) We are the first to explore sparsity over vertically split, secret-shared data in MPC, enabling decompositions of sparse matrices with arbitrary sparsity and isolating computations that can be performed in plaintext without sacrificing privacy.

2) We propose two efficient $2$PC primitives for OP and OSM, both optimally single-round.
Combined with our sparse matrix decomposition approach, our \osmm protocol ($\prosmm$) achieves constant-round communication costs of $O(\numedge)$, reducing memory requirements and avoiding out-of-memory errors for large matrices.
In practice, it saves $99\%+$ communication
%(Table~\ref{table:comm_smm}) 
and reduces ${\sim}72\%$ memory usage over large $(5000\times5000)$ matrices compared with using Beaver triples.
%(Table~\ref{table:mem_smm_sparse}) ${\sim}16\%$-

3) We build an end-to-end secure GCN framework for inference and training over vertically split data, maintaining accuracy on par with plaintext computations.
We will open-source our evaluation code for research and deployment.

To evaluate the performance of $\cgnn$, we conducted extensive experiments over three standard graph datasets (Cora~\cite{aim/SenNBGGE08}, Citeseer~\cite{dl/GilesBL98}, and Pubmed~\cite{ijcnlp/DernoncourtL17}),
reporting communication, memory usage, accuracy, and running time under varying network conditions, along with an ablation study with or without \osmm.
Below, we highlight our key achievements.

\textit{Communication (\S\ref{sec::comm_compare_gcn}).}
$\cgnn$ saves communication by $50$-$80\%$.
(\cf,~CoGNN~\cite{ccs/KotiKPG24}, OblivGNN~\cite{uss/XuL0AYY24}).

\textit{Memory usage (\S\ref{sec::smmmemory}).}
\cgnn alleviates out-of-memory problems of using %the standard 
Beaver-triples~\cite{crypto/Beaver91a} for large datasets.

\textit{Accuracy (\S\ref{sec::acc_compare_gcn}).}
$\cgnn$ achieves inference and training accuracy comparable to plaintext counterparts.
%training accuracy $\{76\%$, $65.1\%$, $75.2\%\}$ comparable to $\{75.7\%$, $65.4\%$, $74.5\%\}$ in plaintext.

{\textit{Computational efficiency (\S\ref{sec::time_net}).}} 
%If the network is worse in bandwidth and better in latency, $\cgnn$ shows more benefits.
$\cgnn$ is faster by $6$-$45\%$ in inference and $28$-$95\%$ in training across various networks and excels in narrow-bandwidth and low-latency~ones.

{\textit{Impact of \osmm (\S\ref{sec:ablation}).}}
Our \osmm protocol shows a $10$-$42\times$ speed-up for $5000\times 5000$ matrices and saves $10$-2$1\%$ memory for ``small'' datasets and up to $90\%$+ for larger ones.


\section{Related Work}
\label{sec:bg}
\section{Related Work}

\subsection{Personalization and Role-Playing}
Recent works have introduced benchmark datasets for personalizing LLM outputs in tasks like email, abstract, and news writing, focusing on shorter outputs (e.g., 300 tokens for product reviews \citep{kumar2024longlamp} and 850 for news writing \citep{shashidhar-etal-2024-unsupervised}). These methods infer user traits from history for task-specific personalization \citep{sun-etal-2024-revealing, sun-etal-2025-persona, pal2024beyond, li2023teach, salemi2025reasoning}. In contrast, we tackle the more subjective problem of long-form story writing, with author stories averaging 1500 tokens. Unlike prior role-playing approaches that use predefined personas (e.g., Tony Stark, Confucius) \citep{wang-etal-2024-rolellm, sadeq-etal-2024-mitigating, tu2023characterchat, xu2023expertprompting}, we propose a novel method to infer story-writing personas from an author’s history to guide role-playing.


\subsection{Story Understanding and Generation}  
Prior work on persona-aware story generation \citep{yunusov-etal-2024-mirrorstories, bae-kim-2024-collective, zhang-etal-2022-persona, chandu-etal-2019-way} defines personas using discrete attributes like personality traits, demographics, or hobbies. Similarly, \citep{zhu-etal-2023-storytrans} explore story style transfer across pre-defined domains (e.g., fairy tales, martial arts, Shakespearean plays). In contrast, we mimic an individual author's writing style based on their history. Our approach differs by (1) inferring long-form author personas—descriptions of an author’s style from their past works, rather than relying on demographics, and (2) handling long-form story generation, averaging 1500 tokens per output, exceeding typical story lengths in prior research.

\section{\method{}}
\label{sec:method}

\section{\Kbase Generation}\label{sec:kb}
The Knowledge Management System module (KMS), is in charge of taking the natural language description of both the environment and the actions that the agents can do, and convert them to a Prolog \kb using a LLM. 
The \kb contains all the necessary elements to define the mapped planning problem introduced in the previous section.

The framework works by considering a high-level and a low-level \kbase. For this reason, the input descriptions are also split into \HL and \LL. The former captures more abstract concepts, e.g., complex actions such as \verb|move_block| or the objects that are present in the environment. The latter captures more concrete and physical aspects of the problem, e.g., the actions that can be actually carried out by the agents such as \verb|move_arm| or the positions of the blocks. An example of this division can be seen in Section~\ref{sssec:runegKMS}.

% Describe how the kb works
The \kbase is divided in the following parts:
\begin{itemize}
    \item General \kb ($K$): contains the grounding predicates, both for the \HL and \LL. These predicates describe parts of the scenario or of the environment that do not change during execution. For example, the predicate \verb|wheeled(a1)|, which states that robot \verb | a1| has wheels, should be part of the general \kb and not of the state. 
    \item Initial ($I$) and final states ($G$): they contain all the fluents that change during the execution of the plan. This could be, for example, the position of blocks in the environment. 
    \item High-level actions ($DA_H$): each high-level action predicate is written as:
\begin{minted}[fontsize=\small,breaklines]{Prolog}
action(
    action_name(args),
    [positive_preconditions],
    [negative_preconditions],
    [grounding_predicates],
    [effects]
).
\end{minted}
    The low-level actions ($DA_L$) have the same structure, but instead of being described as predicates of type \verb|action|, they are described as \verb|ll_action|. The preconditions $\pc{a}$ of an action $a$ are obtained by combining the list of predicates \verb|positive_preconditions| and \verb|negative_preconditions|. The predicates in the list \verb|grounding_predicates| are used to ground the parametrised fluents of the action. For example, the action \verb|move_block| depends on a block, and we can check that the action is correctly picking a block and not another object by querying the \kb in this step. Section~\ref{sssec:runegKMS} clarifies this aspect.  
    \item Mappings ($M$): contains a dictionary of \HL actions $DA_H$ and how they should be mapped to a sequence of \LL actions $DA_L$. As will become apparent in the following, the distinction between \HL and \LL actions induces a significant simplification in the planning phase.
    \item Resources ($R$): the predicate \verb|resource\1| states whether another predicate is part of the resources or not. As mentioned before, this is helpful because it allows one to shrink the complexity of the problem not having to check multiple predicates, but instead they are later allocated during the optimisation part.
\end{itemize}

% Describe the process to validate the initial descriptions
Once the user provides descriptions for the \HL and \LL parts, the framework performs a consistency check to ensure that there are no conflicts between them. It verifies that both descriptions share the same goal, that objects remain consistent across \HL and \LL, and that agents are capable of executing the tasks. This validation is carried out by an LLM, which, if inconsistencies are detected, provides an explanation to help the user make the necessary corrections.

In both this step and the subsequent steps to generate the \kb, the LLM is not used directly out of the box. Instead, we employ the Chain-of-Thought (CoT)~\cite{wei2022chain} approach, which involves providing the LLM with examples to guide its reasoning. This process ensures that the output is not only structurally correct, but also more aligned with the overall goal of the task.

% Describe how the different aspects of the kb are extracted
Examples are particularly important when generating the \kb. Indeed, as we have mentioned before, the \kbase is highly structured and the planner expects to have the different components written correctly. CoT enables the LLM to know these details. 

We tested two different ways of generating the \kb through LLMs:
\begin{itemize}
    \item either we produced the whole \kb for the high-level and the low-level all at once, or
    \item we produced the single parts of the \kbs. 
\end{itemize}

The first approach is quite straightforward: once we have the examples to give to the LLM for the CoT process, we can input the \HL description and query the LLM to first extract the high-level \kb, and then also feed the created \HL \kb to the LLM to generate the \LL \kb, which will contain everything. % Highlight why one would want this. 

Instead, the second approach requires more requests to the LLM. We first focus on the \HL \kb, and then feed the \kb that we have obtained to generate the \LL parts. For the \HL generation, we ask the LLM to generate the general \kb, the initial and final states, and the actions set in this particular order. Each time we provide the LLM with the \HL description and with the elements generated in the previous steps. The same thing is done for the \LL \kb, generating again the four components and feeding each time also the \HL \kb. We include a final step that generates the mappings between the \HL and \LL actions. As for all the other steps, also in this final step, we pass the previously generated elements of the \LL \kb. 
Although generating the entire \kbase at once would reduce token usage and speed up the process, dividing the generation of the \kb into distinct steps enhances the system's accuracy, as demonstrated in the experimental evaluation of Section~\ref{sec:experiments}. This improvement comes because the iterative approach allows the LLM to first focus on generating more homogeneous information (i.e., the high-level) and then leverage the previously generated content to perform a consistency check.  

\subsection{Runing Example -- KMS}
\label{sssec:runegKMS}

We now introduce a running example, which will be used throughout this work to expose the interplay between the different components of the framework. 
This scenario is taken from the blocks-world domain~\cite{blocksworld}, which is frequently used in task planning. In particular, in this scenario we consider a table, blocks, which may either be directly on the table or stack on top of each other, and robotics arms, which move the blocks around. Each block is also associated with a position in the 2D space. 
In this particular example, we start from a situation in which we have two blocks, \verb|b1| and \verb|b2|, which are sat on the table in position (1,1) and (3,1) respectively. The goal is to move \verb|b1| in position (2,2) and then put \verb|b2| on top of it. An iconography of the example can be seen in Figure~\ref{fig:running-example}.

\begin{figure}
    \centering
    \input{figures/running-example}
    \caption{A scheme showing the running example. Two blocks must be moved from their initial position to a new position in which they are also stacked.}
    \label{fig:running-example}
\end{figure}

While this is a trivial example, it highlights very well the
capability of the knowledge management system to generate complex
predicates that can be used for planning and it also shows the
cooperative abilities of the framework. Indeed, while using a single
robotic arm generates a straight-forward plan solution, coordinating
two robotics arms to do the same task reduces the completion time at the price
of a higher planning complexity.


Let's now focus on the \kbase generation. For example, the two queries that could be passed to the system are:

\begin{textbox}{\HL query}
% \textboxtitle{\HL query}{
\footnotesize
In this scenario, there are two blocks on a table. They are placed in positions (1,1) and (3,1). There is a robotic agent that can move the blocks from one position to another. In particular, it can:
\begin{itemize}
    \item move a block from a position on the table, to another position on the table;
    \item move a block from a position on the table, to the top of another block;
    \item move a block from the top of a block, to a position on the table;
    \item move a block from the top of a block, to the top of another block.
\end{itemize}
The goal of this scenario is to move the first block from position (1,1) to position (2,2), and then place the second block on top of it.
% }
\end{textbox}

\begin{textbox}{\LL query}
% \textboxtitle{\LL query}{
\footnotesize
In this scenario, there are two blocks that are in the same positions as described in the \HL query. The robotic agent is a robotic arm and its end-effector is a two-finger gripper that can grasp an object. In particular, the following APIs are made available for the agent to move:
\begin{itemize}
    \item \texttt{move\_arm(Arm, X1, Y1, X2, Y2)}, which allows the end-effector to be moved from one position to another;
    \item \texttt{grip(Arm)}, which allows the gripper to be lowered onto a block and grip the block;
    \item \texttt{release(Arm)}, which allows to lower the block and release the gripper leaving the block in place.
\end{itemize}
% }
\end{textbox}


As mentioned before, the first step is to pass the two queries for validation. In this case, when using GPT-4o, both queries pass the consistency check. If instead of using the previous \HL query, we would modify it by removing the possibility of placing one block on top of the other, then the LLM would return the following error:

\begin{textboxerror}
\footnotesize
The description of the scenario has an issue. The agent is described as being able to move a block from a position on the table to another position on the table, and from the top of a block to a position on the table. However, the goal requires the agent to place the second block on top of the first block, which is not listed as an action that the agent can carry out. The agent's capabilities do not include placing a block on top of another block, which is necessary to achieve the goal.
\end{textboxerror}

\noindent This also allows the user to understand where the mistake is and correct it. 

Once the validation part succeeds, the system generates the \HL \kb. In this particular instance, for space limitation, we present only the general \kbase ($K$), the initial ($I$) and final ($G$) states, and a single action. 

\begin{center}
\begin{minipage}{\linewidth}
    \begin{minipage}{.48\linewidth}
        \begin{codebox}{prolog}{General KB}
% Positions
pos(1,1).
pos(2,2).
pos(3,1).

% Blocks
block(b1).
block(b2).

% Agents
agent(a1).

% Resources
resources(agent(_)).
        \end{codebox}
    \end{minipage}
    \hfill
    \begin{minipage}{.48\linewidth}
        \begin{minipage}{\linewidth}
        \begin{codebox}{prolog}{Initial state ($I$)}
init_state([
  ontable(b1), ontable(b2),
  at(b1,1,1), at(b2,3,1),
  clear(b1), clear(b2),
  available(a1)
]).
        \end{codebox}
        \end{minipage}
        \hspace{1cm}\\
        \begin{minipage}{\linewidth}
        \begin{codebox}{prolog}{Final state ($G$)}
goal_state([
  ontable(b1),
  on(b2, b1),
  at(b1,2,2), at(b2,2,2),
  clear(b2),
  available(a1)
]).
        \end{codebox}
        \end{minipage}
    \end{minipage}
\end{minipage}
\begin{codebox}{prolog}{Action example}
action(move_table_to_table_start(Agent, Block, X1, Y1, X2, Y2), 
  [ontable(Block), at(Block, X1, Y1), available(Agent), clear(Block)],
  [
    at(_, X2, Y2), on(Block, _), moving_table_to_table(_, Block, _, _, _, _), 
    moving_table_to_block(_, Block, _, _, _, _, _)
  ],
  [agent(Agent), pos(X1, Y1), pos(X2, Y2), block(Block)],
  [
    del(available(Agent)), del(clear(Block)), del(ontable(Block)), del(at(Block, X1, Y1)),
    add(moving_table_to_table(Agent, Block, X1, Y1, X2, Y2))
  ]
).
\end{codebox}
\end{center}

The resulting \HL \kb is human-readable and relatively simple (in fulfilment of requirement \textbf{R2}).
The user at this point can make corrections to the \HL \kb, if needed, and finally, \frameworkname will also generate the \LL \kbase. In this case for space limitation, we show the changes made to the previous elements, one low-level action, and one mapping. 

\begin{center}
\begin{minipage}{\linewidth}
    \begin{minipage}{.48\linewidth}
        \begin{codebox}{prolog}{General KB}
% Positions
pos(0,0).
pos(1,1).
pos(2,2).
pos(3,1).

% Blocks
block(b1).
block(b2).

% Agents
agent(a1).

% Low-level predicates
ll_arm(a1).
ll_gripper(a1).

% Resources
resources(agent(_)).
        \end{codebox}
    \end{minipage}
    \hfill
    \begin{minipage}{.48\linewidth}
        \begin{minipage}{\linewidth}
        \begin{codebox}{prolog}{Initial state ($I$)}
init_state([
  ontable(b1), ontable(b2),
  at(b1,1,1), at(b2,3,1),
  clear(b1), clear(b2),
  available(a1),
  ll_arm_at(a1,0,0), 
  ll_gripper(a1,open) 
]).
        \end{codebox}
        \end{minipage}
        \hspace{1cm}\\
        \begin{minipage}{\linewidth}
        \begin{codebox}{prolog}{Final state ($G$)}
goal_state([
  ontable(b1),
  on(b2, b1),
  at(b1,2,2), at(b2,2,2),
  clear(b2),
  available(a1),
  ll_arm_at(a1,_,_), 
  ll_gripper(a1,_)    
]).
        \end{codebox}
        \end{minipage}
    \end{minipage}
\end{minipage}
\begin{codebox}{prolog}{Action example}
ll_action(move_arm_start(Arm, X, Y),
  [],
  [ll_arm_at(_, X, Y), moving_arm(Arm, _, _, _, _), gripping(Arm, _), releasing(Arm)],
  [],
  [ll_arm(Arm), pos(X, Y)],
  [
    add(moving_arm(Arm, X, Y)),
    del(ll_arm_at(Arm, X, Y))
  ]
).
\end{codebox}
\begin{codebox}{prolog}{Mapping example}
mapping(move_table_to_table_start(Agent, Block, X1, Y1, X2, Y2),
  [
    move_arm_start(Agent, X1, Y1),
    move_arm_end(Agent, X1, Y1),
    grip_start(Agent),
    grip_end(Agent),
    move_arm_start(Agent, X2, Y2),
    move_arm_end(Agent, X2, Y2),
    release_start(Agent),
    release_end(Agent)
  ]
).
\end{codebox}
\end{center}

Again, the user can correct possible errors (or anyway refine the \kb) and then move on to the planning phase.


\section{Plan Generation}\label{sec:plangen}
In this section, we describe how the framework uses the information from the \kb to generate a task plan for multiple agents.
Generation takes place in three steps: 
\begin{enumerate*}
    \item Generation of a total-order (TO) plan, 
    \item extraction of a partial-order (PO) plan and of the resources, 
    \item solution of a MILP problem to improve resource allocation and reducing the plan makespan by exploiting the possible parallel executions of actions.
\end{enumerate*}



\subsection{Total-Order Plan Generation}\label{ssec:toplangen}
A total-order plan is a strictly sequential list of actions that drives the system from the initial to the goal state. 
The algorithm used to extract a total-order plan is shown in~\autoref{alg:toplanning} and consists of two distinct steps:
\begin{itemize}
    \item identify a total-order plan for high-level actions, and
    \item recursively map each high-level action to a sequence of actions with a lower level until they are mapped to actions corresponding to the APIs of the available robotic resources.
\end{itemize}

\begin{algorithm}
\footnotesize
\caption{Algorithm generating a TO plan with mappings}\label{alg:toplanning}
\KwData{$TP=(F, DA, I, G, K)$}
\KwResult{Plan solving TP}

\DontPrintSemicolon

\SetKwProg{plan}{TO\_PLAN}{}{}
\SetKwProg{map}{APPLY\_MAP}{}{}
\SetKwProg{action}{APPLY\_ACTION}{}{}
\SetKwProg{maps}{APPLY\_MAPPINGS}{}{}

\SetKwInOut{Input} {In}
\SetKwInOut{Output}{Out}

\plan{(S, P)}{
  \Input{The current state $S$ and the current plan $P$}
  \Output{The final plan}
  \If{$S \neq G$}{
      select\_action($a_i$)\;
      (US, UP) $\gets$ APPLY\_ACTION($a_i$, S, P)\;
      P $\gets$ TO\_PLAN(US, UP)\;
  }
  (US, UP) $\gets$ APPLY\_MAPPINGS(S,P)\;
  \KwRet{P}\;
}

\maps{(S, P)}{
  \Input{The current state $S$ and the current plan $P$}
  \Output{The updated state $US$ and plan $UP$ after the mappings}
  US, UP $\gets$ S, P\;
  \ForEach{$a_i \in P$}{
    \If{\textnormal{is\_start($a_i$) $\wedge$ has\_mapping($a_i$)}}{
      (US, UP) $\gets$ APPLY\_MAP($a_i$, \textnormal{US}, UP)\;
    }
  }
  \KwRet{(US, UP}\;
}

\map{($a$, S, P)}{
  \Input{The action $a$, the current state $S$ and the current plan $P$}
  \Output{The updated state $US$ and plan $UP$ after the mappings}
  M $\gets$ mapping($a$)\;
  \ForEach{$a_i \in M$}{
    (US, UP) $\gets$ APPLY\_ACTION($a$, S, P)\;
  }
  \KwRet{(US, UP)}\;
}

\action{($a, S, P$)}{
  \Input{The action $a$, the current state $S$ and the current plan $P$}
  \Output{The updated state $US$ and plan $UP$ after applying the effects of $a$}
  \eIf{\textnormal{is\_applicable($a_i$)}}{
    US $\gets$ change\_state($a_i$.eff, S)\;
    UP $\gets$ plan\_action($a_i$, P)\;
    \KwRet{(US, UP)}\;
  }{
    \KwRet{(S, P)}
  }
}
\end{algorithm}

This enables the extraction of total-order plans that are consistent with the \kb provided, and we reduce the computational cost of checking all the possible actions at each time step. The \texttt{TO\_PLAN} function is the main function, which takes the initial and final states, and it inspects which actions can be executed given the current state. The \texttt{select\_action} function selects the next action from the set of possible actions. This search is based on the Prolog inference engine, which tries the actions in the order in which they are written in the KB, and hence it is not an informed search. 

The algorithm then moves to the \texttt{APPLY\_ACTION} function, which first checks if the chosen action's preconditions are met in the current state and, if they are, then it applies its effects changing the state (\texttt{change\_state}) and adding the action to the plan (\texttt{plan\_action}). It continues until the current state satisfies the goal state. Whenever the search reaches a fail point, we exploit the Prolog algorithm of resolution to step back and explore alternative possibilities.

Once the algorithm has extracted a high-level total-order plan, it applies the mappings. To do so, it iterates over the actions in the plan, and for each action it checks if it is a start action ($a_\vdash$) and if there are mappings for it. If this is the case, it calls the function \texttt{APPLY\_MAP}, which sequentially applies the actions in the mapping to the current state, also adding the actions to the plan. Notice that to do so, we call the \texttt{APPLY\_ACTION} function, which checks the preconditions of the actions w.r.t. the current state, ensuring that the lower-level actions can actually be applied.
% Also, the functions recursively check if any action from the mapped action has a mapping on its own, ensuring that all the actions have a direct grounding to APIs.

The total-order plan $TO$ extracted from this function is a list of actions that are executed in sequence:
\begin{equation*}
    \forall i \in \{0,\hdots \vert TO\vert-1\}~t(a_i)<t(a_{i+1})
\end{equation*}

\subsubsection{Running Example -- Total-Order Plan}
\label{sssec::runegTOPlan}

Let us consider again the \kb that we generated in Section~\ref{sssec:runegKMS}. Let us now see how \frameworkname extracts the TO plan.

The algorithm starts from the initial state and from the first action in the \kb, which in this case is the one shown in Section~\ref{sssec:runegKMS}. The algorithm takes the grounding predicates in this case:

\begin{minted}[fontsize=\footnotesize]{prolog}
agent(Agent), pos(X1, Y1), pos(X2, Y2), block(Block)
\end{minted}

and checks whether there is an assignment of predicates from the \kbase that satisfies them. For example, the predicate \verb|pos(1,1)| satisfies \verb|pos(X1,Y1)|. Not only this, but since the predicates in this list are grounded w.r.t. the \kb, one can also check some conditions. For example, if we were to assign the values to the previous predicates, it can happen that \verb|X1 = X2| and \verb|Y1 = Y2|, which is useless for an action that moves a block from one position to another. By adding the following predicates, we can ensure that the values are different:

\begin{minted}[fontsize=\footnotesize]{prolog}
agent(Agent), pos(X1, Y1), pos(X2, Y2), block(Block), X1\=X2, Y1\=Y2
\end{minted}

Once an assignment for the predicates inside the grounding list is found, the algorithm checks whether the predicates inside the preconditions are satisfied. Let us consider the preconditions for the \verb|move_table_to_table_start| action from Section~\ref{sssec:runegKMS}:

\begin{minted}[fontsize=\footnotesize]{prolog}
% Positive predicates
[ontable(Block), at(Block, X1, Y1), available(Agent), clear(Block)],
% Negative predicates
[
  at(_, X2, Y2), on(Block, _), moving_table_to_table(_, Block, _, _, _,_), 
  moving_table_to_block(_, Block, _, _, _, _, _)
]
\end{minted}

After the first grounding step, they become the following:

\begin{minted}[fontsize=\footnotesize]{prolog}
% Positive predicates
[ontable(b1), at(b1, 0, 0), available(a1), clear(b1)],
% Negative predicates
[
  at(_, 0, 0), on(b1, _), moving_table_to_table(_, b1, _, _, _,_), 
  moving_table_to_block(_, b1, _, _, _, _, _)
]
\end{minted}

The algorithm checks whether the predicates from the first list are satisfied in the current state and whether the predicates from the second list are not present in the current state. Comparing them with the initial state as shown in Section~\ref{sssec:runegKMS}, we can see that \verb|ontable(b1)| is present, but \verb|at(b1, 0, 0)|, so this combination of predicates would already be discarded. The first grounding that is accepted is that in which \verb | Block = b1, X1 = 1, Y1 = 1, Agent = a1 |. Notice that the predicates that start with \verb|_| mean "any", e.g., the predicate \verb|at(_, 0, 0)| checks if there is any predicate with name \verb|at| and arity 3 that has the last two arguments set to 0, regardless of what the first argument is.

By checking the different combinations of actions, the planner can extract a \HL TO plan. In this case, it would be something like this:

\begin{minted}[fontsize=\footnotesize]{text}
[0] move_table_to_table_start(a1, b1, 1, 1, 2, 2)
[1] move_table_to_table_end(a1, b1, 1, 1, 2, 2)
[2] move_table_to_block_start(a1, b2, 3, 1, 2, 2)
[3] move_table_to_block_end(a1, b2, 3, 1, 2, 2)
\end{minted}

At this point, the algorithm takes the mappings and it applies them to the previous plan. For instance, from Section~\ref{sssec:runegKMS} we saw that the mapping for \verb|move_table_to_table_start| is:
\begin{minted}[fontsize=\footnotesize]{prolog}
mapping(move_table_to_table_start(Agent, Block, X1, Y1, X2, Y2),
  [
    move_arm_start(Agent, X1, Y1), move_arm_end(Agent, X1, Y1),
    grip_start(Agent), grip_end(Agent),
    move_arm_start(Agent, X2, Y2), move_arm_end(Agent, X2, Y2),
    release_start(Agent), release_end(Agent)
  ]
).
\end{minted}

Hence, we would change the previous plan with:

\begin{minted}[fontsize=\footnotesize]{text}
[0] move_table_to_table_start(a1, b1, 1, 1, 2, 2)
[1] move_arm_start(a1, 1, 1)
[2] move_arm_end(a1, 1, 1)
[3] grip_start(a1)
[4] grip_end(a1)
[5] move_arm_start(a1, 2, 2)
[6] move_arm_end(a1, 2, 2)
[7] release_start(a1)
[8] release_end(a1)
[9] move_table_to_table_end(a1, b1, 1, 1, 2, 2)
[10] move_table_to_block_start(a1, b2, 3, 1, 2, 2)
[11] move_arm_start(a3, 3, 1)
[12] move_arm_end(a1, 3, 1)
[13] grip_start(a1)
[14] grip_end(a1)
[15] move_arm_start(a1, 2, 2)
[16] move_arm_end(a1, 2, 2)
[17] release_start(a1)
[18] release_end(a1)
[19] move_table_to_block_end(a1, b2, 3, 1, 2, 2)
\end{minted}


\subsection{Partial-Order Plan Generation}\label{ssec:poplangen}
The next step is to analyse the total-order plan in search of all possible causal relationships. This is done by
looking for actions that enable other actions (enablers). In addition, we extract all the resources that can be allocated
and used for the execution of the task. This step will be important for the next phase of the planning process, the MILP problem, in which 
the resources will be re-allocated allowing for shrinking the makespan of the plan.
%
In this work, the only resource considered is the robotic agent, but this limitation could easily be removed by modifying the \kb.  To this end,  we define a special predicate, named \texttt{resource/1}, that allows us to specify the resources.

Given an action $a_i$, another action $a_j$ is an enabler of $a_i$ if it either adds a literal $l$ satisfying one or more preconditions of $a_i$, or it removes a fluent violating one or more preconditions of $a_i$, and if $a_i$ happens after $a_j$: 

\begin{equation}
\small
\begin{array}{rl}
     a_j \in \ach{a_i} \iff & t(a_i) > t(a_j) \wedge \\
                            & ((l\in \pc{a_i}~ \wedge add(l)\in \eff{a_j}) \vee\\
                            & \,\,(\lnot l\in \pc{a_i} \wedge del(l)\in \eff{a_j}))
\end{array}
\label{eq:enablers}
\end{equation}

It is important to note that we consider an action $a_j\notin\ach{a_i}$ if there is at least a fluent $l$ that is not a resource. If all the fluents and their arguments that would make $a_j$ an enabler of $a_i$ are resources, then $a_j$ is not considered an enabler, as this relationship depends on the assignment of the resources, which comes with the optimisation step. 

Besides the enablers added corresponding to the classical definition, we also enforce the following precedence constraints:
\begin{itemize}
    \item When we expand a mapping $m(\alpha_i)$ of a high-level durative action $\alpha_i$ and reach the ending action $\aEnd{\alpha_i}$, then we add all previous durative actions as enablers until the corresponding start action. For example, assume that $m(\alpha_i)=\{\alpha_j, \alpha_k\}$, this means that the total-order plan will be the sequence $\{\aStart{\alpha_i}, \aStart{\alpha_j}, \aEnd{\alpha_j}, \aStart{\alpha_k}, \aEnd{\alpha_k}, \aEnd{\alpha_i}\}$. It follows that $\aStart{\alpha_i}$ is an enabler of $\aEnd{\alpha_i}$, but also all intermediate actions are part of the set of its enablers as they must be completed in order for $\alpha_i$ to end.
    \begin{equation}
        \bigwedge_{a\in m(\alpha_i)} a\in \ach{\aEnd{\alpha_i}}.
        \label{eq:constraint5}
    \end{equation}
    \item When we expand a mapping, all actions in the mapping must have the start of the higher-level action as one of the enablers. For instance, after the previous example, $\aStart{\alpha_j}, \aEnd{\alpha_j}, \aStart{\alpha_k}, \aEnd{\alpha_k}$ have $\aStart{\alpha_i}$ as an enabler.
    \begin{equation}
        \bigwedge_{a\in m(\alpha_i)} \aStart{\alpha} \in \ach{a_i}.
        \label{eq:constraint4}
    \end{equation}
\end{itemize}

% We then create a graph from which we can extract partial-order plans. To do this, after having obtained a plan from the \texttt{TO\_PLAN} from~\autoref{alg:planning}, we look for the achievers of the actions as shown in~\autoref{alg:po_planning}. 

The algorithm that manages this extraction is shown in~\autoref{alg:poplanning}. For ease of reading, we define $R\subseteq F$ as the set of fluents that are resources.

The algorithm \texttt{FIND\_ENABLERS} takes the total-order plan and, starting with the first action in the plan, it extracts all the causal relationships between the actions. The auxiliary function \texttt{IS\_ENABLER} tests whether an action $a_j$ is an enabler of an action $a_i$ by checking the properties of~\autoref{eq:enablers} plus the precedence constraints just described. Finally, notice that the literal checked to be present (absent) in both additive (subtractive) effects must not contain arguments that are part of the resources $R$. For example, consider the case in which an action $a_i$ needs the precondition $l(x_1, x_2, x_3)$ and $a_j$ provides the predicate, then if at least one of $x_1, x_2, x_3$ is in $R$, $a_j$ is an enabler of $a_i$, otherwise it is not. This ensures that only causal relationships that do not depend on the resources are extracted at this time. The precedence of the resources will be defined and discussed in Section~\ref{ssec:poplanopt}. 

\begin{algorithm}[htp]
\footnotesize
\caption{Algorithm extracting the actions enablers and the resources}
\label{alg:poplanning}
\KwData{$TP=(F, DA, I, G, K)$}
\KwResult{Enablers and resources $R$}

\DontPrintSemicolon

\SetKwProg{findenablers}{FIND\_ENABLERS}{}{}
\SetKwProg{isenabler}{IS\_ENABLER}{}{}
\SetKwProg{findresources}{EXTRACT\_RESOURCES}{}{}

\SetKwInOut{Input} {In}
\SetKwInOut{Output}{Out}

\findenablers{$(\tn{TO\_P}, a_i)$}{
  \Input{The total-order plan TO\_P, the $i$th action}
  \Output{The enablers $E$ for all the actions in the plan}

  \For{$a_j \in \tn{TO\_P}, a_j\neq a_i$}{
    \uIf{$\tn{IS\_ENABLER}(a_j, a_i)$}{
      $E[a_i].add(a_j)$;
    }
  }

  \If{$a_i\neq \tn{TO\_P}.back()$}{
    $E \gets \tn{FIND\_ENABLERS}(\tn{TO\_P}, a_{i+1})$\;
  }
  \KwRet{E}\;
}

\isenabler{$(a_j, a_i)$}{
  \Input{The action $a_j$ to test if it's enabler of $a_i$}
  \Output{True if $a_j$ is enabler of $a_i$}

  \ForEach{$e \in \eff{a_j}$}{
    \uIf{$\left(e=\tn{add}(l) \wedge l\in\pc{a_i})\right)$ OR
         $~\left(e=\tn{del}(l) \wedge \lnot l\in\pc{a_i}\right)$ OR
         $~\left(\tn{isStart}(a_j) \wedge a_i \in m(a_j)\right)$ OR\\
         $~~\left(\tn{isEnd}(a_j) \wedge a_i \in m(a_j)\right)$}
    {
      $X\gets \tn{set of arguments of }e$; 
        
      \uIf{$\not\exists x \in X | x \in R$}{
        \KwRet{True};
      }
    }
  }
  \KwRet{False};
}

\findresources{$()$}{
  \output{A list of resources}
  findall(X, resources(X), AllResources)\;
  $R$ = make\_set(AllResources)\;
  \KwRet{$R$}\;
}

\end{algorithm}

\subsubsection{Running Example -- Partial-Order Plan}
\label{sssec:PORunEx}

Once we have applied the mappings as before, we have the full TO plan. We want to extract information from this, which will then be exploited to improve the plan for multiple agents. This is done by examining all the actions and checking which are their enablers. For instance, the 10th action, \verb|move_table_to_block_start(a1, b2, 3, 1, 2, 2)|, has as a precondition the following predicate \verb|clear(Block2), Block2=b1|, which is true only when the 9th action has applied its effects. Since \verb|b1| is not part of the resources, the algorithm will state that $a_9$ is an enabler of $a_{10}$. 

If the second move were to move a block to another position on the table, hence independent of the first move, then the algorithm would not set $a_9$ as an enabler of $a_{10}$, as the only reason it may do so is if the same agent is used, but this is known only later.

After this step, we know the enablers for the actions (shown in squared brackets in the list below):

\begin{minted}[fontsize=\footnotesize]{text}
[0] init()[]
[1] move_table_to_table_start(a1, b1, 1, 1, 2, 2), [0]
[2] move_arm_start(a1, 1, 1), [0,1]
[3] move_arm_end(a1, 1, 1), [0,1,2]
[4] grip_start(a1), [0,1,2,3]
[5] grip_end(a1), [0,1,2,3,4]
[6] move_arm_start(a1, 2, 2), [0,1,2,3,4,5]
[7] move_arm_end(a1, 2, 2), [0,1,2,3,4,5,6]
[8] release_start(a1), [0,1,2,3,4,5,6,7]
[9] release_end(a1), [0,1,2,3,4,5,6,7,8]
[10] move_table_to_table_end(a1, b1, 1, 1, 2, 2), [0,1,2,3,4,5,6,7,8,9]
[11] move_table_to_block_start(a1, b2, 3, 1, 2, 2), [0,10]
[12] move_arm_start(a1, 3, 1), [0,11]
[13] move_arm_end(a1, 3, 1), [0,11,12]
[14] grip_start(a1), [0,11,12,13]
[15] grip_end(a1), [0,11,12,13,14]
[16] move_arm_start(a1, 2, 2), [0,11,12,13,14,15]
[17] move_arm_end(a1, 2, 2), [0,11,12,13,14,15,16]
[18] release_start(a1), [0,11,12,13,14,15,16,17]
[19] release_end(a1), [0,11,12,13,14,15,16,17,18]
[20] move_table_to_block_end(a1, b2, 3, 1, 2, 2), [0,10,11,12,13,14,15,16,17,18,19]
[21] end(), [0,1,2,3,4,5,6,7,8,9,10,11,12,13,14,15,16,17,18,19,20]
\end{minted}

From this we could already notice that all the actions will be carried out in sequence. We also see that in this step we add two fictitious actions, \verb|init| and \verb|end|. This simply represents the start and the end of the plan, respectively. \verb|init| is an enabler of all the actions in the plan and \verb|end| has all the other actions as enablers, which means that the plan can be considered finished only when all the actions have been executed.

As for the resources, we first extract all the possible resources by looking at the predicates \verb|resource(X)| in the \kb, as shown in Section~\ref{sssec:runegKMS}. Then we assign the type of resources used to each action by checking action per action which resources they are using. This is useful because it will provide MILP with the basis to correctly allocate the different resources to the actions.

\begin{minted}[fontsize=\footnotesize]{text}
Resources:
[0] agent-2
Resources list:
[0] agent-[agent(a1),agent(a2)]
Resources required by action:
[4] 6-[agent]
[9] 1-[agent]
\end{minted}


\subsection{Partial-Order Plan Optimization}\label{ssec:poplanopt}
The last part of the planning module, shown in~\autoref{fig:arch_LLM_pKB}, is the optimisation module which allows for shrinking the plan by scheduling the task (temporal plan) and allocating the resources. In order to do this, we instantiate a MILP problem, the solution of which must satisfy constraints ensuring that we are not violating precedence relationships and invalidating the obtained planned. 

We start by taking the work from~\cite{cimatti_strong_2015}, in which the authors describe how it is possible to obtain a plan with lower makespan by reordering some tasks. In particular, we adopt the following concepts from~\cite{cimatti_strong_2015}:
\begin{itemize}
    \item Let $f(l)=\{a\in DA \vert l\in \eff{a}\}$ be the set of actions that achieve a literal $l$, and 
    \item let $\displaystyle p(l,a,r)\doteq a<r \wedge \bigwedge_{a_i\in f(l)\setminus\{a,r\}}(a_i<a\vee a_i>r)$ be the temporal constraint stating which is the last achiever $a$ of an action $r$ for a literal $l$. 
\end{itemize}
The constraints that must hold are the following:
\begin{equation}
    \label{eq:constraint1_old}
    %\footnotesize
    \bigvee_{a_j\in f(l)\setminus\{a\}} p(l,a_j,a).
\end{equation}
Which states that at least an action with effect $l$ should occur before $a$.
\begin{equation}
    \label{eq:constraint2}
    %\footnotesize
    \bigwedge_{a_j\in f(l)} \left(p(l,a_j,a) \rightarrow \bigwedge_{a_t\in f(\lnot l)\setminus\{a\}}(a_t<a_j \vee a_t>a)\right).
\end{equation}
\begin{equation}
    \label{eq:constraint3}
    %\footnotesize
    \bigwedge_{a_j\in f(\lnot l)\wedge l\in \pc{a}} ((a_j<\aStart{a}) \vee (a_j>\aEnd{a})).
\end{equation}
Which state that between the last achiever $a_j$ of a literal $l$ for an action $a$ and the action $a$ there must not be an action $a_t$ negating said literal. This condition is also enforced by~\autoref{eq:constraint3} that constrains actions negating the literal to happen before the action $a$ has started or after it has finished.

Notice though that in this work, the authors have considered achievers and not enablers. The difference is that an action $a_j$ is an achiever of $a_i$ if $a_j$ \emph{adds} a fluent $l$ that is needed by $a_j$. Enablers instead consider the case in which fluents are also removed. 
%
Since these constraints only consider achievers and not enablers, we need to extend them. We redefine the previous as:
\begin{itemize}
    \item let $f(l)=\{a\in DA \vert add(l)\in \eff{a}\}$ be the set of actions that achieve a literal $l$, and 
    \item let $f(\lnot l)=\{a\in DA \vert del(l) \in \eff{a}\}$ be the set of actions that delete a literal $l$, and
    \item let $F(l) = f(l)\cup f(\lnot l)$ be the union set of $f(l)$ and $f(\lnot l)$, and
    \item let $\displaystyle p(l,a,r)\doteq a<r \wedge \bigwedge_{a_i\in F(l) \setminus\{a,r\}}(a_i<a\vee a_i>r)$ be the last enabler $a$ of an action $r$ for a literal $l$. 
\end{itemize}
Consequently, we need to:
\begin{itemize}
    \item revise~\autoref{eq:constraint1_old} to include all enablers:
        \begin{equation}
            \label{eq:constraint1}
            %\footnotesize
            \bigvee_{a_j\in F(l)\setminus\{a\}} p(l,a_j,a).
        \end{equation}
    \item add two constraints similar to~\autoref{eq:constraint2} and~\autoref{eq:constraint3} to ensure that a predicate that was removed is not added again before the execution of the action:
    \begin{equation}
        \label{eq:constraint2_1}
        %\footnotesize
        \bigwedge_{a_j\in f(\lnot l)} \left(p(l,a_j,a) \rightarrow \bigwedge_{a_t\in f(l)\setminus\{a\}}(a_t<a_j \vee a_t>a)\right).
    \end{equation}
    \begin{equation}
        \label{eq:constraint3_1}
        %\footnotesize
        \bigwedge_{a_j\in f(l)\wedge (\lnot l)\in \pc{a}} ((a_j<\aStart{a}) \vee (a_j>\aEnd{a})).
    \end{equation}    
\end{itemize}

The second aspect of the MILP problem concerns resource allocation. Indeed, as stated before, there are some predicates that are parameterised on resources, e.g., \texttt{available(A)} states whether an agent \texttt{A} is available or not, but it does not ground the value of \texttt{A}. %
One possibility would be to allocate the resources using Prolog, as done in~\cite{saccon2023prolog}, but this choice is greedy since Prolog grounds information with the first predicate that satisfy \texttt{A}. To reduce the makespan of the plan and improve the quality of the same, we delay the grounding to an optimisation phase, leaving Prolog to capture the relationships between actions.

As a first step, we are also going to assume that all the actions coming from a mapping of a higher-level action and that are not mapped into lower-level actions shall maintain the same parameterised predicates as the higher-level action. So the constraint in~\autoref{eq:constraint6} must hold.
\begin{equation}
    \label{eq:constraint6}
    \bigwedge_{a_j\in m(a_i) \wedge m(a_j)\notin M} \left(\bigwedge_{p(x_i) \in \pc{a_i} \wedge p(x_j) \in \pc{a_j}} x_i=x_j \right).
\end{equation}
Moreover, for these constraints, we will consider only predicates that are part of the set $K$, that is predicates that are not resources $R\cap K=\emptyset$.

The objective now is three-fold: 
\begin{itemize}
    \item identify a cost function,
    \item summarise the previous constraints, and
    \item construct a MILP problem to be solved.
\end{itemize}

In this work, the first point is straightforward: we want to minimise the makespan, i.e., the total duration required to complete all tasks or activities.

For the second point, we are trying to find a way to put the previous constraints,~\cref{eq:constraint1,eq:constraint2,eq:constraint2_1,eq:constraint3,eq:constraint3_1,eq:constraint4,eq:constraint5,eq:constraint6} in a compact formulation or structure. We opted to extract the information regarding the enablers using Prolog and to place it into a $N\times N$ matrix $C$, where $N$ is the number of actions and each cell $C_{ij}$ is $1$ if $a_i$ is an enabler of $a_j$ (without considering resources), 0 otherwise. 

We now need to address the resource allocation aspect, specifically, how to distribute the available resources $R$ among the various actions. When performing this task, there are primarily two factors to consider:
\begin{itemize}
    \item A resource cannot be utilised for multiple actions simultaneously.
    \item If two actions share the same resource, they must occur sequentially, meaning one action enables the other.
\end{itemize}

For the first factor, we need to make sure that, for each resource type $r\in R$, the number of actions using the resource at the same time must not be higher than the number of resources of that type available, as shown in~\autoref{eq:resAllocation}.
\begin{equation}
    \displaystyle\forall t \in\{t_0, t_{\tn{END}}\},\,\vert r\vert \geq\sum_{a_i\in TO} t\in\{\aStart{a_i}, \aEnd{a_i}\} \wedge \left( \exists~l(\pmb{x})\in \pc{a_i}\vert r\in\pmb{x}\right).
    \label{eq:resAllocation}
\end{equation}

The second factor must instead be merged with also the precedence constraints embedded in $C$. In particular, we want to express that actions $a_i, a_j$ are in a casual relationship if $C_{ij}=1$ or if they share the same resource. This can be expressed with the following constraint: 
\begin{equation}
    C_{ij} \vee \exists r\in R : r\in\fl{a_i} \wedge r\in\fl{a_j}
    \label{eq:precedence}
\end{equation}
Note that $\fl{a}$ was defined in the problem definition paragraph and represents the set of variables and literals used by the predicates in the preconditions of $a$. 

Finally, we need to set up the MILP problem that consists in finding an assignment of the parameters, of the actions' duration and of the causal relationships, such that the depth of the graph $\mathcal{G}$ representing the plan is minimised. This problem can be expressed as shown in~\autoref{eq:optimization_1}.

%\begin{figure*}[h]
%    \centering
    \begin{equation}
    \everymath={\displaystyle}
    \begin{array}{r@{\hspace*{8mm}}l}
        \label{eq:optimization_1}
        \min_{\mathcal{P}, \mathcal{T}} & t_{\tn{END}} \\
        %&\\
        \textrm{s.t.}   & C_{ij} \vee \exists r\in R : r\in\fl{a_i} \wedge r\in\fl{a_j}, \\
                              & \quad \quad \forall t \in\{t_0, t_{\tn{END}}\}, \\
                              & \quad \quad \quad \quad \vert r\vert > \!\!\sum_{a_i\in TO} \left(t\in\{\aStart{a_i}, \aEnd{a_i}\} \wedge \exists~l(\pmb{x})\in \pc{a_i}\vert r\in\pmb{x}\right).\\
    \end{array}
    \end{equation}
%\end{figure*}

As mentioned before, the MILP part is implemented in Python3 using OR-Tools from Google. The program also checks the consistency of the PO matrix $C$, by making sure that all the actions must have a path to the final actions. 
The output of the MILP solution is basically an STN, which describes both the causal relationship between the actions and also the intervals around the duration of the actions. The initial and final nodes of the STN are factitious as they do not correspond to actual actions, but they simply represent the start and the end of the plan.
The STN is extracted by considering the causal relationship from the $C$ matrix taken as input, and by adding the causal relationship given by the resource allocation task. 
Once we have the STN, we can extract a \bt, which can then be directly executed by integrating it in ROS2. 

\subsubsection{Plan Optimization -- Example}
\label{sssec:PORunExample}
As we said at the end of~\autoref{sssec:PORunEx}  on the running example, that particular plan is not optimisable as the actions are executed in sequence. Let's then consider a slight modification, which consists in finding a plan to move the two blocks in two new positions instead of stacking them in one position. We also have a new agent that can be used to carry out part of the work. 
Our new plan and actions' enablers are the following one:

\begin{minted}[fontsize=\footnotesize]{text}
[0] init()[]
[1] move_table_to_table_start(a1, b1, 1, 1, 1, 2), [0]
[2] move_arm_start(a1, 1, 1), [0,1]
[3] move_arm_end(a1, 1, 1), [0,1,2]
[4] grip_start(a1), [0,1,2,3]
[5] grip_end(a1), [0,1,2,3,4]
[6] move_arm_start(a1, 1, 2), [0,1,2,3,4,5]
[7] move_arm_end(a1, 1, 2), [0,1,2,3,4,5,6]
[8] release_start(a1), [0,1,2,3,4,5,6,7]
[9] release_end(a1), [0,1,2,3,4,5,6,7,8]
[10] move_table_to_table_end(a1, b1, 1, 1, 1, 2), [0,1,2,3,4,5,6,7,8,9]
[11] move_table_to_table_start(a1, b2, 3, 1, 3, 2), [0,10]
[12] move_arm_start(a1, 3, 1), [0,11]
[13] move_arm_end(a1, 3, 1), [0,11,12]
[14] grip_start(a1), [0,11,12,13]
[15] grip_end(a1), [0,11,12,13,14]
[16] move_arm_start(a1, 3, 2), [0,11,12,13,14,15]
[17] move_arm_end(a1, 3, 2), [0,11,12,13,14,15,16]
[18] release_start(a1), [0,11,12,13,14,15,16,17]
[19] release_end(a1), [0,11,12,13,14,15,16,17,18]
[20] move_table_to_table_end(a1, b2, 3, 1, 3, 2), [0,10,11,12,13,14,15,16,17,18,19]
[21] end(), [0,1,2,3,4,5,6,7,8,9,10,11,12,13,14,15,16,17,18,19,20]
\end{minted}

Indeed, action $a_9$ may or may not be an enabler of action $a_{10}$ depending on the resource allocation of the MILP solution. If we have just one agent, then $a_9\in\ach{a_{10}}$, if instead we have more than one agent, then $a_9\not\in\ach{a_{10}}$ and the two actions can be executed at the same time and the plan would be:

\begin{minted}[fontsize=\footnotesize]{text}
[0] init()
[1] move_table_to_table_start(a1, b1, 1, 1, 1, 2)
[2] move_arm_start(a1, 1, 1)
[3] move_arm_end(a1, 1, 1)
[4] grip_start(a1)
[5] grip_end(a1)
[6] move_arm_start(a1, 1, 2)
[7] move_arm_end(a1, 1, 2)
[8] release_start(a1)
[9] release_end(a1)
[10] move_table_to_table_end(a1, b1, 1, 1, 1, 2)
[11] move_table_to_block_start(a2, b2, 3, 1, 3, 2)
[12] move_arm_start(a2, 3, 1)
[13] move_arm_end(a2, 3, 1)
[14] grip_start(a2)
[15] grip_end(a2)
[16] move_arm_start(a2, 3, 2)
[17] move_arm_end(a2, 3, 2)
[18] release_start(a2)
[19] release_end(a2)
[20] move_table_to_block_end(a2, b2, 3, 1, 3, 2)
[21] end()
\end{minted}

% \enrcom{Should I also include a figure? MR: I do not think so!}

% \subsubsection{Plan Generation - Example}

\section{\Btree Generation and Execution}\label{sec:bt}
\newcommand{\seq}[0]{\protect\writings{\texttt{SEQUENCE}}}
\newcommand{\parr}[0]{\protect\writings{\texttt{PARALLEL}}}

% In this section, we first introduce how to convert from a STN to a \bt, and then we provide some details regarding the implementation. 

% \subsubsection{\bt Generation}\label{sssec:btgen}

The conversion from STN to \bt is taken from~\cite{roveriSTNtoBT}. We summarize it here and refer the reader to the main article. 

An STN is a graph with a source and a sink, which can be artificial nodes in the sense that they represent the start and the end of the plan. Each node can have multiple parent and multiple children. Having multiple parents implies that the node cannot be executed as long as all the parents haven not finished and, whereas, having multiple children implies that they will be executed in parallel. 

With this knowledge we can extract a \btree, which is a structure that, starting from the root, ticks all the nodes in the tree until it finishes the last leaf. Nodes in the tree can be of different types:
\begin{itemize}
    \item \emph{action}: they are an action that has to be executed;
    \item \emph{control}: they can be either \seq or \parr and state how the children nodes must be executed;
    \item \emph{condition}: they check whether a condition is correct or not;
\end{itemize}
The ticking of a node means that the node is asked to do its function, e.g., if a \seq node is ticked, then it will tick the children one at a time, while if a condition node is ticked, it will make sure that the condition is satisfied before continuing with the next tick. 

The algorithm %(Algorithm~\ref{alg:stntobt})
to convert the STN to a \bt starts from the fictitious initial node (\verb|init|), and for every node it checks:
\begin{itemize}
    \item The number of children: if there is only one child, then it is a \seq node, otherwise it is a \parr node. 
    \item The number of parents: if there are more than one parents then the node must wait for all the parents to have ticked, before being executed.
    \item The type of the action: if it is a low-level action, then it is inserted into the \bt for execution, otherwise it will not be included.
\end{itemize}

%%%%%%%%%%%%%%%%%%%%%%%%%%%%%%%%%%%%%%%%%

% \begin{algorithm}
% \caption{Algorithm extracting a \btree from an STN.}
% \label{alg:stntobt}
% \KwData{The STN $G$}
% \KwResult{\bt corresponding to the STN}

% \DontPrintSemicolon

% \SetKwProg{extractBT}{EXTRACT\_BT}{}{}

% \SetKwInOut{Input} {In}
% \SetKwInOut{Output}{Out}

% \extractBT{(G)}{
%   \Input{The STN $G$ to convert}
%   \Output{The \bt $\mathcal{T}$}
%   \KwRet{$\mathcal{T}$}\;
% }

% \end{algorithm}

%%%%%%%%%%%%%%%%%%%%%%%%%%%%%%%%%%%%%%%%%

% THIS has been moved to the Implementation details section of Experimental Validation
% \subsubsection{\bt Execution}\label{sssec:btexec}

% As said, the execution of a \btree starts from the root and it gradually ticks the different nodes of the tree until all nodes have been ticked. 

% While \bts have become a de facto standard for executing robotic tasks, no universally accepted framework exists for their creation or execution. Some notable examples include PlanSys2~\cite{martin2021plansys2} and BehaviorTree.CPP~\cite{BehaviorTreeCppWebsite}. PlanSys2 is tightly integrated with ROS2; beyond merely executing \btrees, it can also derive feasible plans from a knowledge base. In contrast, BehaviorTree.CPP is a more general framework that enables the creation and execution of \bts from an XML file. We selected BehaviorTree.CPP since our main objective was to execute APIs from a \bt, which is easily represented using an XML file, while also maintaining maximum generality. Nevertheless, BehaviorTree.CPP also offers a ROS2 wrapper, which can easily be integrated with the flow.

% \enrcom{Maybe it should be moved to Section~\ref{ssec:implementation}?}


\section{Experimental Setup}
\label{sec:exp_setup}
\subsection{Datasets}
\label{sec:exp_setup:datasets}

We evaluate \method{} on three multimodal humor datasets (see examples in Appendix \ref{app:dataset-eg}):

\paragraph{MemeCap \cite{hwang-shwartz-2023-memecap}.} Each instance includes a meme paired with a title (social media post to which the meme was attached). The task is to generate a brief explanation, compared against multiple reference explanations. The task requires interpreting visual metaphors in relation to the text, where models can benefit from reasoning about background knowledge. 
% , and reference captions in the test images can contain up to four captions. 
% includes 5.8k training and validation samples, along with 559 test samples. 

\paragraph{New Yorker Cartoon \cite{hessel-etal-2023-androids}.} We focus on the explanation generation task: given a New Yorker cartoon and its caption, generate an explanation for why the caption is funny given the cartoon, requiring an understanding of the scene, caption, and commonsense and world knowledge.  
%indirect and playful meanings tied to human experience and culture.
% It includes 2.3k training samples, 130 validation samples, and 130 test samples, and offers five different splits. 

\paragraph{YesBut \cite{nandy-etal-2024-yesbut}.} 
%dataset consists of 1k satirical and 1k non-satirical test samples, with our focus on the satirical test set. 
Each instance contains an image with two parts captioned ``yes'' and ``but''. The task is to explain why the image is funny or satirical.
% requiring an understanding of commonsense knowledge, social norms, and cultural references related to everyday objects and situations.

Since our method is unsupervised, we use the test set portions of these datasets. Due to resource and cost constraints, we don't evaluate our method on the full test sets. Instead, from each dataset, we randomly sample 100 test instances. We repeat the process three times using different random seeds to obtain three test splits and report average performance and standard deviation.

\subsection{Models}
\label{sec:exp_setup_models}

We test our method with two closed-source and two open-source VLMs.
\paragraph{GPT-4o \cite{hurst2024gpt}} is an advanced, closed-source multimodal model processing text, audio, images, and video and generating text, audio, and images. It matches GPT-4's performance in English text tasks with improved vision understanding.
\paragraph{Gemini \cite{team2023gemini}} is a closed-source multimodal model from Google, available in multiple variants optimized for different tasks. 
We use \texttt{Gemini 1.5 Flash} for evaluation and \texttt{Gemini 1.5 Flash-8B} for experiments, a smaller, faster variant with comparable performance.
\paragraph{Qwen2 \cite{yang2024qwen2technicalreport}} is an open-source multimodal model built on a vision transformer with strong visual reasoning. We use the \texttt{Qwen2-VL-7B-Instruct} model, competitive with GPT-4o on several benchmarks.
\paragraph{Phi \cite{phi}} is a lightweight, open-source 4.2B-parameter multimodal model, trained on synthetic and web data. We use \texttt{Phi-3.5-Vision-Instruct}, optimized for precise instruction adherence.

\subsection{Baselines}
\label{sec:exp_setup:baselines}

We compare our method to four prompting-based baselines:\footnote{Temperature set to 0.8 for all baselines.} zero-shot (\base{}), Chain-of-Thought (\chain{}) prompting, and self-refinement with (\critic{}) and without (\nocritic{}) a critic.

\base{} generates a final explanation directly from the image and caption using VLM. \chain{} follows a similar setup but instructs the model to produce intermediate reasoning chains \cite{cot}. 
Additionally, we implement \critic{}, a multimodal variant of self-refinement \cite{madaan2023selfrefine}, where a \textit{generator} produces a response, and a \textit{critic} evaluates it based on predefined criteria. The critic's feedback helps refine the output iteratively\footnote{Refinement steps set to 2 for fair comparison.}. Evaluation criteria include correctness, soundness, completeness, faithfulness, and clarity (details in Appendix \ref{app:base-prompts}).
\nocritic{} functions identically to \critic{} but without a \textit{critic model}, refining candidate explanations without feedback. This also serves as an ablation of the implications from \method{}. Prompts for baselines are in Appendix \ref{app:base-prompts}.


\subsection{Evaluation Metrics}
\label{sec:exp_setup:eval}
While human evaluation is often the most reliable option for open-ended tasks like ours \cite{hwang-shwartz-2023-memecap}, it is costly at scale. LLM-based evaluations (e.g., with \texttt{Gemini 1.5 Flash}) offer a more affordable alternative but are not always reliable \cite{biases_paper}. Prior research in fact verification has found that modern closed-source LLMs excel at fact checking when the complex facts are decomposed into simpler, atomic facts and verified individually \cite{gunjal-durrett-2024-molecular, samir-etal-2024-locating}. Inspired by this approach, we propose LLM-based precision and recall scores.

For recall, we decompose the reference $ref$ into atomic facts: $\{y_1, y_2, ..., y_n\}$ and check whether each appears in the predicted response $pred$.
% The percent of facts present in the predicted response forms the recall score:
\[
\text{Recall} = \frac{1}{n} \sum_{i=1}^{n} \mathbbm{1} \big( LLM(y_i, pred) = \text{Yes} \big)
\]
where $n$ is the number of atomic facts in $ref$.

Precision follows the same process in reverse, decomposing $pred$ into a list of atomic facts: $\{x_1, x_2, ..., x_m\}$ and verifying their presence in $ref$:
\[
\text{Precision} = \frac{1}{m} \sum_{i=1}^{m} \mathbbm{1} \big( LLM(x_i, ref) = \text{Yes} \big)
\]
where $m$ is the number of atomic facts in $pred$. Both decomposition and verification use Gemini-Flash-1.5 with a temperature of 0.2.

In preliminary experiments, we observed that human references tend to omit obvious visual details, whereas model-generated answers are often more complete, referencing visual information. To prevent penalizing the models for these facts, we incorporate literal image descriptions (Sec~\ref{sec:method}) into the reference by decomposing them and adding them to the atomic facts for fairer evaluation. Based on the precision and recall scores, we report the macro-$F_1$ score.

To assess the reliability of our metrics, we conducted a human evaluation on 130 random samples across all models and datasets via CloudResearch (details in Appendix \ref{app:cloudresearch}). Human annotators determined whether each atomic sentence appeared in the corresponding text (e.g., reference). The average agreement between the LLM-based evaluator and two human annotators was 77.1\% (\(\kappa = 54.1\)), similar to the agreement between the two annotators: 75.4\% (\(\kappa = 50.8\)), indicating considerable alignment with human judgment. Prompts are in Appendix \ref{app:eval-prompts}.


\section{Results}
\label{sec:results}
\section{Results}
%Below we report quantitative results from the study sessions and survey. 
Among 54 task instances, participants successfully completed the programming task in 50 instances, passing all test cases. 
In 4 instances, the task was halted as participants did not pass all test cases within 30 minutes.
The mean task completion time was 16 minutes 46 seconds, with no significant differences across system conditions, task orders, or tasks.
To understand the effects of proactivity on human-AI programming collaboration (RQ2), we first report participants' user experience comparison between prompt-based AI tools (e.g. ChatGPT), their perceived effort of use, and the sense of disruption.
We then describe participants' evaluation of the \sys{} probe's key design features, including the timing of proactive interventions, the AI agent presence, and context management.
Analyzing the 1004 human-AI interaction episodes, we illustrate how users interacted with the AI agent under different programming processes, as well as discuss participants' preference to utilize proactive AI in different task contexts and workflows (RQ3).
We also discuss the human-AI interplay between users and different versions of the system, covering their reliance and trust towards AI, and their own sense of control, ownership, and level of code understanding while using the tools.
% From a software engineering perspective, we discuss participants' preference to use proactive AI in different programming task contexts and workflow processes (RQ3).
% Finally, we report on task-based metrics to quantitatively evaluate the three versions of the system, as well as an analysis of the 1004 human-AI interaction episodes to illustrate how users interacted and made use of the AI agent under different programming processes. 

% Potential results we can report:
% Talk about the effect of proactivity on disruption and how our design helped
% Awareness of our AI agent design and effect on collaboration experience
% Which stages of programming were most suitable for proactivity?
% Time and effort with proactivity
% CONTEXT: At which stage, which subtask, what type of AI actions are preferred by the users
% PERCEPTION: Is acceptance or usability connected to user's biases or perceptions of AI tools
% Can lead to design principles of proactive AI in different domains



\subsection{\sys{} Reduces Expression Effort and Alleviates Disruptions}
% Increased productivity, compare to base line
Overall, participants found the increased AI proactivity in the CodeGhost and \sys{} conditions led to higher efficiency (P1, P2, P13, P15, P18). 
% P2 commented \textit{``I couldn't accomplish this [task] in a short time, so that's the reason I use that [AI support].'' }
% Similarly, P15 remarked: 
% \begin{quote}
%    \textit{ ``Now that I have experienced this AI assistant, I think that the arguments about AIs are out there taking programming jobs...has some merit to it... Just for the convenience of programming, I would love to have one of these in my home. (P15)''}
% \end{quote}
Participants commented that prompt-based tools, like Github Copilot or the PromptOnly in the study, required more effort to interact with (P7, P8, P10, P12, P14).
This was due to the proactive systems' ability to provide suggestions preemptively (P7), making the interaction feel more natural (P8).
After experiencing the CodeGhost and \sys{} conditions, P10 felt that \textit{``in the third one [PromptOnly], there was not enough [proactivity]. Like I had to keep on prompting and asking.''}

% This result was supported quantitatively based on time to convey and interpret...
The proactive agent interventions also resulted in less effort for the user to interpret each AI action in both CodeGhost and \sys{} compared to in PromptOnly (Figure \ref{fig:time_convey_interpret}). 
Among 857 recorded episodes where both the user and the AI agent had at least one turn of interaction (i.e. AI responded to the user's query or the user engaged with AI proactive intervention), we observed a significant difference in the amount of time to interpret the AI agent's actions (e.g., chat messages, editor code changes, presence cues) per interaction across three conditions (\textit{F}(2,856)= 41.1, \textit{p} < 0.001) using one-way ANOVA.
Using pairwise T-test with Bonferroni Correction, we found the interpretation time significantly higher in PromptOnly ($\mu$ = 34.5 seconds, $\sigma$ = 30.1) than in CodeGhost ($\mu$ = 19.8 seconds, $\sigma$ = 17.2; \textit{p} < 0.001) and \sys{} ($\mu$ = 18.7 seconds, $\sigma$ = 14.9; \textit{p} < 0.001). 
There was no significant difference in the time to interpret between the CodeGhost and \sys{} conditions (\textit{p} = 0.398; Figure \ref{fig:time_convey_interpret}). 
This indicates that when the system was proactive, participants spent less time interpreting AI's response and incorporating them into their own code, potentially due to the context awareness of the assistance to present just-in-time help.
We did not find a significant difference in the time to express user intent to the AI agent per interaction (e.g. respond to AI intervention via chat message, in-line comment, or breakout chat) (\textit{F}(2,652) = 2.36, \textit{p} = 0.095), despite qualitative feedback that the PromptOnly without proactivity was the most effortful to communicate with.


\begin{figure}[t]

\centering
\includegraphics[width=\columnwidth]{figures/Time_to_convey_and_interpret.pdf}
%https://docs.google.com/drawings/d/1CWPtvVLIvQhpS99MhPdVq3V5_PZS9gdYvgYkQgENFII/edit
\caption{\textbf{The time to express user intentions to the AI and the time to interpret the AI response per interaction.} \textnormal{Users' expression time was not significantly different across conditions \textit{F}(2,652) = 2.36, \textit{p} = 0.095). Users' interpretation time varied (\textit{F}(2,856)= 41.1, \textit{p} < 0.001), and was significantly lower for CodeGhost and \sys{} conditions than in PromptOnly.}}
\label{fig:time_convey_interpret}
% Link to google drawing: https://docs.google.com/drawings/d/1CWPtvVLIvQhpS99MhPdVq3V5_PZS9gdYvgYkQgENFII/edit?usp=sharing
\end{figure}

While proactivity allowed participants to feel more productive and efficient, they also experienced an increased sense of disruption.
% A Chi-squared test (\textit{$\chi^2$} = 13.8, \textit{df} = 2, \textit{p} < 0.01) shows significant difference in the number of disruptions 
This was especially prominent in the CodeGhost condition, when the AI agent did not exhibit its presence and provide context management (P1, P9, P10, P14).
% Overall, introducing proactive AI support led to an increased sense of disruption.
Disruptions occurred in different patterns across the three conditions.
In PromptOnly, the scarce disruptions arose from users accidentally triggering AI responses via the in-line comments (similar to Github Copilot's autocompletion) while documenting code or making manual changes during system feedback, leading to interruptions. 
% P10 specifically disliked using comments as instructions for AI: ``I feel like when I think of comments, I think of just writing helpful little notes for myself. Like I don't see them necessarily as instructions. So I feel like it would have been a little distracting right now.''
In CodeGhost, disruptions were due to users' lack of awareness of the AI's state, leading to unanticipated AI actions while they attempted to manually code or move to another task, making interventions feel abrupt. 
For example, P14 found the lack of visual feedback on which part of the code the AI modified made the collaboration chaotic.
Similarly, P12 felt that the automatic response disrupted their flow of thinking, leading to confusion.
In \sys{}, similar disruptions occurred less frequently with the addition of AI presence and threaded interaction.
% However, the additional agent visual signals and organizations of breakout chat messages could be overwhelming and lead to disruptions.
% However, miscommunications about turn-taking between the user and AI sometimes arose, resulting in both parties acting simultaneously and causing interruptions.
% Add quantitative support

Analyzing the Likert-scale survey data (Fig. \ref{fig:survey}) using the Friedman test, participants perceived different levels of disruptions among three conditions (\textit{$\chi^2$} = 22.1, \textit{df} = 2, \textit{p} < 0.001, Fig.\ref{fig:survey} Q1), with the highest in CodeGhost ($\mu$ = 4.61, $\sigma$ = 1.58), then \sys{} ($\mu$ = 3.78, $\sigma$ = 1.86) and PromptOnly ($\mu$ = 1.56, $\sigma$ = 1.15).
Using Wilcoxon signed-rank test with Bonferroni Correction, we found higher perceived disruption in CodeGhost than PromptOnly (\textit{Z} = 3.44, \textit{p} < 0.01), and in \sys{} than PromptOnly (\textit{Z} = 3.10, \textit{p} < 0.01).
We did not find a statistically significant difference in perceived disruption between CodeGhost and \sys{} (\textit{Z} = -1.51, \textit{p} = 0.131).
The perceived disruptions in \sys{} might be due to the additional visual cues exhibited by the AI agent and the breakout chat, which we further discuss in Section \ref{Results:presence_context}.

% The results presented a multifaceted outcome of using proactive AI assistance in programming.
% On the one hand, the AI reduced users' effort to specify and initiate help-seeking, enhancing productivity and efficiency by leveraging LLM's generative capabilities.
% Meanwhile, AI's increased involvement in user workflows created disruptions, but this was alleviated to an extent by our design of \sys{}, although participants perceived level of disruption varied widely (Fig.\ref{fig:survey} Q1).
% We further discuss designs to adapt the salience of the AI presence in the user interface 
% % and improvements to the timing of service 
% in the Discussion.



\begin{figure*}[h]

\centering
\includegraphics[width=0.85\textwidth]{figures/Codellaborator_Timing_Heuristics.pdf}
%https://docs.google.com/drawings/d/1CWPtvVLIvQhpS99MhPdVq3V5_PZS9gdYvgYkQgENFII/edit
\caption{\textbf{Summary of Heuristics for Proactive Assistance Timing.} Overall, we recorded 398 instances of AI proactivity defined by our timing heuristics (Table \ref{table:proactive-features}), with 212 (53.3\%) instances leading to effective user engagement, 48 (12.1\%) instances of disruptions, and 138 (34.7\%) instances of ignored AI proactivity.}
\label{fig:timing_heuristics}
% Link to google drawing: https://docs.google.com/drawings/d/1UW-dSkpbG0QZd4AoKhKW00GZha-VE8aNPmTWE58lVqA/edit
\end{figure*}


\subsection{Measuring Programming Sub-Task Boundary Is Effective to Time Proactive AI Assistance}
% Talk about the frequency, effectiveness of each design
% Discuss any qualitative feedback on each timing of design
To evaluate the design heuristics for the timing of proactive AI assistance (DG1, Table \ref{table:proactive-features}), we analyzed how each system feature and heuristic was utilized. 
Derived from the interaction data, we summarize the frequency, duration, and outcome of each heuristic design (Fig.\ref{fig:timing_heuristics}). 
Overall, we recorded 398 proactivity instances, with 212 (53.3\%) interactions leading to the user's effective engagement (e.g. adapting AI code changes, respond to the agent's message), 48 disruptions (12.1\%), and 138 interactions (34.7\%) where the user did not engage with the proactive agent (i.e. ignored or did not notice).
The most frequently triggered heuristics were code block completion (107 times), program execution (102 times), and user-written in-line comment (78 times). 
Additionally, the most effective heuristics that led to user engagement are multi-line change (73.1\%), user-written comment (69.2\%) and program execution (66.7\%).
Reflecting on the proactivity features, Design Rationale 2 --- intervening at programmer's task boundary --- was the most effective design principle overall.
The only exception is the heuristic of intervening at code block completion, which resulted in \revise{excessive AI responses. Many were affirmatory messages} to acknowledge the completed code and ask if the user needs further help. This led users to ignore around \revise{50\% of the proactive agent signals (Fig.\ref{fig:timing_heuristics})} to avoid disruption to their workflow.

% Add a couple lines on the user's comments on those
\revise{On the other hand}, the implementation of Design Rationale 3 --- intervening based on the user's implicit signals of adding a code comment or selecting a range of code --- resulted in many false positives that led to workflow disruptions.
Code comments and cursor selections conveyed different utilities for different users, which led to misinterpretation of user intent.
For example, P10 did not perceive comments as instructions for AI: ``\textit{I feel like when I think of comments, I think of just writing helpful little notes for myself. Like I don't see them necessarily as instructions. So I feel like it would have been a little distracting right now.}''
Code selection, similarly, was used by some participants as a habitual behavior to focus their own attention on a part of code. Therefore, the agent's proactivity could be perceived as unexpected and unnecessary.

Design Rationale 1 --- intervening at moments of low mental workload --- \revise{was not effectively operationalized}. 
Participants reported that when they were inactive for an extended period, they were likely thinking through the code design or solving an issue, which represents high mental workload. 
While it is likely that idleness is a signal to assist, participants preferred to initiate the help-seeking after they could not resolve the issue themselves, rather than having the AI agent intervene at a potentially mentally occupied moment.
The design rationale requires more involved modeling of the programmer's mental state to render it effective.
We outline the design implications from these finding in the discussion.



\begin{figure*}[t]

\centering
\includegraphics[width=\textwidth]{figures/Codellaborator_User_interaction_journey.pdf}

\caption{\textbf{Human-AI interaction timelines for P1.} \textnormal{For each task, we visualized each interaction initiated by the user and the AI, along with the time spent expressing the user's intent and interpreting the AI agent's response. We also visualized the annotated programming stages over the task. 
The Misc stage colored in black represents when the user was not actively engaged in the task (e.g. performing think-out-loud).
In PromptOnly, we observed the traditional command-response interaction paradigm where the user initiated most interactions. However, P1 unexpectedly triggered the AI agent when documenting code with comments, causing disruptions. In the CodeGhost condition, AI initiated most interactions, but this caused 6 disruptions, mainly during the Organize stage when P1 was making low-level edits and did not expect AI intervention. 
In the \sys{} condition, AI remained proactive but caused fewer disruptions, as P1 engaged in more back-and-forth interactions with higher awareness of the AI's actions and processes. See supplementary material for all timelines.}}
\label{fig:timeline}
% link to google drawing: https://docs.google.com/drawings/d/1Gz6nBWUN-ja1zI0tPYowdSLWafotLV5jUtYCOJjdFmw/edit?usp=sharing
\end{figure*}


\subsection{Users Adapted to AI Proactivity and Established New Collaboration Patterns}
% Describe how the users develop trust to the system, and some users become reliant or over-reliant on the AI
% Throughout the user study session, participants demonstrated calibration of their mental model of the AI agent's capabilities in the editor.
% Users naturally developed more trust and reliance as they used the AI to aid with their tasks, especially after they were exposed to proactive assistance.
Throughout the user study, participants calibrated their mental model of the AI agent's capabilities in the editor, developing a level of trust and reliance after experiencing proactive assistance.
% P15 used an analogy to a leader in a software engineering team, and that as the team establishes a \textit{``good track record of performing well, you're just naturally going to trust it.''}
Half of the participants ($N$ = 9) exhibited a level of reliance on the AI's generative power to tackle the coding task at hand and resorted to an observer and code reviewer role.
% P7 described their mentality shift as such: 
% \textit{``As programmers you're never really going to do extra work if you don't have to...You might as well take a little bit of a backseat on it and kind of only start working on it yourself once it's like complex logic that you need to understand yourself.''}
With this role change, participants shifted their mental process to focus more on high-level task design and away from syntax-level code-writing.
P3 reflected on their shift: \textit{``I kind of shifted more from `I want to try and solve the problem' to what are the keywords to use to get this [AI agent] to solve the problem for me... I could also feel myself paying less attention to what exactly was being written...So I think my shift focus from less like problem-solving and more so like prompts.''}
% P15 had a similar view of the proactive system: 
% \begin{quote}
%    \textit{ ``It [the AI] was the driver and I was the tool. Basically, I was the post-mortem tool, right. I was checking whether his code is correct, right. But it was the driver writing it. So in that case, I was not disrupted at all.. because the paradigm of the workflow has shifted. I am not in a position to be disrupted anymore, right. It was doing the heavy lifting. I was just doing the code review. (P15)''}
% \end{quote}
P10 expressed optimism toward developers' transition from code-writing to more high-level engineering and designing tasks:
\begin{quote}
    \textit{``I think with the increase of... low code, or even no code sort of systems, I feel like the coding part is becoming less and less important. And so I really do see this as a good thing that can really empower software engineers to do more. Like this sort of more wrote software engineering, more wrote code writing is just... it's not needed anymore.''}
\end{quote}

Under this trend of allowing the AI to drive the programming tasks, four participants (P3, P6, P7, P15) commented that they were still able to maintain overall control of the programming collaboration and steer the AI toward their goal.
P7 described their control as they adjusted to the level of AI assistance and navigated division of labor: \textit{``It's great to the point where you have the autonomy and agency to tell it if you want it to implement it for you, or if you want suggestions or something like you can tell [the AI] with the way it's written. It's always kind of like asking you, do you want me to do this for you? And I think that's like, perfect.''}
These findings highlight the potential for users to adopt proactive AI support in their programming workflows, fostering productive and balanced collaborations, provided the systems show clear signals of its capabilities for users to align their understanding to.






\subsection{Users Desired Varying Proactivity at Different Programming Processes}
\label{Results:programming_process}
% Some users expressed ambivalence to using proactive AI, good for efficiency, but bad for code understanding
Through analyzing the 1004 human-AI interaction episodes, we found that participants engaged with the AI the most (38.2\%) during the implement stage, 
followed by debug (26.4\%), 
analyze (i.e., examine existing code or querying technical questions like how to use an API; 11.5\%), design (i.e., planning the implementation; 10.9\%), organize (i.e., formatting, re-arranging code; 6.67\%), refactor (5.48\%), and miscellaneous interactions (e.g., user thanks the AI agent for its help; 0.697\%). 
We visualize P1's user interaction timeline as an example to illustrate different interaction types and frequencies under different programming stages (Fig.\ref{fig:timeline}).
% To perform this analysis, we referenced CUPS, an existing process taxonomy on AI programming usage \cite{Mozannar2022ReadingBT}, and adapted it to our research questions and applicable stages observed in our tasks.
% We acknowledge that the listed programming processes do not comprehensively represent different tasks and software engineering contexts.
% Rather, we cross-reference our analysis with qualitative feedback to identify user experiences in different stages of programming at a high level.
% With this broad categorization, we probed participants during the post-interview and identified processes of programming where proactive AI assistance was desired, and where it was disruptive and unhelpful.
To conduct this analysis, we adapted CUPS, an existing process taxonomy on AI programming usage \cite{Mozannar2022ReadingBT}, to align with our research questions and the stages observed in our tasks. 
By cross-referencing the interaction analysis with qualitative feedback, \revise{we identified programming stages where proactive AI assistance was most desired or disruptive.}
% This broad categorization allowed us to identify where proactive AI assistance was most desired and where it could be disruptive or unhelpful, providing valuable insights into optimizing AI support in programming.

In general, participants preferred to engage with the AI during well-defined boundaries between high-level processes, like providing scaffolds to the initial design or executing the code, and repetitive processes, such as refactoring.
They additionally desired AI intervention when they were stuck, for example during debugging.
In contrast, for more low-level tasks that require high mental focus, like implementing \revise{planned} functionality, participants were more often disrupted by proactive AI support and would prefer to take control and initiate interactions themselves.

This was corroborated by our interaction analysis results.
When examining the number of disruptions, we found that most disruptions occurred during the implementation process (32.7\%, 18 disruptions).
% and the implement (25.7\%, 9 disruptions). A disproportional amount of organize stage disruptions took place when participants were not intentionally moving forward with their processes or trying to advance the task. 
% Instead, participants were conducting low-level code organization (e.g., moving blocks around or adding empty lines between code blocks) to improve readability or satisfy personal preferences. 
% The AI agent's intervention during this stage was considered unnecessary. P3, who often documented and re-arranged their code, lamented about proactivity in P condition intruding their organization process: ``\textit{I would like type a comment, and then like, it would appear...giving me like three different text messages}.''
In contrast, \revise{very few} disruptions occurred during the debugging (7.27\%, 4 disruptions) and refactor phases (1.82\%, 1 disruption), which comprised 26.3\% and 5.48\% of all the interactions, respectively.
Most participants expressed the need to seek help from the AI agent in these stages and anticipated AI intervention as there were clear indications of turn-taking (i.e., program execution) and information to act on (i.e., program output, code to be refactored). 
After experiencing proactive assistance, P9 felt that ``\textit{[PromptOnly] wasn't responsive enough in the sense that when I ran the tests, I was kind of looking for immediate feedback regarding what's wrong with my tests and how I can fix it.}''
This corresponds with our proactivity design guideline to initiate intervention during subtask boundaries (Table \ref{table:proactive-features}, Design Rationale 2). 
In a sense, participants desired meaningful actions to be taken before AI intervened.
As P13 described, \textit{``If I'm like paste [code], something big, I run the program, the proactivity in that way, it's good. But if it's proactive because I'm idle or proactive because of a tiny action or like a fidget, then I don't really like that [AI] initiation.''}
% In different context, they would want different levels of proactivity, the action should match the level of severity
% Despite general agreement on preferred and less preferred programming processes to engage with proactive AI, participants did not reach a consensus and often expressed conflicting views on specific processes.
% For example, while P9, P16, and P17 desired proactive feedback after executing their programs and receiving errors, P14 and P18 were against using proactive AI for debugging as it might recurse into more errors, making the program harder to debug.
% Thus, in addition to adhering to general trends, future systems should also aim to be adaptive to the user's preferences, exhibiting different levels of proactive AI assistance according to best fit the user's personal needs and use cases.
% We propose a detailed design suggestion in Section \ref{Discussion:design_implication}.
\revise{Participants generally expressed preferences for programming processes where they wanted proactive support, but their opinions varied regarding which specific processes required more or less proactive assistance.}
For instance, while P9, P16, and P17 welcomed proactive feedback after program errors, P14 and P18 opposed it, fearing it could lead to more errors and complicate debugging. 
Therefore, future systems should adapt to individual user preferences, offering varying levels of proactive AI assistance based on personal needs and use cases. A detailed design suggestion is provided in Section \ref{Discussion:design_implication}.



\subsection{\sys{} and CodeGhost Feel \revise{More} Like Programming with a Partner than a Tool}
% Another effect of presence and context is a different sense of collaboration
% Another effect we observed from participants using the proactive conditions was an elevated sense of collaboration rather than using the programming assistant as a tool.
% While in all three conditions, the AI agent was initialized with the same prompt that enforces pair programming practices (Appendix \ref{appendix:prompt}), some participants expressed that working with the \sys{} and CodeGhost conditions felt more like collaborating with a more human-like agent with presence ($N$ = 6) than the PromptOnly.
We observed that participants in the proactive conditions perceived an elevated sense of collaboration with the AI, rather than viewing it as just a tool. 
Despite all three conditions using the same pair programming prompt (Appendix \ref{appendix:prompt}), six participants noted that \sys{} and CodeGhost felt more like collaborating with a human-like agent compared to PromptOnly.
% P1 commented that \textit{``it's like a person that's on your side [and says] `that's over here. You add that here' and kind of felt that way.''}
P6 reflected that \textit{``just the fact that it was talking with me and checking in with a code editor. I maybe treated it more like an actual human.''}
A part of this is due to the \revise{local scope of interaction with the agent in} the code editor (DG3), as P14 reflected \textit{``by changing the code that I'm working on instead of like on the side window...it feels more like physically interacting with my task.''}
Even the disruptions arising from the proactive AI actions facilitated a human-like interaction experience.
P9 recalled an interaction where they encountered a conflict in turn-taking with the AI: \textit{``[AI agent] was like, `Do you want to read the import statement? Or should I?' I was like, `No, I'll write it' and it [AI agent] said `Great I'll do it' and it just did it. Okay, yeah. True to the human experience.''}

This different sense of collaboration was \revise{reflected in} the survey results ((\textit{$\chi^2$} = 22.1, \textit{df} = 2, \textit{p} < 0.001, Fig.\ref{fig:survey} Q8).
Participants rated the AI assistant in the PromptOnly to be much like a tool ($\mu$ = 5.67, $\sigma$ = 1.58), while both the \sys{} ($\mu$ = 3.61, $\sigma$ = 1.65) and the CodeGhost conditions ($\mu$ = 4.17, $\sigma$ = 1.72) felt more like a programming partner (both \textit{p} < 0.001 compared to PromptOnly).
This more humanistic collaboration experience introduced by proactive AI systems naturally brings questions to its implications for programmers' workflow. We further share our analysis across programming processes in Section \ref{Results:programming_process} and the corresponding design suggestions in the Discussion.



\begin{figure*}[]

\centering
\includegraphics[width=0.8\textwidth]{figures/Codellaborator_survey.pdf}
% new link to google drawing:
% https://docs.google.com/drawings/d/1oiOvDttY3Xibk0P3hm0y1BJWbRh3cKMkh6or_Gu4XB4/edit?usp=sharing
% link to google drawing: https://docs.google.com/drawings/d/1rELkSVN1rroPm8ccWyVykW72SDOhkBKn3MjoaZZq4ec/edit?usp=sharing
\caption{\textbf{Likert-scale Response displayed in box and whisker plots comparing three conditions}. \textnormal{Anchors are 1 - Strongly disagree and 7 - Strongly agree. The green dotted lines represent the mean values for each question. Using the Friedman test, we identified significant differences in rating in Q1 for disruption, Q5 for awareness, and Q8 for partner versus tool use experience.}}
\label{fig:survey}
\end{figure*}



\subsection{Presence and \revise{Local Scope of Interaction} Increase User Awareness on AI Action and Process}
\label{Results:presence_context}
% To specifically evaluate our design of the \sys{} technology probe, we collected qualitative feedback on the AI presence and context management features, and their effects on the user experience compared to other conditions.
% Eight participants expressed that the AI agent's presence in the editor increased their awareness of the AI's actions, intentions, and processes.
We gathered qualitative feedback on the \sys{} technology probe's AI presence and \revise{breakout} features to assess their impact on user experience. 
Eight participants noted that the AI's presence in the editor enhanced their awareness of its actions, intentions, and processes (DG2).
Visualizing the AI's edit traces in the editor using a caret and cursor helped guide the users (P1, P4, P7, P12, P18) and allowed them to understand the system's focus and thinking (P12, P13, P18).
As P13 commented \textit{``I... like the cursor implementation of like, be able to see what it's highlighting, be able to move that cursor all the way just to see like, what part of the file it's focusing on.''}
The presence features also helped users identify the provenance of code and clarified the human-AI turn-taking.
As P10 remarked, \textit{``it was really clear when the AI was taking the turn with writing out the text and like the cursor versus when I was writing it.''}
On the other hand, the different scopes of interaction further increased users' awareness by reducing their cognitive load and enhancing the granularity of control (DG3).
For example, compared to a standard chat interface where \textit{``everything is just one very long line of like, long stream of chat''}, P6 preferred the threaded breakout conversations that decomposed and organized past exchanges.
P4 also found that the breakout \textit{``could be sort of like a plus towards steerability because you can really highlight what you want it to do.''}

% In survey, more awareness...
% Analyzing the survey response, we found that participants generally found the system to be highly aware of the user's actions, with no significant difference across conditions (\textit{$\chi^2$} = 5.83, \textit{df} = 2, \textit{p} = 0.054, Fig.\ref{fig:survey} Q6).
% Conversely, participants rated their own awareness of the AI differently (\textit{$\chi^2$} = 12.7, \textit{df} = 2, \textit{p} < 0.001, Fig.\ref{fig:survey} Q5), with the highest in PromptOnly ($\mu$ = 6.56, $\sigma$ = 0.511), then \sys{} ($\mu$ = 5.44, $\sigma$ = 1.79), then CodeGhost ($\mu$ = 4.17, $\sigma$ = 1.86).
% Specifically, the users felt like they were less aware of the CodeGhost condition prototype compared to the non-proactive PromptOnly (\textit{Z} = 2.5, \textit{p} < 0.001).
% We did not identify any other significant pairwise comparisons after Bonferroni Correction.
% This suggests that proactivity alone in CodeGhost induced more workflow interruptions, which in turn lowered users' perceived awareness of the AI system's action and process. 
% But similar to alleviating workflow disruptions, the \sys{} condition with visual presence and context management also improved user awareness.
Analyzing the survey responses, we found that participants generally rated the system as highly aware of their actions, with no significant difference across conditions (\textit{$\chi^2$} = 5.83, \textit{df} = 2, \textit{p} = 0.054, Fig.\ref{fig:survey} Q6). 
However, participants' \revise{own} awareness of the AI \revise{agent's actions} varied significantly (\textit{$\chi^2$} = 12.7, \textit{df} = 2, \textit{p} < 0.001, Fig.\ref{fig:survey} Q5), with the highest ratings in PromptOnly ($\mu$ = 6.56, $\sigma$ = 0.511), followed by \sys{} ($\mu$ = 5.44, $\sigma$ = 1.79), and the lowest in CodeGhost ($\mu$ = 4.17, $\sigma$ = 1.86). 
Specifically, participants felt less aware of the AI in CodeGhost compared to the non-proactive PromptOnly condition (\textit{Z} = 2.5, \textit{p} < 0.001). 
No other significant pairwise differences were found after applying Bonferroni correction. 
\revise{This can be attributed to CodeGhost's increased proactivity, which, in the absence of sufficient presence signals and a manageable local interaction scope, resulted in more frequent workflow interruptions. These interruptions, in turn, diminished users' ability to remain aware of the AI's actions and interpret its signals effectively, ultimately reducing their sense of control and understanding during the interaction.}

% These findings suggest that CodeGhost's proactivity led to more workflow interruptions and reduced users' perceived awareness of the AI's actions, presenting drawbacks to proactive assistance.

% However, not everyone was fond of presence and context, some find it distracting, and some don't want to go back to old conversations
% While the \sys{} condition with visual presence and context management improved user awareness,  not every participant found the agent design helpful. 
% Four participants thought that the AI presence could be distracting (P8, P10, P17), and two participants did not prefer the integration of AI in the editor, as they took up screen real estate (P6, P9).
% Similarly, P16 pointed out that their personal workflow would not involve using breakout chats for managing past interactions: \textit{``I also don't think I would have that many discussions with the AI once the coding is done and I have this working, then I'm probably not gonna look back at the discussions I've taken.''}
% The mixed findings on the system design indicate that different users, given different programming styles, workflows, preferences, and task contexts, desire different types of systems. We detail our design implications and suggestions in Section \ref{Discussion:design_implication}.
While the \sys{} condition showed improvements on user awareness, not all participants found the agent design helpful. 
Four participants felt the AI presence was distracting (P8, P10, P17), and two thought it occupied too much screen space (P6, P9). 
P16 noted that their workflow wouldn't involve using breakout chats to manage \revise{interaction context}: \textit{“Once the coding is done and I have this working, then I'm probably not gonna look back at the discussions I've taken.”} 
These mixed responses suggest that users with different programming styles, workflows, and preferences require varied system designs. Design implications are discussed in Section \ref{Discussion:design_implication}.




\subsection{Over-Reliance on Proactive Assistance Led to Loss of Control, Ownership, and Code Understanding}
% Discuss how some users are against overly trusting and relying on AI
% These users express concerns on their loss of control
Despite optimism in adopting proactive AI support in many participants, some participants (P6, P10, P11, P13, P16, P18) voiced concerns about over-relying on AI help, \revise{citing} a loss of control.
P10 felt like they were \textit{``fighting against the AI''} in terms of planning for the coding task, as the agent proactively makes coding changes during the implementation phase.
% P11 described that the AI agent \textit{``didn't let me implement it the way I wanted to implement it, it just kind of implemented it the way it felt fit.''}
They further expanded on the potential limitations of LLM code generation, particularly with regards to devising innovative solutions: \textit{``If it was too proactive with that, it would almost force you into a box of whatever data it's already been trained on, right?... It would probably give you whatever is the most common choice, as opposed to what's best for your specific project (P11).''}

% The effects also affect ownership of the code
The capability to understand rich task context and quickly generate solutions also lowered users' sense of ownership of the completed code.
P7 concluded that \textit{``the more proactivity there was, the less ownership I felt...it feels like the AI is kind of ahead of you in terms of its understanding.''}
% Code understanding, maintainability
This lack of code understanding was referenced by multiple participants (P11, P16, P18), raising issues on the maintainability of the code (P11, P14, P15) and security risks (P18).
As P11 suggested: \textit{``It's... not facilitating code understanding or your knowledge transfer. And yes, it's not very easily understood by others, if they just take a look at it.''}
Additionally, some participants believed that programmers should still invest time and effort to cultivate a deep understanding of the codebase, even if AI took the initiative to write the code.
P4 commented \textit{``I think the more that you leave it up to the AI, the more that you sort of have to take it upon yourself to understand what it's doing, assuming that you're being you know, responsible as [a] programmer.''}
Upon noticing that the AI was overtaking the control, P14 adjusted the way they utilized the proactive assistance and found a more balanced paradigm: \textit{``It's more like a conversation, like I gave him [AI] something so it did something, and then step by step I give another instruction and then you [AI] did something. I was being more involved, which allows me to like step by step understand what the AI is doing also to oversee, I was able to check it.''}

However, not all participants shared this concern. P10 expressed a different opinion as they felt like they are not \textit{``emotionally attached''} to their code, and that in the industry setting, code has been written and modified by many stakeholders anyways, so that \textit{``me typing it versus me asking the AI to type it, it's just not that much of a difference.''}
% This prompts an adaptive and balanced design that emphasizes user's control and reliance...
% This part of our findings uncovered different trade-offs between convenience and productivity from the utilization of more proactive and autonomous AI tools, and the potential loss of control during the programming process and less ownership of the end result.
\revise{This finding highlights the trade-offs between convenient productivity and potential risks in user control and code quality.}
While the system can be used to increase efficiency and free programmers from low-level tasks like learning syntax, documentation, and debugging minor issues, it remains a challenge to design balanced human-AI interaction, where the users' influences are not diminished and developers can work with AI, not driven by AI, to tackle new engineering problems.
We condense our findings into design implications in Section \ref{Discussion:design_implication}.


% \subsection{Old Disruption Subsection}

% % \subsection{Social Transparency Cues Reduced the Number of Disruptions Caused by Proactivity}
% Analyzing the Likert-scale survey data using the Friedman test, participants perceived different levels of disruptions among three conditions (\textit{$\chi^2$} = 17.4, \textit{df} = 2, \textit{p} < 0.001, Fig.\ref{fig:survey} Q1). 
% Using Wilcoxon signed-rank test with Bonferroni Correction, we found higher perceived disruption in the P condition ($\mu$ = 4.75, $\sigma$ = 1.42) than in the B condition ($\mu$ = 1.50, $\sigma$ = 0.90, \textit{Z} = 3.1, \textit{p} < 0.01), as well as a higher level of perceived disruption in condition S than B ($\mu$ = 3.75, $\sigma$ = 1.91, \textit{Z} = 2.9, \textit{p} < 0.01).
% We did not find a significant difference in the perceived level of disruption between condition P and condition S ($\mu$ = 3.75, $\sigma$ = 1.9, \textit{Z} = -1.7, \textit{p} = 0.085).
% The higher level of perceived disruptiveness compared to the baseline might be due to the added AI agent's visual signals burdening users' cognitive load. P6 commented that the presence of the AI cursor and thought bubble ``\textit{take up some, I guess real estate of like the editor itself.}'' Similarly, P7 found that the AI agent is S condition is ``\textit{very proactive and very like present with what you're doing. I think that can be potentially a tad bit overwhelming at times because especially with like the highlighting if you can see like what it's doing and it's doing everything.}''
% This indicates that while social transparency cues allow for more visibility of the AI agent's process, it can be overwhelming for the user to perceive all the visual feedback.

% Comparing all three system conditions using a one-way ANOVA test, we observe a significant difference in the mean number of disruptions participants experience per task (\textit{F(2,33)}= 5.86, \textit{p} < 0.01).
% We then conduct post-hoc analysis using Tukey's Honestly Significant Difference (HSD) for multi-group pairwise comparisons with control in overall familywise error rate.
% The P condition ($\mu$ = 2.08, $\sigma$ = 2.11) resulted in significantly more disruptions than the B condition ($\mu$=0.33, $\sigma$ = 0.89; \textit{p} < 0.01). 
% % Disruptions during the B condition were mainly due to the user unintentionally triggering an AI response when documenting code (i.e., activating a comment action), or the user attempting to manually make code changes while waiting for system feedback but then being disrupted by system feedback. 
% % Disruptions during the P condition stemmed from the lack of awareness the user had about the AI agent's state. Participants perceived no AI actions and desired to take control (i.e., manually coding or examining the file), potentially advancing to the next sub-task with a different context. However, the AI agent reacted to user actions without visibility about the interaction process and existing context, causing the interventions to be jarring and hard to interpret.
% In contrast, participants experienced significantly fewer disruptions during the S condition ($\mu$ = 0.58, $\sigma$ = 0.51) than the P condition (\textit{p} < 0.05). There was not a significant difference between the S and B conditions (\textit{p} = 0.89). 
% This result suggests a mediating effect of social transparency cues for alleviating the disruptions brought upon by a proactive AI agent.

% Disruptions occurred in different patterns across the three conditions.
% In condition B, disruptions were mainly due to the user unintentionally triggering an AI response when documenting code (i.e., activating a comment action), or the user attempting to manually make code changes while waiting for system feedback but then being disrupted by system feedback. 
% Disruptions during the P condition stemmed from the lack of awareness the user had about the AI agent's state. Participants perceived no AI actions and desired to take control (i.e., manually coding or examining the file), potentially advancing to the next sub-task with a different context. However, the AI agent reacted to user actions without visibility about the interaction process and existing context, causing the interventions to be jarring.
% Disruptions during the S condition are similar to those of the P condition, where users expected to take control but were interrupted, but they occurred less frequently. Another type of disruption with the S condition occurs when the user perceives and communicates with the AI about the turn-taking, but the coordination is ambiguous, so both the human and the AI agent act, causing disruptions.
% % Disruptions during the B condition occurred more often when neither the participant nor the AI agent was active, but both attempted to initiate interaction with poor coordination. Due to a period of participant inactivity, the AI agent could not accurately identify the participant's needs and current thought process. Thus, the timing of system intervention appeared more unforeseen.



% % \subsection{Proactivity and Social Transparency's Effects on Disruptions}
% % Comparing all three system conditions using a one-way ANOVA test, we observe a significant difference in the mean number of disruptions participants experience per task (\textit{F}(2,33)= 5.86, \textit{p} < 0.01).
% % We then conduct pairwise comparisons using T-test and Bonferroni Correction, judging statistical significance at \textit{p} < 0.0167 [0.05 / 3]. With this metric, we found that the P condition ($\mu$ = 2.08, $\sigma$ = 2.11) resulted in significantly more disruptions than the B condition ($\mu$=0.33, $\sigma$ = 0.89; \textit{p} < 0.0167). 
% % Participants experienced fewer disruptions during the S condition ($\mu$ = 0.58, $\sigma$ = 0.51), but we did not identify statistical significance when comparing to the P condition (\textit{p} = 0.0256). 
% % There was also not a significant difference between the S and B conditions (\textit{p} = 0.408). 

% % Disruptions during the B condition were mainly due to the user unintentionally triggering an AI response when documenting code (i.e., activating a comment action), or the user attempting to manually make code changes while waiting for system feedback but then being disrupted by system feedback. 
% % Disruptions during the P condition stemmed from the lack of awareness the user had about the AI agent's state. Participants perceived no AI actions and desired to take control (i.e., manually coding or examining the file), potentially advancing to the next sub-task with a different context. However, the AI agent reacted to user actions without visibility about the interaction process and existing context, causing the interventions to be jarring and hard to interpret.
% % Disruptions during the S condition are similar to those of the P condition, where users expected to take control but were interrupted, although less frequent. Another type of disruption with the S condition occurs when the user perceives and communicates with the AI about the turn-taking, but the coordination is ambiguous, so both the human and the AI agent act, causing disruptions.
% % nor the AI agent was active, but both attempted to initiate interaction with poor coordination. Due to a period of participant inactivity, the AI agent could not accurately identify the participant's needs and current thought process. Thus, the timing of system intervention appeared more unforeseen.

% % Analyzing the Likert-scale survey using the Friedman test, participants perceived different levels of disruptions among three conditions (\textit{$\chi^2$} = 17.4, \textit{df} = 2, \textit{p} < 0.001, Fig.\ref{fig:survey} Q1). 
% % Using Wilcoxon signed-rank test with Bonferroni Correction, we found higher perceived disruption in the P condition ($\mu$ = 4.75, $\sigma$ = 1.42) than in the B condition ($\mu$ = 1.50, $\sigma$ = 0.90, \textit{Z} = 3.1, \textit{p} < 0.01), and higher perceived disruption in condition S than B ($\mu$ = 3.75, $\sigma$ = 1.91, \textit{Z} = 2.9, \textit{p} < 0.01).
% % However, we did not find a significant difference in the perceived level of disruption between condition P and condition S (\textit{Z} = -1.7, \textit{p} = 0.085).

% % Despite a lower mean number of disruptions and lower mean rating for perceived disruption in condition S than P, we did not identify an effect of social transparency cues reducing disruptions significantly. 
% % This might be due to the visual signals indicating AI agent status in S condition adds to users' cognitive load. P6 commented that the presence of AI cursor and chat bubble ``\textit{take up some, I guess real estate of like the editor itself.}'' Similarly, P7 found that the AI agent is S condition is ``\textit{very proactive and very like present with what you're doing. I think that can be potentially a tad bit overwhelming at times because especially with like the highlighting if you can see like what it's doing and it's doing everything.}''
% % This indicates that while social transparency cues allow for more visibility of the AI agent's process, it can be overbearing for the user to perceive all the visual feedback.



% \subsection{Time and Effort to Convey to, and Comprehend, the AI}
% % \yc{is there a statement to make in the title? should have one takeaway msg per result as a paragraph-header}
% To examine the effects of proactivity on the facilitation of productivity and better human-AI collaboration, we analyzed the overall completion time for each condition. We also draw from interaction-level data to inspect the effort users needed to convey their intentions to the AI agent and to comprehend the system response.

% All tasks were completed within the 30-minute time allotment except for three. During these three instances, the participants did not come close to fulfilling the specifications of the tasks and were halted without passing all test cases. We observe one failed instance of each coding task, with two occurring during the P condition and one during the S condition. 
% The interaction episodes in the failed instances are still included in the data analysis, as they were natural interactions from first-time users.

% A one-way ANOVA did not reveal a significant difference in overall completion time for each condition (\textit{F}(2,33)= 0.425, \textit{p} = 0.657). This demonstrates that despite feeling more disrupted in proactive AI systems, the interventions do not slow down participants from completing the tasks.
% % , consistent with prior studies \cite{vaithilingam2022expectation}.
% % \yc{this is consistent with prior studies}.
% % or each task (\textit{F(2,33)}= 1.77, \textit{p} = 0.19). 
% % Thus, there is no evidence that the difference in task difficulties was a confounding factor. 
% % There was also no significant effect of ordering (\textit{F(X,X)} = 1.16, \textit{p} = 0.32). However, the range of task completion time is drastic, with the fastest completion being only 169 seconds (task order 1, condition P, budget tracker), which was lower than the time allotted and deviated from the overall mean completion time of 991 seconds. This could have been due to the inconsistent code generation quality from the LLM despite our effort to configure it with the least randomness.
% Despite no significant differences in task completion time, the proactive agent interventions in the P and S conditions resulted in less effort for the user to interpret each AI action than during the B condition (Figure \ref{fig:time_convey_interpret}). 
% We observe a significant difference in the amount of time to interpret the AI agent's actions (e.g., chat messages, editor code changes, presence cues) per interaction across three conditions (\textit{F}(2,637)= 46.7, \textit{p} < 0.001) from ANOVA.
% Using Tukey's HSD, we found the time to interpret per interaction was significantly higher in condition B ($\mu$ = 36.4 seconds, $\sigma$ = 30.8) than in condition P ($\mu$ = 17.7 seconds, $\sigma$ = 16.1; \textit{p} < 0.001) and S ($\mu$ = 17.0 seconds, $\sigma$ = 15.9; \textit{p} < 0.001). There was no significant difference in the time to interpret between the P and S conditions (\textit{p} = 0.90; Figure \ref{fig:time_convey_interpret}). 


% There was also not a significant effect of condition on the time to convey user intention to the AI agent (\textit{F}(2,432) = 0.744, \textit{p} = 0.476). The degree of freedom is lower as only user-initiated interactions required users to convey their intentions. However, five participants expressed that out of the three conditions, they felt that they spent the most effort communicating with the AI agent in condition B since they had to manually draft a message with a specific context (P1, P3, P6, P7, P12).

% Overall, these findings indicate that proactivity in AI programming systems might not result in a significant increase in productivity or efficiency, however, AI-initiated contextualized assistance could potentially contribute to more explainable and interpretable AI, with the decreased effort to understand AI responses and no extra cost to convey user intentions, at the scope of each interaction.

% \subsection{Proactivity Affects Awareness and Collaboration Experience}
% Further analyzing the survey results using Friedman test, we found that across three conditions,
% participants rated different levels on their sense of awareness of the AI agent (\textit{$\chi^2$} = 8.06, \textit{df} = 2, \textit{p} < 0.05, Fig.\ref{fig:survey} Q5). 
% Using Wilcoxon signed-rank tests with Bonferroni Correction, we found that compared to the baseline condition ($\mu$ = 6.33, $\sigma$ = 0.49), participants felt the AI agent was less aware of their actions in P condition ($\mu$ = 4.33, $\sigma$ = 2.06, \textit{Z} = -2.4, \textit{p} < 0.0167 [0.05 / 3]).
% We did not identify any significant difference between conditions B and S ($\mu$ = 5.58, $\sigma$ = 1.62, \textit{Z} = -1.2, \textit{p} = 0.202), nor between conditions P and S (\textit{Z} = 1.7, \textit{p} = 0.092).


% On the other hand, we found a significant difference in users' rating of the AI agent's awareness of the user's actions (\textit{$\chi^2$} = 8.21, \textit{df} = 2, \textit{p} < 0.05, Fig.\ref{fig:survey} Q6).
% % participants perceived different levels of disruptions among three conditions (\textit{$\chi^2$} = 17.61, \textit{df} = 2, \textit{p} < 0.001, Fig.\ref{fig:survey} Q1). 
% % Using Wilcoxon signed-rank tests, we found a similar pattern in higher perceived disruption in P condition ($\mu$ = 4.8, $\sigma$ = 1.4) than in B condition ($\mu$ = 1.5, $\sigma$ = 0.90, \textit{Z} = 3.1, \textit{p} < 0.01).
% Performing pairwise condition comparisons, we found that compared to the baseline condition ($\mu$ = 5.33, $\sigma$ = 0.98), participants felt the AI agent was more aware of their actions in P condition ($\mu$ = 6.42, $\sigma$ = 0.79, \textit{Z} = 2.4, \textit{p} < 0.0167). 
% We did not identify a significant difference between condition B and S ($\mu$ = 6.17, $\sigma$ = 0.83, \textit{Z} = 2.0, \textit{p} = 0.0442) nor between P and S conditions (\textit{Z} = -0.89, \textit{p} = 0.37).


% These results demonstrate that the proactivity in conditions P without social transparency cues simultaneously made users feel like the AI agent was more aware of the user's actions, while they were less aware of the AI agent's actions themselves. 
% % Whereas in condition S, participants did not necessarily feel less aware of the AI. 
% % This potentially indicates that the social transparency cues alleviated the abruptness of the AI agent's proactive actions. While we did not identify a significant pairwise difference between conditions P and S, 

% Interestingly, proactivity also affects users' perspectives on whether the collaboration felt like a partnership or merely utilizing a tool. 
% When asked whether they felt like the AI agent was more like a tool than a programming partner (Fig.\ref{fig:survey} Q8), we discovered significant differences across conditions (\textit{$\chi^2$} = 8.04, \textit{df} = 2, \textit{p} < 0.05). 
% % Similar to the pattern in system awareness, the conditions P ($\mu$ = 3.83, $\sigma$ = 1.59, \textit{Z} = -2.21, \textit{p} < 0.05) and S ($\mu$ = 4.0, $\sigma$ = 1.86, \textit{Z} = -2.35, \textit{p} < 0.05) were both perceived as closer to a programming partner than a tool when compared to the baseline ($\mu$ = 5.25, $\sigma$ = 1.71, \textit{Z} = 2.4, \textit{p} < 0.05). we did not find a significant pairwise difference between conditions P and S (\textit{Z} = 0.42, \textit{p} = 0.67).
% While we did not identify any significant pairwise comparisons between conditions, we received qualitative feedback that conditions P ($\mu$ = 3.83, $\sigma$ = 1.59) and S ($\mu$ = 4.0, $\sigma$ = 1.86) felt less like using a tool and more like a partnership experience than in baseline condition ($\mu$ = 5.25, $\sigma$ = 1.71).
% From the interviews, participants expressed that proactivity and social transparency cues led to a human-like programming collaboration. P1 in condition S commented that ``\textit{it's like a person that's on your side that [says] `here, you add that [code] here'}.'' P8 also expressed that in condition S, ``\textit{it felt a bit more human in the way that it was kind of interrupting you}.'' Similarly, P6 recalled that in conditions P and S, ``\textit{the fact that it was talking with me and checking in with a code editor. I maybe treated it more like an actual human}.'' 

% Based on these findings, AI proactivity could lead to users feeling less aware of the system's actions but more observed by the system with higher AI-to-user awareness. 
% When introduced with social transparency cues in condition S, participants did not feel less aware of the AI compared to the baseline. 
% However, more data needs to be collected to further explore if social transparency mediates the awareness gap when users interact with proactive AI tools.
% Proactivity and social transparency also potentially serve the purpose of facilitating a social presence, creating a partnership-like experience. 
% Future research can explore the implications of proactive AI design and how to best utilize this human-like collaboration experience.

% % The Likert-scale responses revealed that participants felt that the AI agent was more aware of the users' actions during the P ($\mu$ = 6.42, $\sigma$ = XXX; \textit{p} = 0.0071) and S conditions ($\mu$ = 6.17, $\sigma$ = XXX; \textit{p} = 0.035) compared to the B ($\mu$ = 5.33, $\sigma$ = XXX). On the other hand, the participants were less aware of the AI agent's actions during condition P ($\mu$ = 4.33, $\sigma$ = XXX) than condition B ($\mu$ = 6.33, $\sigma$ = XXX; \textit{p} = 0.0035). 

% % During the S condition, participants felt more aware of the AI agent's actions ($\mu$ = 5.58, $\sigma$ = XXX) than condition P ($\mu$ = XXX $\sigma$ = XXX; \textit{p} = 0.11) but less aware than in condition B ($\mu$ = XXX $\sigma$ = XXX; \textit{p} = 0.14). We did not identify statistical significance in the difference. A potential explanation is that social transparency cues mediated the disruptive effects of system proactivity, and provided more visual signals compared to the baseline, adding to users' cognitive load.



% % \subsection{Time and Completion}





\section{Conclusions}
\label{sec:conclusion}
\section{Conclusion}

In this paper, we introduce STeCa, a novel agent learning framework designed to enhance the performance of LLM agents in long-horizon tasks. 
STeCa identifies deviated actions through step-level reward comparisons and constructs calibration trajectories via reflection. 
These trajectories serve as critical data for reinforced training. Extensive experiments demonstrate that STeCa significantly outperforms baseline methods, with additional analyses underscoring its robust calibration capabilities.

\section*{Limitations}
\label{sec:limitations}
We acknowledge several limitations in our work. First, while our CA-based pipeline is effective in knowledge-driven contexts, its applicability to non-knowledge-based conversations, such as opinion-based questions (e.g., ``What would you do in such a scenario?''), remains unclear, as the subjective judgment required in these conversations can be difficult for a generated CA to capture. Additionally, although our pipeline prioritizes informativeness, follow-up questions do not always need to introduce new information \cite{kurkul2018question}---for example, requests for simpler explanations (e.g., ``Can you explain this in an easier-to-understand way?''). In the future, we hope to extend this method to support various types of follow-up questions and integrate it into downstream dialogue-based applications. 
%Moreover, in Section~\ref{sec:analysis:infogain}, we preliminarily explored using generated CAs to assess the informativeness of follow-up questions, highlighting a potential application beyond question generation. However, this approach requires further refinement and rigorous validation to ensure generalizability across domains and question types. We encourage future work to explore hybrid methods that combine human evaluation with automated CA-based informativeness assessment to reduce human effort while maintaining reliability.

% Conversational Agents (CAs) can facilitate information elicitation in various scenarios, such as semi-structured interviews. Current CAs can ask predetermined questions but lack skills for asking follow-up questions. Thus, we designed three approaches for CAs to automatically ask follow-up questions, i.e., follow-ups on concepts, follow-ups on related concepts, and general follow-ups. To investigate their effects, we conducted a user study (N=26) in which a CA interviewer asked follow-up questions generated by algorithms and crafted by human wizards. Our results showed that the CA's follow-up questions were readable and effective in information elicitation. The follow-ups on concepts and related concepts achieved a lower drop rate and better relevance, while the general follow-ups elicited more informative responses. Further qualitative analysis of the human-CA interview data revealed algorithm drawbacks and identified follow-up question techniques used by the human wizards. We provided design implications for improving information elicitation of future CAs based on the results.

% \cite{hu2024designing}

% Inquisitive questions -- open-ended, curiosity-driven questions people ask as they read -- are an integral part of discourse processing (Kehler and Rohde, 2017; Onea, 2016) and comprehension (Prince, 2004). Recent work in NLP has taken advantage of question generation capabilities of LLMs to enhance a wide range of applications. But the space of inquisitive questions is vast: many questions can be evoked from a given context. So which of those should be prioritized to find answers? Linguistic theories, unfortunately, have not yet provided an answer to this question. This paper presents QSALIENCE, a salience predictor of inquisitive questions. QSALIENCE is instruction-tuned over our dataset of linguist-annotated salience scores of 1,766 (context, question) pairs. A question scores high on salience if answering it would greatly enhance the understanding of the text (Van Rooy, 2003). We show that highly salient questions are empirically more likely to be answered in the same article, bridging potential questions (Onea, 2016) with Questions Under Discussion (Roberts, 2012). We further validate our findings by showing that answering salient questions is an indicator of summarization quality in news.

% \cite{wu2024questions}

\section*{Ethics Statement}
\label{sec:ethics}
\paragraph{Data.}
All datasets used in our work, MemeCap, NewYorker, and YesBut, are publicly available. The datasets include images, accompanying texts, and humor interpretations collected from humans and may contain offensive content to some people.

\paragraph{Models.} 
The LLMs and VLMs we used for the experiments are trained on a large-scale web corpora and some of them utilize human feedback. Given their training sources, they could potentially generate content (i.e., descriptions, implications, and explanations) that exhibit societal biases.

\paragraph{Data Collection.} 
We use CloudResearch to collect judgments about model-generated explanations in order to validate our proposed  automatic evaluation method.  To ensure the quality of evaluation, we required that workers were located in English-speaking countries (e.g. US, UK, Canada, Australia, and New Zealand), and had an acceptance rate of at least 93\% on 1,000 prior annotations. We paid \$0.20 for the evaluation task, which means that annotators were compensated with an average hourly wage of \$13, which is comparable to the US minimum wage. We did not use any personal information from annotators. We obtained ethics approval from our institution's research ethics board prior to running the study. 

% TODO: when we prepare the camera-ready version, we should thank Jack in the acknowledgements
\section*{Acknowledgements}
This work was funded, in part, by the Vector Institute for AI, Canada CIFAR AI Chairs program, Accelerate Foundation Models Research Program Award from Microsoft, an NSERC discovery grant, and a research gift from AI2. We thank Jack Hessel, Benyamin Movassagh, Sahithya Ravi, Aditya Chinchure, and Vasile Negrescu for insightful discussions and feedback.

\bibliography{custom,anthology}

\appendix
\section{Dataset Examples}
\label{app:dataset-eg}
Figure \ref{fig:dataset-eg} illustrates example data instances from MemeCap, NewYorker, and YesBut.

\begin{figure*}[t]
  \includegraphics[width=\linewidth]{figures/dataset-eg.pdf} \hfill
  \caption {Dataset Examples on MemeCap, NewYorker, and YesBut.}
  \label{fig:dataset-eg}
\end{figure*}


\section{SentenceSHAP}
\label{app:sentence-shap}
In this section, we introduce SentenceSHAP, an adaptation of TokenSHAP \cite{horovicz-goldshmidt-2024-tokenshap}. While TokenSHAP calculates the importance of individual tokens, SentenceSHAP estimates the importance of individual sentences in the input prompt. The importance score is calculated using Monte Carlo Shapley Estimation, following the same principles as TokenSHAP.

Given an input prompt \( X = \{x_1, x_2, \dots, x_n\} \), where \( x_i \) represents a sentence, we generate all possible combinations of \( X \) by excluding each sentence \( x_i \) (i.e., \( X - \{x_i\} \)). Let \( Z \) represent the set of all combinations where each \( x_i \) is removed. To estimate Shapley values efficiently, we randomly sample from \( Z \) with a specified sampling ratio, resulting in a subset \( Z_s = \{X_1, X_2, \dots, X_s\} \), where each \( X_i = X - \{x_i\} \).

Next, we generate a base response \( r_0 \) using a VLM (or LLM) with the original prompt \( X \), and a set of responses \( R_s = \{r_1, r_2, \dots, r_s\} \), each generated by a prompt from one of the sampled combinations in \( Z_s \).

We then compute the cosine similarity between the base response \( r_0 \) and each response in \( R_s \) using Sentence Transformer (\texttt{BAAI/bge-large-en-v1.5}). The average similarity between combinations with and without \( x_i \) is computed, and the difference between these averages gives the Shapley value for sentence \( x_i \). This is expressed as:
\begin{align}
\notag
\phi(x_i) = \\ \notag
&\frac{1}{s} \sum_{j=1}^{s} \left( \text{cos}(r_0, r_j \mid x_i) - \text{cos}(r_0, r_j \mid \neg x_i) \right)
\end{align}
where \( \phi(x_i) \) represents the Shapley value for sentence \( x_i \), $\text{cos}(r_0, r_j \mid x_i)$ is the cosine similarity between the base response and the response that includes sentence $x_i$, $\text{cos}(r_0, r_j \mid \neg x_i)$ is the cosine similarity between the base response and the response that excludes sentence $x_i$, and $s$ is the number of sampled combinations in $Z_s$.

\section{Error Analysis Based on SentenceSHAP}
Figure \ref{fig:error-analysis} presents two examples of negative impacts from implications: dilution of focus and the introduction of irrelevant information.
\label{app:error-analysis-shap}
\begin{figure*}[t]
  \includegraphics[width=\linewidth]{figures/error-analysis.pdf} \hfill
  \caption {Examples of negative impact from implications from Phi (top) and GPT4o (bottom).}
  \label{fig:error-analysis}
\end{figure*}

\section{Details on human anntations}
\label{app:cloudresearch}
We present the annotation interface on CloudResearch used for human evaluation to validate our evaluation metric in Figure \ref{fig:cloud-research}. Refer to Sec.~\ref{sec:ethics} for details on annotator selection criteria and compensation.

\begin{figure*}[t]
  \includegraphics[width=\linewidth]{figures/cloud-research.pdf} \hfill
  \caption {Annotation interface on CloudResearch used for human evaluation to validate our evaluation metric.}
  \label{fig:cloud-research}
\end{figure*}



\section{Generation Prompts for Selection and Refinement}
\label{app:gen-prompts}
Figures \ref{fig:desc-prompt}, \ref{fig:seed-imp-prompt}, and \ref{fig:nonseed-imp-prompt} show the prompts used for generating image descriptions, seed implications (1st hop), and non-seed implications (2nd hop onward). Figure \ref{fig:cand-prompt} displays the prompt used to generate candidate and final explanations. Image descriptions are used for candidate explanations when existing data is insufficient but are not used for final explanations. For calculating Cross Entropy values (used as a relevance term), we use the prompt in Figure \ref{fig:cand-prompt}, substituting the image with image descriptions, as LLM is used to calculate the cross entropies.

\begin{figure*}[h]
\small
\begin{tcolorbox}[
    title=Prompt for Image Descriptions,
    colback=white,
    colframe=CadetBlue,
    arc=0pt,        % Remove rounded corners
    outer arc=0pt   % Remove outer rounded corners (important for some styles)    
]

Describe the image by focusing on the noun phrases that highlight the actions, expressions, and interactions of the main visible objects, facial expressions, and people.\\
\\
Here are some guidelines when generating image descriptions:\\
* Provide specific and detailed references to the objects, their actions, and expressions. Avoid using pronouns in the description.\\
* Do not include trivial details such as artist signatures, autographs, copyright marks, or any unrelated background information.\\
* Focus only on elements that directly contribute to the meaning, context, or main action of the scene.\\
* If you are unsure about any object, action, or expression, do not make guesses or generate made-up elements.\\
* Write each sentence on a new line.\\
* Limit the description to a maximum of 5 sentences, with each focusing on a distinct and relevant aspect that directly contribute to the meaning, context, or main action of the scene.\\
\\
Here are some examples of desired output:
---\\
\text{[Description]} (example of newyorker cartoon image):\\
Through a window, two women with surprised expressions gaze at a snowman with human arms.\\
---\\
\text{[Description]} (example of newyorker cartoon image):\\
A man and a woman are in a room with a regular looking bookshelf and regular sized books on the wall.\\
In the middle of the room the man is pointing to text written on a giant open book which covers the entire floor.\\
He is talking while the woman with worried expression watches from the doorway.\\
---\\
\text{[Description]} (example of meme):\\
The left side shows a woman angrily pointing with a distressed expression, yelling ``You said memes would work!''.\\
The right side shows a white cat sitting at a table with a plate of food in front of it, looking indifferent or smug with the text above the cat reads, ``I said good memes would work''.\\
---\\
\text{[Description]} (example of yesbut image):\\
The left side shows a hand holding a blue plane ticket marked with a price of ``\$50'', featuring an airplane icon and a barcode, indicating it's a flight ticket.\\
The right side shows a hand holding a smartphone displaying a taxi app, showing a route map labeled ``Airport'' and a price of ``\$65''.\\
---\\

Proceed to generate the description.\\
\text{[Description]}:

\end{tcolorbox}
\caption{A prompt used to generate image descriptions.} % Add a caption to the figure
\label{fig:desc-prompt}
\end{figure*}


%%%%%%%%%%%%%%%%%%%%%%%%%%% Prompt for implications %%%%%%%%%%%%%%%%%%%%%%%%%%%
\begin{figure*}[t]
\small
\begin{tcolorbox}[
    title=Prompt for Seed Implications,
    colback=white,
    colframe=Green,
    arc=0pt,        % Remove rounded corners
    outer arc=0pt,  % Remove outer rounded corners (important for some styles)    
    % breakable,
]

You are provided with the following inputs:\\
- \text{[}Image\text{]}: An image (e.g. meme, new yorker cartoon, yes-but image)\\
- \text{[}Caption\text{]}: A caption written by a human.\\
- \text{[}Descriptions\text{]}: Literal descriptions that detail the image.\\
\\
\#\#\# Your Task:\\
\texttt{[ One-sentence description of the ultimate goal of your task. Customize based on the task. ]}\\
Infer implicit meanings, cultural references, commonsense knowledge, social norms, or contrasts that connect the caption to the described objects, concepts, situations, or facial expressions.\\
\\
\#\#\# Guidelines:\\
- If you are unsure about any details in the caption, description, or implication, refer to the original image for clarification.\\
- Identify connections between the objects, actions, or concepts described in the inputs.\\
- Explore possible interpretations, contrasts, or relationships that arise naturally from the scene, while staying grounded in the provided details.\\
- Avoid repeating or rephrasing existing implications. Ensure each new implication introduces fresh insights or perspectives.\\
- Each implication should be concise (one sentence) and avoid being overly generic or vague.\\
- Be specific in making connections, ensuring they align with the details provided in the caption and descriptions.\\
- Generate up to 3 meaningful implications.\\
\\
\#\#\# Example Outputs:\\
\#\#\#\# Example 1 (example of newyorker cartoon image):\\
\text{[}Caption\text{]}: ``This is the most advanced case of Surrealism I've seen.''\\
\text{[}Descriptions\text{]}: A body in three parts is on an exam table in a doctor's office with the body's arms crossed as though annoyed.\\
\text{[}Connections\text{]}:\\
1. The dismembered body is illogical and impossible, much like Surrealist art, which often explores the absurd.\\
2. The body’s angry posture adds a human emotion to an otherwise bizarre scenario, highlighting the strange contrast.\\
\\
\#\#\#\# Example 2 (example of newyorker cartoon image):\\
\text{[}Caption\text{]}: ``He has a summer job as a scarecrow.''\\
\text{[}Descriptions\text{]}: A snowman with human arms stands in a field.\\
\text{[}Connections\text{]}:\\
1. The snowman, an emblem of winter, represents something out of place in a summer setting, much like a scarecrow's seasonal function.\\
2. The human arms on the snowman suggest that the role of a scarecrow is being played by something unexpected and seasonal.\\
\\
\#\#\#\# Example 3 (example of yesbut image):\\
\text{[}Caption\text{]}: ``The left side shows a hand holding a blue plane ticket marked with a price of `\$50'.''\\
\text{[}Descriptions\text{]}: The screen on the right side shows a route map labeled ``Airport'' and a price of `\$65'.\\
\text{[}Connections\text{]}:\\
1. The discrepancy between the ticket price and the taxi fare highlights the often-overlooked costs of travel beyond just booking a flight.\\
2. The image shows the hidden costs of air travel, with the extra fare representing the added complexity of budgeting for transportation.\\
\\
\#\#\#\# Example 4 (example of meme):\\
\text{[}Caption\text{]}: ``You said memes would work!''\\
\text{[}Descriptions\text{]}: A cat smirks with the text ``I said good memes would work.''\\
\text{[}Connections\text{]}:\\
1. The woman's frustration reflects a common tendency to blame concepts (memes) instead of the quality of execution, as implied by the cat’s response.\\
2. The contrast between the angry human and the smug cat highlights how people often misinterpret success as simple, rather than a matter of quality.\\
\\
\#\#\# Now, proceed to generate output:\\
\text{[}Caption\text{]}: \texttt{[ Caption ]}\\
\\
\text{[}Descriptions\text{]}:\\
\texttt{[ Descriptions ]}\\
\\
\text{[}Connections\text{]}:

\end{tcolorbox}
\caption{A prompt used to generate seed implications.} % Add a caption to the figure
\label{fig:seed-imp-prompt}
\end{figure*}


%%%%%%%%%%%%%%%%%%%%%%%%%%% Prompt for nonseed implications %%%%%%%%%%%%%%%%%%%%%%%%%%%
\begin{figure*}[t]
\small
%  \begin{tcolorbox}[
%  width=\textwidth,
%  colback={white},
%  title={Title},
%  colbacktitle={DarkGreen},
%  coltitle=white,
%  colframe={DarkGreen},
%  breakable
% ]
 % \parskip=5pt

\begin{tcolorbox}[
    % breakable,
    title=Prompt for Non-Seed Implications (2nd hop onward),
    colback=white,
    colframe=Green,
    arc=0pt,        % Remove rounded corners
    outer arc=0pt,  % Remove outer rounded corners (important for some styles)    
    % breakable,
]

You are provided with the following inputs:\\
- \text{[}Image\text{]}: An image (e.g. meme, new yorker cartoon, yes-but image)\\
- \text{[}Caption\text{]}: A caption written by a human.\\
- \text{[}Descriptions\text{]}: Literal descriptions that detail the image.\\
- \text{[}Implication\text{]}: A previously generated implication that suggests a possible connection between the objects or concepts in the caption and description.\\
\\
\#\#\# Your Task:\\
\texttt{[ One-sentence description of the ultimate goal of your task. Customize based on the task. ]}\\
Infer implicit meanings across the objects, concepts, situations, or facial expressions found in the caption, description, and implication. Focus on identifying relevant commonsense knowledge, social norms, or underlying connections.\\
\\
\#\#\# Guidelines:\\
- If you are unsure about any details in the caption, description, or implication, refer to the original image for clarification.\\
- Identify potential connections between the objects, actions, or concepts described in the inputs.\\
- Explore interpretations, contrasts, or relationships that naturally arise from the scene while remaining grounded in the inputs.\\
- Avoid repeating or rephrasing existing implications. Ensure each new implication provides fresh insights or perspectives.\\
- Each implication should be concise (one sentence) and avoid overly generic or vague statements.\\
- Be specific in the connections you make, ensuring they align closely with the details provided.\\
- Generate up to 3 meaningful implications that expand on the implicit meaning of the scene.\\
\\
\#\#\# Example Outputs:\\
\#\#\#\# Example 1 (example of newyorker cartoon image):\\
\text{[}Caption\text{]}: "This is the most advanced case of Surrealism I've seen."\\
\text{[}Descriptions\text{]}: A body in three parts is on an exam table in a doctor's office with the body's arms crossed as though annoyed.\\
\text{[}Implication\text{]}: Surrealism is an art style that emphasizes strange, impossible, or unsettling scenes.\\
\text{[}Connections\text{]}:\\
1. A body in three parts creates an unsettling juxtaposition with the clinical setting, which aligns with Surrealist themes.\\
2. The body’s crossed arms add humor by assigning human emotion to an impossible scenario, reflecting Surrealist absurdity.\\
... \\
\texttt{[ We used sample examples from the prompt for generating seed implications (see Figure \ref{fig:seed-imp-prompt}), following the above format, which includes [Implication]:. ]}
\\
---\\
\\
\#\#\# Proceed to Generate Output:\\
\text{[}Caption\text{]}: \texttt{[ Caption ]}\\
\\
\text{[}Descriptions\text{]}:\\
\texttt{[ Descriptions ]}\\
\\
\text{[}Implication\text{]}:\\
\texttt{[ Implication ]}\\
\\
\text{[}Connections\text{]}:
\end{tcolorbox}
\caption{A prompt used to generate non-seed implications.} % Add a caption to the figure
\label{fig:nonseed-imp-prompt}
\end{figure*}


%%%%%%%%%%%%%%%%%%%%%%%%%%% Prompt for nonseed implications %%%%%%%%%%%%%%%%%%%%%%%%%%%
\begin{figure*}[t]
\small
%  \begin{tcolorbox}[
%  width=\textwidth,
%  colback={white},
%  title={Title},
%  colbacktitle={DarkGreen},
%  coltitle=white,
%  colframe={DarkGreen},
%  breakable
% ]
 % \parskip=5pt

\begin{tcolorbox}[
    % breakable,
    title=Prompt for Candidate and Final Explanations,
    colback=white,
    colframe=RedViolet,
    arc=0pt,        % Remove rounded corners
    outer arc=0pt,  % Remove outer rounded corners (important for some styles)    
    % breakable,
]

You are provided with the following inputs:\\
- **\text{[}Image\text{]}:** A New Yorker cartoon image.\\
- **\text{[}Caption\text{]}:** A caption written by a human to accompany the image.\\
- **\text{[}Image Descriptions\text{]}:** Literal descriptions of the visual elements in the image.\\
- **\text{[}Implications\text{]}:** Possible connections or relationships between objects, concepts, or the caption and the image.\\
- **\text{[}Candidate Answers\text{]}:** Example answers generated in a previous step to provide guidance and context.\\
\\
\#\#\# Your Task:\\
Generate **one concise, specific explanation** that clearly captures why the caption is funny in the context of the image. Your explanation must provide detailed justification and address how the humor arises from the interplay of the caption, image, and associated norms or expectations.\\
\\
\#\#\# Guidelines for Generating Your Explanation:\\
1. **Clarity and Specificity:**  \\
   - Avoid generic or ambiguous phrases.  \\
   - Provide specific details that connect the roles, contexts, or expectations associated with the elements in the image and its caption.  \\
\\
2. **Explain the Humor:**  \\
- Clearly connect the humor to the caption, image, and any cultural, social, or situational norms being subverted or referenced.  \\
- Highlight why the combination of these elements creates an unexpected or amusing contrast.\\
\\
3. **Prioritize Clarity Over Brevity:**  \\
- Justify the humor by explaining all important components clearly and in detail.  \\
- Aim to keep your response concise and under 150 words while ensuring no critical details are omitted.  \\
\\
4. **Use Additional Inputs Effectively:**\\
- **\text{[}Image Descriptions\text{]}:** Provide a foundation for understanding the visual elements."   \\
- **\text{[}Implications\text{]}:** Assist in understanding relationships and connections but do not allow them to dominate or significantly alter the central idea.\\
- **\text{[}Candidate Answers\text{]}:** Adapt your reasoning by leveraging strengths or improving upon weaknesses in the candidate answers.\\
\\
Now, proceed to generate your response based on the provided inputs.\\
\\
\#\#\# Inputs:\\
\text{[}Caption\text{]}: \texttt{\text{[} Caption \text{]}}\\
\\
\text{[}Descriptions\text{]}:\\
\texttt{\text{[} Top-K Implications \text{]}}\\
\\
\text{[}Implications\text{]}:\\
\texttt{\text{[} Top-K Implications \text{]}}\\
\\
\text{[}Candidate Anwers\text{]}:\\
\texttt{\text{[} Top-K Candidate Explanations \text{]}}\\
\\
\text{[}Output\text{]}:\\

\end{tcolorbox}
\caption{A prompt used to generate candidate and final explanations.} % Add a caption to the figure
\label{fig:cand-prompt}
\end{figure*}


\section{Evaluation Prompts}
\label{app:eval-prompts}
Figures \ref{fig:recall-prompt} and \ref{fig:precision-prompt} present the prompts used to calculate recall and precision scores in our LLM-based evaluation, respectively.

%%%%%%%%%%%%%%%%%%%%%%%%%%% Prompt for nonseed implications %%%%%%%%%%%%%%%%%%%%%%%%%%%
\begin{figure*}[t]
\small
\begin{tcolorbox}[
    % breakable,
    title=Prompt for Evaluating Recall Score,
    colback=white,
    colframe=MidnightBlue,
    arc=0pt,        % Remove rounded corners
    outer arc=0pt,  % Remove outer rounded corners (important for some styles)    
    % breakable,
]

Your task is to assess whether \text{[}Sentence1\text{]} is conveyed in \text{[}Sentence2\text{]}. \text{[}Sentence2\text{]} may consist of multiple sentences.\\
\\
Here are the evaluation guidelines:\\
1. Mark 'Yes' if \text{[}Sentence1\text{]} is conveyed in \text{[}Sentence2\text{]}.\\
2. Mark 'No' if \text{[}Sentence2\text{]} does not convey the information in \text{[}Sentence1\text{]}.\\
\\
Proceed to evaluate. \\
\\
\text{[}Sentence1\text{]}: \texttt{[ One Atomic Sentence from Decomposed Reference Explanation ]} \\
\\
\text{[}Sentence2\text{]}: \texttt{[ Predicted Explanation ]}\\
\\
\text{[}Output\text{]}:

\end{tcolorbox}
\caption{Prompt for evaluating recall score.} % Add a caption to the figure
\label{fig:recall-prompt}
\end{figure*}


\begin{figure*}[t]
\small
\begin{tcolorbox}[
    % breakable,
    title=Prompt for Evaluating Precision Score,
    colback=white,
    colframe=MidnightBlue,
    arc=0pt,        % Remove rounded corners
    outer arc=0pt,  % Remove outer rounded corners (important for some styles)    
    % breakable,
]

Your task is to assess whether \text{[}Sentence1\text{]} is inferable from \text{[}Sentence2\text{]}. \text{[}Sentence2\text{]} may consist of multiple sentences.\\
\\
Here are the evaluation guidelines:\\
1. Mark "Yes" if \text{[}Sentence1\text{]} can be inferred from \text{[}Sentence2\text{]} — whether explicitly stated, implicitly conveyed, reworded, or serving as supporting information.\\
2. Mark 'No' if \text{[}Sentence1\text{]} is absent from \text{[}Sentence2\text{]}, cannot be inferred, or contradicts it.\\
\\
Proceed to evaluate. \\
\\
\text{[}Sentence1\text{]}: \texttt{[ One Atomic Sentence from Decomposed Predicted Explanation ]}\\
\\
\text{[}Sentence2\text{]}: \texttt{[ Reference Explanation ]}\\
\\
\text{[}Output\text{]}:


\end{tcolorbox}
\caption{Prompt for evaluating precision score.} % Add a caption to the figure
\label{fig:precision-prompt}
\end{figure*}

\section{Prompts for Baselines}
\label{app:base-prompts}

Figure \ref{fig:base-prompt} presents the prompt used for the ZS, CoT, and SR Generator methods. While the format remains largely the same, we adjust it based on the baseline being tested (e.g., CoT requires generating intermediate reasoning, so we add extra instructions for that).
Figure \ref{fig:critic-prompt} shows the prompt used in the SR critic model. The critic's criteria include: (1) \textit{correctness}, measuring whether the explanation directly addresses why the caption is humorous in relation to the image and its caption; (2) \textit{soundness}, evaluating whether the explanation provides a well-reasoned interpretation of the humor; (3) \textit{completeness}, ensuring all important aspects in the caption and image contributing to the humor are considered; (4) \textit{faithfulness}, verifying that the explanation is factually consistency with the image and caption; and (5) \textit{clarity}, ensuring the explanation is clear, concise, and free from unnecessary ambiguity.
\begin{figure*}
\small
\begin{tcolorbox}[
    % breakable,
    title=Prompt for Baselines,
    colback=white,
    colframe=Black,
    arc=0pt,        % Remove rounded corners
    outer arc=0pt,  % Remove outer rounded corners (important for some styles)    
    % breakable,
]

You are provided with the following inputs:\\
- **\text{[}Image\text{]}:** A New Yorker cartoon image.\\
- **\text{[}Caption\text{]}:** A caption written by a human to accompany the image.\\
\texttt{[ if Self-Refine with Critic is True: ]} \\
- **\text{[}Feedback for Candidate Answer\text{]}:** Feedback that points out some weakness in the current candidate responses.\\
\texttt{[ if Self-Refine is True: ]} \\
- **\text{[}Candidate Answers\text{]}:** Example answers generated in a previous step to provide guidance and context.\\
\\
\#\#\# Your Task:\\
Generate **one concise, specific explanation** that clearly captures why the caption is funny in the context of the image. Your explanation must provide detailed justification and address how the humor arises from the interplay of the caption, image, and associated norms or expectations.\\
\\
\#\#\# Guidelines for Generating Your Explanation:\\
1. **Clarity and Specificity:**  \\
   - Avoid generic or ambiguous phrases.  \\
   - Provide specific details that connect the roles, contexts, or expectations associated with the elements in the image and its caption.  \\
\\
2. **Explain the Humor:**  \\
- Clearly connect the humor to the caption, image, and any cultural, social, or situational norms being subverted or referenced.  \\
- Highlight why the combination of these elements creates an unexpected or amusing contrast.\\
\\
3. **Prioritize Clarity Over Brevity:**  \\
- Justify the humor by explaining all important components clearly and in detail.  \\
- Aim to keep your response concise and under 150 words while ensuring no critical details are omitted.  \\
\\
\texttt{[ if Self-Refine is True: ]}\\
4. **Use Additional Inputs Effectively:**\\
- **[Candidate Answers]:** Adapt your reasoning by leveraging strengths or improving upon weaknesses in candidate answers. \\
\texttt{[ if Self-Refine with Critic is True: ]}\\
- **[Feedback for Candidate Answer]:** Feedback that points out some weaknesses in the current candidate responses.\\
\\
\texttt{ [ if CoT is True: ]} \\
Begin by analyzing the image and the given context, and explain your reasoning briefly before generating your final response. \\
\\
Here is an example format of the output: \\
\{\{ \\
    "Reasoning": "...", \\
    "Explanation": "..."   \\
\}\} \\

Now, proceed to generate your response based on the provided inputs.\\
\\
\#\#\# Inputs:\\
\text{[}Caption\text{]}: \texttt{\text{[} Caption \text{]}}\\
\\
\text{[}Candidate Answers\text{]}: \texttt{\text{[} Candidate Explanations \text{]}}\\
\\
\text{[}[Feedback for Candidate Answer]:\text{]}: \texttt{\text{[} Feedback for Candidate Explanations \text{]}}\\
\\
\text{[}Output\text{]}:\\

\end{tcolorbox}
\caption{A prompt used for baseline methods, with conditions added based on the specific baseline being experimented with.} % Add a caption to the figure
\label{fig:base-prompt}
\end{figure*}


\begin{figure*}
\small
\begin{tcolorbox}[
    % breakable,
    title=Prompt for Self-Refine Critic,
    colback=white,
    colframe=Black,
    arc=0pt,        % Remove rounded corners
    outer arc=0pt,  % Remove outer rounded corners (important for some styles)    
    % breakable,
]
\texttt{[ Customize goal text here: ]} \\
\texttt{MemeCap:} You will be given a meme along with its caption, and a candidate response that describes what meme poster is trying to convey. \\
\texttt{NewYorker:} You will be given an image along with its caption, and a candidate response that explains why the caption is funny for the given image. \\
\texttt{YesBut:} You will be given an image and a candidate response that describes why the image is funny or satirical. \\
\\
Your task is to criticize the candidate response based on the following evaluation criteria: \\
- Correctness: Does the explanation directly address why the caption is funny, considering both the image and its caption? \\
- Soundness: Does the explanation provide a meaningful and well-reasoned interpretation of the humor? \\
- Completeness: Does the explanation address all relevant aspects of the caption and image (e.g., visual details, text) that contribute to the humor? \\
- Faithfulness: Is the explanation factually consistent with the details in the image and caption? \\
- Clarity: Is the explanation clear, concise, and free from unnecessary ambiguity? \\
 \\
Proceed to criticize the candidate response ideally using less than 5 sentences:\\
\\
\text{[}Caption\text{]}: \texttt{[ caption ]}\\
\\
\text{[}Candidate Response\text{]}: \\
 \texttt{\text{[} Candidate Response \text{]}}\\
\\
\text{[}Output\text{]}: \\
\end{tcolorbox}
\caption{A prompt used in SR critic model.} % Add a caption to the figure
\label{fig:critic-prompt}
\end{figure*}

% \begin{figure*}[t]
%   \includegraphics[width=\linewidth]{figures/error-analysis.pdf} \hfill
%   \vspace{-20pt}
%   \caption {Examples of negative impact from implications from Phi (top) and GPT4o (bottom).}
%   \label{fig:error-analysis}
% \end{figure*}

\end{document}
