% This must be in the first 5 lines to tell arXiv to use pdfLaTeX, which is strongly recommended.
\pdfoutput=1
% In particular, the hyperref package requires pdfLaTeX in order to break URLs across lines.

\documentclass[11pt]{article}

% Change "review" to "final" to generate the final (sometimes called camera-ready) version.
% Change to "preprint" to generate a non-anonymous version with page numbers.
\usepackage[preprint]{acl}

% Standard package includes
\usepackage{times}
\usepackage{latexsym}

% For proper rendering and hyphenation of words containing Latin characters (including in bib files)
\usepackage[T1]{fontenc}
% For Vietnamese characters
% \usepackage[T5]{fontenc}
% See https://www.latex-project.org/help/documentation/encguide.pdf for other character sets

% This assumes your files are encoded as UTF8
\usepackage[utf8]{inputenc}

% This is not strictly necessary, and may be commented out,
% but it will improve the layout of the manuscript,
% and will typically save some space.
\usepackage{microtype}

% This is also not strictly necessary, and may be commented out.
% However, it will improve the aesthetics of text in
% the typewriter font.
\usepackage{inconsolata}

%Including images in your LaTeX document requires adding
%additional package(s)
\usepackage{graphicx}
\usepackage{amsmath}
\usepackage{mathtools}
\usepackage{booktabs}
\usepackage{enumitem}
\usepackage{hhline}
\usepackage{multirow}
\usepackage{bbm}
\usepackage{colortbl}
\usepackage{listings}
\usepackage{mdframed}
\usepackage{adjustbox}
\usepackage[most]{tcolorbox}

\definecolor{lightgray}{RGB}{240,240,240}
\definecolor{Gray}{gray}{0.7}
\definecolor{Green}{rgb}{0.05, 0.5, 0.06}
\definecolor{Purple}{rgb}{0.56, 0.0, 1.0}
\definecolor{Orange}{rgb}{1.0, 0.55, 0.0}\definecolor{Blue}{rgb}{0.0, 0.2, 0.4}

% If the title and author information does not fit in the area allocated, uncomment the following
%
%\setlength\titlebox{<dim>}
%
% and set <dim> to something 5cm or larger.

\title{\textsc{BottleHumor}: Self-Informed Humor Explanation using the Information Bottleneck Principle}
% Unsupervised Discovery of Useful Information\\for Multimodal Humor Understanding}

% Author information can be set in various styles:
% For several authors from the same institution:
% \author{Author 1 \and ... \and Author n \\
%         Address line \\ ... \\ Address line}
% if the names do not fit well on one line use
%         Author 1 \\ {\bf Author 2} \\ ... \\ {\bf Author n} \\
% For authors from different institutions:
% \author{Author 1 \\ Address line \\  ... \\ Address line
%         \And  ... \And
%         Author n \\ Address line \\ ... \\ Address line}
% To start a separate ``row'' of authors use \AND, as in
% \author{Author 1 \\ Address line \\  ... \\ Address line
%         \AND
%         Author 2 \\ Address line \\ ... \\ Address line \And
%         Author 3 \\ Address line \\ ... \\ Address line}
\author{EunJeong Hwang$^{1,2}$, Peter West$^{1}$, and Vered Shwartz$^{1,2}$ \\
$^1$ University of British Columbia~~~$^2$ Vector Institute for AI\\
{\tt \{ejhwang,pwest,vshwartz\}@cs.ubc.ca}}

% \author{EunJeong Hwang \\
%   Affiliation / Address line 1 \\
%   Affiliation / Address line 2 \\
%   Affiliation / Address line 3 \\
%   \texttt{email@domain} \\\And
%   Second Author \\
%   Affiliation / Address line 1 \\
%   Affiliation / Address line 2 \\
%   Affiliation / Address line 3 \\
%   \texttt{email@domain} \\}

%\author{
%  \textbf{First Author\textsuperscript{1}},
%  \textbf{Second Author\textsuperscript{1,2}},
%  \textbf{Third T. Author\textsuperscript{1}},
%  \textbf{Fourth Author\textsuperscript{1}},
%\\
%  \textbf{Fifth Author\textsuperscript{1,2}},
%  \textbf{Sixth Author\textsuperscript{1}},
%  \textbf{Seventh Author\textsuperscript{1}},
%  \textbf{Eighth Author \textsuperscript{1,2,3,4}},
%\\
%  \textbf{Ninth Author\textsuperscript{1}},
%  \textbf{Tenth Author\textsuperscript{1}},
%  \textbf{Eleventh E. Author\textsuperscript{1,2,3,4,5}},
%  \textbf{Twelfth Author\textsuperscript{1}},
%\\
%  \textbf{Thirteenth Author\textsuperscript{3}},
%  \textbf{Fourteenth F. Author\textsuperscript{2,4}},
%  \textbf{Fifteenth Author\textsuperscript{1}},
%  \textbf{Sixteenth Author\textsuperscript{1}},
%\\
%  \textbf{Seventeenth S. Author\textsuperscript{4,5}},
%  \textbf{Eighteenth Author\textsuperscript{3,4}},
%  \textbf{Nineteenth N. Author\textsuperscript{2,5}},
%  \textbf{Twentieth Author\textsuperscript{1}}
%\\
%\\
%  \textsuperscript{1}Affiliation 1,
%  \textsuperscript{2}Affiliation 2,
%  \textsuperscript{3}Affiliation 3,
%  \textsuperscript{4}Affiliation 4,
%  \textsuperscript{5}Affiliation 5
%\\
%  \small{
%    \textbf{Correspondence:} \href{mailto:email@domain}{email@domain}
%  }
%}

\newcommand{\method}{\textsc{BottleHumor}}
\newcommand{\base}{\textsc{ZS}}
\newcommand{\chain}{\textsc{CoT}}
\newcommand{\critic}{\textsc{SR}}
\newcommand{\nocritic}{\textsc{SR-noC}}

\begin{document}
\maketitle
\begin{abstract}
% Humor plays a crucial role in understanding ourselves, society, and the world, shaped by cultural, social, and personal factors, and expressed through images, text, and audio. 
Humor is prevalent in online communications and it often relies on more than one modality (e.g., cartoons and memes).  
Interpreting humor in multimodal settings requires drawing on diverse types of knowledge, including metaphorical, sociocultural, and commonsense knowledge. However, identifying the most useful knowledge remains an open question. 
We introduce \method{}, a method inspired by the information bottleneck principle that elicits relevant world knowledge from vision and language models which is iteratively refined for generating an explanation of the humor in an unsupervised manner. Our experiments on three datasets confirm the advantage of our method over a range of baselines. 
Our method can further be adapted in the future for additional tasks that can benefit from eliciting and conditioning on relevant world knowledge and open new research avenues in this direction.

%We provide a detailed analysis of the IB components and the impact of implications on final explanations.


\end{abstract}

\section{Introduction}
\label{sec:intro}
\section{Introduction}
\label{sec:intro}

Foundational models (FMs)~\cite{zhang2024data, zhou2023comprehensive} have shown remarkable progress in the healthcare domain, enabling professional-like assessment of disease diagnosis, treatment decision-making, and monitoring~\cite{zhang2023text, wang2022medclip, lu2023mi-zero}. 
Examples include LLaVA-Med~\cite{li2023llava}, Med-PaLM Multimodal~\cite{tu2024towards}, and Med-Flamingo~\cite{moor2023med}, have demonstrated their capacity on question answering, medical image analysis, and report generation.
These studies follow a predominant top-down model development strategy that requires upstream developers to collect data and train models for downstream tasks. 
Consequently, the developed model capabilities are heavily dependent on the training data, limiting their generalization performance in diverse clinical scenarios. 
For instance, Med-Gemini~\cite{yang2024advancing} reveals promising general capabilities in report generation while it lags behind state-of-the-art (SoTA) models on classification tasks, especially for out-of-domain applications. 
This indicates that while the generalizability of the foundation model is promising, more solutions are expected to meet the various specialized clinical needs.

To address these challenges, multi-center data centralization becomes essential to enhance model capacity and robustness across varied clinical scenarios~\cite{rajpurkar2022ai}. 
Centralizing distributed data can significantly improve model training and inference performance.
However, the process of medical data storage, transfer, and aggregation among centers requires extra efforts to ensure data security and system interoperability~\cite{bradford2020international}.
Moreover, a growing concern for patient privacy makes large-scale multi-center data sharing particularly challenging. 
While efforts like federated learning~\cite{wen2023survey, li2020review} can achieve good model performance on local data, the need for synchronized system coordination presents significant challenges, as clients are unable to update asynchronously. This limitation greatly restricts the practical capability of such approaches.
As a result, without a flexible collaboration, medical community still struggles to fully utilize the isolated data and local computation resources for comprehensive medical AI model development. 
To address this dilemma, open-source platforms encourage public data sharing and knowledge integration~\cite{markiewicz2021openneuro, zenodo}.
However, these platforms focus solely on raw data sharing while seldom providing collaborative model training or cooperation between different institutions.
Recently, collaborative learning has emerged as a viable approach for enhancing multi-model robustness~\cite{boulemtafes2020review}. 
For instance, software-like model development~\cite{raffel2023building} mimics software engineering practices by introducing structured workflows, enabling merging, version control, and continuous model integration.
Under this design, model ability can be strengthened with incremental knowledge updates similar to the version updating in software development. 

Although collaborative learning provides a multi-model collaboration, two key challenges remain in the leakage of raw data during collaboration~\cite{huang2023lorahub} and the synchronization of multiple collaborators~\cite{mcmahan2017communication} in the medical AI community. It is still challenging to integrate decentralized, privacy-sensitive data across institutions, leading to under-utilized insights and fragmented knowledge sharing~\cite{kaissis2020secure, rajpurkar2022ai, abdullah2021ethics}.
 To address these challenges, inspired by the collaborative software development, we propose \textbf{Med}ical \textbf{Fo}undation Models Me\textbf{rg}ing (\textbf{MedForge}), a cooperative workflow enabling continuously community-driven foundation model (FM) development.
MedForge enables a lightweight manner for individual centers to share their knowledge among multiple centers, minimizing the burden of data transmission and integration while enhancing model robustness.
Meanwhile, MedForge facilitates asynchronous and flexible collaboration, allowing individual centers to continuously update and improve medical FMs without the need for real-time synchronization.
Similar to open-source software development, MedForge incrementally updates medical knowledge and follows a sustainable model development scheme. 
This key design emphasizes a bottom-up construction of a multi-task medical FM, allowing downstream users to collaboratively build, refine, and update the upstream model according to their local resources. Our major contributions of MedForge are as below: 
\begin{enumerate}
    \item[$\bullet$] We introduce a collaborative workflow to promote the merging scheme of open-source software development. Our proposed MedForge allows distributed clinical centers to asynchronously contribute to comprehensive medical model construction while reducing transmitting costs among centers and avoiding the leakage of raw data, thus enhancing the utilization of private resources in the healthcare system. 
    \item[$\bullet$] We propose two effective knowledge-merging strategies for the asynchronous branch contribution. The MedForge-Fusion strategy updates the plugin module parameters of the main model during the merging phase, whereas the MedForge-Mixture strategy integrates the output of the plugin module by memorizing each contributor's coefficient. These strategies make MedForge more flexible and versatile. MedForge-Fusion is friendly to implement, while the MedForge-Mixture offers better performance and robustness.
    \item[$\bullet$]  We comprehensively evaluate model merging strategies to accumulate medical knowledge among multiple branch plugin modules. MedForge yields superior performance on medical classification tasks compared to other collaborative baselines across multiple datasets. We demonstrate the robustness of MedForge by shuffling the task order and evaluating various configurations of plugin modules and dataset distillation methods.
\end{enumerate}




\section{Related Work}
\label{sec:bg}
\paragraph{Uncertainty-based hallucination detection methods.}
Various approaches have been proposed to detect hallucinated content in LLMs generation.
Unlike other methods that require external knowledge sources for fact-checking~\citep{gou2024critic, chen-etal-2024-complex, min-etal-2023-factscore, huo2023retrieving}, uncertainty-based approaches are reference-free and rely only on LLM internal states or behaviors to determine hallucination~\citep{10.1145/3703155}. 
For instance, sampling-based approaches generate multiple responses and measure the diversity in meaning among them~\citep{fomicheva-etal-2020-unsupervised, kuhn2023semantic, lin2024generating}, while density-based approaches approximate the training data distribution and provide probabilities or unnormalized scores to assess how likely a generated response belongs to the distribution~\citep{yoo-etal-2022-detection, ren2023outofdistribution, vazhentsev-etal-2023-hybrid}.

In this paper, we focus on uncertainty quantification methods that rely on token-level likelihood or entropy~\citep{guerreiro-etal-2023-looking, malinin2021uncertainty}. 
Recent works have explored refining likelihood estimation by incorporating semantic relationships or reweighting token importance. For instance, Claim-Conditioned Probability (CCP)~\citep{fadeeva-etal-2024-fact} was introduced to recalculate likelihood according to semantical equivalence; while \citet{zhang-etal-2023-enhancing-uncertainty} and \citet{duan-etal-2024-shifting} adjust token weights to better convey meaning in uncertainty aggregation. \emph{Although these approaches leverage token-level information, they are typically evaluated at the sentence level, raising questions about their reliability}. To address this, we conduct a comprehensive analysis of entity-level hallucination detection for finer-grained performance insights.


\paragraph{Fine-grained hallucination detection benchmark.}

Most hallucination detection benchmarks are in sentence or paragraph level. For example, CoQA~\citep{reddy-etal-2019-coqa}, TriviaQA~\citep{joshi-etal-2017-triviaqa}, TruthfulQA~\citep{lin-etal-2022-truthfulqa}, and HaluEval~\citep{li-etal-2023-halueval}. These benchmarks classify each generated response as either hallucinated or correct. However, instance-level detection cannot pinpoint specific hallucinated content, which is crucial for correcting misinformation~\citep{cattan2024localizingfactualinconsistenciesattributable}. This limitation becomes particularly problematic in long-form text, where a single response often combines supported and unsupported information, making binary quality judgments inadequate~\citep{min-etal-2023-factscore}.

To address these challenges, recent works have advanced benchmarks for more granular hallucination detection. For example, \citet{min-etal-2023-factscore} introduced \textsc{FActScore}, which decomposes LLM-generated text into atomic facts---short sentences conveying a single piece of information---for more precise evaluation. In parallel, \citet{cattan2024localizingfactualinconsistenciesattributable} introduced \textsc{QASemConsistency}, decomposing LLM generated text with QA-SRL, a semantic formalism, to form simple QA pairs, where each QA pair represent one verifiable fact. \emph{However, these methods do not enable entity-level hallucination detection, as they lack explicit entity-level labeling (hallucinated or not) in the original generated text}.  
Beyond decomposition-based approaches, datasets like \textsc{HaDes}~\citep{liu-etal-2022-token} and CLIFF~\citep{cao-wang-2021-cliff} create token-level hallucinated content by perturbing human-written text, allowing token-level annotation on the same text. These perturbed hallucinated content, however, could be unrealistic, biased, and overly synthetic due to the limitations of models they used to perturb words. 
To bridge this gap, we create a new dataset with entity-level hallucination labels on the same LLMs generated text. This allows us to evaluate uncertainty-based hallucination detection approaches on a finer-grained level and analyze their reliability.






\section{\method{}}
\label{sec:method}

\section{\Kbase Generation}\label{sec:kb}
The Knowledge Management System module (KMS), is in charge of taking the natural language description of both the environment and the actions that the agents can do, and convert them to a Prolog \kb using a LLM. 
The \kb contains all the necessary elements to define the mapped planning problem introduced in the previous section.

The framework works by considering a high-level and a low-level \kbase. For this reason, the input descriptions are also split into \HL and \LL. The former captures more abstract concepts, e.g., complex actions such as \verb|move_block| or the objects that are present in the environment. The latter captures more concrete and physical aspects of the problem, e.g., the actions that can be actually carried out by the agents such as \verb|move_arm| or the positions of the blocks. An example of this division can be seen in Section~\ref{sssec:runegKMS}.

% Describe how the kb works
The \kbase is divided in the following parts:
\begin{itemize}
    \item General \kb ($K$): contains the grounding predicates, both for the \HL and \LL. These predicates describe parts of the scenario or of the environment that do not change during execution. For example, the predicate \verb|wheeled(a1)|, which states that robot \verb | a1| has wheels, should be part of the general \kb and not of the state. 
    \item Initial ($I$) and final states ($G$): they contain all the fluents that change during the execution of the plan. This could be, for example, the position of blocks in the environment. 
    \item High-level actions ($DA_H$): each high-level action predicate is written as:
\begin{minted}[fontsize=\small,breaklines]{Prolog}
action(
    action_name(args),
    [positive_preconditions],
    [negative_preconditions],
    [grounding_predicates],
    [effects]
).
\end{minted}
    The low-level actions ($DA_L$) have the same structure, but instead of being described as predicates of type \verb|action|, they are described as \verb|ll_action|. The preconditions $\pc{a}$ of an action $a$ are obtained by combining the list of predicates \verb|positive_preconditions| and \verb|negative_preconditions|. The predicates in the list \verb|grounding_predicates| are used to ground the parametrised fluents of the action. For example, the action \verb|move_block| depends on a block, and we can check that the action is correctly picking a block and not another object by querying the \kb in this step. Section~\ref{sssec:runegKMS} clarifies this aspect.  
    \item Mappings ($M$): contains a dictionary of \HL actions $DA_H$ and how they should be mapped to a sequence of \LL actions $DA_L$. As will become apparent in the following, the distinction between \HL and \LL actions induces a significant simplification in the planning phase.
    \item Resources ($R$): the predicate \verb|resource\1| states whether another predicate is part of the resources or not. As mentioned before, this is helpful because it allows one to shrink the complexity of the problem not having to check multiple predicates, but instead they are later allocated during the optimisation part.
\end{itemize}

% Describe the process to validate the initial descriptions
Once the user provides descriptions for the \HL and \LL parts, the framework performs a consistency check to ensure that there are no conflicts between them. It verifies that both descriptions share the same goal, that objects remain consistent across \HL and \LL, and that agents are capable of executing the tasks. This validation is carried out by an LLM, which, if inconsistencies are detected, provides an explanation to help the user make the necessary corrections.

In both this step and the subsequent steps to generate the \kb, the LLM is not used directly out of the box. Instead, we employ the Chain-of-Thought (CoT)~\cite{wei2022chain} approach, which involves providing the LLM with examples to guide its reasoning. This process ensures that the output is not only structurally correct, but also more aligned with the overall goal of the task.

% Describe how the different aspects of the kb are extracted
Examples are particularly important when generating the \kb. Indeed, as we have mentioned before, the \kbase is highly structured and the planner expects to have the different components written correctly. CoT enables the LLM to know these details. 

We tested two different ways of generating the \kb through LLMs:
\begin{itemize}
    \item either we produced the whole \kb for the high-level and the low-level all at once, or
    \item we produced the single parts of the \kbs. 
\end{itemize}

The first approach is quite straightforward: once we have the examples to give to the LLM for the CoT process, we can input the \HL description and query the LLM to first extract the high-level \kb, and then also feed the created \HL \kb to the LLM to generate the \LL \kb, which will contain everything. % Highlight why one would want this. 

Instead, the second approach requires more requests to the LLM. We first focus on the \HL \kb, and then feed the \kb that we have obtained to generate the \LL parts. For the \HL generation, we ask the LLM to generate the general \kb, the initial and final states, and the actions set in this particular order. Each time we provide the LLM with the \HL description and with the elements generated in the previous steps. The same thing is done for the \LL \kb, generating again the four components and feeding each time also the \HL \kb. We include a final step that generates the mappings between the \HL and \LL actions. As for all the other steps, also in this final step, we pass the previously generated elements of the \LL \kb. 
Although generating the entire \kbase at once would reduce token usage and speed up the process, dividing the generation of the \kb into distinct steps enhances the system's accuracy, as demonstrated in the experimental evaluation of Section~\ref{sec:experiments}. This improvement comes because the iterative approach allows the LLM to first focus on generating more homogeneous information (i.e., the high-level) and then leverage the previously generated content to perform a consistency check.  

\subsection{Runing Example -- KMS}
\label{sssec:runegKMS}

We now introduce a running example, which will be used throughout this work to expose the interplay between the different components of the framework. 
This scenario is taken from the blocks-world domain~\cite{blocksworld}, which is frequently used in task planning. In particular, in this scenario we consider a table, blocks, which may either be directly on the table or stack on top of each other, and robotics arms, which move the blocks around. Each block is also associated with a position in the 2D space. 
In this particular example, we start from a situation in which we have two blocks, \verb|b1| and \verb|b2|, which are sat on the table in position (1,1) and (3,1) respectively. The goal is to move \verb|b1| in position (2,2) and then put \verb|b2| on top of it. An iconography of the example can be seen in Figure~\ref{fig:running-example}.

\begin{figure}
    \centering
    \input{figures/running-example}
    \caption{A scheme showing the running example. Two blocks must be moved from their initial position to a new position in which they are also stacked.}
    \label{fig:running-example}
\end{figure}

While this is a trivial example, it highlights very well the
capability of the knowledge management system to generate complex
predicates that can be used for planning and it also shows the
cooperative abilities of the framework. Indeed, while using a single
robotic arm generates a straight-forward plan solution, coordinating
two robotics arms to do the same task reduces the completion time at the price
of a higher planning complexity.


Let's now focus on the \kbase generation. For example, the two queries that could be passed to the system are:

\begin{textbox}{\HL query}
% \textboxtitle{\HL query}{
\footnotesize
In this scenario, there are two blocks on a table. They are placed in positions (1,1) and (3,1). There is a robotic agent that can move the blocks from one position to another. In particular, it can:
\begin{itemize}
    \item move a block from a position on the table, to another position on the table;
    \item move a block from a position on the table, to the top of another block;
    \item move a block from the top of a block, to a position on the table;
    \item move a block from the top of a block, to the top of another block.
\end{itemize}
The goal of this scenario is to move the first block from position (1,1) to position (2,2), and then place the second block on top of it.
% }
\end{textbox}

\begin{textbox}{\LL query}
% \textboxtitle{\LL query}{
\footnotesize
In this scenario, there are two blocks that are in the same positions as described in the \HL query. The robotic agent is a robotic arm and its end-effector is a two-finger gripper that can grasp an object. In particular, the following APIs are made available for the agent to move:
\begin{itemize}
    \item \texttt{move\_arm(Arm, X1, Y1, X2, Y2)}, which allows the end-effector to be moved from one position to another;
    \item \texttt{grip(Arm)}, which allows the gripper to be lowered onto a block and grip the block;
    \item \texttt{release(Arm)}, which allows to lower the block and release the gripper leaving the block in place.
\end{itemize}
% }
\end{textbox}


As mentioned before, the first step is to pass the two queries for validation. In this case, when using GPT-4o, both queries pass the consistency check. If instead of using the previous \HL query, we would modify it by removing the possibility of placing one block on top of the other, then the LLM would return the following error:

\begin{textboxerror}
\footnotesize
The description of the scenario has an issue. The agent is described as being able to move a block from a position on the table to another position on the table, and from the top of a block to a position on the table. However, the goal requires the agent to place the second block on top of the first block, which is not listed as an action that the agent can carry out. The agent's capabilities do not include placing a block on top of another block, which is necessary to achieve the goal.
\end{textboxerror}

\noindent This also allows the user to understand where the mistake is and correct it. 

Once the validation part succeeds, the system generates the \HL \kb. In this particular instance, for space limitation, we present only the general \kbase ($K$), the initial ($I$) and final ($G$) states, and a single action. 

\begin{center}
\begin{minipage}{\linewidth}
    \begin{minipage}{.48\linewidth}
        \begin{codebox}{prolog}{General KB}
% Positions
pos(1,1).
pos(2,2).
pos(3,1).

% Blocks
block(b1).
block(b2).

% Agents
agent(a1).

% Resources
resources(agent(_)).
        \end{codebox}
    \end{minipage}
    \hfill
    \begin{minipage}{.48\linewidth}
        \begin{minipage}{\linewidth}
        \begin{codebox}{prolog}{Initial state ($I$)}
init_state([
  ontable(b1), ontable(b2),
  at(b1,1,1), at(b2,3,1),
  clear(b1), clear(b2),
  available(a1)
]).
        \end{codebox}
        \end{minipage}
        \hspace{1cm}\\
        \begin{minipage}{\linewidth}
        \begin{codebox}{prolog}{Final state ($G$)}
goal_state([
  ontable(b1),
  on(b2, b1),
  at(b1,2,2), at(b2,2,2),
  clear(b2),
  available(a1)
]).
        \end{codebox}
        \end{minipage}
    \end{minipage}
\end{minipage}
\begin{codebox}{prolog}{Action example}
action(move_table_to_table_start(Agent, Block, X1, Y1, X2, Y2), 
  [ontable(Block), at(Block, X1, Y1), available(Agent), clear(Block)],
  [
    at(_, X2, Y2), on(Block, _), moving_table_to_table(_, Block, _, _, _, _), 
    moving_table_to_block(_, Block, _, _, _, _, _)
  ],
  [agent(Agent), pos(X1, Y1), pos(X2, Y2), block(Block)],
  [
    del(available(Agent)), del(clear(Block)), del(ontable(Block)), del(at(Block, X1, Y1)),
    add(moving_table_to_table(Agent, Block, X1, Y1, X2, Y2))
  ]
).
\end{codebox}
\end{center}

The resulting \HL \kb is human-readable and relatively simple (in fulfilment of requirement \textbf{R2}).
The user at this point can make corrections to the \HL \kb, if needed, and finally, \frameworkname will also generate the \LL \kbase. In this case for space limitation, we show the changes made to the previous elements, one low-level action, and one mapping. 

\begin{center}
\begin{minipage}{\linewidth}
    \begin{minipage}{.48\linewidth}
        \begin{codebox}{prolog}{General KB}
% Positions
pos(0,0).
pos(1,1).
pos(2,2).
pos(3,1).

% Blocks
block(b1).
block(b2).

% Agents
agent(a1).

% Low-level predicates
ll_arm(a1).
ll_gripper(a1).

% Resources
resources(agent(_)).
        \end{codebox}
    \end{minipage}
    \hfill
    \begin{minipage}{.48\linewidth}
        \begin{minipage}{\linewidth}
        \begin{codebox}{prolog}{Initial state ($I$)}
init_state([
  ontable(b1), ontable(b2),
  at(b1,1,1), at(b2,3,1),
  clear(b1), clear(b2),
  available(a1),
  ll_arm_at(a1,0,0), 
  ll_gripper(a1,open) 
]).
        \end{codebox}
        \end{minipage}
        \hspace{1cm}\\
        \begin{minipage}{\linewidth}
        \begin{codebox}{prolog}{Final state ($G$)}
goal_state([
  ontable(b1),
  on(b2, b1),
  at(b1,2,2), at(b2,2,2),
  clear(b2),
  available(a1),
  ll_arm_at(a1,_,_), 
  ll_gripper(a1,_)    
]).
        \end{codebox}
        \end{minipage}
    \end{minipage}
\end{minipage}
\begin{codebox}{prolog}{Action example}
ll_action(move_arm_start(Arm, X, Y),
  [],
  [ll_arm_at(_, X, Y), moving_arm(Arm, _, _, _, _), gripping(Arm, _), releasing(Arm)],
  [],
  [ll_arm(Arm), pos(X, Y)],
  [
    add(moving_arm(Arm, X, Y)),
    del(ll_arm_at(Arm, X, Y))
  ]
).
\end{codebox}
\begin{codebox}{prolog}{Mapping example}
mapping(move_table_to_table_start(Agent, Block, X1, Y1, X2, Y2),
  [
    move_arm_start(Agent, X1, Y1),
    move_arm_end(Agent, X1, Y1),
    grip_start(Agent),
    grip_end(Agent),
    move_arm_start(Agent, X2, Y2),
    move_arm_end(Agent, X2, Y2),
    release_start(Agent),
    release_end(Agent)
  ]
).
\end{codebox}
\end{center}

Again, the user can correct possible errors (or anyway refine the \kb) and then move on to the planning phase.


\section{Plan Generation}\label{sec:plangen}
In this section, we describe how the framework uses the information from the \kb to generate a task plan for multiple agents.
Generation takes place in three steps: 
\begin{enumerate*}
    \item Generation of a total-order (TO) plan, 
    \item extraction of a partial-order (PO) plan and of the resources, 
    \item solution of a MILP problem to improve resource allocation and reducing the plan makespan by exploiting the possible parallel executions of actions.
\end{enumerate*}



\subsection{Total-Order Plan Generation}\label{ssec:toplangen}
A total-order plan is a strictly sequential list of actions that drives the system from the initial to the goal state. 
The algorithm used to extract a total-order plan is shown in~\autoref{alg:toplanning} and consists of two distinct steps:
\begin{itemize}
    \item identify a total-order plan for high-level actions, and
    \item recursively map each high-level action to a sequence of actions with a lower level until they are mapped to actions corresponding to the APIs of the available robotic resources.
\end{itemize}

\begin{algorithm}
\footnotesize
\caption{Algorithm generating a TO plan with mappings}\label{alg:toplanning}
\KwData{$TP=(F, DA, I, G, K)$}
\KwResult{Plan solving TP}

\DontPrintSemicolon

\SetKwProg{plan}{TO\_PLAN}{}{}
\SetKwProg{map}{APPLY\_MAP}{}{}
\SetKwProg{action}{APPLY\_ACTION}{}{}
\SetKwProg{maps}{APPLY\_MAPPINGS}{}{}

\SetKwInOut{Input} {In}
\SetKwInOut{Output}{Out}

\plan{(S, P)}{
  \Input{The current state $S$ and the current plan $P$}
  \Output{The final plan}
  \If{$S \neq G$}{
      select\_action($a_i$)\;
      (US, UP) $\gets$ APPLY\_ACTION($a_i$, S, P)\;
      P $\gets$ TO\_PLAN(US, UP)\;
  }
  (US, UP) $\gets$ APPLY\_MAPPINGS(S,P)\;
  \KwRet{P}\;
}

\maps{(S, P)}{
  \Input{The current state $S$ and the current plan $P$}
  \Output{The updated state $US$ and plan $UP$ after the mappings}
  US, UP $\gets$ S, P\;
  \ForEach{$a_i \in P$}{
    \If{\textnormal{is\_start($a_i$) $\wedge$ has\_mapping($a_i$)}}{
      (US, UP) $\gets$ APPLY\_MAP($a_i$, \textnormal{US}, UP)\;
    }
  }
  \KwRet{(US, UP}\;
}

\map{($a$, S, P)}{
  \Input{The action $a$, the current state $S$ and the current plan $P$}
  \Output{The updated state $US$ and plan $UP$ after the mappings}
  M $\gets$ mapping($a$)\;
  \ForEach{$a_i \in M$}{
    (US, UP) $\gets$ APPLY\_ACTION($a$, S, P)\;
  }
  \KwRet{(US, UP)}\;
}

\action{($a, S, P$)}{
  \Input{The action $a$, the current state $S$ and the current plan $P$}
  \Output{The updated state $US$ and plan $UP$ after applying the effects of $a$}
  \eIf{\textnormal{is\_applicable($a_i$)}}{
    US $\gets$ change\_state($a_i$.eff, S)\;
    UP $\gets$ plan\_action($a_i$, P)\;
    \KwRet{(US, UP)}\;
  }{
    \KwRet{(S, P)}
  }
}
\end{algorithm}

This enables the extraction of total-order plans that are consistent with the \kb provided, and we reduce the computational cost of checking all the possible actions at each time step. The \texttt{TO\_PLAN} function is the main function, which takes the initial and final states, and it inspects which actions can be executed given the current state. The \texttt{select\_action} function selects the next action from the set of possible actions. This search is based on the Prolog inference engine, which tries the actions in the order in which they are written in the KB, and hence it is not an informed search. 

The algorithm then moves to the \texttt{APPLY\_ACTION} function, which first checks if the chosen action's preconditions are met in the current state and, if they are, then it applies its effects changing the state (\texttt{change\_state}) and adding the action to the plan (\texttt{plan\_action}). It continues until the current state satisfies the goal state. Whenever the search reaches a fail point, we exploit the Prolog algorithm of resolution to step back and explore alternative possibilities.

Once the algorithm has extracted a high-level total-order plan, it applies the mappings. To do so, it iterates over the actions in the plan, and for each action it checks if it is a start action ($a_\vdash$) and if there are mappings for it. If this is the case, it calls the function \texttt{APPLY\_MAP}, which sequentially applies the actions in the mapping to the current state, also adding the actions to the plan. Notice that to do so, we call the \texttt{APPLY\_ACTION} function, which checks the preconditions of the actions w.r.t. the current state, ensuring that the lower-level actions can actually be applied.
% Also, the functions recursively check if any action from the mapped action has a mapping on its own, ensuring that all the actions have a direct grounding to APIs.

The total-order plan $TO$ extracted from this function is a list of actions that are executed in sequence:
\begin{equation*}
    \forall i \in \{0,\hdots \vert TO\vert-1\}~t(a_i)<t(a_{i+1})
\end{equation*}

\subsubsection{Running Example -- Total-Order Plan}
\label{sssec::runegTOPlan}

Let us consider again the \kb that we generated in Section~\ref{sssec:runegKMS}. Let us now see how \frameworkname extracts the TO plan.

The algorithm starts from the initial state and from the first action in the \kb, which in this case is the one shown in Section~\ref{sssec:runegKMS}. The algorithm takes the grounding predicates in this case:

\begin{minted}[fontsize=\footnotesize]{prolog}
agent(Agent), pos(X1, Y1), pos(X2, Y2), block(Block)
\end{minted}

and checks whether there is an assignment of predicates from the \kbase that satisfies them. For example, the predicate \verb|pos(1,1)| satisfies \verb|pos(X1,Y1)|. Not only this, but since the predicates in this list are grounded w.r.t. the \kb, one can also check some conditions. For example, if we were to assign the values to the previous predicates, it can happen that \verb|X1 = X2| and \verb|Y1 = Y2|, which is useless for an action that moves a block from one position to another. By adding the following predicates, we can ensure that the values are different:

\begin{minted}[fontsize=\footnotesize]{prolog}
agent(Agent), pos(X1, Y1), pos(X2, Y2), block(Block), X1\=X2, Y1\=Y2
\end{minted}

Once an assignment for the predicates inside the grounding list is found, the algorithm checks whether the predicates inside the preconditions are satisfied. Let us consider the preconditions for the \verb|move_table_to_table_start| action from Section~\ref{sssec:runegKMS}:

\begin{minted}[fontsize=\footnotesize]{prolog}
% Positive predicates
[ontable(Block), at(Block, X1, Y1), available(Agent), clear(Block)],
% Negative predicates
[
  at(_, X2, Y2), on(Block, _), moving_table_to_table(_, Block, _, _, _,_), 
  moving_table_to_block(_, Block, _, _, _, _, _)
]
\end{minted}

After the first grounding step, they become the following:

\begin{minted}[fontsize=\footnotesize]{prolog}
% Positive predicates
[ontable(b1), at(b1, 0, 0), available(a1), clear(b1)],
% Negative predicates
[
  at(_, 0, 0), on(b1, _), moving_table_to_table(_, b1, _, _, _,_), 
  moving_table_to_block(_, b1, _, _, _, _, _)
]
\end{minted}

The algorithm checks whether the predicates from the first list are satisfied in the current state and whether the predicates from the second list are not present in the current state. Comparing them with the initial state as shown in Section~\ref{sssec:runegKMS}, we can see that \verb|ontable(b1)| is present, but \verb|at(b1, 0, 0)|, so this combination of predicates would already be discarded. The first grounding that is accepted is that in which \verb | Block = b1, X1 = 1, Y1 = 1, Agent = a1 |. Notice that the predicates that start with \verb|_| mean "any", e.g., the predicate \verb|at(_, 0, 0)| checks if there is any predicate with name \verb|at| and arity 3 that has the last two arguments set to 0, regardless of what the first argument is.

By checking the different combinations of actions, the planner can extract a \HL TO plan. In this case, it would be something like this:

\begin{minted}[fontsize=\footnotesize]{text}
[0] move_table_to_table_start(a1, b1, 1, 1, 2, 2)
[1] move_table_to_table_end(a1, b1, 1, 1, 2, 2)
[2] move_table_to_block_start(a1, b2, 3, 1, 2, 2)
[3] move_table_to_block_end(a1, b2, 3, 1, 2, 2)
\end{minted}

At this point, the algorithm takes the mappings and it applies them to the previous plan. For instance, from Section~\ref{sssec:runegKMS} we saw that the mapping for \verb|move_table_to_table_start| is:
\begin{minted}[fontsize=\footnotesize]{prolog}
mapping(move_table_to_table_start(Agent, Block, X1, Y1, X2, Y2),
  [
    move_arm_start(Agent, X1, Y1), move_arm_end(Agent, X1, Y1),
    grip_start(Agent), grip_end(Agent),
    move_arm_start(Agent, X2, Y2), move_arm_end(Agent, X2, Y2),
    release_start(Agent), release_end(Agent)
  ]
).
\end{minted}

Hence, we would change the previous plan with:

\begin{minted}[fontsize=\footnotesize]{text}
[0] move_table_to_table_start(a1, b1, 1, 1, 2, 2)
[1] move_arm_start(a1, 1, 1)
[2] move_arm_end(a1, 1, 1)
[3] grip_start(a1)
[4] grip_end(a1)
[5] move_arm_start(a1, 2, 2)
[6] move_arm_end(a1, 2, 2)
[7] release_start(a1)
[8] release_end(a1)
[9] move_table_to_table_end(a1, b1, 1, 1, 2, 2)
[10] move_table_to_block_start(a1, b2, 3, 1, 2, 2)
[11] move_arm_start(a3, 3, 1)
[12] move_arm_end(a1, 3, 1)
[13] grip_start(a1)
[14] grip_end(a1)
[15] move_arm_start(a1, 2, 2)
[16] move_arm_end(a1, 2, 2)
[17] release_start(a1)
[18] release_end(a1)
[19] move_table_to_block_end(a1, b2, 3, 1, 2, 2)
\end{minted}


\subsection{Partial-Order Plan Generation}\label{ssec:poplangen}
The next step is to analyse the total-order plan in search of all possible causal relationships. This is done by
looking for actions that enable other actions (enablers). In addition, we extract all the resources that can be allocated
and used for the execution of the task. This step will be important for the next phase of the planning process, the MILP problem, in which 
the resources will be re-allocated allowing for shrinking the makespan of the plan.
%
In this work, the only resource considered is the robotic agent, but this limitation could easily be removed by modifying the \kb.  To this end,  we define a special predicate, named \texttt{resource/1}, that allows us to specify the resources.

Given an action $a_i$, another action $a_j$ is an enabler of $a_i$ if it either adds a literal $l$ satisfying one or more preconditions of $a_i$, or it removes a fluent violating one or more preconditions of $a_i$, and if $a_i$ happens after $a_j$: 

\begin{equation}
\small
\begin{array}{rl}
     a_j \in \ach{a_i} \iff & t(a_i) > t(a_j) \wedge \\
                            & ((l\in \pc{a_i}~ \wedge add(l)\in \eff{a_j}) \vee\\
                            & \,\,(\lnot l\in \pc{a_i} \wedge del(l)\in \eff{a_j}))
\end{array}
\label{eq:enablers}
\end{equation}

It is important to note that we consider an action $a_j\notin\ach{a_i}$ if there is at least a fluent $l$ that is not a resource. If all the fluents and their arguments that would make $a_j$ an enabler of $a_i$ are resources, then $a_j$ is not considered an enabler, as this relationship depends on the assignment of the resources, which comes with the optimisation step. 

Besides the enablers added corresponding to the classical definition, we also enforce the following precedence constraints:
\begin{itemize}
    \item When we expand a mapping $m(\alpha_i)$ of a high-level durative action $\alpha_i$ and reach the ending action $\aEnd{\alpha_i}$, then we add all previous durative actions as enablers until the corresponding start action. For example, assume that $m(\alpha_i)=\{\alpha_j, \alpha_k\}$, this means that the total-order plan will be the sequence $\{\aStart{\alpha_i}, \aStart{\alpha_j}, \aEnd{\alpha_j}, \aStart{\alpha_k}, \aEnd{\alpha_k}, \aEnd{\alpha_i}\}$. It follows that $\aStart{\alpha_i}$ is an enabler of $\aEnd{\alpha_i}$, but also all intermediate actions are part of the set of its enablers as they must be completed in order for $\alpha_i$ to end.
    \begin{equation}
        \bigwedge_{a\in m(\alpha_i)} a\in \ach{\aEnd{\alpha_i}}.
        \label{eq:constraint5}
    \end{equation}
    \item When we expand a mapping, all actions in the mapping must have the start of the higher-level action as one of the enablers. For instance, after the previous example, $\aStart{\alpha_j}, \aEnd{\alpha_j}, \aStart{\alpha_k}, \aEnd{\alpha_k}$ have $\aStart{\alpha_i}$ as an enabler.
    \begin{equation}
        \bigwedge_{a\in m(\alpha_i)} \aStart{\alpha} \in \ach{a_i}.
        \label{eq:constraint4}
    \end{equation}
\end{itemize}

% We then create a graph from which we can extract partial-order plans. To do this, after having obtained a plan from the \texttt{TO\_PLAN} from~\autoref{alg:planning}, we look for the achievers of the actions as shown in~\autoref{alg:po_planning}. 

The algorithm that manages this extraction is shown in~\autoref{alg:poplanning}. For ease of reading, we define $R\subseteq F$ as the set of fluents that are resources.

The algorithm \texttt{FIND\_ENABLERS} takes the total-order plan and, starting with the first action in the plan, it extracts all the causal relationships between the actions. The auxiliary function \texttt{IS\_ENABLER} tests whether an action $a_j$ is an enabler of an action $a_i$ by checking the properties of~\autoref{eq:enablers} plus the precedence constraints just described. Finally, notice that the literal checked to be present (absent) in both additive (subtractive) effects must not contain arguments that are part of the resources $R$. For example, consider the case in which an action $a_i$ needs the precondition $l(x_1, x_2, x_3)$ and $a_j$ provides the predicate, then if at least one of $x_1, x_2, x_3$ is in $R$, $a_j$ is an enabler of $a_i$, otherwise it is not. This ensures that only causal relationships that do not depend on the resources are extracted at this time. The precedence of the resources will be defined and discussed in Section~\ref{ssec:poplanopt}. 

\begin{algorithm}[htp]
\footnotesize
\caption{Algorithm extracting the actions enablers and the resources}
\label{alg:poplanning}
\KwData{$TP=(F, DA, I, G, K)$}
\KwResult{Enablers and resources $R$}

\DontPrintSemicolon

\SetKwProg{findenablers}{FIND\_ENABLERS}{}{}
\SetKwProg{isenabler}{IS\_ENABLER}{}{}
\SetKwProg{findresources}{EXTRACT\_RESOURCES}{}{}

\SetKwInOut{Input} {In}
\SetKwInOut{Output}{Out}

\findenablers{$(\tn{TO\_P}, a_i)$}{
  \Input{The total-order plan TO\_P, the $i$th action}
  \Output{The enablers $E$ for all the actions in the plan}

  \For{$a_j \in \tn{TO\_P}, a_j\neq a_i$}{
    \uIf{$\tn{IS\_ENABLER}(a_j, a_i)$}{
      $E[a_i].add(a_j)$;
    }
  }

  \If{$a_i\neq \tn{TO\_P}.back()$}{
    $E \gets \tn{FIND\_ENABLERS}(\tn{TO\_P}, a_{i+1})$\;
  }
  \KwRet{E}\;
}

\isenabler{$(a_j, a_i)$}{
  \Input{The action $a_j$ to test if it's enabler of $a_i$}
  \Output{True if $a_j$ is enabler of $a_i$}

  \ForEach{$e \in \eff{a_j}$}{
    \uIf{$\left(e=\tn{add}(l) \wedge l\in\pc{a_i})\right)$ OR
         $~\left(e=\tn{del}(l) \wedge \lnot l\in\pc{a_i}\right)$ OR
         $~\left(\tn{isStart}(a_j) \wedge a_i \in m(a_j)\right)$ OR\\
         $~~\left(\tn{isEnd}(a_j) \wedge a_i \in m(a_j)\right)$}
    {
      $X\gets \tn{set of arguments of }e$; 
        
      \uIf{$\not\exists x \in X | x \in R$}{
        \KwRet{True};
      }
    }
  }
  \KwRet{False};
}

\findresources{$()$}{
  \output{A list of resources}
  findall(X, resources(X), AllResources)\;
  $R$ = make\_set(AllResources)\;
  \KwRet{$R$}\;
}

\end{algorithm}

\subsubsection{Running Example -- Partial-Order Plan}
\label{sssec:PORunEx}

Once we have applied the mappings as before, we have the full TO plan. We want to extract information from this, which will then be exploited to improve the plan for multiple agents. This is done by examining all the actions and checking which are their enablers. For instance, the 10th action, \verb|move_table_to_block_start(a1, b2, 3, 1, 2, 2)|, has as a precondition the following predicate \verb|clear(Block2), Block2=b1|, which is true only when the 9th action has applied its effects. Since \verb|b1| is not part of the resources, the algorithm will state that $a_9$ is an enabler of $a_{10}$. 

If the second move were to move a block to another position on the table, hence independent of the first move, then the algorithm would not set $a_9$ as an enabler of $a_{10}$, as the only reason it may do so is if the same agent is used, but this is known only later.

After this step, we know the enablers for the actions (shown in squared brackets in the list below):

\begin{minted}[fontsize=\footnotesize]{text}
[0] init()[]
[1] move_table_to_table_start(a1, b1, 1, 1, 2, 2), [0]
[2] move_arm_start(a1, 1, 1), [0,1]
[3] move_arm_end(a1, 1, 1), [0,1,2]
[4] grip_start(a1), [0,1,2,3]
[5] grip_end(a1), [0,1,2,3,4]
[6] move_arm_start(a1, 2, 2), [0,1,2,3,4,5]
[7] move_arm_end(a1, 2, 2), [0,1,2,3,4,5,6]
[8] release_start(a1), [0,1,2,3,4,5,6,7]
[9] release_end(a1), [0,1,2,3,4,5,6,7,8]
[10] move_table_to_table_end(a1, b1, 1, 1, 2, 2), [0,1,2,3,4,5,6,7,8,9]
[11] move_table_to_block_start(a1, b2, 3, 1, 2, 2), [0,10]
[12] move_arm_start(a1, 3, 1), [0,11]
[13] move_arm_end(a1, 3, 1), [0,11,12]
[14] grip_start(a1), [0,11,12,13]
[15] grip_end(a1), [0,11,12,13,14]
[16] move_arm_start(a1, 2, 2), [0,11,12,13,14,15]
[17] move_arm_end(a1, 2, 2), [0,11,12,13,14,15,16]
[18] release_start(a1), [0,11,12,13,14,15,16,17]
[19] release_end(a1), [0,11,12,13,14,15,16,17,18]
[20] move_table_to_block_end(a1, b2, 3, 1, 2, 2), [0,10,11,12,13,14,15,16,17,18,19]
[21] end(), [0,1,2,3,4,5,6,7,8,9,10,11,12,13,14,15,16,17,18,19,20]
\end{minted}

From this we could already notice that all the actions will be carried out in sequence. We also see that in this step we add two fictitious actions, \verb|init| and \verb|end|. This simply represents the start and the end of the plan, respectively. \verb|init| is an enabler of all the actions in the plan and \verb|end| has all the other actions as enablers, which means that the plan can be considered finished only when all the actions have been executed.

As for the resources, we first extract all the possible resources by looking at the predicates \verb|resource(X)| in the \kb, as shown in Section~\ref{sssec:runegKMS}. Then we assign the type of resources used to each action by checking action per action which resources they are using. This is useful because it will provide MILP with the basis to correctly allocate the different resources to the actions.

\begin{minted}[fontsize=\footnotesize]{text}
Resources:
[0] agent-2
Resources list:
[0] agent-[agent(a1),agent(a2)]
Resources required by action:
[4] 6-[agent]
[9] 1-[agent]
\end{minted}


\subsection{Partial-Order Plan Optimization}\label{ssec:poplanopt}
The last part of the planning module, shown in~\autoref{fig:arch_LLM_pKB}, is the optimisation module which allows for shrinking the plan by scheduling the task (temporal plan) and allocating the resources. In order to do this, we instantiate a MILP problem, the solution of which must satisfy constraints ensuring that we are not violating precedence relationships and invalidating the obtained planned. 

We start by taking the work from~\cite{cimatti_strong_2015}, in which the authors describe how it is possible to obtain a plan with lower makespan by reordering some tasks. In particular, we adopt the following concepts from~\cite{cimatti_strong_2015}:
\begin{itemize}
    \item Let $f(l)=\{a\in DA \vert l\in \eff{a}\}$ be the set of actions that achieve a literal $l$, and 
    \item let $\displaystyle p(l,a,r)\doteq a<r \wedge \bigwedge_{a_i\in f(l)\setminus\{a,r\}}(a_i<a\vee a_i>r)$ be the temporal constraint stating which is the last achiever $a$ of an action $r$ for a literal $l$. 
\end{itemize}
The constraints that must hold are the following:
\begin{equation}
    \label{eq:constraint1_old}
    %\footnotesize
    \bigvee_{a_j\in f(l)\setminus\{a\}} p(l,a_j,a).
\end{equation}
Which states that at least an action with effect $l$ should occur before $a$.
\begin{equation}
    \label{eq:constraint2}
    %\footnotesize
    \bigwedge_{a_j\in f(l)} \left(p(l,a_j,a) \rightarrow \bigwedge_{a_t\in f(\lnot l)\setminus\{a\}}(a_t<a_j \vee a_t>a)\right).
\end{equation}
\begin{equation}
    \label{eq:constraint3}
    %\footnotesize
    \bigwedge_{a_j\in f(\lnot l)\wedge l\in \pc{a}} ((a_j<\aStart{a}) \vee (a_j>\aEnd{a})).
\end{equation}
Which state that between the last achiever $a_j$ of a literal $l$ for an action $a$ and the action $a$ there must not be an action $a_t$ negating said literal. This condition is also enforced by~\autoref{eq:constraint3} that constrains actions negating the literal to happen before the action $a$ has started or after it has finished.

Notice though that in this work, the authors have considered achievers and not enablers. The difference is that an action $a_j$ is an achiever of $a_i$ if $a_j$ \emph{adds} a fluent $l$ that is needed by $a_j$. Enablers instead consider the case in which fluents are also removed. 
%
Since these constraints only consider achievers and not enablers, we need to extend them. We redefine the previous as:
\begin{itemize}
    \item let $f(l)=\{a\in DA \vert add(l)\in \eff{a}\}$ be the set of actions that achieve a literal $l$, and 
    \item let $f(\lnot l)=\{a\in DA \vert del(l) \in \eff{a}\}$ be the set of actions that delete a literal $l$, and
    \item let $F(l) = f(l)\cup f(\lnot l)$ be the union set of $f(l)$ and $f(\lnot l)$, and
    \item let $\displaystyle p(l,a,r)\doteq a<r \wedge \bigwedge_{a_i\in F(l) \setminus\{a,r\}}(a_i<a\vee a_i>r)$ be the last enabler $a$ of an action $r$ for a literal $l$. 
\end{itemize}
Consequently, we need to:
\begin{itemize}
    \item revise~\autoref{eq:constraint1_old} to include all enablers:
        \begin{equation}
            \label{eq:constraint1}
            %\footnotesize
            \bigvee_{a_j\in F(l)\setminus\{a\}} p(l,a_j,a).
        \end{equation}
    \item add two constraints similar to~\autoref{eq:constraint2} and~\autoref{eq:constraint3} to ensure that a predicate that was removed is not added again before the execution of the action:
    \begin{equation}
        \label{eq:constraint2_1}
        %\footnotesize
        \bigwedge_{a_j\in f(\lnot l)} \left(p(l,a_j,a) \rightarrow \bigwedge_{a_t\in f(l)\setminus\{a\}}(a_t<a_j \vee a_t>a)\right).
    \end{equation}
    \begin{equation}
        \label{eq:constraint3_1}
        %\footnotesize
        \bigwedge_{a_j\in f(l)\wedge (\lnot l)\in \pc{a}} ((a_j<\aStart{a}) \vee (a_j>\aEnd{a})).
    \end{equation}    
\end{itemize}

The second aspect of the MILP problem concerns resource allocation. Indeed, as stated before, there are some predicates that are parameterised on resources, e.g., \texttt{available(A)} states whether an agent \texttt{A} is available or not, but it does not ground the value of \texttt{A}. %
One possibility would be to allocate the resources using Prolog, as done in~\cite{saccon2023prolog}, but this choice is greedy since Prolog grounds information with the first predicate that satisfy \texttt{A}. To reduce the makespan of the plan and improve the quality of the same, we delay the grounding to an optimisation phase, leaving Prolog to capture the relationships between actions.

As a first step, we are also going to assume that all the actions coming from a mapping of a higher-level action and that are not mapped into lower-level actions shall maintain the same parameterised predicates as the higher-level action. So the constraint in~\autoref{eq:constraint6} must hold.
\begin{equation}
    \label{eq:constraint6}
    \bigwedge_{a_j\in m(a_i) \wedge m(a_j)\notin M} \left(\bigwedge_{p(x_i) \in \pc{a_i} \wedge p(x_j) \in \pc{a_j}} x_i=x_j \right).
\end{equation}
Moreover, for these constraints, we will consider only predicates that are part of the set $K$, that is predicates that are not resources $R\cap K=\emptyset$.

The objective now is three-fold: 
\begin{itemize}
    \item identify a cost function,
    \item summarise the previous constraints, and
    \item construct a MILP problem to be solved.
\end{itemize}

In this work, the first point is straightforward: we want to minimise the makespan, i.e., the total duration required to complete all tasks or activities.

For the second point, we are trying to find a way to put the previous constraints,~\cref{eq:constraint1,eq:constraint2,eq:constraint2_1,eq:constraint3,eq:constraint3_1,eq:constraint4,eq:constraint5,eq:constraint6} in a compact formulation or structure. We opted to extract the information regarding the enablers using Prolog and to place it into a $N\times N$ matrix $C$, where $N$ is the number of actions and each cell $C_{ij}$ is $1$ if $a_i$ is an enabler of $a_j$ (without considering resources), 0 otherwise. 

We now need to address the resource allocation aspect, specifically, how to distribute the available resources $R$ among the various actions. When performing this task, there are primarily two factors to consider:
\begin{itemize}
    \item A resource cannot be utilised for multiple actions simultaneously.
    \item If two actions share the same resource, they must occur sequentially, meaning one action enables the other.
\end{itemize}

For the first factor, we need to make sure that, for each resource type $r\in R$, the number of actions using the resource at the same time must not be higher than the number of resources of that type available, as shown in~\autoref{eq:resAllocation}.
\begin{equation}
    \displaystyle\forall t \in\{t_0, t_{\tn{END}}\},\,\vert r\vert \geq\sum_{a_i\in TO} t\in\{\aStart{a_i}, \aEnd{a_i}\} \wedge \left( \exists~l(\pmb{x})\in \pc{a_i}\vert r\in\pmb{x}\right).
    \label{eq:resAllocation}
\end{equation}

The second factor must instead be merged with also the precedence constraints embedded in $C$. In particular, we want to express that actions $a_i, a_j$ are in a casual relationship if $C_{ij}=1$ or if they share the same resource. This can be expressed with the following constraint: 
\begin{equation}
    C_{ij} \vee \exists r\in R : r\in\fl{a_i} \wedge r\in\fl{a_j}
    \label{eq:precedence}
\end{equation}
Note that $\fl{a}$ was defined in the problem definition paragraph and represents the set of variables and literals used by the predicates in the preconditions of $a$. 

Finally, we need to set up the MILP problem that consists in finding an assignment of the parameters, of the actions' duration and of the causal relationships, such that the depth of the graph $\mathcal{G}$ representing the plan is minimised. This problem can be expressed as shown in~\autoref{eq:optimization_1}.

%\begin{figure*}[h]
%    \centering
    \begin{equation}
    \everymath={\displaystyle}
    \begin{array}{r@{\hspace*{8mm}}l}
        \label{eq:optimization_1}
        \min_{\mathcal{P}, \mathcal{T}} & t_{\tn{END}} \\
        %&\\
        \textrm{s.t.}   & C_{ij} \vee \exists r\in R : r\in\fl{a_i} \wedge r\in\fl{a_j}, \\
                              & \quad \quad \forall t \in\{t_0, t_{\tn{END}}\}, \\
                              & \quad \quad \quad \quad \vert r\vert > \!\!\sum_{a_i\in TO} \left(t\in\{\aStart{a_i}, \aEnd{a_i}\} \wedge \exists~l(\pmb{x})\in \pc{a_i}\vert r\in\pmb{x}\right).\\
    \end{array}
    \end{equation}
%\end{figure*}

As mentioned before, the MILP part is implemented in Python3 using OR-Tools from Google. The program also checks the consistency of the PO matrix $C$, by making sure that all the actions must have a path to the final actions. 
The output of the MILP solution is basically an STN, which describes both the causal relationship between the actions and also the intervals around the duration of the actions. The initial and final nodes of the STN are factitious as they do not correspond to actual actions, but they simply represent the start and the end of the plan.
The STN is extracted by considering the causal relationship from the $C$ matrix taken as input, and by adding the causal relationship given by the resource allocation task. 
Once we have the STN, we can extract a \bt, which can then be directly executed by integrating it in ROS2. 

\subsubsection{Plan Optimization -- Example}
\label{sssec:PORunExample}
As we said at the end of~\autoref{sssec:PORunEx}  on the running example, that particular plan is not optimisable as the actions are executed in sequence. Let's then consider a slight modification, which consists in finding a plan to move the two blocks in two new positions instead of stacking them in one position. We also have a new agent that can be used to carry out part of the work. 
Our new plan and actions' enablers are the following one:

\begin{minted}[fontsize=\footnotesize]{text}
[0] init()[]
[1] move_table_to_table_start(a1, b1, 1, 1, 1, 2), [0]
[2] move_arm_start(a1, 1, 1), [0,1]
[3] move_arm_end(a1, 1, 1), [0,1,2]
[4] grip_start(a1), [0,1,2,3]
[5] grip_end(a1), [0,1,2,3,4]
[6] move_arm_start(a1, 1, 2), [0,1,2,3,4,5]
[7] move_arm_end(a1, 1, 2), [0,1,2,3,4,5,6]
[8] release_start(a1), [0,1,2,3,4,5,6,7]
[9] release_end(a1), [0,1,2,3,4,5,6,7,8]
[10] move_table_to_table_end(a1, b1, 1, 1, 1, 2), [0,1,2,3,4,5,6,7,8,9]
[11] move_table_to_table_start(a1, b2, 3, 1, 3, 2), [0,10]
[12] move_arm_start(a1, 3, 1), [0,11]
[13] move_arm_end(a1, 3, 1), [0,11,12]
[14] grip_start(a1), [0,11,12,13]
[15] grip_end(a1), [0,11,12,13,14]
[16] move_arm_start(a1, 3, 2), [0,11,12,13,14,15]
[17] move_arm_end(a1, 3, 2), [0,11,12,13,14,15,16]
[18] release_start(a1), [0,11,12,13,14,15,16,17]
[19] release_end(a1), [0,11,12,13,14,15,16,17,18]
[20] move_table_to_table_end(a1, b2, 3, 1, 3, 2), [0,10,11,12,13,14,15,16,17,18,19]
[21] end(), [0,1,2,3,4,5,6,7,8,9,10,11,12,13,14,15,16,17,18,19,20]
\end{minted}

Indeed, action $a_9$ may or may not be an enabler of action $a_{10}$ depending on the resource allocation of the MILP solution. If we have just one agent, then $a_9\in\ach{a_{10}}$, if instead we have more than one agent, then $a_9\not\in\ach{a_{10}}$ and the two actions can be executed at the same time and the plan would be:

\begin{minted}[fontsize=\footnotesize]{text}
[0] init()
[1] move_table_to_table_start(a1, b1, 1, 1, 1, 2)
[2] move_arm_start(a1, 1, 1)
[3] move_arm_end(a1, 1, 1)
[4] grip_start(a1)
[5] grip_end(a1)
[6] move_arm_start(a1, 1, 2)
[7] move_arm_end(a1, 1, 2)
[8] release_start(a1)
[9] release_end(a1)
[10] move_table_to_table_end(a1, b1, 1, 1, 1, 2)
[11] move_table_to_block_start(a2, b2, 3, 1, 3, 2)
[12] move_arm_start(a2, 3, 1)
[13] move_arm_end(a2, 3, 1)
[14] grip_start(a2)
[15] grip_end(a2)
[16] move_arm_start(a2, 3, 2)
[17] move_arm_end(a2, 3, 2)
[18] release_start(a2)
[19] release_end(a2)
[20] move_table_to_block_end(a2, b2, 3, 1, 3, 2)
[21] end()
\end{minted}

% \enrcom{Should I also include a figure? MR: I do not think so!}

% \subsubsection{Plan Generation - Example}

\section{\Btree Generation and Execution}\label{sec:bt}
\newcommand{\seq}[0]{\protect\writings{\texttt{SEQUENCE}}}
\newcommand{\parr}[0]{\protect\writings{\texttt{PARALLEL}}}

% In this section, we first introduce how to convert from a STN to a \bt, and then we provide some details regarding the implementation. 

% \subsubsection{\bt Generation}\label{sssec:btgen}

The conversion from STN to \bt is taken from~\cite{roveriSTNtoBT}. We summarize it here and refer the reader to the main article. 

An STN is a graph with a source and a sink, which can be artificial nodes in the sense that they represent the start and the end of the plan. Each node can have multiple parent and multiple children. Having multiple parents implies that the node cannot be executed as long as all the parents haven not finished and, whereas, having multiple children implies that they will be executed in parallel. 

With this knowledge we can extract a \btree, which is a structure that, starting from the root, ticks all the nodes in the tree until it finishes the last leaf. Nodes in the tree can be of different types:
\begin{itemize}
    \item \emph{action}: they are an action that has to be executed;
    \item \emph{control}: they can be either \seq or \parr and state how the children nodes must be executed;
    \item \emph{condition}: they check whether a condition is correct or not;
\end{itemize}
The ticking of a node means that the node is asked to do its function, e.g., if a \seq node is ticked, then it will tick the children one at a time, while if a condition node is ticked, it will make sure that the condition is satisfied before continuing with the next tick. 

The algorithm %(Algorithm~\ref{alg:stntobt})
to convert the STN to a \bt starts from the fictitious initial node (\verb|init|), and for every node it checks:
\begin{itemize}
    \item The number of children: if there is only one child, then it is a \seq node, otherwise it is a \parr node. 
    \item The number of parents: if there are more than one parents then the node must wait for all the parents to have ticked, before being executed.
    \item The type of the action: if it is a low-level action, then it is inserted into the \bt for execution, otherwise it will not be included.
\end{itemize}

%%%%%%%%%%%%%%%%%%%%%%%%%%%%%%%%%%%%%%%%%

% \begin{algorithm}
% \caption{Algorithm extracting a \btree from an STN.}
% \label{alg:stntobt}
% \KwData{The STN $G$}
% \KwResult{\bt corresponding to the STN}

% \DontPrintSemicolon

% \SetKwProg{extractBT}{EXTRACT\_BT}{}{}

% \SetKwInOut{Input} {In}
% \SetKwInOut{Output}{Out}

% \extractBT{(G)}{
%   \Input{The STN $G$ to convert}
%   \Output{The \bt $\mathcal{T}$}
%   \KwRet{$\mathcal{T}$}\;
% }

% \end{algorithm}

%%%%%%%%%%%%%%%%%%%%%%%%%%%%%%%%%%%%%%%%%

% THIS has been moved to the Implementation details section of Experimental Validation
% \subsubsection{\bt Execution}\label{sssec:btexec}

% As said, the execution of a \btree starts from the root and it gradually ticks the different nodes of the tree until all nodes have been ticked. 

% While \bts have become a de facto standard for executing robotic tasks, no universally accepted framework exists for their creation or execution. Some notable examples include PlanSys2~\cite{martin2021plansys2} and BehaviorTree.CPP~\cite{BehaviorTreeCppWebsite}. PlanSys2 is tightly integrated with ROS2; beyond merely executing \btrees, it can also derive feasible plans from a knowledge base. In contrast, BehaviorTree.CPP is a more general framework that enables the creation and execution of \bts from an XML file. We selected BehaviorTree.CPP since our main objective was to execute APIs from a \bt, which is easily represented using an XML file, while also maintaining maximum generality. Nevertheless, BehaviorTree.CPP also offers a ROS2 wrapper, which can easily be integrated with the flow.

% \enrcom{Maybe it should be moved to Section~\ref{ssec:implementation}?}


\section{Experimental Setup}
\label{sec:exp_setup}
\subsection{Datasets}
\label{sec:exp_setup:datasets}

We evaluate \method{} on three multimodal humor datasets (see examples in Appendix \ref{app:dataset-eg}):

\paragraph{MemeCap \cite{hwang-shwartz-2023-memecap}.} Each instance includes a meme paired with a title (social media post to which the meme was attached). The task is to generate a brief explanation, compared against multiple reference explanations. The task requires interpreting visual metaphors in relation to the text, where models can benefit from reasoning about background knowledge. 
% , and reference captions in the test images can contain up to four captions. 
% includes 5.8k training and validation samples, along with 559 test samples. 

\paragraph{New Yorker Cartoon \cite{hessel-etal-2023-androids}.} We focus on the explanation generation task: given a New Yorker cartoon and its caption, generate an explanation for why the caption is funny given the cartoon, requiring an understanding of the scene, caption, and commonsense and world knowledge.  
%indirect and playful meanings tied to human experience and culture.
% It includes 2.3k training samples, 130 validation samples, and 130 test samples, and offers five different splits. 

\paragraph{YesBut \cite{nandy-etal-2024-yesbut}.} 
%dataset consists of 1k satirical and 1k non-satirical test samples, with our focus on the satirical test set. 
Each instance contains an image with two parts captioned ``yes'' and ``but''. The task is to explain why the image is funny or satirical.
% requiring an understanding of commonsense knowledge, social norms, and cultural references related to everyday objects and situations.

Since our method is unsupervised, we use the test set portions of these datasets. Due to resource and cost constraints, we don't evaluate our method on the full test sets. Instead, from each dataset, we randomly sample 100 test instances. We repeat the process three times using different random seeds to obtain three test splits and report average performance and standard deviation.

\subsection{Models}
\label{sec:exp_setup_models}

We test our method with two closed-source and two open-source VLMs.
\paragraph{GPT-4o \cite{hurst2024gpt}} is an advanced, closed-source multimodal model processing text, audio, images, and video and generating text, audio, and images. It matches GPT-4's performance in English text tasks with improved vision understanding.
\paragraph{Gemini \cite{team2023gemini}} is a closed-source multimodal model from Google, available in multiple variants optimized for different tasks. 
We use \texttt{Gemini 1.5 Flash} for evaluation and \texttt{Gemini 1.5 Flash-8B} for experiments, a smaller, faster variant with comparable performance.
\paragraph{Qwen2 \cite{yang2024qwen2technicalreport}} is an open-source multimodal model built on a vision transformer with strong visual reasoning. We use the \texttt{Qwen2-VL-7B-Instruct} model, competitive with GPT-4o on several benchmarks.
\paragraph{Phi \cite{phi}} is a lightweight, open-source 4.2B-parameter multimodal model, trained on synthetic and web data. We use \texttt{Phi-3.5-Vision-Instruct}, optimized for precise instruction adherence.

\subsection{Baselines}
\label{sec:exp_setup:baselines}

We compare our method to four prompting-based baselines:\footnote{Temperature set to 0.8 for all baselines.} zero-shot (\base{}), Chain-of-Thought (\chain{}) prompting, and self-refinement with (\critic{}) and without (\nocritic{}) a critic.

\base{} generates a final explanation directly from the image and caption using VLM. \chain{} follows a similar setup but instructs the model to produce intermediate reasoning chains \cite{cot}. 
Additionally, we implement \critic{}, a multimodal variant of self-refinement \cite{madaan2023selfrefine}, where a \textit{generator} produces a response, and a \textit{critic} evaluates it based on predefined criteria. The critic's feedback helps refine the output iteratively\footnote{Refinement steps set to 2 for fair comparison.}. Evaluation criteria include correctness, soundness, completeness, faithfulness, and clarity (details in Appendix \ref{app:base-prompts}).
\nocritic{} functions identically to \critic{} but without a \textit{critic model}, refining candidate explanations without feedback. This also serves as an ablation of the implications from \method{}. Prompts for baselines are in Appendix \ref{app:base-prompts}.


\subsection{Evaluation Metrics}
\label{sec:exp_setup:eval}
While human evaluation is often the most reliable option for open-ended tasks like ours \cite{hwang-shwartz-2023-memecap}, it is costly at scale. LLM-based evaluations (e.g., with \texttt{Gemini 1.5 Flash}) offer a more affordable alternative but are not always reliable \cite{biases_paper}. Prior research in fact verification has found that modern closed-source LLMs excel at fact checking when the complex facts are decomposed into simpler, atomic facts and verified individually \cite{gunjal-durrett-2024-molecular, samir-etal-2024-locating}. Inspired by this approach, we propose LLM-based precision and recall scores.

For recall, we decompose the reference $ref$ into atomic facts: $\{y_1, y_2, ..., y_n\}$ and check whether each appears in the predicted response $pred$.
% The percent of facts present in the predicted response forms the recall score:
\[
\text{Recall} = \frac{1}{n} \sum_{i=1}^{n} \mathbbm{1} \big( LLM(y_i, pred) = \text{Yes} \big)
\]
where $n$ is the number of atomic facts in $ref$.

Precision follows the same process in reverse, decomposing $pred$ into a list of atomic facts: $\{x_1, x_2, ..., x_m\}$ and verifying their presence in $ref$:
\[
\text{Precision} = \frac{1}{m} \sum_{i=1}^{m} \mathbbm{1} \big( LLM(x_i, ref) = \text{Yes} \big)
\]
where $m$ is the number of atomic facts in $pred$. Both decomposition and verification use Gemini-Flash-1.5 with a temperature of 0.2.

In preliminary experiments, we observed that human references tend to omit obvious visual details, whereas model-generated answers are often more complete, referencing visual information. To prevent penalizing the models for these facts, we incorporate literal image descriptions (Sec~\ref{sec:method}) into the reference by decomposing them and adding them to the atomic facts for fairer evaluation. Based on the precision and recall scores, we report the macro-$F_1$ score.

To assess the reliability of our metrics, we conducted a human evaluation on 130 random samples across all models and datasets via CloudResearch (details in Appendix \ref{app:cloudresearch}). Human annotators determined whether each atomic sentence appeared in the corresponding text (e.g., reference). The average agreement between the LLM-based evaluator and two human annotators was 77.1\% (\(\kappa = 54.1\)), similar to the agreement between the two annotators: 75.4\% (\(\kappa = 50.8\)), indicating considerable alignment with human judgment. Prompts are in Appendix \ref{app:eval-prompts}.


\section{Results}
\label{sec:results}
\section{Results}
% For chunking, It could be condensed -> First, show the final results (line chunking vs ideal chunking) and the naive chunking stats. Then, describe the trends and example of sections not in any naive chunking strategy.
\subsection{(RQ1) Impact of tailored components}
\label{subsec: rq1_result}

\subsubsection{Hierarchy-aware Chunking}
\label{subsubsec: chunking_result}

From the defined metrics that are used to quantify the information loss on each type of chunking in \S\ref{subsubsec: chunk_setup}, we can conclude that a good chunking strategy should minimize \textit{Fail Chunk Ratio}, \textit{Fail Section Ratio} and \textit{Uncovered Section Ratio}.
%
Minimizing these metrics will reduce the information loss of some sections or parts of sections that are missing from the chunks.
%
Additionally, \textit{Sections/Chunk} and \textit{Chunks/Section} should be close to 1 in order for sections \emph{not to be} split into multiple chunks and retain atomicity within each chunk.

We evaluate multiple configurations of chunking strategies, chunk sizes, and overlaps as described in \S\ref{subsubsec: chunk_setup} and present the average metrics for each chunking strategy in Table~\ref{table: chunking_by_type}. 
%
It is observed that the chunking strategy most closely resembling the output of our hierarchy-aware chunking strategy is line-based chunking. 

However, across all strategies, approximately 30\% of sections are not referenced in any chunks, and at least 41.7\% of sections are not fully contained within a single chunk. 
%
Further analysis indicates that around 20\% of all sections cannot be fully covered in a single chunk under any naive chunking strategy due to their extended length. 
%
This necessitates the retrieval model to retrieve multiple chunks to provide sufficient context.

An example of such a section is Section 44 of the \textbf{Emergency Decree on Digital Asset Businesses, B.E. 2561}, which cannot be fully covered in a single chunk across any naive chunking strategy. 
%
This is attributed to its lengthy content and the presence of multiple subsections separated by newline characters, which are commonly used as delimiters in many naive chunking approaches.

% Should this be translated?
\begin{quote}
    \textbf{Section 44 of Emergency Decree on Digital Asset Businesses, B.E. 2561}
    
    It shall be presumed that the following persons, who exhibit behavior involving the buying or selling of digital tokens or engaging in forward contracts related to digital tokens in an unusual manner for themselves, are persons who possess or are aware of inside information as defined under Section 42:
    
    (1) Holders of digital tokens exceeding 5\% of the total tokens sold in each series by the issuer of digital tokens. This includes digital tokens held by their spouses, cohabiting partners in the manner of husband and wife, and their minor children.
    
    (2) Directors, executives, controlling persons, employees, or staff members of the affiliated entities of the digital token issuer who are in positions or roles responsible for, or with access to, inside information.
    
    (3) Ascendants, descendants, adoptive parents, or adopted children of persons specified under Section 43.
    
    (4) Siblings sharing the same father and mother, or the same father or mother as persons specified under Section 43.
    
    (5) Spouses or cohabiting partners in the manner of husband and wife of persons specified under Section 43 or individuals listed under (3) or (4).
    
    The term \enquote{affiliated entities of the digital token issuer} under (2) refers to parent companies, subsidiaries, or associated companies of the digital token issuer, as defined by the criteria set forth by the SEC Board's announcements.
\end{quote}

\begin{table}[!ht]
\centering

\resizebox{\textwidth}{!}{%
\renewcommand{\arraystretch}{1.3} % This increases the cell height by 1.5 times
\small % or \scriptsize
\begin{tabular}{@{}lccccc@{}}
\toprule
\textbf{Chunking Strategy} & \multicolumn{1}{l}{\textbf{Section/Chunk $\rightarrow$1}} & \multicolumn{1}{l}{\textbf{Chunk/Section $\rightarrow$1}} & \multicolumn{1}{l}{\textbf{Fail Chunk Ratio $\downarrow$}} & \multicolumn{1}{l}{\textbf{Fail Section Ratio $\downarrow$}} & \multicolumn{1}{l}{\textbf{Uncovered Section Ratio $\downarrow$}} \\ \midrule
\cellcolor{lightgray}Hierarchy-aware  & \cellcolor{lightgray}{1.000}   & \cellcolor{lightgray}{1.000}    & \cellcolor{lightgray}{0.000}  & \cellcolor{lightgray}{0.000}  & \cellcolor{lightgray}{0.000}                                       \\
Character  & 3.098                             & 1.710                              & 0.819                                & 0.675                                  & 0.397                                       \\

Line       & \textbf{1.689}                    & \textbf{1.234}                    & \textbf{0.658}                       & \textbf{0.417}                         & \textbf{0.294}                              \\

Recursive  & \underline{1.793}                       &\underline{1.27}                        & \underline{0.741}                          & \underline{0.504}                            & \underline{0.381}                                 \\ \bottomrule
\end{tabular}
}
\caption{Information loss comparison between hierarchy-aware chunking compared to other naive chunking strategies. Since hierarchy-aware chunking consistently parses into a single law section, it was treated as an upper bound because no information loss occurred.}
\label{table: chunking_by_type}
\end{table}

For the specific configuration of line chunking that produces chunks most similar to hierarchy-aware chunking, we fix the chunking strategy while varying the chunk overlap and chunk size parameters. 
%
Increasing the chunk size results in more text per chunk, leading to higher \textbf{Sections/Chunk} and \textbf{Chunks/Section} values while reducing the \textbf{Fail Chunk Ratio}, \textbf{Fail Section Ratio}, and \textbf{Uncovered Section Ratio}. 
%
Similarly, increasing the overlap effectively increases the chunk size, producing comparable effects to directly increasing the chunk size.
%
Based on these observations, we select the optimal configuration for naive chunking as line chunking with a chunk size of 553 characters and a chunk overlap of 50 characters. The detailed results for this configuration are displayed in Appendix~\ref{appendix: chunk_hyper}.

Finally, the metrics for the selected naive chunking configuration are compared against hierarchy-aware chunking in Table~\ref{table: chunking_compare_metric}.

\begin{table}[!ht]
\centering

\resizebox{\textwidth}{!}{%
\renewcommand{\arraystretch}{1.3} % This increases the cell height by 1.5 times
\small % or \scriptsize
\begin{tabular}{@{}lccccc@{}}
\toprule
\textbf{Chunking Strategy} & \multicolumn{1}{l}{\textbf{Section/Chunk $\rightarrow$1}} & \multicolumn{1}{l}{\textbf{Chunk/Section $\rightarrow$1}} & \multicolumn{1}{l}{\textbf{Fail Chunk Ratio $\downarrow$}} & \multicolumn{1}{l}{\textbf{Fail Section Ratio $\downarrow$}} & \multicolumn{1}{l}{\textbf{Uncovered Section Ratio $\downarrow$}} \\ \midrule
Hierarchy-aware chunking  & 1.000   & 1.000    & 0.000  & 0.000  & 0.000                                       \\
Line chunking (553 chunk size and 50 chunk overlap)  & 1.956                             & 1.180                              & 0.521                                & 0.323                                  & 0.156                                       \\ \bottomrule
\end{tabular}
}
\caption{Information loss comparison between perfect chunking strategy (hierarchy-aware chunking) and the best naive chunking setup.}
\label{table: chunking_compare_metric}
\end{table}

% Next, we also show results on our benchmark as well
Apart from the evaluation of chunking in isolation in terms of information loss, we also present the evaluation results on our benchmark in Table~\ref{table: chunk_e2e_main}.

\begin{table}[ht!]
\centering
\resizebox{\textwidth}{!}{%
\begin{tabular}{@{}lccccccc@{}}
    \toprule
    \textbf{Settings} & \textbf{Retriever Multi MRR ($\uparrow$)} & \textbf{Retriever Recall ($\uparrow$)} & \textbf{Coverage ($\uparrow$)} & \textbf{Contradiction ($\downarrow$)} & \textbf{E2E Recall ($\uparrow$)} & \textbf{E2E Precision ($\uparrow$)} & \textbf{E2E F1 ($\uparrow$)} \\
    \midrule
    Naïve Chunking            & 0.786 & 0.935 & 86.6 & \textbf{0.050} & 0.882 & 0.613 & 0.722 \\
    Hierarchy-aware Chunking  & \textbf{0.834} & \textbf{0.942} & \textbf{86.7} & 0.054 & \textbf{0.894} & \textbf{0.630} & \textbf{0.739} \\
    \bottomrule
\end{tabular}%
}
\caption{Effect of chunking configuration on E2E performance on NitiBench-CCL}
\label{table: chunk_e2e_main}
\end{table}


% From table~\ref{table: chunk_e2e_main}, the naive chunking strategy performs significantly worse than section-based chunking in terms of retrieval performance on both WCX and Tax Case datasets. This discrepancy likely stems from two factors: First, naive chunking discards chunks that do not fully contain a section. Second, it often splits single sections across multiple chunks, rendering these fragmented sections unusable for evaluation and practical application, as legal responses require complete sections.

% The E2E performance also agrees with the retrieval performance in that the section-based chunking significantly outperforms line chunking. Interestingly,with line chunking, end-to-end (E2E) recall (based on sections cited by the LLM) exceeds retrieval recall. This stems from two factors: 1) LLM sometimes cites unretrieved sections, either from its internal knowledge or through hallucination; and 2) Line chunking’s mapping of chunks to single, fully covered sections can lead to partial section coverage within a chunk, causing LLM to cite portions outside the mapped section. Consequently, line chunking’s E2E recall can surpass its retrieval recall.

From Table~\ref{table: chunk_e2e_main}, the naive chunking strategy performs worse than hierarchy-aware chunking in terms of retrieval performance. 
%
This discrepancy likely arises because naive chunks often contain content from multiple sections, introducing \enquote{noise} that can negatively impact the retrieval model's ranking of relevant documents.  

However, in terms of end-to-end (E2E) performance, the system using hierarchy-aware chunking only slightly outperforms the one using naive chunking. 
%
We suspect that this is because the LLM can effectively filter out the \enquote{noise} in the retrieved sections during answer generation. 
%
As a result, the coverage and contradiction scores are not significantly different between the two systems.
%
Nevertheless, there remains a discrepancy in the E2E citation score.  

In conclusion, \textbf{hierarchy-aware chunking achieves a slight but consistent advantage over the naive chunking strategy.}

\subsubsection{NitiLink}
\label{subsubsec: referencer_result}

The evaluation results of the experiment described in \S\ref{subsubsec: referencer_setup} are presented in Table~\ref{table: augmenter_e2e_main}. 
%
In the table, ``Ref Depth 1'' denotes a RAG system that incorporates a NitiLink component with a maximum depth of 1, while \enquote{No Ref} represents a RAG system without NitiLink. 
%
For the metrics, ``NitiLink'' indicates retrieval metrics calculated on the augmented context, which includes both the initially retrieved sections and the additional sections fetched by NitiLink.

% \begin{table}[!ht]
% \centering
% \caption{Effect of augmenter configuration on E2E performance}
% \renewcommand{\arraystretch}{1.5} % This increases the cell height by 1.5 times
% \label{table: augmenter_e2e_main}
% \begin{tabular}{@{}c|cc|cc@{}}
% \toprule
% Dataset                                  & \multicolumn{2}{c|}{Tax}        & \multicolumn{2}{c}{WCX}         \\ \midrule
% Setting                                  & Ref Depth 1    & No Ref         & Ref Depth 1    & No Ref         \\ \midrule
% Retriever MRR                            & 0.574          & 0.574          & 0.809          & 0.809          \\
% \multicolumn{1}{l|}{Retriever Multi MRR} & 0.333          & 0.333          & 0.809          & 0.809          \\
% Retriever Recall                         & 0.499          & 0.499          & 0.938          & 0.938          \\
% Referencer MRR                           & \textbf{0.582} & 0.574          & 0.800          & \textbf{0.809} \\
% Referencer Multi MRR                     & \textbf{0.345} & 0.333          & 0.800          & \textbf{0.809} \\
% Referencer Recall                        & \textbf{0.602} & 0.499          & \textbf{0.940} & 0.938          \\
% Coverage                                 & 45.0           & \textbf{50.0}  & \textbf{86.3}  & 85.2           \\
% Contradiction                            & 0.520          & \textbf{0.460} & \textbf{0.051} & 0.055          \\
% E2E Recall                               & \textbf{0.354} & 0.333          & \textbf{0.885} & 0.880          \\
% E2E Precision                            & 0.630          & \textbf{0.64}  & 0.579          & \textbf{0.601} \\
% E2E F1                                   & \textbf{0.453} & 0.438          & 0.700          & \textbf{0.714} \\ \bottomrule
% \end{tabular}%
% \end{table}
\begin{table}[!ht]
\centering
\renewcommand{\arraystretch}{1.3}
\newcommand{\gray}{\cellcolor{gray!15}}
\newcommand{\pos}[1]{\textcolor{darkgreen}{(#1\%)}}
\newcommand{\negv}[1]{\textcolor{red}{(#1\%)}}

\begin{tabular}{lcccccc}
\toprule
\multirow{2}{*}{\textbf{Metric}} & \multicolumn{3}{c}{\textbf{NitiBench-CCL}} & \multicolumn{3}{c}{\textbf{NitiBench-Tax}} \\ 
 & \textbf{No Ref} & \gray \textbf{Ref Depth 1} & $\Delta$ & \textbf{No Ref} & \gray \textbf{Ref Depth 1} & $\Delta$ \\ 
\midrule
\multicolumn{7}{c}{\textbf{Retriever Metrics}} \\ 
\midrule
MRR ($\uparrow$)       & \multicolumn{2}{c}{0.809} & -  & \multicolumn{2}{c}{0.574} & -  \\
Multi MRR ($\uparrow$) & \multicolumn{2}{c}{0.809} & -  & \multicolumn{2}{c}{0.333} & -  \\
Recall ($\uparrow$)    & \multicolumn{2}{c}{0.938} & -  & \multicolumn{2}{c}{0.499} & -  \\
\midrule
\multicolumn{7}{c}{\textbf{NitiLink Metrics}} \\ 
\midrule
MRR ($\uparrow$)             & 0.809  & \gray 0.800  & \negv{-1.11}  & 0.574  & \gray \textbf{0.582}  & \pos{+1.39}  \\
Multi MRR ($\uparrow$)       & 0.809  & \gray 0.800  & \negv{-1.11}  & 0.333  & \gray \textbf{0.345}  & \pos{+3.60}  \\
Recall ($\uparrow$)          & 0.938  & \gray \textbf{0.940}  & \pos{+0.21}  & 0.499  & \gray \textbf{0.602}  & \pos{+20.6}  \\
Coverage ($\uparrow$)        & 85.2   & \gray \textbf{86.3}  & \pos{+1.29}  & \textbf{50.0}   & \gray 45.0   & \negv{-10.0}  \\
Contradiction ($\downarrow$) & 0.055  & \gray \textbf{0.051}  & \pos{-7.27}  & \textbf{0.460}  & \gray 0.520  & \negv{+13.0}  \\
E2E Recall ($\uparrow$)      & 0.880  & \gray \textbf{0.885}  & \pos{+0.57}  & 0.333  & \gray \textbf{0.354}  & \pos{+6.31}  \\
E2E Precision ($\uparrow$)   & 0.601  & \gray 0.579  & \negv{-3.66}  & \textbf{0.640}  & \gray 0.630  & \negv{-1.56}  \\
E2E F1 ($\uparrow$)          & \textbf{0.714}  & \gray 0.700  & \negv{-1.96}  & 0.438  & \gray \textbf{0.453}  & \pos{+3.42}  \\
\bottomrule
\end{tabular}
\caption{Effect of NitiLink augmenter configuration on E2E performance. The $\Delta$ column shows the relative percentage change compared to ``No Ref'', with dark green indicating improvement and red indicating degradation.}
\label{table: augmenter_e2e_main}
\end{table}




The results from Table~\ref{table: augmenter_e2e_main} show that there is no clear significant advantage when employing NitiLink in a RAG system. 
%
The results also highlight the differing impacts of incorporating NitiLink across datasets.

\textbf{NitiBench-Tax} For this dataset, we can clearly see that the recall was substantially improved from 0.499 to 0.602.
%
The improvement of recall suggested that NitiLink does provide an additional correct law section to the retrieved documents. 
%
Despite significant improvement over recall, we only see marginal improvements over MRR and Multi MRR.
%
Since we're using a depth-first augmented strategy (see \S\ref{subsubsec: referencer_setup}), this suggested that the document that cited more positives by NitiLink is ranked at the bottom of the retrieved documents.
%
Surprisingly, despite a major improvement in recall, some E2E metrics declined.
%
This might be due to NitiBench-Tax's query complexity, which often demands advanced reasoning capabilities that the LLM, even with the correct documents, struggles to provide. 
%
Another reason that might affect the performance decline even with more relevant documents provided to the LLM is the longer context that the LLM needs to process due to the higher amount of content added by NitiLink.

\textbf{NitiBench-CCL} For NitiBench-CCL showed no significant change in retrieval metrics and most E2E metrics.
%
Incorporating NitiLink yields very little recall gain, while MRR is slightly lower. 
%
This means that NitiLink often pushed the positive lower in the ranking as we're using a depth-first augmentation strategy (see \S\ref{subsubsec: referencer_setup}).
%
We highlight several factors that might contribute to the limited recall gain in this dataset:
\begin{enumerate}
    \item \textbf{Binary recall nature:} NitiBench-CCL queries typically involve a single relevant law, making recall binary and thus harder to improve.
    %
    \item \textbf{Simplicity of NitiBench-CCL queries:} Simple, non-specific NitiBench-CCL queries often rely on many relevant law sections that are similar semantically rather than hierarchically. 
    %
    This is opposed to NitiBench-Tax, where referenced law sections are necessary for legal reasoning.
    %
    This simplicity stems from the fact that the dataset was created by letting the annotator craft a question based on a given law section.
    %
    This explicitly provides bias toward the dataset since the question was created without a referenced law section.
    %
    \item \textbf{Hierarchical limiation:} The hierarchical structure itself presents challenges. 
    %
    Although NitiLink augmented the retrieved law section mentioned in the retrieved document (children reference), it lacks a law section that references retrieved law sections (parent reference).  
    %
    Thus, this version of NitiLink that lacks the ability to fetch parent law sections could result in a suboptimal performance.
    %
    %\item The hierarchical structure itself presents challenges. Many sections in the Revenue Code lack hierarchical connections (41\% have no children, 45\% no parents, and 24\% neither), limiting the referencer's effectiveness. Furthermore, some queries (e.g., those about criminal penalties) require retrieving laws that reference multiple others. The current referencer, retrieving only children of initially retrieved laws, struggles with these, as it cannot retrieve the parent law (which might be the ground truth).
\end{enumerate}

Despite the limited recall gains in NitiBench-CCL, we can see that there's a slight improvement in coverage score as well as recall.
%
This suggests that even small recall improvements can enhance the LLM's ability to answer NitiBench-CCL queries effectively.

\subsection{(RQ2) Impact of Retriever and LLM}
\label{subsec: rq2_result}

\subsubsection{Retriever}
\label{subsubsec: retriever_result}

\textbf{NitiBench-CCL} 
%
Table~\ref{table: retrieval_wangchan} presents the retrieval performance of 8 models as described in \S\ref{subsubsec: retriever_setup} on NitiBench-CCL with hierarchy-aware chunking. 
%
Because each query has only one positive label (as mentioned in \S\ref{subsubsec: wcx_dataset}), the multi-hit-rate and multi-MRR metrics are equivalent to their single-label counterparts. This also applies to recall and hit rate as well. Thus, for this dataset, we only showed Recall (Recall@K) and MRR (MRR@K) since other metrics are considered redundant.

The best-performing model is the human-reranked fine-tuned BGE-M3, achieving an MRR@5 of 0.805. 
%
Close behind are the auto-reranked fine-tuned BGE-M3 (0.800 MRR@5) and the base BGE-M3 (0.579 MRR@5). 
%
BGE-M3's strong performance is likely due to its use of three embedding types for relevance calculation, further enhanced by fine-tuning on in-domain data. 
%
Notably, the auto-reranked version nearly matches human-reranked performance without requiring costly human annotation. 
%

\textbf{Based on these findings, we recommend a cost-effective in-domain adaptation pipeline, notably Auto-Finetuned BGE-M3, that uses a strong LLM to generate synthetic training pairs, retrieves top-k passages with BGE-M3, and then applies a BGE-M3 Reranker. 
%
As shown in the results, this approach closely matches human-reranked performance while significantly saving annotation costs in an in-domain setup.
}

The commercially available Cohere embedding model ranks just below the top-performing BGE-M3 models and is followed by the ColBERT-based and dense embedding models, JINA ColBERT v2 and JINA embeddings v3, respectively. 
%
Among the tested retrievers, NV-Embed v1 shows the lowest performance among non-baseline models (0.713 MRR@5), likely due to its decoder-based architecture and reliance on prefix instruction prompts. 
%
Overall, retrieval performance on NitiBench-CCL is strong, with most models delivering comparable results, except for NV-Embed v1 and BM25. 
%
However, despite this strong performance, a gap between hit-rate and MRR when $k=\{5,10\}$ indicates that \textbf{while relevant documents are frequently retrieved, they are not consistently ranked first, potentially impacting end-to-end performance.}

\begin{table}[!ht]
\centering

% \renewcommand{\arraystretch}{1.5}
\small
\begin{tabular}{@{}clcc@{}}
\toprule
\textbf{Top-K} & \textbf{Model} & \textbf{HR/Recall@k} & \textbf{MRR@k} \\ \midrule
\multirow{8}{*}{k=1} 
  & BM25                   & .481 & .481 \\
  & JINA V2                & .681 & .681 \\
  & JINA V3                & .587 & .587 \\
  & NV-Embed V1            & .492 & .492 \\
  & BGE-M3                 & .700 & .700 \\
  & Human-Finetuned BGE-M3 & \textbf{.735} & \textbf{.735} \\
  & Auto-Finetuned BGE-M3  & \underline{.731} & \underline{.731} \\
  & Cohere                 & .676 & .676 \\ \midrule
\multirow{8}{*}{k=5} 
  & BM25                   & .658 & .548 \\
  & JINA V2                & .852 & .750 \\
  & JINA V3                & .821 & .681 \\
  & NV-Embed V1            & .713 & .579 \\
  & BGE-M3                 & .880 & .773 \\
  & Human-Finetuned BGE-M3 & \textbf{.906} & \textbf{.805} \\
  & Auto-Finetuned BGE-M3  & \underline{.900} & \underline{.800} \\
  & Cohere                 & .870 & .754 \\ \midrule
\multirow{8}{*}{k=10} 
  & BM25                   & .715 & .556 \\
  & JINA V2                & .889 & .755 \\
  & JINA V3                & .875 & .688 \\
  & NV-Embed V1            & .776 & .587 \\
  & BGE-M3                 & .919 & .778 \\
  & Human-Finetuned BGE-M3 & \textbf{.938} & \textbf{.809} \\
  & Auto-Finetuned BGE-M3  & \underline{.934} & \underline{.804} \\
  & Cohere                 & .912 & .760 \\ \bottomrule
\end{tabular}
\caption{Retrieval Evaluation Result on NitiBench-CCL with hierarchy-aware chunking. Since the test split contains a single positive (as mentioned in \S \ref{subsubsec: wcx_dataset}), we collapsed metrics that are duplicated, such as HitRate (HR)/ Recall/ Multi-HitRate and MultiMRR / MRR.}
\label{table: retrieval_wangchan}
\end{table}


\textbf{NitiBench-Tax} Table~\ref{table: retrieval_tax} presents the retrieval performance of various models on NitiBench-Tax using hierarchy-aware chunking. 
%
Unlike NitiBench-CCL, this dataset includes multi-label queries, resulting in different values for single-label and multi-label metrics.

Overall performance is significantly lower on this dataset compared to NitiBench-CCL, likely due to the considerably longer and more nuanced queries in NitiBench-Tax.
%
JINA v3 and BGE-M3 (base, auto-fine-tuned, and human-finetuned) consistently perform among the top, achieving Multi-MRR@10 scores of 0.311, 0.354, 0.345, and 0.333, respectively. 
%
Conversely, JINA v2 and NV-Embed v1 consistently underperform compared to the baseline, potentially because NitiBench-Tax is out-of-distribution relative to their training data, considering the complexity of the query.
%
This is particularly evident with JINA v2, whose Multi-MRR@10 drops dramatically from 0.750 on NitiBench-CCL to 0.091 on NitiBench-Tax.

Similarly, the Human-Finetuned BGE-M3 variants are often outperformed by the base BGE-M3, suggesting different data distributions between NitiBench-CCL and NitiBench-Tax, hindering cross-dataset generalization. 
%
While some models achieve reasonable single-label hit rates, multi-label hit-rate performance is poor across all models. 
%
This, combined with low recall and significantly lower multi-label MRR compared to single-label MRR, indicates that while models can often retrieve some relevant documents, they struggle to retrieve all relevant documents for a given query. 
%
This limitation is critical, as comprehensive legal responses require consideration of all relevant legal sections. 
%
\textbf{Although the proposed pipeline (Human-Finetuned BGE-M3) performs strongly on in-domain data (as seen with NitiBench-CCL), these Tax Case results underscore the critical need for sufficiently diverse in-domain training data, since a narrow domain distribution can lead to inconsistent or contradictory outcomes in real-world settings.}

Despite its lower overall performance, NitiBench-Tax benefits more from increasing the number of top-k retrieved documents compared to NitiBench-CCL. 
%
Its hit rate and recall improve at a faster rate as more documents are retrieved compared to NitiBench-CCL.

\begin{table}[!ht]
\centering
\small
\begin{tabular}{@{}clccccc@{}}
\toprule
\textbf{Top-K} & \textbf{Model}                  & \textbf{HR@k}          & \textbf{Multi HR@k}    & \textbf{Recall@k}      & \textbf{MRR@k}         & \textbf{Multi MRR@k}   \\ \midrule
k=1   & BM25                   & .220          & .080          & .118          & .220          & .118          \\
      & JINA V2                & .140          & .040          & .068          & .140          & .068          \\
      & JINA V3                & .400          & .100          & .203          & .400          & .203          \\
      & NV-Embed V1            & .100          & .020          & .035          & .100          & .035          \\
      & BGE-M3                 & \underline{.500}    & \underline{.140}    & \underline{.269}    & \underline{.500}    & \underline{.269}    \\
      & Human-Finetuned BGE-M3 & .480          & \underline{.140}    & .255          & .480          & .255          \\
      & Auto-Finetuned BGE-M3  & \textbf{.520} & \textbf{.160} & \textbf{.281} & \textbf{.520} & \textbf{.281} \\
      & Cohere                 & .340          & .100          & .179          & .340          & .179          \\ \midrule
k=5   & BM25                   & .480          & .120          & .254          & .318          & .171          \\
      & JINA V2                & .200          & .080          & .114          & .165          & .085          \\
      & JINA V3                & \underline{.720}    & \textbf{.260} & \textbf{.448} & .508          & .297          \\
      & NV-Embed V1            & .200          & .020          & .081          & .126          & .050          \\
      & BGE-M3                 & \underline{.720}    & \underline{.240}    & \underline{.435}    & \underline{.580}    & \textbf{.337} \\
      & Human-Finetuned BGE-M3 & \textbf{.740} & .220          & .411          & .565          & .320          \\
      & Auto-Finetuned BGE-M3  & .700          & .200          & .382          & \textbf{.587} & \underline{.329}    \\
      & Cohere                 & .620          & .200          & .363          & .447          & .256          \\ \midrule
k=10  & BM25                   & .540          & .160          & .320          & .327          & .183          \\
      & JINA V2                & .240          & .100          & .147          & .171          & .091          \\
      & JINA V3                & \textbf{.840} & \underline{.340}    & \underline{.549}    & .524          & .311          \\
      & NV-Embed V1            & .220          & .040          & .097          & .128          & .052          \\
      & BGE-M3                 & \underline{.820}    & \textbf{.360} & \textbf{.555} & \underline{.593}    & \textbf{.354} \\
      & Human-Finetuned BGE-M3 & .800          & .280          & .499          & .574          & .333          \\
      & Auto-Finetuned BGE-M3  & .780          & .260          & .483          & \textbf{.600} & \underline{.345}    \\
      & Cohere                 & .680          & .200          & .414          & .454          & .263          \\ \bottomrule
\end{tabular}
\caption{Retrieval Evaluation Result on NitiBench-Tax with hierarchy-aware chunking. This split contains multiple positives per question.}
\label{table: retrieval_tax}
\end{table}



\subsubsection{LLM}
\label{subsubsec: llm_result}

The evaluation results of the experiments described in \S\ref{subsubsec: llm_setup} are presented in Table~\ref{table: llm_e2e_main_wcx} for NitiBench-CCL and Table~\ref{table: llm_e2e_main_tax} for NitiBench-Tax. 
%
Since experiments in \S\ref{subsubsec: referencer_result} do not provide conclusive results on whether the inclusion of NitiLink is necessary, we also vary the inclusion of NitiLink in this experiment as well.


\begin{table}[!ht]
\centering
\renewcommand{\arraystretch}{1.2}
\resizebox{\textwidth}{!}{%
\begin{tabular}{@{}lcccccccc@{}}
\toprule
\textbf{Setting} & \textbf{NitiLink} & \textbf{Retriever MRR ($\uparrow$)} & \textbf{Retriever Recall ($\uparrow$)} & \textbf{E2E Recall ($\uparrow$)} & \textbf{E2E Precision ($\uparrow$)} & \textbf{E2E F1 ($\uparrow$)} & \textbf{Coverage ($\uparrow$)} & \textbf{Contradiction ($\downarrow$)} \\ \midrule
\multirow{3}{*}{\texttt{gpt-4o-2024-08-06}} 
& No Ref      & \multirow{3}{*}{0.809} & \multirow{3}{*}{0.938} & 0.880  & \textbf{0.601}  & \textbf{0.714}  & 85.2  & 0.055  \\
& \cellcolor{lightgray}Ref Depth 1 &                      &                     & \cellcolor{lightgray}0.885  & \cellcolor{lightgray}\underline{0.579}  & \cellcolor{lightgray}\underline{0.700}  & \cellcolor{lightgray}86.3  & \cellcolor{lightgray}0.051  \\
& $\Delta$    &                      &                     & \textcolor{darkgreen}{+0.6\%}  & \textcolor{red}{-3.7\%}  & \textcolor{red}{-2.0\%}  & \textcolor{darkgreen}{+1.3\%}  & \textcolor{darkgreen}{-7.3\%}  \\ \midrule
\multirow{3}{*}{\texttt{gemini-1.5-pro-002}} 
& No Ref      & \multirow{3}{*}{0.809} & \multirow{3}{*}{0.938} & 0.892  & 0.512  & 0.651  & 86.5  & 0.048  \\
& \cellcolor{lightgray}Ref Depth 1 &                      &                     & \cellcolor{lightgray}\underline{0.895}  & \cellcolor{lightgray}0.491  & \cellcolor{lightgray}0.634  & \cellcolor{lightgray}87.3  & \cellcolor{lightgray}\underline{0.042}  \\
& $\Delta$    &                      &                     & \textcolor{darkgreen}{+0.3\%}  & \textcolor{red}{-4.1\%}  & \textcolor{red}{-2.6\%}  & \textcolor{darkgreen}{+0.9\%}  & \textcolor{darkgreen}{-12.5\%} \\ \midrule
\multirow{3}{*}{\texttt{claude-3-5-sonnet-20240620}} 
& No Ref      & \multirow{3}{*}{0.809} & \multirow{3}{*}{0.938} & \textbf{0.901} & 0.444  & 0.595  & \textbf{89.7} & \textbf{0.040}  \\ 
& \cellcolor{lightgray}Ref Depth 1 &                      &                     & \cellcolor{lightgray}0.894  & \cellcolor{lightgray}0.443  & \cellcolor{lightgray}0.592  & \cellcolor{lightgray}\underline{89.5} & \cellcolor{lightgray}0.044  \\
& $\Delta$    &                      &                     & \textcolor{red}{-0.8\%} & \textcolor{red}{-0.2\%} & \textcolor{red}{-0.5\%}  & \textcolor{red}{-0.2\%}  & \textcolor{red}{+10.0\%}  \\ \midrule
\multirow{3}{*}{\texttt{typhoon-v2-70b-instruct}} 
& No Ref      & \multirow{3}{*}{0.809} & \multirow{3}{*}{0.938} & 0.862  & 0.537  & 0.662  & 81.2  & 0.076  \\
& \cellcolor{lightgray}Ref Depth 1 &                      &                     & \cellcolor{lightgray}0.845  & \cellcolor{lightgray}0.573  & \cellcolor{lightgray}0.683  & \cellcolor{lightgray}79.9  & \cellcolor{lightgray}0.080  \\
& $\Delta$    &                      &                     & \textcolor{red}{-2.0\%} & \textcolor{darkgreen}{+6.7\%} & \textcolor{darkgreen}{+3.2\%}  & \textcolor{red}{-1.6\%}  & \textcolor{red}{+5.3\%}  \\ \midrule
\multirow{3}{*}{\texttt{typhoon-v2-8b-instruct}} 
& No Ref      & \multirow{3}{*}{0.809} & \multirow{3}{*}{0.938} & 0.775  & 0.387  & 0.516  & 70.8  & 0.134  \\
& \cellcolor{lightgray}Ref Depth 1 &                      &                     & \cellcolor{lightgray}0.718  & \cellcolor{lightgray}0.385  & \cellcolor{lightgray}0.501  & \cellcolor{lightgray}68.5  & \cellcolor{lightgray}0.145  \\
& $\Delta$    &                      &                     & \textcolor{red}{-7.4\%} & \textcolor{red}{-0.5\%}  & \textcolor{red}{-2.9\%}  & \textcolor{red}{-3.3\%}  & \textcolor{red}{+8.2\%}  \\ \bottomrule
\end{tabular}
}
\caption{Effect of LLM configuration on E2E performance on NitiBench-CCL. $\Delta$ values are computed relative to No Ref and normalized to percentage change.}
\label{table: llm_e2e_main_wcx}
\end{table}


\begin{table}[!ht]
\centering
\renewcommand{\arraystretch}{1.2}
\resizebox{\textwidth}{!}{%
\begin{tabular}{@{}lccccccccc@{}}
\toprule
{\textbf{Setting}} & {\textbf{NitiLink}} 
& {\textbf{Retriever MRR ($\uparrow$)}} 
& {\textbf{Retriever Multi MRR ($\uparrow$)}} 
& {\textbf{Retriever Recall ($\uparrow$)}} 
& {\textbf{E2E Recall ($\uparrow$)}} 
& {\textbf{E2E Precision ($\uparrow$)}} 
& {\textbf{E2E F1 ($\uparrow$)}} 
& {\textbf{Coverage ($\uparrow$)}} 
& {\textbf{Contradiction ($\downarrow$)}} \\ 
\midrule

%---------------- gpt-4o-2024-08-06 ----------------
\multirow{3}{*}{\texttt{gpt-4o-2024-08-06}} 
& No Ref            
  & \multirow{3}{*}{0.574} 
  & \multirow{3}{*}{0.333}  
  & \multirow{3}{*}{0.499}
  & 0.333 
  & \underline{0.640}
  & 0.438 
  & 50.0  
  & \underline{0.46} \\
& \cellcolor{lightgray}Ref Depth 1  
  & 
  & 
  & 
  & \cellcolor{lightgray}0.354 
  & \cellcolor{lightgray}0.630 
  & \cellcolor{lightgray}0.453 
  & \cellcolor{lightgray}45.0 
  & \cellcolor{lightgray}0.52 \\
& $\Delta$           
  & 
  & 
  & 
  & \textcolor{darkgreen}{+6.3\%}
  & \textcolor{red}{-1.6\%}
  & \textcolor{darkgreen}{+3.4\%}
  & \textcolor{red}{-10.0\%}
  & \textcolor{red}{+13.0\%} \\
\midrule

%---------------- gemini-1.5-pro-002 ----------------
\multirow{3}{*}{\texttt{gemini-1.5-pro-002}} 
& No Ref            
  & \multirow{3}{*}{0.574} & \multirow{3}{*}{0.333} & \multirow{3}{*}{0.499}
  & 0.361 & 0.308 & 0.332 
  & 44.0  & 0.48 \\
& \cellcolor{lightgray}Ref Depth 1  
  & 
  & 
  & 
  & \cellcolor{lightgray}0.354 
  & \cellcolor{lightgray}0.347 
  & \cellcolor{lightgray}0.351 
  & \cellcolor{lightgray}45.0 
  & \cellcolor{lightgray}0.48 \\
& $\Delta$           
  & 
  & 
  & 
  & \textcolor{red}{-1.9\%}
  & \textcolor{darkgreen}{+12.7\%}
  & \textcolor{darkgreen}{+5.7\%}
  & \textcolor{darkgreen}{+2.3\%}
  & \textcolor{black}{+0.0\%} \\
\midrule

%---------------- claude-3-5-sonnet-20240620 ----------------
\multirow{3}{*}{\texttt{claude-3-5-sonnet-20240620}} 
& No Ref            
  & \multirow{3}{*}{0.574} & \multirow{3}{*}{0.333} & \multirow{3}{*}{0.499 }
  & \underline{0.389} & 0.554 & \underline{0.457}
  & \underline{51.0}  & \textbf{0.44} \\
& \cellcolor{lightgray}Ref Depth 1  
  & 
  & 
  & 
  & \cellcolor{lightgray}\textbf{0.417} 
  & \cellcolor{lightgray}0.577 
  & \cellcolor{lightgray}\textbf{0.484} 
  & \cellcolor{lightgray}49.0 
  & \cellcolor{lightgray}0.56 \\
& $\Delta$           
  & 
  & 
  & 
  & \textcolor{darkgreen}{+7.2\%}
  & \textcolor{darkgreen}{+4.2\%}
  & \textcolor{darkgreen}{+5.9\%}
  & \textcolor{red}{-3.9\%}
  & \textcolor{red}{+27.3\%} \\
\midrule

%---------------- typhoon-v2-70b-instruct ----------------
\multirow{3}{*}{\texttt{typhoon-v2-70b-instruct}} 
& No Ref            
  & \multirow{3}{*}{0.574} & \multirow{3}{*}{0.333} & \multirow{3}{*}{0.499}
  & 0.326 & \textbf{0.662} & 0.437 
  & 42.0  & 0.58 \\
& \cellcolor{lightgray}Ref Depth 1  
  & 
  & 
  &  
  & \cellcolor{lightgray}0.333 
  & \cellcolor{lightgray}0.453 
  & \cellcolor{lightgray}0.384 
  & \cellcolor{lightgray}\textbf{54.0 }
  & \cellcolor{lightgray}\underline{0.46} \\
& $\Delta$           
  & 
  & 
  & 
  & \textcolor{darkgreen}{+2.1\%}
  & \textcolor{red}{-31.6\%}
  & \textcolor{red}{-12.1\%}
  & \textcolor{darkgreen}{+28.6\%}
  & \textcolor{darkgreen}{-20.7\%} \\
\midrule

%---------------- typhoon-v2-8b-instruct ----------------
\multirow{3}{*}{\texttt{typhoon-v2-8b-instruct}} 
& No Ref            
  & \multirow{3}{*}{0.574} & \multirow{3}{*}{0.333} & \multirow{3}{*}{0.499 }
  & 0.278 & 0.471 & 0.349 
  & 37.0  & 0.60 \\
& \cellcolor{lightgray}Ref Depth 1  
  & 
  & 
  & 
  & \cellcolor{lightgray}0.319 
  & \cellcolor{lightgray}0.561 
  & \cellcolor{lightgray}0.407 
  & \cellcolor{lightgray}35.0 
  & \cellcolor{lightgray}0.54 \\
& $\Delta$           
  & 
  & 
  & 
  & \textcolor{darkgreen}{+14.7\%}
  & \textcolor{darkgreen}{+19.1\%}
  & \textcolor{darkgreen}{+16.6\%}
  & \textcolor{red}{-5.4\%}
  & \textcolor{darkgreen}{-10.0\%} \\
\bottomrule
\end{tabular}
}
\caption{Effect of LLM configuration on E2E performance on NitiBench-Tax. 
\(\Delta\) values are computed relative to No Ref and normalized to percentage change.}
\label{table: llm_e2e_main_tax}
\end{table}


\textbf{NitiBench-CCL} On NitiBench-CCL, \texttt{claude-3.5-sonnet} excels in Coverage, Contradiction, and E2E Recall. 
%
However, the typhoon family of models struggles to match the performance of the closed-source models in this broader Thai legal QA domain. 
%
As seen with NitiBench-Tax, both \texttt{claude-3.5-sonnet} and \texttt{gemini-1.5-pro-002} exhibit low E2E Precision, leading to lower F1 scores compared to \texttt{gpt-4o}. 
%
This underscores a trade-off between recall and precision, particularly for \texttt{claude-3.5-sonnet} and \texttt{gemini-1.5-pro-002}. 
%
The causes of this precision drop are further analyzed in a later section.

\textbf{NitiBench-Tax} The results on NitiBench-Tax demonstrate that \texttt{claude-3.5-sonnet} also achieves top-2 performance across most end-to-end metrics. 
%
Interestingly, \texttt{typhoon-v2-70b-instruct}, an open-sourced model, delivers comparable results and outperforms others on NitiBench-Tax with the highest coverage score and E2E precision. 
%
However, its smaller variant, \texttt{typhoon-v2-8b-instruct}, ranks the lowest among the tested models.f
%`
Despite its limited parameter size, it manages to avoid falling significantly behind, showcasing a reasonable performance given its constraints.

Notably, both \texttt{claude-3.5-sonnet} and \texttt{gemini-1.5-pro-002} exhibit considerably lower E2E precision compared to \texttt{gpt-4o} and \texttt{typhoon-v2-70b-instruct}. This compromises their suitability for precision-critical applications, even though they excel in other areas. Additionally, the NitiLink module fails to consistently enhance performance, with mixed results indicating no definitive advantage in its current configuration.

\subsubsection{E2E results using best setups}
\label{subsubsec: e2e_best_result}

The results from experiments conducted under the four settings described in \S\ref{subsubsec: e2e_best_setup} are presented in Table~\ref{table: main_exp_full}. 
%
Both the Proposed RAG and Naive RAG settings utilize Human-Finetuned BGE-M3 as the retriever, as it demonstrated the best performance in previous experiments (Section~\ref{subsubsec: retriever_result}). 
%
Similarly, Claude 3.5 Sonnet is chosen as the generator based on its superior results in \S\ref{subsubsec: llm_result}. 
%
NitiLink is excluded in both settings due to its inconclusive impact on E2E performance (Section~\ref{subsubsec: referencer_result} and~\ref{subsubsec: llm_result}). 
%
We opted for this choice instead of the opposite because omitting NitiLink yields similar performance while reducing API costs. 
%
The primary distinction between Naive RAG and Proposed RAG lies in their chunking strategies: Naive RAG employs a naive chunking approach, whereas Proposed RAG utilizes hierarchy-aware chunking.

\begin{table}[!ht]
\centering
\renewcommand{\arraystretch}{1.2} % Increase cell height
\resizebox{\textwidth}{!}{%
\begin{tabular}{@{}lcccccccc@{}}
\toprule
{\textbf{Setting}} & \textbf{{Retriever MRR} ($\uparrow$)} & \textbf{{Retriever Multi-MRR} ($\uparrow$)} & \textbf{{Retriever Recall} ($\uparrow$)} & \textbf{{Coverage} ($\uparrow$)} & \textbf{{Contradiction} ($\downarrow$)} & \textbf{{E2E Recall} ($\uparrow$)} & \textbf{{E2E Precision} ($\uparrow$)} & \textbf{{E2E F1} ($\uparrow$)} \\ 
\midrule
\multicolumn{9}{c}{\textbf{NitiBench-CCL}} \\ \midrule
Parametric     
 & -     
 & -     
 & -     
 & 60.3     
 & 0.199   
 & 0.188   
 & 0.141   
 & 0.161  
\\
Na\"ive RAG    
 & 0.120 
 & 0.048 
 & 0.062 
 & 77.3     
 & 0.097   
 & 0.745   
 & 0.370   
 & 0.495  
\\
Proposed RAG   
 & 0.809 
 & 0.809 
 & 0.938 
 & \textbf{89.7}     
 & \textbf{0.040}   
 & \textbf{0.901}   
 & \textbf{0.444}   
 & \textbf{0.595}  
\\
\rowcolor{gray!15}
Golden Context 
 & 1.0   
 & 1.0   
 & 1.0   
 & 93.4     
 & 0.034   
 & 0.999   
 & 1.000   
 & 1.000  
\\ 
\midrule
\multicolumn{9}{c}{\textbf{NitiBench-Tax}} \\ \midrule
Parametric     
 & -     
 & -     
 & -     
 & 46.0     
 & 0.480   
 & \textbf{0.458}   
 & \textbf{0.629}   
 & \textbf{0.530}  
\\
Na\"ive RAG    
 & 0.120 
 & 0.048 
 & 0.062 
 & 50.0     
 & 0.460   
 & 0.306   
 & 0.463   
 & 0.368  
\\
Proposed RAG   
 & 0.574 
 & 0.333 
 & 0.499 
 & \textbf{51.0}     
 & \textbf{0.440}   
 & {0.389}   
 & {0.554}   
 & {0.457}  
\\
\rowcolor{gray!15}
Golden Context 
 & 1.0   
 & 1.0   
 & 1.0   
 & 52.0     
 & 0.460   
 & 0.694   
 & 1.000   
 & 0.820  
\\ 
\bottomrule
\end{tabular}
}
\caption{E2E Experiment results on NitiBench-CCL and NitiBench-Tax comparing various RAG setups on Human-Finetuned BGE-M3 retriever and \texttt{claude-3-5-sonnet-20240620} as a LLM.}
\label{table: main_exp_full}
\end{table}



\textbf{NitiBench-Tax} On NitiBench-Tax, the four main settings perform similarly, except for the parametric setting's slightly lower coverage and higher contradiction. Two key observations emerge:

First, the parametric setting achieves the second-highest E2E recall and precision despite lacking a retriever. 
%
To investigate this further, we inspected the cited law section generated by LLM. 
%
Surprisingly, we found that out of 105 law sections cited from LLM parametric knowledge, 58 of them aren't even retrieved by the best retriever. 
%
Among those 58 cited documents, 26 were in the correct law section. 
%
In contrast, only 5 of 101 sections cited by the proposed RAG system are \emph{not} retrieved. 
%
This indicates that retriever performance significantly constrains RAG systems, especially with complex queries like those in NitiBench-Tax. 
%
RAG system generators seem discouraged from using internal knowledge, which might sometimes provide better answers. 
%
Furthermore, the substantial disparity between retriever and E2E recall shows that the LLM often underutilizes relevant retrieved sections, particularly those containing primarily terminology (as discussed in \S\ref{subsec: retriever_re_error_analysis_tax}).
%
% We suspect that the high performance of parametric knowledge could be due to Thai law data or the case data from the revenue department website might be contaminated in the pretraining data since it's publicly available.
%
% However, we didn't further investigate this due to time constraint and we leave this to future works.

Second, Table~\ref{table: main_exp_full} (NitiBench-Tax) shows no clear relationship between E2E citation scores and coverage/contradiction. 
%
This suggests the LLM struggles to apply cited sections correctly in its reasoning, leading to incorrect or erroneously reasoned answers. 
%
This resembles the issue in \S\ref{subsubsec: referencer_result}, where improved retriever recall from NitiLink didn't consistently improve E2E metrics. 
%
Here, increased E2E citations with Claude 3.5 Sonnet don't necessarily improve coverage or contradiction.

\textbf{NitiBench-CCL} For NitiBench-CCL (Table~\ref{table: main_exp_full}), the results are as expected: parametric performs worst, followed by Naive RAG, then Proposed RAG, and finally Golden Context (best).

\subsection{(RQ3) Performance of Long-Context LLM (LCLM)}
\label{subsec: rq3_result}

The evaluation results of the Thai legal QA system based on LCLM are presented in Table~\ref{table: main_exp_sampled} in comparison to the same baselines in Table~\ref{table: main_exp_full}. 
%
The evaluation is conducted on a stratified 20\% sample of NitiBench-CCL and the full NitiBench-Tax. 
%
As detailed in \S\ref{subsec: setup_rq3}, the LCLM system processes all 35 Thai financial laws simultaneously as context, with special tokens inserted to serve as section identifiers. 
%
These tokens enable the LCLM to cite relevant sections explicitly when generating responses to queries.

\begin{table}[!ht]
\centering
\renewcommand{\arraystretch}{1.2} % Increase cell height
\resizebox{\textwidth}{!}{%
\begin{tabular}{@{}lcccccccc@{}}
\toprule
\textbf{Setting} & \textbf{Retriever MRR ($\uparrow$)} & \textbf{Retriever Multi-MRR ($\uparrow$)} & \textbf{Retriever Recall ($\uparrow$)} & \textbf{Coverage ($\uparrow$)} & \textbf{Contradiction ($\downarrow$)} & \textbf{E2E Recall ($\uparrow$)} & \textbf{E2E Precision ($\uparrow$)} & \textbf{E2E F1 ($\uparrow$)} \\ 
\midrule
\multicolumn{9}{c}{\textbf{NitiBench-CCL (20\% subsampled)}} \\ \midrule
Parametric     & -     & -     & -     & 60.6  & 0.198 & 0.197 & 0.147 & 0.169 \\
Na\"ive RAG    & 0.549 & 0.549 & 0.649 & 77.7  & 0.092 & 0.740 & 0.379 & 0.501 \\
Proposed RAG   & 0.825 & 0.825 & 0.945 & \textbf{90.1}  & \textbf{0.028} & \textbf{0.920} & 0.453 & 0.607 \\
LCLM (Gemini)  & -     & -     & -     & 83.2  & 0.063 & 0.765 & \textbf{0.514} & \textbf{0.615} \\
\rowcolor{gray!15}
Golden Context & 1.0   & 1.0   & 1.0   & 94.2  & 0.025 & 0.999 & 1.000 & 0.999 \\
\midrule
\multicolumn{9}{c}{\textbf{NitiBench-Tax}} \\ \midrule
Parametric     & -     & -     & -     & 46.0  & 0.480 & \textbf{0.458} & \textbf{0.629} & \textbf{0.530} \\
Na\"ive RAG    & 0.120 & 0.048 & 0.062 & 50.0  & 0.460 & 0.306 & 0.463 & 0.368 \\
Proposed RAG   & 0.574 & 0.333 & 0.499 & \textbf{51.0}  & \textbf{0.440} & 0.389 & 0.554 & 0.457 \\
LCLM (Gemini)  & -     & -     & -     & 36.0  & 0.620 & 0.410 & 0.484 & 0.444 \\
\rowcolor{gray!15}
Golden Context & 1.0   & 1.0   & 1.0   & 52.0  & 0.460 & 0.694 & 1.000 & 0.820 \\ 
\bottomrule
\end{tabular}
}
\caption{Experiment results on sampled NitiBench-CCL and full NitiBench-Tax. In the LCLM setup, we used the Gemini without retriever section, where full legislation books were parsed as a context.}
\label{table: main_exp_sampled}
\end{table}



From Table~\ref{table: main_exp_sampled}, the LCLM-based system performs comparably to the parametric setting on NitiBench-Tax and to the Naive RAG system on NitiBench-CCL. 
%
This performance gap may stem from degradation when processing extremely long contexts (1.2 million tokens). 
%
Regardless of the exact cause, the results suggest that while an LCLM-based Thai legal QA system is feasible, its performance remains significantly behind RAG-based counterparts, highlighting areas for further improvement.

Apart from utilizing LCLM to process the legislations and respond directly to the queries, we also explored using it as a retriever. 
%
As stated in \S\ref{subsec: setup_rq3}, Gemini 1.5 Pro is provided with all 35 legislations and tasked to retrieve 20 relevant laws given a query. 
%
This experiment is also conducted on the same sample of NitiBench-CCL as the previous experiment and the full NitiBench-Tax. 
%
The results are shown in Table~\ref{table: retrieval_wcx_lclm} and~\ref{table: retrieval_tax_lclm}.

\begin{table}[!ht]
\centering
\small
\begin{tabular}{@{}clcc@{}}
\toprule
\textbf{Top-K} & \textbf{Model} & \textbf{HR/Recall@k} & \textbf{MRR@k} \\ \midrule
\multirow{9}{*}{k=1} 
  & BM25                   & .480           & .480 \\
  & JINA V2                & .698           & .698 \\
  & JINA V3                & .601           & .601 \\
  & NV-Embed V1            & .496           & .496 \\
  & BGE-M3                 & .708           & .708 \\
  & Human-Finetuned BGE-M3 & \textbf{.757}  & \textbf{.757} \\
  & Auto-Finetuned BGE-M3  & \underline{.741} & \underline{.741} \\
  & Cohere                 & .707           & .707 \\
  & LCLM-as-a-retriever (Gemini)                   & .590           & .590 \\ \midrule
\multirow{9}{*}{k=5} 
  & BM25                   & .663           & .549 \\
  & JINA V2                & .858           & .761 \\
  & JINA V3                & .828           & .693 \\
  & NV-Embed V1            & .711           & .585 \\
  & BGE-M3                 & \underline{.888} & .779 \\
  & Human-Finetuned BGE-M3 & \textbf{.909}  & \textbf{.819} \\
  & Auto-Finetuned BGE-M3  & \textbf{.909}  & \underline{.807} \\
  & Cohere                 & .867           & .772 \\
  & LCLM-as-a-retriever (Gemini)                   & .776           & .667 \\ \midrule
\multirow{9}{*}{k=10} 
  & BM25                   & .733           & .559 \\
  & JINA V2                & .891           & .766 \\
  & JINA V3                & .878           & .700 \\
  & NV-Embed V1            & .794           & .596 \\
  & BGE-M3                 & .926           & .784 \\
  & Human-Finetuned BGE-M3 & \textbf{.945}  & \textbf{.824} \\
  & Auto-Finetuned BGE-M3  & \underline{.941} & \underline{.812} \\
  & Cohere                 & .913           & .778 \\
  & LCLM-as-a-retriever (Gemini)                   & .807           & .671 \\ \bottomrule
\end{tabular}
\caption{Retrieval Evaluation Result on a 20\% subset of NitiBench-CCL with hierarchy-aware chunking with Long-Context Retriever. Since the test split of NitiBench-CCL is single labeled, duplicated metrics (HR/Recall and MRR) have been collapsed.}
\label{table: retrieval_wcx_lclm}
\end{table}



\begin{table}[!ht]
\centering
\small
\begin{tabular}{@{}clccccc@{}}
\toprule
\textbf{Top-K} & \textbf{Model}                  & \textbf{HR@k}          & \textbf{Multi HR@k}    & \textbf{Recall@k}      & \textbf{MRR@k}         & \textbf{Multi MRR@k}   \\ \midrule
k=1   & BM25                   & .220          & .080          & .118          & .220          & .118          \\
      & JINA V2                & .140          & .040          & .068          & .140          & .068          \\
      & JINA V3                & .400          & .100          & .203          & .400          & .203          \\
      & NV-Embed V1            & .100          & .020          & .035          & .100          & .035          \\
      & BGE-M3                 & \underline{.500}    & \underline{.140}    & \underline{.269}    & \underline{.500}    & \underline{.269}    \\
      & Human-Finetuned BGE-M3 & .480          & \underline{.140}    & .255          & .480          & .255          \\
      & Auto-Finetuned BGE-M3  & \textbf{.520} & \textbf{.160} & \textbf{.281} & \textbf{.520} & \textbf{.281} \\
      & Cohere                 & .340          & .100          & .179          & .340          & .179          \\
      & LCLM                   & .480          & .120          & .227          & .480          & .227          \\ \midrule
k=5   & BM25                   & .480          & .120          & .254          & .318          & .171          \\
      & JINA V2                & .200          & .080          & .114          & .165          & .085          \\
      & JINA V3                & .720          & \underline{.260}    & \underline{.448}    & .508          & .297          \\
      & NV-Embed V1            & .200          & .020          & .081          & .126          & .050          \\
      & BGE-M3                 & .720          & .240          & .435          & \underline{.580}    & \underline{.337}    \\
      & Human-Finetuned BGE-M3 & \underline{.740}    & .220          & .411          & .565          & .320          \\
      & Auto-Finetuned BGE-M3  & .700          & .200          & .382          & \textbf{.587} & \underline{.329}    \\
      & Cohere                 & .620          & .200          & .363          & .447          & .256          \\
      & LCLM-as-a-retriever (Gemini)                   & \textbf{.760} & \textbf{.320} & \textbf{.515} & \textbf{.587} & \textbf{.370} \\ \midrule
k=10  & BM25                   & .540          & .160          & .320          & .327          & .183          \\
      & JINA V2                & .240          & .100          & .147          & .171          & .091          \\
      & JINA V3                & \textbf{.840} & \underline{.340}    & .549          & .524          & .311          \\
      & NV-Embed V1            & .220          & .040          & .097          & .128          & .052          \\
      & BGE-M3                 & \underline{.820}    & \textbf{.360} & \underline{.555}    & \underline{.593}    & \underline{.354}    \\
      & Human-Finetuned BGE-M3 & .800          & .280          & .499          & .574          & .333          \\
      & Auto-Finetuned BGE-M3  & .780          & .260          & .483          & \textbf{.600} & \underline{.345}    \\
      & Cohere                 & .680          & .200          & .414          & .454          & .263          \\
      & LCLM-as-a-retriever (Gemini)                   & .780          & \textbf{.360} & \textbf{.566} & .590          & \textbf{.379} \\ \bottomrule
\end{tabular}
\caption{Retrieval Evaluation Result on NitiBench-Tax with hierarchy-aware chunking. This split contains multiple positives per question.}
\label{table: retrieval_tax_lclm}
\end{table}


The results indicate that the LCLM retriever performs comparably to embedding-based retrievers on NitiBench-Tax, likely due to its superior reasoning capabilities. 
%
However, a noticeable performance gap exists when compared to the best retriever on NitiBench-CCL.
%
Additionally, increasing the number of retrieved documents for the LCLM yields minimal performance improvements relative to other models. 
%
We hypothesize that this limited gain is a result of LLMs' next-token prediction mechanism, which may hinder their ability to effectively retrieve and output relevant laws when those laws are distant from the query context or when the model attempts to generate relevant laws ranked lower in the retrieval order.












\section{Conclusions}
\label{sec:conclusion}
\section{Conclusion}\label{sec:conclusion}

In this paper, we proposed a prototype ASL generation system aimed at improving the naturalness, comprehensiveness, and overall quality of generated signs, addressing key limitations in existing approaches. Our technical evaluations indicate that our proposed approaches improve these aspects, enhancing the quality of generated ASL content. Feedback from DHH participants was mixed; while there was general interest in the system, concerns regarding visual quality and naturalness were noted. Reflecting on our design process and study findings, we discuss key insights and identify key areas for future improvement. While further work is needed, our study takes an initial step toward developing sign language generation systems that better meet the needs of the DHH and signing communities, offering real-world value.

\section*{Limitations}
\label{sec:limitations}
\section{Limitations}

The limitations of this work can be summarized in two main aspects.

First, while {\name} demonstrates lower accuracy degradation compared to existing methods while accessing a comparable proportion of KV cache,
it still exhibits non-negligible performance degradation.
This suggests opportunities for future work to investigate adaptive attention sparsity allocation strategies that dynamically optimize the sparsity ratios across layers and attention heads, based on their contextual importance.

Second, while {\name} increases long context serving throughput by up to \(2.7 \times\), our current implementation is limited to single-GPU deployment.
Future research could further explore (1) distributed multi-GPU system designs for scaled deployment, (2) integration with disaggregated LLM serving architectures like MoonCake~\citep{mooncake}.

\section*{Ethics Statement}
\label{sec:ethics}
All annotators involved in the human evaluation for this research were fairly compensated, with payment rates exceeding the local minimum wage to ensure equitable remuneration for their time and effort. Prior to recruiting annotators, ethical approval was obtained from the research ethics board at the authors’ institution, ensuring that the human evaluation process adhered to ethical guidelines and that no harm was caused to any individual involved. Additionally, the \fd{} dataset is publicly available, and we commit to making our dataset, including the augmented data and generated comprehensive answers, accessible to promote transparency and reproducibility in future work.


% TODO: when we prepare the camera-ready version, we should thank Jack in the acknowledgements
\section*{Acknowledgements}
This work was funded, in part, by the Vector Institute for AI, Canada CIFAR AI Chairs program, Accelerate Foundation Models Research Program Award from Microsoft, an NSERC discovery grant, and a research gift from AI2. We thank Jack Hessel, Benyamin Movassagh, Sahithya Ravi, Aditya Chinchure, and Vasile Negrescu for insightful discussions and feedback.

\bibliography{custom,anthology}

\appendix
\clearpage
\begin{appendices}

\section{Production Fault Trace}
\label{appendix:production-fault-trace}
The production fault trace was collected from an 8-GPU node pretrain cluster with 2880 GPUs over a period of 160 days. The trace includes details such as fault start time, fault end time, and the ID of the faulty node. \figref{fig:simulation:trace:timetrace} and \figref{fig:simulation:trace:cdf} provide a macro-level overview of the production fault trace. On average, the ratio of faulty 8-GPU nodes at any given time is $3.83\%$, with a p99 value of $7.22\%$.

\begin{figure}[h!t]
    \centering
    \begin{subfigure}[b]{0.23\textwidth}
        \centering
        \includegraphics[width=\textwidth]{figs/evaluation/fault_server_ratio.pdf}
        \caption{Fault Node Ratio Trace.}
        \label{fig:simulation:trace:timetrace}
    \end{subfigure}
    \hspace{2pt}
    \begin{subfigure}[b]{0.23\textwidth}
        \centering
        \includegraphics[width=\textwidth]{figs/evaluation/fault_server_cdf.pdf}
        \caption{Cumulative Distribution.}
        \label{fig:simulation:trace:cdf}
    \end{subfigure}
    \vspace{-2ex}
    \caption{Fault node trace in the production AI DC.}
    \label{fig:simulation:trace}
\end{figure}

Since most of failure events are GPU faults, we normalized the trace of 8-GPU nodes to generate 4-GPU nodes trace. Assuming that the fault rates of GPUs are i.i.d. with a fault probability of $p$ for each GPU, and considering that a node is deemed faulty if any GPU within it fails, the fault rate of an 8-GPU node is calculated as:  

\vspace{-1em}
$$
P_{fault}(8\text{-GPU}) = 1 - (1-p)^8 = 3.83\%.
$$  

From this, we derive $p = 0.49\%$. The fault rate for a 4-GPU node is then:  
$$
P_{fault}(4\text{-GPU}) = 1 - (1-p)^4 = 1.93\%.
$$  

The fault event of 4-GPU node is generate with Bayesian Equation, as:


\begin{align*}\label{eq:convert-trace}
& P_{fault}( \text{4-GPU} \mid  \text{8-GPU})\\ 
    &=\frac{P_{fault}(\text{8-GPU} \mid \text{4-GPU}) P_{fault}(\text{4-GPU})}{P_{fault}(\text{8-GPU})} \\ 
    & =  \frac{1 \times 1.93\%}{3.83\%} = 50.39\% \\
\end{align*}

Thus, whenever a fault occurs in an 8-GPU node in the original trace, each of the two corresponding 4-GPU nodes at the same location has a $50.39\%$ probability of fault. This method is used to convert the traces.

As node faults are i.i.d., the simulator linearly maps the fault trace to different network architectures.

\section{GPT-MoE Architecture}
\label{appendix:gpt-moe}
This model is a mixture-of-experts (MoE) model with the following configuration:

\para{Model Configuration:}
\begin{itemize}
    \item \textbf{Number of Layers:} 192
    \item \textbf{Inner Layer Dimension:} 49152
    \item \textbf{Embedding Dimension:} 12288
    \item \textbf{Hidden Dimension:} 12288
    \item \textbf{Vocabulary Size:} 64000
    \item \textbf{Number of Attention Heads:} 128
    \item \textbf{Maximum Sequence Length:} 2048
    \item \textbf{Number of Experts:} 8
    \item \textbf{MoE Layer Ratio:} 0.5
    \item \textbf{Top-K Experts:} 2
\end{itemize}

\para{Runtime Configuration:}
\begin{itemize}
    \item \textbf{Virtual Pipeline Parallelism:} 3
    \item \textbf{Micro Batch Size:} 1
    \item \textbf{Global Batch Size:} 1536
    \item \textbf{Max Sequence Length:} 2048
\end{itemize}




\section{Theoretical analysis of wasted GPU ratio for \sys}
\label{appendix:ft-anay}

The count of backup lines as $2K - 2$ will significantly influence the fault tolerance of \sys. We use the expectation of waste ratio caused by GPU failure and fragmentation problem to evaluate this design, the result is shown in \tabref{table:design:1.5ratio}.

For one single working server in the middle of line, the count of breakpoints $B$ on its two sides has the expectation as:

\vspace{-1em}
\begin{equation*}
E_B(\eta = 1,middle) = 2(P_s^K + P_s^{2K})
\end{equation*}

Where $P_s$ is the fail probability of GPU server, and $\eta$ is count of servers. The expectation of breakpoints count is:

Once the distance between one server and the tail of line is $\alpha < K$, it will connect to all servers between itself and the last one, so there will be no breakpoints on this side, and the expectation of breakpoints count is less than servers in the middle of line.
Then, for any server in the line topology:

\vspace{-1em}
$$
E_B(\eta = 1) \leq E_B(\eta = 1,middle) 
$$

When the distance between two servers is $\beta \geq K$, the breakpoints among them can be calculated as independent.
Once the distance $\beta < K$, as all servers in this range are connected to these two servers, there will be no breakpoints between them. So, the expectation is less than two independent servers. Then,



\vspace{-1em}
\begin{align*}
E_B(\eta =& 2) < E_B(\eta = 2, \beta \geq K) =  2E(\eta = 1)   \\ 
 E_B(\eta =& N_s) \leq N_s E_B(\eta = 1) 
\end{align*}

For a LLM job which require a ring communication size (TP .etc) as $N_t$, \sys   will cut the whole line topology into several sub lines with the length of $N_t/R$.
Once \sys is cutting a new sub line from the remaining servers in the line, 
all $N_t$ GPU will be wasted when one break point exist in the middle of this sub line required, shown in \fig{fig:subline-waste}. 
Then the expectation for waste GPU caused by one single break point is:

\vspace{-1em}
$$
E_W(B=1) = N_t R\cdot (1 - (N_t/R)^{-1} ) = R(N_t -R)
$$

\begin{figure}[h!t]
    \centering
    \includegraphics[width=0.8\linewidth]{figs/design/intra-topo/break-topo.drawio.pdf}
    \caption{Break point can cause server waste compare to ideal situation.}
    \vspace{-1em}
    \label{fig:subline-waste}
\end{figure}

As the influence between two break points only reduce the expectation of wasted GPUs, we can have this for $X$ break points:

\vspace{-1em}
\begin{equation*}
E_W(B = X) \leq XE_W(B=1) = XR(N_t-R)
\end{equation*}

So the expectation of wasted GPU for a servers cluster with $N_s$ GPU servers is:

\vspace{-1em}
\begin{align*}
E_W(\eta = N_s) &\leq \sum P(B=X ,\eta = N_s) \cdot X\cdot  E_W(B=1)\\
&= E_B(\eta = N_s)\cdot E_W(B=1)\\
&\leq  \lim_{P_s\rightarrow 0}2N_s\cdot R \cdot (N_t-R)P_s^K
\end{align*}



The final expectation of GPUs waste ratio is \eqref{eq:design:ratio}:

\begin{equation}
E_{WR}(\eta = N_s) = \frac{E_W(\eta = N_s)}{N_g} \leq 2(N_t-R)(P_s)^K
\label{eq:design:ratio}
\end{equation}

In our trace for a 160 days long pre-train job on 10K-GPU, the p99 failure rate for 8-card machines is 7\%. If a TP32 jobs is running on \sys, we can get the upper bond for waste ratio expectation for various configuration in \tabref{table:design:1.5ratio}.

\begin{table}[h!t]
\centering
\begin{tabular}{cccc}
    \toprule
        & $K=2$&$K=3$&$K=4$\\
    \midrule
     R=4& $7.35\%$ & $0.26\%$ & $9.00\times 10^{-4}$ \\
     R=8& $27.4\%$ & $1.92\%$ & $0.13\%$ \\
     \bottomrule
\end{tabular}
\caption{Upper bond for waste ratio expectation of GPU, where GPU failure rate is 0.875\% and X is 32}
\vspace{-2em}
\label{table:design:1.5ratio}
\end{table}

As shown in the table, for 4 GPU server ($R=4$) 3 bundles ($K = 3$) design, the additional waste of GPU is less than 0.26\%, while the waste ratio for $R=8,K=4$ is less than 0.13\%. This is sufficient for production clusters. 

\section{Orchestration For Fat-Tree}
\label{appendix:orch-algo}
In this section, we introduce the orchestration algorithm under Fat-Tree DCN in detail.

\para{Notations}
\label{appendix:orch-algo:notation}
To ensure rigorous mathematical reasoning, we introduce the following notations:

\begin{itemize}
    \item {
        $n$: number of nodes in the data-center.
    }
    \item {
        $K$: \docs{} bundle (see \S\ref{section:design:topology}).
    }
    \item {
        $S_{all}$: ordered set, represents all nodes numbered from 1 according to their physical connection order in DCN fabric. $|S_{all}|=n$.
    }
    \item {
        $S$: ordered subset, represents nodes, $\forall u \in S, u \in S_{all}$. Adjacent elements in $S$ are also adjacent from the perspective of the \SYS{} topology. 
    }
    \item{
        $E$: The set of edges across $S$, should be equal to $\{ (S_i, S_j) \mid 1 \leq i < j \leq n, j - i \leq K \} $, representing the connections between nodes, including both primary and backup links, and $O(|E|) = O(K|S|)$.
    }
    \item {
        $InfHBD=<S,E>$: the topology of \SYS{} as an undirected graph.
    }
    \item {
        $F$: faulty nodes.
    }
    \item {
        $HealthyHBD=<H,HE>$: healthy node subgraph where the set of healthy nodes $H = S - F$ and the edge set $HE = \{ (u, v) \mid u \in H \text{ and } v \in H \text{ and } (u, v) \in E \}$.
    }
    \item{
        $t$: TP size, number of GPUs in one TP Group.
    }
    \item{
        $r$: GPU ranks per node.
    }
    \item{
        $m=t/r$: number of nodes in a TP group.
    }
    % \item{
    %     $k$: number of rails in rail-optimized network.
    % }
    \item{
        $s$: job scale, number of GPUs required for the job.
    }
    \item{
        $d$: Aggregation-Switches Domain size. Number of nodes under coverage of one group of Aggregation-Switches.
    }
    \item{
        $n_{constrains}$: number of applied constraints in binary-search-based orchestration algorithm.
    }
    \item{
        $p$: number of nodes under each ToR.
    }
    \item{
        $l$: shortest sub-line length under fat-tree orchestration.
    }
    \item{
        $n_{maxsubline}=\lfloor \frac{nd}{p} \rfloor$: max number of sub-lines.
    }
    \item{
        $G_{deploy}=<S_{deploy},E_{deploy}>$: deployed topology. After applying the deployment strategy, the topology from the perspective of \SYS{} is described as follows: $S_{\text{deploy}}$ is an ordered set where adjacent elements correspond to adjacent nodes in \SYS{}, and $E_{\text{deploy}}$ represents the connections between nodes.
    }
    
\end{itemize}


% For the \SYS{} the orchestration algorithm in ideal conditions is relatively straightforward. The detailed steps of the algorithm are outlined in \algref{alg:orchestration-ideal}.

% Assume that the \SYS{}(with \docs{} direction $K$) is represented as an undirected graph $ \text{InfHBD} = \langle S, E \rangle $, where the ordered set of nodes $ S $ represents nodes. Adjacent elements in $S$ are also adjacent from the perspective of the \SYS{} topology. The set of edges $E$ should be equal to $\{ (S_i, S_j) \mid 1 \leq i < j \leq n, j - i \leq K \} $, representing the connections between nodes, including both primary and backup links, and $O(|E|) = O(K|S|)$. The set of faulty nodes is denoted as $ F \subseteq S $.

% The algorithm proceeds as follows:

% \begin{enumerate}
%     \item {\textbf{Extract the Healthy Node Subgraph:} First, extract the subgraph $\text{HealthyHBD} = \langle H, HE \rangle$ where the set of healthy nodes $H = S - F$ and the edge set $HE = \{ (u, v) \mid u \in H \text{ and } v \in H \text{ and } (u, v) \in E \}$. See \algref{alg:orchestration-ideal}.
%     }
%     \item {\textbf{Identify Connected Components:} Next, identify all connected components in the graph $\text{HealthyHBD}$. Faulty nodes may cause disconnections in the \SYS{} fabric, splitting the original cluster into multiple sub-HBDs. These sub-HBDs are the connected components, and TP Groups cannot span across these disconnected sub-HBDs. We use a simple Depth-First Search (DFS) algorithm here. See \algref{alg:dfs}.}
%     \item {\textbf{Generate Placement Scheme:} Given the excellent physical properties of the \SYS{}, TP Groups can be arranged sequentially within each connected component to generate placement scheme maximizing GPU utilization. See \algref{alg:orchestration-ideal}.
%     }
% \end{enumerate}

% Since each of the three steps involves traversing the entire graph's edges and nodes only once, 
The orchestration algorithm (\algref{alg:orchestration-ideal}) without considering DCN has the overall time complexity $3\cdot O(|H| + |HE|) = O(|S| + |E|) = O((K+1)|S|) = O(|S|)$.

% \begin{algorithm}[!h]
% \small
% \caption{Connected-Component-DFS}
% \label{alg:dfs}
% \SetAlgoNlRelativeSize{-1}
% \SetAlgoNlRelativeSize{1}
%  \KwIn{ $node$, $HealthyHBD$, $visited$}
%  \KwOut{ $component$}

%  Initialize $stack = [node]$ \;
%  Initialize $component = []$\;

% \While{ stack is not empty}
% {
%      $current = stack.pop()$\;
%     \If{$current$ not in $visited$}
%     {
%          Add $current$ to $visited$\;
%          Add $current$ to $component$\;
%         \For{ each neighbor in $HealthyHBD.neighbors(current)$}
%         {
%              $stack.push(neighbor)$\;
%         }
%     }
% }
        
% \KwRet{$component$}
% \end{algorithm}

\begin{algorithm}[!h]
\small
\caption{Orchestration-DCN-Free}
\label{alg:orchestration-ideal}
\SetAlgoNlRelativeSize{-1}
\SetAlgoNlRelativeSize{1}
\KwIn{$\text{InfHBD}=\langle S, E \rangle$, $F$, $m$}
\KwOut{ Placement scheme maximizing GPU utilization}

 Initialize $H = S - F$\;
 Initialize $HE = \{ (u, v) \mid u \in H \text{ and } v \in H \text{ and } (u, v) \in E \}$\;
 Create subgraph $HealthyHBD = \langle H, HE \rangle$\;
 Initialize $component\_list = []$\;
 Initialize $visited = \{\}$\;
 Initialize $placement\_scheme= \{\}$\;

\For{ each node $s$ in $H$}
{
    \uIf{ $s$ not in $visited$}
    {
         $component = Connected-Component-DFS(s, HealthyHBD, visited)$\;
         Add $component.sortedinHBD()$ to $component\_list$\;
    }
}
\For{ each $component$ in $component\_list$}
{
    \While{ $component.size()\geq m$}
    {
         Add $component.pop(m)$ to $placement\_scheme$\;
    }
}
        
 \KwRet{$placement\_scheme$}
 \end{algorithm}
 
% \subsection{Algorithms under Rail-Optimized Network}
% \label{appendix:orch-algo:rail-optimized}

% This subsection provides a detailed description of the orchestration algorithm for Rail-Optimized network.  

% The rail-optimized network topology is specifically designed for highly regular machine learning workload traffic patterns, making it a commonly used and effective architecture. As illustrated in \fig{fig:rail-topo}, Rail Switch $i$ connects to GPU $i$ in node, dividing the network into multiple rails. Let $r$ denote the GPU ranks per node, and $k$ the number of rails. In traditional rail-optimized networks, $k = r$, and a typical training strategy involves running TP $r$ within the single-node HBD, while DP operates between HBDs. Since in DP, GPUs only communicate with GPUs of the same rank in different TP groups, in other words, DP traffic is confined to the rail itself. Therefore, the Rail-Optimized topology perfectly meets this requirement.

% % \begin{figure}[!h]
% %     \centering
% %     \includegraphics[width=\linewidth]{figs/design/Orchestration/rail-optimized.drawio.pdf}
% %     \caption{Rail-Optimized Network: GPU ranks per node $r=4$, Number of rails $k=8$, Aggregation-Switches Domain size $d$, Number of Aggregation-Switches Domain $nd$, Node IDs from 1 to $nd\cdot d$. }
% %     \label{fig:rail-topo}
% % \end{figure}

% \para{Orchestration Constraints. }To minimize the cross-rail traffic which can lead to congestion and latency, the rail-optimized network introduces two key constraints for orchestration algorithms:


% \begin{itemize}
%     \item {
%         \textbf{Aggregation-Switches Domain Coverage Constraint. }
%         The coverage domian of a group of Aggregation-Switches is limited, meaning that TP groups spanning across Aggregation-Switches domains would result in cross-rail traffic, which should be avoided as much as possible.
%     }
%     \item {
%         \textbf{Node Rail State Constraint. }When$ k = r$, this constraint does not apply, as there is no cross-rail traffic.However, as HBDs extend beyond single nodes and the need for larger DP scales due to the expansion of LLM scale, scenarios with $k = p \cdot r$ may arise. This results in $p$ different node states within the data center, with each state occupying $r$ rails, and inter-state communication leads to cross-rail traffic. The specific form of this constraint depends on the deployment strategy.
%     }
% \end{itemize}

% \para{Deployment Strategy. }If the \SYS{} connections continue to follow the physical layout of nodes on the DCN Fabric, avoiding cross-rail traffic would require each TP Group to have an equal number of nodes from each state, making the algorithm to maximize GPU Utilization NP-Complete (see Appendix.\ref{appendix:np-hard-orchestration}). However, by altering the physical connection sequence of \SYS{}, this NP-Complete problem can be reduced to polynomial time. As shown in \fig{fig:parallel-line}, nodes of each state are arranged into $p$ parallel sub-lines, which are then connected end-to-end to form a single line. By restricting DP to operate within sub-lines, all DP traffic remains within the rails, effectively reducing the $k = p * r$ scenario to $k = r$. 

% % \begin{figure}[!h]
% %     \centering
% %     \includegraphics[width=\linewidth]{figs/design/Orchestration/parallel-line.drawio.pdf}
% %     \caption{The deployment strategy example with $p=4$ and Aggregation-Switches Domain size $K=8$. Node IDs from 1 to n are arranged according to their connection order in the DCN Fabric.}
% %     \label{fig:parallel-line}
% % \end{figure}

% \para{The binary search-based Orchestration algorithm.} Based on the above-mentioned constraints and the deployment strategy, we developed an orchestration algorithm that maximizes the number of constraints satisfied while meeting the job scale requirements. This is achieved using a binary search approach with the number of satisfied constraints as the variable. Both types of constraints essentially involve splitting the Line into sub-lines. Therefore, controlling the number of constraints translates to managing the number of sub-lines: fewer sub-lines mean longer sub-lines, leading to higher GPU Utilization. Since the Ideal orchestration algorithm with complexity $O(n)$ can be applied within sub-lines.

% \algref{alg:orchestration-fat-tree} is the main binary-search-based orchestration algorithm. It begins by generating the topology from the perspective of \SYS{} based on the hardware deployment strategy (\algref{alg:deployment-strategy}). Using the number of satisfied constraints as a variable, the algorithm performs a binary search to identify the placement scheme that maximizes the number of satisfied constraints while meeting the job scale requirements.  

% \algref{alg:placement-rail-optimized} calculates the placement scheme for a given number of constraints. It divides the topology into multiple ideal sub-lines and applies the ideal-case orchestration algorithm (\algref{alg:orchestration-ideal}) to each sub-line.  

% Since the time complexity of \algref{alg:orchestration-ideal} is $O(|S|)$, the complexity of \algref{alg:placement-rail-optimized} is 

% \begin{align*}
% &\sum_{i=1}^{n_{constraints}} O(|S_{subline}|) \\
% &= O(\sum_{i=1}^{n_{constraints}} |S_{subline}|) \\
% &= O(|S_{all}|) = O(n)
% \end{align*}

% Thus, the overall time complexity of \algref{alg:orchestration-rail-optimized} is $O(n \log n)$.

\begin{algorithm}[!h]
\small
\caption{Deployment-Strategy}
\label{alg:deployment-strategy}
\SetAlgoNlRelativeSize{-1}
\SetAlgoNlRelativeSize{1}
 \KwIn{Node ordered set $S$, \docs{} direction $K$, parallel factor $p$}
 \KwOut{Deployment topology $G_{deploy}=<S_{deploy},E_{deploy}>$}
 Initialize ordered set $S_{deploy}=[]$\;
 Initialize $l=\lfloor \frac{|S|}{p}\rfloor$\;
\For{$i$ in $0...p-1$}
{
    \For{$j$ in $0...l-1$}{
         Add $i+j\cdot p$ to $S_{deploy}$\;}
}
 Create $E_{deploy}=\{(S_{deploy}^i,S_{deploy}^j)|1\leq i\le j\leq |S_{deploy}|, j-i\leq K \}$\;
 \KwRet{$G_{deploy}=<S_{deploy},E_{deploy}>$}
\end{algorithm}


% \begin{algorithm}[!h]
% \small
% \caption{Placement-Rail-Optimized}
% \label{alg:placement-rail-optimized}
% \SetAlgoNlRelativeSize{-1}
% \SetAlgoNlRelativeSize{1}
%  \KwIn{Deployment topology $G_{deploy}=<S_{deploy},E_{deploy}>$, Number of applied constraints $n_{constraints}$, Faulty node $F$, Sub-line length $l$, Number of node in one TP group $m$}
%  \KwOut{Placement scheme}
%  Initialize $placement\_scheme=\{\}$\;
% \For{$i$ in $1..n_{constraints}$}
% {
%      $S_{subline}=S_{deploy}.pop(l)$\;
%      $E_{subline}=\{(u,v)\mid u\in S_{subline} \text{ and } v\in S_{subline} \text{ and } (u,v)\in E_{subline}\}$\;
%      $F_{subline}=F\cap S_{subline}$\;
%      $placement\_scheme=placement\_scheme\cup \text{Orchestration-Ideal}(<S_{subline},E_{subline}>, F_{subline}, m)$\;
% }
%  $E_{res}=\{(u,v)\mid u \in S_{deploy} \text{ and } v \in S_{deploy} \text{ and } (u,v) \in E_{deploy}\}$\;
%  $F_{res}=F\cap S_{deploy}$\;
%  $placement\_scheme=placement\_scheme\cup \text{Orchestration-Ideal}(<S_{deploy},E_{res}>, F_{res},m)$\;
%  \KwRet{$placement\_scheme$}
% \end{algorithm}


% \begin{algorithm}[!h]
% \small
% \caption{Orchestration-Rail-Optimized}
% \label{alg:orchestration-rail-optimized}
% \SetAlgoNlRelativeSize{-1}
% \SetAlgoNlRelativeSize{1}
%  \KwIn{Node ordered set $S$ (from 1 to n in DCN Fabric), GPU ranks per node $r$, Number of rails $k$, Faulty set $F$, TP size $t$, Job scale $s$ (number of GPUs required for the job), Aggregation-Switches Domain size $d$, \docs{} directions $K$.}
%  \KwOut{Placement scheme that satisfies job scale and minimizes cross-rail traffic.}
%  Initialize $p=k/r$, $m=t/r$, $n=|S|$, $l=\lfloor \frac{d}{p}\rfloor$\;
%  Create graph $G_{deploy}=<S_{deploy},E_{deploy}>=\text{Deployment-Strategy}(S,K,p)$\;
%  Initialize $high=\lfloor\frac{nd}{p}\rfloor$\;
%  Initialize $low=0$\;
%  Initialize $placement\_scheme=\{\}$\;
% \While{ $low \leq$ high}
% {
%      $mid=\lfloor \frac{low+high}{2} \rfloor$\;
%      $placement\_scheme=\text{Placement-Rail-Optimized}(G_{deploy},mid,F,l,m)$\;
%     \eIf {$|placement\_scheme|\cdot m\cdot r\ge s$}
%     {
%          $low=mid+1$\;
%     }
%     {
%          $high=mid-1$\;
%     }
% }
    
% \eIf{$|placement\_scheme|\cdot m\cdot r\ge s$}
% {
%   \KwRet {$placement\_scheme$}
% }
% {
%     \KwRet {None}
% }
% \end{algorithm}
  

Fat-Tree topology is another common data center topology. A typical training strategy for this topology aims to maximize the bandwidth utilization under ToR (Top of Rack) Switches. Using Meta's two-stage clos topology\cite{sigcomm2024meta} as a reference, it can be observed that there is an attempt to run CP under ToR.

\para{Deployment Strategy:} Assuming there are $p$ nodes under each ToR, nodes with the same index under each ToR are deployed along the same parallel sub-line, and the $p$ sub-lines are connected end-to-end, as shown in \fig{fig:fat-tree-topo}. The training strategy involves running CP $p$ across the sub-lines and running TP within them.

\para{Orchestration Constraints. }To maximize the utilization of ToR bandwidth and minimize cross-ToR traffic, the fat-tree topology introduces two constraints:

\begin{packeditemize}
    \item {
        \textbf{Aggregation-Switches Domain Constraint: }The coverage domian of a group of Aggregation Switches is limited, meaning that TP groups spanning across Aggregation Switches domains would result in cross-rail traffic, which should be avoided as much as possible.
    }
    \item {
        \textbf{TP Group Alignment Constraint: } A CP Group consists of TP Groups across parallel sub-lines. To keep CP traffic within the ToR, the TP Groups must be aligned. If a node fails under one ToR, all nodes under that ToR are considered failed, expanding the failure radius by a factor of $p$. 
    }
\end{packeditemize}

\para{Binary-Search-Based Orchestration Algorithm.} Based on the constraints and deployment strategy, we develop a binary search orchestration algorithm (see \algref{alg:orchestration-fat-tree}) that adjusts the number of satisfied constraints. The binary search first relaxes the TP Group alignment constraints within the Aggregation-Switches Domain and then relaxes the TP Group crossing constraints between Aggregation-Switch domains (see \algref{alg:placement-fat-tree}). This process is monotonic.


% \begin{figure}[!h]
%     \centering
%     \includegraphics[width=\linewidth]{figs/design/Orchestration/meta-topo.drawio.pdf}
%     \caption{Orchestration example for Fat-Tree Topology under single Aggregation-Switches Domain with $p=2$. Green indicates active node, red indicates faulty node and yellow indicates idle nodes}
%     \label{fig:meta-topo}
% \end{figure}


The time complexity of \algref{alg:orchestration-ideal} is $O(|S|)$, and the complexity of \algref{alg:placement-fat-tree} is 

$$\sum_{i=1}^{n_{subline}} O(|S_{subline}|) = O(\sum_{i=1}^{n_{subline}} |S_{subline}|) = O(|S_{all}|) = O(n)$$  

Thus, the overall time complexity of \algref{alg:orchestration-fat-tree} is $O(n \log n)$.

\begin{algorithm}[!h]
\small
\caption{Placement-Fat-Tree}
\label{alg:placement-fat-tree}
\SetAlgoNlRelativeSize{-1}
\SetAlgoNlRelativeSize{1}
 \KwIn{$G_{deploy}=<S_{deploy},E_{deploy}>$, $n_{constraints}$, $F$, $l$, $m$, $n_{maxsubline}$, $d$, $p$}
 \KwOut{Placement scheme}
 Initialize $placement\_scheme=\{\}$\;
 Initialize $n_{align}=max(0,n_{constraints}-n_{maxsubline})$, $n_{subline}=min(n_{maxsubline},n_{constraints})$\;
 
\For{$i$ in $0..n_{align}-1$}
{
    \For{$j$ in $1..d$}
    {
        $sid=i*d+j$\;
        \If{$sid \in F$}
        {
            $F\cup \{\lfloor \frac{sid-1}{p}\rfloor\cdot p+1..(\lfloor \frac{sid-1}{p}\rfloor+1)\cdot p \}$\;
        }
    }
}
\For{$i$ in $1..n_{subline}$}
{
     $S_{subline}=S_{deploy}.pop(l)$\;
     $E_{subline}=\{(u,v)\mid u\in S_{subline} \text{ and } v\in S_{subline} \text{ and } (u,v)\in E_{subline}\}$\;
     $F_{subline}=F\cap S_{subline}$\;
     $placement\_scheme=placement\_scheme\cup \text{Orchestration-Ideal}(<S_{subline},E_{subline}>, F_{subline}, m)$\;
}
 $E_{res}=\{(u,v)\mid u \in S_{deploy} \text{ and } v \in S_{deploy} \text{ and } (u,v) \in E_{deploy}\}$\;
 $F_{res}=F\cap S_{deploy}$\;
 $placement\_scheme=placement\_scheme\cup \text{Orchestration-Ideal}(<S_{deploy},E_{res}>, F_{res},m)$\;
 \KwRet{$placement\_scheme$}
\end{algorithm}

\begin{algorithm}[!h]
\small
\caption{Orchestration-Fat-Tree}
\label{alg:orchestration-fat-tree}
\SetAlgoNlRelativeSize{-1}
\SetAlgoNlRelativeSize{1}
 \KwIn{$S$, $r$, $p$, $F$, $t$, $s$, $d$, $K$.}
 \KwOut{Placement scheme that satisfies job scale and minimizes cross-rail traffic.}
 Initialize $m=t/r$, $n=|S|$, $l=\lfloor\frac{d}{p}\rfloor$\, $n_{domain}=\lfloor\frac{n}{d}\rfloor$, $n_{maxsubline}=\lfloor\frac{nd}{p}\rfloor$\;
 Create graph $G_{deploy}=<S_{deploy},E_{deploy}>=\text{Deployment-Strategy}(S,K,p)$\;
 Initialize $high=n_{domain}+n_{maxsubline}$\;
 Initialize $low=0$\;
 Initialize $placement\_scheme=\{\}$\;
\While{ $low \leq$ high}
{
     $mid=\lfloor \frac{low+high}{2} \rfloor$\;
     $placement\_scheme=\text{Placement-Fat-Tree}(G_{deploy},mid,F,l,m,n_{maxsubline},d,p)$\;
    \eIf {$|placement\_scheme|\cdot m\cdot r\ge s$}
    {
         $low=mid+1$\;
    }
    {
         $high=mid-1$\;
    }
}
    
\eIf{$|placement\_scheme|\cdot m\cdot r\ge s$}
{
    \KwRet {$placement\_scheme$}
}
{
    \KwRet {None}
}
\end{algorithm}





\section{Additional Simulation Results for Fault Resilience}
\label{appendix:wasted-GPUs-ratio}
This section presents additional simulation results related to \S\ref{sec:simulation:fault}. \figref{fig:simulation:wasted-trace} shows the variation of the GPU waste ratio over time under the production fault trace. \figref{fig:simulation:waste-cdf:gr4:supple} presents the CDF data for the GPU waste ratio. \figref{fig:simulation:model:wasted-gr4} illustrates the waste GPU ratio for different HBD architectures under various node failure rates, including the results for TP-8 to TP-64. \figref{fig:simulation:breakdown-duration-supple} shows the proportion of job-fault waiting time relative to total time for different job scales. All the aforementioned experiments include results for TP-8, TP-16, TP-32, and TP-64 configurations.








\begin{figure*}[h!t]
    \centering
    \begin{subfigure}[b]{0.23\linewidth}
        \centering
        \includegraphics[width=\linewidth]{figs/evaluation/fault_trace_based/frag_trace_tp8_gr4.pdf}
        \caption{TP-8.}
        \label{fig:simulation:wasted-trace:tp8-4gpu}
    \end{subfigure}
    \hspace{2pt}
    \begin{subfigure}[b]{0.23\linewidth}
        \centering
        \includegraphics[width=\linewidth]{figs/evaluation/fault_trace_based/frag_trace_tp16_gr4.pdf}
        \caption{TP-16.}
        \label{fig:simulation:wasted-trace:tp16-4gpu}
    \end{subfigure}
    \hspace{2pt}
    \begin{subfigure}[b]{0.23\linewidth}
        \centering
        \includegraphics[width=\linewidth]{figs/evaluation/fault_trace_based/frag_trace_tp32_gr4.pdf}
        \caption{TP-32.}
        \label{fig:simulation:wasted-trace:tp32-4gpu}
    \end{subfigure}
    \hspace{2pt}
    \begin{subfigure}[b]{0.23\linewidth}
        \centering
        \includegraphics[width=\linewidth]{figs/evaluation/fault_trace_based/frag_trace_tp64_gr4.pdf}
        \caption{TP-64.}
        \label{fig:simulation:wasted-trace:tp64-4gpu}
    \end{subfigure}

    \vspace{-1ex}
    \caption{GPU waste ratio over production fault trace, 4 GPU node.}
    \label{fig:simulation:wasted-trace}
\end{figure*}


\begin{figure*}[h!t]
    \centering
    \begin{subfigure}[b]{0.23\linewidth}
        \centering
        \includegraphics[width=\linewidth]{figs/evaluation/fault_trace_based/cdf_trace_waste_tp8_gr4.pdf}
        \caption{TP-8.}
        \label{fig:simulation:waste-cdf:tp8-gr4}
    \end{subfigure}
    \hspace{2pt}
    \begin{subfigure}[b]{0.23\linewidth}
        \centering
        \includegraphics[width=\linewidth]{figs/evaluation/fault_trace_based/cdf_trace_waste_tp16_gr4.pdf}
        \caption{TP-16.}
        \label{fig:simulation:waste-cdf:tp16-gr4}
    \end{subfigure}
    \hspace{2pt}
    \begin{subfigure}[b]{0.23\linewidth}
        \centering
        \includegraphics[width=\linewidth]{figs/evaluation/fault_trace_based/cdf_trace_waste_tp32_gr4.pdf}
        \caption{TP-32.}
        \label{fig:simulation:waste-cdf:tp32-gr4}
    \end{subfigure}
    \hspace{2pt}
    \begin{subfigure}[b]{0.23\linewidth}
        \centering
        \includegraphics[width=\linewidth]{figs/evaluation/fault_trace_based/cdf_trace_waste_tp64_gr4.pdf}
        \caption{TP-64.}
        \label{fig:simulation:waste-cdf:tp64-gr4}
    \end{subfigure}
    \vspace{-1ex}
    \caption{CDF of GPU waste ratio over production fault trace, 4 GPU node.}
    \label{fig:simulation:waste-cdf:gr4:supple}
\end{figure*}


\begin{figure*}[h!t]
    \centering
    \begin{subfigure}[b]{0.23\linewidth}
        \centering
        \includegraphics[width=\linewidth]{figs/evaluation/fault_model_based/frag_ratio_tp8_gr4.pdf}
        \caption{TP-8.}
        \label{fig:simulation:model:wasted:tp8}
    \end{subfigure}
    \hspace{2pt}
    \begin{subfigure}[b]{0.23\linewidth}
        \centering
        \includegraphics[width=\linewidth]{figs/evaluation/fault_model_based/frag_ratio_tp16_gr4.pdf}
        \caption{TP-16.}
        \label{fig:simulation:model:wasted:tp16}
    \end{subfigure}
    \hspace{2pt}
    \begin{subfigure}[b]{0.23\linewidth}
        \centering
        \includegraphics[width=\linewidth]{figs/evaluation/fault_model_based/frag_ratio_tp32_gr4.pdf}
        \caption{TP-32.}
        \label{fig:simulation:model:wasted:tp32}
    \end{subfigure}
    \hspace{2pt}
    \begin{subfigure}[b]{0.23\linewidth}
        \centering
        \includegraphics[width=\linewidth]{figs/evaluation/fault_model_based/frag_ratio_tp64_gr4.pdf}
        \caption{TP-64.}
        \label{fig:simulation:model:wasted:tp64}
    \end{subfigure}
    \vspace{-1ex}
    \caption{GPU wastes ratio with different GPU fault ratio, 4-GPU node.}
    \label{fig:simulation:model:wasted-gr4}
\end{figure*}



\begin{figure*}[h!t]
    \centering
    \begin{subfigure}[b]{0.23\linewidth}
        \centering
        \includegraphics[width=\linewidth]{figs/evaluation/fault_trace_based/breakdown_ratio_tp8_gr4.pdf}
        \caption{TP-8.}
        \label{fig:simulation:breakdown-duration:tp8-4gpu}
    \end{subfigure}
    \hspace{2pt}
    \begin{subfigure}[b]{0.23\linewidth}
        \centering
        \includegraphics[width=\linewidth]{figs/evaluation/fault_trace_based/breakdown_ratio_tp16_gr4.pdf}
        \caption{TP-16.}
        \label{fig:simulation:breakdown-duration:tp16-4gpu}
    \end{subfigure}
    \hspace{2pt}
    \begin{subfigure}[b]{0.23\linewidth}
        \centering
        \includegraphics[width=\linewidth]{figs/evaluation/fault_trace_based/breakdown_ratio_tp32_gr4.pdf}
        \caption{TP-32.}
        \label{fig:simulation:breakdown-duration:tp32-4gpu}
    \end{subfigure}
    \hspace{2pt}
    \begin{subfigure}[b]{0.23\linewidth}
        \centering
        \includegraphics[width=\linewidth]{figs/evaluation/fault_trace_based/breakdown_ratio_tp64_gr4.pdf}
        \caption{TP-64.}
        \label{fig:simulation:breakdown-duration:tp64-4gpu}
    \end{subfigure}
    \vspace{-1ex}
    \caption{Job fault-waiting duration with different levels of job-scale, 4 GPU node}
    \label{fig:simulation:breakdown-duration-supple}
\end{figure*}





\vspace{-12em}
\section{Detailed Cost and power consumption Analysis}
\label{appendix:cost}
In this section, \tabref{tab:eval:components} provides a detailed description of the quantity, cost, bandwidth, and power consumption of the interconnect components in various network architectures, including Google TPUv4~\cite{isca2023tpu}, NVIDIA GB200 NVL series~\cite{nvl72}, Alibaba HPN\cite{sigcomm2024hpn}, and \sys{}.


\begin{table*}[h!t] \small
    \centering
    \begin{tabular}{lllll}
    \toprule
    
    \textbf{Component} & \textbf{Quantity} & \textbf{Unit Cost (\$)}  & \textbf{Unit Bandwidth (GBps)} & \textbf{Unit Power (W)} \\

    \midrule
    \multicolumn{5}{c}{\textbf{Google TPUv4\cite{isca2023tpu} with 4096 GPU, bandwidth 300GBps/GPU}} \\
    
    \midrule
    OCS\cite{sigcomm2023lightwave} & 48 & 80000 & 6400 & 108 \\
    DAC Cable\cite{400G_DAC} & 5120 & 63.60 & 50 & 0.1 \\
    Optical Module\cite{400G_OPTICAL_MODULE} & 6144 & 360 & 50 & 12  \\
    Fiber\cite{FIBER}& 6144 & 6.80 & 50 & 0 \\
    
    \midrule
    \multicolumn{5}{c}{\textbf{NVIDIA GB200 NVL-36\cite{SEMIANALYSIS_GB200} with 36 GPU, bandwidth 900GBps/GPU}}\\
    \midrule
    NVLink Switch\cite{SEMIANALYSIS_Power} & 9 & 28000 & 3600 & 275 \\
    DAC Cable\cite{200G_DAC} & 2592 & 35.60 & 25 & 0.1 \\
    
    \midrule
    \multicolumn{5}{c}{\textbf{NVIDIA GB200 NVL-72\cite{nvl72}\cite{SEMIANALYSIS_GB200} with 72 GPU, bandwidth 900GBps/GPU}}\\
    \midrule
    NVLink Switch\cite{SEMIANALYSIS_Power} & 18 & 28000 & 3600 & 275 \\
    DAC Cable\cite{200G_DAC} & 5184 & 35.60 & 25 & 0.1 \\
    \midrule
    \multicolumn{5}{c}{\textbf{NVIDIA GB200 NVL-36x2\cite{SEMIANALYSIS_GB200} with 72 GPU, bandwidth 900GBps/GPU}}\\
    \midrule
    NVLink Switch\cite{SEMIANALYSIS_Power} & 36 & 28000 & 3600 &  275\\
    DAC Cable\cite{200G_DAC} & 6480 & 35.60 & 25 & 0.1 \\
    ACC Cable\cite{SEMIANALYSIS_Power} & 162 & 320 & 200 & 2.5 \\

    \midrule
    \multicolumn{5}{c}{\textbf{NVIDIA GB200 NVL-576\cite{SEMIANALYSIS_GB200} with 576 GPU, bandwidth 900GBps/GPU}}\\
    \midrule
    NVLink Switch\cite{SEMIANALYSIS_Power} & 432 & 28000 & 3600 & 275 \\
    DAC Cable\cite{200G_DAC} & 41472 & 35.60 & 25 & 0.1 \\
    Optical Module\cite{OSFPXD} & 4608 & 850 & 200 & 25 \\
    Fiber\cite{FIBER} & 4608 & 6.80 & 200 & 0 \\

    \midrule
    \multicolumn{5}{c}{\textbf{Alibaba HPN\cite{sigcomm2024hpn} with 16320 GPU, bandwidth 50GBps/GPU}}\\
    \midrule
    EPS\cite{51.2T_EPS} & 360 & 14960 & 6400 & 3145 \\
    DAC Cable\cite{200G_DAC} & 32640 & 35.60 & 25 & 0.1\\
    Optical Module\cite{400G_OPTICAL_MODULE} & 28800 & 360 & 50 & 12 \\
    Fiber\cite{FIBER} & 14400 & 6.80 & 50 & 0 \\

    \midrule
    \multicolumn{5}{c}{\textbf{\SYS{}($K=2$)  with 4 GPU, bandwidth 800GBps/GPU}}\\
    \midrule
    DAC Cable\cite{1.6T_DAC}& 4 & 199.60 & 200 & 0.1\\
    dOCS Module & 16 & 600 & 100 & 12 \\
    Fiber\cite{FIBER} & 16 & 6.80 & 100 & 0 \\

    \midrule
    \multicolumn{5}{c}{\textbf{\SYS{}($K=3$)  with 4 GPU, bandwidth 800GBps/GPU}}\\
    \midrule
    DAC Cable\cite{1.6T_DAC} & 2 & 199.60 & 200 & 0.1\\
    dOCS Module & 24 & 600 & 100 & 12 \\
    Fiber\cite{FIBER} & 24 & 6.80 & 100 & 0 \\
    \bottomrule
    \end{tabular}
    \caption{Interconnect cost and power consumption of components used in different network architectures.}
    \label{tab:eval:components}
\end{table*}


\end{appendices}





\end{document}
