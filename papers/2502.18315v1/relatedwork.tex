\section{Related work}
In the dynamic landscape of resume information extraction, diverse methodologies have been explored, each with distinct approaches and advantages. Traditional techniques like keyword search-based methods, rule-based methods, and semantic-based methods have been prominent, yet they often struggle to grasp the intricate nuances of skills and experiences due to the ever-evolving nature of job descriptions and applicant backgrounds \cite{chen2018two}.

Recent strides have introduced innovative methods leveraging graph-based structures for skill processing.\cite{velampalli2016novel} pioneered a graph-based approach utilizing the MapReduce Programming model, enabling the extraction of common skill-sets from extensive resume datasets. Their foundational work integrated graph theory into talent analytics, offering a promising avenue for nuanced skill assessment.

The integration of sentiment analysis into skill evaluation marks a groundbreaking development. \cite{maheshwari2010approach}
introduced a methodology focused on feature selection, enhancing the efficiency of the resume selection process. Their research showcased a substantial reduction (50-94\%) in the number of features recruiters needed to review, highlighting the potential of advanced techniques in streamlining hiring.

 Additionally, \cite{nasr2019assessment} proposed a method combining the modified Boyer-Moore Method and Dice metrics-based string similarity verification. This hybrid approach empowers employees to sift through overwhelming resume databases efficiently, ensuring accurate query results.

The emergence of video resumes has presented novel challenges and opportunities. \cite{chen2018two} introduced a framework for processing video resumes, analyzing the formation of personality traits and hirability impressions. This multimodal approach, integrating visual and verbal cues, provides a holistic understanding of candidates, revolutionizing how recruiters assess potential hires.

In recent years, the fusion of natural language processing and deep learning techniques has significantly enhanced resume parsing and skill extraction. \cite{johnston2001information} proposed methods enhancing resume parsing through natural language processing techniques, offering a more nuanced understanding of applicant qualifications. \cite{jones2021deep} delved into semantic resume parsing and skill extraction using deep learning, providing a more sophisticated approach to understanding the context and relevance of skills mentioned in resumes.

Furthermore, \cite{wang2022graph} introduced a graph-based resume analysis method tailored for job matching. Their approach utilized graph structures to represent both job requirements and applicant skills, enabling a more comprehensive and accurate matching process. This innovative method has opened new horizons in the realm of resume analysis and job matching.

In our pioneering work, we have taken a significant leap by attributing weights to sentiment words associated with skills. This innovative step enables us to attach these sentiment weights to edges within the graph structure. This nuanced approach not only refines the ranking method for jobseekers and organizations but also opens avenues for more granular skill evaluation. By integrating these latest advancements with our innovative graph-based methodology, we aim to revolutionize talent analytics, providing a comprehensive and accurate means of assessing candidates' skills and experiences.