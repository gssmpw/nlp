% CVPR 2025 Paper Template; see https://github.com/cvpr-org/author-kit

\documentclass[10pt,twocolumn,letterpaper]{article}

%%%%%%%%% PAPER TYPE  - PLEASE UPDATE FOR FINAL VERSION
% \usepackage{cvpr}              % To produce the CAMERA-READY version
% \usepackage[review]{cvpr}      % To produce the REVIEW version
\usepackage{multirow}
\usepackage{graphicx} 
\usepackage{float} 
\usepackage{array}
\usepackage{svg}
\usepackage{subcaption}
\usepackage{colortbl} 
% \usepackage{xcolor} 
% \usepackage{parskip}
\usepackage{amssymb}
\usepackage{pifont}
\usepackage{fontawesome}
\usepackage{siunitx}


\definecolor{lightblue}{RGB}{197, 234, 228} 
\newcommand{\cmark}{\textcolor{green}{\ding{51}}} % Green check mark
\newcommand{\xmark}{\textcolor{red}{\ding{55}}}   % Red cross mark


\usepackage[pagenumbers]{cvpr} % To force page numbers, e.g. for an arXiv version

% Import additional packages in the preamble file, before hyperref
%
% --- inline annotations
%
\newcommand{\red}[1]{{\color{red}#1}}
\newcommand{\todo}[1]{{\color{red}#1}}
\newcommand{\TODO}[1]{\textbf{\color{red}[TODO: #1]}}
% --- disable by uncommenting  
% \renewcommand{\TODO}[1]{}
% \renewcommand{\todo}[1]{#1}



\newcommand{\VLM}{LVLM\xspace} 
\newcommand{\ours}{PeKit\xspace}
\newcommand{\yollava}{Yo’LLaVA\xspace}

\newcommand{\thisismy}{This-Is-My-Img\xspace}
\newcommand{\myparagraph}[1]{\noindent\textbf{#1}}
\newcommand{\vdoro}[1]{{\color[rgb]{0.4, 0.18, 0.78} {[V] #1}}}
% --- disable by uncommenting  
% \renewcommand{\TODO}[1]{}
% \renewcommand{\todo}[1]{#1}
\usepackage{slashbox}
% Vectors
\newcommand{\bB}{\mathcal{B}}
\newcommand{\bw}{\mathbf{w}}
\newcommand{\bs}{\mathbf{s}}
\newcommand{\bo}{\mathbf{o}}
\newcommand{\bn}{\mathbf{n}}
\newcommand{\bc}{\mathbf{c}}
\newcommand{\bp}{\mathbf{p}}
\newcommand{\bS}{\mathbf{S}}
\newcommand{\bk}{\mathbf{k}}
\newcommand{\bmu}{\boldsymbol{\mu}}
\newcommand{\bx}{\mathbf{x}}
\newcommand{\bg}{\mathbf{g}}
\newcommand{\be}{\mathbf{e}}
\newcommand{\bX}{\mathbf{X}}
\newcommand{\by}{\mathbf{y}}
\newcommand{\bv}{\mathbf{v}}
\newcommand{\bz}{\mathbf{z}}
\newcommand{\bq}{\mathbf{q}}
\newcommand{\bff}{\mathbf{f}}
\newcommand{\bu}{\mathbf{u}}
\newcommand{\bh}{\mathbf{h}}
\newcommand{\bb}{\mathbf{b}}

\newcommand{\rone}{\textcolor{green}{R1}}
\newcommand{\rtwo}{\textcolor{orange}{R2}}
\newcommand{\rthree}{\textcolor{red}{R3}}
\usepackage{amsmath}
%\usepackage{arydshln}
\DeclareMathOperator{\similarity}{sim}
\DeclareMathOperator{\AvgPool}{AvgPool}

\newcommand{\argmax}{\mathop{\mathrm{argmax}}}     



% It is strongly recommended to use hyperref, especially for the review version.
% hyperref with option pagebackref eases the reviewers' job.
% Please disable hyperref *only* if you encounter grave issues, 
% e.g. with the file validation for the camera-ready version.
%
% If you comment hyperref and then uncomment it, you should delete *.aux before re-running LaTeX.
% (Or just hit 'q' on the first LaTeX run, let it finish, and you should be clear).
\definecolor{cvprblue}{rgb}{0.21,0.49,0.74}
\usepackage[pagebackref,breaklinks,colorlinks,allcolors=cvprblue]{hyperref}

\newcommand{\yapeng}[1]{\textcolor{red}{[Yapeng: #1]}}
\newcommand{\wj}[1]{\textcolor{blue}{[Wenjie: #1]}}
\newcommand{\yunhui}[1]{\textcolor{green}{[Yunhui: #1]}}
\newcommand{\ziru}[1]{\textcolor{purple}{[Ziru: #1]}}
\newcommand{\jia}[1]{\textcolor{magenta}{[Jia: #1]}}

%%%%%%%%% PAPER ID  - PLEASE UPDATE
\def\paperID{11939} % *** Enter the Paper ID here
\def\confName{CVPR}
\def\confYear{2025}

%%%%%%%%% TITLE - PLEASE UPDATE
\title{Do Audio-Visual Segmentation Models Truly Segment Sounding Objects?}

%%%%%%%%% AUTHORS - PLEASE UPDATE
\author{
Jia Li$^{1}$ \hspace{1cm} Wenjie Zhao $^{1}$ \hspace{1cm} Ziru Huang$^{2}$ \hspace{1cm} Yunhui Guo$^{1}$ \hspace{1cm} Yapeng Tian$^{1}$ \\
$^{1}$ The University of Texas at Dallas \hspace{1cm}
$^{2}$ Tsinghua University\\
% Richardson, Texas, U.S.\\
$^{1}${\tt\small  \{Jia.Li, Wenjie.Zhao,  Yunhui.Guo, Yapeng Tian\}@utdallas.edu} $^{2}$ {\tt\small huangzr21@mails.tsinghua.edu.cn}
}
% \and
% Wenjie Zhao\\
% The University of Texas at Dallas\\
% Richardson, Texas, U.S.\\
% {\tt\small Wenjie.Zhao@utdallas.edu}
% \and
% Ziru Huang\\
% Tsinghua University\\
% Beijing, China\\
% {\tt\small huangzr21@mails.tsinghua.edu.cn}
% \and
% Yunhui Guo\\
% The University of Texas at Dallas\\
% Richardson, Texas, U.S.\\
% {\tt\small Yunhui.Guo@utdallas.edu}
% \and
% Yapeng Tian\\
% The University of Texas at Dallas\\
% Richardson, Texas, U.S.\\
% {\tt\small Yapeng.Tian@utdallas.edu}
% }



\begin{document}
\maketitle
% \begin{abstract}


The choice of representation for geographic location significantly impacts the accuracy of models for a broad range of geospatial tasks, including fine-grained species classification, population density estimation, and biome classification. Recent works like SatCLIP and GeoCLIP learn such representations by contrastively aligning geolocation with co-located images. While these methods work exceptionally well, in this paper, we posit that the current training strategies fail to fully capture the important visual features. We provide an information theoretic perspective on why the resulting embeddings from these methods discard crucial visual information that is important for many downstream tasks. To solve this problem, we propose a novel retrieval-augmented strategy called RANGE. We build our method on the intuition that the visual features of a location can be estimated by combining the visual features from multiple similar-looking locations. We evaluate our method across a wide variety of tasks. Our results show that RANGE outperforms the existing state-of-the-art models with significant margins in most tasks. We show gains of up to 13.1\% on classification tasks and 0.145 $R^2$ on regression tasks. All our code and models will be made available at: \href{https://github.com/mvrl/RANGE}{https://github.com/mvrl/RANGE}.

\end{abstract}

  
% \section{Introduction}
Backdoor attacks pose a concealed yet profound security risk to machine learning (ML) models, for which the adversaries can inject a stealth backdoor into the model during training, enabling them to illicitly control the model's output upon encountering predefined inputs. These attacks can even occur without the knowledge of developers or end-users, thereby undermining the trust in ML systems. As ML becomes more deeply embedded in critical sectors like finance, healthcare, and autonomous driving \citep{he2016deep, liu2020computing, tournier2019mrtrix3, adjabi2020past}, the potential damage from backdoor attacks grows, underscoring the emergency for developing robust defense mechanisms against backdoor attacks.

To address the threat of backdoor attacks, researchers have developed a variety of strategies \cite{liu2018fine,wu2021adversarial,wang2019neural,zeng2022adversarial,zhu2023neural,Zhu_2023_ICCV, wei2024shared,wei2024d3}, aimed at purifying backdoors within victim models. These methods are designed to integrate with current deployment workflows seamlessly and have demonstrated significant success in mitigating the effects of backdoor triggers \cite{wubackdoorbench, wu2023defenses, wu2024backdoorbench,dunnett2024countering}.  However, most state-of-the-art (SOTA) backdoor purification methods operate under the assumption that a small clean dataset, often referred to as \textbf{auxiliary dataset}, is available for purification. Such an assumption poses practical challenges, especially in scenarios where data is scarce. To tackle this challenge, efforts have been made to reduce the size of the required auxiliary dataset~\cite{chai2022oneshot,li2023reconstructive, Zhu_2023_ICCV} and even explore dataset-free purification techniques~\cite{zheng2022data,hong2023revisiting,lin2024fusing}. Although these approaches offer some improvements, recent evaluations \cite{dunnett2024countering, wu2024backdoorbench} continue to highlight the importance of sufficient auxiliary data for achieving robust defenses against backdoor attacks.

While significant progress has been made in reducing the size of auxiliary datasets, an equally critical yet underexplored question remains: \emph{how does the nature of the auxiliary dataset affect purification effectiveness?} In  real-world  applications, auxiliary datasets can vary widely, encompassing in-distribution data, synthetic data, or external data from different sources. Understanding how each type of auxiliary dataset influences the purification effectiveness is vital for selecting or constructing the most suitable auxiliary dataset and the corresponding technique. For instance, when multiple datasets are available, understanding how different datasets contribute to purification can guide defenders in selecting or crafting the most appropriate dataset. Conversely, when only limited auxiliary data is accessible, knowing which purification technique works best under those constraints is critical. Therefore, there is an urgent need for a thorough investigation into the impact of auxiliary datasets on purification effectiveness to guide defenders in  enhancing the security of ML systems. 

In this paper, we systematically investigate the critical role of auxiliary datasets in backdoor purification, aiming to bridge the gap between idealized and practical purification scenarios.  Specifically, we first construct a diverse set of auxiliary datasets to emulate real-world conditions, as summarized in Table~\ref{overall}. These datasets include in-distribution data, synthetic data, and external data from other sources. Through an evaluation of SOTA backdoor purification methods across these datasets, we uncover several critical insights: \textbf{1)} In-distribution datasets, particularly those carefully filtered from the original training data of the victim model, effectively preserve the model’s utility for its intended tasks but may fall short in eliminating backdoors. \textbf{2)} Incorporating OOD datasets can help the model forget backdoors but also bring the risk of forgetting critical learned knowledge, significantly degrading its overall performance. Building on these findings, we propose Guided Input Calibration (GIC), a novel technique that enhances backdoor purification by adaptively transforming auxiliary data to better align with the victim model’s learned representations. By leveraging the victim model itself to guide this transformation, GIC optimizes the purification process, striking a balance between preserving model utility and mitigating backdoor threats. Extensive experiments demonstrate that GIC significantly improves the effectiveness of backdoor purification across diverse auxiliary datasets, providing a practical and robust defense solution.

Our main contributions are threefold:
\textbf{1) Impact analysis of auxiliary datasets:} We take the \textbf{first step}  in systematically investigating how different types of auxiliary datasets influence backdoor purification effectiveness. Our findings provide novel insights and serve as a foundation for future research on optimizing dataset selection and construction for enhanced backdoor defense.
%
\textbf{2) Compilation and evaluation of diverse auxiliary datasets:}  We have compiled and rigorously evaluated a diverse set of auxiliary datasets using SOTA purification methods, making our datasets and code publicly available to facilitate and support future research on practical backdoor defense strategies.
%
\textbf{3) Introduction of GIC:} We introduce GIC, the \textbf{first} dedicated solution designed to align auxiliary datasets with the model’s learned representations, significantly enhancing backdoor mitigation across various dataset types. Our approach sets a new benchmark for practical and effective backdoor defense.



% \section{Related Work}
\label{sec:related-works}
\subsection{Novel View Synthesis}
Novel view synthesis is a foundational task in the computer vision and graphics, which aims to generate unseen views of a scene from a given set of images.
% Many methods have been designed to solve this problem by posing it as 3D geometry based rendering, where point clouds~\cite{point_differentiable,point_nfs}, mesh~\cite{worldsheet,FVS,SVS}, planes~\cite{automatci_photo_pop_up,tour_into_the_picture} and multi-plane images~\cite{MINE,single_view_mpi,stereo_magnification}, \etal
Numerous methods have been developed to address this problem by approaching it as 3D geometry-based rendering, such as using meshes~\cite{worldsheet,FVS,SVS}, MPI~\cite{MINE,single_view_mpi,stereo_magnification}, point clouds~\cite{point_differentiable,point_nfs}, etc.
% planes~\cite{automatci_photo_pop_up,tour_into_the_picture}, 


\begin{figure*}[!t]
    \centering
    \includegraphics[width=1.0\linewidth]{figures/overview-v7.png}
    %\caption{\textbf{Overview.} Given a set of images, our method obtains both camera intrinsics and extrinsics, as well as a 3DGS model. First, we obtain the initial camera parameters, global track points from image correspondences and monodepth with reprojection loss. Then we incorporate the global track information and select Gaussian kernels associated with track points. We jointly optimize the parameters $K$, $T_{cw}$, 3DGS through multi-view geometric consistency $L_{t2d}$, $L_{t3d}$, $L_{scale}$ and photometric consistency $L_1$, $L_{D-SSIM}$.}
    \caption{\textbf{Overview.} Given a set of images, our method obtains both camera intrinsics and extrinsics, as well as a 3DGS model. During the initialization, we extract the global tracks, and initialize camera parameters and Gaussians from image correspondences and monodepth with reprojection loss. We determine Gaussian kernels with recovered 3D track points, and then jointly optimize the parameters $K$, $T_{cw}$, 3DGS through the proposed global track constraints (i.e., $L_{t2d}$, $L_{t3d}$, and $L_{scale}$) and original photometric losses (i.e., $L_1$ and $L_{D-SSIM}$).}
    \label{fig:overview}
\end{figure*}

Recently, Neural Radiance Fields (NeRF)~\cite{2020NeRF} provide a novel solution to this problem by representing scenes as implicit radiance fields using neural networks, achieving photo-realistic rendering quality. Although having some works in improving efficiency~\cite{instant_nerf2022, lin2022enerf}, the time-consuming training and rendering still limit its practicality.
Alternatively, 3D Gaussian Splatting (3DGS)~\cite{3DGS2023} models the scene as explicit Gaussian kernels, with differentiable splatting for rendering. Its improved real-time rendering performance, lower storage and efficiency, quickly attract more attentions.
% Different from NeRF-based methods which need MLPs to model the scene and huge computational cost for rendering, 3DGS has stronger real-time performance, higher storage and computational efficiency, benefits from its explicit representation and gradient backpropagation.

\subsection{Optimizing Camera Poses in NeRFs and 3DGS}
Although NeRF and 3DGS can provide impressive scene representation, these methods all need accurate camera parameters (both intrinsic and extrinsic) as additional inputs, which are mostly obtained by COLMAP~\cite{colmap2016}.
% This strong reliance on COLMAP significantly limits their use in real-world applications, so optimizing the camera parameters during the scene training becomes crucial.
When the prior is inaccurate or unknown, accurately estimating camera parameters and scene representations becomes crucial.

% In early works, only photometric constraints are used for scene training and camera pose estimation. 
% iNeRF~\cite{iNerf2021} optimizes the camera poses based on a pre-trained NeRF model.
% NeRFmm~\cite{wang2021nerfmm} introduce a joint optimization process, which estimates the camera poses and trains NeRF model jointly.
% BARF~\cite{barf2021} and GARF~\cite{2022GARF} provide new positional encoding strategy to handle with the gradient inconsistency issue of positional embedding and yield promising results.
% However, they achieve satisfactory optimization results when only the pose initialization is quite closed to the ground-truth, as the photometric constrains can only improve the quality of camera estimation within a small range.
% Later, more prior information of geometry and correspondence, \ie monocular depth and feature matching, are introduced into joint optimisation to enhance the capability of camera poses estimation.
% SC-NeRF~\cite{SCNeRF2021} minimizes a projected ray distance loss based on correspondence of adjacent frames.
% NoPe-NeRF~\cite{bian2022nopenerf} chooses monocular depth maps as geometric priors, and defines undistorted depth loss and relative pose constraints for joint optimization.
In earlier studies, scene training and camera pose estimation relied solely on photometric constraints. iNeRF~\cite{iNerf2021} refines the camera poses using a pre-trained NeRF model. NeRFmm~\cite{wang2021nerfmm} introduces a joint optimization approach that simultaneously estimates camera poses and trains the NeRF model. BARF~\cite{barf2021} and GARF~\cite{2022GARF} propose a new positional encoding strategy to address the gradient inconsistency issues in positional embedding, achieving promising results. However, these methods only yield satisfactory optimization when the initial pose is very close to the ground truth, as photometric constraints alone can only enhance camera estimation quality within a limited range. Subsequently, 
% additional prior information on geometry and correspondence, such as monocular depth and feature matching, has been incorporated into joint optimization to improve the accuracy of camera pose estimation. 
SC-NeRF~\cite{SCNeRF2021} minimizes a projected ray distance loss based on correspondence between adjacent frames. NoPe-NeRF~\cite{bian2022nopenerf} utilizes monocular depth maps as geometric priors and defines undistorted depth loss and relative pose constraints.

% With regard to 3D Gaussian Splatting, CF-3DGS~\cite{CF-3DGS-2024} also leverages mono-depth information to constrain the optimization of local 3DGS for relative pose estimation and later learn a global 3DGS progressively in a sequential manner.
% InstantSplat~\cite{fan2024instantsplat} focus on sparse view scenes, first use DUSt3R~\cite{dust3r2024cvpr} to generate a set of densely covered and pixel-aligned points for 3D Gaussian initialization, then introduce a parallel grid partitioning strategy in joint optimization to speed up.
% % Jiang et al.~\cite{Jiang_2024sig} proposed to build the scene continuously and progressively, to next unregistered frame, they use registration and adjustment to adjust the previous registered camera poses and align unregistered monocular depths, later refine the joint model by matching detected correspondences in screen-space coordinates.
% \gjh{Jiang et al.~\cite{Jiang_2024sig} also implemented an incremental approach for reconstructing camera poses and scenes. Initially, they perform feature matching between the current image and the image rendered by a differentiable surface renderer. They then construct matching point errors, depth errors, and photometric errors to achieve the registration and adjustment of the current image. Finally, based on the depth map, the pixels of the current image are projected as new 3D Gaussians. However, this method still exhibits limitations when dealing with complex scenes and unordered images.}
% % CG-3DGS~\cite{sun2024correspondenceguidedsfmfree3dgaussian} follows CF-3DGS, first construct a coarse point cloud from mono-depth maps to train a 3DGS model, then progressively estimate camera poses based on this pre-trained model by constraining the correspondences between rendering view and ground-truth.
% \gjh{Similarly, CG-3DGS~\cite{sun2024correspondenceguidedsfmfree3dgaussian} first utilizes monocular depth estimation and the camera parameters from the first frame to initialize a set of 3D Gaussians. It then progressively estimates camera poses based on this pre-trained model by constraining the correspondences between the rendered views and the ground truth.}
% % Free-SurGS~\cite{freesurgs2024} matches the projection flow derived from 3D Gaussians with optical flow to estimate the poses, to compensate for the limitations of photometric loss.
% \gjh{Free-SurGS~\cite{freesurgs2024} introduces the first SfM-free 3DGS approach for surgical scene reconstruction. Due to the challenges posed by weak textures and photometric inconsistencies in surgical scenes, Free-SurGS achieves pose estimation by minimizing the flow loss between the projection flow and the optical flow. Subsequently, it keeps the camera pose fixed and optimizes the scene representation by minimizing the photometric loss, depth loss and flow loss.}
% \gjh{However, most current works assume camera intrinsics are known and primarily focus on optimizing camera poses. Additionally, these methods typically rely on sequentially ordered image inputs and incrementally optimize camera parameters and scene representation. This inevitably leads to drift errors, preventing the achievement of globally consistent results. Our work aims to address these issues.}

Regarding 3D Gaussian Splatting, CF-3DGS~\cite{CF-3DGS-2024} utilizes mono-depth information to refine the optimization of local 3DGS for relative pose estimation and subsequently learns a global 3DGS in a sequential manner. InstantSplat~\cite{fan2024instantsplat} targets sparse view scenes, initially employing DUSt3R~\cite{dust3r2024cvpr} to create a densely covered, pixel-aligned point set for initializing 3D Gaussian models, and then implements a parallel grid partitioning strategy to accelerate joint optimization. Jiang \etal~\cite{Jiang_2024sig} develops an incremental method for reconstructing camera poses and scenes, but it struggles with complex scenes and unordered images. 
% Similarly, CG-3DGS~\cite{sun2024correspondenceguidedsfmfree3dgaussian} progressively estimates camera poses using a pre-trained model by aligning the correspondences between rendered views and actual scenes. Free-SurGS~\cite{freesurgs2024} pioneers an SfM-free 3DGS method for reconstructing surgical scenes, overcoming challenges such as weak textures and photometric inconsistencies by minimizing the discrepancy between projection flow and optical flow.
%\pb{SF-3DGS-HT~\cite{ji2024sfmfree3dgaussiansplatting} introduced VFI into training as additional photometric constraints. They separated the whole scene into several local 3DGS models and then merged them hierarchically, which leads to a significant improvement on simple and dense view scenes.}
HT-3DGS~\cite{ji2024sfmfree3dgaussiansplatting} interpolates frames for training and splits the scene into local clips, using a hierarchical strategy to build 3DGS model. It works well for simple scenes, but fails with dramatic motions due to unstable interpolation and low efficiency.
% {While effective for simple scenes, it struggles with dramatic motion due to unstable view interpolation and suffers from low computational efficiency.}

However, most existing methods generally depend on sequentially ordered image inputs and incrementally optimize camera parameters and 3DGS, which often leads to drift errors and hinders achieving globally consistent results. Our work seeks to overcome these limitations.
 
% \input{sec/3_Problem_Definition}
% \section{MDGD: Modality-Decoupled Gradient Regularization and Descent}
Motivated by the visual forgetting problem caused by the degradation of multimodal encoding in Eq.~\eqref{eq:erank_degradation}, we introduce a modality-decoupling gradient regularization (\textbf{MDGD}) to approximate orthogonal gradients between visual understanding drift and downstream task optimization. Specifically, leveraging modality-decoupled gradients $\Bar{g}_\theta$ and $\Bar{g}_\phi$ derived from the current MLLM and a pre-trained MLLM respectively, we propose a gradient regularization term $\Tilde{g}_\theta$ for more efficient multimodal instruction tuning, which promotes the alignment of downstream tasks while mitigating visual forgetting \cite{zhu2024model}. Since MDGD requires the estimation of parameter gradients, we could not directly apply parameter-efficient fine-tuning methods (\emph{e.g.}, LoRA \cite{hu2021lora}). Thus, we alternatively formulate the regularization as a gradient mask $M_{\Tilde{g}_\theta}$, which allows efficient fine-tuning only on a subset of masked model parameters.

\subsection{Modality Decoupling}
Based on the information bottleneck objective in Eq.~\eqref{eq:ib_vision}, the objective encourages the model to maximize $I(y; Z)$ while compressing $I(X^v; Z)$ \cite{tishby2000information, alemi2016deep}. 
In practice, this compression may discard useful visual details, leading to visual forgetting. To mitigate such compression and preserve the pre-trained visual knowledge, we follow the KL divergence loss
$D_{\text{KL}}\Bigl(\mu_\phi(X^v) \,\Big\|\, \pi_\theta (X^v)\Bigr)$
to constrain the current model’s visual representation $\pi_\theta(X^v)$ to remain close to the pre-trained distribution $\mu_\phi(X^v)$, 
thereby preserving the mutual information $I(X^v; Z)$ that would otherwise be reduced by the compression \cite{hinton2015distilling, lopez2018information}. 
However, since MLLMs cannot directly track the distributions of image tokens, we instead introduce an auxiliary loss function
\begin{equation}\label{eq:visual_loss}
    \mathcal{L}_v(\phi,\theta) = \|\mu(X^v|\phi) - \pi(X^v|\theta)\|_1,
\end{equation}
which approximates the KL divergence loss \cite{zhu2022wdibs,zhu2017unpaired} by penalizing discrepancies between the pre-trained visual representation and that obtained during instruction tuning. 

In the MLLM instruction tuning, the visual output tokens (e.g., $\{z^{vl}_k\}_{k=1}^M$) are encoded as latent representations. 
Such visual encoding cannot be directly supervised by any learning objective but is learned through textual gradient propagation of the negative log-likelihood loss in downstream tasks. 
To approximate the visual optimization direction, we derive the gradients of $\mathcal{L}_v(\phi,\theta)$ for both the pre-trained MLLM $\pi_\phi$ and the current MLLM $\pi_\theta$:
\begin{align*}
    h_{\phi} &= \nabla_{\phi}\mathcal{L}_v(\phi) = \boldsymbol{\lambda}(\phi,\theta) \cdot \nabla_\phi \mu(X^v|\phi), \\
    h_{\theta} &= \nabla_{\theta}\mathcal{L}_v(\theta) = -\boldsymbol{\lambda}(\phi,\theta) \cdot \nabla_\theta \pi(X^v|\theta),
\end{align*}
where $\boldsymbol{\lambda}(\phi,\theta) = \text{sign}\left( \mu(X^v|\phi) - \pi(X^v|\theta) \right)$.
Intuitively, when the MLLM's visual understanding drift causes visual forgetting, we further derive the orthogonal task gradients $\Bar{g}_\phi$ and $\Bar{g}_\theta$:
\begin{align}\label{eq:orth}
    \Bar{g}_\phi &= \nabla_{\phi}\mathcal{L}_{vl}(\phi) - \frac{\nabla_{\phi}\mathcal{L}_{vl}(\phi)^\top h_{\phi}}{\|h_{\phi}\|^2} \cdot h_{\phi}, \\
    \Bar{g}_\theta &= \nabla_{\theta}\mathcal{L}_{vl}(\theta) - \frac{\nabla_{\theta}\mathcal{L}_{vl}(\theta)^\top h_{\theta}}{\|h_{\theta}\|^2} \cdot h_{\theta},
\end{align}
which enables \textbf{modality decoupling} of the downstream task loss gradient in Eq.\eqref{eq:task_loss} orthogonal to the visual understanding drift
for the pretrained MLLM $\Bar{g}_{\phi} \perp h_{\phi}$ and current MLLM $\Bar{g}_{\theta} \perp h_{\theta}$.


\begin{figure}[t]
    \centering
    \includegraphics[width=0.85\linewidth]{contents//figure/opt.pdf}
    \caption{Illustration of the proposed method. To mitigate suboptimal optimization and prevent visual forgetting, we first project \textcolor{darkblue}{\( \nabla_{\theta} \mathcal{L}_{vl} \)} onto the direction orthogonal to \( \nabla_{\theta} \mathcal{L}_{v} \), obtaining \textcolor{darkgreen}{\( \bar{g}_{\theta} \)}. Next, we project \textcolor{darkgreen}{\( \bar{g}_{\theta} \)} onto the direction of \( \bar{g}_\phi \), yielding \textcolor{darkred}{\( \tilde{g}_\theta \)}. This process guides the gradient towards the optimal region without visual forgetting.
    }
    \label{fig:opt}
    \vspace{-1em}
\end{figure}

\begin{algorithm*}[t]
\caption{Verified Diversification for ambiguous queries}
\begin{algorithmic}[1]
\Require Question $q$,
LLM $\texttt{LLM}(\cdot)$
\Ensure Pairs of clarification question and answer $\hat{\mathcal{Q}}=\{(\hat{q}, \hat{y})\}$

\State $q^\prime$ $\leftarrow$ Relax $q$ for high-recall universe
\State %
$U$ $\leftarrow$ Top-$k$ retrieved passages from the retriever, using $q^\prime$ as query

\State $\mathcal{Q} \leftarrow \{\}$
\For{$i=1$ to $k$}
    \State $(\hat{q}_i, \hat{y}_i)$ $\leftarrow$ $\texttt{LLM}(q, p_i; I_{\textrm{E}})$ \Comment{Extracting interpretation from passage $p_i$ with execution feedback}
    \If {$\hat{q}_i$ is not \texttt{None}}
        \State $\mathcal{Q} \leftarrow \mathcal{Q} \cup \left\{ (\hat{q}_i, \hat{y}_i) \right\}$
    \EndIf
\EndFor

\State $\mathcal{C}$ $\leftarrow$ Cluster into partitions $\mathcal{C}_i$'s of $\mathcal{Q}$ based on embeddings $f(\hat{q}_i; \hat{y}_i)$

\State $\hat{\mathcal{Q}} \leftarrow \{\}$
\For{$j=1$ to $k$}
    \State $(\check{q}_j, \check{y}_j)$ $\leftarrow$ $\argmax_{({q}^*,y^*)\in \mathcal{C}_j} \sum_{(\hat{q},\hat{y})\in \mathcal{C}_j} \mathrm{sim}(f(\hat{q};\hat{y}), f(q^*;y^*))$ \Comment{Medoid of the $j$-th cluster $\mathcal{C}_j$}
    \State $\hat{\mathcal{Q}} \leftarrow \hat{\mathcal{Q}} \cup \left\{ (\check{q}_j, \check{y}_j) \right\}$
\EndFor

\State \textbf{return} $\hat{\mathcal{Q}}$
\end{algorithmic}
\label{alg:ours}
\end{algorithm*}






\subsection{Regularized Gradient Descent}
The auxiliary loss in Eq.~\eqref{eq:visual_loss} preserves the visual representation at a distribution level via the feature alignment auxiliary loss in Eq.~\eqref{eq:visual_loss}. 
However, the information bottleneck framework indicates that the gradient component compressing $I(X^v; Z)$ (\emph{i.e.}, $\nabla_\theta I(X^v; Z)$), 
can harm visual preservation by reducing the effective rank of the features \cite{achille2018information,lee2021compressive}.

To address this compression-induced drift, we incorporate an orthogonal gradient as a regularize. 
Motivated by multi-task orthogonal gradient optimization \cite{yu2020gradient, zhu2022gradient, dong2022gdod}, 
we leverage the gradient $\Bar{g}_\phi$ from the pre-trained model $\mu_\phi$, which reflects the accumulated visual drift and approximates a global orthogonal learning effect in the downstream task. 
We then project the current model’s gradient onto this direction:
\begin{equation}\label{eq:gd}
    \Tilde{g}_\theta = \frac{\Bar{g}_\theta^\top \Bar{g}_\phi}{\|\Bar{g}_\phi\|^2}\cdot \Bar{g}_\phi.
\end{equation}

In addition, to prevent discrepancies between the regularization and task gradients, we include the feature alignment auxiliary loss (Eq.~\eqref{eq:visual_loss}) in the overall objective. The final parameter update is:
\begin{equation}\label{eq:opt-gd}
    \pi_\theta \leftarrow \pi_\theta - \nabla_\theta\mathcal{L}_{vl}(\theta) - \nabla_\theta\mathcal{L}_v(\theta) - \Tilde{g}_\theta.
\end{equation}

\subsection{Enabling Parameter-efficient Fine-tuning of MDGD via Gradient Masking}
Parameter-efficient fine-tuning (PEFT) methods, such as adapters \cite{houlsby2019parameter} and LoRA \cite{hu2021lora}, aim to reduce the computational cost and memory usage when fine-tuning models on downstream tasks under practical constraints \cite{han2024parameter}. 
However, due to the requirement of directly estimating gradient directions on the pre-trained model parameters, MDGD cannot be directly applied to these PEFT methods, which introduce additional model parameters whose gradients are separate from the original model weights. 

To address this challenge, we propose a variant, MDGD-GM, by formulating the gradient regularization term in Eq.~\eqref{eq:gd} as gradient masking that selects model weights with efficient gradient directions. Specifically, we define the gradient mask as
\begin{equation}\label{eq:masking}
    M_{\Tilde{g}_\theta} = \mathbf{1}\left\{\frac{\Bar{g}_\theta^\top \Bar{g}_\phi}{\|\Bar{g}_\phi\| \|\Bar{g}_\theta\|} \geq T_\alpha \right\},
\end{equation}
where $T_\alpha$ is determined by a percentile $\alpha$ of trainable parameters with the highest similarity scores between $\Bar{g}_\theta$ and $\Bar{g}_\phi$. Consequently, the optimization in Eq.~\eqref{eq:opt-gd} is reformulated as
\begin{equation}\label{eq:opt-mask}
    \pi_\theta \leftarrow \pi_\theta - M_{\Tilde{g}_\theta} \cdot \left(\nabla_\theta\mathcal{L}_{vl}(\theta) + \nabla_\theta\mathcal{L}_v(\theta)\right).
\end{equation}
We summarize and illustrate the optimization process of MDGD and MDGD-GM in Algorithm~\ref{alg}.

% \section{Experiments}
\label{section5}

In this section, we conduct extensive experiments to show that \ourmethod~can significantly speed up the sampling of existing MR Diffusion. To rigorously validate the effectiveness of our method, we follow the settings and checkpoints from \cite{luo2024daclip} and only modify the sampling part. Our experiment is divided into three parts. Section \ref{mainresult} compares the sampling results for different NFE cases. Section \ref{effects} studies the effects of different parameter settings on our algorithm, including network parameterizations and solver types. In Section \ref{analysis}, we visualize the sampling trajectories to show the speedup achieved by \ourmethod~and analyze why noise prediction gets obviously worse when NFE is less than 20.


\subsection{Main results}\label{mainresult}

Following \cite{luo2024daclip}, we conduct experiments with ten different types of image degradation: blurry, hazy, JPEG-compression, low-light, noisy, raindrop, rainy, shadowed, snowy, and inpainting (see Appendix \ref{appd1} for details). We adopt LPIPS \citep{zhang2018lpips} and FID \citep{heusel2017fid} as main metrics for perceptual evaluation, and also report PSNR and SSIM \citep{wang2004ssim} for reference. We compare \ourmethod~with other sampling methods, including posterior sampling \citep{luo2024posterior} and Euler-Maruyama discretization \citep{kloeden1992sde}. We take two tasks as examples and the metrics are shown in Figure \ref{fig:main}. Unless explicitly mentioned, we always use \ourmethod~based on SDE solver, with data prediction and uniform $\lambda$. The complete experimental results can be found in Appendix \ref{appd3}. The results demonstrate that \ourmethod~converges in a few (5 or 10) steps and produces samples with stable quality. Our algorithm significantly reduces the time cost without compromising sampling performance, which is of great practical value for MR Diffusion.


\begin{figure}[!ht]
    \centering
    \begin{minipage}[b]{0.45\textwidth}
        \centering
        \includegraphics[width=1\textwidth, trim=0 20 0 0]{figs/main_result/7_lowlight_fid.pdf}
        \subcaption{FID on \textit{low-light} dataset}
        \label{fig:main(a)}
    \end{minipage}
    \begin{minipage}[b]{0.45\textwidth}
        \centering
        \includegraphics[width=1\textwidth, trim=0 20 0 0]{figs/main_result/7_lowlight_lpips.pdf}
        \subcaption{LPIPS on \textit{low-light} dataset}
        \label{fig:main(b)}
    \end{minipage}
    \begin{minipage}[b]{0.45\textwidth}
        \centering
        \includegraphics[width=1\textwidth, trim=0 20 0 0]{figs/main_result/10_motion_fid.pdf}
        \subcaption{FID on \textit{motion-blurry} dataset}
        \label{fig:main(c)}
    \end{minipage}
    \begin{minipage}[b]{0.45\textwidth}
        \centering
        \includegraphics[width=1\textwidth, trim=0 20 0 0]{figs/main_result/10_motion_lpips.pdf}
        \subcaption{LPIPS on \textit{motion-blurry} dataset}
        \label{fig:main(d)}
    \end{minipage}
    \caption{\textbf{Perceptual evaluations on \textit{low-light} and \textit{motion-blurry} datasets.}}
    \label{fig:main}
\end{figure}

\subsection{Effects of parameter choice}\label{effects}

In Table \ref{tab:ablat_param}, we compare the results of two network parameterizations. The data prediction shows stable performance across different NFEs. The noise prediction performs similarly to data prediction with large NFEs, but its performance deteriorates significantly with smaller NFEs. The detailed analysis can be found in Section \ref{section5.3}. In Table \ref{tab:ablat_solver}, we compare \ourmethod-ODE-d-2 and \ourmethod-SDE-d-2 on the \textit{inpainting} task, which are derived from PF-ODE and reverse-time SDE respectively. SDE-based solver works better with a large NFE, whereas ODE-based solver is more effective with a small NFE. In general, neither solver type is inherently better.


% In Table \ref{tab:hazy}, we study the impact of two step size schedules on the results. On the whole, uniform $\lambda$ performs slightly better than uniform $t$. Our algorithm follows the method of \cite{lu2022dpmsolverplus} to estimate the integral part of the solution, while the analytical part does not affect the error.  Consequently, our algorithm has the same global truncation error, that is $\mathcal{O}\left(h_{max}^{k}\right)$. Note that the initial and final values of $\lambda$ depend on noise schedule and are fixed. Therefore, uniform $\lambda$ scheduling leads to the smallest $h_{max}$ and works better.

\begin{table}[ht]
    \centering
    \begin{minipage}{0.5\textwidth}
    \small
    \renewcommand{\arraystretch}{1}
    \centering
    \caption{Ablation study of network parameterizations on the Rain100H dataset.}
    % \vspace{8pt}
    \resizebox{1\textwidth}{!}{
        \begin{tabular}{cccccc}
			\toprule[1.5pt]
            % \multicolumn{6}{c}{Rainy} \\
            % \cmidrule(lr){1-6}
             NFE & Parameterization      & LPIPS\textdownarrow & FID\textdownarrow &  PSNR\textuparrow & SSIM\textuparrow  \\
            \midrule[1pt]
            \multirow{2}{*}{50}
             & Noise Prediction & \textbf{0.0606}     & \textbf{27.28}   & \textbf{28.89}     & \textbf{0.8615}    \\
             & Data Prediction & 0.0620     & 27.65   & 28.85     & 0.8602    \\
            \cmidrule(lr){1-6}
            \multirow{2}{*}{20}
              & Noise Prediction & 0.1429     & 47.31   & 27.68     & 0.7954    \\
              & Data Prediction & \textbf{0.0635}     & \textbf{27.79}   & \textbf{28.60}     & \textbf{0.8559}    \\
            \cmidrule(lr){1-6}
            \multirow{2}{*}{10}
              & Noise Prediction & 1.376     & 402.3   & 6.623     & 0.0114    \\
              & Data Prediction & \textbf{0.0678}     & \textbf{29.54}   & \textbf{28.09}     & \textbf{0.8483}    \\
            \cmidrule(lr){1-6}
            \multirow{2}{*}{5}
              & Noise Prediction & 1.416     & 447.0   & 5.755     & 0.0051    \\
              & Data Prediction & \textbf{0.0637}     & \textbf{26.92}   & \textbf{28.82}     & \textbf{0.8685}    \\       
            \bottomrule[1.5pt]
        \end{tabular}}
        \label{tab:ablat_param}
    \end{minipage}
    \hspace{0.01\textwidth}
    \begin{minipage}{0.46\textwidth}
    \small
    \renewcommand{\arraystretch}{1}
    \centering
    \caption{Ablation study of solver types on the CelebA-HQ dataset.}
    % \vspace{8pt}
        \resizebox{1\textwidth}{!}{
        \begin{tabular}{cccccc}
			\toprule[1.5pt]
            % \multicolumn{6}{c}{Raindrop} \\     
            % \cmidrule(lr){1-6}
             NFE & Solver Type     & LPIPS\textdownarrow & FID\textdownarrow &  PSNR\textuparrow & SSIM\textuparrow  \\
            \midrule[1pt]
            \multirow{2}{*}{50}
             & ODE & 0.0499     & 22.91   & 28.49     & 0.8921    \\
             & SDE & \textbf{0.0402}     & \textbf{19.09}   & \textbf{29.15}     & \textbf{0.9046}    \\
            \cmidrule(lr){1-6}
            \multirow{2}{*}{20}
              & ODE & 0.0475    & 21.35   & 28.51     & 0.8940    \\
              & SDE & \textbf{0.0408}     & \textbf{19.13}   & \textbf{28.98}    & \textbf{0.9032}    \\
            \cmidrule(lr){1-6}
            \multirow{2}{*}{10}
              & ODE & \textbf{0.0417}    & 19.44   & \textbf{28.94}     & \textbf{0.9048}    \\
              & SDE & 0.0437     & \textbf{19.29}   & 28.48     & 0.8996    \\
            \cmidrule(lr){1-6}
            \multirow{2}{*}{5}
              & ODE & \textbf{0.0526}     & 27.44   & \textbf{31.02}     & \textbf{0.9335}    \\
              & SDE & 0.0529    & \textbf{24.02}   & 28.35     & 0.8930    \\
            \bottomrule[1.5pt]
        \end{tabular}}
        \label{tab:ablat_solver}
    \end{minipage}
\end{table}


% \renewcommand{\arraystretch}{1}
%     \centering
%     \caption{Ablation study of step size schedule on the RESIDE-6k dataset.}
%     % \vspace{8pt}
%         \resizebox{1\textwidth}{!}{
%         \begin{tabular}{cccccc}
% 			\toprule[1.5pt]
%             % \multicolumn{6}{c}{Raindrop} \\     
%             % \cmidrule(lr){1-6}
%              NFE & Schedule      & LPIPS\textdownarrow & FID\textdownarrow &  PSNR\textuparrow & SSIM\textuparrow  \\
%             \midrule[1pt]
%             \multirow{2}{*}{50}
%              & uniform $t$ & 0.0271     & 5.539   & 30.00     & 0.9351    \\
%              & uniform $\lambda$ & \textbf{0.0233}     & \textbf{4.993}   & \textbf{30.19}     & \textbf{0.9427}    \\
%             \cmidrule(lr){1-6}
%             \multirow{2}{*}{20}
%               & uniform $t$ & 0.0313     & 6.000   & 29.73     & 0.9270    \\
%               & uniform $\lambda$ & \textbf{0.0240}     & \textbf{5.077}   & \textbf{30.06}    & \textbf{0.9409}    \\
%             \cmidrule(lr){1-6}
%             \multirow{2}{*}{10}
%               & uniform $t$ & 0.0309     & 6.094   & 29.42     & 0.9274    \\
%               & uniform $\lambda$ & \textbf{0.0246}     & \textbf{5.228}   & \textbf{29.65}     & \textbf{0.9372}    \\
%             \cmidrule(lr){1-6}
%             \multirow{2}{*}{5}
%               & uniform $t$ & 0.0256     & 5.477   & \textbf{29.91}     & 0.9342    \\
%               & uniform $\lambda$ & \textbf{0.0228}     & \textbf{5.174}   & 29.65     & \textbf{0.9416}    \\
%             \bottomrule[1.5pt]
%         \end{tabular}}
%         \label{tab:ablat_schedule}



\subsection{Analysis}\label{analysis}
\label{section5.3}

\begin{figure}[ht!]
    \centering
    \begin{minipage}[t]{0.6\linewidth}
        \centering
        \includegraphics[width=\linewidth, trim=0 20 10 0]{figs/trajectory_a.pdf} %trim左下右上
        \subcaption{Sampling results.}
        \label{fig:traj(a)}
    \end{minipage}
    \begin{minipage}[t]{0.35\linewidth}
        \centering
        \includegraphics[width=\linewidth, trim=0 0 0 0]{figs/trajectory_b.pdf} %trim左下右上
        \subcaption{Trajectory.}
        \label{fig:traj(b)}
    \end{minipage}
    \caption{\textbf{Sampling trajectories.} In (a), we compare our method (with order 1 and order 2) and previous sampling methods (i.e., posterior sampling and Euler discretization) on a motion blurry image. The numbers in parentheses indicate the NFE. In (b), we illustrate trajectories of each sampling method. Previous methods need to take many unnecessary paths to converge. With few NFEs, they fail to reach the ground truth (i.e., the location of $\boldsymbol{x}_0$). Our methods follow a more direct trajectory.}
    \label{fig:traj}
\end{figure}

\textbf{Sampling trajectory.}~ Inspired by the design idea of NCSN \citep{song2019ncsn}, we provide a new perspective of diffusion sampling process. \cite{song2019ncsn} consider each data point (e.g., an image) as a point in high-dimensional space. During the diffusion process, noise is added to each point $\boldsymbol{x}_0$, causing it to spread throughout the space, while the score function (a neural network) \textit{remembers} the direction towards $\boldsymbol{x}_0$. In the sampling process, we start from a random point by sampling a Gaussian distribution and follow the guidance of the reverse-time SDE (or PF-ODE) and the score function to locate $\boldsymbol{x}_0$. By connecting each intermediate state $\boldsymbol{x}_t$, we obtain a sampling trajectory. However, this trajectory exists in a high-dimensional space, making it difficult to visualize. Therefore, we use Principal Component Analysis (PCA) to reduce $\boldsymbol{x}_t$ to two dimensions, obtaining the projection of the sampling trajectory in 2D space. As shown in Figure \ref{fig:traj}, we present an example. Previous sampling methods \citep{luo2024posterior} often require a long path to find $\boldsymbol{x}_0$, and reducing NFE can lead to cumulative errors, making it impossible to locate $\boldsymbol{x}_0$. In contrast, our algorithm produces more direct trajectories, allowing us to find $\boldsymbol{x}_0$ with fewer NFEs.

\begin{figure*}[ht]
    \centering
    \begin{minipage}[t]{0.45\linewidth}
        \centering
        \includegraphics[width=\linewidth, trim=0 0 0 0]{figs/convergence_a.pdf} %trim左下右上
        \subcaption{Sampling results.}
        \label{fig:convergence(a)}
    \end{minipage}
    \begin{minipage}[t]{0.43\linewidth}
        \centering
        \includegraphics[width=\linewidth, trim=0 20 0 0]{figs/convergence_b.pdf} %trim左下右上
        \subcaption{Ratio of convergence.}
        \label{fig:convergence(b)}
    \end{minipage}
    \caption{\textbf{Convergence of noise prediction and data prediction.} In (a), we choose a low-light image for example. The numbers in parentheses indicate the NFE. In (b), we illustrate the ratio of components of neural network output that satisfy the Taylor expansion convergence requirement.}
    \label{fig:converge}
\end{figure*}

\textbf{Numerical stability of parameterizations.}~ From Table 1, we observe poor sampling results for noise prediction in the case of few NFEs. The reason may be that the neural network parameterized by noise prediction is numerically unstable. Recall that we used Taylor expansion in Eq.(\ref{14}), and the condition for the equality to hold is $|\lambda-\lambda_s|<\boldsymbol{R}(s)$. And the radius of convergence $\boldsymbol{R}(t)$ can be calculated by
\begin{equation}
\frac{1}{\boldsymbol{R}(t)}=\lim_{n\rightarrow\infty}\left|\frac{\boldsymbol{c}_{n+1}(t)}{\boldsymbol{c}_n(t)}\right|,
\end{equation}
where $\boldsymbol{c}_n(t)$ is the coefficient of the $n$-th term in Taylor expansion. We are unable to compute this limit and can only compute the $n=0$ case as an approximation. The output of the neural network can be viewed as a vector, with each component corresponding to a radius of convergence. At each time step, we count the ratio of components that satisfy $\boldsymbol{R}_i(s)>|\lambda-\lambda_s|$ as a criterion for judging the convergence, where $i$ denotes the $i$-th component. As shown in Figure \ref{fig:converge}, the neural network parameterized by data prediction meets the convergence criteria at almost every step. However, the neural network parameterized by noise prediction always has components that cannot converge, which will lead to large errors and failed sampling. Therefore, data prediction has better numerical stability and is a more recommended choice.


% \paragraph{Summary}
Our findings provide significant insights into the influence of correctness, explanations, and refinement on evaluation accuracy and user trust in AI-based planners. 
In particular, the findings are three-fold: 
(1) The \textbf{correctness} of the generated plans is the most significant factor that impacts the evaluation accuracy and user trust in the planners. As the PDDL solver is more capable of generating correct plans, it achieves the highest evaluation accuracy and trust. 
(2) The \textbf{explanation} component of the LLM planner improves evaluation accuracy, as LLM+Expl achieves higher accuracy than LLM alone. Despite this improvement, LLM+Expl minimally impacts user trust. However, alternative explanation methods may influence user trust differently from the manually generated explanations used in our approach.
% On the other hand, explanations may help refine the trust of the planner to a more appropriate level by indicating planner shortcomings.
(3) The \textbf{refinement} procedure in the LLM planner does not lead to a significant improvement in evaluation accuracy; however, it exhibits a positive influence on user trust that may indicate an overtrust in some situations.
% This finding is aligned with prior works showing that iterative refinements based on user feedback would increase user trust~\cite{kunkel2019let, sebo2019don}.
Finally, the propensity-to-trust analysis identifies correctness as the primary determinant of user trust, whereas explanations provided limited improvement in scenarios where the planner's accuracy is diminished.

% In conclusion, our results indicate that the planner's correctness is the dominant factor for both evaluation accuracy and user trust. Therefore, selecting high-quality training data and optimizing the training procedure of AI-based planners to improve planning correctness is the top priority. Once the AI planner achieves a similar correctness level to traditional graph-search planners, strengthening its capability to explain and refine plans will further improve user trust compared to traditional planners.

\paragraph{Future Research} Future steps in this research include expanding user studies with larger sample sizes to improve generalizability and including additional planning problems per session for a more comprehensive evaluation. Next, we will explore alternative methods for generating plan explanations beyond manual creation to identify approaches that more effectively enhance user trust. 
Additionally, we will examine user trust by employing multiple LLM-based planners with varying levels of planning accuracy to better understand the interplay between planning correctness and user trust. 
Furthermore, we aim to enable real-time user-planner interaction, allowing users to provide feedback and refine plans collaboratively, thereby fostering a more dynamic and user-centric planning process.


\begin{abstract}
\vspace{-5pt}

% % >>>>>>>>>>>>>>>>>>>>>>>>>>>>>>>>>>>>>>>>>>>>>>
% % 1. What is AVS
% The task of Audio-Visual Segmentation (AVS) involves generating pixel-level segmentation maps of objects that produce sound in synchronization with visual frames. 
% % enabling more sophisticated, audio-driven scene understanding in dynamic environments. 
% % >>>>>>>>>>>>>>>>>>>>>>>>>>>>>>>>>>>>>>>>>>>>>>
% % 2. What kind of flaws we find in current SOTA methods
% % Although recent advances in state-of-the-art (SOTA) AVS models have shown promise, they still face critical limitations: (i) an over-reliance on visual cues, reducing their capacity for robust audio-visual integration, and (ii) evaluation methods that largely focus on "positive" scenarios where sounds align with visible objects, neglecting cases where sounds are unrelated to the visual scene.
% While existing AVS methods have shown promising results, they exhibit a fundamental bias: generating segmentation masks primarily based on visual salience regardless of audio context, leading to unreliable predictions when sounds are absent or irrelevant.
% % , or originate from off-screen sources.
% % >>>>>>>>>>>>>>>>>>>>>>>>>>>>>>>>>>>>>>>>>>>>>>
% % 3. In this paper, what did we proposed
% % To address these challenges, we propose an expanded evaluation framework that includes a diverse set of "negative" audio samples, such as silence, ambient noise, and off-screen sounds. This framework introduces new metrics for assessing how well AVS models perform when visual and audio cues do not align, a crucial aspect for real-world applicability. Additionally, we design novel loss functions and a gating mechanism to improve model training, allowing the model to more accurately distinguish between relevant and irrelevant audio-visual pairs.
% % This benchmark enables systematic evaluation of AVS models' ability to suppress false predictions when no valid audio-visual correspondence exists.
% To address this, we introduce AVSBench-Robust, a comprehensive benchmark incorporating diverse negative audio scenarios including silence, ambient noise, and off-screen sounds. Additionally, we propose a simple yet effective approach that combines balanced training with negative samples with BCE-guided similarity learning.
% % and Binary Cross-Entropy (BCE) guided similarity learning to explicitly teach models when segmentation should or should not occur.
% % >>>>>>>>>>>>>>>>>>>>>>>>>>>>>>>>>>>>>>>>>>>>>>
% % 4. What we found.
% % Our experiments reveal that current SOTA models frequently misinterpret diverse audio cues due to their dependence on visual information. In contrast, our approach demonstrates enhanced resilience and accuracy, particularly in complex audio-visual scenarios, with a significant reduction in false positives in unrelated audio contexts. By broadening AVS evaluation standards and offering refined training mechanisms, this work advances AVS towards more robust and realistic applications in audio-driven segmentation tasks.
% Our extensive experiments reveal that state-of-the-art AVS methods consistently fail under negative audio conditions, demonstrating the prevalence of visual bias. In contrast, our approach achieves remarkable improvements in both standard metrics and robustness measures, maintaining near-perfect false positive rates while preserving high-quality segmentation performance. These results highlight the importance of addressing the bias problem in AVS and demonstrate the effectiveness of our solution in enabling more reliable audio-visual segmentation for real-world applications.

% \jia{Another outline: (1) What is AVS, (2) We propose a new benchmark, (3) Observed that all the SOTA methods faild, (4) Our approach}

%Audio-Visual Segmentation (AVS) is the task of generating pixel-level segmentation maps for objects that produce sound in sync with visual frames. 
Unlike traditional visual segmentation, audio-visual segmentation (AVS) requires the model not only to identify and segment objects but also to determine whether they are sound sources.
Recent AVS approaches, leveraging transformer architectures and powerful foundation models like SAM, have achieved impressive performance on standard benchmarks. Yet, an important question remains: Do these models genuinely integrate audio-visual cues to segment sounding objects?
In this paper, we systematically investigate this issue in the context of robust AVS. Our study reveals a fundamental bias in current methods: they tend to generate segmentation masks based predominantly on visual salience, irrespective of the audio context. This bias results in unreliable predictions when sounds are absent or irrelevant.
To address this challenge, we introduce AVSBench-Robust, a comprehensive benchmark incorporating diverse negative audio scenarios including silence, ambient noise, and off-screen sounds. We also propose a simple yet effective approach combining balanced training with negative samples and classifier-guided similarity learning.
Our extensive experiments show that state-of-the-art AVS methods consistently fail under negative audio conditions, demonstrating the prevalence of visual bias. In contrast, our approach achieves remarkable improvements in both standard metrics and robustness measures, maintaining near-perfect false positive rates while preserving high-quality segmentation performance. %These results highlight the importance of addressing the bias problem in AVS and demonstrate the effectiveness of our solution in enabling more robust AVS for real-world applications. 
%\yapeng{Feel free to edit this new version -- if it is too long, we could remove the last sentence.}

\vspace{-3mm}

\end{abstract}

\section{Introduction}
\label{sec:intro}

\begin{figure}[h]
    \centering
    \includegraphics[width=0.48\textwidth]{fig1.pdf}
    \vspace{-7mm}
    \caption{\textbf{Performance in Different Audio Scenarios.} The top row shows an ambulance image under different audio conditions: \textit{Ambulance sound} (positive), \textit{Silence}, \textit{Noise}, and \textit{Offscreen sounds} (negative). Each subsequent row displays the segmentation output various SOTA AVS models~\cite{zhou2022audio, gao2024avsegformer, chen2024cavp} and our model under each audio condition. In negative scenarios, existing models segment the ambulance due to ``visual prior" bias, mistakenly associating it with unrelated audio. In contrast, our model accurately segments only in the presence of relevant audio, demonstrating improved alignment between audio cues and visual segmentation.
    }
    \vspace{-7mm}

    % \caption{Limitations of Current AVS Models in Negative Audio Scenarios: The top row shows an input image featuring an ambulance. Each column represents different audio conditions: \textit{Ambulance sound}, \textit{Silence}, \textit{Noise}, and \textit{Offscreen sounds}. The last three conditions are examples of "negative" audio samples, where the sound not align with visible objects. Notably, existing models often highlight the ambulance regardless of whether the sound aligns with it (e.g., in silence or noise cases), showing a "visual prior" bias. In contrast, our approach shows improved segmentation by not relying solely on visual presence and offering more accurate alignment with relevant audio cues.}
    \label{fig:fig1}
\end{figure}


% >>>>>>>>>>>>>>>>>>>>>>>>>>>>>>>>>>>>>>>>>>>
% 1. task(What can it do? Why do we need this?)
Audio-Visual Segmentation (AVS) aims to identify and segment sounding objects within visual scenes~\cite{zhou2022audio,gao2024avsegformer}. This essential multimodal task mirrors a fundamental aspect of human perception: the integration of auditory and visual stimuli to focus attention on relevant sources~\cite{zhou2022audio, small2005odor, chen2020vggsound}. For instance, when hearing a baby cry, people naturally locate the sound’s visual source.  Simulating this ability in machines could open up valuable cross-modal applications, such as improved multimedia analysis~\cite{arandjelovic2017look, arandjelovic2018objects, hu2022mix, mo2024multi}, enhanced human-computer interaction~\cite{yang2024analyzing, wang2024audiobench, johansen2022characterising, lv2022deep, fu2021design}, and autonomous systems capable of interpreting sound-emitting objects in complex environments~\cite{schmidt2020acoustic, dutt2020self, topcu2020assured}. 

% >>>>>>>>>>>>>>>>>>>>>>>>>>>>>>>>>>>>>>>>>>>
% 2. state-of-the-art(Briefly cite recent work (direct competition), )
Recent years have witnessed remarkable progress in AVS.  State-of-the-art (SOTA) methods leverage multimodal information, utilizing encoder-decoder structures with audio-visual interaction~\cite{zhou2022audio}, multimodal transformer architectures~\cite{gao2024avsegformer,li2024qdformer,liu2023AuTR}, audio query-guided designs~\cite{liu2023AuTR, sun2024biasinAVS}, and strong vision foundation models~\cite{mo2023av,liu2024annofree, wang2024GAVS, sun2024biasinAVS} like SAM~\cite{kirillov2023sam}and Mask2Former~\cite{cheng2022mask2former}. 
These innovations have driven impressive performance on standard benchmark datasets: AVSBench-S4 and AVSBench-MS3 ~\cite{zhou2022audio}.

%These innovations have not only driven impressive performance on benchmark datasets~\cite{xxx} but also extended AVS to weakly-supervised~\cite{xxx} and open-vocabulary~\cite{xxx} learning settings, significantly broadening its applicability. \ytian{add refs; feel free to edit}


% Significant progress has architures been made in AVS through the development of state-of-the-art (SOTA) methods, which combine audio and visual features at different processing stages to pinpoint sound-emitting objects by integrating information across both modalities, or leveraged large visual foundation models \cite{zhou2022audio, gao2024avsegformer, li2023catr, chen2024cavp, liu2024annofree, ma2024stepping, wang2024GAVS, guo2024open, sun2024biasinAVS, li2024qdformer, yang2024combo, huang2023aqformer, ling2023hear2seg, liu2023AuTR, wang2024ref}, offering a promising degree of generalization and precision across both single-source and multi-source audio-visual tasks.

% which fall into two main categories\cite{sun2024biasinAVS}: fusion-based\cite{zhou2022audio} and prompt-based approaches\cite{gao2024avsegformer, chen2024cavp}. Fusion-based methods combine audio and visual features at different processing stages to pinpoint sound-emitting objects by integrating information across both modalities. For instance, AVSBench\cite{zhou2022audio} achieves high accuracy in aligning sounds and visual features by utilizing temporal and multi-scale data, improving model performance in dynamic scenes. Recently, prompt-based models like AVSegFormer\cite{gao2024avsegformer} and GAVS\cite{gao2024avsegformer} have leveraged large visual foundation models, using tailored audio prompts to directly query visual features. These models excel in zero-shot and few-shot learning scenarios, offering a promising degree of generalization and precision across both single-source and multi-source audio-visual tasks.


% >>>>>>>>>>>>>>>>>>>>>>>>>>>>>>>>>>>>>>>>>>>
% 3. flaw in state-of-the-art(What can’t we do yet? Why should the reader care? No, really? Why can’t X, Y or Z be used to solve this? Fig 1)
% 4. your idea/solution (Keep it brief, )
However, a critical question arises: \textit{are these models truly performing audio-visual segmentation, or simply conducting visual segmentation with minimal audio integration?}  AVS, by definition, introduces a crucial constraint: only objects acting as sound sources should be segmented. For example, an AVS model should not segment a visually salient yet silent dog.  Current AVS models, primarily trained and evaluated on ``positive'' cases where visual objects correspond to audio cues, often neglect scenarios with unrelated sounds, such as silence or off-screen sources.

%AVS extends beyond the traditional object segmentation by introducing a crucial constraint: only objects acting as sound sources should be segmented. For example, an AVS model should not segment a visually salient yet silent dog in a video. However, the current AVS models are primarily trained and evaluated on ``positive'' cases where visual objects correspond to audio cues, neglecting scenarios with unrelated sounds, such as silence or off-screen sources. This raises a critical question: \textit{are these models truly performing audio-visual segmentation, or simply conducting visual segmentation with minimal audio integration?}


% \yapeng{better to include some statistics about the new benchmark} 
To systematically investigate whether AVS models truly integrate audio-visual information, we introduce AVS-Robust, a comprehensive benchmark comprising 4,932 single-source and 424 multi-source videos across 20 diverse object classes from AVSBench~\cite{zhou2022audio}. We incorporate four different audio conditions for each video: original audio, silence, ambient noise, and off-screen sounds. Each condition represents 25\% of the evaluation scenarios. Our study reveals a concerning bias in existing SOTA methods: they tend to generate segmentation masks based primarily on visual salience, irrespective of the audio context. For instance, these models may segment an ambulance even in the presence of silence or unrelated ambient sounds, indicating an over-reliance on visual cues rather than genuine audio-visual integration (Fig.~\ref{fig:fig1}).

% However, do these methods truly integrate audio-visual cues to segment sounding objects?
% To address this gap, we systematically investigate this issue in the context of robust AVS. We introduce AVS-Robust, a comprehensive benchmark that includes diverse negative audio scenarios to challenge AVS models under conditions where audio cues do not align with visual information.

%To address this critical gap in AVS evaluation, we introduce the AVS-Robust Benchmark, a comprehensive benchmark that includes diverse negative audio scenarios. We conducted a systematic study of AVS in multiple sound environments and reveals a concerning bias in current approaches - the existing SOTA methods tend to generate segmentation masks based predominantly on visual salience, regardless of the audio context. For example, existing models will segment an ambulance even when presented with silence or unrelated ambient sounds, indicating an over-reliance on visual cues rather than true audio-visual integration (Fig \ref{fig:fig1}). 

% Building upon these insights, we introduce negative audio-visual pairs into training and propose a balanced training approach that incorporates both matched and mismatched audio-visual pairs, utilizing classifier guidance for feature alignment and segmentation loss to ensure accurate segmentation. 
%\yapeng{This version is okay but not great. Can we highlight that simply adding negative pairs does not work? We should also highlight the significance of the proposed approach. We mainly motivated why this research but did not clearly motivate the proposed approach}
Building upon these insights, we explore solutions to address this visual bias. While incorporating negative audio-visual pairs during training seems intuitive, this approach alone presents a challenge: without explicit guidance for audio-visual integration, models struggle to determine whether to segment objects based solely on visual information. To overcome this, we propose a debiasing approach with two key components:
\textit{(1) Balanced Training with Negative Samples:} Incorporating both positive and negative audio-visual pairs during training to expose models to a wider range of audio-visual relationships. \textit{(2) Classifier-Guided Similarity Learning:} Utilizing a classifier to guide the model in learning effective audio-visual feature representations and promoting similarity between corresponding audio and visual features.

% >>>>>>>>>>>>>>>>>>>>>>>>>>>>>>>>>>>>>>>>>>>
% 5. proof it works(Say a bit about evaluation and baselines, our method lowers reconstruction error by xx% compared to previous methods)
Extensive experiments using our new benchmark yield several crucial findings. Recent SOTA methods, including SAMA-AVS \cite{liu2024annofree}, Stepping-Stones \cite{ma2024stepping}, and CAVP \cite{chen2024cavp}, consistently fail under negative audio conditions, exhibiting high False Positive Rates (FPR). When evaluated with our comprehensive metrics—such as G-mIoU, G-F, and G-FPR, as discussed in Sec.~\ref{sec:problem_benchmark}—these models show significant performance degradation compared to their reported results on standard benchmarks. In contrast, our approach achieves superior performance across all robustness metrics while maintaining competitive segmentation quality on positive audio inputs in both single- and multi-source scenarios.
% However, current AVS models face significant limitations in real-world audio-visual scenarios. Most AVS models are evaluated only on positive cases—where visible objects align with audio cues—while excluding scenarios where sounds are unrelated to visual objects, such as silence, ambient noise, or off-screen sounds. This limitation results in an incomplete assessment of model performance, potentially overestimating their audio-visual alignment capability. Although some methods consider off-screen sounds, they lack dedicated metrics to evaluate such cases, providing limited insight into model behavior in negative audio scenarios. As shown in Figure \ref{fig:fig1}, current models segment an ambulance even in negative audio conditions, such as silence or off-screen sounds, due to an over-reliance on visual information alone
% To address these gaps, we propose a two-part solution. First, we introduce a balanced training strategy that includes both matched (positive) and mismatched (negative) audio-visual pairs. This strategy employs Binary Cross-Entropy (BCE) loss to align audio and visual features, while Segmentation Loss ensures accurate segmentation only when audio-visual correspondence exists. Second, we introduce an expanded evaluation framework incorporating negative audio cases, such as silence, ambient noise, and off-screen sounds. This comprehensive benchmark enables a more accurate assessment of AVS models, providing a closer reflection of real-world audio-visual interactions.

Our main contributions are summarized as follows:
\begin{itemize}    
    \item  We conduct a systematic study on audio robustness in AVS and introduce AVSBench-Robust along with our new robustness evaluation protocols. This benchmark rigorously evaluates AVS models under both standard conditions and challenging negative scenarios, assessing their ability to effectively integrate audio-visual information.
    \item We propose a training strategy for robust AVS by incorporating diverse negative audio scenarios and employing classifier-guided similarity learning, which enhances model robustness and preserves segmentation quality.
    \item Extensive experiments demonstrate that our approach substantially outperforms current SOTA methods in terms of our robustness metrics while achieving competitive performance on standard AVS benchmarks. %Our results establish new benchmarks for robust audio-visual segmentation and provide insights for future research in this direction.
\end{itemize}


% \begin{itemize}
%     \item We propose AVSBench-Robust, a novel evaluation benchmark encompassing a broad range of audio scenarios, designed to assess both the segmentation performance on paired audio-video inputs and the model’s robustness when audio and video are misaligned.
%     \item
%     We introduce a training strategy that  balances positive and negative audio scenarios, incorporating BCE-guided similarity learning to help models effectively distinguish valid from invalid audio-visual correspondences while maintaining hight segmentation quality.
%     \item We present extensive empirical evidence showing our approach's robustness and precision in handling complex audio-visual scenarios. 
% \end{itemize}
% \item We develop a novel training framework that addresses the visual bias problem in AVS through two key components: (1) balanced sampling of positive and negative audio-visual pairs, and (2) classifier-guided similarity learning that explicitly enforces audio-visual correspondence. This approach significantly improves model robustness while maintaining high-quality segmentation performance.
% \item Through extensive experiments on both our benchmark and existing datasets, we demonstrate that our approach substantially outperforms current state-of-the-art methods in terms of robustness metrics (FPR, G-mIoU, G-F) while achieving competitive performance on standard AVS tasks. Our results establish new benchmarks for robust audio-visual segmentation and provide insights for future research in this direction.




% Together, these contributions advance AVS towards more robust, real-world applications, supporting next-generation models in achieving nuanced, human-like multi-modal perception.

% Despite recent progress, existing Audio-Visual Segmentation (AVS) methods face significant limitations in handling real-world audio-visual scenarios. Most AVS models are evaluated only on positive audio cases—where sound-producing objects are visible in the scene. This limited scope risks drawing inaccurate conclusions about a model’s ability to interpret audio, as it fails to address cases involving silent objects or offscreen sounds. Although some methods\cite{sun2024biasinAVS, li2024qdformer, chen2024cavp} consider offscreen sounds, they lack metrics for separate evaluation, leaving the model’s behavior on negative sounds largely unexamined.

\section{Related work}
\label{sec:related_new}
\vspace{-4pt}
% >>>>>>>>>>>>>>>>>>>>>>>>>>>>>>>>>>>>>>>>>>>
% Organization
% >>>>>>>>>>>>>>>>>>>>>>>>>>>>>>>>>>>>>>>>>>>
% In this section, we review the current state of research in three key areas relevant to our study: Sound Source Localization (SSL)(\ref{subsec:ssl}), Audio-Visual Segmentation (AVS) \wj{maybe we don't need this in related work because every familiry with this graph.}(\ref{subsec:avs_related}), and Imbalanced Multimodal Learning(\ref{subsec:imbalanced_multimodal_learning}). 
% Each area contributes differently to our understanding of multimodal interactions and challenges in audio-visual processing, and we highlight both the advancements and limitations that frame the context for our proposed approach.


% >>>>>>>>>>>>>>>>>>>>>>>>>>>>>>>>>>>>>>>>>>>
% Sound Localization
% >>>>>>>>>>>>>>>>>>>>>>>>>>>>>>>>>>>>>>>>>>>
\textbf{Sound Source Localization.}
% 1. --- Define Task & Early Work
This task is closely related to AVS, focusing on localizing sound sources within visual scenes \cite{arandjelovic2018objects, senocak2018learning,mo2022closer, chen2021localizing,mahmud2024t}. This task advances cross-modal understanding through various technical approaches, from basic feature fusion strategies to sophisticated attention mechanisms \cite{senocak2018learning, mo2022closer, hu2020discriminative, qian2020multiple}.
% 2. --- Recent advances
Recent sound source localization approaches have significantly improved sound source discrimination through multiple innovations: contrastive learning with hard-mining strategies enhances complex region distinction~\cite{chen2021localizing, hu2020discriminative, mo2022closer}, while class-aware approaches and dual-phase feature alignment enable robust multi-source localization without explicit pairwise annotations \cite{hu2020discriminative, qian2020multiple, chen2021localizing}.
However, the predicted sounding object heatmaps lack the fine-grained precision offered by AVS's pixel-level segmentation capabilities.


% >>>>>>>>>>>>>>>>>>>>>>>>>>>>>>>>>>>>>>>>>>>
% AVS
% >>>>>>>>>>>>>>>>>>>>>>>>>>>>>>>>>>>>>>>>>>>
\vspace{2mm}
\noindent
\textbf{Audio-Visual Segmentation.}
% \label{subsec:avs_related}
% --- 1. Define Task & Early Work
AVS task focuses on identifying and segmenting sound-producing objects through pixel-level mask prediction. 
% --- 2. Recent advances
The field has progressed significantly since its inception~\cite{zhou2022audio,gao2024avsegformer, li2023catr, chen2024cavp, liu2024annofree, ma2024stepping, wang2024GAVS, guo2024open, sun2024biasinAVS}. Most approaches follow an encoder-decoder design, with early works focusing on effective fusion strategies for audio-visual information~\cite{zhou2022audio}. Subsequent developments explored more sophisticated architectures, incorporating multimodal transformers and audio query guided mechanisms~\cite{gao2024avsegformer, liu2023AuTR, sun2024biasinAVS, wang2024ref, li2023catr, ma2024stepping, yang2024combo} to enhance cross-modal understanding. Recently, methods leveraging vision foundation models~\cite{liu2024annofree, wang2024GAVS, sun2024biasinAVS} and LLMs~\cite{wang2024can, he2024mlmseg}  have demonstrated improved segmentation capabilities. 

% The integration of large language models has further extend AVS to open-vocabulary scenarios~\cite{, }, enabling more flexible applications.
% Recent approaches have advanced AVS capabilities through various technical innovations, such as fusion-based architectures \cite{zhou2022audio, huang2023aqformer, li2023catr, ling2023hear2seg, liu2023AuTR} and prompt-based methods \cite{gao2024avsegformer, liu2024annofree, ma2024stepping, wang2024GAVS, wang2024ref} \yapeng{make it consistent with the descriptions in intro}. Some advances have enabled more sophisticated applications, from multi-source segmentation to open-vocabulary scenarios~\cite{wang2024GAVS, guo2024open}.

% --- 3. Bias Problems & Existing Solutions
Despite these developments, we observe that AVS models commonly suffer from ``visual prior'' bias, where models generate predictions primarily based on visual salience regardless of audio context~\cite{sun2024biasinAVS, chen2024cavp, li2024qdformer}. While recent efforts address this through contrastive learning~\cite{chen2024cavp} and semantic enhancement~\cite{sun2024biasinAVS, li2024qdformer}, they still lead to disregard audio features (see Fig.~\ref{fig:fig1}) or face limited evaluation scope. 
% --- 4. Our Approach

% In contrast to these complex architectural solutions, our approach offers a simple yet effective strategy that can be integrated into existing AVS models toward more robust AVS. 
In contrast to complex architectural solutions, we adopt a straightforward yet effective approach to enhancing AVS model robustness. We systematically analyze audio robustness in AVS and introduce AVSBench-Robust, a benchmark designed to evaluate models under both original audio conditions and challenging negative audio scenarios. 
A concurrent study \cite{juanola2024critical} identified visual bias in SSL models, introducing the same negative audio scenarios along with evaluation metrics, but primarily focuses on assessing model performance.
In contrast, our work not only evaluates robustness but also presents a targeted training strategy to strengthen AVS models against misleading audio-visual cues.
% However, AVS systems commonly exhibit problematic biases, particularly "visual prior" where models over-rely on visual features while disregarding audio context \cite{sun2024biasinAVS, chen2024cavp, li2024qdformer}. While recent works attempt to address these limitations through contrastive learning \cite{chen2024cavp} and semantic feature enhancement \cite{sun2024biasinAVS, li2024qdformer}, Due to CAVP\cite{chen2024cavp} heavy reliance on negative samples in the contrastive learning process inadvertently, the model to disregard audio features entirely, resulting in predictions that remain largely unaffected by audio variations. (see figure \ref{fig:fig1}), while other approaches \cite{sun2024biasinAVS, li2024qdformer} face limited evaluation scope due to restricted implementation availability.

% >>>>>>>>>>>>>>>>>>>>>>>>>>>>>>>>>>>>>>>>>>>
% Imbalanced Multimodal Learning
% >>>>>>>>>>>>>>>>>>>>>>>>>>>>>>>>>>>>>>>>>>>
\vspace{2mm}
\noindent
\textbf{Imbalanced Multimodal Learning.}
%\label{subsec:imbalanced_multimodal_learning}
Recent studies in audio-visual learning highlight significant challenges in balancing different modalities during training, where dominant modalities often overshadow others in the learning process \cite{wang2020mmlhard,tian2020unified,wei2024mmpareto, peng2022balanced}. Various solutions such as modality-specific optimization, gradient modulation, and Pareto optimization~\cite{wang2020mmlhard, wei2024mmpareto, peng2022balanced} have been proposed to address this imbalance, aiming to preserve the contribution of each modality. However, these approaches primarily target tasks where separate losses are applied for joint modalities and each individual modality ~\cite{wang2020mmlhard, wei2024mmpareto}. In contrast, the AVS task typically involves applying a single loss after fusing audio and visual features, which requires us to design a new strategy for effective modality balancing in AVS.

\section{Problem and Benchmark}
\label{sec:problem_benchmark}

% AVS aims to identify and segment objects in video frames that correspond to the accompanying audio signals. Given a dataset consisting of video frames, their corresponding audio signals, and ground truth segmentation masks, the goal is to learn a mapping function that can accurately localize and segment sound-producing objects in videos.
% In this section, we first formulate the AVS task and highlight its two-fold nature, then analyze core challenges in current approaches, and finally introduce our benchmark for robust AVS evaluation.

In this section, we first present the formulation of the AVS task and its associated challenges in Sec.~\ref{subsec:task}. In Sec.~\ref{subsec:benchmark}, we introduce our new benchmark for robust AVS. Finally, we present our evaluation protocols in Sec.~\ref{subsec:eval}.

% ==============================================
% 1. AVS Task Formulation
\subsection{Task and Challenges}
\label{subsec:task}
Given \( T \) non-overlapping video and audio clips \( \{V^t, A^t\}_{t=1}^{T} \), the goal of the AVS task is to predict a segmentation mask \( \mathcal{M_{\text{pred}}}^t \in \mathbb{R}^{H \times W} \) that labels sounding pixels in each video frame of the clips, where \( H \) and \( W \) denote the frame dimensions, and the mask is binary. Following previous studies~\cite{zhou2022audio,gao2024avsegformer, li2023catr}, we extract a single video frame at the end of each second and set \( T = 5 \) in practice, so each clip contains only one extracted frame. 

Unlike purely visual segmentation, AVS inherently addresses two subtasks simultaneously: segmenting visual objects and determining whether they are sound sources. Therefore, predictions should satisfy two key requirements: (1) when objects are producing sound, the model should generate accurate segmentation masks for those objects, and (2) when no audio-visual correspondence exists, the model should produce empty masks to avoid false predictions.


 

%Given a dataset $\mathcal{D} = \{V^i, A^i, \mathcal{M}_{gt}^i\}_{i=1}^n$ consisting of triplets of video frames $V^i \in V$, audio clips $A^i \in A$, and ground truth masks $\mathcal{M}_{gt}^i \in \mathcal{M}_{gt}$, the AVS task aims to learn a mapping $\mathcal{G}(V, A) \rightarrow \mathcal{M}_{gt}$ that generates pixel-wise segmentation masks for sound-producing objects. This mapping should satisfy two key requirements: (1) when objects are producing sound, generate accurate segmentation masks corresponding to those objects, and (2) when no valid audio-visual correspondence exists, produce empty masks to avoid false predictions.

% We evaluate our approach on two benchmark settings: (1) AVSBench-S4: a semi-supervised single sound source segmentation task where only the first frame of each video is labeled during training, while requiring predictions for all frames during evaluation, and (2) AVSBench-MS3: a fully-supervised multiple sound source segmentation task where ground truth masks for all frames are available during training.


% ==============================================
% 2. AVS bias Problem
% \subsection{Core Challenges}  
While current AVS approaches have shown promising segmentation results on standard benchmark protocols, they face a fundamental limitation: evaluations have primarily focused on positive cases where audio and visual signals fully align, with salient objects in video frames typically being sound sources in the datasets. This setup allows AVS models to potentially rely solely on visual information to achieve high performance, bypassing true multimodal integration.
This focus on positive cases overlooks the equally important ability to suppress predictions when no valid audio-visual correspondence exists. To comprehensively assess the multimodal learning capabilities of AVS models, a more robust benchmark is needed.

%This incomplete evaluation masks a critical bias in existing models—they consistently predict sound-producing regions regardless of audio context, leading to three common issues: (1) Silent Object Bias, where models segment visually prominent objects even in silence; (2) Background Noise Bias; and (3) Irrelevant Sound Bias, where irrelevant sounds are incorrectly attributed to visible objects (see fig \ref{fig:fig1}). 

% A critical limitation in existing AVS approaches is their inherent bias towards always predicting some sound-producing regions, even when no corresponding audio signal is present. This bias manifests in two problematic scenarios:
% (1) Silent Object Bias: Models tend to generate segmentation masks for visually prominent objects even during moments of silence, failing to recognize that visible objects may not always produce sound.
% (2) Background Noise Bias: When presented with background noise or irrelevant sounds, models often incorrectly attribute these sounds to visible objects in the frame, leading to false positive segmentations.

% This bias problem fundamentally challenges the reliability of current AVS systems in real-world applications, where distinguishing between sound-producing and silent objects is crucial for accurate scene understanding.

\begin{table}[t]
\centering
\footnotesize
\vspace{-5mm}
\renewcommand{\arraystretch}{1.1}  % Increases row height
\setlength{\tabcolsep}{10pt}  % Adjusts horizontal padding

\begin{tabular}{>{\raggedright\arraybackslash}m{0.15\linewidth}|>{\raggedright\arraybackslash}m{0.65\linewidth}}
\hline
\textbf{Category} & \textbf{Representative  Examples} \\
\hline
\faMusic \textbf{ Music} & Guitar, Violin, Piano, Tabla, Marimba, Ukulele, Playing Acoustic Guitar, Playing Glockenspiel, Playing Violin, Playing Ukulele \\
\hline
\faChild \textbf{ Human Voice} & Male Speech, Female Speech, Male Singing, Female Singing, Baby Crying, Baby Laughter \\
\hline
\faPaw \textbf{ Animals} & Dog Barking, Lion Roaring, Cat Meowing, Bird Chirping, Wolf Howling, Horse Neighing, Coyote Howling, Mynah Bird Singing \\
\hline
\faCar \textbf{ Devices, Machines} & Helicopter, Ambulance Siren, Car Horn, Lawn Mower, Chainsaw, Bus Engine, Typing on Computer Keyboard, Cap Gun Shooting, Emergency Car, Driving Buses, Race Car \\
\hline
\end{tabular}
\caption{Semantic Categories and Examples in AVSBench-Robust: To ensure clear evaluation of cross-modal understanding, we organize sounds into distinct semantic categories, which particularly crucial for the off-screen audio condition.}
\label{tab:sound-categories}
\vspace{-5mm} 

\end{table}
% ==============================================
% 3. Our Benchmark
\subsection{AVSBench-Robust}  
\label{subsec:benchmark}
% To advance the development of more robust AVS models, we introduce AVSBench-Robust, a comprehensive framework that fundamentally extends the original AVSBench dataset\cite{zhou2022audio}. Rather than focusing solely on aligned audio-visual pairs, our benchmark introduces three challenging negative audio conditions that models must handle appropriately:

To address the limitations of current evaluation frameworks and facilitate the development of more robust AVS models, we introduce {AVSBench-Robust}. Building upon AVSBench~\cite{zhou2022audio}, this benchmark includes two evaluation scenarios: (1) the single-source subset (S4), containing 4,932 videos (3,452 for training, 740 for validation, and 740 for testing), and (2) the multi-source subset (MS3), comprising 424 videos (296 for training, 64 for validation, and 64 for testing). For each video from AVSBench~\cite{zhou2022audio}, we create three additional negative audio conditions alongside the original positive audio, effectively quadrupling the number of audio-visual pairs for a comprehensive evaluation.

The benchmark spans 20 diverse sound-producing object classes across four major categories: machine (32.2\%), music (32.1\%), animal (23.2\%), and human (12.5\%). To evaluate model robustness, each video is paired with four types of audio conditions:

\noindent
\underline{\textit{Positive Pair:}} Original audio of the video from AVSBench, where the audio accurately reflects the visible objects.

\noindent
\underline{\textit{Silence Scenario:}} Test cases without audio, where objects are visually present but silent in the video.

\noindent
\underline{\textit{Noise Condition:}} Background audio noise, testing the model's ability to differentiate between meaningful and irrelevant audio signals.

\noindent
\underline{\textit{Off-screen Audio:}} Semantically unrelated sounds from different categories, as outlined in Table~\ref{tab:sound-categories}, testing the model’s ability to maintain accurate audio-visual correspondence. For example, pairing animal visuals with device sounds requires models to learn true cross-modal relationships rather than relying solely on visual cues.



By incorporating unpaired audio clips, we create negative audio-visual pairs that enable the study of potential visual bias issues in AVS, specifically: (1) {silent object bias}, where models segment visually salient but silent objects; (2) {background noise bias}; and (3) {irrelevant sound bias}, where unrelated sounds are misattributed to visible objects.






% \begin{figure}[h]
%     \centering
%     \includesvg[width=0.5\textwidth]{image/benchmark_s4.svg}
%     \caption{}
%     \label{fig:fig1}
% \end{figure}


% We provide detailed statistics of AVSBench-Robust in Figure \ref{}. The benchmark consists of two subsets: a robust single-source subset-robust S4 with 4,932 videos (3,452/740/740 for train/val/test) and a robust multi-source subset-robust M4 containing 424 videos (296/64/64 for train/val/test). The robust S4 subset contains highly diverse instances across 20 classes, with detailed per-class distributions shown in Fig \ref{}. These classes are grouped into four major categories: device (32.2\%), music (32.1\%), animal (23.2\%), and human (12.5\%), as shown in the pie chart. The MS3 subset focuses on multi-source scenarios, where videos contain multiple sound-producing objects, with distributions across different category combinations shown in Fig \ref{}.

% For each video in both subsets (3,452/740/740 for robust S4 train/val/test and 296/64/64 for robust MS3), we create a comprehensive evaluation suite by augmenting it with four types of audio conditions, effectively quadrupling the number of audio-visual pairs. These conditions are evenly distributed (25\% each) among: (1) original positive pairs where audio-visual signals align, (2) silence scenarios, (3) background noise conditions, and (4) off-screen sounds(Fig \ref{}). This balanced design enables thorough evaluation of models' ability to both generate accurate segmentation masks when appropriate and suppress false predictions when no valid audio-visual correspondence exists.

% ==============================================
% 4. Evaluation Metrics
\subsection{Evaluation Protocols}
\label{subsec:eval}
A robust AVS model should not only accurately segment sound-producing objects but also reliably suppress predictions when no valid audio-visual correspondence exists. To enable this comprehensive evaluation, we propose new metrics that assess both aspects of AVS performance.


Let $\mathcal{P}$ and $\mathcal{N}$ denote sets of positive and negative samples, respectively. For positive samples, following established protocols~\cite{zhou2022audio, gao2024avsegformer, ma2024stepping}, we employ mean Intersection over Union (mIoU) and F-score to evaluate segmentation accuracy. For negative ones, we introduce complementary metrics to capture different aspects of model robustness.

\noindent
\textit{False Positive Rate (FPR):}
\begin{equation}
\text{FPR} = \frac{\sum_{x\in \mathcal{M_{\text{pred}}} } m(x)}{H \cdot W},
\end{equation}
% \yapeng{define m(x)}
where $m(x)$ denotes the binary indicator (0 or 1) for pixel $x$ in the predicted mask.
FPR measures the proportion of incorrectly activated pixels in negative scenarios, directly assessing the model's tendency to generate false predictions.

To evaluate overall performance across both positive and negative cases, we propose three global metrics.

\noindent
\textit{Global mIoU (G-mIoU):} 
\begin{equation} \label{eq:gmiou}
\text{G-mIoU} = \frac{2 \cdot \text{mIoU}_\mathcal{P} \cdot (1 - \text{mIoU}_\mathcal{N})}{\text{mIoU}_\mathcal{P} + (1 - \text{mIoU}_\mathcal{N})}.
\end{equation}
where $\text{mIoU}_\mathcal{P}$ is the mIoU for positive samples, and $\text{mIoU}_\mathcal{N}$ is for negative samples. G-mIoU balances region-level accuracy, emphasizing the model's ability to maintain precise segmentation boundaries while suppressing false activations. A high score indicates accurate object delineation in positive cases and clean masks in negative cases.

\noindent
\textit{Global F-score (G-F):} 
\begin{equation} \label{eq:gf}
\text{G-F} = \frac{2 \cdot \text{F}_\mathcal{P} \cdot (1 - \text{F}_\mathcal{N})}{\text{F}_\mathcal{P} + (1 - \text{F}_\mathcal{N})}.
\end{equation}
G-F provides a pixel-level assessment that equally weighs precision and recall, which is essential for evaluating both false positives and false negatives. This metric is particularly sensitive to small errors that may be overlooked by IoU-based measures.


\noindent
\textit{Global False Positive Rate (G-FPR):}
\begin{equation} \label{eq:gfpr}
\text{G-FPR} = \frac{1}{|\mathcal{N}|} \sum_{i \in \mathcal{N}} \text{FPR}_{i}.
\end{equation}
This metric specifically focuses on false activation suppression across all negative conditions. While G-mIoU and G-F balance positive and negative performance, G-FPR provides a dedicated measure of a model's robustness against different types of audio distractors.

The combination of these metrics provides a comprehensive evaluation framework: G-mIoU captures region-level accuracy, G-F ensures pixel-level precision, and G-FPR specifically measures robustness to negative conditions. Together, they enable thorough assessment of both segmentation quality and prediction suppression capabilities.


% To ensure comprehensive evaluation, we introduce metrics that assess both alignment accuracy and suppression of irrelevant activations. 

% Aligning with established protocols \cite{zhou2022audio, gao2024avsegformer, ma2024stepping, chen2024cavp, yang2024combo}, we using mean Intersection over Union (mIoU) and F-score as the performance metrics for positive samples.


% \textbf{False Positive Rate (FPR):} $text{FPR} = \frac{\sum_x m(x)}{H \cdot W}$. FPR quantifies suppression effectiveness in negative samples by measuring the proportion of activated pixels in irrelevant audio scenarios.

% To assess model performance across both sample types, we introduce global metrics:

% A higher G-mIoU score reflects a model that successfully segments relevant regions for positive cases while minimizing activations in irrelevant areas for negative cases.


% where $\text{F}_\mathcal{P}$ denotes the F-score for positive samples, and $\text{F}_\mathcal{N}$ is the average F-score for negative samples. A high G-F score indicates that the model performs well in detecting relevant regions when audio and visual inputs are aligned, while effectively suppressing false activations in negative cases.

% A lower G-FPR indicates that the model reliably suppresses activations when there is no relevant audio-visual correspondence, demonstrating robustness against irrelevant sounds or silence.

% % ==============================================
% % 3. Existing Solutions and Their Limitations
% \subsection{Existing Solutions and Their Limitations}  
% Recent works have recognized this bias problem in AVS. CAVP\cite{chen2024cavp} attempts to address it through contrastive learning with mismatched audio-visual pairs. However, their heavy reliance on negative samples in the contrastive learning process inadvertently causes the model to disregard audio features entirely, resulting in predictions that remain largely unaffected by audio variations. While other approaches like BiasAVS\cite{sun2024biasinAVS} and QDFormer\cite{li2024qdformer} propose incorporating semantic information into audio representations, their restricted availability limits comprehensive evaluation and comparison.




\begin{figure*}[ht]
    \centering

        \centering
        \includegraphics[width=0.9\textwidth]{AVS_framework.pdf}
        \vspace{-8mm}
        \caption{\textbf{Framework Overview.} Given video frames and an audio clip as inputs, our approach can robustly identify and segment sounding objects in video frames. Positive audio-visual pairs represent aligned sound sources, while negative pairs, such as silence or offscreen sounds, correspond to empty masks. The model uses separate visual and audio encoders to extract modality-specific features, applies similarity-based alignment optimized with classifier guidance in a contrastive manner, and integrates features through a fusion module. Positive pairs maximize similarity, while negative pairs minimize it, using a small portion (10\%) of the dataset for improved boundary delineation. This dual-stream design facilitates segmentation by distinguishing sound-relevant regions in complex scenes.}
        \label{fig:AVSBench}

    \vspace{-3mm}
\end{figure*}


\section{Method}
\label{sec:method}

In this section, we first present a framework overview in Sec.~\ref{subsec:framework_overview}. Upon the framework, we detail our approach to address the bias problem in AVS through three key components: balanced audio-visual pair construction (Sec. \ref{subsec:Learning_Balanced}), classifier-guided similarity learning (Sec. \ref{subsec:bce_guide} ), and joint segmentation training (Sec. \ref{subsec:total_loss}). To validate our approach, we apply it to two representative AVS models: AVSBench \cite{zhou2022audio} and AVSegFormer \cite{gao2024avsegformer}. The architecture of AVSBench is described in Sec.~\ref{subsec:preliminary}. Due to space constraints, implementation details for \cite{gao2024avsegformer} are provided in the appendix.
% The framework effectively handles both positive and negative audio scenarios while maintaining high-quality segmentation performance.

% Our method was tested using two different baseline models to validate its effectiveness and robustness: AVSBench \cite{zhou2022audio} and AVSegFormer \cite{gao2024avsegformer}. While we focus on the implementation details based on the AVSBench \cite{zhou2022audio} baseline within this section, the specifics of our experiments with the AVSegFormer \cite{gao2024avsegformer} model are provided in the appendix due to space constraints.


\subsection{Preliminary: AVS Architecture}
\label{subsec:preliminary}

\textbf{Encoder:} We employ an encoder structure that separately processes audio clip $A$ and visual frames $V$. Specifically, input audio is converted into spectrograms and processed through a VGGish-based network \cite{hershey2017vggish}, pre-trained on AudioSet \cite{gemmeke2017audioset}, to generate audio feature $\mathcal{F}_A \in \mathbb{R}^d$ where $d = 128$. For visual inputs $V$, we utilize a transformer-based backbone~\cite{wang2022pvt} to extract hierarchical visual features.$\mathcal{F}_{V_i} \in \mathbb{R}^{h_i \times w_i \times C_i}$, where $(h_i,w_i) = (H,W)/2^{i+1}, i = 1,\ldots,n$. The number of levels is set to $n = 4$ in all experiments. %\yapeng{dimension of this feature}.
% $\mathcal{F}_V$

\noindent \textbf{Cross-Modal Fusion:} Following the work in \cite{zhou2022audio}, the fusion process involves an Atrous Spatial Pyramid Pooling (ASPP) module~\cite{chen2017aspp} that manipulates the visual feature maps to enhance object recognition capabilities in varying receptive fields. Subsequently, audio features are integrated to reinforce the identification of sounding objects, crucial for precise segmentation in mixed audio scenarios.

\noindent\textbf{Decoder:} The decoder leverages a Panoptic-FPN \cite{kirillov2019panoptic} architecture, which sequentially processes outputs from the fusion stage and refines them through upsampling, aiming to recover detailed segmentations at the original scale.

\noindent \textbf{Segmentation Loss:} The segmentation objective is the binary cross-entropy loss for basic segmentation accuracy.
\begin{equation} 
\mathcal{L}_{\text{Seg}} = \mathcal{L}_{BCE}(\mathcal{M}_{pred}, \mathcal{M}_{gt}),  
\end{equation}
where $\mathcal{M}_{pred}$ is the predicted segmentation mask, $\mathcal{M}_{gt}$ is the ground-truth (GT) mask.


\subsection{Framework Overview}
\label{subsec:framework_overview}

Our framework, as illustrated in Fig.~\ref{fig:AVSBench}, processes both positive and negative audio-visual pairs to learn robust correspondence for segmentation. Built upon the presented AVS architecture, our model achieves balanced training by incorporating negative audio-visual pairs, enhancing robustness in AVS. Within this framework, audio and visual features are extracted and used to compute cosine similarity scores for both positive pairs $\mathcal{P}$ and negative pairs $\mathcal{N}$, allowing the model to differentiate aligned from unaligned audio-visual pairs. For mask prediction, we employ a segmentation module that combines a fusion module and an FPN decoder, enabling precise segmentation of sound-producing objects. The dual-stream design allows the model to accurately identify sound-relevant regions in complex scenes while suppressing predictions when no valid audio-visual correspondence exists. The following sections detail each component and their integration within the framework.
% Given a dataset $\mathcal{D} = \{V^i, A^i, M_{gt}^i\}_{i=1}^n$ containing video frames $V$, audio clips $A$, and ground truth masks $M_{gt}$, we use pre-trained visual and audio models to extract visual features $\mathcal{F}_V$ and audio feature $\mathcal{F}_A$ respectively. Taking extracted audio and visual features as inputs, cosine similarity scores are computed for both positive pairs $\mathcal{P}$ and negative pairs $\mathcal{N}$, helping the model distinguish between aligned and unaligned audio-visual pairs.
% To predict the sound-producing object mask, we use segmentation module where the framework integrates fusion module and Decoder(FPN). This setup enables precise segmentation of sound-producing objects. The dual-stream design allows the model to effectively identify sound-relevant regions in complex scenes while suppressing predictions when no valid audio-visual correspondence exists. The following sections provide a detailed explanation of each component and their integration within the framework.

\subsection{Learning with Balanced Audio-Visual Pairs}
\label{subsec:Learning_Balanced}

In real-world scenarios, audio-visual correspondence is inherently dynamic~\cite{chen2022comprehensive, chakraborty2023multimodal, yang2024combo}. A visible object may or may not be producing sound at any given moment—for instance, a person may be speaking or silent, and a car may be running or stationary. Additionally, sounds may come from off-screen sources or be ambient noise. This variability requires AVS models to learn true audio-visual association rather than assume that all visible objects are sound sources.

Existing AVS models have been trained with predominantly \textit{positive} audio-visual pairs, where audio and visual signals align, and salient objects are typically the sound sources. This encourages AVS models to rely solely on visual information, bypassing true multimodal integration.

Motivated by this insight, we propose a critical requirement: models must be trained with both positive and negative audio-visual pairs. This balanced approach ensures that the model learns not only when to segment objects that make sounds but also, crucially, when to suppress segmentation predictions for visually salient but silent objects.

% This insight points to a crucial requirement: the model must be exposed to both positive and negative audio-visual pairs during training. Such balanced training ensures that the model learns not only when to segment objects but also, critically, when to suppress segmentation predictions.

Given a video clip with its corresponding audio signal, we construct two types of pairs:


\textit{Positive Pairs ($\mathcal{P}$):} Original audio-visual pairs where the audio corresponds to visual objects in the frame. These pairs represent valid correspondence cases and constitute the majority (approximately 90\%) of training samples.


\textit{Negative Pairs ($\mathcal{N}$):} We deliberately create challenging negative scenarios by:
 1) Replacing the original audio with silence;
 2) Replacing the original audio background noise or ambient sounds;
 3) Using off-screen sounds that are semantically distinct from visible objects.

We maintain a 10\% of negative pairs during training, which we empirically found to optimally balance robustness, segmentation accuracy, and training efficiency.  Expanding the diversity of training samples is anticipated to further enhance the model's robustness.

% This balanced training strategy serves as the foundation for our framework, enabling the model to develop robust audio-visual correspondence understanding. 

\subsection{Classifier-Guided Feature Alignment}
\label{subsec:bce_guide}
However, we observed that simply introducing negative pairs is insufficient to mitigate the visual bias, as show in Table~\ref{tab:only_negative}. Due to the inherent bias in existing models, which often fail to effectively utilize audio information, the model tends to behave more like a purely visual segmentation model. Without explicit guidance, adding negative pairs can lead to confusion during training, as the model alternates between predicting object masks and empty masks. This ultimately degrades performance, not only on the original dataset but also in negative conditions, where the model may continue to produce object masks despite the absence of valid audio-visual correspondence.

While balanced training with positive and negative pairs exposes the model to diverse scenarios, it needs explicit guidance to learn when audio and visual features truly correspond. To address this, we propose using a classifier to directly supervise audio-visual similarity learning, creating clear decision boundaries for correspondence detection.

% Observation of Table \ref{tab:only_negative} reveals that merely introducing negative pairs is insufficient. 
% Due to the inherent bias in existing models, which often fail to effectively utilize audio information, the model tends to behave more like a segmentation model trained solely on visual cues. Without additional guidance, the introduction of negative pairs can lead to confusion during training, as the model alternates between predicting object masks and empty masks. This lack of clarity ultimately results in degraded performance not only on the original dataset but also in negative conditions, where the model may continue to produce object masks despite the absence of valid audio-visual correspondence.

% While balanced training pairs provide diverse scenarios, the model needs explicit guidance to learn when audio and visual features correspond. We propose using Binary Cross-Entropy (BCE) loss to directly supervise the audio-visual similarity learning, creating clear decision boundaries for correspondence detection.


Given multi-scale visual features $\mathcal{F}_i \in \mathbb{R}^{h_i \times w_i \times C_i}$ from the backbone, we use the final-stage features $\mathcal{F}_4 \in \mathbb{R}^{h_4\times w_4 \times C_4}$ and audio features $\mathcal{F}_A \in \mathbb{R}^{D_a}$ for similarity computation. We project $\mathcal{F}_A$ to $C_4$ dimensions via a linear layer and apply spatial pooling to $\mathcal{F}_4$ to obtain aligned features $\hat{\mathcal{F}}_A, \hat{\mathcal{F}}_V \in \mathbb{R}^{C_4}$. 
% , we first align their feature dimensions. Specifically, $\mathcal{F}_A$ is projected to a $D_v$-dimensional space through a linear layer, while $\mathcal{F}_V$ is spatially pooled to obtain global visual features. This yields aligned features $\hat{\mathcal{F}}_A, \hat{\mathcal{F}}_V \in \mathbb{R}^{D_v}$. 
Their correspondence is then computed through cosine similarity:
\begin{equation}
   s(F_A, F_V) = \text{cos}(\hat{\mathcal{F}}_A, \hat{\mathcal{F}}_V).
\end{equation}

We then apply BCE loss to explicitly guide similarity learning in a contrastive manner:
\begin{equation}
\begin{split}
   \mathcal{L}_{\text{BCE}} = -\frac{1}{|\mathcal{P}| + |\mathcal{N}|} & \sum_{j=1}^{|\mathcal{P}| + |\mathcal{N}|}  \left( y_j \log \sigma(s_j) \right. \\
   & \left. + (1 - y_j) \log (1 - \sigma(s_j)) \right),
\end{split}
\end{equation}
where $\sigma(\cdot)$ is the sigmoid function, $y_j$ is the binary label (1 for positive pairs, 0 for negative pairs), and $|\mathcal{P}| + |\mathcal{N}|$ is the total number of positive pairs and the total number of negative pairs respectively. By explicitly supervising the similarity learning, the BCE loss forces the model to maximize similarity for positive pairs (where valid audio-visual correspondence exists) and minimize it for negative pairs (where no correspondence is present). This guidance helps the model learn to interpret audio as a cue for segmentation only when there is a meaningful alignment with the visual input, reducing confusion in cases without correspondence. 


\subsection{Joint Training with Segmentation}
\label{subsec:total_loss}
% Our overall objective is to minimize the following loss function, which combines Binary Cross-Entropy (BCE) loss for correspondence guidance and segmentation loss for prediction quality:
Our total loss objective function $\mathcal{L}$ can be computed as follows:
% combines BCE loss for correspondence guidance and segmentation loss for prediction quality:
\begin{equation}
   \mathcal{L} = \lambda \mathcal{L}_{\text{BCE}} + \mathcal{L}_{\text{Seg}},
\end{equation}
where $\lambda$ is a balancing weight. Together, these loss terms enforce robust and effective learning in AVS models:
1) The first term determines whether segmentation should occur based on audio-visual correspondence;
2) The second term ensures correct segmentation masks when correspondence exists;
3) For negative pairs, the empty GT masks naturally guide the segmentation loss to suppress predictions.

This simple, well-motivated approach can achieve strong performance without relying on complex model modifications, making our method easier to implement, tune, and integrate with existing AVS architectures.

% and the segmentation loss ensures accurate pixel-wise segmentation.
% This joint training creates a natural division of responsibilities: BCE loss determines whether segmentation should occur based on audio-visual correspondence; Segmentation loss ensures high-quality segmentation masks when correspondence exists; And for negative pairs, the empty ground truth masks naturally guide the Segmentation loss to suppress predictions.

% Compared to more complex alternatives that rely on elaborate architectural modifications or sophisticated loss functions, our approach achieves robust performance through simple, well-motivated components, makes our method easier to implement and tune.



\begin{table*}[!htbp]
\centering
\resizebox{\textwidth}{!}{%
\begin{tabular}{c|c|cc|ccc|ccc|ccc|ccc}
\hline
& & \multicolumn{2}{c|}{\textbf{Positive audio input}}&  \multicolumn{9}{c|}{\textbf{Negative audio input}}& \multicolumn{3}{c}{\textbf{Global metric}}\\
 & & \multicolumn{2}{c|}{}&  \multicolumn{3}{c}{\textbf{Slience}}&\multicolumn{3}{c}{\textbf{Noise}}&   \multicolumn{3}{c|}{\textbf{Offscreen sound}}& \multicolumn{3}{c}{}\\ \hline
 \textbf{Test set}&  \textbf{Model}& \textbf{mIoU ↑}& \textbf{F-score ↑}& \textbf{mIoU ↓}&  \textbf{F-score ↓}&\textbf{FPR ↓}&\textbf{mIoU ↓}& \textbf{F-score ↓}& \textbf{FPR ↓}& \textbf{mIoU ↓}&  \textbf{F-score ↓}&\textbf{FPR ↓}& \textbf{G-mIoU↑}& \textbf{G-F↑}&\textbf{G-FPR↓}\\ \hline
\multirow{7}{*}{\textbf{AVSBench-S4}}& AVSBench~\cite{zhou2022audio}& 78.7& 87.9& 76.6&   87.1&0.19&77.6& 88.0& 0.18& 78.2&  88.2&0.19
& 35.032&  21.479&0.186
\\
 & AVSegFormer~\cite{gao2024avsegformer}& 82.1& 89.9& 83.0&   90.4&0.19&83.0& 90.4& 0.19& 83.0&  90.4&0.19
& 28.199&  17.355&0.188
\\
 & Stepping-Stones~\cite{ma2024stepping}& 83.2& 91.3& 82.2&   91.3&0.19&82.2& 91.3& 0.19& 82.5&  91.3&0.19
& 28.980&  15.806&0.190
\\
 & SAMA-AVS~\cite{liu2024annofree}& 83.1& 90.0& 56.2&   69.1&0.17&59.3& 73.8& 0.13& 68.7&  79.0&
0.17& 52.688&  40.417&0.155
\\
 & CAVP~\cite{chen2024cavp}& 78.7& 88.8& 78.7&   88.8&0.19&78.7& 88.8& 0.19& 78.7&  88.8&0.19
& 33.526&  19.891&0.185\\
 & COMBO~\cite{yang2024combo}& \textbf{84.7}& \textbf{91.9}& 84.6&   91.9&0.19&84.6& 91.9& 0.19& 84.6&  91.9&
0.19& 26.062&  14.888&0.190\\
     
     & \cellcolor{lightblue} \textbf{AVSBench + Ours} & \cellcolor{lightblue} 78.1& \cellcolor{lightblue} 88.2& \cellcolor{lightblue} \textbf{0.2}& \cellcolor{lightblue} \textbf{22.6}& \cellcolor{lightblue} \textbf{0.00}& \cellcolor{lightblue} \textbf{0.2}& \cellcolor{lightblue} \textbf{22.6}& \cellcolor{lightblue} \textbf{0.00}& \cellcolor{lightblue} \textbf{0.2}& \cellcolor{lightblue} \textbf{22.6}& \cellcolor{lightblue} \textbf{0.00}& \cellcolor{lightblue} \textbf{87.672}& \cellcolor{lightblue} \textbf{82.461}& \cellcolor{lightblue} \textbf{0.000}\\
    
     & \cellcolor{lightblue} \textbf{AVSegFormer + Ours} & \cellcolor{lightblue} 74.2& \cellcolor{lightblue} 84.8& \cellcolor{lightblue} \textbf{0.2}& \cellcolor{lightblue} \textbf{22.6}& \cellcolor{lightblue} \textbf{0.00}& \cellcolor{lightblue} \textbf{0.3}& \cellcolor{lightblue} \textbf{22.7}& \cellcolor{lightblue} \textbf{0.00}& \cellcolor{lightblue} \textbf{0.5}& \cellcolor{lightblue} \textbf{22.9}& \cellcolor{lightblue} \textbf{0.00}&  \cellcolor{lightblue}\textbf{85.069}& \cellcolor{lightblue} \textbf{80.849}& \cellcolor{lightblue}\textbf{0.001}\\ \hline
    

\multirow{7}{*}{\textbf{AVSBench-MS3}}& AVSBench~\cite{zhou2022audio}& 54.0& 64.5& 27.6&   53.5&0.05&31.7& 57.4& 0.05& 42.2&  62.4&0.09
& 59.468&  51.036&0.072
\\
 & AVSegFormer~\cite{gao2024avsegformer}& 61.3& 73.8& 53.2&   68.2&0.13&47.5& 63.8& 0.09& 50.3&  66.0&0.11
& 54.889&  46.571&0.103
\\
 & Stepping-Stones~\cite{ma2024stepping}& 67.3& 77.6& 45.6&   
72.5&0.09&43.8& 
72.3& 0.08& 41.0&  63.3&0.15& 61.439&  43.937&0.114
\\
 & SAMA-AVS~\cite{liu2024annofree}& \textbf{68.6}& \textbf{78.3}& 29.7&   
36.8&0.09&39.2& 
46.6& 0.12& 44.1&  49.9&
0.14& 65.308&  65.038&0.125
\\
 & CAVP~\cite{chen2024cavp}& 45.8& 61.7& 45.8&   61.7&0.11&45.8& 61.7& 0.11& 45.8&  61.7&0.11
& 49.647&  47.262&0.110\\
 & COMBO \cite{yang2024combo}& 59.2& 71.2& -&   
-&-&-& -& -& -&  -&
-& -&  -&-\\ 

    & \cellcolor{lightblue} \textbf{AVSBench + Ours} & \cellcolor{lightblue} 51.3& \cellcolor{lightblue} 64.5& \cellcolor{lightblue} 9.8& \cellcolor{lightblue} 17.7& \cellcolor{lightblue} \textbf{0.00}& \cellcolor{lightblue} \textbf{9.9}& \cellcolor{lightblue} \textbf{25.8}& \cellcolor{lightblue} \textbf{0.00}& \cellcolor{lightblue} \textbf{9.1}& \cellcolor{lightblue} \textbf{20.3}& \cellcolor{lightblue} \textbf{0.00}& \cellcolor{lightblue} \textbf{65.427}& \cellcolor{lightblue} \textbf{70.911}& \cellcolor{lightblue} \textbf{0.001}\\ 

    & \cellcolor{lightblue} \textbf{AVSegFormer + Ours} & \cellcolor{lightblue} 61.5& \cellcolor{lightblue} 74.0& \cellcolor{lightblue} \textbf{9.1}& \cellcolor{lightblue} \textbf{17.0}& \cellcolor{lightblue} \textbf{0.00}& \cellcolor{lightblue} \textbf{9.4}& \cellcolor{lightblue} \textbf{17.2}& \cellcolor{lightblue} \textbf{0.00}& \cellcolor{lightblue} \textbf{9.1}& \cellcolor{lightblue} \textbf{17.0}& \cellcolor{lightblue} \textbf{0.00}& \cellcolor{lightblue} \textbf{73.354}& \cellcolor{lightblue} \textbf{78.244}& \cellcolor{lightblue} \textbf{0.000}\\ 

\hline
\end{tabular}%
}
\vspace{-2mm}
\caption{Performance comparison of various models on different audio input types and global metrics.}
\label{tab:main_table}
\end{table*}


%=================================================
% \subsection{Challenges in Audio-Visual Correspondence}

% The bias problem in Audio-Visual Segmentation presents several fundamental challenges that make it particularly difficult to solve:

% \textbf{Inherent Visual Dominance:} Visual features often provide stronger and more consistent cues for object segmentation compared to audio signals. This natural imbalance leads models to overly rely on visual information, making it challenging to properly integrate audio cues in the decision process. Even when audio signals are absent or irrelevant, visually salient objects can trigger false segmentation predictions.

% \textbf{Dataset Limitations:} Existing datasets predominantly contain positive audio-visual pairs where sounds align with visible objects. This creates a strong training bias where models learn to always generate segmentation masks when they detect visually prominent objects. The lack of diverse negative scenarios in training data makes it difficult for models to learn when not to predict segmentation masks.

% \textbf{Architectural Constraints:} Traditional AVS architectures are designed to fuse audio and visual features for mask generation, but they lack explicit mechanisms to suppress predictions when audio-visual correspondence is absent. The challenge lies in designing architectures that can not only detect positive correspondences but also confidently identify and handle scenarios where no valid correspondence exists.

% These challenges are further compounded in real-world applications where audio conditions can be highly variable and unpredictable, making robust audio-visual segmentation a critical yet difficult problem to solve.
% % =================================================


\section{Experiment}

% \begin{figure*}[h]
%     \centering
%     \includesvg[width=1.0\textwidth]{image/examples_s4.svg}
%     \caption{Qualitative results of S4 dataset on AVSBench baseline}
%     \label{fig:examples}
% \end{figure*}

% \begin{figure*}[h]
%     \centering
%     \includesvg[width=1.0\textwidth]{image/examples_ms3_2.svg}
%     \caption{Qualitative results of MS3 dataset on AVSBench baseline}
%     \label{fig:examples}
% \end{figure*}

\begin{figure*}[h]
\vspace{-2mm}
    \centering
    \includegraphics[width=1.0\textwidth]{avs_examples.pdf}
    % \caption{Qualitative results on S4 dataset. We can see that \yapeng{xxxx} \yapeng{We also need qualitative results on MS3. If we do not have space, we should add the results in appendix}}

    \caption{\textbf{Performance comparison of different AVS models under various audio conditions on Robust-S4 dataset}. Existing SOTA methods \cite{liu2024annofree, ma2024stepping, chen2024cavp} segment objects primarily based on visual salience, exhibiting a strong visual bias. In contrast, our approach achieves accurate segmentation with original audio while successfully reject predict in negative scenarios (e.g., silence, noise, off-screen).}
    \label{fig:examples}
\vspace{-4mm}
\end{figure*}


% \subsection{Results}

\subsection{Setup}


\noindent\textbf{Dataset.} We utilize the AVSBench-Robust Benchmark for our evaluation, which is designed to rigorously assess AVS capabilities. Further details on the dataset specifics and video categories have been discussed in Sec. \ref{sec:problem_benchmark}.

\noindent\textbf{Baselines.} We benchmark our model against notable methods including AVSBench~\cite{zhou2022audio} and AVSegFormer~\cite{gao2024avsegformer}, representing fusion-based and prompt-based approaches, respectively. We also compared our method with the CAVP~\cite{chen2024cavp}, Stepping-Stones~\cite{ma2024stepping}, SAMA-AVS~\cite{liu2024annofree} and COMBO~\cite{yang2024combo}. These baselines allow us to demonstrate the broad applicability of our method by comparing it against state-of-the-art models designed to address different aspects of audio-visual segmentation.


\noindent\textbf{Evaluation Metrics.} 
%The evaluation of our model's performance on the AVSBench-Robust benchmark employs key metrics previously introduced in Sec.~\ref{sec:problem_benchmark}.  
Evaluation metrics, including mIoU, F1 score, FPR, and G-mIou, G-F, G-FPR, are used to assess the segmentation accuracy and robustness of AVS models.
%assess accuracy in segmenting aligned audio-visual pairs and model robustness against negative samples. 
%Further details about these metrics and their specific applications in AVS can be found in the aforementioned section.

\noindent\textbf{Implementation:} Our implementation for the AVSBench model employ the Pyramid Vision Transformer (PVT-v2)~\cite{wang2022pvt} pretrained on the ImageNet dataset~\cite{russakovsky2015imagenet} as the visual backbone, which processes video frames of size $H \times W = 224 \times 224$ and output multiple scales visual feature $\mathcal{F}_{V_i} \in \mathbb{R}^{h_i \times w_i \times C_i}$ for $i = 1,\ldots,4$, The channel dimensions $C_i$ correspond to \{64, 128, 320, 512\} for each respective scale.
% This transformer-based architecture, pretrained on the ImageNet dataset~\cite{russakovsky2015imagenet}, processes video frames resized to $H \times W = 224 \times 224$. The architecture's stages have channel dimensions set at $C_1, C_2, C_3, C_4 = [64, 128, 320, 512]$, allowing for detailed feature extraction across multiple scales. 
For audio input, we employ VGGish~\cite{hershey2017vggish} pretrained on AudioSet~\cite{gemmeke2017audioset} to extract features $\mathcal{F}_A \in \mathbb{R}^{128}$ from each one-second audio clips.
% For audio data, we employ the VGGish model, a VGG-style network pretrained on the AudioSet dataset, which processes audio inputs segmented into one-second clips. 
This model is trained using the Adam~\cite{kingma2014adam} optimizer with a learning rate of $1 \times 10^{-4}$, batch size of 4, and loss weighting factor $\lambda = 1$. Training durations are 15 epochs for the semi-supervised S4 setup and 30 epochs for the fully-supervised MS3 setup on an NVIDIA RTX A5000 GPU.


\subsection{Experimental Comparison}
Our extensive experimental comparisons reveal several significant findings in AVS performance as shown in Table~\ref{tab:main_table}. 

\noindent
\textbf{SOTA methods fail under negative audio conditions}, demonstrating a strong visual bias and ineffective audio-visual integration.
Surprisingly, recent methods like Stepping-Stones~\cite{ma2024stepping} and CAVP~\cite{chen2024cavp} achieve nearly identical mIoU and F-scores regardless of whether the input audio is silent, irrelevant, or noisy.  These methods consistently exhibit high False Positive Rates (FPR), ranging from 0.17 to 0.19 across all negative scenarios on AVSBench-S4, indicating a significant reliance on visual cues.  This issue also impacts their global metrics, with G-mIoU scores between 28.19 and 35.03. These results suggest that these methods fail to effectively leverage audio information in this multimodal segmentation task. While this issue is somewhat less pronounced on the MS3 dataset, it remains present.

\vspace{1mm}
\noindent
\textbf{Our method resolves bias while maintaining performance.} When integrated with AVSBench~\cite{zhou2022audio}, our approach performs comparable positive audio performance (mIoU: 78.1, F-score: 88.2) while achieving perfect robustness to negative audio inputs with an FPR of 0.00 across all negative conditions. Similarly, our AVSegFormer~\cite{gao2024avsegformer} variant demonstrates only minimal degradation in positive audio metrics while achieving perfect FPR scores. Most notably, our approach achieves superior global metrics, with our AVSBench variant reaching a G-mIoU of 87.672 and G-F score of 82.461, substantially outperforming existing methods. The consistent improvement in robustness across two very different AVS architectures demonstrates the effectiveness and generality of our approach. Fig.~\ref{fig:examples} provides examples of S4 dataset visualizations. Visualizations for MS3 dataset are included in the supplementary material.


\vspace{1mm}
\noindent
\textbf{Our method excels in global metrics across scenarios}, showing consistent improvements in both the single-source and more complex multi-source settings. In the MS3, it maintains perfect robustness with an FPR of 0.00 in all negative conditions and achieves impressive global metrics; the AVSegFormer variant records a G-mIoU of 73.354 and a G-F score of 78.244. These results confirm our method's scalability and its significant advancement in addressing the longstanding limitations of existing AVS methods.







% \begin{figure}
%     \centering  
%     \begin{subfigure}{0.25\textwidth}
%         \centering
%         \includegraphics[width=\textwidth]{image/AVSBench_S4_cos_original_offscreen_comparison_histogram.png}
%         \subcaption{Baseline Model}
%         \label{fig:similarity_baseline}
%     \end{subfigure}%
%     \begin{subfigure}{0.25\textwidth}
%         \centering
%         \includegraphics[width=\textwidth]{image/AVSBench_S4_10ood_cos_original_offscreen_comparison_histogram.png}
%         \subcaption{Our Model}
%         \label{fig:similarity_our_model}
%     \end{subfigure}
%     \caption{Comparison of Normalized Similarity Distributions for Audio-Visual Pairs.
%     Each plot shows the distribution of similarity scores (x-axis) between audio and visual embeddings, normalized using a sigmoid function. The y-axis represents the density of similarity scores. The blue bars and lines indicate the distribution for original (matched) audio-visual pairs, while the red bars and lines represent offscreen (mismatched) pairs. The solid lines show kernel density estimates for each category. In the baseline model (left), there is significant overlap between the two distributions, indicating poor separability between matched and mismatched pairs. In contrast, our model (right) demonstrates a clear distinction, with minimal overlap between the distributions, highlighting improved discriminative capability.}
%     \label{fig:similarity_comparison}
% \end{figure}




\begin{figure}[t]
\vspace{-2mm}
    \centering
    \begin{subfigure}[t]{0.24\textwidth}
        \centering
        \includegraphics[width=\textwidth]{avs_s4_sim_distribution_before.pdf}
        \caption{Before}
    \end{subfigure}%
    \begin{subfigure}[t]{0.24\textwidth}
        \centering
        \includegraphics[width=\textwidth]{avs_s4_sim_distribution_after.pdf}
        \caption{After}
    \end{subfigure}
    % \caption{\yapeng{add detailed captions -- seems not discussed in the paper. }}    
    \caption{Cosine similarity distributions between paired features before and after training.(a) Positive and negative pairs exhibit similar distributions, indicating the model’s limited ability to distinguish audio-visual correspondence. (b) After training with classifier-guided similarity learning, the distributions are well-separated, demonstrating the model's enhanced capability to identify valid audio-visual pairs. }
    \label{fig:similarity_hist}
\vspace{-3mm}
\end{figure}



\subsection{Impact of Positive-Negative Pair Ratio}

% \begin{table}[t]
% \centering
% \resizebox{\columnwidth}{!}{
% \begin{tabular}{c|c|c|c|c|c|c}
% \hline
%  & Model & Pos Pairs & Neg Pairs & G-mIoU & G-F & G-FPR \\
% \hline
% \multirow{4}{*}{S4} & Baseline & 100\% & 0\% & 35.032 & 21.479 & 0.186 \\
% \cline{2-7}
% & \multirow{3}{*}{Ours} & 90\% & 10\% & \underline{87.672} & \underline{82.461} & 0.000 \\
% & & 80\% & 20\% & \textbf{87.780} & \textbf{82.114} & 0.000 \\
% & & 70\% & 30\% & 87.204 & 82.233 & 0.000 \\
% \hline
% \multirow{4}{*}{MS3} & Baseline & 100\% & 0\% & 59.468 & 51.036 & 0.072 \\
% \cline{2-7}
% & \multirow{3}{*}{Ours} & 90\% & 10\% & \underline{65.427} & \underline{70.911} & 0.001 \\
% & & 80\% & 20\% & \textbf{67.909} & \textbf{72.572} & 0.003 \\
% & & 70\% & 30\% & 66.251 & 72.908 & 0.000 \\
% \hline
% \end{tabular}
% }
% \caption{Impact of positive-negative ratio on AVS performance}
% \label{tab:pair_ratio_study}
% \end{table}



Our investigation into the ratio of positive to negative audio-visual pairs reveals important insights about training data composition for robust audio-visual segmentation.

As illustrated in Table~\ref{tab:pair_ratio_study}, we found that \textit{introducing negative samples, even in small proportions, dramatically improves performance.} Without negative samples, the baseline model shows poor performance with the G-mIoU of 35.032 and a high FPR of 0.186 on the S4 dataset.
Introducing just 10\% negative samples yields substantial gains, improving G-mIoU to 87.672 and reducing FPR to 0.000. Similar improvements are observed in the MS3 dataset, where G-mIoU increases from 59.468 to 65.427 and FPR drops from 0.072 to 0.001, demonstrating the crucial role of negative samples in developing robust models.

While further increasing the proportion of negative samples (to 20\% or 30\%) maintains similar performance levels, the marginal gains are minimal compared to the 10\% setting. For instance, the G-mIoU difference between 10\% and 20\% negative samples is less than 0.3\% on S4 and 2.5\% on MS3. Considering that adding 10\% negative pairs only increases training time by approximately 10\% while achieving nearly optimal performance, we adopt this ratio as our default setting, offering an efficient balance between robustness and training cost.





\subsection{Abaltion Study}

% \subsubsection{Effectiveness of Negative Samples and Loss Design}

% \begin{table}[t]
% \centering
% \resizebox{\columnwidth}{!}{  
% \begin{tabular}{c|c|c|c|c|c}
% \hline
%  & Negative samples & $\mathcal{L}_{\text{BCE}}$& G-mIoU↑ & G-F↑ & G-FPR↓ \\
% \hline
%  \multirow{3}{*}{S4} & \xmark& \xmark& 35.032& 21.479& 0.186
% \\
%  & \cmark& \xmark& 34.847& 21.993& 0.189\\
%  & \cmark& \cmark& \textbf{87.672}& \textbf{82.461}& \textbf{0.000}
% \\ \hline
%  \multirow{3}{*}{MS3} & \xmark& \xmark& 59.468& 51.036& 0.072
% \\
%  & \cmark& \xmark& 55.489& 30.057& 0.095\\
%  & \cmark& \cmark& \textbf{66.605}& \textbf{70.590}& \textbf{0.004}\\
% \hline
% \end{tabular}
% }
% \caption{Ablation study on negative samples and BCE loss}
% \label{tab:only_negative}
% \end{table}
%A common assumption would be that simply adding negative samples (\eg, silent or background noise) might improve the model's ability to differentiate between sound-producing and non-sound-producing regions. To validate this, we conducted an ablation study with AVSBench as the baseline~\cite{zhou2022audio}, analyzing configurations with and without negative samples and classifier guidance, as summarized in Tab.~\ref{tab:only_negative}.


A common assumption might be that simply adding negative samples would enhance the model's ability to distinguish between sound-producing and non-sound-producing visual regions in AVS. To test this hypothesis, we conducted an ablation study using AVSBench~\cite{zhou2022audio} as the baseline, comparing configurations with and without negative samples and classifier guidance, as summarized in Table~\ref{tab:only_negative}.
 % (\eg, silent or background noise)
\noindent
\textbf{Negative samples alone fail to address the bias problem.} 
Our experimental results highlight the limitations of using only negative samples. Without explicit loss guidance, adding negative pairs not only fails to improve performance but can even lead to significant degradation, particularly on the challenging MS3 dataset. Here, G-mIoU drops from 59.47 to 55.49, and G-F decreases dramatically from 51.04 to 30.06. This degradation occurs because the model becomes confused when alternately exposed to scenarios requiring empty predictions and those with salient object mask predictions, ultimately compromising performance even on standard positive audio inputs.

%Experimental results confirm this limitation - without appropriate loss guidance, model performance significantly deteriorates, particularly in the challenging MS3 dataset where G-mIoU drops from 59.468 to 55.489 and G-F decreases dramatically from 51.036 to 30.057. This performance degradation occurs because the model becomes confused when alternately exposed to scenarios requiring empty predictions and ground-truth mask predictions, leading to compromised performance even in standard positive audio scenarios.

\noindent
\textbf{Combining negative samples with classifier guidance enables robust segmentation.}  Our full approach shows substantial improvements across all metrics. On the Robust S4 dataset, we achieve G-mIoU of 87.672, G-F of 82.461, and perfect G-FPR of 0. Similar gains are observed on MS3, with G-mIoU of 66.605, G-F of 70.590, and near-perfect G-FPR of 0.004. 
The effectiveness of classifier guidance is illustrated in Fig. \ref{fig:similarity_hist}: initially, audio-visual feature similarities cluster around 0.5 for both positive and negative pairs; after training, they are well-separated (0.75 for positive vs. 0.30 for negative), demonstrating enhanced discrimination. This improved feature alignment enables strong  performance on positive cases while accurately suppressing predictions in negative scenarios.
% The classifier guidance with BCE loss provides explicit feature alignment, allowing the model to effectively differentiate between sound-producing and silent regions. This approach enables strong performance on positive cases while accurately handling negative scenarios.
% The ablation results underscore that while negative samples provide essential training data diversity, they must be accompanied by appropriate loss guidance to be effective. Without such guidance, the model struggles to reconcile conflicting objectives, leading to degraded performance. 
The classifier guidance serves as a critical learning framework for effectively utilize negative samples while maintaining its original capabilities, resulting in a robust AVS system. % that can reliably handle both positive and negative audio conditions.

Further experiments on unseen audio categories demonstrate the generalization capability of our approach. Due to space constraints, we refer readers to the supplementary material for detailed results.



\begin{table}[t]
\vspace{-2mm} 
\centering
\resizebox{\columnwidth}{!}{
\begin{tabular}{c|c|c|c|c|c|c}
\hline
 & Model & Pos Pairs & Neg Pairs & G-mIoU & G-F & G-FPR \\
\hline
\multirow{4}{*}{S4} & Baseline & 100\% & 0\% & 35.032 & 21.479 & 0.186 \\
\cline{2-7}
& \multirow{3}{*}{Ours} & 90\% & 10\% & \underline{87.672} & \underline{82.461} & 0.000 \\
& & 80\% & 20\% & \textbf{87.780} & \textbf{82.114} & 0.000 \\
& & 70\% & 30\% & 87.204 & 82.233 & 0.000 \\
\hline
\multirow{4}{*}{MS3} & Baseline & 100\% & 0\% & 59.468 & 51.036 & 0.072 \\
\cline{2-7}
& \multirow{3}{*}{Ours} & 90\% & 10\% & \underline{65.427} & \underline{70.911} & 0.001 \\
& & 80\% & 20\% & \textbf{67.909} & \textbf{72.572} & 0.003 \\
& & 70\% & 30\% & 66.251 & 72.908 & 0.000 \\
\hline
\end{tabular}
}
\vspace{-2mm} 
\caption{Impact of positive-negative ratio on AVS performance}
\label{tab:pair_ratio_study}
\end{table}


\begin{table}[t]
\centering
\resizebox{\columnwidth}{!}{  
\begin{tabular}{c|c|c|c|c|c}
\hline
 & Negative samples & $\mathcal{L}_{\text{BCE}}$& G-mIoU↑ & G-F↑ & G-FPR↓ \\
\hline
 \multirow{3}{*}{S4} & \xmark& \xmark& 35.032& 21.479& 0.186
\\
 & \cmark& \xmark& 34.847& 21.993& 0.189\\
 & \cmark& \cmark& \textbf{87.672}& \textbf{82.461}& \textbf{0.000}
\\ \hline
 \multirow{3}{*}{MS3} & \xmark& \xmark& 59.468& 51.036& 0.072
\\
 & \cmark& \xmark& 55.489& 30.057& 0.095\\
 & \cmark& \cmark& \textbf{66.605}& \textbf{70.590}& \textbf{0.004}\\
\hline
\end{tabular}
}
\vspace{-2mm} 
\caption{Effects of negative samples and classifier guidance.}
\label{tab:only_negative}
\vspace{-3mm} 
\end{table}







%  ==================================

% \textbf{Dataset.} Our evaluation employs the AVSBench-Robust Benchmark, a rigorous framework designed to assess models on their capacity to segment sound-producing objects from video frames. The benchmark is divided into two subsets: the Single-Source Audio Segmentation Subset (S4), containing 4,932 videos across 23 diverse categories, and the Multi-Source Audio Segmentation Subset (MS3), which includes 424 videos in equally varied categories, challenging models to manage scenarios with multiple concurrent sound sources.

% \textbf{Evaluation Metrics.} The performance of our model under the AVSBench-Robust benchmark is quantified using key metrics designed for comprehensive assessment. These include the mean Intersection over Union (mIoU) for measuring accuracy in segmenting positively aligned audio-visual pairs, the F1 score for overall precision and recall balance, and the False Positive Rate (FPR) to evaluate the model's robustness in handling negative samples. A detailed discussion of these metrics and their relevance to AVS research is provided in Section \ref{sec:problem_benchmark}.

%  ==================================
% \textbf{Finding 1: Existing SOTA methods consistently fail under negative audio conditions (silence, noise, and offscreen sound), demonstrating a concerning bias in their audio-visual alignment capabilities.} Traditional methods, including SAMA-AVS\cite{liu2024annofree}, Stepping-Stones\cite{ma2024stepping}, and CAVP\cite{chen2024cavp}, exhibit persistently high False Positive Rates (FPR) ranging from 0.17 to 0.19 across all negative audio scenarios on AVSBench-S4. This limitation is further reflected in their global metrics, where baseline methods achieve relatively low G-mIoU scores between 28.199 and 35.032.

% \textbf{Finding 3: The improvements are consistent across both single-source (AVSBench-S4) and more complex multi-source (AVSBench-MS3) scenarios, demonstrating the scalability of our approach.} In the challenging multi-source setting of AVSBench-MS3, our method maintains its robust performance with an FPR of 0.00 across all negative conditions, while achieving strong global metrics with our AVSegFormer variant reaching a G-mIoU of 73.354 and G-F score of 78.244. These comprehensive results demonstrate that our approach successfully addresses a critical limitation in existing audio-visual segmentation methods while maintaining competitive performance on standard tasks, representing a significant advancement in the field's state-of-the-art.


%  ====== Old table findings ========
% Table \ref{tab:main_table} provides a detailed comparison of performance metrics of our models, which outperform other state-of-the-art methods in challenging negative audio conditions.

% \textbf{Superior Handling of Negative Audio Conditions:}
% A critical observation from our experiments is that, aside from our models, other methods struggle significantly under negative audio conditions. While methods like Stepping-Stones\cite{ma2024stepping}, SAMA-AVS\cite{liu2024annofree} and CAVP\cite{chen2024cavp} traditionally perform well with positive audio inputs, their performance degrades under negative conditions such as silence, noise, and offscreen sound, often resulting in higher False Positive Rates (FPR). In stark contrast, both variants of our models—Ours(AVSBench) and Ours(AVSegFormer)—maintain an FPR of 0.00 across all negative scenarios, highlighting their exceptional robustness and precision in avoiding false positives.

% \textbf{Comparison with AVSBench and AVSegFormer:}
% When comparing Ours(AVSBench) to the standard AVSBench, it is notable that our model achieves comparable results in mIoU and F-score under positive audio inputs but excels in negative audio conditions. Specifically, while the standard AVSBench shows an FPR of up to 0.19 in negative scenarios, Ours(AVSBench) consistently maintains a zero FPR, demonstrating a significant improvement in specificity and error reduction.

% Similarly, Ours(AVSegFormer) shows a minimal decrease in performance metrics such as mIoU and F-score compared to the baseline AVSegFormer under positive audio conditions. However, it vastly outperforms in negative audio scenarios, again maintaining a zero FPR compared to the baseline's higher rates. This performance underscores the enhanced noise discrimination capabilities of our model, making it highly effective for applications requiring high fidelity audio-visual alignment.


% \textbf{Global Metric Performance:}
% On a global scale, both Ours(AVSBench) and Ours(AVSegFormer) score impressively high in Global mIoU (G-mIoU) and Global F-score (G-F), reflecting their robust overall performance across different test conditions and scenarios. These metrics not only substantiate the models' effectiveness in handling complex audio-visual environments but also their adaptability and reliability across both single-source and multi-source settings.
%  ==================================




% \textbf{Increasing negative samples consistently yields near-zero FPR, with an optimal ratio enhancing performance. } Our experiments show that the model maintains remarkable robustness with FPR values of 0.000-0.003 across all configurations after introducing negative samples. However, the segmentation performance peaks at 20\% negative samples, achieving optimal G-mIoU of 87.780 for S4 and 67.909 for MS3. Further increasing to 30\% leads to slight performance degradation, particularly in the more challenging MS3 dataset where G-mIoU drops to 66.251. This reveals that while our approach reliably eliminates false positives regardless of the exact ratio, maintaining approximately 20\% negative samples provides the best trade-off for overall segmentation performance across both single-source and multi-source scenarios.

%  ==================================

% \subsubsection{Positive Audio Input Performance}

% In scenarios with positive audio input, where the sound source aligns with visible objects, our approach demonstrates comparable performance to leading SOTA methods. Specifically, our model, denoted as "Ours (AVSBench)" and "Ours (AVSegFormer)" in Table \ref{tab:main_table}, achieves competitive mIoU and F-score values on both AVSBench-S4 and AVSBench-MS3 datasets, indicating that the incorporation of negative audio samples does not compromise the model's segmentation performance on positive cases. On the AVSBench-S4 subset, the best-performing model, COMBO, achieves an mIoU of 84.7 and F-score of 91.9, while our model reaches an mIoU of 78.1 and F-score of 88.2, demonstrating robust segmentation capabilities. This consistent performance across positive cases supports the adaptability of our model to scenarios with aligned audio-visual cues.

% \subsubsection{Negative Audio Input Performance}

% One of the main objectives of our approach is to address limitations in existing AVS models by improving performance on negative audio inputs, including silence, noise, and offscreen sounds. For these cases, the goal is to suppress segmentation activation, thereby reducing false positives. Table \ref{tab:main_table} shows that our model significantly outperforms other methods in this regard, achieving near-zero FPR and mIoU values on negative audio samples. For instance, on the AVSBench-S4 dataset, our model attains an FPR of 0.00 for both silence and noise cases, indicating that it successfully minimizes unnecessary activations in the absence of relevant audio-visual correspondence. Similarly, our model achieves low mIoU and F-score values on negative audio cases, which demonstrates its capability to effectively ignore visual content when no corresponding audio signal exists.


%  ==================================
% \textbf{Simply incorporating negative samples without appropriate loss guidance not only fails to address the bias problem but also disrupts the model's original capability in handling positive samples.} When we only added negative samples to the training set without modifying the loss function, the model's performance deteriorated, particularly in the challenging MS3 dataset where G-mIoU dropped from 59.468 to 55.489 and G-F plummeted from 51.036 to 30.057. This significant performance degradation reveals a fundamental issue: without proper guidance, the model becomes confused when alternately exposed to scenarios requiring empty predictions (negative audio) and ground-truth mask predictions (positive audio). The original biased model, which primarily relied on visual information, struggles to reconcile these conflicting objectives, leading to compromised performance even in standard positive audio scenarios.

% \textbf{The synergistic combination of negative samples and BCE loss is crucial for achieving robust audio-visual segmentation.} Our complete method, which integrates both components, demonstrates remarkable improvements across all metrics. In the S4 dataset, this combination leads to dramatic performance gains (G-mIoU: 87.672, G-F: 82.461) while achieving a perfect G-FPR of 0.000. Similar substantial improvements are observed in the more challenging MS3 dataset, where our method achieves G-mIoU of 66.605 and G-F of 70.590, with near-perfect G-FPR of 0.004. The BCE loss provides explicit guidance for the model to effectively distinguish between sound-producing and silent regions, enabling it to maintain strong performance in positive scenarios while correctly handling negative cases.
%  ==================================


% \textbf{Adding Negative Samples Only:} 
% When negative samples were added without BCE loss, we observed limited improvements in segmentation metrics. For the S4 dataset, the G-mIoU and G-F scores showed only marginal gains, while on the more complex MS3 dataset, performance actually declined, with G-mIoU dropping from 59.468 to 55.489 and G-F from 51.036 to 30.057. These results indicate that merely adding negative samples is insufficient to guide the model effectively, as it lacks the necessary framework to differentiate positive and negative regions accurately.

% \textbf{Combining Negative Samples with BCE Loss:} 
% Our full method, which integrates negative samples with BCE loss, yielded substantial improvements across both datasets. On the S4 dataset, G-mIoU and G-F increased dramatically to 87.672 and 82.461, respectively, with a perfect FPR of 0.000, indicating no false positives in negative audio conditions. Similar gains were observed for MS3, where G-mIoU and G-F rose to 66.605 and 70.590, with FPR reduced to 0.004. This combination enables the model to align similarity scores with binary labels, allowing it to distinguish between relevant and irrelevant regions with high precision.

%  ==================================
% For both the S4 and MS3 datasets, adding negative samples alone led to minimal improvement in segmentation metrics and even slightly worsened performance on MS3, as seen by drops in G-mIoU and G-F scores. This suggests that without further guidance, negative samples alone may not provide the necessary information to effectively distinguish non-sound-producing regions.

% Our full method, which combines negative samples with the Binary Cross-Entropy (BCE) loss, achieves significant improvements across both datasets. By aligning labels with similarity scores through BCE loss, the model is better equipped to separate positive and negative samples, resulting in higher G-mIoU and G-F scores and a near-zero false positive rate for negative samples.
%  ==================================



% \subsubsection{Percentage of negative samples during training}

% To further investigate the impact of negative samples, we conducted an ablation study on the ratio of positive to negative audio-visual pairs during training, as shown in Table \ref{tab:pair_ratio_study}. The study examines how increasing the percentage of negative samples affects model performance on the S4 and MS3 datasets, specifically in terms of G-mIoU, G-F score, and False Positive Rate (FPR).

% \textbf{Increasing Negative Sample Percentage:}
% The baseline model, trained without any negative samples, demonstrates the lowest performance on both datasets. For instance, on the S4 dataset, the baseline achieves a G-mIoU of 35.032 and a G-F of 21.479, with an average FPR of 0.186. Similarly, on the MS3 dataset, the baseline shows a G-mIoU of 59.468 and a G-F of 51.036, with a higher FPR of 0.072, indicating a pronounced difficulty in distinguishing non-sound-producing regions.

% As the percentage of negative samples increases, we observe a substantial improvement in model performance across both datasets. With 10\% negative samples, our model on the S4 dataset achieves a G-mIoU of 87.672 and a G-F of 82.461, with an FPR reduced to zero. Increasing the negative sample percentage to 20\% and 30\% maintains this high level of segmentation accuracy and robustness, with only slight variations in G-mIoU and G-F scores.

% On the more complex MS3 dataset, introducing 20\% negative samples yields the highest G-mIoU of 67.909 and a G-F score of 72.572, while the FPR remains low at 0.003. At 30\% negative samples, the G-mIoU slightly decreases to 66.251, but the FPR is again reduced to zero. These results suggest that, while increasing negative samples generally enhances performance, an optimal balance around 20\% negative samples maximizes the model’s ability to differentiate sound-related from non-sound regions without overfitting.



% The results on the MS3 dataset demonstrate that introducing negative samples improves model performance in distinguishing sound-related regions. The baseline (0\% negative samples) shows the lowest G-mIoU and G-F scores, along with a high false positive rate (FPR) of 0.072, indicating difficulty in identifying non-sound regions.

% Adding 5-20\% negative samples leads to substantial gains, with the best results at 20\% negative samples (G-mIoU: 67.909, G-F: 72.572, FPR: 0.003), highlighting that a balanced inclusion of negative examples helps refine segmentation accuracy. At 30\% negative samples, performance stabilizes with only slight variations, suggesting that adding more may yield diminishing returns.



\section{Conclusion and Discussion}
% Our systematic investigation into AVS models reveals a critical gap between their intended functionality and actual behavior. Through 
Our comprehensive study using AVSBench-Robust reveals  that current SOTA methods exhibit strong visual bias, generating segmentation masks based predominantly on visual salience regardless of audio context. To address this issue, we introduce a simple yet effective approach combining balanced training with negative audio-visual pairs and classifier-guided feature alignment, which significantly improves model robustness while maintaining competitive performance on standard AVS tasks.
% , such as varying levels of background noise or multiple overlapping sounds
While our method effectively addresses the robustness issue, several challenges remain. Our approach is constrained by the baseline model's performance on positive samples, and real-world applications may encounter even more challenging conditions than those covered in our benchmark. 
We hope our work could inspire further research in this significant and worthwhile field.
% We believe our work opens up new directions for developing truly robust AVS systems in real-world scenarios.


{
    \small
    \bibliographystyle{ieeenat_fullname}
    \bibliography{main}
}

% \clearpage
\pagenumbering{gobble}
\maketitlesupplementary

\section{Additional Results on Embodied Tasks}

To evaluate the broader applicability of our EgoAgent's learned representation beyond video-conditioned 3D human motion prediction, we test its ability to improve visual policy learning for embodiments other than the human skeleton.
Following the methodology in~\cite{majumdar2023we}, we conduct experiments on the TriFinger benchmark~\cite{wuthrich2020trifinger}, which involves a three-finger robot performing two tasks: reach cube and move cube. 
We freeze the pretrained representations and use a 3-layer MLP as the policy network, training each task with 100 demonstrations.

\begin{table}[h]
\centering
\caption{Success rate (\%) on the TriFinger benchmark, where each model's pretrained representation is fixed, and additional linear layers are trained as the policy network.}
\label{tab:trifinger}
\resizebox{\linewidth}{!}{%
\begin{tabular}{llcc}
\toprule
Methods       & Training Dataset & Reach Cube & Move Cube \\
\midrule
DINO~\cite{caron2021emerging}         & WT Venice        & 78.03     & 47.42     \\
DoRA~\cite{venkataramanan2023imagenet}          & WT Venice        & 81.62     & 53.76     \\
DoRA~\cite{venkataramanan2023imagenet}          & WT All           & 82.40     & 48.13     \\
\midrule
EgoAgent-300M & WT+Ego-Exo4D      & 82.61    & 54.21      \\
EgoAgent-1B   & WT+Ego-Exo4D      & \textbf{85.72}      & \textbf{57.66}   \\
\bottomrule
\end{tabular}%
}
\end{table}

As shown in Table~\ref{tab:trifinger}, EgoAgent achieves the highest success rates on both tasks, outperforming the best models from DoRA~\cite{venkataramanan2023imagenet} with increases of +3.32\% and +3.9\% respectively.
This result shows that by incorporating human action prediction into the learning process, EgoAgent demonstrates the ability to learn more effective representations that benefit both image classification and embodied manipulation tasks.
This highlights the potential of leveraging human-centric motion data to bridge the gap between visual understanding and actionable policy learning.



\section{Additional Results on Egocentric Future State Prediction}

In this section, we provide additional qualitative results on the egocentric future state prediction task. Additionally, we describe our approach to finetune video diffusion model on the Ego-Exo4D dataset~\cite{grauman2024ego} and generate future video frames conditioned on initial frames as shown in Figure~\ref{fig:opensora_finetune}.

\begin{figure}[b]
    \centering
    \includegraphics[width=\linewidth]{figures/opensora_finetune.pdf}
    \caption{Comparison of OpenSora V1.1 first-frame-conditioned video generation results before and after finetuning on Ego-Exo4D. Fine-tuning enhances temporal consistency, but the predicted pixel-space future states still exhibit errors, such as inaccuracies in the basketball's trajectory.}
    \label{fig:opensora_finetune}
\end{figure}

\subsection{Visualizations and Comparisons}

More visualizations of our method, DoRA, and OpenSora in different scenes (as shown in Figure~\ref{fig:supp pred}). For OpenSora, when predicting the states of $t_k$, we use all the ground truth frames from $t_{0}$ to $t_{k-1}$ as conditions. As OpenSora takes only past observations as input and neglects human motion, it performs well only when the human has relatively small motions (see top cases in Figure~\ref{fig:supp pred}), but can not adjust to large movements of the human body or quick viewpoint changes (see bottom cases in Figure~\ref{fig:supp pred}).

\begin{figure*}
    \centering
    \includegraphics[width=\linewidth]{figures/supp_pred.pdf}
    \caption{Retrieval and generation results for egocentric future state prediction. Correct and wrong retrieval images are marked with green and red boundaries, respectively.}
    \label{fig:supp pred}
\end{figure*}

\begin{figure*}[t]
    \centering
    \includegraphics[width=0.9\linewidth]{figures/motion_prediction.pdf}
    \vspace{-0.5mm}
    \caption{Motion prediction results in scenes with minor changes in observation.}
    \vspace{-1.5mm}
    \label{fig:motion_prediction}
\end{figure*}

\subsection{Finetuning OpenSora on Ego-Exo4D}

OpenSora V1.1~\cite{opensora}, initially trained on internet videos and images, produces severely inconsistent results when directly applied to infer future videos on the Ego-Exo4D dataset, as illustrated in Figure~\ref{fig:opensora_finetune}.
To address the gap between general internet content and egocentric video data, we fine-tune the official checkpoint on the Ego-Exo4D training set for 50 epochs.
OpenSora V1.1 proposed a random mask strategy during training to enable video generation by image and video conditioning. We adopted the default masking rate, which applies: 75\% with no masking, 2.5\% with random masking of 1 frame to 1/4 of the total frames, 2.5\% with masking at either the beginning or the end for 1 frame to 1/4 of the total frames, and 5\% with random masking spanning 1 frame to 1/4 of the total frames at both the beginning and the end.

As shown in Fig.~\ref{fig:opensora_finetune}, despite being trained on a large dataset, OpenSora struggles to generalize to the Ego-Exo4D dataset, producing future video frames with minimal consistency relative to the conditioning frame. While fine-tuning improves temporal consistency, the moving trajectories of objects like the basketball and soccer ball still deviate from realistic physical laws. Compared with our feature space prediction results, this suggests that training world models in a reconstructive latent space is more challenging than training them in a feature space.


\section{Additional Results on 3D Human Motion Prediction}

We present additional qualitative results for the 3D human motion prediction task, highlighting a particularly challenging scenario where egocentric observations exhibit minimal variation. This scenario poses significant difficulties for video-conditioned motion prediction, as the model must effectively capture and interpret subtle changes. As demonstrated in Fig.~\ref{fig:motion_prediction}, EgoAgent successfully generates accurate predictions that closely align with the ground truth motion, showcasing its ability to handle fine-grained temporal dynamics and nuanced contextual cues.

\section{OpenSora for Image Classification}

In this section, we detail the process of extracting features from OpenSora V1.1~\cite{opensora} (without fine-tuning) for an image classification task. Following the approach of~\cite{xiang2023denoising}, we leverage the insight that diffusion models can be interpreted as multi-level denoising autoencoders. These models inherently learn linearly separable representations within their intermediate layers, without relying on auxiliary encoders. The quality of the extracted features depends on both the layer depth and the noise level applied during extraction.


\begin{table}[h]
\centering
\caption{$k$-NN evaluation results of OpenSora V1.1 features from different layer depths and noising scales on ImageNet-100. Top1 and Top5 accuracy (\%) are reported.}
\label{tab:opensora-knn}
\resizebox{0.95\linewidth}{!}{%
\begin{tabular}{lcccccc}
\toprule
\multirow{2}{*}{Timesteps} & \multicolumn{2}{c}{First Layer} & \multicolumn{2}{c}{Middle Layer} & \multicolumn{2}{c}{Last Layer} \\
\cmidrule(r){2-3}   \cmidrule(r){4-5}  \cmidrule(r){6-7}  & Top1           & Top5           & Top1            & Top5           & Top1           & Top5          \\
\midrule
32        &  6.10           & 18.20             & 34.04               & 59.50             & 30.40             & 55.74             \\
64        & 6.12              & 18.48              & 36.04               & 61.84              & 31.80         & 57.06         \\
128       & 5.84             & 18.14             & 38.08               & 64.16              & 33.44       & 58.42 \\
256       & 5.60             & 16.58              & 30.34               & 56.38              &28.14          & 52.32        \\
512       & 3.66              & 11.70            & 6.24              & 17.62              & 7.24              & 19.44  \\ 
\bottomrule
\end{tabular}%
}
\end{table}

As shown in Table~\ref{tab:opensora-knn}, we first evaluate $k$-NN classification performance on the ImageNet-100 dataset using three intermediate layers and five different noise scales. We find that a noise timestep of 128 yields the best results, with the middle and last layers performing significantly better than the first layer.
We then test this optimal configuration on ImageNet-1K and find that the last layer with 128 noising timesteps achieves the best classification accuracy.

\section{Data Preprocess}
For egocentric video sequences, we utilize videos from the Ego-Exo4D~\cite{grauman2024ego} and WT~\cite{venkataramanan2023imagenet} datasets.
The original resolution of Ego-Exo4D videos is 1408×1408, captured at 30 fps. We sample one frame every five frames and use the original resolution to crop local views (224×224) for computing the self-supervised representation loss. For computing the prediction and action loss, the videos are downsampled to 224×224 resolution.
WT primarily consists of 4K videos (3840×2160) recorded at 60 or 30 fps. Similar to Ego-Exo4D, we use the original resolution and downsample the frame rate to 6 fps for representation loss computation.
As Ego-Exo4D employs fisheye cameras, we undistort the images to a pinhole camera model using the official Project Aria Tools to align them with the WT videos.

For motion sequences, the Ego-Exo4D dataset provides synchronized 3D motion annotations and camera extrinsic parameters for various tasks and scenes. While some annotations are manually labeled, others are automatically generated using 3D motion estimation algorithms from multiple exocentric views. To maximize data utility and maintain high-quality annotations, manual labels are prioritized wherever available, and automated annotations are used only when manual labels are absent.
Each pose is converted into the egocentric camera's coordinate system using transformation matrices derived from the camera extrinsics. These matrices also enable the computation of trajectory vectors for each frame in a sequence. Beyond the x, y, z coordinates, a visibility dimension is appended to account for keypoints invisible to all exocentric views. Finally, a sliding window approach segments sequences into fixed-size windows to serve as input for the model. Note that we do not downsample the frame rate of 3D motions.

\section{Training Details}
\subsection{Architecture Configurations}
In Table~\ref{tab:arch}, we provide detailed architecture configurations for EgoAgent following the scaling-up strategy of InternLM~\cite{team2023internlm}. To ensure the generalization, we do not modify the internal modules in InternML, \emph{i.e.}, we adopt the RMSNorm and 1D RoPE. We show that, without specific modules designed for vision tasks, EgoAgent can perform well on vision and action tasks.

\begin{table}[ht]
  \centering
  \caption{Architecture configurations of EgoAgent.}
  \resizebox{0.8\linewidth}{!}{%
    \begin{tabular}{lcc}
    \toprule
          & EgoAgent-300M & EgoAgent-1B \\
          \midrule
    Depth & 22    & 22 \\
    Embedding dim & 1024  & 2048 \\
    Number of heads & 8     & 16 \\
    MLP ratio &    8/3   & 8/3 \\
    $\#$param.  & 284M & 1.13B \\
    \bottomrule
    \end{tabular}%
    }
  \label{tab:arch}%
\end{table}%

Table~\ref{tab:io_structure} presents the detailed configuration of the embedding and prediction modules in EgoAgent, including the image projector ($\text{Proj}_i$), representation head/state prediction head ($\text{MLP}_i$), action projector ($\text{Proj}_a$) and action prediction head ($\text{MLP}_a$).
Note that the representation head and the state prediction head share the same architecture but have distinct weights.

\begin{table}[t]
\centering
\caption{Architecture of the embedding ($\text{Proj}_i$, $\text{Proj}_a$) and prediction ($\text{MLP}_i$, $\text{MLP}_a$) modules in EgoAgent. For details on module connections and functions, please refer to Fig.~2 in the main paper.}
\label{tab:io_structure}
\resizebox{\linewidth}{!}{%
\begin{tabular}{lcl}
\toprule
       & \multicolumn{1}{c}{Norm \& Activation} & \multicolumn{1}{c}{Output Shape}  \\
\midrule
\multicolumn{3}{l}{$\text{Proj}_i$ (\textit{Image projector})} \\
\midrule
Input image  & -          & 3$\times$224$\times$224 \\
Conv 2D (16$\times$16) & -       & Embedding dim$\times$14$\times$14    \\
\midrule
\multicolumn{3}{l}{$\text{MLP}_i$ (\textit{State prediction head} \& \textit{Representation head)}} \\
\midrule
Input embedding  & -          & Embedding dim \\
Linear & GELU       & 2048          \\
Linear & GELU       & 2048          \\
Linear & -          & 256           \\
Linear & -          & 65536     \\
\midrule
\multicolumn{3}{l}{$\text{Proj}_a$ (\textit{Action projector})} \\
\midrule
Input pose sequence  & -          & 4$\times$5$\times$17 \\
Conv 2D (5$\times$17) & LN, GELU   & Embedding dim$\times$1$\times$1    \\
\midrule
\multicolumn{3}{l}{$\text{MLP}_a$ (\textit{Action prediction head})} \\
\midrule
Input embedding  & -          & Embedding dim$\times$1$\times$1 \\
Linear & -          & 4$\times$5$\times$17     \\
\bottomrule
\end{tabular}%
}
\end{table}


\subsection{Training Configurations}
In Table~\ref{tab:training hyper}, we provide the detailed training hyper-parameters for experiments in the main manuscripts.

\begin{table}[ht]
  \centering
  \caption{Hyper-parameters for training EgoAgent.}
  \resizebox{0.86\linewidth}{!}{%
    \begin{tabular}{lc}
    \toprule
    Training Configuration & EgoAgent-300M/1B \\
    \midrule
    Training recipe: &  \\
    optimizer & AdamW~\cite{loshchilov2017decoupled} \\
    optimizer momentum & $\beta_1=0.9, \beta_2=0.999$ \\
    \midrule
    Learning hyper-parameters: &  \\
    base learning rate & 6.0E-04 \\
    learning rate schedule & cosine \\
    base weight decay & 0.04 \\
    end weight decay & 0.4 \\
    batch size & 1920 \\
    training iters & 72,000 \\
    lr warmup iters & 1,800 \\
    warmup schedule & linear \\
    gradient clip & 1.0 \\
    data type & float16 \\
    norm epsilon & 1.0E-06 \\
    \midrule
    EMA hyper-parameters: &  \\
    momentum & 0.996 \\
    \bottomrule
    \end{tabular}%
    }
  \label{tab:training hyper}%
\end{table}%

\clearpage


\section*{Supplementary Material}


\subsection*{A. Additional Experimental Results}

\subsubsection*{A.1 Evaluation on Unseen Audio Categories}
% Test setup for unseen categories
    % Test Audio:
    % Thunderstorm, 
    % Chicken, rooster
    % Sheep
    % Tuning fork
    % Washing machine, Shower, Subway, Human Cough
    
% Performance metrics
% Comparison with baseline models

To assess generalization capability, we evaluate model performance on four diverse unseen audio (tuning fork, rooster, sheep, and thunderstorm) across both S4 and MS3 datasets.



Tables~\ref{tab:unseen_audio_s4} and~\ref{tab:unseen_audio_ms3} present comparative results between the baseline AVSBench and our approach. On S4, the baseline exhibits significant visual bias, maintaining consistently high performance (mIoU: $\sim$78\%, F-score: $\sim$88\%) regardless of audio input. Our approach effectively mitigates this bias, reducing mIoU to near-zero and decreasing false positive ratios from 0.187 to 10\textsuperscript{-5}.


The improvement is equally pronounced in the more challenging multi-source scenario (MS3). Our method effectively suppresses false predictions, achieving very low mIoU ($\sim 0.09$) and F-scores ($\sim 0.17$) across all unseen categories. For certain categories (e.g., rooster), our approach achieves complete suppression with zero false positives, demonstrating robust audio-visual correspondence even in complex multi-source scenarios.



\subsubsection*{A.2 Multi-Source (MS3) Dataset Visualizations}

Figure~\ref{fig:ms3_vis} illustrates our method's performance on complex multi-source scenarios. The visualizations demonstrate segmentation results under various audio conditions: original audio (positive), silence, noise, and off-screen sounds, providing qualitative evidence of our model's effectiveness in handling diverse acoustic environments.


\begin{table}[t]
    \footnotesize
    \centering
    \begin{tabular}{l|c|c|c|c}
        \hline
        Category & Method & mIoU $\downarrow$ & F-score $\downarrow$ & FPR $\downarrow$ \\
        \hline
        \multirow{2}{*}{Tuning Fork} & AVSBench & 78.14 & 88.16 & 0.187 \\
        & AVSBench $+$ Ours & \textbf{0.003} & \textbf{0.248} & \textbf{0.0001} \\
        \hline
        \multirow{2}{*}{Rooster} & AVSBench & 78.19 & 88.26 & 0.186 \\
        & AVSBench $+$ Ours & \textbf{0.002} & \textbf{0.234} & \textbf{3.82e-5} \\
        \hline
        \multirow{2}{*}{Sheep} & AVSBench & 78.26 & 88.25 & 0.187 \\
        & AVSBench $+$ Ours & \textbf{0.002} & \textbf{0.236} & \textbf{8.43e-5} \\
        \hline
        \multirow{2}{*}{Thunder} & AVSBench & 78.19 & 88.23 & 0.187 \\
        & AVSBench $+$ Ours & \textbf{0.002} & \textbf{0.233} & \textbf{4.03e-5} \\
        \hline
    \end{tabular}
    \caption{Performance comparison on unseen audio categories for single-source (S4) dataset. Lower values indicate better ability to avoid false predictions for unfamiliar sounds.}
    \label{tab:unseen_audio_s4}    
\end{table}

\begin{table}[t]
    \footnotesize
    \centering
    \begin{tabular}{l|c|c|c|c}
        \hline
        Category & Method & mIoU $\downarrow$ & F-score $\downarrow$ & FPR $\downarrow$ \\
        \hline
        \multirow{2}{*}{Tuning Fork} & AVSBench & 41.44 & 62.19 & 0.091 \\
        & AVSBench $+$ Ours & \textbf{0.103} & \textbf{0.186} & \textbf{0.007} \\
        \hline
        \multirow{2}{*}{Rooster} & AVSBench & 44.52 & 63.04 & 0.102 \\
        & AVSBench $+$ Ours & \textbf{0.091} & \textbf{0.169} & \textbf{0.000} \\
        \hline
        \multirow{2}{*}{Sheep} & AVSBench & 40.14 & 62.13 & 0.082 \\
        & AVSBench $+$ Ours & \textbf{0.091} & \textbf{0.179} & \textbf{0.0001} \\
        \hline
        \multirow{2}{*}{Thunder} & AVSBench & 38.16 & 61.82 & 0.074 \\
        & AVSBench $+$ Ours & \textbf{0.091} & \textbf{0.178} & \textbf{9.07e-5} \\
        \hline
    \end{tabular}
    \caption{Performance comparison on unseen audio categories for multi-source (MS3) dataset. Lower values indicate better ability to avoid false predictions for unfamiliar sounds.}
    \label{tab:unseen_audio_ms3}    
\end{table}

\begin{figure*}[t]
\vspace{-2mm}
    \centering
    \includegraphics[width=1.0\textwidth]{avs_examples_ms3.pdf}
    % \caption{Qualitative results on S4 dataset. We can see that \yapeng{xxxx} \yapeng{We also need qualitative results on MS3. If we do not have space, we should add the results in appendix}}

    \caption{Performance comparison of different AVS models under various audio conditions on Robust-MS3 dataset. Existing SOTA methods \cite{liu2024annofree, ma2024stepping, chen2024cavp} segment objects primarily based on visual salience, exhibiting a strong visual bias. In contrast, our approach achieves accurate segmentation with original audio while successfully reject predict in negative scenarios (e.g., silence, noise, off-screen).}
    \label{fig:ms3_vis}
\end{figure*}


\subsubsection*{A.3 Effect of feature alignment strategies}

To thoroughly evaluate our choice of cosine similarity for feature alignment, we compare it with two intuitive alternatives: Euclidean distance, which directly measures feature space proximity, and concatenation-based alignment, which preserves complete feature information. Table~\ref{table:feature_alignment} presents comparative results across both S4 and MS3 datasets.


\begin{table*}[htbp]
\centering
\resizebox{\textwidth}{!}{%
\begin{tabular}{c|c|c|c|c|c|c|c|c|c|c|c|c|c|c|c}
\hline
& \multirow{3}{*}{\begin{tabular}[c]{@{}c@{}}Guide Method\end{tabular}} & \multicolumn{2}{c|}{\textbf{Positive audio input}} & \multicolumn{9}{c|}{\textbf{Negative audio input}} & \multicolumn{3}{c}{\textbf{Global metric}} \\
\cline{5-13}
& & \multicolumn{2}{c|}{} & \multicolumn{3}{c|}{\textbf{Silence}} & \multicolumn{3}{c|}{\textbf{Noise}} & \multicolumn{3}{c|}{\textbf{Offscreen sound}} & \multicolumn{3}{c}{} \\
\cline{3-16}
\textbf{Test set} & & \textbf{mIoU↑} & \textbf{F-score↑} & \textbf{mIoU↓} & \textbf{F-score↓} & \textbf{FPR↓} & \textbf{mIoU↓} & \textbf{F-score↓} & \textbf{FPR↓} & \textbf{mIoU↓} & \textbf{F-score↓} & \textbf{FPR↓} & \textbf{G-mIoU↑} & \textbf{G-F↑} & \textbf{G-FPR↓} \\
\hline
\multirow{3}{*}{S4} & cosine & \textbf{78.1} & \textbf{88.2} & \textbf{0.2} & \textbf{22.6} & \textbf{0.000} & \textbf{0.2} & \textbf{22.6} & \textbf{0.000} & \textbf{0.2} & \textbf{22.6} & \textbf{0.00} & \textbf{87.672} & \textbf{82.461} & \textbf{0.000} \\
& Euclidean& 69.3& 82.5& 5.9& 33.8& 0.032& 0.2& 23.8& 0.000& 0.2& 24.3& 0.000& 81.144& 77.283& 0.004\\
& Concat& 77.5& 87.5& 1.0& 23.1& 0.003& 0.2& 22.6& 0.000& 0.2& 22.6& 0.000& 87.139& 82.047& 0.000\\
\hline
\multirow{3}{*}{MS3} & cosine 
& \textbf{51.3} & \textbf{64.5} & \textbf{9.8} & \textbf{17.7} & \textbf{0.00} & \textbf{9.9} & \textbf{25.8} & \textbf{0.000} & \textbf{9.1} & \textbf{20.3} & \textbf{0.000} & \textbf{66.605} & \textbf{70.590} & \textbf{0.004} \\
& Euclidean
& 43.5& 56.4& 10.7& 21.8& 0.011& 11.1& 24.2& 0.028& 9.5& 28.3& 0.001& 58.526& 64.447& 0.014\\
& Concat& 49.7& 62.8& 12.1& 21.0& 0.006& 14.8& 32.3& 0.012& 14.1& 29.0& 0.011& 63.053& 67.333& 0.011\\
\hline
\end{tabular}
}
\caption{Comparison of different feature alignment strategies. Results show performance across positive and negative audio scenarios, as well as global metrics.}
\label{table:feature_alignment}
\end{table*}

All three methods demonstrate effectiveness in audio-visual feature alignment, with each achieving reasonable performance on positive samples. However, cosine similarity exhibits superior performance, particularly in negative suppression scenarios. On S4, while concatenation-based alignment maintains competitive positive metrics (mIoU: 77.5\%), cosine similarity achieves better balance between positive performance (mIoU: 78.1\%) and negative suppression (FPR: 0.00). This pattern extends to the more challenging MS3 dataset, where cosine similarity shows notably better global metrics (G-mIoU: 66.61, G-F: 70.59) compared to both alternatives.

The advantage of cosine similarity likely stems from its inherent normalization property and focus on directional relationships, making it particularly suitable for cross-modal feature comparison. While Euclidean distance is sensitive to feature magnitude variations and concatenation may preserve redundant information, cosine similarity captures the essential semantic alignment between modalities while maintaining computational efficiency.


\noindent\textbf{Implementation Details:} For all methods, audio features are first projected from 128 to 256 dimensions to match the visual feature dimension, and visual features undergo adaptive average pooling to obtain global representations. The methods then differ in their alignment computation:
\begin{itemize}
\item \textbf{Cosine Similarity} computes normalized directional alignment using the standard cosine similarity function: $\text{similarity} = \cos(\hat{\mathcal{F}}_A, \hat{\mathcal{F}}_V)$, where $\hat{\mathcal{F}}_A$ and $\hat{\mathcal{F}}_V$ are the projected audio and visual features respectively.
\item \textbf{Euclidean Distance} measures direct feature space proximity through L2 norm: $\text{similarity} = -||\hat{\mathcal{F}}_A - \hat{\mathcal{F}}_V||_2$. The negative sign converts distance to similarity, ensuring larger values indicate stronger alignment.
\item \textbf{Concatenation-based} alignment employs a three-layer MLP that processes the concatenated features $[\hat{\mathcal{F}}_A; \hat{\mathcal{F}}_V]$ (512 dimensions). The network progressively reduces dimensionality (512 → 256 → 128 → 1) with ReLU activation and dropout (0.1) for regularization, learning more complex non-linear relationships between modalities.
\end{itemize}
% For concatenation-based alignment, we implement a multi-layer perceptron (MLP) architecture that processes the concatenated audio-visual features. The audio features are first projected from 128 to 256 dimensions to match the visual feature dimension. Visual features undergo adaptive average pooling to obtain a global representation. The concatenated 512-dimensional features are then processed through a three-layer MLP with ReLU activation and dropout (0.1) for regularization. The network progressively reduces feature dimensionality (512 → 256 → 128 → 1) before outputting the final alignment score. This architecture allows the model to learn complex non-linear relationships between the modalities while maintaining computational efficiency.


\subsubsection*{A.4 Sensitivity to hyperparameter choices}

We evaluate the model's sensitivity to the classifier guidance weight $\lambda_{\text{BCE}}$ by varying its value from 0.2 to 1.0. Table~\ref{table:lambda_sensitivity} presents the quantitative results across both S4 and MS3 datasets.

\begin{table*}[t]
    \centering
    % \caption{Performance analysis with different $\lambda_{\text{BCE}}$ values on S4 and MS3 datasets. Results show the trade-off between positive performance and negative suppression across different weight settings.}
    \resizebox{\linewidth}{!}{
    \begin{tabular}{c|c|cc|ccc|ccc|ccc|ccc}
        \hline
        Dataset & $\lambda_{\text{BCE}}$ & \multicolumn{2}{c|}{Positive} & \multicolumn{3}{c|}{Silence} & \multicolumn{3}{c|}{Noise} & \multicolumn{3}{c|}{Offscreen} & \multicolumn{3}{c}{Global} \\
        \cline{3-16}
        & & mIoU$\uparrow$ & F$\uparrow$ & mIoU$\downarrow$ & F$\downarrow$ & FPR$\downarrow$ & mIoU$\downarrow$ & F$\downarrow$ & FPR$\downarrow$ & mIoU$\downarrow$ & F$\downarrow$ & FPR$\downarrow$ & G-mIoU$\uparrow$ & G-F$\uparrow$ & G-FPR$\downarrow$ \\
        \hline
        \multirow{5}{*}{S4} & 1.0 & 78.15 & 88.22 & 0.16 & 22.59 & 0.00 & 0.16 & 22.59 & 0.00 & 0.16 & 22.59 & 0.00 & 87.67 & 82.46 & 0.000 \\
        & 0.8 & 78.97 & 88.78 & 0.16 & 22.60 & 0.00 & 0.16 & 22.60 & 0.00 & 0.16 & 22.60 & 0.00 & 88.18 & 82.70 & 0.000 \\
        & 0.6 & 77.98 & 88.00 & 0.18 & 22.64 & 0.00 & 0.17 & 22.63 & 0.00 & 0.17 & 22.64 & 0.00 & 87.56 & 82.34 & 0.000 \\
        & 0.4 & 78.76 & 88.56 & 1.06 & 32.21 & 0.00 & 0.16 & 22.59 & 0.00 & 0.16 & 22.65 & 0.00 & 87.54 & 80.73 & 0.000 \\
        & 0.2 & 78.95 & 88.68 & 0.22 & 22.93 & 0.00 & 0.17 & 22.85 & 0.00 & 0.18 & 22.85 & 0.00 & 88.16 & 82.50 & 0.000 \\
        \hline
        \multirow{5}{*}{MS3} & 1.0 & 51.27 & 64.51 & 9.84 & 17.72 & 0.00 & 9.93 & 25.78 & 0.00 & 9.11 & 20.33 & 0.00 & 65.43 & 70.91 & 0.001 \\
        & 0.8 & 53.32 & 66.32 & 9.64 & 18.09 & 0.00 & 11.75 & 35.07 & 0.01 & 9.81 & 31.23 & 0.00 & 66.86 & 68.98 & 0.004 \\
        & 0.6 & 51.23 & 64.94 & 13.37 & 21.38 & 0.03 & 15.20 & 37.80 & 0.02 & 9.67 & 26.79 & 0.00 & 64.55 & 67.99 & 0.011 \\
        & 0.4 & 52.75 & 65.85 & 13.66 & 26.41 & 0.02 & 10.33 & 35.82 & 0.01 & 10.33 & 35.82 & 0.01 & 66.12 & 66.57 & 0.007 \\
        & 0.2 & 52.81 & 65.56 & 9.81 & 18.01 & 0.00 & 12.99 & 35.02 & 0.01 & 9.81 & 21.94 & 0.00 & 66.32 & 69.97 & 0.005 \\
        \hline
    \end{tabular}
    }
    \caption{Performance analysis with different $\lambda_{\text{BCE}}$ values. Results demonstrate the model's robustness to this hyperparameter choice.}    
    \label{table:lambda_sensitivity}
    
    
\end{table*}

Experimental results demonstrate strong robustness to this hyperparameter choice. On S4, the G-mIoU variation remains minimal (87.54-88.18\%, $\Delta$=0.64\%), with consistent perfect negative suppression (FPR: 0.000) across all settings. The MS3 dataset shows slightly larger but still modest variations (G-mIoU: 64.55-66.86\%, $\Delta$=2.31\%), attributable to its inherently more complex multi-source scenarios. This stability suggests that the model's performance is not heavily dependent on precise hyperparameter tuning.

\subsubsection*{A.5 Detailed Analysis of the Effect of Negative Samples and Classifier Guidance}
To dissect the individual contributions of our key components, we conduct ablation experiments on both negative sample integration and classifier guidance. Table~\ref{table:full_comparison}, compare to Table 4 presents comprehensive results across different configurations.
Adding negative samples alone proves insufficient and can even degrade performance. On S4, while the baseline achieves 78.7\% mIoU with high false positives (FPR: ~0.19), incorporating only negative samples marginally improves robustness but compromises positive performance (mIoU: 79.0\%). This effect is more pronounced on MS3, where positive performance significantly degrades (mIoU drops from 54.0\% to 51.6\%) while maintaining high false positive rates. This degradation occurs because the model, without explicit guidance, struggles to establish clear decision boundaries between valid and invalid audio-visual correspondences.
The integration of classifier guidance ($\mathcal{L}_{\text{BCE}}$) with negative samples yields substantial improvements. This combination achieves 78.1\% mIoU on S4 while reducing FPR to nearly zero across all negative scenarios. Similarly on MS3, it maintains competitive positive performance (51.3\% mIoU) while effectively suppressing false predictions (FPR: 0.004). 


\begin{table*}[htbp]
\centering
\resizebox{\textwidth}{!}{%
\begin{tabular}{c|c|c|c|c|c|c|c|c|c|c|c|c|c|c|c|c}
\hline
& \multirow{3}{*}{\begin{tabular}[c]{@{}c@{}}Negative\\samples\end{tabular}} & \multirow{3}{*}{$\mathcal{L}_{\text{BCE}}$} & \multicolumn{2}{c|}{\textbf{Positive audio input}} & \multicolumn{9}{c|}{\textbf{Negative audio input}} & \multicolumn{3}{c}{\textbf{Global metric}} \\
\cline{6-14}
& & & \multicolumn{2}{c|}{} & \multicolumn{3}{c|}{\textbf{Silence}} & \multicolumn{3}{c|}{\textbf{Noise}} & \multicolumn{3}{c|}{\textbf{Offscreen sound}} & \multicolumn{3}{c}{} \\
\cline{4-17}
\textbf{Test set} & & & \textbf{mIoU↑} & \textbf{F-score↑} & \textbf{mIoU↓} & \textbf{F-score↓} & \textbf{FPR↓} & \textbf{mIoU↓} & \textbf{F-score↓} & \textbf{FPR↓} & \textbf{mIoU↓} & \textbf{F-score↓} & \textbf{FPR↓} & \textbf{G-mIoU↑} & \textbf{G-F↑} & \textbf{G-FPR↓} \\
\hline
\multirow{3}{*}{S4} & \xmark & \xmark & 78.7 & 87.9 & 76.6 & 87.1 & 0.19 & 77.6 & 88.0 & 0.18 & 78.2 & 88.2 & 0.19 & 35.032 & 21.479 & 0.186 \\
& \cmark & \xmark & 79.0 & 88.4 & 76.6 & 86.6 & 0.19 & 77.9 & 87.7 & 0.20 & 78.5 & 88.0 & 0.19 & 34.847 & 21.993 & 0.189 \\
& \cmark & \cmark & 78.1 & 88.2 & 0.2 & 22.6 & 0.00 & 0.2 & 22.6 & 0.00 & 0.2 & 22.6 & 0.00 & 87.672 & 82.461 & 0.000 \\
\hline
\multirow{3}{*}{MS3} & \xmark & \xmark & 54.0 & 64.5 & 27.6 & 53.5 & 0.05 & 31.7 & 57.4 & 0.05 & 42.2 & 62.4 & 0.09 & 59.468 & 51.036 & 0.072 \\
& \cmark & \xmark & 51.6 & 23.3 & 34.4 & 52.1 & 0.10 & 40.2 & 58.6 & 0.10 & 45.2 & 62.3 & 0.10 & 55.489 & 30.057 & 0.095 \\
& \cmark & \cmark & 51.3 & 64.5 & 9.8 & 17.7 & 0.00 & 9.9 & 25.8 & 0.00 & 9.1 & 20.3 & 0.00 & 66.605 & 70.590 & 0.004 \\
\hline
\end{tabular}
}
\caption{Ablation study of negative samples and classifier guidance. Results show performance on positive and negative audio inputs, as well as global metrics. The checkmarks (\cmark) and crosses (\xmark) indicate whether negative samples and classifier guidance ($\mathcal{L}_{\text{BCE}}$) are used in each configuration}
\label{table:full_comparison}
\end{table*}



\subsection*{B Detailed Analysis of Audio Conditions}
% Statistical analysis

\noindent\textbf{S4 Dataset Composition:} The Robust S4 dataset spans four major categories with diverse audio-visual characteristics, as shown in Figure~\ref{fig:category_dist}. The distribution is relatively balanced across device (32.2\%), music (32.1\%), and animal (23.2\%) categories, with human sounds comprising 12.5\% of the dataset. This balanced distribution helps ensure robust evaluation across different types of audio-visual scenarios.

\begin{figure}[t]
    \centering
    \includegraphics[width=\linewidth]{benchmark_s4.pdf}
    \caption{Distribution of cross-category combinations in multi-source scenarios. This analysis reveals common co-occurrence patterns, with music-human and animal-human being the most frequent combinations.}
    \label{fig:category_dist}
\end{figure}


% \noindent\textbf{Category Characteristics:}
% \begin{itemize}
%     \item \textbf{Music (32.1\%)}: Includes various instruments (guitar, piano, violin, tabla) with distinct timbre and visual appearances, providing diverse audio-visual correlation patterns.
    
%     \item \textbf{Device (32.2\%)}: Encompasses mechanical sounds (helicopter, keyboard, car) with characteristic audio signatures and varying visual scales.
    
%     \item \textbf{Animal (23.2\%)}: Features natural sounds (dog, lion, bird) with unique audio-visual correspondence patterns and challenging temporal dynamics.
    
%     \item \textbf{Human (12.5\%)}: Contains speech and other human-generated sounds, presenting complex scenarios with both audio and visual variations.
% \end{itemize}

\begin{figure}[t]
    \centering
    \includegraphics[width=\linewidth]{benchmark_ms3.pdf}
    \caption{Distribution of multi-source audio category combinations in our dataset. The horizontal bars show the frequency of different category pairs including individual categories (music, human, animal, device) and their combinations (e.g., human\_music, animal\_human).}
    \label{fig:cross_category}
\end{figure}

\noindent\textbf{MS3 Cross-Category Analysis:} For multi-source scenarios (Figure~\ref{fig:cross_category}), we analyze the distribution of audio category combinations in our dataset. Music and human categories have the highest individual counts, followed by animal and device categories. Among cross-category combinations, human\_music shows the highest frequency, followed by animal\_human. Other combinations such as device\_human and animal\_device occur less frequently in our dataset.





\subsection*{C. Failure Case Analysis}

Our method inherits certain limitations from the base segmentation model. Specifically, in cases where the underlying model fails to correctly segment the target object, our approach will also produce incorrect results. This cascading failure occurs because our method builds upon and depends on the initial segmentation output.

For example, when the base model misidentifies object boundaries or fails to detect the target object entirely, our method cannot compensate for these fundamental segmentation errors.

Future work could explore ways to make our approach more robust to initial segmentation errors, possibly through the incorporation of additional verification mechanisms or multi-model ensemble approaches.

% \subsection*{E. Computational Analysis}
% \subsubsection*{E.1 Runtime Performance}
% \subsubsection*{E.2 Model Complexity}











\subsection*{D. AVSegFormer Implementation Details}
\subsubsection*{D.1 Preliminary: AVSegFormer Architecture}
\label{subsec:avSegformer}

AVSegFormer~\cite{gao2024avsegformer} introduces several key architectural innovations while maintaining some fundamental components from AVSBench~\cite{zhou2022audio}. As illustrated in Figure~\ref{fig:avsegformer_arch}, the framework consists of five main components: audio-visual backbone encoders, a query generator that produces audio-conditioned queries, a transformer encoder for multi-scale feature processing, an audio-visual mixer for cross-modal feature fusion, and a transformer decoder for final mask generation. The query-based design enables the model to adaptively focus on audio-relevant regions in the visual frame. Here we detail its implementation.


\noindent\textbf{Encoder:} The encoding pathway remains consistent with AVSBench, employing VGGish~\cite{hershey2017vggish} to generate audio features $\mathcal{F}_A \in \mathbb{R}^d$ ($d = 128$). For visual processing, we utilizes Pyramid Vision Transformer~\cite{wang2022pvt} to extract hierarchical features $\mathcal{F}_{V_i}$, where $i \in [1,4]$ denotes the multi-resolution stage.

\noindent\textbf{Query Generator:} A key innovation in AVSegFormer~\cite{gao2024avsegformer} is its query-based architecture. The generator processes:
\begin{itemize}
    \item Initial query: $\mathcal{Q}_{\text{init}} \in \mathbb{R}^{T\times N_{\text{query}}\times D}$
    \item Audio feature: $\mathcal{F}_{\text{audio}} \in \mathbb{R}^{T\times D}$
    \item Learnable query: $\mathcal{Q}_{\text{learn}}$
\end{itemize}
Through cross-attention mechanisms, these components are transformed into mixed queries $\mathcal{Q}_{\text{mixed}}$, which help the model adaptively focus on audio-relevant regions in the visual frame. The inclusion of learnable queries enhances the model's capability to handle various audio-visual scenarios and dataset-specific characteristics.


\noindent\textbf{Transformer Encoder:} The transformer encoder processes visual features at three different resolutions (1/8, 1/16, and 1/32 of the original size). These multi-scale features are first flattened and concatenated to form a unified representation. The concatenated features are then processed through transformer layers, after which they are reshaped back to their original spatial dimensions. The 1/8-scale features are specifically upsampled by a factor of 2 and combined with the 1/4-scale features from the visual backbone through addition, producing the final mask features $\mathcal{F}_{\text{mask}}$ at 1/4 resolution. This multi-scale processing ensures the model captures both fine details and broader contextual information.

\begin{figure}[t]
    \centering
    \includegraphics[width=0.5\textwidth]{avsegformer_arch.pdf}
    \caption{Overview of AVSegformer architecture. The framework processes audio and visual inputs through parallel backbone networks, generates audio-conditioned queries, and employs transformer-based encoder-decoder architecture with a specialized mixer for audio-visual feature fusion.}
    \label{fig:avsegformer_arch}
\end{figure}

\noindent\textbf{Audio-Visual Mixer:} The mixer implements channel attention mechanism through:
\begin{equation}
    \omega = \text{softmax}(\frac{\mathcal{F}_{\text{audio}}\mathcal{F}_{\text{mask}}^T}{\sqrt{D/n_{\text{head}}}})
\end{equation}
\begin{equation}
    \hat{\mathcal{F}}_{\text{mask}} = \mathcal{F}_{\text{mask}} + \mathcal{F}_{\text{mask}} \odot \omega
\end{equation}
where $n_{\text{head}}=8$ and $\odot$ represents element-wise multiplication.

\noindent\textbf{Transformer Decoder:} The decoder utilizes the mixed query $\mathcal{Q}_{\text{mixed}}$ as input and processes multi-scale visual features as key/value pairs. Through the decoding process, output queries $\mathcal{Q}_{\text{output}}$ continuously aggregate visual features and combine with audio information. The final mask is generated through:
\begin{equation}
    \mathcal{M} = \text{FC}(\hat{\mathcal{F}}_{\text{mask}} + \text{MLP}(\hat{\mathcal{F}}_{\text{mask}} \cdot \mathcal{Q}_{\text{output}}))
\end{equation}
where MLP and FC layers integrate different channels to produce the final segmentation prediction.

\noindent\textbf{Loss Functions:} The original AVSegFormer~\cite{gao2024avsegformer} employs two complementary loss terms for training:
\begin{itemize}
\item $\mathcal{L}_{\text{IoU}}$ is a Dice loss comparing predicted segmentation masks with ground truth masks. This loss is particularly effective for handling the class imbalance inherent in AVS tasks where sounding objects often occupy small portions of the frame.
\item $\mathcal{L}_{\text{mix}}$ supervises the audio-visual mixer by computing Dice loss between a predicted binary mask (aggregated from mixed features) and combined foreground labels. This loss enhances the model's ability to handle complex scenes with multiple sound sources.
\end{itemize}


\subsubsection*{D.2 Learning with Balanced Audio-Visual Pairs, Classifier-Guided Feature Alignment}

As illustrated in Figure~\ref{fig:framework}, our approach enhances existing AVS architectures with three key components to address the visual bias problem. Given video frames and audio inputs, we first construct balanced positive and negative audio-visual pairs, where positive pairs represent aligned sound sources while negative pairs correspond to scenarios like silence or off-screen sounds. The model processes these pairs through separate visual and audio encoders to extract modality-specific features. These features undergo similarity-based alignment, optimized through classifier guidance in a contrastive manner. Finally, the aligned features are through the Query Generator module and the Transformer-based Head part before generating the final segmentation masks.

The framework is designed to maintain high segmentation accuracy for positive pairs while effectively suppressing predictions for negative pairs. Positive pairs are trained to maximize feature similarity, while negative pairs minimize it. By processing both types of pairs through this pipeline, the model learns to distinguish between valid and invalid audio-visual correspondences, enabling more robust segmentation in complex real-world scenarios.

\begin{figure}[t]
    \centering
    \includegraphics[width=\linewidth]{avsegformer_arch_overview.pdf}
    \caption{Overview of our robust AVS framework based on AVSegformer~\cite{gao2024avsegformer}. The model processes both positive and negative audio-visual pairs through separate encoders, employs classifier-guided similarity learning for feature alignment, and integrates the features for mask prediction. Positive pairs maximize similarity while negative pairs minimize it, enabling effective discrimination between valid and invalid audio-visual correspondences.}
    \label{fig:framework}
\end{figure}


Following Section~\ref{sec:method} of the main paper, we implement balanced training by maintaining a 10\% ratio of negative pairs during training. For classifier-guided similarity learning, we compute cosine similarity between aligned audio and visual features:
\begin{equation}
    s(\mathcal{F}_A, \mathcal{F}_V) = \text{cos}(\hat{\mathcal{F}}_A, \hat{\mathcal{F}}_V)
\end{equation}

The binary cross-entropy loss guides similarity learning:
\begin{equation}
\begin{split}
   \mathcal{L}_{\text{BCE}} = -\frac{1}{|\mathcal{P}| + |\mathcal{N}|} & \sum_{j=1}^{|\mathcal{P}| + |\mathcal{N}|}  \left( y_j \log \sigma(s_j) \right. \\
   & \left. + (1 - y_j) \log (1 - \sigma(s_j)) \right),
\end{split}
\end{equation}

The final training objective combines three complementary components:
\begin{equation}
\mathcal{L}_{\text{total}} = \mathcal{L}_{\text{IoU}} + \lambda_1\mathcal{L}_{\text{mix}} + \lambda_{\text{BCE}}\mathcal{L}_{\text{BCE}}
\end{equation}
where $\mathcal{L}_{\text{BCE}}$ guides the similarity learning between audio and visual features, helping the model distinguish between valid and invalid audio-visual correspondences through contrastive learning. The weighting factors $\lambda_1=0.1$ and $\lambda_{\text{BCE}}=1.0$ balance the contributions of each loss component.

\subsubsection*{D.3 Training Details}
Following AVSegFormer~\cite{gao2024avsegformer}'s setting, we adjust the input resolution to $512 \times 512$ to better capture detailed visual information. This model's training employs the AdamW optimizer~\cite{loshchilov2017adamw}, with a batch size of 1 and an initial learning rate of $2 \times 10^{-5}$. The encoder and decoder consist of 6 layers each, with an embedding size of 256. The training protocol extends to 60 epochs for the MS3 dataset and 30 epochs for the S4 dataset, conducted on an NVIDIA RTX A6000 GPU.



\end{document}
