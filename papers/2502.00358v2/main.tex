% CVPR 2025 Paper Template; see https://github.com/cvpr-org/author-kit

\documentclass[10pt,twocolumn,letterpaper]{article}

%%%%%%%%% PAPER TYPE  - PLEASE UPDATE FOR FINAL VERSION
% \usepackage{cvpr}              % To produce the CAMERA-READY version
% \usepackage[review]{cvpr}      % To produce the REVIEW version
\usepackage{multirow}
\usepackage{graphicx} 
\usepackage{float} 
\usepackage{array}
\usepackage{svg}
\usepackage{subcaption}
\usepackage{colortbl} 
% \usepackage{xcolor} 
% \usepackage{parskip}
\usepackage{amssymb}
\usepackage{pifont}
\usepackage{fontawesome}
\usepackage{siunitx}


\definecolor{lightblue}{RGB}{197, 234, 228} 
\newcommand{\cmark}{\textcolor{green}{\ding{51}}} % Green check mark
\newcommand{\xmark}{\textcolor{red}{\ding{55}}}   % Red cross mark


\usepackage[pagenumbers]{cvpr} % To force page numbers, e.g. for an arXiv version

% Import additional packages in the preamble file, before hyperref
\newcommand{\CG}{\mathcal{G}\xspace}
\newcommand{\CV}{\mathcal{V}\xspace}
\newcommand{\CE}{\mathcal{E}\xspace}
\newcommand{\CA}{\mathcal{A}\xspace}
\newcommand{\CF}{\mathcal{F}\xspace}
\newcommand{\CR}{\mathcal{R}\xspace}
\newcommand{\CB}{\mathcal{B}\xspace}
\newcommand{\CX}{\mathcal{X}\xspace}
\newcommand{\CK}{\mathcal{K}\xspace}
\newcommand{\CM}{\mathcal{M}\xspace}
\newcommand{\CC}{\mathcal{C}\xspace}
\newcommand{\CL}{\mathcal{L}\xspace}
\newcommand{\CI}{\mathcal{I}\xspace}
\newcommand{\CQ}{\mathcal{Q}\xspace}
\newcommand{\CO}{\mathcal{O}\xspace}
\newcommand{\CP}{\mathcal{P}\xspace}
\newcommand{\CS}{\mathcal{S}\xspace}
\newcommand{\CT}{\mathcal{T}\xspace}
\newcommand{\CJ}{\mathcal{J}\xspace}
\usepackage[para]{footmisc}
\usepackage{subfig}
% \usepackage{subcaption}
% \usepackage{array}
% \usepackage{colortbl}



% It is strongly recommended to use hyperref, especially for the review version.
% hyperref with option pagebackref eases the reviewers' job.
% Please disable hyperref *only* if you encounter grave issues, 
% e.g. with the file validation for the camera-ready version.
%
% If you comment hyperref and then uncomment it, you should delete *.aux before re-running LaTeX.
% (Or just hit 'q' on the first LaTeX run, let it finish, and you should be clear).
\definecolor{cvprblue}{rgb}{0.21,0.49,0.74}
\usepackage[pagebackref,breaklinks,colorlinks,allcolors=cvprblue]{hyperref}

\newcommand{\yapeng}[1]{\textcolor{red}{[Yapeng: #1]}}
\newcommand{\wj}[1]{\textcolor{blue}{[Wenjie: #1]}}
\newcommand{\yunhui}[1]{\textcolor{green}{[Yunhui: #1]}}
\newcommand{\ziru}[1]{\textcolor{purple}{[Ziru: #1]}}
\newcommand{\jia}[1]{\textcolor{magenta}{[Jia: #1]}}

%%%%%%%%% PAPER ID  - PLEASE UPDATE
\def\paperID{11939} % *** Enter the Paper ID here
\def\confName{CVPR}
\def\confYear{2025}

%%%%%%%%% TITLE - PLEASE UPDATE
\title{Do Audio-Visual Segmentation Models Truly Segment Sounding Objects?}

%%%%%%%%% AUTHORS - PLEASE UPDATE
\author{
Jia Li$^{1}$ \hspace{1cm} Wenjie Zhao $^{1}$ \hspace{1cm} Ziru Huang$^{2}$ \hspace{1cm} Yunhui Guo$^{1}$ \hspace{1cm} Yapeng Tian$^{1}$ \\
$^{1}$ The University of Texas at Dallas \hspace{1cm}
$^{2}$ Tsinghua University\\
% Richardson, Texas, U.S.\\
$^{1}${\tt\small  \{Jia.Li, Wenjie.Zhao,  Yunhui.Guo, Yapeng Tian\}@utdallas.edu} $^{2}$ {\tt\small huangzr21@mails.tsinghua.edu.cn}
}
% \and
% Wenjie Zhao\\
% The University of Texas at Dallas\\
% Richardson, Texas, U.S.\\
% {\tt\small Wenjie.Zhao@utdallas.edu}
% \and
% Ziru Huang\\
% Tsinghua University\\
% Beijing, China\\
% {\tt\small huangzr21@mails.tsinghua.edu.cn}
% \and
% Yunhui Guo\\
% The University of Texas at Dallas\\
% Richardson, Texas, U.S.\\
% {\tt\small Yunhui.Guo@utdallas.edu}
% \and
% Yapeng Tian\\
% The University of Texas at Dallas\\
% Richardson, Texas, U.S.\\
% {\tt\small Yapeng.Tian@utdallas.edu}
% }



\begin{document}
\maketitle
% \begin{abstract}

% Recent works to jointly reconstruct 3D human and object from a single RGB image, are mostly model-based, that fail to capture the fine details of the clothed human body and object surface. In this paper, we introduce ReCHOR, a novel, model-free, first-method to produce realistic clothed human-object reconstructions from a monocular view. This is extremely challenging due to human-object occlusions, diverse interactions and depth ambiguity, as it needs to infer both 3D spatial awareness and high resolution details. Our core idea is based on estimating neural implicit representations for human and object respectively by an attention-based neural implicit model that attends to pixel-aligned features from both the global human-object image for spatial awareness and  the local separate view of human and object images for high quality details. Additionally, the network is conditioned on semantic features from an initial estimated human-object pose prior and a generative diffusion model that inpaints occluded regions, thus enabling the retrieval of details from them.
% We also propose a synthetic dataset with rendered scenes of diverse, inter-occluded 3D human and object scans, to train our network. We evaluate our method on the synthetic and real world BEHAVE dataset. Our experiments show that our method outperforms the SOTA in achieving realistic clothed human-object reconstructions.
Recent approaches to jointly reconstruct 3D humans and objects from a single RGB image represent 3D shapes with template-based or coarse models, which fail to capture details of loose clothing on human bodies. In this paper, we introduce a novel implicit approach for jointly reconstructing realistic 3D clothed humans and objects from a monocular view. For the first time, we model both the human and the object with an implicit representation, allowing to capture more realistic details such as clothing. This task is extremely challenging due to human-object occlusions and the lack of 3D information in 2D images, often leading to poor detail reconstruction and depth ambiguity. To address these problems, we propose a novel attention-based neural implicit model that leverages image pixel alignment from both the input human-object image for a global understanding of the human-object scene and from local separate views of the human and object images to improve realism with, for example, clothing details. Additionally, the network is conditioned on semantic features derived from an estimated human-object pose prior, which provides 3D spatial information about the shared space of humans and objects. To handle human occlusion caused by objects, we use a generative diffusion model that inpaints the occluded regions, recovering otherwise lost details. For training and evaluation, we introduce a synthetic dataset featuring rendered scenes of inter-occluded 3D human scans and diverse objects. Extensive evaluation on both synthetic and real-world datasets demonstrates the superior quality of the proposed human-object reconstructions over competitive methods.
\end{abstract}  
% \section{Introduction}
\label{sec:intro}
% Image editing methods in diffusion models depend on user-defined control directions - users can unlock their creativity using these methods by specifying the desired manipulation through prompts~\cite{gandikota2023concept}, reference images~\cite{ruiz2022dreambooth, kumari2022customdiffusion, gal2022image, chen2024trainingfreeregionalpromptingdiffusion}, or attribute vectors~\cite{parmar2023zero,hertz2022prompt}. In this work, we ask a fundamentally different question: \emph{Can we automatically discover the underlying visual structure of a concept within diffusion model's knowledge?} %Rather than requiring user-specified controls, we aim to decompose the model's internal knowledge into meaningful directions.

% This question touches on a fundamental limitation in how we interact with diffusion models. Current control methods ~\cite{zhang2023addingconditionalcontroltexttoimage, gandikota2023concept, ye2023ipadaptertextcompatibleimage,ye2023ipadaptertextcompatibleimage, hertz2024stylealignedimagegeneration, li2023photomaker, shi2024instantbooth, chen2024trainingfreeregionalpromptingdiffusion} require users to specify their desired manipulations in advance, limiting interactive creativity. This contrasts with natural human artistic workflows, where creators dynamically explore creative ideas while jointly refining them toward meaningful artistic outcomes~\cite{hoffmann2016modeling}. This synergy between specification and exploration is not new to generative models. Early GAN architectures naturally developed disentangled latent spaces that enabled continuous\cite{harkonen2020ganspace,radford2015unsupervised, wu2021stylespace, shen2020interfacegan}, compositional control over generated images. Users could explore these spaces to discover interesting variations that would be difficult to describe in words~\cite{wu2021stylespace}, then combine them to achieve their creative goals~\cite{grabe2022towards}. 


% While diffusion models have largely superseded GANs in conditional image synthesis~\cite{dhariwal2021diffusion},  their underlying structure remains less understood. Diffusion models achieve remarkable diversity through high-dimensional latents, unlike GANs' compact latent spaces.  With a single prompt, diffusion models can generate radically different variations through different random initializations of input noise. We ask - Is it possible to discover interpretable structure within this vast space of variations?

Text-to-image diffusion models are capable of generating remarkable visual variations from a single prompt through different random initializations. However, this vast creative potential remains largely opaque to users---while we can generate diverse images, we lack understanding of the underlying structure of these variations. This presents a fundamental challenge: how can we discover and expose the latent visual capabilities encoded within these models?

\let\thefootnote\relax \footnote{$^{*}$Correspondence to \texttt{gandikota.ro@northeastern.edu}}

The challenge touches on a key limitation in how we interact with diffusion models today. Current control methods require users to explicitly specify their desired edits in advance through prompts~\cite{gandikota2023concept}, reference images~\cite{zhang2023addingconditionalcontroltexttoimage, chen2024trainingfreeregionalpromptingdiffusion, ruiz2022dreambooth,kumari2022customdiffusion, Ryu_lora, hu2021lora}, or attribute vectors~\cite{ye2023ipadaptertextcompatibleimage, hertz2024stylealignedimagegeneration, li2023photomaker, shi2024instantbooth,parmar2023zero,hertz2022prompt}. That contrasts sharply with natural human creative workflows, where artists dynamically explore creative ideas and jointly refine them toward meaningful artistic outcomes~\cite{hoffmann2016modeling}. The need for pre-specified controls creates a barrier between users and the full creative potential of these models.

Interestingly, earlier generative models like GANs~\cite{gans,karras2019style,brock2018large} naturally developed more interpretable internal structures. Their compact latent spaces often exhibited emergent disentanglement~\cite{harkonen2020ganspace,radford2015unsupervised, wu2021stylespace, shen2020interfacegan}, enabling continuous and compositional control over generated images. Users could explore these spaces to discover interesting variations that would be difficult to describe in words~\cite{wu2021stylespace}, then combine them to achieve their creative goals~\cite{grabe2022towards}.

Diffusion models have largely superseded GANs in conditional image synthesis~\cite{dhariwal2021diffusion}, achieving greater diversity through much higher-dimensional latents. And yet an understanding of the underlying structure of these larger latent spaces has remained elusive. In this work, we ask a fundamental question: \emph{Can we automatically discover the visual structure within a diffusion model's knowledge of a concept?} Rather than requiring user-specified controls, we aim to decompose the model's internal representations into expressive directions that users can explore and combine.

To address these needs, we present \textbf{SliderSpace}, a framework that brings systematic explorability to diffusion models. Given just a text prompt, SliderSpace discovers a canonical set of meaningful, diverse, and controllable directions within the model's knowledge of that concept. Each direction is implemented as a low-rank adapter~\cite{hu2021lora} that can be scaled and composed with others, allowing users to explore and smoothly combine different aspects of variation, as shown in Figure~\ref{fig:intro}.

We ground SliderSpace discovery in three key requirements for meaningful decomposition of a diffusion model's visual manifold: 
\begin{enumerate}
    \item \textbf{Unsupervised Discovery:} The decomposition process should emerge from the intrinsic structure of the model's learned representation, rather than being guided by predefined attributes. This ensures we capture the true topology of the model's knowledge space rather than projecting our assumptions onto it.
    
    \item \textbf{Semantic Orthogonality:} Each discovered control must represent a distinct semantic direction. This is enforced in a semantic feature space, like CLIP, where every slider has an orthogonal effect in embeddings. This prevents discovering multiple controls that create similar semantic effects, making the system more efficient and easier.
    
    \item \textbf{Distribution Consistency:} Directions must induce consistent transformations across both random seeds and prompt variations. 
\end{enumerate}

These requirements naturally lead to our proposed framework, which we formalize in Section~\ref{sec:method}. As we show in our experiments, SliderSpace is architecture-agnostic, working with both conventional U-Net based models like Stable Diffusion~\cite{rombach2022high, rombach2022sd20, podell2023sdxl, turbo, dmd} and recent transformer-based architectures like Flux~\cite{flux}.

We demonstrate the expressiveness of SliderSpace through three applications: First, we show how SliderSpace can decompose high-level concepts into diverse and expressive components, revealing the natural axes of variation in the model's understanding. Second, we explore artistic style variation, where SliderSpace discovers directions that match or exceed the diversity of manually curated artist lists while being judged more useful by human evaluators. Finally, we show how SliderSpace can help reverse the mode collapse commonly observed in distilled diffusion models, restoring diversity while maintaining generation speed.

Beyond providing practical creative control, SliderSpace opens new avenues for understanding and utilizing the latent capabilities of diffusion models. By mapping these models' visual potential into intuitive, composable directions, we take a step toward making their creative possibilities more accessible and interpretable to users.

% Image editing methods in diffusion models unlock the creativity of users. In this work we ask an alternate question: \emph{Can we organize and expose what of the diffusion model is already capable of?}.
% Existing methods for controlling image generation typically require users to manually specify edit directions for desired changes. This process is time-consuming, requires technical expertise, and limits the spontaneity of the creative process. For instance, if a user wants to adjust the smile of a generated person, they must explicitly request this edit, often through imprecise prompt engineering or model fine-tuning. This approach of predefined controls or manual specifications restricts users from fully exploring the latent capabilities of the model. There may be interesting stylistic variations or attributes that the model can generate, but users have no easy way to discover or utilize these.

% Natural visual disentanglement was an emergent property in the latent space of Generative Adversarial Models (GANs) \cite{harkonen2020ganspace,radford2015unsupervised, wu2021stylespace, shen2020interfacegan}. In particular, it has been observed that StyleGAN~\cite{karras2019style} stylespace neurons offer detailed control over many meaningful aspects of images that would be difficult to describe in words~\cite{wu2021stylespace}. However, diffusion models do not share such a compact latent space~\cite{park2023unsupervised}; and efforts to uncover such a space in the semantic embeddings of the text conditioning have met with limited success \nik{Nick - is there a specific citation you were thinking about?}.

% In this work we introduce \textbf{SliderSpace}, which takes a step towards uncovering an analogous low dimensional representation of diffusion models' visual breadth; in essence treating the diffusion model as many generators sharing parameters, where a particular generator is defined by a specific prompt. For a given prompt we sample many random seeds (and optionally prompt expansions using an LLM), generate the corresponding images, and apply an off the shelf feature extractor (in this work CLIP, but our method can be applied to any differentiable feature extractor). We use PCA to analyze these features, and for each of the leading $k$ principal components we train a LoRA \cite{} which causes the diffusion model to produces images which increase the feature magnitude along that component when passed back through the same feature extractor. This leads to a 'Slider' for each principal component, because each LoRA can be scaled and applied to the original diffusion model, continuously varying those visual features in the generated results (as measured, in our case, by CLIP).

% There are many other works that enhance the controllability of diffusion models. One common approach is enabling users to add spatial constraints to a generation either manually, or via a reference image \cite{zhang2023addingconditionalcontroltexttoimage, chen2024trainingfreeregionalpromptingdiffusion}, a second is leveraging more abstract embeddings (e.g. identity, style) extracted from a reference image \cite{ye2023ipadaptertextcompatibleimage, hertz2024stylealignedimagegeneration, li2023photomaker, shi2024instantbooth}, a third is finetuning a foundation model to better generate a concept important to the user \cite{ruiz2022dreambooth, kumari2022customdiffusion, Ryu_lora, hu2021lora}, and a fourth (most relevant to this work) is finding low-rank adaptors of the model based on a prompt or small training set which can be scaled to provide continous control over one aspect of generated image (e.g. night vs day, basic vs luxury, etc.) \cite{gandikota2023concept}. SliderSpace is complementary to all of these methods and offers something distinct. All of the other methods we are aware require the user (and / or model designer) to know in advance what type of control they want. In contrast SliderSpace assists users in discovering and controlling hidden capabilities present in the diffusion model's distribution of possible generations.

%We propose that truly intuitive creative control in a text-to-image model should meet three key criteria: \emph{discoverability}, \emph{intuitiveness}, and \emph{specificity}. The model should reveal controllable attributes that may not be immediately obvious, offer controls that are easy to understand and manipulate, and ensure each control affects a distinct attribute of the generated image.

% We demonstrate the utility and power of SliderSpace using three applications built on top of SDXL-DMD \cite{dmd}, because its fast generation speed lends itself well to the continuous control offered by SliderSpace.

% First, we study concept decomposition (Section \ref{sec:concept_exp}), where we learn sliders for a specific concept (e.g. 'monster', 'waterfall', 'car'). Through quantitative metrics of diversity and text alignment we demonstrate that the learned sliders dramatically boost the diversity of generations when randomly applied without harming text alignment; we also ask humans to qualitatively judge these results in a user study where they find the SliderSpace results to be more 'Diverse', 'Useful', and 'Creative' than our baselines.

% Second, we attempt to compare the automatic discoveries of SliderSpace to a large scale manual study of artistic styles (Section \ref{sec:art_exp}), open-sourced by ParrotZone \cite{parrotzone}. In this study SDXL was prompted with over 4300 artist names,  and based on visual inspection the cases of successful stylistic mimicry recorded. Quantitatively SliderSpace more closely matches the distribution of artistic variation discovered by ParrotZone than other baselines, and in our user studies was judged to be significantly more 'Diverse' and 'Useful' than the baselines. To our surprise humans even judged SliderSpace results to be slightly more 'Diverse' than the results generated by the manually discovered artist names of \cite{parrotzone}.

% Third, we attempt to use SliderSpace to reverse the mode collapse commonly observed in distilled few-step diffusion models relative to the original teacher model (Section \ref{sec:diverse_exp}). We quantitatively demonstrate that applying SliderSpace to SDXL-DMD leads to more closely matching the distribution of images by the original teacher, SDXL.

%Through extensive experiments on various state-of-the-art text-to-image models, we demonstrate that SliderSpace significantly enhances user control and creative expression in AI-assisted image generation tasks. Our method enables a range of applications, including concept decomposition and control, diversity improvement in generated images, customization dissection and edits, and the exploration of artistic styles inherent in the model.

% SliderSpace goes beyond providing a practical tool for enhanced creative control. By mapping the visual potential of diffusion models it can open new avenues for generative creativity and deepens our understanding of each model's hidden potential.
% \section{Related Work}

\paragraph{LLMs for Agent tasks.}

Our research is related to deploying large language models (LLMs) as agents for decision-making tasks in interactive environments~\citep{liu2023agentbench,zhou2023webarena,shridhar2020alfred,toyama2021androidenv}. Earlier works, such as~\citep{yao2023webshopscalablerealworldweb}, fine-tuned models like BERT~\citep{devlin2019bertpretrainingdeepbidirectional} for decision-making in simplified environments, such as online shopping or mobile phone manipulation. With the advent of large language models~\citep{brown2020languagemodelsfewshotlearners,openai2024gpt4technicalreport}, it became feasible to perform decision-making tasks through zero-shot or few-shot in-context learning. To better assess the capabilities of LLMs as agents, several models have been developed~\citep{deng2024mind2web,xiong2024watch,hong2023cogagent,yan2023gpt}. Most approaches~\citep{zheng2024seeact,deng2024mind2web} provide the agent with observation and action history, and the language model predicts the next action via in-context learning. Additionally, some methods~\citep{zhang2023building,li2023camel,song2024trial} attempt to distill trajectories from state-of-the-art language models to train more effective policy models. In contrast, our paper introduces a novel framework that automatically learns a reward model from LLM agent navigation, using it to guide the agents in making more effective plans.

\textbf{LLM Planning.} Our paper is also related to planning with large language models. Early researchers~\citep{brown2020languagemodelsfewshotlearners} often prompted large language models to directly perform agent tasks. Later, \citet{yao2022react} proposed ReAct, which combined LLMs for action prediction with chain-of-thought prompting~\citep{wei2022chain}. Several other works~\citep{yao2023treethoughtsdeliberateproblem,hao2023reasoning,zhao2023large,qiao2024agentplanningworldknowledge} have focused on enhancing multi-step reasoning capabilities by integrating LLMs with tree search methods. Our model differs from these previous studies in several significant ways. First, rather than solely focusing on text generation tasks, our pipeline addresses multi-step action planning tasks in interactive environments, where we must consider not only historical input but also multimodal feedback from the environment. Additionally, our pipeline involves automatic learning of the reward model from the environment without relying on human-annotated data, whereas previous works rely on prompting-based frameworks that require large commercial LLMs like GPT-4~\citep{openai2024gpt4technicalreport} to learn action prediction. Furthermore, \Model supports a variety of planning algorithms beyond tree search.

\textbf{Learning from AI Feedback.} In contrast to prior work on LLM planning, our approach also draws on recent advances in learning from AI feedback~\citep{bai2022constitutional,lee2023rlaif,yuan2024self,sharma2024critical,pan2024autonomous,koh2024tree}. These studies initially prompt state-of-the-art large language models to generate text responses that adhere to predefined principles and then potentially fine-tune the LLMs with reinforcement learning. Like previous studies, we also prompt large language models to generate synthetic data. However, unlike them, we focus not on fine-tuning a better generative model but on developing a classification model that evaluates how well action trajectories fulfill the intended instructions. This approach is simpler, requires no reliance on state-of-the-art LLMs, and is more efficient. We also demonstrate that our learned reward model can integrate with various LLMs and planning algorithms, consistently improving their performance.

\textbf{Inference-Time Scaling.} ~\citet{snell2024scaling} validates the efficacy of inference-time scaling for language models. Based on inference-time scaling, various methods have been proposed, such as random sampling~\citep{wang2022self} and tree-search methods~\citep{hao2023reasoning, zhang2024accessing, guan2025rstar}. Concurrently, several works have also leveraged inference-time scaling to improve the performance of agentic tasks. ~\citet{koh2024tree} adopts a training-free approach, employing MCTS to enhance policy model performance during inference and prompting the LLM to return the reward. ~\citet{gu2024your} introduces a novel speculative reasoning approach to bypass irreversible actions by leveraging LLMs or VLMs. It also employs tree search to improve performance and prompts an LLM to output rewards. ~\citet{yu2024exact} proposes Reflective-MCTS to perform tree search and fine-tune the GPT model, leading to improvements in ~\citet{koh2024visualwebarena}. ~\citet{putta2024agent} also utilizes MCTS to enhance performance on web-based tasks such as ~\citet{yao2023webshopscalablerealworldweb} and real-world booking environments. ~\cite{lin2025qlass} utilizes the stepwise reward to give effective intermediate guidance across different agentic tasks. Our work differs from previous efforts in two key aspects: (1) Broader Application Domain. Unlike prior studies that primarily focus on tasks from a single domain, our method demonstrates strong generalizability across web agents, mathematical reasoning, and scientific discovery domains, further proving its effectiveness. (2) Flexible and Effective Reward Modeling. Instead of simply prompting an LLM as a reward model, we finetune a small scale VLM~\citep{lin2023vila} to evaluate input trajectories. %Our reward scores range continuously between 0 and 1, in contrast to existing methods that rely on discrete scoring (e.g., 0 and 1, or 0, 0.5, and 1) through direct LLM prompting.

% Concurrently, several works have also leveraged inference-time scaling to improve the performance of agentic tasks. ~\citet{pan2024autonomous} demonstrates that LLMs and VLMs, such as the GPT series, can function as evaluators or reward models to provide guidance for fine-tuning or reflection, thereby enhancing digital agents. This lays the groundwork for subsequent studies that directly prompt LLMs as reward models. ~\citet{koh2024tree} adopts a training-free approach, employing MCTS to enhance policy model performance during inference. However, it is limited to web environments~\citep{koh2024visualwebarena}. Moreover, its value function relies on prompting an LLM, which is less effective than our proposed method. We validate our approach through ablation studies, demonstrating that our fine-tuned reward model is more effective. ~\citet{gu2024your} introduces a novel speculative reasoning approach to bypass irreversible actions, such as purchasing a product, by leveraging LLMs or VLMs. It also employs tree search to improve performance, but it remains restricted to the web domain~\citep{koh2024visualwebarena, deng2024mind2web}. Additionally, it lacks reward modeling and instead prompts an LLM to output rewards. ~\citet{yu2024exact} proposes Reflective-MCTS to perform tree search and fine-tune the GPT model, leading to improvements in ~\citep{koh2024visualwebarena}. However, this work focuses solely on a single web agent task, and its reward modeling is derived from multi-agent debate, differing from our more effective and efficient reward modeling approach. ~\citet{putta2024agent} also utilizes MCTS to enhance performance, but it is limited to web-based tasks such as ~\citep{yao2023webshopscalablerealworldweb} and real-world booking environments. 
% \input{sec/3_Problem_Definition}
% \section{MDGD: Modality-Decoupled Gradient Regularization and Descent}
Motivated by the visual forgetting problem caused by the degradation of multimodal encoding in Eq.~\eqref{eq:erank_degradation}, we introduce a modality-decoupling gradient regularization (\textbf{MDGD}) to approximate orthogonal gradients between visual understanding drift and downstream task optimization. Specifically, leveraging modality-decoupled gradients $\Bar{g}_\theta$ and $\Bar{g}_\phi$ derived from the current MLLM and a pre-trained MLLM respectively, we propose a gradient regularization term $\Tilde{g}_\theta$ for more efficient multimodal instruction tuning, which promotes the alignment of downstream tasks while mitigating visual forgetting \cite{zhu2024model}. Since MDGD requires the estimation of parameter gradients, we could not directly apply parameter-efficient fine-tuning methods (\emph{e.g.}, LoRA \cite{hu2021lora}). Thus, we alternatively formulate the regularization as a gradient mask $M_{\Tilde{g}_\theta}$, which allows efficient fine-tuning only on a subset of masked model parameters.

\subsection{Modality Decoupling}
Based on the information bottleneck objective in Eq.~\eqref{eq:ib_vision}, the objective encourages the model to maximize $I(y; Z)$ while compressing $I(X^v; Z)$ \cite{tishby2000information, alemi2016deep}. 
In practice, this compression may discard useful visual details, leading to visual forgetting. To mitigate such compression and preserve the pre-trained visual knowledge, we follow the KL divergence loss
$D_{\text{KL}}\Bigl(\mu_\phi(X^v) \,\Big\|\, \pi_\theta (X^v)\Bigr)$
to constrain the current model’s visual representation $\pi_\theta(X^v)$ to remain close to the pre-trained distribution $\mu_\phi(X^v)$, 
thereby preserving the mutual information $I(X^v; Z)$ that would otherwise be reduced by the compression \cite{hinton2015distilling, lopez2018information}. 
However, since MLLMs cannot directly track the distributions of image tokens, we instead introduce an auxiliary loss function
\begin{equation}\label{eq:visual_loss}
    \mathcal{L}_v(\phi,\theta) = \|\mu(X^v|\phi) - \pi(X^v|\theta)\|_1,
\end{equation}
which approximates the KL divergence loss \cite{zhu2022wdibs,zhu2017unpaired} by penalizing discrepancies between the pre-trained visual representation and that obtained during instruction tuning. 

In the MLLM instruction tuning, the visual output tokens (e.g., $\{z^{vl}_k\}_{k=1}^M$) are encoded as latent representations. 
Such visual encoding cannot be directly supervised by any learning objective but is learned through textual gradient propagation of the negative log-likelihood loss in downstream tasks. 
To approximate the visual optimization direction, we derive the gradients of $\mathcal{L}_v(\phi,\theta)$ for both the pre-trained MLLM $\pi_\phi$ and the current MLLM $\pi_\theta$:
\begin{align*}
    h_{\phi} &= \nabla_{\phi}\mathcal{L}_v(\phi) = \boldsymbol{\lambda}(\phi,\theta) \cdot \nabla_\phi \mu(X^v|\phi), \\
    h_{\theta} &= \nabla_{\theta}\mathcal{L}_v(\theta) = -\boldsymbol{\lambda}(\phi,\theta) \cdot \nabla_\theta \pi(X^v|\theta),
\end{align*}
where $\boldsymbol{\lambda}(\phi,\theta) = \text{sign}\left( \mu(X^v|\phi) - \pi(X^v|\theta) \right)$.
Intuitively, when the MLLM's visual understanding drift causes visual forgetting, we further derive the orthogonal task gradients $\Bar{g}_\phi$ and $\Bar{g}_\theta$:
\begin{align}\label{eq:orth}
    \Bar{g}_\phi &= \nabla_{\phi}\mathcal{L}_{vl}(\phi) - \frac{\nabla_{\phi}\mathcal{L}_{vl}(\phi)^\top h_{\phi}}{\|h_{\phi}\|^2} \cdot h_{\phi}, \\
    \Bar{g}_\theta &= \nabla_{\theta}\mathcal{L}_{vl}(\theta) - \frac{\nabla_{\theta}\mathcal{L}_{vl}(\theta)^\top h_{\theta}}{\|h_{\theta}\|^2} \cdot h_{\theta},
\end{align}
which enables \textbf{modality decoupling} of the downstream task loss gradient in Eq.\eqref{eq:task_loss} orthogonal to the visual understanding drift
for the pretrained MLLM $\Bar{g}_{\phi} \perp h_{\phi}$ and current MLLM $\Bar{g}_{\theta} \perp h_{\theta}$.

\begin{figure}
\centering

\definecolor{mPurple}{rgb}{0.58,0,0.82}
\lstset{
  basicstyle=\footnotesize\fontfamily{ttfamily}\selectfont, % set the font
  keywordstyle={\color{mPurple}}, % set the keyword style
  morecomment=[l]{//},
  commentstyle=\rmfamily\slshape,
  % commentstyle=\itshape, % set the comment style
  showstringspaces=false, % don't show spaces in strings
  columns=fullflexible, % use proportional spacing
  morekeywords={fun,func,let,val,in,end,case,of,SOME, NONE, and, structure, if, else, then, return, def}, % define additional keywords
  mathescape=true, % enable math mode
  escapechar={@},
  keepspaces=true,
  breaklines=true,
  numbers=left,
  numbersep=4pt,
  xleftmargin=2em}

\begin{minipage}[t]{0.32\textwidth}
\begin{lstlisting}
$\Omega$: int
$\oracle{}$: circuit $\rightarrow$ circuit
$\cost{}$: circuit $\rightarrow$ int
compact: circuit $\rightarrow$ circuit

def $\mathsf{\algname{}}(C)$:
  $C'$ = segopt(compact($C$))
  if $C' = C$:
    return $C$
  else:
    return $\mathsf{\algname{}}(C')$
\end{lstlisting}
\end{minipage}
\begin{minipage}[t]{0.32\textwidth}
\begin{lstlisting}[firstnumber=12]
def segopt($C$):
  $d$ = length$(C)$
  if $d \leq 2\Omega$:
    $C'$ = $\oracle{}(C)$
    if $\costof{C'} < \costof{C}$:@\label{line:guard1start}@
      return $C'$
    else:
      return $C$@\label{line:guard1stop}@
  else:
    $m$ = $\lfloor d/2 \rfloor$
    $C_1$ = $C[0 : m]$
    $C_2$ = $C[m : d]$
    $C_1'$ = segopt($C_1$)
    $C_2'$ = segopt($C_2$)
    return meld$(C_1', C_2')$
\end{lstlisting}
\end{minipage}
\begin{minipage}[t]{0.345\textwidth}
\begin{lstlisting}[firstnumber=27]
def meld$(C_1, C_2)$:
  $d_1$ = length($C_1$)
  $d_2$ = length($C_2$)
  $W$ = $C_1 [d_1 - \Omega : d_1] + C_2 [0 : \Omega]$@\label{line:combine}@
  $W'$ = $\oracle{}(W)$@\label{line:guard2start}@
  if $\costof{W'} = \costof{W}$:
    return $(C_1 ; C_2)$@\label{line:guard2stop}@  // concatenate
  else:
    $M$ = meld$(C_1[0 : d_1 - \Omega], W')$@\label{line:mrec}@
    return meld($M, C_2[\Omega : d_2]$)
\end{lstlisting}
\end{minipage}

% $\Omega$: int
% Oracle: circuit $\rightarrow$ circuit
% cost: circuit $\rightarrow$ int

% def meld$(C_1, C_2)$:
%   $P$, $W_1$ = splitLastLayers$(C_1, \Omega)$
%   $W_2$, $S$ = splitFirstLayers$(C_2, \Omega)$
%   $W = W_1 + W_2$
%   $W'$ = Oracle$(W)$
%   if $\costof{W'} = \costof{W}$:
%     return concat$(P, W, S)$
%   else:
%     // assert ($\costof{W'} < \costof{W}$)
%     return meld(meld$(P, W')$, $S$)

% $c_1 + c_2$ = c $\mathit{where}\ \sizeof{c_1} = \sizeof{c}/2$
% $p_1 + s_1$ = $c_1$  $\mathit{where}\ \sizeof{s_1} = \Omega$
% $p_2 + s_2$ = $c_2$  $\mathit{where}\ \sizeof{p_2} = \Omega$
% $\textit{wd} \gets s_1 + p_2$
% $\textit{wd}' \gets \mathcal{O}(\textit{wd})$



% \begin{lstlisting}
% fun meld (c1, c2) =
%   let
%     val (c1s, c1p) = C.splitEnd (c1, $\Omega$)
%     val c2' = C.merge (c1p, c2)
%   in
%     case opt_prefix (2*$\Omega$, c2') of
%       SOME c2'' => meld (c1s, c2'')
%     | NONE => C.merge (c1, c2')
%   end

% and fun opt_prefix (prefix_size, c) =
%   let
%     val (cp, cs) = C.split (c, prefix_size)
%   in
%     case $\mathcal{O}$(cp) of
%       SOME cp' => SOME (meld (cp', cs))
%     | NONE => NONE
%   end
% \end{lstlisting}

\caption{
Algorithm \algname{} produces locally optimal circuits with
respect to a given $\oracle{}$, $\cost{}$, and segment length $\Omega$.
%
To achieve local optimality,
\algname{} only uses the oracle on small segments of length $2\Omega$.
%
The algorithm repeatedly optimizes and compacts the circuit
until convergence.
%
The function \textsf{segopt}$()$ implements our optimization algorithm
and uses $\mathsf{meld}()$ to efficiently produce \wopttext{} circuits.
}
% \vspace{-1.5in}
\label{fig:lopt-code}
\end{figure}


\section{Main Algorithm}
\label{sec:main_alg}
We present our main result in this section and explain the algorithm in a top-down manner.  The algorithm is based on the localization framework of  
\cite{FKT20}; see Algorithm~\ref{alg:loacalizatioin} in the Appendix for details. The main result is stated formally below:
\begin{theorem}
\label{thm:main_result}
Under Assumptions~\ref{assum:lispchitz_smooth} and \ref{assump:dia_dominant}, suppose $\beta\le\frac{G}{D}(\frac{\sqrt{n}\epsilon}{\sqrt{m}\log(nmd/\delta)}+\frac{\sqrt{d\log(1/\delta)\log(nmd)}}{\sqrt{m}\epsilon})$, $\epsilon\le O(1),n\ge \log^2(nd/\delta)/\epsilon$ and $ m\le n^{O(\log\log n)}$.
Setting $\eta\le\frac{D}{G}\cdot \min\{ \frac{B\sqrt{m}}{\sqrt{n}} ,   \frac{\sqrt{m}\epsilon}{\sqrt{d\log(1/\delta)\log( nmd)}}\}$, $B=100\log(mnd/\delta)/\epsilon$, $\tau=O(G\log(nmd)/\sqrt{m})$ and $\upsilon=0.9B+\frac{2\log(T/\delta)}{\epsilon}$, Algorithm~\ref{alg:loacalizatioin}  is $(\epsilon,\delta)$-user-level-DP. 
When the $nm$ functions in dataset $\calD$ are i.i.d. drawn from the underlying distribution $\calP$, it takes $mn$ gradient computations and outputs $x_S$ such that
    \begin{align*}
        \E[F(x_S)-F(x^*)]\le \Tilde{O} \left(\frac{d}{\sqrt{nm}}+\frac{d^{3/2}}{n\epsilon^2\sqrt{m}} \right).
    \end{align*}
\end{theorem}

We briefly describe the localization framework.  
In the first phase, it runs (non-private) SGD using half of the dataset, and averages the iterates to get $\bx_1$.
Roughly speaking, the solution $\bx_1$ already provides a good approximation with a small population loss when the datasets are  drawn i.i.d. from the underlying distribution. However, to ensure privacy, we require a  
sensitivity bound on $\|\bx_1\|$ and add noise $\zeta_1$ correspondingly to privatize $\bx_1$, yielding the private solution $x_1 \leftarrow \bx_1 + \zeta_1$.  

A naive bound on the excess loss due to the privatization is given by  
\[
\E[F(x_1) - F(\bx_1)] \leq G\|\zeta_1\|_2,
\]  
but the magnitude of the noise $\|\zeta_1\|_2$ is typically too large  
to achieve a good utility guarantee. Nevertheless, this process yields  
a much better initial point $x_1$ compared to the original starting  
point $x_0$. 
As a result, a smaller dataset and a smaller step size are sufficient  
to find the next good solution $\bx_2$ in expectation, with smaller noise $\|\zeta_2\|_2$ added to privatize $\bx_2$.  

This process is repeated over $O(\log n)$ phases, where each subsequent solution $\bx_S$ is progressively refined, and the Gaussian noise  
$\|\zeta_S\|_2$ becomes negligible. Ultimately, this iterative refinement  
balances privacy and utility, as established in Theorem~\ref{thm:main_result}.  
The formal argument about the utility guarantee and proof can be found in Lemma~\ref{lm:localization}.  

Our main contribution is in Algorithm~\ref{alg:dpsgd},  
which uses a novel gradient estimation sub-procedure.

% Given some initial point $x$, we define the Projected Gradient Descent sequences $\{x_{t}^Z\}_{t\in[m/K]}$, denoted by $\PGD(x,\eta,Z,K)$ for each user $Z$ with batch size $K$ as follows:
% \begin{align}
% \label{eq:PGD_each_user}
%     x_t^Z= \Pi_{\calX}(x^Z_{t-1}-\eta\frac{1}{K}\sum_{j\in[K]}\nabla f(x^Z_{t-1};z_{t,j})),
% \end{align}
% where $\{z_{t,j}\}_{j\in[K]}$ is a set of unused item functions of size $K$.
% This is simply running GD for each user for $m/K$ iterations, with batch size $K$ for each iteration.

\begin{algorithm2e}
\caption{SGD for User-level DP-SCO}
\label{alg:dpsgd}
\textbf{ Input:} dataset $\calD$, privacy parameters $\epsilon,\delta$, other parameters $\eta,\tau,\upsilon,B$, initial point $x_0$\;
%\textbf{ Process:} \\
Divide $\calD$ into {B} disjoint subsets of equal size, denoted by $\{\calD_i\}_{i\in[B]}$,
$\calD_i=\{Z_{i,t}\}_{t\in[|\calD|/B]}$\; 
Set $T=|\calD|/B$\;
\For{Step $t=1,\ldots,T$}
{
For each $i\in[B]$, get $q_t(Z_{i,t}):=\frac{1}{m}\sum_{z\in Z_{i,t}}\nabla f(x_{t-1};z)$\;
Let $g_{t-1}$ be the output of Algorithm~\ref{alg:robust_gradient_est} with inputs $\{q_t(Z_{i,t})\}_{i\in[B]}$ and threshold $1/\tau$\;
$x_{t}=\Pi_\calX(x_{t-1}-\eta g_{t-1})$
}
\tcc{Concentration Test}
\tcc{Recall the query $q_t(Z_{i,t})$ for each $t\in[T], i\in[B]$ from above}
Run Algorithm~\ref{alg:out_rem} with query $\{q_t\}_{t\in[T]}$ and parameters $\calD_t,\epsilon,\frac{\delta}{2Tmnd},\tau,\upsilon$ to get answers $\{a_t\}_{t\in [T]}$ \;
\If{$a_t=\top,\forall t\in[T]$}
{
\textbf{ Return:} Average iterate $\bar{x}=\frac{1}{T}\sum_{t\in[T]}x_t$\;
}
\Else
{
\textbf{ Output:} Initial point $x_0$\;
}
\end{algorithm2e}

\paragraph{ Iteration Sensitivity of Algorithm~\ref{alg:dpsgd}:}
The contractivity of gradient descent plays a crucial role in the sensitivity analysis, for which we need the Hessians to be diagonally  dominant
(Assumption~\ref{assump:dia_dominant}). 

\begin{restatable}{lem}{contractivity}[Contractivity]
    \label{lm:contractivity}
Suppose $f:\calX\to\R$ is a convex and $\beta$-smooth function satisfying Assumption~\ref{assump:dia_dominant}, then for any two points $x,y\in \calX$, with step size $\eta\le 2/\beta$, we have
    \begin{align*}
        \|(x-\eta \nabla f(x)) - (y-\eta \nabla f(y))\|_\infty\le \|x-y\|_\infty.
    \end{align*}
\end{restatable}

Now, we discuss Algorithm~\ref{alg:dpsgd}.  
Given the dataset $\calD$, we proceed in $T = |\calD|/B$ steps.  
At the $t$th step, we draw $B$ users $\{Z_{i,t}\}_{i \in [B]}$ and compute the average gradient of each user. 
We then apply our gradient estimation algorithm (Algorithm~\ref{alg:robust_gradient_est}) and perform normal gradient descent for $T$ steps.  

In the second phase of Algorithm~\ref{alg:dpsgd}, we perform the concentration test  
(Algorithm~\ref{alg:out_rem}) on the $B$ gradients at each step based on $\AboTh$ (Algorithm~\ref{alg:mean_est_with_AT}).  
If the concentration test passes for all steps (i.e., $a_t = \top$  
for all $t \in [T]$), we output the average iterate. Otherwise, the  
algorithm fails and returns the initial point.  
As mentioned in the Introduction, the crucial novelty of Algorithm~\ref{alg:dpsgd}  
and Algorithm~\ref{alg:robust_gradient_est} lies in bounding the sensitivity  
of each solution $x_t$ by incorporating the (coordinate-wise) robust  
statistics into SGD.

% As discussed before, we apply the (coordinate-wise) geometric median into the SGD algorithm, and show that the iteration-sensitivity can always be bounded in terms of the $\ell_\infty$ norm when the number of ``bad'' users does not exceed the ``break point''.

% Our algorithm framework is based on SGD.
% For the $t$-th phase, we get solution $x_t$ and then take a batch of $B$ users, denoted by $\{Z_{i,t}\}_{i\in[B]}$.
% Each user owns $m$ functions and can run their own gradient descent freely with batch size $K$ from $[1,m]$.
% We take $K=m$ for simplicity; that is, each user takes the average of the $m$ gradients and does one descent step, and we get $\{x_1^{Z_{i,t}}\}_{i\in[B]}$.
% Then we let $x_{t+1}:=\arg\min_{x}\sum_{i\in[B]}\|x-x_1^{Z_{i,t}}\|_\infty$ be the geometric median over the $B$ points. 

\begin{algorithm2e}
\caption{Gradient Estimation based on Robust Statistics}
\label{alg:robust_gradient_est}
\textbf{ Input:} a set of $d$-dimensional vectors $\{X_i\}_{i\in[B]}$, threshold parameter $\varsigma>0$\;
%Initialize a zero vector $X_{est}=\mathbf{0}$\;
\For{Each dimension $j=1,\ldots,d$}
{
Compute the robust statistics $X_{\rs}[j]$, and the mean $\bx[j]$ over $\{X_{i}[j]\}_{i\in[B]}$\;
\If{$|X_{\rs}[j]-\bx[j]|\ge \varsigma$}
{
Set $X_{est}[j]=\Pi_{B(Y_j,\varsigma)}(\bx[j])$\;
}
\Else
{
Set $X_{est}[j]=\bx[j]$\;
}
}
\textbf{ Return $X_{est}$}
\end{algorithm2e}


We utilize robust statistics in the  
gradient estimation sub-procedure. 
We make the following assumptions regarding the robust statistics used:

\begin{assumption}
\label{assum:prop_geo_median}
    Given a set $\{X_i\}_{i \in [B]}$ of vectors, let $X_{\rs}$ be  
    any robust statistic satisfying the following properties:
    
    (i) For any $\rho \ge 0$, if there exists a point $X'$ such  
        that more than $B/2$ points lie within $B_\infty(X', \rho)$,  
        then $X_{\rs} \in B_\infty(X', \rho)$.
        
(ii) If we perturb each point $Y_i = X_i + \iota_i$ such that  
        $\|\iota_i\|_\infty \le \Delta$ for any $\Delta \ge 0$, and let  
        $Y_{\rs}$ be the robust statistic of $\{Y_i\}$, then  
        $\|X_{\rs} - Y_{\rs}\|_\infty \le \Delta$.
        
    (iii) For any real numbers $a$ and $b$, if $Z_i = aX_i + b$ for  
        each $i \in [B]$, and $Z_{\rs}$ is the corresponding robust  
        statistic of $\{Z_i\}_{i \in [B]}$, then $Z_{\rs} = aX_{\rs} + b$.  
\end{assumption}

\begin{remark}
    Common robust statistics, such as the (coordinate-wise) median and trimmed mean,  
    satisfy Assumption~\ref{assum:prop_geo_median}.
    %Pasin: I'm commenting the following out since I don't think all robust statistics are computed in coordinate-wise manner.
    %One can verify  
    %whether the robust statistic satisfies Assumption~\ref{assum:prop_geo_median}  
    %in one dimension, as robust statistics can be computed in a  
    %coordinate-wise manner.
\end{remark}
\vspace{-2mm}
In Algorithm~\ref{alg:robust_gradient_est}, we output means if they are close to the robust statistics to control the bias, and project the means onto the sphere around the robust statistics in a coordinate-wise manner when they are far apart.  
However, we still need to ensure that the sensitivity remains bounded when the projection is operated.  
The following technical lemma plays a crucial role in establishing iteration sensitivity to deal with the sensitivity with potential projection operations.
% Its proof can be found in the Appendix:
\vspace{-1mm}

\begin{restatable}{lem}{projmeantors}
\label{lm:proj_mean_to_rs}
Consider four real numbers $a,b,c,d$, such that $|a-b|\le 1$, and $|c-d|\le 1$.
Let $c'=\Pi_{B(a,r)}(c)$ and $d'=\Pi_{B(b,r)}(d)$ for any $r\ge 0$.
Then, we have $|c'-d'|\le 1.$
\end{restatable} 


Unfortunately, we are unaware of any robust statistic satisfying  
Assumption~\ref{assum:prop_geo_median} in high dimensions under the  
$\ell_2$-norm, and Lemma~\ref{lm:proj_mean_to_rs} does not hold in high  
dimensions either. These limitations restrict the applicability of our  
techniques in high-dimensional Euclidean spaces; see Section~\ref{sec:discussion}.  

Let $\{x_t\}_{t \in [T]}$ and $\{y_t\}_{t \in [T]}$ be two trajectories  
corresponding to neighboring datasets that differ by one user. The  
crucial technical novelty is that, for any $t \in [T]$, we can control  
$\|x_t - y_t\|_{\infty}$ as long as the number of ``bad'' users in each  
phase ($B$ in total) does not exceed the ``break point'', say $2B/3$.  
Without loss of generality, assume that $Z_{1,1} \neq Z_{1,1}'$ is the  
differing user in the neighboring dataset pairs $(\calD, \calD')$  
considered in the following proof.  

The first property of Assumption~\ref{assum:prop_geo_median} ensures that when the number of ``bad'' users in each phase does not exceed the  ``break point'' $2B/3$, the robust statistic remains close to most of the gradients, allowing us to control $\|x_1 - y_1\|_\infty$.  
To formalize this, we say that the neighboring dataset pair 
$(\calD, \calD')$ is $\rho$-\textit{aligned} if there exist points  
$X'$ and $Y'$ such that $|X_{\good}| \ge 2B/3$ and  
$|Y_{\good}| \ge 2B/3$, where  
\[
    X_{\good} = \{q_1(Z_{i,1}) : q_1(Z_{i,1}) \in B_{\infty}(X', \rho),  
    i \in [B]\},  \text{ and }
\]  
\[
    Y_{\good} = \{q_1'(Z_{i,1}') : q_1'(Z_{i,1}') \in B_{\infty}(Y', \rho),  
    i \in [B]\}.  
\]  
This definition essentially states that the number of ``bad'' users does  
not exceed the ``break point'' in either $\calD$ or $\calD'$, ensuring  
that most gradients remain well-aligned within a bounded region.

\begin{restatable}{lem}{itesensitivitybase}
    \label{lm:ite_sensitivity_base}
    For some (unknown) parameter $\rho > 0$, suppose $(\calD, \calD')$  
    is $\rho$-aligned. Then, by running Algorithm~\ref{alg:robust_gradient_est}  
    with threshold parameter $\varsigma \ge 0$, we have $\|x_1 - y_1\|_\infty \le \eta(4\rho + 2\varsigma)$.
\end{restatable}


% Now the sequential sensitivity can be bounded by induction, for which the base case, $\|x_1-y_1\|_\infty$ is already bounded.
% Say $\|x_{t-1}-y_{t-1}\|_\infty$ is bounded,
% then by Lemma~\ref{lm:contractivity}, we can show that $\|x_{j}^{Z_{i,t}}-y_{j}^{Z_{i,t}}\|_\infty\le \|x_{t-1}-y_{t-1}\|_\infty$.
% We then treat $x_{j}^{Z_{i,t}}-y_{j}^{Z_{i,t}}$ as the perturbation and apply the second property in Assumption~\ref{assum:prop_geo_median}, which leads to that $\|x_t-y_t\|_\infty\le \|x_{t-1}-y_{t-1}\|_\infty$.
% The formal statements can be found in Lemma~\ref{lm:iteration_sensitivity}.
The sequential sensitivity can be bounded using induction, with the base  
case $\|x_1 - y_1\|_\infty$ already established. The formal statement  
is provided in Lemma~\ref{lm:iteration_sensitivity}.  

\begin{algorithm2e}
\caption{Concentration Test}
\label{alg:out_rem}
\textbf{ Input:} Dataset $\calD=(Z_1,\ldots,Z_B)$, privacy parameters $\epsilon,\delta$, parameters $\tau,\upsilon$\;
\For{$t=1,\ldots,T$}
{ 
Receive a new concentration query $q_t:\calZ\to\R^d$\;
Define the concentration score
\begin{align}
\label{eq:concentration_score_def}
    \qcon_t(\calD,\tau):=\frac{1}{B}\sum_{Z\in\calD}\sum_{Z'\in \calD}\exp(-\tau\|q_t(Z)-q_t(Z')\|_\infty)\;
\end{align}
\textbf{ Return }$\AboTh(\qcon_t, \epsilon/2, \upsilon)$
}
\end{algorithm2e}


\begin{restatable}[Iteration Sensitivity]{lem}{iterationsensitivity}
\label{lm:iteration_sensitivity}  
    If we use a robust statistic satisfying Assumption~\ref{assum:prop_geo_median}  
    in Algorithm~\ref{alg:robust_gradient_est}, then for all $t \in [T]$, we have  $\|x_t - y_t\|_\infty \le \|x_1 - y_1\|_\infty$.
\end{restatable}

Lemmas~\ref{lm:ite_sensitivity_base} and \ref{lm:iteration_sensitivity}  
together establish the iteration sensitivity of Algorithm~\ref{alg:dpsgd}.

\paragraph{ Query Sensitivity of Concentration Test (Algorithm~\ref{alg:out_rem}):}
We have established iteration sensitivity for any aligned neighboring  
dataset pair $(\calD, \calD')$. Next, we analyze the influence of the  
concentration test, which we use to check if the number of ``bad'' users exceed the ``break point''.

To apply the privacy guarantee of $\AboTh$  
(Lemma~\ref{thm:Above_Threshold}), it suffices to bound the sensitivity  
of each query in the concentration test.  
Recall that we assume $Z_{1,1} \neq Z_{1,1}'$ in the neighboring datasets.  
Thus, by the definition (Equation~\eqref{eq:concentration_score_def}), it is straightforward to observe that  
\begin{align}
\label{eq:query_sensitivity_qcon_one}
    |\qcon_1(\calD, \tau) - \qcon_1(\calD', \tau)| \le 2.  
\end{align}  

Next, we consider the sensitivity of $\qcon_t$ for $t \ge 2$.  
The sensitivity is proportional to $\|x_t - y_t\|_\infty$, which we have  
already bounded by $\|x_1 - y_1\|_\infty$.  
Note that we can bound the iteration sensitivity if the neighboring  
datasets are aligned, meaning the number of ``bad'' users does not  
exceed the ``break point''. We first show that if the number of ``bad''  
users exceeds the ``break point'', the algorithm is likely to halt  
after the first step by failing the first test.


\begin{restatable}{lem}{sensitivitybase}
    \label{lm:sensitivity_base}
Suppose $B\ge \frac{100\log(T/\delta)}{\epsilon}, \epsilon\le O(1)$ and we set $\upsilon=0.9B+\frac{2\log(T/\delta)}{\epsilon}$.
Suppose for any point $Y$, we get $|X_{\good}|<B/3$ where $X_{\good}=\{q_1(Z_{i,1}):q_1(Z_{i,1})\in B_{\infty}(Y,1/\tau),i\in[B]\}$.
Then with probability at least $1-\delta/T\exp(\epsilon)$, the $\AboTh$ returns $a_1=\bot$.
\end{restatable}






% \begin{lemma}[Query Sensitivity: Part One]
% \label{lm:query_diff}
% Consider two initial points $x$ and $y$ such that $\|x-y\|_\infty \le \nu$, and get $\{x_{t}^Z\}_{t\in[m/K]}$ and $\{y_{t}^Z\}_{t\in[m/K]}$ respectively from running $\PGD(x,\eta,Z,K)$ and $\PGD(y,\eta,Z,K)$, defined from Equation~\eqref{eq:PGD_each_user}.
% Similarly, we get $\{x_t^{Z'}\}_{t\in[m/K]}$ and $\{y_t^{Z'}\}_{t\in[m/K]}$ for another user $Z'$.
% Then under Assumptions~\ref{assum:lispchitz_smooth}, with $\eta\beta\le1$, for any $t\in[m/K]$, we have
% \begin{align*}
%     \Bigg|\|x_t^Z-x_t^{Z'}\|_\infty-\|y_t^{Z}-y_t^{Z'}\|_\infty\Bigg|\le 2\nu \eta\beta.
% \end{align*}
% \end{lemma}

% \begin{proof}
%     % We prove the statement by induction.
%     % As for the basic case when $t=1$, we have
%     % \begin{align*}
%     %    & \|x_t^Z-x_t^{Z'}\|_\infty-\|y_t^Z-y_t^{Z'}\|_\infty\\
%     %    =& \|x_t^Z-x-(x_t^{Z'}-x)\|_\infty-\|y_t^Z-y-(y_t^{Z'}-y)\|_\infty\\
%     %    \le & \|x_t^Z-x-(y_t^Z-y)+(y_t^{Z'}-y)-(x_t^{Z'}-x)\|_\infty\\
%     %    \le & \|x_t^Z-x-(y_t^Z-y)\|_\infty+\|(y_t^{Z'}-y)-(x_t^{Z'}-x)\|_\infty\\
%     %    \le & 2\eta\nu\beta,
%     % \end{align*}
%     % where the last inequality comes from the assumption on smoothness.

%     % Now suppose the condition holds for any $t\le t'$, consider the case when $t=t'+1$.
%     Letting $x_0^Z=x_0^{Z'}=x$ and $y_0^Z=y_0^{Z'}$, notice that
%     \begin{align*}
%         &\|(x_t^Z-x_{t-1}^Z)-(y_t^Z-y_{t-1}^Z)\|_\infty \\
%         \le&  \eta/K\|\sum_{j\in[K]}\nabla f(x_{t-1}^Z;z_{t,j})-\nabla f(y_{t-1}^Z;z_{t,z})\|_\infty \\
%         \le & 2\eta\beta \|x_{t-1}-y_{t-1}^Z\|_\infty\\
%         \le & 2\eta\beta\nu.
%     \end{align*}
    
%  Hence, we have
%     \begin{align*}
%         & \|x_t^Z-x_t^{Z'}\|_\infty-\|y_t^Z-y_t^{Z'}\|_\infty\\
%        =& \|\sum_{i=1}^{t}(x_i^Z-x_{i-1}^Z)-\sum_{i=1}^t(x_i^{Z'}-x_{i-1}^{Z'})\|_\infty-\|\sum_{i=1}^{t}(y_i^Z-y_{i-1}^Z)-\sum_{i=1}^t(y_i^{Z'}-y_{i-1}^{Z'})\|_\infty\\
%        \le & \|\sum_{i=1}^{t} (x_i^Z-x_{i-1}^Z)-(y_i^Z-y_{i-1}^Z)\|_\infty+\|\sum_{i=1}^t(y_i^{Z'}-y_{i-1}^{Z'})-(x_i^{Z'}-x_{i-1}^{Z'})\|_\infty\\
%        \le & 2t\nu\eta\beta.
%     \end{align*}
%     This completes the proof.
% \end{proof}

We now analyze the query sensitivity between the aligned neighboring  
datasets.

\begin{restatable}[Query Sensitivity]{lem}{querysensitivity}
    \label{lm:query_sensitivity}
Suppose $6\beta\eta B\le1$.
Suppose $(\calD,\calD')$ is $(1/\tau)$-aligned and set threshold parameter $\varsigma=1/\tau$ in Algorithm~\ref{alg:mean_est_with_AT}, the sensitivity of the query is bounded by at most $2$.
That is,
\begin{align*}
    |\qcon_t(\calD,\tau)-\qcon_1(\calD',\tau)|\le 2, & & \forall t\ge 2.
\end{align*}
\end{restatable}



Equation~\eqref{eq:query_sensitivity_qcon_one} shows that the sensitivity is always bounded for $\qcon_1$.  
Lemma~\ref{lm:sensitivity_base} shows that if the number of ``bad''  
users exceeds the ``break point'', we obtain $a_1 = \bot$, and  
the query sensitivities of the subsequent queries do not need to be considered.  
Lemma~\ref{lm:query_sensitivity} establishes the query sensitivity  
in the concentration test when the neighboring datasets are aligned,  
and the number of "bad" users is below the threshold.

\paragraph{Privacy proof.}
% Consider the implementation on two neighboring datasets $\calD$ and $\calD'$.
% Without loss of generality, we assume that the different users appeared in the first batch, that is, $t=1$.

%Now, we can complete the proof of the privacy guarantee.

%The final privacy guarantee is stated below. 
The final privacy guarantee--stated formally below--now easily follows from the previous lemmas.
% Due to space constraint, 
The full proof is deferred to Appendix~\ref{app:privacy-proof}.

\begin{restatable}[Privacy Guarantee]{lem}{privacyguarantee}
    \label{lm:privacy_guarantee}
    Under Assumption~\ref{assum:lispchitz_smooth} and Assumption~\ref{assump:dia_dominant}, suppose $\epsilon\le O(1), B\ge\frac{100\log(T/\delta)}{\epsilon}$, then Algorithm~\ref{alg:loacalizatioin} is $(\epsilon,\delta)$-user-level-DP.
\end{restatable}





\paragraph{Utility proof.}
We apply the localization framework in private optimization to finish the utility argument.
We analyze the utility guarantee of Algorithm~\ref{alg:dpsgd} based on the classic convergence rate of SGD on smooth convex functions (Lemma~\ref{lm:sgd_smooth}) as follows:
% The following classic result states the convergence rate of SGD for smooth convex functions.



% \Daogao{Clean the notations..}
% We have the following lemma:

% \begin{lemma}[\cite{LLA24}]
% \label{lemma_tech_core}
% Assume $f(\cdot, z)$ is convex, $G$-Lipschitz, and $\beta$-smooth on $\calX$ with $\eta \leq 1/\beta$. Let $\tilde{x} \gets SGD(D, \eta, T, x_0,1)$ and $\tilde{y} \gets SGD(D', \eta, T, x_0,1)$ be two independent runs of projected SGD, where
% $D, D' \sim \calP^T$ are i.i.d. Then, with probability at least $1 - \zeta$, we have \[
% \|\tilde{x} - \tilde{y}\|_2 \lesssim \eta G\sqrt{T \log(dT/\zeta)}.
% \]
% \end{lemma}

% As may be noticed, the naive bound we can get is $\|\title{x}-\title{y}\|_2\le 2\eta LT$.
% Hence, the distributional assumption on $\calD$ and $\calD'$ improves the stability from $\Tilde{O}(\eta LT)$ to $\Tilde{O}(\eta L\sqrt{T})$, which is crucial in getting improved results in user-level setting.

% We generalize it into a batched version of SGD, that is the batch size of each iterate is captured by a parameter $K\ge 1$:
% \Daogao{Overuse notation $T$...}

% \begin{lemma}
% \label{lm:batched_tech_core}
%     Assume $f(\cdot, z)$ is convex, $G$-Lipschitz, and $\beta$-smooth on $\calX$ with $\eta \leq 1/\beta$. Let $\tilde{x} \gets SGD(D, \eta, T, x_0,K)$ and $\tilde{y} \gets SGD(D', \eta, T, x_0,K)$ be two independent runs of projected SGD, where
% $D, D' \sim \calP^{TK}$ are i.i.d. Then, with probability at least $1 - \zeta$, we have \[
% \|\tilde{x} - \tilde{y}\|_2 \lesssim \eta G\sqrt{T \log(dT/\zeta)/K}.
% \]
% \end{lemma}

% \begin{proof}
% Let $g_t:= \frac{1}{K}\sum_{i\in[K]}\nabla f(x_t, z_{t,i})$ for $\{z_{t,i}\}_{i\in[K]}$ drawn uniformly from $D$ without replacement and $g_t':=  \frac{1}{K}\sum_{i\in[K]}\nabla f(y_t, z'_{t,i})$ for $z'_{t,i}$ drawn uniformly from $D'$ without replacement. Let $F(x) := \E_{z \sim \calP}[f(x,z)]$. 

% We will prove that $\|x_t - y_t\| \lesssim \eta L\sqrt{T \log(dT/\zeta)/K}$ with probability at least $1 - \zeta/t$ for all $t \in [T]$. Note that this implies the lemma. We proceed by induction. The base case, when $t=0$, is trivially true since $x_0 = y_0$. For the inductive hypothesis, suppose there is an absolute constant $c > 0$ such that with probability at least $1-t\zeta/T$, we have 
% \begin{align*}
%     \|x_{i}-y_i\|\le  c  \eta L\sqrt{i\cdot \log(dT/\zeta)/K},
% \end{align*}
% $\forall i \le t$. Then, for the inductive step, we have by non-expansiveness of projection onto convex sets, that
% \begin{align}
% \label{eq: thingy}
%     \|x_{t+1}-y_{t+1}\|^2 \le &~ \|x_t-\eta g_t-(y_t-\eta g_t')\|^2 \nonumber \\
%     =& ~ \|x_t-\eta \nabla F(x_t)-(y_t-\eta \nabla F(y_t))-\eta (g_t- \nabla F(x_t)-g_t'+\nabla F(y_t)   )\|^2 \nonumber\\
%     =& ~ \|x_t-\eta\nabla F(x_t)-(y_t-\eta \nabla F(y_t))\|^2 \nonumber\\ 
%     &~-2\eta \langle  x_t-\eta\nabla F(x_t)-(y_t-\eta \nabla F(y_t)),g_t- \nabla F(x_t)-g_t'+\nabla F(y_t)\rangle \nonumber\\
%     &~+ \eta^2 \|g_t-\nabla F(x_t)-g_t'+\nabla F(y_t)\|^2 \nonumber \\
%     \stackrel{(i)}{\le} & ~ \|x_t-y_t\|^2-2\eta \langle  x_t-\eta\nabla F(x_t)-(y_t-\eta \nabla F(y_t)),g_t- \nabla F(x_t)-g_t'+\nabla F(y_t)\rangle \nonumber \\
%     &~+ \eta^2 \|g_t-\nabla F(x_t)-g_t'+\nabla F(y_t)\|^2, 
% \end{align}
% where $(i)$ follows from the non-expansive property of gradient descent on smooth convex function for $\eta \le 1/\beta$~\cite{hardt16}.

% For any $t\in T$, we have
% \begin{align*}
%  \Pr\Big[\eta^2 \|g_t-\nabla F(x_t)-g_t'+\nabla F(y_t)\|^2\ge 4\log(Td/\zeta)\eta^2L^2/K\Big]\le \zeta/T .  
% \end{align*}
% Conditional on this event in the following argument.

% Define $a_t:=-2\eta \langle  x_t-\eta\nabla F(x_t)-(y_t-\eta \nabla F(y_t)),g_t- \nabla F(x_t)-g_t'+\nabla F(y_t)\rangle$.
% By Inequality~\eqref{eq: thingy} and the inductive hypothesis, we obtain
% \begin{align*}
%     \|x_{t+1}-y_{t+1}\|^2\le 4t\log(Td/\zeta)\eta^2L^2/K+\sum_{i=1}^{t}a_t.
% \end{align*}
% It remains to bound $\sum_{i=1}^t a_i$.
% Note that $\E[a_i \mid a_1,\cdots,a_{i-1}]=0$ and $g_t-\nabla F(x_t)$ is $\nSG(\log(d/\zeta)/\sqrt{K})$. 
% By Lemma~\ref{lm:inner_product_nSG}, we know there is a constant $c' > 0$ such that $a_i$ is $\nSG(c' \eta L \|x_i-y_i\|/\sqrt{K})$ for all $i$.
% Hence by Theorem~\ref{thm:hoeffding_nSG}, we know
% \begin{align*}
% \mathbb{P}\left[\left|\sum_{i=1}^{t}a_i\right|\ge c'\eta L\sqrt{\log(dT/\zeta)\sum_{i\le t}\|x_i-y_i\|^2/K}\right]\le 1-\zeta/T.
% \end{align*}

% Conditional on the event that $\|x_{i}-y_{i}\|\le c\sqrt{\log(dT/\zeta)}\eta L\sqrt{i/K}$ for all $i \leq t$ (which happens with probability $1-t\zeta/T$ by the  inductive hypothesis), we know
% \begin{align*}
% \mathbb{P}\left[\left|\sum_{i=1}^{t}a_i\right|\ge (cc')tL^2\eta^2\log(dT/\zeta)/K \middle| \|x_i-y_i\|\le c\log(dT/\zeta)\eta L\sqrt{i/K},\forall i\le t\right]\le 1-\zeta/T.
% \end{align*}
% Hence, as long as $4t+cc't\le c^2(t+1)$, we know
% \begin{align*}
%     \mathbb{P}\left[\|x_{t+1}-y_{t+1}\|^2\ge c^2\log(dT/\zeta)\eta^2L^2(t+1)/K \middle|  \|x_i-y_i\|\le c\log(dT/\zeta)\eta L\sqrt{i/K},\forall i\le t \right]\le 1-\zeta/T.
% \end{align*}
% Combining the above elements completes the inductive step, showing that 
% \[\|x_{t+1}-y_{t+1}\|\le c\sqrt{(t+1)\log(dT/\zeta)/K}\eta L\]
% with probability at least $1-(t+1)\zeta/T$.
% This completes the proof.
% \end{proof}


% \begin{lemma}
% \label{lem:concentrated_grd}
% Under Assumption~\ref{assum:lispchitz_smooth},
% for any fixed $x$ and for each $Z_i$, if each item in $Z_i$ is drawn i.i.d. from $\calP$, with probability at least $1-\gamma/n$, we have
% \begin{align*}
%     \|\nabla F(x;Z_i)-\nabla F_\calP(x)\|_2\lesssim \frac{G\log(nd/\gamma)}{\sqrt{m}},
% \end{align*}
% \end{lemma}



% \subsection{Further Potential Improvements}
% It is interesting to see if we can generalize the results to the $\ell_2$ norm by utilizing other estimators.
% In the $\ell_2$ norm, we may use and generalize the following stability lemma from \cite{LLA24}.



\begin{restatable}{lem}{dgsgdutility}
    \label{lm:dpsgd_utility}
    Let $x\in\calX$ be any point in the domain.
    Suppose the data set $\calD$ of the users, whose size $|\calD|$ is larger than $\frac{100\log(T/\delta)}{\epsilon}$, is drawn i.i.d. from the distribution $\calP$.
    Setting $\tau=G\log(nmd/\omega)/\sqrt{m}$
    then the final output $\bar{x}$ of Algorithm~\ref{alg:dpsgd} satisfies that
    \begin{align*}
        \E[F(\bar{x})-F(x)]\lesssim \left(\beta+\frac{1}{\eta} \right)\frac{\E[\|x_0-x\|^2]}{T}+\frac{\eta G^2d}{Bm}+GDd\omega.
    \end{align*}
\end{restatable}



% \begin{proof}
% We reindex the iterates by $y_{i,(t-1)m/K+j}=x_{j}^{Z_{i,t}}$ when $j\neq m/K$, ane define $y_{i,tm/K}=x_{t}$ for $0\le t\le T$.

% Then the average iterate $\bar{x}=\frac{K}{mTB}\sum_{i\in[B],j\in[m/K],t\in[T]}x_j^{Z_{0,t}}=\frac{K}{mTB}\sum_{i\in[B],j\in[Tm/K]}y_{i,j}$.

% To prove the statement, without loss of generality, it suffices to show
% \begin{align*}
%     \E[F(\frac{K}{Tm}\sum_{j\in[Tm/K]}y_{1,j})-F(x)]\lesssim (\beta+\frac{1}{\eta})\frac{\E[\|x_0-x\|^2]}{Tm/K}+\frac{\eta(TG^2d m/K^2) }{Tm/K}.
% \end{align*}

% Let $g_j$ be the gradient estimate, that is
% \begin{align*}
%     y_{1,j}=\Pi_\calX(y_{1,j-1}-\eta g_j).
% \end{align*}

% By Lemma~\ref{lm:sgd_smooth}, 
% it suffices to bound $\sum_{j\in[Tm/K]}\E\|g_j-\nabla F(y_{1,j})\|_2^2\le \eta TG^2dm/K^2$.

% For any $j\mod (m/K)\neq 0$, we know $\E g_j=\nabla F(y_{1,j})$ and $\E\|g_j-\nabla F(1,j)\|^2_2\le G^2/K$.
% For the other case when $j\mod (m/K)\equiv 0$, define $\Tilde{g}_j$ be the gradient estimator for which 
% \begin{align*}
%     x_{m/K}^{Z_{1,jK/m}}=\Pi_{\calX}(x_{m/K-1}^{Z_{1,jK/m}}-\eta \Tilde{g}_j).
% \end{align*}

% We have
% \begin{align*}
%     \E\|g_j- \nabla F(y_{1,j})\|_2^2\le & 2\E\|g_j-\Tilde{g}_j\|_2^2+2\E\|\Tilde{g}_j-\nabla F(y_{1,j})\|_2^2.
% \end{align*}
% Similarly, we can bound $\E\|\Tilde{g}_j-\nabla F(y_{1,j})\|_2^2\le G^2/K$ by the assumptions on i.i.d. and Lipschitz.
% It remains to bound $\E\|g_j-\tg_j\|_2^2$.

% By Lemma~\ref{lm:batched_tech_core} and Lemma~\ref{lm:prop_geo_median}, we know that 
% $
% \|y_{1,j}-x_{m/K}^{Z_{1,jK/m}}\|_\infty\lesssim \eta G\sqrt{m\log(dm)/K^2}$,
% and hence we can bound 
% $\E\|g_j-\tg_j\|_2^2\le\|y_{1,j}-x_{m/K}^{Z_{1,jK/m}}\|_2^2/\eta^2+G^2\lesssim G^2md\log(dm)/K^2$.
% This completes the proof.
% \Daogao{We need a new lemma to handle the shift for the utility when $K$ is large...}
% \end{proof}

Now we apply the localization framework.
We set $\omega=1/(nmd)^3$ to make the term depending on it negligible.
The proof of the following lemma mostly follows from \cite{FKT20}.

\begin{restatable}[Localization]{lem}{Localization}
    \label{lm:localization}
Under Assumption~\ref{assum:lispchitz_smooth} and Assumption~\ref{assump:dia_dominant}, suppose $\beta\le\frac{G}{D}(\frac{\sqrt{n}\epsilon}{\sqrt{m}\log(nmd/\delta})$, $n\ge \log^2(nd/\delta)/\epsilon, \epsilon\le O(1)$ and $ m\le n^{O(\log\log n)}$.
Set $\eta\le\frac{D}{G}\cdot \min\{ \frac{B\sqrt{m}}{\sqrt{n}} ,  \frac{\sqrt{m}\epsilon}{\sqrt{d\log(1/\delta)\log( nmd)}}\}$, $B=100\log(mnd/\delta)/\epsilon$, $\tau=O(G\log(nmd)/\sqrt{m})$ and $\upsilon=0.9B+\frac{2\log(T/\delta)}{\epsilon}$.
If the dataset is drawn i.i.d. from the distribution $\calP$,
the final output $x_S$ for Algorithm~\ref{alg:loacalizatioin} satisfies
\begin{align*}
    \E[F(x_S)-F(x^*)]\le \Tilde{O}\Big(GD\Big(\frac{d}{\sqrt{mn}}+\frac{d^{3/2}}{n\epsilon^2\sqrt{m}}\Big)\Big).
\end{align*}
\end{restatable}

\noindent\textbf{ Main Result:}
Theorem~\ref{thm:main_result} directly follows from Lemma~\ref{lm:localization} and Lemma~\ref{lm:privacy_guarantee}.


% \begin{lemma}[Localization]
% \label{lm:localization}
% Under Assumption~\ref{assum:lispchitz_smooth} and Assumption~\ref{assump:dia_dominant},
% the final output $x_k$ for Algorithm~\ref{alg:loacalizatioin} satisfies that
% \begin{align*}
%     \E[F(x_k)-F(x^*)]\le \Tilde{O}(\frac{\sqrt{d}}{\sqrt{mn\epsilon}}+\frac{d}{n\epsilon\sqrt{m}}).
% \end{align*}
% \end{lemma}

% \begin{proof}
% We need $\eta\le\min\{ \frac{K}{\epsilon\sqrt{nmd}} ,   \frac{K\epsilon}{d\sqrt{m}}\}$.
% Let $\bx_0=x^*$ and $\zeta_0=x_0-x^*$.
% By the assumption, we know $\|\zeta_0\|_2\le D\sqrt{d}$.
% Recall that by definition $\eta\le \frac{D}{G}\cdot\frac{K\epsilon}{d\sqrt{m}}$, for all $t\ge 0$,
% \begin{align*}
%     \E[\|\zeta_t\|_2^2]=d\sigma_t^2=\frac{G^2d^2m}{K^2\epsilon^2}\cdot\frac{D^2K^2\epsilon^2}{mdG^2(\log m)^{2t}}\le (\frac{D}{\log^{-t} m})^2.
% \end{align*}
% Then by Lemma~\ref{lm:dpsgd_utility}, we have
% \begin{align*}
%     \E[F(x_k)]-F(x^*)=&\sum_{t=1}^{k}\E[F(\bx_{t}-\bx_{t-1})]+\E[F(x_k)-F(\bx_k)]\\
%     \le & \sum_{t=1}^{k}(\frac{\E[\|\zeta_{i-1}\|_2^2]}{\eta_i(T_im/K)}+\frac{\eta_iG^2d}{2K})+G\E[\|\zeta_k\|_2]\\
%     \le& \sum_{i=1}^{k}(\frac{\log m}{2})^{-i}(\frac{D^2}{\eta nm\epsilon/K}+\frac{\eta G^2d}{2K})+\frac{GD}{(\log m)^{\log n}}\\
%     \lesssim &   GD(\frac{\sqrt{d}}{\sqrt{nm}}+\frac{d}{n\sqrt{m}\epsilon}).
% \end{align*}
% \end{proof}








\subsection{Regularized Gradient Descent}
The auxiliary loss in Eq.~\eqref{eq:visual_loss} preserves the visual representation at a distribution level via the feature alignment auxiliary loss in Eq.~\eqref{eq:visual_loss}. 
However, the information bottleneck framework indicates that the gradient component compressing $I(X^v; Z)$ (\emph{i.e.}, $\nabla_\theta I(X^v; Z)$), 
can harm visual preservation by reducing the effective rank of the features \cite{achille2018information,lee2021compressive}.

To address this compression-induced drift, we incorporate an orthogonal gradient as a regularize. 
Motivated by multi-task orthogonal gradient optimization \cite{yu2020gradient, zhu2022gradient, dong2022gdod}, 
we leverage the gradient $\Bar{g}_\phi$ from the pre-trained model $\mu_\phi$, which reflects the accumulated visual drift and approximates a global orthogonal learning effect in the downstream task. 
We then project the current model’s gradient onto this direction:
\begin{equation}\label{eq:gd}
    \Tilde{g}_\theta = \frac{\Bar{g}_\theta^\top \Bar{g}_\phi}{\|\Bar{g}_\phi\|^2}\cdot \Bar{g}_\phi.
\end{equation}

In addition, to prevent discrepancies between the regularization and task gradients, we include the feature alignment auxiliary loss (Eq.~\eqref{eq:visual_loss}) in the overall objective. The final parameter update is:
\begin{equation}\label{eq:opt-gd}
    \pi_\theta \leftarrow \pi_\theta - \nabla_\theta\mathcal{L}_{vl}(\theta) - \nabla_\theta\mathcal{L}_v(\theta) - \Tilde{g}_\theta.
\end{equation}

\subsection{Enabling Parameter-efficient Fine-tuning of MDGD via Gradient Masking}
Parameter-efficient fine-tuning (PEFT) methods, such as adapters \cite{houlsby2019parameter} and LoRA \cite{hu2021lora}, aim to reduce the computational cost and memory usage when fine-tuning models on downstream tasks under practical constraints \cite{han2024parameter}. 
However, due to the requirement of directly estimating gradient directions on the pre-trained model parameters, MDGD cannot be directly applied to these PEFT methods, which introduce additional model parameters whose gradients are separate from the original model weights. 

To address this challenge, we propose a variant, MDGD-GM, by formulating the gradient regularization term in Eq.~\eqref{eq:gd} as gradient masking that selects model weights with efficient gradient directions. Specifically, we define the gradient mask as
\begin{equation}\label{eq:masking}
    M_{\Tilde{g}_\theta} = \mathbf{1}\left\{\frac{\Bar{g}_\theta^\top \Bar{g}_\phi}{\|\Bar{g}_\phi\| \|\Bar{g}_\theta\|} \geq T_\alpha \right\},
\end{equation}
where $T_\alpha$ is determined by a percentile $\alpha$ of trainable parameters with the highest similarity scores between $\Bar{g}_\theta$ and $\Bar{g}_\phi$. Consequently, the optimization in Eq.~\eqref{eq:opt-gd} is reformulated as
\begin{equation}\label{eq:opt-mask}
    \pi_\theta \leftarrow \pi_\theta - M_{\Tilde{g}_\theta} \cdot \left(\nabla_\theta\mathcal{L}_{vl}(\theta) + \nabla_\theta\mathcal{L}_v(\theta)\right).
\end{equation}
We summarize and illustrate the optimization process of MDGD and MDGD-GM in Algorithm~\ref{alg}.

% \section{Experiment}
In this section, we conduct extensive experiments to evaluate the performance of various LLMs on our Hellaswag-Pro benchmark. Our study is guided by three key research questions:
\textbf{RQ1}: How do different LLMs perform across all variants?
\textbf{RQ2}: What is the relative difficulty of different variants?
\textbf{RQ3}: How robust are LLMs to diverse prompts during evaluation?

\subsection{Experiment Setup} 
\subsubsection{Model Selection and Implementation Details}
We select 41 representative commercial and open-source models, including English LLMs, such as GPT-4o, Claude-3.5-Sonnet, Gemini-1.5-Pro,Mistral series, Llama3 series and Chinese LLMs, like Qwen-Max,  Qwen2.5 series, InternLM-2.5 series, Yi-1.5 series, Baichuan-2 series and DeepSeek series.

We integrate both Chinese HellaSwag and HellaSwagPro into the lm-evaluation-harness platform. For the open-source models, we use the default settings of lm-evaluation-harness: do\_sample is set to false and the temperature is set to the default value of the hugging-face library. For the closed-source models, we set the temperature to 0.7. In addition, we set the maximum output length to 1024.

\subsubsection{Prompt Strategy}
Taking into account the influence of language and shot, we design 9 prompting strategies, including Direct, CN-CoT, EN-CoT, CN-XLT and EN-XLT. The last four setups include both zero-shot and few-shot variants.\footnote{
For open-source models, Direct adopts an approach similar to the official implementation of HellaSwag, computing the log-likelihood for each option and selecting the one with the highest log-likelihood. And we report normalized accuracy that accounts for the impact of option length. Other prompting strategies use a generation setup and report accuracy based on exact match.}
\textbf {(1)Direct}: LLMs makes the selection directly without any CoT process.
\textbf{(2)CN-CoT}: LLMs performs CoT in Chinese, regardless of dataset language.
\textbf{(3)EN-CoT}: Similar to CN-CoT, but CoT is conducted in English. 
\textbf{(4)CN-XLT}: LLMs are instructed to first translate English questions and options to Chinese, and then reason in Chinese.
\textbf{(5)EN-XLT}: Similar to CN-XLT, but translates from Chinese dataset to English and reasons in English. 

%\textbf {CN-CoT}: LLMs perform Chinese reasoning and then output the answer and 3 shots are provided.
%\textbf {CN-CoT}: Similar as CNCoTFewShot without any shots.
%\textbf {EN-CoT}: The reasoning process in English is executed and then the answer is output and 3 shots are provided.
%\textbf {CN-XLT}: Inspired by this, we instruct LLMs to translate questions in Chinese and then output the answer after performing reasoning in Chinese too. And 3 shots are provided.
%\textbf {EN-XLT}: Inspired by this, we instruct LLMs to translate questions in Englsih and then output the answer after performing reasoning in Englsih too. Three shots are provided.

\subsubsection{Evaluation metric}

To comprehensively evaluate the robustness of each LLM, we consider four metrics: 
% Original Accuracy (\textbf{OA}), Average Robust Accuracy (\textbf{ARA}), Robust Loss Accuracy (\textbf{RLA}), and  Consistent Robust Accuracy (\textbf{CRA}).
\noindent %
\textbf{- Original Accuracy (OA)} measures accuracy on original problems.
\begin{equation}\label{eq1}
OA=\frac{\sum_{(x, y) \in D} \mathds{1}[L M(x), y]}{|D|}.
\end{equation}
\noindent %
\textbf{- Average Robust Accuracy  (ARA)} represents average accuracy across all variants, gauging overall performance on the robustness tasks.
\begin{equation}\label{eq2}
ARA=\frac{\sum_{\left(x^{\prime}, y^{\prime}\right) \in D_{R}} \mathds{1}\left(L M\left(x^{\prime}, y^{\prime}\right)\right.}{\left|D_{R}\right|}.
\end{equation}

\noindent %
\textbf{- Robust Loss Accuracy (RLA)} is the difference between ARA and OA, indicating performance degradation on robustness data versus original data.
%\begin{tiny}
%\begin{equation}\label{eq3}
%RLA=\frac{\sum_{\left(x^{\prime}, y^{\prime}\right) \in D_{R}} %\mathds{1}\left(L M\left(x^{\prime}, y^{\prime}\right)\right.}{\left|D_{R}\right|}-\frac{\sum_{(x, y) \in D}\mathds{1}[L M(x), y]}{|D|}
%\end{equation}
%\end{tiny}
\begin{equation}\label{eq3}
RLA= OA - ARA.
\end{equation}
\noindent %
\textbf{- Consistent Robust Accuracy (CRA)} shows accuracy when the model correctly answers both original and variant data, reflecting the model do understand the problem.
% consistency in problem-solving.
\begin{equation}\label{eq4}
CRA=\frac{\sum_{x, y, x^{\prime}, y^{\prime}}\mathds{1}[L M(x), y] \cdot \mathds{1}[L M(x^{\prime}), y^{\prime}]}{\left|D_{R}\right|}.
\end{equation}
For all equation above, $D$ denotes the original dataset, where $x$ represents the input question and options, and $y$ represents the correct label, while $D_{R}$ is the robust dataset with $x^{\prime}$ and $y^{\prime}$ representing similar to $x$ and $y$.


\begin{table*}[ht]
\centering
\setlength{\tabcolsep}{5pt}
% \footnotesize
\scalebox{0.6}{
% Please add the following required packages to your document preamble:
% \usepackage{multirow}
% \usepackage[table,xcdraw]{xcolor}
% Beamer presentation requires \usepackage{colortbl} instead of \usepackage[table,xcdraw]{xcolor}
% Please add the following required packages to your document preamble:
% \usepackage{multirow}
% \usepackage[table,xcdraw]{xcolor}
% Beamer presentation requires \usepackage{colortbl} instead of \usepackage[table,xcdraw]{xcolor}
\begin{tabular}{ccccccccccccc}
\hline
\multicolumn{1}{c|}{{ }}& \multicolumn{4}{c|}{Chinese}& \multicolumn{4}{c|}{English}& \multicolumn{4}{c}{AVG}\\ \cline{2-13} 
\multicolumn{1}{c|}{\multirow{-2}{*}{{ Model}}} & { OA(\%)$\uparrow$}& { ARA(\%)$\uparrow$} & {RLA(\%)$\downarrow$}& \multicolumn{1}{l|}{{CRA(\%)$\uparrow$}} & { OA(\%)$\uparrow$}& { ARA(\%)$\uparrow$} & { RLA(\%)$\downarrow$}& \multicolumn{1}{l|}{{CRA(\%)$\uparrow$}} & {OA(\%)$\uparrow$}& { ARA(\%)$\uparrow$} & {RLA(\%)$\downarrow$}& { CRA(\%)$\uparrow$} \\ \hline
\multicolumn{1}{c|}{{ Human}} & 96.41& 97.79& -1.38 & \multicolumn{1}{l|}{92.03}& 95.56& 96.04& -0.48 & \multicolumn{1}{l|}{90.02}& 95.99 & 96.92 & -0.93& 91.03 \\ \hline
\multicolumn{13}{c}{\textit{Close-source LLMs}}\\ 
\multicolumn{1}{c|}{{ GPT-4o}}& { 91.37} & { 81.97} & { 9.40}& \multicolumn{1}{l|}{{ 75.55}} & { \textbf{88.63}} & { \textbf{70.17}} & { \textbf{18.46}} & \multicolumn{1}{l|}{{ \textbf{63.06}}} & { 90.00} & { \textbf{76.07}} & { \textbf{13.93}} & { \textbf{69.31}} \\
\multicolumn{1}{c|}{{ Claude3.5}}& { \textbf{95.37}} & { 80.15} & { 15.22} & \multicolumn{1}{l|}{{ 75.04}} & { 85.11} & { 66.02} & { 19.08} & \multicolumn{1}{l|}{{ 57.20}} & { 90.24} & { 73.09} & { 17.15} & { 66.12} \\
\multicolumn{1}{c|}{{ Gemini-1.5-Pro}}& { 90.62} & { 78.36} & { 12.26} & \multicolumn{1}{l|}{{ 70.48}} & { 87.75} & { 60.74} & { 27.01} & \multicolumn{1}{l|}{{ 58.27}} & { 89.19} & { 69.55} & { 19.63} & { 64.38} \\
\multicolumn{1}{c|}{{ Qwen-Max}}& { 93.50} & { \textbf{84.82}} & { \textbf{8.68}}& \multicolumn{1}{l|}{{ \textbf{78.91}}} & { 87.60} & { 62.61} & { 24.99} & \multicolumn{1}{l|}{{ 59.65}} & { \textbf{90.55}} & { 73.72} & { 16.83} & { 69.28} \\ \hline
\multicolumn{13}{c}{\textit{Chinese open-source LLMs}} \\ 
\multicolumn{1}{c|}{{ Qwen2.5-0.5B}}& { 60.75} & { 45.18} & { \textbf{15.57}} & \multicolumn{1}{l|}{{ 28.70}} & { 49.50} & { 38.21} & { \textbf{11.29}} & \multicolumn{1}{l|}{{ 20.57}} & { 55.13} & { 41.70} & { \textbf{13.43}} & { 24.64} \\
\multicolumn{1}{c|}{{ Qwen2.5-1.5B}}& { 63.25} & { 46.16} & { 17.09} & \multicolumn{1}{l|}{{ 29.89}} & { 56.88} & { 39.57} & { 17.30} & \multicolumn{1}{l|}{{ 23.48}} & { 60.06} & { 42.87} & { 17.20} & { 26.69} \\
\multicolumn{1}{c|}{{ Qwen2.5-3B}}& { 67.50} & { 48.75} & { 18.75} & \multicolumn{1}{l|}{{ 33.79}} & { 61.75} & { 39.98} & { 21.77} & \multicolumn{1}{l|}{{ 25.75}} & { 64.63} & { 44.37} & { 20.26} & { 29.77} \\
\multicolumn{1}{c|}{{ Qwen2.5-7B}}& { 67.63} & { 50.59} & { 17.04} & \multicolumn{1}{l|}{{ 35.62}} & { 65.63} & { 43.93} & { 21.70} & \multicolumn{1}{l|}{{ 30.77}} & { 66.63} & { 47.26} & { 19.37} & { 33.20} \\
\multicolumn{1}{c|}{{ Qwen2.5-14B}} & { 69.00} & { 51.41} & { 17.59} & \multicolumn{1}{l|}{{ 35.84}} & { 68.50} & { 45.20} & { 23.30} & \multicolumn{1}{l|}{{ 32.12}} & { 68.75} & { 48.30} & { 20.45} & { 33.98} \\
\multicolumn{1}{c|}{{ Qwen2.5-32B}} & { 69.75} & { 53.11} & { 16.64} & \multicolumn{1}{l|}{{ 37.54}} & { 70.00} & { 46.10} & { 23.90} & \multicolumn{1}{l|}{{ 32.68}} & { 69.88} & { 49.61} & { 20.27} & { 35.11} \\
\multicolumn{1}{c|}{{ Qwen2.5-72B}} & { \textbf{70.87}} & { \textbf{54.75}} & { 16.12} & \multicolumn{1}{l|}{{ \textbf{39.64}}} & { \textbf{72.00}} & { \textbf{47.75}} & { 24.25} & \multicolumn{1}{l|}{{\textbf{ 35.12}}} & { \textbf{71.44}} & { \textbf{51.25}} & {20.19} & { \textbf{37.38}} \\ \hdashline[0.5pt/5pt]
\multicolumn{1}{c|}{{ Baichuan2-7B}}& { 67.00} & { 46.16} & { 20.84} & \multicolumn{1}{l|}{{ 31.50}} & { 60.62} & { 39.04} & { 21.58} & \multicolumn{1}{l|}{{ 25.21}} & { 63.81} & { 42.60} & { 21.21} & { 28.36} \\
\multicolumn{1}{c|}{{ Baichua2-13B}}& { 69.13} & { 46.98} & { 22.15} & \multicolumn{1}{l|}{{ 33.45}} & { 64.62} & { 38.82} & { 25.80} & \multicolumn{1}{l|}{{ 26.07}} & { 66.88} & { 42.90} & { 23.97} & { 29.76} \\ \hdashline[0.5pt/5pt]
\multicolumn{1}{c|}{{ DeepSeek-7B}} & { 68.13} & { 47.96} & { 20.17} & \multicolumn{1}{l|}{{ 33.30}} & { 63.38} & { 40.39} & { 22.99} & \multicolumn{1}{l|}{{ 26.70}} & { 65.76} & { 44.18} & { 21.58} & { 30.00} \\
\multicolumn{1}{c|}{{ DeepSeek-67B}}& { 71.50} & { 49.21} & { 22.29} & \multicolumn{1}{l|}{{ 35.89}} & { 71.37} & { 40.63} & { 30.75} & \multicolumn{1}{l|}{{ 29.71}} & { 71.44} & { 44.92} & { 26.52} & { 32.80} \\ \hdashline[0.5pt/5pt]
\multicolumn{1}{c|}{{ InternLM2.5-1.8B}}& { 61.62} & { 42.07} & { 19.55} & \multicolumn{1}{l|}{{ 26.99}} & { 55.37} & { 38.46} & { 16.91} & \multicolumn{1}{l|}{{ 22.61}} & { 58.50} & { 40.27} & { 18.23} & { 24.80} \\
\multicolumn{1}{c|}{{ InternLM2.5-7B}}& { 67.25} & { 49.77} & { 17.48} & \multicolumn{1}{l|}{{ 34.57}} & { 69.50} & { 40.89} & { 28.61} & \multicolumn{1}{l|}{{ 29.75}} & { 68.38} & { 45.33} & { 23.04} & { 32.16} \\
\multicolumn{1}{c|}{{ InternLM2.5-20B}} & { 67.37} & { 48.08} & { 19.29} & \multicolumn{1}{l|}{{ 33.21}} & { 73.62} & { 41.11} & { 32.51} & \multicolumn{1}{l|}{{ 31.23}} & { 70.50} & { 44.60} & { 25.90} & { 32.22} \\ \hdashline[0.5pt/5pt]
\multicolumn{1}{c|}{{ Yi-1.5-6B}} & { 67.00} & { 49.59} & { 17.41} & \multicolumn{1}{l|}{{ 34.27}} & { 64.38} & { 39.37} & { 25.01} & \multicolumn{1}{l|}{{ 26.62}} & { 65.69} & { 44.48} & { 21.21} & { 30.45} \\
\multicolumn{1}{c|}{{ Yi-1.5-9B}} & { 68.50} & { 50.18} & { 18.32} & \multicolumn{1}{l|}{{ 35.55}} & { 66.37} & { 39.58} & { 26.79} & \multicolumn{1}{l|}{{ 27.48}} & { 67.44} & { 44.88} & { 22.56} & { 31.52} \\
\multicolumn{1}{c|}{{ Yi-1.5-34B}}& { 71.00} & { 52.23} & { 18.77} & \multicolumn{1}{l|}{{ 38.09}} & { 71.00} & { 40.75} & { 30.25} & \multicolumn{1}{l|}{{ 29.91}} & { 71.00} & { 46.49} & { 24.51} & { 34.00} \\ \hline
\multicolumn{13}{c}{\textit{English open-source LLMs}} \\ 
\multicolumn{1}{c|}{{ Llama3-8B}} & { 59.13} & { 46.62} & { 12.51} & \multicolumn{1}{l|}{{ 28.23}} & { 66.25} & { 40.21} & { 26.04} & \multicolumn{1}{l|}{{ 27.34}} & { 62.69} & { 43.42} & { 19.27} & { 27.79} \\
\multicolumn{1}{c|}{{ Llama3-70B}}& { 65.75} & { 48.63} & { 17.12} & \multicolumn{1}{l|}{{ 32.70}} & { \textbf{72.50}} & { 41.27} & { 31.23} & \multicolumn{1}{l|}{{\textbf{ 30.63}}} & {\textbf{ 69.13}} & { 44.95} & { 24.18} & { 31.67} \\ \hdashline[0.5pt/5pt]
\multicolumn{1}{c|}{{ Mistral-7B-v0.2}} & { 57.75} & { 46.25} & { \textbf{11.50}} & \multicolumn{1}{l|}{{ 27.57}} & { 67.50} & { \textbf{41.52}} & { 25.98} & \multicolumn{1}{l|}{{ 28.93}} & { 62.63} & { 43.88} & { 18.74} & { 28.25} \\
\multicolumn{1}{c|}{{ Mixtral-8x7B-v0.1}} & { 63.62} & { 46.80} & { 16.82} & \multicolumn{1}{l|}{{ 30.82}} & { 69.75} & { 41.21} & { 28.54} & \multicolumn{1}{l|}{{ 29.39}} & { 66.69} & { 44.01} & { 22.68} & { 30.11} \\
\multicolumn{1}{c|}{{ Mixtral-8x22B-v0.1}}& { 66.00} & {\textbf{ 50.73}} & { 15.27} & \multicolumn{1}{l|}{{ \textbf{34.32}}} & { 72.12} & { 41.25} & { 30.87} & \multicolumn{1}{l|}{{ 30.61}} & { 69.06} & { \textbf{45.99}} & { 23.07} & { \textbf{32.47}} \\ \hdashline[0.5pt/5pt]
\multicolumn{1}{c|}{{ Gemma-2-2B}}& { 61.88} & { 45.38} & { 16.51} & \multicolumn{1}{l|}{{ 29.02}} & { 59.62} & { 39.13} & { \textbf{20.50}} & \multicolumn{1}{l|}{{ 24.88}} & { 60.75} & { 42.25} & {\textbf{ 18.50}} & { 26.95} \\
\multicolumn{1}{c|}{{ Gemma-2-9B}}& { \textbf{69.13}} & { 46.75} & { 22.38} & \multicolumn{1}{l|}{{ 33.29}} & { 64.88} & { 39.80} & { 25.08} & \multicolumn{1}{l|}{{ 26.91}} & { 67.01} & { 43.28} & { 23.73} & { 30.10} \\
\multicolumn{1}{c|}{{ Gemma-2-27B}} & { 63.38} & { 48.52} & { 14.86} & \multicolumn{1}{l|}{{ 31.96}} & { 71.88} & { 40.91} & { 30.97} & \multicolumn{1}{l|}{{ 30.25}} & { 67.63} & { 44.71} & { 22.92} & { 31.11} \\ \hline
\end{tabular}
}
\caption{TODO: bolded is not result. Results of existing LLMs on our HellaSwag-Pro dataset using \textbf{Direct} prompt. ``AVG'' indicates the average performance of each model on Chinese and English parts of the dataset.
The best results for each metric in each model category are \textbf{bolded}. }
\label{tab:main experiment.}
\end{table*}

\subsection{Model Performance (RQ1)}
\paragraph{Overall Performance}
Table \ref{tab:main experiment.} provides a comprehensive evaluation of various LLMs across four performance metrics\footnote{The results of instruct/chat models of Qwen2.5, Llama3 and Mixtral latest series are shown in Appendix.}. The main observations are as follow:
\begin{itemize}[leftmargin=*,topsep=0pt]
% \setlength{}{0}
    \item Upon evaluating all available models, we found that all performed well in overall accuracy (e.g., GPT-4 scored 90.00 in AVG OA, Claude 3.5 scored 90.24 in AVG OA). However, all models struggled with variations of the questions, as evidenced by a positive RLA value for each model. In contrast, humans received a negative RLA value, suggesting that the question variants were not more challenging than the originals. This disparity further illustrates that current LLMs lack a true understanding of the reasoning process and can easily be misled by question variants.
    \item When comparing open-source and close-source models, the close-source models demonstrate stronger capabilities in both OA and ARA scores, similar to most existing benchmarks. Overall, the RLA values for close-source models are also smaller, indicating that they are more robust in commonsense reasoning tasks compared to open-source models.
    \item When we compare models within the same series (e.g., Qwen, Llama), we observe that larger models often achieve higher scores on OA, ARA, and CRA. However, they are also more susceptible to variations, i.e., they have higher RLA values, a phenomenon particularly evident in English datasets. We attribute this phenomenon to the fact that larger models, compared to smaller ones, may have memorized more data, allowing them to rely on memorization to solve some problems more easily and making them more prone to the influence of variations~\cite{}.
\end{itemize}
% 1. When evaluating all available models, We find although 
% 2. When comparing the opensource LLMs and close source LLMs, 
% 3. When looking into each serious details
% \noindent
% \textbf{Overall Model Performance.}
% 1. close-source > open-source 2. the large the better 3. all have a performance decline when meeting varients.

% To evaluate the performance of various models, we observed patterns consistent with current mainstream trends: closed-source models generally outperform open-source models across metrics. 
% For instance, the closed-source model GPT-4o achieved scores of 90.00 in OA, 76.07 in ARA, and 69.31 in CRA, whereas the open-source model Qwen2.5-72B scored 71.44, 51.25, and 37.38, respectively. 
% Furthermore, within each model series, performance tends to improve with larger model sizes. 
% Nevertheless, even the strongest closed-source models struggle with variations in questions, as indicated by positive values in RLA for all models. In contrast, human performance yields a negative RLA value, highlighting that current LLMs do not genuinely grasp the reasoning process and are prone to falling into traps set by question variants. 
% This suggests that there is still significant room for improvement in developing models that can robustly understand and reason through complex linguistic challenges.
% It reveals a consistent pattern across Chinese, English, and average scores, with close-sourced LLMs generally outperforming open-sourced models. 
% However, all models exhibit a significant drop in performance when faced with robust variants, as indicated by RLA and CRA. Among closed-source models, GPT-4o demonstrates the highest ARA of 76.07\% in average scores, demonstrating its overwhelming superiority. Among open-sourced models, larger models tend to perform better, with Qwen2.5-72B achieving the highest OA (71.44\%) and ARA (51.25\%) in the average scores. However, even these top performers still struggle with robustness, as evidenced by the substantial RLA of 13.93\% for GPT-4o and 20.19\% for Qwen2.5-72B. Interestingly, some English open-sourced models, such as Llama3-70B and Mixtral-8x22B-v0.1, show competitive performance in English tasks but lag in Chinese tasks, highlighting the importance of language-specific training.

% \noindent
% \textbf{Chinese Models vs English Models.}
% Chinese models generally demonstrate higher OA in Chinese tasks compared to English tasks, with Qwen-Max achieving 93.50\% OA in Chinese versus 87.60\% in English. Conversely, English models tend to perform better in English tasks, exemplified by Llama3-70B's 72.50\% OA in English compared to 65.75\% in Chinese. 
% However, both Chinese and English models exhibit important drops in ARA across languages, indicating challenges in maintaining performance when faced with variations. This trend suggests that while models may excel in their primary language, they struggle with robustness across linguistic boundaries. 
% Notably, larger models tend to achieve higher ARA scores but also experience more substantial RLA, as seen with Qwen2.5-0.5B (41.70\% ARA, 13.43\% RLA in total) and Qwen2.5-72B (51.25\% ARA, 20.19\% RLA in total). 
% This pattern indicates that while increased model size enhances overall performance, it doesn't necessarily improve robustness proportionally. 
% The discrepancy between OA and ARA across languages underscores the need for improved cross-lingual robustness in language models, particularly as they scale in size and capability.


% \noindent
% \textbf{Comparison between Chinese and English datasets.}
% Generally, models demonstrate higher accuracy on the Chinese dataset compared to the English one, as evidenced by the consistently higher OA, ARA and CRA scores. For instance, GPT-4o achieves an OA of 91.37\%, an ARA of 81.97\% , an CRA of 75.55\% on the Chinese dataset, compared to 88.63\% and 70.17\% respectively on the English dataset. This trend is observed across most models, suggesting that the Chinese dataset is easier than English one. Moreover, the RLA values are typically lower for Chinese, indicating smaller performance drops when dealing with robust variants of Chinese questions. For example, Qwen-Max shows an RLA of 8.68\% for Chinese versus 24.99\% for English, highlighting a more consistent performance in Chinese. The CRA scores further reinforce this observation, with models generally maintaining higher consistency in correct answers for both original and variant Chinese questions.
% We attribute this phenomenon to the fact that blablabla

\noindent
\textbf{Reasoning Transferable Capability.}
% 为了进一步
To further analyze whether the model can transfer reasoning ability from the original question to its variant, Figure \ref{consis} presents the distribution of model performance on the original question and variant pairs. For all models, the pairs of (HellaSwag \ding{51} HellaSwag-Pro \ding{55}) occupy a significant proportion, indicating a challenge in transferring reasoning capabilities for current LLMs to more complex scenarios. Looking deeply, closed-source models like GPT-4 and Qwen-Max achieve around a 69\% portion of (HellaSwag \ding{51} HellaSwag-Pro \ding{51}) and a 3\% portion of (HellaSwag \ding{55} HellaSwag-Pro \ding{55}), while in contrast, open-source models struggle with around a 30\% portion of (HellaSwag \ding{51} HellaSwag-Pro \ding{51}) and a 20\% portion of (HellaSwag \ding{55} HellaSwag-Pro \ding{55}), further indicating the robustness of reasoning abilities in closed-source models.
% If a model can get both the original question and the variant right, we consider it to have transferable reasoning ability. Table \ref{consis} presents the distribution of model performance on the original question and variant pairs. Among all models, the pairs of (HellaSwag \ding{51}HellaSwag-Pro \ding{55}) account for a considerable proportion, i 
% The closed-source models like GPT-4o and Qwen-Max achieve around 69\% portion of (HellaSwag \ding{51}HellaSwag-Pro \ding{51}) and 3\% portion of (HellaSwag \ding{55} HellaSwag-Pro \ding{55}), indicating stronger reasoning transfer ability than other models. In contrast, open-source models struggle more, with around 30\% portion of (HellaSwag \ding{51}HellaSwag-Pro \ding{51}) and 20\% portion of (HellaSwag \ding{55} HellaSwag-Pro \ding{55}). 
% A notable trend is observed among the Qwen2.5 series, where increasing model size from 7B to 72B parameters correlates with improved performance on correct answers for both datasets (33.20\% to 37.38\%) and decreased failure rates (17.69\% to 14.7\%). It underscores the importance of model size in commonsense reasoning tasks.

\begin{figure}[t]
\centering
\setlength{\abovecaptionskip}{0.1cm}
\setlength{\belowcaptionskip}{0cm}
\includegraphics[width=\linewidth,scale=1.00]{images/consis.pdf}
\caption{Analysis of the transferable ability of model reasoning based on question pair performance. The green part, where both the original and the variant data are right, represents the transferable performance of model reasoning.}
\label{consis}
\vspace{-15pt}
\end{figure}

\begin{figure*}[ht]
\centering
\setlength{\abovecaptionskip}{0.1cm}
\setlength{\belowcaptionskip}{0cm}
\includegraphics[width=\linewidth,scale=1.00]{images/xing.pdf}
\caption{The impact of different few-shot prompts on model performance. With - as the separator, the first two parts of the legend represent the prompt name, and the third part represents the language of the dataset.}
\label{xing}
\vspace{-15pt}
\end{figure*}

\begin{figure}[ht]
\centering
\setlength{\abovecaptionskip}{0.1cm}
\setlength{\belowcaptionskip}{0cm}
\includegraphics[width=1.05\linewidth,scale=1.05]{images/zhu.pdf}
\caption{The RLA Distribution for 7 variants of commonsense reasoning. Parts below the 0 axis indicate that the model’s performance on the variant is improved compared to the original problem.}
\label{fig:zhu}
\vspace{-15pt}
\end{figure}


\subsection{Variant Analysis (RQ2)}
To further analyze the impact of different variants, we assessed the contribution of each variant to the RLA score. A higher contribution indicates that the model is more likely to make errors in that type. Figure~\ref{fig:zhu} presents the overall results, and the key observations are as follows:
\begin{itemize}[leftmargin=*]
    \item For problem restatement, causal inference, and sentence ordering, these three categories are the least challenging. Almost all models, particularly the close-source and Qwen series models, perform well on these variants, indicating that current LLMs can effectively handle these forms and we do not pay more attention on this kind of varients.
    \item For reverse conversion and critical testing, these two varients each contribute about 10\% to the RLA score. This indicates that current LLMs struggle to fully generalize to these simple scenarios, possibly because these types of questions are not commonly encountered, and reaserchers should pay some attention to this type of varients.
    \item For negative transformation and scenario refinement, this are the two most difficult tasks, with negative transformation being particularly challenging. For almost all models, these two varients accounts for more than 50\% of the RLA score. This may be due to intuitively counterintuitive questions—such as the use of "will not"  or counterfactual scenarios in scenario refinement. These setups are less common in LLM training data and cannot be easily tackled through memory alone. Only those LLMs which truely understand the question could answer the varient correctly, wihch better reflect the true performance of the model.. In the future, researchers should focus more on enhancing LLM's capability to address such types of questions.
\end{itemize}

% 1. Problem restCausal Inference 
% To further analysis the impact of different varients, we further 
% Figure \ref{fig: zhu} presents a comprehensive analysis of various LLMs' performance across different variant types. Negative transformation emerges as the most challenging task for all models, with scores consistently above 50.00\% and peaking at 78.38\% for Gemini-1.5-Pro. Conversely, problem restatement appears to be the least challenging, with most models scoring in the negative range. Intriguingly, smaller models like Qwen2.5-0.5B demonstrate unexpected strengths in certain areas, such as sentence sorting (7.75\%), outperforming some larger counterparts. A detailed analysis of each variant type follows.

% \noindent
% \textbf{Causal inference.} In this category, scores vary widely from -4.73\% for Qwen-Max to 12.25\% for Baichuan2-13B, illustrating differing degrees of sensitivity to causal reasoning among the models. Smaller models, such as Qwen2.5-0.5B and Qwen2.5-1.5B, achieve better scores, indicating relatively stronger robustness in causal reasoning. Conversely, larger models, like Baichuan2-13B, have higher scores, suggesting greater sensitivity to the challenges of inferring causality.

% \noindent
% \textbf{Critical testing.} Larger models, including Qwen2.5-72B and DeepSeek-67B, exhibit higher RLA scores of 30.50\% and 31.37\%, respectively, suggesting increased sensitivity when dealing with incomplete key information. In contrast, GPT-4o achieves the lowest score, highlighting its superior robustness in critical reasoning. This trend indicates that more complex models might struggle to handle incomplete contexts, underscoring potential areas for improvement in sophisticated architectures.

% \noindent
% \textbf{Negative transformation.} This aspect remains consistently challenging for all models, with scores ranging from 48.88\% to 78.38\%. Advanced commercial models like Gemini-1.5-Pro and Claude-3.5 also score higher (78.38\% and 76.43\%, respectively), indicating a prevalent sensitivity issue in reasoning processes when handling negations, irrespective of model size or architecture.

% \noindent
% \textbf{Problem restatement.} The negative values in this category for nearly all models suggest it is not particularly challenging. This is surprising, given that previous models were quite sensitive to sentence representation.

% \noindent
% \textbf{Reverse conversion.} This variation, which involves swapping the roles of the question and answer, seems to specifically impact larger models. For example, Qwen2.5-72B and DeepSeek-67B exhibit higher RLA scores of 24.38\% and 27.43\%, respectively, indicating heightened sensitivity to reverse reasoning compared to their performance on original questions.

% \noindent
% \textbf{Scenario refinement.} The scores range from 16.06\% for Gemma-2-2B to 32.56\% for Qwen2.5-72B, with larger models displaying more sensitivity in adapting to counterfactual predictions. This suggests that larger models may rely more heavily on general commonsense rather than flexibly adapting to specific contexts. Consequently, increased model complexity might adversely affect adaptability to scenario changes, underscoring the need for enhanced flexibility in advanced models.

% \noindent
% \textbf{Sentence sorting.} This category exhibits the most varied results across models. Some larger models like DeepSeek-67B and InternLM2.5-20B display higher scores (26.69\% and 26.68\%), indicating sensitivity, while others like Qwen2.5-72B and Gemini-1.5-Pro excel with lower scores (-9.88\% and -1.07\%, respectively). This suggests that sentence sorting ability may depend more on specific training approaches rather than being solely contingent on model size.


\subsection{Prompt Robustness (RQ3)}
% To investigate how prompt  influence our benchmark, we apply sereral prompt strategy on our datasets and showcase the average performance of all models on different kind of prompt strategies.
% Table~\ref{prompt} illustrates the final results. For both Chinese and English datasets, CN LLMs achieve the highest performance using CN-CoT-Few-Shot, followed closely by EN-CoT-Few-Shot, with overall performance scores of 67.36\% and 67.03\%, respectively. In contrast, English LLMs perform best with the EN-CoT-Few-Shot, reaching 67.55\% on the Chinese dataset and 60.36\% on the English dataset.
% Contrary to previous findings, translating the dataset to the model's advantage language before performing reasoning does not enhance performance. Moreover, Figure~\ref{xing} also shows the similar phenomenon. Conducting CoT reasoning in the model’s advantage language generally leads to better outcomes compared to Direct. Additionally, increasing the number of shots consistently improves performance across most configurations, highlighting the benefits of exposing models to multiple examples. 
To explore the impact of various prompt strategies on our benchmarks, we evaluated several approaches across our datasets and present the average performance of all models using different prompting techniques. Table~\ref{prompt} summarizes the results. For both Chinese and English datasets, Chinese LLMs performed best with the CN-CoT-Few-Shot strategy, followed closely by EN-CoT-Few-Shot, achieving overall scores of 67.36\% and 67.03\%, respectively. Conversely, English LLMs showed optimal performance with the EN-CoT-Few-Shot approach, attaining 67.55\% on the Chinese dataset and 60.36\% on the English dataset.
Besides, translating datasets into the model's native language before reasoning did not enhance performance. This phenomenon is further illustrated in Figure~\ref{xing}. Conducting CoT reasoning in the model's native language generally yields better results compared to direct reasoning. Furthermore, increasing the number of examples (shots) consistently boosts performance across most configurations, emphasizing the advantages of exposing models to multiple examples.
% Overall, the interaction between question language, prompt language, and the number of shots underscores the importance of aligning these factors to optimize task performance and robustness in LLMs.



% Please add the following required packages to your document preamble:
% \usepackage{multirow}
% Please add the following required packages to your document preamble:
% \usepackage{multirow}
\begin{table}[t]
\setlength{\tabcolsep}{8pt}
% \footnotesize
\scalebox{0.65}{
\begin{tabular}{c|l|lll}
\hline
\multicolumn{1}{l|}{Dataset}  & Prompt  & CN LLMs & EN LLMs &  LLMs \\ \hline
\multirow{7}{*}{\begin{tabular}[c]{@{}c@{}}Chinese\\ HellaSwag-Pro\end{tabular}} & Direct  & 48.95& 41.16& 45.06  \\
& CN-CoT-Few  & \textbf{71.04}& 51.90& 61.47  \\
& EN-CoT-Few  & 70.95& \textbf{67.55}& \textbf{69.25}  \\
& EN-XLT-Few  & 41.48& 28.69& 35.09  \\
& CN-CoT-Zero & 44.82& 23.89& 34.36  \\
& EN-CoT-Zero & 45.38& 31.39& 38.39  \\
& EN-XLT-Zero & 28.57& 12.93& 20.75  \\ \hline
\multirow{7}{*}{\begin{tabular}[c]{@{}c@{}}English\\ HellaSwag-Pro\end{tabular}} & Direct  & 47.46& 40.66& 44.06  \\
& CN-CoT-Few  & \textbf{63.67}& 47.24& 55.46  \\
& EN-CoT-Few  & 63.12& \textbf{60.36}& \textbf{61.74}  \\
& CN-XLT-Few  & 48.77& 16.61& 32.69  \\
& CN-CoT-Zero & 34.89& 18.25& 26.57  \\
& EN-CoT-Zero & 42.41& 31.03& 36.72  \\
& CN-XLT-Zero & 16.36& 11.22& 13.79  \\ \hline
\multirow{9}{*}{HellaSwag-Pro}& Direct  & 48.21& 40.91& 44.83  \\
& CN-CoT-Few  & \textbf{67.36}& 49.57& 58.46  \\
& EN-CoT-Few  & 67.03& \textbf{63.95}& \textbf{65.49}  \\
& CN-XLT-Few  & 59.91& 34.26& 47.08  \\
& EN-XLT-Few  & 52.30& 44.52& 48.41  \\
& CN-CoT-Zero & 39.86& 21.07& 30.46  \\
& EN-CoT-Zero & 43.90& 31.21& 37.55  \\
& CN-XLT-Zero & 30.59& 17.55& 24.07  \\
& EN-XLT-Zero & 35.49& 21.98& 28.74  \\ \hline
\end{tabular}
}
\caption{Average ARA of all open-source models on different prompts. CN-LLMs contains 17 LLMs, and EN-LLMs contains 7 LLMs. The bast results for each dataset are \textbf{bolded}.}
\label{prompt}
\end{table}




% \section{Concluding Remarks}
In this paper, we proposed a novel approach utilizing multimodal LLMs to generate gesture-aware speech recognition transcripts for patients with language disorders. Our framework integrates verbal speech and iconic gestures, enabling the generation of enriched transcripts that capture the latent meaning conveyed through both modalities. Through extensive experimentation, we demonstrated that the proposed method effectively contextualizes incomplete or disfluent speech by incorporating gesture information, leading to more accurate and meaningful representations of the speaker's intent. These findings highlight the potential of our approach to significantly contribute to the field of speech and language therapy, offering innovative tools that can enhance the quality of life for individuals with language disorders by facilitating better communication and assessment methods.

\subsection{Ethical Statement} 
Our dataset was obtained from AphasiaBank with the approval of the Institutional Review Board (IRB) and adheres to the data sharing guidelines set by TalkBank\footnote{https://talkbank.org/share/ethics.html}. This includes complying with the Ground Rules for all TalkBank databases, which are based on the American Psychological Association Code of Ethics~\cite{american2002ethical}.

\subsection{Limitation \& Future Work} 
%This study represents a preliminary investigation into using multimodal LLMs to generate gesture-aware speech recognition transcripts. 
While the results are promising, we recognize several limitations and outline our plans to extend this work further.

One primary limitation is the absence of a definitive ground truth for quantitative evaluation. Since our model generates transcripts by synthesizing speech and gesture data from scratch, traditional benchmarks, such as comparisons with standard speech recognition outputs, are insufficient. Moreover, existing original transcripts lack gesture annotations, making direct comparisons challenging. In future work, we aim to address this gap by collaborating with certified pathologists to conduct qualitative assessments, such as A-B preference tests, to evaluate the effectiveness of gesture-enriched transcripts in accurately conveying the speaker's intentions.

To support quantitative evaluations, we plan to develop novel metrics that assess transcript quality, including grammar accuracy, semantic consistency, and the integration of multimodal information. Such metrics will provide a more objective basis for assessing our model's performance and facilitate comparisons with other multimodal and unimodal approaches.

Another limitation of this study is its focus on structured gestures from a specific task, the Peanut Butter Sandwich Task. While this task offers a controlled context for testing our approach, it does not encompass the diversity of gestures and communication patterns seen in everyday scenarios. As part of our future work, we plan to expand the scope of our model to include tasks such as the Cinderella Story Recall Task~\cite{bird1996cinderella}, which involves unstructured and complex narrative gestures. This expansion will allow us to evaluate the adaptability and robustness of our model in handling varied linguistic and gestural contexts.

In summary, while this study establishes a strong foundation for gesture-aware speech recognition, we aim to refine and extend our methods through collaborative qualitative evaluations, the development of robust quantitative metrics, and broader task applications. These efforts will ensure that our approach continues to evolve, ultimately contributing to more effective communication tools and interventions for individuals with language disorders.





\begin{abstract}
\vspace{-5pt}

% % >>>>>>>>>>>>>>>>>>>>>>>>>>>>>>>>>>>>>>>>>>>>>>
% % 1. What is AVS
% The task of Audio-Visual Segmentation (AVS) involves generating pixel-level segmentation maps of objects that produce sound in synchronization with visual frames. 
% % enabling more sophisticated, audio-driven scene understanding in dynamic environments. 
% % >>>>>>>>>>>>>>>>>>>>>>>>>>>>>>>>>>>>>>>>>>>>>>
% % 2. What kind of flaws we find in current SOTA methods
% % Although recent advances in state-of-the-art (SOTA) AVS models have shown promise, they still face critical limitations: (i) an over-reliance on visual cues, reducing their capacity for robust audio-visual integration, and (ii) evaluation methods that largely focus on "positive" scenarios where sounds align with visible objects, neglecting cases where sounds are unrelated to the visual scene.
% While existing AVS methods have shown promising results, they exhibit a fundamental bias: generating segmentation masks primarily based on visual salience regardless of audio context, leading to unreliable predictions when sounds are absent or irrelevant.
% % , or originate from off-screen sources.
% % >>>>>>>>>>>>>>>>>>>>>>>>>>>>>>>>>>>>>>>>>>>>>>
% % 3. In this paper, what did we proposed
% % To address these challenges, we propose an expanded evaluation framework that includes a diverse set of "negative" audio samples, such as silence, ambient noise, and off-screen sounds. This framework introduces new metrics for assessing how well AVS models perform when visual and audio cues do not align, a crucial aspect for real-world applicability. Additionally, we design novel loss functions and a gating mechanism to improve model training, allowing the model to more accurately distinguish between relevant and irrelevant audio-visual pairs.
% % This benchmark enables systematic evaluation of AVS models' ability to suppress false predictions when no valid audio-visual correspondence exists.
% To address this, we introduce AVSBench-Robust, a comprehensive benchmark incorporating diverse negative audio scenarios including silence, ambient noise, and off-screen sounds. Additionally, we propose a simple yet effective approach that combines balanced training with negative samples with BCE-guided similarity learning.
% % and Binary Cross-Entropy (BCE) guided similarity learning to explicitly teach models when segmentation should or should not occur.
% % >>>>>>>>>>>>>>>>>>>>>>>>>>>>>>>>>>>>>>>>>>>>>>
% % 4. What we found.
% % Our experiments reveal that current SOTA models frequently misinterpret diverse audio cues due to their dependence on visual information. In contrast, our approach demonstrates enhanced resilience and accuracy, particularly in complex audio-visual scenarios, with a significant reduction in false positives in unrelated audio contexts. By broadening AVS evaluation standards and offering refined training mechanisms, this work advances AVS towards more robust and realistic applications in audio-driven segmentation tasks.
% Our extensive experiments reveal that state-of-the-art AVS methods consistently fail under negative audio conditions, demonstrating the prevalence of visual bias. In contrast, our approach achieves remarkable improvements in both standard metrics and robustness measures, maintaining near-perfect false positive rates while preserving high-quality segmentation performance. These results highlight the importance of addressing the bias problem in AVS and demonstrate the effectiveness of our solution in enabling more reliable audio-visual segmentation for real-world applications.

% \jia{Another outline: (1) What is AVS, (2) We propose a new benchmark, (3) Observed that all the SOTA methods faild, (4) Our approach}

%Audio-Visual Segmentation (AVS) is the task of generating pixel-level segmentation maps for objects that produce sound in sync with visual frames. 
Unlike traditional visual segmentation, audio-visual segmentation (AVS) requires the model not only to identify and segment objects but also to determine whether they are sound sources.
Recent AVS approaches, leveraging transformer architectures and powerful foundation models like SAM, have achieved impressive performance on standard benchmarks. Yet, an important question remains: Do these models genuinely integrate audio-visual cues to segment sounding objects?
In this paper, we systematically investigate this issue in the context of robust AVS. Our study reveals a fundamental bias in current methods: they tend to generate segmentation masks based predominantly on visual salience, irrespective of the audio context. This bias results in unreliable predictions when sounds are absent or irrelevant.
To address this challenge, we introduce AVSBench-Robust, a comprehensive benchmark incorporating diverse negative audio scenarios including silence, ambient noise, and off-screen sounds. We also propose a simple yet effective approach combining balanced training with negative samples and classifier-guided similarity learning.
Our extensive experiments show that state-of-the-art AVS methods consistently fail under negative audio conditions, demonstrating the prevalence of visual bias. In contrast, our approach achieves remarkable improvements in both standard metrics and robustness measures, maintaining near-perfect false positive rates while preserving high-quality segmentation performance. %These results highlight the importance of addressing the bias problem in AVS and demonstrate the effectiveness of our solution in enabling more robust AVS for real-world applications. 
%\yapeng{Feel free to edit this new version -- if it is too long, we could remove the last sentence.}

\vspace{-3mm}

\end{abstract}

\section{Introduction}
\label{sec:intro}

\begin{figure}[h]
    \centering
    \includegraphics[width=0.48\textwidth]{fig1.pdf}
    \vspace{-7mm}
    \caption{\textbf{Performance in Different Audio Scenarios.} The top row shows an ambulance image under different audio conditions: \textit{Ambulance sound} (positive), \textit{Silence}, \textit{Noise}, and \textit{Offscreen sounds} (negative). Each subsequent row displays the segmentation output various SOTA AVS models~\cite{zhou2022audio, gao2024avsegformer, chen2024cavp} and our model under each audio condition. In negative scenarios, existing models segment the ambulance due to ``visual prior" bias, mistakenly associating it with unrelated audio. In contrast, our model accurately segments only in the presence of relevant audio, demonstrating improved alignment between audio cues and visual segmentation.
    }
    \vspace{-7mm}

    % \caption{Limitations of Current AVS Models in Negative Audio Scenarios: The top row shows an input image featuring an ambulance. Each column represents different audio conditions: \textit{Ambulance sound}, \textit{Silence}, \textit{Noise}, and \textit{Offscreen sounds}. The last three conditions are examples of "negative" audio samples, where the sound not align with visible objects. Notably, existing models often highlight the ambulance regardless of whether the sound aligns with it (e.g., in silence or noise cases), showing a "visual prior" bias. In contrast, our approach shows improved segmentation by not relying solely on visual presence and offering more accurate alignment with relevant audio cues.}
    \label{fig:fig1}
\end{figure}


% >>>>>>>>>>>>>>>>>>>>>>>>>>>>>>>>>>>>>>>>>>>
% 1. task(What can it do? Why do we need this?)
Audio-Visual Segmentation (AVS) aims to identify and segment sounding objects within visual scenes~\cite{zhou2022audio,gao2024avsegformer}. This essential multimodal task mirrors a fundamental aspect of human perception: the integration of auditory and visual stimuli to focus attention on relevant sources~\cite{zhou2022audio, small2005odor, chen2020vggsound}. For instance, when hearing a baby cry, people naturally locate the sound’s visual source.  Simulating this ability in machines could open up valuable cross-modal applications, such as improved multimedia analysis~\cite{arandjelovic2017look, arandjelovic2018objects, hu2022mix, mo2024multi}, enhanced human-computer interaction~\cite{yang2024analyzing, wang2024audiobench, johansen2022characterising, lv2022deep, fu2021design}, and autonomous systems capable of interpreting sound-emitting objects in complex environments~\cite{schmidt2020acoustic, dutt2020self, topcu2020assured}. 

% >>>>>>>>>>>>>>>>>>>>>>>>>>>>>>>>>>>>>>>>>>>
% 2. state-of-the-art(Briefly cite recent work (direct competition), )
Recent years have witnessed remarkable progress in AVS.  State-of-the-art (SOTA) methods leverage multimodal information, utilizing encoder-decoder structures with audio-visual interaction~\cite{zhou2022audio}, multimodal transformer architectures~\cite{gao2024avsegformer,li2024qdformer,liu2023AuTR}, audio query-guided designs~\cite{liu2023AuTR, sun2024biasinAVS}, and strong vision foundation models~\cite{mo2023av,liu2024annofree, wang2024GAVS, sun2024biasinAVS} like SAM~\cite{kirillov2023sam}and Mask2Former~\cite{cheng2022mask2former}. 
These innovations have driven impressive performance on standard benchmark datasets: AVSBench-S4 and AVSBench-MS3 ~\cite{zhou2022audio}.

%These innovations have not only driven impressive performance on benchmark datasets~\cite{xxx} but also extended AVS to weakly-supervised~\cite{xxx} and open-vocabulary~\cite{xxx} learning settings, significantly broadening its applicability. \ytian{add refs; feel free to edit}


% Significant progress has architures been made in AVS through the development of state-of-the-art (SOTA) methods, which combine audio and visual features at different processing stages to pinpoint sound-emitting objects by integrating information across both modalities, or leveraged large visual foundation models \cite{zhou2022audio, gao2024avsegformer, li2023catr, chen2024cavp, liu2024annofree, ma2024stepping, wang2024GAVS, guo2024open, sun2024biasinAVS, li2024qdformer, yang2024combo, huang2023aqformer, ling2023hear2seg, liu2023AuTR, wang2024ref}, offering a promising degree of generalization and precision across both single-source and multi-source audio-visual tasks.

% which fall into two main categories\cite{sun2024biasinAVS}: fusion-based\cite{zhou2022audio} and prompt-based approaches\cite{gao2024avsegformer, chen2024cavp}. Fusion-based methods combine audio and visual features at different processing stages to pinpoint sound-emitting objects by integrating information across both modalities. For instance, AVSBench\cite{zhou2022audio} achieves high accuracy in aligning sounds and visual features by utilizing temporal and multi-scale data, improving model performance in dynamic scenes. Recently, prompt-based models like AVSegFormer\cite{gao2024avsegformer} and GAVS\cite{gao2024avsegformer} have leveraged large visual foundation models, using tailored audio prompts to directly query visual features. These models excel in zero-shot and few-shot learning scenarios, offering a promising degree of generalization and precision across both single-source and multi-source audio-visual tasks.


% >>>>>>>>>>>>>>>>>>>>>>>>>>>>>>>>>>>>>>>>>>>
% 3. flaw in state-of-the-art(What can’t we do yet? Why should the reader care? No, really? Why can’t X, Y or Z be used to solve this? Fig 1)
% 4. your idea/solution (Keep it brief, )
However, a critical question arises: \textit{are these models truly performing audio-visual segmentation, or simply conducting visual segmentation with minimal audio integration?}  AVS, by definition, introduces a crucial constraint: only objects acting as sound sources should be segmented. For example, an AVS model should not segment a visually salient yet silent dog.  Current AVS models, primarily trained and evaluated on ``positive'' cases where visual objects correspond to audio cues, often neglect scenarios with unrelated sounds, such as silence or off-screen sources.

%AVS extends beyond the traditional object segmentation by introducing a crucial constraint: only objects acting as sound sources should be segmented. For example, an AVS model should not segment a visually salient yet silent dog in a video. However, the current AVS models are primarily trained and evaluated on ``positive'' cases where visual objects correspond to audio cues, neglecting scenarios with unrelated sounds, such as silence or off-screen sources. This raises a critical question: \textit{are these models truly performing audio-visual segmentation, or simply conducting visual segmentation with minimal audio integration?}


% \yapeng{better to include some statistics about the new benchmark} 
To systematically investigate whether AVS models truly integrate audio-visual information, we introduce AVS-Robust, a comprehensive benchmark comprising 4,932 single-source and 424 multi-source videos across 20 diverse object classes from AVSBench~\cite{zhou2022audio}. We incorporate four different audio conditions for each video: original audio, silence, ambient noise, and off-screen sounds. Each condition represents 25\% of the evaluation scenarios. Our study reveals a concerning bias in existing SOTA methods: they tend to generate segmentation masks based primarily on visual salience, irrespective of the audio context. For instance, these models may segment an ambulance even in the presence of silence or unrelated ambient sounds, indicating an over-reliance on visual cues rather than genuine audio-visual integration (Fig.~\ref{fig:fig1}).

% However, do these methods truly integrate audio-visual cues to segment sounding objects?
% To address this gap, we systematically investigate this issue in the context of robust AVS. We introduce AVS-Robust, a comprehensive benchmark that includes diverse negative audio scenarios to challenge AVS models under conditions where audio cues do not align with visual information.

%To address this critical gap in AVS evaluation, we introduce the AVS-Robust Benchmark, a comprehensive benchmark that includes diverse negative audio scenarios. We conducted a systematic study of AVS in multiple sound environments and reveals a concerning bias in current approaches - the existing SOTA methods tend to generate segmentation masks based predominantly on visual salience, regardless of the audio context. For example, existing models will segment an ambulance even when presented with silence or unrelated ambient sounds, indicating an over-reliance on visual cues rather than true audio-visual integration (Fig \ref{fig:fig1}). 

% Building upon these insights, we introduce negative audio-visual pairs into training and propose a balanced training approach that incorporates both matched and mismatched audio-visual pairs, utilizing classifier guidance for feature alignment and segmentation loss to ensure accurate segmentation. 
%\yapeng{This version is okay but not great. Can we highlight that simply adding negative pairs does not work? We should also highlight the significance of the proposed approach. We mainly motivated why this research but did not clearly motivate the proposed approach}
Building upon these insights, we explore solutions to address this visual bias. While incorporating negative audio-visual pairs during training seems intuitive, this approach alone presents a challenge: without explicit guidance for audio-visual integration, models struggle to determine whether to segment objects based solely on visual information. To overcome this, we propose a debiasing approach with two key components:
\textit{(1) Balanced Training with Negative Samples:} Incorporating both positive and negative audio-visual pairs during training to expose models to a wider range of audio-visual relationships. \textit{(2) Classifier-Guided Similarity Learning:} Utilizing a classifier to guide the model in learning effective audio-visual feature representations and promoting similarity between corresponding audio and visual features.

% >>>>>>>>>>>>>>>>>>>>>>>>>>>>>>>>>>>>>>>>>>>
% 5. proof it works(Say a bit about evaluation and baselines, our method lowers reconstruction error by xx% compared to previous methods)
Extensive experiments using our new benchmark yield several crucial findings. Recent SOTA methods, including SAMA-AVS \cite{liu2024annofree}, Stepping-Stones \cite{ma2024stepping}, and CAVP \cite{chen2024cavp}, consistently fail under negative audio conditions, exhibiting high False Positive Rates (FPR). When evaluated with our comprehensive metrics—such as G-mIoU, G-F, and G-FPR, as discussed in Sec.~\ref{sec:problem_benchmark}—these models show significant performance degradation compared to their reported results on standard benchmarks. In contrast, our approach achieves superior performance across all robustness metrics while maintaining competitive segmentation quality on positive audio inputs in both single- and multi-source scenarios.
% However, current AVS models face significant limitations in real-world audio-visual scenarios. Most AVS models are evaluated only on positive cases—where visible objects align with audio cues—while excluding scenarios where sounds are unrelated to visual objects, such as silence, ambient noise, or off-screen sounds. This limitation results in an incomplete assessment of model performance, potentially overestimating their audio-visual alignment capability. Although some methods consider off-screen sounds, they lack dedicated metrics to evaluate such cases, providing limited insight into model behavior in negative audio scenarios. As shown in Figure \ref{fig:fig1}, current models segment an ambulance even in negative audio conditions, such as silence or off-screen sounds, due to an over-reliance on visual information alone
% To address these gaps, we propose a two-part solution. First, we introduce a balanced training strategy that includes both matched (positive) and mismatched (negative) audio-visual pairs. This strategy employs Binary Cross-Entropy (BCE) loss to align audio and visual features, while Segmentation Loss ensures accurate segmentation only when audio-visual correspondence exists. Second, we introduce an expanded evaluation framework incorporating negative audio cases, such as silence, ambient noise, and off-screen sounds. This comprehensive benchmark enables a more accurate assessment of AVS models, providing a closer reflection of real-world audio-visual interactions.

Our main contributions are summarized as follows:
\begin{itemize}    
    \item  We conduct a systematic study on audio robustness in AVS and introduce AVSBench-Robust along with our new robustness evaluation protocols. This benchmark rigorously evaluates AVS models under both standard conditions and challenging negative scenarios, assessing their ability to effectively integrate audio-visual information.
    \item We propose a training strategy for robust AVS by incorporating diverse negative audio scenarios and employing classifier-guided similarity learning, which enhances model robustness and preserves segmentation quality.
    \item Extensive experiments demonstrate that our approach substantially outperforms current SOTA methods in terms of our robustness metrics while achieving competitive performance on standard AVS benchmarks. %Our results establish new benchmarks for robust audio-visual segmentation and provide insights for future research in this direction.
\end{itemize}


% \begin{itemize}
%     \item We propose AVSBench-Robust, a novel evaluation benchmark encompassing a broad range of audio scenarios, designed to assess both the segmentation performance on paired audio-video inputs and the model’s robustness when audio and video are misaligned.
%     \item
%     We introduce a training strategy that  balances positive and negative audio scenarios, incorporating BCE-guided similarity learning to help models effectively distinguish valid from invalid audio-visual correspondences while maintaining hight segmentation quality.
%     \item We present extensive empirical evidence showing our approach's robustness and precision in handling complex audio-visual scenarios. 
% \end{itemize}
% \item We develop a novel training framework that addresses the visual bias problem in AVS through two key components: (1) balanced sampling of positive and negative audio-visual pairs, and (2) classifier-guided similarity learning that explicitly enforces audio-visual correspondence. This approach significantly improves model robustness while maintaining high-quality segmentation performance.
% \item Through extensive experiments on both our benchmark and existing datasets, we demonstrate that our approach substantially outperforms current state-of-the-art methods in terms of robustness metrics (FPR, G-mIoU, G-F) while achieving competitive performance on standard AVS tasks. Our results establish new benchmarks for robust audio-visual segmentation and provide insights for future research in this direction.




% Together, these contributions advance AVS towards more robust, real-world applications, supporting next-generation models in achieving nuanced, human-like multi-modal perception.

% Despite recent progress, existing Audio-Visual Segmentation (AVS) methods face significant limitations in handling real-world audio-visual scenarios. Most AVS models are evaluated only on positive audio cases—where sound-producing objects are visible in the scene. This limited scope risks drawing inaccurate conclusions about a model’s ability to interpret audio, as it fails to address cases involving silent objects or offscreen sounds. Although some methods\cite{sun2024biasinAVS, li2024qdformer, chen2024cavp} consider offscreen sounds, they lack metrics for separate evaluation, leaving the model’s behavior on negative sounds largely unexamined.

\section{Related work}
\label{sec:related_new}
\vspace{-4pt}
% >>>>>>>>>>>>>>>>>>>>>>>>>>>>>>>>>>>>>>>>>>>
% Organization
% >>>>>>>>>>>>>>>>>>>>>>>>>>>>>>>>>>>>>>>>>>>
% In this section, we review the current state of research in three key areas relevant to our study: Sound Source Localization (SSL)(\ref{subsec:ssl}), Audio-Visual Segmentation (AVS) \wj{maybe we don't need this in related work because every familiry with this graph.}(\ref{subsec:avs_related}), and Imbalanced Multimodal Learning(\ref{subsec:imbalanced_multimodal_learning}). 
% Each area contributes differently to our understanding of multimodal interactions and challenges in audio-visual processing, and we highlight both the advancements and limitations that frame the context for our proposed approach.


% >>>>>>>>>>>>>>>>>>>>>>>>>>>>>>>>>>>>>>>>>>>
% Sound Localization
% >>>>>>>>>>>>>>>>>>>>>>>>>>>>>>>>>>>>>>>>>>>
\textbf{Sound Source Localization.}
% 1. --- Define Task & Early Work
This task is closely related to AVS, focusing on localizing sound sources within visual scenes \cite{arandjelovic2018objects, senocak2018learning,mo2022closer, chen2021localizing,mahmud2024t}. This task advances cross-modal understanding through various technical approaches, from basic feature fusion strategies to sophisticated attention mechanisms \cite{senocak2018learning, mo2022closer, hu2020discriminative, qian2020multiple}.
% 2. --- Recent advances
Recent sound source localization approaches have significantly improved sound source discrimination through multiple innovations: contrastive learning with hard-mining strategies enhances complex region distinction~\cite{chen2021localizing, hu2020discriminative, mo2022closer}, while class-aware approaches and dual-phase feature alignment enable robust multi-source localization without explicit pairwise annotations \cite{hu2020discriminative, qian2020multiple, chen2021localizing}.
However, the predicted sounding object heatmaps lack the fine-grained precision offered by AVS's pixel-level segmentation capabilities.


% >>>>>>>>>>>>>>>>>>>>>>>>>>>>>>>>>>>>>>>>>>>
% AVS
% >>>>>>>>>>>>>>>>>>>>>>>>>>>>>>>>>>>>>>>>>>>
\vspace{2mm}
\noindent
\textbf{Audio-Visual Segmentation.}
% \label{subsec:avs_related}
% --- 1. Define Task & Early Work
AVS task focuses on identifying and segmenting sound-producing objects through pixel-level mask prediction. 
% --- 2. Recent advances
The field has progressed significantly since its inception~\cite{zhou2022audio,gao2024avsegformer, li2023catr, chen2024cavp, liu2024annofree, ma2024stepping, wang2024GAVS, guo2024open, sun2024biasinAVS}. Most approaches follow an encoder-decoder design, with early works focusing on effective fusion strategies for audio-visual information~\cite{zhou2022audio}. Subsequent developments explored more sophisticated architectures, incorporating multimodal transformers and audio query guided mechanisms~\cite{gao2024avsegformer, liu2023AuTR, sun2024biasinAVS, wang2024ref, li2023catr, ma2024stepping, yang2024combo} to enhance cross-modal understanding. Recently, methods leveraging vision foundation models~\cite{liu2024annofree, wang2024GAVS, sun2024biasinAVS} and LLMs~\cite{wang2024can, he2024mlmseg}  have demonstrated improved segmentation capabilities. 

% The integration of large language models has further extend AVS to open-vocabulary scenarios~\cite{, }, enabling more flexible applications.
% Recent approaches have advanced AVS capabilities through various technical innovations, such as fusion-based architectures \cite{zhou2022audio, huang2023aqformer, li2023catr, ling2023hear2seg, liu2023AuTR} and prompt-based methods \cite{gao2024avsegformer, liu2024annofree, ma2024stepping, wang2024GAVS, wang2024ref} \yapeng{make it consistent with the descriptions in intro}. Some advances have enabled more sophisticated applications, from multi-source segmentation to open-vocabulary scenarios~\cite{wang2024GAVS, guo2024open}.

% --- 3. Bias Problems & Existing Solutions
Despite these developments, we observe that AVS models commonly suffer from ``visual prior'' bias, where models generate predictions primarily based on visual salience regardless of audio context~\cite{sun2024biasinAVS, chen2024cavp, li2024qdformer}. While recent efforts address this through contrastive learning~\cite{chen2024cavp} and semantic enhancement~\cite{sun2024biasinAVS, li2024qdformer}, they still lead to disregard audio features (see Fig.~\ref{fig:fig1}) or face limited evaluation scope. 
% --- 4. Our Approach

% In contrast to these complex architectural solutions, our approach offers a simple yet effective strategy that can be integrated into existing AVS models toward more robust AVS. 
In contrast to complex architectural solutions, we adopt a straightforward yet effective approach to enhancing AVS model robustness. We systematically analyze audio robustness in AVS and introduce AVSBench-Robust, a benchmark designed to evaluate models under both original audio conditions and challenging negative audio scenarios. 
A concurrent study \cite{juanola2024critical} identified visual bias in SSL models, introducing the same negative audio scenarios along with evaluation metrics, but primarily focuses on assessing model performance.
In contrast, our work not only evaluates robustness but also presents a targeted training strategy to strengthen AVS models against misleading audio-visual cues.
% However, AVS systems commonly exhibit problematic biases, particularly "visual prior" where models over-rely on visual features while disregarding audio context \cite{sun2024biasinAVS, chen2024cavp, li2024qdformer}. While recent works attempt to address these limitations through contrastive learning \cite{chen2024cavp} and semantic feature enhancement \cite{sun2024biasinAVS, li2024qdformer}, Due to CAVP\cite{chen2024cavp} heavy reliance on negative samples in the contrastive learning process inadvertently, the model to disregard audio features entirely, resulting in predictions that remain largely unaffected by audio variations. (see figure \ref{fig:fig1}), while other approaches \cite{sun2024biasinAVS, li2024qdformer} face limited evaluation scope due to restricted implementation availability.

% >>>>>>>>>>>>>>>>>>>>>>>>>>>>>>>>>>>>>>>>>>>
% Imbalanced Multimodal Learning
% >>>>>>>>>>>>>>>>>>>>>>>>>>>>>>>>>>>>>>>>>>>
\vspace{2mm}
\noindent
\textbf{Imbalanced Multimodal Learning.}
%\label{subsec:imbalanced_multimodal_learning}
Recent studies in audio-visual learning highlight significant challenges in balancing different modalities during training, where dominant modalities often overshadow others in the learning process \cite{wang2020mmlhard,tian2020unified,wei2024mmpareto, peng2022balanced}. Various solutions such as modality-specific optimization, gradient modulation, and Pareto optimization~\cite{wang2020mmlhard, wei2024mmpareto, peng2022balanced} have been proposed to address this imbalance, aiming to preserve the contribution of each modality. However, these approaches primarily target tasks where separate losses are applied for joint modalities and each individual modality ~\cite{wang2020mmlhard, wei2024mmpareto}. In contrast, the AVS task typically involves applying a single loss after fusing audio and visual features, which requires us to design a new strategy for effective modality balancing in AVS.

\section{Problem and Benchmark}
\label{sec:problem_benchmark}

% AVS aims to identify and segment objects in video frames that correspond to the accompanying audio signals. Given a dataset consisting of video frames, their corresponding audio signals, and ground truth segmentation masks, the goal is to learn a mapping function that can accurately localize and segment sound-producing objects in videos.
% In this section, we first formulate the AVS task and highlight its two-fold nature, then analyze core challenges in current approaches, and finally introduce our benchmark for robust AVS evaluation.

In this section, we first present the formulation of the AVS task and its associated challenges in Sec.~\ref{subsec:task}. In Sec.~\ref{subsec:benchmark}, we introduce our new benchmark for robust AVS. Finally, we present our evaluation protocols in Sec.~\ref{subsec:eval}.

% ==============================================
% 1. AVS Task Formulation
\subsection{Task and Challenges}
\label{subsec:task}
Given \( T \) non-overlapping video and audio clips \( \{V^t, A^t\}_{t=1}^{T} \), the goal of the AVS task is to predict a segmentation mask \( \mathcal{M_{\text{pred}}}^t \in \mathbb{R}^{H \times W} \) that labels sounding pixels in each video frame of the clips, where \( H \) and \( W \) denote the frame dimensions, and the mask is binary. Following previous studies~\cite{zhou2022audio,gao2024avsegformer, li2023catr}, we extract a single video frame at the end of each second and set \( T = 5 \) in practice, so each clip contains only one extracted frame. 

Unlike purely visual segmentation, AVS inherently addresses two subtasks simultaneously: segmenting visual objects and determining whether they are sound sources. Therefore, predictions should satisfy two key requirements: (1) when objects are producing sound, the model should generate accurate segmentation masks for those objects, and (2) when no audio-visual correspondence exists, the model should produce empty masks to avoid false predictions.


 

%Given a dataset $\mathcal{D} = \{V^i, A^i, \mathcal{M}_{gt}^i\}_{i=1}^n$ consisting of triplets of video frames $V^i \in V$, audio clips $A^i \in A$, and ground truth masks $\mathcal{M}_{gt}^i \in \mathcal{M}_{gt}$, the AVS task aims to learn a mapping $\mathcal{G}(V, A) \rightarrow \mathcal{M}_{gt}$ that generates pixel-wise segmentation masks for sound-producing objects. This mapping should satisfy two key requirements: (1) when objects are producing sound, generate accurate segmentation masks corresponding to those objects, and (2) when no valid audio-visual correspondence exists, produce empty masks to avoid false predictions.

% We evaluate our approach on two benchmark settings: (1) AVSBench-S4: a semi-supervised single sound source segmentation task where only the first frame of each video is labeled during training, while requiring predictions for all frames during evaluation, and (2) AVSBench-MS3: a fully-supervised multiple sound source segmentation task where ground truth masks for all frames are available during training.


% ==============================================
% 2. AVS bias Problem
% \subsection{Core Challenges}  
While current AVS approaches have shown promising segmentation results on standard benchmark protocols, they face a fundamental limitation: evaluations have primarily focused on positive cases where audio and visual signals fully align, with salient objects in video frames typically being sound sources in the datasets. This setup allows AVS models to potentially rely solely on visual information to achieve high performance, bypassing true multimodal integration.
This focus on positive cases overlooks the equally important ability to suppress predictions when no valid audio-visual correspondence exists. To comprehensively assess the multimodal learning capabilities of AVS models, a more robust benchmark is needed.

%This incomplete evaluation masks a critical bias in existing models—they consistently predict sound-producing regions regardless of audio context, leading to three common issues: (1) Silent Object Bias, where models segment visually prominent objects even in silence; (2) Background Noise Bias; and (3) Irrelevant Sound Bias, where irrelevant sounds are incorrectly attributed to visible objects (see fig \ref{fig:fig1}). 

% A critical limitation in existing AVS approaches is their inherent bias towards always predicting some sound-producing regions, even when no corresponding audio signal is present. This bias manifests in two problematic scenarios:
% (1) Silent Object Bias: Models tend to generate segmentation masks for visually prominent objects even during moments of silence, failing to recognize that visible objects may not always produce sound.
% (2) Background Noise Bias: When presented with background noise or irrelevant sounds, models often incorrectly attribute these sounds to visible objects in the frame, leading to false positive segmentations.

% This bias problem fundamentally challenges the reliability of current AVS systems in real-world applications, where distinguishing between sound-producing and silent objects is crucial for accurate scene understanding.

\begin{table}[t]
\centering
\footnotesize
\vspace{-5mm}
\renewcommand{\arraystretch}{1.1}  % Increases row height
\setlength{\tabcolsep}{10pt}  % Adjusts horizontal padding

\begin{tabular}{>{\raggedright\arraybackslash}m{0.15\linewidth}|>{\raggedright\arraybackslash}m{0.65\linewidth}}
\hline
\textbf{Category} & \textbf{Representative  Examples} \\
\hline
\faMusic \textbf{ Music} & Guitar, Violin, Piano, Tabla, Marimba, Ukulele, Playing Acoustic Guitar, Playing Glockenspiel, Playing Violin, Playing Ukulele \\
\hline
\faChild \textbf{ Human Voice} & Male Speech, Female Speech, Male Singing, Female Singing, Baby Crying, Baby Laughter \\
\hline
\faPaw \textbf{ Animals} & Dog Barking, Lion Roaring, Cat Meowing, Bird Chirping, Wolf Howling, Horse Neighing, Coyote Howling, Mynah Bird Singing \\
\hline
\faCar \textbf{ Devices, Machines} & Helicopter, Ambulance Siren, Car Horn, Lawn Mower, Chainsaw, Bus Engine, Typing on Computer Keyboard, Cap Gun Shooting, Emergency Car, Driving Buses, Race Car \\
\hline
\end{tabular}
\caption{Semantic Categories and Examples in AVSBench-Robust: To ensure clear evaluation of cross-modal understanding, we organize sounds into distinct semantic categories, which particularly crucial for the off-screen audio condition.}
\label{tab:sound-categories}
\vspace{-5mm} 

\end{table}
% ==============================================
% 3. Our Benchmark
\subsection{AVSBench-Robust}  
\label{subsec:benchmark}
% To advance the development of more robust AVS models, we introduce AVSBench-Robust, a comprehensive framework that fundamentally extends the original AVSBench dataset\cite{zhou2022audio}. Rather than focusing solely on aligned audio-visual pairs, our benchmark introduces three challenging negative audio conditions that models must handle appropriately:

To address the limitations of current evaluation frameworks and facilitate the development of more robust AVS models, we introduce {AVSBench-Robust}. Building upon AVSBench~\cite{zhou2022audio}, this benchmark includes two evaluation scenarios: (1) the single-source subset (S4), containing 4,932 videos (3,452 for training, 740 for validation, and 740 for testing), and (2) the multi-source subset (MS3), comprising 424 videos (296 for training, 64 for validation, and 64 for testing). For each video from AVSBench~\cite{zhou2022audio}, we create three additional negative audio conditions alongside the original positive audio, effectively quadrupling the number of audio-visual pairs for a comprehensive evaluation.

The benchmark spans 20 diverse sound-producing object classes across four major categories: machine (32.2\%), music (32.1\%), animal (23.2\%), and human (12.5\%). To evaluate model robustness, each video is paired with four types of audio conditions:

\noindent
\underline{\textit{Positive Pair:}} Original audio of the video from AVSBench, where the audio accurately reflects the visible objects.

\noindent
\underline{\textit{Silence Scenario:}} Test cases without audio, where objects are visually present but silent in the video.

\noindent
\underline{\textit{Noise Condition:}} Background audio noise, testing the model's ability to differentiate between meaningful and irrelevant audio signals.

\noindent
\underline{\textit{Off-screen Audio:}} Semantically unrelated sounds from different categories, as outlined in Table~\ref{tab:sound-categories}, testing the model’s ability to maintain accurate audio-visual correspondence. For example, pairing animal visuals with device sounds requires models to learn true cross-modal relationships rather than relying solely on visual cues.



By incorporating unpaired audio clips, we create negative audio-visual pairs that enable the study of potential visual bias issues in AVS, specifically: (1) {silent object bias}, where models segment visually salient but silent objects; (2) {background noise bias}; and (3) {irrelevant sound bias}, where unrelated sounds are misattributed to visible objects.






% \begin{figure}[h]
%     \centering
%     \includesvg[width=0.5\textwidth]{image/benchmark_s4.svg}
%     \caption{}
%     \label{fig:fig1}
% \end{figure}


% We provide detailed statistics of AVSBench-Robust in Figure \ref{}. The benchmark consists of two subsets: a robust single-source subset-robust S4 with 4,932 videos (3,452/740/740 for train/val/test) and a robust multi-source subset-robust M4 containing 424 videos (296/64/64 for train/val/test). The robust S4 subset contains highly diverse instances across 20 classes, with detailed per-class distributions shown in Fig \ref{}. These classes are grouped into four major categories: device (32.2\%), music (32.1\%), animal (23.2\%), and human (12.5\%), as shown in the pie chart. The MS3 subset focuses on multi-source scenarios, where videos contain multiple sound-producing objects, with distributions across different category combinations shown in Fig \ref{}.

% For each video in both subsets (3,452/740/740 for robust S4 train/val/test and 296/64/64 for robust MS3), we create a comprehensive evaluation suite by augmenting it with four types of audio conditions, effectively quadrupling the number of audio-visual pairs. These conditions are evenly distributed (25\% each) among: (1) original positive pairs where audio-visual signals align, (2) silence scenarios, (3) background noise conditions, and (4) off-screen sounds(Fig \ref{}). This balanced design enables thorough evaluation of models' ability to both generate accurate segmentation masks when appropriate and suppress false predictions when no valid audio-visual correspondence exists.

% ==============================================
% 4. Evaluation Metrics
\subsection{Evaluation Protocols}
\label{subsec:eval}
A robust AVS model should not only accurately segment sound-producing objects but also reliably suppress predictions when no valid audio-visual correspondence exists. To enable this comprehensive evaluation, we propose new metrics that assess both aspects of AVS performance.


Let $\mathcal{P}$ and $\mathcal{N}$ denote sets of positive and negative samples, respectively. For positive samples, following established protocols~\cite{zhou2022audio, gao2024avsegformer, ma2024stepping}, we employ mean Intersection over Union (mIoU) and F-score to evaluate segmentation accuracy. For negative ones, we introduce complementary metrics to capture different aspects of model robustness.

\noindent
\textit{False Positive Rate (FPR):}
\begin{equation}
\text{FPR} = \frac{\sum_{x\in \mathcal{M_{\text{pred}}} } m(x)}{H \cdot W},
\end{equation}
% \yapeng{define m(x)}
where $m(x)$ denotes the binary indicator (0 or 1) for pixel $x$ in the predicted mask.
FPR measures the proportion of incorrectly activated pixels in negative scenarios, directly assessing the model's tendency to generate false predictions.

To evaluate overall performance across both positive and negative cases, we propose three global metrics.

\noindent
\textit{Global mIoU (G-mIoU):} 
\begin{equation} \label{eq:gmiou}
\text{G-mIoU} = \frac{2 \cdot \text{mIoU}_\mathcal{P} \cdot (1 - \text{mIoU}_\mathcal{N})}{\text{mIoU}_\mathcal{P} + (1 - \text{mIoU}_\mathcal{N})}.
\end{equation}
where $\text{mIoU}_\mathcal{P}$ is the mIoU for positive samples, and $\text{mIoU}_\mathcal{N}$ is for negative samples. G-mIoU balances region-level accuracy, emphasizing the model's ability to maintain precise segmentation boundaries while suppressing false activations. A high score indicates accurate object delineation in positive cases and clean masks in negative cases.

\noindent
\textit{Global F-score (G-F):} 
\begin{equation} \label{eq:gf}
\text{G-F} = \frac{2 \cdot \text{F}_\mathcal{P} \cdot (1 - \text{F}_\mathcal{N})}{\text{F}_\mathcal{P} + (1 - \text{F}_\mathcal{N})}.
\end{equation}
G-F provides a pixel-level assessment that equally weighs precision and recall, which is essential for evaluating both false positives and false negatives. This metric is particularly sensitive to small errors that may be overlooked by IoU-based measures.


\noindent
\textit{Global False Positive Rate (G-FPR):}
\begin{equation} \label{eq:gfpr}
\text{G-FPR} = \frac{1}{|\mathcal{N}|} \sum_{i \in \mathcal{N}} \text{FPR}_{i}.
\end{equation}
This metric specifically focuses on false activation suppression across all negative conditions. While G-mIoU and G-F balance positive and negative performance, G-FPR provides a dedicated measure of a model's robustness against different types of audio distractors.

The combination of these metrics provides a comprehensive evaluation framework: G-mIoU captures region-level accuracy, G-F ensures pixel-level precision, and G-FPR specifically measures robustness to negative conditions. Together, they enable thorough assessment of both segmentation quality and prediction suppression capabilities.


% To ensure comprehensive evaluation, we introduce metrics that assess both alignment accuracy and suppression of irrelevant activations. 

% Aligning with established protocols \cite{zhou2022audio, gao2024avsegformer, ma2024stepping, chen2024cavp, yang2024combo}, we using mean Intersection over Union (mIoU) and F-score as the performance metrics for positive samples.


% \textbf{False Positive Rate (FPR):} $text{FPR} = \frac{\sum_x m(x)}{H \cdot W}$. FPR quantifies suppression effectiveness in negative samples by measuring the proportion of activated pixels in irrelevant audio scenarios.

% To assess model performance across both sample types, we introduce global metrics:

% A higher G-mIoU score reflects a model that successfully segments relevant regions for positive cases while minimizing activations in irrelevant areas for negative cases.


% where $\text{F}_\mathcal{P}$ denotes the F-score for positive samples, and $\text{F}_\mathcal{N}$ is the average F-score for negative samples. A high G-F score indicates that the model performs well in detecting relevant regions when audio and visual inputs are aligned, while effectively suppressing false activations in negative cases.

% A lower G-FPR indicates that the model reliably suppresses activations when there is no relevant audio-visual correspondence, demonstrating robustness against irrelevant sounds or silence.

% % ==============================================
% % 3. Existing Solutions and Their Limitations
% \subsection{Existing Solutions and Their Limitations}  
% Recent works have recognized this bias problem in AVS. CAVP\cite{chen2024cavp} attempts to address it through contrastive learning with mismatched audio-visual pairs. However, their heavy reliance on negative samples in the contrastive learning process inadvertently causes the model to disregard audio features entirely, resulting in predictions that remain largely unaffected by audio variations. While other approaches like BiasAVS\cite{sun2024biasinAVS} and QDFormer\cite{li2024qdformer} propose incorporating semantic information into audio representations, their restricted availability limits comprehensive evaluation and comparison.




\begin{figure*}[ht]
    \centering

        \centering
        \includegraphics[width=0.9\textwidth]{AVS_framework.pdf}
        \vspace{-8mm}
        \caption{\textbf{Framework Overview.} Given video frames and an audio clip as inputs, our approach can robustly identify and segment sounding objects in video frames. Positive audio-visual pairs represent aligned sound sources, while negative pairs, such as silence or offscreen sounds, correspond to empty masks. The model uses separate visual and audio encoders to extract modality-specific features, applies similarity-based alignment optimized with classifier guidance in a contrastive manner, and integrates features through a fusion module. Positive pairs maximize similarity, while negative pairs minimize it, using a small portion (10\%) of the dataset for improved boundary delineation. This dual-stream design facilitates segmentation by distinguishing sound-relevant regions in complex scenes.}
        \label{fig:AVSBench}

    \vspace{-3mm}
\end{figure*}


\section{Method}
\label{sec:method}

In this section, we first present a framework overview in Sec.~\ref{subsec:framework_overview}. Upon the framework, we detail our approach to address the bias problem in AVS through three key components: balanced audio-visual pair construction (Sec. \ref{subsec:Learning_Balanced}), classifier-guided similarity learning (Sec. \ref{subsec:bce_guide} ), and joint segmentation training (Sec. \ref{subsec:total_loss}). To validate our approach, we apply it to two representative AVS models: AVSBench \cite{zhou2022audio} and AVSegFormer \cite{gao2024avsegformer}. The architecture of AVSBench is described in Sec.~\ref{subsec:preliminary}. Due to space constraints, implementation details for \cite{gao2024avsegformer} are provided in the appendix.
% The framework effectively handles both positive and negative audio scenarios while maintaining high-quality segmentation performance.

% Our method was tested using two different baseline models to validate its effectiveness and robustness: AVSBench \cite{zhou2022audio} and AVSegFormer \cite{gao2024avsegformer}. While we focus on the implementation details based on the AVSBench \cite{zhou2022audio} baseline within this section, the specifics of our experiments with the AVSegFormer \cite{gao2024avsegformer} model are provided in the appendix due to space constraints.


\subsection{Preliminary: AVS Architecture}
\label{subsec:preliminary}

\textbf{Encoder:} We employ an encoder structure that separately processes audio clip $A$ and visual frames $V$. Specifically, input audio is converted into spectrograms and processed through a VGGish-based network \cite{hershey2017vggish}, pre-trained on AudioSet \cite{gemmeke2017audioset}, to generate audio feature $\mathcal{F}_A \in \mathbb{R}^d$ where $d = 128$. For visual inputs $V$, we utilize a transformer-based backbone~\cite{wang2022pvt} to extract hierarchical visual features.$\mathcal{F}_{V_i} \in \mathbb{R}^{h_i \times w_i \times C_i}$, where $(h_i,w_i) = (H,W)/2^{i+1}, i = 1,\ldots,n$. The number of levels is set to $n = 4$ in all experiments. %\yapeng{dimension of this feature}.
% $\mathcal{F}_V$

\noindent \textbf{Cross-Modal Fusion:} Following the work in \cite{zhou2022audio}, the fusion process involves an Atrous Spatial Pyramid Pooling (ASPP) module~\cite{chen2017aspp} that manipulates the visual feature maps to enhance object recognition capabilities in varying receptive fields. Subsequently, audio features are integrated to reinforce the identification of sounding objects, crucial for precise segmentation in mixed audio scenarios.

\noindent\textbf{Decoder:} The decoder leverages a Panoptic-FPN \cite{kirillov2019panoptic} architecture, which sequentially processes outputs from the fusion stage and refines them through upsampling, aiming to recover detailed segmentations at the original scale.

\noindent \textbf{Segmentation Loss:} The segmentation objective is the binary cross-entropy loss for basic segmentation accuracy.
\begin{equation} 
\mathcal{L}_{\text{Seg}} = \mathcal{L}_{BCE}(\mathcal{M}_{pred}, \mathcal{M}_{gt}),  
\end{equation}
where $\mathcal{M}_{pred}$ is the predicted segmentation mask, $\mathcal{M}_{gt}$ is the ground-truth (GT) mask.


\subsection{Framework Overview}
\label{subsec:framework_overview}

Our framework, as illustrated in Fig.~\ref{fig:AVSBench}, processes both positive and negative audio-visual pairs to learn robust correspondence for segmentation. Built upon the presented AVS architecture, our model achieves balanced training by incorporating negative audio-visual pairs, enhancing robustness in AVS. Within this framework, audio and visual features are extracted and used to compute cosine similarity scores for both positive pairs $\mathcal{P}$ and negative pairs $\mathcal{N}$, allowing the model to differentiate aligned from unaligned audio-visual pairs. For mask prediction, we employ a segmentation module that combines a fusion module and an FPN decoder, enabling precise segmentation of sound-producing objects. The dual-stream design allows the model to accurately identify sound-relevant regions in complex scenes while suppressing predictions when no valid audio-visual correspondence exists. The following sections detail each component and their integration within the framework.
% Given a dataset $\mathcal{D} = \{V^i, A^i, M_{gt}^i\}_{i=1}^n$ containing video frames $V$, audio clips $A$, and ground truth masks $M_{gt}$, we use pre-trained visual and audio models to extract visual features $\mathcal{F}_V$ and audio feature $\mathcal{F}_A$ respectively. Taking extracted audio and visual features as inputs, cosine similarity scores are computed for both positive pairs $\mathcal{P}$ and negative pairs $\mathcal{N}$, helping the model distinguish between aligned and unaligned audio-visual pairs.
% To predict the sound-producing object mask, we use segmentation module where the framework integrates fusion module and Decoder(FPN). This setup enables precise segmentation of sound-producing objects. The dual-stream design allows the model to effectively identify sound-relevant regions in complex scenes while suppressing predictions when no valid audio-visual correspondence exists. The following sections provide a detailed explanation of each component and their integration within the framework.

\subsection{Learning with Balanced Audio-Visual Pairs}
\label{subsec:Learning_Balanced}

In real-world scenarios, audio-visual correspondence is inherently dynamic~\cite{chen2022comprehensive, chakraborty2023multimodal, yang2024combo}. A visible object may or may not be producing sound at any given moment—for instance, a person may be speaking or silent, and a car may be running or stationary. Additionally, sounds may come from off-screen sources or be ambient noise. This variability requires AVS models to learn true audio-visual association rather than assume that all visible objects are sound sources.

Existing AVS models have been trained with predominantly \textit{positive} audio-visual pairs, where audio and visual signals align, and salient objects are typically the sound sources. This encourages AVS models to rely solely on visual information, bypassing true multimodal integration.

Motivated by this insight, we propose a critical requirement: models must be trained with both positive and negative audio-visual pairs. This balanced approach ensures that the model learns not only when to segment objects that make sounds but also, crucially, when to suppress segmentation predictions for visually salient but silent objects.

% This insight points to a crucial requirement: the model must be exposed to both positive and negative audio-visual pairs during training. Such balanced training ensures that the model learns not only when to segment objects but also, critically, when to suppress segmentation predictions.

Given a video clip with its corresponding audio signal, we construct two types of pairs:


\textit{Positive Pairs ($\mathcal{P}$):} Original audio-visual pairs where the audio corresponds to visual objects in the frame. These pairs represent valid correspondence cases and constitute the majority (approximately 90\%) of training samples.


\textit{Negative Pairs ($\mathcal{N}$):} We deliberately create challenging negative scenarios by:
 1) Replacing the original audio with silence;
 2) Replacing the original audio background noise or ambient sounds;
 3) Using off-screen sounds that are semantically distinct from visible objects.

We maintain a 10\% of negative pairs during training, which we empirically found to optimally balance robustness, segmentation accuracy, and training efficiency.  Expanding the diversity of training samples is anticipated to further enhance the model's robustness.

% This balanced training strategy serves as the foundation for our framework, enabling the model to develop robust audio-visual correspondence understanding. 

\subsection{Classifier-Guided Feature Alignment}
\label{subsec:bce_guide}
However, we observed that simply introducing negative pairs is insufficient to mitigate the visual bias, as show in Table~\ref{tab:only_negative}. Due to the inherent bias in existing models, which often fail to effectively utilize audio information, the model tends to behave more like a purely visual segmentation model. Without explicit guidance, adding negative pairs can lead to confusion during training, as the model alternates between predicting object masks and empty masks. This ultimately degrades performance, not only on the original dataset but also in negative conditions, where the model may continue to produce object masks despite the absence of valid audio-visual correspondence.

While balanced training with positive and negative pairs exposes the model to diverse scenarios, it needs explicit guidance to learn when audio and visual features truly correspond. To address this, we propose using a classifier to directly supervise audio-visual similarity learning, creating clear decision boundaries for correspondence detection.

% Observation of Table \ref{tab:only_negative} reveals that merely introducing negative pairs is insufficient. 
% Due to the inherent bias in existing models, which often fail to effectively utilize audio information, the model tends to behave more like a segmentation model trained solely on visual cues. Without additional guidance, the introduction of negative pairs can lead to confusion during training, as the model alternates between predicting object masks and empty masks. This lack of clarity ultimately results in degraded performance not only on the original dataset but also in negative conditions, where the model may continue to produce object masks despite the absence of valid audio-visual correspondence.

% While balanced training pairs provide diverse scenarios, the model needs explicit guidance to learn when audio and visual features correspond. We propose using Binary Cross-Entropy (BCE) loss to directly supervise the audio-visual similarity learning, creating clear decision boundaries for correspondence detection.


Given multi-scale visual features $\mathcal{F}_i \in \mathbb{R}^{h_i \times w_i \times C_i}$ from the backbone, we use the final-stage features $\mathcal{F}_4 \in \mathbb{R}^{h_4\times w_4 \times C_4}$ and audio features $\mathcal{F}_A \in \mathbb{R}^{D_a}$ for similarity computation. We project $\mathcal{F}_A$ to $C_4$ dimensions via a linear layer and apply spatial pooling to $\mathcal{F}_4$ to obtain aligned features $\hat{\mathcal{F}}_A, \hat{\mathcal{F}}_V \in \mathbb{R}^{C_4}$. 
% , we first align their feature dimensions. Specifically, $\mathcal{F}_A$ is projected to a $D_v$-dimensional space through a linear layer, while $\mathcal{F}_V$ is spatially pooled to obtain global visual features. This yields aligned features $\hat{\mathcal{F}}_A, \hat{\mathcal{F}}_V \in \mathbb{R}^{D_v}$. 
Their correspondence is then computed through cosine similarity:
\begin{equation}
   s(F_A, F_V) = \text{cos}(\hat{\mathcal{F}}_A, \hat{\mathcal{F}}_V).
\end{equation}

We then apply BCE loss to explicitly guide similarity learning in a contrastive manner:
\begin{equation}
\begin{split}
   \mathcal{L}_{\text{BCE}} = -\frac{1}{|\mathcal{P}| + |\mathcal{N}|} & \sum_{j=1}^{|\mathcal{P}| + |\mathcal{N}|}  \left( y_j \log \sigma(s_j) \right. \\
   & \left. + (1 - y_j) \log (1 - \sigma(s_j)) \right),
\end{split}
\end{equation}
where $\sigma(\cdot)$ is the sigmoid function, $y_j$ is the binary label (1 for positive pairs, 0 for negative pairs), and $|\mathcal{P}| + |\mathcal{N}|$ is the total number of positive pairs and the total number of negative pairs respectively. By explicitly supervising the similarity learning, the BCE loss forces the model to maximize similarity for positive pairs (where valid audio-visual correspondence exists) and minimize it for negative pairs (where no correspondence is present). This guidance helps the model learn to interpret audio as a cue for segmentation only when there is a meaningful alignment with the visual input, reducing confusion in cases without correspondence. 


\subsection{Joint Training with Segmentation}
\label{subsec:total_loss}
% Our overall objective is to minimize the following loss function, which combines Binary Cross-Entropy (BCE) loss for correspondence guidance and segmentation loss for prediction quality:
Our total loss objective function $\mathcal{L}$ can be computed as follows:
% combines BCE loss for correspondence guidance and segmentation loss for prediction quality:
\begin{equation}
   \mathcal{L} = \lambda \mathcal{L}_{\text{BCE}} + \mathcal{L}_{\text{Seg}},
\end{equation}
where $\lambda$ is a balancing weight. Together, these loss terms enforce robust and effective learning in AVS models:
1) The first term determines whether segmentation should occur based on audio-visual correspondence;
2) The second term ensures correct segmentation masks when correspondence exists;
3) For negative pairs, the empty GT masks naturally guide the segmentation loss to suppress predictions.

This simple, well-motivated approach can achieve strong performance without relying on complex model modifications, making our method easier to implement, tune, and integrate with existing AVS architectures.

% and the segmentation loss ensures accurate pixel-wise segmentation.
% This joint training creates a natural division of responsibilities: BCE loss determines whether segmentation should occur based on audio-visual correspondence; Segmentation loss ensures high-quality segmentation masks when correspondence exists; And for negative pairs, the empty ground truth masks naturally guide the Segmentation loss to suppress predictions.

% Compared to more complex alternatives that rely on elaborate architectural modifications or sophisticated loss functions, our approach achieves robust performance through simple, well-motivated components, makes our method easier to implement and tune.



\begin{table*}[!htbp]
\centering
\resizebox{\textwidth}{!}{%
\begin{tabular}{c|c|cc|ccc|ccc|ccc|ccc}
\hline
& & \multicolumn{2}{c|}{\textbf{Positive audio input}}&  \multicolumn{9}{c|}{\textbf{Negative audio input}}& \multicolumn{3}{c}{\textbf{Global metric}}\\
 & & \multicolumn{2}{c|}{}&  \multicolumn{3}{c}{\textbf{Slience}}&\multicolumn{3}{c}{\textbf{Noise}}&   \multicolumn{3}{c|}{\textbf{Offscreen sound}}& \multicolumn{3}{c}{}\\ \hline
 \textbf{Test set}&  \textbf{Model}& \textbf{mIoU ↑}& \textbf{F-score ↑}& \textbf{mIoU ↓}&  \textbf{F-score ↓}&\textbf{FPR ↓}&\textbf{mIoU ↓}& \textbf{F-score ↓}& \textbf{FPR ↓}& \textbf{mIoU ↓}&  \textbf{F-score ↓}&\textbf{FPR ↓}& \textbf{G-mIoU↑}& \textbf{G-F↑}&\textbf{G-FPR↓}\\ \hline
\multirow{7}{*}{\textbf{AVSBench-S4}}& AVSBench~\cite{zhou2022audio}& 78.7& 87.9& 76.6&   87.1&0.19&77.6& 88.0& 0.18& 78.2&  88.2&0.19
& 35.032&  21.479&0.186
\\
 & AVSegFormer~\cite{gao2024avsegformer}& 82.1& 89.9& 83.0&   90.4&0.19&83.0& 90.4& 0.19& 83.0&  90.4&0.19
& 28.199&  17.355&0.188
\\
 & Stepping-Stones~\cite{ma2024stepping}& 83.2& 91.3& 82.2&   91.3&0.19&82.2& 91.3& 0.19& 82.5&  91.3&0.19
& 28.980&  15.806&0.190
\\
 & SAMA-AVS~\cite{liu2024annofree}& 83.1& 90.0& 56.2&   69.1&0.17&59.3& 73.8& 0.13& 68.7&  79.0&
0.17& 52.688&  40.417&0.155
\\
 & CAVP~\cite{chen2024cavp}& 78.7& 88.8& 78.7&   88.8&0.19&78.7& 88.8& 0.19& 78.7&  88.8&0.19
& 33.526&  19.891&0.185\\
 & COMBO~\cite{yang2024combo}& \textbf{84.7}& \textbf{91.9}& 84.6&   91.9&0.19&84.6& 91.9& 0.19& 84.6&  91.9&
0.19& 26.062&  14.888&0.190\\
     
     & \cellcolor{lightblue} \textbf{AVSBench + Ours} & \cellcolor{lightblue} 78.1& \cellcolor{lightblue} 88.2& \cellcolor{lightblue} \textbf{0.2}& \cellcolor{lightblue} \textbf{22.6}& \cellcolor{lightblue} \textbf{0.00}& \cellcolor{lightblue} \textbf{0.2}& \cellcolor{lightblue} \textbf{22.6}& \cellcolor{lightblue} \textbf{0.00}& \cellcolor{lightblue} \textbf{0.2}& \cellcolor{lightblue} \textbf{22.6}& \cellcolor{lightblue} \textbf{0.00}& \cellcolor{lightblue} \textbf{87.672}& \cellcolor{lightblue} \textbf{82.461}& \cellcolor{lightblue} \textbf{0.000}\\
    
     & \cellcolor{lightblue} \textbf{AVSegFormer + Ours} & \cellcolor{lightblue} 74.2& \cellcolor{lightblue} 84.8& \cellcolor{lightblue} \textbf{0.2}& \cellcolor{lightblue} \textbf{22.6}& \cellcolor{lightblue} \textbf{0.00}& \cellcolor{lightblue} \textbf{0.3}& \cellcolor{lightblue} \textbf{22.7}& \cellcolor{lightblue} \textbf{0.00}& \cellcolor{lightblue} \textbf{0.5}& \cellcolor{lightblue} \textbf{22.9}& \cellcolor{lightblue} \textbf{0.00}&  \cellcolor{lightblue}\textbf{85.069}& \cellcolor{lightblue} \textbf{80.849}& \cellcolor{lightblue}\textbf{0.001}\\ \hline
    

\multirow{7}{*}{\textbf{AVSBench-MS3}}& AVSBench~\cite{zhou2022audio}& 54.0& 64.5& 27.6&   53.5&0.05&31.7& 57.4& 0.05& 42.2&  62.4&0.09
& 59.468&  51.036&0.072
\\
 & AVSegFormer~\cite{gao2024avsegformer}& 61.3& 73.8& 53.2&   68.2&0.13&47.5& 63.8& 0.09& 50.3&  66.0&0.11
& 54.889&  46.571&0.103
\\
 & Stepping-Stones~\cite{ma2024stepping}& 67.3& 77.6& 45.6&   
72.5&0.09&43.8& 
72.3& 0.08& 41.0&  63.3&0.15& 61.439&  43.937&0.114
\\
 & SAMA-AVS~\cite{liu2024annofree}& \textbf{68.6}& \textbf{78.3}& 29.7&   
36.8&0.09&39.2& 
46.6& 0.12& 44.1&  49.9&
0.14& 65.308&  65.038&0.125
\\
 & CAVP~\cite{chen2024cavp}& 45.8& 61.7& 45.8&   61.7&0.11&45.8& 61.7& 0.11& 45.8&  61.7&0.11
& 49.647&  47.262&0.110\\
 & COMBO \cite{yang2024combo}& 59.2& 71.2& -&   
-&-&-& -& -& -&  -&
-& -&  -&-\\ 

    & \cellcolor{lightblue} \textbf{AVSBench + Ours} & \cellcolor{lightblue} 51.3& \cellcolor{lightblue} 64.5& \cellcolor{lightblue} 9.8& \cellcolor{lightblue} 17.7& \cellcolor{lightblue} \textbf{0.00}& \cellcolor{lightblue} \textbf{9.9}& \cellcolor{lightblue} \textbf{25.8}& \cellcolor{lightblue} \textbf{0.00}& \cellcolor{lightblue} \textbf{9.1}& \cellcolor{lightblue} \textbf{20.3}& \cellcolor{lightblue} \textbf{0.00}& \cellcolor{lightblue} \textbf{65.427}& \cellcolor{lightblue} \textbf{70.911}& \cellcolor{lightblue} \textbf{0.001}\\ 

    & \cellcolor{lightblue} \textbf{AVSegFormer + Ours} & \cellcolor{lightblue} 61.5& \cellcolor{lightblue} 74.0& \cellcolor{lightblue} \textbf{9.1}& \cellcolor{lightblue} \textbf{17.0}& \cellcolor{lightblue} \textbf{0.00}& \cellcolor{lightblue} \textbf{9.4}& \cellcolor{lightblue} \textbf{17.2}& \cellcolor{lightblue} \textbf{0.00}& \cellcolor{lightblue} \textbf{9.1}& \cellcolor{lightblue} \textbf{17.0}& \cellcolor{lightblue} \textbf{0.00}& \cellcolor{lightblue} \textbf{73.354}& \cellcolor{lightblue} \textbf{78.244}& \cellcolor{lightblue} \textbf{0.000}\\ 

\hline
\end{tabular}%
}
\vspace{-2mm}
\caption{Performance comparison of various models on different audio input types and global metrics.}
\label{tab:main_table}
\end{table*}


%=================================================
% \subsection{Challenges in Audio-Visual Correspondence}

% The bias problem in Audio-Visual Segmentation presents several fundamental challenges that make it particularly difficult to solve:

% \textbf{Inherent Visual Dominance:} Visual features often provide stronger and more consistent cues for object segmentation compared to audio signals. This natural imbalance leads models to overly rely on visual information, making it challenging to properly integrate audio cues in the decision process. Even when audio signals are absent or irrelevant, visually salient objects can trigger false segmentation predictions.

% \textbf{Dataset Limitations:} Existing datasets predominantly contain positive audio-visual pairs where sounds align with visible objects. This creates a strong training bias where models learn to always generate segmentation masks when they detect visually prominent objects. The lack of diverse negative scenarios in training data makes it difficult for models to learn when not to predict segmentation masks.

% \textbf{Architectural Constraints:} Traditional AVS architectures are designed to fuse audio and visual features for mask generation, but they lack explicit mechanisms to suppress predictions when audio-visual correspondence is absent. The challenge lies in designing architectures that can not only detect positive correspondences but also confidently identify and handle scenarios where no valid correspondence exists.

% These challenges are further compounded in real-world applications where audio conditions can be highly variable and unpredictable, making robust audio-visual segmentation a critical yet difficult problem to solve.
% % =================================================


\section{Experiment}

% \begin{figure*}[h]
%     \centering
%     \includesvg[width=1.0\textwidth]{image/examples_s4.svg}
%     \caption{Qualitative results of S4 dataset on AVSBench baseline}
%     \label{fig:examples}
% \end{figure*}

% \begin{figure*}[h]
%     \centering
%     \includesvg[width=1.0\textwidth]{image/examples_ms3_2.svg}
%     \caption{Qualitative results of MS3 dataset on AVSBench baseline}
%     \label{fig:examples}
% \end{figure*}

\begin{figure*}[h]
\vspace{-2mm}
    \centering
    \includegraphics[width=1.0\textwidth]{avs_examples.pdf}
    % \caption{Qualitative results on S4 dataset. We can see that \yapeng{xxxx} \yapeng{We also need qualitative results on MS3. If we do not have space, we should add the results in appendix}}

    \caption{\textbf{Performance comparison of different AVS models under various audio conditions on Robust-S4 dataset}. Existing SOTA methods \cite{liu2024annofree, ma2024stepping, chen2024cavp} segment objects primarily based on visual salience, exhibiting a strong visual bias. In contrast, our approach achieves accurate segmentation with original audio while successfully reject predict in negative scenarios (e.g., silence, noise, off-screen).}
    \label{fig:examples}
\vspace{-4mm}
\end{figure*}


% \subsection{Results}

\subsection{Setup}


\noindent\textbf{Dataset.} We utilize the AVSBench-Robust Benchmark for our evaluation, which is designed to rigorously assess AVS capabilities. Further details on the dataset specifics and video categories have been discussed in Sec. \ref{sec:problem_benchmark}.

\noindent\textbf{Baselines.} We benchmark our model against notable methods including AVSBench~\cite{zhou2022audio} and AVSegFormer~\cite{gao2024avsegformer}, representing fusion-based and prompt-based approaches, respectively. We also compared our method with the CAVP~\cite{chen2024cavp}, Stepping-Stones~\cite{ma2024stepping}, SAMA-AVS~\cite{liu2024annofree} and COMBO~\cite{yang2024combo}. These baselines allow us to demonstrate the broad applicability of our method by comparing it against state-of-the-art models designed to address different aspects of audio-visual segmentation.


\noindent\textbf{Evaluation Metrics.} 
%The evaluation of our model's performance on the AVSBench-Robust benchmark employs key metrics previously introduced in Sec.~\ref{sec:problem_benchmark}.  
Evaluation metrics, including mIoU, F1 score, FPR, and G-mIou, G-F, G-FPR, are used to assess the segmentation accuracy and robustness of AVS models.
%assess accuracy in segmenting aligned audio-visual pairs and model robustness against negative samples. 
%Further details about these metrics and their specific applications in AVS can be found in the aforementioned section.

\noindent\textbf{Implementation:} Our implementation for the AVSBench model employ the Pyramid Vision Transformer (PVT-v2)~\cite{wang2022pvt} pretrained on the ImageNet dataset~\cite{russakovsky2015imagenet} as the visual backbone, which processes video frames of size $H \times W = 224 \times 224$ and output multiple scales visual feature $\mathcal{F}_{V_i} \in \mathbb{R}^{h_i \times w_i \times C_i}$ for $i = 1,\ldots,4$, The channel dimensions $C_i$ correspond to \{64, 128, 320, 512\} for each respective scale.
% This transformer-based architecture, pretrained on the ImageNet dataset~\cite{russakovsky2015imagenet}, processes video frames resized to $H \times W = 224 \times 224$. The architecture's stages have channel dimensions set at $C_1, C_2, C_3, C_4 = [64, 128, 320, 512]$, allowing for detailed feature extraction across multiple scales. 
For audio input, we employ VGGish~\cite{hershey2017vggish} pretrained on AudioSet~\cite{gemmeke2017audioset} to extract features $\mathcal{F}_A \in \mathbb{R}^{128}$ from each one-second audio clips.
% For audio data, we employ the VGGish model, a VGG-style network pretrained on the AudioSet dataset, which processes audio inputs segmented into one-second clips. 
This model is trained using the Adam~\cite{kingma2014adam} optimizer with a learning rate of $1 \times 10^{-4}$, batch size of 4, and loss weighting factor $\lambda = 1$. Training durations are 15 epochs for the semi-supervised S4 setup and 30 epochs for the fully-supervised MS3 setup on an NVIDIA RTX A5000 GPU.


\subsection{Experimental Comparison}
Our extensive experimental comparisons reveal several significant findings in AVS performance as shown in Table~\ref{tab:main_table}. 

\noindent
\textbf{SOTA methods fail under negative audio conditions}, demonstrating a strong visual bias and ineffective audio-visual integration.
Surprisingly, recent methods like Stepping-Stones~\cite{ma2024stepping} and CAVP~\cite{chen2024cavp} achieve nearly identical mIoU and F-scores regardless of whether the input audio is silent, irrelevant, or noisy.  These methods consistently exhibit high False Positive Rates (FPR), ranging from 0.17 to 0.19 across all negative scenarios on AVSBench-S4, indicating a significant reliance on visual cues.  This issue also impacts their global metrics, with G-mIoU scores between 28.19 and 35.03. These results suggest that these methods fail to effectively leverage audio information in this multimodal segmentation task. While this issue is somewhat less pronounced on the MS3 dataset, it remains present.

\vspace{1mm}
\noindent
\textbf{Our method resolves bias while maintaining performance.} When integrated with AVSBench~\cite{zhou2022audio}, our approach performs comparable positive audio performance (mIoU: 78.1, F-score: 88.2) while achieving perfect robustness to negative audio inputs with an FPR of 0.00 across all negative conditions. Similarly, our AVSegFormer~\cite{gao2024avsegformer} variant demonstrates only minimal degradation in positive audio metrics while achieving perfect FPR scores. Most notably, our approach achieves superior global metrics, with our AVSBench variant reaching a G-mIoU of 87.672 and G-F score of 82.461, substantially outperforming existing methods. The consistent improvement in robustness across two very different AVS architectures demonstrates the effectiveness and generality of our approach. Fig.~\ref{fig:examples} provides examples of S4 dataset visualizations. Visualizations for MS3 dataset are included in the supplementary material.


\vspace{1mm}
\noindent
\textbf{Our method excels in global metrics across scenarios}, showing consistent improvements in both the single-source and more complex multi-source settings. In the MS3, it maintains perfect robustness with an FPR of 0.00 in all negative conditions and achieves impressive global metrics; the AVSegFormer variant records a G-mIoU of 73.354 and a G-F score of 78.244. These results confirm our method's scalability and its significant advancement in addressing the longstanding limitations of existing AVS methods.







% \begin{figure}
%     \centering  
%     \begin{subfigure}{0.25\textwidth}
%         \centering
%         \includegraphics[width=\textwidth]{image/AVSBench_S4_cos_original_offscreen_comparison_histogram.png}
%         \subcaption{Baseline Model}
%         \label{fig:similarity_baseline}
%     \end{subfigure}%
%     \begin{subfigure}{0.25\textwidth}
%         \centering
%         \includegraphics[width=\textwidth]{image/AVSBench_S4_10ood_cos_original_offscreen_comparison_histogram.png}
%         \subcaption{Our Model}
%         \label{fig:similarity_our_model}
%     \end{subfigure}
%     \caption{Comparison of Normalized Similarity Distributions for Audio-Visual Pairs.
%     Each plot shows the distribution of similarity scores (x-axis) between audio and visual embeddings, normalized using a sigmoid function. The y-axis represents the density of similarity scores. The blue bars and lines indicate the distribution for original (matched) audio-visual pairs, while the red bars and lines represent offscreen (mismatched) pairs. The solid lines show kernel density estimates for each category. In the baseline model (left), there is significant overlap between the two distributions, indicating poor separability between matched and mismatched pairs. In contrast, our model (right) demonstrates a clear distinction, with minimal overlap between the distributions, highlighting improved discriminative capability.}
%     \label{fig:similarity_comparison}
% \end{figure}




\begin{figure}[t]
\vspace{-2mm}
    \centering
    \begin{subfigure}[t]{0.24\textwidth}
        \centering
        \includegraphics[width=\textwidth]{avs_s4_sim_distribution_before.pdf}
        \caption{Before}
    \end{subfigure}%
    \begin{subfigure}[t]{0.24\textwidth}
        \centering
        \includegraphics[width=\textwidth]{avs_s4_sim_distribution_after.pdf}
        \caption{After}
    \end{subfigure}
    % \caption{\yapeng{add detailed captions -- seems not discussed in the paper. }}    
    \caption{Cosine similarity distributions between paired features before and after training.(a) Positive and negative pairs exhibit similar distributions, indicating the model’s limited ability to distinguish audio-visual correspondence. (b) After training with classifier-guided similarity learning, the distributions are well-separated, demonstrating the model's enhanced capability to identify valid audio-visual pairs. }
    \label{fig:similarity_hist}
\vspace{-3mm}
\end{figure}



\subsection{Impact of Positive-Negative Pair Ratio}

% \begin{table}[t]
% \centering
% \resizebox{\columnwidth}{!}{
% \begin{tabular}{c|c|c|c|c|c|c}
% \hline
%  & Model & Pos Pairs & Neg Pairs & G-mIoU & G-F & G-FPR \\
% \hline
% \multirow{4}{*}{S4} & Baseline & 100\% & 0\% & 35.032 & 21.479 & 0.186 \\
% \cline{2-7}
% & \multirow{3}{*}{Ours} & 90\% & 10\% & \underline{87.672} & \underline{82.461} & 0.000 \\
% & & 80\% & 20\% & \textbf{87.780} & \textbf{82.114} & 0.000 \\
% & & 70\% & 30\% & 87.204 & 82.233 & 0.000 \\
% \hline
% \multirow{4}{*}{MS3} & Baseline & 100\% & 0\% & 59.468 & 51.036 & 0.072 \\
% \cline{2-7}
% & \multirow{3}{*}{Ours} & 90\% & 10\% & \underline{65.427} & \underline{70.911} & 0.001 \\
% & & 80\% & 20\% & \textbf{67.909} & \textbf{72.572} & 0.003 \\
% & & 70\% & 30\% & 66.251 & 72.908 & 0.000 \\
% \hline
% \end{tabular}
% }
% \caption{Impact of positive-negative ratio on AVS performance}
% \label{tab:pair_ratio_study}
% \end{table}



Our investigation into the ratio of positive to negative audio-visual pairs reveals important insights about training data composition for robust audio-visual segmentation.

As illustrated in Table~\ref{tab:pair_ratio_study}, we found that \textit{introducing negative samples, even in small proportions, dramatically improves performance.} Without negative samples, the baseline model shows poor performance with the G-mIoU of 35.032 and a high FPR of 0.186 on the S4 dataset.
Introducing just 10\% negative samples yields substantial gains, improving G-mIoU to 87.672 and reducing FPR to 0.000. Similar improvements are observed in the MS3 dataset, where G-mIoU increases from 59.468 to 65.427 and FPR drops from 0.072 to 0.001, demonstrating the crucial role of negative samples in developing robust models.

While further increasing the proportion of negative samples (to 20\% or 30\%) maintains similar performance levels, the marginal gains are minimal compared to the 10\% setting. For instance, the G-mIoU difference between 10\% and 20\% negative samples is less than 0.3\% on S4 and 2.5\% on MS3. Considering that adding 10\% negative pairs only increases training time by approximately 10\% while achieving nearly optimal performance, we adopt this ratio as our default setting, offering an efficient balance between robustness and training cost.





\subsection{Abaltion Study}

% \subsubsection{Effectiveness of Negative Samples and Loss Design}

% \begin{table}[t]
% \centering
% \resizebox{\columnwidth}{!}{  
% \begin{tabular}{c|c|c|c|c|c}
% \hline
%  & Negative samples & $\mathcal{L}_{\text{BCE}}$& G-mIoU↑ & G-F↑ & G-FPR↓ \\
% \hline
%  \multirow{3}{*}{S4} & \xmark& \xmark& 35.032& 21.479& 0.186
% \\
%  & \cmark& \xmark& 34.847& 21.993& 0.189\\
%  & \cmark& \cmark& \textbf{87.672}& \textbf{82.461}& \textbf{0.000}
% \\ \hline
%  \multirow{3}{*}{MS3} & \xmark& \xmark& 59.468& 51.036& 0.072
% \\
%  & \cmark& \xmark& 55.489& 30.057& 0.095\\
%  & \cmark& \cmark& \textbf{66.605}& \textbf{70.590}& \textbf{0.004}\\
% \hline
% \end{tabular}
% }
% \caption{Ablation study on negative samples and BCE loss}
% \label{tab:only_negative}
% \end{table}
%A common assumption would be that simply adding negative samples (\eg, silent or background noise) might improve the model's ability to differentiate between sound-producing and non-sound-producing regions. To validate this, we conducted an ablation study with AVSBench as the baseline~\cite{zhou2022audio}, analyzing configurations with and without negative samples and classifier guidance, as summarized in Tab.~\ref{tab:only_negative}.


A common assumption might be that simply adding negative samples would enhance the model's ability to distinguish between sound-producing and non-sound-producing visual regions in AVS. To test this hypothesis, we conducted an ablation study using AVSBench~\cite{zhou2022audio} as the baseline, comparing configurations with and without negative samples and classifier guidance, as summarized in Table~\ref{tab:only_negative}.
 % (\eg, silent or background noise)
\noindent
\textbf{Negative samples alone fail to address the bias problem.} 
Our experimental results highlight the limitations of using only negative samples. Without explicit loss guidance, adding negative pairs not only fails to improve performance but can even lead to significant degradation, particularly on the challenging MS3 dataset. Here, G-mIoU drops from 59.47 to 55.49, and G-F decreases dramatically from 51.04 to 30.06. This degradation occurs because the model becomes confused when alternately exposed to scenarios requiring empty predictions and those with salient object mask predictions, ultimately compromising performance even on standard positive audio inputs.

%Experimental results confirm this limitation - without appropriate loss guidance, model performance significantly deteriorates, particularly in the challenging MS3 dataset where G-mIoU drops from 59.468 to 55.489 and G-F decreases dramatically from 51.036 to 30.057. This performance degradation occurs because the model becomes confused when alternately exposed to scenarios requiring empty predictions and ground-truth mask predictions, leading to compromised performance even in standard positive audio scenarios.

\noindent
\textbf{Combining negative samples with classifier guidance enables robust segmentation.}  Our full approach shows substantial improvements across all metrics. On the Robust S4 dataset, we achieve G-mIoU of 87.672, G-F of 82.461, and perfect G-FPR of 0. Similar gains are observed on MS3, with G-mIoU of 66.605, G-F of 70.590, and near-perfect G-FPR of 0.004. 
The effectiveness of classifier guidance is illustrated in Fig. \ref{fig:similarity_hist}: initially, audio-visual feature similarities cluster around 0.5 for both positive and negative pairs; after training, they are well-separated (0.75 for positive vs. 0.30 for negative), demonstrating enhanced discrimination. This improved feature alignment enables strong  performance on positive cases while accurately suppressing predictions in negative scenarios.
% The classifier guidance with BCE loss provides explicit feature alignment, allowing the model to effectively differentiate between sound-producing and silent regions. This approach enables strong performance on positive cases while accurately handling negative scenarios.
% The ablation results underscore that while negative samples provide essential training data diversity, they must be accompanied by appropriate loss guidance to be effective. Without such guidance, the model struggles to reconcile conflicting objectives, leading to degraded performance. 
The classifier guidance serves as a critical learning framework for effectively utilize negative samples while maintaining its original capabilities, resulting in a robust AVS system. % that can reliably handle both positive and negative audio conditions.

Further experiments on unseen audio categories demonstrate the generalization capability of our approach. Due to space constraints, we refer readers to the supplementary material for detailed results.



\begin{table}[t]
\vspace{-2mm} 
\centering
\resizebox{\columnwidth}{!}{
\begin{tabular}{c|c|c|c|c|c|c}
\hline
 & Model & Pos Pairs & Neg Pairs & G-mIoU & G-F & G-FPR \\
\hline
\multirow{4}{*}{S4} & Baseline & 100\% & 0\% & 35.032 & 21.479 & 0.186 \\
\cline{2-7}
& \multirow{3}{*}{Ours} & 90\% & 10\% & \underline{87.672} & \underline{82.461} & 0.000 \\
& & 80\% & 20\% & \textbf{87.780} & \textbf{82.114} & 0.000 \\
& & 70\% & 30\% & 87.204 & 82.233 & 0.000 \\
\hline
\multirow{4}{*}{MS3} & Baseline & 100\% & 0\% & 59.468 & 51.036 & 0.072 \\
\cline{2-7}
& \multirow{3}{*}{Ours} & 90\% & 10\% & \underline{65.427} & \underline{70.911} & 0.001 \\
& & 80\% & 20\% & \textbf{67.909} & \textbf{72.572} & 0.003 \\
& & 70\% & 30\% & 66.251 & 72.908 & 0.000 \\
\hline
\end{tabular}
}
\vspace{-2mm} 
\caption{Impact of positive-negative ratio on AVS performance}
\label{tab:pair_ratio_study}
\end{table}


\begin{table}[t]
\centering
\resizebox{\columnwidth}{!}{  
\begin{tabular}{c|c|c|c|c|c}
\hline
 & Negative samples & $\mathcal{L}_{\text{BCE}}$& G-mIoU↑ & G-F↑ & G-FPR↓ \\
\hline
 \multirow{3}{*}{S4} & \xmark& \xmark& 35.032& 21.479& 0.186
\\
 & \cmark& \xmark& 34.847& 21.993& 0.189\\
 & \cmark& \cmark& \textbf{87.672}& \textbf{82.461}& \textbf{0.000}
\\ \hline
 \multirow{3}{*}{MS3} & \xmark& \xmark& 59.468& 51.036& 0.072
\\
 & \cmark& \xmark& 55.489& 30.057& 0.095\\
 & \cmark& \cmark& \textbf{66.605}& \textbf{70.590}& \textbf{0.004}\\
\hline
\end{tabular}
}
\vspace{-2mm} 
\caption{Effects of negative samples and classifier guidance.}
\label{tab:only_negative}
\vspace{-3mm} 
\end{table}







%  ==================================

% \textbf{Dataset.} Our evaluation employs the AVSBench-Robust Benchmark, a rigorous framework designed to assess models on their capacity to segment sound-producing objects from video frames. The benchmark is divided into two subsets: the Single-Source Audio Segmentation Subset (S4), containing 4,932 videos across 23 diverse categories, and the Multi-Source Audio Segmentation Subset (MS3), which includes 424 videos in equally varied categories, challenging models to manage scenarios with multiple concurrent sound sources.

% \textbf{Evaluation Metrics.} The performance of our model under the AVSBench-Robust benchmark is quantified using key metrics designed for comprehensive assessment. These include the mean Intersection over Union (mIoU) for measuring accuracy in segmenting positively aligned audio-visual pairs, the F1 score for overall precision and recall balance, and the False Positive Rate (FPR) to evaluate the model's robustness in handling negative samples. A detailed discussion of these metrics and their relevance to AVS research is provided in Section \ref{sec:problem_benchmark}.

%  ==================================
% \textbf{Finding 1: Existing SOTA methods consistently fail under negative audio conditions (silence, noise, and offscreen sound), demonstrating a concerning bias in their audio-visual alignment capabilities.} Traditional methods, including SAMA-AVS\cite{liu2024annofree}, Stepping-Stones\cite{ma2024stepping}, and CAVP\cite{chen2024cavp}, exhibit persistently high False Positive Rates (FPR) ranging from 0.17 to 0.19 across all negative audio scenarios on AVSBench-S4. This limitation is further reflected in their global metrics, where baseline methods achieve relatively low G-mIoU scores between 28.199 and 35.032.

% \textbf{Finding 3: The improvements are consistent across both single-source (AVSBench-S4) and more complex multi-source (AVSBench-MS3) scenarios, demonstrating the scalability of our approach.} In the challenging multi-source setting of AVSBench-MS3, our method maintains its robust performance with an FPR of 0.00 across all negative conditions, while achieving strong global metrics with our AVSegFormer variant reaching a G-mIoU of 73.354 and G-F score of 78.244. These comprehensive results demonstrate that our approach successfully addresses a critical limitation in existing audio-visual segmentation methods while maintaining competitive performance on standard tasks, representing a significant advancement in the field's state-of-the-art.


%  ====== Old table findings ========
% Table \ref{tab:main_table} provides a detailed comparison of performance metrics of our models, which outperform other state-of-the-art methods in challenging negative audio conditions.

% \textbf{Superior Handling of Negative Audio Conditions:}
% A critical observation from our experiments is that, aside from our models, other methods struggle significantly under negative audio conditions. While methods like Stepping-Stones\cite{ma2024stepping}, SAMA-AVS\cite{liu2024annofree} and CAVP\cite{chen2024cavp} traditionally perform well with positive audio inputs, their performance degrades under negative conditions such as silence, noise, and offscreen sound, often resulting in higher False Positive Rates (FPR). In stark contrast, both variants of our models—Ours(AVSBench) and Ours(AVSegFormer)—maintain an FPR of 0.00 across all negative scenarios, highlighting their exceptional robustness and precision in avoiding false positives.

% \textbf{Comparison with AVSBench and AVSegFormer:}
% When comparing Ours(AVSBench) to the standard AVSBench, it is notable that our model achieves comparable results in mIoU and F-score under positive audio inputs but excels in negative audio conditions. Specifically, while the standard AVSBench shows an FPR of up to 0.19 in negative scenarios, Ours(AVSBench) consistently maintains a zero FPR, demonstrating a significant improvement in specificity and error reduction.

% Similarly, Ours(AVSegFormer) shows a minimal decrease in performance metrics such as mIoU and F-score compared to the baseline AVSegFormer under positive audio conditions. However, it vastly outperforms in negative audio scenarios, again maintaining a zero FPR compared to the baseline's higher rates. This performance underscores the enhanced noise discrimination capabilities of our model, making it highly effective for applications requiring high fidelity audio-visual alignment.


% \textbf{Global Metric Performance:}
% On a global scale, both Ours(AVSBench) and Ours(AVSegFormer) score impressively high in Global mIoU (G-mIoU) and Global F-score (G-F), reflecting their robust overall performance across different test conditions and scenarios. These metrics not only substantiate the models' effectiveness in handling complex audio-visual environments but also their adaptability and reliability across both single-source and multi-source settings.
%  ==================================




% \textbf{Increasing negative samples consistently yields near-zero FPR, with an optimal ratio enhancing performance. } Our experiments show that the model maintains remarkable robustness with FPR values of 0.000-0.003 across all configurations after introducing negative samples. However, the segmentation performance peaks at 20\% negative samples, achieving optimal G-mIoU of 87.780 for S4 and 67.909 for MS3. Further increasing to 30\% leads to slight performance degradation, particularly in the more challenging MS3 dataset where G-mIoU drops to 66.251. This reveals that while our approach reliably eliminates false positives regardless of the exact ratio, maintaining approximately 20\% negative samples provides the best trade-off for overall segmentation performance across both single-source and multi-source scenarios.

%  ==================================

% \subsubsection{Positive Audio Input Performance}

% In scenarios with positive audio input, where the sound source aligns with visible objects, our approach demonstrates comparable performance to leading SOTA methods. Specifically, our model, denoted as "Ours (AVSBench)" and "Ours (AVSegFormer)" in Table \ref{tab:main_table}, achieves competitive mIoU and F-score values on both AVSBench-S4 and AVSBench-MS3 datasets, indicating that the incorporation of negative audio samples does not compromise the model's segmentation performance on positive cases. On the AVSBench-S4 subset, the best-performing model, COMBO, achieves an mIoU of 84.7 and F-score of 91.9, while our model reaches an mIoU of 78.1 and F-score of 88.2, demonstrating robust segmentation capabilities. This consistent performance across positive cases supports the adaptability of our model to scenarios with aligned audio-visual cues.

% \subsubsection{Negative Audio Input Performance}

% One of the main objectives of our approach is to address limitations in existing AVS models by improving performance on negative audio inputs, including silence, noise, and offscreen sounds. For these cases, the goal is to suppress segmentation activation, thereby reducing false positives. Table \ref{tab:main_table} shows that our model significantly outperforms other methods in this regard, achieving near-zero FPR and mIoU values on negative audio samples. For instance, on the AVSBench-S4 dataset, our model attains an FPR of 0.00 for both silence and noise cases, indicating that it successfully minimizes unnecessary activations in the absence of relevant audio-visual correspondence. Similarly, our model achieves low mIoU and F-score values on negative audio cases, which demonstrates its capability to effectively ignore visual content when no corresponding audio signal exists.


%  ==================================
% \textbf{Simply incorporating negative samples without appropriate loss guidance not only fails to address the bias problem but also disrupts the model's original capability in handling positive samples.} When we only added negative samples to the training set without modifying the loss function, the model's performance deteriorated, particularly in the challenging MS3 dataset where G-mIoU dropped from 59.468 to 55.489 and G-F plummeted from 51.036 to 30.057. This significant performance degradation reveals a fundamental issue: without proper guidance, the model becomes confused when alternately exposed to scenarios requiring empty predictions (negative audio) and ground-truth mask predictions (positive audio). The original biased model, which primarily relied on visual information, struggles to reconcile these conflicting objectives, leading to compromised performance even in standard positive audio scenarios.

% \textbf{The synergistic combination of negative samples and BCE loss is crucial for achieving robust audio-visual segmentation.} Our complete method, which integrates both components, demonstrates remarkable improvements across all metrics. In the S4 dataset, this combination leads to dramatic performance gains (G-mIoU: 87.672, G-F: 82.461) while achieving a perfect G-FPR of 0.000. Similar substantial improvements are observed in the more challenging MS3 dataset, where our method achieves G-mIoU of 66.605 and G-F of 70.590, with near-perfect G-FPR of 0.004. The BCE loss provides explicit guidance for the model to effectively distinguish between sound-producing and silent regions, enabling it to maintain strong performance in positive scenarios while correctly handling negative cases.
%  ==================================


% \textbf{Adding Negative Samples Only:} 
% When negative samples were added without BCE loss, we observed limited improvements in segmentation metrics. For the S4 dataset, the G-mIoU and G-F scores showed only marginal gains, while on the more complex MS3 dataset, performance actually declined, with G-mIoU dropping from 59.468 to 55.489 and G-F from 51.036 to 30.057. These results indicate that merely adding negative samples is insufficient to guide the model effectively, as it lacks the necessary framework to differentiate positive and negative regions accurately.

% \textbf{Combining Negative Samples with BCE Loss:} 
% Our full method, which integrates negative samples with BCE loss, yielded substantial improvements across both datasets. On the S4 dataset, G-mIoU and G-F increased dramatically to 87.672 and 82.461, respectively, with a perfect FPR of 0.000, indicating no false positives in negative audio conditions. Similar gains were observed for MS3, where G-mIoU and G-F rose to 66.605 and 70.590, with FPR reduced to 0.004. This combination enables the model to align similarity scores with binary labels, allowing it to distinguish between relevant and irrelevant regions with high precision.

%  ==================================
% For both the S4 and MS3 datasets, adding negative samples alone led to minimal improvement in segmentation metrics and even slightly worsened performance on MS3, as seen by drops in G-mIoU and G-F scores. This suggests that without further guidance, negative samples alone may not provide the necessary information to effectively distinguish non-sound-producing regions.

% Our full method, which combines negative samples with the Binary Cross-Entropy (BCE) loss, achieves significant improvements across both datasets. By aligning labels with similarity scores through BCE loss, the model is better equipped to separate positive and negative samples, resulting in higher G-mIoU and G-F scores and a near-zero false positive rate for negative samples.
%  ==================================



% \subsubsection{Percentage of negative samples during training}

% To further investigate the impact of negative samples, we conducted an ablation study on the ratio of positive to negative audio-visual pairs during training, as shown in Table \ref{tab:pair_ratio_study}. The study examines how increasing the percentage of negative samples affects model performance on the S4 and MS3 datasets, specifically in terms of G-mIoU, G-F score, and False Positive Rate (FPR).

% \textbf{Increasing Negative Sample Percentage:}
% The baseline model, trained without any negative samples, demonstrates the lowest performance on both datasets. For instance, on the S4 dataset, the baseline achieves a G-mIoU of 35.032 and a G-F of 21.479, with an average FPR of 0.186. Similarly, on the MS3 dataset, the baseline shows a G-mIoU of 59.468 and a G-F of 51.036, with a higher FPR of 0.072, indicating a pronounced difficulty in distinguishing non-sound-producing regions.

% As the percentage of negative samples increases, we observe a substantial improvement in model performance across both datasets. With 10\% negative samples, our model on the S4 dataset achieves a G-mIoU of 87.672 and a G-F of 82.461, with an FPR reduced to zero. Increasing the negative sample percentage to 20\% and 30\% maintains this high level of segmentation accuracy and robustness, with only slight variations in G-mIoU and G-F scores.

% On the more complex MS3 dataset, introducing 20\% negative samples yields the highest G-mIoU of 67.909 and a G-F score of 72.572, while the FPR remains low at 0.003. At 30\% negative samples, the G-mIoU slightly decreases to 66.251, but the FPR is again reduced to zero. These results suggest that, while increasing negative samples generally enhances performance, an optimal balance around 20\% negative samples maximizes the model’s ability to differentiate sound-related from non-sound regions without overfitting.



% The results on the MS3 dataset demonstrate that introducing negative samples improves model performance in distinguishing sound-related regions. The baseline (0\% negative samples) shows the lowest G-mIoU and G-F scores, along with a high false positive rate (FPR) of 0.072, indicating difficulty in identifying non-sound regions.

% Adding 5-20\% negative samples leads to substantial gains, with the best results at 20\% negative samples (G-mIoU: 67.909, G-F: 72.572, FPR: 0.003), highlighting that a balanced inclusion of negative examples helps refine segmentation accuracy. At 30\% negative samples, performance stabilizes with only slight variations, suggesting that adding more may yield diminishing returns.



\section{Conclusion and Discussion}
% Our systematic investigation into AVS models reveals a critical gap between their intended functionality and actual behavior. Through 
Our comprehensive study using AVSBench-Robust reveals  that current SOTA methods exhibit strong visual bias, generating segmentation masks based predominantly on visual salience regardless of audio context. To address this issue, we introduce a simple yet effective approach combining balanced training with negative audio-visual pairs and classifier-guided feature alignment, which significantly improves model robustness while maintaining competitive performance on standard AVS tasks.
% , such as varying levels of background noise or multiple overlapping sounds
While our method effectively addresses the robustness issue, several challenges remain. Our approach is constrained by the baseline model's performance on positive samples, and real-world applications may encounter even more challenging conditions than those covered in our benchmark. 
We hope our work could inspire further research in this significant and worthwhile field.
% We believe our work opens up new directions for developing truly robust AVS systems in real-world scenarios.


{
    \small
    \bibliographystyle{ieeenat_fullname}
    \bibliography{main}
}

% 
\clearpage
% \setcounter{page}{1}
% \maketitlesupplementary
\begin{center}
Supplementary Material
\end{center}

% {
%     \onecolumn
%     \centering
%     \Large
%     \textbf{\thetitle}\\
%     \vspace{0.5em}Supplementary Material \\
%     \vspace{1.0em}
% }

\section{Proof of \cref{theorem:dr}}
We require some additional regularity assumptions:
\begin{assumption} 1) The number of classes $C$ is bounded w.r.t the number of samples $N$, 2) the missingness mechanism $P(A=1|Y,\theta)$, as well as its estimated counterpart $P(A=1|Y,\theta)$, are bounded below by some constant $\epsilon > 0$, 3) the quantities $P(Y|X,\theta)$ and $P(A|Y,\theta)$ are estimated using auxiliary samples independent of samples used for the sample averaging.
\label{assumption:extra}
\end{assumption}
Assumptions 1 and 2 are natural. For the missingness mechanism, the ground truth being bounded means that there is a non-vanishing proportion of samples for every class. The boundedness of the estimate can be enforced by clipping the estimate. Assumption 3 is called sample splitting in \cite{kennedy-dr}.

For convenience we use operator $\E_N$ to denote the average of $N$ samples i.e. $\frac{1}{N}\sum_{i=1}^N$. Note that this is by itself a random variable, in contrast to $\E$ which is a fixed number.

\begin{proof}[Proof of \cref{theorem:dr}] Because $C$ is bounded (assumption \ref{assumption:extra}), we can fix a class $c$ and prove the theorem.
Let us define the influence function $\phi$, parameterized by $\theta$, as
\begin{equation}
\phi(O | \theta)(c) = P(Y=c|X,\theta) + \frac{\one(A=1)}{P(A=1|Y,\theta)} (\one(Y=c) - P(Y=c|X,\theta)) - P(Y=c)
\end{equation}
As we have done in the main text, we use $\phi(O)$ to denote the same function but all estimated quantities are replaced with their truths. In other words, we use $\phi(O)$ for $\phi(O|\theta_0)$ where $\theta_0$ is the truth, given that our model contains $\theta_0$ e.g. when the model is consistent.

Recall that:
\begin{equation}
\begin{aligned}
\Psi_{dr}(\theta)(c) &= \frac{1}{N}\sum_{i=1}^N \left\{P(Y=c|X,\theta) + \frac{\one(A=1)}{P(A=1|Y,\theta)} (\one(Y=c) - P(Y=c|X,\theta))\right\}\\
&= \E_N [\phi(O|\theta)(c)] + P(Y=c)
\end{aligned}
\end{equation}

We will show that:
\begin{equation}
\Psi_{dr}(\theta)(c) - P(Y=c) = (\E_N - \E)[\phi(O)(c)] + o_P(N^{-1/2})
\label{eq:proof-linearity}
\end{equation}
To do that, we use the following decomposition
\begin{equation}
\begin{aligned}
\Psi_{dr}(\theta)(c) - P(Y=c) &= \E_N [\phi(O|\theta)(c)] \\
&= (\E_N - \E)[\phi(O)(c)] + (\E_N - \E)[\phi(O|\theta)(c) - \phi(O)(c)] + \E[\phi(O|\theta)(c)]
% &+ (\E_n - \E)[\phi(O;\theta) - \phi(O)]\\
% &+ \E[P(Y=c|X,\theta)] - \E[P(Y=c|X)] + \E[\phi(O,\theta)]
\end{aligned}
\end{equation}
and analyze the second and third term. The third term is:
\begin{equation}
\begin{aligned}
\E[\phi(O|\theta)(c)] &= \E[P(Y=c|X,\theta)] + \E\left[\frac{\one(A=1)}{P(A=1|Y,\theta)}(\one(Y=c) - P(Y=c|X,\theta))\right]- P(Y=c) \\
&= \E\left[P(Y=c|X,\theta) + \frac{P(A=1|Y)}{P(A=1|Y,\theta)}(P(Y=c|X) - P(Y=c|X,\theta))\right] - \E[P(Y=c|X)]\\
&= \E\left[(P(Y=c|X,\theta) - P(Y=c|X)) (P(A=1|Y,\theta) -P(A=1|Y)) \frac{1}{P(A=1|Y,\theta)}\right]\\
\end{aligned}
\end{equation}
by Cauchy-Schwarz inequality:
\begin{equation}
\begin{aligned}
\E[\phi(O|\theta)(c)] &\le \frac{1}{\epsilon} \|P(A=1|Y,\theta) - P(A=1|Y)\|_2 \|P(Y=c|X,\theta) - P(Y=c|X)\|_{L_2(P)}\\
&= \frac{1}{\epsilon} o_P(N^{-1/4} N^{-1/4}) = o_P(N^{-1/2})
\end{aligned}
\end{equation}
by assumption \ref{assumption:4th-root-n} and that $P(A=1|Y,\theta) > \epsilon$ (assumption \ref{assumption:extra}). The second term can be bounded by Chebyshev inequality
% \begin{equation}
% \begin{aligned}
% \E[\E_N[\phi(O|\theta)(c) - \phi(O)(c)]] &= \E[\phi(O|\theta)(c) - \phi(O)(c)]\\
% \var[\E_N[\phi(O|\theta)(c) - \phi(O)(c)]] &= \frac{1}{N}\var[\phi(O|\theta)(c) - \phi(O)(c)] \le 
% \end{aligned}
% \end{equation}
\begin{equation}
P(|(\E_N - \E)[\phi(O|\theta)(c) - \phi(O)(c)]| \ge t) \le \frac{\var[\E_N[\phi(O|\theta)(c) - \phi(O)(c)]]}{t^2} = \frac{\var[\phi(O|\theta)(c) - \phi(O)(c)]}{Nt^2}
\end{equation}
note here that $\theta$ is independent of the samples used for $\E_N$ by assumption \ref{assumption:extra}. For any $\varepsilon > 0$, by picking $t = \frac{1}{\sqrt{N\varepsilon}}$ we get
\begin{equation}
P\left(\left|\frac{(\E_N - \E)[\phi(O|\theta)(c) - \phi(O)(c)]}{N^{-1/2}}\right| \ge \frac{1}{\sqrt{\varepsilon}}\right) \le \varepsilon \var[\phi(O|\theta)(c) - \phi(O)(c)]
\end{equation}
by the definition of $O_P$, we then get
\begin{equation}
(\E_N - \E)[\phi(O|\theta)(c) - \phi(O)(c)] = O_P(N^{-1/2}\var[\phi(O|\theta)(c) - \phi(O)(c)])
\end{equation}
Because $\phi$ is a continuous function of $P(Y|X,\theta)$ and $P(A|Y,\theta)$ (given $P(A|Y,\theta) > \epsilon$, assumption \ref{assumption:extra}), by the continuous mapping theorem and the fact that $P(Y|X,\theta)$ and $P(A|Y,\theta)$ are convergent in probability (assumption \ref{assumption:4th-root-n}), we get $\var[\phi(O|\theta)(c) - \phi(O)(c)] = o_P(1)$. This gives
\begin{equation}
(\E_N - \E)[\phi(O|\theta)(c) - \phi(O)(c)] = o_P(N^{-1/2})
\end{equation}
Therefore, we have shown that the second and third term are both $o_P(N^{-1/2})$, proving \cref{eq:proof-linearity}. As the final step, multiply both sides of this equation by $\sqrt{N}$ we get:
\begin{equation}
\sqrt{N}(\Psi_{dr}(\theta)(c) - P(Y=c)) = \sqrt{N} (\E_N - \E)[\phi(O)(c)] + o_P(1) \rightsquigarrow \mathcal{N}(0, \var[\phi(O)(c)])
\end{equation}
by the central limit theorem, and $\var[\phi(O)(c)] = \E[\phi(O)(c)^2]$ because $\E[\phi(O)(c)] = 0$.
\end{proof}

While we started with the definition of $\phi$, \cref{eq:proof-linearity} shows that $\phi$ is indeed an influence function. Now we show that $\phi$ is also the efficient influence function, by using the characterization of the model's tangent space \cite{tsiatis-missingdata}. Note that the joint probability factorizes as $P(X,A,Y) = P(X)P(Y|X)P(A|Y)$, therefore the tangent space $\mathcal{T}$ factorizes as $\mathcal{T} = \mathcal{T}_{X} \oplus \mathcal{T}_{Y|X} \oplus \mathcal{T}_{A|Y}$ where $\mathcal{T}_X = \{h(X): \E[h] = 0\}$, $\mathcal{T}_{Y|X} = \{h(X,Y): \E[h|X] = 0\}$, $\mathcal{T}_{A|Y} = \{h(A,Y): \E[h|Y] = 0\}$, and the 3 subspaces are pairwise orthogonal. All influence functions are orthogonal to the tangent space, but the influence function that is also in the tangent space has the smallest variance and is called the efficient influence function. As $\phi$ is already an influence function, we need only show that $\phi$ is in $\mathcal{T}$. We write $\phi$ as
\begin{equation}
\phi(O)(c) = (P(Y=c|X) - P(Y=c)) + \left[\frac{\one(A=1)}{P(A=1|Y)} - 1\right](\one(Y=c) - P(Y=c|X)) + (\one(Y=c) - P(Y=c|X))
\end{equation}
and note that the first, second and third term are in $\mathcal{T}_X$, $\mathcal{T}_{A|Y}$ and $\mathcal{T}_{Y|X}$ respectively. Therefore, $\phi$ is indeed in $\mathcal{T}$. The efficient influence function has the smallest variance of all influence function, and therefore our estimator being asymptotically linear in $\phi$ (\cref{eq:proof-linearity}) has the smallest mean squared error in a local asymptotic minimax sense \cite{kennedy-dr, asymptoticstatistics}

\section{Further background and related work}
\paragraph{Discussion on semi-supervised EM.}
It appears that semi-supervised EM was first used for parameter estimation when the missingness mechanism is non-ignorable in \cite{ibrahim1996parameter}, but has not been used for label shift estimation.
Perhaps this is because the semi-supervised situation where additional unlabeled data is available during training is rarer than the test-time adaptation case. EM is well suited to take advantage of the extra unlabeled data to improve the classifier under very scarce and long-tailed labeled data. While the connection between pseudo-labeling and EM has been explored before \cite{entropyminimization}, the situation with label shift has not until recently \cite{simpro}. Here the application of EM is much more interesting, because other than simply giving pseudo-labeling a rigorous formulation, EM also estimates the missingness mechanism (equivalently the label distribution shift), which is important for shift correction and thus high-quality pseudo-labels \cite{acr}. The application of confidence thresholding can be seen as a sparse variant of EM \cite{neal1998view}.

\paragraph{The doubly-robust risk.} 
\label{subsec:dr-risk}
A technique that also derives from the theory of semi-parametric efficiency is orthogonal statistical learning \citep{foster2023orthogonal}. The idea is to minimize the doubly-robust risk:
\label{subsec:method-dr-risk}
\begin{equation}
\label{eq:dr-risk}
\mathcal{R}(\theta_2) = \frac{1}{N} \sum_{i=1}^N \Bigg[ l(x_i, \hat y_i|\theta_2) + \frac{\one(a_i=1)}{P(A=a_i|Y=y_i, \theta_1)} (l(x_i, y_i | \theta_2) - l(x_i, \hat y_i | \theta_2))\Bigg]
\end{equation}
where $l(x,y|\theta) = -\sum_{c=1}^C [y]_c \log P(Y=c|X=x,\theta)$ is the negative cross-entropy. 
The notation $[y]_c$ means that we are using the $c$-entry in a C-dimension probability vector $y$. 
Thus, $y_i$ denotes the one-hot label of observation $i$, while $\hat y_i$ denotes the pseudo-label, which can be one-hot or all-zero. 
Finally, we use $\theta_1$ to denote that $P(a|y,\theta_1)$ is an estimation from a previous stage, but it can be estimated with $\theta_2$ as well. 
The risk $\mathcal{R}(\theta_2)$ can be used as a training loss in a straightforward fashion. 
Similar to the doubly robust estimation of $P(Y)$, the doubly robust risk provides approximately unbiased estimation of the risk. 
This property has been used in \citep{arelabelsinformative, onnonrandommissinglabels, drst} also in the semi-supervised learning setting.
More broadly, it is at the heart of one of the core techniques in heterogenous treatment effect estimation in causal estimation \cite{kennedy2023towards, foster2023orthogonal, wager2018estimation}. 
The focus here is not the estimation of $\mathcal{R}(\theta_2)$ per se, but the quality of the learned model \cite{foster2023orthogonal}.
By using the doubly-robust risk, we can achieve an optimality result similar in spirit to our theorem \cref{theorem:dr}, but for the generalization error.
While this is appealing, in practice there are 2 problems with this approach. First, the inverse probability weight $P(A=a_i|Y=y_i,\theta_1)$ can be very large if the class ratio is highly unlabeled, making training unstable \cite{kallus2020deepmatch, pham2023stable}. 
This problem exists for our estimation as well. However, it is much easier to control for estimation than for training because of the iterative nature of model update. Secondly, we can further write $\mathcal{R}$ as:
\begin{equation}
\mathcal{R}(\theta_2) = \frac{1}{N}\sum_{i=1}^N l\left(x_i, \hat y_i + \frac{\one(a_i=1)}{P(A=a_i|Y=y_i,\theta_1)} (y_i - \hat y_i)\Bigg\vert\theta_2\right)
\end{equation}
which is a cross-entropy loss with new meta-pseudo-labels. However, these labels are not meant to be learned exactly, and furthermore they can be negative. Thus, theoretical works have to put stringent assumptions on the models. In \cref{subsec:ablation-1}, we show that experimentally that the instability problem makes doubly-robust risk performance worse than our 2-stage approach.

\section{Training and hyperparameter settings.}
\label{subsec:training-setting}
For neural network training, we follow the implementation and hyperparameter settings of \cite{simpro}. In particular, we adapt the core code of SimPro for Supervised, MLE and EM. For MLE, we update $P(A|Y)$ using the Adam optimizer with learning rate 1e-3, while for EM we use a momentum update similar to SimPro's update of $P(Y|A)$ because it has a a closed-form solution at each mini-batch. We use Wide ResNet-28-2 on all methods and all datasets in this section, including Imagenet-127, because we are motivated by the fact that stage-1's goal is not classification accuracy but the estimation of a finite-dimensional parameter. When using Wide ResNet-28-2 for Imagenet-127, we use the hyperparameters of CIFAR-100, except we lower the batch size of unlabeled data to 2 times that of labeled data instead of 8 for memory reason. We do not perform additional hyperparameter tuning. All experiments can be performed on 1 A6000 RTX GPU, and are run 3 times. We report the total variation distance between the estimated and the ground truth unlabeled class distribution, similar to its usage in Theorem 3.1 of \cite{lsc}, and the top-1 classification accuracy.

In the second stage of our algorithm, we freeze our estimation and plug it in SimPro and BOAT.
We keep exactly the same hyperparameter settings that SimPro and BOAT use. In particular, for Imagenet-127, we now use ResNet-50 and run each experiment once.
In SimPro, we set the unlabeled class distribution $P(Y|A=0)$ at the E-step;  however, we still keep a running estimate of the class distribution $P(Y)$ in the logit adjustment loss \cref{eq:simpro-la-loss}. While it is possible to use the first stage estimate in the logit adjustment loss, we observe that doing so results in lower accuracy than using the the running average. This is conceptually consistent with the role of the running average - serving not as an accurate estimate of $P(Y)$ but to make the classifier's class distribution uniform through the logit adjustment loss, which is good for the test set. Similarly, in BOAT, we only replace $\Delta_c = \log P(Y|A=1) - \log P(Y|A=0)$ in equation (4) of \cite{boat}, which is adjusting a classifier's predictions from the labeled to the unlabeled class distribution, with our SimPro + DR estimate instead of their on-the-fly estimate. 


% \section{Additional experiments}
% % \begin{table*}[t]
\centering
\caption{Total Variation Distance on CIFAR-10-LT ($N_l = 500$, $M_l = 4000$) with different class imbalance ratios $\gamma_l$ and $\gamma_u$ under five different unlabeled class distributions.}
\label{tab:cifar10-tv}
\resizebox{\textwidth}{!}{
\begin{tabular}{lccccccccccc}
\toprule
& & \multicolumn{2}{c}{consistent} & \multicolumn{2}{c}{uniform} & \multicolumn{2}{c}{reversed} & \multicolumn{2}{c}{middle} & \multicolumn{2}{c}{head-tail} \\
\cmidrule(lr){3-4} \cmidrule(lr){5-6} \cmidrule(lr){7-8} \cmidrule(lr){9-10} \cmidrule(lr){11-12}
& & $\gamma_l = 150$ & $\gamma_l = 100$ & $\gamma_l = 150$ & $\gamma_l = 100$ & $\gamma_l = 150$ & $\gamma_l = 100$ & $\gamma_l = 150$ & $\gamma_l = 100$ & $\gamma_l = 150$ & $\gamma_l = 100$ \\
Model & Estimator & $\gamma_u = 150$ & $\gamma_u = 100$ & $\gamma_u = 1$ & $\gamma_u = 1$ & $\gamma_u = 1/150$ & $\gamma_u = 1/100$ & $\gamma_u = 150$ & $\gamma_u = 100$ & $\gamma_u = 150$ & $\gamma_u = 100$ \\
\midrule
Supervised & MLLS & 0.269 ± 0.252 & 0.038 ± 0.006 & 0.251 ± 0.046 & 0.255 ± 0.060 & 0.429 ± 0.028 & 0.493 ± 0.050 & 0.333 ± 0.042 & 0.320 ± 0.009 & 0.457 ± 0.034 & 0.444 ± 0.043 \\
Supervised & RLLS & 0.043 ± 0.001 & 0.044 ± 0.010 & 0.348 ± 0.034 & 0.305 ± 0.068 & 0.769 ± 0.016 & 0.678 ± 0.028 & 0.430 ± 0.008 & 0.368 ± 0.013 & 0.539 ± 0.018 & 0.503 ± 0.020 \\
\midrule
MLE & IPW & 0.027 ± 0.001 & 0.027 ± 0.000 & 0.319 ± 0.072 & 0.243 ± 0.010 & 0.674 ± 0.020 & 0.646 ± 0.041 & 0.438 ± 0.020 & 0.454 ± 0.026 & 0.547 ± 0.049 & 0.491 ± 0.059 \\
MLE & OR & 0.045 ± 0.004 & 0.042 ± 0.000 & 0.215 ± 0.026 & 0.203 ± 0.032 & 0.433 ± 0.017 & 0.395 ± 0.033 & 0.193 ± 0.006 & 0.209 ± 0.037 & 0.307 ± 0.147 & 0.249 ± 0.130 \\
MLE & DR & 0.090 ± 0.002 & 0.079 ± 0.000 & 0.407 ± 0.027 & 0.360 ± 0.007 & 0.425 ± 0.007 & 0.421 ± 0.029 & 0.256 ± 0.001 & 0.286 ± 0.031 & 0.435 ± 0.136 & 0.362 ± 0.122 \\
\midrule
EM & IPW & 0.035 ± 0.002 & 0.040 ± 0.001 & 0.021 ± 0.001 & 0.029 ± 0.015 & 0.303 ± 0.187 & 0.091 ± 0.010 & 0.119 ± 0.011 & 0.105 ± 0.022 & 0.104 ± 0.026 & 0.104 ± 0.051 \\
EM & OR & 0.037 ± 0.003 & 0.042 ± 0.002 & 0.016 ± 0.001 & 0.024 ± 0.012 & 0.269 ± 0.183 & 0.090 ± 0.008 & 0.122 ± 0.012 & 0.103 ± 0.022 & 0.072 ± 0.012 & 0.073 ± 0.024 \\
EM & DR & 0.034 ± 0.004 & 0.037 ± 0.001 & 0.014 ± 0.001 & 0.027 ± 0.020 & 0.264 ± 0.191 & 0.092 ± 0.005 & 0.111 ± 0.019 & 0.097 ± 0.026 & 0.077 ± 0.016 & 0.073 ± 0.028 \\
\midrule
SimPro & IPW & 0.070 ± 0.011 & 0.058 ± 0.000 & 0.046 ± 0.001 & 0.049 ± 0.005 & 0.254 ± 0.074 & 0.223 ± 0.098 & 0.097 ± 0.025 & 0.067 ± 0.002 & 0.105 ± 0.066 & 0.110 ± 0.079 \\
SimPro & OR & 0.071 ± 0.012 & 0.058 ± 0.000 & 0.045 ± 0.001 & 0.049 ± 0.006 & 0.040 ± 0.003 & 0.059 ± 0.017 & 0.074 ± 0.006 & 0.075 ± 0.002 & 0.033 ± 0.003 & 0.033 ± 0.003 \\
SimPro & DR & 0.017 ± 0.004 & 0.026 ± 0.001 & 0.019 ± 0.002 & 0.018 ± 0.003 & 0.039 ± 0.003 & 0.058 ± 0.025 & 0.091 ± 0.007 & 0.031 ± 0.001 & 0.015 ± 0.003 & 0.019 ± 0.007 \\
\bottomrule
\end{tabular}
}
\end{table*}
% 

\begin{table*}[t]
\centering
\caption{Total Variation Distance on CIFAR-100-LT ($N_l = 50$, $M_l = 400$) with different class imbalance ratios $\gamma_l$ and $\gamma_u$ under five different unlabeled class distributions.}
\label{tab:cifar100-tv}
\resizebox{\textwidth}{!}{
\begin{tabular}{lccccccccccc}
\toprule
& & \multicolumn{2}{c}{consistent} & \multicolumn{2}{c}{uniform} & \multicolumn{2}{c}{reversed} & \multicolumn{2}{c}{middle} & \multicolumn{2}{c}{head-tail} \\
\cmidrule(lr){3-4} \cmidrule(lr){5-6} \cmidrule(lr){7-8} \cmidrule(lr){9-10} \cmidrule(lr){11-12}
& & $\gamma_l = 20$ & $\gamma_l = 10$ & $\gamma_l = 20$ & $\gamma_l = 10$ & $\gamma_l = 20$ & $\gamma_l = 10$ & $\gamma_l = 20$ & $\gamma_l = 10$ & $\gamma_l = 20$ & $\gamma_l = 10$ \\
Model & Estimator & $\gamma_u = 20$ & $\gamma_u = 10$ & $\gamma_u = 1$ & $\gamma_u = 1$ & $\gamma_u = 1/20$ & $\gamma_u = 1/10$ & $\gamma_u = 20$ & $\gamma_u = 10$ & $\gamma_u = 20$ & $\gamma_u = 10$ \\
\midrule
Supervised & MLLS & 0.707 ± 0.016 & 0.313 ± 0.100 & 0.445 ± 0.172 & 0.309 ± 0.119 & 0.383 ± 0.075 & 0.397 ± 0.006 & 0.570 ± 0.001 & 0.373 ± 0.107 & 0.543 ± 0.009 & 0.231 ± 0.057 \\
Supervised & RLLS & 0.520 ± 0.007 & 0.133 ± 0.003 & 0.337 ± 0.125 & 0.253 ± 0.082 & 0.424 ± 0.060 & 0.463 ± 0.003 & 0.454 ± 0.021 & 0.306 ± 0.074 & 0.460 ± 0.028 & 0.241 ± 0.040 \\
\midrule
MLE & IPW & 0.075 ± 0.000 & 0.071 ± 0.001 & 0.229 ± 0.001 & 0.167 ± 0.002 & 0.565 ± 0.005 & 0.443 ± 0.007 & 0.415 ± 0.000 & 0.311 ± 0.005 & 0.343 ± 0.000 & 0.280 ± 0.001 \\
MLE & OR & 0.065 ± 0.002 & 0.061 ± 0.001 & 0.200 ± 0.007 & 0.143 ± 0.001 & 0.526 ± 0.011 & 0.399 ± 0.023 & 0.360 ± 0.003 & 0.256 ± 0.012 & 0.328 ± 0.003 & 0.266 ± 0.005 \\
MLE & DR & 0.149 ± 0.019 & 0.145 ± 0.010 & 0.243 ± 0.004 & 0.214 ± 0.019 & 0.568 ± 0.005 & 0.464 ± 0.014 & 0.403 ± 0.014 & 0.309 ± 0.012 & 0.365 ± 0.007 & 0.320 ± 0.004 \\
\midrule
EM & IPW & 0.097 ± 0.008 & 0.092 ± 0.004 & 0.239 ± 0.007 & 0.179 ± 0.003 & 0.478 ± 0.012 & 0.329 ± 0.020 & 0.262 ± 0.016 & 0.202 ± 0.003 & 0.312 ± 0.002 & 0.227 ± 0.001 \\
EM & OR & 0.121 ± 0.007 & 0.108 ± 0.005 & 0.261 ± 0.007 & 0.189 ± 0.004 & 0.489 ± 0.013 & 0.335 ± 0.020 & 0.274 ± 0.016 & 0.211 ± 0.004 & 0.336 ± 0.003 & 0.235 ± 0.001 \\
EM & DR & 0.125 ± 0.005 & 0.111 ± 0.004 & 0.269 ± 0.007 & 0.194 ± 0.005 & 0.497 ± 0.010 & 0.336 ± 0.024 & 0.281 ± 0.019 & 0.219 ± 0.008 & 0.336 ± 0.007 & 0.233 ± 0.004 \\
\midrule
SimPro & IPW & 0.125 ± 0.001 & 0.100 ± 0.005 & 0.166 ± 0.007 & 0.141 ± 0.009 & 0.353 ± 0.023 & 0.261 ± 0.008 & 0.202 ± 0.003 & 0.158 ± 0.005 & 0.277 ± 0.009 & 0.197 ± 0.003 \\
SimPro & OR & 0.133 ± 0.005 & 0.100 ± 0.004 & 0.160 ± 0.007 & 0.138 ± 0.010 & 0.322 ± 0.014 & 0.253 ± 0.008 & 0.202 ± 0.003 & 0.156 ± 0.005 & 0.269 ± 0.006 & 0.191 ± 0.004 \\
SimPro & DR & 0.122 ± 0.003 & 0.106 ± 0.006 & 0.188 ± 0.009 & 0.149 ± 0.006 & 0.343 ± 0.023 & 0.257 ± 0.007 & 0.219 ± 0.010 & 0.172 ± 0.002 & 0.279 ± 0.007 & 0.198 ± 0.004 \\
\bottomrule
\end{tabular}
}
\end{table*}

\section*{Supplementary Material}


\subsection*{A. Additional Experimental Results}

\subsubsection*{A.1 Evaluation on Unseen Audio Categories}
% Test setup for unseen categories
    % Test Audio:
    % Thunderstorm, 
    % Chicken, rooster
    % Sheep
    % Tuning fork
    % Washing machine, Shower, Subway, Human Cough
    
% Performance metrics
% Comparison with baseline models

To assess generalization capability, we evaluate model performance on four diverse unseen audio (tuning fork, rooster, sheep, and thunderstorm) across both S4 and MS3 datasets.



Tables~\ref{tab:unseen_audio_s4} and~\ref{tab:unseen_audio_ms3} present comparative results between the baseline AVSBench and our approach. On S4, the baseline exhibits significant visual bias, maintaining consistently high performance (mIoU: $\sim$78\%, F-score: $\sim$88\%) regardless of audio input. Our approach effectively mitigates this bias, reducing mIoU to near-zero and decreasing false positive ratios from 0.187 to 10\textsuperscript{-5}.


The improvement is equally pronounced in the more challenging multi-source scenario (MS3). Our method effectively suppresses false predictions, achieving very low mIoU ($\sim 0.09$) and F-scores ($\sim 0.17$) across all unseen categories. For certain categories (e.g., rooster), our approach achieves complete suppression with zero false positives, demonstrating robust audio-visual correspondence even in complex multi-source scenarios.



\subsubsection*{A.2 Multi-Source (MS3) Dataset Visualizations}

Figure~\ref{fig:ms3_vis} illustrates our method's performance on complex multi-source scenarios. The visualizations demonstrate segmentation results under various audio conditions: original audio (positive), silence, noise, and off-screen sounds, providing qualitative evidence of our model's effectiveness in handling diverse acoustic environments.


\begin{table}[t]
    \footnotesize
    \centering
    \begin{tabular}{l|c|c|c|c}
        \hline
        Category & Method & mIoU $\downarrow$ & F-score $\downarrow$ & FPR $\downarrow$ \\
        \hline
        \multirow{2}{*}{Tuning Fork} & AVSBench & 78.14 & 88.16 & 0.187 \\
        & AVSBench $+$ Ours & \textbf{0.003} & \textbf{0.248} & \textbf{0.0001} \\
        \hline
        \multirow{2}{*}{Rooster} & AVSBench & 78.19 & 88.26 & 0.186 \\
        & AVSBench $+$ Ours & \textbf{0.002} & \textbf{0.234} & \textbf{3.82e-5} \\
        \hline
        \multirow{2}{*}{Sheep} & AVSBench & 78.26 & 88.25 & 0.187 \\
        & AVSBench $+$ Ours & \textbf{0.002} & \textbf{0.236} & \textbf{8.43e-5} \\
        \hline
        \multirow{2}{*}{Thunder} & AVSBench & 78.19 & 88.23 & 0.187 \\
        & AVSBench $+$ Ours & \textbf{0.002} & \textbf{0.233} & \textbf{4.03e-5} \\
        \hline
    \end{tabular}
    \caption{Performance comparison on unseen audio categories for single-source (S4) dataset. Lower values indicate better ability to avoid false predictions for unfamiliar sounds.}
    \label{tab:unseen_audio_s4}    
\end{table}

\begin{table}[t]
    \footnotesize
    \centering
    \begin{tabular}{l|c|c|c|c}
        \hline
        Category & Method & mIoU $\downarrow$ & F-score $\downarrow$ & FPR $\downarrow$ \\
        \hline
        \multirow{2}{*}{Tuning Fork} & AVSBench & 41.44 & 62.19 & 0.091 \\
        & AVSBench $+$ Ours & \textbf{0.103} & \textbf{0.186} & \textbf{0.007} \\
        \hline
        \multirow{2}{*}{Rooster} & AVSBench & 44.52 & 63.04 & 0.102 \\
        & AVSBench $+$ Ours & \textbf{0.091} & \textbf{0.169} & \textbf{0.000} \\
        \hline
        \multirow{2}{*}{Sheep} & AVSBench & 40.14 & 62.13 & 0.082 \\
        & AVSBench $+$ Ours & \textbf{0.091} & \textbf{0.179} & \textbf{0.0001} \\
        \hline
        \multirow{2}{*}{Thunder} & AVSBench & 38.16 & 61.82 & 0.074 \\
        & AVSBench $+$ Ours & \textbf{0.091} & \textbf{0.178} & \textbf{9.07e-5} \\
        \hline
    \end{tabular}
    \caption{Performance comparison on unseen audio categories for multi-source (MS3) dataset. Lower values indicate better ability to avoid false predictions for unfamiliar sounds.}
    \label{tab:unseen_audio_ms3}    
\end{table}

\begin{figure*}[t]
\vspace{-2mm}
    \centering
    \includegraphics[width=1.0\textwidth]{avs_examples_ms3.pdf}
    % \caption{Qualitative results on S4 dataset. We can see that \yapeng{xxxx} \yapeng{We also need qualitative results on MS3. If we do not have space, we should add the results in appendix}}

    \caption{Performance comparison of different AVS models under various audio conditions on Robust-MS3 dataset. Existing SOTA methods \cite{liu2024annofree, ma2024stepping, chen2024cavp} segment objects primarily based on visual salience, exhibiting a strong visual bias. In contrast, our approach achieves accurate segmentation with original audio while successfully reject predict in negative scenarios (e.g., silence, noise, off-screen).}
    \label{fig:ms3_vis}
\end{figure*}


\subsubsection*{A.3 Effect of feature alignment strategies}

To thoroughly evaluate our choice of cosine similarity for feature alignment, we compare it with two intuitive alternatives: Euclidean distance, which directly measures feature space proximity, and concatenation-based alignment, which preserves complete feature information. Table~\ref{table:feature_alignment} presents comparative results across both S4 and MS3 datasets.


\begin{table*}[htbp]
\centering
\resizebox{\textwidth}{!}{%
\begin{tabular}{c|c|c|c|c|c|c|c|c|c|c|c|c|c|c|c}
\hline
& \multirow{3}{*}{\begin{tabular}[c]{@{}c@{}}Guide Method\end{tabular}} & \multicolumn{2}{c|}{\textbf{Positive audio input}} & \multicolumn{9}{c|}{\textbf{Negative audio input}} & \multicolumn{3}{c}{\textbf{Global metric}} \\
\cline{5-13}
& & \multicolumn{2}{c|}{} & \multicolumn{3}{c|}{\textbf{Silence}} & \multicolumn{3}{c|}{\textbf{Noise}} & \multicolumn{3}{c|}{\textbf{Offscreen sound}} & \multicolumn{3}{c}{} \\
\cline{3-16}
\textbf{Test set} & & \textbf{mIoU↑} & \textbf{F-score↑} & \textbf{mIoU↓} & \textbf{F-score↓} & \textbf{FPR↓} & \textbf{mIoU↓} & \textbf{F-score↓} & \textbf{FPR↓} & \textbf{mIoU↓} & \textbf{F-score↓} & \textbf{FPR↓} & \textbf{G-mIoU↑} & \textbf{G-F↑} & \textbf{G-FPR↓} \\
\hline
\multirow{3}{*}{S4} & cosine & \textbf{78.1} & \textbf{88.2} & \textbf{0.2} & \textbf{22.6} & \textbf{0.000} & \textbf{0.2} & \textbf{22.6} & \textbf{0.000} & \textbf{0.2} & \textbf{22.6} & \textbf{0.00} & \textbf{87.672} & \textbf{82.461} & \textbf{0.000} \\
& Euclidean& 69.3& 82.5& 5.9& 33.8& 0.032& 0.2& 23.8& 0.000& 0.2& 24.3& 0.000& 81.144& 77.283& 0.004\\
& Concat& 77.5& 87.5& 1.0& 23.1& 0.003& 0.2& 22.6& 0.000& 0.2& 22.6& 0.000& 87.139& 82.047& 0.000\\
\hline
\multirow{3}{*}{MS3} & cosine 
& \textbf{51.3} & \textbf{64.5} & \textbf{9.8} & \textbf{17.7} & \textbf{0.00} & \textbf{9.9} & \textbf{25.8} & \textbf{0.000} & \textbf{9.1} & \textbf{20.3} & \textbf{0.000} & \textbf{66.605} & \textbf{70.590} & \textbf{0.004} \\
& Euclidean
& 43.5& 56.4& 10.7& 21.8& 0.011& 11.1& 24.2& 0.028& 9.5& 28.3& 0.001& 58.526& 64.447& 0.014\\
& Concat& 49.7& 62.8& 12.1& 21.0& 0.006& 14.8& 32.3& 0.012& 14.1& 29.0& 0.011& 63.053& 67.333& 0.011\\
\hline
\end{tabular}
}
\caption{Comparison of different feature alignment strategies. Results show performance across positive and negative audio scenarios, as well as global metrics.}
\label{table:feature_alignment}
\end{table*}

All three methods demonstrate effectiveness in audio-visual feature alignment, with each achieving reasonable performance on positive samples. However, cosine similarity exhibits superior performance, particularly in negative suppression scenarios. On S4, while concatenation-based alignment maintains competitive positive metrics (mIoU: 77.5\%), cosine similarity achieves better balance between positive performance (mIoU: 78.1\%) and negative suppression (FPR: 0.00). This pattern extends to the more challenging MS3 dataset, where cosine similarity shows notably better global metrics (G-mIoU: 66.61, G-F: 70.59) compared to both alternatives.

The advantage of cosine similarity likely stems from its inherent normalization property and focus on directional relationships, making it particularly suitable for cross-modal feature comparison. While Euclidean distance is sensitive to feature magnitude variations and concatenation may preserve redundant information, cosine similarity captures the essential semantic alignment between modalities while maintaining computational efficiency.


\noindent\textbf{Implementation Details:} For all methods, audio features are first projected from 128 to 256 dimensions to match the visual feature dimension, and visual features undergo adaptive average pooling to obtain global representations. The methods then differ in their alignment computation:
\begin{itemize}
\item \textbf{Cosine Similarity} computes normalized directional alignment using the standard cosine similarity function: $\text{similarity} = \cos(\hat{\mathcal{F}}_A, \hat{\mathcal{F}}_V)$, where $\hat{\mathcal{F}}_A$ and $\hat{\mathcal{F}}_V$ are the projected audio and visual features respectively.
\item \textbf{Euclidean Distance} measures direct feature space proximity through L2 norm: $\text{similarity} = -||\hat{\mathcal{F}}_A - \hat{\mathcal{F}}_V||_2$. The negative sign converts distance to similarity, ensuring larger values indicate stronger alignment.
\item \textbf{Concatenation-based} alignment employs a three-layer MLP that processes the concatenated features $[\hat{\mathcal{F}}_A; \hat{\mathcal{F}}_V]$ (512 dimensions). The network progressively reduces dimensionality (512 → 256 → 128 → 1) with ReLU activation and dropout (0.1) for regularization, learning more complex non-linear relationships between modalities.
\end{itemize}
% For concatenation-based alignment, we implement a multi-layer perceptron (MLP) architecture that processes the concatenated audio-visual features. The audio features are first projected from 128 to 256 dimensions to match the visual feature dimension. Visual features undergo adaptive average pooling to obtain a global representation. The concatenated 512-dimensional features are then processed through a three-layer MLP with ReLU activation and dropout (0.1) for regularization. The network progressively reduces feature dimensionality (512 → 256 → 128 → 1) before outputting the final alignment score. This architecture allows the model to learn complex non-linear relationships between the modalities while maintaining computational efficiency.


\subsubsection*{A.4 Sensitivity to hyperparameter choices}

We evaluate the model's sensitivity to the classifier guidance weight $\lambda_{\text{BCE}}$ by varying its value from 0.2 to 1.0. Table~\ref{table:lambda_sensitivity} presents the quantitative results across both S4 and MS3 datasets.

\begin{table*}[t]
    \centering
    % \caption{Performance analysis with different $\lambda_{\text{BCE}}$ values on S4 and MS3 datasets. Results show the trade-off between positive performance and negative suppression across different weight settings.}
    \resizebox{\linewidth}{!}{
    \begin{tabular}{c|c|cc|ccc|ccc|ccc|ccc}
        \hline
        Dataset & $\lambda_{\text{BCE}}$ & \multicolumn{2}{c|}{Positive} & \multicolumn{3}{c|}{Silence} & \multicolumn{3}{c|}{Noise} & \multicolumn{3}{c|}{Offscreen} & \multicolumn{3}{c}{Global} \\
        \cline{3-16}
        & & mIoU$\uparrow$ & F$\uparrow$ & mIoU$\downarrow$ & F$\downarrow$ & FPR$\downarrow$ & mIoU$\downarrow$ & F$\downarrow$ & FPR$\downarrow$ & mIoU$\downarrow$ & F$\downarrow$ & FPR$\downarrow$ & G-mIoU$\uparrow$ & G-F$\uparrow$ & G-FPR$\downarrow$ \\
        \hline
        \multirow{5}{*}{S4} & 1.0 & 78.15 & 88.22 & 0.16 & 22.59 & 0.00 & 0.16 & 22.59 & 0.00 & 0.16 & 22.59 & 0.00 & 87.67 & 82.46 & 0.000 \\
        & 0.8 & 78.97 & 88.78 & 0.16 & 22.60 & 0.00 & 0.16 & 22.60 & 0.00 & 0.16 & 22.60 & 0.00 & 88.18 & 82.70 & 0.000 \\
        & 0.6 & 77.98 & 88.00 & 0.18 & 22.64 & 0.00 & 0.17 & 22.63 & 0.00 & 0.17 & 22.64 & 0.00 & 87.56 & 82.34 & 0.000 \\
        & 0.4 & 78.76 & 88.56 & 1.06 & 32.21 & 0.00 & 0.16 & 22.59 & 0.00 & 0.16 & 22.65 & 0.00 & 87.54 & 80.73 & 0.000 \\
        & 0.2 & 78.95 & 88.68 & 0.22 & 22.93 & 0.00 & 0.17 & 22.85 & 0.00 & 0.18 & 22.85 & 0.00 & 88.16 & 82.50 & 0.000 \\
        \hline
        \multirow{5}{*}{MS3} & 1.0 & 51.27 & 64.51 & 9.84 & 17.72 & 0.00 & 9.93 & 25.78 & 0.00 & 9.11 & 20.33 & 0.00 & 65.43 & 70.91 & 0.001 \\
        & 0.8 & 53.32 & 66.32 & 9.64 & 18.09 & 0.00 & 11.75 & 35.07 & 0.01 & 9.81 & 31.23 & 0.00 & 66.86 & 68.98 & 0.004 \\
        & 0.6 & 51.23 & 64.94 & 13.37 & 21.38 & 0.03 & 15.20 & 37.80 & 0.02 & 9.67 & 26.79 & 0.00 & 64.55 & 67.99 & 0.011 \\
        & 0.4 & 52.75 & 65.85 & 13.66 & 26.41 & 0.02 & 10.33 & 35.82 & 0.01 & 10.33 & 35.82 & 0.01 & 66.12 & 66.57 & 0.007 \\
        & 0.2 & 52.81 & 65.56 & 9.81 & 18.01 & 0.00 & 12.99 & 35.02 & 0.01 & 9.81 & 21.94 & 0.00 & 66.32 & 69.97 & 0.005 \\
        \hline
    \end{tabular}
    }
    \caption{Performance analysis with different $\lambda_{\text{BCE}}$ values. Results demonstrate the model's robustness to this hyperparameter choice.}    
    \label{table:lambda_sensitivity}
    
    
\end{table*}

Experimental results demonstrate strong robustness to this hyperparameter choice. On S4, the G-mIoU variation remains minimal (87.54-88.18\%, $\Delta$=0.64\%), with consistent perfect negative suppression (FPR: 0.000) across all settings. The MS3 dataset shows slightly larger but still modest variations (G-mIoU: 64.55-66.86\%, $\Delta$=2.31\%), attributable to its inherently more complex multi-source scenarios. This stability suggests that the model's performance is not heavily dependent on precise hyperparameter tuning.

\subsubsection*{A.5 Detailed Analysis of the Effect of Negative Samples and Classifier Guidance}
To dissect the individual contributions of our key components, we conduct ablation experiments on both negative sample integration and classifier guidance. Table~\ref{table:full_comparison}, compare to Table 4 presents comprehensive results across different configurations.
Adding negative samples alone proves insufficient and can even degrade performance. On S4, while the baseline achieves 78.7\% mIoU with high false positives (FPR: ~0.19), incorporating only negative samples marginally improves robustness but compromises positive performance (mIoU: 79.0\%). This effect is more pronounced on MS3, where positive performance significantly degrades (mIoU drops from 54.0\% to 51.6\%) while maintaining high false positive rates. This degradation occurs because the model, without explicit guidance, struggles to establish clear decision boundaries between valid and invalid audio-visual correspondences.
The integration of classifier guidance ($\mathcal{L}_{\text{BCE}}$) with negative samples yields substantial improvements. This combination achieves 78.1\% mIoU on S4 while reducing FPR to nearly zero across all negative scenarios. Similarly on MS3, it maintains competitive positive performance (51.3\% mIoU) while effectively suppressing false predictions (FPR: 0.004). 


\begin{table*}[htbp]
\centering
\resizebox{\textwidth}{!}{%
\begin{tabular}{c|c|c|c|c|c|c|c|c|c|c|c|c|c|c|c|c}
\hline
& \multirow{3}{*}{\begin{tabular}[c]{@{}c@{}}Negative\\samples\end{tabular}} & \multirow{3}{*}{$\mathcal{L}_{\text{BCE}}$} & \multicolumn{2}{c|}{\textbf{Positive audio input}} & \multicolumn{9}{c|}{\textbf{Negative audio input}} & \multicolumn{3}{c}{\textbf{Global metric}} \\
\cline{6-14}
& & & \multicolumn{2}{c|}{} & \multicolumn{3}{c|}{\textbf{Silence}} & \multicolumn{3}{c|}{\textbf{Noise}} & \multicolumn{3}{c|}{\textbf{Offscreen sound}} & \multicolumn{3}{c}{} \\
\cline{4-17}
\textbf{Test set} & & & \textbf{mIoU↑} & \textbf{F-score↑} & \textbf{mIoU↓} & \textbf{F-score↓} & \textbf{FPR↓} & \textbf{mIoU↓} & \textbf{F-score↓} & \textbf{FPR↓} & \textbf{mIoU↓} & \textbf{F-score↓} & \textbf{FPR↓} & \textbf{G-mIoU↑} & \textbf{G-F↑} & \textbf{G-FPR↓} \\
\hline
\multirow{3}{*}{S4} & \xmark & \xmark & 78.7 & 87.9 & 76.6 & 87.1 & 0.19 & 77.6 & 88.0 & 0.18 & 78.2 & 88.2 & 0.19 & 35.032 & 21.479 & 0.186 \\
& \cmark & \xmark & 79.0 & 88.4 & 76.6 & 86.6 & 0.19 & 77.9 & 87.7 & 0.20 & 78.5 & 88.0 & 0.19 & 34.847 & 21.993 & 0.189 \\
& \cmark & \cmark & 78.1 & 88.2 & 0.2 & 22.6 & 0.00 & 0.2 & 22.6 & 0.00 & 0.2 & 22.6 & 0.00 & 87.672 & 82.461 & 0.000 \\
\hline
\multirow{3}{*}{MS3} & \xmark & \xmark & 54.0 & 64.5 & 27.6 & 53.5 & 0.05 & 31.7 & 57.4 & 0.05 & 42.2 & 62.4 & 0.09 & 59.468 & 51.036 & 0.072 \\
& \cmark & \xmark & 51.6 & 23.3 & 34.4 & 52.1 & 0.10 & 40.2 & 58.6 & 0.10 & 45.2 & 62.3 & 0.10 & 55.489 & 30.057 & 0.095 \\
& \cmark & \cmark & 51.3 & 64.5 & 9.8 & 17.7 & 0.00 & 9.9 & 25.8 & 0.00 & 9.1 & 20.3 & 0.00 & 66.605 & 70.590 & 0.004 \\
\hline
\end{tabular}
}
\caption{Ablation study of negative samples and classifier guidance. Results show performance on positive and negative audio inputs, as well as global metrics. The checkmarks (\cmark) and crosses (\xmark) indicate whether negative samples and classifier guidance ($\mathcal{L}_{\text{BCE}}$) are used in each configuration}
\label{table:full_comparison}
\end{table*}



\subsection*{B Detailed Analysis of Audio Conditions}
% Statistical analysis

\noindent\textbf{S4 Dataset Composition:} The Robust S4 dataset spans four major categories with diverse audio-visual characteristics, as shown in Figure~\ref{fig:category_dist}. The distribution is relatively balanced across device (32.2\%), music (32.1\%), and animal (23.2\%) categories, with human sounds comprising 12.5\% of the dataset. This balanced distribution helps ensure robust evaluation across different types of audio-visual scenarios.

\begin{figure}[t]
    \centering
    \includegraphics[width=\linewidth]{benchmark_s4.pdf}
    \caption{Distribution of cross-category combinations in multi-source scenarios. This analysis reveals common co-occurrence patterns, with music-human and animal-human being the most frequent combinations.}
    \label{fig:category_dist}
\end{figure}


% \noindent\textbf{Category Characteristics:}
% \begin{itemize}
%     \item \textbf{Music (32.1\%)}: Includes various instruments (guitar, piano, violin, tabla) with distinct timbre and visual appearances, providing diverse audio-visual correlation patterns.
    
%     \item \textbf{Device (32.2\%)}: Encompasses mechanical sounds (helicopter, keyboard, car) with characteristic audio signatures and varying visual scales.
    
%     \item \textbf{Animal (23.2\%)}: Features natural sounds (dog, lion, bird) with unique audio-visual correspondence patterns and challenging temporal dynamics.
    
%     \item \textbf{Human (12.5\%)}: Contains speech and other human-generated sounds, presenting complex scenarios with both audio and visual variations.
% \end{itemize}

\begin{figure}[t]
    \centering
    \includegraphics[width=\linewidth]{benchmark_ms3.pdf}
    \caption{Distribution of multi-source audio category combinations in our dataset. The horizontal bars show the frequency of different category pairs including individual categories (music, human, animal, device) and their combinations (e.g., human\_music, animal\_human).}
    \label{fig:cross_category}
\end{figure}

\noindent\textbf{MS3 Cross-Category Analysis:} For multi-source scenarios (Figure~\ref{fig:cross_category}), we analyze the distribution of audio category combinations in our dataset. Music and human categories have the highest individual counts, followed by animal and device categories. Among cross-category combinations, human\_music shows the highest frequency, followed by animal\_human. Other combinations such as device\_human and animal\_device occur less frequently in our dataset.





\subsection*{C. Failure Case Analysis}

Our method inherits certain limitations from the base segmentation model. Specifically, in cases where the underlying model fails to correctly segment the target object, our approach will also produce incorrect results. This cascading failure occurs because our method builds upon and depends on the initial segmentation output.

For example, when the base model misidentifies object boundaries or fails to detect the target object entirely, our method cannot compensate for these fundamental segmentation errors.

Future work could explore ways to make our approach more robust to initial segmentation errors, possibly through the incorporation of additional verification mechanisms or multi-model ensemble approaches.

% \subsection*{E. Computational Analysis}
% \subsubsection*{E.1 Runtime Performance}
% \subsubsection*{E.2 Model Complexity}











\subsection*{D. AVSegFormer Implementation Details}
\subsubsection*{D.1 Preliminary: AVSegFormer Architecture}
\label{subsec:avSegformer}

AVSegFormer~\cite{gao2024avsegformer} introduces several key architectural innovations while maintaining some fundamental components from AVSBench~\cite{zhou2022audio}. As illustrated in Figure~\ref{fig:avsegformer_arch}, the framework consists of five main components: audio-visual backbone encoders, a query generator that produces audio-conditioned queries, a transformer encoder for multi-scale feature processing, an audio-visual mixer for cross-modal feature fusion, and a transformer decoder for final mask generation. The query-based design enables the model to adaptively focus on audio-relevant regions in the visual frame. Here we detail its implementation.


\noindent\textbf{Encoder:} The encoding pathway remains consistent with AVSBench, employing VGGish~\cite{hershey2017vggish} to generate audio features $\mathcal{F}_A \in \mathbb{R}^d$ ($d = 128$). For visual processing, we utilizes Pyramid Vision Transformer~\cite{wang2022pvt} to extract hierarchical features $\mathcal{F}_{V_i}$, where $i \in [1,4]$ denotes the multi-resolution stage.

\noindent\textbf{Query Generator:} A key innovation in AVSegFormer~\cite{gao2024avsegformer} is its query-based architecture. The generator processes:
\begin{itemize}
    \item Initial query: $\mathcal{Q}_{\text{init}} \in \mathbb{R}^{T\times N_{\text{query}}\times D}$
    \item Audio feature: $\mathcal{F}_{\text{audio}} \in \mathbb{R}^{T\times D}$
    \item Learnable query: $\mathcal{Q}_{\text{learn}}$
\end{itemize}
Through cross-attention mechanisms, these components are transformed into mixed queries $\mathcal{Q}_{\text{mixed}}$, which help the model adaptively focus on audio-relevant regions in the visual frame. The inclusion of learnable queries enhances the model's capability to handle various audio-visual scenarios and dataset-specific characteristics.


\noindent\textbf{Transformer Encoder:} The transformer encoder processes visual features at three different resolutions (1/8, 1/16, and 1/32 of the original size). These multi-scale features are first flattened and concatenated to form a unified representation. The concatenated features are then processed through transformer layers, after which they are reshaped back to their original spatial dimensions. The 1/8-scale features are specifically upsampled by a factor of 2 and combined with the 1/4-scale features from the visual backbone through addition, producing the final mask features $\mathcal{F}_{\text{mask}}$ at 1/4 resolution. This multi-scale processing ensures the model captures both fine details and broader contextual information.

\begin{figure}[t]
    \centering
    \includegraphics[width=0.5\textwidth]{avsegformer_arch.pdf}
    \caption{Overview of AVSegformer architecture. The framework processes audio and visual inputs through parallel backbone networks, generates audio-conditioned queries, and employs transformer-based encoder-decoder architecture with a specialized mixer for audio-visual feature fusion.}
    \label{fig:avsegformer_arch}
\end{figure}

\noindent\textbf{Audio-Visual Mixer:} The mixer implements channel attention mechanism through:
\begin{equation}
    \omega = \text{softmax}(\frac{\mathcal{F}_{\text{audio}}\mathcal{F}_{\text{mask}}^T}{\sqrt{D/n_{\text{head}}}})
\end{equation}
\begin{equation}
    \hat{\mathcal{F}}_{\text{mask}} = \mathcal{F}_{\text{mask}} + \mathcal{F}_{\text{mask}} \odot \omega
\end{equation}
where $n_{\text{head}}=8$ and $\odot$ represents element-wise multiplication.

\noindent\textbf{Transformer Decoder:} The decoder utilizes the mixed query $\mathcal{Q}_{\text{mixed}}$ as input and processes multi-scale visual features as key/value pairs. Through the decoding process, output queries $\mathcal{Q}_{\text{output}}$ continuously aggregate visual features and combine with audio information. The final mask is generated through:
\begin{equation}
    \mathcal{M} = \text{FC}(\hat{\mathcal{F}}_{\text{mask}} + \text{MLP}(\hat{\mathcal{F}}_{\text{mask}} \cdot \mathcal{Q}_{\text{output}}))
\end{equation}
where MLP and FC layers integrate different channels to produce the final segmentation prediction.

\noindent\textbf{Loss Functions:} The original AVSegFormer~\cite{gao2024avsegformer} employs two complementary loss terms for training:
\begin{itemize}
\item $\mathcal{L}_{\text{IoU}}$ is a Dice loss comparing predicted segmentation masks with ground truth masks. This loss is particularly effective for handling the class imbalance inherent in AVS tasks where sounding objects often occupy small portions of the frame.
\item $\mathcal{L}_{\text{mix}}$ supervises the audio-visual mixer by computing Dice loss between a predicted binary mask (aggregated from mixed features) and combined foreground labels. This loss enhances the model's ability to handle complex scenes with multiple sound sources.
\end{itemize}


\subsubsection*{D.2 Learning with Balanced Audio-Visual Pairs, Classifier-Guided Feature Alignment}

As illustrated in Figure~\ref{fig:framework}, our approach enhances existing AVS architectures with three key components to address the visual bias problem. Given video frames and audio inputs, we first construct balanced positive and negative audio-visual pairs, where positive pairs represent aligned sound sources while negative pairs correspond to scenarios like silence or off-screen sounds. The model processes these pairs through separate visual and audio encoders to extract modality-specific features. These features undergo similarity-based alignment, optimized through classifier guidance in a contrastive manner. Finally, the aligned features are through the Query Generator module and the Transformer-based Head part before generating the final segmentation masks.

The framework is designed to maintain high segmentation accuracy for positive pairs while effectively suppressing predictions for negative pairs. Positive pairs are trained to maximize feature similarity, while negative pairs minimize it. By processing both types of pairs through this pipeline, the model learns to distinguish between valid and invalid audio-visual correspondences, enabling more robust segmentation in complex real-world scenarios.

\begin{figure}[t]
    \centering
    \includegraphics[width=\linewidth]{avsegformer_arch_overview.pdf}
    \caption{Overview of our robust AVS framework based on AVSegformer~\cite{gao2024avsegformer}. The model processes both positive and negative audio-visual pairs through separate encoders, employs classifier-guided similarity learning for feature alignment, and integrates the features for mask prediction. Positive pairs maximize similarity while negative pairs minimize it, enabling effective discrimination between valid and invalid audio-visual correspondences.}
    \label{fig:framework}
\end{figure}


Following Section~\ref{sec:method} of the main paper, we implement balanced training by maintaining a 10\% ratio of negative pairs during training. For classifier-guided similarity learning, we compute cosine similarity between aligned audio and visual features:
\begin{equation}
    s(\mathcal{F}_A, \mathcal{F}_V) = \text{cos}(\hat{\mathcal{F}}_A, \hat{\mathcal{F}}_V)
\end{equation}

The binary cross-entropy loss guides similarity learning:
\begin{equation}
\begin{split}
   \mathcal{L}_{\text{BCE}} = -\frac{1}{|\mathcal{P}| + |\mathcal{N}|} & \sum_{j=1}^{|\mathcal{P}| + |\mathcal{N}|}  \left( y_j \log \sigma(s_j) \right. \\
   & \left. + (1 - y_j) \log (1 - \sigma(s_j)) \right),
\end{split}
\end{equation}

The final training objective combines three complementary components:
\begin{equation}
\mathcal{L}_{\text{total}} = \mathcal{L}_{\text{IoU}} + \lambda_1\mathcal{L}_{\text{mix}} + \lambda_{\text{BCE}}\mathcal{L}_{\text{BCE}}
\end{equation}
where $\mathcal{L}_{\text{BCE}}$ guides the similarity learning between audio and visual features, helping the model distinguish between valid and invalid audio-visual correspondences through contrastive learning. The weighting factors $\lambda_1=0.1$ and $\lambda_{\text{BCE}}=1.0$ balance the contributions of each loss component.

\subsubsection*{D.3 Training Details}
Following AVSegFormer~\cite{gao2024avsegformer}'s setting, we adjust the input resolution to $512 \times 512$ to better capture detailed visual information. This model's training employs the AdamW optimizer~\cite{loshchilov2017adamw}, with a batch size of 1 and an initial learning rate of $2 \times 10^{-5}$. The encoder and decoder consist of 6 layers each, with an embedding size of 256. The training protocol extends to 60 epochs for the MS3 dataset and 30 epochs for the S4 dataset, conducted on an NVIDIA RTX A6000 GPU.



\end{document}
