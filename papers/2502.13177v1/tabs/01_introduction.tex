\section{Introduction}

\begin{figure}[!t]
  \includegraphics[width=\columnwidth]{figures/monotonic_logit_image.pdf}
  \caption{\method{} adaptively controls $\beta$ corresponding to the KL penalty strength for each preference pair by checking whether the log-likelihood ratio of the chosen response and the rejected response changes monotonically with the perturbation of $\beta$ used during training. It is equivalent to checking the monotonicity of the logits as a preference model induced by the DPO and estimating the advantage of the change in $\beta$ by the change of train-time inverse temperature to the preference confidence under the same test-time temperature.}
  \label{fig:monotonic_logit}
\end{figure}
\section{Overview}

\revision{In this section, we first explain the foundational concept of Hausdorff distance-based penetration depth algorithms, which are essential for understanding our method (Sec.~\ref{sec:preliminary}).
We then provide a brief overview of our proposed RT-based penetration depth algorithm (Sec.~\ref{subsec:algo_overview}).}



\section{Preliminaries }
\label{sec:Preliminaries}

% Before we introduce our method, we first overview the important basics of 3D dynamic human modeling with Gaussian splatting. Then, we discuss the diffusion-based 3d generation techniques, and how they can be applied to human modeling.
% \ZY{I stopp here. TBC.}
% \subsection{Dynamic human modeling with Gaussian splatting}
\subsection{3D Gaussian Splatting}
3D Gaussian splatting~\cite{kerbl3Dgaussians} is an explicit scene representation that allows high-quality real-time rendering. The given scene is represented by a set of static 3D Gaussians, which are parameterized as follows: Gaussian center $x\in {\mathbb{R}^3}$, color $c\in {\mathbb{R}^3}$, opacity $\alpha\in {\mathbb{R}}$, spatial rotation in the form of quaternion $q\in {\mathbb{R}^4}$, and scaling factor $s\in {\mathbb{R}^3}$. Given these properties, the rendering process is represented as:
\begin{equation}
  I = Splatting(x, c, s, \alpha, q, r),
  \label{eq:splattingGA}
\end{equation}
where $I$ is the rendered image, $r$ is a set of query rays crossing the scene, and $Splatting(\cdot)$ is a differentiable rendering process. We refer readers to Kerbl et al.'s paper~\cite{kerbl3Dgaussians} for the details of Gaussian splatting. 



% \ZY{I would suggest move this part to the method part.}
% GaissianAvatar is a dynamic human generation model based on Gaussian splitting. Given a sequence of RGB images, this method utilizes fitted SMPLs and sampled points on its surface to obtain a pose-dependent feature map by a pose encoder. The pose-dependent features and a geometry feature are fed in a Gaussian decoder, which is employed to establish a functional mapping from the underlying geometry of the human form to diverse attributes of 3D Gaussians on the canonical surfaces. The parameter prediction process is articulated as follows:
% \begin{equation}
%   (\Delta x,c,s)=G_{\theta}(S+P),
%   \label{eq:gaussiandecoder}
% \end{equation}
%  where $G_{\theta}$ represents the Gaussian decoder, and $(S+P)$ is the multiplication of geometry feature S and pose feature P. Instead of optimizing all attributes of Gaussian, this decoder predicts 3D positional offset $\Delta{x} \in {\mathbb{R}^3}$, color $c\in\mathbb{R}^3$, and 3D scaling factor $ s\in\mathbb{R}^3$. To enhance geometry reconstruction accuracy, the opacity $\alpha$ and 3D rotation $q$ are set to fixed values of $1$ and $(1,0,0,0)$ respectively.
 
%  To render the canonical avatar in observation space, we seamlessly combine the Linear Blend Skinning function with the Gaussian Splatting~\cite{kerbl3Dgaussians} rendering process: 
% \begin{equation}
%   I_{\theta}=Splatting(x_o,Q,d),
%   \label{eq:splatting}
% \end{equation}
% \begin{equation}
%   x_o = T_{lbs}(x_c,p,w),
%   \label{eq:LBS}
% \end{equation}
% where $I_{\theta}$ represents the final rendered image, and the canonical Gaussian position $x_c$ is the sum of the initial position $x$ and the predicted offset $\Delta x$. The LBS function $T_{lbs}$ applies the SMPL skeleton pose $p$ and blending weights $w$ to deform $x_c$ into observation space as $x_o$. $Q$ denotes the remaining attributes of the Gaussians. With the rendering process, they can now reposition these canonical 3D Gaussians into the observation space.



\subsection{Score Distillation Sampling}
Score Distillation Sampling (SDS)~\cite{poole2022dreamfusion} builds a bridge between diffusion models and 3D representations. In SDS, the noised input is denoised in one time-step, and the difference between added noise and predicted noise is considered SDS loss, expressed as:

% \begin{equation}
%   \mathcal{L}_{SDS}(I_{\Phi}) \triangleq E_{t,\epsilon}[w(t)(\epsilon_{\phi}(z_t,y,t)-\epsilon)\frac{\partial I_{\Phi}}{\partial\Phi}],
%   \label{eq:SDSObserv}
% \end{equation}
\begin{equation}
    \mathcal{L}_{\text{SDS}}(I_{\Phi}) \triangleq \mathbb{E}_{t,\epsilon} \left[ w(t) \left( \epsilon_{\phi}(z_t, y, t) - \epsilon \right) \frac{\partial I_{\Phi}}{\partial \Phi} \right],
  \label{eq:SDSObservGA}
\end{equation}
where the input $I_{\Phi}$ represents a rendered image from a 3D representation, such as 3D Gaussians, with optimizable parameters $\Phi$. $\epsilon_{\phi}$ corresponds to the predicted noise of diffusion networks, which is produced by incorporating the noise image $z_t$ as input and conditioning it with a text or image $y$ at timestep $t$. The noise image $z_t$ is derived by introducing noise $\epsilon$ into $I_{\Phi}$ at timestep $t$. The loss is weighted by the diffusion scheduler $w(t)$. 
% \vspace{-3mm}

\subsection{Overview of the RTPD Algorithm}\label{subsec:algo_overview}
Fig.~\ref{fig:Overview} presents an overview of our RTPD algorithm.
It is grounded in the Hausdorff distance-based penetration depth calculation method (Sec.~\ref{sec:preliminary}).
%, similar to that of Tang et al.~\shortcite{SIG09HIST}.
The process consists of two primary phases: penetration surface extraction and Hausdorff distance calculation.
We leverage the RTX platform's capabilities to accelerate both of these steps.

\begin{figure*}[t]
    \centering
    \includegraphics[width=0.8\textwidth]{Image/overview.pdf}
    \caption{The overview of RT-based penetration depth calculation algorithm overview}
    \label{fig:Overview}
\end{figure*}

The penetration surface extraction phase focuses on identifying the overlapped region between two objects.
\revision{The penetration surface is defined as a set of polygons from one object, where at least one of its vertices lies within the other object. 
Note that in our work, we focus on triangles rather than general polygons, as they are processed most efficiently on the RTX platform.}
To facilitate this extraction, we introduce a ray-tracing-based \revision{Point-in-Polyhedron} test (RT-PIP), significantly accelerated through the use of RT cores (Sec.~\ref{sec:RT-PIP}).
This test capitalizes on the ray-surface intersection capabilities of the RTX platform.
%
Initially, a Geometry Acceleration Structure (GAS) is generated for each object, as required by the RTX platform.
The RT-PIP module takes the GAS of one object (e.g., $GAS_{A}$) and the point set of the other object (e.g., $P_{B}$).
It outputs a set of points (e.g., $P_{\partial B}$) representing the penetration region, indicating their location inside the opposing object.
Subsequently, a penetration surface (e.g., $\partial B$) is constructed using this point set (e.g., $P_{\partial B}$) (Sec.~\ref{subsec:surfaceGen}).
%
The generated penetration surfaces (e.g., $\partial A$ and $\partial B$) are then forwarded to the next step. 

The Hausdorff distance calculation phase utilizes the ray-surface intersection test of the RTX platform (Sec.~\ref{sec:RT-Hausdorff}) to compute the Hausdorff distance between two objects.
We introduce a novel Ray-Tracing-based Hausdorff DISTance algorithm, RT-HDIST.
It begins by generating GAS for the two penetration surfaces, $P_{\partial A}$ and $P_{\partial B}$, derived from the preceding step.
RT-HDIST processes the GAS of a penetration surface (e.g., $GAS_{\partial A}$) alongside the point set of the other penetration surface (e.g., $P_{\partial B}$) to compute the penetration depth between them.
The algorithm operates bidirectionally, considering both directions ($\partial A \to \partial B$ and $\partial B \to \partial A$).
The final penetration depth between the two objects, A and B, is determined by selecting the larger value from these two directional computations.

%In the Hausdorff distance calculation step, we compute the Hausdorff distance between given two objects using a ray-surface-intersection test. (Sec.~\ref{sec:RT-Hausdorff}) Initially, we construct the GAS for both $\partial A$ and $\partial B$ to utilize the RT-core effectively. The RT-based Hausdorff distance algorithms then determine the Hausdorff distance by processing the GAS of one object (e.g. $GAS_{\partial A}$) and set of the vertices of the other (e.g. $P_{\partial B}$). Following the Hausdorff distance definition (Eq.~\ref{equation:hausdorff_definition}), we compute the Hausdorff distance to both directions ($\partial A \to \partial B$) and ($\partial B \to \partial A$). As a result, the bigger one is the final Hausdorff distance, and also it is the penetration depth between input object $A$ and $B$.


%the proposed RT-based penetration depth calculation pipeline.
%Our proposed methods adopt Tang's Hausdorff-based penetration depth methods~\cite{SIG09HIST}. The pipeline is divided into the penetration surface extraction step and the Hausdorff distance calculation between the penetration surface steps. However, since Tang's approach is not suitable for the RT platform in detail, we modified and applied it with appropriate methods.

%The penetration surface extraction step is extracting overlapped surfaces on other objects. To utilize the RT core, we use the ray-intersection-based PIP(Point-In-Polygon) algorithms instead of collision detection between two objects which Tang et al.~\cite{SIG09HIST} used. (Sec.~\ref{sec:RT-PIP})
%RT core-based PIP test uses a ray-surface intersection test. For purpose this, we generate the GAS(Geometry Acceleration Structure) for each object. RT core-based PIP test takes the GAS of one object (e.g. $GAS_{A}$) and a set of vertex of another one (e.g. $P_{B}$). Then this computes the penetrated vertex set of another one (e.g. $P_{\partial B}$). To calculate the Hausdorff distance, these vertex sets change to objects constructed by penetrated surface (e.g. $\partial B$). Finally, the two generated overlapped surface objects $\partial A$ and $\partial B$ are used in the Hausdorff distance calculation step.

Aligning large language models with human preferences for helpfulness and harmless principles~\cite{askell2021general, bai2022training, cui2023ultrafeedback} is a crucial requirement for general chatbot agents. Reinforcement Learning from Human Feedback (RLHF)~\cite{ziegler2020finetuning} is the pioneering approach that regards the alignment of large language models as a reward maximization problem and solves it by reinforcement learning~\cite{schulman2017proximal}. However, the complicated training pipeline of RLHF increases the training complexity and computation cost of the rollout for online reinforcement learning, in addition to the difficulty of collecting human preference datasets. Moreover, introducing a trained reward model as a proxy reward function to replace the intractable ground-truth human preference reward function makes large language models suffer from the side effect of reward over-optimization~\cite{gao2023scaling} inherited from the reward models.

Direct Preference Optimization (DPO)~\cite{rafailov2023direct} proposes an approach to reform the limitation of RLHF by converting the policy optimization problem into a preference modeling problem and performing alignment using only offline learning. It shows comparable performance while skipping the reward modeling process required by RLHF and has become an effective alternative approach for alignment. In particular, subsequent studies with various modifications to the DPO loss objective open a new research domain called direct alignment algorithms~\cite{rafailov2024scaling}, which perform alignment directly from offline preference datasets without training separated reward models.

However, DPO assumes that $\beta$ and the reference policy, which define a KL penalty that prevents excessive deviations from the reference model in RLHF, are fixed for exploiting the existence of a closed-form solution derived from the objective function of the RLHF. However, this assumption can lead to suboptimal results, since the KL penalty can be regarded as a Lagrangian relaxation of the constraint optimization defined by the trust region~\cite{schulman2017proximal}. In this regard, $\beta$-DPO~\cite{wu2024beta} argues that $\beta$ should be adaptively chosen according to the quality of the preference pair but fails to control $\beta$ at the instance-level and proposes a batch-level control method. On the other hand, TR-DPO~\cite{gorbatovski2024learn} claims to periodically update the reference policy to reduce over-optimization~\cite{rafailov2024scaling}, but it may induce unnecessary KL divergence for improvement since the update is not adaptive.

In this paper, we present \textbf{\methodbfull{} (\methodb{})}, an instance-level adaptive KL penalty control for DPO that neither TR-DPO nor $\beta$-DPO achieves. Specifically, we check the advantage of adjusting $\beta$ for each preference pair by observing the monotonicity of the log-likelihood ratio between the chosen response and the rejected response when the $\beta$ used during training is perturbed, as described in \Cref{fig:monotonic_logit}. Here, the criterion for controlling $\beta$ does not require batch-level statistics, and the policy under the perturbed $\beta$ can be estimated by reusing the current policy and reference policy logits. This criterion results in independence from the choice of micro-batch size and no additional computation requirements for model updates, unlike $\beta$-DPO and TR-DPO.

Experimental results demonstrate that \method{} outperforms $\beta$-DPO, TR-DPO, and most direct alignment algorithms that modify DPO loss objective~\cite{yuan2023rrhf, zhao2023slic, azar2024general, xu2024contrastive, ethayarajh2024kto, hong2024orpo, park2024disentangling, meng2024simpo}, highlighting the importance of adequate KL penalty relaxation for DPO. Furthermore, we confirm that the variation of $\beta$ determined by the adaptive criterion in \method{} reflects the confusion as a preference model, which is not addressed by the adaptive criterion of $\beta$-DPO. We also find that the adaptive KL penalty control of \method{} is crucial for an efficient KL trade-off compared to TR-DPO, which is not an adaptive KL penalty control.

In summary, our work shows the following:
\begin{itemize}
    \setlength\itemsep{0em}
  \item \method{} provides a simple criterion to improve DPO through KL penalty relaxation.
  \item \method{} adaptively adjusts $\beta$ in instance-level reflecting confusion as a preference model.
  \item \method{} efficiently controls $\beta$ in trade-off between KL divergence and performance.
\end{itemize}