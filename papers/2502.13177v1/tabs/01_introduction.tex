\section{Introduction}

\begin{figure}[!t]
  \includegraphics[width=\columnwidth]{figures/monotonic_logit_image.pdf}
  \caption{\method{} adaptively controls $\beta$ corresponding to the KL penalty strength for each preference pair by checking whether the log-likelihood ratio of the chosen response and the rejected response changes monotonically with the perturbation of $\beta$ used during training. It is equivalent to checking the monotonicity of the logits as a preference model induced by the DPO and estimating the advantage of the change in $\beta$ by the change of train-time inverse temperature to the preference confidence under the same test-time temperature.}
  \label{fig:monotonic_logit}
\end{figure}
\begin{figure*}[t]
\begin{center}
\includegraphics[width=.85\linewidth]{fig_overview_v3.pdf}
\end{center}
\caption{
FastAtlas Overview: In each frame, we compute charts spanning fully or partially visible triangles (a), determine texture space bounding boxes for the visible portions of the view-space projections of each chart, and tightly pack these boxes into atlases (b, here $2K \times 2K$). We simultaneously bijectively parameterize and shade the charts into their atlas boxes, obtaining high quality texture space shading (c), and use this shading to render the shaded frames (d).}
\label{fig:overview}
\label{fig:alg_overview}
\end{figure*}

\section{Overview}
\label{sec:overview}
Our work has two core contributions: a real-time, GPU-based algorithm for tight packing of general parameterized charts into compact atlases; and a real-time TSS method that
utilizes this packing.  

\paragraph*{FastAtlas Packing.}
FastAtlas runs entirely on the GPU as a series of compute shaders. It takes the bounding boxes of parameterized charts as input, and packs them into an atlas (Fig~\ref{fig:overview}b, Sec.~\ref{sec:pack}). As such, the only input it requires are the dimensions of the bounding boxes.
Its outputs are deterministic; identical input charts are packed into identical atlases. This is critical for TSS and similar applications, as it ensures that consecutive frames taken from the same camera view have the same shading. Even minute shading differences across such frames can cause sampling jitter, leading to undesirable flicker \cite{baker2012rock}. 
While prior methods such as \cite{mueller2018shading,hladky2019tessellated,hladky2021snakebinning,Neff2022MSA} cap the dimensions of the charts that can be packed as-is for a given atlas size, and scale down all charts that exceed these dimensions, we scale all charts by the same factor, and do so only when strictly necessary to achieve packing success (Figs~\ref{fig:atlas},~\ref{fig:sas_issues}). 

\paragraph*{TSS using FastAtlas.}
Our end-to-end TSS atlas generation method combines the packing method above with a novel approach for computing seamless per-frame charts. 
We define our charts as the connected components of the visible surfaces in each frame (Fig.~\ref{fig:overview}a), and efficiently compute them using a parallel union-find algorithm (Sec.~\ref{sec:visible}). Since the boundaries of these charts coincide with the contours of the rendered surface, they are {\em invisible} to the viewer. This approach 
eliminates the artifacts caused by shading discontinuities along visible seams (Fig.~\ref{fig:seams}). 

\begin{parWithWrapFigure}
\begin{wrapfigure}{l}{.27\columnwidth}%
\includegraphics[width=\linewidth]{fig_inset_view_plane.pdf}%
\end{wrapfigure}
We bijectively parametrize the {\em visible portions} of our charts by projecting them to view space (inset). This maps a constant number of texels to each pixel in the final rendered output, evenly distributing residual undersampling error across all image pixels. While conceptually straightforward, efficiently parameterizing charts containing partially visible triangles using viewspace projection is non-trivial, as the visible portions may no longer be triangular (e.g. green triangle in the inset); applying naive projection to triangles with vertices behind the camera may produce ill-posed results. Clipping triangles before projection is both computationally expensive and significantly complicates downstream operations. We avoid explicit clipping by observing that all that is required for atlas packing is the dimensions of, potentially conservative, bounding boxes of these projected visible portions. We compute such bounding boxes without explicit chart clipping by adapting a conservative screen coverage estimator \shortcite{Blinn:CalculatingScreenCoverage} (Sec.~\ref{sec:box}). We then pack the computed boxes using FastAtlas. 
\end{parWithWrapFigure}

Finally, we shade the visible portion of each chart into its corresponding atlas bounding box (Fig~\ref{fig:overview}c). 
The resulting texture is then used during rasterization as a standard texture map (Fig. ~\ref{fig:overview}d). 
Our framework is compatible with all existing approaches for texture space shading, including forward shading, raytraced illumination, or deferred shading in texture space \cite{baker:2016}. In the examples shown, we use the standard forward shading based rendering pipeline included in the G3D Innovation Engine \cite{G3D17}, a commercial grade renderer.


Aligning large language models with human preferences for helpfulness and harmless principles~\cite{askell2021general, bai2022training, cui2023ultrafeedback} is a crucial requirement for general chatbot agents. Reinforcement Learning from Human Feedback (RLHF)~\cite{ziegler2020finetuning} is the pioneering approach that regards the alignment of large language models as a reward maximization problem and solves it by reinforcement learning~\cite{schulman2017proximal}. However, the complicated training pipeline of RLHF increases the training complexity and computation cost of the rollout for online reinforcement learning, in addition to the difficulty of collecting human preference datasets. Moreover, introducing a trained reward model as a proxy reward function to replace the intractable ground-truth human preference reward function makes large language models suffer from the side effect of reward over-optimization~\cite{gao2023scaling} inherited from the reward models.

Direct Preference Optimization (DPO)~\cite{rafailov2023direct} proposes an approach to reform the limitation of RLHF by converting the policy optimization problem into a preference modeling problem and performing alignment using only offline learning. It shows comparable performance while skipping the reward modeling process required by RLHF and has become an effective alternative approach for alignment. In particular, subsequent studies with various modifications to the DPO loss objective open a new research domain called direct alignment algorithms~\cite{rafailov2024scaling}, which perform alignment directly from offline preference datasets without training separated reward models.

However, DPO assumes that $\beta$ and the reference policy, which define a KL penalty that prevents excessive deviations from the reference model in RLHF, are fixed for exploiting the existence of a closed-form solution derived from the objective function of the RLHF. However, this assumption can lead to suboptimal results, since the KL penalty can be regarded as a Lagrangian relaxation of the constraint optimization defined by the trust region~\cite{schulman2017proximal}. In this regard, $\beta$-DPO~\cite{wu2024beta} argues that $\beta$ should be adaptively chosen according to the quality of the preference pair but fails to control $\beta$ at the instance-level and proposes a batch-level control method. On the other hand, TR-DPO~\cite{gorbatovski2024learn} claims to periodically update the reference policy to reduce over-optimization~\cite{rafailov2024scaling}, but it may induce unnecessary KL divergence for improvement since the update is not adaptive.

In this paper, we present \textbf{\methodbfull{} (\methodb{})}, an instance-level adaptive KL penalty control for DPO that neither TR-DPO nor $\beta$-DPO achieves. Specifically, we check the advantage of adjusting $\beta$ for each preference pair by observing the monotonicity of the log-likelihood ratio between the chosen response and the rejected response when the $\beta$ used during training is perturbed, as described in \Cref{fig:monotonic_logit}. Here, the criterion for controlling $\beta$ does not require batch-level statistics, and the policy under the perturbed $\beta$ can be estimated by reusing the current policy and reference policy logits. This criterion results in independence from the choice of micro-batch size and no additional computation requirements for model updates, unlike $\beta$-DPO and TR-DPO.

Experimental results demonstrate that \method{} outperforms $\beta$-DPO, TR-DPO, and most direct alignment algorithms that modify DPO loss objective~\cite{yuan2023rrhf, zhao2023slic, azar2024general, xu2024contrastive, ethayarajh2024kto, hong2024orpo, park2024disentangling, meng2024simpo}, highlighting the importance of adequate KL penalty relaxation for DPO. Furthermore, we confirm that the variation of $\beta$ determined by the adaptive criterion in \method{} reflects the confusion as a preference model, which is not addressed by the adaptive criterion of $\beta$-DPO. We also find that the adaptive KL penalty control of \method{} is crucial for an efficient KL trade-off compared to TR-DPO, which is not an adaptive KL penalty control.

In summary, our work shows the following:
\begin{itemize}
    \setlength\itemsep{0em}
  \item \method{} provides a simple criterion to improve DPO through KL penalty relaxation.
  \item \method{} adaptively adjusts $\beta$ in instance-level reflecting confusion as a preference model.
  \item \method{} efficiently controls $\beta$ in trade-off between KL divergence and performance.
\end{itemize}