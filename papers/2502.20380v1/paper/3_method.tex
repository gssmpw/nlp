\section{\method: Multi-turn Code Generation}

We compare the performance of agents trained on data from the InSTA pipeline to agents trained on human demonstrations from WebLINX \citep{WebLINX} and Mind2Web \citep{Mind2Web}, two recent and popular benchmarks for web navigation. Recent works that mix synthetic data with real data control the real data sampling probability in the batch $p_{\text{real}}$ independently from data size \citep{DAFusion}. We employ $p_{\text{real}} = 0.5$ in few-shot experiments and $p_{\text{real}} = 0.8$ otherwise. Shown in Figure~\ref{fig:data-statistics}, our data have a wide spread in performance, so we apply several filtering rules to select high-quality training data. First, we require the evaluator to return \texttt{conf} = 1 that the task was successfully completed, and that the agent was on the right track (this selects data where the actions are reliable, and directly caused the task to be solved). Second, we filter data where the trajectory contains at least three actions. Third, we remove data where the agent encountered any type of server error, was presented with a captcha, or was blocked at any point. These steps produce $7,463$ high-quality demonstrations in which agents successfully completed tasks on diverse websites. We sample 500 demonstrations uniformly at random from this pool to create a diverse test set, and employ the remaining $6,963$ demonstrations to train agents on a mix of real and synthetic data.

\subsection{Improving Data-Efficiency}
\label{sec:few-shot}

\begin{wrapfigure}{r}{0.48\textwidth}
    \centering
    \vspace{-0.8cm}
    \includegraphics[width=\linewidth]{assets/few_shot_results_weblinx_mind2web.pdf}
    \vspace{-0.3cm}
    \caption{\small \textbf{Data from InSTA improves efficiency.} Language model agents trained on mixtures of our data and human demonstrations scale faster than agents trained on human data. In a setting with 32 human actions, adding our data improves \textit{Step Accuracy} by +89.5\% relative to human data for Mind2Web, and +122.1\% relative to human data for WebLINX.}
    \vspace{-0.2cm}
    \label{fig:few-shot-results}
\end{wrapfigure}

In a data-limited setting derived from WebLINX \citep{WebLINX} and Mind2Web \citep{Mind2Web}, agents trained on our data \textit{scale faster with increasing data size} than human data alone. Without requiring laborious human annotations, the data produced by our pipeline leads to improvements on Mind2Web that range from +89.5\% in \textit{Step Accuracy} (the rate at which the correct element is selected and the correct action is performed on that element) with 32 human actions, to +77.5\% with 64 human actions, +13.8\% with 128 human actions, and +12.1\% with 256 human actions. For WebLINX, our data improves by +122.1\% with 32 human actions, and +24.6\% with 64 human actions, and +6.2\% for 128 human actions. Adding our data is comparable in performance gained to doubling the amount of human data from 32 to 64 actions. Performance on the original test sets for Mind2Web and WebLINX appears to saturate as the amount of human data increases, but these benchmark only test agent capabilities for a limited set of 150 popular sites.

\subsection{Improving Generalization} 
\label{sec:generalization}

\begin{wrapfigure}{r}{0.48\textwidth}
    \centering
    \vspace{-1.0cm}
    \includegraphics[width=\linewidth]{assets/diverse_results_weblinx_mind2web.pdf}
    \vspace{-0.3cm}
    \caption{\small \textbf{Our data improves generalization.} We train agents with all human data from the WebLINX and Mind2Web training sets, and resulting agents struggle to generalize to more diverse test data. Adding our data improves generalization by +149.0\% for WebLINX, and +156.3\% for Mind2Web.}
    \vspace{-0.3cm}
    \label{fig:generalization-results}
\end{wrapfigure}

To understand how agents trained on data from our pipeline generalize to diverse real-world sites, we construct a more diverse test set than Mind2Web and WebLINX using 500 held-out demonstrations produced by our pipeline. Shown in Figure~\ref{fig:generalization-results}, we train agents using all human data in the training sets for WebLINX and Mind2Web, and compare the performance with agents trained on 80\% human data, and 20\% data from our pipeline. Agents trained with our data achieve comparable performance to agents trained purely on human data on the official test sets for the WebLINX and Mind2Web benchmarks, suggesting that when enough human data are available, synthetic data may not be necessary. However, when evaluated in a more diverse test set that includes 500 sites not considered by existing benchmarks, agents trained purely on existing human data struggle to generalize. Training with our data improves generalization to these sites by +149.0\% for WebLINX agents, and +156.3\% for Mind2Web agents, with the largest gains in generalization \textit{Step Accuracy} appearing for harder tasks.

We propose \method, a simple and scalable algorithm for multi-turn code generation using execution feedback. \method follows an \emph{expert iteration}~\citep{anthony2017thinkingfastslowdeep} framework with a \emph{local search expert}. 
\method iteratively trains two components -- a \emph{learned verifier} $R_{\phi}$ to score code snippets (Section~\ref{sec:training_learned_verifiers}), and a \emph{generator} $\pi_\theta$ to imitate local search with the verifier (Section~\ref{sec:training_generator}). This iterative process allows the generator and expert to bootstrap off each other, leading to continuous improvement. At inference time, both the generator and verifier are used as BoN search to select and execute code (Section~\ref{sec:inference_time}). Finally, we analyze the performance of \method in Section~\ref{sec:analysis}.

\subsection{The \method Algorithm}
\label{sec:mucode}
Algorithm~\ref{algo:pseudo_training_mucode} presents the iterative training procedure. At an iteration $i$, the current generator $\pi_{\theta}$ is rolled out in the multi-turn code environment $\mathcal{E}$ to generate interaction data $\mathcal{D}_i\leftarrow\{(x,s_t,y_t, r_t)\}$. Every turn $t$ in $\mathcal{D}_i$ includes the prompt $x$, interaction history $s_t$, code generated $y_t$ and the correctness score from the oracle verifier $r_t=R(x, y_t)$. 
This data is then aggregated $\mathcal{D} \gets \mathcal{D} \cup \mathcal{D}_i$. 
The learned verifier $R_{\phi}^i$ is trained on the aggregated data $\mathcal{D}$.
An expert is created using $R_\phi^i$ to perform local search to find the optimal action $\pi_\star^i(x) = \arg\max_{y \in \mathcal{D}(x)} R_\phi^i(x,y)$, where $\mathcal{D}(x)$ are all the code completions for a given prompt $x$. The expert $\pi_\star^i(x)$ relabels the data $\mathcal{D}$ with the optimal action. The generator $\pi_\theta^i$ is then trained via fine-tuning (FT) on $\mathcal{D}$. This process iterates $M$ times, and the best generator and verifier pair on the validation dataset are returned.

\subsection{Training Verifier}
\label{sec:training_learned_verifiers}
The learned verifier provides dense scores to code solutions for a given problem. At train time, this is used by the expert to perform local search to obtain optimal code. At inference time, the verifier is used for multi-turn BoN~(\ref{sec:inference_time}) for efficient search. 
The learned verifier has two distinct advantages over process reward functions typically used in multi-turn RL:
(1) It is conditioned only on the initial prompt and the current solution, and is not dependent on previous states (2) It is trained via supervised learning on oracle reward labels. We explore two different losses: % for the learned verifier:

\textbf{Binary Cross-Entropy loss}~(BCE): 
The nominal way to train the verifier is to directly predict the oracle reward ~\citep{cobbe2021training}:
\begin{equation}
\begin{aligned}
    \mathcal{L}_{\rm BCE}(\phi) = -\mathbb{E}_{(x, y, r) \sim \mathcal{D}}[r \log R_{\phi}(x,y) \\ - (1 - r) \log R_{\phi}(x, y)]
\end{aligned}
\end{equation}

\textbf{Bradley Terry Model}~(BT):
Since the goal of the verifier is to relatively rank code solutions rather than predict absolute reward, we create a preference dataset and then train with a  Bradley Terry loss~\citep{ouyang2022traininglanguagemodelsfollow}.
For every prompt $x$, we create pairs of correct $y^{+}$ (where $r=1$) and incorrect $y^{-}$ (where $r=0$) code and define the following loss:
\begin{equation}
    \small
    \mathcal{L}_{BT}(\phi) = -\mathbb{E}_{(x, y^+, y^-) \sim \mathcal{D}} [\log \sigma(R_{\phi}(x,y^+) - R_{\phi}(x,y^-))].
\end{equation}
where $\sigma(.)$ is the sigmoid function. 
We hypothesize that BT is strictly easier to optimize as the verifier has to only focus on relative performance. This is also consistent with observations made for training process reward models, where the advantage function is easier to optimize than the absolute Q function~\citep{setlur2024rewarding}. 



\subsection{Training Generator}
\label{sec:training_generator}

\method comprises a generator~$\pi_\theta$ trained to produce code solutions conditioned on the initial problem and execution observations from previous turns.
Given a dataset $\mathcal{D}$, \method iteratively trains the generator to find the optimal code solution labeled using the local expert over the learned verifier.
% \method constructs an expert using local search and the learned verifier.
For this step, \method extracts all code solutions from $\mathcal{D}$ for every problem $x$.
An expert is then created by picking the best solution, $y^\star$, which achieves the highest score using with the learned verifier~$R_{\phi}(x, y)$ and is given by
\begin{equation}
    y^\star = \pi_\star(x) = \arg\max_{y \in \mathcal{D}(x)} R_\phi(x,y).
\end{equation}
Using this expert dataset, we relabel the dataset $\mathcal{D}$ with the optimal solutions for each prompt: 
\begin{equation}
    \mathcal{D}_\star = \{(x, s_t, y^\star)~|~(x, s_t) \sim \mathcal{D}\},
\end{equation}
where $\mathcal{D}_\star$ represents the expert dataset. 
%As a result of this relabeling, every trajectory for a problem \(x\) ends with the same optimal solution. 
The generator $\pi_{\theta}$ is then trained via fine-tuning~(FT) on this expert dataset $\mathcal{D}_\star$.

\section{Balancing Functional and Communicative Actions in Human-Robot Collaborative Transport}

We describe a control framework that leverages implicit communication to support efficient and fluent collaboration in human-robot collaborative transport. By reasoning about its partner's uncertainty over the \emph{way} the task is being executed, the robot balances between communicative, uncertainty-reducing actions, and functional, task-driven actions. This balance is not prescribed, but rather dynamically adaptive to the robot's belief about the uncertainty of its partner.


\subsection{Formalizing Joint Strategies of Workspace Traversal}

Collaborative tasks involving multiple agents working together require consensus on a \emph{joint strategy} $\psi$, i.e., a qualitatively distinct way of completing the task, out of the set of all possible joint strategies, $\Psi$. Often, this consensus is not established \textit{a priori}; rather, it is dynamically negotiated during execution. The abstraction of a joint strategy effectively captures critical domain knowledge at a representation level. While prior work on collaborative transport has emphasized \emph{role} assignment across the team (i.e., whether the robot or the human are leading or following each other)~\citep{mortl2012role, nikolaidis2013human, jarrasse2014slaves}, realistic, obstacle-cluttered environments present additional important challenges, such as the decision over \emph{how to pass through} an obstacle-cluttered workspace.

In this work, we formalize the space of workspace traversal strategies using tools from homotopy theory~\citep{knepper2012equivalence}. The human-robot team is tasked with transporting an object from its initial pose $p_0$ to a final pose $g$, resulting in an object trajectory $\boldsymbol{p}:[0,1]\to\mathcal{W}$, where $\boldsymbol{p}(0)=p_0$ and $\boldsymbol{p}(1)=g$, belonging to an appropriate space of trajectories $\mathcal{P}$.
Obstacles, defined as the connected components of $\mathcal{O}$, naturally partition $\mathcal{P}$ into equivalence classes $\Psi$, where each $\psi\in\Psi$ represents a distinct workspace traversal strategy under which the transported object can travel from $p_0$ to $g$, i.e.,
\begin{equation}
    \begin{split}
    & \mathcal{P} = \underset{\psi \in \Psi}{\bigcup} \psi \\
    & \forall \psi^i, \psi^j \in \Psi : (\psi^i \cap \psi^j \neq \emptyset) \implies (\psi^i = \psi^j) \\
    & \forall \boldsymbol{p}^i, \boldsymbol{p}^j \in \psi : \boldsymbol{p}^i \sim \boldsymbol{p}^j
    \end{split}
    \mbox{.}
\end{equation}

These classes can be identified using a notion of topological invariance. The works of~\citet{vernaza2013winding,kretzschmar2016irl,mavrogiannis2023winding} use winding numbers to describe topological relationships between the robot and obstacles or humans navigating around it. Here, we adapt this idea to collaborative transport by enumerating the set of homotopy classes between the object trajectory and obstacles in the workspace. Specifically, for any object trajectory $\boldsymbol{p}$ embedded in a space with $m$ obstacles $o_1,\dots, o_m$, we can define winding numbers
\begin{equation}
    w_i = \frac{1}{2\pi}\sum_{t} \Delta\theta_t^i, \quad i=1,\dots, m\mbox{,}
\end{equation}
where $\Delta\theta^i_t = \angle \left(p_t - o_i, p_{t-1}-o_i\right)$ denotes an angular displacement corresponding to the transfer of the object from $p_{t-1}$ to $p_t$ (see~\figref{fig:winding}). The sign of $w_i$ represents the passing side between the object and the $i$-th obstacle, and its absolute value represents the number of times the object encircled the $i$-th obstacle.
For a trajectory $\boldsymbol{p}$, the tuple of winding number signs
\begin{equation}
    W=(\sign{w_1},\dots, \sign{w_m})
\end{equation}
represents an equivalence class describing how the human-robot team transported the object past all obstacles in the environment. In this work, we model the space of joint strategies $\Psi$ as set of distinct $W$, i.e., $|\Psi|=2^m$.


\begin{figure}[t]
\begin{subfigure}{\linewidth}
    \centering
    \includegraphics[width=\linewidth]{figures/winding-intuition.png}
    \caption{
    Identification of workspace traversal strategies based on path homotopy~\citep{kretzschmar2016irl}. By integrating the angle of the vector between the obstacle and the object as it is being transported along a path $\boldsymbol{p}$, we extract a winding number $w(\boldsymbol{p})$ identifying the strategy of workspace traversal. Here, $w(\boldsymbol{p}^2)=w(\boldsymbol{p}^3)$ since both $\boldsymbol{p}^2$, $\boldsymbol{p}^3$ passed on the right of $o_1$.}
    \label{fig:winding}
\end{subfigure}
\begin{subfigure}{\linewidth}
    \includegraphics[width=\linewidth]{figures/windings.png}
    \caption{Representing workspace traversal strategies as tuples of winding number signs, $W$. In this scene with two obstacles, there are four possible strategies represented as continuous curves. The red curve highlights a strategy corresponding to passing on the right of $o_1$ ($w_1>0$), and the left of $o_2$ ($w_2<0$). This representation is applicable to any number of obstacles.}
    \label{fig:passing-strategy}
\end{subfigure}
\caption{Illustration of our topological abstraction for representing strategies of workspace traversal.\label{fig:workspace-traversal}}
\end{figure}


\subsection{Inferring Strategies of Workspace Traversal}

We describe an inference mechanism that maps observations of team actions to a belief over a workspace traversal strategy. This mechanism is agnostic to the specific definition of the strategy. At time $t$, we assume that the robot observes the joint action $\alpha = (a, u)$, the object state $p$, and the task context $c = (g, \mathcal{O})$. Given $\alpha$, $p$, and $c$, our goal is to infer the unfolding workspace traversal strategy, $\psi$, i.e.,
\begin{equation}
    \mathbb{P}(\psi \mid \alpha, p, c)\mbox{.}\label{eq:inference}
\end{equation}
Using Bayes' rule, we can expand~\eqref{eq:inference} as
\begin{equation}
    \mathbb{P}(\psi \mid \alpha, p, c) = \frac{1}{\eta} \mathbb{P}(\alpha \mid \psi, p, c)\,\mathbb{P}(\psi \mid p, c)\mbox{,}\label{eq:inference-bayes}
\end{equation}
where the left-hand side expression is the \emph{posterior distribution} of the joint strategy $\psi$, and on the right-hand side, $\eta$ is a normalizer across $\alpha$, $\mathbb{P}(\alpha \mid \psi, p, c)$ is the \emph{joint action likelihood distribution} and $\mathbb{P}(\psi \mid p, c)$ is a \emph{prior distribution} of the joint strategy before observing the joint action. We can rewrite the joint action likelihood distribution as
\begin{equation}
    \mathbb{P}(\alpha \mid \psi, p, c) = \mathbb{P}(a \mid \psi, p, c)\,\mathbb{P}(u \mid \psi, p, c)\mbox{,}
\end{equation}
since the two agents choose their actions independently.

The distribution of \eqref{eq:inference} allows the robot to represent the belief of its partner over the unfolding traversal strategy. A natural measure of uncertainty over the observer's belief regarding that strategy can be acquired by computing the information entropy of $\psi$, conditioned on known $\alpha, p, c$:
\begin{equation}
    H\left(\psi \mid \alpha,p,c\right) = - \sum_{\psi \in \Psi} \mathbb{P}(\psi\mid\alpha,p,c) \log \mathbb{P}(\psi\mid\alpha,p,c)\label{eq:entropy}    \mbox{.}
\end{equation}
Intuitively, the higher $H$ is, the higher the uncertainty of the user over the unfolding $\psi$ is assumed to be.


\subsection{Integrating Human Inferences into Robot Control}\label{sec:control}

We integrate the inference mechanism of \eqref{eq:inference} into a model predictive control (MPC) algorithm by using its entropy \eqref{eq:entropy} as a cost. Given the context $c = (g, \mathcal{O})$ and the object state $p$ at time $t$, the goal of the MPC is to find the sequence of future robot actions $\boldsymbol{u}^*$ that minimizes a cost function $J$ over a horizon $T$. At every control cycle, the MPC solves the following planning problem:

\begin{equation}
\begin{split}
\left(u_{t:t+T}\right)^{*} = \underset{u_{t:t+T}} {\arg\,\min}\; & J(p_{t:t+T}, u_{t:t+T})\\
    s.t.\: & p_{k+1} = f(p_k, a_k, u_k) \\
           & a_k \in \mathcal{A} \\
           & u_k\in\mathcal{U}
   \label{eq:mpc}
\end{split}\mbox{,}
\end{equation}
We split $J$ into a running cost $J_k$ and a terminal cost $J_T$
\begin{equation}
\begin{split}
     J(p_{t:t+T}, u_{t:t+T}) = & \sum_{k=0}^{T} \gamma^k J_k(p_{t+k}, u_{t+k})\\
        &+ J_T(p_{t+T}, u_{t+T})
\end{split}
        \mbox{,}
\end{equation}
where $\gamma$ is a discount factor, and the terminal cost penalizes distance from the object's goal pose $g$:
\begin{equation}
    J_T(p_{t+k}, u_{t+k}) = || p_{t+k} - g ||^2
    \mbox{.}
\end{equation}
The running cost $J_k$ is a weighted sum of two terms, i.e.,
\begin{equation}
    \begin{split}
        J_k(p_{t+k}, u_{t+k}) =\,& w_{obs} J_{obs}(p_{t+k}, u_{t+k})\\ &+ w_{ent} J_{ent}(p_{t+k}, u_{t+k})\mbox{,}
    \end{split}\label{eq:mpc-running-cost}
\end{equation}
where
\begin{equation}
    \begin{split}
    J_{obs}(p_{t+k}, u_{t+k}) &=\\ \max & \left(0, -\log\left(\underset{o \in \mathcal{O}}{\min} \frac{||p_{t+k} - o||}{\delta}\right)\right)
    \end{split}
    \mbox{,}
\end{equation}
is a collision avoidance cost penalizing proximity to obstacles, $\delta$ is a clearance threshold, $J_{ent}$ is a cost proportional to the entropy defined in \eqref{eq:entropy}, and $w_{obs}$, $w_{ent}$ are weights.

We refer to this control framework as \emph{Implicit Communication MPC}, or \textbf{IC-MPC}. At every control cycle, IC-MPC plans a future robot trajectory that balances between functional objectives (collision avoidance, progress to goal) and communicative objectives (minimizing the partner's uncertainty over the upcoming joint strategy). The robot executes the first action $u_t$ from the planned trajectory and then replans. This process is repeated in fixed control cycles until the task is completed.




\subsection{Inference: Multi-turn Best-of-N}
\label{sec:inference_time}
At inference time, the goal is to generate a code solution with a fixed inference budget -- denoting the number of times generators can provide one complete solution.
In this work, we propose to leverage the learned verifier to improve search and code generations over successive turns with \textit{multi-turn Best-of-N}~(BoN).
To achieve this, \method uses a natural extension of BoN to the multi-turn setting.
At each turn, the generator produces $N$ one-step rollouts $\{y^n_t\}_{n=1}^N \sim\pi_\theta(.|s_{t})$ and the learned verifier picks the most promising code solution among these candidates using
\begin{equation}
    y_t^*=\arg \max_{n} R_{\phi}(x,y^{n}_t). 
\end{equation}
The selected code~$y_t^*$ is executed in the environment over public tests to obtain the execution feedback $o_t$.
This solution and the feedback is provided as context to the generator at the next turn to repeat this procedure.
The search ends once~$y_t^*$ passes all public tests or when the turn limit is reached. Consequently, even if $R_\phi(\cdot)$ grants a high score to a code solution, inference continues until the solution has successfully cleared all public tests, thus mitigating potential errors by $R_\phi(\cdot)$.
The final response $y^*_t$ is then passed through the oracle verifier to check its correctness.
Algorithm~\ref{algo:pseudo_inference_mucode} describes a description of multi-turn BoN.
We found it beneficial to use the reward model trained with samples of the latest generator $\pi_\theta$ (see Table~\ref{tab:bon_results}).

\subsection{Analysis}
\label{sec:analysis}
\method effectively treats multi-turn code generation as an interactive imitation learning problem by collecting rollouts from a learned policy and re-labeling them with an expert. It circumvents the exploration burden of generic reinforcement learning which has exponentially higher sample complexity~\cite{sun2017deeply}. We briefly analyze why this problem is amenable to imitation learning and prove performance bounds for \method.

\begin{definition}[One-Step Recoverable MDP]
\label{def:one_step_recoverability}
A MDP $\mathcal{M} = (\mathcal{S}, \mathcal{A}, P, R, \gamma)$ with horizon $T$ is \emph{one-step recoverable} if the advantage function of the optimal policy $\pi^*$, defined as $A^*(s, a) = Q^*(s, a) - V^*(s)$, is uniformly bounded for all $(s, a)$, i.e. $A^*(s, a) \leq 1$.
\end{definition}

\paragraph{Code generation is one-step recoverable MDP.}
Multi-turn code generation satisfies one-step recoverability because the optimal policy $\pi^*(y_t | s_t)$ depends only on the problem prompt $x$ and not the interaction history $s_t = (x, y_1, o_1, \dots, y_{t-1}, o_{t-1})$. Since the correctness of a code snippet $y_t$ is fully determined by $x$, the optimal Q-function satisfies $Q^*(s_t, y_t) = R(x, y_t)$, where $R(x, y_t) \in \{0,1\}$. The optimal value function is $V^*(s_t) = \max_{y_t} R(x, y_t)$, so the advantage function simplifies to $A^*(s_t, y_t) = R(x, y_t) - \max_{y_t'} R(x, y_t') \leq 1$. 

\paragraph{Code generation enables efficient imitation learning.} There are two challenges to applying interactive imitation learning ~\cite{ross2011reduction, ross2014reinforcement} -- (1) Existence of expert policies or value functions, and (2) Recoverability of expert from arbitrary states. First, for code generation, the expert is simply the one-step reward maximizer $\arg\max_y R(x, y)$. We can efficiently estimate $R_\phi(x, y)$ to compute the expert, without needing to compute value function backups. Second, even if the learner fails to imitate the expert at any given state, the expert can perfectly recover from the next state. 
This results in the best possible performance bounds for imitation learning, which we formalize below. 
\begin{theorem}[Performance bound for \method]
\label{thm:one_step_bound} 
For a one-step recoverable MDP $\mathcal{M}$ with horizon $T$, running $N$ iterations of \method yields at least one policy $\pi$ such that
\begin{equation}
    J(\pi^*) - J(\pi) \leq O(T (\epsilon + \gamma(N))).
\end{equation}
where $\pi^*$ is the expert policy, $\epsilon$ is the realizability error, and $\gamma(N)$ is the average regret.
\end{theorem}
Proof is in Appendix~\ref{appendix:proof}. The bound $O(\epsilon T)$ is much better than the worst-case scenario of $O(\epsilon T^2)$ for unrecoverable MDPs~\cite{swamy2021moments}. Thus, \method exploits the structure of multi-turn code generation to enable imitation learning, bypassing the need for hierarchical credit assignment. More generally, this analysis suggests that for any task where the optimal action is history-independent and recoverable in one step, reinforcement learning can be reduced to efficient imitation learning without loss of performance.