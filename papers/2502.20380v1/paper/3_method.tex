\section{\method: Multi-turn Code Generation}


% \section{With what \emph{data}?}\label{sec:data}
\section{How do we \emph{train models}?}\label{sec:model_training}

% Which scales? How? 
%  - [DONE] Too small may not hold
%  - [DONE] number of models can affect confidence intervals
%  - [DONE] counting size also matters data (tokens vs bits) and flop counts kaplan vs hoffman CITE; non-embedding vs embedding 
%  - DITCH? how you scale architecture matters; for eg - clark shows scaling N and E is needed to separate; enc and decoder scaling can be different for architectures; in the 
%  - [DONE] 6ND is a common approximation - does not hold for very long context (these days 128k to 1M)


In order to fit a scaling law, one needs to train a range of models across multiple orders of magnitude in model size and/or dataset size. Researchers must first decide the range and distribution of $N$ and $D$ values for their training runs, in order to achieve stable convergence to a solution with high confidence, while limiting the total compute budget of all experiments. Many papers did not specify the number of data points used to fit each scaling law; those that did range from 4 to several hundred, but most used fewer than 50 data points. The specific $N$ and $D$ values also skew the optimization process towards a certain range of $N/D$ ratios, which may be too narrow to include the true optimum. Some approaches, such as using IsoFLOPs \citep{hoffmann2022training}, additionally dictate rules for choosing $N$ and $D$ values. Moreover, using a minimum $N$ or $D$ value may result in outlier values that may need to be dropped \citep{porian2024resolving,shin2023scaling,henighan2020scaling}. We investigate this choice in Section \S\ref{sec:repl-model_training}

The definition of $N$, $D$, or compute cost $C$ can affect the results of a scaling study. For example, if a study studies variation in tokenizers, a definition of training data size based on character count may be more appropriate than one based on token count \citep{tao2024scaling}. The inclusion or exclusion of embedding layer compute and parameters, may also skew the results of a study - a major factor in the different in optimal ratios determined by \cite{kaplan2020scaling} and \cite{hoffmann2022training} has been attributed to not factoring embedding FLOPs into the final compute cost \citep{pearce2024reconcilingkaplanchinchillascaling, porian2024resolving}. Given the increase in extremely long context models (128k-1M) \cite{reid2024gemini}, the commonly used training FLOPs approximation $C = 6 ND$ (see Appendix \ref{app:full-details}) may not hold for such models, given the additional cost proportional to the context length and model dimension - \citet{bi2024deepseek} introduce a new terms non-embedding FLOPs/token to account for this.

% \ml{we've gotta decide where this discussion should go }

% \luke{I agree we could probably cut some of the next few paragraphs for space if needed. The last paragraph ends well though I think.}
% \srk{ DITCH? how you scale architecture matters; for eg - clark shows scaling N and E is needed to separate; enc and decoder scaling can be different for architectures}

 % - counting size also matters data (tokens vs bits) and flop counts kaplan vs hoffman CITE; non-embedding vs embedding 

% The goal of scaling laws is generally to extrapolate findings to larger compute budgets. It is unclear which $N$ and $D$ values should be included in the data in order to predict loss at a larger scale. 


% \srk{discuss outliers being dropped and therefore, need to be sure about the scale of training}



% The scaling law form identifies the relevant variables (e.g., $N$ and $D$). However, there remain many decisions affecting the way in which data is selected to for scaling law optimization.

%  - [DONE] knowing data composition matters because knowing data quality can change exponent across different studies ofc CITE


% Moreover, hyperparams can matter
%  - [DONE] For example, learning rate schedule can changes results CITE
%  - [DONE] batch size can change
%  - [DONE] optimal hparams change with scale so determining those matters
%  - [DONE] embedding size has been shown to matter - part of scaling law 



Scaling law fit depends on the performance of each individual checkpoint, which is highly dependent on factors such as training data source, architecture and hyperparameter choice. \citet{bansal2022data} and \citet{goyal2024scaling}, for instance, discuss the effect of data quality and composition on power law exponents and constants. Repeating data has also been found to yield different scaling patterns in large language models \citep{muennighoff2024scaling,goyal2024scaling}. 

Researchers have also studied the effect of architecture choice on scaling - \citet{hestness2017deep} find that architectural improvements only shift the irreducible loss, while \citet{poli2024mechanistic} suggest that these improvements may be more significant. The way in which a model is scaled can also affect results. Within the same architecture family, \citet{clark2022unified} show that increasing the number of experts in a routed language model has diminishing returns beyond a point, while \citet{ghorbani2021scaling} find that scaling the encoder and decoder have different effects on model performance. Scaling embedding size can also drastically change scaling trends \citep{tao2024scaling}.

The optimal hyperparameters to train a model changes with scale. Changing batch size, for example, can change model performance \cite{mccandlish2018empirical, kaplan2020scaling}. Optimal learning rate is another hyperparameter shown to change with scale, though techniques such as those proposed in Tensor Programs series of papers \citep{yang2022tensor} can keep this factor constant with simple changes to initialization. More specifically, changing the learning rate schedule from a cosine decay to a constant learning rate with a cooldown (or even changing the learning rate hyperparameters) has been found to greatly affect the results of scaling laws studies \citep{hu2024minicpm,porian2024resolving,hagele2024scaling, hoffmann2022training}.

% \ml{add paragraph about max model params}

% \srk{talk abt }

% - LR \citep{hu2024minicpm}.
% - batch size

% Some factors, such as training data distribution, do not have a single optimal value and may simply be held constant for all experiments, but they change the absolute value of all power law parameters, e.g. it is not meaningful to compare power laws fit to experiments on different training data distributions. Other factors, such as batch size, may have optimal values depending on other hyperparameters, data budget, architecture, etc. It is sometimes possible to fix these factors in relation to others (e.g., scale batch size with the parameter count). There are other factors for which there is no smooth interpolation between points. For example, the width and depth of a transformer are limited to integer values, and further limited by the convention of using widths equal to small multiples of powers of 2.

% \srk{we should concretely discuss learning rate schedules and cite miniCPM, the LR schedule rate paper}
One common motivation for fitting a scaling laws is extrapolation to higher compute budgets. However, there is no consensus on the orders of magnitude up that one can project a scaling law and still find it accurate, nor on the breadth of compute budgets that should be covered by the data. We find that the range of model size $N$ and dataset size $D$ greatly varies, with the maximum value of $N$ in each paper ranging from 10M parameters to around 7B and that of $D$ being as large as 400B tokens. 
% \textcolor{blue}{
For most papers we survey, the scales are relatively modest: 13 of 51 papers train models beyond 2B parameters; most only train models smaller than 1B parameters.
% }
% \srk{idk what this refers to: Some papers show projected results at as much as 6 or 7 orders of magnitude greater than the compute budgets they experimented with. } 
It has been shown, with some controversy \cite{schaeffer2023emergent}, that scaling to significantly larger scales can result in new abilities that did not appear in smaller models \citep{wei2022emergent}. Forecasting limits to extrapolation and the appearance of new abilities at new scales is an open question.

% 
% - while SOTA models go till 100s of B; 
% - scaling laws often only go until 1b
% - it remains unclear what the corrcet scale is

% \srk{mention emergent abilities; and counterpt}

% \srk{Training data [DONE] and downstream metrics chosen may completely change exponent?}

We propose \method, a simple and scalable algorithm for multi-turn code generation using execution feedback. \method follows an \emph{expert iteration}~\citep{anthony2017thinkingfastslowdeep} framework with a \emph{local search expert}. 
\method iteratively trains two components -- a \emph{learned verifier} $R_{\phi}$ to score code snippets (Section~\ref{sec:training_learned_verifiers}), and a \emph{generator} $\pi_\theta$ to imitate local search with the verifier (Section~\ref{sec:training_generator}). This iterative process allows the generator and expert to bootstrap off each other, leading to continuous improvement. At inference time, both the generator and verifier are used as BoN search to select and execute code (Section~\ref{sec:inference_time}). Finally, we analyze the performance of \method in Section~\ref{sec:analysis}.

\subsection{The \method Algorithm}
\label{sec:mucode}
Algorithm~\ref{algo:pseudo_training_mucode} presents the iterative training procedure. At an iteration $i$, the current generator $\pi_{\theta}$ is rolled out in the multi-turn code environment $\mathcal{E}$ to generate interaction data $\mathcal{D}_i\leftarrow\{(x,s_t,y_t, r_t)\}$. Every turn $t$ in $\mathcal{D}_i$ includes the prompt $x$, interaction history $s_t$, code generated $y_t$ and the correctness score from the oracle verifier $r_t=R(x, y_t)$. 
This data is then aggregated $\mathcal{D} \gets \mathcal{D} \cup \mathcal{D}_i$. 
The learned verifier $R_{\phi}^i$ is trained on the aggregated data $\mathcal{D}$.
An expert is created using $R_\phi^i$ to perform local search to find the optimal action $\pi_\star^i(x) = \arg\max_{y \in \mathcal{D}(x)} R_\phi^i(x,y)$, where $\mathcal{D}(x)$ are all the code completions for a given prompt $x$. The expert $\pi_\star^i(x)$ relabels the data $\mathcal{D}$ with the optimal action. The generator $\pi_\theta^i$ is then trained via fine-tuning (FT) on $\mathcal{D}$. This process iterates $M$ times, and the best generator and verifier pair on the validation dataset are returned.

\subsection{Training Verifier}
\label{sec:training_learned_verifiers}
The learned verifier provides dense scores to code solutions for a given problem. At train time, this is used by the expert to perform local search to obtain optimal code. At inference time, the verifier is used for multi-turn BoN~(\ref{sec:inference_time}) for efficient search. 
The learned verifier has two distinct advantages over process reward functions typically used in multi-turn RL:
(1) It is conditioned only on the initial prompt and the current solution, and is not dependent on previous states (2) It is trained via supervised learning on oracle reward labels. We explore two different losses: % for the learned verifier:

\textbf{Binary Cross-Entropy loss}~(BCE): 
The nominal way to train the verifier is to directly predict the oracle reward ~\citep{cobbe2021training}:
\begin{equation}
\begin{aligned}
    \mathcal{L}_{\rm BCE}(\phi) = -\mathbb{E}_{(x, y, r) \sim \mathcal{D}}[r \log R_{\phi}(x,y) \\ - (1 - r) \log R_{\phi}(x, y)]
\end{aligned}
\end{equation}

\textbf{Bradley Terry Model}~(BT):
Since the goal of the verifier is to relatively rank code solutions rather than predict absolute reward, we create a preference dataset and then train with a  Bradley Terry loss~\citep{ouyang2022traininglanguagemodelsfollow}.
For every prompt $x$, we create pairs of correct $y^{+}$ (where $r=1$) and incorrect $y^{-}$ (where $r=0$) code and define the following loss:
\begin{equation}
    \small
    \mathcal{L}_{BT}(\phi) = -\mathbb{E}_{(x, y^+, y^-) \sim \mathcal{D}} [\log \sigma(R_{\phi}(x,y^+) - R_{\phi}(x,y^-))].
\end{equation}
where $\sigma(.)$ is the sigmoid function. 
We hypothesize that BT is strictly easier to optimize as the verifier has to only focus on relative performance. This is also consistent with observations made for training process reward models, where the advantage function is easier to optimize than the absolute Q function~\citep{setlur2024rewarding}. 



\subsection{Training Generator}
\label{sec:training_generator}

\method comprises a generator~$\pi_\theta$ trained to produce code solutions conditioned on the initial problem and execution observations from previous turns.
Given a dataset $\mathcal{D}$, \method iteratively trains the generator to find the optimal code solution labeled using the local expert over the learned verifier.
% \method constructs an expert using local search and the learned verifier.
For this step, \method extracts all code solutions from $\mathcal{D}$ for every problem $x$.
An expert is then created by picking the best solution, $y^\star$, which achieves the highest score using with the learned verifier~$R_{\phi}(x, y)$ and is given by
\begin{equation}
    y^\star = \pi_\star(x) = \arg\max_{y \in \mathcal{D}(x)} R_\phi(x,y).
\end{equation}
Using this expert dataset, we relabel the dataset $\mathcal{D}$ with the optimal solutions for each prompt: 
\begin{equation}
    \mathcal{D}_\star = \{(x, s_t, y^\star)~|~(x, s_t) \sim \mathcal{D}\},
\end{equation}
where $\mathcal{D}_\star$ represents the expert dataset. 
%As a result of this relabeling, every trajectory for a problem \(x\) ends with the same optimal solution. 
The generator $\pi_{\theta}$ is then trained via fine-tuning~(FT) on this expert dataset $\mathcal{D}_\star$.

\begin{figure*}[t]
    \centering
    \vspace{-1mm}
    \includegraphics[width=1\linewidth]{figures/inference.pdf}
\vspace{-7mm}
    \caption{\textbf{Overview of our inference pipeline for illumination estimation.} A neutral color checker is pasted onto the input image, which is then encoded into the latent space. The input latent is processed through Laplacian composition before being concatenated with the masked image latent and the resized mask. The modified U-Net generates an inpainted result at fixed timestep $T$. After inverse gamma correction, we sample the color checker patches to obtain the final RGB illumination value. We highlight the steps and components that are different from the training pipeline.}
    \label{fig:inference}
\end{figure*}



\subsection{Inference: Multi-turn Best-of-N}
\label{sec:inference_time}
At inference time, the goal is to generate a code solution with a fixed inference budget -- denoting the number of times generators can provide one complete solution.
In this work, we propose to leverage the learned verifier to improve search and code generations over successive turns with \textit{multi-turn Best-of-N}~(BoN).
To achieve this, \method uses a natural extension of BoN to the multi-turn setting.
At each turn, the generator produces $N$ one-step rollouts $\{y^n_t\}_{n=1}^N \sim\pi_\theta(.|s_{t})$ and the learned verifier picks the most promising code solution among these candidates using
\begin{equation}
    y_t^*=\arg \max_{n} R_{\phi}(x,y^{n}_t). 
\end{equation}
The selected code~$y_t^*$ is executed in the environment over public tests to obtain the execution feedback $o_t$.
This solution and the feedback is provided as context to the generator at the next turn to repeat this procedure.
The search ends once~$y_t^*$ passes all public tests or when the turn limit is reached. Consequently, even if $R_\phi(\cdot)$ grants a high score to a code solution, inference continues until the solution has successfully cleared all public tests, thus mitigating potential errors by $R_\phi(\cdot)$.
The final response $y^*_t$ is then passed through the oracle verifier to check its correctness.
Algorithm~\ref{algo:pseudo_inference_mucode} describes a description of multi-turn BoN.
We found it beneficial to use the reward model trained with samples of the latest generator $\pi_\theta$ (see Table~\ref{tab:bon_results}).

\subsection{Analysis}
\label{sec:analysis}
\method effectively treats multi-turn code generation as an interactive imitation learning problem by collecting rollouts from a learned policy and re-labeling them with an expert. It circumvents the exploration burden of generic reinforcement learning which has exponentially higher sample complexity~\cite{sun2017deeply}. We briefly analyze why this problem is amenable to imitation learning and prove performance bounds for \method.

\begin{definition}[One-Step Recoverable MDP]
\label{def:one_step_recoverability}
A MDP $\mathcal{M} = (\mathcal{S}, \mathcal{A}, P, R, \gamma)$ with horizon $T$ is \emph{one-step recoverable} if the advantage function of the optimal policy $\pi^*$, defined as $A^*(s, a) = Q^*(s, a) - V^*(s)$, is uniformly bounded for all $(s, a)$, i.e. $A^*(s, a) \leq 1$.
\end{definition}

\paragraph{Code generation is one-step recoverable MDP.}
Multi-turn code generation satisfies one-step recoverability because the optimal policy $\pi^*(y_t | s_t)$ depends only on the problem prompt $x$ and not the interaction history $s_t = (x, y_1, o_1, \dots, y_{t-1}, o_{t-1})$. Since the correctness of a code snippet $y_t$ is fully determined by $x$, the optimal Q-function satisfies $Q^*(s_t, y_t) = R(x, y_t)$, where $R(x, y_t) \in \{0,1\}$. The optimal value function is $V^*(s_t) = \max_{y_t} R(x, y_t)$, so the advantage function simplifies to $A^*(s_t, y_t) = R(x, y_t) - \max_{y_t'} R(x, y_t') \leq 1$. 

\paragraph{Code generation enables efficient imitation learning.} There are two challenges to applying interactive imitation learning ~\cite{ross2011reduction, ross2014reinforcement} -- (1) Existence of expert policies or value functions, and (2) Recoverability of expert from arbitrary states. First, for code generation, the expert is simply the one-step reward maximizer $\arg\max_y R(x, y)$. We can efficiently estimate $R_\phi(x, y)$ to compute the expert, without needing to compute value function backups. Second, even if the learner fails to imitate the expert at any given state, the expert can perfectly recover from the next state. 
This results in the best possible performance bounds for imitation learning, which we formalize below. 
\begin{theorem}[Performance bound for \method]
\label{thm:one_step_bound} 
For a one-step recoverable MDP $\mathcal{M}$ with horizon $T$, running $N$ iterations of \method yields at least one policy $\pi$ such that
\begin{equation}
    J(\pi^*) - J(\pi) \leq O(T (\epsilon + \gamma(N))).
\end{equation}
where $\pi^*$ is the expert policy, $\epsilon$ is the realizability error, and $\gamma(N)$ is the average regret.
\end{theorem}
Proof is in Appendix~\ref{appendix:proof}. The bound $O(\epsilon T)$ is much better than the worst-case scenario of $O(\epsilon T^2)$ for unrecoverable MDPs~\cite{swamy2021moments}. Thus, \method exploits the structure of multi-turn code generation to enable imitation learning, bypassing the need for hierarchical credit assignment. More generally, this analysis suggests that for any task where the optimal action is history-independent and recoverable in one step, reinforcement learning can be reduced to efficient imitation learning without loss of performance.