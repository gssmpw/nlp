
\documentclass{article} % For LaTeX2e
\usepackage{iclr2025_conference,times}

% Optional math commands from https://github.com/goodfeli/dlbook_notation.
%%%%% NEW MATH DEFINITIONS %%%%%

% \usepackage{amsmath,amsfonts,bm}
\usepackage{amsmath,amsfonts}

\usepackage{pifont}


\newcommand{\R}{\mathbb{R}}


\def\va{{\mathbf{a}}}
\def\vg{{\mathbf{g}}}

% Sets
\def\sR{\mathbb{R}}
\def\sC{\mathbb{C}}
\def\sZ{\mathbb{Z}}
\def\sN{\mathbb{N}}
\def\sQ{\mathbb{Q}}

\def\sS{\mathcal{S}}



% Vectors
\def\vzero{{\mathbf{0}}}
\def\vone{{\mathbf{1}}}
\def\vmu{{\mathbf{\mu}}}
\def\vtheta{{\mathbf{\theta}}}
\def\va{{\mathbf{a}}}
\def\vb{{\mathbf{b}}}
\def\vc{{\mathbf{c}}}
\def\vd{{\mathbf{d}}}
\def\ve{{\mathbf{e}}}
\def\vf{{\mathbf{f}}}
\def\vg{{\mathbf{g}}}
\def\vh{{\mathbf{h}}}
\def\vi{{\mathbf{i}}}
\def\vj{{\mathbf{j}}}
\def\vk{{\mathbf{k}}}
\def\vl{{\mathbf{l}}}
\def\vm{{\mathbf{m}}}
\def\vn{{\mathbf{n}}}
\def\vo{{\mathbf{o}}}
\def\vp{{\mathbf{p}}}
\def\vq{{\mathbf{q}}}
\def\vr{{\mathbf{r}}}
\def\vs{{\mathbf{s}}}
\def\vt{{\mathbf{t}}}
\def\vu{{\mathbf{u}}}
\def\vv{{\mathbf{v}}}
\def\vw{{\mathbf{w}}}
\def\vx{{\mathbf{x}}}
\def\vy{{\mathbf{y}}}
\def\vz{{\mathbf{z}}}
\def\vzeta{{\mathbf{\zeta}}}

% Matrix
\def\mA{{\mathbf{A}}}
\def\mB{{\mathbf{B}}}
\def\mC{{\mathbf{C}}}
\def\mD{{\mathbf{D}}}
\def\mE{{\mathbf{E}}}
\def\mF{{\mathbf{F}}}
\def\mG{{\mathbf{G}}}
\def\mH{{\mathbf{H}}}
\def\mI{{\mathbf{I}}}
\def\mJ{{\mathbf{J}}}
\def\mK{{\mathbf{K}}}
\def\mL{{\mathbf{L}}}
\def\mM{{\mathbf{M}}}
\def\mN{{\mathbf{N}}}
\def\mO{{\mathbf{O}}}
\def\mP{{\mathbf{P}}}
\def\mQ{{\mathbf{Q}}}
\def\mR{{\mathbf{R}}}
\def\mS{{\mathbf{S}}}
\def\mT{{\mathbf{T}}}
\def\mU{{\mathbf{U}}}
\def\mV{{\mathbf{V}}}
\def\mW{{\mathbf{W}}}
\def\mX{{\mathbf{X}}}
\def\mY{{\mathbf{Y}}}
\def\mZ{{\mathbf{Z}}}
\def\mBeta{{\mathbf{\beta}}}
\def\mPhi{{\mathbf{\Phi}}}
\def\mLambda{{\mathbf{\Lambda}}}
\def\mSigma{{\mathbf{\Sigma}}}


% Expectation
% \def\eE{\mathop{\mathbb{E}}\limits}
\def\eE{\mathbb{E}}

% Probability
\def\pP{\mathbb{P}}

% Tilde
\def\tf{\tilde{f}}
\def\tS{\tilde{S}}
\def\wtF{\widetilde{\mathcal{F}}}
\def\whR{\widehat{R}}
\def\tvx{\tilde{\mathbf{x}}}
\def\ty{\tilde{y}}


\def\defeq{\overset{\textup{def}}{=}}
% \def\defeq{\overset{.}{=}}
\def\defone{\overset{\text{\ding{172}}}{=}}
\def\deftwo{\overset{\text{\ding{173}}}{=}}
\def\leqone{\overset{\text{\ding{172}}}{\leq}}
\def\leqtwo{\overset{\text{\ding{173}}}{\leq}}
\def\leqthree{\overset{\text{\ding{174}}}{\leq}}
\def\leqfour{\overset{\text{\ding{175}}}{\leq}}
\def\eqone{\overset{\text{\ding{172}}}{=}}
\def\eqtwo{\overset{\text{\ding{173}}}{=}}
\def\eqthree{\overset{\text{\ding{174}}}{=}}
\def\eqfour{\overset{\text{\ding{175}}}{=}}
\def\geqfive{\overset{\text{\ding{176}}}{\geq}}

\usepackage{hyperref}
\usepackage{url}
\usepackage{pdfpages}
\usepackage{xcolor}
\usepackage{booktabs}
\usepackage{multirow}
\usepackage{graphicx}
\usepackage{nth}

\definecolor{changran}{RGB}{255,100,0} % Orange color
\definecolor{yunhao}{RGB}{0,0,255} % Green color


\title{DeepRTL: Bridging Verilog Understanding and Generation with a Unified Representation Model}

% Authors must not appear in the submitted version. They should be hidden
% as long as the \iclrfinalcopy macro remains commented out below.
% Non-anonymous submissions will be rejected without review.

\author{Yi Liu, Changran Xu, Yunhao Zhou, Zeju Li, Qiang Xu \\
Department of Computer Science and Engineering\\
The Chinese University of Hong Kong\\
\texttt{\{yliu22,zjli24,qxu\}@cse.cuhk.edu.hk}\\
\texttt{\{xxuchangran,yunhaoz.cs\}@gmail.com}
}

% The \author macro works with any number of authors. There are two commands
% used to separate the names and addresses of multiple authors: \And and \AND.
%
% Using \And between authors leaves it to \LaTeX{} to determine where to break
% the lines. Using \AND forces a linebreak at that point. So, if \LaTeX{}
% puts 3 of 4 authors names on the first line, and the last on the second
% line, try using \AND instead of \And before the third author name.

\newcommand{\fix}{\marginpar{FIX}}
\newcommand{\new}{\marginpar{NEW}}

\iclrfinalcopy % Uncomment for camera-ready version, but NOT for submission.
\begin{document}
\maketitle

\begin{abstract}

% Recent works to jointly reconstruct 3D human and object from a single RGB image, are mostly model-based, that fail to capture the fine details of the clothed human body and object surface. In this paper, we introduce ReCHOR, a novel, model-free, first-method to produce realistic clothed human-object reconstructions from a monocular view. This is extremely challenging due to human-object occlusions, diverse interactions and depth ambiguity, as it needs to infer both 3D spatial awareness and high resolution details. Our core idea is based on estimating neural implicit representations for human and object respectively by an attention-based neural implicit model that attends to pixel-aligned features from both the global human-object image for spatial awareness and  the local separate view of human and object images for high quality details. Additionally, the network is conditioned on semantic features from an initial estimated human-object pose prior and a generative diffusion model that inpaints occluded regions, thus enabling the retrieval of details from them.
% We also propose a synthetic dataset with rendered scenes of diverse, inter-occluded 3D human and object scans, to train our network. We evaluate our method on the synthetic and real world BEHAVE dataset. Our experiments show that our method outperforms the SOTA in achieving realistic clothed human-object reconstructions.
Recent approaches to jointly reconstruct 3D humans and objects from a single RGB image represent 3D shapes with template-based or coarse models, which fail to capture details of loose clothing on human bodies. In this paper, we introduce a novel implicit approach for jointly reconstructing realistic 3D clothed humans and objects from a monocular view. For the first time, we model both the human and the object with an implicit representation, allowing to capture more realistic details such as clothing. This task is extremely challenging due to human-object occlusions and the lack of 3D information in 2D images, often leading to poor detail reconstruction and depth ambiguity. To address these problems, we propose a novel attention-based neural implicit model that leverages image pixel alignment from both the input human-object image for a global understanding of the human-object scene and from local separate views of the human and object images to improve realism with, for example, clothing details. Additionally, the network is conditioned on semantic features derived from an estimated human-object pose prior, which provides 3D spatial information about the shared space of humans and objects. To handle human occlusion caused by objects, we use a generative diffusion model that inpaints the occluded regions, recovering otherwise lost details. For training and evaluation, we introduce a synthetic dataset featuring rendered scenes of inter-occluded 3D human scans and diverse objects. Extensive evaluation on both synthetic and real-world datasets demonstrates the superior quality of the proposed human-object reconstructions over competitive methods.
\end{abstract}
\section{Introduction}
\label{sec:intro}
% Image editing methods in diffusion models depend on user-defined control directions - users can unlock their creativity using these methods by specifying the desired manipulation through prompts~\cite{gandikota2023concept}, reference images~\cite{ruiz2022dreambooth, kumari2022customdiffusion, gal2022image, chen2024trainingfreeregionalpromptingdiffusion}, or attribute vectors~\cite{parmar2023zero,hertz2022prompt}. In this work, we ask a fundamentally different question: \emph{Can we automatically discover the underlying visual structure of a concept within diffusion model's knowledge?} %Rather than requiring user-specified controls, we aim to decompose the model's internal knowledge into meaningful directions.

% This question touches on a fundamental limitation in how we interact with diffusion models. Current control methods ~\cite{zhang2023addingconditionalcontroltexttoimage, gandikota2023concept, ye2023ipadaptertextcompatibleimage,ye2023ipadaptertextcompatibleimage, hertz2024stylealignedimagegeneration, li2023photomaker, shi2024instantbooth, chen2024trainingfreeregionalpromptingdiffusion} require users to specify their desired manipulations in advance, limiting interactive creativity. This contrasts with natural human artistic workflows, where creators dynamically explore creative ideas while jointly refining them toward meaningful artistic outcomes~\cite{hoffmann2016modeling}. This synergy between specification and exploration is not new to generative models. Early GAN architectures naturally developed disentangled latent spaces that enabled continuous\cite{harkonen2020ganspace,radford2015unsupervised, wu2021stylespace, shen2020interfacegan}, compositional control over generated images. Users could explore these spaces to discover interesting variations that would be difficult to describe in words~\cite{wu2021stylespace}, then combine them to achieve their creative goals~\cite{grabe2022towards}. 


% While diffusion models have largely superseded GANs in conditional image synthesis~\cite{dhariwal2021diffusion},  their underlying structure remains less understood. Diffusion models achieve remarkable diversity through high-dimensional latents, unlike GANs' compact latent spaces.  With a single prompt, diffusion models can generate radically different variations through different random initializations of input noise. We ask - Is it possible to discover interpretable structure within this vast space of variations?

Text-to-image diffusion models are capable of generating remarkable visual variations from a single prompt through different random initializations. However, this vast creative potential remains largely opaque to users---while we can generate diverse images, we lack understanding of the underlying structure of these variations. This presents a fundamental challenge: how can we discover and expose the latent visual capabilities encoded within these models?

\let\thefootnote\relax \footnote{$^{*}$Correspondence to \texttt{gandikota.ro@northeastern.edu}}

The challenge touches on a key limitation in how we interact with diffusion models today. Current control methods require users to explicitly specify their desired edits in advance through prompts~\cite{gandikota2023concept}, reference images~\cite{zhang2023addingconditionalcontroltexttoimage, chen2024trainingfreeregionalpromptingdiffusion, ruiz2022dreambooth,kumari2022customdiffusion, Ryu_lora, hu2021lora}, or attribute vectors~\cite{ye2023ipadaptertextcompatibleimage, hertz2024stylealignedimagegeneration, li2023photomaker, shi2024instantbooth,parmar2023zero,hertz2022prompt}. That contrasts sharply with natural human creative workflows, where artists dynamically explore creative ideas and jointly refine them toward meaningful artistic outcomes~\cite{hoffmann2016modeling}. The need for pre-specified controls creates a barrier between users and the full creative potential of these models.

Interestingly, earlier generative models like GANs~\cite{gans,karras2019style,brock2018large} naturally developed more interpretable internal structures. Their compact latent spaces often exhibited emergent disentanglement~\cite{harkonen2020ganspace,radford2015unsupervised, wu2021stylespace, shen2020interfacegan}, enabling continuous and compositional control over generated images. Users could explore these spaces to discover interesting variations that would be difficult to describe in words~\cite{wu2021stylespace}, then combine them to achieve their creative goals~\cite{grabe2022towards}.

Diffusion models have largely superseded GANs in conditional image synthesis~\cite{dhariwal2021diffusion}, achieving greater diversity through much higher-dimensional latents. And yet an understanding of the underlying structure of these larger latent spaces has remained elusive. In this work, we ask a fundamental question: \emph{Can we automatically discover the visual structure within a diffusion model's knowledge of a concept?} Rather than requiring user-specified controls, we aim to decompose the model's internal representations into expressive directions that users can explore and combine.

To address these needs, we present \textbf{SliderSpace}, a framework that brings systematic explorability to diffusion models. Given just a text prompt, SliderSpace discovers a canonical set of meaningful, diverse, and controllable directions within the model's knowledge of that concept. Each direction is implemented as a low-rank adapter~\cite{hu2021lora} that can be scaled and composed with others, allowing users to explore and smoothly combine different aspects of variation, as shown in Figure~\ref{fig:intro}.

We ground SliderSpace discovery in three key requirements for meaningful decomposition of a diffusion model's visual manifold: 
\begin{enumerate}
    \item \textbf{Unsupervised Discovery:} The decomposition process should emerge from the intrinsic structure of the model's learned representation, rather than being guided by predefined attributes. This ensures we capture the true topology of the model's knowledge space rather than projecting our assumptions onto it.
    
    \item \textbf{Semantic Orthogonality:} Each discovered control must represent a distinct semantic direction. This is enforced in a semantic feature space, like CLIP, where every slider has an orthogonal effect in embeddings. This prevents discovering multiple controls that create similar semantic effects, making the system more efficient and easier.
    
    \item \textbf{Distribution Consistency:} Directions must induce consistent transformations across both random seeds and prompt variations. 
\end{enumerate}

These requirements naturally lead to our proposed framework, which we formalize in Section~\ref{sec:method}. As we show in our experiments, SliderSpace is architecture-agnostic, working with both conventional U-Net based models like Stable Diffusion~\cite{rombach2022high, rombach2022sd20, podell2023sdxl, turbo, dmd} and recent transformer-based architectures like Flux~\cite{flux}.

We demonstrate the expressiveness of SliderSpace through three applications: First, we show how SliderSpace can decompose high-level concepts into diverse and expressive components, revealing the natural axes of variation in the model's understanding. Second, we explore artistic style variation, where SliderSpace discovers directions that match or exceed the diversity of manually curated artist lists while being judged more useful by human evaluators. Finally, we show how SliderSpace can help reverse the mode collapse commonly observed in distilled diffusion models, restoring diversity while maintaining generation speed.

Beyond providing practical creative control, SliderSpace opens new avenues for understanding and utilizing the latent capabilities of diffusion models. By mapping these models' visual potential into intuitive, composable directions, we take a step toward making their creative possibilities more accessible and interpretable to users.

% Image editing methods in diffusion models unlock the creativity of users. In this work we ask an alternate question: \emph{Can we organize and expose what of the diffusion model is already capable of?}.
% Existing methods for controlling image generation typically require users to manually specify edit directions for desired changes. This process is time-consuming, requires technical expertise, and limits the spontaneity of the creative process. For instance, if a user wants to adjust the smile of a generated person, they must explicitly request this edit, often through imprecise prompt engineering or model fine-tuning. This approach of predefined controls or manual specifications restricts users from fully exploring the latent capabilities of the model. There may be interesting stylistic variations or attributes that the model can generate, but users have no easy way to discover or utilize these.

% Natural visual disentanglement was an emergent property in the latent space of Generative Adversarial Models (GANs) \cite{harkonen2020ganspace,radford2015unsupervised, wu2021stylespace, shen2020interfacegan}. In particular, it has been observed that StyleGAN~\cite{karras2019style} stylespace neurons offer detailed control over many meaningful aspects of images that would be difficult to describe in words~\cite{wu2021stylespace}. However, diffusion models do not share such a compact latent space~\cite{park2023unsupervised}; and efforts to uncover such a space in the semantic embeddings of the text conditioning have met with limited success \nik{Nick - is there a specific citation you were thinking about?}.

% In this work we introduce \textbf{SliderSpace}, which takes a step towards uncovering an analogous low dimensional representation of diffusion models' visual breadth; in essence treating the diffusion model as many generators sharing parameters, where a particular generator is defined by a specific prompt. For a given prompt we sample many random seeds (and optionally prompt expansions using an LLM), generate the corresponding images, and apply an off the shelf feature extractor (in this work CLIP, but our method can be applied to any differentiable feature extractor). We use PCA to analyze these features, and for each of the leading $k$ principal components we train a LoRA \cite{} which causes the diffusion model to produces images which increase the feature magnitude along that component when passed back through the same feature extractor. This leads to a 'Slider' for each principal component, because each LoRA can be scaled and applied to the original diffusion model, continuously varying those visual features in the generated results (as measured, in our case, by CLIP).

% There are many other works that enhance the controllability of diffusion models. One common approach is enabling users to add spatial constraints to a generation either manually, or via a reference image \cite{zhang2023addingconditionalcontroltexttoimage, chen2024trainingfreeregionalpromptingdiffusion}, a second is leveraging more abstract embeddings (e.g. identity, style) extracted from a reference image \cite{ye2023ipadaptertextcompatibleimage, hertz2024stylealignedimagegeneration, li2023photomaker, shi2024instantbooth}, a third is finetuning a foundation model to better generate a concept important to the user \cite{ruiz2022dreambooth, kumari2022customdiffusion, Ryu_lora, hu2021lora}, and a fourth (most relevant to this work) is finding low-rank adaptors of the model based on a prompt or small training set which can be scaled to provide continous control over one aspect of generated image (e.g. night vs day, basic vs luxury, etc.) \cite{gandikota2023concept}. SliderSpace is complementary to all of these methods and offers something distinct. All of the other methods we are aware require the user (and / or model designer) to know in advance what type of control they want. In contrast SliderSpace assists users in discovering and controlling hidden capabilities present in the diffusion model's distribution of possible generations.

%We propose that truly intuitive creative control in a text-to-image model should meet three key criteria: \emph{discoverability}, \emph{intuitiveness}, and \emph{specificity}. The model should reveal controllable attributes that may not be immediately obvious, offer controls that are easy to understand and manipulate, and ensure each control affects a distinct attribute of the generated image.

% We demonstrate the utility and power of SliderSpace using three applications built on top of SDXL-DMD \cite{dmd}, because its fast generation speed lends itself well to the continuous control offered by SliderSpace.

% First, we study concept decomposition (Section \ref{sec:concept_exp}), where we learn sliders for a specific concept (e.g. 'monster', 'waterfall', 'car'). Through quantitative metrics of diversity and text alignment we demonstrate that the learned sliders dramatically boost the diversity of generations when randomly applied without harming text alignment; we also ask humans to qualitatively judge these results in a user study where they find the SliderSpace results to be more 'Diverse', 'Useful', and 'Creative' than our baselines.

% Second, we attempt to compare the automatic discoveries of SliderSpace to a large scale manual study of artistic styles (Section \ref{sec:art_exp}), open-sourced by ParrotZone \cite{parrotzone}. In this study SDXL was prompted with over 4300 artist names,  and based on visual inspection the cases of successful stylistic mimicry recorded. Quantitatively SliderSpace more closely matches the distribution of artistic variation discovered by ParrotZone than other baselines, and in our user studies was judged to be significantly more 'Diverse' and 'Useful' than the baselines. To our surprise humans even judged SliderSpace results to be slightly more 'Diverse' than the results generated by the manually discovered artist names of \cite{parrotzone}.

% Third, we attempt to use SliderSpace to reverse the mode collapse commonly observed in distilled few-step diffusion models relative to the original teacher model (Section \ref{sec:diverse_exp}). We quantitatively demonstrate that applying SliderSpace to SDXL-DMD leads to more closely matching the distribution of images by the original teacher, SDXL.

%Through extensive experiments on various state-of-the-art text-to-image models, we demonstrate that SliderSpace significantly enhances user control and creative expression in AI-assisted image generation tasks. Our method enables a range of applications, including concept decomposition and control, diversity improvement in generated images, customization dissection and edits, and the exploration of artistic styles inherent in the model.

% SliderSpace goes beyond providing a practical tool for enhanced creative control. By mapping the visual potential of diffusion models it can open new avenues for generative creativity and deepens our understanding of each model's hidden potential.
\section{Related Work}
\label{sec:related_work}

The original investigation \cite{gibson1979ecological} on the relationship between visual perception and human action defines \emph{affordance} as the opportunities for interaction with the surrounding environment. Behavioral studies on regular and cognitively impaired persons have shown evidence that perception results in both visual and motor signals in the human brain. An extended study \cite{anderson2002attentional} shows that visual attention to the spatial characteristics of the perceived objects initiates automatic motor signals for different actions. In computer vision, human affordance learning involves novel pose prediction such that the estimated pose represents a valid human action within the scene context. The task is fundamental to many problems requiring robust semantic reasoning about the environment, such as human motion synthesis \cite{wang2021scene} and scene-aware human pose generation \cite{wang2017binge, roy2016multi, zhang2022inpaint, yao2023scene}.

Earlier methods of affordance learning have explored knowledge mining \cite{zhu2014reasoning} and multimodal feature cues \cite{roy2016multi} to address the problem. In \cite{zhu2014reasoning}, the authors use a Markov Logic Network for constructing a knowledge base by extracting several object attributes from different image and metadata sources, which can perform various downstream visual inference tasks without any additional classifier, including zero-shot affordance prediction. In \cite{roy2016multi}, the authors use depth map, surface normals, and segmentation map as multimodal cues to train a multi-scale convolutional neural network (CNN) for scene-level semantic label assignment associated with specific human actions. In \cite{do2018affordancenet}, the authors design a multi-branch end-to-end CNN with two separate pathways for object detection and affordance label assignment to achieve high real-time inference throughput. Researchers \cite{chuang2018learning} have also explored socially imposed constraints for affordance learning. In \cite{chuang2018learning}, the authors propose a graph neural network (GNN) to propagate contextual scene information from egocentric views for action-object affordance reasoning.

Probabilistic modeling of scene-aware human motion generation also involves semantic reasoning of human interaction with the environment. Initial works on human motion synthesis have taken different architectural approaches, such as sequence-to-sequence models \cite{barsoum2018hp}, generative adversarial networks (GAN) \cite{barsoum2018hp, cai2018deep, yang2018pose}, graph convolutional networks (GCN) \cite{yan2019convolutional}, and variational autoencoders (VAE) \cite{guo2020action2motion}. However, these methods have mostly ignored the role of environmental semantics. Due to potential uncertainty in human motion, in a recent approach \cite{wang2021scene}, the authors address such motion synthesis with a GAN conditioned on scene attributes and motion trajectory to predict probable body pose dynamics.

One key challenge of human affordance generation in 2D scenes is the lack of large-scale datasets with rich pose annotations. In \cite{wang2017binge}, the authors compile the only public dataset of annotated human body poses in complex 2D indoor scenes by extracting frames from sitcom videos. Aiming to generate a contextually valid human affordance at a user-defined location, the authors propose sampling the scale and deformation parameters for an existing human pose template using a VAE conditioned on the localized image patches as scene context. In \cite{zhang2022inpaint}, the authors introduce a two-stage GAN architecture for achieving a similar goal by estimating the affine bounding box parameters to localize a probable human in the scene and then generating a potential body pose at that location. The method uses the input scene, corresponding depth, and segmentation maps as semantic guidance. In \cite{yao2023scene}, the authors propose a transformer-based approach with knowledge distillation for generating human affordances in 2D indoor scenes.


\section{Dataset and Understanding Benchmark}
\begin{figure}[ht]
    \centering
    \includegraphics[width=0.6\linewidth]{fig/cot_v6.pdf}
    \vspace{-12pt}
    \caption{The overview of the data annotation process. We employ the CoT approach and the SOTA LLM, GPT-4, for annotation. Annotations span three levels—line, block, and module—providing both detailed specifications and high-level functional descriptions.}
    \label{fig:cot}
\end{figure}
In this section, we introduce our dataset designed to enhance Verilog understanding and generation, which aligns natural language with Verilog code across line, block, and module levels with detailed and high-level descriptions.
By integrating both open-source and proprietary code, the dataset offers a diverse and robust collection that spans a broad spectrum of hardware design complexities.
% By integrating open-source and proprietary code, we ensure a diverse and robust dataset encompassing a wide range of hardware designs. 
We employ GPT-4 along with the CoT approach for annotation, achieving about $90\%$ accuracy in human evaluations, confirming the dataset's high quality.
We also introduce the first benchmark for Verilog understanding, setting a new standard for evaluating LLMs' capabilities in interpreting Verilog code.





\subsection{Dataset Source}
Our dataset comprises both open-source and proprietary Verilog code. For the open-source part, we gather \texttt{.v} files from GitHub repositories using the keyword \texttt{Verilog}.
These files are segmented into individual modules, each representing a distinct functional unit within the Verilog design.
This segmentation is crucial given the limited context length of current LLMs, improving the efficiency and accuracy of the subsequent annotation and fine-tuning processes.
We employ MinHash and Jaccard similarity metrics~\citep{yan2017privmin} to deduplicate these modules and exclude those predominantly made up of comments or lacking complete \texttt{module} and \texttt{endmodule} structures.
Finally, this process results in a total of 61,755 distinct Verilog modules.
For the proprietary portion, we incorporate a set of purchased intellectual properties (IPs) that enhance the variety and functional diversity of our dataset. This component includes a total of 213 high-quality, industry-standard Verilog modules. These IPs not only offer a range of advanced functions but also provide unique insights that complement the open-source data. Integrating these elements ensures a comprehensive dataset that captures a wide spectrum of hardware design practices.


\subsection{Dataset Annotation}
We employ different annotation strategies for open-source and proprietary code. For open-source code, we utilize the CoT approach with the SOTA LLM, GPT-4, to provide annotations at multiple levels. As illustrated in Figure~\ref{fig:cot}, 
%we begin by removing all comments from the original Verilog module (resulting in refined code) and counting the number of tokens using CodeT5+'s tokenizer. 
we initially remove all comments from the original Verilog code (resulting in refined code) to avoid training complications from incorrect or misleading comments.
If the token count of a complete module exceeds $2048$, the maximum context length for CodeT5+, we utilize GPT-4 to segment the module into smaller, manageable blocks such as \texttt{always} blocks. 
If the resulting blocks still exceed $2048$ tokens, we will discard them. 
For modules and blocks with a token count below $2048$ (qualified code), we then use GPT-4 to add informative comments, resulting in commented code (Step 1).
% If the token count is below $2048$, we treat the module as a complete module and use GPT-4 to add informative comments, resulting in commented code (Step 1). 
% From this commented code, we can extract line-level descriptions. To ensure accuracy, GPT-4 is employed to verify that these line-level descriptions are strictly confined to the context of each individual line, avoiding any external or unrelated information. For example, [insert specific examples here].
From this commented code, we can extract line-level descriptions (pairings of single lines of code with natural language descriptions). To guarantee the accuracy and relevance of the inline comments, we use GPT-4o-mini to rigorously check each comment, ensuring that all line-level descriptions are strictly confined to the context of their respective lines without incorporating any extraneous or irrelevant information. For example, consider the line \texttt{"O = I1;"} annotated with \texttt{"Assign the value of I1 to the output O when S is high."}.
Since we cannot deduce from this single line that \texttt{O} is the output and \texttt{S} is related, such descriptions are deemed inaccurate and are consequently excluded from the dataset to maintain training effectiveness.
% we cannot directly infer the information that O is the output and S is related from the single line of code. This type of line-level description is considered ineffective and detrimental to model training, and thus we will discard it.
% \textcolor{changran}{
% Otherwise, we employ GPT-4 to divide the module into smaller manageable code blocks, such as \texttt{always} blocks, which enhances the utilization of the available Verilog code. [state whether certain techniques are used here]. Blocks with token counts below $2048$ are returned to Step 1 for further processing.
% }
% Once we have the commented code, we use GPT-4 to generate a detailed specification for the code (Step 2). This specification comprises of two key components: a summary of the code's functionality (what it does) and a detailed explanation of the implementation process, including data transitions between registers (how it works). This dual-layered specification provides a deeper understanding of the code. [do we only have what+how for specification?]
In Step 2, we use GPT-4 to generate a detailed specification for the commented code from Step1. 
This specification includes two main components: a summary of the code's functionality (what it does) and a comprehensive explanation of the implementation process (how it works). 
% Specifically, for the implementation process, we will require GPT to include descriptions of the input and output ports, an explanation of the internal workings of the module/block, a description of the logical implementation process, an overview of the algorithmic logic used in the implementation, and an explanation of the internally defined signals. 
Finally, in Step 3, we combine the qualified code from Step 1 with the detailed specification generated in Step 2 to create high-level functional descriptions. 
% To ensure precision, we instruct GPT-4 to prioritize the refined code, using the detailed specification as a reference. 
To ensure precision, we instruct GPT-4 to focus on the qualified code, using the detailed specification only as reference. 
% This approach prevents GPT-4 from potentially overlooking some details in lengthy segments of code.
The resulting high-level descriptions succinctly summarize the code's functionality (what it does) and provide a concise overview of the implementation process (how it works).
This annotation phase is the most critical and challenging as it demands that the model captures the code's high-level semantics, requiring a profound understanding of Verilog. In current benchmarks and practical applications, users typically prompt the model with high-level functional descriptions rather than detailed specifications. Otherwise, they would need to invest significant effort in writing exhaustive implementation details, making the process time-consuming and requiring extensive expertise. For detailed prompts used in this annotation process, please refer to Appendix~\ref{appendix:prompt}.
And a detailed explanation of why we discard Verilog modules or blocks exceeding $2048$ tokens can be found in Appendix~\ref{appendix:discard}.


Given the industrial-grade quality of the proprietary code, we engage professional hardware engineers to maintain high annotation standards. Adhering to rigorous industry-level standards, these experts ensure precise and accurate annotations, capturing intricate details and significantly enhancing the dataset's value for advanced applications.
Unlike GPT-generated annotations, these human-annotated ones incorporate an additional layer of granularity with medium-detail block descriptions.
For detailed annotation standards and processes, please refer to Appendix~\ref{appendix:standard}.

% Due to the industrial-grade quality of the proprietary code, we have high annotation standards and thus hire several professional hardware engineers for annotation. 
% We adhere to rigorous industry-level standards for annotation. Consequently, we have engaged several professional hardware engineers to perform the annotation tasks.
% Their expertise ensures precise and accurate annotations, capturing intricate details and enhancing the dataset's overall value for advanced applications. Specifically, [complete the annotation details].

\begin{table}[ht]
\centering
\vspace{-10pt}
\caption{The overall statistics of the annotation results for our dataset.}
\vspace{5pt}
\begin{tabular}{|c|c|c|}
\hline
\textbf{Comment Level} & \textbf{Granularity} & \textbf{Count} \\ \hline
Line Level             & N/A                  & 434697 \\ \hline
\multirow{3}{*}{Block Level}  & High-level Description    & 892    \\ \cline{2-3} 
                              & Medium-Detail Description & 1306    \\ \cline{2-3} 
                              & Detailed Description      & 894    \\ \hline
\multirow{2}{*}{Module Level} & High-level Functional Description    & 59448 \\ \cline{2-3} 
                              & Detailed Specification             & 59503 \\ \hline
\end{tabular}

\label{tab:dataset_statistics}
\end{table}

We present the overall statistics of the annotation results in Table~\ref{tab:dataset_statistics}. 
Additionally, Figure~\ref{fig:comment_example} illustrates an example of our comprehensive annotation for a complete Verilog module. 
Notably, the overall dataset encompassing descriptions of various details across multiple levels is used for training.
A similar work to ours is the MG-Verilog dataset introduced by~\citet{zhang2024mg}, including 11,000 Verilog code samples and corresponding natural language descriptions at various levels of details.
However, it has several limitations compared to ours. Firstly, MG-Verilog is relatively small in size and lacks proprietary Verilog code, which diminishes its diversity and applicability. Secondly, it employs direct annotation rather than the CoT approach, which we have found to enhance annotation accuracy as demonstrated in Section~\ref{sec:dataset_evaluation}. 
Besides, our annotation is more comprehensive than that of MG-Verilog, which lacks granularity. We cover line, block, and module levels with both detailed and high-level descriptions, ensuring a strong alignment between natural language and Verilog code.
%Besides, our annotation is more comprehensive than that of MG-Verilog which lacks granularity, covering line, block, and module levels with both detailed and high-level descriptions, ensuring strong alignment between natural language and Verilog code. 
Lastly, MG-Verilog relies on the open-source LLM LLaMA2-70B-Chat for annotation, whereas we use the SOTA LLM GPT-4. In Section~\ref{sec:understanding_evaluation}, we demonstrate that LLaMA2-70B-Chat has a poor understanding of Verilog code, leading to inferior annotation quality in MG-Verilog.
\vspace{-10pt}



\begin{figure}[ht]
    \centering
    \includegraphics[width=0.88\linewidth]{fig/comment_example.pdf}
    \vspace{-12pt}
    \caption{An example of our comprehensive annotation for a complete Verilog module.}
    \label{fig:comment_example}
\end{figure}

\vspace{-10pt}
\subsection{Dataset Evaluation}
\label{sec:dataset_evaluation}
To ensure the quality of our dataset, we assess annotations generated from the CoT process. We randomly sample 200 Verilog modules and engage four professional Verilog designers to evaluate the accuracy of annotations at various levels. This human evaluation indicates that annotations describing high-level functions achieve an accuracy of $91\%$, while those providing detailed specifications attain an accuracy of $88\%$. For line-level annotations, the accuracy is $98\%$. Additionally, we compare the CoT method with the direct annotation approach, where annotations are generated straightforwardly from the original code. This direct annotation method yields only a $67\%$ accuracy, highlighting the significant advantage of integrating the CoT process.

Recent studies in natural language processing (NLP) have demonstrated that LLMs fine-tuned with synthetic instruction data can better understand natural language instructions and show improved alignment with corresponding tasks~\citep{wang2022self,ouyang2022training,taori2023stanford}.
It is important to note that in our work, we also utilize data generated by language models for fine-tuning, including annotations at various levels. While not all annotations are perfectly accurate, we achieve a commendable accuracy of approximately $90\%$. Motivated by~\citet{wang2022self}, we treat those inaccuracies as data noise, and the fine-tuned model on this dataset still derives significant benefits.

\subsection{Understanding Benchmark}
\label{sec:understanding_benchmark}
As the first work to consider the task of Verilog understanding, we introduce a pioneering benchmark to evaluate LLMs' capabilities in interpreting Verilog code. This benchmark consists of 100 high-quality Verilog modules, selected to ensure comprehensive coverage of diverse hardware functionalities, providing a broad assessment scope across different types of hardware designs. We have engaged four experienced hardware engineers to provide precise annotations on each module’s functionalities and the specific operations involved in their implementations. These initial annotations are then rigorously cross-verified by three additional engineers to guarantee accuracy and establish a high standard for future model evaluations. This benchmark fills a critical gap by providing a standardized means to assess LLMs on interpreting Verilog code and will be released later. For detailed examples included in the benchmark, please refer to Appendix~\ref{appendix:benchmark}.

\begin{figure}[ht]
    \centering
    \includegraphics[width=0.7\linewidth]{fig/instruction_v2.pdf}
    \vspace{-12pt}
    \caption{The overview of the instruction construction process and the curriculum learning strategy. For instruction construction, we integrate various settings, \textit{e.g.}, task type, granularity, and comment level, to create tailored instructions for specific scenarios. The curriculum learning strategy involves three hierarchical stages: training progresses from line-level to module-level code (\nth{1} stage), transitioning from detailed to high-level descriptions at each level (\nth{2} stage), and advancing from GPT-annotated to human-annotated descriptions for each granularity (\nth{3} stage).}
    \label{fig:instruction}
\end{figure}


\section{Model and Evaluation}

In this section, we introduce DeepRTL and elaborate on the preparation of our instruction tuning dataset and how we adapt curriculum learning for training.
% our progressive training strategy.
% we need an overview paragraph here
%In this section, we introduce DeepRTL and elaborate on the preparation of our instruction tuning dataset, as well as our progressive training strategy.
%, along with specific training and inference settings that optimize model performance. 
Additionally, we detail the benchmarks and metrics used to evaluate our model's performance in both Verilog understanding and generation tasks.
To accurately assess the semantic precision of the generated descriptions, 
we take the initiative to apply embedding similarity and GPT score for evaluation,
% we introduce two novel metrics: embedding similarity and GPT score, 
which are designed to quantitatively measure the semantic similarity between the model's outputs and the ground truth.

\subsection{Model}
In our work, we have chosen to fine-tune CodeT5+~\citep{wang2023codet5+}, a family of encoder-decoder code foundation LLMs for a wide range of code understanding and generation tasks. CodeT5+ employs a ``shallow encoder and deep decoder" architecture~\citep{li2022competition}, where both encoder and decoder are initialized from pre-trained checkpoints and connected by cross-attention layers. 
We choose to fine-tune CodeT5+ for its extensive pre-training on a vast corpus of software code, 
with the intent to transfer its acquired knowledge to hardware code tasks.
Also, the model's flexible architecture allows for the customization of various training tasks, making it highly adaptable for specific downstream applications. Furthermore, CodeT5+ adopts an efficient fine-tuning strategy where the deep decoder is frozen and only the shallow encoder and cross-attention layers are allowed to train, significantly reducing the number of trainable parameters.
Specifically, we have fine-tuned two versions of CodeT5+, codet5p-220m-bimodal\footnote{\url{https://huggingface.co/Salesforce/codet5p-220m-bimodal}} (CodeT5+-220m) and instructcodet5p-16b\footnote{\url{https://huggingface.co/Salesforce/instructcodet5p-16b}} (CodeT5+-16b), on our dataset, resulting in DeepRTL-220m and DeepRTL-16b, respectively. 
For more information on the model selection, please refer to Appendix~\ref{appendix:model_selection}.



\subsection{Instruction Tuning Dataset}
During the fine-tuning process, we adopt the instruction tuning strategy to enhance the adaptability of LLMs, which is particularly effective when handling diverse types of data and tasks.
Given that our dataset features descriptions at multiple levels and our model is fine-tuned for both Verilog understanding and generation tasks, there is diversity in both the data types and tasks.
To accommodate this diversity, we carefully design specific instructions for each scenario, ensuring the model can adjust its output to align with the intended instructions. Figure~\ref{fig:instruction} illustrates how we combine various settings, \textit{e.g.}, task type, granularity, and comment level, to construct tailored instructions for each specific scenario, fostering a structured approach to instruction-based tuning that optimizes the fine-tuning efficacy. For details on the instructions for different scenarios, please refer to Appendix~\ref{appendix:instruction}.

\subsection{Curriculum Learning for DeepRTL}
We adapt curriculum learning for the fine-tuning process, leveraging our structured dataset that features descriptions of various details across multiple levels.
% For the fine-tuning process, we implement a progressive training strategy, 
%recognizing that the model is more influenced by the data it encounters most recently. 
Initially, the model is fine-tuned on line-level and block-level data, subsequently progressing to module-level data. At each level, we start by aligning the detailed specifications with the code before moving to the high-level functional descriptions. 
And fine-tuning typically starts with GPT-annotated data, followed by human-annotated data for each annotation granularity.
Figure~\ref{fig:instruction} provides an illustration of this process.
We adopt such strategy because a particular focus is placed on aligning Verilog modules with their high-level functional descriptions, which poses the greatest challenge and offers substantial practical applications.
This curriculum learning strategy enables the model to incrementally build knowledge from simpler to more complex scenarios. As a result, the models demonstrate impressive performance across both Verilog understanding and generation benchmarks.
Note that we exclude the cases in the benchmarks from our training dataset.
We primarily follow the instruction tuning script of CodeT5+\footnote{\url{https://github.com/salesforce/CodeT5}} in the fine-tuning process, with a modification to expand the input context length to the maximum of $2048$ tokens.  We utilize the distributed framework, DeepSpeed, to efficiently fine-tune the model across a cluster equipped with eight NVIDIA A800 GPUs, each with 80GB of memory. During inference, we adjust the temperature to 0.8 for understanding tasks and to 0.5 for generation tasks, while other hyperparameters remain at their default settings to ensure optimal performance. 
Further details on the adopted curriculum learning strategy are provided in Appendix~\ref{appendix:explanation_curriculum_learning}.


\subsection{Understanding Evaluation}
\label{sec:understanding_evaluation}
For evaluating LLMs' capabilities in Verilog understanding, we utilize the benchmark introduced in Section~\ref{sec:understanding_benchmark}. The evaluation measures the similarity between the generated descriptions and the ground truth summaries. Previous works usually use BLEU~\citep{papineni2002bleu} and ROUGE~\citep{lin2004rouge} scores for this purpose~\citep{wang2023codet5+}. 
BLEU assesses how many $n$-grams, \textit{i.e.}, sequences of $n$ words, in the machine-generated text appear in the reference text (focusing on precision). In contrast, ROUGE counts how many $n$-grams from the reference appear in the generated text (focusing on recall). 
However, both metrics primarily capture lexical rather than semantic similarity, which may not fully reflect the accuracy of the generated descriptions.
To address this limitation, we take the initiative to apply embedding similarity and GPT score for evaluation.
% we propose two innovative metrics, embedding similarity and GPT score. 
Embedding similarity calculates the cosine similarity between vector representations of generated and ground truth descriptions, using embeddings derived from OpenAI's text-embedding-3-large model. Meanwhile, GPT score uses GPT-4 to quantify the semantic coherence between descriptions by assigning a similarity score from 0 to 1, where 1 indicates perfect semantic alignment.
These metrics provide a more nuanced evaluation by capturing the semantic essence of the descriptions, thus offering a more accurate assessment than previous methods.
For details on the prompt used to calculate the GPT score, please refer to Appendix~\ref{appendix:gpt_score}.

\begin{table}[ht]
    \centering
    \vspace{-10pt}
    \caption{Evaluation results on Verilog understanding using the benchmark proposed in Section~\ref{sec:understanding_benchmark}. BLEU-4 denotes the smoothed BLEU-4 score, and Emb. Sim. represents the embedding similarity metric. Best results are highlighted in bold.}
    \vspace{5pt}
    \label{tab:understanding_results}
    \small{
    \begin{tabular}{@{}l|ccccccc@{}}
    \toprule
       Model  & BLEU-4 & ROUGE-1 & ROUGE-2 & ROUGE-L & Emb. Sim. & GPT Score\\
    \midrule
       GPT-3.5 & 4.75 & 35.46 & 12.64 & 32.07 & 0.802 & 0.641 \\ 
       GPT-4 & 5.36 & 34.31 & 11.31 & 30.66 & 0.824 & 0.683 \\
       o1-preview & 6.06 & 34.27 & 12.25 & 31.01 & 0.806 & 0.643\\
    \midrule
       CodeT5+-220m & 0.28 & 7.10 & 0.34 & 6.18 & 0.313 & 0.032 \\ 
       CodeT5+-16b & 0.10 & 1.37 & 0.00 & 1.37 & 0.228 & 0.014 \\
       LLaMA2-70B-Chat & 2.86 & 28.15 & 10.09 & 26.12 & 0.633 & 0.500 \\ 
       DeepRTL-220m-direct & 11.99 & 40.05 & 20.56 & 37.09 & 0.793 & 0.572\\ 
       DeepRTL-16b-direct & 11.06 & 38.12 & 18.15 & 34.85 & 0.778 & 0.533 \\ 
    % \midrule
    %    CodeT5+-220m & \\
    %    CodeT5+-16b & \\
    %    LLaMA-70B-Chat & \\
    \midrule
       DeepRTL-220m & 18.66 & \textbf{47.69} & \textbf{29.49} & 44.02 & \textbf{0.837} & \textbf{0.705}\\
       DeepRTL-16b & \textbf{18.94} & 47.27 & 29.46 & \textbf{44.13} & 0.830 & 0.694\\
    \bottomrule
    \end{tabular}
    }
    \vspace{-10pt}
\end{table}



\subsection{Generation Evaluation}
To evaluate LLMs' capabilities in Verilog generation, we adopt the latest benchmark introduced by~\citet{chang2024natural}, which is an expansion based on the previous well-established RTLLM benchmark~\citep{lu2024rtllm}.
The benchmark by~\citet{chang2024natural} encompasses a broad spectrum of complexities across three categories: arithmetic, digital circuit logic, and advanced hardware designs.
This benchmark extends beyond previous efforts by incorporating a wider range of more challenging and practical Verilog designs, thus providing a more thorough assessment of the models' capabilities in generating Verilog code.

The evaluation focuses on two critical aspects: syntax correctness and functional accuracy. We use the open-source simulator iverilog~\citep{williams2002icarus} to assess both syntactic and functional correctness of Verilog code generated by LLMs. 
For the evaluation metric, we adopt the prevalent Pass@$k$ metric, which considers a problem solved if any of the $k$ generated code samples pass the compilation or functional tests~\citep{pei2024betterv}. For this study, we set $k$ values of 1 and 5, where a higher Pass@$k$ score indicates better model performance.
To further delineate the models' capabilities, we track the proportion of cases that pass out of 5 generated samples and compute the average as the success rate. For syntax correctness, this success rate measures the proportion of code samples that successfully compile and, for functional accuracy, the fraction that passes unit tests.


% evaluation benchmark
% the scope of the benchmark is also important
% comparing the results with the software programming language
% we use less data to achieve comparable performance of models trained with abundant software codes
% maybe we could state that the simple functional description is a more challenging case
% some works even do not compare the results with GPT-3.5/GPT-4, like Thakur's work
% we need to state that the evaluation benchmark is not overlapped with the training dataset
% we could mention that the previous software code understanding and generation tasks are relatively simple by showing that the maximum input and output lengths are relatively small for the software code model
% failure case analysis may also be interesting





\section{Experiment}
In this section, we conduct extensive experiments to evaluate the performance of various LLMs on our Hellaswag-Pro benchmark. Our study is guided by three key research questions:
\textbf{RQ1}: How do different LLMs perform across all variants?
\textbf{RQ2}: What is the relative difficulty of different variants?
\textbf{RQ3}: How robust are LLMs to diverse prompts during evaluation?

\subsection{Experiment Setup} 
\subsubsection{Model Selection and Implementation Details}
We select 41 representative commercial and open-source models, including English LLMs, such as GPT-4o, Claude-3.5-Sonnet, Gemini-1.5-Pro,Mistral series, Llama3 series and Chinese LLMs, like Qwen-Max,  Qwen2.5 series, InternLM-2.5 series, Yi-1.5 series, Baichuan-2 series and DeepSeek series.

We integrate both Chinese HellaSwag and HellaSwagPro into the lm-evaluation-harness platform. For the open-source models, we use the default settings of lm-evaluation-harness: do\_sample is set to false and the temperature is set to the default value of the hugging-face library. For the closed-source models, we set the temperature to 0.7. In addition, we set the maximum output length to 1024.

\subsubsection{Prompt Strategy}
Taking into account the influence of language and shot, we design 9 prompting strategies, including Direct, CN-CoT, EN-CoT, CN-XLT and EN-XLT. The last four setups include both zero-shot and few-shot variants.\footnote{
For open-source models, Direct adopts an approach similar to the official implementation of HellaSwag, computing the log-likelihood for each option and selecting the one with the highest log-likelihood. And we report normalized accuracy that accounts for the impact of option length. Other prompting strategies use a generation setup and report accuracy based on exact match.}
\textbf {(1)Direct}: LLMs makes the selection directly without any CoT process.
\textbf{(2)CN-CoT}: LLMs performs CoT in Chinese, regardless of dataset language.
\textbf{(3)EN-CoT}: Similar to CN-CoT, but CoT is conducted in English. 
\textbf{(4)CN-XLT}: LLMs are instructed to first translate English questions and options to Chinese, and then reason in Chinese.
\textbf{(5)EN-XLT}: Similar to CN-XLT, but translates from Chinese dataset to English and reasons in English. 

%\textbf {CN-CoT}: LLMs perform Chinese reasoning and then output the answer and 3 shots are provided.
%\textbf {CN-CoT}: Similar as CNCoTFewShot without any shots.
%\textbf {EN-CoT}: The reasoning process in English is executed and then the answer is output and 3 shots are provided.
%\textbf {CN-XLT}: Inspired by this, we instruct LLMs to translate questions in Chinese and then output the answer after performing reasoning in Chinese too. And 3 shots are provided.
%\textbf {EN-XLT}: Inspired by this, we instruct LLMs to translate questions in Englsih and then output the answer after performing reasoning in Englsih too. Three shots are provided.

\subsubsection{Evaluation metric}

To comprehensively evaluate the robustness of each LLM, we consider four metrics: 
% Original Accuracy (\textbf{OA}), Average Robust Accuracy (\textbf{ARA}), Robust Loss Accuracy (\textbf{RLA}), and  Consistent Robust Accuracy (\textbf{CRA}).
\noindent %
\textbf{- Original Accuracy (OA)} measures accuracy on original problems.
\begin{equation}\label{eq1}
OA=\frac{\sum_{(x, y) \in D} \mathds{1}[L M(x), y]}{|D|}.
\end{equation}
\noindent %
\textbf{- Average Robust Accuracy  (ARA)} represents average accuracy across all variants, gauging overall performance on the robustness tasks.
\begin{equation}\label{eq2}
ARA=\frac{\sum_{\left(x^{\prime}, y^{\prime}\right) \in D_{R}} \mathds{1}\left(L M\left(x^{\prime}, y^{\prime}\right)\right.}{\left|D_{R}\right|}.
\end{equation}

\noindent %
\textbf{- Robust Loss Accuracy (RLA)} is the difference between ARA and OA, indicating performance degradation on robustness data versus original data.
%\begin{tiny}
%\begin{equation}\label{eq3}
%RLA=\frac{\sum_{\left(x^{\prime}, y^{\prime}\right) \in D_{R}} %\mathds{1}\left(L M\left(x^{\prime}, y^{\prime}\right)\right.}{\left|D_{R}\right|}-\frac{\sum_{(x, y) \in D}\mathds{1}[L M(x), y]}{|D|}
%\end{equation}
%\end{tiny}
\begin{equation}\label{eq3}
RLA= OA - ARA.
\end{equation}
\noindent %
\textbf{- Consistent Robust Accuracy (CRA)} shows accuracy when the model correctly answers both original and variant data, reflecting the model do understand the problem.
% consistency in problem-solving.
\begin{equation}\label{eq4}
CRA=\frac{\sum_{x, y, x^{\prime}, y^{\prime}}\mathds{1}[L M(x), y] \cdot \mathds{1}[L M(x^{\prime}), y^{\prime}]}{\left|D_{R}\right|}.
\end{equation}
For all equation above, $D$ denotes the original dataset, where $x$ represents the input question and options, and $y$ represents the correct label, while $D_{R}$ is the robust dataset with $x^{\prime}$ and $y^{\prime}$ representing similar to $x$ and $y$.


\begin{table*}[ht]
\centering
\setlength{\tabcolsep}{5pt}
% \footnotesize
\scalebox{0.6}{
% Please add the following required packages to your document preamble:
% \usepackage{multirow}
% \usepackage[table,xcdraw]{xcolor}
% Beamer presentation requires \usepackage{colortbl} instead of \usepackage[table,xcdraw]{xcolor}
% Please add the following required packages to your document preamble:
% \usepackage{multirow}
% \usepackage[table,xcdraw]{xcolor}
% Beamer presentation requires \usepackage{colortbl} instead of \usepackage[table,xcdraw]{xcolor}
\begin{tabular}{ccccccccccccc}
\hline
\multicolumn{1}{c|}{{ }}& \multicolumn{4}{c|}{Chinese}& \multicolumn{4}{c|}{English}& \multicolumn{4}{c}{AVG}\\ \cline{2-13} 
\multicolumn{1}{c|}{\multirow{-2}{*}{{ Model}}} & { OA(\%)$\uparrow$}& { ARA(\%)$\uparrow$} & {RLA(\%)$\downarrow$}& \multicolumn{1}{l|}{{CRA(\%)$\uparrow$}} & { OA(\%)$\uparrow$}& { ARA(\%)$\uparrow$} & { RLA(\%)$\downarrow$}& \multicolumn{1}{l|}{{CRA(\%)$\uparrow$}} & {OA(\%)$\uparrow$}& { ARA(\%)$\uparrow$} & {RLA(\%)$\downarrow$}& { CRA(\%)$\uparrow$} \\ \hline
\multicolumn{1}{c|}{{ Human}} & 96.41& 97.79& -1.38 & \multicolumn{1}{l|}{92.03}& 95.56& 96.04& -0.48 & \multicolumn{1}{l|}{90.02}& 95.99 & 96.92 & -0.93& 91.03 \\ \hline
\multicolumn{13}{c}{\textit{Close-source LLMs}}\\ 
\multicolumn{1}{c|}{{ GPT-4o}}& { 91.37} & { 81.97} & { 9.40}& \multicolumn{1}{l|}{{ 75.55}} & { \textbf{88.63}} & { \textbf{70.17}} & { \textbf{18.46}} & \multicolumn{1}{l|}{{ \textbf{63.06}}} & { 90.00} & { \textbf{76.07}} & { \textbf{13.93}} & { \textbf{69.31}} \\
\multicolumn{1}{c|}{{ Claude3.5}}& { \textbf{95.37}} & { 80.15} & { 15.22} & \multicolumn{1}{l|}{{ 75.04}} & { 85.11} & { 66.02} & { 19.08} & \multicolumn{1}{l|}{{ 57.20}} & { 90.24} & { 73.09} & { 17.15} & { 66.12} \\
\multicolumn{1}{c|}{{ Gemini-1.5-Pro}}& { 90.62} & { 78.36} & { 12.26} & \multicolumn{1}{l|}{{ 70.48}} & { 87.75} & { 60.74} & { 27.01} & \multicolumn{1}{l|}{{ 58.27}} & { 89.19} & { 69.55} & { 19.63} & { 64.38} \\
\multicolumn{1}{c|}{{ Qwen-Max}}& { 93.50} & { \textbf{84.82}} & { \textbf{8.68}}& \multicolumn{1}{l|}{{ \textbf{78.91}}} & { 87.60} & { 62.61} & { 24.99} & \multicolumn{1}{l|}{{ 59.65}} & { \textbf{90.55}} & { 73.72} & { 16.83} & { 69.28} \\ \hline
\multicolumn{13}{c}{\textit{Chinese open-source LLMs}} \\ 
\multicolumn{1}{c|}{{ Qwen2.5-0.5B}}& { 60.75} & { 45.18} & { \textbf{15.57}} & \multicolumn{1}{l|}{{ 28.70}} & { 49.50} & { 38.21} & { \textbf{11.29}} & \multicolumn{1}{l|}{{ 20.57}} & { 55.13} & { 41.70} & { \textbf{13.43}} & { 24.64} \\
\multicolumn{1}{c|}{{ Qwen2.5-1.5B}}& { 63.25} & { 46.16} & { 17.09} & \multicolumn{1}{l|}{{ 29.89}} & { 56.88} & { 39.57} & { 17.30} & \multicolumn{1}{l|}{{ 23.48}} & { 60.06} & { 42.87} & { 17.20} & { 26.69} \\
\multicolumn{1}{c|}{{ Qwen2.5-3B}}& { 67.50} & { 48.75} & { 18.75} & \multicolumn{1}{l|}{{ 33.79}} & { 61.75} & { 39.98} & { 21.77} & \multicolumn{1}{l|}{{ 25.75}} & { 64.63} & { 44.37} & { 20.26} & { 29.77} \\
\multicolumn{1}{c|}{{ Qwen2.5-7B}}& { 67.63} & { 50.59} & { 17.04} & \multicolumn{1}{l|}{{ 35.62}} & { 65.63} & { 43.93} & { 21.70} & \multicolumn{1}{l|}{{ 30.77}} & { 66.63} & { 47.26} & { 19.37} & { 33.20} \\
\multicolumn{1}{c|}{{ Qwen2.5-14B}} & { 69.00} & { 51.41} & { 17.59} & \multicolumn{1}{l|}{{ 35.84}} & { 68.50} & { 45.20} & { 23.30} & \multicolumn{1}{l|}{{ 32.12}} & { 68.75} & { 48.30} & { 20.45} & { 33.98} \\
\multicolumn{1}{c|}{{ Qwen2.5-32B}} & { 69.75} & { 53.11} & { 16.64} & \multicolumn{1}{l|}{{ 37.54}} & { 70.00} & { 46.10} & { 23.90} & \multicolumn{1}{l|}{{ 32.68}} & { 69.88} & { 49.61} & { 20.27} & { 35.11} \\
\multicolumn{1}{c|}{{ Qwen2.5-72B}} & { \textbf{70.87}} & { \textbf{54.75}} & { 16.12} & \multicolumn{1}{l|}{{ \textbf{39.64}}} & { \textbf{72.00}} & { \textbf{47.75}} & { 24.25} & \multicolumn{1}{l|}{{\textbf{ 35.12}}} & { \textbf{71.44}} & { \textbf{51.25}} & {20.19} & { \textbf{37.38}} \\ \hdashline[0.5pt/5pt]
\multicolumn{1}{c|}{{ Baichuan2-7B}}& { 67.00} & { 46.16} & { 20.84} & \multicolumn{1}{l|}{{ 31.50}} & { 60.62} & { 39.04} & { 21.58} & \multicolumn{1}{l|}{{ 25.21}} & { 63.81} & { 42.60} & { 21.21} & { 28.36} \\
\multicolumn{1}{c|}{{ Baichua2-13B}}& { 69.13} & { 46.98} & { 22.15} & \multicolumn{1}{l|}{{ 33.45}} & { 64.62} & { 38.82} & { 25.80} & \multicolumn{1}{l|}{{ 26.07}} & { 66.88} & { 42.90} & { 23.97} & { 29.76} \\ \hdashline[0.5pt/5pt]
\multicolumn{1}{c|}{{ DeepSeek-7B}} & { 68.13} & { 47.96} & { 20.17} & \multicolumn{1}{l|}{{ 33.30}} & { 63.38} & { 40.39} & { 22.99} & \multicolumn{1}{l|}{{ 26.70}} & { 65.76} & { 44.18} & { 21.58} & { 30.00} \\
\multicolumn{1}{c|}{{ DeepSeek-67B}}& { 71.50} & { 49.21} & { 22.29} & \multicolumn{1}{l|}{{ 35.89}} & { 71.37} & { 40.63} & { 30.75} & \multicolumn{1}{l|}{{ 29.71}} & { 71.44} & { 44.92} & { 26.52} & { 32.80} \\ \hdashline[0.5pt/5pt]
\multicolumn{1}{c|}{{ InternLM2.5-1.8B}}& { 61.62} & { 42.07} & { 19.55} & \multicolumn{1}{l|}{{ 26.99}} & { 55.37} & { 38.46} & { 16.91} & \multicolumn{1}{l|}{{ 22.61}} & { 58.50} & { 40.27} & { 18.23} & { 24.80} \\
\multicolumn{1}{c|}{{ InternLM2.5-7B}}& { 67.25} & { 49.77} & { 17.48} & \multicolumn{1}{l|}{{ 34.57}} & { 69.50} & { 40.89} & { 28.61} & \multicolumn{1}{l|}{{ 29.75}} & { 68.38} & { 45.33} & { 23.04} & { 32.16} \\
\multicolumn{1}{c|}{{ InternLM2.5-20B}} & { 67.37} & { 48.08} & { 19.29} & \multicolumn{1}{l|}{{ 33.21}} & { 73.62} & { 41.11} & { 32.51} & \multicolumn{1}{l|}{{ 31.23}} & { 70.50} & { 44.60} & { 25.90} & { 32.22} \\ \hdashline[0.5pt/5pt]
\multicolumn{1}{c|}{{ Yi-1.5-6B}} & { 67.00} & { 49.59} & { 17.41} & \multicolumn{1}{l|}{{ 34.27}} & { 64.38} & { 39.37} & { 25.01} & \multicolumn{1}{l|}{{ 26.62}} & { 65.69} & { 44.48} & { 21.21} & { 30.45} \\
\multicolumn{1}{c|}{{ Yi-1.5-9B}} & { 68.50} & { 50.18} & { 18.32} & \multicolumn{1}{l|}{{ 35.55}} & { 66.37} & { 39.58} & { 26.79} & \multicolumn{1}{l|}{{ 27.48}} & { 67.44} & { 44.88} & { 22.56} & { 31.52} \\
\multicolumn{1}{c|}{{ Yi-1.5-34B}}& { 71.00} & { 52.23} & { 18.77} & \multicolumn{1}{l|}{{ 38.09}} & { 71.00} & { 40.75} & { 30.25} & \multicolumn{1}{l|}{{ 29.91}} & { 71.00} & { 46.49} & { 24.51} & { 34.00} \\ \hline
\multicolumn{13}{c}{\textit{English open-source LLMs}} \\ 
\multicolumn{1}{c|}{{ Llama3-8B}} & { 59.13} & { 46.62} & { 12.51} & \multicolumn{1}{l|}{{ 28.23}} & { 66.25} & { 40.21} & { 26.04} & \multicolumn{1}{l|}{{ 27.34}} & { 62.69} & { 43.42} & { 19.27} & { 27.79} \\
\multicolumn{1}{c|}{{ Llama3-70B}}& { 65.75} & { 48.63} & { 17.12} & \multicolumn{1}{l|}{{ 32.70}} & { \textbf{72.50}} & { 41.27} & { 31.23} & \multicolumn{1}{l|}{{\textbf{ 30.63}}} & {\textbf{ 69.13}} & { 44.95} & { 24.18} & { 31.67} \\ \hdashline[0.5pt/5pt]
\multicolumn{1}{c|}{{ Mistral-7B-v0.2}} & { 57.75} & { 46.25} & { \textbf{11.50}} & \multicolumn{1}{l|}{{ 27.57}} & { 67.50} & { \textbf{41.52}} & { 25.98} & \multicolumn{1}{l|}{{ 28.93}} & { 62.63} & { 43.88} & { 18.74} & { 28.25} \\
\multicolumn{1}{c|}{{ Mixtral-8x7B-v0.1}} & { 63.62} & { 46.80} & { 16.82} & \multicolumn{1}{l|}{{ 30.82}} & { 69.75} & { 41.21} & { 28.54} & \multicolumn{1}{l|}{{ 29.39}} & { 66.69} & { 44.01} & { 22.68} & { 30.11} \\
\multicolumn{1}{c|}{{ Mixtral-8x22B-v0.1}}& { 66.00} & {\textbf{ 50.73}} & { 15.27} & \multicolumn{1}{l|}{{ \textbf{34.32}}} & { 72.12} & { 41.25} & { 30.87} & \multicolumn{1}{l|}{{ 30.61}} & { 69.06} & { \textbf{45.99}} & { 23.07} & { \textbf{32.47}} \\ \hdashline[0.5pt/5pt]
\multicolumn{1}{c|}{{ Gemma-2-2B}}& { 61.88} & { 45.38} & { 16.51} & \multicolumn{1}{l|}{{ 29.02}} & { 59.62} & { 39.13} & { \textbf{20.50}} & \multicolumn{1}{l|}{{ 24.88}} & { 60.75} & { 42.25} & {\textbf{ 18.50}} & { 26.95} \\
\multicolumn{1}{c|}{{ Gemma-2-9B}}& { \textbf{69.13}} & { 46.75} & { 22.38} & \multicolumn{1}{l|}{{ 33.29}} & { 64.88} & { 39.80} & { 25.08} & \multicolumn{1}{l|}{{ 26.91}} & { 67.01} & { 43.28} & { 23.73} & { 30.10} \\
\multicolumn{1}{c|}{{ Gemma-2-27B}} & { 63.38} & { 48.52} & { 14.86} & \multicolumn{1}{l|}{{ 31.96}} & { 71.88} & { 40.91} & { 30.97} & \multicolumn{1}{l|}{{ 30.25}} & { 67.63} & { 44.71} & { 22.92} & { 31.11} \\ \hline
\end{tabular}
}
\caption{TODO: bolded is not result. Results of existing LLMs on our HellaSwag-Pro dataset using \textbf{Direct} prompt. ``AVG'' indicates the average performance of each model on Chinese and English parts of the dataset.
The best results for each metric in each model category are \textbf{bolded}. }
\label{tab:main experiment.}
\end{table*}

\subsection{Model Performance (RQ1)}
\paragraph{Overall Performance}
Table \ref{tab:main experiment.} provides a comprehensive evaluation of various LLMs across four performance metrics\footnote{The results of instruct/chat models of Qwen2.5, Llama3 and Mixtral latest series are shown in Appendix.}. The main observations are as follow:
\begin{itemize}[leftmargin=*,topsep=0pt]
% \setlength{}{0}
    \item Upon evaluating all available models, we found that all performed well in overall accuracy (e.g., GPT-4 scored 90.00 in AVG OA, Claude 3.5 scored 90.24 in AVG OA). However, all models struggled with variations of the questions, as evidenced by a positive RLA value for each model. In contrast, humans received a negative RLA value, suggesting that the question variants were not more challenging than the originals. This disparity further illustrates that current LLMs lack a true understanding of the reasoning process and can easily be misled by question variants.
    \item When comparing open-source and close-source models, the close-source models demonstrate stronger capabilities in both OA and ARA scores, similar to most existing benchmarks. Overall, the RLA values for close-source models are also smaller, indicating that they are more robust in commonsense reasoning tasks compared to open-source models.
    \item When we compare models within the same series (e.g., Qwen, Llama), we observe that larger models often achieve higher scores on OA, ARA, and CRA. However, they are also more susceptible to variations, i.e., they have higher RLA values, a phenomenon particularly evident in English datasets. We attribute this phenomenon to the fact that larger models, compared to smaller ones, may have memorized more data, allowing them to rely on memorization to solve some problems more easily and making them more prone to the influence of variations~\cite{}.
\end{itemize}
% 1. When evaluating all available models, We find although 
% 2. When comparing the opensource LLMs and close source LLMs, 
% 3. When looking into each serious details
% \noindent
% \textbf{Overall Model Performance.}
% 1. close-source > open-source 2. the large the better 3. all have a performance decline when meeting varients.

% To evaluate the performance of various models, we observed patterns consistent with current mainstream trends: closed-source models generally outperform open-source models across metrics. 
% For instance, the closed-source model GPT-4o achieved scores of 90.00 in OA, 76.07 in ARA, and 69.31 in CRA, whereas the open-source model Qwen2.5-72B scored 71.44, 51.25, and 37.38, respectively. 
% Furthermore, within each model series, performance tends to improve with larger model sizes. 
% Nevertheless, even the strongest closed-source models struggle with variations in questions, as indicated by positive values in RLA for all models. In contrast, human performance yields a negative RLA value, highlighting that current LLMs do not genuinely grasp the reasoning process and are prone to falling into traps set by question variants. 
% This suggests that there is still significant room for improvement in developing models that can robustly understand and reason through complex linguistic challenges.
% It reveals a consistent pattern across Chinese, English, and average scores, with close-sourced LLMs generally outperforming open-sourced models. 
% However, all models exhibit a significant drop in performance when faced with robust variants, as indicated by RLA and CRA. Among closed-source models, GPT-4o demonstrates the highest ARA of 76.07\% in average scores, demonstrating its overwhelming superiority. Among open-sourced models, larger models tend to perform better, with Qwen2.5-72B achieving the highest OA (71.44\%) and ARA (51.25\%) in the average scores. However, even these top performers still struggle with robustness, as evidenced by the substantial RLA of 13.93\% for GPT-4o and 20.19\% for Qwen2.5-72B. Interestingly, some English open-sourced models, such as Llama3-70B and Mixtral-8x22B-v0.1, show competitive performance in English tasks but lag in Chinese tasks, highlighting the importance of language-specific training.

% \noindent
% \textbf{Chinese Models vs English Models.}
% Chinese models generally demonstrate higher OA in Chinese tasks compared to English tasks, with Qwen-Max achieving 93.50\% OA in Chinese versus 87.60\% in English. Conversely, English models tend to perform better in English tasks, exemplified by Llama3-70B's 72.50\% OA in English compared to 65.75\% in Chinese. 
% However, both Chinese and English models exhibit important drops in ARA across languages, indicating challenges in maintaining performance when faced with variations. This trend suggests that while models may excel in their primary language, they struggle with robustness across linguistic boundaries. 
% Notably, larger models tend to achieve higher ARA scores but also experience more substantial RLA, as seen with Qwen2.5-0.5B (41.70\% ARA, 13.43\% RLA in total) and Qwen2.5-72B (51.25\% ARA, 20.19\% RLA in total). 
% This pattern indicates that while increased model size enhances overall performance, it doesn't necessarily improve robustness proportionally. 
% The discrepancy between OA and ARA across languages underscores the need for improved cross-lingual robustness in language models, particularly as they scale in size and capability.


% \noindent
% \textbf{Comparison between Chinese and English datasets.}
% Generally, models demonstrate higher accuracy on the Chinese dataset compared to the English one, as evidenced by the consistently higher OA, ARA and CRA scores. For instance, GPT-4o achieves an OA of 91.37\%, an ARA of 81.97\% , an CRA of 75.55\% on the Chinese dataset, compared to 88.63\% and 70.17\% respectively on the English dataset. This trend is observed across most models, suggesting that the Chinese dataset is easier than English one. Moreover, the RLA values are typically lower for Chinese, indicating smaller performance drops when dealing with robust variants of Chinese questions. For example, Qwen-Max shows an RLA of 8.68\% for Chinese versus 24.99\% for English, highlighting a more consistent performance in Chinese. The CRA scores further reinforce this observation, with models generally maintaining higher consistency in correct answers for both original and variant Chinese questions.
% We attribute this phenomenon to the fact that blablabla

\noindent
\textbf{Reasoning Transferable Capability.}
% 为了进一步
To further analyze whether the model can transfer reasoning ability from the original question to its variant, Figure \ref{consis} presents the distribution of model performance on the original question and variant pairs. For all models, the pairs of (HellaSwag \ding{51} HellaSwag-Pro \ding{55}) occupy a significant proportion, indicating a challenge in transferring reasoning capabilities for current LLMs to more complex scenarios. Looking deeply, closed-source models like GPT-4 and Qwen-Max achieve around a 69\% portion of (HellaSwag \ding{51} HellaSwag-Pro \ding{51}) and a 3\% portion of (HellaSwag \ding{55} HellaSwag-Pro \ding{55}), while in contrast, open-source models struggle with around a 30\% portion of (HellaSwag \ding{51} HellaSwag-Pro \ding{51}) and a 20\% portion of (HellaSwag \ding{55} HellaSwag-Pro \ding{55}), further indicating the robustness of reasoning abilities in closed-source models.
% If a model can get both the original question and the variant right, we consider it to have transferable reasoning ability. Table \ref{consis} presents the distribution of model performance on the original question and variant pairs. Among all models, the pairs of (HellaSwag \ding{51}HellaSwag-Pro \ding{55}) account for a considerable proportion, i 
% The closed-source models like GPT-4o and Qwen-Max achieve around 69\% portion of (HellaSwag \ding{51}HellaSwag-Pro \ding{51}) and 3\% portion of (HellaSwag \ding{55} HellaSwag-Pro \ding{55}), indicating stronger reasoning transfer ability than other models. In contrast, open-source models struggle more, with around 30\% portion of (HellaSwag \ding{51}HellaSwag-Pro \ding{51}) and 20\% portion of (HellaSwag \ding{55} HellaSwag-Pro \ding{55}). 
% A notable trend is observed among the Qwen2.5 series, where increasing model size from 7B to 72B parameters correlates with improved performance on correct answers for both datasets (33.20\% to 37.38\%) and decreased failure rates (17.69\% to 14.7\%). It underscores the importance of model size in commonsense reasoning tasks.

\begin{figure}[t]
\centering
\setlength{\abovecaptionskip}{0.1cm}
\setlength{\belowcaptionskip}{0cm}
\includegraphics[width=\linewidth,scale=1.00]{images/consis.pdf}
\caption{Analysis of the transferable ability of model reasoning based on question pair performance. The green part, where both the original and the variant data are right, represents the transferable performance of model reasoning.}
\label{consis}
\vspace{-15pt}
\end{figure}

\begin{figure*}[ht]
\centering
\setlength{\abovecaptionskip}{0.1cm}
\setlength{\belowcaptionskip}{0cm}
\includegraphics[width=\linewidth,scale=1.00]{images/xing.pdf}
\caption{The impact of different few-shot prompts on model performance. With - as the separator, the first two parts of the legend represent the prompt name, and the third part represents the language of the dataset.}
\label{xing}
\vspace{-15pt}
\end{figure*}

\begin{figure}[ht]
\centering
\setlength{\abovecaptionskip}{0.1cm}
\setlength{\belowcaptionskip}{0cm}
\includegraphics[width=1.05\linewidth,scale=1.05]{images/zhu.pdf}
\caption{The RLA Distribution for 7 variants of commonsense reasoning. Parts below the 0 axis indicate that the model’s performance on the variant is improved compared to the original problem.}
\label{fig:zhu}
\vspace{-15pt}
\end{figure}


\subsection{Variant Analysis (RQ2)}
To further analyze the impact of different variants, we assessed the contribution of each variant to the RLA score. A higher contribution indicates that the model is more likely to make errors in that type. Figure~\ref{fig:zhu} presents the overall results, and the key observations are as follows:
\begin{itemize}[leftmargin=*]
    \item For problem restatement, causal inference, and sentence ordering, these three categories are the least challenging. Almost all models, particularly the close-source and Qwen series models, perform well on these variants, indicating that current LLMs can effectively handle these forms and we do not pay more attention on this kind of varients.
    \item For reverse conversion and critical testing, these two varients each contribute about 10\% to the RLA score. This indicates that current LLMs struggle to fully generalize to these simple scenarios, possibly because these types of questions are not commonly encountered, and reaserchers should pay some attention to this type of varients.
    \item For negative transformation and scenario refinement, this are the two most difficult tasks, with negative transformation being particularly challenging. For almost all models, these two varients accounts for more than 50\% of the RLA score. This may be due to intuitively counterintuitive questions—such as the use of "will not"  or counterfactual scenarios in scenario refinement. These setups are less common in LLM training data and cannot be easily tackled through memory alone. Only those LLMs which truely understand the question could answer the varient correctly, wihch better reflect the true performance of the model.. In the future, researchers should focus more on enhancing LLM's capability to address such types of questions.
\end{itemize}

% 1. Problem restCausal Inference 
% To further analysis the impact of different varients, we further 
% Figure \ref{fig: zhu} presents a comprehensive analysis of various LLMs' performance across different variant types. Negative transformation emerges as the most challenging task for all models, with scores consistently above 50.00\% and peaking at 78.38\% for Gemini-1.5-Pro. Conversely, problem restatement appears to be the least challenging, with most models scoring in the negative range. Intriguingly, smaller models like Qwen2.5-0.5B demonstrate unexpected strengths in certain areas, such as sentence sorting (7.75\%), outperforming some larger counterparts. A detailed analysis of each variant type follows.

% \noindent
% \textbf{Causal inference.} In this category, scores vary widely from -4.73\% for Qwen-Max to 12.25\% for Baichuan2-13B, illustrating differing degrees of sensitivity to causal reasoning among the models. Smaller models, such as Qwen2.5-0.5B and Qwen2.5-1.5B, achieve better scores, indicating relatively stronger robustness in causal reasoning. Conversely, larger models, like Baichuan2-13B, have higher scores, suggesting greater sensitivity to the challenges of inferring causality.

% \noindent
% \textbf{Critical testing.} Larger models, including Qwen2.5-72B and DeepSeek-67B, exhibit higher RLA scores of 30.50\% and 31.37\%, respectively, suggesting increased sensitivity when dealing with incomplete key information. In contrast, GPT-4o achieves the lowest score, highlighting its superior robustness in critical reasoning. This trend indicates that more complex models might struggle to handle incomplete contexts, underscoring potential areas for improvement in sophisticated architectures.

% \noindent
% \textbf{Negative transformation.} This aspect remains consistently challenging for all models, with scores ranging from 48.88\% to 78.38\%. Advanced commercial models like Gemini-1.5-Pro and Claude-3.5 also score higher (78.38\% and 76.43\%, respectively), indicating a prevalent sensitivity issue in reasoning processes when handling negations, irrespective of model size or architecture.

% \noindent
% \textbf{Problem restatement.} The negative values in this category for nearly all models suggest it is not particularly challenging. This is surprising, given that previous models were quite sensitive to sentence representation.

% \noindent
% \textbf{Reverse conversion.} This variation, which involves swapping the roles of the question and answer, seems to specifically impact larger models. For example, Qwen2.5-72B and DeepSeek-67B exhibit higher RLA scores of 24.38\% and 27.43\%, respectively, indicating heightened sensitivity to reverse reasoning compared to their performance on original questions.

% \noindent
% \textbf{Scenario refinement.} The scores range from 16.06\% for Gemma-2-2B to 32.56\% for Qwen2.5-72B, with larger models displaying more sensitivity in adapting to counterfactual predictions. This suggests that larger models may rely more heavily on general commonsense rather than flexibly adapting to specific contexts. Consequently, increased model complexity might adversely affect adaptability to scenario changes, underscoring the need for enhanced flexibility in advanced models.

% \noindent
% \textbf{Sentence sorting.} This category exhibits the most varied results across models. Some larger models like DeepSeek-67B and InternLM2.5-20B display higher scores (26.69\% and 26.68\%), indicating sensitivity, while others like Qwen2.5-72B and Gemini-1.5-Pro excel with lower scores (-9.88\% and -1.07\%, respectively). This suggests that sentence sorting ability may depend more on specific training approaches rather than being solely contingent on model size.


\subsection{Prompt Robustness (RQ3)}
% To investigate how prompt  influence our benchmark, we apply sereral prompt strategy on our datasets and showcase the average performance of all models on different kind of prompt strategies.
% Table~\ref{prompt} illustrates the final results. For both Chinese and English datasets, CN LLMs achieve the highest performance using CN-CoT-Few-Shot, followed closely by EN-CoT-Few-Shot, with overall performance scores of 67.36\% and 67.03\%, respectively. In contrast, English LLMs perform best with the EN-CoT-Few-Shot, reaching 67.55\% on the Chinese dataset and 60.36\% on the English dataset.
% Contrary to previous findings, translating the dataset to the model's advantage language before performing reasoning does not enhance performance. Moreover, Figure~\ref{xing} also shows the similar phenomenon. Conducting CoT reasoning in the model’s advantage language generally leads to better outcomes compared to Direct. Additionally, increasing the number of shots consistently improves performance across most configurations, highlighting the benefits of exposing models to multiple examples. 
To explore the impact of various prompt strategies on our benchmarks, we evaluated several approaches across our datasets and present the average performance of all models using different prompting techniques. Table~\ref{prompt} summarizes the results. For both Chinese and English datasets, Chinese LLMs performed best with the CN-CoT-Few-Shot strategy, followed closely by EN-CoT-Few-Shot, achieving overall scores of 67.36\% and 67.03\%, respectively. Conversely, English LLMs showed optimal performance with the EN-CoT-Few-Shot approach, attaining 67.55\% on the Chinese dataset and 60.36\% on the English dataset.
Besides, translating datasets into the model's native language before reasoning did not enhance performance. This phenomenon is further illustrated in Figure~\ref{xing}. Conducting CoT reasoning in the model's native language generally yields better results compared to direct reasoning. Furthermore, increasing the number of examples (shots) consistently boosts performance across most configurations, emphasizing the advantages of exposing models to multiple examples.
% Overall, the interaction between question language, prompt language, and the number of shots underscores the importance of aligning these factors to optimize task performance and robustness in LLMs.



% Please add the following required packages to your document preamble:
% \usepackage{multirow}
% Please add the following required packages to your document preamble:
% \usepackage{multirow}
\begin{table}[t]
\setlength{\tabcolsep}{8pt}
% \footnotesize
\scalebox{0.65}{
\begin{tabular}{c|l|lll}
\hline
\multicolumn{1}{l|}{Dataset}  & Prompt  & CN LLMs & EN LLMs &  LLMs \\ \hline
\multirow{7}{*}{\begin{tabular}[c]{@{}c@{}}Chinese\\ HellaSwag-Pro\end{tabular}} & Direct  & 48.95& 41.16& 45.06  \\
& CN-CoT-Few  & \textbf{71.04}& 51.90& 61.47  \\
& EN-CoT-Few  & 70.95& \textbf{67.55}& \textbf{69.25}  \\
& EN-XLT-Few  & 41.48& 28.69& 35.09  \\
& CN-CoT-Zero & 44.82& 23.89& 34.36  \\
& EN-CoT-Zero & 45.38& 31.39& 38.39  \\
& EN-XLT-Zero & 28.57& 12.93& 20.75  \\ \hline
\multirow{7}{*}{\begin{tabular}[c]{@{}c@{}}English\\ HellaSwag-Pro\end{tabular}} & Direct  & 47.46& 40.66& 44.06  \\
& CN-CoT-Few  & \textbf{63.67}& 47.24& 55.46  \\
& EN-CoT-Few  & 63.12& \textbf{60.36}& \textbf{61.74}  \\
& CN-XLT-Few  & 48.77& 16.61& 32.69  \\
& CN-CoT-Zero & 34.89& 18.25& 26.57  \\
& EN-CoT-Zero & 42.41& 31.03& 36.72  \\
& CN-XLT-Zero & 16.36& 11.22& 13.79  \\ \hline
\multirow{9}{*}{HellaSwag-Pro}& Direct  & 48.21& 40.91& 44.83  \\
& CN-CoT-Few  & \textbf{67.36}& 49.57& 58.46  \\
& EN-CoT-Few  & 67.03& \textbf{63.95}& \textbf{65.49}  \\
& CN-XLT-Few  & 59.91& 34.26& 47.08  \\
& EN-XLT-Few  & 52.30& 44.52& 48.41  \\
& CN-CoT-Zero & 39.86& 21.07& 30.46  \\
& EN-CoT-Zero & 43.90& 31.21& 37.55  \\
& CN-XLT-Zero & 30.59& 17.55& 24.07  \\
& EN-XLT-Zero & 35.49& 21.98& 28.74  \\ \hline
\end{tabular}
}
\caption{Average ARA of all open-source models on different prompts. CN-LLMs contains 17 LLMs, and EN-LLMs contains 7 LLMs. The bast results for each dataset are \textbf{bolded}.}
\label{prompt}
\end{table}




\section{Concluding Remarks}
In this paper, we proposed a novel approach utilizing multimodal LLMs to generate gesture-aware speech recognition transcripts for patients with language disorders. Our framework integrates verbal speech and iconic gestures, enabling the generation of enriched transcripts that capture the latent meaning conveyed through both modalities. Through extensive experimentation, we demonstrated that the proposed method effectively contextualizes incomplete or disfluent speech by incorporating gesture information, leading to more accurate and meaningful representations of the speaker's intent. These findings highlight the potential of our approach to significantly contribute to the field of speech and language therapy, offering innovative tools that can enhance the quality of life for individuals with language disorders by facilitating better communication and assessment methods.

\subsection{Ethical Statement} 
Our dataset was obtained from AphasiaBank with the approval of the Institutional Review Board (IRB) and adheres to the data sharing guidelines set by TalkBank\footnote{https://talkbank.org/share/ethics.html}. This includes complying with the Ground Rules for all TalkBank databases, which are based on the American Psychological Association Code of Ethics~\cite{american2002ethical}.

\subsection{Limitation \& Future Work} 
%This study represents a preliminary investigation into using multimodal LLMs to generate gesture-aware speech recognition transcripts. 
While the results are promising, we recognize several limitations and outline our plans to extend this work further.

One primary limitation is the absence of a definitive ground truth for quantitative evaluation. Since our model generates transcripts by synthesizing speech and gesture data from scratch, traditional benchmarks, such as comparisons with standard speech recognition outputs, are insufficient. Moreover, existing original transcripts lack gesture annotations, making direct comparisons challenging. In future work, we aim to address this gap by collaborating with certified pathologists to conduct qualitative assessments, such as A-B preference tests, to evaluate the effectiveness of gesture-enriched transcripts in accurately conveying the speaker's intentions.

To support quantitative evaluations, we plan to develop novel metrics that assess transcript quality, including grammar accuracy, semantic consistency, and the integration of multimodal information. Such metrics will provide a more objective basis for assessing our model's performance and facilitate comparisons with other multimodal and unimodal approaches.

Another limitation of this study is its focus on structured gestures from a specific task, the Peanut Butter Sandwich Task. While this task offers a controlled context for testing our approach, it does not encompass the diversity of gestures and communication patterns seen in everyday scenarios. As part of our future work, we plan to expand the scope of our model to include tasks such as the Cinderella Story Recall Task~\cite{bird1996cinderella}, which involves unstructured and complex narrative gestures. This expansion will allow us to evaluate the adaptability and robustness of our model in handling varied linguistic and gestural contexts.

In summary, while this study establishes a strong foundation for gesture-aware speech recognition, we aim to refine and extend our methods through collaborative qualitative evaluations, the development of robust quantitative metrics, and broader task applications. These efforts will ensure that our approach continues to evolve, ultimately contributing to more effective communication tools and interventions for individuals with language disorders.




\newpage

% \subsubsection*{Author Contributions}
% If you'd like to, you may include  a section for author contributions as is done
% in many journals. This is optional and at the discretion of the authors.

\section*{Acknowledgments}
% Use unnumbered third level headings for the acknowledgments. All
% acknowledgments, including those to funding agencies, go at the end of the paper.
This work was supported in part by the General Research Fund of the Hong Kong Research Grants Council (RGC) under Grant No. 14212422 and 14202824, and in part by National Technology Innovation Center for EDA.

% \bibliography{iclr2025_conference}
% \bibliographystyle{iclr2025_conference}

\begin{thebibliography}{51}
\providecommand{\natexlab}[1]{#1}
\providecommand{\url}[1]{\texttt{#1}}
\expandafter\ifx\csname urlstyle\endcsname\relax
  \providecommand{\doi}[1]{doi: #1}\else
  \providecommand{\doi}{doi: \begingroup \urlstyle{rm}\Url}\fi

\bibitem[Achiam et~al.(2023)Achiam, Adler, Agarwal, Ahmad, Akkaya, Aleman, Almeida, Altenschmidt, Altman, Anadkat, et~al.]{achiam2023gpt}
Josh Achiam, Steven Adler, Sandhini Agarwal, Lama Ahmad, Ilge Akkaya, Florencia~Leoni Aleman, Diogo Almeida, Janko Altenschmidt, Sam Altman, Shyamal Anadkat, et~al.
\newblock Gpt-4 technical report.
\newblock \emph{arXiv preprint arXiv:2303.08774}, 2023.

\bibitem[Bengio et~al.(2009)Bengio, Louradour, Collobert, and Weston]{bengio2009curriculum}
Yoshua Bengio, J{\'e}r{\^o}me Louradour, Ronan Collobert, and Jason Weston.
\newblock Curriculum learning.
\newblock In \emph{Proceedings of the 26th annual international conference on machine learning}, pp.\  41--48, 2009.

\bibitem[Blocklove et~al.(2023)Blocklove, Garg, Karri, and Pearce]{blocklove2023chip}
Jason Blocklove, Siddharth Garg, Ramesh Karri, and Hammond Pearce.
\newblock Chip-chat: Challenges and opportunities in conversational hardware design.
\newblock In \emph{2023 ACM/IEEE 5th Workshop on Machine Learning for CAD (MLCAD)}, pp.\  1--6. IEEE, 2023.

\bibitem[Campos(2021)]{campos2021curriculum}
Daniel Campos.
\newblock Curriculum learning for language modeling.
\newblock \emph{arXiv preprint arXiv:2108.02170}, 2021.

\bibitem[Chang et~al.(2023)Chang, Wang, Ren, Wang, Liang, Han, Li, and Li]{chang2023chipgpt}
Kaiyan Chang, Ying Wang, Haimeng Ren, Mengdi Wang, Shengwen Liang, Yinhe Han, Huawei Li, and Xiaowei Li.
\newblock Chipgpt: How far are we from natural language hardware design.
\newblock \emph{arXiv preprint arXiv:2305.14019}, 2023.

\bibitem[Chang et~al.(2024{\natexlab{a}})Chang, Chen, Zhou, Zhu, Xu, Li, Wang, Liang, Li, Han, et~al.]{chang2024natural}
Kaiyan Chang, Zhirong Chen, Yunhao Zhou, Wenlong Zhu, Haobo Xu, Cangyuan Li, Mengdi Wang, Shengwen Liang, Huawei Li, Yinhe Han, et~al.
\newblock Natural language is not enough: Benchmarking multi-modal generative ai for verilog generation.
\newblock \emph{arXiv preprint arXiv:2407.08473}, 2024{\natexlab{a}}.

\bibitem[Chang et~al.(2024{\natexlab{b}})Chang, Wang, Yang, Wang, Jin, Zhu, Chen, Li, Yan, Zhou, et~al.]{chang2024data}
Kaiyan Chang, Kun Wang, Nan Yang, Ying Wang, Dantong Jin, Wenlong Zhu, Zhirong Chen, Cangyuan Li, Hao Yan, Yunhao Zhou, et~al.
\newblock Data is all you need: Finetuning llms for chip design via an automated design-data augmentation framework.
\newblock \emph{arXiv preprint arXiv:2403.11202}, 2024{\natexlab{b}}.

\bibitem[Chen et~al.(2024)Chen, Chen, Chu, Fang, Ho, Huang, Khan, Li, Li, Liang, et~al.]{chen2024dawn}
Lei Chen, Yiqi Chen, Zhufei Chu, Wenji Fang, Tsung-Yi Ho, Yu~Huang, Sadaf Khan, Min Li, Xingquan Li, Yun Liang, et~al.
\newblock The dawn of ai-native eda: Promises and challenges of large circuit models.
\newblock \emph{arXiv preprint arXiv:2403.07257}, 2024.

\bibitem[Dubey et~al.(2024)Dubey, Jauhri, Pandey, Kadian, Al-Dahle, Letman, Mathur, Schelten, Yang, Fan, et~al.]{dubey2024llama}
Abhimanyu Dubey, Abhinav Jauhri, Abhinav Pandey, Abhishek Kadian, Ahmad Al-Dahle, Aiesha Letman, Akhil Mathur, Alan Schelten, Amy Yang, Angela Fan, et~al.
\newblock The llama 3 herd of models.
\newblock \emph{arXiv preprint arXiv:2407.21783}, 2024.

\bibitem[Fu et~al.(2023)Fu, Zhang, Yu, Li, Ye, Li, Wan, and Lin]{fu2023gpt4aigchip}
Yonggan Fu, Yongan Zhang, Zhongzhi Yu, Sixu Li, Zhifan Ye, Chaojian Li, Cheng Wan, and Yingyan~Celine Lin.
\newblock Gpt4aigchip: Towards next-generation ai accelerator design automation via large language models.
\newblock In \emph{2023 IEEE/ACM International Conference on Computer Aided Design (ICCAD)}, pp.\  1--9. IEEE, 2023.

\bibitem[Guo et~al.(2024)Guo, Zhu, Yang, Xie, Dong, Zhang, Chen, Bi, Wu, Li, et~al.]{guo2024deepseek}
Daya Guo, Qihao Zhu, Dejian Yang, Zhenda Xie, Kai Dong, Wentao Zhang, Guanting Chen, Xiao Bi, Yu~Wu, YK~Li, et~al.
\newblock Deepseek-coder: When the large language model meets programming--the rise of code intelligence.
\newblock \emph{arXiv preprint arXiv:2401.14196}, 2024.

\bibitem[Husain et~al.(2019)Husain, Wu, Gazit, Allamanis, and Brockschmidt]{husain2019codesearchnet}
Hamel Husain, Ho-Hsiang Wu, Tiferet Gazit, Miltiadis Allamanis, and Marc Brockschmidt.
\newblock Codesearchnet challenge: Evaluating the state of semantic code search.
\newblock \emph{arXiv preprint arXiv:1909.09436}, 2019.

\bibitem[Jin et~al.(2023)Jin, Liu, Zheng, Li, Zhao, Zhang, Zheng, Zhou, and Liu]{jin2023adapt}
Bu~Jin, Xinyu Liu, Yupeng Zheng, Pengfei Li, Hao Zhao, Tong Zhang, Yuhang Zheng, Guyue Zhou, and Jingjing Liu.
\newblock Adapt: Action-aware driving caption transformer.
\newblock In \emph{2023 IEEE International Conference on Robotics and Automation (ICRA)}, pp.\  7554--7561. IEEE, 2023.

\bibitem[Li et~al.(2022)Li, Choi, Chung, Kushman, Schrittwieser, Leblond, Eccles, Keeling, Gimeno, Dal~Lago, et~al.]{li2022competition}
Yujia Li, David Choi, Junyoung Chung, Nate Kushman, Julian Schrittwieser, R{\'e}mi Leblond, Tom Eccles, James Keeling, Felix Gimeno, Agustin Dal~Lago, et~al.
\newblock Competition-level code generation with alphacode.
\newblock \emph{Science}, 378\penalty0 (6624):\penalty0 1092--1097, 2022.

\bibitem[Lin(2004)]{lin2004rouge}
Chin-Yew Lin.
\newblock Rouge: A package for automatic evaluation of summaries.
\newblock In \emph{Text summarization branches out}, pp.\  74--81, 2004.

\bibitem[Liu et~al.(2023{\natexlab{a}})Liu, Ene, Kirby, Cheng, Pinckney, Liang, Alben, Anand, Banerjee, Bayraktaroglu, et~al.]{liu2023chipnemo}
Mingjie Liu, Teodor-Dumitru Ene, Robert Kirby, Chris Cheng, Nathaniel Pinckney, Rongjian Liang, Jonah Alben, Himyanshu Anand, Sanmitra Banerjee, Ismet Bayraktaroglu, et~al.
\newblock Chipnemo: Domain-adapted llms for chip design.
\newblock \emph{arXiv preprint arXiv:2311.00176}, 2023{\natexlab{a}}.

\bibitem[Liu et~al.(2023{\natexlab{b}})Liu, Pinckney, Khailany, and Ren]{liu2023verilogeval}
Mingjie Liu, Nathaniel Pinckney, Brucek Khailany, and Haoxing Ren.
\newblock Verilogeval: Evaluating large language models for verilog code generation.
\newblock In \emph{2023 IEEE/ACM International Conference on Computer Aided Design (ICCAD)}, pp.\  1--8. IEEE, 2023{\natexlab{b}}.

\bibitem[Liu et~al.(2024)Liu, Fang, Lu, Wang, Zhang, Zhang, and Xie]{liu2024rtlcoder}
Shang Liu, Wenji Fang, Yao Lu, Jing Wang, Qijun Zhang, Hongce Zhang, and Zhiyao Xie.
\newblock Rtlcoder: Fully open-source and efficient llm-assisted rtl code generation technique.
\newblock \emph{IEEE Transactions on Computer-Aided Design of Integrated Circuits and Systems}, 2024.

\bibitem[Lu et~al.(2024)Lu, Liu, Zhang, and Xie]{lu2024rtllm}
Yao Lu, Shang Liu, Qijun Zhang, and Zhiyao Xie.
\newblock Rtllm: An open-source benchmark for design rtl generation with large language model.
\newblock In \emph{2024 29th Asia and South Pacific Design Automation Conference (ASP-DAC)}, pp.\  722--727. IEEE, 2024.

\bibitem[Na et~al.(2024)Na, Yamani, Lhadj, Baghdadi, et~al.]{na2024curriculum}
Marwa Na, Kamel Yamani, Lynda Lhadj, Riyadh Baghdadi, et~al.
\newblock Curriculum learning for small code language models.
\newblock In \emph{Proceedings of the 62nd Annual Meeting of the Association for Computational Linguistics (Volume 4: Student Research Workshop)}, pp.\  531--542, 2024.

\bibitem[Narvekar et~al.(2020)Narvekar, Peng, Leonetti, Sinapov, Taylor, and Stone]{narvekar2020curriculum}
Sanmit Narvekar, Bei Peng, Matteo Leonetti, Jivko Sinapov, Matthew~E Taylor, and Peter Stone.
\newblock Curriculum learning for reinforcement learning domains: A framework and survey.
\newblock \emph{Journal of Machine Learning Research}, 21\penalty0 (181):\penalty0 1--50, 2020.

\bibitem[Ouyang et~al.(2022)Ouyang, Wu, Jiang, Almeida, Wainwright, Mishkin, Zhang, Agarwal, Slama, Ray, et~al.]{ouyang2022training}
Long Ouyang, Jeffrey Wu, Xu~Jiang, Diogo Almeida, Carroll Wainwright, Pamela Mishkin, Chong Zhang, Sandhini Agarwal, Katarina Slama, Alex Ray, et~al.
\newblock Training language models to follow instructions with human feedback.
\newblock \emph{Advances in neural information processing systems}, 35:\penalty0 27730--27744, 2022.

\bibitem[Papineni et~al.(2002)Papineni, Roukos, Ward, and Zhu]{papineni2002bleu}
Kishore Papineni, Salim Roukos, Todd Ward, and Wei-Jing Zhu.
\newblock Bleu: a method for automatic evaluation of machine translation.
\newblock In \emph{Proceedings of the 40th annual meeting of the Association for Computational Linguistics}, pp.\  311--318, 2002.

\bibitem[Pearce et~al.(2020)Pearce, Tan, and Karri]{pearce2020dave}
Hammond Pearce, Benjamin Tan, and Ramesh Karri.
\newblock Dave: Deriving automatically verilog from english.
\newblock In \emph{Proceedings of the 2020 ACM/IEEE Workshop on Machine Learning for CAD}, pp.\  27--32, 2020.

\bibitem[Pei et~al.(2024)Pei, Zhen, Yuan, Huang, and Yu]{pei2024betterv}
Zehua Pei, Hui-Ling Zhen, Mingxuan Yuan, Yu~Huang, and Bei Yu.
\newblock Betterv: Controlled verilog generation with discriminative guidance.
\newblock \emph{arXiv preprint arXiv:2402.03375}, 2024.

\bibitem[Taori et~al.(2023)Taori, Gulrajani, Zhang, Dubois, Li, Guestrin, Liang, and Hashimoto]{taori2023stanford}
Rohan Taori, Ishaan Gulrajani, Tianyi Zhang, Yann Dubois, Xuechen Li, Carlos Guestrin, Percy Liang, and Tatsunori~B Hashimoto.
\newblock Stanford alpaca: An instruction-following llama model, 2023.

\bibitem[Thakur et~al.(2023)Thakur, Ahmad, Fan, Pearce, Tan, Karri, Dolan-Gavitt, and Garg]{thakur2023benchmarking}
Shailja Thakur, Baleegh Ahmad, Zhenxing Fan, Hammond Pearce, Benjamin Tan, Ramesh Karri, Brendan Dolan-Gavitt, and Siddharth Garg.
\newblock Benchmarking large language models for automated verilog rtl code generation.
\newblock In \emph{2023 Design, Automation \& Test in Europe Conference \& Exhibition (DATE)}, pp.\  1--6. IEEE, 2023.

\bibitem[Thakur et~al.(2024)Thakur, Ahmad, Pearce, Tan, Dolan-Gavitt, Karri, and Garg]{thakur2024verigen}
Shailja Thakur, Baleegh Ahmad, Hammond Pearce, Benjamin Tan, Brendan Dolan-Gavitt, Ramesh Karri, and Siddharth Garg.
\newblock Verigen: A large language model for verilog code generation.
\newblock \emph{ACM Transactions on Design Automation of Electronic Systems}, 29\penalty0 (3):\penalty0 1--31, 2024.

\bibitem[Wang et~al.(2022)Wang, Kordi, Mishra, Liu, Smith, Khashabi, and Hajishirzi]{wang2022self}
Yizhong Wang, Yeganeh Kordi, Swaroop Mishra, Alisa Liu, Noah~A Smith, Daniel Khashabi, and Hannaneh Hajishirzi.
\newblock Self-instruct: Aligning language models with self-generated instructions.
\newblock \emph{arXiv preprint arXiv:2212.10560}, 2022.

\bibitem[Wang et~al.(2023{\natexlab{a}})Wang, Le, Gotmare, Bui, Li, and Hoi]{wang2023codet5+}
Yue Wang, Hung Le, Akhilesh~Deepak Gotmare, Nghi~DQ Bui, Junnan Li, and Steven~CH Hoi.
\newblock Codet5+: Open code large language models for code understanding and generation.
\newblock \emph{arXiv preprint arXiv:2305.07922}, 2023{\natexlab{a}}.

\bibitem[Wang et~al.(2023{\natexlab{b}})Wang, Yue, Lu, Liu, Zhong, Song, and Huang]{wang2023efficienttrain}
Yulin Wang, Yang Yue, Rui Lu, Tianjiao Liu, Zhao Zhong, Shiji Song, and Gao Huang.
\newblock Efficienttrain: Exploring generalized curriculum learning for training visual backbones.
\newblock In \emph{Proceedings of the IEEE/CVF International Conference on Computer Vision}, pp.\  5852--5864, 2023{\natexlab{b}}.

\bibitem[Wei et~al.(2024)Wei, Wang, Lu, Xu, Liu, Zhao, Chen, and Wang]{wei2024editable}
Yuxi Wei, Zi~Wang, Yifan Lu, Chenxin Xu, Changxing Liu, Hao Zhao, Siheng Chen, and Yanfeng Wang.
\newblock Editable scene simulation for autonomous driving via collaborative llm-agents.
\newblock In \emph{Proceedings of the IEEE/CVF Conference on Computer Vision and Pattern Recognition}, pp.\  15077--15087, 2024.

\bibitem[Williams \& Baxter(2002)Williams and Baxter]{williams2002icarus}
Stephen Williams and Michael Baxter.
\newblock Icarus verilog: open-source verilog more than a year later.
\newblock \emph{Linux Journal}, 2002\penalty0 (99):\penalty0 3, 2002.

\bibitem[Wu et~al.(2024)Wu, He, Zhang, Yao, Zheng, Zheng, and Yu]{wu2024chateda}
Haoyuan Wu, Zhuolun He, Xinyun Zhang, Xufeng Yao, Su~Zheng, Haisheng Zheng, and Bei Yu.
\newblock Chateda: A large language model powered autonomous agent for eda.
\newblock \emph{IEEE Transactions on Computer-Aided Design of Integrated Circuits and Systems}, 2024.

\bibitem[Xu et~al.(2020)Xu, Zhang, Mao, Wang, Xie, and Zhang]{xu2020curriculum}
Benfeng Xu, Licheng Zhang, Zhendong Mao, Quan Wang, Hongtao Xie, and Yongdong Zhang.
\newblock Curriculum learning for natural language understanding.
\newblock In \emph{Proceedings of the 58th Annual Meeting of the Association for Computational Linguistics}, pp.\  6095--6104, 2020.

\bibitem[Yan et~al.(2017)Yan, Liu, Li, Han, and Qiu]{yan2017privmin}
Ziqi Yan, Jiqiang Liu, Gang Li, Zhen Han, and Shuo Qiu.
\newblock Privmin: Differentially private minhash for jaccard similarity computation.
\newblock \emph{arXiv preprint arXiv:1705.07258}, 2017.

\bibitem[Zhang et~al.(2024)Zhang, Yu, Fu, Wan, et~al.]{zhang2024mg}
Yongan Zhang, Zhongzhi Yu, Yonggan Fu, Cheng Wan, et~al.
\newblock Mg-verilog: Multi-grained dataset towards enhanced llm-assisted verilog generation.
\newblock \emph{arXiv preprint arXiv:2407.01910}, 2024.

\end{thebibliography}


\newpage
\appendix
% \section{Appendix}

\section{Introduction of Verilog}
\label{appendix:verilog_introduction}

Verilog is the most widely used hardware description language (HDL) for modeling digital integrated circuits. It enables designers to specify both the behavioral and structural aspects of hardware systems, such as processors, controllers, and digital logic circuits. Verilog operates at a relatively low level, focusing on gates, registers, and signal assignments—each representing physical hardware components. While Verilog supports behavioral constructs (\textit{e.g.}, \texttt{if-else}, \texttt{case}) that are somewhat similar to software programming languages, their use is constrained by synthesizable coding styles required for hardware implementation.
Verilog differs from software programming languages like Python and C++ in several key ways:


\begin{enumerate}
    \item \textbf{Parallelism:} Verilog inherently models hardware’s concurrnet nature, with multiple statements executing simultaneously. In contrast, software languages like Python typically follow a sequential execution model.
    \item \textbf{Timing:} Timing is a fundamental concept in Verilog that directly influences how digital circuits are designed and simulated. Verilog relies on clocks to synchronize sequential logic behaviors, enabling the precise modeling of synthronous circuits. In contrast, software programming languages generally do not have an inherent need for explicit timing.
    \item \textbf{Syntax and Constructs:} Verilog’s syntax is tailored to describe the behavior and structure of digital circuits, reflecting the parallel nature of hardware. Key constructs of Verilog include:
    
    \begin{itemize}
        \item {\textbf{Modules:}}
        The basic unit of Verilog, used to define a hardware block or component. Each module in Verilog encapsulates inputs, outputs, and internal logic, and modules can be instantiated within other modules, enabling hierarchical designs that mirror the complexity of real-world systems. And each module instantiation results in the generation of a corresponding circuit block.
        \item {\textbf{Always block:}}
        In an \texttt{always} block, circuit designers can model circuits using high-level behavioral descriptions. However, this does not imply that a broad range of programming language syntax is available. In practice, Verilog supports only a limited subset of programming-like constructs, primarily \texttt{if-else} and \texttt{case} statements. Statements in multiple \texttt{always} blocks are executed in parallel and the resulting circuit continuously performs its operations.
        \item {\textbf{Sensitivity list:}}
        In an \texttt{always} block, the sensitivity list specifies the signals that trigger the block’s execution when they change.
        \item {\textbf{Assign statements:}}
        \texttt{assign} statements are used to describe continuous assignments of signal values in parallel, reflecting the inherent concurrency of hardware.
        \item {\textbf{Registers (\texttt{reg}) and Wires (\texttt{wire}):}}
        \texttt{reg} is used for variables that retain their values (\textit{e.g.}, flip-flops or memory), and \texttt{wire} is used for connections that propagate values through the circuit.
        
    \end{itemize}

    In contrast, software programming languages like C, Python, or Java employ a more conventional syntax for defining algorithms, control flow, and data manipulation. These languages use constructs like loops (\texttt{for}, \texttt{while}), conditionals (\texttt{if}, \texttt{else}), and functions or methods for structuring code, with data types such as integers, strings, and floats for variable storage.

\end{enumerate}


\section{Prompt Details for CoT Annotation}
\label{appendix:prompt}

\begin{figure}[ht]
    \centering
    \includegraphics[width=\linewidth]{fig/prompt_v2.pdf}
    \caption{Detailed prompts used in the CoT annotation process.}
    \label{fig:prompt}
\end{figure}

As shown in Figure~\ref{fig:prompt}, we present the detailed prompts used in our annotation process. 
For each task, we supplement the primary prompt with several human-reviewed input-output pair examples, serving as in-context learning examples to enhance GPT's understanding of task requirements and expectations.
%In addition to the prompt listed, for each task, we will also provide GPT with several human-reviewed input-output pair examples as initial input to help it better understand the task requirements and expectations. 
These examples will serve as guidance for the model to correctly interpret and execute tasks in accordance with the prompt, ensuring more accurate and contextually relevant outputs.

\section{Discarding Verilog Code Exceeding $2048$ Tokens}
\label{appendix:discard}
In the main submission, we state that Verilog modules and blocks exceeding $2048$ tokens are excluded, as $2048$ is the maximum input length supported by CodeT5+. Beyond this limitation, several additional factors motivate this decision:
\newpage

\begin{figure}[ht]
    \centering
    \vspace{-20pt}
    \includegraphics[width=0.9\linewidth]{fig/distribution.jpg}
    \vspace{-10pt}
    \caption{The distribution of the token lengths of the generation benchmark by~\citet{chang2024natural}.}
    \label{fig:distribution}
\end{figure}
\begin{enumerate}
    \item \textbf{Generation Capabilities of Existing LLMs Are Limited to Small Designs}

    Existing benchmarks for Verilog generation, including the one used in our work~\citep{chang2024natural}, do not include designs exceeding $2048$ tokens, with the maximum token length observed in the benchmark being $1851$. As shown in Table~\ref{tab:generation_results} of the main submission, even the state-of-the-art LLM, o1-preview, is capable of accurately generating only simple designs and struggles with more complex ones. 
    Figure~\ref{fig:distribution} illustrates the token length distribution across the benchmark, further justifying our decision to exclude Verilog modules and blocks exceeding $2048$ tokens.

    \item \textbf{Segmentation as a Common Practice}

    Segmenting longer code into smaller chunks that fit within the predefined context window and discarding those that exceed it is a widely accepted practice in both Verilog-related research~\citep{chang2024data,pei2024betterv} and studies on software programming language~\citep{wang2023codet5+}. This approach ensures compatibility with current LLMs while maintaining the integrity and usability of the dataset. It is worth noting that the default maximum sequence length in CodeT5+ is $512$ tokens, and our work extends this limit to $2048$ tokens to better accommodate Verilog designs.

    \item \textbf{Empirical Findings and Practical Challenges}
    
    Our experiments reveal an important empirical observation: existing LLMs, such as GPT-4, consistently produce accurate descriptions for shorter Verilog modules but struggle with correctness when handling longer ones. Specifically, During the annotation process, we divide the dataset into two sections: Verilog designs with fewer than $2048$ tokens, and designs with token lengths between $2048$ and $4096$ tokens. Our human evaluation finds that descriptions for Verilog designs with fewer than $2048$ tokens are approximately 90\% accurate, while descriptions for designs with token lengths between $2048$ and $4096$ tokens have accuracy rates of only 60\%–70\%. And accuracy further decreases for designs exceeding $4096$ tokens. Since our datasets rely on LLM-generated annotations, restricting the dataset to Verilog modules within the $2048$-token limit helps maintain the quality and accuracy of annotations. This, in turn, facilitates higher-quality dataset creation and more efficient fine-tuning. For the potential negative impact of incorporating Verilog designs larger than $2048$ tokens, please refer to Appendix~\ref{appendix:negative_impact}.
    And we examine the impact of varying context window lengths in Appendix~\ref{appendix:varying_context_window_length}.
\end{enumerate}


\section{Standards and Processes for Manual Code Annotation}
\label{appendix:standard}

Given the industrial-grade quality of the proprietary code, we employ professional hardware engineers for manual annotation. We have established the following standards and processes to guide engineers in crafting accurate and detailed descriptions with example annotations shown in Figure~\ref{fig:engineer}:

\begin{enumerate}
    \item \textbf{Standards:} The hardware engineers are required to provide descriptions at both the module and block levels.
    
    \begin{itemize}
        \item For module-level descriptions, two levels are defined:
        \begin{itemize}
            \item[i.] \textbf{H (High-level):} The role of this module in the overall design (IP/Chip).
            \item[ii.] \textbf{D (Detailed):} What functions this module performs (overview) and how it is implemented (implementation details). This description should adhere to a top-down structure and consist of approximately 2-5 sentences.
        \end{itemize}
        \textbf{Note:} If the summary statements for H and D are identical, both must be provided.
        
        \item For block-level descriptions, particularly \texttt{always} blocks, descriptions are required at three distinct levels:
        \begin{itemize}
            \item[i.] \textbf{H (High-level):} The role of this block in the overall design (\textit{e.g.}, across modules).
            \item[ii.] \textbf{M (Medium-detail):} Contextual explanations.
            \item[iii.] \textbf{D (Detailed):} Descriptions specific to the block following a top-down structure. If details are absent, they may be omitted; do not guess based on signal names.
        \end{itemize}
    \end{itemize}
    
    \item \textbf{Processes:} Initially, we provide engineers with a set of descriptions generated by GPT-4 for reference. They are then expected to revise and enhance these GPT-generated descriptions using their expertise and relevant supplementary materials, such as README files and register tables.

\end{enumerate}

\begin{figure}[ht]
    \centering
    \includegraphics[width=\linewidth]{fig/engineer.pdf}
    \caption{Human-annotated examples for the proprietary code.}
    \label{fig:engineer}
\end{figure}

\section{Examples of Verilog Understanding Benchmark}
\label{appendix:benchmark}

To construct a high-quality benchmark, we first remove comments from the original code, and then submit it to experienced hardware engineers for annotation, ultimately producing the code and description pairs as shown in Figure~\ref{fig:verified_data}.


\begin{figure}[ht]
    \centering
    \includegraphics[width=\linewidth]{fig/verified_data_v2.pdf}
    \caption{Examples from the Verilog understanding benchmark.}
    \label{fig:verified_data}
\end{figure}


\section{Model Selection}
\label{appendix:model_selection}
In this work, we choose CodeT5+, a family of encoder-decoder code foundation models, as the base model for training DeepRTL for two primary reasons. First, as we aim to develop a unified model for Verilog understanding and generation, T5-like models are particularly well-suited due to their ability to effectively handle both tasks, as evidenced by~\citet{wang2023codet5+}. Second, the encoder component of CodeT5+ enables the natural extraction of Verilog representations, which can be potentially utilized for various downstream tasks in EDA at the RTL stage. Examples include PPA (Power, Performance, Area) prediction, which estimates the power consumption, performance, and area of an RTL design, and verification, which ensures that the RTL design correctly implements its intended functionality and meets specification requirements. Both tasks are crucial in the hardware design process. This capability distinguishes it from decoder-only models, which are typically less suited for producing standalone, reusable intermediate representations. In future work, we plan to explore how DeepRTL can further enhance productivity in the hardware design process.

To further demonstrate the superiority of CodeT5+ as a base model, we fine-tune two additional models, deepseek-coder-1.3b-instruct\footnote{\url{https://huggingface.co/deepseek-ai/deepseek-coder-1.3b-instruct}} (deepseek-coder)~\citep{guo2024deepseek} and Llama-3.2-1B-Instruct\footnote{\url{https://huggingface.co/meta-llama/Llama-3.2-1B-Instruct}} (llama-3.2)~\citep{dubey2024llama}, using the same dataset as DeepRTL and the adopted curriculum learning strategy.

In Table~\ref{tab:understanding_additional} and Table~\ref{tab:decoder_compare}, we present the performance of both the original base models and their fine-tuned counterparts on Verilog understanding and generation tasks. The improvement in performance from the original base models to the fine-tuned models highlights the effectiveness of our dataset and the curriculum learning-based fine-tuning strategy. Compared to the results in Table~\ref{tab:understanding_results} and Table~\ref{tab:generation_results}, the superior performance of DeepRTL-220m on both tasks, despite its smaller model size, underscores the architectural advantages of our approach.


\begin{table}[ht]
\centering
\vspace{-20pt}
\caption{Evaluation results on Verilog understanding using the benchmark proposed in Section~\ref{sec:understanding_benchmark}. BLEU-4 denotes the smoothed BLEU-4 score, and Emb. Sim. represents the embedding similarity metric. Specifically, this table presents the performance of decoder-only models, where ``long'' indicates models fine-tuned on the dataset containing longer Verilog designs, and those fine-tuned specifically on Verilog. $^\dag$ indicates performance evaluated on designs shorter than $512$ tokens.}
\vspace{2pt}
\label{tab:understanding_additional}
% \small{
\resizebox{\columnwidth}{!}{%
\begin{tabular}{@{}l|ccccccc@{}}
    \toprule
Model & BLEU-4 & ROUGE-1 & ROUGE-2 & ROUGE-L & Emb. Sim. & GPT Score \\
\midrule
deepseek-coder (original) & 1.04 & 21.43 & 4.38 & 19.77 & 0.678 & 0.557 \\
deepseek-coder (fine-tuned) & 11.96 & 40.49 & 19.77 & 36.14 & 0.826 & 0.664 \\
deepseek-coder (long) & 11.27 & 40.28 & 18.95 & 35.93 & 0.825 & 0.649 \\
\midrule
llama-3.2 (original) & 0.88 & 19.26 & 3.60 & 17.64 & 0.615 & 0.449 \\
llama-3.2 (fine-tuned) & 12.11 & 39.95 & 19.47 & 35.29 & 0.825 & 0.620 \\
llama-3.2 (long) & 11.32 & 39.60 & 18.67 & 34.94 & 0.814 & 0.610 \\
\midrule
RTLCoder & 1.08 & 21.83 & 4.68 & 20.30 & 0.687 & 0.561 \\
VeriGen & 0.09 & 6.54 & 0.35 & 6.08 & 0.505 & 0.311 \\
\midrule
DeepRTL-220m-512$^\dag$	& 14.98	& 44.27	& 23.11	& 40.08	& 0.780	& 0.567 \\
DeepRTL-220m$^\dag$	& 18.74	& 48.41	& 29.82	& 45.01	& 0.855	& 0.743 \\
\bottomrule
\end{tabular}%
}
\end{table}


\begin{table}[!ht]
\centering
\vspace{-5pt}
\caption{Evaluation results on Verilog generation. Each cell displays the percentage of code samples,
out of five trials, that successfully pass compilation (syntax column) or functional unit tests (function
column). This table presents the performance of decoder-only models, where ``o'' denotes the original model and ``f'' denotes the fine-tuned model.}
\vspace{5pt}
\label{tab:decoder_compare}
% {\tiny
\resizebox{\columnwidth}{!}{%
\begin{tabular}{|cl|cc|cc|cc|cc|}
\hline
\multicolumn{2}{|c|}{\multirow{2}{*}{Benchmark}} & \multicolumn{2}{c|}{deepseek-coder (o)} & \multicolumn{2}{c|}{deepseek-coder (f)} & \multicolumn{2}{c|}{llama-3.2 (o)} & \multicolumn{2}{c|}{llama-3.2 (f)} \\ \cline{3-10} 
\multicolumn{2}{|c|}{} & \multicolumn{1}{c|}{syntax} & function & \multicolumn{1}{c|}{syntax} & function & \multicolumn{1}{c|}{syntax} & function & \multicolumn{1}{c|}{syntax} & function \\ \hline
\multicolumn{1}{|c|}{\multirow{10}{*}{Logic}} & Johnson\_Counter & \multicolumn{1}{c|}{100\%} & 0\% & \multicolumn{1}{c|}{100\%} & 0\% & \multicolumn{1}{c|}{100\%} & 0\% & \multicolumn{1}{c|}{100\%} & 0\% \\ \cline{2-10} 
\multicolumn{1}{|c|}{} & alu & \multicolumn{1}{c|}{0\%} & 0\% & \multicolumn{1}{c|}{0\%} & 0\% & \multicolumn{1}{c|}{0\%} & 0\% & \multicolumn{1}{c|}{0\%} & 0\% \\ \cline{2-10} 
\multicolumn{1}{|c|}{} & edge\_detect & \multicolumn{1}{c|}{60\%} & 0\% & \multicolumn{1}{c|}{80\%} & 20\% & \multicolumn{1}{c|}{60\%} & 0\% & \multicolumn{1}{c|}{80\%} & 0\% \\ \cline{2-10} 
\multicolumn{1}{|c|}{} & freq\_div & \multicolumn{1}{c|}{80\%} & 0\% & \multicolumn{1}{c|}{100\%} & 0\% & \multicolumn{1}{c|}{80\%} & 0\% & \multicolumn{1}{c|}{100\%} & 0\% \\ \cline{2-10} 
\multicolumn{1}{|c|}{} & mux & \multicolumn{1}{c|}{60\%} & 0\% & \multicolumn{1}{c|}{100\%} & 100\% & \multicolumn{1}{c|}{60\%} & 0\% & \multicolumn{1}{c|}{60\%} & 60\% \\ \cline{2-10} 
\multicolumn{1}{|c|}{} & parallel2serial & \multicolumn{1}{c|}{80\%} & 0\% & \multicolumn{1}{c|}{100\%} & 0\% & \multicolumn{1}{c|}{80\%} & 0\% & \multicolumn{1}{c|}{100\%} & 0\% \\ \cline{2-10} 
\multicolumn{1}{|c|}{} & pulse\_detect & \multicolumn{1}{c|}{60\%} & 0\% & \multicolumn{1}{c|}{80\%} & 40\% & \multicolumn{1}{c|}{60\%} & 20\% & \multicolumn{1}{c|}{60\%} & 40\% \\ \cline{2-10} 
\multicolumn{1}{|c|}{} & right\_shifter & \multicolumn{1}{c|}{20\%} & 0\% & \multicolumn{1}{c|}{80\%} & 80\% & \multicolumn{1}{c|}{20\%} & 0\% & \multicolumn{1}{c|}{40\%} & 40\% \\ \cline{2-10} 
\multicolumn{1}{|c|}{} & serial2parallel & \multicolumn{1}{c|}{100\%} & 0\% & \multicolumn{1}{c|}{100\%} & 0\% & \multicolumn{1}{c|}{100\%} & 0\% & \multicolumn{1}{c|}{100\%} & 0\% \\ \cline{2-10} 
\multicolumn{1}{|c|}{} & width\_8to16 & \multicolumn{1}{c|}{100\%} & 0\% & \multicolumn{1}{c|}{100\%} & 0\% & \multicolumn{1}{c|}{100\%} & 0\% & \multicolumn{1}{c|}{100\%} & 0\% \\ \hline
\multicolumn{1}{|c|}{\multirow{11}{*}{Arithmetic}} & accu & \multicolumn{1}{c|}{80\%} & 0\% & \multicolumn{1}{c|}{100\%} & 0\% & \multicolumn{1}{c|}{80\%} & 0\% & \multicolumn{1}{c|}{100\%} & 0\% \\ \cline{2-10} 
\multicolumn{1}{|c|}{} & adder\_16bit & \multicolumn{1}{c|}{20\%} & 0\% & \multicolumn{1}{c|}{40\%} & 20\% & \multicolumn{1}{c|}{20\%} & 0\% & \multicolumn{1}{c|}{20\%} & 20\% \\ \cline{2-10} 
\multicolumn{1}{|c|}{} & adder\_16bit\_csa & \multicolumn{1}{c|}{0\%} & 0\% & \multicolumn{1}{c|}{0\%} & 20\% & \multicolumn{1}{c|}{0\%} & 20\% & \multicolumn{1}{c|}{20\%} & 20\% \\ \cline{2-10} 
\multicolumn{1}{|c|}{} & adder\_32bit & \multicolumn{1}{c|}{0\%} & 0\% & \multicolumn{1}{c|}{20\%} & 0\% & \multicolumn{1}{c|}{0\%} & 0\% & \multicolumn{1}{c|}{20\%} & 20\% \\ \cline{2-10} 
\multicolumn{1}{|c|}{} & adder\_64bit & \multicolumn{1}{c|}{0\%} & 0\% & \multicolumn{1}{c|}{20\%} & 0\% & \multicolumn{1}{c|}{0\%} & 0\% & \multicolumn{1}{c|}{40\%} & 0\% \\ \cline{2-10} 
\multicolumn{1}{|c|}{} & adder\_8bit & \multicolumn{1}{c|}{40\%} & 0\% & \multicolumn{1}{c|}{80\%} & 20\% & \multicolumn{1}{c|}{40\%} & 0\% & \multicolumn{1}{c|}{60\%} & 20\% \\ \cline{2-10} 
\multicolumn{1}{|c|}{} & div\_16bit & \multicolumn{1}{c|}{0\%} & 0\% & \multicolumn{1}{c|}{20\%} & 0\% & \multicolumn{1}{c|}{0\%} & 0\% & \multicolumn{1}{c|}{0\%} & 0\% \\ \cline{2-10} 
\multicolumn{1}{|c|}{} & multi\_16bit & \multicolumn{1}{c|}{60\%} & 0\% & \multicolumn{1}{c|}{80\%} & 0\% & \multicolumn{1}{c|}{60\%} & 0\% & \multicolumn{1}{c|}{80\%} & 0\% \\ \cline{2-10} 
\multicolumn{1}{|c|}{} & multi\_booth & \multicolumn{1}{c|}{40\%} & 0\% & \multicolumn{1}{c|}{60\%} & 0\% & \multicolumn{1}{c|}{40\%} & 0\% & \multicolumn{1}{c|}{60\%} & 0\% \\ \cline{2-10} 
\multicolumn{1}{|c|}{} & multi\_pipe\_4bit & \multicolumn{1}{c|}{100\%} & 0\% & \multicolumn{1}{c|}{100\%} & 100\% & \multicolumn{1}{c|}{100\%} & 0\% & \multicolumn{1}{c|}{100\%} & 100\% \\ \cline{2-10} 
\multicolumn{1}{|c|}{} & multi\_pipe\_8bit & \multicolumn{1}{c|}{0\%} & 0\% & \multicolumn{1}{c|}{0\%} & 0\% & \multicolumn{1}{c|}{0\%} & 0\% & \multicolumn{1}{c|}{0\%} & 0\% \\ \hline
\multicolumn{1}{|c|}{\multirow{10}{*}{Advanced}} & 1x2nocpe & \multicolumn{1}{c|}{60\%} & 0\% & \multicolumn{1}{c|}{20\%} & 40\% & \multicolumn{1}{c|}{60\%} & 20\% & \multicolumn{1}{c|}{60\%} & 20\% \\ \cline{2-10} 
\multicolumn{1}{|c|}{} & 1x4systolic & \multicolumn{1}{c|}{20\%} & 0\% & \multicolumn{1}{c|}{100\%} & 100\% & \multicolumn{1}{c|}{20\%} & 0\% & \multicolumn{1}{c|}{20\%} & 20\% \\ \cline{2-10} 
\multicolumn{1}{|c|}{} & 2x2systolic & \multicolumn{1}{c|}{0\%} & 0\% & \multicolumn{1}{c|}{0\%} & 0\% & \multicolumn{1}{c|}{0\%} & 0\% & \multicolumn{1}{c|}{0\%} & 0\% \\ \cline{2-10} 
\multicolumn{1}{|c|}{} & 4x4spatialacc & \multicolumn{1}{c|}{0\%} & 0\% & \multicolumn{1}{c|}{0\%} & 0\% & \multicolumn{1}{c|}{0\%} & 0\% & \multicolumn{1}{c|}{0\%} & 0\% \\ \cline{2-10} 
\multicolumn{1}{|c|}{} & fsm & \multicolumn{1}{c|}{80\%} & 0\% & \multicolumn{1}{c|}{100\%} & 100\% & \multicolumn{1}{c|}{80\%} & 0\% & \multicolumn{1}{c|}{100\%} & 100\% \\ \cline{2-10} 
\multicolumn{1}{|c|}{} & macpe & \multicolumn{1}{c|}{0\%} & 0\% & \multicolumn{1}{c|}{0\%} & 0\% & \multicolumn{1}{c|}{0\%} & 0\% & \multicolumn{1}{c|}{0\%} & 0\% \\ \cline{2-10} 
\multicolumn{1}{|c|}{} & 5state\_fsm & \multicolumn{1}{c|}{80\%} & 0\% & \multicolumn{1}{c|}{100\%} & 20\% & \multicolumn{1}{c|}{80\%} & 0\% & \multicolumn{1}{c|}{100\%} & 100\% \\ \cline{2-10} 
\multicolumn{1}{|c|}{} & 3state\_fsm & \multicolumn{1}{c|}{0\%} & 0\% & \multicolumn{1}{c|}{100\%} & 80\% & \multicolumn{1}{c|}{20\%} & 20\% & \multicolumn{1}{c|}{100\%} & 100\% \\ \cline{2-10} 
\multicolumn{1}{|c|}{} & 4state\_fsm & \multicolumn{1}{c|}{80\%} & 0\% & \multicolumn{1}{c|}{100\%} & 40\% & \multicolumn{1}{c|}{80\%} & 20\% & \multicolumn{1}{c|}{100\%} & 20\% \\ \cline{2-10} 
\multicolumn{1}{|c|}{} & 2state\_fsm & \multicolumn{1}{c|}{60\%} & 0\% & \multicolumn{1}{c|}{100\%} & 20\% & \multicolumn{1}{c|}{60\%} & 0\% & \multicolumn{1}{c|}{100\%} & 20\% \\ \hline
\multicolumn{2}{|c|}{Success Rate} & \multicolumn{1}{c|}{44.52\%} & 0.00\% & \multicolumn{1}{c|}{63.87\%} & 25.81\% & \multicolumn{1}{c|}{45.16\%} & 3.23\% & \multicolumn{1}{c|}{58.71\%} & 22.58\% \\ \hline
\multicolumn{2}{|c|}{Pass @ 1} & \multicolumn{1}{c|}{12.90\%} & 0.00\% & \multicolumn{1}{c|}{61.29\%} & 22.58\% & \multicolumn{1}{c|}{12.90\%} & 0.00\% & \multicolumn{1}{c|}{54.84\%} & 19.35\% \\ \hline
\multicolumn{2}{|c|}{Pass @ 5} & \multicolumn{1}{c|}{67.74\%} & 0.00\% & \multicolumn{1}{c|}{80.65\%} & 48.39\% & \multicolumn{1}{c|}{70.97\%} & 16.13\% & \multicolumn{1}{c|}{80.65\%} & 48.39\% \\ \hline
\end{tabular}%
}
% }
\end{table}

\section{Instructions for Different Scenarios}
\label{appendix:instruction}
Figure~\ref{fig:instruction_example} presents detailed instruction samples for different scenarios, following the instruction construction process illustrated in Figure~\ref{fig:instruction}.
Additionally, it includes a special module-level task, which involves completing the source code based on the functional descriptions of varying granularity and the predefined module header.

\begin{figure}[ht]
    \centering
    \includegraphics[width=0.98\linewidth]{fig/instruction_example_v2.pdf}
    \caption{Instruction tuning data samples for different scenarios.}
    \label{fig:instruction_example}
\end{figure}

\section{Further Explanation of the Adopted Curriculum Learning Strategy}
\label{appendix:explanation_curriculum_learning}
Our dataset includes three levels of annotation: line, block, and module, with each level containing descriptions that span various levels of detail—from detailed specifications to high-level functional descriptions. And the entire dataset is utilized for training. To fully leverage the potential of this dataset, we employ a curriculum learning strategy, enabling the model to incrementally build knowledge by starting with simpler cases and advancing to more complex ones.

The curriculum learning strategy involves transitioning from more granular to less granular annotations across hierarchical levels, which can be conceptualized as a tree structure with the following components (as shown in Figure~\ref{fig:tree}):

\begin{figure}[ht]
    \centering
    \includegraphics[width=\linewidth]{fig/tree.pdf}
    \caption{The adopted curriculum learning strategy visualized as a tree structure. Specifically, the terminals of the tree, enclosed by blue dotted boxes, represent specific training datasets. Our curriculum learning strategy follows a pre-order traversal of this tree structure.}
    \label{fig:tree}
\end{figure}

\begin{enumerate}
    \item \textbf{Hierarchical Levels (First Layer)}
    
    The training process transitions sequentially across the three hierarchical levels—line, block, and module. Each level is fully trained before moving to the next, ensuring a solid foundation at simpler levels before addressing more complex tasks.
    \item \textbf{Granularity of Descriptions (Second Layer)}
    
    Within each hierarchical level, the annotations transition from detailed descriptions to high-level descriptions. This progression ensures that the model learns finer details first and then builds an understanding of higher-level abstractions.
    \newpage
    \item \textbf{Annotation Source Transition (Third Layer)}
    
    At each level and granularity, training starts with GPT-annotated data and is followed by human-annotated data. This sequence leverages large-scale machine-generated annotations first and refines the model with high-quality, human-curated data.
    
    \item \textbf{Instruction Blending}
    
    Each terminal node in this tree represents a specific training dataset, which blends tasks for Verilog understanding and Verilog generation. This enables the model to perform well across diverse tasks.
\end{enumerate}

The training process mirrors a pre-order traversal of this tree structure:
\begin{enumerate}
    \item Starting at the root, training begins with the line level.
    \item The model progresses through the second layer (detailed, medium-detail, and high-level descriptions).
    \item Within each granularity, training transitions through the third layer (GPT-annotated data first, followed by human-annotated data).
    \item Once the line level is complete, the process repeats for the block level and then the module level.
\end{enumerate}

% This progressive training strategy aligns closely with the principles of curriculum learning, where simpler concepts are introduced first, and the knowledge gained is transferred incrementally to handle more complex scenarios.

To validate the effectiveness of this strategy, we conduct an ablation study where the model is trained on the entire dataset all at once without progression. The results, presented in Table~\ref{tab:understanding_results} of the main submission, demonstrate that the curriculum learning strategy significantly outperforms this baseline approach. Moreover, to the best of our knowledge, this is one of the first applications of a curriculum-like training strategy in the code-learning domain. Unlike existing Verilog-related models that establish simple and weak alignments between natural language and Verilog code~\citep{chang2024data}, or general software code datasets like CodeSearchNet\footnote{\url{https://huggingface.co/datasets/code-search-net/code_search_net}}~\citep{husain2019codesearchnet} that only provide single-level docstring annotations, our approach incorporates multi-level and multi-granularity annotations in a structured training process.


\section{Prompt for Calculating GPT Score}
\label{appendix:gpt_score}
To calculate the GPT score, we input the model’s generated descriptions (model\_output) and the ground truth annotations (ground\_truth) to GPT-4, using the prompt displayed in Figure~\ref{fig:gpt_score}. This metric is designed to assess the semantic accuracy of the generated functional descriptions.

\begin{figure}[ht]
    \centering
    \includegraphics[width=0.72\linewidth]{fig/gptscore.pdf}
    \caption{Prompt used to calculate the GPT score.}
    \label{fig:gpt_score}
\end{figure}

\section{Comparison with Models Specifically Trained on Verilog}
\label{appendix:comparison}

To further demonstrate the superiority of DeepRTL, we conduct experiments comparing it with models specifically trained on Verilog. 
We do not select~\citep{chang2024data,zhang2024mg} for comparison, as their models are not open-sourced, and it is non-trivial to reproduce their experiments. Additionally, the reported performance in their original papers is either comparable to, and in some cases inferior to, that of GPT-3.5. 
In Table~\ref{tab:understanding_additional} and Table~\ref{tab:verilog_specific}, we show the performance of two state-of-the-art Verilog generation models, RTLCoder-Deepseek-v1.1\footnote{\url{https://huggingface.co/ishorn5/RTLCoder-Deepseek-v1.1}} (RTLCoder)~\citep{liu2024rtlcoder} and fine-tuned-codegen-16B-Verilog\footnote{\url{https://huggingface.co/shailja/fine-tuned-codegen-16B-Verilog}} (VeriGen)~\citep{thakur2024verigen} on both Verilog understanding and generation benchmarks. It is noteworthy that RTLCoder is fine-tuned on DeepSeek-coder-6.7B, and VeriGen is fine-tuned on CodeGen-multi-16B, both of which have significantly larger parameter sizes than DeepRTL-220m. Despite this, the superior performance of DeepRTL-220m further underscores the effectiveness of our proposed dataset and the adopted curriculum learning strategy.


\begin{table}[!ht]
\centering
\caption{Evaluation results on Verilog generation. Each cell displays the percentage of code samples,
out of five trials, that successfully pass compilation (syntax column) or functional unit tests (function
column). This table presents the performance of models specifically trained on Verilog.}
\vspace{5pt}
\label{tab:verilog_specific}
{\tiny
\begin{tabular}{|cl|cc|cc|}
\hline
\multicolumn{2}{|c|}{\multirow{2}{*}{Benchmark}} & \multicolumn{2}{c|}{RTLCoder} & \multicolumn{2}{c|}{VeriGen} \\ \cline{3-6} 
\multicolumn{2}{|c|}{} & \multicolumn{1}{c|}{syntax} & function & \multicolumn{1}{c|}{syntax} & function \\ \hline
\multicolumn{1}{|c|}{\multirow{10}{*}{Logic}} & Johnson\_Counter & \multicolumn{1}{c|}{40\%} & 0\% & \multicolumn{1}{c|}{100\%} & 0\% \\ \cline{2-6} 
\multicolumn{1}{|c|}{} & alu & \multicolumn{1}{c|}{0\%} & 0\% & \multicolumn{1}{c|}{0\%} & 0\% \\ \cline{2-6} 
\multicolumn{1}{|c|}{} & edge\_detect & \multicolumn{1}{c|}{100\%} & 100\% & \multicolumn{1}{c|}{100\%} & 20\% \\ \cline{2-6} 
\multicolumn{1}{|c|}{} & freq\_div & \multicolumn{1}{c|}{60\%} & 0\% & \multicolumn{1}{c|}{100\%} & 0\% \\ \cline{2-6} 
\multicolumn{1}{|c|}{} & mux & \multicolumn{1}{c|}{60\%} & 40\% & \multicolumn{1}{c|}{80\%} & 20\% \\ \cline{2-6} 
\multicolumn{1}{|c|}{} & parallel2serial & \multicolumn{1}{c|}{60\%} & 0\% & \multicolumn{1}{c|}{100\%} & 0\% \\ \cline{2-6} 
\multicolumn{1}{|c|}{} & pulse\_detect & \multicolumn{1}{c|}{20\%} & 0\% & \multicolumn{1}{c|}{40\%} & 0\% \\ \cline{2-6} 
\multicolumn{1}{|c|}{} & right\_shifter & \multicolumn{1}{c|}{80\%} & 80\% & \multicolumn{1}{c|}{100\%} & 100\% \\ \cline{2-6} 
\multicolumn{1}{|c|}{} & serial2parallel & \multicolumn{1}{c|}{60\%} & 0\% & \multicolumn{1}{c|}{80\%} & 0\% \\ \cline{2-6} 
\multicolumn{1}{|c|}{} & width\_8to16 & \multicolumn{1}{c|}{60\%} & 0\% & \multicolumn{1}{c|}{100\%} & 0\% \\ \hline
\multicolumn{1}{|c|}{\multirow{11}{*}{Arithmetic}} & accu & \multicolumn{1}{c|}{0\%} & 0\% & \multicolumn{1}{c|}{0\%} & 0\% \\ \cline{2-6} 
\multicolumn{1}{|c|}{} & adder\_16bit & \multicolumn{1}{c|}{40\%} & 20\% & \multicolumn{1}{c|}{20\%} & 0\% \\ \cline{2-6} 
\multicolumn{1}{|c|}{} & adder\_16bit\_csa & \multicolumn{1}{c|}{80\%} & 80\% & \multicolumn{1}{c|}{0\%} & 0\% \\ \cline{2-6} 
\multicolumn{1}{|c|}{} & adder\_32bit & \multicolumn{1}{c|}{80\%} & 0\% & \multicolumn{1}{c|}{0\%} & 0\% \\ \cline{2-6} 
\multicolumn{1}{|c|}{} & adder\_64bit & \multicolumn{1}{c|}{40\%} & 0\% & \multicolumn{1}{c|}{40\%} & 0\% \\ \cline{2-6} 
\multicolumn{1}{|c|}{} & adder\_8bit & \multicolumn{1}{c|}{80\%} & 40\% & \multicolumn{1}{c|}{40\%} & 40\% \\ \cline{2-6} 
\multicolumn{1}{|c|}{} & div\_16bit & \multicolumn{1}{c|}{0\%} & 0\% & \multicolumn{1}{c|}{0\%} & 0\% \\ \cline{2-6} 
\multicolumn{1}{|c|}{} & multi\_16bit & \multicolumn{1}{c|}{80\%} & 0\% & \multicolumn{1}{c|}{80\%} & 0\% \\ \cline{2-6} 
\multicolumn{1}{|c|}{} & multi\_booth & \multicolumn{1}{c|}{20\%} & 0\% & \multicolumn{1}{c|}{20\%} & 0\% \\ \cline{2-6} 
\multicolumn{1}{|c|}{} & multi\_pipe\_4bit & \multicolumn{1}{c|}{60\%} & 20\% & \multicolumn{1}{c|}{80\%} & 20\% \\ \cline{2-6} 
\multicolumn{1}{|c|}{} & multi\_pipe\_8bit & \multicolumn{1}{c|}{0\%} & 0\% & \multicolumn{1}{c|}{0\%} & 0\% \\ \hline
\multicolumn{1}{|c|}{\multirow{10}{*}{Advanced}} & 1x2nocpe & \multicolumn{1}{c|}{40\%} & 40\% & \multicolumn{1}{c|}{100\%} & 100\% \\ \cline{2-6} 
\multicolumn{1}{|c|}{} & 1x4systolic & \multicolumn{1}{c|}{100\%} & 100\% & \multicolumn{1}{c|}{20\%} & 20\% \\ \cline{2-6} 
\multicolumn{1}{|c|}{} & 2x2systolic & \multicolumn{1}{c|}{0\%} & 0\% & \multicolumn{1}{c|}{0\%} & 0\% \\ \cline{2-6} 
\multicolumn{1}{|c|}{} & 4x4spatialacc & \multicolumn{1}{c|}{0\%} & 0\% & \multicolumn{1}{c|}{0\%} & 0\% \\ \cline{2-6} 
\multicolumn{1}{|c|}{} & fsm & \multicolumn{1}{c|}{100\%} & 60\% & \multicolumn{1}{c|}{80\%} & 20\% \\ \cline{2-6} 
\multicolumn{1}{|c|}{} & macpe & \multicolumn{1}{c|}{0\%} & 0\% & \multicolumn{1}{c|}{0\%} & 0\% \\ \cline{2-6} 
\multicolumn{1}{|c|}{} & 5state\_fsm & \multicolumn{1}{c|}{60\%} & 40\% & \multicolumn{1}{c|}{80\%} & 0\% \\ \cline{2-6} 
\multicolumn{1}{|c|}{} & 3state\_fsm & \multicolumn{1}{c|}{80\%} & 0\% & \multicolumn{1}{c|}{80\%} & 20\% \\ \cline{2-6} 
\multicolumn{1}{|c|}{} & 4state\_fsm & \multicolumn{1}{c|}{80\%} & 0\% & \multicolumn{1}{c|}{80\%} & 20\% \\ \cline{2-6} 
\multicolumn{1}{|c|}{} & 2state\_fsm & \multicolumn{1}{c|}{20\%} & 0\% & \multicolumn{1}{c|}{60\%} & 0\% \\ \hline
\multicolumn{2}{|c|}{Success Rate} & \multicolumn{1}{c|}{48.39\%} & 20.00\% & \multicolumn{1}{c|}{50.97\%} & 12.26\% \\ \hline
\multicolumn{2}{|c|}{Pass @ 1} & \multicolumn{1}{c|}{41.94\%} & 16.13\% & \multicolumn{1}{c|}{48.39\%} & 9.68\% \\ \hline
\multicolumn{2}{|c|}{Pass @ 5} & \multicolumn{1}{c|}{77.42\%} & 35.48\% & \multicolumn{1}{c|}{70.97\%} & 32.26\% \\ \hline
\end{tabular}%
}
\end{table}


\begin{table}[!ht]
\centering
\caption{Evaluation results on Verilog generation. Each cell displays the percentage of code samples,
out of five trials, that successfully pass compilation (syntax column) or functional unit tests (function
column). This table presents the performance of decoder-only models fine-tuned on the dataset containing longer Verilog designs.}
\label{tab:decoder_model_with_longer_designs}
\vspace{5pt}
{\tiny
% \resizebox{\columnwidth}{!} & 0\% & \multicolumn{1}{c|}{100\%} & 0\% \\ \cline{2-6} 
\multicolumn{1}{|c|}{} & alu & \multicolumn{1}{c|}{0\%} & 0\% & \multicolumn{1}{c|}{0\%} & 0\% \\ \cline{2-6} 
\multicolumn{1}{|c|}{} & edge\_detect & \multicolumn{1}{c|}{80\%} & 0\% & \multicolumn{1}{c|}{80\%} & 0\% \\ \cline{2-6} 
\multicolumn{1}{|c|}{} & freq\_div & \multicolumn{1}{c|}{100\%} & 0\% & \multicolumn{1}{c|}{100\%} & 0\% \\ \cline{2-6} 
\multicolumn{1}{|c|}{} & mux & \multicolumn{1}{c|}{100\%} & 100\% & \multicolumn{1}{c|}{60\%} & 60\% \\ \cline{2-6} 
\multicolumn{1}{|c|}{} & parallel2serial & \multicolumn{1}{c|}{100\%} & 0\% & \multicolumn{1}{c|}{100\%} & 0\% \\ \cline{2-6} 
\multicolumn{1}{|c|}{} & pulse\_detect & \multicolumn{1}{c|}{80\%} & 40\% & \multicolumn{1}{c|}{60\%} & 40\% \\ \cline{2-6} 
\multicolumn{1}{|c|}{} & right\_shifter & \multicolumn{1}{c|}{80\%} & 80\% & \multicolumn{1}{c|}{40\%} & 40\% \\ \cline{2-6} 
\multicolumn{1}{|c|}{} & serial2parallel & \multicolumn{1}{c|}{100\%} & 0\% & \multicolumn{1}{c|}{100\%} & 0\% \\ \cline{2-6} 
\multicolumn{1}{|c|}{} & width\_8to16 & \multicolumn{1}{c|}{100\%} & 0\% & \multicolumn{1}{c|}{100\%} & 0\% \\ \hline
\multicolumn{1}{|c|}{\multirow{11}{*}{Arithmetic}} & accu & \multicolumn{1}{c|}{100\%} & 0\% & \multicolumn{1}{c|}{100\%} & 0\% \\ \cline{2-6} 
\multicolumn{1}{|c|}{} & adder\_16bit & \multicolumn{1}{c|}{20\%} & 20\% & \multicolumn{1}{c|}{20\%} & 20\% \\ \cline{2-6} 
\multicolumn{1}{|c|}{} & adder\_16bit\_csa & \multicolumn{1}{c|}{20\%} & 20\% & \multicolumn{1}{c|}{20\%} & 20\% \\ \cline{2-6} 
\multicolumn{1}{|c|}{} & adder\_32bit & \multicolumn{1}{c|}{0\%} & 0\% & \multicolumn{1}{c|}{20\%} & 20\% \\ \cline{2-6} 
\multicolumn{1}{|c|}{} & adder\_64bit & \multicolumn{1}{c|}{0\%} & 0\% & \multicolumn{1}{c|}{0\%} & 0\% \\ \cline{2-6} 
\multicolumn{1}{|c|}{} & adder\_8bit & \multicolumn{1}{c|}{80\%} & 20\% & \multicolumn{1}{c|}{60\%} & 20\% \\ \cline{2-6} 
\multicolumn{1}{|c|}{} & div\_16bit & \multicolumn{1}{c|}{20\%} & 0\% & \multicolumn{1}{c|}{0\%} & 0\% \\ \cline{2-6} 
\multicolumn{1}{|c|}{} & multi\_16bit & \multicolumn{1}{c|}{80\%} & 0\% & \multicolumn{1}{c|}{80\%} & 0\% \\ \cline{2-6} 
\multicolumn{1}{|c|}{} & multi\_booth & \multicolumn{1}{c|}{60\%} & 0\% & \multicolumn{1}{c|}{60\%} & 0\% \\ \cline{2-6} 
\multicolumn{1}{|c|}{} & multi\_pipe\_4bit & \multicolumn{1}{c|}{100\%} & 100\% & \multicolumn{1}{c|}{100\%} & 100\% \\ \cline{2-6} 
\multicolumn{1}{|c|}{} & multi\_pipe\_8bit & \multicolumn{1}{c|}{0\%} & 0\% & \multicolumn{1}{c|}{0\%} & 0\% \\ \hline
\multicolumn{1}{|c|}{\multirow{10}{*}{Advanced}} & 1x2nocpe & \multicolumn{1}{c|}{40\%} & 40\% & \multicolumn{1}{c|}{60\%} & 20\% \\ \cline{2-6} 
\multicolumn{1}{|c|}{} & 1x4systolic & \multicolumn{1}{c|}{20\%} & 20\% & \multicolumn{1}{c|}{20\%} & 20\% \\ \cline{2-6} 
\multicolumn{1}{|c|}{} & 2x2systolic & \multicolumn{1}{c|}{0\%} & 0\% & \multicolumn{1}{c|}{0\%} & 0\% \\ \cline{2-6} 
\multicolumn{1}{|c|}{} & 4x4spatialacc & \multicolumn{1}{c|}{0\%} & 0\% & \multicolumn{1}{c|}{0\%} & 0\% \\ \cline{2-6} 
\multicolumn{1}{|c|}{} & fsm & \multicolumn{1}{c|}{100\%} & 100\% & \multicolumn{1}{c|}{100\%} & 100\% \\ \cline{2-6} 
\multicolumn{1}{|c|}{} & macpe & \multicolumn{1}{c|}{0\%} & 0\% & \multicolumn{1}{c|}{0\%} & 0\% \\ \cline{2-6} 
\multicolumn{1}{|c|}{} & 5state\_fsm & \multicolumn{1}{c|}{100\%} & 0\% & \multicolumn{1}{c|}{100\%} & 100\% \\ \cline{2-6} 
\multicolumn{1}{|c|}{} & 3state\_fsm & \multicolumn{1}{c|}{80\%} & 80\% & \multicolumn{1}{c|}{100\%} & 100\% \\ \cline{2-6} 
\multicolumn{1}{|c|}{} & 4state\_fsm & \multicolumn{1}{c|}{100\%} & 0\% & \multicolumn{1}{c|}{100\%} & 0\% \\ \cline{2-6} 
\multicolumn{1}{|c|}{} & 2state\_fsm & \multicolumn{1}{c|}{100\%} & 20\% & \multicolumn{1}{c|}{100\%} & 20\% \\ \hline
\multicolumn{2}{|c|}{Success Rate} & \multicolumn{1}{c|}{60.00\%} & 20.65\% & \multicolumn{1}{c|}{57.42\%} & 21.94\% \\ \hline
\multicolumn{2}{|c|}{Pass @ 1} & \multicolumn{1}{c|}{38.71\%} & 19.35\% & \multicolumn{1}{c|}{38.71\%} & 19.35\% \\ \hline
\multicolumn{2}{|c|}{Pass @ 5} & \multicolumn{1}{c|}{77.42\%} & 38.71\% & \multicolumn{1}{c|}{77.42\%} & 45.16\% \\ \hline
\end{tabular}%
}
\end{table}


\section{Negative Impact of Incorporating Verilog Designs Exceeding $2048$ Tokens}
\label{appendix:negative_impact}
Notably, the maximum input length for DeepSeek-coder is 16k tokens, while for LLaMA-3.2, it is 128k tokens. To assess the potential negative impact of including Verilog designs exceeding $2048$ tokens, we conduct an ablation study in which we do not exclude such modules for these two models and instead use the dataset containing longer designs for training. As shown in Table~\ref{tab:decoder_model_with_longer_designs}, and by comparing the results in Table~\ref{tab:understanding_additional}, the performance of the fine-tuned models on both Verilog understanding and generation tasks significantly degrades compared to the results in Table~\ref{tab:decoder_compare}, where these models are fine-tuned using the same dataset as DeepRTL. This further validates the rationale behind our decision to exclude Verilog modules and blocks exceeding $2048$ tokens.


\section{Additional Experiments Investigating the Impact of Varying Context Window Lengths}
\label{appendix:varying_context_window_length}
To address concerns regarding the potential bias introduced by excluding examples longer than $2048$ tokens, we investigate the impact of context window length. Specifically, we exclude all Verilog modules exceeding $512$ tokens and use the truncated dataset to train a new model, DeepRTL-220m-512 utilizing the curriculum learning strategy, which has a maximum input length of $512$ tokens. We then evaluate both DeepRTL-220m-512 and DeepRTL-220m on Verilog understanding benchmark samples, where the module lengths are below $512$ tokens, and present the results in Table~\ref{tab:understanding_additional}. For the generation task, DeepRTL-220m-512 shows near-zero performance, with nearly 0\% accuracy for both syntax and functional correctness. This result refutes the concern that ``a model accommodating longer context windows could potentially offer superior performance on the general task, but not for this tailored dataset," as it does not hold true in our case.

\end{document}
