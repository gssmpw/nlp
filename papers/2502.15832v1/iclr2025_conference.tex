
\documentclass{article} % For LaTeX2e
\usepackage{iclr2025_conference,times}

% Optional math commands from https://github.com/goodfeli/dlbook_notation.
%%%%% NEW MATH DEFINITIONS %%%%%

\usepackage{amsmath,amsfonts,bm}
\usepackage{derivative}
% Mark sections of captions for referring to divisions of figures
\newcommand{\figleft}{{\em (Left)}}
\newcommand{\figcenter}{{\em (Center)}}
\newcommand{\figright}{{\em (Right)}}
\newcommand{\figtop}{{\em (Top)}}
\newcommand{\figbottom}{{\em (Bottom)}}
\newcommand{\captiona}{{\em (a)}}
\newcommand{\captionb}{{\em (b)}}
\newcommand{\captionc}{{\em (c)}}
\newcommand{\captiond}{{\em (d)}}

% Highlight a newly defined term
\newcommand{\newterm}[1]{{\bf #1}}

% Derivative d 
\newcommand{\deriv}{{\mathrm{d}}}

% Figure reference, lower-case.
\def\figref#1{figure~\ref{#1}}
% Figure reference, capital. For start of sentence
\def\Figref#1{Figure~\ref{#1}}
\def\twofigref#1#2{figures \ref{#1} and \ref{#2}}
\def\quadfigref#1#2#3#4{figures \ref{#1}, \ref{#2}, \ref{#3} and \ref{#4}}
% Section reference, lower-case.
\def\secref#1{section~\ref{#1}}
% Section reference, capital.
\def\Secref#1{Section~\ref{#1}}
% Reference to two sections.
\def\twosecrefs#1#2{sections \ref{#1} and \ref{#2}}
% Reference to three sections.
\def\secrefs#1#2#3{sections \ref{#1}, \ref{#2} and \ref{#3}}
% Reference to an equation, lower-case.
\def\eqref#1{equation~\ref{#1}}
% Reference to an equation, upper case
\def\Eqref#1{Equation~\ref{#1}}
% A raw reference to an equation---avoid using if possible
\def\plaineqref#1{\ref{#1}}
% Reference to a chapter, lower-case.
\def\chapref#1{chapter~\ref{#1}}
% Reference to an equation, upper case.
\def\Chapref#1{Chapter~\ref{#1}}
% Reference to a range of chapters
\def\rangechapref#1#2{chapters\ref{#1}--\ref{#2}}
% Reference to an algorithm, lower-case.
\def\algref#1{algorithm~\ref{#1}}
% Reference to an algorithm, upper case.
\def\Algref#1{Algorithm~\ref{#1}}
\def\twoalgref#1#2{algorithms \ref{#1} and \ref{#2}}
\def\Twoalgref#1#2{Algorithms \ref{#1} and \ref{#2}}
% Reference to a part, lower case
\def\partref#1{part~\ref{#1}}
% Reference to a part, upper case
\def\Partref#1{Part~\ref{#1}}
\def\twopartref#1#2{parts \ref{#1} and \ref{#2}}

\def\ceil#1{\lceil #1 \rceil}
\def\floor#1{\lfloor #1 \rfloor}
\def\1{\bm{1}}
\newcommand{\train}{\mathcal{D}}
\newcommand{\valid}{\mathcal{D_{\mathrm{valid}}}}
\newcommand{\test}{\mathcal{D_{\mathrm{test}}}}

\def\eps{{\epsilon}}


% Random variables
\def\reta{{\textnormal{$\eta$}}}
\def\ra{{\textnormal{a}}}
\def\rb{{\textnormal{b}}}
\def\rc{{\textnormal{c}}}
\def\rd{{\textnormal{d}}}
\def\re{{\textnormal{e}}}
\def\rf{{\textnormal{f}}}
\def\rg{{\textnormal{g}}}
\def\rh{{\textnormal{h}}}
\def\ri{{\textnormal{i}}}
\def\rj{{\textnormal{j}}}
\def\rk{{\textnormal{k}}}
\def\rl{{\textnormal{l}}}
% rm is already a command, just don't name any random variables m
\def\rn{{\textnormal{n}}}
\def\ro{{\textnormal{o}}}
\def\rp{{\textnormal{p}}}
\def\rq{{\textnormal{q}}}
\def\rr{{\textnormal{r}}}
\def\rs{{\textnormal{s}}}
\def\rt{{\textnormal{t}}}
\def\ru{{\textnormal{u}}}
\def\rv{{\textnormal{v}}}
\def\rw{{\textnormal{w}}}
\def\rx{{\textnormal{x}}}
\def\ry{{\textnormal{y}}}
\def\rz{{\textnormal{z}}}

% Random vectors
\def\rvepsilon{{\mathbf{\epsilon}}}
\def\rvphi{{\mathbf{\phi}}}
\def\rvtheta{{\mathbf{\theta}}}
\def\rva{{\mathbf{a}}}
\def\rvb{{\mathbf{b}}}
\def\rvc{{\mathbf{c}}}
\def\rvd{{\mathbf{d}}}
\def\rve{{\mathbf{e}}}
\def\rvf{{\mathbf{f}}}
\def\rvg{{\mathbf{g}}}
\def\rvh{{\mathbf{h}}}
\def\rvu{{\mathbf{i}}}
\def\rvj{{\mathbf{j}}}
\def\rvk{{\mathbf{k}}}
\def\rvl{{\mathbf{l}}}
\def\rvm{{\mathbf{m}}}
\def\rvn{{\mathbf{n}}}
\def\rvo{{\mathbf{o}}}
\def\rvp{{\mathbf{p}}}
\def\rvq{{\mathbf{q}}}
\def\rvr{{\mathbf{r}}}
\def\rvs{{\mathbf{s}}}
\def\rvt{{\mathbf{t}}}
\def\rvu{{\mathbf{u}}}
\def\rvv{{\mathbf{v}}}
\def\rvw{{\mathbf{w}}}
\def\rvx{{\mathbf{x}}}
\def\rvy{{\mathbf{y}}}
\def\rvz{{\mathbf{z}}}

% Elements of random vectors
\def\erva{{\textnormal{a}}}
\def\ervb{{\textnormal{b}}}
\def\ervc{{\textnormal{c}}}
\def\ervd{{\textnormal{d}}}
\def\erve{{\textnormal{e}}}
\def\ervf{{\textnormal{f}}}
\def\ervg{{\textnormal{g}}}
\def\ervh{{\textnormal{h}}}
\def\ervi{{\textnormal{i}}}
\def\ervj{{\textnormal{j}}}
\def\ervk{{\textnormal{k}}}
\def\ervl{{\textnormal{l}}}
\def\ervm{{\textnormal{m}}}
\def\ervn{{\textnormal{n}}}
\def\ervo{{\textnormal{o}}}
\def\ervp{{\textnormal{p}}}
\def\ervq{{\textnormal{q}}}
\def\ervr{{\textnormal{r}}}
\def\ervs{{\textnormal{s}}}
\def\ervt{{\textnormal{t}}}
\def\ervu{{\textnormal{u}}}
\def\ervv{{\textnormal{v}}}
\def\ervw{{\textnormal{w}}}
\def\ervx{{\textnormal{x}}}
\def\ervy{{\textnormal{y}}}
\def\ervz{{\textnormal{z}}}

% Random matrices
\def\rmA{{\mathbf{A}}}
\def\rmB{{\mathbf{B}}}
\def\rmC{{\mathbf{C}}}
\def\rmD{{\mathbf{D}}}
\def\rmE{{\mathbf{E}}}
\def\rmF{{\mathbf{F}}}
\def\rmG{{\mathbf{G}}}
\def\rmH{{\mathbf{H}}}
\def\rmI{{\mathbf{I}}}
\def\rmJ{{\mathbf{J}}}
\def\rmK{{\mathbf{K}}}
\def\rmL{{\mathbf{L}}}
\def\rmM{{\mathbf{M}}}
\def\rmN{{\mathbf{N}}}
\def\rmO{{\mathbf{O}}}
\def\rmP{{\mathbf{P}}}
\def\rmQ{{\mathbf{Q}}}
\def\rmR{{\mathbf{R}}}
\def\rmS{{\mathbf{S}}}
\def\rmT{{\mathbf{T}}}
\def\rmU{{\mathbf{U}}}
\def\rmV{{\mathbf{V}}}
\def\rmW{{\mathbf{W}}}
\def\rmX{{\mathbf{X}}}
\def\rmY{{\mathbf{Y}}}
\def\rmZ{{\mathbf{Z}}}

% Elements of random matrices
\def\ermA{{\textnormal{A}}}
\def\ermB{{\textnormal{B}}}
\def\ermC{{\textnormal{C}}}
\def\ermD{{\textnormal{D}}}
\def\ermE{{\textnormal{E}}}
\def\ermF{{\textnormal{F}}}
\def\ermG{{\textnormal{G}}}
\def\ermH{{\textnormal{H}}}
\def\ermI{{\textnormal{I}}}
\def\ermJ{{\textnormal{J}}}
\def\ermK{{\textnormal{K}}}
\def\ermL{{\textnormal{L}}}
\def\ermM{{\textnormal{M}}}
\def\ermN{{\textnormal{N}}}
\def\ermO{{\textnormal{O}}}
\def\ermP{{\textnormal{P}}}
\def\ermQ{{\textnormal{Q}}}
\def\ermR{{\textnormal{R}}}
\def\ermS{{\textnormal{S}}}
\def\ermT{{\textnormal{T}}}
\def\ermU{{\textnormal{U}}}
\def\ermV{{\textnormal{V}}}
\def\ermW{{\textnormal{W}}}
\def\ermX{{\textnormal{X}}}
\def\ermY{{\textnormal{Y}}}
\def\ermZ{{\textnormal{Z}}}

% Vectors
\def\vzero{{\bm{0}}}
\def\vone{{\bm{1}}}
\def\vmu{{\bm{\mu}}}
\def\vtheta{{\bm{\theta}}}
\def\vphi{{\bm{\phi}}}
\def\va{{\bm{a}}}
\def\vb{{\bm{b}}}
\def\vc{{\bm{c}}}
\def\vd{{\bm{d}}}
\def\ve{{\bm{e}}}
\def\vf{{\bm{f}}}
\def\vg{{\bm{g}}}
\def\vh{{\bm{h}}}
\def\vi{{\bm{i}}}
\def\vj{{\bm{j}}}
\def\vk{{\bm{k}}}
\def\vl{{\bm{l}}}
\def\vm{{\bm{m}}}
\def\vn{{\bm{n}}}
\def\vo{{\bm{o}}}
\def\vp{{\bm{p}}}
\def\vq{{\bm{q}}}
\def\vr{{\bm{r}}}
\def\vs{{\bm{s}}}
\def\vt{{\bm{t}}}
\def\vu{{\bm{u}}}
\def\vv{{\bm{v}}}
\def\vw{{\bm{w}}}
\def\vx{{\bm{x}}}
\def\vy{{\bm{y}}}
\def\vz{{\bm{z}}}

% Elements of vectors
\def\evalpha{{\alpha}}
\def\evbeta{{\beta}}
\def\evepsilon{{\epsilon}}
\def\evlambda{{\lambda}}
\def\evomega{{\omega}}
\def\evmu{{\mu}}
\def\evpsi{{\psi}}
\def\evsigma{{\sigma}}
\def\evtheta{{\theta}}
\def\eva{{a}}
\def\evb{{b}}
\def\evc{{c}}
\def\evd{{d}}
\def\eve{{e}}
\def\evf{{f}}
\def\evg{{g}}
\def\evh{{h}}
\def\evi{{i}}
\def\evj{{j}}
\def\evk{{k}}
\def\evl{{l}}
\def\evm{{m}}
\def\evn{{n}}
\def\evo{{o}}
\def\evp{{p}}
\def\evq{{q}}
\def\evr{{r}}
\def\evs{{s}}
\def\evt{{t}}
\def\evu{{u}}
\def\evv{{v}}
\def\evw{{w}}
\def\evx{{x}}
\def\evy{{y}}
\def\evz{{z}}

% Matrix
\def\mA{{\bm{A}}}
\def\mB{{\bm{B}}}
\def\mC{{\bm{C}}}
\def\mD{{\bm{D}}}
\def\mE{{\bm{E}}}
\def\mF{{\bm{F}}}
\def\mG{{\bm{G}}}
\def\mH{{\bm{H}}}
\def\mI{{\bm{I}}}
\def\mJ{{\bm{J}}}
\def\mK{{\bm{K}}}
\def\mL{{\bm{L}}}
\def\mM{{\bm{M}}}
\def\mN{{\bm{N}}}
\def\mO{{\bm{O}}}
\def\mP{{\bm{P}}}
\def\mQ{{\bm{Q}}}
\def\mR{{\bm{R}}}
\def\mS{{\bm{S}}}
\def\mT{{\bm{T}}}
\def\mU{{\bm{U}}}
\def\mV{{\bm{V}}}
\def\mW{{\bm{W}}}
\def\mX{{\bm{X}}}
\def\mY{{\bm{Y}}}
\def\mZ{{\bm{Z}}}
\def\mBeta{{\bm{\beta}}}
\def\mPhi{{\bm{\Phi}}}
\def\mLambda{{\bm{\Lambda}}}
\def\mSigma{{\bm{\Sigma}}}

% Tensor
\DeclareMathAlphabet{\mathsfit}{\encodingdefault}{\sfdefault}{m}{sl}
\SetMathAlphabet{\mathsfit}{bold}{\encodingdefault}{\sfdefault}{bx}{n}
\newcommand{\tens}[1]{\bm{\mathsfit{#1}}}
\def\tA{{\tens{A}}}
\def\tB{{\tens{B}}}
\def\tC{{\tens{C}}}
\def\tD{{\tens{D}}}
\def\tE{{\tens{E}}}
\def\tF{{\tens{F}}}
\def\tG{{\tens{G}}}
\def\tH{{\tens{H}}}
\def\tI{{\tens{I}}}
\def\tJ{{\tens{J}}}
\def\tK{{\tens{K}}}
\def\tL{{\tens{L}}}
\def\tM{{\tens{M}}}
\def\tN{{\tens{N}}}
\def\tO{{\tens{O}}}
\def\tP{{\tens{P}}}
\def\tQ{{\tens{Q}}}
\def\tR{{\tens{R}}}
\def\tS{{\tens{S}}}
\def\tT{{\tens{T}}}
\def\tU{{\tens{U}}}
\def\tV{{\tens{V}}}
\def\tW{{\tens{W}}}
\def\tX{{\tens{X}}}
\def\tY{{\tens{Y}}}
\def\tZ{{\tens{Z}}}


% Graph
\def\gA{{\mathcal{A}}}
\def\gB{{\mathcal{B}}}
\def\gC{{\mathcal{C}}}
\def\gD{{\mathcal{D}}}
\def\gE{{\mathcal{E}}}
\def\gF{{\mathcal{F}}}
\def\gG{{\mathcal{G}}}
\def\gH{{\mathcal{H}}}
\def\gI{{\mathcal{I}}}
\def\gJ{{\mathcal{J}}}
\def\gK{{\mathcal{K}}}
\def\gL{{\mathcal{L}}}
\def\gM{{\mathcal{M}}}
\def\gN{{\mathcal{N}}}
\def\gO{{\mathcal{O}}}
\def\gP{{\mathcal{P}}}
\def\gQ{{\mathcal{Q}}}
\def\gR{{\mathcal{R}}}
\def\gS{{\mathcal{S}}}
\def\gT{{\mathcal{T}}}
\def\gU{{\mathcal{U}}}
\def\gV{{\mathcal{V}}}
\def\gW{{\mathcal{W}}}
\def\gX{{\mathcal{X}}}
\def\gY{{\mathcal{Y}}}
\def\gZ{{\mathcal{Z}}}

% Sets
\def\sA{{\mathbb{A}}}
\def\sB{{\mathbb{B}}}
\def\sC{{\mathbb{C}}}
\def\sD{{\mathbb{D}}}
% Don't use a set called E, because this would be the same as our symbol
% for expectation.
\def\sF{{\mathbb{F}}}
\def\sG{{\mathbb{G}}}
\def\sH{{\mathbb{H}}}
\def\sI{{\mathbb{I}}}
\def\sJ{{\mathbb{J}}}
\def\sK{{\mathbb{K}}}
\def\sL{{\mathbb{L}}}
\def\sM{{\mathbb{M}}}
\def\sN{{\mathbb{N}}}
\def\sO{{\mathbb{O}}}
\def\sP{{\mathbb{P}}}
\def\sQ{{\mathbb{Q}}}
\def\sR{{\mathbb{R}}}
\def\sS{{\mathbb{S}}}
\def\sT{{\mathbb{T}}}
\def\sU{{\mathbb{U}}}
\def\sV{{\mathbb{V}}}
\def\sW{{\mathbb{W}}}
\def\sX{{\mathbb{X}}}
\def\sY{{\mathbb{Y}}}
\def\sZ{{\mathbb{Z}}}

% Entries of a matrix
\def\emLambda{{\Lambda}}
\def\emA{{A}}
\def\emB{{B}}
\def\emC{{C}}
\def\emD{{D}}
\def\emE{{E}}
\def\emF{{F}}
\def\emG{{G}}
\def\emH{{H}}
\def\emI{{I}}
\def\emJ{{J}}
\def\emK{{K}}
\def\emL{{L}}
\def\emM{{M}}
\def\emN{{N}}
\def\emO{{O}}
\def\emP{{P}}
\def\emQ{{Q}}
\def\emR{{R}}
\def\emS{{S}}
\def\emT{{T}}
\def\emU{{U}}
\def\emV{{V}}
\def\emW{{W}}
\def\emX{{X}}
\def\emY{{Y}}
\def\emZ{{Z}}
\def\emSigma{{\Sigma}}

% entries of a tensor
% Same font as tensor, without \bm wrapper
\newcommand{\etens}[1]{\mathsfit{#1}}
\def\etLambda{{\etens{\Lambda}}}
\def\etA{{\etens{A}}}
\def\etB{{\etens{B}}}
\def\etC{{\etens{C}}}
\def\etD{{\etens{D}}}
\def\etE{{\etens{E}}}
\def\etF{{\etens{F}}}
\def\etG{{\etens{G}}}
\def\etH{{\etens{H}}}
\def\etI{{\etens{I}}}
\def\etJ{{\etens{J}}}
\def\etK{{\etens{K}}}
\def\etL{{\etens{L}}}
\def\etM{{\etens{M}}}
\def\etN{{\etens{N}}}
\def\etO{{\etens{O}}}
\def\etP{{\etens{P}}}
\def\etQ{{\etens{Q}}}
\def\etR{{\etens{R}}}
\def\etS{{\etens{S}}}
\def\etT{{\etens{T}}}
\def\etU{{\etens{U}}}
\def\etV{{\etens{V}}}
\def\etW{{\etens{W}}}
\def\etX{{\etens{X}}}
\def\etY{{\etens{Y}}}
\def\etZ{{\etens{Z}}}

% The true underlying data generating distribution
\newcommand{\pdata}{p_{\rm{data}}}
\newcommand{\ptarget}{p_{\rm{target}}}
\newcommand{\pprior}{p_{\rm{prior}}}
\newcommand{\pbase}{p_{\rm{base}}}
\newcommand{\pref}{p_{\rm{ref}}}

% The empirical distribution defined by the training set
\newcommand{\ptrain}{\hat{p}_{\rm{data}}}
\newcommand{\Ptrain}{\hat{P}_{\rm{data}}}
% The model distribution
\newcommand{\pmodel}{p_{\rm{model}}}
\newcommand{\Pmodel}{P_{\rm{model}}}
\newcommand{\ptildemodel}{\tilde{p}_{\rm{model}}}
% Stochastic autoencoder distributions
\newcommand{\pencode}{p_{\rm{encoder}}}
\newcommand{\pdecode}{p_{\rm{decoder}}}
\newcommand{\precons}{p_{\rm{reconstruct}}}

\newcommand{\laplace}{\mathrm{Laplace}} % Laplace distribution

\newcommand{\E}{\mathbb{E}}
\newcommand{\Ls}{\mathcal{L}}
\newcommand{\R}{\mathbb{R}}
\newcommand{\emp}{\tilde{p}}
\newcommand{\lr}{\alpha}
\newcommand{\reg}{\lambda}
\newcommand{\rect}{\mathrm{rectifier}}
\newcommand{\softmax}{\mathrm{softmax}}
\newcommand{\sigmoid}{\sigma}
\newcommand{\softplus}{\zeta}
\newcommand{\KL}{D_{\mathrm{KL}}}
\newcommand{\Var}{\mathrm{Var}}
\newcommand{\standarderror}{\mathrm{SE}}
\newcommand{\Cov}{\mathrm{Cov}}
% Wolfram Mathworld says $L^2$ is for function spaces and $\ell^2$ is for vectors
% But then they seem to use $L^2$ for vectors throughout the site, and so does
% wikipedia.
\newcommand{\normlzero}{L^0}
\newcommand{\normlone}{L^1}
\newcommand{\normltwo}{L^2}
\newcommand{\normlp}{L^p}
\newcommand{\normmax}{L^\infty}

\newcommand{\parents}{Pa} % See usage in notation.tex. Chosen to match Daphne's book.

\DeclareMathOperator*{\argmax}{arg\,max}
\DeclareMathOperator*{\argmin}{arg\,min}

\DeclareMathOperator{\sign}{sign}
\DeclareMathOperator{\Tr}{Tr}
\let\ab\allowbreak


\usepackage{hyperref}
\usepackage{url}
\usepackage{pdfpages}
\usepackage{xcolor}
\usepackage{booktabs}
\usepackage{multirow}
\usepackage{graphicx}
\usepackage{nth}

\definecolor{changran}{RGB}{255,100,0} % Orange color
\definecolor{yunhao}{RGB}{0,0,255} % Green color


\title{DeepRTL: Bridging Verilog Understanding and Generation with a Unified Representation Model}

% Authors must not appear in the submitted version. They should be hidden
% as long as the \iclrfinalcopy macro remains commented out below.
% Non-anonymous submissions will be rejected without review.

\author{Yi Liu, Changran Xu, Yunhao Zhou, Zeju Li, Qiang Xu \\
Department of Computer Science and Engineering\\
The Chinese University of Hong Kong\\
\texttt{\{yliu22,zjli24,qxu\}@cse.cuhk.edu.hk}\\
\texttt{\{xxuchangran,yunhaoz.cs\}@gmail.com}
}

% The \author macro works with any number of authors. There are two commands
% used to separate the names and addresses of multiple authors: \And and \AND.
%
% Using \And between authors leaves it to \LaTeX{} to determine where to break
% the lines. Using \AND forces a linebreak at that point. So, if \LaTeX{}
% puts 3 of 4 authors names on the first line, and the last on the second
% line, try using \AND instead of \And before the third author name.

\newcommand{\fix}{\marginpar{FIX}}
\newcommand{\new}{\marginpar{NEW}}

\iclrfinalcopy % Uncomment for camera-ready version, but NOT for submission.
\begin{document}
\maketitle

\begin{abstract}


The choice of representation for geographic location significantly impacts the accuracy of models for a broad range of geospatial tasks, including fine-grained species classification, population density estimation, and biome classification. Recent works like SatCLIP and GeoCLIP learn such representations by contrastively aligning geolocation with co-located images. While these methods work exceptionally well, in this paper, we posit that the current training strategies fail to fully capture the important visual features. We provide an information theoretic perspective on why the resulting embeddings from these methods discard crucial visual information that is important for many downstream tasks. To solve this problem, we propose a novel retrieval-augmented strategy called RANGE. We build our method on the intuition that the visual features of a location can be estimated by combining the visual features from multiple similar-looking locations. We evaluate our method across a wide variety of tasks. Our results show that RANGE outperforms the existing state-of-the-art models with significant margins in most tasks. We show gains of up to 13.1\% on classification tasks and 0.145 $R^2$ on regression tasks. All our code and models will be made available at: \href{https://github.com/mvrl/RANGE}{https://github.com/mvrl/RANGE}.

\end{abstract}


\section{Introduction}
Backdoor attacks pose a concealed yet profound security risk to machine learning (ML) models, for which the adversaries can inject a stealth backdoor into the model during training, enabling them to illicitly control the model's output upon encountering predefined inputs. These attacks can even occur without the knowledge of developers or end-users, thereby undermining the trust in ML systems. As ML becomes more deeply embedded in critical sectors like finance, healthcare, and autonomous driving \citep{he2016deep, liu2020computing, tournier2019mrtrix3, adjabi2020past}, the potential damage from backdoor attacks grows, underscoring the emergency for developing robust defense mechanisms against backdoor attacks.

To address the threat of backdoor attacks, researchers have developed a variety of strategies \cite{liu2018fine,wu2021adversarial,wang2019neural,zeng2022adversarial,zhu2023neural,Zhu_2023_ICCV, wei2024shared,wei2024d3}, aimed at purifying backdoors within victim models. These methods are designed to integrate with current deployment workflows seamlessly and have demonstrated significant success in mitigating the effects of backdoor triggers \cite{wubackdoorbench, wu2023defenses, wu2024backdoorbench,dunnett2024countering}.  However, most state-of-the-art (SOTA) backdoor purification methods operate under the assumption that a small clean dataset, often referred to as \textbf{auxiliary dataset}, is available for purification. Such an assumption poses practical challenges, especially in scenarios where data is scarce. To tackle this challenge, efforts have been made to reduce the size of the required auxiliary dataset~\cite{chai2022oneshot,li2023reconstructive, Zhu_2023_ICCV} and even explore dataset-free purification techniques~\cite{zheng2022data,hong2023revisiting,lin2024fusing}. Although these approaches offer some improvements, recent evaluations \cite{dunnett2024countering, wu2024backdoorbench} continue to highlight the importance of sufficient auxiliary data for achieving robust defenses against backdoor attacks.

While significant progress has been made in reducing the size of auxiliary datasets, an equally critical yet underexplored question remains: \emph{how does the nature of the auxiliary dataset affect purification effectiveness?} In  real-world  applications, auxiliary datasets can vary widely, encompassing in-distribution data, synthetic data, or external data from different sources. Understanding how each type of auxiliary dataset influences the purification effectiveness is vital for selecting or constructing the most suitable auxiliary dataset and the corresponding technique. For instance, when multiple datasets are available, understanding how different datasets contribute to purification can guide defenders in selecting or crafting the most appropriate dataset. Conversely, when only limited auxiliary data is accessible, knowing which purification technique works best under those constraints is critical. Therefore, there is an urgent need for a thorough investigation into the impact of auxiliary datasets on purification effectiveness to guide defenders in  enhancing the security of ML systems. 

In this paper, we systematically investigate the critical role of auxiliary datasets in backdoor purification, aiming to bridge the gap between idealized and practical purification scenarios.  Specifically, we first construct a diverse set of auxiliary datasets to emulate real-world conditions, as summarized in Table~\ref{overall}. These datasets include in-distribution data, synthetic data, and external data from other sources. Through an evaluation of SOTA backdoor purification methods across these datasets, we uncover several critical insights: \textbf{1)} In-distribution datasets, particularly those carefully filtered from the original training data of the victim model, effectively preserve the model’s utility for its intended tasks but may fall short in eliminating backdoors. \textbf{2)} Incorporating OOD datasets can help the model forget backdoors but also bring the risk of forgetting critical learned knowledge, significantly degrading its overall performance. Building on these findings, we propose Guided Input Calibration (GIC), a novel technique that enhances backdoor purification by adaptively transforming auxiliary data to better align with the victim model’s learned representations. By leveraging the victim model itself to guide this transformation, GIC optimizes the purification process, striking a balance between preserving model utility and mitigating backdoor threats. Extensive experiments demonstrate that GIC significantly improves the effectiveness of backdoor purification across diverse auxiliary datasets, providing a practical and robust defense solution.

Our main contributions are threefold:
\textbf{1) Impact analysis of auxiliary datasets:} We take the \textbf{first step}  in systematically investigating how different types of auxiliary datasets influence backdoor purification effectiveness. Our findings provide novel insights and serve as a foundation for future research on optimizing dataset selection and construction for enhanced backdoor defense.
%
\textbf{2) Compilation and evaluation of diverse auxiliary datasets:}  We have compiled and rigorously evaluated a diverse set of auxiliary datasets using SOTA purification methods, making our datasets and code publicly available to facilitate and support future research on practical backdoor defense strategies.
%
\textbf{3) Introduction of GIC:} We introduce GIC, the \textbf{first} dedicated solution designed to align auxiliary datasets with the model’s learned representations, significantly enhancing backdoor mitigation across various dataset types. Our approach sets a new benchmark for practical and effective backdoor defense.



\section{Related Work}

\subsection{Large 3D Reconstruction Models}
Recently, generalized feed-forward models for 3D reconstruction from sparse input views have garnered considerable attention due to their applicability in heavily under-constrained scenarios. The Large Reconstruction Model (LRM)~\cite{hong2023lrm} uses a transformer-based encoder-decoder pipeline to infer a NeRF reconstruction from just a single image. Newer iterations have shifted the focus towards generating 3D Gaussian representations from four input images~\cite{tang2025lgm, xu2024grm, zhang2025gslrm, charatan2024pixelsplat, chen2025mvsplat, liu2025mvsgaussian}, showing remarkable novel view synthesis results. The paradigm of transformer-based sparse 3D reconstruction has also successfully been applied to lifting monocular videos to 4D~\cite{ren2024l4gm}. \\
Yet, none of the existing works in the domain have studied the use-case of inferring \textit{animatable} 3D representations from sparse input images, which is the focus of our work. To this end, we build on top of the Large Gaussian Reconstruction Model (GRM)~\cite{xu2024grm}.

\subsection{3D-aware Portrait Animation}
A different line of work focuses on animating portraits in a 3D-aware manner.
MegaPortraits~\cite{drobyshev2022megaportraits} builds a 3D Volume given a source and driving image, and renders the animated source actor via orthographic projection with subsequent 2D neural rendering.
3D morphable models (3DMMs)~\cite{blanz19993dmm} are extensively used to obtain more interpretable control over the portrait animation. For example, StyleRig~\cite{tewari2020stylerig} demonstrates how a 3DMM can be used to control the data generated from a pre-trained StyleGAN~\cite{karras2019stylegan} network. ROME~\cite{khakhulin2022rome} predicts vertex offsets and texture of a FLAME~\cite{li2017flame} mesh from the input image.
A TriPlane representation is inferred and animated via FLAME~\cite{li2017flame} in multiple methods like Portrait4D~\cite{deng2024portrait4d}, Portrait4D-v2~\cite{deng2024portrait4dv2}, and GPAvatar~\cite{chu2024gpavatar}.
Others, such as VOODOO 3D~\cite{tran2024voodoo3d} and VOODOO XP~\cite{tran2024voodooxp}, learn their own expression encoder to drive the source person in a more detailed manner. \\
All of the aforementioned methods require nothing more than a single image of a person to animate it. This allows them to train on large monocular video datasets to infer a very generic motion prior that even translates to paintings or cartoon characters. However, due to their task formulation, these methods mostly focus on image synthesis from a frontal camera, often trading 3D consistency for better image quality by using 2D screen-space neural renderers. In contrast, our work aims to produce a truthful and complete 3D avatar representation from the input images that can be viewed from any angle.  

\subsection{Photo-realistic 3D Face Models}
The increasing availability of large-scale multi-view face datasets~\cite{kirschstein2023nersemble, ava256, pan2024renderme360, yang2020facescape} has enabled building photo-realistic 3D face models that learn a detailed prior over both geometry and appearance of human faces. HeadNeRF~\cite{hong2022headnerf} conditions a Neural Radiance Field (NeRF)~\cite{mildenhall2021nerf} on identity, expression, albedo, and illumination codes. VRMM~\cite{yang2024vrmm} builds a high-quality and relightable 3D face model using volumetric primitives~\cite{lombardi2021mvp}. One2Avatar~\cite{yu2024one2avatar} extends a 3DMM by anchoring a radiance field to its surface. More recently, GPHM~\cite{xu2025gphm} and HeadGAP~\cite{zheng2024headgap} have adopted 3D Gaussians to build a photo-realistic 3D face model. \\
Photo-realistic 3D face models learn a powerful prior over human facial appearance and geometry, which can be fitted to a single or multiple images of a person, effectively inferring a 3D head avatar. However, the fitting procedure itself is non-trivial and often requires expensive test-time optimization, impeding casual use-cases on consumer-grade devices. While this limitation may be circumvented by learning a generalized encoder that maps images into the 3D face model's latent space, another fundamental limitation remains. Even with more multi-view face datasets being published, the number of available training subjects rarely exceeds the thousands, making it hard to truly learn the full distibution of human facial appearance. Instead, our approach avoids generalizing over the identity axis by conditioning on some images of a person, and only generalizes over the expression axis for which plenty of data is available. 

A similar motivation has inspired recent work on codec avatars where a generalized network infers an animatable 3D representation given a registered mesh of a person~\cite{cao2022authentic, li2024uravatar}.
The resulting avatars exhibit excellent quality at the cost of several minutes of video capture per subject and expensive test-time optimization.
For example, URAvatar~\cite{li2024uravatar} finetunes their network on the given video recording for 3 hours on 8 A100 GPUs, making inference on consumer-grade devices impossible. In contrast, our approach directly regresses the final 3D head avatar from just four input images without the need for expensive test-time fine-tuning.


\section{Dataset and Understanding Benchmark}
\begin{figure}[ht]
    \centering
    \includegraphics[width=0.6\linewidth]{fig/cot_v6.pdf}
    \vspace{-12pt}
    \caption{The overview of the data annotation process. We employ the CoT approach and the SOTA LLM, GPT-4, for annotation. Annotations span three levels—line, block, and module—providing both detailed specifications and high-level functional descriptions.}
    \label{fig:cot}
\end{figure}
In this section, we introduce our dataset designed to enhance Verilog understanding and generation, which aligns natural language with Verilog code across line, block, and module levels with detailed and high-level descriptions.
By integrating both open-source and proprietary code, the dataset offers a diverse and robust collection that spans a broad spectrum of hardware design complexities.
% By integrating open-source and proprietary code, we ensure a diverse and robust dataset encompassing a wide range of hardware designs. 
We employ GPT-4 along with the CoT approach for annotation, achieving about $90\%$ accuracy in human evaluations, confirming the dataset's high quality.
We also introduce the first benchmark for Verilog understanding, setting a new standard for evaluating LLMs' capabilities in interpreting Verilog code.





\subsection{Dataset Source}
Our dataset comprises both open-source and proprietary Verilog code. For the open-source part, we gather \texttt{.v} files from GitHub repositories using the keyword \texttt{Verilog}.
These files are segmented into individual modules, each representing a distinct functional unit within the Verilog design.
This segmentation is crucial given the limited context length of current LLMs, improving the efficiency and accuracy of the subsequent annotation and fine-tuning processes.
We employ MinHash and Jaccard similarity metrics~\citep{yan2017privmin} to deduplicate these modules and exclude those predominantly made up of comments or lacking complete \texttt{module} and \texttt{endmodule} structures.
Finally, this process results in a total of 61,755 distinct Verilog modules.
For the proprietary portion, we incorporate a set of purchased intellectual properties (IPs) that enhance the variety and functional diversity of our dataset. This component includes a total of 213 high-quality, industry-standard Verilog modules. These IPs not only offer a range of advanced functions but also provide unique insights that complement the open-source data. Integrating these elements ensures a comprehensive dataset that captures a wide spectrum of hardware design practices.


\subsection{Dataset Annotation}
We employ different annotation strategies for open-source and proprietary code. For open-source code, we utilize the CoT approach with the SOTA LLM, GPT-4, to provide annotations at multiple levels. As illustrated in Figure~\ref{fig:cot}, 
%we begin by removing all comments from the original Verilog module (resulting in refined code) and counting the number of tokens using CodeT5+'s tokenizer. 
we initially remove all comments from the original Verilog code (resulting in refined code) to avoid training complications from incorrect or misleading comments.
If the token count of a complete module exceeds $2048$, the maximum context length for CodeT5+, we utilize GPT-4 to segment the module into smaller, manageable blocks such as \texttt{always} blocks. 
If the resulting blocks still exceed $2048$ tokens, we will discard them. 
For modules and blocks with a token count below $2048$ (qualified code), we then use GPT-4 to add informative comments, resulting in commented code (Step 1).
% If the token count is below $2048$, we treat the module as a complete module and use GPT-4 to add informative comments, resulting in commented code (Step 1). 
% From this commented code, we can extract line-level descriptions. To ensure accuracy, GPT-4 is employed to verify that these line-level descriptions are strictly confined to the context of each individual line, avoiding any external or unrelated information. For example, [insert specific examples here].
From this commented code, we can extract line-level descriptions (pairings of single lines of code with natural language descriptions). To guarantee the accuracy and relevance of the inline comments, we use GPT-4o-mini to rigorously check each comment, ensuring that all line-level descriptions are strictly confined to the context of their respective lines without incorporating any extraneous or irrelevant information. For example, consider the line \texttt{"O = I1;"} annotated with \texttt{"Assign the value of I1 to the output O when S is high."}.
Since we cannot deduce from this single line that \texttt{O} is the output and \texttt{S} is related, such descriptions are deemed inaccurate and are consequently excluded from the dataset to maintain training effectiveness.
% we cannot directly infer the information that O is the output and S is related from the single line of code. This type of line-level description is considered ineffective and detrimental to model training, and thus we will discard it.
% \textcolor{changran}{
% Otherwise, we employ GPT-4 to divide the module into smaller manageable code blocks, such as \texttt{always} blocks, which enhances the utilization of the available Verilog code. [state whether certain techniques are used here]. Blocks with token counts below $2048$ are returned to Step 1 for further processing.
% }
% Once we have the commented code, we use GPT-4 to generate a detailed specification for the code (Step 2). This specification comprises of two key components: a summary of the code's functionality (what it does) and a detailed explanation of the implementation process, including data transitions between registers (how it works). This dual-layered specification provides a deeper understanding of the code. [do we only have what+how for specification?]
In Step 2, we use GPT-4 to generate a detailed specification for the commented code from Step1. 
This specification includes two main components: a summary of the code's functionality (what it does) and a comprehensive explanation of the implementation process (how it works). 
% Specifically, for the implementation process, we will require GPT to include descriptions of the input and output ports, an explanation of the internal workings of the module/block, a description of the logical implementation process, an overview of the algorithmic logic used in the implementation, and an explanation of the internally defined signals. 
Finally, in Step 3, we combine the qualified code from Step 1 with the detailed specification generated in Step 2 to create high-level functional descriptions. 
% To ensure precision, we instruct GPT-4 to prioritize the refined code, using the detailed specification as a reference. 
To ensure precision, we instruct GPT-4 to focus on the qualified code, using the detailed specification only as reference. 
% This approach prevents GPT-4 from potentially overlooking some details in lengthy segments of code.
The resulting high-level descriptions succinctly summarize the code's functionality (what it does) and provide a concise overview of the implementation process (how it works).
This annotation phase is the most critical and challenging as it demands that the model captures the code's high-level semantics, requiring a profound understanding of Verilog. In current benchmarks and practical applications, users typically prompt the model with high-level functional descriptions rather than detailed specifications. Otherwise, they would need to invest significant effort in writing exhaustive implementation details, making the process time-consuming and requiring extensive expertise. For detailed prompts used in this annotation process, please refer to Appendix~\ref{appendix:prompt}.
And a detailed explanation of why we discard Verilog modules or blocks exceeding $2048$ tokens can be found in Appendix~\ref{appendix:discard}.


Given the industrial-grade quality of the proprietary code, we engage professional hardware engineers to maintain high annotation standards. Adhering to rigorous industry-level standards, these experts ensure precise and accurate annotations, capturing intricate details and significantly enhancing the dataset's value for advanced applications.
Unlike GPT-generated annotations, these human-annotated ones incorporate an additional layer of granularity with medium-detail block descriptions.
For detailed annotation standards and processes, please refer to Appendix~\ref{appendix:standard}.

% Due to the industrial-grade quality of the proprietary code, we have high annotation standards and thus hire several professional hardware engineers for annotation. 
% We adhere to rigorous industry-level standards for annotation. Consequently, we have engaged several professional hardware engineers to perform the annotation tasks.
% Their expertise ensures precise and accurate annotations, capturing intricate details and enhancing the dataset's overall value for advanced applications. Specifically, [complete the annotation details].

\begin{table}[ht]
\centering
\vspace{-10pt}
\caption{The overall statistics of the annotation results for our dataset.}
\vspace{5pt}
\begin{tabular}{|c|c|c|}
\hline
\textbf{Comment Level} & \textbf{Granularity} & \textbf{Count} \\ \hline
Line Level             & N/A                  & 434697 \\ \hline
\multirow{3}{*}{Block Level}  & High-level Description    & 892    \\ \cline{2-3} 
                              & Medium-Detail Description & 1306    \\ \cline{2-3} 
                              & Detailed Description      & 894    \\ \hline
\multirow{2}{*}{Module Level} & High-level Functional Description    & 59448 \\ \cline{2-3} 
                              & Detailed Specification             & 59503 \\ \hline
\end{tabular}

\label{tab:dataset_statistics}
\end{table}

We present the overall statistics of the annotation results in Table~\ref{tab:dataset_statistics}. 
Additionally, Figure~\ref{fig:comment_example} illustrates an example of our comprehensive annotation for a complete Verilog module. 
Notably, the overall dataset encompassing descriptions of various details across multiple levels is used for training.
A similar work to ours is the MG-Verilog dataset introduced by~\citet{zhang2024mg}, including 11,000 Verilog code samples and corresponding natural language descriptions at various levels of details.
However, it has several limitations compared to ours. Firstly, MG-Verilog is relatively small in size and lacks proprietary Verilog code, which diminishes its diversity and applicability. Secondly, it employs direct annotation rather than the CoT approach, which we have found to enhance annotation accuracy as demonstrated in Section~\ref{sec:dataset_evaluation}. 
Besides, our annotation is more comprehensive than that of MG-Verilog, which lacks granularity. We cover line, block, and module levels with both detailed and high-level descriptions, ensuring a strong alignment between natural language and Verilog code.
%Besides, our annotation is more comprehensive than that of MG-Verilog which lacks granularity, covering line, block, and module levels with both detailed and high-level descriptions, ensuring strong alignment between natural language and Verilog code. 
Lastly, MG-Verilog relies on the open-source LLM LLaMA2-70B-Chat for annotation, whereas we use the SOTA LLM GPT-4. In Section~\ref{sec:understanding_evaluation}, we demonstrate that LLaMA2-70B-Chat has a poor understanding of Verilog code, leading to inferior annotation quality in MG-Verilog.
\vspace{-10pt}



\begin{figure}[ht]
    \centering
    \includegraphics[width=0.88\linewidth]{fig/comment_example.pdf}
    \vspace{-12pt}
    \caption{An example of our comprehensive annotation for a complete Verilog module.}
    \label{fig:comment_example}
\end{figure}

\vspace{-10pt}
\subsection{Dataset Evaluation}
\label{sec:dataset_evaluation}
To ensure the quality of our dataset, we assess annotations generated from the CoT process. We randomly sample 200 Verilog modules and engage four professional Verilog designers to evaluate the accuracy of annotations at various levels. This human evaluation indicates that annotations describing high-level functions achieve an accuracy of $91\%$, while those providing detailed specifications attain an accuracy of $88\%$. For line-level annotations, the accuracy is $98\%$. Additionally, we compare the CoT method with the direct annotation approach, where annotations are generated straightforwardly from the original code. This direct annotation method yields only a $67\%$ accuracy, highlighting the significant advantage of integrating the CoT process.

Recent studies in natural language processing (NLP) have demonstrated that LLMs fine-tuned with synthetic instruction data can better understand natural language instructions and show improved alignment with corresponding tasks~\citep{wang2022self,ouyang2022training,taori2023stanford}.
It is important to note that in our work, we also utilize data generated by language models for fine-tuning, including annotations at various levels. While not all annotations are perfectly accurate, we achieve a commendable accuracy of approximately $90\%$. Motivated by~\citet{wang2022self}, we treat those inaccuracies as data noise, and the fine-tuned model on this dataset still derives significant benefits.

\subsection{Understanding Benchmark}
\label{sec:understanding_benchmark}
As the first work to consider the task of Verilog understanding, we introduce a pioneering benchmark to evaluate LLMs' capabilities in interpreting Verilog code. This benchmark consists of 100 high-quality Verilog modules, selected to ensure comprehensive coverage of diverse hardware functionalities, providing a broad assessment scope across different types of hardware designs. We have engaged four experienced hardware engineers to provide precise annotations on each module’s functionalities and the specific operations involved in their implementations. These initial annotations are then rigorously cross-verified by three additional engineers to guarantee accuracy and establish a high standard for future model evaluations. This benchmark fills a critical gap by providing a standardized means to assess LLMs on interpreting Verilog code and will be released later. For detailed examples included in the benchmark, please refer to Appendix~\ref{appendix:benchmark}.

\begin{figure}[ht]
    \centering
    \includegraphics[width=0.7\linewidth]{fig/instruction_v2.pdf}
    \vspace{-12pt}
    \caption{The overview of the instruction construction process and the curriculum learning strategy. For instruction construction, we integrate various settings, \textit{e.g.}, task type, granularity, and comment level, to create tailored instructions for specific scenarios. The curriculum learning strategy involves three hierarchical stages: training progresses from line-level to module-level code (\nth{1} stage), transitioning from detailed to high-level descriptions at each level (\nth{2} stage), and advancing from GPT-annotated to human-annotated descriptions for each granularity (\nth{3} stage).}
    \label{fig:instruction}
\end{figure}


\section{Model and Evaluation}

In this section, we introduce DeepRTL and elaborate on the preparation of our instruction tuning dataset and how we adapt curriculum learning for training.
% our progressive training strategy.
% we need an overview paragraph here
%In this section, we introduce DeepRTL and elaborate on the preparation of our instruction tuning dataset, as well as our progressive training strategy.
%, along with specific training and inference settings that optimize model performance. 
Additionally, we detail the benchmarks and metrics used to evaluate our model's performance in both Verilog understanding and generation tasks.
To accurately assess the semantic precision of the generated descriptions, 
we take the initiative to apply embedding similarity and GPT score for evaluation,
% we introduce two novel metrics: embedding similarity and GPT score, 
which are designed to quantitatively measure the semantic similarity between the model's outputs and the ground truth.

\subsection{Model}
In our work, we have chosen to fine-tune CodeT5+~\citep{wang2023codet5+}, a family of encoder-decoder code foundation LLMs for a wide range of code understanding and generation tasks. CodeT5+ employs a ``shallow encoder and deep decoder" architecture~\citep{li2022competition}, where both encoder and decoder are initialized from pre-trained checkpoints and connected by cross-attention layers. 
We choose to fine-tune CodeT5+ for its extensive pre-training on a vast corpus of software code, 
with the intent to transfer its acquired knowledge to hardware code tasks.
Also, the model's flexible architecture allows for the customization of various training tasks, making it highly adaptable for specific downstream applications. Furthermore, CodeT5+ adopts an efficient fine-tuning strategy where the deep decoder is frozen and only the shallow encoder and cross-attention layers are allowed to train, significantly reducing the number of trainable parameters.
Specifically, we have fine-tuned two versions of CodeT5+, codet5p-220m-bimodal\footnote{\url{https://huggingface.co/Salesforce/codet5p-220m-bimodal}} (CodeT5+-220m) and instructcodet5p-16b\footnote{\url{https://huggingface.co/Salesforce/instructcodet5p-16b}} (CodeT5+-16b), on our dataset, resulting in DeepRTL-220m and DeepRTL-16b, respectively. 
For more information on the model selection, please refer to Appendix~\ref{appendix:model_selection}.



\subsection{Instruction Tuning Dataset}
During the fine-tuning process, we adopt the instruction tuning strategy to enhance the adaptability of LLMs, which is particularly effective when handling diverse types of data and tasks.
Given that our dataset features descriptions at multiple levels and our model is fine-tuned for both Verilog understanding and generation tasks, there is diversity in both the data types and tasks.
To accommodate this diversity, we carefully design specific instructions for each scenario, ensuring the model can adjust its output to align with the intended instructions. Figure~\ref{fig:instruction} illustrates how we combine various settings, \textit{e.g.}, task type, granularity, and comment level, to construct tailored instructions for each specific scenario, fostering a structured approach to instruction-based tuning that optimizes the fine-tuning efficacy. For details on the instructions for different scenarios, please refer to Appendix~\ref{appendix:instruction}.

\subsection{Curriculum Learning for DeepRTL}
We adapt curriculum learning for the fine-tuning process, leveraging our structured dataset that features descriptions of various details across multiple levels.
% For the fine-tuning process, we implement a progressive training strategy, 
%recognizing that the model is more influenced by the data it encounters most recently. 
Initially, the model is fine-tuned on line-level and block-level data, subsequently progressing to module-level data. At each level, we start by aligning the detailed specifications with the code before moving to the high-level functional descriptions. 
And fine-tuning typically starts with GPT-annotated data, followed by human-annotated data for each annotation granularity.
Figure~\ref{fig:instruction} provides an illustration of this process.
We adopt such strategy because a particular focus is placed on aligning Verilog modules with their high-level functional descriptions, which poses the greatest challenge and offers substantial practical applications.
This curriculum learning strategy enables the model to incrementally build knowledge from simpler to more complex scenarios. As a result, the models demonstrate impressive performance across both Verilog understanding and generation benchmarks.
Note that we exclude the cases in the benchmarks from our training dataset.
We primarily follow the instruction tuning script of CodeT5+\footnote{\url{https://github.com/salesforce/CodeT5}} in the fine-tuning process, with a modification to expand the input context length to the maximum of $2048$ tokens.  We utilize the distributed framework, DeepSpeed, to efficiently fine-tune the model across a cluster equipped with eight NVIDIA A800 GPUs, each with 80GB of memory. During inference, we adjust the temperature to 0.8 for understanding tasks and to 0.5 for generation tasks, while other hyperparameters remain at their default settings to ensure optimal performance. 
Further details on the adopted curriculum learning strategy are provided in Appendix~\ref{appendix:explanation_curriculum_learning}.


\subsection{Understanding Evaluation}
\label{sec:understanding_evaluation}
For evaluating LLMs' capabilities in Verilog understanding, we utilize the benchmark introduced in Section~\ref{sec:understanding_benchmark}. The evaluation measures the similarity between the generated descriptions and the ground truth summaries. Previous works usually use BLEU~\citep{papineni2002bleu} and ROUGE~\citep{lin2004rouge} scores for this purpose~\citep{wang2023codet5+}. 
BLEU assesses how many $n$-grams, \textit{i.e.}, sequences of $n$ words, in the machine-generated text appear in the reference text (focusing on precision). In contrast, ROUGE counts how many $n$-grams from the reference appear in the generated text (focusing on recall). 
However, both metrics primarily capture lexical rather than semantic similarity, which may not fully reflect the accuracy of the generated descriptions.
To address this limitation, we take the initiative to apply embedding similarity and GPT score for evaluation.
% we propose two innovative metrics, embedding similarity and GPT score. 
Embedding similarity calculates the cosine similarity between vector representations of generated and ground truth descriptions, using embeddings derived from OpenAI's text-embedding-3-large model. Meanwhile, GPT score uses GPT-4 to quantify the semantic coherence between descriptions by assigning a similarity score from 0 to 1, where 1 indicates perfect semantic alignment.
These metrics provide a more nuanced evaluation by capturing the semantic essence of the descriptions, thus offering a more accurate assessment than previous methods.
For details on the prompt used to calculate the GPT score, please refer to Appendix~\ref{appendix:gpt_score}.

\begin{table}[ht]
    \centering
    \vspace{-10pt}
    \caption{Evaluation results on Verilog understanding using the benchmark proposed in Section~\ref{sec:understanding_benchmark}. BLEU-4 denotes the smoothed BLEU-4 score, and Emb. Sim. represents the embedding similarity metric. Best results are highlighted in bold.}
    \vspace{5pt}
    \label{tab:understanding_results}
    \small{
    \begin{tabular}{@{}l|ccccccc@{}}
    \toprule
       Model  & BLEU-4 & ROUGE-1 & ROUGE-2 & ROUGE-L & Emb. Sim. & GPT Score\\
    \midrule
       GPT-3.5 & 4.75 & 35.46 & 12.64 & 32.07 & 0.802 & 0.641 \\ 
       GPT-4 & 5.36 & 34.31 & 11.31 & 30.66 & 0.824 & 0.683 \\
       o1-preview & 6.06 & 34.27 & 12.25 & 31.01 & 0.806 & 0.643\\
    \midrule
       CodeT5+-220m & 0.28 & 7.10 & 0.34 & 6.18 & 0.313 & 0.032 \\ 
       CodeT5+-16b & 0.10 & 1.37 & 0.00 & 1.37 & 0.228 & 0.014 \\
       LLaMA2-70B-Chat & 2.86 & 28.15 & 10.09 & 26.12 & 0.633 & 0.500 \\ 
       DeepRTL-220m-direct & 11.99 & 40.05 & 20.56 & 37.09 & 0.793 & 0.572\\ 
       DeepRTL-16b-direct & 11.06 & 38.12 & 18.15 & 34.85 & 0.778 & 0.533 \\ 
    % \midrule
    %    CodeT5+-220m & \\
    %    CodeT5+-16b & \\
    %    LLaMA-70B-Chat & \\
    \midrule
       DeepRTL-220m & 18.66 & \textbf{47.69} & \textbf{29.49} & 44.02 & \textbf{0.837} & \textbf{0.705}\\
       DeepRTL-16b & \textbf{18.94} & 47.27 & 29.46 & \textbf{44.13} & 0.830 & 0.694\\
    \bottomrule
    \end{tabular}
    }
    \vspace{-10pt}
\end{table}



\subsection{Generation Evaluation}
To evaluate LLMs' capabilities in Verilog generation, we adopt the latest benchmark introduced by~\citet{chang2024natural}, which is an expansion based on the previous well-established RTLLM benchmark~\citep{lu2024rtllm}.
The benchmark by~\citet{chang2024natural} encompasses a broad spectrum of complexities across three categories: arithmetic, digital circuit logic, and advanced hardware designs.
This benchmark extends beyond previous efforts by incorporating a wider range of more challenging and practical Verilog designs, thus providing a more thorough assessment of the models' capabilities in generating Verilog code.

The evaluation focuses on two critical aspects: syntax correctness and functional accuracy. We use the open-source simulator iverilog~\citep{williams2002icarus} to assess both syntactic and functional correctness of Verilog code generated by LLMs. 
For the evaluation metric, we adopt the prevalent Pass@$k$ metric, which considers a problem solved if any of the $k$ generated code samples pass the compilation or functional tests~\citep{pei2024betterv}. For this study, we set $k$ values of 1 and 5, where a higher Pass@$k$ score indicates better model performance.
To further delineate the models' capabilities, we track the proportion of cases that pass out of 5 generated samples and compute the average as the success rate. For syntax correctness, this success rate measures the proportion of code samples that successfully compile and, for functional accuracy, the fraction that passes unit tests.


% evaluation benchmark
% the scope of the benchmark is also important
% comparing the results with the software programming language
% we use less data to achieve comparable performance of models trained with abundant software codes
% maybe we could state that the simple functional description is a more challenging case
% some works even do not compare the results with GPT-3.5/GPT-4, like Thakur's work
% we need to state that the evaluation benchmark is not overlapped with the training dataset
% we could mention that the previous software code understanding and generation tasks are relatively simple by showing that the maximum input and output lengths are relatively small for the software code model
% failure case analysis may also be interesting





\section{Experiments}
\label{section5}

In this section, we conduct extensive experiments to show that \ourmethod~can significantly speed up the sampling of existing MR Diffusion. To rigorously validate the effectiveness of our method, we follow the settings and checkpoints from \cite{luo2024daclip} and only modify the sampling part. Our experiment is divided into three parts. Section \ref{mainresult} compares the sampling results for different NFE cases. Section \ref{effects} studies the effects of different parameter settings on our algorithm, including network parameterizations and solver types. In Section \ref{analysis}, we visualize the sampling trajectories to show the speedup achieved by \ourmethod~and analyze why noise prediction gets obviously worse when NFE is less than 20.


\subsection{Main results}\label{mainresult}

Following \cite{luo2024daclip}, we conduct experiments with ten different types of image degradation: blurry, hazy, JPEG-compression, low-light, noisy, raindrop, rainy, shadowed, snowy, and inpainting (see Appendix \ref{appd1} for details). We adopt LPIPS \citep{zhang2018lpips} and FID \citep{heusel2017fid} as main metrics for perceptual evaluation, and also report PSNR and SSIM \citep{wang2004ssim} for reference. We compare \ourmethod~with other sampling methods, including posterior sampling \citep{luo2024posterior} and Euler-Maruyama discretization \citep{kloeden1992sde}. We take two tasks as examples and the metrics are shown in Figure \ref{fig:main}. Unless explicitly mentioned, we always use \ourmethod~based on SDE solver, with data prediction and uniform $\lambda$. The complete experimental results can be found in Appendix \ref{appd3}. The results demonstrate that \ourmethod~converges in a few (5 or 10) steps and produces samples with stable quality. Our algorithm significantly reduces the time cost without compromising sampling performance, which is of great practical value for MR Diffusion.


\begin{figure}[!ht]
    \centering
    \begin{minipage}[b]{0.45\textwidth}
        \centering
        \includegraphics[width=1\textwidth, trim=0 20 0 0]{figs/main_result/7_lowlight_fid.pdf}
        \subcaption{FID on \textit{low-light} dataset}
        \label{fig:main(a)}
    \end{minipage}
    \begin{minipage}[b]{0.45\textwidth}
        \centering
        \includegraphics[width=1\textwidth, trim=0 20 0 0]{figs/main_result/7_lowlight_lpips.pdf}
        \subcaption{LPIPS on \textit{low-light} dataset}
        \label{fig:main(b)}
    \end{minipage}
    \begin{minipage}[b]{0.45\textwidth}
        \centering
        \includegraphics[width=1\textwidth, trim=0 20 0 0]{figs/main_result/10_motion_fid.pdf}
        \subcaption{FID on \textit{motion-blurry} dataset}
        \label{fig:main(c)}
    \end{minipage}
    \begin{minipage}[b]{0.45\textwidth}
        \centering
        \includegraphics[width=1\textwidth, trim=0 20 0 0]{figs/main_result/10_motion_lpips.pdf}
        \subcaption{LPIPS on \textit{motion-blurry} dataset}
        \label{fig:main(d)}
    \end{minipage}
    \caption{\textbf{Perceptual evaluations on \textit{low-light} and \textit{motion-blurry} datasets.}}
    \label{fig:main}
\end{figure}

\subsection{Effects of parameter choice}\label{effects}

In Table \ref{tab:ablat_param}, we compare the results of two network parameterizations. The data prediction shows stable performance across different NFEs. The noise prediction performs similarly to data prediction with large NFEs, but its performance deteriorates significantly with smaller NFEs. The detailed analysis can be found in Section \ref{section5.3}. In Table \ref{tab:ablat_solver}, we compare \ourmethod-ODE-d-2 and \ourmethod-SDE-d-2 on the \textit{inpainting} task, which are derived from PF-ODE and reverse-time SDE respectively. SDE-based solver works better with a large NFE, whereas ODE-based solver is more effective with a small NFE. In general, neither solver type is inherently better.


% In Table \ref{tab:hazy}, we study the impact of two step size schedules on the results. On the whole, uniform $\lambda$ performs slightly better than uniform $t$. Our algorithm follows the method of \cite{lu2022dpmsolverplus} to estimate the integral part of the solution, while the analytical part does not affect the error.  Consequently, our algorithm has the same global truncation error, that is $\mathcal{O}\left(h_{max}^{k}\right)$. Note that the initial and final values of $\lambda$ depend on noise schedule and are fixed. Therefore, uniform $\lambda$ scheduling leads to the smallest $h_{max}$ and works better.

\begin{table}[ht]
    \centering
    \begin{minipage}{0.5\textwidth}
    \small
    \renewcommand{\arraystretch}{1}
    \centering
    \caption{Ablation study of network parameterizations on the Rain100H dataset.}
    % \vspace{8pt}
    \resizebox{1\textwidth}{!}{
        \begin{tabular}{cccccc}
			\toprule[1.5pt]
            % \multicolumn{6}{c}{Rainy} \\
            % \cmidrule(lr){1-6}
             NFE & Parameterization      & LPIPS\textdownarrow & FID\textdownarrow &  PSNR\textuparrow & SSIM\textuparrow  \\
            \midrule[1pt]
            \multirow{2}{*}{50}
             & Noise Prediction & \textbf{0.0606}     & \textbf{27.28}   & \textbf{28.89}     & \textbf{0.8615}    \\
             & Data Prediction & 0.0620     & 27.65   & 28.85     & 0.8602    \\
            \cmidrule(lr){1-6}
            \multirow{2}{*}{20}
              & Noise Prediction & 0.1429     & 47.31   & 27.68     & 0.7954    \\
              & Data Prediction & \textbf{0.0635}     & \textbf{27.79}   & \textbf{28.60}     & \textbf{0.8559}    \\
            \cmidrule(lr){1-6}
            \multirow{2}{*}{10}
              & Noise Prediction & 1.376     & 402.3   & 6.623     & 0.0114    \\
              & Data Prediction & \textbf{0.0678}     & \textbf{29.54}   & \textbf{28.09}     & \textbf{0.8483}    \\
            \cmidrule(lr){1-6}
            \multirow{2}{*}{5}
              & Noise Prediction & 1.416     & 447.0   & 5.755     & 0.0051    \\
              & Data Prediction & \textbf{0.0637}     & \textbf{26.92}   & \textbf{28.82}     & \textbf{0.8685}    \\       
            \bottomrule[1.5pt]
        \end{tabular}}
        \label{tab:ablat_param}
    \end{minipage}
    \hspace{0.01\textwidth}
    \begin{minipage}{0.46\textwidth}
    \small
    \renewcommand{\arraystretch}{1}
    \centering
    \caption{Ablation study of solver types on the CelebA-HQ dataset.}
    % \vspace{8pt}
        \resizebox{1\textwidth}{!}{
        \begin{tabular}{cccccc}
			\toprule[1.5pt]
            % \multicolumn{6}{c}{Raindrop} \\     
            % \cmidrule(lr){1-6}
             NFE & Solver Type     & LPIPS\textdownarrow & FID\textdownarrow &  PSNR\textuparrow & SSIM\textuparrow  \\
            \midrule[1pt]
            \multirow{2}{*}{50}
             & ODE & 0.0499     & 22.91   & 28.49     & 0.8921    \\
             & SDE & \textbf{0.0402}     & \textbf{19.09}   & \textbf{29.15}     & \textbf{0.9046}    \\
            \cmidrule(lr){1-6}
            \multirow{2}{*}{20}
              & ODE & 0.0475    & 21.35   & 28.51     & 0.8940    \\
              & SDE & \textbf{0.0408}     & \textbf{19.13}   & \textbf{28.98}    & \textbf{0.9032}    \\
            \cmidrule(lr){1-6}
            \multirow{2}{*}{10}
              & ODE & \textbf{0.0417}    & 19.44   & \textbf{28.94}     & \textbf{0.9048}    \\
              & SDE & 0.0437     & \textbf{19.29}   & 28.48     & 0.8996    \\
            \cmidrule(lr){1-6}
            \multirow{2}{*}{5}
              & ODE & \textbf{0.0526}     & 27.44   & \textbf{31.02}     & \textbf{0.9335}    \\
              & SDE & 0.0529    & \textbf{24.02}   & 28.35     & 0.8930    \\
            \bottomrule[1.5pt]
        \end{tabular}}
        \label{tab:ablat_solver}
    \end{minipage}
\end{table}


% \renewcommand{\arraystretch}{1}
%     \centering
%     \caption{Ablation study of step size schedule on the RESIDE-6k dataset.}
%     % \vspace{8pt}
%         \resizebox{1\textwidth}{!}{
%         \begin{tabular}{cccccc}
% 			\toprule[1.5pt]
%             % \multicolumn{6}{c}{Raindrop} \\     
%             % \cmidrule(lr){1-6}
%              NFE & Schedule      & LPIPS\textdownarrow & FID\textdownarrow &  PSNR\textuparrow & SSIM\textuparrow  \\
%             \midrule[1pt]
%             \multirow{2}{*}{50}
%              & uniform $t$ & 0.0271     & 5.539   & 30.00     & 0.9351    \\
%              & uniform $\lambda$ & \textbf{0.0233}     & \textbf{4.993}   & \textbf{30.19}     & \textbf{0.9427}    \\
%             \cmidrule(lr){1-6}
%             \multirow{2}{*}{20}
%               & uniform $t$ & 0.0313     & 6.000   & 29.73     & 0.9270    \\
%               & uniform $\lambda$ & \textbf{0.0240}     & \textbf{5.077}   & \textbf{30.06}    & \textbf{0.9409}    \\
%             \cmidrule(lr){1-6}
%             \multirow{2}{*}{10}
%               & uniform $t$ & 0.0309     & 6.094   & 29.42     & 0.9274    \\
%               & uniform $\lambda$ & \textbf{0.0246}     & \textbf{5.228}   & \textbf{29.65}     & \textbf{0.9372}    \\
%             \cmidrule(lr){1-6}
%             \multirow{2}{*}{5}
%               & uniform $t$ & 0.0256     & 5.477   & \textbf{29.91}     & 0.9342    \\
%               & uniform $\lambda$ & \textbf{0.0228}     & \textbf{5.174}   & 29.65     & \textbf{0.9416}    \\
%             \bottomrule[1.5pt]
%         \end{tabular}}
%         \label{tab:ablat_schedule}



\subsection{Analysis}\label{analysis}
\label{section5.3}

\begin{figure}[ht!]
    \centering
    \begin{minipage}[t]{0.6\linewidth}
        \centering
        \includegraphics[width=\linewidth, trim=0 20 10 0]{figs/trajectory_a.pdf} %trim左下右上
        \subcaption{Sampling results.}
        \label{fig:traj(a)}
    \end{minipage}
    \begin{minipage}[t]{0.35\linewidth}
        \centering
        \includegraphics[width=\linewidth, trim=0 0 0 0]{figs/trajectory_b.pdf} %trim左下右上
        \subcaption{Trajectory.}
        \label{fig:traj(b)}
    \end{minipage}
    \caption{\textbf{Sampling trajectories.} In (a), we compare our method (with order 1 and order 2) and previous sampling methods (i.e., posterior sampling and Euler discretization) on a motion blurry image. The numbers in parentheses indicate the NFE. In (b), we illustrate trajectories of each sampling method. Previous methods need to take many unnecessary paths to converge. With few NFEs, they fail to reach the ground truth (i.e., the location of $\boldsymbol{x}_0$). Our methods follow a more direct trajectory.}
    \label{fig:traj}
\end{figure}

\textbf{Sampling trajectory.}~ Inspired by the design idea of NCSN \citep{song2019ncsn}, we provide a new perspective of diffusion sampling process. \cite{song2019ncsn} consider each data point (e.g., an image) as a point in high-dimensional space. During the diffusion process, noise is added to each point $\boldsymbol{x}_0$, causing it to spread throughout the space, while the score function (a neural network) \textit{remembers} the direction towards $\boldsymbol{x}_0$. In the sampling process, we start from a random point by sampling a Gaussian distribution and follow the guidance of the reverse-time SDE (or PF-ODE) and the score function to locate $\boldsymbol{x}_0$. By connecting each intermediate state $\boldsymbol{x}_t$, we obtain a sampling trajectory. However, this trajectory exists in a high-dimensional space, making it difficult to visualize. Therefore, we use Principal Component Analysis (PCA) to reduce $\boldsymbol{x}_t$ to two dimensions, obtaining the projection of the sampling trajectory in 2D space. As shown in Figure \ref{fig:traj}, we present an example. Previous sampling methods \citep{luo2024posterior} often require a long path to find $\boldsymbol{x}_0$, and reducing NFE can lead to cumulative errors, making it impossible to locate $\boldsymbol{x}_0$. In contrast, our algorithm produces more direct trajectories, allowing us to find $\boldsymbol{x}_0$ with fewer NFEs.

\begin{figure*}[ht]
    \centering
    \begin{minipage}[t]{0.45\linewidth}
        \centering
        \includegraphics[width=\linewidth, trim=0 0 0 0]{figs/convergence_a.pdf} %trim左下右上
        \subcaption{Sampling results.}
        \label{fig:convergence(a)}
    \end{minipage}
    \begin{minipage}[t]{0.43\linewidth}
        \centering
        \includegraphics[width=\linewidth, trim=0 20 0 0]{figs/convergence_b.pdf} %trim左下右上
        \subcaption{Ratio of convergence.}
        \label{fig:convergence(b)}
    \end{minipage}
    \caption{\textbf{Convergence of noise prediction and data prediction.} In (a), we choose a low-light image for example. The numbers in parentheses indicate the NFE. In (b), we illustrate the ratio of components of neural network output that satisfy the Taylor expansion convergence requirement.}
    \label{fig:converge}
\end{figure*}

\textbf{Numerical stability of parameterizations.}~ From Table 1, we observe poor sampling results for noise prediction in the case of few NFEs. The reason may be that the neural network parameterized by noise prediction is numerically unstable. Recall that we used Taylor expansion in Eq.(\ref{14}), and the condition for the equality to hold is $|\lambda-\lambda_s|<\boldsymbol{R}(s)$. And the radius of convergence $\boldsymbol{R}(t)$ can be calculated by
\begin{equation}
\frac{1}{\boldsymbol{R}(t)}=\lim_{n\rightarrow\infty}\left|\frac{\boldsymbol{c}_{n+1}(t)}{\boldsymbol{c}_n(t)}\right|,
\end{equation}
where $\boldsymbol{c}_n(t)$ is the coefficient of the $n$-th term in Taylor expansion. We are unable to compute this limit and can only compute the $n=0$ case as an approximation. The output of the neural network can be viewed as a vector, with each component corresponding to a radius of convergence. At each time step, we count the ratio of components that satisfy $\boldsymbol{R}_i(s)>|\lambda-\lambda_s|$ as a criterion for judging the convergence, where $i$ denotes the $i$-th component. As shown in Figure \ref{fig:converge}, the neural network parameterized by data prediction meets the convergence criteria at almost every step. However, the neural network parameterized by noise prediction always has components that cannot converge, which will lead to large errors and failed sampling. Therefore, data prediction has better numerical stability and is a more recommended choice.


\paragraph{Summary}
Our findings provide significant insights into the influence of correctness, explanations, and refinement on evaluation accuracy and user trust in AI-based planners. 
In particular, the findings are three-fold: 
(1) The \textbf{correctness} of the generated plans is the most significant factor that impacts the evaluation accuracy and user trust in the planners. As the PDDL solver is more capable of generating correct plans, it achieves the highest evaluation accuracy and trust. 
(2) The \textbf{explanation} component of the LLM planner improves evaluation accuracy, as LLM+Expl achieves higher accuracy than LLM alone. Despite this improvement, LLM+Expl minimally impacts user trust. However, alternative explanation methods may influence user trust differently from the manually generated explanations used in our approach.
% On the other hand, explanations may help refine the trust of the planner to a more appropriate level by indicating planner shortcomings.
(3) The \textbf{refinement} procedure in the LLM planner does not lead to a significant improvement in evaluation accuracy; however, it exhibits a positive influence on user trust that may indicate an overtrust in some situations.
% This finding is aligned with prior works showing that iterative refinements based on user feedback would increase user trust~\cite{kunkel2019let, sebo2019don}.
Finally, the propensity-to-trust analysis identifies correctness as the primary determinant of user trust, whereas explanations provided limited improvement in scenarios where the planner's accuracy is diminished.

% In conclusion, our results indicate that the planner's correctness is the dominant factor for both evaluation accuracy and user trust. Therefore, selecting high-quality training data and optimizing the training procedure of AI-based planners to improve planning correctness is the top priority. Once the AI planner achieves a similar correctness level to traditional graph-search planners, strengthening its capability to explain and refine plans will further improve user trust compared to traditional planners.

\paragraph{Future Research} Future steps in this research include expanding user studies with larger sample sizes to improve generalizability and including additional planning problems per session for a more comprehensive evaluation. Next, we will explore alternative methods for generating plan explanations beyond manual creation to identify approaches that more effectively enhance user trust. 
Additionally, we will examine user trust by employing multiple LLM-based planners with varying levels of planning accuracy to better understand the interplay between planning correctness and user trust. 
Furthermore, we aim to enable real-time user-planner interaction, allowing users to provide feedback and refine plans collaboratively, thereby fostering a more dynamic and user-centric planning process.

\newpage

% \subsubsection*{Author Contributions}
% If you'd like to, you may include  a section for author contributions as is done
% in many journals. This is optional and at the discretion of the authors.

\section*{Acknowledgments}
% Use unnumbered third level headings for the acknowledgments. All
% acknowledgments, including those to funding agencies, go at the end of the paper.
This work was supported in part by the General Research Fund of the Hong Kong Research Grants Council (RGC) under Grant No. 14212422 and 14202824, and in part by National Technology Innovation Center for EDA.

% \bibliography{iclr2025_conference}
% \bibliographystyle{iclr2025_conference}

\begin{thebibliography}{51}
\providecommand{\natexlab}[1]{#1}
\providecommand{\url}[1]{\texttt{#1}}
\expandafter\ifx\csname urlstyle\endcsname\relax
  \providecommand{\doi}[1]{doi: #1}\else
  \providecommand{\doi}{doi: \begingroup \urlstyle{rm}\Url}\fi

\bibitem[Achiam et~al.(2023)Achiam, Adler, Agarwal, Ahmad, Akkaya, Aleman, Almeida, Altenschmidt, Altman, Anadkat, et~al.]{achiam2023gpt}
Josh Achiam, Steven Adler, Sandhini Agarwal, Lama Ahmad, Ilge Akkaya, Florencia~Leoni Aleman, Diogo Almeida, Janko Altenschmidt, Sam Altman, Shyamal Anadkat, et~al.
\newblock Gpt-4 technical report.
\newblock \emph{arXiv preprint arXiv:2303.08774}, 2023.

\bibitem[Bengio et~al.(2009)Bengio, Louradour, Collobert, and Weston]{bengio2009curriculum}
Yoshua Bengio, J{\'e}r{\^o}me Louradour, Ronan Collobert, and Jason Weston.
\newblock Curriculum learning.
\newblock In \emph{Proceedings of the 26th annual international conference on machine learning}, pp.\  41--48, 2009.

\bibitem[Blocklove et~al.(2023)Blocklove, Garg, Karri, and Pearce]{blocklove2023chip}
Jason Blocklove, Siddharth Garg, Ramesh Karri, and Hammond Pearce.
\newblock Chip-chat: Challenges and opportunities in conversational hardware design.
\newblock In \emph{2023 ACM/IEEE 5th Workshop on Machine Learning for CAD (MLCAD)}, pp.\  1--6. IEEE, 2023.

\bibitem[Campos(2021)]{campos2021curriculum}
Daniel Campos.
\newblock Curriculum learning for language modeling.
\newblock \emph{arXiv preprint arXiv:2108.02170}, 2021.

\bibitem[Chang et~al.(2023)Chang, Wang, Ren, Wang, Liang, Han, Li, and Li]{chang2023chipgpt}
Kaiyan Chang, Ying Wang, Haimeng Ren, Mengdi Wang, Shengwen Liang, Yinhe Han, Huawei Li, and Xiaowei Li.
\newblock Chipgpt: How far are we from natural language hardware design.
\newblock \emph{arXiv preprint arXiv:2305.14019}, 2023.

\bibitem[Chang et~al.(2024{\natexlab{a}})Chang, Chen, Zhou, Zhu, Xu, Li, Wang, Liang, Li, Han, et~al.]{chang2024natural}
Kaiyan Chang, Zhirong Chen, Yunhao Zhou, Wenlong Zhu, Haobo Xu, Cangyuan Li, Mengdi Wang, Shengwen Liang, Huawei Li, Yinhe Han, et~al.
\newblock Natural language is not enough: Benchmarking multi-modal generative ai for verilog generation.
\newblock \emph{arXiv preprint arXiv:2407.08473}, 2024{\natexlab{a}}.

\bibitem[Chang et~al.(2024{\natexlab{b}})Chang, Wang, Yang, Wang, Jin, Zhu, Chen, Li, Yan, Zhou, et~al.]{chang2024data}
Kaiyan Chang, Kun Wang, Nan Yang, Ying Wang, Dantong Jin, Wenlong Zhu, Zhirong Chen, Cangyuan Li, Hao Yan, Yunhao Zhou, et~al.
\newblock Data is all you need: Finetuning llms for chip design via an automated design-data augmentation framework.
\newblock \emph{arXiv preprint arXiv:2403.11202}, 2024{\natexlab{b}}.

\bibitem[Chen et~al.(2024)Chen, Chen, Chu, Fang, Ho, Huang, Khan, Li, Li, Liang, et~al.]{chen2024dawn}
Lei Chen, Yiqi Chen, Zhufei Chu, Wenji Fang, Tsung-Yi Ho, Yu~Huang, Sadaf Khan, Min Li, Xingquan Li, Yun Liang, et~al.
\newblock The dawn of ai-native eda: Promises and challenges of large circuit models.
\newblock \emph{arXiv preprint arXiv:2403.07257}, 2024.

\bibitem[Dubey et~al.(2024)Dubey, Jauhri, Pandey, Kadian, Al-Dahle, Letman, Mathur, Schelten, Yang, Fan, et~al.]{dubey2024llama}
Abhimanyu Dubey, Abhinav Jauhri, Abhinav Pandey, Abhishek Kadian, Ahmad Al-Dahle, Aiesha Letman, Akhil Mathur, Alan Schelten, Amy Yang, Angela Fan, et~al.
\newblock The llama 3 herd of models.
\newblock \emph{arXiv preprint arXiv:2407.21783}, 2024.

\bibitem[Fu et~al.(2023)Fu, Zhang, Yu, Li, Ye, Li, Wan, and Lin]{fu2023gpt4aigchip}
Yonggan Fu, Yongan Zhang, Zhongzhi Yu, Sixu Li, Zhifan Ye, Chaojian Li, Cheng Wan, and Yingyan~Celine Lin.
\newblock Gpt4aigchip: Towards next-generation ai accelerator design automation via large language models.
\newblock In \emph{2023 IEEE/ACM International Conference on Computer Aided Design (ICCAD)}, pp.\  1--9. IEEE, 2023.

\bibitem[Guo et~al.(2024)Guo, Zhu, Yang, Xie, Dong, Zhang, Chen, Bi, Wu, Li, et~al.]{guo2024deepseek}
Daya Guo, Qihao Zhu, Dejian Yang, Zhenda Xie, Kai Dong, Wentao Zhang, Guanting Chen, Xiao Bi, Yu~Wu, YK~Li, et~al.
\newblock Deepseek-coder: When the large language model meets programming--the rise of code intelligence.
\newblock \emph{arXiv preprint arXiv:2401.14196}, 2024.

\bibitem[Husain et~al.(2019)Husain, Wu, Gazit, Allamanis, and Brockschmidt]{husain2019codesearchnet}
Hamel Husain, Ho-Hsiang Wu, Tiferet Gazit, Miltiadis Allamanis, and Marc Brockschmidt.
\newblock Codesearchnet challenge: Evaluating the state of semantic code search.
\newblock \emph{arXiv preprint arXiv:1909.09436}, 2019.

\bibitem[Jin et~al.(2023)Jin, Liu, Zheng, Li, Zhao, Zhang, Zheng, Zhou, and Liu]{jin2023adapt}
Bu~Jin, Xinyu Liu, Yupeng Zheng, Pengfei Li, Hao Zhao, Tong Zhang, Yuhang Zheng, Guyue Zhou, and Jingjing Liu.
\newblock Adapt: Action-aware driving caption transformer.
\newblock In \emph{2023 IEEE International Conference on Robotics and Automation (ICRA)}, pp.\  7554--7561. IEEE, 2023.

\bibitem[Li et~al.(2022)Li, Choi, Chung, Kushman, Schrittwieser, Leblond, Eccles, Keeling, Gimeno, Dal~Lago, et~al.]{li2022competition}
Yujia Li, David Choi, Junyoung Chung, Nate Kushman, Julian Schrittwieser, R{\'e}mi Leblond, Tom Eccles, James Keeling, Felix Gimeno, Agustin Dal~Lago, et~al.
\newblock Competition-level code generation with alphacode.
\newblock \emph{Science}, 378\penalty0 (6624):\penalty0 1092--1097, 2022.

\bibitem[Lin(2004)]{lin2004rouge}
Chin-Yew Lin.
\newblock Rouge: A package for automatic evaluation of summaries.
\newblock In \emph{Text summarization branches out}, pp.\  74--81, 2004.

\bibitem[Liu et~al.(2023{\natexlab{a}})Liu, Ene, Kirby, Cheng, Pinckney, Liang, Alben, Anand, Banerjee, Bayraktaroglu, et~al.]{liu2023chipnemo}
Mingjie Liu, Teodor-Dumitru Ene, Robert Kirby, Chris Cheng, Nathaniel Pinckney, Rongjian Liang, Jonah Alben, Himyanshu Anand, Sanmitra Banerjee, Ismet Bayraktaroglu, et~al.
\newblock Chipnemo: Domain-adapted llms for chip design.
\newblock \emph{arXiv preprint arXiv:2311.00176}, 2023{\natexlab{a}}.

\bibitem[Liu et~al.(2023{\natexlab{b}})Liu, Pinckney, Khailany, and Ren]{liu2023verilogeval}
Mingjie Liu, Nathaniel Pinckney, Brucek Khailany, and Haoxing Ren.
\newblock Verilogeval: Evaluating large language models for verilog code generation.
\newblock In \emph{2023 IEEE/ACM International Conference on Computer Aided Design (ICCAD)}, pp.\  1--8. IEEE, 2023{\natexlab{b}}.

\bibitem[Liu et~al.(2024)Liu, Fang, Lu, Wang, Zhang, Zhang, and Xie]{liu2024rtlcoder}
Shang Liu, Wenji Fang, Yao Lu, Jing Wang, Qijun Zhang, Hongce Zhang, and Zhiyao Xie.
\newblock Rtlcoder: Fully open-source and efficient llm-assisted rtl code generation technique.
\newblock \emph{IEEE Transactions on Computer-Aided Design of Integrated Circuits and Systems}, 2024.

\bibitem[Lu et~al.(2024)Lu, Liu, Zhang, and Xie]{lu2024rtllm}
Yao Lu, Shang Liu, Qijun Zhang, and Zhiyao Xie.
\newblock Rtllm: An open-source benchmark for design rtl generation with large language model.
\newblock In \emph{2024 29th Asia and South Pacific Design Automation Conference (ASP-DAC)}, pp.\  722--727. IEEE, 2024.

\bibitem[Na et~al.(2024)Na, Yamani, Lhadj, Baghdadi, et~al.]{na2024curriculum}
Marwa Na, Kamel Yamani, Lynda Lhadj, Riyadh Baghdadi, et~al.
\newblock Curriculum learning for small code language models.
\newblock In \emph{Proceedings of the 62nd Annual Meeting of the Association for Computational Linguistics (Volume 4: Student Research Workshop)}, pp.\  531--542, 2024.

\bibitem[Narvekar et~al.(2020)Narvekar, Peng, Leonetti, Sinapov, Taylor, and Stone]{narvekar2020curriculum}
Sanmit Narvekar, Bei Peng, Matteo Leonetti, Jivko Sinapov, Matthew~E Taylor, and Peter Stone.
\newblock Curriculum learning for reinforcement learning domains: A framework and survey.
\newblock \emph{Journal of Machine Learning Research}, 21\penalty0 (181):\penalty0 1--50, 2020.

\bibitem[Ouyang et~al.(2022)Ouyang, Wu, Jiang, Almeida, Wainwright, Mishkin, Zhang, Agarwal, Slama, Ray, et~al.]{ouyang2022training}
Long Ouyang, Jeffrey Wu, Xu~Jiang, Diogo Almeida, Carroll Wainwright, Pamela Mishkin, Chong Zhang, Sandhini Agarwal, Katarina Slama, Alex Ray, et~al.
\newblock Training language models to follow instructions with human feedback.
\newblock \emph{Advances in neural information processing systems}, 35:\penalty0 27730--27744, 2022.

\bibitem[Papineni et~al.(2002)Papineni, Roukos, Ward, and Zhu]{papineni2002bleu}
Kishore Papineni, Salim Roukos, Todd Ward, and Wei-Jing Zhu.
\newblock Bleu: a method for automatic evaluation of machine translation.
\newblock In \emph{Proceedings of the 40th annual meeting of the Association for Computational Linguistics}, pp.\  311--318, 2002.

\bibitem[Pearce et~al.(2020)Pearce, Tan, and Karri]{pearce2020dave}
Hammond Pearce, Benjamin Tan, and Ramesh Karri.
\newblock Dave: Deriving automatically verilog from english.
\newblock In \emph{Proceedings of the 2020 ACM/IEEE Workshop on Machine Learning for CAD}, pp.\  27--32, 2020.

\bibitem[Pei et~al.(2024)Pei, Zhen, Yuan, Huang, and Yu]{pei2024betterv}
Zehua Pei, Hui-Ling Zhen, Mingxuan Yuan, Yu~Huang, and Bei Yu.
\newblock Betterv: Controlled verilog generation with discriminative guidance.
\newblock \emph{arXiv preprint arXiv:2402.03375}, 2024.

\bibitem[Taori et~al.(2023)Taori, Gulrajani, Zhang, Dubois, Li, Guestrin, Liang, and Hashimoto]{taori2023stanford}
Rohan Taori, Ishaan Gulrajani, Tianyi Zhang, Yann Dubois, Xuechen Li, Carlos Guestrin, Percy Liang, and Tatsunori~B Hashimoto.
\newblock Stanford alpaca: An instruction-following llama model, 2023.

\bibitem[Thakur et~al.(2023)Thakur, Ahmad, Fan, Pearce, Tan, Karri, Dolan-Gavitt, and Garg]{thakur2023benchmarking}
Shailja Thakur, Baleegh Ahmad, Zhenxing Fan, Hammond Pearce, Benjamin Tan, Ramesh Karri, Brendan Dolan-Gavitt, and Siddharth Garg.
\newblock Benchmarking large language models for automated verilog rtl code generation.
\newblock In \emph{2023 Design, Automation \& Test in Europe Conference \& Exhibition (DATE)}, pp.\  1--6. IEEE, 2023.

\bibitem[Thakur et~al.(2024)Thakur, Ahmad, Pearce, Tan, Dolan-Gavitt, Karri, and Garg]{thakur2024verigen}
Shailja Thakur, Baleegh Ahmad, Hammond Pearce, Benjamin Tan, Brendan Dolan-Gavitt, Ramesh Karri, and Siddharth Garg.
\newblock Verigen: A large language model for verilog code generation.
\newblock \emph{ACM Transactions on Design Automation of Electronic Systems}, 29\penalty0 (3):\penalty0 1--31, 2024.

\bibitem[Wang et~al.(2022)Wang, Kordi, Mishra, Liu, Smith, Khashabi, and Hajishirzi]{wang2022self}
Yizhong Wang, Yeganeh Kordi, Swaroop Mishra, Alisa Liu, Noah~A Smith, Daniel Khashabi, and Hannaneh Hajishirzi.
\newblock Self-instruct: Aligning language models with self-generated instructions.
\newblock \emph{arXiv preprint arXiv:2212.10560}, 2022.

\bibitem[Wang et~al.(2023{\natexlab{a}})Wang, Le, Gotmare, Bui, Li, and Hoi]{wang2023codet5+}
Yue Wang, Hung Le, Akhilesh~Deepak Gotmare, Nghi~DQ Bui, Junnan Li, and Steven~CH Hoi.
\newblock Codet5+: Open code large language models for code understanding and generation.
\newblock \emph{arXiv preprint arXiv:2305.07922}, 2023{\natexlab{a}}.

\bibitem[Wang et~al.(2023{\natexlab{b}})Wang, Yue, Lu, Liu, Zhong, Song, and Huang]{wang2023efficienttrain}
Yulin Wang, Yang Yue, Rui Lu, Tianjiao Liu, Zhao Zhong, Shiji Song, and Gao Huang.
\newblock Efficienttrain: Exploring generalized curriculum learning for training visual backbones.
\newblock In \emph{Proceedings of the IEEE/CVF International Conference on Computer Vision}, pp.\  5852--5864, 2023{\natexlab{b}}.

\bibitem[Wei et~al.(2024)Wei, Wang, Lu, Xu, Liu, Zhao, Chen, and Wang]{wei2024editable}
Yuxi Wei, Zi~Wang, Yifan Lu, Chenxin Xu, Changxing Liu, Hao Zhao, Siheng Chen, and Yanfeng Wang.
\newblock Editable scene simulation for autonomous driving via collaborative llm-agents.
\newblock In \emph{Proceedings of the IEEE/CVF Conference on Computer Vision and Pattern Recognition}, pp.\  15077--15087, 2024.

\bibitem[Williams \& Baxter(2002)Williams and Baxter]{williams2002icarus}
Stephen Williams and Michael Baxter.
\newblock Icarus verilog: open-source verilog more than a year later.
\newblock \emph{Linux Journal}, 2002\penalty0 (99):\penalty0 3, 2002.

\bibitem[Wu et~al.(2024)Wu, He, Zhang, Yao, Zheng, Zheng, and Yu]{wu2024chateda}
Haoyuan Wu, Zhuolun He, Xinyun Zhang, Xufeng Yao, Su~Zheng, Haisheng Zheng, and Bei Yu.
\newblock Chateda: A large language model powered autonomous agent for eda.
\newblock \emph{IEEE Transactions on Computer-Aided Design of Integrated Circuits and Systems}, 2024.

\bibitem[Xu et~al.(2020)Xu, Zhang, Mao, Wang, Xie, and Zhang]{xu2020curriculum}
Benfeng Xu, Licheng Zhang, Zhendong Mao, Quan Wang, Hongtao Xie, and Yongdong Zhang.
\newblock Curriculum learning for natural language understanding.
\newblock In \emph{Proceedings of the 58th Annual Meeting of the Association for Computational Linguistics}, pp.\  6095--6104, 2020.

\bibitem[Yan et~al.(2017)Yan, Liu, Li, Han, and Qiu]{yan2017privmin}
Ziqi Yan, Jiqiang Liu, Gang Li, Zhen Han, and Shuo Qiu.
\newblock Privmin: Differentially private minhash for jaccard similarity computation.
\newblock \emph{arXiv preprint arXiv:1705.07258}, 2017.

\bibitem[Zhang et~al.(2024)Zhang, Yu, Fu, Wan, et~al.]{zhang2024mg}
Yongan Zhang, Zhongzhi Yu, Yonggan Fu, Cheng Wan, et~al.
\newblock Mg-verilog: Multi-grained dataset towards enhanced llm-assisted verilog generation.
\newblock \emph{arXiv preprint arXiv:2407.01910}, 2024.

\end{thebibliography}


\newpage
\appendix
% \section{Appendix}

\section{Introduction of Verilog}
\label{appendix:verilog_introduction}

Verilog is the most widely used hardware description language (HDL) for modeling digital integrated circuits. It enables designers to specify both the behavioral and structural aspects of hardware systems, such as processors, controllers, and digital logic circuits. Verilog operates at a relatively low level, focusing on gates, registers, and signal assignments—each representing physical hardware components. While Verilog supports behavioral constructs (\textit{e.g.}, \texttt{if-else}, \texttt{case}) that are somewhat similar to software programming languages, their use is constrained by synthesizable coding styles required for hardware implementation.
Verilog differs from software programming languages like Python and C++ in several key ways:


\begin{enumerate}
    \item \textbf{Parallelism:} Verilog inherently models hardware’s concurrnet nature, with multiple statements executing simultaneously. In contrast, software languages like Python typically follow a sequential execution model.
    \item \textbf{Timing:} Timing is a fundamental concept in Verilog that directly influences how digital circuits are designed and simulated. Verilog relies on clocks to synchronize sequential logic behaviors, enabling the precise modeling of synthronous circuits. In contrast, software programming languages generally do not have an inherent need for explicit timing.
    \item \textbf{Syntax and Constructs:} Verilog’s syntax is tailored to describe the behavior and structure of digital circuits, reflecting the parallel nature of hardware. Key constructs of Verilog include:
    
    \begin{itemize}
        \item {\textbf{Modules:}}
        The basic unit of Verilog, used to define a hardware block or component. Each module in Verilog encapsulates inputs, outputs, and internal logic, and modules can be instantiated within other modules, enabling hierarchical designs that mirror the complexity of real-world systems. And each module instantiation results in the generation of a corresponding circuit block.
        \item {\textbf{Always block:}}
        In an \texttt{always} block, circuit designers can model circuits using high-level behavioral descriptions. However, this does not imply that a broad range of programming language syntax is available. In practice, Verilog supports only a limited subset of programming-like constructs, primarily \texttt{if-else} and \texttt{case} statements. Statements in multiple \texttt{always} blocks are executed in parallel and the resulting circuit continuously performs its operations.
        \item {\textbf{Sensitivity list:}}
        In an \texttt{always} block, the sensitivity list specifies the signals that trigger the block’s execution when they change.
        \item {\textbf{Assign statements:}}
        \texttt{assign} statements are used to describe continuous assignments of signal values in parallel, reflecting the inherent concurrency of hardware.
        \item {\textbf{Registers (\texttt{reg}) and Wires (\texttt{wire}):}}
        \texttt{reg} is used for variables that retain their values (\textit{e.g.}, flip-flops or memory), and \texttt{wire} is used for connections that propagate values through the circuit.
        
    \end{itemize}

    In contrast, software programming languages like C, Python, or Java employ a more conventional syntax for defining algorithms, control flow, and data manipulation. These languages use constructs like loops (\texttt{for}, \texttt{while}), conditionals (\texttt{if}, \texttt{else}), and functions or methods for structuring code, with data types such as integers, strings, and floats for variable storage.

\end{enumerate}


\section{Prompt Details for CoT Annotation}
\label{appendix:prompt}

\begin{figure}[ht]
    \centering
    \includegraphics[width=\linewidth]{fig/prompt_v2.pdf}
    \caption{Detailed prompts used in the CoT annotation process.}
    \label{fig:prompt}
\end{figure}

As shown in Figure~\ref{fig:prompt}, we present the detailed prompts used in our annotation process. 
For each task, we supplement the primary prompt with several human-reviewed input-output pair examples, serving as in-context learning examples to enhance GPT's understanding of task requirements and expectations.
%In addition to the prompt listed, for each task, we will also provide GPT with several human-reviewed input-output pair examples as initial input to help it better understand the task requirements and expectations. 
These examples will serve as guidance for the model to correctly interpret and execute tasks in accordance with the prompt, ensuring more accurate and contextually relevant outputs.

\section{Discarding Verilog Code Exceeding $2048$ Tokens}
\label{appendix:discard}
In the main submission, we state that Verilog modules and blocks exceeding $2048$ tokens are excluded, as $2048$ is the maximum input length supported by CodeT5+. Beyond this limitation, several additional factors motivate this decision:
\newpage

\begin{figure}[ht]
    \centering
    \vspace{-20pt}
    \includegraphics[width=0.9\linewidth]{fig/distribution.jpg}
    \vspace{-10pt}
    \caption{The distribution of the token lengths of the generation benchmark by~\citet{chang2024natural}.}
    \label{fig:distribution}
\end{figure}
\begin{enumerate}
    \item \textbf{Generation Capabilities of Existing LLMs Are Limited to Small Designs}

    Existing benchmarks for Verilog generation, including the one used in our work~\citep{chang2024natural}, do not include designs exceeding $2048$ tokens, with the maximum token length observed in the benchmark being $1851$. As shown in Table~\ref{tab:generation_results} of the main submission, even the state-of-the-art LLM, o1-preview, is capable of accurately generating only simple designs and struggles with more complex ones. 
    Figure~\ref{fig:distribution} illustrates the token length distribution across the benchmark, further justifying our decision to exclude Verilog modules and blocks exceeding $2048$ tokens.

    \item \textbf{Segmentation as a Common Practice}

    Segmenting longer code into smaller chunks that fit within the predefined context window and discarding those that exceed it is a widely accepted practice in both Verilog-related research~\citep{chang2024data,pei2024betterv} and studies on software programming language~\citep{wang2023codet5+}. This approach ensures compatibility with current LLMs while maintaining the integrity and usability of the dataset. It is worth noting that the default maximum sequence length in CodeT5+ is $512$ tokens, and our work extends this limit to $2048$ tokens to better accommodate Verilog designs.

    \item \textbf{Empirical Findings and Practical Challenges}
    
    Our experiments reveal an important empirical observation: existing LLMs, such as GPT-4, consistently produce accurate descriptions for shorter Verilog modules but struggle with correctness when handling longer ones. Specifically, During the annotation process, we divide the dataset into two sections: Verilog designs with fewer than $2048$ tokens, and designs with token lengths between $2048$ and $4096$ tokens. Our human evaluation finds that descriptions for Verilog designs with fewer than $2048$ tokens are approximately 90\% accurate, while descriptions for designs with token lengths between $2048$ and $4096$ tokens have accuracy rates of only 60\%–70\%. And accuracy further decreases for designs exceeding $4096$ tokens. Since our datasets rely on LLM-generated annotations, restricting the dataset to Verilog modules within the $2048$-token limit helps maintain the quality and accuracy of annotations. This, in turn, facilitates higher-quality dataset creation and more efficient fine-tuning. For the potential negative impact of incorporating Verilog designs larger than $2048$ tokens, please refer to Appendix~\ref{appendix:negative_impact}.
    And we examine the impact of varying context window lengths in Appendix~\ref{appendix:varying_context_window_length}.
\end{enumerate}


\section{Standards and Processes for Manual Code Annotation}
\label{appendix:standard}

Given the industrial-grade quality of the proprietary code, we employ professional hardware engineers for manual annotation. We have established the following standards and processes to guide engineers in crafting accurate and detailed descriptions with example annotations shown in Figure~\ref{fig:engineer}:

\begin{enumerate}
    \item \textbf{Standards:} The hardware engineers are required to provide descriptions at both the module and block levels.
    
    \begin{itemize}
        \item For module-level descriptions, two levels are defined:
        \begin{itemize}
            \item[i.] \textbf{H (High-level):} The role of this module in the overall design (IP/Chip).
            \item[ii.] \textbf{D (Detailed):} What functions this module performs (overview) and how it is implemented (implementation details). This description should adhere to a top-down structure and consist of approximately 2-5 sentences.
        \end{itemize}
        \textbf{Note:} If the summary statements for H and D are identical, both must be provided.
        
        \item For block-level descriptions, particularly \texttt{always} blocks, descriptions are required at three distinct levels:
        \begin{itemize}
            \item[i.] \textbf{H (High-level):} The role of this block in the overall design (\textit{e.g.}, across modules).
            \item[ii.] \textbf{M (Medium-detail):} Contextual explanations.
            \item[iii.] \textbf{D (Detailed):} Descriptions specific to the block following a top-down structure. If details are absent, they may be omitted; do not guess based on signal names.
        \end{itemize}
    \end{itemize}
    
    \item \textbf{Processes:} Initially, we provide engineers with a set of descriptions generated by GPT-4 for reference. They are then expected to revise and enhance these GPT-generated descriptions using their expertise and relevant supplementary materials, such as README files and register tables.

\end{enumerate}

\begin{figure}[ht]
    \centering
    \includegraphics[width=\linewidth]{fig/engineer.pdf}
    \caption{Human-annotated examples for the proprietary code.}
    \label{fig:engineer}
\end{figure}

\section{Examples of Verilog Understanding Benchmark}
\label{appendix:benchmark}

To construct a high-quality benchmark, we first remove comments from the original code, and then submit it to experienced hardware engineers for annotation, ultimately producing the code and description pairs as shown in Figure~\ref{fig:verified_data}.


\begin{figure}[ht]
    \centering
    \includegraphics[width=\linewidth]{fig/verified_data_v2.pdf}
    \caption{Examples from the Verilog understanding benchmark.}
    \label{fig:verified_data}
\end{figure}


\section{Model Selection}
\label{appendix:model_selection}
In this work, we choose CodeT5+, a family of encoder-decoder code foundation models, as the base model for training DeepRTL for two primary reasons. First, as we aim to develop a unified model for Verilog understanding and generation, T5-like models are particularly well-suited due to their ability to effectively handle both tasks, as evidenced by~\citet{wang2023codet5+}. Second, the encoder component of CodeT5+ enables the natural extraction of Verilog representations, which can be potentially utilized for various downstream tasks in EDA at the RTL stage. Examples include PPA (Power, Performance, Area) prediction, which estimates the power consumption, performance, and area of an RTL design, and verification, which ensures that the RTL design correctly implements its intended functionality and meets specification requirements. Both tasks are crucial in the hardware design process. This capability distinguishes it from decoder-only models, which are typically less suited for producing standalone, reusable intermediate representations. In future work, we plan to explore how DeepRTL can further enhance productivity in the hardware design process.

To further demonstrate the superiority of CodeT5+ as a base model, we fine-tune two additional models, deepseek-coder-1.3b-instruct\footnote{\url{https://huggingface.co/deepseek-ai/deepseek-coder-1.3b-instruct}} (deepseek-coder)~\citep{guo2024deepseek} and Llama-3.2-1B-Instruct\footnote{\url{https://huggingface.co/meta-llama/Llama-3.2-1B-Instruct}} (llama-3.2)~\citep{dubey2024llama}, using the same dataset as DeepRTL and the adopted curriculum learning strategy.

In Table~\ref{tab:understanding_additional} and Table~\ref{tab:decoder_compare}, we present the performance of both the original base models and their fine-tuned counterparts on Verilog understanding and generation tasks. The improvement in performance from the original base models to the fine-tuned models highlights the effectiveness of our dataset and the curriculum learning-based fine-tuning strategy. Compared to the results in Table~\ref{tab:understanding_results} and Table~\ref{tab:generation_results}, the superior performance of DeepRTL-220m on both tasks, despite its smaller model size, underscores the architectural advantages of our approach.


\begin{table}[ht]
\centering
\vspace{-20pt}
\caption{Evaluation results on Verilog understanding using the benchmark proposed in Section~\ref{sec:understanding_benchmark}. BLEU-4 denotes the smoothed BLEU-4 score, and Emb. Sim. represents the embedding similarity metric. Specifically, this table presents the performance of decoder-only models, where ``long'' indicates models fine-tuned on the dataset containing longer Verilog designs, and those fine-tuned specifically on Verilog. $^\dag$ indicates performance evaluated on designs shorter than $512$ tokens.}
\vspace{2pt}
\label{tab:understanding_additional}
% \small{
\resizebox{\columnwidth}{!}{%
\begin{tabular}{@{}l|ccccccc@{}}
    \toprule
Model & BLEU-4 & ROUGE-1 & ROUGE-2 & ROUGE-L & Emb. Sim. & GPT Score \\
\midrule
deepseek-coder (original) & 1.04 & 21.43 & 4.38 & 19.77 & 0.678 & 0.557 \\
deepseek-coder (fine-tuned) & 11.96 & 40.49 & 19.77 & 36.14 & 0.826 & 0.664 \\
deepseek-coder (long) & 11.27 & 40.28 & 18.95 & 35.93 & 0.825 & 0.649 \\
\midrule
llama-3.2 (original) & 0.88 & 19.26 & 3.60 & 17.64 & 0.615 & 0.449 \\
llama-3.2 (fine-tuned) & 12.11 & 39.95 & 19.47 & 35.29 & 0.825 & 0.620 \\
llama-3.2 (long) & 11.32 & 39.60 & 18.67 & 34.94 & 0.814 & 0.610 \\
\midrule
RTLCoder & 1.08 & 21.83 & 4.68 & 20.30 & 0.687 & 0.561 \\
VeriGen & 0.09 & 6.54 & 0.35 & 6.08 & 0.505 & 0.311 \\
\midrule
DeepRTL-220m-512$^\dag$	& 14.98	& 44.27	& 23.11	& 40.08	& 0.780	& 0.567 \\
DeepRTL-220m$^\dag$	& 18.74	& 48.41	& 29.82	& 45.01	& 0.855	& 0.743 \\
\bottomrule
\end{tabular}%
}
\end{table}


\begin{table}[!ht]
\centering
\vspace{-5pt}
\caption{Evaluation results on Verilog generation. Each cell displays the percentage of code samples,
out of five trials, that successfully pass compilation (syntax column) or functional unit tests (function
column). This table presents the performance of decoder-only models, where ``o'' denotes the original model and ``f'' denotes the fine-tuned model.}
\vspace{5pt}
\label{tab:decoder_compare}
% {\tiny
\resizebox{\columnwidth}{!}{%
\begin{tabular}{|cl|cc|cc|cc|cc|}
\hline
\multicolumn{2}{|c|}{\multirow{2}{*}{Benchmark}} & \multicolumn{2}{c|}{deepseek-coder (o)} & \multicolumn{2}{c|}{deepseek-coder (f)} & \multicolumn{2}{c|}{llama-3.2 (o)} & \multicolumn{2}{c|}{llama-3.2 (f)} \\ \cline{3-10} 
\multicolumn{2}{|c|}{} & \multicolumn{1}{c|}{syntax} & function & \multicolumn{1}{c|}{syntax} & function & \multicolumn{1}{c|}{syntax} & function & \multicolumn{1}{c|}{syntax} & function \\ \hline
\multicolumn{1}{|c|}{\multirow{10}{*}{Logic}} & Johnson\_Counter & \multicolumn{1}{c|}{100\%} & 0\% & \multicolumn{1}{c|}{100\%} & 0\% & \multicolumn{1}{c|}{100\%} & 0\% & \multicolumn{1}{c|}{100\%} & 0\% \\ \cline{2-10} 
\multicolumn{1}{|c|}{} & alu & \multicolumn{1}{c|}{0\%} & 0\% & \multicolumn{1}{c|}{0\%} & 0\% & \multicolumn{1}{c|}{0\%} & 0\% & \multicolumn{1}{c|}{0\%} & 0\% \\ \cline{2-10} 
\multicolumn{1}{|c|}{} & edge\_detect & \multicolumn{1}{c|}{60\%} & 0\% & \multicolumn{1}{c|}{80\%} & 20\% & \multicolumn{1}{c|}{60\%} & 0\% & \multicolumn{1}{c|}{80\%} & 0\% \\ \cline{2-10} 
\multicolumn{1}{|c|}{} & freq\_div & \multicolumn{1}{c|}{80\%} & 0\% & \multicolumn{1}{c|}{100\%} & 0\% & \multicolumn{1}{c|}{80\%} & 0\% & \multicolumn{1}{c|}{100\%} & 0\% \\ \cline{2-10} 
\multicolumn{1}{|c|}{} & mux & \multicolumn{1}{c|}{60\%} & 0\% & \multicolumn{1}{c|}{100\%} & 100\% & \multicolumn{1}{c|}{60\%} & 0\% & \multicolumn{1}{c|}{60\%} & 60\% \\ \cline{2-10} 
\multicolumn{1}{|c|}{} & parallel2serial & \multicolumn{1}{c|}{80\%} & 0\% & \multicolumn{1}{c|}{100\%} & 0\% & \multicolumn{1}{c|}{80\%} & 0\% & \multicolumn{1}{c|}{100\%} & 0\% \\ \cline{2-10} 
\multicolumn{1}{|c|}{} & pulse\_detect & \multicolumn{1}{c|}{60\%} & 0\% & \multicolumn{1}{c|}{80\%} & 40\% & \multicolumn{1}{c|}{60\%} & 20\% & \multicolumn{1}{c|}{60\%} & 40\% \\ \cline{2-10} 
\multicolumn{1}{|c|}{} & right\_shifter & \multicolumn{1}{c|}{20\%} & 0\% & \multicolumn{1}{c|}{80\%} & 80\% & \multicolumn{1}{c|}{20\%} & 0\% & \multicolumn{1}{c|}{40\%} & 40\% \\ \cline{2-10} 
\multicolumn{1}{|c|}{} & serial2parallel & \multicolumn{1}{c|}{100\%} & 0\% & \multicolumn{1}{c|}{100\%} & 0\% & \multicolumn{1}{c|}{100\%} & 0\% & \multicolumn{1}{c|}{100\%} & 0\% \\ \cline{2-10} 
\multicolumn{1}{|c|}{} & width\_8to16 & \multicolumn{1}{c|}{100\%} & 0\% & \multicolumn{1}{c|}{100\%} & 0\% & \multicolumn{1}{c|}{100\%} & 0\% & \multicolumn{1}{c|}{100\%} & 0\% \\ \hline
\multicolumn{1}{|c|}{\multirow{11}{*}{Arithmetic}} & accu & \multicolumn{1}{c|}{80\%} & 0\% & \multicolumn{1}{c|}{100\%} & 0\% & \multicolumn{1}{c|}{80\%} & 0\% & \multicolumn{1}{c|}{100\%} & 0\% \\ \cline{2-10} 
\multicolumn{1}{|c|}{} & adder\_16bit & \multicolumn{1}{c|}{20\%} & 0\% & \multicolumn{1}{c|}{40\%} & 20\% & \multicolumn{1}{c|}{20\%} & 0\% & \multicolumn{1}{c|}{20\%} & 20\% \\ \cline{2-10} 
\multicolumn{1}{|c|}{} & adder\_16bit\_csa & \multicolumn{1}{c|}{0\%} & 0\% & \multicolumn{1}{c|}{0\%} & 20\% & \multicolumn{1}{c|}{0\%} & 20\% & \multicolumn{1}{c|}{20\%} & 20\% \\ \cline{2-10} 
\multicolumn{1}{|c|}{} & adder\_32bit & \multicolumn{1}{c|}{0\%} & 0\% & \multicolumn{1}{c|}{20\%} & 0\% & \multicolumn{1}{c|}{0\%} & 0\% & \multicolumn{1}{c|}{20\%} & 20\% \\ \cline{2-10} 
\multicolumn{1}{|c|}{} & adder\_64bit & \multicolumn{1}{c|}{0\%} & 0\% & \multicolumn{1}{c|}{20\%} & 0\% & \multicolumn{1}{c|}{0\%} & 0\% & \multicolumn{1}{c|}{40\%} & 0\% \\ \cline{2-10} 
\multicolumn{1}{|c|}{} & adder\_8bit & \multicolumn{1}{c|}{40\%} & 0\% & \multicolumn{1}{c|}{80\%} & 20\% & \multicolumn{1}{c|}{40\%} & 0\% & \multicolumn{1}{c|}{60\%} & 20\% \\ \cline{2-10} 
\multicolumn{1}{|c|}{} & div\_16bit & \multicolumn{1}{c|}{0\%} & 0\% & \multicolumn{1}{c|}{20\%} & 0\% & \multicolumn{1}{c|}{0\%} & 0\% & \multicolumn{1}{c|}{0\%} & 0\% \\ \cline{2-10} 
\multicolumn{1}{|c|}{} & multi\_16bit & \multicolumn{1}{c|}{60\%} & 0\% & \multicolumn{1}{c|}{80\%} & 0\% & \multicolumn{1}{c|}{60\%} & 0\% & \multicolumn{1}{c|}{80\%} & 0\% \\ \cline{2-10} 
\multicolumn{1}{|c|}{} & multi\_booth & \multicolumn{1}{c|}{40\%} & 0\% & \multicolumn{1}{c|}{60\%} & 0\% & \multicolumn{1}{c|}{40\%} & 0\% & \multicolumn{1}{c|}{60\%} & 0\% \\ \cline{2-10} 
\multicolumn{1}{|c|}{} & multi\_pipe\_4bit & \multicolumn{1}{c|}{100\%} & 0\% & \multicolumn{1}{c|}{100\%} & 100\% & \multicolumn{1}{c|}{100\%} & 0\% & \multicolumn{1}{c|}{100\%} & 100\% \\ \cline{2-10} 
\multicolumn{1}{|c|}{} & multi\_pipe\_8bit & \multicolumn{1}{c|}{0\%} & 0\% & \multicolumn{1}{c|}{0\%} & 0\% & \multicolumn{1}{c|}{0\%} & 0\% & \multicolumn{1}{c|}{0\%} & 0\% \\ \hline
\multicolumn{1}{|c|}{\multirow{10}{*}{Advanced}} & 1x2nocpe & \multicolumn{1}{c|}{60\%} & 0\% & \multicolumn{1}{c|}{20\%} & 40\% & \multicolumn{1}{c|}{60\%} & 20\% & \multicolumn{1}{c|}{60\%} & 20\% \\ \cline{2-10} 
\multicolumn{1}{|c|}{} & 1x4systolic & \multicolumn{1}{c|}{20\%} & 0\% & \multicolumn{1}{c|}{100\%} & 100\% & \multicolumn{1}{c|}{20\%} & 0\% & \multicolumn{1}{c|}{20\%} & 20\% \\ \cline{2-10} 
\multicolumn{1}{|c|}{} & 2x2systolic & \multicolumn{1}{c|}{0\%} & 0\% & \multicolumn{1}{c|}{0\%} & 0\% & \multicolumn{1}{c|}{0\%} & 0\% & \multicolumn{1}{c|}{0\%} & 0\% \\ \cline{2-10} 
\multicolumn{1}{|c|}{} & 4x4spatialacc & \multicolumn{1}{c|}{0\%} & 0\% & \multicolumn{1}{c|}{0\%} & 0\% & \multicolumn{1}{c|}{0\%} & 0\% & \multicolumn{1}{c|}{0\%} & 0\% \\ \cline{2-10} 
\multicolumn{1}{|c|}{} & fsm & \multicolumn{1}{c|}{80\%} & 0\% & \multicolumn{1}{c|}{100\%} & 100\% & \multicolumn{1}{c|}{80\%} & 0\% & \multicolumn{1}{c|}{100\%} & 100\% \\ \cline{2-10} 
\multicolumn{1}{|c|}{} & macpe & \multicolumn{1}{c|}{0\%} & 0\% & \multicolumn{1}{c|}{0\%} & 0\% & \multicolumn{1}{c|}{0\%} & 0\% & \multicolumn{1}{c|}{0\%} & 0\% \\ \cline{2-10} 
\multicolumn{1}{|c|}{} & 5state\_fsm & \multicolumn{1}{c|}{80\%} & 0\% & \multicolumn{1}{c|}{100\%} & 20\% & \multicolumn{1}{c|}{80\%} & 0\% & \multicolumn{1}{c|}{100\%} & 100\% \\ \cline{2-10} 
\multicolumn{1}{|c|}{} & 3state\_fsm & \multicolumn{1}{c|}{0\%} & 0\% & \multicolumn{1}{c|}{100\%} & 80\% & \multicolumn{1}{c|}{20\%} & 20\% & \multicolumn{1}{c|}{100\%} & 100\% \\ \cline{2-10} 
\multicolumn{1}{|c|}{} & 4state\_fsm & \multicolumn{1}{c|}{80\%} & 0\% & \multicolumn{1}{c|}{100\%} & 40\% & \multicolumn{1}{c|}{80\%} & 20\% & \multicolumn{1}{c|}{100\%} & 20\% \\ \cline{2-10} 
\multicolumn{1}{|c|}{} & 2state\_fsm & \multicolumn{1}{c|}{60\%} & 0\% & \multicolumn{1}{c|}{100\%} & 20\% & \multicolumn{1}{c|}{60\%} & 0\% & \multicolumn{1}{c|}{100\%} & 20\% \\ \hline
\multicolumn{2}{|c|}{Success Rate} & \multicolumn{1}{c|}{44.52\%} & 0.00\% & \multicolumn{1}{c|}{63.87\%} & 25.81\% & \multicolumn{1}{c|}{45.16\%} & 3.23\% & \multicolumn{1}{c|}{58.71\%} & 22.58\% \\ \hline
\multicolumn{2}{|c|}{Pass @ 1} & \multicolumn{1}{c|}{12.90\%} & 0.00\% & \multicolumn{1}{c|}{61.29\%} & 22.58\% & \multicolumn{1}{c|}{12.90\%} & 0.00\% & \multicolumn{1}{c|}{54.84\%} & 19.35\% \\ \hline
\multicolumn{2}{|c|}{Pass @ 5} & \multicolumn{1}{c|}{67.74\%} & 0.00\% & \multicolumn{1}{c|}{80.65\%} & 48.39\% & \multicolumn{1}{c|}{70.97\%} & 16.13\% & \multicolumn{1}{c|}{80.65\%} & 48.39\% \\ \hline
\end{tabular}%
}
% }
\end{table}

\section{Instructions for Different Scenarios}
\label{appendix:instruction}
Figure~\ref{fig:instruction_example} presents detailed instruction samples for different scenarios, following the instruction construction process illustrated in Figure~\ref{fig:instruction}.
Additionally, it includes a special module-level task, which involves completing the source code based on the functional descriptions of varying granularity and the predefined module header.

\begin{figure}[ht]
    \centering
    \includegraphics[width=0.98\linewidth]{fig/instruction_example_v2.pdf}
    \caption{Instruction tuning data samples for different scenarios.}
    \label{fig:instruction_example}
\end{figure}

\section{Further Explanation of the Adopted Curriculum Learning Strategy}
\label{appendix:explanation_curriculum_learning}
Our dataset includes three levels of annotation: line, block, and module, with each level containing descriptions that span various levels of detail—from detailed specifications to high-level functional descriptions. And the entire dataset is utilized for training. To fully leverage the potential of this dataset, we employ a curriculum learning strategy, enabling the model to incrementally build knowledge by starting with simpler cases and advancing to more complex ones.

The curriculum learning strategy involves transitioning from more granular to less granular annotations across hierarchical levels, which can be conceptualized as a tree structure with the following components (as shown in Figure~\ref{fig:tree}):

\begin{figure}[ht]
    \centering
    \includegraphics[width=\linewidth]{fig/tree.pdf}
    \caption{The adopted curriculum learning strategy visualized as a tree structure. Specifically, the terminals of the tree, enclosed by blue dotted boxes, represent specific training datasets. Our curriculum learning strategy follows a pre-order traversal of this tree structure.}
    \label{fig:tree}
\end{figure}

\begin{enumerate}
    \item \textbf{Hierarchical Levels (First Layer)}
    
    The training process transitions sequentially across the three hierarchical levels—line, block, and module. Each level is fully trained before moving to the next, ensuring a solid foundation at simpler levels before addressing more complex tasks.
    \item \textbf{Granularity of Descriptions (Second Layer)}
    
    Within each hierarchical level, the annotations transition from detailed descriptions to high-level descriptions. This progression ensures that the model learns finer details first and then builds an understanding of higher-level abstractions.
    \newpage
    \item \textbf{Annotation Source Transition (Third Layer)}
    
    At each level and granularity, training starts with GPT-annotated data and is followed by human-annotated data. This sequence leverages large-scale machine-generated annotations first and refines the model with high-quality, human-curated data.
    
    \item \textbf{Instruction Blending}
    
    Each terminal node in this tree represents a specific training dataset, which blends tasks for Verilog understanding and Verilog generation. This enables the model to perform well across diverse tasks.
\end{enumerate}

The training process mirrors a pre-order traversal of this tree structure:
\begin{enumerate}
    \item Starting at the root, training begins with the line level.
    \item The model progresses through the second layer (detailed, medium-detail, and high-level descriptions).
    \item Within each granularity, training transitions through the third layer (GPT-annotated data first, followed by human-annotated data).
    \item Once the line level is complete, the process repeats for the block level and then the module level.
\end{enumerate}

% This progressive training strategy aligns closely with the principles of curriculum learning, where simpler concepts are introduced first, and the knowledge gained is transferred incrementally to handle more complex scenarios.

To validate the effectiveness of this strategy, we conduct an ablation study where the model is trained on the entire dataset all at once without progression. The results, presented in Table~\ref{tab:understanding_results} of the main submission, demonstrate that the curriculum learning strategy significantly outperforms this baseline approach. Moreover, to the best of our knowledge, this is one of the first applications of a curriculum-like training strategy in the code-learning domain. Unlike existing Verilog-related models that establish simple and weak alignments between natural language and Verilog code~\citep{chang2024data}, or general software code datasets like CodeSearchNet\footnote{\url{https://huggingface.co/datasets/code-search-net/code_search_net}}~\citep{husain2019codesearchnet} that only provide single-level docstring annotations, our approach incorporates multi-level and multi-granularity annotations in a structured training process.


\section{Prompt for Calculating GPT Score}
\label{appendix:gpt_score}
To calculate the GPT score, we input the model’s generated descriptions (model\_output) and the ground truth annotations (ground\_truth) to GPT-4, using the prompt displayed in Figure~\ref{fig:gpt_score}. This metric is designed to assess the semantic accuracy of the generated functional descriptions.

\begin{figure}[ht]
    \centering
    \includegraphics[width=0.72\linewidth]{fig/gptscore.pdf}
    \caption{Prompt used to calculate the GPT score.}
    \label{fig:gpt_score}
\end{figure}

\section{Comparison with Models Specifically Trained on Verilog}
\label{appendix:comparison}

To further demonstrate the superiority of DeepRTL, we conduct experiments comparing it with models specifically trained on Verilog. 
We do not select~\citep{chang2024data,zhang2024mg} for comparison, as their models are not open-sourced, and it is non-trivial to reproduce their experiments. Additionally, the reported performance in their original papers is either comparable to, and in some cases inferior to, that of GPT-3.5. 
In Table~\ref{tab:understanding_additional} and Table~\ref{tab:verilog_specific}, we show the performance of two state-of-the-art Verilog generation models, RTLCoder-Deepseek-v1.1\footnote{\url{https://huggingface.co/ishorn5/RTLCoder-Deepseek-v1.1}} (RTLCoder)~\citep{liu2024rtlcoder} and fine-tuned-codegen-16B-Verilog\footnote{\url{https://huggingface.co/shailja/fine-tuned-codegen-16B-Verilog}} (VeriGen)~\citep{thakur2024verigen} on both Verilog understanding and generation benchmarks. It is noteworthy that RTLCoder is fine-tuned on DeepSeek-coder-6.7B, and VeriGen is fine-tuned on CodeGen-multi-16B, both of which have significantly larger parameter sizes than DeepRTL-220m. Despite this, the superior performance of DeepRTL-220m further underscores the effectiveness of our proposed dataset and the adopted curriculum learning strategy.


\begin{table}[!ht]
\centering
\caption{Evaluation results on Verilog generation. Each cell displays the percentage of code samples,
out of five trials, that successfully pass compilation (syntax column) or functional unit tests (function
column). This table presents the performance of models specifically trained on Verilog.}
\vspace{5pt}
\label{tab:verilog_specific}
{\tiny
\begin{tabular}{|cl|cc|cc|}
\hline
\multicolumn{2}{|c|}{\multirow{2}{*}{Benchmark}} & \multicolumn{2}{c|}{RTLCoder} & \multicolumn{2}{c|}{VeriGen} \\ \cline{3-6} 
\multicolumn{2}{|c|}{} & \multicolumn{1}{c|}{syntax} & function & \multicolumn{1}{c|}{syntax} & function \\ \hline
\multicolumn{1}{|c|}{\multirow{10}{*}{Logic}} & Johnson\_Counter & \multicolumn{1}{c|}{40\%} & 0\% & \multicolumn{1}{c|}{100\%} & 0\% \\ \cline{2-6} 
\multicolumn{1}{|c|}{} & alu & \multicolumn{1}{c|}{0\%} & 0\% & \multicolumn{1}{c|}{0\%} & 0\% \\ \cline{2-6} 
\multicolumn{1}{|c|}{} & edge\_detect & \multicolumn{1}{c|}{100\%} & 100\% & \multicolumn{1}{c|}{100\%} & 20\% \\ \cline{2-6} 
\multicolumn{1}{|c|}{} & freq\_div & \multicolumn{1}{c|}{60\%} & 0\% & \multicolumn{1}{c|}{100\%} & 0\% \\ \cline{2-6} 
\multicolumn{1}{|c|}{} & mux & \multicolumn{1}{c|}{60\%} & 40\% & \multicolumn{1}{c|}{80\%} & 20\% \\ \cline{2-6} 
\multicolumn{1}{|c|}{} & parallel2serial & \multicolumn{1}{c|}{60\%} & 0\% & \multicolumn{1}{c|}{100\%} & 0\% \\ \cline{2-6} 
\multicolumn{1}{|c|}{} & pulse\_detect & \multicolumn{1}{c|}{20\%} & 0\% & \multicolumn{1}{c|}{40\%} & 0\% \\ \cline{2-6} 
\multicolumn{1}{|c|}{} & right\_shifter & \multicolumn{1}{c|}{80\%} & 80\% & \multicolumn{1}{c|}{100\%} & 100\% \\ \cline{2-6} 
\multicolumn{1}{|c|}{} & serial2parallel & \multicolumn{1}{c|}{60\%} & 0\% & \multicolumn{1}{c|}{80\%} & 0\% \\ \cline{2-6} 
\multicolumn{1}{|c|}{} & width\_8to16 & \multicolumn{1}{c|}{60\%} & 0\% & \multicolumn{1}{c|}{100\%} & 0\% \\ \hline
\multicolumn{1}{|c|}{\multirow{11}{*}{Arithmetic}} & accu & \multicolumn{1}{c|}{0\%} & 0\% & \multicolumn{1}{c|}{0\%} & 0\% \\ \cline{2-6} 
\multicolumn{1}{|c|}{} & adder\_16bit & \multicolumn{1}{c|}{40\%} & 20\% & \multicolumn{1}{c|}{20\%} & 0\% \\ \cline{2-6} 
\multicolumn{1}{|c|}{} & adder\_16bit\_csa & \multicolumn{1}{c|}{80\%} & 80\% & \multicolumn{1}{c|}{0\%} & 0\% \\ \cline{2-6} 
\multicolumn{1}{|c|}{} & adder\_32bit & \multicolumn{1}{c|}{80\%} & 0\% & \multicolumn{1}{c|}{0\%} & 0\% \\ \cline{2-6} 
\multicolumn{1}{|c|}{} & adder\_64bit & \multicolumn{1}{c|}{40\%} & 0\% & \multicolumn{1}{c|}{40\%} & 0\% \\ \cline{2-6} 
\multicolumn{1}{|c|}{} & adder\_8bit & \multicolumn{1}{c|}{80\%} & 40\% & \multicolumn{1}{c|}{40\%} & 40\% \\ \cline{2-6} 
\multicolumn{1}{|c|}{} & div\_16bit & \multicolumn{1}{c|}{0\%} & 0\% & \multicolumn{1}{c|}{0\%} & 0\% \\ \cline{2-6} 
\multicolumn{1}{|c|}{} & multi\_16bit & \multicolumn{1}{c|}{80\%} & 0\% & \multicolumn{1}{c|}{80\%} & 0\% \\ \cline{2-6} 
\multicolumn{1}{|c|}{} & multi\_booth & \multicolumn{1}{c|}{20\%} & 0\% & \multicolumn{1}{c|}{20\%} & 0\% \\ \cline{2-6} 
\multicolumn{1}{|c|}{} & multi\_pipe\_4bit & \multicolumn{1}{c|}{60\%} & 20\% & \multicolumn{1}{c|}{80\%} & 20\% \\ \cline{2-6} 
\multicolumn{1}{|c|}{} & multi\_pipe\_8bit & \multicolumn{1}{c|}{0\%} & 0\% & \multicolumn{1}{c|}{0\%} & 0\% \\ \hline
\multicolumn{1}{|c|}{\multirow{10}{*}{Advanced}} & 1x2nocpe & \multicolumn{1}{c|}{40\%} & 40\% & \multicolumn{1}{c|}{100\%} & 100\% \\ \cline{2-6} 
\multicolumn{1}{|c|}{} & 1x4systolic & \multicolumn{1}{c|}{100\%} & 100\% & \multicolumn{1}{c|}{20\%} & 20\% \\ \cline{2-6} 
\multicolumn{1}{|c|}{} & 2x2systolic & \multicolumn{1}{c|}{0\%} & 0\% & \multicolumn{1}{c|}{0\%} & 0\% \\ \cline{2-6} 
\multicolumn{1}{|c|}{} & 4x4spatialacc & \multicolumn{1}{c|}{0\%} & 0\% & \multicolumn{1}{c|}{0\%} & 0\% \\ \cline{2-6} 
\multicolumn{1}{|c|}{} & fsm & \multicolumn{1}{c|}{100\%} & 60\% & \multicolumn{1}{c|}{80\%} & 20\% \\ \cline{2-6} 
\multicolumn{1}{|c|}{} & macpe & \multicolumn{1}{c|}{0\%} & 0\% & \multicolumn{1}{c|}{0\%} & 0\% \\ \cline{2-6} 
\multicolumn{1}{|c|}{} & 5state\_fsm & \multicolumn{1}{c|}{60\%} & 40\% & \multicolumn{1}{c|}{80\%} & 0\% \\ \cline{2-6} 
\multicolumn{1}{|c|}{} & 3state\_fsm & \multicolumn{1}{c|}{80\%} & 0\% & \multicolumn{1}{c|}{80\%} & 20\% \\ \cline{2-6} 
\multicolumn{1}{|c|}{} & 4state\_fsm & \multicolumn{1}{c|}{80\%} & 0\% & \multicolumn{1}{c|}{80\%} & 20\% \\ \cline{2-6} 
\multicolumn{1}{|c|}{} & 2state\_fsm & \multicolumn{1}{c|}{20\%} & 0\% & \multicolumn{1}{c|}{60\%} & 0\% \\ \hline
\multicolumn{2}{|c|}{Success Rate} & \multicolumn{1}{c|}{48.39\%} & 20.00\% & \multicolumn{1}{c|}{50.97\%} & 12.26\% \\ \hline
\multicolumn{2}{|c|}{Pass @ 1} & \multicolumn{1}{c|}{41.94\%} & 16.13\% & \multicolumn{1}{c|}{48.39\%} & 9.68\% \\ \hline
\multicolumn{2}{|c|}{Pass @ 5} & \multicolumn{1}{c|}{77.42\%} & 35.48\% & \multicolumn{1}{c|}{70.97\%} & 32.26\% \\ \hline
\end{tabular}%
}
\end{table}


\begin{table}[!ht]
\centering
\caption{Evaluation results on Verilog generation. Each cell displays the percentage of code samples,
out of five trials, that successfully pass compilation (syntax column) or functional unit tests (function
column). This table presents the performance of decoder-only models fine-tuned on the dataset containing longer Verilog designs.}
\label{tab:decoder_model_with_longer_designs}
\vspace{5pt}
{\tiny
% \resizebox{\columnwidth}{!} & 0\% & \multicolumn{1}{c|}{100\%} & 0\% \\ \cline{2-6} 
\multicolumn{1}{|c|}{} & alu & \multicolumn{1}{c|}{0\%} & 0\% & \multicolumn{1}{c|}{0\%} & 0\% \\ \cline{2-6} 
\multicolumn{1}{|c|}{} & edge\_detect & \multicolumn{1}{c|}{80\%} & 0\% & \multicolumn{1}{c|}{80\%} & 0\% \\ \cline{2-6} 
\multicolumn{1}{|c|}{} & freq\_div & \multicolumn{1}{c|}{100\%} & 0\% & \multicolumn{1}{c|}{100\%} & 0\% \\ \cline{2-6} 
\multicolumn{1}{|c|}{} & mux & \multicolumn{1}{c|}{100\%} & 100\% & \multicolumn{1}{c|}{60\%} & 60\% \\ \cline{2-6} 
\multicolumn{1}{|c|}{} & parallel2serial & \multicolumn{1}{c|}{100\%} & 0\% & \multicolumn{1}{c|}{100\%} & 0\% \\ \cline{2-6} 
\multicolumn{1}{|c|}{} & pulse\_detect & \multicolumn{1}{c|}{80\%} & 40\% & \multicolumn{1}{c|}{60\%} & 40\% \\ \cline{2-6} 
\multicolumn{1}{|c|}{} & right\_shifter & \multicolumn{1}{c|}{80\%} & 80\% & \multicolumn{1}{c|}{40\%} & 40\% \\ \cline{2-6} 
\multicolumn{1}{|c|}{} & serial2parallel & \multicolumn{1}{c|}{100\%} & 0\% & \multicolumn{1}{c|}{100\%} & 0\% \\ \cline{2-6} 
\multicolumn{1}{|c|}{} & width\_8to16 & \multicolumn{1}{c|}{100\%} & 0\% & \multicolumn{1}{c|}{100\%} & 0\% \\ \hline
\multicolumn{1}{|c|}{\multirow{11}{*}{Arithmetic}} & accu & \multicolumn{1}{c|}{100\%} & 0\% & \multicolumn{1}{c|}{100\%} & 0\% \\ \cline{2-6} 
\multicolumn{1}{|c|}{} & adder\_16bit & \multicolumn{1}{c|}{20\%} & 20\% & \multicolumn{1}{c|}{20\%} & 20\% \\ \cline{2-6} 
\multicolumn{1}{|c|}{} & adder\_16bit\_csa & \multicolumn{1}{c|}{20\%} & 20\% & \multicolumn{1}{c|}{20\%} & 20\% \\ \cline{2-6} 
\multicolumn{1}{|c|}{} & adder\_32bit & \multicolumn{1}{c|}{0\%} & 0\% & \multicolumn{1}{c|}{20\%} & 20\% \\ \cline{2-6} 
\multicolumn{1}{|c|}{} & adder\_64bit & \multicolumn{1}{c|}{0\%} & 0\% & \multicolumn{1}{c|}{0\%} & 0\% \\ \cline{2-6} 
\multicolumn{1}{|c|}{} & adder\_8bit & \multicolumn{1}{c|}{80\%} & 20\% & \multicolumn{1}{c|}{60\%} & 20\% \\ \cline{2-6} 
\multicolumn{1}{|c|}{} & div\_16bit & \multicolumn{1}{c|}{20\%} & 0\% & \multicolumn{1}{c|}{0\%} & 0\% \\ \cline{2-6} 
\multicolumn{1}{|c|}{} & multi\_16bit & \multicolumn{1}{c|}{80\%} & 0\% & \multicolumn{1}{c|}{80\%} & 0\% \\ \cline{2-6} 
\multicolumn{1}{|c|}{} & multi\_booth & \multicolumn{1}{c|}{60\%} & 0\% & \multicolumn{1}{c|}{60\%} & 0\% \\ \cline{2-6} 
\multicolumn{1}{|c|}{} & multi\_pipe\_4bit & \multicolumn{1}{c|}{100\%} & 100\% & \multicolumn{1}{c|}{100\%} & 100\% \\ \cline{2-6} 
\multicolumn{1}{|c|}{} & multi\_pipe\_8bit & \multicolumn{1}{c|}{0\%} & 0\% & \multicolumn{1}{c|}{0\%} & 0\% \\ \hline
\multicolumn{1}{|c|}{\multirow{10}{*}{Advanced}} & 1x2nocpe & \multicolumn{1}{c|}{40\%} & 40\% & \multicolumn{1}{c|}{60\%} & 20\% \\ \cline{2-6} 
\multicolumn{1}{|c|}{} & 1x4systolic & \multicolumn{1}{c|}{20\%} & 20\% & \multicolumn{1}{c|}{20\%} & 20\% \\ \cline{2-6} 
\multicolumn{1}{|c|}{} & 2x2systolic & \multicolumn{1}{c|}{0\%} & 0\% & \multicolumn{1}{c|}{0\%} & 0\% \\ \cline{2-6} 
\multicolumn{1}{|c|}{} & 4x4spatialacc & \multicolumn{1}{c|}{0\%} & 0\% & \multicolumn{1}{c|}{0\%} & 0\% \\ \cline{2-6} 
\multicolumn{1}{|c|}{} & fsm & \multicolumn{1}{c|}{100\%} & 100\% & \multicolumn{1}{c|}{100\%} & 100\% \\ \cline{2-6} 
\multicolumn{1}{|c|}{} & macpe & \multicolumn{1}{c|}{0\%} & 0\% & \multicolumn{1}{c|}{0\%} & 0\% \\ \cline{2-6} 
\multicolumn{1}{|c|}{} & 5state\_fsm & \multicolumn{1}{c|}{100\%} & 0\% & \multicolumn{1}{c|}{100\%} & 100\% \\ \cline{2-6} 
\multicolumn{1}{|c|}{} & 3state\_fsm & \multicolumn{1}{c|}{80\%} & 80\% & \multicolumn{1}{c|}{100\%} & 100\% \\ \cline{2-6} 
\multicolumn{1}{|c|}{} & 4state\_fsm & \multicolumn{1}{c|}{100\%} & 0\% & \multicolumn{1}{c|}{100\%} & 0\% \\ \cline{2-6} 
\multicolumn{1}{|c|}{} & 2state\_fsm & \multicolumn{1}{c|}{100\%} & 20\% & \multicolumn{1}{c|}{100\%} & 20\% \\ \hline
\multicolumn{2}{|c|}{Success Rate} & \multicolumn{1}{c|}{60.00\%} & 20.65\% & \multicolumn{1}{c|}{57.42\%} & 21.94\% \\ \hline
\multicolumn{2}{|c|}{Pass @ 1} & \multicolumn{1}{c|}{38.71\%} & 19.35\% & \multicolumn{1}{c|}{38.71\%} & 19.35\% \\ \hline
\multicolumn{2}{|c|}{Pass @ 5} & \multicolumn{1}{c|}{77.42\%} & 38.71\% & \multicolumn{1}{c|}{77.42\%} & 45.16\% \\ \hline
\end{tabular}%
}
\end{table}


\section{Negative Impact of Incorporating Verilog Designs Exceeding $2048$ Tokens}
\label{appendix:negative_impact}
Notably, the maximum input length for DeepSeek-coder is 16k tokens, while for LLaMA-3.2, it is 128k tokens. To assess the potential negative impact of including Verilog designs exceeding $2048$ tokens, we conduct an ablation study in which we do not exclude such modules for these two models and instead use the dataset containing longer designs for training. As shown in Table~\ref{tab:decoder_model_with_longer_designs}, and by comparing the results in Table~\ref{tab:understanding_additional}, the performance of the fine-tuned models on both Verilog understanding and generation tasks significantly degrades compared to the results in Table~\ref{tab:decoder_compare}, where these models are fine-tuned using the same dataset as DeepRTL. This further validates the rationale behind our decision to exclude Verilog modules and blocks exceeding $2048$ tokens.


\section{Additional Experiments Investigating the Impact of Varying Context Window Lengths}
\label{appendix:varying_context_window_length}
To address concerns regarding the potential bias introduced by excluding examples longer than $2048$ tokens, we investigate the impact of context window length. Specifically, we exclude all Verilog modules exceeding $512$ tokens and use the truncated dataset to train a new model, DeepRTL-220m-512 utilizing the curriculum learning strategy, which has a maximum input length of $512$ tokens. We then evaluate both DeepRTL-220m-512 and DeepRTL-220m on Verilog understanding benchmark samples, where the module lengths are below $512$ tokens, and present the results in Table~\ref{tab:understanding_additional}. For the generation task, DeepRTL-220m-512 shows near-zero performance, with nearly 0\% accuracy for both syntax and functional correctness. This result refutes the concern that ``a model accommodating longer context windows could potentially offer superior performance on the general task, but not for this tailored dataset," as it does not hold true in our case.

\end{document}
