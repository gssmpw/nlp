\section{Related Work}
\textbf{Long-term Causal Inference}
For decades, many studies have explored the validity of a surrogate, i.e., what kind of short-term outcomes can reliably predict long-term causal effects. 
Various criteria are proposed for a valid surrogate, e.g., prentice criteria ____, principal criteria ____, strong surrogate criteria ____, causal effect predictiveness ____, and consistent surrogate and its variants ____.
Recently, many works have studied estimating long-term causal effects based on surrogates. 
One prominent line of research assumes the unconfoundedness assumption.
Under the unconfoundedness assumption, LTEE ____ and Laser ____ are based on different designed neural networks for long-term causal inference.
EETE ____ studies the data efficiency from the surrogate and proposes efficient estimation for treatment effect.
ORL ____ proposes a doubly robust estimator for average treatment effects with only short-term experiments, additionally assuming stationarity conditions between short and long-term outcomes.
____ proposes a policy learning method for balancing short-term and long-term rewards.
\underline{Different} from these works, we do not assume the unconfoundedness assumption, and we use the data combination technique to solve the problem of unobserved confounders.
Another line of research, which also avoids the unconfoundedness assumption, tackles the issue by combining experimental and observational data — a setting known as data combination.
This setting is initialized by the method proposed by Athey et~al. ____, 
which, under surrogacy assumption, constructs the so-called Surrogate Index as the substitutions for long-term outcomes in the experimental data to achieve effect identification.
As follow-up work, ____ assumes latent unconfoundedness assumption, i.e., short-term potential outcomes can mediate the long-term potential outcomes, to identify long-term causal effects.
Other feasible assumptions ____  are proposed to replace the latent unconfoundedness assumption, e.g., the additive equi-confounding bias assumption.
Based on proximal methods, the sequential structure surrogates are studied ____.
Learn ____ proposes a reweighting schema to align observational data and experimental data, enabling effect identification.
\underline{However}, these works mostly focus on the average treatment effects or do not consider double/multiple robust estimators for heterogeneous causal effects.
Different from these works above, we address the overlooked problem by providing several heterogeneous long-term causal effect estimators, including regression-based, propensity score-based, and multiple robust estimators, and provide a comprehensive theoretical analysis of their properties.

\textbf{Double/Multiple Robustness}
A double/multiple Robust estimator is an estimator that remains consistent when part of nuisance functions are inconsistent.
Regarding average treatment effect estimation, the most well-known estimator is the augmented inverse propensity weighted (AIPW) estimator ____ in the traditional scenario. 
AIPW consists of a regression model and a propensity model ____, and it is consistent as long as one of the models is consistent.
Similarly, doubly robust estimators for average causal effects are proposed in various scenarios.
____ and ____ propose a doubly and multiple robust estimator, respectively, for average causal effect in the instrumental variable (IV) setting.
____ proposes a multiple robust estimator for mediation analysis.
For continuous average effect estimation, ____ proposes a nonparametric estimator leveraging kernel methods.
More related to our work, ____ proposes a multiple robust estimator for long-term average effects in the same setting as ours. 
\underline{However}, these works above are not applicable to estimate the heterogeneous effects.
Different from them, our work focuses on designing multiple robust heterogeneous effect estimator instead of average effect estimators.
Additionally, many works also study the double/multiple robust estimator for heterogeneous effects.
____ analyzes the doubly robust estimator in the standard setting and derives doubly robust convergence rates.
____ extends to the IV setting and proposes a corresponding multiple robust estimator.
\underline{However}, the multiple robust estimation for long-term heterogeneous effects is still an understudied problem. 
In this paper, we propose a multiple robust heterogeneous effect estimator based on neural networks and also provide a detailed theoretical analysis.