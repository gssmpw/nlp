\begin{table}[tp] 
\tiny
\centering
\caption{The rejection prompt used in the AutoGUI pipeline in UI manipulation cases. This example shows how the LLM assigns a low score to a sample that exhibits meaningless and unpredictable UI content changes.}
\label{tab:supp:rejection prompt}
\begin{tabular}{|l|}
\hline
\begin{tabular}[c]{@{}p{0.9\linewidth}@{}}\color{blue}{(Requirements for rejection)} \\Your primary objective is to determine whether the changes in the webpage's content are sufficient for predicting the functionality of the webpage element causing these changes after being interacted with.\\

Instructions:\\
1. You will be shown the outcome (webpage changes) resulting from interacting with the element. The outcome can take one of two forms:  changes to the webpage description, or line-by-line differences. For the latter form, here's what each prefix indicates:\\
Unchanged: Lines that are identical before and after the interaction.\\
Added: New lines that appear after the interaction.\\
Deleted: Lines that were present before the interaction but removed afterward.\\
Renaming: Lines indicating elements that were renamed due to the interaction.\\
Attribute Update: Lines showing elements whose attributes were updated during the interaction.\\
Repositioned: Elements that were moved to a different part of the webpage.\\
2. Analyze the provided outcome and provide detailed reasoning for whether this outcome helps to predict the element's functionality, considering the following stringent criteria:\\
1) Explicitness of Changes: Rate how directly the changes suggest the element's functionality. Score 1-3 for clear, unambiguous changes. Clearer changes obtain a higher score. No scores for vague, meaningless, or non-specific changes.\\
Positive Example: A button labeled ``Show More'' that, upon interaction, clearly adds new content below it. The direct addition of content clearly indicates a content expansion functionality. Score: 3\\
Negative Example: After clicking a ``Details'' button, the page layout changes subtly without adding relevant information or altering content in a meaningful way. The changes do not clearly relate to the button's presumed functionality. Score: 0\\

2) Relevance of Changes: Evaluate the significance of the modifications in relation to the element's intended function. Score 1-3 for changes that enhance understanding of the element’s role. Highly related modifications obtain a high score. No scores for irrelevant or unrelated content changes.\\
Positive Example: Clicking on a ``Contact Us'' button opens a form to fill out, which is highly relevant to the button's intended functionality. Score: 3\\
Negative Example: Clicking on a ``View Profile'' link leads to a page refresh without displaying the profile or any related information, making the change irrelevant to the link's intended purpose. Score: 0\\

3) Predictability of Outcome: Assess how anticipated the interaction outcome is based on the changes, considering common web conventions and user experience principles. Score 1-3 for highly predictable outcomes. Highly predictable changes obtain a high score. No scores for outcomes that are moderate, unexpected, or counterintuitive.\\
Positive Example: Clicking or hovering over a ``Help'' icon reveals a tooltip with information, an outcome that users can easily predict given the icon's universal symbol for help. Score: 3\\
Negative Example: Hovering over a ``Settings'' icon changes its color but does not display any settings options, tooltips, or lead to a settings page, making the outcome unpredictable and the icon's purpose unclear. Score: 0\\

Provide a detailed analysis based on these criteria, ending with an overall score between 0-9.\\

4. Your output MUST follow this format:\\
Reasoning:\\
- Explicitness of Changes: [Detailed reasoning and score]\\
- Relevance of Changes: [Detailed reasoning and score]\\
- Predictability of Outcome: [Detailed reasoning and score]\\

Overall score: <score>[criterium 1 score] + [criterium 2 score] + [criterium 3 score] = [total score]</score> (output an additive equation enclosed with a pair of <score></score>)\\

Now it's your turn.\\
The interacted element is \color{blue}{a <button> element named ``Cancel'' (interaction target)}\\
Outcome:\\
\color{blue}{(An example of interaction outcome)}\\
After clicking a <button> element named ``Cancel'', the webpage exhibits the following variations:\\
Unchanged 			StaticText `Email Address'\\
Unchanged 			textbox `Email Address' required: True\\
Unchanged 			button `Submit'\\
Unchanged 				StaticText `Submit'\\
Before Attribute Update 			button `Cancel'\\
After Attribute Update 			button `Cancel' focused: True\\
Unchanged 				StaticText `Cancel'\\
Unchanged 	contentinfo `'\\
Unchanged 		StaticText `Your contact information is used to deliver requested updates or to access your subscriber preferences.'\\ \\

Now please think step-by-step, examine the content variation in detail, and finally calculate scores according to the criteria. Do follow the output format ``Reasoning: ... Overall score: <score>...</score>''.\\ \midrule
\color{blue}{(LLM response)}\\
Reasoning:\\

- Explicitness of Changes: The changes resulting from interacting with the ``Cancel'' button are mostly unchanged elements, except for the ``Cancel'' button itself, which now has a ``focused: True'' attribute. This change is not directly related to the button's functionality, as it does not provide clear insight into what the button does when clicked. Score: 1\\

- Relevance of Changes: The changes do not provide any significant information about the button's intended function. The button's name ``Cancel'' suggests that it might cancel some action or process, but the changes do not reveal what exactly is being canceled or how the button's functionality relates to the surrounding elements. Score: 1\\

- Predictability of Outcome: Given the name ``Cancel'' and the context of a form with an ``Email Address'' field and a ``Submit'' button, a user might expect the ``Cancel'' button to reset the form, clear the input field, or abort a submission process. However, the outcome does not provide any clear indication of what happens when the button is clicked. The changes are not predictable based on common web conventions and user experience principles. Score: 1\\

Overall score: <score>1 + 1 + 1 = 3</score>

\end{tabular} \\ \hline
\end{tabular}
\end{table}