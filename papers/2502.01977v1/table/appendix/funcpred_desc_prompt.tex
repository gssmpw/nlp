\begin{table}[tp]
\tiny
\centering
\caption{The functionality annotation prompt used in the AutoGUI pipeline in UI navigation cases. This example shows how the LLM }
\label{tab:supp:funcpred nav prompt}
\begin{tabular}{|l|}
\hline
\begin{tabular}[c]{@{}p{0.9\linewidth}@{}}\color{blue}{(Requirements for annotation)} \\
Objective: Your mission, as a digital navigation specialist, is to deduce and articulate the function and usage of a specific webpage element. This deduction should be based on your analysis of the differences in webpage content before and after interacting with said element.\\

Instructions:\\
1. You will be given descriptions of a webpage before and after interaction with an element. Your primary task is to meticulously analyze the differences in content resulting from this interaction to understand what the functionality of the element is in the webpage context.\\
2. You must present a detailed reasoning process before finally summarizing the element's overall purpose based on your analysis.\\
3. Prioritize examining changes in the webpage's regional content over individual element variations. This approach will provide a more holistic view of the element's impact on the webpage.\\
4. You should emphasize on the main content changes and pay less attention to less meaningful regions, such as headers, navigation bars, and footers.\\
5. Your output MUST follow this format:\\
Reasoning: (Examine the webpage variation carefully to figure out how the interacted element changes the webpage)\\
Summary: This element ... (Provide a high-level description of the element's functionality. This description should contain the meaningful feature of the element in its context.)\\
6. Avoid mentioning specific elements from the webpage prior to interaction in the Summary. Instead, focus directly on the outcome of the interaction. For instance, rather than detailing element changes, like ``This element triggers the disappearance of the header and language selector elements and the emergence of a login form'', simplify it to ``This element triggers the display of a login page.'' This approach keeps the summary concise and focused on functionality, eliminating unnecessary details.\\
7. The final summary must emphasize the distinctive functionality of the element and not be too vague. For example, ``This element redirects the user to a specific website, allowing them to access its content and services''  is too broad and lacks meaningful detail. Instead, it should precisely describe how the element provides users with unique value\\

Now it's your turn.\\
The description of the initial webpage:\\
\color{blue}{(An example description of the UI before the interaction)}\\
Region 1 (Header): Navigation and login functionality\\
Region 2 (Main Content): Announcing updates to Instagram Direct Message, providing a blog post with a title, date, and content\\
Region 3 (Footer): Additional links and buttons (less meaningful region, summarized in a single sentence)\\
Overall Functionality: A blog post webpage announcing updates to Instagram Direct Message, providing a title, date, and content, with navigation and login functionality at the top.\\

After \color{blue}{clicking a <link> element named ``'' (an example of the interaction target)}, we jump to a new webpage.\\
The description of the new webpage is:\\
\color{blue}{(An example description of the UI after the interaction)}\\
Region 1 (Header): Displays the webpage title and heading.\\
Region 2 (Login Form): Allows users to log in to their Facebook account by providing their email address or phone number and password.\\
Region 3 (Forgot Password): Provides a link to recover a forgotten account password.\\
Region 4 (Create New Account): Offers an option to create a new Facebook account.\\
Overall Functionality: Enables users to access their existing Facebook account or create a new one, providing a gateway to the social media platform.\\ \\

Now please think step-by-step, examine the content variation in detail, and finally summarize the high-level usage of the interacted element. Do follow the output format ``Reasoning: ... Summary: This element ...''.\\ \midrule
\color{blue}{(LLM response)}\\
Reasoning: Upon examining the webpage variations, I notice that the initial webpage is a blog post announcing updates to Instagram Direct Message, with navigation and login functionality at the top. After clicking the `<link>` element named ``'', the webpage transitions to a new page with a Facebook title and heading. The main content of the new webpage is a login form, allowing users to log in to their Facebook account. Additionally, there are options to recover a forgotten password and create a new Facebook account.\\
Summary: This element redirects the user to a Facebook login page, allowing them to access their existing account or create a new one. \\


\end{tabular} \\ \hline
\end{tabular}
\end{table}