\begin{table}[tp]
\tiny
\centering
\caption{The functionality annotation prompt used in the AutoGUI pipeline in UI manipulation cases.}
\label{tab:supp:funcpred manip prompt}
\begin{tabular}{|l|}
\hline
\begin{tabular}[c]{@{}p{0.9\linewidth}@{}}\color{blue}{(Requirements for annotation)} \\
Objective: As an Internet expert, your task is to describe the usage and functionality of a webpage element based on the changes observed in the webpage contents before and after interacting with the element.\\

Instructions: \\
1. You will be shown line-by-line differences between the webpage content before and after interacting with the element. Here's what each prefix indicates:\\
Unchanged: Lines that are identical before and after the interaction.\\
Added: New lines that appear after the interaction.\\
Deleted: Lines that were present before the interaction but removed afterward.\\
Renaming: Lines indicating elements that were renamed due to the interaction.\\
Attribute Update: Lines showing elements whose attributes were updated during the interaction.\\
Repositioned: Elements that were moved to a different part of the webpage.\\
2. You MUST thoroughly analyze the changes in webpage content (Added, Deleted, Unchanged lines) caused by interacting with the element, present a detailed reasoning process elucidating how the element affects the webpage, and finally summarize the element's overall purpose based on your analysis\\
3. Avoid detailing every specific functionality of the webpage element. Instead, focus on describing its broader impact on the webpage experience. For example, if interacting with a ``Products'' button reveals a dropdown menu, do not catalog the subsequent webpage changes in exhaustive detail.\\
4. Your output MUST follow this format:\\
Reasoning: (Examine the webpage variation carefully to figure out how the interacted element changes the webpage)\\
Summary: This element ... (Provide a concise high-level description of the element's function. This description should contain the meaningful feature of the element in its context.)\\
5. Avoid mentioning specific elements from the webpage before interaction in the Summary. Instead, focus directly on the outcome of the interaction. For instance, rather than detailing a transition from one element to another, like ``This element switches the focus from the `Destination' input field to the `Check-in Date' field, providing a date picker,'' simplify it to ``This element triggers a date picker for users to select their check-in date.'' This approach keeps the description concise and focused on functionality, eliminating unnecessary details.\\
6. For simplicity, you MUST not mention the text of the element in the summary.\\
Now it's your turn.\\ \\
Element interaction: After \color{blue}{clicking a <link> element named ``Community submenu'' (an example of the interaction target)}, \color{black}{the webpage exhibits the following variations:}\\
\color{blue}{(UI content changes)}\\
Unchanged 	button `Extend submenu' expanded: False \\
Unchanged 		StaticText `Extend'\\
Unchanged 	button `Learn submenu' expanded: False\\
Unchanged 		StaticText `Learn'\\
Before Attribute Update 	button `Community submenu' expanded: False\\
After Attribute Update 	button `Community submenu' focused: True expanded: True\\
Unchanged 		StaticText `Community'\\
Added 	link `Make WordPress'\\
Added 	link `Photo Directory'\\
Added 	link `Five for the Future'\\
Added 	link `Events'\\
Added 	link `Job Board'\\
Unchanged 	button `About submenu' expanded: False\\
Unchanged 		StaticText `About'\\
Unchanged 	button `Open Search' hasPopup: dialog\\
Unchanged 	link `Get WordPress'\\ \\

Now please think step-by-step, examine the content variation in detail, and finally summarize the high-level usage of the interacted element. Do follow the output format ``Reasoning: ... Summary: This element ...''.\\ \midrule
\color{blue}{(LLM response)}\\
Reasoning: After interacting with the ``Community submenu'' button, the webpage content changes significantly. The button's ``expanded'' attribute is updated to ``True'', indicating that the button is now expanded. Additionally, the button gains focus, as indicated by the ``focused: True'' attribute. The most notable change, however, is the appearance of new. These new elements suggest that the button has triggered the display of a submenu or dropdown list related to the ``Community'' theme. \\

Summary: This element reveals a submenu of community-related links and resources.
\end{tabular} \\ \hline
\end{tabular}
\end{table}