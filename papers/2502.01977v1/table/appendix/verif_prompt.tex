\begin{table}[tp] 
\tiny
\centering
\caption{The self-verification prompt used in the AutoGUI pipeline in UI manipulation cases. This example shows how the LLM assigns a low score to the incorrect functionality.}
\label{tab:supp:verif prompt}
\begin{tabular}{|l|}
\hline
\begin{tabular}[c]{@{}p{0.9\linewidth}@{}}
\color{blue}{(Requirements for self-verification)}\\
Given the following inputs:\\
1) Webpage content: This input represents the hierarchical structure of a webpage's elements, emphasizing semantic information and relationships. Each node in the tree includes details such as the element's role (e.g., button, link, heading), relevant attributes (e.g., expanded), and hierarchical relationships with other elements.\\

2) Task Description: This describes the action a user intends to perform (such as submitting a form, navigating to a particular section, or adjusting settings) or the information they seek (such as a specific content piece or form field). It also introduces a candidate element for evaluation and then presents the webpage changes caused by interacting with this element. Your task is to assess whether this element effectively facilitates the specified user action.\\

Your job is to:\\
1) Analyze the provided webpage content to understand the structure and semantics of the webpage's elements.\\
2) Evaluate the Candidate Element: Determine the suitability of the specified candidate element for the described action. Consider the element's role, attributes, and position within the hierarchy. Your evaluation should be grounded in how well these aspects align with the required functionality for the user's intended action.\\
3) Score the Element: Assign a score ranging from 0 to 3, enclosed within <score></score> tags. This score should reflect the degree to which the candidate element meets the action's requirements:\\
0: The element does not support the action in any capacity.\\
1: The element provides minimal support for the action.\\
2: The element supports the action but with limitations.\\
3: The element fully supports the action without significant limitations.\\
4) Provide Reasoning: Before presenting your score, offer a detailed explanation of your reasoning. This should cover your analysis of the webpage content, the relationship between the candidate element and the specified action, and how these factors informed your scoring decision.\\
5) Format for Your Answer:\\
Reasoning: (Provide a comprehensive analysis covering the webpage's insights, the relationship between the specified action and the candidate element, and the rationale behind your scoring decision.)\\

Score: <score>$[Your Score Here]$</score>

\color{blue}{(An in-context exemplar)}\\
Example:\\
Webpage content:\\
$[0]$ RootWebArea `Rental Cars at Low, Affordable Rates'\\
$[1]$ dialog `Vehicle Class' modal: True\\
$[2]$ radiogroup `Vans' \\
$[3]$ 	radio `Minivans' checked: false\\
$[4]$ 	radio `Passenger Vans' checked: false\\
$[5]$ 	radio `Cargo Vans' checked: false\\
$[6]$ button `Cancel'\\
$[7]$ button `Apply Filter'\\

Task Description: Please identify the target element. The element helps users narrow down their vehicle choices to minivans specifically\\

Candidate element: $[3]$ 	radio `Minivans'\\
After interacting with the candidate element, the webpage exhibits these changes:\\
Upon clicking the ``Minivans'' <input> element, a new ``Remove Filter'' button is added to the dialog modal. The radio buttons for different vehicle classes remain unchanged, but the ``Minivans'' radio button is checked after the interaction.\\

Reference response:\\
Reasoning: The provided webpage content outlines a clear hierarchical structure for selecting vehicle types on a car rental webpage, categorized into sections like Cars, Vans, and Trucks, each with its own set of options represented as radio buttons. \\
The task involves narrowing vehicle choices to minivans. The candidate element is part of the `Vans' radiogroup on a car rental webpage. This directly supports the user's action of narrowing choices to minivans. The element's interaction leads to its checking and introduces a ``Remove Filter'' button. Its role, position, and functionality support the user's intent.\\

Score: <score>3</score>

\color{blue}{(The self-verification task)}\\
Now it's your turn.\\
Current webpage content:
\color{blue}{(A UI content example)} \\
$[0]$ RootWebArea `X for Business | X tips, tools, and best practices' focused: True \\
$[2]$ 			link `Get started in your ads account'\\
$[4]$ 			link `Campaign setup'\\
$[6]$ 			link `Editing and optimization'\\
$[8]$ 			link `Measurement'\\
$[9]$ 	heading ``Be what's happening''\\
$[10]$ 	StaticText `Spread your message, grow your audience, and get more traffic with Twitter Ads.'\\
=> $[11]$ 	link `Learn more' \color{blue}{(The interaction target)}\\
$[13]$ 	heading `Twitter is where people come to discover what’s happening.'\\
$[14]$ 	StaticText `With over 500 million Tweets being sent each day, ...'\\
$[15]$ 	StaticText `Whether you’re a CEO strengthening their personal brand...'\\
$[16]$ 	link `Get started with Twitter for Business'\\
$[17]$ 		button `Get started with Twitter for Business'\\
$[18]$ 	heading ``What's new on Twitter Business Outlook''\\
$[20]$ 	link `Google'\\

Task Description: Please identify the target element. This element \color{blue}{redirects users to a non-existent page, providing a custom error message and alternative content instead of the expected information. (The functionality annotation of the element)}\\

Candidate element: \color{blue}{$[11]$ 	link `Learn more'  (The interaction target)}\\
\color{blue}{(The interaction outcome)}\\
After interacting with the candidate element, we navigate to a new webpage that contains these contents:\\
Region 1 (Header): Contains the website's logo and navigation links.\\
Region 2 (Error Message): Displays an error message indicating that the page does not exist, accompanied by a humorous image and a link to potentially related content.\\
Region 3 (Footer): Provides links to various policy and information pages, as well as copyright information.\\
Overall Functionality: This webpage serves as a custom 404 error page, providing a lighthearted and engaging way to notify users that the requested page is not available.\\ \\

Now please do your job. Your answer: \\ \midrule
\color{blue}{(LLM response)}\\
Reasoning: The candidate element, a link labeled `Learn more', is part of the webpage's main content, surrounded by descriptions and headings related to Twitter for Business. However, upon interacting with this element, the webpage exhibits significant changes, redirecting users to a custom 404 error page with a humorous image and a link to potentially related content. In fact, the element's interaction leads to a completely different webpage with an error message and alternative content, which is unrelated to the initial webpage's purpose. The element's role, position, and functionality do not align with the user's intended action.\\

Score: <score>0</score>
\end{tabular} \\ \hline
\end{tabular}
\end{table}