\documentclass[]{arxiv}
\usepackage[english]{babel}
\usepackage{amsmath,amssymb,amsfonts}

\usepackage{xspace}
\usepackage{natbib}
\bibliographystyle{apalike}
\newcommand{\newblock}{}

\TheoremsNumberedThrough
% Set page size and margins
% Replace `letterpaper' with `a4paper' for UK/EU standard size
\usepackage[letterpaper,top=2cm,bottom=2cm,left=3cm,right=3cm,marginparwidth=1.75cm]{geometry}
\usepackage[shortlabels]{enumitem}
% Useful packages
\usepackage{dsfont}
\usepackage{mathtools}

\usepackage{caption}
\usepackage{subcaption}
\usepackage{graphicx}
\usepackage[colorlinks=true, allcolors=blue]{hyperref}
\usepackage[size=tiny]{todonotes}

\newcommand{\todof}[2][]{\todo[size=\scriptsize,color=magenta!20!white,#1]{FS: #2}}
\newcommand{\todog}[2][]{\todo[size=\scriptsize,color=cyan!20!white,#1]{GM: #2}}
\newcommand{\todoc}[2][]{\todo[size=\scriptsize,color=red!20!white,#1]{Cs: #2}}
\newcommand{\todoi}[2][]{\todo[size=\scriptsize,color=orange!20!white,#1]{IL: #2}}

\newcommand{\sgn}{\textrm{sgn}}

\usepackage{titlesec}
\setcounter{secnumdepth}{4}

\titleformat{\paragraph}
{\normalfont\normalsize\bfseries}{\theparagraph}{1em}{}
\titlespacing*{\paragraph}
{0pt}{3.25ex plus 1ex minus .2ex}{1.5ex plus .2ex}

\usepackage[capitalize]{cleveref}
\usepackage[ruled,linesnumbered]{algorithm2e}
\newcommand{\algoname}{\textsc{OtO}\xspace}

\crefname{algocf}{Algorithm}{algs.}
\Crefname{algocf}{Algorithm}{Algorithms}

\newcommand{\algucb}{\textsc{UCB}\xspace}
\newcommand{\alglcb}{\textsc{LCB}\xspace}

\usepackage{tikz}
\usepackage{standalone}
\ARTICLEAUTHORS{%
\AUTHOR{Flore Sentenac}
\AFF{HEC Paris,
\EMAIL{sentenac@hec.fr}}
\AUTHOR{Ilbin Lee}
\AFF{University of Alberta,
\EMAIL{}}
\AUTHOR{Csaba Szepesvari}
\AFF{University of Alberta,
Deepmind London}
}

\usepackage{soul}
\usepackage{booktabs}

\usepackage{multirow}

\TITLE{Balancing optimism and pessimism in offline-to-online learning}
%\author{You}

\newcommand{\cP}{\mathcal{P}}
\newcommand{\R}{\mathbb{R}}


% Language setting
% Replace `english' with e.g. `spanish' to change the document language

\begin{document}


\ABSTRACT{We consider what we call the offline-to-online learning setting, focusing on stochastic finite-armed bandit problems.
In offline-to-online learning, a learner starts with offline data collected from interactions with an unknown environment in a way that is not under the learner’s control. Given this data, the learner begins interacting with the environment, gradually improving its initial strategy as it collects more data to maximize its total reward.
The learner in this setting faces a fundamental dilemma: if the policy is deployed for only a short period, a suitable strategy (in a number of senses) is the Lower Confidence Bound (LCB) algorithm, which is based on pessimism. LCB can effectively compete with any policy that is sufficiently "covered" by the offline data. However, for longer time horizons, a preferred strategy is the Upper Confidence Bound (UCB) algorithm, which is based on optimism. Over time, UCB converges to the performance of the optimal policy at a rate that is nearly the best possible among all online algorithms.
In offline-to-online learning, however, UCB initially explores excessively, leading to worse short-term performance compared to LCB. This suggests that a learner not in control of how long its policy will be in use should start with LCB for short horizons and gradually transition to a UCB-like strategy as more rounds are played. This article explores how and why this transition should occur.
Our main result shows that our new algorithm performs nearly as well as the better of LCB and UCB at any point in time. The core idea behind our algorithm is broadly applicable, and we anticipate that our results will extend beyond the multi-armed bandit setting.
}

%\KEYWORDS{Offline-to-online;learning;optimism;pessimism;bandits}

\maketitle

\section{Introduction}


\begin{figure}[t]
\centering
\includegraphics[width=0.6\columnwidth]{figures/evaluation_desiderata_V5.pdf}
\vspace{-0.5cm}
\caption{\systemName is a platform for conducting realistic evaluations of code LLMs, collecting human preferences of coding models with real users, real tasks, and in realistic environments, aimed at addressing the limitations of existing evaluations.
}
\label{fig:motivation}
\end{figure}

\begin{figure*}[t]
\centering
\includegraphics[width=\textwidth]{figures/system_design_v2.png}
\caption{We introduce \systemName, a VSCode extension to collect human preferences of code directly in a developer's IDE. \systemName enables developers to use code completions from various models. The system comprises a) the interface in the user's IDE which presents paired completions to users (left), b) a sampling strategy that picks model pairs to reduce latency (right, top), and c) a prompting scheme that allows diverse LLMs to perform code completions with high fidelity.
Users can select between the top completion (green box) using \texttt{tab} or the bottom completion (blue box) using \texttt{shift+tab}.}
\label{fig:overview}
\end{figure*}

As model capabilities improve, large language models (LLMs) are increasingly integrated into user environments and workflows.
For example, software developers code with AI in integrated developer environments (IDEs)~\citep{peng2023impact}, doctors rely on notes generated through ambient listening~\citep{oberst2024science}, and lawyers consider case evidence identified by electronic discovery systems~\citep{yang2024beyond}.
Increasing deployment of models in productivity tools demands evaluation that more closely reflects real-world circumstances~\citep{hutchinson2022evaluation, saxon2024benchmarks, kapoor2024ai}.
While newer benchmarks and live platforms incorporate human feedback to capture real-world usage, they almost exclusively focus on evaluating LLMs in chat conversations~\citep{zheng2023judging,dubois2023alpacafarm,chiang2024chatbot, kirk2024the}.
Model evaluation must move beyond chat-based interactions and into specialized user environments.



 

In this work, we focus on evaluating LLM-based coding assistants. 
Despite the popularity of these tools---millions of developers use Github Copilot~\citep{Copilot}---existing
evaluations of the coding capabilities of new models exhibit multiple limitations (Figure~\ref{fig:motivation}, bottom).
Traditional ML benchmarks evaluate LLM capabilities by measuring how well a model can complete static, interview-style coding tasks~\citep{chen2021evaluating,austin2021program,jain2024livecodebench, white2024livebench} and lack \emph{real users}. 
User studies recruit real users to evaluate the effectiveness of LLMs as coding assistants, but are often limited to simple programming tasks as opposed to \emph{real tasks}~\citep{vaithilingam2022expectation,ross2023programmer, mozannar2024realhumaneval}.
Recent efforts to collect human feedback such as Chatbot Arena~\citep{chiang2024chatbot} are still removed from a \emph{realistic environment}, resulting in users and data that deviate from typical software development processes.
We introduce \systemName to address these limitations (Figure~\ref{fig:motivation}, top), and we describe our three main contributions below.


\textbf{We deploy \systemName in-the-wild to collect human preferences on code.} 
\systemName is a Visual Studio Code extension, collecting preferences directly in a developer's IDE within their actual workflow (Figure~\ref{fig:overview}).
\systemName provides developers with code completions, akin to the type of support provided by Github Copilot~\citep{Copilot}. 
Over the past 3 months, \systemName has served over~\completions suggestions from 10 state-of-the-art LLMs, 
gathering \sampleCount~votes from \userCount~users.
To collect user preferences,
\systemName presents a novel interface that shows users paired code completions from two different LLMs, which are determined based on a sampling strategy that aims to 
mitigate latency while preserving coverage across model comparisons.
Additionally, we devise a prompting scheme that allows a diverse set of models to perform code completions with high fidelity.
See Section~\ref{sec:system} and Section~\ref{sec:deployment} for details about system design and deployment respectively.



\textbf{We construct a leaderboard of user preferences and find notable differences from existing static benchmarks and human preference leaderboards.}
In general, we observe that smaller models seem to overperform in static benchmarks compared to our leaderboard, while performance among larger models is mixed (Section~\ref{sec:leaderboard_calculation}).
We attribute these differences to the fact that \systemName is exposed to users and tasks that differ drastically from code evaluations in the past. 
Our data spans 103 programming languages and 24 natural languages as well as a variety of real-world applications and code structures, while static benchmarks tend to focus on a specific programming and natural language and task (e.g. coding competition problems).
Additionally, while all of \systemName interactions contain code contexts and the majority involve infilling tasks, a much smaller fraction of Chatbot Arena's coding tasks contain code context, with infilling tasks appearing even more rarely. 
We analyze our data in depth in Section~\ref{subsec:comparison}.



\textbf{We derive new insights into user preferences of code by analyzing \systemName's diverse and distinct data distribution.}
We compare user preferences across different stratifications of input data (e.g., common versus rare languages) and observe which affect observed preferences most (Section~\ref{sec:analysis}).
For example, while user preferences stay relatively consistent across various programming languages, they differ drastically between different task categories (e.g. frontend/backend versus algorithm design).
We also observe variations in user preference due to different features related to code structure 
(e.g., context length and completion patterns).
We open-source \systemName and release a curated subset of code contexts.
Altogether, our results highlight the necessity of model evaluation in realistic and domain-specific settings.





%!TEX root =  main.tex
\section{Related Works}

\textbf{Offline Learning}: We begin by providing an overview of the research conducted in offline (or batch) learning, with a particular focus on works addressing the multi-armed bandit (MAB) setting in detail. A central challenge in offline learning is data coverage: does the offline dataset contain sufficient information about an optimal or near-optimal policy? There are at least two extreme offline data types \citep{rashidinejad2023bridgingofflinereinforcementlearning}: expert data sets, produced by a near-optimal policy, and uniform coverage data sets, which provide equal representation of all actions. In expert data sets, imitation learning algorithms are shown to have a small sub-optimality gap against the logging policy \citep{Imitationlearningross,rajaraman2020fundamentallimitsimitationlearning}. Meanwhile, theoretical guarantees for many offline RL algorithms depend on the coverage of offline data—often quantified through a concentrability coefficient that measures how well the dataset covers the policies being evaluated \citep{rashidinejad2023bridgingofflinereinforcementlearning}; \cite{rashidinejad2023bridgingofflinereinforcementlearning}) or the set of policies with which the algorithms compete is limited to those covered in offline data \citep{cheng2022adversarially,  yin2020nearoptimalprovableuniformconvergence}. In this paper, we also consider different offline data compositions (uniform vs. skewed), but the main focus is the spectrum from offline to online learning.



A widely accepted intuition in offline learning is that a good policy should avoid under-explored regions of the environment. This motivates the use of the pessimism principle, which manifests in various forms: learning a pessimistic value function \citep{swaminathan2015counterfactualriskminimizationlearning,wu2019behaviorregularizedofflinereinforcement,li2022pessimismofflinelinearcontextual}, pessimistic surrogate \citep{buckman2020importancepessimismfixeddatasetpolicy}, or planning with a pessimistic model \citep{MorelPessimistisOfflineLearning,yu2020mopomodelbasedofflinepolicy}. 
Despite the surge of work in that direction, a theoretical base for the principle has only recently been developed. 
\cite{xiao2021optimalitybatchpolicyoptimization} analyzed pessimism in MABs, proving that both UCB and \alglcb are minimax optimal (up to logarithmic factors) but that \alglcb outperforms UCB under a weighted minimax criterion reflecting the difficulty of learning the optimal policy.  Notably, this result implies that \alglcb outperforms \algucb in cases where the offline data is generated by a near-optimal policy. \cite{rashidinejad2023bridgingofflinereinforcementlearning} further studied \alglcb, introducing a concentrability coefficient to measure how close the offline dataset is to being an expert dataset. They showed that \alglcb is adaptively optimal in the entire concentrability coefficient range for contextual bandits and MDP. They also showed that for MAB, \alglcb is also adaptively optimal for a wide set of concentrability coefficients, excluding only datasets where the optimal arm is drawn with probability larger than $1/2$ and where "play the most played arm" achieves an exponential convergence rate to the optimal policy.
It is important to note that these findings do not contradict the results of \cite{xiao2021optimalitybatchpolicyoptimization}.  \cite{rashidinejad2023bridgingofflinereinforcementlearning} fixed the concentrability coefficient in the minimax analysis, whereas \cite{xiao2021optimalitybatchpolicyoptimization} did not. Finally, \cite{rashidinejad2023bridgingofflinereinforcementlearning} also showed that the \alglcb algorithm can compete with any target policy that is covered in the offline data, regardless of the performance of that policy.

Two key insights emerge from these theoretical studies:
minimax regret does not fully capture the performance gap between UCB and \alglcb in offline learning, and \alglcb outperforms UCB when the logging policy is close to optimal. 
Because it is crucial to find a policy that is at least comparable to a previously deployed policy, there has been a series of works in offline RL literature that studied the regret against the logging policy or any baseline policy. Some of those methods are based on the pessimism principle. \cite{xie2022armormodelbasedframeworkimproving} proposed optimizing the worst-case regret against a baseline policy, which they called relative pessimism. They showed that the method finds a policy that performs as well as the baseline policy and learns the best policy among those supported in offline data when the baseline policy is also supported in the data. \cite{cheng2022adversarially} proposed a different optimization formulation also based on relative pessimism and showed that the algorithm improves over the logging policy. They also showed that the method can compete with policies covered in offline data. Other methods to improve over the logging policy that are not (directly) based on pessimism have been proposed as well, for example, those based on the idea that the learned policy should not be too far from the logging policy (e.g., \citealt{fujimoto2021minimalist}). However, these methods focus exclusively on the offline setting and thus only explore one extreme of the offline-to-online spectrum. Our work shares their objective of competing against the logging policy but extends beyond purely offline learning.  

\textbf{Offline-to-Online} 
While the literature on offline-to-online learning is still new, it is rapidly growing. A first line of work can be considered answering the following question: how can we adapt an online learning algorithm to achieve good performance when the offline data set is rich enough to learn a near-optimal policy? In some cases, classical online learning methods that incorporate offline data naturally achieve strong performance. For MABs, \cite{MABwithHistory} showed that a logarithmic amount of offline data can reduce the regret from logarithmic to constant, and their algorithm implements optimism but incorporates offline data. Results of \cite{gur2020adaptive} imply a lower bound for regret in offline-to-online MABs (see Appendix G of \citealt{bu2021onlinepricingofflinedata} for a summary), offering more insights into how additional data influences achievable regret. Their setting is broader than ours, allowing for new information to arrive sequentially, with offline data as a special case. They, too, employed classical online learning methods, including Thompson sampling and UCB-like algorithms augmented with offline data.
This question was also studied in dynamic pricing by \cite{bu2021onlinepricingofflinedata},  a class of contextual bandits. The authors of this last paper proposed an optimism-driven algorithm and showed that the achievable regret as a function of $T$ decreases notably as the size of the offline data set increases, which the authors coined as a ``phase transition''. It is also notable that for MABs, the optimal regret phase transition is achieved only when offline data is balanced over the arms \citep{bu2021onlinepricingofflinedata}. \cite{chen2022data} also investigated offline-to-online learning, focusing on healthcare applications where offline and online data distributions may differ. Their approach, like the others, is based on optimism or Thompson sampling.

Our work builds on this literature in a fundamental way. The above papers enhance classical online learning methods, such as UCB-like or Thompson-sampling, by incorporating additional offline data and studying how this affects regret against optimality. Here, we bring offline learning methods and metrics into the offline-to-online learning setting and thus put both offline and online learning into perspective. 
%We also refined some results that were obtained by \cite{gur2020adaptive}. In particular, we refine some of their bounds that were tight in regimes of their more general setting but are loose in regimes of key interest in this paper, most importantly for small values of the horizon $T$.

Another stream of offline-to-online works focuses on minimizing computational and sample costs rather than minimizing regret over a horizon. This setting with that objective is more commonly called hybrid reinforcement learning. Their goal is to find a good policy using both offline and online data while minimizing the number of samples or computational resources required \citep{Song2022-vg,xie2022policyfinetuningbridgingsampleefficient,ball2023efficient,
wagenmaker2023leveraging,Li2023-lv, li2024reward, zhou2023offlinedataenhancedonpolicy}. These works differ from ours in both goal and methodology: they do not address the central question of how best to navigate the transition from offline to online learning as TT varies. Furthermore, our focus on MABs sidesteps computational efficiency concerns.

Despite these prior contributions, the theoretical understanding of offline-to-online learning remains incomplete, particularly regarding how to balance conservatism and exploration as learning shifts from purely offline to purely online. Our paper directly addresses this gap, with a specific focus on MABs.
We study (stochastic) gradient descent on the empirical risk
\begin{equation*}
\cL(w) = \frac{1}{n}\sum_{i=1}^n l(p_i(w))\, ,
\end{equation*}
where the loss function $l$ and the functions  $(p_i)_{i=1}^n$  are specified in the following assumptions. Note that the empirical risk for binary classification from Equation~\eqref{def:emp_risk_intro} is a special case of the above objective.

\begin{assumption}\label{hyp:loss_exp_log}\phantom{=}
  \begin{enumerate}[label=\roman*)]
    \item The loss is either the exponential loss, $l(q) = e^{-q}$, or the logistic loss, $l(q) = \log(1{+}e^{-q})$.
    \item There exists an integer $L \in \mathbb{N}^*$  such that, for all $1 \leq i \leq n$, the function $p_i$ is $L$-homogeneous\footnote{We recall that a mapping $f : \mathbb{R}^d \rightarrow \mathbb{R}$ is positively $L$-homogeneous if $f(\lambda w) = \lambda^L f(w)$ for all $w \in \mathbb{R}^d$ and $\lambda >0$.}, locally Lipschitz continuous and semialgebraic.
  \end{enumerate}
\end{assumption}
If the $p_i$'s were differentiable with respect to $w$, the chain rule would guarantee that
\begin{align*}
\nabla \mathcal{L}(w) = \frac{1}{n}\sum_{i=1}^n  l'(p_i(w)) \nabla p_i(w)\enspace.
\end{align*}
However, we only assume that the $p_i$'s are semialgebraic. While we could consider Clarke subgradients, the Clarke subgradient of operations on functions (e.g., addition, composition, and minimum) is only contained within the composition of the respective Clarke subgradients. This, as noted in Section~\ref{sec:cons_field}, implies that the output of backpropagation is usually not an element of a Clarke subgradient but a selection of some conservative set-valued field.
Consequently, for $1\leq i \leq n$, we consider $D_i : \bbR^d \rightrightarrows\bbR^d$, a conservative set-valued field of $p_i$, and a function $\sa_i : \bbR^d \rightarrow \bbR^d$ such that for all $w \in \bbR^d$, $\sa_i(w) \in D_i(w)$. Given a step-size $\gamma >0$, gradient descent (GD)\footnote{More precisely, this refers to conservative gradient descent. We use the term GD for simplicity, as conservative gradients behave similarly to standard gradients.} is then expressed as
\begin{equation*}\label{eq:gd_new}\tag{GD}
  w_{k+1} = w_k - \frac{\gamma}{n} \sum_{i=1}^n l'(p_i(w_k))\sa_i(w_k)\,.
\end{equation*}
For its stochastic counterpart, stochastic gradient descent (SGD), we fix a batch-size $1\leq n_b \leq n$. At each iteration $k \in \bbN$, we randomly and uniformly draw a batch $B_k \subset \{1, \ldots, n \}$ of size $n_b$. The update rule is then given by 
\begin{equation*}\label{eq:sgd_new}\tag{SGD}
  w_{k+1} = w_k -  \frac{\gamma}{n_b}\sum_{i\in B_k} l'(p_i(w_k)) \sa_i(w_k)\, .
\end{equation*}
The considered conservative set-valued fields will satisfy an Euler lemma-type assumption.
%\nic{Smoother transition}
\begin{assumption}\phantom{=}\label{hyp:conserv}
  For every $i \leq n$, $\sa_i$ is measurable and $D_i$ is semialgebraic. Moreover, for every $w \in \bbR^d$ and $\lambda \geq 0$, $\sa_i(w)  \in D_i(w)$,
  \begin{equation*}
    D_i(\lambda w) = \lambda^{L-1} D_i(w)\, , \textrm{ and } \quad   L p_i(w) = \scalarp{\sa_i(w)}{w}\, .
  \end{equation*}
\end{assumption}
%\nic{Smoother transition}
Having in mind the binary classification setting, in which $p_i(w) = y_i \Phi(x_i, w)$, we define the margin
\begin{equation}\label{def:marg}
  \sm: \bbR^d \rightarrow \bbR, \quad \sm(w) = \min_{1\leq i \leq n} p_i(w)\, .
\end{equation}
It quantifies the quality of a prediction rule $\Phi(\cdot, w)$. In particular,  the training data is perfectly separated when $\sm(w) >0$. A binary prediction for $x$ is given by the sign of $\Phi(x, w)$, and under the homogeneity assumption, it depends only on the normalized direction $w / \norm{w}$. Consequently, we will focus on the sequence of directions $u_k := w_k / \norm{w_k}$. Our final assumption ensures that the normalized directions $(u_k)$ have stabilized in a region where the training data is correctly classified.

\begin{assumption}\label{hyp:marg_lowb}
  Almost surely, $\liminf \sm(u_k) >0$.
\end{assumption}
Before presenting our main result, we comment on our assumptions.

\paragraph{On Assumption~\ref{hyp:loss_exp_log}.} As discussed in the introduction, the primary example we consider is when $p_i(w) = y_i \Phi(x_i;w)$ is the signed prediction of a feedforward neural network without biases and with piecewise linear activation functions on a labeled dataset $((x_i,y_i))_{i \leq n}$. In this case,
\begin{equation}\label{eq:NN}
 p_i(w) = y_i \Phi(w;x_i) = y_i V_L(W_L) \sigma(V_{L-1}(W_{L-1}) \sigma(V_{L-1}(W_{L-2}) \ldots \sigma(V_{1}(W_1 x_i))))\, ,
\end{equation}
where $w = [W_1, \ldots, W_L]$, $W_i$ represents the weights of the $i$-th layer, $V_i$ is a linear function in the space of matrices (with $V_i$ being the identity for fully-connected layers) and $\sigma$ is a coordinate-wise activation function such as $z \mapsto \max(0,z)$ ($\ReLU$), $z \mapsto \max(az, z)$ for a small parameter $a>0$ (LeakyReLu) or $z \mapsto z$. Note that the mapping $w \mapsto p_i(w)$ is semialgebraic and $L$-homogeneous for any of these activation functions. Regarding the loss functions, the logistic and exponential losses are among the most commonly studied and widely used. In Appendix~\ref{app:gen_sett}, we extend our results to a broader class of losses, including $l(q) = e^{-q^a}$ and $l(q) = \ln (1 + e^{-q^a})$ for any $a \geq 1$.

\paragraph{On Assumption~\ref{hyp:conserv}.} Assumption~\ref{hyp:conserv} holds automatically  if $D_i$ is the Clarke subgradient of $p_i$. Indeed, at any vector $w \in \bbR^d$, where $p_i$ is differentiable it holds that $p_i(\lambda w) = \lambda^{L} p_i(w)$. Differentiating relatively to $w$ and $\lambda$ (noting that $p_i$ remains differentiable at $\lambda w$ due to homogeneity), we obtain $\lambda \nabla p_i(\lambda w) = \lambda^{L} \nabla p_i(w)$ and $\scalarp{\nabla p_i(\lambda w)}{w} = L \lambda^{L-1} p_i(w)$. The expression for any element of the Clarke subgradient then follows from~\eqref{eq:def_clarke}. 

However, for an arbitrary conservative set-valued field, Assumption~\ref{hyp:conserv} does not necessarily hold. For instance, $D(x) = \mathds{1}(x \in \mathbb{N})$ is a conservative set-valued field for $p \equiv 0$, which does not satisfy Assumption~\ref{hyp:conserv}. Nevertheless, in practice, conservative set-valued fields naturally arise from a formal application of the chain rule. For a non-smooth but homogeneous activation function $\sigma$, one selects an element $e \in \partial \sigma (0)$, and computes $\sa_i(w)$ via backpropagation. Whenever a gradient candidate of $\sigma$ at zero is required (i.e., in~\eqref{eq:NN}, for some $j$, $V_j(W_j)$ contains a zero entry), it is replaced by $e$. 
Since $V_j(W_j)$ and $V_j(\lambda W_j)$ have the same zero elements, it follows that for every such $w$, $
\sa_i(\lambda w) = \lambda^L \sa_i(w)$. The conservative set-valued field $D_i$ is then obtained by associating to each $w$ the set of all possible outcomes of the chain rule, with $e$ ranging over all elements of $\partial \sigma(0)$. Thus, for such fields, Assumption~\ref{hyp:conserv} holds.


\paragraph{On Assumption~\ref{hyp:marg_lowb}.} Training typically continues even after the training error reaches zero.
Assumption~\ref{hyp:marg_lowb} characterizes this late-training phase, where our result applies. 
As noted earlier, since $\sm$ is $L$-homogeneous, the classification rule is determined by the direction of the  iterates $u_k=w_k/\norm{w_k}$. Assumption~\ref{hyp:marg_lowb} then states that, beyond some iteration, the normalized margin remains positive. 
This assumption is natural in the context of studying the implicit bias of SGD: we \emph{assume} that we reached the phase in which the dataset is correctly classified and \emph{then} characterize the limit points. A similar perspective was taken in  \cite{nacson2019lexicographic}, where the implicit bias of GF was analyzed under the assumption that the sequence of directions and the loss converge. However, unlike their approach, ours does not require assuming such convergence a priori.

Earlier works such as \cite{ji2020directional,Lyu_Li_maxmargin}, which analyze subgradient flow or smooth GD, establish convergence by assuming the existence of a single iterate $w_{k_0}$ satisfying $\sm(w_{k_0}) > \varepsilon$ and then proving that $\lim \sm(u_{k}) > 0$. Their approach relies on constructing a smooth approximation of the margin, which increases during training, ensuring that $\sm(u_k) > 0$ for all iterates with $k \geq k_0$. This is feasible in their setting, as they study either subgradient flow or GD with smooth $p_i$’s, allowing them to leverage the descent lemma.

In contrast, our analysis considers a nonsmooth and stochastic setting, in which, even if an iterate $w_{k_0}$ satisfying $\sm(w_{k_0}) > \varepsilon$ exists, there is no a priori assurance that subsequent iterates remain in the region where Assumption~\ref{hyp:marg_lowb} holds. From this perspective, Assumption~\ref{hyp:marg_lowb} can be viewed as a stability assumption, ensuring that iterates continue to classify the dataset correctly. Establishing stability for stochastic and nonsmooth algorithms is notoriously hard, and only partial results in restrictive settings exist \cite{borkar2000ode,ramaswamy2017generalization,josz2024global}.

%Finally, note that Assumption~\ref{hyp:marg_lowb} only needs to hold almost surely. Specifically, with probability 1, there exist $k_0$ and $\varepsilon$ such that for all $k \geq k_0$, $\sm(u_k) \geq \varepsilon > 0$. In the case of~\eqref{eq:sgd_new}, $k_0$ and $\delta$ are random variables and may take different values across different realizations. 

%\paragraph{On constant stepsizes.}
%We allow the step size to be a constant of arbitrary magnitude, subject to the stability Assumption~\ref{hyp:marg_lowb}. This may seem surprising in a nonsmooth and stochastic setting, where a vanishing step size is typically required to ensure convergence (see, e.g., \cite{majewski2018analysis, dav-dru-kak-lee-19, bolte2023subgradient, le2024nonsmooth}).
\begin{algorithm}[ht!]
\caption{\textit{NovelSelect}}
\label{alg:novelselect}
\begin{algorithmic}[1]
\State \textbf{Input:} Data pool $\mathcal{X}^{all}$, data budget $n$
\State Initialize an empty dataset, $\mathcal{X} \gets \emptyset$
\While{$|\mathcal{X}| < n$}
    \State $x^{new} \gets \arg\max_{x \in \mathcal{X}^{all}} v(x)$
    \State $\mathcal{X} \gets \mathcal{X} \cup \{x^{new}\}$
    \State $\mathcal{X}^{all} \gets \mathcal{X}^{all} \setminus \{x^{new}\}$
\EndWhile
\State \textbf{return} $\mathcal{X}$
\end{algorithmic}
\end{algorithm}

%!TEX root =  main.tex
\section{Regret Analysis of UCB and LCB}\label{sec:intersetanalysis}

In this section, we systematically investigate the performance of UCB and LCB as promised. The goal here is to paint a picture as comprehensive as possible over the whole spectrum of offline-to-online learning and different compositions of offline data, which may be of interest on its own. The first step will be to focus on the minimax regret against an optimal arm. Then, we will study the two approaches'  regret against the logging policy. 
 

Before presenting the precise statements of the results, we provide a summary with tables and figures for illustration. The first line of results presented concerns the pseudo regret against optimality and is summarized in \cref{fig:pseudoregret}. 
The first and second rows of the table correspond to UCB and LCB, respectively. The columns are labeled with a specific horizon. Here, the choices $T=1$ and $T\gg m$ correspond to short and long horizons, respectively. That horizon $T=m$ corresponds to the case when the amount of online collected data matches the size of the offline data; this is the time when we expect the online data to make some difference for the first time. 


First, it is shown that \algucb\ achieves minimax optimality (up to poly-logarithmic factors) for all horizons $T\geq 0$, with a more refined dependence on the composition of the offline data compared to prior results in the literature. As for \alglcb, it is also minimax optimal for $T=1$, but this is no longer the case as $T$ increases. It would be tempting to stop here and simply accept that \algucb, being minimax optimal over the whole range of $T$, is a good algorithm for offline-to-online setting. However, the second line of results paints a different picture.

\begin{table}[bth] 
\begin{center}
\renewcommand{\arraystretch}{1.4}
\begin{tabular}{>{\centering\arraybackslash}m{3cm} | |>{\centering\arraybackslash}m{3cm} |>{\centering\arraybackslash}m{3cm} |>{\centering\arraybackslash}m{3cm} }
\toprule
$T$ & $T = 1$ & $T = m$ & $T \gg m$ \\
 \hline\hline
 \addlinespace

$\mathcal{R}_{\text{UCB}}(T)$ & $\sqrt{\frac{1}{\min_i m_i}}$ & $m\sqrt{\frac{K}{\sum_i m_i}}$ & $\sqrt{KT}$ \\
\addlinespace
\hline
\addlinespace

$\mathcal{R}_{\text{LCB}}(T)$ & $\sqrt{\frac{1}{\min_i m_i}}$ & $m\sqrt{\frac{1}{\min_i m_i}}$ & $T\sqrt{\frac{1}{\min_i m_i}}$ \\
\bottomrule
\end{tabular}
\vspace{0.2cm}
\caption{Evolution of the pseudo regret of \alglcb\ and \algucb\ as $T$ grows (matching upper and lower bounds-ignoring poly log terms-are displayed, exact expressions in the lemmas).}
\label{fig:pseudoregret}
\end{center}
\end{table}





\cite{Xiao2021OnTO} showed that the minimax criterion was not enough to distinguish \algucb{} and \alglcb, and suggested an additional criterion. In offline learning, despite the similar minimax guarantees, \alglcb{} is often favored over \algucb. There is an intuitive explanation for this: even when there is only a single iteration, \algucb\ will explore if a single arm does not have any sample, and the gathered knowledge has no chance of being exploited. \alglcb{} does the opposite in offline learning and sticks to what the offline data indicates as a ``good enough'' arm. Intuition suggests that if offline data are focused on good arms, then exploration may affect the performance negatively, whereas \alglcb{} exploits the information immediately. Also, in the literature, pessimistic algorithms have been shown to be competitive against the logging policy in offline learning. These suggest that competing against the data generating policy, i.e., the logging policy, can be another reasonable criterion for offline-to-online learning. This is what we set out to do in this second line of results, summarized in \cref{fig:regretlogging}.
 First, we observe that at $T=1$, which corresponds to offline learning, LCB has a lower regret whenever offline data is not perfectly balanced. The gap is greater when offline data are completely concentrated on one arm, i.e., $m_i=m$ for some $i \in [K]$. As $T$ increases, the gap between the two guarantees for the two algorithms decreases slowly. The point at which it closes depends on the repartition of offline data.


\begin{table}[htb]
\begin{center}
\renewcommand{\arraystretch}{1.4}
\begin{tabular}{>{\centering\arraybackslash}m{1.5cm} | |>{\centering\arraybackslash}m{1.5cm}|>{\centering\arraybackslash}m{1.5cm} |>{\centering\arraybackslash}m{3.cm}|>{\centering\arraybackslash}m{2cm}  |>{\centering\arraybackslash}m{1.5cm}|>{\centering\arraybackslash}m{1.5cm} }
 \toprule
 $T$ & \multicolumn{2}{c|}{$1$}&\multicolumn{2}{c|}{$T=m$} &\multicolumn{2}{c}{$T\gg m$ } \\ 
  \cline{2-7}
    & LB & UB  &LB & UB&LB & UB\\
     \hline
     \hline
\addlinespace
 $\mathcal{R}^{\text{log}}_\algucb(T)$ & \multicolumn{2}{c|}{$\sqrt{\frac{1}{\min_i m_i}}$} & {\small $\sum_{i=1}^K\left(\frac{m}{K}-m_i\right)\rho_i$}&$\sqrt{KT}$& $0$&$\sqrt{KT}$ \\ 
 \addlinespace
  \hline
\addlinespace
 $\mathcal{R}^{\text{log}}_\alglcb(T)$ & $\frac{\sqrt{m_2}}{m}$&$\frac{\sum_{i}\sqrt{m_i}}{m} $ &  $\sqrt{m_2}$ & $\sum_{i=1}^m\sqrt{m_i}$&$T\frac{\sqrt{m_2}}{m}$&$T\frac{\sum_{i}\sqrt{m_i}}{m} $\\ 
 \addlinespace
 \hline
\end{tabular}
\end{center}
\vspace{0.2cm}
\caption{Evolution of the regrets against the logging policy as $T$ grows (ignoring poly log terms, exact expressions in the Lemmas), assuming wlog $m_1\geq m_2\geq \ldots\geq m_K$, and with $\rho_i=\left[\sqrt{\frac{1}{m_i+\frac{m}{K}}}-\sqrt{\frac{1}{m_1+\frac{m}{K}}}\right]$.}
\label{fig:regretlogging}
\end{table}

One lesson from these two lines of results is that if $T\gg m$, then \algucb\ dominates \alglcb\ in both objectives. It is also the case when offline data are uniformly spread over the arms, i.e., $m_i=\frac{m}{K}$ for all $i \in [K]$. However, when the horizon is short ($m\gg T$) and offline data do not cover the arms uniformly, then \alglcb\ dominates \algucb in both objectives. Thus, neither algorithm dominates the other for all horizons and all possible data coverage. The algorithm we introduced, \algoname, finds a trade-off between the two strategies.

\begin{figure}[htb]
\begin{center}
  \includestandalone[scale=0.8]{plots/pareto_same_sample}%     without .tex extension
  \caption{Evolution of the two regret measures when offline samples are uniformly spread between the arms, i.e., $m_i=\frac{m}{K}$ for all $ I \in [K]$. The best algorithms are the ones closest to $(0,0)$. From left to right, the horizon is $T=1$, $T=m$ and $T\gg m. $ In the last plot, we assume $T\gg m$. The \algoname\ in the plot uses $\alpha=1$.}
\label{fig:tradeoffucblcbbanlanced}
\end{center}
\end{figure}

\cref{fig:tradeoffucblcbbanlanced,fig:tradeoffucblcbconcentrated} illustrate these points visually. In the figures, we plot the upper bound on the regret (ignoring poly-log terms) of all three algorithms for each specific composition of the offline arm counts and varying values of $T$. For the chosen offline arm counts, the upper bounds on \algucb\ and \alglcb\ are tight up to poly-log terms. 
In \cref{fig:tradeoffucblcbbanlanced}, we see the evolution of the regret for perfectly balanced datasets. Here, \algucb\ dominates \alglcb\ for all horizons, and \algoname\ behaves like \algucb.
In \cref{fig:tradeoffucblcbconcentrated}, we see the evolution of the regret for highly skewed offline datasets: all the samples are concentrated on the first two arms. 
Here, the picture is somewhat more complicated: \algucb\ dominates in the regret against optimality, whereas \alglcb\ dominates in the regret with respect to the logging policy. We see that \algoname finds a trade-off between the two, and the trade-off point depends on the value of the parameter $\alpha$ chosen. With $\alpha=\sqrt{K}$, for example, \algoname\ improves upon the regret against the logging policy of \algucb without downgrading (up to a multiplicative constant that does not appear on the plot) the regret against optimality. With $\alpha=1$, \algoname\ improves on the regret against optimality of \alglcb, without downgrading (again up to a multiplicative constant) the regret against the logging policy.

\begin{figure}[htb]
\begin{center}
  \includestandalone[scale=0.8]{plots/pareto_one_sample}%     without .tex extension
  % or use \input{mytikz}
  \caption{Evolution of the two regrets when all offline samples are concentrated on two arms, i.e. $m_1=m_2=\frac{m}{2K}$. From left to right, the horizon is $T=1$ and $T=m$. For readability purposes, we do not plot $T\gg m$, as the relative behavior of the algorithms for that horizon would be quite similar to the one at $T=m$, but with an even larger ratio between the horizontal and vertical axis. }
\label{fig:tradeoffucblcbconcentrated}
\end{center}
\end{figure}


The rest of the section is dedicated to stating the results summarized in \cref{fig:pseudoregret,fig:regretlogging} precisely.


\subsection{Minimax regret}
We will start by proving a lower bound on the minimax regret of any algorithm for offline-to-online learning. We also derive a matching upper bound given by \algucb. 

\begin{theorem}[Lower bound on the minimax regret of any algorithm for offline-to-online learning]\label{prop:lowerboundminimax}
For any $T\geq 1$ and for any algorithm $\mathcal{A}$, we have
    \begin{equation*}
 \mathcal{R}_{\mathcal{A}}(T)\geq \frac{1}{31} T\sqrt{\max_{J\subseteq [K]}\frac{|J|}{T-1+\sum_{j\in J} m_j}}.
\end{equation*}
\end{theorem}
\textit{Sketch of Proof}: The proof is a refined application of classical information-theoretic lower bounds. The main technical challenge lies in defining an appropriate threshold, $\Delta$, which determines when arms cannot be reliably identified as suboptimal. This threshold is given by:  
\[
\Delta = \sqrt{\max_{J \subseteq [K]} \frac{|J|}{2(T - 1 + \sum_{j \in J} m_j)}}.
\]  Details can be found in \cref{sec:omittedproofs}.


The above bound may be hard to interpret. To get a sense of the behavior of the lower bound, we look at two special cases. Assume that $m_1\geq \ldots \geq m_K$. Then letting $J=\left\{m_2,\ldots, m_K\right\}$, we get
\begin{equation*}
 \mathcal{R}_{\mathcal{A}}(T)\geq \frac{1}{31} T\sqrt{\frac{(K-1)}{T-1+m-\max_i m_i}},
\end{equation*}
which recovers the $\Omega(\sqrt{TK})$ lower bound for $T$ large. Now, when $J=\left\{ m_K\right\}$, we get
\begin{equation*}
\mathcal{R}_{\mathcal{A}}(T)\geq \frac{1}{31} T\sqrt{\frac{1}{T-1+\min_i m_i}},
\end{equation*}
 which tells us that for $T$ small, the minimum count in the offline data sets a lower limit on the regret. 

We now study the regret of \algucb\ and \alglcb. We start with \algucb, showing that it is minimax optimal. We also provide an instance-dependent bound on the regret.

\begin{theorem}[\algucb's upper bound on the minimax regret]\label{prop:regretminimaxucb}
    For any $T\geq 1$ and any $\theta\in \Theta$, with probability at least $1-2T^2\delta$,
    \[
R_\algucb(T)\leq \sum_{i=1}^K \Delta_i\left( \frac{2}{\Delta_i^2} \log(K/\delta)-m_i\right)_++\sum_{i=1}^K \Delta_i.
    \]
    Also, we have the following instance-independent bound: 
    \[
\mathcal{R}_{\algucb}(T)\leq \min\left( \max_{J\subseteq [K]}2T\sqrt{\frac{2|J|\log(K/\delta)}{T+\sum_{j\in J}m_j}}+|J|;T\sqrt{\frac{2\log(K/\delta)}{\min_i m_i}}\right)+2T^2\delta.
    \]
 \end{theorem}
\textit{Sketch of Proof}: The first bound is obtained through the usual techniques of upper bounding the number of pulls from each arm. The second approach demands more intricate algebraic computations. It entails separately bounding the regret for two categories of arms: those whose gap exceeds a "detection threshold" and those whose gap falls below it. The proof can be found in \cref{sec:omittedproofs}. \hfill \(\Box\)

Theorems~\ref{prop:lowerboundminimax} and \ref{prop:regretminimaxucb} illustrate the difficulty of offline-to-online learning measured by the minimax regret and show how the difficulty depends on the composition of offline data. The derived bounds align with those obtained by \cite{pmlr-v235-cheung24a}, although the proof techniques differ. These differing techniques result in distinct implicit expressions for the bounds. However, it can be shown that the solution to the linear program (LP) in \cite{pmlr-v235-cheung24a} and the inverse of the solution to our maximization problem are within multiplicative constants of each other.



When it comes to \alglcb, the following result shows that the algorithm fully depends on the quality of the offline data. This is perhaps unsurprising, as the algorithm has no built-in exploration: its knowledge of the arms may not improve with the online interactions. 

\begin{proposition}[Minimax regret of \alglcb]\label{prop:minimaxregretlcb}
    For $T\geq 1$, we have
    \[
   \min\left(0.07 T, 0.15 T\sqrt{\frac{1}{\min_i m_i}}\right)\leq  \mathcal{R}_\alglcb(T) \leq T\sqrt{\frac{2\log(K/\delta)}{\min_i m_i}}+ 2T^2\delta.
    \]
\end{proposition}
\textit{Sketch of proof}: For the lower bound, we construct an instance where the optimal arm is the arm with the minimum count in offline samples, and all other arms have deterministic rewards. By the structure of \alglcb, if the optimal arm is not picked in the first iteration, it will never be picked. We show that this happens with a constant probability with a proper choice of distribution parameters. The upper bound is derived by bounding the gap between the mean of the chosen arm and $\mu_*$. We defer the technical steps of the proof to \cref{sec:omittedproofs}. \hfill \(\Box\)



\subsection{Regret against the logging policy}


In this section, we present the results shown in Table~\ref{fig:regretlogging}. All proofs are in Section~\ref{sec:omittedproofs}. 

\begin{proposition}[\algucb's regret against the logging policy for $T=1$]\label{prop:loginucbT=1}
We have
\[
0.07\min \left(1;\sqrt{\frac{1}{\min_i m_i}}\right)\leq \mathcal{R}^{\text{log}}_\textsc{UCB}(1) \leq \sqrt{\frac{2\log(\frac{K}{\delta})}{\min_i m_i}}+2\delta.
\]
\end{proposition}

\textit{Sketch of Proof}: For the lower bound, we construct an instance in which all arms have the same mean, except the arm with the minimal count, which has a slightly lower mean. With a proper choice of distribution parameters, \algucb\ picks the worst arm with a constant probability. The upper bound is a consequence of the upper bound on the minimax regret obtained in \cref{prop:regretminimaxucb}. \hfill \( \Box\)

\begin{proposition}[\algucb's regret against the logging policy for general $T$]\label{prop:lowerboundUCBanyT}
For any $T>0$, $\frac{T}{K}\in \mathbb{N}$, we have
    \[\mathcal{R}^{\text{log}}_{\textsc{UCB}}(T)\geq T\sum_{i=1}^K\left(\frac{1}{K}-\frac{m_i}{m}\right)\left[\sqrt{\frac{1}{2(m_i+\frac{T}{K})}}-\sqrt{\frac{1}{2(\max_{j\in [K]}m_j+\frac{T}{K})}}\right],\]
and
\[
\mathcal{R}^{\text{log}}_{\textsc{UCB}}(T)\leq \mathcal{R}_{\textsc{UCB}}(T).
\]
\end{proposition}

\textit{Sketch of Proof}: The upper bound is a consequence of the definitions, as $\mu_*\geq \mu_0$ always holds. For the lower bound, we construct an instance where the mean of each arm decreases with the number of offline samples for that arm, and all rewards are deterministic. Then, using the property that \algucb matches the upper bounds of arms (up to rounding effects, roughly speaking), we show that for a proper choice of mean parameters, each arm is sampled at least $\frac{T}{K}$ times in the online phase.  \hfill \( \Box\)

The above bound may be hard to interpret. It is easier to interpret when instantiated for extreme values of offline arm counts. For instance, if all of the offline samples come from a single arm, i.e., $m_1=m$ and $m_i=0$ for any $i>1$, we obtain
    \[\mathcal{R}^{\text{log}}_{\textsc{UCB}}(T)\geq T\frac{K-1}{K}\left[\sqrt{\frac{K}{2T}}-\sqrt{\frac{1}{2(m+\frac{T}{K})}}\right].\]
For any $T\leq K m$, this entails
    \[\mathcal{R}^{\text{log}}_{\textsc{UCB}}(T)\geq \frac{1}{10}\sqrt{KT}.\]
Similarly, if we have balanced samples for half of the arms ($m_i=\frac{2m}{K}$, for each $i \leq \frac{K}{2}$, and $m_i=0$ for $i\geq \frac{K}{2}$), then, for any $T\leq m$, we have
        $\mathcal{R}^{\text{log}}_{\textsc{UCB}}(T)\geq \frac{1}{10}\sqrt{KT}.$
On the other hand, if $m_i=\frac{m}{K}$ for all $i \in [K]$, we get $\mathcal{R}^{\text{log}}_{\textsc{UCB}}(T)\geq 0$, which may not be a strong lower bound, but it aligns with the well-known observation that a uniform offline dataset is favorable for \algucb's theoretical guarantee. 

\begin{remark}
    The upper bound obtained here does not match the lower bound for all offline count repartitions. For instance, in the case where the offline samples are uniformly spread, the former is much larger than the latter. It remains an open question to find matching bounds in all regimes. 
\end{remark}

\begin{proposition}[ \alglcb's regret against the logging policy for general $T$]\label{prop:regretlogginglcb}
We have
\[
\mathcal{R}^{\text{log}}_\textsc{LCB}(T) \leq T\frac{\sum_i \sqrt{m_i}}{\sum_i m_i}\sqrt{2\log\left(\frac{K}{\delta}\right)}+2T^2\delta.
\]
Moreover, assuming  $m_1\geq m_2\geq \ldots\geq m_K$,
\[
\mathcal{R}^{\text{log}}_\textsc{LCB}(T)\geq 0.15 T\frac{\sqrt{m_2}}{m}.
\]
\end{proposition}



It is again informative to instantiate the above bound in extreme regimes of the offline count repartition. If $m_1=m$ and $m_i=0$ for any $i>1$, we obtain
    \[\mathcal{R}^{\text{log}}_\textsc{LCB}(T) \leq T\sqrt{\frac{2\log\left(\frac{K}{\delta}\right)}{m}}+2T^2\delta.\]  
On the other hand, if $m_i=\frac{m}{K}$ for all $i \in [K]$, we get:      \[\mathcal{R}^{\text{log}}_\textsc{LCB}(T) \leq T\sqrt{\frac{2K\log\left(\frac{K}{\delta}\right)}{m}}+2T^2\delta.\]


\section{Experiments}
\label{sec:experiments}
The experiments are designed to address two key research questions.
First, \textbf{RQ1} evaluates whether the average $L_2$-norm of the counterfactual perturbation vectors ($\overline{||\perturb||}$) decreases as the model overfits the data, thereby providing further empirical validation for our hypothesis.
Second, \textbf{RQ2} evaluates the ability of the proposed counterfactual regularized loss, as defined in (\ref{eq:regularized_loss2}), to mitigate overfitting when compared to existing regularization techniques.

% The experiments are designed to address three key research questions. First, \textbf{RQ1} investigates whether the mean perturbation vector norm decreases as the model overfits the data, aiming to further validate our intuition. Second, \textbf{RQ2} explores whether the mean perturbation vector norm can be effectively leveraged as a regularization term during training, offering insights into its potential role in mitigating overfitting. Finally, \textbf{RQ3} examines whether our counterfactual regularizer enables the model to achieve superior performance compared to existing regularization methods, thus highlighting its practical advantage.

\subsection{Experimental Setup}
\textbf{\textit{Datasets, Models, and Tasks.}}
The experiments are conducted on three datasets: \textit{Water Potability}~\cite{kadiwal2020waterpotability}, \textit{Phomene}~\cite{phomene}, and \textit{CIFAR-10}~\cite{krizhevsky2009learning}. For \textit{Water Potability} and \textit{Phomene}, we randomly select $80\%$ of the samples for the training set, and the remaining $20\%$ for the test set, \textit{CIFAR-10} comes already split. Furthermore, we consider the following models: Logistic Regression, Multi-Layer Perceptron (MLP) with 100 and 30 neurons on each hidden layer, and PreactResNet-18~\cite{he2016cvecvv} as a Convolutional Neural Network (CNN) architecture.
We focus on binary classification tasks and leave the extension to multiclass scenarios for future work. However, for datasets that are inherently multiclass, we transform the problem into a binary classification task by selecting two classes, aligning with our assumption.

\smallskip
\noindent\textbf{\textit{Evaluation Measures.}} To characterize the degree of overfitting, we use the test loss, as it serves as a reliable indicator of the model's generalization capability to unseen data. Additionally, we evaluate the predictive performance of each model using the test accuracy.

\smallskip
\noindent\textbf{\textit{Baselines.}} We compare CF-Reg with the following regularization techniques: L1 (``Lasso''), L2 (``Ridge''), and Dropout.

\smallskip
\noindent\textbf{\textit{Configurations.}}
For each model, we adopt specific configurations as follows.
\begin{itemize}
\item \textit{Logistic Regression:} To induce overfitting in the model, we artificially increase the dimensionality of the data beyond the number of training samples by applying a polynomial feature expansion. This approach ensures that the model has enough capacity to overfit the training data, allowing us to analyze the impact of our counterfactual regularizer. The degree of the polynomial is chosen as the smallest degree that makes the number of features greater than the number of data.
\item \textit{Neural Networks (MLP and CNN):} To take advantage of the closed-form solution for computing the optimal perturbation vector as defined in (\ref{eq:opt-delta}), we use a local linear approximation of the neural network models. Hence, given an instance $\inst_i$, we consider the (optimal) counterfactual not with respect to $\model$ but with respect to:
\begin{equation}
\label{eq:taylor}
    \model^{lin}(\inst) = \model(\inst_i) + \nabla_{\inst}\model(\inst_i)(\inst - \inst_i),
\end{equation}
where $\model^{lin}$ represents the first-order Taylor approximation of $\model$ at $\inst_i$.
Note that this step is unnecessary for Logistic Regression, as it is inherently a linear model.
\end{itemize}

\smallskip
\noindent \textbf{\textit{Implementation Details.}} We run all experiments on a machine equipped with an AMD Ryzen 9 7900 12-Core Processor and an NVIDIA GeForce RTX 4090 GPU. Our implementation is based on the PyTorch Lightning framework. We use stochastic gradient descent as the optimizer with a learning rate of $\eta = 0.001$ and no weight decay. We use a batch size of $128$. The training and test steps are conducted for $6000$ epochs on the \textit{Water Potability} and \textit{Phoneme} datasets, while for the \textit{CIFAR-10} dataset, they are performed for $200$ epochs.
Finally, the contribution $w_i^{\varepsilon}$ of each training point $\inst_i$ is uniformly set as $w_i^{\varepsilon} = 1~\forall i\in \{1,\ldots,m\}$.

The source code implementation for our experiments is available at the following GitHub repository: \url{https://anonymous.4open.science/r/COCE-80B4/README.md} 

\subsection{RQ1: Counterfactual Perturbation vs. Overfitting}
To address \textbf{RQ1}, we analyze the relationship between the test loss and the average $L_2$-norm of the counterfactual perturbation vectors ($\overline{||\perturb||}$) over training epochs.

In particular, Figure~\ref{fig:delta_loss_epochs} depicts the evolution of $\overline{||\perturb||}$ alongside the test loss for an MLP trained \textit{without} regularization on the \textit{Water Potability} dataset. 
\begin{figure}[ht]
    \centering
    \includegraphics[width=0.85\linewidth]{img/delta_loss_epochs.png}
    \caption{The average counterfactual perturbation vector $\overline{||\perturb||}$ (left $y$-axis) and the cross-entropy test loss (right $y$-axis) over training epochs ($x$-axis) for an MLP trained on the \textit{Water Potability} dataset \textit{without} regularization.}
    \label{fig:delta_loss_epochs}
\end{figure}

The plot shows a clear trend as the model starts to overfit the data (evidenced by an increase in test loss). 
Notably, $\overline{||\perturb||}$ begins to decrease, which aligns with the hypothesis that the average distance to the optimal counterfactual example gets smaller as the model's decision boundary becomes increasingly adherent to the training data.

It is worth noting that this trend is heavily influenced by the choice of the counterfactual generator model. In particular, the relationship between $\overline{||\perturb||}$ and the degree of overfitting may become even more pronounced when leveraging more accurate counterfactual generators. However, these models often come at the cost of higher computational complexity, and their exploration is left to future work.

Nonetheless, we expect that $\overline{||\perturb||}$ will eventually stabilize at a plateau, as the average $L_2$-norm of the optimal counterfactual perturbations cannot vanish to zero.

% Additionally, the choice of employing the score-based counterfactual explanation framework to generate counterfactuals was driven to promote computational efficiency.

% Future enhancements to the framework may involve adopting models capable of generating more precise counterfactuals. While such approaches may yield to performance improvements, they are likely to come at the cost of increased computational complexity.


\subsection{RQ2: Counterfactual Regularization Performance}
To answer \textbf{RQ2}, we evaluate the effectiveness of the proposed counterfactual regularization (CF-Reg) by comparing its performance against existing baselines: unregularized training loss (No-Reg), L1 regularization (L1-Reg), L2 regularization (L2-Reg), and Dropout.
Specifically, for each model and dataset combination, Table~\ref{tab:regularization_comparison} presents the mean value and standard deviation of test accuracy achieved by each method across 5 random initialization. 

The table illustrates that our regularization technique consistently delivers better results than existing methods across all evaluated scenarios, except for one case -- i.e., Logistic Regression on the \textit{Phomene} dataset. 
However, this setting exhibits an unusual pattern, as the highest model accuracy is achieved without any regularization. Even in this case, CF-Reg still surpasses other regularization baselines.

From the results above, we derive the following key insights. First, CF-Reg proves to be effective across various model types, ranging from simple linear models (Logistic Regression) to deep architectures like MLPs and CNNs, and across diverse datasets, including both tabular and image data. 
Second, CF-Reg's strong performance on the \textit{Water} dataset with Logistic Regression suggests that its benefits may be more pronounced when applied to simpler models. However, the unexpected outcome on the \textit{Phoneme} dataset calls for further investigation into this phenomenon.


\begin{table*}[h!]
    \centering
    \caption{Mean value and standard deviation of test accuracy across 5 random initializations for different model, dataset, and regularization method. The best results are highlighted in \textbf{bold}.}
    \label{tab:regularization_comparison}
    \begin{tabular}{|c|c|c|c|c|c|c|}
        \hline
        \textbf{Model} & \textbf{Dataset} & \textbf{No-Reg} & \textbf{L1-Reg} & \textbf{L2-Reg} & \textbf{Dropout} & \textbf{CF-Reg (ours)} \\ \hline
        Logistic Regression   & \textit{Water}   & $0.6595 \pm 0.0038$   & $0.6729 \pm 0.0056$   & $0.6756 \pm 0.0046$  & N/A    & $\mathbf{0.6918 \pm 0.0036}$                     \\ \hline
        MLP   & \textit{Water}   & $0.6756 \pm 0.0042$   & $0.6790 \pm 0.0058$   & $0.6790 \pm 0.0023$  & $0.6750 \pm 0.0036$    & $\mathbf{0.6802 \pm 0.0046}$                    \\ \hline
%        MLP   & \textit{Adult}   & $0.8404 \pm 0.0010$   & $\mathbf{0.8495 \pm 0.0007}$   & $0.8489 \pm 0.0014$  & $\mathbf{0.8495 \pm 0.0016}$     & $0.8449 \pm 0.0019$                    \\ \hline
        Logistic Regression   & \textit{Phomene}   & $\mathbf{0.8148 \pm 0.0020}$   & $0.8041 \pm 0.0028$   & $0.7835 \pm 0.0176$  & N/A    & $0.8098 \pm 0.0055$                     \\ \hline
        MLP   & \textit{Phomene}   & $0.8677 \pm 0.0033$   & $0.8374 \pm 0.0080$   & $0.8673 \pm 0.0045$  & $0.8672 \pm 0.0042$     & $\mathbf{0.8718 \pm 0.0040}$                    \\ \hline
        CNN   & \textit{CIFAR-10} & $0.6670 \pm 0.0233$   & $0.6229 \pm 0.0850$   & $0.7348 \pm 0.0365$   & N/A    & $\mathbf{0.7427 \pm 0.0571}$                     \\ \hline
    \end{tabular}
\end{table*}

\begin{table*}[htb!]
    \centering
    \caption{Hyperparameter configurations utilized for the generation of Table \ref{tab:regularization_comparison}. For our regularization the hyperparameters are reported as $\mathbf{\alpha/\beta}$.}
    \label{tab:performance_parameters}
    \begin{tabular}{|c|c|c|c|c|c|c|}
        \hline
        \textbf{Model} & \textbf{Dataset} & \textbf{No-Reg} & \textbf{L1-Reg} & \textbf{L2-Reg} & \textbf{Dropout} & \textbf{CF-Reg (ours)} \\ \hline
        Logistic Regression   & \textit{Water}   & N/A   & $0.0093$   & $0.6927$  & N/A    & $0.3791/1.0355$                     \\ \hline
        MLP   & \textit{Water}   & N/A   & $0.0007$   & $0.0022$  & $0.0002$    & $0.2567/1.9775$                    \\ \hline
        Logistic Regression   &
        \textit{Phomene}   & N/A   & $0.0097$   & $0.7979$  & N/A    & $0.0571/1.8516$                     \\ \hline
        MLP   & \textit{Phomene}   & N/A   & $0.0007$   & $4.24\cdot10^{-5}$  & $0.0015$    & $0.0516/2.2700$                    \\ \hline
       % MLP   & \textit{Adult}   & N/A   & $0.0018$   & $0.0018$  & $0.0601$     & $0.0764/2.2068$                    \\ \hline
        CNN   & \textit{CIFAR-10} & N/A   & $0.0050$   & $0.0864$ & N/A    & $0.3018/
        2.1502$                     \\ \hline
    \end{tabular}
\end{table*}

\begin{table*}[htb!]
    \centering
    \caption{Mean value and standard deviation of training time across 5 different runs. The reported time (in seconds) corresponds to the generation of each entry in Table \ref{tab:regularization_comparison}. Times are }
    \label{tab:times}
    \begin{tabular}{|c|c|c|c|c|c|c|}
        \hline
        \textbf{Model} & \textbf{Dataset} & \textbf{No-Reg} & \textbf{L1-Reg} & \textbf{L2-Reg} & \textbf{Dropout} & \textbf{CF-Reg (ours)} \\ \hline
        Logistic Regression   & \textit{Water}   & $222.98 \pm 1.07$   & $239.94 \pm 2.59$   & $241.60 \pm 1.88$  & N/A    & $251.50 \pm 1.93$                     \\ \hline
        MLP   & \textit{Water}   & $225.71 \pm 3.85$   & $250.13 \pm 4.44$   & $255.78 \pm 2.38$  & $237.83 \pm 3.45$    & $266.48 \pm 3.46$                    \\ \hline
        Logistic Regression   & \textit{Phomene}   & $266.39 \pm 0.82$ & $367.52 \pm 6.85$   & $361.69 \pm 4.04$  & N/A   & $310.48 \pm 0.76$                    \\ \hline
        MLP   &
        \textit{Phomene} & $335.62 \pm 1.77$   & $390.86 \pm 2.11$   & $393.96 \pm 1.95$ & $363.51 \pm 5.07$    & $403.14 \pm 1.92$                     \\ \hline
       % MLP   & \textit{Adult}   & N/A   & $0.0018$   & $0.0018$  & $0.0601$     & $0.0764/2.2068$                    \\ \hline
        CNN   & \textit{CIFAR-10} & $370.09 \pm 0.18$   & $395.71 \pm 0.55$   & $401.38 \pm 0.16$ & N/A    & $1287.8 \pm 0.26$                     \\ \hline
    \end{tabular}
\end{table*}

\subsection{Feasibility of our Method}
A crucial requirement for any regularization technique is that it should impose minimal impact on the overall training process.
In this respect, CF-Reg introduces an overhead that depends on the time required to find the optimal counterfactual example for each training instance. 
As such, the more sophisticated the counterfactual generator model probed during training the higher would be the time required. However, a more advanced counterfactual generator might provide a more effective regularization. We discuss this trade-off in more details in Section~\ref{sec:discussion}.

Table~\ref{tab:times} presents the average training time ($\pm$ standard deviation) for each model and dataset combination listed in Table~\ref{tab:regularization_comparison}.
We can observe that the higher accuracy achieved by CF-Reg using the score-based counterfactual generator comes with only minimal overhead. However, when applied to deep neural networks with many hidden layers, such as \textit{PreactResNet-18}, the forward derivative computation required for the linearization of the network introduces a more noticeable computational cost, explaining the longer training times in the table.

\subsection{Hyperparameter Sensitivity Analysis}
The proposed counterfactual regularization technique relies on two key hyperparameters: $\alpha$ and $\beta$. The former is intrinsic to the loss formulation defined in (\ref{eq:cf-train}), while the latter is closely tied to the choice of the score-based counterfactual explanation method used.

Figure~\ref{fig:test_alpha_beta} illustrates how the test accuracy of an MLP trained on the \textit{Water Potability} dataset changes for different combinations of $\alpha$ and $\beta$.

\begin{figure}[ht]
    \centering
    \includegraphics[width=0.85\linewidth]{img/test_acc_alpha_beta.png}
    \caption{The test accuracy of an MLP trained on the \textit{Water Potability} dataset, evaluated while varying the weight of our counterfactual regularizer ($\alpha$) for different values of $\beta$.}
    \label{fig:test_alpha_beta}
\end{figure}

We observe that, for a fixed $\beta$, increasing the weight of our counterfactual regularizer ($\alpha$) can slightly improve test accuracy until a sudden drop is noticed for $\alpha > 0.1$.
This behavior was expected, as the impact of our penalty, like any regularization term, can be disruptive if not properly controlled.

Moreover, this finding further demonstrates that our regularization method, CF-Reg, is inherently data-driven. Therefore, it requires specific fine-tuning based on the combination of the model and dataset at hand.
\section{Proofs}
\label{sec:appendix}


\section{Conclusion}


\sdeni{}{We introduced convex-concave generative-adversarial characterization of inverse Nash equilibria for a large class of games, including normal-form, finite state and action Markov games, and a number of continuous state and action Markov games. This novel formulation then allowed us to obtain polynomial-time computation guarantees for inverse equilibria in these games, a rather surprising result since the computation of a Nash equilibrium is in general PPAD-complete. Our result can be thus seen as a positive computation result for game theory. We then extended our characterization to a multiagent apprenticeship learning setting, where we souught to not only rationalize the observed behavior as an inverse Nash equilibrium but also make predictions based off the inverse Nash equilibrium, and have shown in experiments on prices in Spanish electricity markets that our approach to solving multiagent apprenticship learning can be effective at predicting behavior in multiagent systems. The approach to inverse game theory that we provided in this paper is a highly flexible one and thus can be used to solve inverse equilibrium beyond inverse Nash equilibria and future work could explore ways to extend our approach to other game-theoretic settings and equilibrium concepts.}


\amy{discuss extenstion to other eqm concepts? CE, CCE, etc.}

\amy{and other idea from yesterday. check text thread?}



\section{Acknowledgements}





\bibliography{sample}
\newpage
\subsection{Lloyd-Max Algorithm}
\label{subsec:Lloyd-Max}
For a given quantization bitwidth $B$ and an operand $\bm{X}$, the Lloyd-Max algorithm finds $2^B$ quantization levels $\{\hat{x}_i\}_{i=1}^{2^B}$ such that quantizing $\bm{X}$ by rounding each scalar in $\bm{X}$ to the nearest quantization level minimizes the quantization MSE. 

The algorithm starts with an initial guess of quantization levels and then iteratively computes quantization thresholds $\{\tau_i\}_{i=1}^{2^B-1}$ and updates quantization levels $\{\hat{x}_i\}_{i=1}^{2^B}$. Specifically, at iteration $n$, thresholds are set to the midpoints of the previous iteration's levels:
\begin{align*}
    \tau_i^{(n)}=\frac{\hat{x}_i^{(n-1)}+\hat{x}_{i+1}^{(n-1)}}2 \text{ for } i=1\ldots 2^B-1
\end{align*}
Subsequently, the quantization levels are re-computed as conditional means of the data regions defined by the new thresholds:
\begin{align*}
    \hat{x}_i^{(n)}=\mathbb{E}\left[ \bm{X} \big| \bm{X}\in [\tau_{i-1}^{(n)},\tau_i^{(n)}] \right] \text{ for } i=1\ldots 2^B
\end{align*}
where to satisfy boundary conditions we have $\tau_0=-\infty$ and $\tau_{2^B}=\infty$. The algorithm iterates the above steps until convergence.

Figure \ref{fig:lm_quant} compares the quantization levels of a $7$-bit floating point (E3M3) quantizer (left) to a $7$-bit Lloyd-Max quantizer (right) when quantizing a layer of weights from the GPT3-126M model at a per-tensor granularity. As shown, the Lloyd-Max quantizer achieves substantially lower quantization MSE. Further, Table \ref{tab:FP7_vs_LM7} shows the superior perplexity achieved by Lloyd-Max quantizers for bitwidths of $7$, $6$ and $5$. The difference between the quantizers is clear at 5 bits, where per-tensor FP quantization incurs a drastic and unacceptable increase in perplexity, while Lloyd-Max quantization incurs a much smaller increase. Nevertheless, we note that even the optimal Lloyd-Max quantizer incurs a notable ($\sim 1.5$) increase in perplexity due to the coarse granularity of quantization. 

\begin{figure}[h]
  \centering
  \includegraphics[width=0.7\linewidth]{sections/figures/LM7_FP7.pdf}
  \caption{\small Quantization levels and the corresponding quantization MSE of Floating Point (left) vs Lloyd-Max (right) Quantizers for a layer of weights in the GPT3-126M model.}
  \label{fig:lm_quant}
\end{figure}

\begin{table}[h]\scriptsize
\begin{center}
\caption{\label{tab:FP7_vs_LM7} \small Comparing perplexity (lower is better) achieved by floating point quantizers and Lloyd-Max quantizers on a GPT3-126M model for the Wikitext-103 dataset.}
\begin{tabular}{c|cc|c}
\hline
 \multirow{2}{*}{\textbf{Bitwidth}} & \multicolumn{2}{|c|}{\textbf{Floating-Point Quantizer}} & \textbf{Lloyd-Max Quantizer} \\
 & Best Format & Wikitext-103 Perplexity & Wikitext-103 Perplexity \\
\hline
7 & E3M3 & 18.32 & 18.27 \\
6 & E3M2 & 19.07 & 18.51 \\
5 & E4M0 & 43.89 & 19.71 \\
\hline
\end{tabular}
\end{center}
\end{table}

\subsection{Proof of Local Optimality of LO-BCQ}
\label{subsec:lobcq_opt_proof}
For a given block $\bm{b}_j$, the quantization MSE during LO-BCQ can be empirically evaluated as $\frac{1}{L_b}\lVert \bm{b}_j- \bm{\hat{b}}_j\rVert^2_2$ where $\bm{\hat{b}}_j$ is computed from equation (\ref{eq:clustered_quantization_definition}) as $C_{f(\bm{b}_j)}(\bm{b}_j)$. Further, for a given block cluster $\mathcal{B}_i$, we compute the quantization MSE as $\frac{1}{|\mathcal{B}_{i}|}\sum_{\bm{b} \in \mathcal{B}_{i}} \frac{1}{L_b}\lVert \bm{b}- C_i^{(n)}(\bm{b})\rVert^2_2$. Therefore, at the end of iteration $n$, we evaluate the overall quantization MSE $J^{(n)}$ for a given operand $\bm{X}$ composed of $N_c$ block clusters as:
\begin{align*}
    \label{eq:mse_iter_n}
    J^{(n)} = \frac{1}{N_c} \sum_{i=1}^{N_c} \frac{1}{|\mathcal{B}_{i}^{(n)}|}\sum_{\bm{v} \in \mathcal{B}_{i}^{(n)}} \frac{1}{L_b}\lVert \bm{b}- B_i^{(n)}(\bm{b})\rVert^2_2
\end{align*}

At the end of iteration $n$, the codebooks are updated from $\mathcal{C}^{(n-1)}$ to $\mathcal{C}^{(n)}$. However, the mapping of a given vector $\bm{b}_j$ to quantizers $\mathcal{C}^{(n)}$ remains as  $f^{(n)}(\bm{b}_j)$. At the next iteration, during the vector clustering step, $f^{(n+1)}(\bm{b}_j)$ finds new mapping of $\bm{b}_j$ to updated codebooks $\mathcal{C}^{(n)}$ such that the quantization MSE over the candidate codebooks is minimized. Therefore, we obtain the following result for $\bm{b}_j$:
\begin{align*}
\frac{1}{L_b}\lVert \bm{b}_j - C_{f^{(n+1)}(\bm{b}_j)}^{(n)}(\bm{b}_j)\rVert^2_2 \le \frac{1}{L_b}\lVert \bm{b}_j - C_{f^{(n)}(\bm{b}_j)}^{(n)}(\bm{b}_j)\rVert^2_2
\end{align*}

That is, quantizing $\bm{b}_j$ at the end of the block clustering step of iteration $n+1$ results in lower quantization MSE compared to quantizing at the end of iteration $n$. Since this is true for all $\bm{b} \in \bm{X}$, we assert the following:
\begin{equation}
\begin{split}
\label{eq:mse_ineq_1}
    \tilde{J}^{(n+1)} &= \frac{1}{N_c} \sum_{i=1}^{N_c} \frac{1}{|\mathcal{B}_{i}^{(n+1)}|}\sum_{\bm{b} \in \mathcal{B}_{i}^{(n+1)}} \frac{1}{L_b}\lVert \bm{b} - C_i^{(n)}(b)\rVert^2_2 \le J^{(n)}
\end{split}
\end{equation}
where $\tilde{J}^{(n+1)}$ is the the quantization MSE after the vector clustering step at iteration $n+1$.

Next, during the codebook update step (\ref{eq:quantizers_update}) at iteration $n+1$, the per-cluster codebooks $\mathcal{C}^{(n)}$ are updated to $\mathcal{C}^{(n+1)}$ by invoking the Lloyd-Max algorithm \citep{Lloyd}. We know that for any given value distribution, the Lloyd-Max algorithm minimizes the quantization MSE. Therefore, for a given vector cluster $\mathcal{B}_i$ we obtain the following result:

\begin{equation}
    \frac{1}{|\mathcal{B}_{i}^{(n+1)}|}\sum_{\bm{b} \in \mathcal{B}_{i}^{(n+1)}} \frac{1}{L_b}\lVert \bm{b}- C_i^{(n+1)}(\bm{b})\rVert^2_2 \le \frac{1}{|\mathcal{B}_{i}^{(n+1)}|}\sum_{\bm{b} \in \mathcal{B}_{i}^{(n+1)}} \frac{1}{L_b}\lVert \bm{b}- C_i^{(n)}(\bm{b})\rVert^2_2
\end{equation}

The above equation states that quantizing the given block cluster $\mathcal{B}_i$ after updating the associated codebook from $C_i^{(n)}$ to $C_i^{(n+1)}$ results in lower quantization MSE. Since this is true for all the block clusters, we derive the following result: 
\begin{equation}
\begin{split}
\label{eq:mse_ineq_2}
     J^{(n+1)} &= \frac{1}{N_c} \sum_{i=1}^{N_c} \frac{1}{|\mathcal{B}_{i}^{(n+1)}|}\sum_{\bm{b} \in \mathcal{B}_{i}^{(n+1)}} \frac{1}{L_b}\lVert \bm{b}- C_i^{(n+1)}(\bm{b})\rVert^2_2  \le \tilde{J}^{(n+1)}   
\end{split}
\end{equation}

Following (\ref{eq:mse_ineq_1}) and (\ref{eq:mse_ineq_2}), we find that the quantization MSE is non-increasing for each iteration, that is, $J^{(1)} \ge J^{(2)} \ge J^{(3)} \ge \ldots \ge J^{(M)}$ where $M$ is the maximum number of iterations. 
%Therefore, we can say that if the algorithm converges, then it must be that it has converged to a local minimum. 
\hfill $\blacksquare$


\begin{figure}
    \begin{center}
    \includegraphics[width=0.5\textwidth]{sections//figures/mse_vs_iter.pdf}
    \end{center}
    \caption{\small NMSE vs iterations during LO-BCQ compared to other block quantization proposals}
    \label{fig:nmse_vs_iter}
\end{figure}

Figure \ref{fig:nmse_vs_iter} shows the empirical convergence of LO-BCQ across several block lengths and number of codebooks. Also, the MSE achieved by LO-BCQ is compared to baselines such as MXFP and VSQ. As shown, LO-BCQ converges to a lower MSE than the baselines. Further, we achieve better convergence for larger number of codebooks ($N_c$) and for a smaller block length ($L_b$), both of which increase the bitwidth of BCQ (see Eq \ref{eq:bitwidth_bcq}).


\subsection{Additional Accuracy Results}
%Table \ref{tab:lobcq_config} lists the various LOBCQ configurations and their corresponding bitwidths.
\begin{table}
\setlength{\tabcolsep}{4.75pt}
\begin{center}
\caption{\label{tab:lobcq_config} Various LO-BCQ configurations and their bitwidths.}
\begin{tabular}{|c||c|c|c|c||c|c||c|} 
\hline
 & \multicolumn{4}{|c||}{$L_b=8$} & \multicolumn{2}{|c||}{$L_b=4$} & $L_b=2$ \\
 \hline
 \backslashbox{$L_A$\kern-1em}{\kern-1em$N_c$} & 2 & 4 & 8 & 16 & 2 & 4 & 2 \\
 \hline
 64 & 4.25 & 4.375 & 4.5 & 4.625 & 4.375 & 4.625 & 4.625\\
 \hline
 32 & 4.375 & 4.5 & 4.625& 4.75 & 4.5 & 4.75 & 4.75 \\
 \hline
 16 & 4.625 & 4.75& 4.875 & 5 & 4.75 & 5 & 5 \\
 \hline
\end{tabular}
\end{center}
\end{table}

%\subsection{Perplexity achieved by various LO-BCQ configurations on Wikitext-103 dataset}

\begin{table} \centering
\begin{tabular}{|c||c|c|c|c||c|c||c|} 
\hline
 $L_b \rightarrow$& \multicolumn{4}{c||}{8} & \multicolumn{2}{c||}{4} & 2\\
 \hline
 \backslashbox{$L_A$\kern-1em}{\kern-1em$N_c$} & 2 & 4 & 8 & 16 & 2 & 4 & 2  \\
 %$N_c \rightarrow$ & 2 & 4 & 8 & 16 & 2 & 4 & 2 \\
 \hline
 \hline
 \multicolumn{8}{c}{GPT3-1.3B (FP32 PPL = 9.98)} \\ 
 \hline
 \hline
 64 & 10.40 & 10.23 & 10.17 & 10.15 &  10.28 & 10.18 & 10.19 \\
 \hline
 32 & 10.25 & 10.20 & 10.15 & 10.12 &  10.23 & 10.17 & 10.17 \\
 \hline
 16 & 10.22 & 10.16 & 10.10 & 10.09 &  10.21 & 10.14 & 10.16 \\
 \hline
  \hline
 \multicolumn{8}{c}{GPT3-8B (FP32 PPL = 7.38)} \\ 
 \hline
 \hline
 64 & 7.61 & 7.52 & 7.48 &  7.47 &  7.55 &  7.49 & 7.50 \\
 \hline
 32 & 7.52 & 7.50 & 7.46 &  7.45 &  7.52 &  7.48 & 7.48  \\
 \hline
 16 & 7.51 & 7.48 & 7.44 &  7.44 &  7.51 &  7.49 & 7.47  \\
 \hline
\end{tabular}
\caption{\label{tab:ppl_gpt3_abalation} Wikitext-103 perplexity across GPT3-1.3B and 8B models.}
\end{table}

\begin{table} \centering
\begin{tabular}{|c||c|c|c|c||} 
\hline
 $L_b \rightarrow$& \multicolumn{4}{c||}{8}\\
 \hline
 \backslashbox{$L_A$\kern-1em}{\kern-1em$N_c$} & 2 & 4 & 8 & 16 \\
 %$N_c \rightarrow$ & 2 & 4 & 8 & 16 & 2 & 4 & 2 \\
 \hline
 \hline
 \multicolumn{5}{|c|}{Llama2-7B (FP32 PPL = 5.06)} \\ 
 \hline
 \hline
 64 & 5.31 & 5.26 & 5.19 & 5.18  \\
 \hline
 32 & 5.23 & 5.25 & 5.18 & 5.15  \\
 \hline
 16 & 5.23 & 5.19 & 5.16 & 5.14  \\
 \hline
 \multicolumn{5}{|c|}{Nemotron4-15B (FP32 PPL = 5.87)} \\ 
 \hline
 \hline
 64  & 6.3 & 6.20 & 6.13 & 6.08  \\
 \hline
 32  & 6.24 & 6.12 & 6.07 & 6.03  \\
 \hline
 16  & 6.12 & 6.14 & 6.04 & 6.02  \\
 \hline
 \multicolumn{5}{|c|}{Nemotron4-340B (FP32 PPL = 3.48)} \\ 
 \hline
 \hline
 64 & 3.67 & 3.62 & 3.60 & 3.59 \\
 \hline
 32 & 3.63 & 3.61 & 3.59 & 3.56 \\
 \hline
 16 & 3.61 & 3.58 & 3.57 & 3.55 \\
 \hline
\end{tabular}
\caption{\label{tab:ppl_llama7B_nemo15B} Wikitext-103 perplexity compared to FP32 baseline in Llama2-7B and Nemotron4-15B, 340B models}
\end{table}

%\subsection{Perplexity achieved by various LO-BCQ configurations on MMLU dataset}


\begin{table} \centering
\begin{tabular}{|c||c|c|c|c||c|c|c|c|} 
\hline
 $L_b \rightarrow$& \multicolumn{4}{c||}{8} & \multicolumn{4}{c||}{8}\\
 \hline
 \backslashbox{$L_A$\kern-1em}{\kern-1em$N_c$} & 2 & 4 & 8 & 16 & 2 & 4 & 8 & 16  \\
 %$N_c \rightarrow$ & 2 & 4 & 8 & 16 & 2 & 4 & 2 \\
 \hline
 \hline
 \multicolumn{5}{|c|}{Llama2-7B (FP32 Accuracy = 45.8\%)} & \multicolumn{4}{|c|}{Llama2-70B (FP32 Accuracy = 69.12\%)} \\ 
 \hline
 \hline
 64 & 43.9 & 43.4 & 43.9 & 44.9 & 68.07 & 68.27 & 68.17 & 68.75 \\
 \hline
 32 & 44.5 & 43.8 & 44.9 & 44.5 & 68.37 & 68.51 & 68.35 & 68.27  \\
 \hline
 16 & 43.9 & 42.7 & 44.9 & 45 & 68.12 & 68.77 & 68.31 & 68.59  \\
 \hline
 \hline
 \multicolumn{5}{|c|}{GPT3-22B (FP32 Accuracy = 38.75\%)} & \multicolumn{4}{|c|}{Nemotron4-15B (FP32 Accuracy = 64.3\%)} \\ 
 \hline
 \hline
 64 & 36.71 & 38.85 & 38.13 & 38.92 & 63.17 & 62.36 & 63.72 & 64.09 \\
 \hline
 32 & 37.95 & 38.69 & 39.45 & 38.34 & 64.05 & 62.30 & 63.8 & 64.33  \\
 \hline
 16 & 38.88 & 38.80 & 38.31 & 38.92 & 63.22 & 63.51 & 63.93 & 64.43  \\
 \hline
\end{tabular}
\caption{\label{tab:mmlu_abalation} Accuracy on MMLU dataset across GPT3-22B, Llama2-7B, 70B and Nemotron4-15B models.}
\end{table}


%\subsection{Perplexity achieved by various LO-BCQ configurations on LM evaluation harness}

\begin{table} \centering
\begin{tabular}{|c||c|c|c|c||c|c|c|c|} 
\hline
 $L_b \rightarrow$& \multicolumn{4}{c||}{8} & \multicolumn{4}{c||}{8}\\
 \hline
 \backslashbox{$L_A$\kern-1em}{\kern-1em$N_c$} & 2 & 4 & 8 & 16 & 2 & 4 & 8 & 16  \\
 %$N_c \rightarrow$ & 2 & 4 & 8 & 16 & 2 & 4 & 2 \\
 \hline
 \hline
 \multicolumn{5}{|c|}{Race (FP32 Accuracy = 37.51\%)} & \multicolumn{4}{|c|}{Boolq (FP32 Accuracy = 64.62\%)} \\ 
 \hline
 \hline
 64 & 36.94 & 37.13 & 36.27 & 37.13 & 63.73 & 62.26 & 63.49 & 63.36 \\
 \hline
 32 & 37.03 & 36.36 & 36.08 & 37.03 & 62.54 & 63.51 & 63.49 & 63.55  \\
 \hline
 16 & 37.03 & 37.03 & 36.46 & 37.03 & 61.1 & 63.79 & 63.58 & 63.33  \\
 \hline
 \hline
 \multicolumn{5}{|c|}{Winogrande (FP32 Accuracy = 58.01\%)} & \multicolumn{4}{|c|}{Piqa (FP32 Accuracy = 74.21\%)} \\ 
 \hline
 \hline
 64 & 58.17 & 57.22 & 57.85 & 58.33 & 73.01 & 73.07 & 73.07 & 72.80 \\
 \hline
 32 & 59.12 & 58.09 & 57.85 & 58.41 & 73.01 & 73.94 & 72.74 & 73.18  \\
 \hline
 16 & 57.93 & 58.88 & 57.93 & 58.56 & 73.94 & 72.80 & 73.01 & 73.94  \\
 \hline
\end{tabular}
\caption{\label{tab:mmlu_abalation} Accuracy on LM evaluation harness tasks on GPT3-1.3B model.}
\end{table}

\begin{table} \centering
\begin{tabular}{|c||c|c|c|c||c|c|c|c|} 
\hline
 $L_b \rightarrow$& \multicolumn{4}{c||}{8} & \multicolumn{4}{c||}{8}\\
 \hline
 \backslashbox{$L_A$\kern-1em}{\kern-1em$N_c$} & 2 & 4 & 8 & 16 & 2 & 4 & 8 & 16  \\
 %$N_c \rightarrow$ & 2 & 4 & 8 & 16 & 2 & 4 & 2 \\
 \hline
 \hline
 \multicolumn{5}{|c|}{Race (FP32 Accuracy = 41.34\%)} & \multicolumn{4}{|c|}{Boolq (FP32 Accuracy = 68.32\%)} \\ 
 \hline
 \hline
 64 & 40.48 & 40.10 & 39.43 & 39.90 & 69.20 & 68.41 & 69.45 & 68.56 \\
 \hline
 32 & 39.52 & 39.52 & 40.77 & 39.62 & 68.32 & 67.43 & 68.17 & 69.30  \\
 \hline
 16 & 39.81 & 39.71 & 39.90 & 40.38 & 68.10 & 66.33 & 69.51 & 69.42  \\
 \hline
 \hline
 \multicolumn{5}{|c|}{Winogrande (FP32 Accuracy = 67.88\%)} & \multicolumn{4}{|c|}{Piqa (FP32 Accuracy = 78.78\%)} \\ 
 \hline
 \hline
 64 & 66.85 & 66.61 & 67.72 & 67.88 & 77.31 & 77.42 & 77.75 & 77.64 \\
 \hline
 32 & 67.25 & 67.72 & 67.72 & 67.00 & 77.31 & 77.04 & 77.80 & 77.37  \\
 \hline
 16 & 68.11 & 68.90 & 67.88 & 67.48 & 77.37 & 78.13 & 78.13 & 77.69  \\
 \hline
\end{tabular}
\caption{\label{tab:mmlu_abalation} Accuracy on LM evaluation harness tasks on GPT3-8B model.}
\end{table}

\begin{table} \centering
\begin{tabular}{|c||c|c|c|c||c|c|c|c|} 
\hline
 $L_b \rightarrow$& \multicolumn{4}{c||}{8} & \multicolumn{4}{c||}{8}\\
 \hline
 \backslashbox{$L_A$\kern-1em}{\kern-1em$N_c$} & 2 & 4 & 8 & 16 & 2 & 4 & 8 & 16  \\
 %$N_c \rightarrow$ & 2 & 4 & 8 & 16 & 2 & 4 & 2 \\
 \hline
 \hline
 \multicolumn{5}{|c|}{Race (FP32 Accuracy = 40.67\%)} & \multicolumn{4}{|c|}{Boolq (FP32 Accuracy = 76.54\%)} \\ 
 \hline
 \hline
 64 & 40.48 & 40.10 & 39.43 & 39.90 & 75.41 & 75.11 & 77.09 & 75.66 \\
 \hline
 32 & 39.52 & 39.52 & 40.77 & 39.62 & 76.02 & 76.02 & 75.96 & 75.35  \\
 \hline
 16 & 39.81 & 39.71 & 39.90 & 40.38 & 75.05 & 73.82 & 75.72 & 76.09  \\
 \hline
 \hline
 \multicolumn{5}{|c|}{Winogrande (FP32 Accuracy = 70.64\%)} & \multicolumn{4}{|c|}{Piqa (FP32 Accuracy = 79.16\%)} \\ 
 \hline
 \hline
 64 & 69.14 & 70.17 & 70.17 & 70.56 & 78.24 & 79.00 & 78.62 & 78.73 \\
 \hline
 32 & 70.96 & 69.69 & 71.27 & 69.30 & 78.56 & 79.49 & 79.16 & 78.89  \\
 \hline
 16 & 71.03 & 69.53 & 69.69 & 70.40 & 78.13 & 79.16 & 79.00 & 79.00  \\
 \hline
\end{tabular}
\caption{\label{tab:mmlu_abalation} Accuracy on LM evaluation harness tasks on GPT3-22B model.}
\end{table}

\begin{table} \centering
\begin{tabular}{|c||c|c|c|c||c|c|c|c|} 
\hline
 $L_b \rightarrow$& \multicolumn{4}{c||}{8} & \multicolumn{4}{c||}{8}\\
 \hline
 \backslashbox{$L_A$\kern-1em}{\kern-1em$N_c$} & 2 & 4 & 8 & 16 & 2 & 4 & 8 & 16  \\
 %$N_c \rightarrow$ & 2 & 4 & 8 & 16 & 2 & 4 & 2 \\
 \hline
 \hline
 \multicolumn{5}{|c|}{Race (FP32 Accuracy = 44.4\%)} & \multicolumn{4}{|c|}{Boolq (FP32 Accuracy = 79.29\%)} \\ 
 \hline
 \hline
 64 & 42.49 & 42.51 & 42.58 & 43.45 & 77.58 & 77.37 & 77.43 & 78.1 \\
 \hline
 32 & 43.35 & 42.49 & 43.64 & 43.73 & 77.86 & 75.32 & 77.28 & 77.86  \\
 \hline
 16 & 44.21 & 44.21 & 43.64 & 42.97 & 78.65 & 77 & 76.94 & 77.98  \\
 \hline
 \hline
 \multicolumn{5}{|c|}{Winogrande (FP32 Accuracy = 69.38\%)} & \multicolumn{4}{|c|}{Piqa (FP32 Accuracy = 78.07\%)} \\ 
 \hline
 \hline
 64 & 68.9 & 68.43 & 69.77 & 68.19 & 77.09 & 76.82 & 77.09 & 77.86 \\
 \hline
 32 & 69.38 & 68.51 & 68.82 & 68.90 & 78.07 & 76.71 & 78.07 & 77.86  \\
 \hline
 16 & 69.53 & 67.09 & 69.38 & 68.90 & 77.37 & 77.8 & 77.91 & 77.69  \\
 \hline
\end{tabular}
\caption{\label{tab:mmlu_abalation} Accuracy on LM evaluation harness tasks on Llama2-7B model.}
\end{table}

\begin{table} \centering
\begin{tabular}{|c||c|c|c|c||c|c|c|c|} 
\hline
 $L_b \rightarrow$& \multicolumn{4}{c||}{8} & \multicolumn{4}{c||}{8}\\
 \hline
 \backslashbox{$L_A$\kern-1em}{\kern-1em$N_c$} & 2 & 4 & 8 & 16 & 2 & 4 & 8 & 16  \\
 %$N_c \rightarrow$ & 2 & 4 & 8 & 16 & 2 & 4 & 2 \\
 \hline
 \hline
 \multicolumn{5}{|c|}{Race (FP32 Accuracy = 48.8\%)} & \multicolumn{4}{|c|}{Boolq (FP32 Accuracy = 85.23\%)} \\ 
 \hline
 \hline
 64 & 49.00 & 49.00 & 49.28 & 48.71 & 82.82 & 84.28 & 84.03 & 84.25 \\
 \hline
 32 & 49.57 & 48.52 & 48.33 & 49.28 & 83.85 & 84.46 & 84.31 & 84.93  \\
 \hline
 16 & 49.85 & 49.09 & 49.28 & 48.99 & 85.11 & 84.46 & 84.61 & 83.94  \\
 \hline
 \hline
 \multicolumn{5}{|c|}{Winogrande (FP32 Accuracy = 79.95\%)} & \multicolumn{4}{|c|}{Piqa (FP32 Accuracy = 81.56\%)} \\ 
 \hline
 \hline
 64 & 78.77 & 78.45 & 78.37 & 79.16 & 81.45 & 80.69 & 81.45 & 81.5 \\
 \hline
 32 & 78.45 & 79.01 & 78.69 & 80.66 & 81.56 & 80.58 & 81.18 & 81.34  \\
 \hline
 16 & 79.95 & 79.56 & 79.79 & 79.72 & 81.28 & 81.66 & 81.28 & 80.96  \\
 \hline
\end{tabular}
\caption{\label{tab:mmlu_abalation} Accuracy on LM evaluation harness tasks on Llama2-70B model.}
\end{table}

%\section{MSE Studies}
%\textcolor{red}{TODO}


\subsection{Number Formats and Quantization Method}
\label{subsec:numFormats_quantMethod}
\subsubsection{Integer Format}
An $n$-bit signed integer (INT) is typically represented with a 2s-complement format \citep{yao2022zeroquant,xiao2023smoothquant,dai2021vsq}, where the most significant bit denotes the sign.

\subsubsection{Floating Point Format}
An $n$-bit signed floating point (FP) number $x$ comprises of a 1-bit sign ($x_{\mathrm{sign}}$), $B_m$-bit mantissa ($x_{\mathrm{mant}}$) and $B_e$-bit exponent ($x_{\mathrm{exp}}$) such that $B_m+B_e=n-1$. The associated constant exponent bias ($E_{\mathrm{bias}}$) is computed as $(2^{{B_e}-1}-1)$. We denote this format as $E_{B_e}M_{B_m}$.  

\subsubsection{Quantization Scheme}
\label{subsec:quant_method}
A quantization scheme dictates how a given unquantized tensor is converted to its quantized representation. We consider FP formats for the purpose of illustration. Given an unquantized tensor $\bm{X}$ and an FP format $E_{B_e}M_{B_m}$, we first, we compute the quantization scale factor $s_X$ that maps the maximum absolute value of $\bm{X}$ to the maximum quantization level of the $E_{B_e}M_{B_m}$ format as follows:
\begin{align}
\label{eq:sf}
    s_X = \frac{\mathrm{max}(|\bm{X}|)}{\mathrm{max}(E_{B_e}M_{B_m})}
\end{align}
In the above equation, $|\cdot|$ denotes the absolute value function.

Next, we scale $\bm{X}$ by $s_X$ and quantize it to $\hat{\bm{X}}$ by rounding it to the nearest quantization level of $E_{B_e}M_{B_m}$ as:

\begin{align}
\label{eq:tensor_quant}
    \hat{\bm{X}} = \text{round-to-nearest}\left(\frac{\bm{X}}{s_X}, E_{B_e}M_{B_m}\right)
\end{align}

We perform dynamic max-scaled quantization \citep{wu2020integer}, where the scale factor $s$ for activations is dynamically computed during runtime.

\subsection{Vector Scaled Quantization}
\begin{wrapfigure}{r}{0.35\linewidth}
  \centering
  \includegraphics[width=\linewidth]{sections/figures/vsquant.jpg}
  \caption{\small Vectorwise decomposition for per-vector scaled quantization (VSQ \citep{dai2021vsq}).}
  \label{fig:vsquant}
\end{wrapfigure}
During VSQ \citep{dai2021vsq}, the operand tensors are decomposed into 1D vectors in a hardware friendly manner as shown in Figure \ref{fig:vsquant}. Since the decomposed tensors are used as operands in matrix multiplications during inference, it is beneficial to perform this decomposition along the reduction dimension of the multiplication. The vectorwise quantization is performed similar to tensorwise quantization described in Equations \ref{eq:sf} and \ref{eq:tensor_quant}, where a scale factor $s_v$ is required for each vector $\bm{v}$ that maps the maximum absolute value of that vector to the maximum quantization level. While smaller vector lengths can lead to larger accuracy gains, the associated memory and computational overheads due to the per-vector scale factors increases. To alleviate these overheads, VSQ \citep{dai2021vsq} proposed a second level quantization of the per-vector scale factors to unsigned integers, while MX \citep{rouhani2023shared} quantizes them to integer powers of 2 (denoted as $2^{INT}$).

\subsubsection{MX Format}
The MX format proposed in \citep{rouhani2023microscaling} introduces the concept of sub-block shifting. For every two scalar elements of $b$-bits each, there is a shared exponent bit. The value of this exponent bit is determined through an empirical analysis that targets minimizing quantization MSE. We note that the FP format $E_{1}M_{b}$ is strictly better than MX from an accuracy perspective since it allocates a dedicated exponent bit to each scalar as opposed to sharing it across two scalars. Therefore, we conservatively bound the accuracy of a $b+2$-bit signed MX format with that of a $E_{1}M_{b}$ format in our comparisons. For instance, we use E1M2 format as a proxy for MX4.

\begin{figure}
    \centering
    \includegraphics[width=1\linewidth]{sections//figures/BlockFormats.pdf}
    \caption{\small Comparing LO-BCQ to MX format.}
    \label{fig:block_formats}
\end{figure}

Figure \ref{fig:block_formats} compares our $4$-bit LO-BCQ block format to MX \citep{rouhani2023microscaling}. As shown, both LO-BCQ and MX decompose a given operand tensor into block arrays and each block array into blocks. Similar to MX, we find that per-block quantization ($L_b < L_A$) leads to better accuracy due to increased flexibility. While MX achieves this through per-block $1$-bit micro-scales, we associate a dedicated codebook to each block through a per-block codebook selector. Further, MX quantizes the per-block array scale-factor to E8M0 format without per-tensor scaling. In contrast during LO-BCQ, we find that per-tensor scaling combined with quantization of per-block array scale-factor to E4M3 format results in superior inference accuracy across models. 

\end{document}