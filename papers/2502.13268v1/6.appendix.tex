% \newpage

\appendix

\section{Appendix}

\subsection{Case Study Samples}
\label{appendix}
The following are paraphrased samples of text extracted from various LLM documentations. They were employed during the second phase of our remote interviews with participants, in which they vocally reflected on each statement and considered possible assumptions.

\smallskip
\noindent \textbf{PaLM 2 \cite[p.~64]{anil2023palm}.} \textit{We see major differences in observed representations of people within pre-training data across all dimensions we consider. For sexuality, we see that while marked references are only found in 3\% of documents, most references are related to ``gay'' (53\%) and ``homosexuality'' (22\%)}.

\smallskip
\noindent \textbf{BLOOM \cite[p.~41]{le2023bloom}.} \textit{The library relies on minimal pairs to compare a stereotyped statement and a non-stereotyped statement (e.g. ``Women can’t drive.'' is a gender stereotype while ``Men can’t drive'' is not). The two statements differ only by the social category targeted by the stereotype and that social category is present in the stereotyped statement and not in the non-stereotyped statement. The evaluation aims at assessing systematic preference of models for stereotyped statements.}

\smallskip
\noindent \textbf{LLAMA 2 \cite[p.~20]{touvron2023llama}.} \textit{We followed Meta’s standard privacy and legal review processes for each dataset used in training...We excluded data from certain sites known to contain a high volume of personal information about private individuals...Sharing our models broadly will reduce the need for others to train similar models. No additional filtering was conducted on the datasets, to allow the LLM to be more widely usable across tasks (e.g., it can be better used for hate speech classification), while avoiding the potential for the accidental demographic erasure sometimes caused by over-scrubbing.}

\smallskip
\noindent \textbf{PaLM 2 \cite[p.~20]{anil2023palm}.} \textit{We additionally analyze potential toxic language harms across languages, datasets, and prompts referencing identity terms...Similarly, when disaggregating by identity term we find biases in how potential toxic language harm vary across language. For instance, queries referencing the ``Black'' and ``White'' identity group lead to higher toxicity rates in English, German and Portuguese compared to other languages, and queries referencing ``Judaism'' and ``Islam'' produce toxic responses more often as well. In the other languages we measure, dialog-prompting methods appear to control toxic language harms more effectively.}

\subsection{\rev{Interview Guide}}
\label{interview}

\noindent \textbf{Themes to focus:}
\begin{enumerate}
\item General perceptions
 \item Identifying assumptions
    \item Handling/Using assumptions
\end{enumerate}

\smallskip
\noindent \textbf{General perceptions:}
\begin{enumerate}
    \item What is your role, and what do you work on specifically for AI?
    \item Throughout the lifecycle, what are the things that are given, taken for granted, unstated reasons?
    \begin{enumerate}
        \item Why do you think they are given?
        \item How do these "givens" influence how you do your work?
        \item Which of these are easy to identify and handle, and which are not?
    \end{enumerate}
    \item Which stage of the ML lifecycle do you focus more or less on while identifying and handling these “taken-for-granted”? Why?
    \item What is an assumption, according to you? Why? 
    \item In your work, what do you characterize as limitations?
    \begin{enumerate}
        \item Why are these limitations? Examples?
        \item Contextual: What assumptions do you make that inform these limitations?
        \item How do these limitations influence your work?
    \end{enumerate} 
    \item How do you limit or cap the number of assumptions you can identify and handle?
\end{enumerate}

\smallskip
\noindent \textbf{Identifying assumptions:}
\begin{enumerate}
    \item What tools or methods do you use to KNOW that you are assuming something?
    \item How separate is the assumption identification and handling process from your typical workflow?
    \begin{enumerate}
        \item How much effort did you put into finding and handling the assumptions?
        \item Have you used any Responsible AI-related artifacts (documentation, toolkits, etc.) to elicit or discuss assumptions? 
    \end{enumerate}
    \item How difficult is it to examine your or others’ assumptions and deal with them?
    \item How would you identify and discuss someone else’s assumption that is relevant to your work?
    \begin{enumerate}
        \item If they need to make that assumption, what action of theirs will convince you? How would you evaluate their response?
    \end{enumerate}
    \item Have you observed anything where assumptions are craftily embedded somehow? Can you give some examples?
    \item Have you ever wondered how you missed an assumption after you have already missed it?
\end{enumerate}

\smallskip
\noindent \textbf{Handling/Using assumptions:}
\begin{enumerate}
    \item After you identify an assumption, what do you do?
    \begin{enumerate}
        \item Do you inspect how it influences your argument or analysis? What do you do?
    \end{enumerate}
    \item How often do you explicitly use an assumption in your work? Can you give some examples? Why do you think they are necessary?
    \item When you explicitly formulate (come up with or use) an assumption, how do you check if you have installed the assumption accurately? Can you give any examples?
    \item When you need an assumption for your argument, how do you justify or evaluate it? Do you do this explicitly or implicitly?
    \begin{enumerate}
        \item If your assumption does not have a clear or universally acceptable justification, then what do you do?
    \end{enumerate}
    \item How do you explain your assumptions to someone who needs to understand your work? 
    \begin{enumerate}
        \item How do you imagine this process should go, and how does it actually go? 
    \end{enumerate}
    \item When you make a decision, do you think about the assumptions that accompany it? If so, how?
    \begin{enumerate}
        \item Have you changed your decisions after analyzing an assumption? What does the process look like?
    \end{enumerate}
    \item Do you make any distinctions between the assumptions? If so, what are they, and why do you do that? If not, why do you treat all assumptions homogenously?
\end{enumerate}

\smallskip
\noindent \textbf{Case studies:}
\begin{enumerate}
    \item Present the case and explain the situation
    \item What do they think about the text? (open-ended)
    \item How will they articulate their assumptions? (reflects how they identify and use)
    \item How should others have articulated the identified assumptions?
    \item What more is required to better articulate? List form vs. any other forms? Will detailed annotation help?
\end{enumerate}
