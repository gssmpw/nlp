\section{Methodology}

\smallskip
\noindent \textbf{Participant }\rev{\textbf{Recruitment and Demographics.}} Once receiving approval from our institutions' ethics boards, we posted an open call for participants in several AI-oriented online communities \rev{on Slack and LinkedIn}. The call invited practitioners involved in some capacity with the research, development, design, or implementation of ML to participate in in-depth qualitative interviews on how they conceptualize, identify, and handle assumptions within their work. 52 individuals responded to our call, out of which we recruited 22 respondents for remote semi-structured interviews through purposive sampling \cite{sharma2017pros}. While this may not yield a statistically representative sample, it still allowed us to explore rich and unique insights into the experiences of the participants we felt most capable of answering the research questions in our study \cite{sharma2017pros,roy2015sampling}. Those who demonstrated significant experience working on ML projects, either as developers, data scientists, or product managers, as well as individuals closely involved with responsible ML artifacts, were ultimately chosen to participate in our interviews.
\rev{Most of our participants were from Global North locations and identified as males. Table \ref{tab:demographics} provides more details about our participants.}

\begin{table*}[]
\centering
\begin{tabular}{|l|l|}
\hline
\textbf{Dimension} & \textbf{Distribution} \\ \hline
Gender & Male: 16, Female: 6 \\ \hline
Region & Global North: 18, Global South: 4 \\ \hline
Role & ML/Data Engineer: 7, ML/Data Scientist: 6,  Management: 5, \\ 
 & Others (Designer/Ethicist/Academic): 4, \\ \hline
Organization Type & Tech company: 10, NGO/Civil Society: 5, Consulting: 4, \\
& Academia: 2, Government: 1 \\ \hline
\end{tabular}
\caption{\rev{Participant Demographics}}
\label{tab:demographics}
\end{table*}

\smallskip
\noindent \textbf{Interview }\rev{\textbf{Design.}} \citet{brookfield1992uncovering} emphasizes that the key to uncovering assumptions lies in analyzing the lived experience of the assumer in order to embed a specific practice within a realistic context. This motivated how we framed our questions to be reflective, allowing participants to answer in a way that stepped outside their typical frames of reference and assess their assumptions by explicitly thinking about them. The questions were also designed to explore participants' experiences without presuming outcomes while allowing participants to refute our own underlying assumptions \cite{kvale2009interviews}. The downside of this direct approach is that unconscious assumptions---the ones that inform a participant's intuition without them being privy to their existence and persistence---may fall through. 
To remedy this, we offered a second part of the interview in which we extracted specific phrases from model documentation of three popular large language models--- PaLM 2 \cite{anil2023palm}, BLOOM \cite{le2023bloom}, and Llama 2 \cite{touvron2023llama}---and asked participants to vocally analyze them. We chose these models as they varied across different dimensions of openness \cite{liesenfeld2024rethinking}. The sample texts were selected because they most directly offered an argument that follows a typical premise-conclusion structure with deliberate non-technical language that may prompt confusion at first glance. The samples are provided in appendix \ref{appendix}.

Our approach in framing the lifecycle of an assumption by inquiring about how it is conceptualized\footnote{\rev{Some readers may wonder why we focus only on conceptualization and not consider the \textit{operationalization} of an assumption. However, in our experience and based on our interviews with practitioners, ML stakeholders do not operationalize the construct ``assumption'' in practice but operationalize only the content of a specific assumption (e.g., the usage of ``representative'' in the assumption ``this data is representative''). In this view, assumptions function at a meta-level as discussed in prior works in Critical Thinking and Informal Logic (section \ref{rel:core}), and so we focus only on the conceptualization of assumptions in this work to uncover the confusion associated with the practical use of this term in ML. We leave alternative explorations to future work.}}, then identified, then handled aligns with \citet{berman2001opening}'s breakdown of an assumption as a single entity composed of assuming, feeling, thinking, and behaving. By organizing questions through assessing \textit{functions} of assumptions rather than conveying them holistically, we are able to easily distinguish what specific elements contribute to confusion around assumptions, and how participants react to that confusion. Furthermore, following the logic that initial assumptions are likely to predicate how future assumptions are handled, we attempted to frame questions in a way that allowed us to form a narrative of a participant's assumptions.

Questions are also informed by our personal experience in ML ecosystems, aligning with established practices in \textit{reflexive} qualitative research \cite{berger2015now}. The idea of assumptions being present in technical ecosystems and the motivation for the study in assessing their influence is driven by our own observations working within the space and examining it from a critical lens derived from our past and current positions as responsible ML researchers. We make this position explicit to enhance the rigor, credibility, and trustworthiness of the study and allow readers to understand the lens through which we interpreted responses.

\smallskip
\noindent \rev{\textbf{Interview Procedure.}}
\rev{The interview guide was developed by the first and second authors and was thoroughly discussed and approved by all authors. We share our complete interview guide in Appendix \ref{interview}.
We sent our consent letters ahead of the interviews and gave our participants the option of either returning the signed letter or providing verbal consent during the interview.
All our interviews were conducted in English via Zoom. While the first and second authors conducted 9 interviews together, the first author conducted 12 interviews independently, and the second author conducted 1 independently. 
We recorded our calls upon consent and manually took notes of participants who were uncomfortable with recording. 
Our participants were given the option to exit the interviews whenever they needed. 
Our interviews lasted for 60 minutes on average. We compensated participants with 30\$ for their time and contribution.}    

\smallskip
\noindent \rev{\textbf{Data }}\textbf{Analysis.} Our virtual interviews yielded approximately 25 hours of recorded audio, paired with auto-generated transcripts from Zoom. \rev{The interview data and notes were stored in the first author's institutional cloud storage.} 
\rev{As described in our interview design, our questions were broadly framed to extract how assumptions are perceived, identified, handled, and used in practice.}
Given the nature of our more open questions, we employed interpretative and descriptive qualitative analysis \cite{merriam2019qualitative} to decipher \rev{insights} within the transcribed responses. 
\rev{The first and second authors conducted the bulk of the data analysis, and the final themes were discussed and finalized among all authors. The analysis began with multiple readings of the transcripts followed by open-coding on the transcribed data, independently and manually, by the first two authors. They then iteratively went through each other's codes manually, extracted and recorded commonalities, cross-checked with one another for reliability, and finalized the codes after resolving critical disagreements by open discussion.}

\rev{In the next phase, the codes were interpreted through the assumption argument lens (section \ref{rel:core}), mapped, categorized, and structured into themes and sub-themes over multiple iterations.} 
\rev{For instance, several sub-themes such as ``forgotton assumptions'' and ``recording style'' were grouped into one of the main themes, ``informal documentation.'' These sub-themes were created by grouping several codes that revolved around how practitioners noted down their and others' assumptions. 
Further, while some sub-themes, such as ``chained assumptions'' and ``granularity'' had overlapping codes, we categorized these sub-themes into distinct themes (elaborated in sections \ref{subsec:integrate} and \ref{subsec:doc} respectively) as it offers a better frame to understand the confusions around assumptions.
Overall, as key takeaways were found around how participants personally and professionally interacted with assumptions, we were able to form ontological distinctions, procedural inconsistencies, and other confusing elements that helped us craft clear constructions of an assumption, how workflows perpetuate unchecked assumptions, and what practitioners (can) do about it.  
Our findings in section \ref{sec:findings} reflects how we inferred and organized the main themes in our data.
}
% - Answers were categorized in themes - how participants defined and identified assumptions, what givens they had going into the ML process, how they handled assumptions, and how they responded to the case study.


% \smallskip
\noindent \textbf{Limitations.} A study about assumptions will naturally possess a few assumptions itself. First, the premise of the study requires a consensus between the authors and the participants that assumptions in an ML workflow have a significance that needs to be addressed, potentially influencing answers toward having a more proactive stance toward them. Second, the samples chosen in the case study portion of the interviews were pointers extracted from lengthier and more contextualized model documentation; our selection was informed by our own assumptions about what may elicit rich responses. The samples shown were also the same for all participants. While this provided an equal frame of reference, future works could reinforce our findings through comparing similar perceptions in more diverse samples.