%%
%% This is file `sample-sigconf-authordraft.tex',
%% generated with the docstrip utility.
%%
%% The original source files were:
%%
%% samples.dtx  (with options: `all,proceedings,bibtex,authordraft')
%% 
%% IMPORTANT NOTICE:
%% 
%% For the copyright see the source file.
%% 
%% Any modified versions of this file must be renamed
%% with new filenames distinct from sample-sigconf-authordraft.tex.
%% 
%% For distribution of the original source see the terms
%% for copying and modification in the file samples.dtx.
%% 
%% This generated file may be distributed as long as the
%% original source files, as listed above, are part of the
%% same distribution. (The sources need not necessarily be
%% in the same archive or directory.)
%%
%%
%% Commands for TeXCount
%TC:macro \cite [option:text,text]
%TC:macro \citep [option:text,text]
%TC:macro \citet [option:text,text]
%TC:envir table 0 1
%TC:envir table* 0 1
%TC:envir tabular [ignore] word
%TC:envir displaymath 0 word
%TC:envir math 0 word
%TC:envir comment 0 0
%%
%%
%% The first command in your LaTeX source must be the \documentclass
%% command.
%%
%% For submission and review of your manuscript please change the
%% command to \documentclass[manuscript, screen, review]{acmart}.
%%
%% When submitting camera ready or to TAPS, please change the command
%% to \documentclass[sigconf]{acmart} or whichever template is required
%% for your publication.
%%
%%
% \documentclass[manuscript,review,anonymous]{acmart}
 % \documentclass[sigconf]{acmart} % for CHI camera ready - also enable ack
\documentclass[manuscript]{acmart} % for arxiv


\usepackage{graphics}
\usepackage{soul}
\usepackage{multirow}

%%
%% \BibTeX command to typeset BibTeX logo in the docs
\AtBeginDocument{%
  \providecommand\BibTeX{{%
    Bib\TeX}}}

%% Rights management information.  This information is sent to you
%% when you complete the rights form.  These commands have SAMPLE
%% values in them; it is your responsibility as an author to replace
%% the commands and values with those provided to you when you
%% complete the rights form.
% \setcopyright{acmlicensed}
% \copyrightyear{2018}
% \acmYear{2018}
% \acmDOI{XXXXXXX.XXXXXXX}
\copyrightyear{2025} 
\acmYear{2025} 
\setcopyright{acmlicensed}\acmConference[CHI '25]{CHI Conference on Human Factors in Computing Systems}{April 26-May 1, 2025}{Yokohama, Japan}
\acmBooktitle{CHI Conference on Human Factors in Computing Systems (CHI '25), April 26-May 1, 2025, Yokohama, Japan}
\acmDOI{10.1145/3706598.3713958}
\acmISBN{979-8-4007-1394-1/25/04}


%% These commands are for a PROCEEDINGS abstract or paper.
% \acmConference[Conference acronym 'XX]{Make sure to enter the correct
%   conference title from your rights confirmation emai}{June 03--05,
%   2018}{Woodstock, NY}
% %%
% %%  Uncomment \acmBooktitle if the title of the proceedings is different
% %%  from ``Proceedings of ...''!
% %%
% %%\acmBooktitle{Woodstock '18: ACM Symposium on Neural Gaze Detection,
% %%  June 03--05, 2018, Woodstock, NY}
% \acmISBN{978-1-4503-XXXX-X/18/06}


%%
%% Submission ID.
%% Use this when submitting an article to a sponsored event. You'll
%% receive a unique submission ID from the organizers
%% of the event, and this ID should be used as the parameter to this command.
%%\acmSubmissionID{123-A56-BU3}

%%
%% For managing citations, it is recommended to use bibliography
%% files in BibTeX format.
%%
%% You can then either use BibTeX with the ACM-Reference-Format style,
%% or BibLaTeX with the acmnumeric or acmauthoryear sytles, that include
%% support for advanced citation of software artefact from the
%% biblatex-software package, also separately available on CTAN.
%%
%% Look at the sample-*-biblatex.tex files for templates showcasing
%% the biblatex styles.
%%

%%
%% The majority of ACM publications use numbered citations and
%% references.  The command \citestyle{authoryear} switches to the
%% "author year" style.
%%
%% If you are preparing content for an event
%% sponsored by ACM SIGGRAPH, you must use the "author year" style of
%% citations and references.
%% Uncommenting
%% the next command will enable that style.
%%\citestyle{acmauthoryear}

\AtBeginDocument{\colorlet{defaultcolor}{.}}

\newif{\ifhidecomments}
% \hidecommentsfalse
\hidecommentstrue
\ifhidecomments
    \newcommand{\rev}[1]{\textcolor{defaultcolor}{#1}}
\else
    \newcommand{\rev}[1]{\textcolor{purple}{#1}}
\fi

%%
%% end of the preamble, start of the body of the document source.
\begin{document}

%%
%% The "title" command has an optional parameter,
%% allowing the author to define a "short title" to be used in page headers.
\title{Talking About the Assumption in the Room}

\author{Ramaravind Kommiya Mothilal}
\email{ram.mothilal@mail.utoronto.ca}
\affiliation{%
  \institution{University of Toronto}
  \country{Canada}
}

\author{Faisal M. Lalani}
\email{faisalmlalani@gmail.com}
\affiliation{%
  \institution{University of Illinois Urbana-Champaign}
  \country{USA}
}

\author{Syed Ishtiaque Ahmed}
\email{ishtiaque@cs.toronto.edu}
\authornotemark[1]
\affiliation{%
  \institution{University of Toronto}
  \country{Canada}
}

\author{Shion Guha}
\email{shion.guha@utoronto.ca}
\authornote{Served as co-supervisors and contributed equally to this work.}
\affiliation{%
  \institution{University of Toronto}
  \country{Canada}
}

\author{Sharifa Sultana}
\email{sharifas@illinois.edu}
\affiliation{%
  \institution{University of Illinois Urbana-Champaign}
  \country{USA}
}
%%
%% The "author" command and its associated commands are used to define
%% the authors and their affiliations.
%% Of note is the shared affiliation of the first two authors, and the
%% "authornote" and "authornotemark" commands
%% used to denote shared contribution to the research.

%%
%% By default, the full list of authors will be used in the page
%% headers. Often, this list is too long, and will overlap
%% other information printed in the page headers. This command allows
%% the author to define a more concise list
%% of authors' names for this purpose.
\renewcommand{\shortauthors}{Kommiya Mothilal et al.}

%%
%% The abstract is a short summary of the work to be presented in the
%% article.

% The presence and controversy of assumptions in technical ecosystems is well-established in responsible AI discourse. What is seldom touched upon is the conceptualization around assumptions, and how AI practitioners identify and handle them throughout their workflows. We present a theoretical framework of how these practitioners generally navigate assumptions and offer recommendations on how to incorporate reflective practices that allow organizations to better utilize these assumptions in avoiding potential harms.

\begin{abstract}
The reference to \textit{assumptions} in how practitioners use or interact with machine learning (ML) systems is ubiquitous in HCI and responsible ML discourse. However, what remains unclear from prior works is the conceptualization of assumptions and how practitioners identify and handle assumptions throughout their workflows. This leads to confusion about what assumptions are and what needs to be done with them. We use the concept of an \textit{argument} from Informal Logic, a branch of Philosophy, to offer a new perspective to understand and explicate the confusions surrounding assumptions. Through semi-structured interviews with 22 ML practitioners, we find what contributes most to these confusions is how \textit{independently} assumptions are constructed, how \textit{reactively} and \textit{reflectively} they are handled, and how \textit{nebulously} they are recorded. Our study brings the peripheral discussion of assumptions in ML to the center and presents recommendations for practitioners to better think about and work with assumptions. 
\end{abstract}

%%
%% The code below is generated by the tool at http://dl.acm.org/ccs.cfm.
%% Please copy and paste the code instead of the example below.
%%
\begin{CCSXML}
<ccs2012>
   <concept>
       <concept_id>10003120.10003121.10011748</concept_id>
       <concept_desc>Human-centered computing~Empirical studies in HCI</concept_desc>
       <concept_significance>500</concept_significance>
       </concept>
 </ccs2012>
\end{CCSXML}

\ccsdesc[500]{Human-centered computing~Empirical studies in HCI}

%%
%% Keywords. The author(s) should pick words that accurately describe
%% the work being presented. Separate the keywords with commas.
\keywords{Assumption, Machine Learning, Responsible ML, Informal Logic, Critical Thinking, ML Practitioners}
%% A "teaser" image appears between the author and affiliation
%% information and the body of the document, and typically spans the
%% page.

% \received{20 February 2007}
% \received[revised]{12 March 2009}
% \received[accepted]{5 June 2009}

%%
%% This command processes the author and affiliation and title
%% information and builds the first part of the formatted document.
\maketitle

 
\section{Introduction}

\begin{figure}[h]
    \centering
    \begin{overpic}[trim=0cm 0cm 0cm 0cm,clip,angle=0,origin=c,width=.4\linewidth]{images/teaser_absolute.png}
        %  trim={<left> <lower> <right> <upper>}
        %  \put(horiz, vert)
        %  \put(horiz, vert){\rotatebox{90}{Text}}
        %
        \put(107, 32){$\mathbf{\to}$}
    \end{overpic}\hspace{1cm}
    \begin{overpic}[trim=0cm 0cm 0cm 0cm,clip,angle=0,origin=c,width=.4\linewidth]{images/teaser_translated_yellow.png}
        %  trim={<left> <lower> <right> <upper>}
        %  \put(horiz, vert)
        %  \put(horiz, vert){\rotatebox{90}{Text}}
        %
    \end{overpic}
    \caption{Using translation methods, a controller trained on an environment with a given visual variation \textit{(left)} can be reused without any training or fine-tuning on a different environment (\textit{right}) with comparable performance. In red we see the trajectory of a car driven by the same controller when connected to two different encoders, one for each visual variation.
    }
    \label{fig:teaser}
\end{figure}

Deep Reinforcement Learning (RL) has enabled agents to achieve remarkable performance in complex decision-making tasks, from robotic manipulation to high-dimensional games (Mnih et al., 2015; Silver et al., 2017). 
Although recent RL techniques achieved strong improvements over sample efficiency \citep{yarats2021drqv2, kostrikov2020image}, training new agents remains a costly process, both in computational and temporal terms.
Despite these advances, most methods still require at least partial retraining when dealing with domain shifts such as visual appearance, reward functions, or action spaces \citep{pmlr-v97-cobbe19a, zhang2020learning}. These domain changes typically require expensive retraining, which can be prohibitive for real-world settings that require millions of interactions.

A variety of approaches have been proposed to address these shifting conditions. Domain randomization \citep{tobin2017domain, sadeghi2016cad2rl} trains agents across diverse visual styles or physics settings, promoting invariant features but demanding broader coverage of possible variations. Multi-task RL \citep{parisotto2015actor, teh2017distral} attempts to learn shared representations across multiple tasks.

In the supervised setting, recent representation learning techniques \citep{Moschella2022-yf,maiorca2023latent, norelli2022b, cannistraci2023bricks}, show that it is possible to zero-shot recombine encoders and decoders to perform new tasks across different modalities (images, text..) and tasks (classification, reconstruction) and even architectures.
In RL, methods adopting the relative representation framework \citep{Moschella2022-yf} have shown promising results in adapting encoders to different controllers with zero or few-shots adaptation, for robotic control from proprioceptive states \citep{jian2021adversarial} or for playing games in the Gymnasium suite \citep{towers2024gymnasium} from pixels \citep{ricciardi2025r3lrelativerepresentationsreinforcement}.
These methods, however, still require training models to use the new relative representations.

By contrast, \cite{maiorca2023latent} suggest that modules from independently trained neural networks can be connected via a simple linear or affine transformation, with no training constraint or fine-tuning required, if such transformations can be reliably estimated from a small set of “anchor” samples, pairs of states or observations deemed semantically equivalent.

Our main contribution is the implementation of a RL method based on semantic alignment to map between latent spaces of different neural models, so that their encoders and controllers can be stitched with the goal of creating new agents that can act on visual-task combinations never seen together in training. This includes the use of the transformations to map modules from different networks, and the collection of anchor samples used to estimate these transformations. We call our method Semantic Alignment for Policy Stitching (\textbf{SAPS}).
We perform analyses and empirical tests on the CarRacing and LunarLander environments to show the performance of new agents created via zero-shot stitching of encoders and controllers trained on different visual-task variations, demonstrating significant gains compared to existing zero-shot methods.
\section{Related Literature}

Making simplifying assumptions to operate at a required abstract level is central to machine learning practice \cite{selbst2019fairness,saitta2013abstraction}. Abstraction inherently involves making assumptions about what is necessary and what is not. For instance, ignoring certain features or choosing a specific representation in data abstracts out certain social contexts and interactions that may be assumed as nonessential. As much as these traits have contributed to the rise of ML applications in diverse domains, the last decade has seen an increasing number of concerns arising from abstraction and assumptions made about human behaviors and characteristics in automated decision-making systems \cite{benjamin2019race,noble2018algorithms,o2017weapons,eubanks2018automating}. While assumptions form an essential constituent of many prior works on how practitioners use ML, in section \ref{rel:periphery}, we review how \rev{prior works in HCI and related disciplines} often place assumptions on the periphery. Then, in section \ref{rel:core}, we discuss how the concept of an \textit{argument} in Informal Logic can offer a new perspective to think about and act on the confusions surrounding assumptions in ML.

\subsection{Assumptions on the Periphery}
\label{rel:periphery}

% \rev{\textbf{Assumptions as Marginal Disruptor.}} 
Prior works in HCI, ML fairness, and AI ethics have extensively looked into how practitioners use ML systems and interact with different phases of an ML workflow (for e.g., \rev{\cite{zhangHowDataScience2020,yang2018investigating,wang2023designing,muller2020interrogating}}.) Many of these works have brief discussions about assumptions or at least mention the word ``assumption'' when analyzing practitioners' interactions with ML-based systems. However, most of these references assign a marginal causal agency to assumptions for disrupting a desired state or chain of actions: some common references to assumptions include phrases such as \textit{``The result is often erroneous assumptions [made by practitioners] about what users would want from AI.''} \rev{\cite[p.~12]{subramonyam2022human}}, \textit{``...they assumed that succinct answers were sufficient.''} (indicating undesired documentation practices) \rev{\cite[p.~17]{heger2022understanding}}, and \textit{``participants who assumed sex was a sensitive feature attempted to mitigate biases in the ML pipeline by simply removing...''} (explaining undesired actions sequence) \rev{\cite[p.~5]{dengExploringHowMachine2022}}.

\rev{In addition to model efficiency,} the desired state or actions discussed in these prior works often revolve around sociotechnical concerns related to fairness, transparency, or collaboration. For instance, some prior works refer to practitioners' assumptions as one of the factors hampering their collaborative efforts with stakeholders of different technical backgrounds \cite{yang2018investigating,varanasi2023currently,wang2023designing}; a few others discuss how assumptions distort practitioners' understanding of ethical or fairness issues \cite{boyd2021datasheets,dengExploringHowMachine2022,aragon2022human,jarrahi2022principles}. 
\rev{In all of these works, assumptions are often discussed peripherally to the main discussion about a desired state or action, such as efficiency or fairness. 
Specifically, along with other factors such as institutional constraints and incentives, assumptions are discussed as one of the factors that affect practitioners' attainment of a goal.}

% Though assumptions are only superficially discussed per se in studying practitioners' usage, 
% \smallskip
% \noindent \textbf{Assumptions as Object of Enumeration.}
\rev{As assumptions are often discussed in terms of their effect,}
their influence in practice is well-appreciated in responsible ML discourse \cite{mitchell2021algorithmic,jarrahi2022principles,aragon2022human,malik2020hierarchy}. 
Consequently, research in HCI and responsible ML has developed numerous toolkits \cite{wong2023seeing} (refer to frameworks, guidelines, etc.) to invoke assumptions in everyday practice \rev{\cite{Sadek2024,Lavin2022,Smith_undated,rismani2023plane}}. There are also a significant number of these toolkits---some very popular such as model cards \cite{mitchellModelCardsModel2019} or datasheets \cite{gebruDatasheetsDatasets2021}---that do not include direct interrogative questions to invoke assumptions, but rather frame questions on various phases of an ML workflow intended to unearth hidden assumptions \rev{\cite{Pushkarna2022,Kawakami2024,Raji2020,elsayed2023responsible}}. 
\rev{In a recent work, \citet{kommiya2024towards} recommend that practitioners deliberately connect, in addition to listing down, the known assumptions to the desired states.}   
However, most of \rev{these} toolkits \rev{and guidelines} are only suggestive, simply prompting the practitioners to list down and document assumptions in some form. While the authors of these toolkits argue that this process intrinsically will help mitigate the undesired consequences of making assumptions, it remains unclear how practitioners actually \textit{conceptualize} and \textit{work through assumptions} when prompted with ``what are the assumptions'' or ``list down the assumptions'' type questions. Though one of the primary intentions of these toolkits is to make practitioners elicit and reflect on the assumptions, the methodology adopted in the toolkits gives meager attention to \rev{the practical steps that practitioners undertake during} assumption articulation, identification, and handling. 

% -- also consequently a substantive number of works disuxss waht assumptions are missed and ignored by practitioner 
% -- consequently enumeration and recording 

Overall, prior works in HCI and responsible ML mostly make peripheral references to assumptions \rev{as objects of enumeration to avoid disrupting a desired state or action.
Several studies provide a review of common assumptions that practitioners ignore or miss during statistical modeling \cite{wang2022against,mitchell2021algorithmic,malik2020hierarchy}.
However, it remains unclear if assumptions in ML can be objectively and uniformly viewed and acted upon, especially when practitioners work with diverse stakeholders to build an ML system.
Hence, to supplement prior works, we examine this fundamental problem of confusion and uncertainties that practitioners might face when conceptualizing and working through assumptions.}
% Overall, prior works in HCI and ML mostly make peripheral references to assumptions in studying practitioners' use of ML systems and in the development of toolkits. 
To our knowledge, there is no work in an ML context that places assumptions in the center stage and investigates the surrounding confusions. In the next section, we introduce \rev{the concept of an \textit{argument}} from philosophical thinking to the HCI community to (re)think about assumptions in ML.

% Finally, we also find several prior works that make no or sparse reference to assumptions  
% With recent focus on improving the regulation on AI systems, governance and risk management framework such as NIST's RMF, 

% - overally it seems like assumptions are on the periphery and not getging enough attneiont

% - how in these tools - while some encourage reflectivess - but do not go deeper,
% - others just do not enocurage reflective practice or gives rise to further assumptions and causes confusion
    
% c. So this sidelining and unclarity in what do people do with assumptions encouraged us to ask our RQ.
% Our findings show that this practitoiners face confusion and unclarity concerning assumptions due to this current setup and infrastrucutre they are in and have

% - last para - many works do not mention assmtipns at all or use other terms but implitly they mean assumptions - so the result is confusion
%     - data workflow is a good example
    
% Prior work has discussed how ML and data practitioners draw on their beliefs and experiences to make subjective decisions at various stages of an ML lifecycle (cite).
% Our findings contribute to and reinterpret this line of work through the assumptions and the surrounding uncertainties and tensions that (a) prompts practitioners to make those subjective decisions, and (b) practitioners assess in others' interpretations and actions.


\subsection{Bringing Assumptions to the Center}
\label{rel:core}
The sub-field of \textit{Informal Logic} in Philosophy emerged as a response to the inefficiency of tools and criteria of formal \textit{Logic}\footnote{The prefix ``informal'' in Informal Logic is contentious and it is sometimes argued as a formal enterprise in a different sense \cite{woods2000philosophical,levi2000defense,johnson1999relation}.} in analyzing and evaluating natural language discursive arguments \cite{blair2012informal,walton2008informal,scriven1980philosophical,scriven1976reasoning}.
Informal Logic appreciates the structural complexity of everyday language use, the formulation of unstated assumptions, and the epistemological questions surrounding argumentation, among others, that analytic and normative tools of Logic ignores or over-simplifies, distorting the meaning of arguments \cite{anthony2015informal,johnson2014rise}. 
Arguments are also extensively studied in \textit{Critical Thinking}, often associated with Informal Logic \cite{weinstein1990towards,johnson2012informal,crews2007critical}, that studies a mode of thinking about an object involving active interpretation, clear articulation and analysis of reasons, assumptions and conclusions, logical evaluation of explanations and evidences, self-regulation, and holding a disposition to use the above-mentioned skills \rev{\cite{fisher1997critical,hitchcock2018critical,emis1962concept,facione1990critical}}.
% While critical thinking can also be about information, communication, and observations, a large overlap between Critical Thinking and Informal Logic lies in their focus on formulating, analyzing, and evaluating \textit{arguments} presented in some form (written, spoken, pictorial, etc. or multi-modal \cite{groarke2015going}.)

In this study, we refer to the conception of an \textit{argument} put forward by \citet{hitchcock2021concept} and review prior works to establish a connection between an assumptions and an argument \cite{kingsbury2002teaching,hitchcock2007informal,govier1992good,goddu2009refining}. By referring to the structuring of arguments as discussed in Informal Logic, we situate assumptions as core elements of arguments that ML practitioners make or engage in \textit{implicitly} or \textit{explicitly}.
We use this assumption-argument paradigm to offer a new perspective to understand and explicate the confusions surrounding assumptions in ML for two reasons. First, assumptions do not exist in a vacuum; they exist as part of arguments that are expressed or implied \rev{\cite{plumer2017presumptions,delin1994assumption,ennis1982identifying}}.
Therefore, critically analyzing the structure of an argument can explain \textit{how} and \textit{why} assumptions are made and what contributes to the confusion. Second, when analyzing an assumption and the surrounding confusion, we are essentially making arguments ourselves to critically think how assumptions shape a resulting argument \cite{berman2001opening,brookfield1992uncovering,delin1994assumption,ennis1982identifying}.

\rev{\citet[p.~105]{hitchcock2021concept}} formulates a simple argument as a premise-conclusion complex as follows:
\begin{quote}
    \textit{A simple argument consists of one or more of the types of expression that can function as reasons, a ``target'' (any type of expression), and an indicator of whether the reasons count for or against the target.}
\end{quote}
Reasons in the above definition refer to the \textit{premises} of an argument that perform a specific function: they commit the author of an argument to \textit{``something's being the case''} either assertively or hypothetically \rev{\cite[p.~10]{hitchcock2007informal,searle1976classification}}. 
% \cite{hitchcock2007informal,searle1976classification}.
In other words, premises constitute the propositions and the accompanying intention (or the illocutionary act \cite{searle1975taxonomy}, in philosophical terminology). 
For instance, when we use ``suppose the data is not representative'' as a premise for an argument, we express the proposition that the data is not representative \textit{hypothetically}.  The \textit{target} or conclusion is also a proposition but can be an illocutionary act type of different kinds, including a directive, a commissive, an expressive, and a declarative \cite{toulmin1958uses,ennis2006probably}. Finally, arguments can also be complex where the premise of one argument can be the target of another and so on \cite{hitchcock2021concept}.

When an ML practitioner makes or uses an assumption, it is often made or used \textit{for} a particular purpose.
For instance, when a feature is assumed to be unnecessary, it is often for realizing a particular objective such as to reduce the complexity of feature space. 
Similarly, when a performance metric is chosen, it is done so that optimized model outcomes are relevant for decision-subjects. The structure of an argument, as described above, then suggests that assumptions can be perceived as premises for attaining a target.
\citet{ennis1982identifying} discuss how these premise-type assumptions back-up or fill gaps for realizing the conclusion, and so the falsity of these premise-type assumptions weakens the support provided for the target. In other words, assumptions now become an essential component of an argument that a practitioner makes or implies. We recognize that there could be other forms of arguments, such as using vivid descriptions to display an identity or marching in protests \cite{jacobs2000rhetoric,hample2015arguing}, but we interpret the actions and expressions of ML practitioners as a premise-conclusion complex in this study and leave further exploration to future studies.
We also recognize the possibilities of interpreting different assumptions in an ML workflow as categories other than premises, such as conclusions and presuppositions, which might require a new lens to analyze \cite{walton2008argumentation,ennis1982identifying}.

Now, there can be situations where a practitioner may \textit{only} be making a premise-type assumption, but an analyst will be the one who is inferring the argument and making a distinction between premise and target. 
For instance, an ML developer can exclude certain text sources from the data and proceed with training their language model, but it is the safety expert in their organization who actually attempts to dissect the reasoning behind the data exclusion assumption\footnote{\rev{Prior works in HCI and responsible ML often do not make a distinction between first- and higher-order assumptions. In other words, assumptions made by the practitioners are not distinguished from those that are interpreted by the authors or someone else. Our point is not to doubt the inferential validity of these works but instead call attention to the complexity of assumptions, which may influence how they are examined and handled \cite{berman2001opening,atkin2017investigating,korzybski1958science}.}}. 
In other situations, a practitioner might need an assumption that they did not explicitly use, but which an analyst could infer. Or practitioners may not need an assumption but could have unintentionally used an assumption in making an argument. It is important to note that arguments are not necessarily localized to what practitioners do or write about in their documentations and reports, instead any (un)intentional action performed by a practitioner, such as choosing a specific algorithm, can be interpreted as an argument whose assumptions could be unearthed by an analyst \cite{kjeldsen2015study,groarke2015going}. Overall, this \textit{premise-target lens of an argument} deconstructs an assumption to understand why it was made (by identifying and relating it to the target) and how it was made (for instance, implicitly or explicitly), and thereby can support us in understanding how and why confusions exist around assumptions in ML practice.

% \textbf{relevance to hci}

% - what is assumtion - thinking
% - analyzing an assumption angle - critical and reflective thinking bit in papers

% - argument not only made but also had or engaged - this in itself has arguments made - implied argument - doc can be seen as that

% - now, conclusions can also be assumptions - but usually straightforward and leae to future work
% - explain why analyzing arguments help us uncover confusions about assumptions

% - taxomony of presmie tyle assumptions
% - how tools of critical thinking helps us as a method and as a theory


% In philosophical language, a premise is a illocutionary act of a proposition, either expressed assertively or hypothetically. 
% Consider two simple arguments, A1 and A2 below, inspired from prior philosophical works on this topic:
% \begin{itemize}
%     \item \textit{A1: Suppose that the data is not representative of the target population. Then it makes sense to add more data points.} 
%     \item \textit{A2: The data is not representative of the target population. So it makes sense to add more data points.}
% \end{itemize}
% Though the premises in both A1 and A2 different, they express the same proposition that the data is not representative.
% The difference lies in the illocutionary act performed by the 

\section{Methodology}

\smallskip
\noindent \textbf{Participant }\rev{\textbf{Recruitment and Demographics.}} Once receiving approval from our institutions' ethics boards, we posted an open call for participants in several AI-oriented online communities \rev{on Slack and LinkedIn}. The call invited practitioners involved in some capacity with the research, development, design, or implementation of ML to participate in in-depth qualitative interviews on how they conceptualize, identify, and handle assumptions within their work. 52 individuals responded to our call, out of which we recruited 22 respondents for remote semi-structured interviews through purposive sampling \cite{sharma2017pros}. While this may not yield a statistically representative sample, it still allowed us to explore rich and unique insights into the experiences of the participants we felt most capable of answering the research questions in our study \cite{sharma2017pros,roy2015sampling}. Those who demonstrated significant experience working on ML projects, either as developers, data scientists, or product managers, as well as individuals closely involved with responsible ML artifacts, were ultimately chosen to participate in our interviews.
\rev{Most of our participants were from Global North locations and identified as males. Table \ref{tab:demographics} provides more details about our participants.}

\begin{table*}[]
\centering
\begin{tabular}{|l|l|}
\hline
\textbf{Dimension} & \textbf{Distribution} \\ \hline
Gender & Male: 16, Female: 6 \\ \hline
Region & Global North: 18, Global South: 4 \\ \hline
Role & ML/Data Engineer: 7, ML/Data Scientist: 6,  Management: 5, \\ 
 & Others (Designer/Ethicist/Academic): 4, \\ \hline
Organization Type & Tech company: 10, NGO/Civil Society: 5, Consulting: 4, \\
& Academia: 2, Government: 1 \\ \hline
\end{tabular}
\caption{\rev{Participant Demographics}}
\label{tab:demographics}
\end{table*}

\smallskip
\noindent \textbf{Interview }\rev{\textbf{Design.}} \citet{brookfield1992uncovering} emphasizes that the key to uncovering assumptions lies in analyzing the lived experience of the assumer in order to embed a specific practice within a realistic context. This motivated how we framed our questions to be reflective, allowing participants to answer in a way that stepped outside their typical frames of reference and assess their assumptions by explicitly thinking about them. The questions were also designed to explore participants' experiences without presuming outcomes while allowing participants to refute our own underlying assumptions \cite{kvale2009interviews}. The downside of this direct approach is that unconscious assumptions---the ones that inform a participant's intuition without them being privy to their existence and persistence---may fall through. 
To remedy this, we offered a second part of the interview in which we extracted specific phrases from model documentation of three popular large language models--- PaLM 2 \cite{anil2023palm}, BLOOM \cite{le2023bloom}, and Llama 2 \cite{touvron2023llama}---and asked participants to vocally analyze them. We chose these models as they varied across different dimensions of openness \cite{liesenfeld2024rethinking}. The sample texts were selected because they most directly offered an argument that follows a typical premise-conclusion structure with deliberate non-technical language that may prompt confusion at first glance. The samples are provided in appendix \ref{appendix}.

Our approach in framing the lifecycle of an assumption by inquiring about how it is conceptualized\footnote{\rev{Some readers may wonder why we focus only on conceptualization and not consider the \textit{operationalization} of an assumption. However, in our experience and based on our interviews with practitioners, ML stakeholders do not operationalize the construct ``assumption'' in practice but operationalize only the content of a specific assumption (e.g., the usage of ``representative'' in the assumption ``this data is representative''). In this view, assumptions function at a meta-level as discussed in prior works in Critical Thinking and Informal Logic (section \ref{rel:core}), and so we focus only on the conceptualization of assumptions in this work to uncover the confusion associated with the practical use of this term in ML. We leave alternative explorations to future work.}}, then identified, then handled aligns with \citet{berman2001opening}'s breakdown of an assumption as a single entity composed of assuming, feeling, thinking, and behaving. By organizing questions through assessing \textit{functions} of assumptions rather than conveying them holistically, we are able to easily distinguish what specific elements contribute to confusion around assumptions, and how participants react to that confusion. Furthermore, following the logic that initial assumptions are likely to predicate how future assumptions are handled, we attempted to frame questions in a way that allowed us to form a narrative of a participant's assumptions.

Questions are also informed by our personal experience in ML ecosystems, aligning with established practices in \textit{reflexive} qualitative research \cite{berger2015now}. The idea of assumptions being present in technical ecosystems and the motivation for the study in assessing their influence is driven by our own observations working within the space and examining it from a critical lens derived from our past and current positions as responsible ML researchers. We make this position explicit to enhance the rigor, credibility, and trustworthiness of the study and allow readers to understand the lens through which we interpreted responses.

\smallskip
\noindent \rev{\textbf{Interview Procedure.}}
\rev{The interview guide was developed by the first and second authors and was thoroughly discussed and approved by all authors. We share our complete interview guide in Appendix \ref{interview}.
We sent our consent letters ahead of the interviews and gave our participants the option of either returning the signed letter or providing verbal consent during the interview.
All our interviews were conducted in English via Zoom. While the first and second authors conducted 9 interviews together, the first author conducted 12 interviews independently, and the second author conducted 1 independently. 
We recorded our calls upon consent and manually took notes of participants who were uncomfortable with recording. 
Our participants were given the option to exit the interviews whenever they needed. 
Our interviews lasted for 60 minutes on average. We compensated participants with 30\$ for their time and contribution.}    

\smallskip
\noindent \rev{\textbf{Data }}\textbf{Analysis.} Our virtual interviews yielded approximately 25 hours of recorded audio, paired with auto-generated transcripts from Zoom. \rev{The interview data and notes were stored in the first author's institutional cloud storage.} 
\rev{As described in our interview design, our questions were broadly framed to extract how assumptions are perceived, identified, handled, and used in practice.}
Given the nature of our more open questions, we employed interpretative and descriptive qualitative analysis \cite{merriam2019qualitative} to decipher \rev{insights} within the transcribed responses. 
\rev{The first and second authors conducted the bulk of the data analysis, and the final themes were discussed and finalized among all authors. The analysis began with multiple readings of the transcripts followed by open-coding on the transcribed data, independently and manually, by the first two authors. They then iteratively went through each other's codes manually, extracted and recorded commonalities, cross-checked with one another for reliability, and finalized the codes after resolving critical disagreements by open discussion.}

\rev{In the next phase, the codes were interpreted through the assumption argument lens (section \ref{rel:core}), mapped, categorized, and structured into themes and sub-themes over multiple iterations.} 
\rev{For instance, several sub-themes such as ``forgotton assumptions'' and ``recording style'' were grouped into one of the main themes, ``informal documentation.'' These sub-themes were created by grouping several codes that revolved around how practitioners noted down their and others' assumptions. 
Further, while some sub-themes, such as ``chained assumptions'' and ``granularity'' had overlapping codes, we categorized these sub-themes into distinct themes (elaborated in sections \ref{subsec:integrate} and \ref{subsec:doc} respectively) as it offers a better frame to understand the confusions around assumptions.
Overall, as key takeaways were found around how participants personally and professionally interacted with assumptions, we were able to form ontological distinctions, procedural inconsistencies, and other confusing elements that helped us craft clear constructions of an assumption, how workflows perpetuate unchecked assumptions, and what practitioners (can) do about it.  
Our findings in section \ref{sec:findings} reflects how we inferred and organized the main themes in our data.
}
% - Answers were categorized in themes - how participants defined and identified assumptions, what givens they had going into the ML process, how they handled assumptions, and how they responded to the case study.


% \smallskip
\noindent \textbf{Limitations.} A study about assumptions will naturally possess a few assumptions itself. First, the premise of the study requires a consensus between the authors and the participants that assumptions in an ML workflow have a significance that needs to be addressed, potentially influencing answers toward having a more proactive stance toward them. Second, the samples chosen in the case study portion of the interviews were pointers extracted from lengthier and more contextualized model documentation; our selection was informed by our own assumptions about what may elicit rich responses. The samples shown were also the same for all participants. While this provided an equal frame of reference, future works could reinforce our findings through comparing similar perceptions in more diverse samples. 
\section{Findings}
\label{sec:findings}
We find that the confusions concerning assumptions in ML largely revolve around what assumptions are (section \ref{sec:ontology}) and what is being done about them (section \ref{sec:procedure}). While many confusions are due to vague conceptualizations of assumptions, institutional fragmentation, and a general lack of clarity on response, others stem from holding unique and inconsistent views and procedures that deal with assumptions. Further, practitioners differ in their characterization of assumptions based on the role they take: whether they are the ones assuming (\textit{assumer}) or analyzing (\textit{analyst}) something. \rev{Table 1 summarizes our findings.}

\begin{table*}[]
\centering
\resizebox{\textwidth}{!}{%
\begin{tabular}{|l||l|l|l|}
\hline
\textbf{Axis of Confusion} & \textbf{Theme} & \textbf{Key Takeaways} & \textbf{Premise-Target Lens} \\ \hline
\multirow{2}{*}{\begin{tabular}[c]{@{}l@{}} Conceptualization of \\ what assumptions are\end{tabular}} & \begin{tabular}[c]{@{}l@{}}Independent\\ construction\end{tabular} & \begin{tabular}[c]{@{}l@{}}- viewed as being outside of the ML workflow \\ \\ - solidified as axioms or hailed as requirements \\   or relegated as limitations\end{tabular} & \begin{tabular}[c]{@{}l@{}}- target of the assumption \\   often remains unstated\\ \\ - premise is seldom evaluated\end{tabular} \\ \cline{2-4}  
 & \begin{tabular}[c]{@{}l@{}}Relative\\ construction\end{tabular} & \begin{tabular}[c]{@{}l@{}} - defined in relation to data quality, model \\   specifications, or business objectives\\ \\ - rationalized in relation to ML workflow and \\   prevents deeper and more inclusive assessment\end{tabular} & \begin{tabular}[c]{@{}l@{}}- premise, target, and argument\\   are explicit and clear\\ \\ - identification and evaluation\\   of premises are easier\end{tabular} \\ \hline \hline
\multirow{2}{*}{\begin{tabular}[c]{@{}l@{}}Uncertainty about \\ what is being done \\ with assumptions\end{tabular}} & \begin{tabular}[c]{@{}l@{}}Integration with\\ existing workflow\end{tabular} & \begin{tabular}[c]{@{}l@{}}  \textbf{Reactive handling:}\\ - reactive and iterative approach to ML extends\\   to assumptions identification and handling\\ \\ - assumptions constructed relative to ML \\   workflows are often reactively handled\\ \\ \textbf{Unreflective quantification:}\\ - the incentive to quantify assumptions \\   evaluation obstructs reflective practice \\ \\ \textbf{Circle of ambiguity:}\\ - no mechanisms to capture evidence of \\   assumptions, creating more uncertainties\\ \\ - knowledge and communication gap in ML\\   between stakeholders creates circling ambiguity\end{tabular} & \begin{tabular}[c]{@{}l@{}}- formulation of argument's\\   target depends on how surprising \\ 
 assumption's  consequences are\\ \\ \\ - ambiguities in premises and \\   implied arguments are disguised\\ \\ \\ - target of one assumption forms\\   the premise of other targets\end{tabular} \\ \cline{2-4} 
 & \begin{tabular}[c]{@{}l@{}}Unstructured\\ documentation\end{tabular} & \begin{tabular}[c]{@{}l@{}}  \textbf{Informal and implicit recording:}\\ - distinction between formulation and installation\\   of assumptions is not clearly documented\\ \\ - site, content, and style of assumption recording\\   is strongly associated with role, resulting in conflict\\ \\ \textbf{Granularity of recording:}\\ - insistence to understand the rationale behind\\   assumptions lead to further assumptions based\\   on lived experience\\ \\ - no structured prompt to record the required level\\   of details in assumptions recording\end{tabular} & \begin{tabular}[c]{@{}l@{}}- lack of distinction between \\   formulation and installation \\   of premises in documentation\\   \\ \\ \\ \\ - premises are often recorded as \\   declarative statements with no\\   relation to target of the argument\end{tabular} \\ \hline
\end{tabular}
}
\caption{\rev{A summary of key themes, takeaways, and the premise-target theoretical lens we adopt to deconstruct and understand the confusing factors about assumptions in ML.}}
\label{tab:findings}
\end{table*}


\subsection {Ontological Differences}
\label{sec:ontology}

The conceptualization of an ``assumption'' varied greatly between participants. While some gravitated toward an understanding that coincided with their preexisting technical workflows, others perceived it as an isolated consideration, existing outside the normative bounds of development. These inconsistencies may be rooted in a systemic de-emphasis of abstract thinking, incentivizing participants to view socio-technical concepts as a static, external object rather than an embedded mentality \cite{selbst2019fairness,wang2022towards,malik2020hierarchy,fazelpour2020algorithmic}. When prompted explicitly to define ``assumptions'', many participants described it as preliminary, something to be ironed out, built upon, or invalidated. These initial assumptions also predicated on how future assumptions were handled. The elaboration of how this characterization plays out throughout the development process differed, with some participants conceiving of an assumption independent of internal components (section \ref{subsec:ind}) and others associating it directly or indirectly with other entities (section \ref{subsec:rel}). Both these constructions create uncertainties and confusion in their own ways, both in how they shape institutional handling of downstream tasks and potential harms. 
% Redundant:
% 
% Ultimately, it was the background and experience of the participant that dictated how they distinguished assumptions. 

\subsubsection{Independent Construction}
\label{subsec:ind}
\citet{delin1994assumption} describe how average discourse around an assumption implies that it is a sort of abstract entity existing in one's mind, and to interact with one is akin to finding, identifying, and examining a ``thing.'' This interpretation best characterizes what we label an \textit{independent} construction of an assumption. This type of assumption exists as an external \textit{other} to the primary subject---the ML workflow. The \textit{purity} of this workflow is a common theme among participants with a technical background. The idea of the model and the deference to ML work is a foundational mentality that conceptualizations of risk \cite{saxenaRethinkingRiskAlgorithmic2023,zanotti2024ai}, bias \cite{andrusWhatWeCan2021,kernAssumptionsBiasData}, and, in this context, assumptions must work around. 

This is best seen through how technologists frame assumptions as independent ideas that exist to serve the technology. For instance, for many participants, we introduced the concept of assumptions by inquiring the participant about \textit{givens}, or what information or knowledge the participant takes for granted. These givens are necessary \textit{premises} to begin a project as they unquestionably validate the primary heuristics of the project. In other words, the \textit{target} of independently constructed assumptions remains unstated or, in the best case, unscrutinized if it is explicitly mentioned. This is often because these assumptions lay the foundation of a project, and they are often immutable and the rest of the work must be accommodated. This immutability is justified as the organic nature of an AI model, with assumptions being reflective of its surrounding context rather than the technology itself. P1, an ML developer, explains the unspoken nature of these givens:

% Redundant:
% The emphasis of the technology as the central component of the product lifecycle further contributes to this mindset.
% P1, a technical product manager, elaborates on this:
% \begin{quote}
% \textit{"How I describe an assumption is something that’s taken for granted. Something where I don’t have to think about it. And it’s a fact or a basis for the work that I’m about to do. I don’t really need to question, hey, what’s happening? Is this what I need to do? It just happens, or it’s there, and there’s no question about the validity of it."} 
% \end{quote}
\begin{quote}
\textit{``I guess the most basic assumption is that the human behavior can be modeled with numbers...Because if you know these things are unquantifiable, then there’s no work...It's a set of axioms, I guess, from which I can draw conclusions. And if these axioms are violated, then, you know there’s no guarantee how the system will turn out.''} 
\end{quote}

% These givens are also provided through a proxy that is perceived trustworthy and thus rarely questioned. These may be institutionalized teams or roles that convey the givens directly or the background and experience of the technologist. 

Recent research suggests that the assumptions and choices of technical practitioners, in particular, are often found to be more subtle \cite{kery2019towards,kommiya2024towards,wang2019data}. These types of assumptions reflect a desired \textit{implicitness} to certain thinking that allows the technology to be developed in the first place. The assumption then possesses an \textit{interpretive flexibility} \cite{meyer2006three,star1989structure,leigh2010not}, being swept under the rug but still requiring further assumptions to validate it and allow for it to remain unspoken. This process may entail the transformation of an assumption into a \textit{requirement} or a \textit{limitation}. The former is managed through the validity of the assumption by an authority, which is either the assumer themselves and their expertise or a separate role or team that explicitly assigns it credence. They become \textit{informed} assumptions, legally justified and deliberated upon by personal decision-making. The latter may be designated as such through real-world constraints that prevent deeper internalization. Assumptions that are relegated to limitations may also be the byproduct of a ``perfection is the enemy of good'' culture \cite{sylvester2018applied,green2019good}. The assumptions that are not solidified as axioms, or hailed as requirements, or relegated as limitations, are then needed to be empirically validated and conceptually clarified in relation to the target towards which the assumptions are directed.
% in order to assess risk.

% Independent assumptions are treated as \textit{boundary objects} \cite{star1989structure,leigh2010not}, where the back-and-forth between ill-structured and well-structured forms are to be expected. 

% \begin{quote}
% \textit{I’m sure you know, the model would not get things correctly, like 100\% of the time. But you know, we don’t aim for perfection. We aim for some number like 99.9\% of the time. Maybe that’s good enough.} 
% \end{quote} (P2)
% \textit{…it feels like for a lot of product teams, they get hindered because of the years of risk. So they just really want to move forward so that they can actually deliver something.} 
% (P3)

\subsubsection{Relative Construction}
\label{subsec:rel}
Other participants had a more \textit{embedded} perspective on assumptions. While independent construction allowed the assumption to exist as its own object flowing through and being manipulated by the ML workflow, \textit{relative construction} implies that assumptions exist \textit{in relation} to existing phases of the development process. Returning to \citet{delin1994assumption}'s characterizations, this type of assumption involves examining the ``mental-event-or-state'' of an individual or institution instead of perceiving the assumption as an independent entity. In other words, relative assumptions exist as a byproduct of the practitioner's \textit{mentality} in a workflow rather than an external factor.

In particular, participants embodying this perspective defined assumptions through technical framing, citing decisions around data quality or model specifications. The assumptions must live within the confines of a technically-driven approach and be subject to dissection through that lens. This integration of assumptions into a technical dimension can help practitioners investigate deeper issues within their work, despite the purview being narrower. Since the target of the assumptions is often explicit in this case, identifying the argument is relatively easier than in independent construction. This has both upsides and downsides: it allows the practitioner to navigate the assumption through a familiar paradigm, but it also presents assumptions as an inevitability through the sheer breadth of available information. This could allow assumptions to go unchecked, but in a justified resignation, as demonstrated by P2, an ML scientist:

\begin{quote}
\textit{``...we kind of don't have the bandwidth to check each and everything. So that's maybe one assumption we make. We're also kind of assuming that...all of this leads to data correctness...but we end up making some assumptions like, for example, data is about accuracy. If data is empty, then you should just treat it as empty and not treat it, as you know, something meaningful...They are assumptions, and they will sort of, you know, go into the model and be baked into the process.''}
\end{quote}

% Relative assumptions, like independent assumptions, originate outside the realm of the ML workflow, but are \textit{actualized} through technical framing. 

Relative assumptions too possess an inherent implicitness that may unintentionally inform understanding of the model. But if an independent construction allows for an assumption to persist as its own object with the possibility of transforming into something beneficial to the technical process, then a relative construction attempts to integrate an assumption directly \textit{into} the process. The formulation of problems is an example, as they usually become intertwined with relative assumptions: P2 described how associations with specific demographics, for instance, may be less scrutinized due to the expectations of the technical team. In other words, if the data is labeled or categorized in a certain way that conforms to the lived experience of the practitioner, it may prevent deeper assessment. And because relative assumptions are tied directly to workflow components and workflow components are necessary, there is an incentive for the assumption to be rationalized.

% \textit{If you know, your assumptions start becoming givens, and you know ideally at the end of your project you should have only knowns and no assumptions, nothing left to assumption.} (P5)
% \\
% \textit{….because at the end of the day, bias in your model means bias in data and bias in data is just like reflecting what exists in the real world…So there those are certain assumptions in terms of like, 'oh, this bias already exists and you’re conforming to it in a certain sense…'} (P6)

Relative assumptions need not necessarily be attached to \textit{technical} processes only. A few practitioners in management roles described how assumptions can linger throughout business objectives and outcomes; what is deemed critical to operationalizing the company vision is often an assumption in and of itself. These assumptions are often second or third-order assumptions \cite{berman2001opening,korzybski1958science}, meaning that the assumer is multiple degrees removed from the original observation that incited the assumption\footnote{\rev{To understand the confusions around assumptions, for the sake of clarity, we almost exclusively treat an assumption independently of other assumptions. We leave explorations of higher order and connected assumptions to future work and briefly point to related works at the end of section \ref{disc:articulate}.}}. More so, the authority attached to the central teams that assert these claims makes it easier to internalize relative assumptions because the task of proving them right is often the primary function of the business. 


% \begin{itemize}
%     \item 			i. Data collection quality - 30-35 age not checked - if model performs badly for this distribution, we won’t know
% \item			ii. Data correctness 
% 	\item		iii. Lack of data - loss to follow up before death - how to impute missing data
% 	\item		iv. Data quality is often equated to data accuracy
% \item First data quality is investigated because that is often taken for granted
% \end{itemize}
% -  Investigate assumptions if personally relevant - else take altruistic stance

% <1 para>
% - Another conception of assumptions has an implicit relation with limitations
% - though when practitioners talk about limitations, assumptions are inherently present, they are unable to describe it explicitly.
% - Unable to describe assumptions as limitations - the latter is always in terms of outcomes - except a few - Assumptions are ignorant decision or requirements - can be seen as limitations
% - Though assumptions inform various actions/decisions, focus is on latter and not on former

% <unused for now>
% - Assumptions in operationalizing business vision - economic impact vs revenue example - 40\% assumptions comes from objective or mission statements - intentionally used - anticipate


\subsection {Procedural Uncertainties}
\label{sec:procedure}

Other than ontological differences, practitioners also face several challenges in determining what to do with the assumptions. In this section, we lay out the uncertainties that accompany various \textit{procedures} practitioners employ to identify and handle assumptions. We find that these uncertainties either correspond to integration methods with existing workflows (section \ref{subsec:integrate}) or the documentation practices for collaboration and accountability (section \ref{subsec:doc}). 

% Finally, we discuss the strategies practitioners employ, and the challenges they face to bring some structure to the articulation of assumptions for alleviating some of the surrounding confusion (section \ref{sec:articulate}.) 

\subsubsection{Integration with existing workflow}
\label{subsec:integrate}
\citet{brookfield1992uncovering} argues that the investigation of assumptions must be a deliberate and reflective process to assess and validate various decisions that go into a process. However, most practitioners we spoke to shared that their typical ML lifecycles in practice seldom offer them avenues to reflectively identify and handle assumptions.

\smallskip
\noindent \textbf{Reactive Handling.} Machine learning in practice is usually an iterative and outcome-oriented pursuit \cite{ashmore2021assuring,malik2020hierarchy}: almost all our participants start their ML lifecycles with lesser information than they ideally need, but then iterate and learn about assumptions as they go. Because there is no appropriate structure for them to inquire about assumptions, this iterative process often puts practitioners in ambiguous situations where there are no clear and accepted procedures to follow when one is found. This, in turn, demands practitioners to adopt a \textit{reactive} approach to identify and handle assumptions. This is most prominent in relative assumptions, in which the assumption is directly embedded within the ML workflow. For instance, some practitioners, such as P4, raise a red flag and look for assumptions in data when presented with a ``neat classification problem'' in which all data is categorized too perfectly. For a few others, assumptions are investigated only during exploratory data analysis (EDA for short) when extreme and skewed patterns about data are observed. \rev{P3, a technical lead, provides an example of this mentality by demonstrating how their team took certain factors for granted until a ``surprising'' result was found:}
\begin{quote}
    \textit{\rev{``So in any modeling process, we do some basic EDA to understand the quality of the data. But some things we take for granted. We then build a model. And sometimes, surprisingly, results will be like, very good than what you anticipate. Then we start looking into the top features. And then we think about, okay, is there any data leakage? This is very hard to understand when you're doing initial EDAs, right? But once we build the model, once you're seeing surprising results, then we go back to business. Then we discuss its implications...is there anything wrong with that feature?''}}
\end{quote}

Because relative assumptions can be associated with both technical and business processes, what is deemed surprising by the latter may inform the reaction of the former. This is demonstrated by P5, a lead data scientist, who articulated how the inquiry of assumptions often takes place only when the outcomes have unanticipated business implications in the typical ML workflow.
While the above-discussed ``innocent until proven guilty mindset'' (P6) is prevalent among ML teams, some practitioners in management-related roles highlighted that product teams generally use existing collaboration infrastructure, such as GitLab\footnote{https://about.gitlab.com/}, Asana\footnote{https://asana.com/}, or other business management software to inadvertently integrate assumptions discussions into the typical workflow. Though not a typical practice of ML teams, a couple of practitioners also shared that dedicated ``assumptions trackers'' are sometimes used to encourage documentation and discussion of assumptions at work. 

However, P6 shared that the use of these tools is largely descriptive and only results in noting down the simplistic description of assumptions, subjective association with the argument's target, and mapped ownership of assumptions redressal. 
Overall, assumptions are communicated through these tools, but it is unclear if they are reasoned and rationalized: questions such as why the assumptions are made, why they are necessary, what kind of assumptions they are, what their implications for ML lifecycle are, etc. are sparsely addressed by these tools. We expand upon this tendency to examine assumptions only at a surface-level in the next section.
% quote : i have a decade long experience in client facing roles - with great confidence i can say that this is not the case

% - though assumptions tracker etc. could be helpful, it often requires a historian type method - As historian, methodology revolves around recognition of “contingency” - reflective endorsement
% -- case study analysis


% Finally, integration into existing workflow also is contingent on the cost of 
% <1 para> - hold this for now
% With recent increase in RML attention, there has been some efforts to invoke assumptions - however
% 1. RML docs are not universal - best used for system documentations
% 3. RML toolkits are not used in many organizations 
%     a. Assumptions work comes at a cost - RML doesn’t help - Mediterranean addition example
% 4. RML just borrows inputs from various laws - we need dedicated legal frameworks.

\smallskip
\noindent \textbf{Unreflective Practice through Quantification.} A crucial constituent of assumptions identification and assessment is the \textit{reflectiveness}, if not \textit{critical reflectiveness}, of the process \cite{paul1993critical,paul1993workshop}. 
However, \textit{the seductions of quantification} \cite{merry2009seductions} often derail practitioners from performing reflective exercises on assumptions and instead reroute them to aspire for a fictitious objective and unequivocal state. \rev{P10, an engineering consultant, elaborated on her perceived futility in quantifying assumptions:}

\begin{quote}
   \rev{ \textit{``...we do have some in-house metrics…we also really want to know which task is being performed the best without any sort of assumptions or biases so we have inbuilt like a comparator for ourselves, based on particular metrics… I believe these metrics can be very much subjective, based on the use case a particular company or a particular product has...there is no way we can quantify what assumptions a particular model is taking…''}
}
\end{quote}

\textit{Independent} assumptions are most prone to this instinct---they must adhere to the workflow. Practitioners noted that the organizational constraints and standard ML workflow practices do not incentivize many technical counterparts to perform reflective exercises but only make them care about data coverage and infrastructure-related concerns. But this interaction between assumption and workflow is bi-directional. Both in their own practice and during our case analyses, practitioners suggested ways of using various computational methods \cite{yanga2024exploratory,zhang2018empirical} to debug and act on their assumptions. However, as an extreme example, one practitioner supported the use of software plugin-type tools to automatically check all assumptions and provide a score to quantify assumptions evaluation. 

In our case analyses with practitioners, we find that many of them spend more time understanding and justifying various computational steps in LLM reports but focus less on reflecting on the premises and conclusions that inform the various steps they are analyzing. We observe that practitioners with some exposure to responsible ML principles often recognize that over-reliance on computational methods just disguises the underlying ambiguities, not removes them \cite{green2018myth,fazelpour2020algorithmic}. This means that the organizational instinct and incentive to adhere to the ML workflow implies a bias to a relative construction of assumptions. They also shared that such disguising has the danger of pushing practitioners to ignore the assumptions and the implied argument, which often peek through at later phases of ML lifecycles in various forms. \rev{P7, a designer working with state organizations, shared how a risk assessment tool unnecessarily quantified risk evaluation:}

\begin{quote}
    \textit{\rev{``so these risk assessment questionnaires are designed such that you're kind of casting a wide net like, yeah, you're probably at risk of something. After they do the risk assessment, the general next step is, your team has to come up with a control plan in detail explaining how you're going to mitigate those risks. So you know, obviously, some of the feedback that we got was like, okay, so we could come up with the most horrific AI system that has all these risks to end users. And it's all okay, because we can just write a control plan, saying that we'll mitigate right? So there was a lot of like contention with that approach.''}}
\end{quote}

While the quantification of assumptions is intended to reduce uncertainty around decision-making, further assumptions emerge when model performance is negotiated with other sociotechnical concerns, such as safety and fairness. For many practitioners, this leads to a nuanced tension they face when, for instance, deciding on a lower threshold for the maximum number of security violations. For example, while a 90\% violation is clearly red-flagged, negotiation between 0.5\% and 1.0\% creates debates and confusion. We expand on the cyclical nature of assumptions in the next section.



% benchmarks curb reflectivity?
% <-- finally benchamarks are blindly used (cite gorilla etc.) - see quote about assumptions>

% - it was not a one side affair - some practiotners also sggested ways of reflectively using compuational methods - some also did it in our case studies
% -- while some computational techniques aid assumptiosn inquirt reflective - CF and knowledge base
% - however, the infrastructure (connect to worlflow + documnet next)

% -- embarassement score    

% When participants were asked to identify assumptions more indirectly through analyzing model documentations at face-value, some associated anything distinct from technical terminology as an assumption.

\smallskip
\noindent \textbf{The Circle of Ambiguity.}
While assumptions are often invoked to mitigate the ambiguity around unresolved questions, ironically, due to their very implicit nature, they instead raise several questions and concerns they are trying to ameliorate in the first place. Consider the use of automated risk-assessment tools used in many organizations in recent years: practitioners find that many responses to assessment software require extrapolation and reflection. However, the checklist-type design and quantification of risk evaluation in these tools obscure the \textit{consequences} of underlying assumptions. A few participants shared that several sub-modules of these tools in their organization require practitioners to invoke assumptions in their own responses, but there are typically no mechanisms to capture the evidence of these assumptions, creating more uncertainties. Therefore, the trajectory of an assumption in an ML workflow is one that may beget more assumptions. This ambiguity is discussed in \textit{Critical Thinking} as complex or \textit{chain of reasoning} arguments, where the assumption that supports a target may be the target of one or more assumptions, and so on indefinitely \cite{hitchcock2021concept}. 

% Looking through the analytical lens our practitioners took in the case studies, our findings highlight nuanced uncertainties that motivate technical practitioners to make subtle assumptions.  

Consider P4, an ML scientist, who expresses how the confusion around calibration scores \cite{pleiss2017fairness} among different non-technical stakeholders motivated their subjective decision on how to present the model outcomes:

\begin{quote}
    \textit{``...for multiple projects that I was on...I saw that the easiest thing to do was to provide high, medium, low kind of bucketing of confidences rather than to provide a confidence score. The assumption that kind of affords a user is that both significant digits are significant, that there is a difference between, you know, 0.95 and 0.97 or something like that. And that really wasn't the case. As a practitioner, I would see that and just be like, okay, that's not that significant. But how does especially someone who is not used to reasoning about the processes that produce these numbers know that?''
}
\end{quote}

The ordinal categories expected to resolve misinterpretation of double-precision metrics yield further confusion for both technical and non-technical stakeholders depending on how risk levels are construed. A possible explanation for this circling ambiguity, as indicated by some of our participants, could be attributed to the knowledge and communication gaps between people with and without ML backgrounds \cite{chen2021beyond,kommiya2024towards,varanasi2023currently}. P7 shared that their ML scientists \textit{``did not feel like there was a lot of risk involved''} when the models they work with are more interpretable to them, in contrast to what risk-averse employees felt about the model and its policy implications. This logic informs another assumption: to fix biases, what is needed is \textit{more} knowledge and training. However, most management-related practitioners in our sample concurred that technical knowledge in their organization is generally very isolated and emphasized the need to document differences in interpretations and motivating assumptions.

% Because products are made in the perspective of the organization, they can be easily dismissed to fall under the responsibility of a different position.

% \begin{quote}
%     \textit{At least in my experience, providing confidence scores, was not always a benefit, because the ways in which users interpreted those confidence scores varied a lot right. Some people would really set a lot of store by the difference between 0.90 and 0.92, and not care about the difference between 0.9 and 0.7. And so for multiple projects that I was on you know, I saw that the the easiest thing to do was to provide high, medium, low kind of like bucketing of confidences rather than to provide a confidence score. The assumption that kind of affords a user is that both significant digits are significant, that there is a difference between, you know, 0.95 and 0.97 or something like that. And that really like that wasn't the case. As a practitioner, I would see that and just be like, Okay, that's not that significant. But how do especially someone who is not used to reasoning about the processes that produce these numbers know that?
% }
% \end{quote}
% - exclusion complex
% - Some organizations have top-down assumptions handling 
% - 

% - we found similar observations - Mihir doc - quote - assumptins  techincal - infor  subjective deicon - rationale lost



\subsubsection {Unstructured Documentation}
\label{subsec:doc}

Though the successful adoption of responsible ML principles in organizational settings is open to debate \cite{deshpande2022responsible,rakova2021responsible,raiAdoption}, practitioners are undoubtedly getting increased exposure to different toolkits and frameworks for responsible ML \cite{liang2024systematic,yang2024navigating}. While most of these support systems have some reference for practitioners to elicit and discuss assumptions behind various decisions (section \ref{rel:periphery}), prior works are unclear on how and to what extent assumptions are documented through these toolkits from the perspective of a practitioner. Below, we discuss two characteristics of documentation practices that contribute to the confusion around assumptions.
% For instance, a recent study found that the use of framework such as modelcards have rapidly risen in the last few years.

\smallskip
\noindent \textbf{Informal and Implicit Recording.} \citet{delin1994assumption}, in their discussion about assumptions, make a distinction between formulating and installing an assumption: while \textit{formulating} is about the expression of intent to install an assumption, \textit{installation} corresponds to aligning the execution to the intention within given constraints. In one of the excerpts in our case study (appendix \ref{appendix} \cite[p.~64]{anil2023palm}), the authors use ``marked references'' or annotated characterizations of identity\footnote{Some participants speculated the assumptions behind the usage of this term, but for this discussion, this can be perceived as typical annotations in ML data pipelines.} to assess the representational bias of their language model. Though our participants did not use the terms referenced in the \textit{Informal Logic} literature exactly, they did observe that if the authors had intended to operationalize representativeness with these annotations, then they must have correctly \textit{installed} the assumptions but did not explicitly document the \textit{formulation}. While the authors of the case study vaguely note down this assumption in the limitations section of the report, many of our participants were dissatisfied and shared that assumptions that are not articulated at the time of their identification get forgotten over time. 

Because the recording of assumptions is often informal and unstructured, system documentation and other related trackers are rarely modified after an assumption is acted upon. While some practitioners stated that their organizations require them to record assumptions in separate sections in accordance with a design or product requirement, others just recorded them offhand in change logs. Both approaches offered no structured prompt or instructions about how to aptly record them. As such, many participants hinted that documentation is not often given serious thought for internal projects. Below, P9, another technical lead, admits how these details are either not documented explicitly or get lost in an attempt to simplify the language.

\begin{quote}
\textit{``...when I run into an assumption like that, I try to distill it sometimes through multiple rewrites into a simple, concise, clear declaratory statement which can be connected to others...I would forget what I had in mind when I was writing those things down...'' 
}
\end{quote} 

The site, content, and style of assumptions documentation also have a strong relation with the primary role and responsibility of the practitioner, leading to conflicting situations when one party (say an ML scientist) has to consume and work with assumptions recorded by another party (say a product manager). For instance, writing and analyzing compliance-related reports are often perceived as the responsibility of safety or legal teams. Their articulation of assumptions might differ from that of developers and data scientists, who usually write in terms of inputs and outputs. If these technical practitioners are tasked to write about assumptions, they are less likely to appreciate the implications of their assumptions and choices in their reports. As the data scientist P5 mentioned, when they use LLMs, they go directly to model docs, datasets, and benchmarks, and treat the system cards and other reports accompanying these LLMs (with safety and RAI analysis) as just \textit{``terms and conditions.''} This fundamental asymmetry presents a fragmentation of thinking and recording that has consequences for how assumptions are handled. 

% 		ii. RML just borrows inputs from various laws - we need dedicated legal frameworks.
% Some implicit decisions require technical effort - need better documentations

% In this section, we will use an example from our case study where the LLM authors mention that they \textit{``excluded data from certain sites known to contain a high volume of personal information about private individuals''} (cite).

\smallskip 
\noindent \textbf{Granularity of recording.} We leverage the \textit{analytical} perspective our practitioners took in the case studies in assessing how deliberate they were in recording assumptions. As \citet{brookfield1992uncovering} argues, this particular type of perspective is well-positioned to step out of a familiar interpretive frame of reference and look at assumptions through an unfamiliar lens. We observe that the most common practice the participants followed is simply listing down the identified assumptions in some style and form, though at varied levels of formality, such as they would in a business requirement document or vocally at a team meeting. However, practitioners note that this method generally relies on their own unjustified claims. In other words, there was no distinction between the premises, target, and the argument that the assumption-target complex makes.

While listing down assumptions found in the case study, most of the practitioners sought further clarification on \textit{why} particular assumptions were made. This phase is when the interpretive lens of a practitioner begins to crack, and their primary disciplinary training starts to dominate; the choice of assumption to expand on and what needs to be explained starts to depend again on their organizational role or lived experience. For instance, when technical practitioners observe relative qualifiers such as ``higher'' or ``lower'', their scrutiny often stops at the sight of a quantified relation, such as ``40\% higher.'' \rev{Participants shared that many technical practitioners} end up not reflecting on how benchmarks are only the \textit{indications} and not the equivalence of the capability they are measuring. \rev{In the words of P11, an AI ethicist:}

\begin{quote}
    \textit{\rev{``The fact that an LLM could perform well on an LSAT benchmark means not necessarily that it's capable of, you know, of solving legal problems. But it could be quite capable of delivering to you the outputs that mimic responses to those questions and that's valuable. Where we can go wrong is to infer that, you know, because the model can pass a test that must mean that its feature representations allow it to understand the material of the test. This is a very different question and that I feel like needs holistic evaluations. You know, once they're developed, there's a huge evaluation gap here.''}}
\end{quote}

% \rev{Below, P4 explains how the granularity of assumptions inquiry ends when benchmarks are conflated with the capability they are intended to measure:}

% \begin{quote}
%     \textit{\rev{``if your model can solve this entailment data set, the idea is that it can do textual entailment more broadly. But there are several kinds of subpopulations that you need to reason about before you can actually take that to the task that you actually care about. It's reasonable with the intentional definition of entailment that is presented to say, like, oh, my model can understand that 3 tenths and 30\% of the same thing. That is like a linguistic equivalence. But models don't do that right. So if you are building an entailment based application that relies on that kind of numerical reasoning, your application will break. That's not an assumption that you can actually...do textual entailment as a capability rather than textual entailment on MultiNLI as a data set.''}}
% \end{quote}

How a clarification is or ought to be justified is a critical step in assumption recording, the complexities of which are expanded upon in the following section. P8, an AI governance architect, makes a distinction between \textit{visibility} and \textit{explainability} in recording assumptions to avoid confusion. While the former could be one or more declarative statements that add a marginal level of detail, the latter is about reasoning in every step with logical validity. For instance, in many of the LLM reports, our participants observed that data quality is justified by arguing that the followed data processing steps led to ``good'' model outcome in terms of some metrics. However, the above aligns more with the visibility-level of clarification. An explainability-level of granularity would involve expanding how quality is operationalized (for example, showing no outliers, following a representative distribution, etc.) and then logically explaining how the data can be assessed on both these parameters and the model outcome.

Finally, the maximum level of granularity in which an assumption can be recorded is ambiguous and context-dependent. While some practitioners seek ``expert'' intervention, which entails transferring the authority to a team lead or an executive, most practitioners instead advocate for a diverse collaborative effort to avoid biases. In section 5.1, we discuss a framework inspired by argument structuring in Informal Logic to articulate assumptions that facilitate such discussions.

% To do:
% 1. change quotes repetitive participants
% 2. include citations
% 3. check tense consistency

% We observe that practitioners implicitly or explicitly follow three distinct levels of granularity in recording assumptions. 
\section{Limitations and future directions}
In this work, we introduce a theoretical framework for understanding the mechanism of weak-to-strong (W2S) generalization in the variance-dominated regime where both the student and teacher have sufficient capacities for the downstream task. Leveraging the low intrinsic dimensionality of finetuning (FT), we characterize model capacities from three perspectives: FT approximation errors for ``accuracy'', intrinsic dimensions for ``complexity'', and student-teacher correlation for ``alignment''. Our analysis shows that W2S generalization is driven by variance reduction in the discrepancy between the weak teacher and strong student features. 
This generalization analysis is followed by a case study on the relative W2S performance in terms of performance gap recovery (PGR) and outperforming ratio (OPR). We show that while larger sample sizes imply better W2S generalization in an absolute sense, the relative W2S performance can degenerate as the sample size increases.
Our results provide theoretical insights into the choice of weak teachers and sample sizes in W2S pipelines. 

An interesting implication of our analysis is that the mechanism of W2S may differ as the balance between variance and bias shifts. In the variance-dominated regime studied in this work, W2S can benefit from a lower intrinsic dimension of the strong student due to the resulting variance reduction in the subspace of discrepancy from the weak teacher. In contrast, in the bias-dominated regime, the lower approximation error of the strong student is generally brought by the larger ``capacity'' of the strong model corresponding to a higher intrinsic dimension~\citep{ildiz2024high,wu2024provable}. 
This calls for future studies on unified views and transitions between the two regimes, which will provide a more comprehensive understanding of W2S.
Toward this goal, a limitation of our analysis is the quantification for the advantage of W2S in bias (see \Cref{fn:bias_strong}), which could be a promising next step.


% \newpage
\bibliographystyle{ACM-Reference-Format}
\bibliography{references}

\appendix
% \clearpage
\onecolumn
\clearpage
\appendix
\appendixpage  % if you use a package that provides an appendix title page
\hypersetup{linkcolor=black}
\startcontents[sections]
\printcontents[sections]{l}{1}

\hypersetup{linkcolor=hrefblue}
\glsresetall

\section{Additional related works}\label{apx:related_works}

\paragraph{Knowledge distillation.}
Knowledge distillation (KD)~\citep{hinton2015distilling,gou2021knowledge} is closely connected to W2S generalization regarding the teacher-student setup, while W2S reverts the capacities of teacher and student in KD. In KD, a strong teacher model guides a weak student model to learn the teacher's knowledge. In contrast, W2S generalization occurs when a strong student model surpasses a weak teacher model under weak supervision.
\citet{phuong2019towards,stanton2021does,ojha2023knowledge,nagarajan2023student,dong2024cluster,ildiz2024high} conducted rigorous statistical analyses for the student's generalization from knowledge distillation. 
From the analysis perspective, a key difference between KD and W2S is that W2S is usually analyzed in the context of finetuning since the notions of “weak” and “strong” are built upon pretraining. This finetuning perspective introduces distinct angles from KD for examining intrinsic dimension~\citep{li2018measuring} and student-teacher correlation in W2S. 

\paragraph{Self-distillation and self-training.}
In contrast to W2S that considers distinct student and teacher models, self-distillation~\citep{zhang2019your,zhang2021self} and related paradigms such as Born-Again Networks~\citep{furlanello2018born} use the same or progressively refined architectures to iteratively distill knowledge from a ``previous version'' of the model. There have been extensive theoretical analyses toward understanding the mechanism behind self-distillation~\citep{mobahi2020self,das2023understanding,borup2023self,pareek2024understanding}.

Self-training~\citep{scudder1965probability,lee2013pseudo} is a closely related method to self-distillation that takes a single model's confident predictions to create pseudo-labels for unlabeled data and refines that model iteratively. 
\citet{wei2020theoretical,oymak2021theoretical,frei2022self} provide theoretical insights into the generalization of self-training. 
In particular, \citet{wei2020theoretical} introduced a theoretical framework based on neighborhood expansion, which was later on extended to various settings of weakly supervised learning, including domain adaptation~\citep{cai2021theory}, contrastive learning~\citep{shen2022connect}, consistency regularization~\citep{yang2023sample}, and now weak-to-strong generalization~\citep{lang2024theoretical,shin2024weak}.




\section{Proofs in \Cref{sec:single_task_ft}}

\begin{lemma}\label{lem:low_est_err_ft}    
    Given the FT approximation errors $\rho_s$ and $\rho_w$ in \Cref{def:ft_est_err}, we have
    \begin{align*}
        \rho_s(n) \le n \rho_s \quad \text{and} \quad \rho_w(n) \le n \rho_w \quad \forall\ n \in \N.
    \end{align*}
\end{lemma}

\begin{proof}[Proof of \Cref{lem:low_est_err_ft}]
    Let $\thetab_* = \argmin_{\thetab \in \R^d}\ \E_{\xb \sim \Dcal}[(\phi_w(\xb)^\top \thetab - f_*(\xb))^2]$ such that
    \begin{align*}
        \E_{\xb \sim \Dcal}[(\phi_w(\xb)^\top \thetab_* - f_*(\xb))^2] = \rho_w.
    \end{align*}
    Then, by observing that conditioned on $\Xb$,
    \begin{align*}
        \phi_w(\Xb)^\dagger f_*(\Xb) = \argmin_{\thetab \in \R^d}\ \| \phi_w(\Xb) \thetab - f_*(\Xb) \|_2^2,
    \end{align*} 
    we have
    \begin{align*}
        \rho_w(n) &= \E_{\Xb \sim \Dcal^n}\sbr{\| \phi_w(\Xb) \phi_w(\Xb)^\dagger f_*(\Xb) - f_*(\Xb) \|_2^2} \\
        &\le \E_{\Xb \sim \Dcal^n}\sbr{\| \phi_w(\Xb) \thetab_* - f_*(\Xb) \|_2^2} \\
        &= n\ \E_{\Xb \sim \Dcal^n}\sbr{\frac{1}{n} \| \phi_w(\Xb) \thetab_* - f_*(\Xb) \|_2^2} \\
        &= n\ \E_{\xb \sim \Dcal}\sbr{(\phi_w(\xb)^\top \thetab_* - f_*(\xb))^2} \\
        &= n\ \rho_w.
    \end{align*}
    The proof for $\rho_s(n)$ follows analogously.
\end{proof}



\subsection{Proof of \Cref{thm:w2s_ft}}\label{apx:pf_w2s_ft}

\begin{theorem}[Formal restatement of \Cref{thm:w2s_ft}]\label{thm:w2s_ft_formal}
    Consider $f_\wts(\xb) = \phi_s(\xb)^\top \thetab_\wts$ finetuned as in \eqref{eq:sft_weak}, \eqref{eq:w2s_ft} with both $\alpha_w, \alpha_\wts \to 0$. Under \Cref{asm:features,asm:ft_data}, when $n \ge \Omega(d_w)$, the excess risk $\exrisk(f_\wts) = \vari(f_\wts) + \bias(f_\wts)$ satisfies
    \begin{align*}
        &\bias(f_\wts) \le \frac{\rho_w(n)}{n} + \frac{\rho_s(N)}{N} \le \rho_w + \rho_s, \\
        &\vari(f_\wts) \lesssim \frac{\sigma^2}{n} \rbr{d_{s \wedge w} + \frac{d_s}{N} (d_w - d_{s \wedge w})}.
    \end{align*}
    In particular, when ${\rho_w(n)}/{n} > 0$ and $d_s < d_w$, the inequality for $\bias(f_\wts)$ is strict.

    Moreover, when $\phi_w(\xb) \sim \Ncal(\b0_d, \Sigmab_w)$, for any $n > d_w + 1$, we have 
    \begin{align*}
        &\vari(f_\wts) = \frac{\sigma^2}{n-d_w-1} \rbr{d_{s \wedge w} + \frac{d_s}{N} (d_w - d_{s \wedge w})}.
    \end{align*}
\end{theorem}

\begin{proof}[Proof of \Cref{thm:w2s_ft} and \Cref{thm:w2s_ft_formal}]
    We first observe that the solution of \eqref{eq:sft_weak} as $\alpha_w \to 0$ is given by
    \begin{align*}
        \thetab_w = \wt\Phib_w^\dagger \wt\yb = \wt\Phib_w^\dagger (\wt\fb_* + \wt\zb),
    \end{align*}
    where $\wt\zb \sim \Ncal(\b0_n, \sigma^2 \Ib_n)$.
    Meanwhile, the solution of \eqref{eq:w2s_ft} as $\alpha_\wts \to 0$ is given by
    \begin{align*}
        \thetab_\wts = \Phib_s^\dagger \Phib_w \thetab_w = \Phib_s^\dagger \Phib_w \wt\Phib_w^\dagger (\wt\fb_* + \wt\zb).
    \end{align*}  
    
    Then, the excess risk of $f_\wts$ can be decomposed into variance and bias as follows:
    \begin{align*}
        \exrisk(f_\wts) &= \E_{\xb \sim \Dcal}\sbr{\E_{f_\wts}\sbr{(f_\wts(\xb) - f_*(\xb))^2}} \\
        &= \E_{\Scal_x}\sbr{\E_{\wt\Scal}\sbr{\frac{1}{N}\nbr{\Phib_s \thetab_\wts - \fb_*}_2^2}} \\
        &=\E_{\Scal_x, \wt\Scal}\sbr{\frac{1}{N} \nbr{(\Phib_s \Phib_s^\dagger \Phib_w \wt\Phib_w^\dagger \wt\fb_* - \fb_*) + \Phib_s \Phib_s^\dagger \Phib_w \wt\Phib_w^\dagger \wt\zb}_2^2} \\
        &= \underbrace{\frac{1}{N} \E_{\Scal_x, \wt\Scal}\sbr{\nbr{\Phib_s \Phib_s^\dagger \Phib_w \wt\Phib_w^\dagger \wt\zb}_2^2}}_{\vari(f_\wts)} + \underbrace{\frac{1}{N} \E_{\Scal_x, \wt\Scal}\sbr{\nbr{\Phib_s \Phib_s^\dagger \Phib_w \wt\Phib_w^\dagger \wt\fb_* - \fb_*}_2^2}}_{\bias(f_\wts)}.
    \end{align*}

    \paragraph{Bias.}
    For the bias term, by observing that $\Pb_s = \Phib_s \Phib_s^\dagger$ is an $N \times N$ orthogonal projection, we can decompose the bias term as
    \begin{align*}
        \bias(f_\wts) &= \E_{\Scal_x, \wt\Scal}\sbr{\frac{1}{N} \nbr{\Pb_s \rbr{\Phib_w \wt\Phib_w^\dagger \wt\fb_* - \fb_*}}_2^2} + \frac{1}{N} \E_{\Scal_x}\sbr{\nbr{\rbr{\Ib_N - \Pb_s} \fb_*}_2^2},
    \end{align*}
    where $\E_{\Scal_x}\sbr{\nbr{\rbr{\Ib_N - \Pb_s} \fb_*}_2^2} = \rho_s(N)$ by \Cref{def:ft_est_err}.

    For the first term, 
    \begin{align*}
        \E_{\Scal_x, \wt\Scal}\sbr{\frac{1}{N} \nbr{\Pb_s \rbr{\Phib_w \wt\Phib_w^\dagger \wt\fb_* - \fb_*}}_2^2} &\le \E_{\Scal_x, \wt\Scal}\sbr{\frac{1}{N} \nbr{\Phib_w \wt\Phib_w^\dagger \wt\fb_* - \fb_*}_2^2} \\
        &= \E_{\wt\Scal}\sbr{\frac{1}{n} \nbr{\wt\Phib_w \wt\Phib_w^\dagger \wt\fb_* - \wt\fb_*}_2^2} \\
        &= \frac{\rho_w(n)}{n}.
    \end{align*}
    Notice that when ${\rho_w(n)}/{n} > 0$, this inequality is strict if $d_s < d_w$, where $\Phib_w \wt\Phib_w^\dagger \wt\fb_* - \wt\fb_* \notin \range(\Phib_s)$ almost surely.

    Overall, we have
    \begin{align*}
        \bias(f_\wts) \le \frac{\rho_w(n)}{n} + \frac{\rho_s(N)}{N} \le \rho_w + \rho_s,
    \end{align*}
    where the second inequality follows from \Cref{lem:low_est_err_ft}.

    \paragraph{Variance.}
    For the variance term, we observe that
    \begin{align*}
    \begin{split}
        \vari(f_\wts) &= \frac{1}{N} \E_{\Scal_x, \wt\Scal}\sbr{\nbr{\Pb_s \Phib_w \wt\Phib_w^\dagger \wt\zb}_2^2} \\
        &= \frac{1}{N} \E_{\Scal_x, \wt\Scal}\sbr{\tr\rbr{\Phib_w^\top \Pb_s \Phib_w \wt\Phib_w^\dagger \wt\zb \wt\zb^\top (\wt\Phib_w^\dagger)^\top}} \\
        &= \frac{\sigma^2}{N} \E_{\Scal_x, \wt\Scal}\sbr{\tr\rbr{\Phib_w^\top \Pb_s \Phib_w (\wt\Phib_w^\top \wt\Phib_w)^\dagger}},
    \end{split}
    \end{align*}
    which implies
    \begin{align}\label{eq:pf_var_w2s}
    \begin{split}
        \vari(f_\wts) = \frac{\sigma^2}{N} \tr\rbr{\E_{\Scal_x}\sbr{\Sigmab_w^{-1/2} \Phib_w^\top \Pb_s \Phib_w \Sigmab_w^{-1/2}} \E_{\wt\Scal}\sbr{\rbr{\Sigmab_w^{-1/2} \wt\Phib_w^\top \wt\Phib_w \Sigmab_w^{-1/2}}^\dagger}}.
    \end{split}
    \end{align}

    Recall the spectral decomposition $\Sigmab_w = \Vb_w \Lambdab_w \Vb_w^\top$. 
    Since $\E_{\xb \sim \Dcal}[\phi_w(\xb) \phi_w(\xb)^\top] = \Sigmab_w$, for each $\xb \sim \Dcal$, we can write $\phi_w(\xb) = \Sigmab_w^{1/2} \gammab$, where $\gammab \in \R^{d}$ is an independent random vector that is zero-mean and isotropic (\ie $\E[\gammab] = \b0_{d}$ and $\E[\gammab \gammab^\top] = \Ib_{d}$). The same holds for $\Sigmab_s = \Vb_s \Lambdab_s \Vb_s^\top$ and $\phi_s(\xb) = \Sigmab_s^{1/2} \gammab$.

    Then, for $\Scal$ and $\wt\Scal$, there exist independent random matrices $\Gammab = [\gammab_1, \ldots, \gammab_N]^\top \in \R^{N \times d}$ and $\wt\Gammab = [\wt\gammab_1, \ldots, \wt\gammab_n]^\top \in \R^{n \times d}$ consisting of $\iid$ zero-mean isotropic rows such that
    \begin{align}\label{eq:pf_var_w2s_subgaussian_asm}
    \begin{split}
        &\Phib_w \Sigmab_w^{-1/2} = \Gammab \Sigmab_w^{1/2} \Sigmab_w^{-1/2} = \Gammab \Vb_w \Vb_w^\top, \\
        &\wt\Phib_w \Sigmab_w^{-1/2} = \wt\Gammab \Sigmab_w^{1/2} \Sigmab_w^{-1/2} = \wt\Gammab \Vb_w \Vb_w^\top, \\
        &\Phib_s \Sigmab_s^{-1/2} = \Gammab \Sigmab_s^{1/2} \Sigmab_s^{-1/2} = \Gammab \Vb_s \Vb_s^\top, \\
        &\wt\Phib_s \Sigmab_s^{-1/2} = \wt\Gammab \Sigmab_s^{1/2} \Sigmab_s^{-1/2} = \wt\Gammab \Vb_s \Vb_s^\top.
    \end{split}
    \end{align}
    Let $\Gammab_w = \Gammab \Vb_w \in \R^{N \times d_w}$ and $\wt\Gammab_w = \wt\Gammab \Vb_w \in \R^{n \times d_w}$. We observe that
    \begin{align*}
        \E_{\wt\Scal}\sbr{\rbr{\Sigmab_w^{-1/2} \wt\Phib_w^\top \wt\Phib_w \Sigmab_w^{-1/2}}^\dagger}
        = \E_{\wt\Scal}\sbr{\rbr{\Vb_w \wt\Gammab_w^\top \wt\Gammab_w \Vb_w^\top}^\dagger} 
        = \Vb_w \E_{\wt\Scal}\sbr{\rbr{\wt\Gammab_w^\top \wt\Gammab_w}^\dagger} \Vb_w^\top.
    \end{align*}

    Now, we consider the following two cases for the feature distribution of $\phi_w(\xb)$, corresponding to the distribution of $\Gammab_w$ and $\wt\Gammab_w$:
    \begin{enumerate}[label=(\alph*)]
        \item \b{Gaussian features}: In \Cref{thm:w2s_ft}, assuming $\phi_w(\xb) \sim \Ncal(\b0_d, \Sigmab_w)$ such that $\wt\Gammab_w$ consists of $\iid$ Gaussian rows, we have $\wt\gammab_i \sim \Ncal(\b0_{d_w}, \Ib_{d_w})$. Notice that under the assumption $n > d_w + 1$, $\rank(\wt\Gammab_w) = d_w$ almost surely, and therefore $\wt\Gammab_w^\top \wt\Gammab_w$ is invertible.
        
        Meanwhile, with $\wt\gammab_i \sim \Ncal(\b0_{d_w}, \Ib_{d_w})$ for all $i \in [n]$, $(\wt\Gammab_w^\top \wt\Gammab_w) \sim \Wcal(\Ib_{d_w},n)$ follows the Wishart distribution~\citep[Definition 3.4.1]{wishart1928generalised} with $n$ degrees of freedom and scale matrix $\Ib_{d_w}$. 
        Therefore, $(\wt\Gammab_w^\top \wt\Gammab_w)^{-1} \sim \Wcal^{-1}(\Ib_{d_w},n)$ follows the inverse Wishart distribution~\citep[\S 3.8]{mardia2024multivariate}, whose mean takes the form~\citep[(3.8.3)]{mardia2024multivariate}
        \begin{align*}
            \E_{\wt\Scal}\sbr{(\wt\Gammab_w^\top \wt\Gammab_w)^\dagger} = \frac{1}{n - d_w -1} \Ib_{d_w}.
        \end{align*}
        Then, we have
        \begin{align*}
            \E_{\wt\Scal}\sbr{\rbr{\Sigmab_w^{-1/2} \wt\Phib_w^\top \wt\Phib_w \Sigmab_w^{-1/2}}^\dagger}
            = \frac{1}{n - d_w -1} \Vb_w \Vb_w^\top.
        \end{align*}
        Therefore, \eqref{eq:pf_var_w2s} implies
        \begin{align}\label{eq:pf_var_w2s_1}
        \begin{split}
            \vari(f_\wts) &= \frac{\sigma^2}{N}\ \frac{1}{n - d_w -1}\ \tr\rbr{\Vb_w^\top \E_{\Scal_x}\sbr{\Sigmab_w^{-1/2} \Phib_w^\top \Pb_s \Phib_w \Sigmab_w^{-1/2}} \Vb_w} \\
            &= \frac{\sigma^2}{N}\ \frac{1}{n - d_w -1}\ \tr\rbr{\E_{\Scal_x}\sbr{\Vb_w^\top \Vb_w \Gammab_w^\top \Pb_s \Gammab_w \Vb_w^\top \Vb_w}} \\
            &= \frac{\sigma^2}{N}\ \frac{1}{n - d_w -1}\ \tr\rbr{\E_{\Scal_x}\sbr{\Gammab_w^\top \Pb_s \Gammab_w}}.
        \end{split}
        \end{align}
        Recall that $\Pb_s = \Phib_s \Phib_s^\dagger$. Let $\Gammab_s = \Gammab \Vb_s \in \R^{N \times d_s}$, and we can write
        \begin{align*}
            \Pb_s = (\Phib_s \Sigmab_s^{-1/2}) (\Phib_s \Sigmab_s^{-1/2})^\dagger = (\Gammab_s \Vb_s^\top) (\Gammab_s \Vb_s^\top)^\dagger = \Gammab_s \Gammab_s^\dagger.
        \end{align*}
        Therefore, with $\Gammab_w = \Gammab \Vb_w$ and $\Gammab_s = \Gammab \Vb_s$, we can decompose
        \begin{align*}
            \tr\rbr{\E_{\Scal_x}\sbr{\Gammab_w^\top \Pb_s \Gammab_w}} 
            &= \E_{\Scal_x}\sbr{\tr\rbr{\Gammab_w^\top \Gammab_s \Gammab_s^\dagger \Gammab_w}} \\
            &= \E_{\Scal_x}\sbr{\tr\rbr{\Vb_w^\top \Vb_s \Vb_s^\top \Vb_w \Gammab_w^\top \Gammab_s \Gammab_s^\dagger \Gammab_w}} \\
            &+ \E_{\Scal_x}\sbr{\tr\rbr{\Vb_w^\top (\Ib_d - \Vb_s \Vb_s^\top) \Vb_w \Gammab_w^\top \Gammab_s \Gammab_s^\dagger \Gammab_w}}.
        \end{align*}
        For the first term, since $\Gammab_w \Vb_w^\top \Vb_s = \Gammab \Vb_w \Vb_w^\top \Vb_s$ and $\Gammab_s = \Gammab \Vb_s$, the range of $\Gammab_w \Vb_w^\top \Vb_s$ is a subspace of that of $\Gammab_s$ and therefore,
        \begin{align*}
            \E_{\Scal_x}\sbr{\tr\rbr{\Vb_w^\top \Vb_s \Vb_s^\top \Vb_w \Gammab_w^\top \Gammab_s \Gammab_s^\dagger \Gammab_w}} 
            &= \E_{\Scal_x}\sbr{\tr\rbr{ \Vb_s^\top \Vb_w \Gammab_w^\top \Gammab_s \Gammab_s^\dagger \Gammab_w \Vb_w^\top \Vb_s}} \\
            &= \E_{\Scal_x}\sbr{\tr\rbr{ \Vb_s^\top \Vb_w \Gammab_w^\top \Gammab_w \Vb_w^\top \Vb_s}} \\
            &= \tr\rbr{\Vb_s^\top \Vb_w \E_{\Scal_x}\sbr{\Gammab_w^\top \Gammab_w} \Vb_w^\top \Vb_s}.
        \end{align*}
        Since $\E_{\Scal_x}\sbr{\Gammab_w^\top \Gammab_w} = N \Ib_{d_w}$, we have
        \begin{align*}
            \E_{\Scal_x}\sbr{\tr\rbr{\Vb_w^\top \Vb_s \Vb_s^\top \Vb_w \Gammab_w^\top \Gammab_s \Gammab_s^\dagger \Gammab_w}} 
            &= N \tr\rbr{\Vb_s^\top \Vb_w \Vb_w^\top \Vb_s} \\
            &= N \nbr{\Vb_s^\top \Vb_w}_F^2 \\
            &= N d_{s \wedge w}.
        \end{align*}
        For the second term, we first observe that the row space of $\Gammab_w \Vb_w^\top (\Ib_d - \Vb_s \Vb_s^\top)$ is orthogonal to that of $\Gammab_s = \Gammab \Vb_s$, and therefore, $\Gammab_w \Vb_w^\top (\Ib_d - \Vb_s \Vb_s^\top)$ and $\Gammab_s$ are independent, which implies
        \begin{align*}
            \E_{\Scal_x}\sbr{\tr\rbr{\Vb_w^\top (\Ib_d - \Vb_s \Vb_s^\top) \Vb_w \Gammab_w^\top \Gammab_s \Gammab_s^\dagger \Gammab_w}} 
            &= \tr\rbr{\E\sbr{\Gammab_w \Vb_w^\top (\Ib_d - \Vb_s \Vb_s^\top) \Vb_w \Gammab_w^\top} \E\sbr{\Gammab_s \Gammab_s^\dagger}}.
        \end{align*}
        Since $\Gammab$ consists of independent isotropic rows, so do $\Gammab_s = \Gammab \Vb_s \in \R^{N \times d_s}$ and $\Gammab_w = \Gammab \Vb_w \in \R^{N \times d_w}$, which implies
        \begin{align*}
            \E\sbr{\Gammab_s \Gammab_s^\dagger} = \frac{d_s}{N}\ \Ib_N \quad \t{and} \quad \E\sbr{\Gammab_w^\top \Gammab_w} = N\ \Ib_{d_w}.
        \end{align*}
        Then, we have
        \begin{align*}
            \E_{\Scal_x}\sbr{\tr\rbr{\Vb_w^\top (\Ib_d - \Vb_s \Vb_s^\top) \Vb_w \Gammab_w^\top \Gammab_s \Gammab_s^\dagger \Gammab_w}} 
            &= \tr\rbr{\E\sbr{\Gammab_w \Vb_w^\top (\Ib_d - \Vb_s \Vb_s^\top) \Vb_w \Gammab_w^\top} \E\sbr{\Gammab_s \Gammab_s^\dagger}} \\
            &= \frac{d_s}{N} \tr\rbr{\E\sbr{\Gammab_w \Vb_w^\top (\Ib_d - \Vb_s \Vb_s^\top) \Vb_w \Gammab_w^\top}} \\
            &= \frac{d_s}{N} \tr\rbr{\Vb_w^\top (\Ib_d - \Vb_s \Vb_s^\top) \Vb_w \E\sbr{\Gammab_w^\top \Gammab_w}} \\
            &= \frac{d_s}{N} N \tr\rbr{\Vb_w^\top (\Ib_d - \Vb_s \Vb_s^\top) \Vb_w} \\
            &= d_s (d_w - d_{s \wedge w}).
        \end{align*}
        Combining the two terms, we have
        \begin{align*}
            \tr\rbr{\E_{\Scal_x}\sbr{\Gammab_w^\top \Pb_s \Gammab_w}} = N d_{s \wedge w} + d_s (d_w - d_{s \wedge w}).
        \end{align*}
        Then, by \eqref{eq:pf_var_w2s_1}, the variance is exactly characterized by
        \begin{align*}
            \vari(f_\wts) 
            &= \frac{\sigma^2}{N}\ \frac{N d_{s \wedge w} + d_s (d_w - d_{s \wedge w})}{n - d_w -1} \\
            &= \frac{\sigma^2}{n-d_w-1} \rbr{d_{s \wedge w} + \frac{d_s}{N} (d_w - d_{s \wedge w})}.
        \end{align*}

        \item \b{Sub-gaussian features}: Relaxing the Gaussian feature assumption, when $\wt\Gammab_w$ consists of $\iid$ sub-gaussian random vectors that are zero-mean and isotropic (\ie $\E[\wt\gammab_i] = \b0_{d_w}$ and $\E[\wt\gammab_i \wt\gammab_i^\top] = \Ib_{d_w}$), with $n \ge \Omega(d_w)$, \Cref{lem:trace_inv_subgaussian} implies that
        \begin{align*}
            \E_{\wt\Scal}\sbr{(\wt\Gammab_w^\top \wt\Gammab_w)^\dagger} \aleq O\rbr{\frac{1}{n}} \Ib_{d_w},
        \end{align*}
        and therefore,
        \begin{align*}
            \E_{\wt\Scal}\sbr{\rbr{\Sigmab_w^{-1/2} \wt\Phib_w^\top \wt\Phib_w \Sigmab_w^{-1/2}}^\dagger} \aleq O\rbr{\frac{1}{n}} \Vb_w \Vb_w^\top.
        \end{align*}
        Then, via an analogous argument as \eqref{eq:pf_var_w2s_1}, \eqref{eq:pf_var_w2s} implies that 
        \begin{align}\label{eq:pf_var_w2s_2}
        \begin{split}
            \vari(f_\wts) \le \frac{\sigma^2}{N}\ O\rbr{\frac{1}{n}}\ \tr\rbr{\E_{\Scal_x}\sbr{\Gammab_w^\top \Pb_s \Gammab_w}}.
        \end{split}
        \end{align}
        We observe that in the analysis of the Gaussian feature case, the characterization
        \begin{align*}
            \tr\rbr{\E_{\Scal_x}\sbr{\Gammab_w^\top \Pb_s \Gammab_w}} = (N - d_s) d_{s \wedge w} + d_s d_w
        \end{align*}
        does not involve the Gaussianity of $\Gammab$ and therefore holds for general subgaussian features.
        This leads to an upper bound on the variance:
        \begin{align*}
            \vari(f_\wts) 
            &\le \frac{\sigma^2}{N}\ O\rbr{\frac{1}{n}}\ \rbr{N d_{s \wedge w} + d_s (d_w - d_{s \wedge w})} \\
            &\lesssim \frac{\sigma^2}{n} \rbr{d_{s \wedge w} + \frac{d_s}{N} (d_w - d_{s \wedge w})}.
        \end{align*}
    \end{enumerate}
\end{proof}


\begin{lemma}[Adapting \cite{vershynin2010introduction} Theorem 5.39]\label{lem:trace_inv_subgaussian}
    Let $\wt\Gammab_w = [\wt\gammab_1, \ldots, \wt\gammab_n]^\top$ be an $n \times d_w$ matrix whose rows $\wt\gammab_1, \ldots, \wt\gammab_n$ consist of $\iid$ sub-gaussian random vectors that are zero-mean and isotropic (\ie $\E[\wt\gammab_i] = \b0_{d_w}$ and $\E[\wt\gammab_i \wt\gammab_i^\top] = \Ib_{d_w}$). When $n \ge \Omega(d_w)$, we have
    \begin{align*}
        \E\sbr{\nbr{\rbr{\wt\Gammab_w^\top \wt\Gammab_w}^\dagger}_2} \le O\rbr{\frac{1}{n}},
    \end{align*}
    where $\Omega(\cdot)$ and $O(\cdot)$ suppresses constants that depend only on the sub-gaussian norm $\nbr{\wt\gammab_i}_{\psi_2} = \sup_{\vb \in \SSS^{d_w-1}} \sup_{p \ge 1} (\E[|\wt\gammab_i^\top \vb|^p])^{1/p} / \sqrt{p}$, independent of $n, d_w$.
\end{lemma}

\begin{proof}[Proof of \Cref{lem:trace_inv_subgaussian}]
    Let $\sigma_{\min}(\wt\Gammab_w^\top \wt\Gammab_w)$ be the smallest singular value of $\wt\Gammab_w^\top \wt\Gammab_w$.
    Leveraging \cite{vershynin2010introduction} Theorem 5.39, we notice that for $n \ge \Omega(d_w)$, there exist constants $c_1, c_2 > 0$ that depend only on the sub-gaussian norm $\nbr{\wt\gammab_i}_{\psi_2}$ such that
    \begin{align*}
        \Pr\sbr{\sigma_{\min}(\wt\Gammab_w^\top \wt\Gammab_w) < \rbr{\sqrt{n} - c_1\sqrt{d_w} - t}^2} \le \exp\rbr{-c_2 t^2}.
    \end{align*}
    Therefore, we have 
    \begin{align*}
        \Pr\sbr{\frac{1}{\sigma_{\min}(\wt\Gammab_w^\top \wt\Gammab_w)} > t} \le \exp\rbr{-c_2 \rbr{\sqrt{n} - c_1 \sqrt{d_w} - \sqrt{\frac{1}{t}}}^2}.
    \end{align*}

    Notice that for any non-negative random variable $Z$ with a cumulative density function $F_Z(z)$, 
    \begin{align*}
        \E\sbr{Z} &= \int_0^\infty z d F_Z(z) 
        = - \int_0^\infty z d \rbr{1 - F_Z(z)} \\
        &= \sbr{z \rbr{1 - F_Z(z)}}_0^\infty + \int_0^\infty \rbr{1 - F_Z(z)} dz \\
        &= \int_0^\infty \Pr\sbr{Z > z} dz.
    \end{align*}
    Therefore, we have
    \begin{align*}
        \E\sbr{\frac{1}{\sigma_{\min}(\wt\Gammab_w^\top \wt\Gammab_w)}} \le \int_0^\infty \exp\rbr{-c_2 \rbr{\sqrt{n} - c_1 \sqrt{d_w} - \sqrt{\frac{1}{t}}}^2} d t.
    \end{align*}
    Let $t_0 = 1 / \rbr{\sqrt{n} - c_1 \sqrt{d_w}}^2$ such that $\sqrt{n} - c_1 \sqrt{d_w} - \sqrt{\frac{1}{t}}=0$ and 
    \begin{align*}
        \int_{0}^{t_0} \exp\rbr{-c_2 \rbr{\sqrt{n} - c_1 \sqrt{d_w} - \sqrt{\frac{1}{t}}}^2} d t \le t_0
    \end{align*}
    Then, we have
    \begin{align*}
        &\E\sbr{\frac{1}{\sigma_{\min}(\wt\Gammab_w^\top \wt\Gammab_w)}} 
        \le \int_0^\infty \exp\rbr{-c_2 \rbr{\sqrt{n} - c_1 \sqrt{d_w} - \sqrt{\frac{1}{t}}}^2} d t \\
        &\le t_0 + \int_{t_0}^\infty \exp\rbr{-c_2 \rbr{\sqrt{n} - c_1 \sqrt{d_w} - \sqrt{\frac{1}{t}}}^2} d t \\
        &= t_0 + 2 \int_{0}^{\sqrt{n}-c_1\sqrt{d_w}} \exp\rbr{-c_2 u^2} \rbr{\sqrt{n} - c_1 \sqrt{d_w} - u}^{-3} d u \\
        &= t_0 + \frac{2}{\rbr{\sqrt{n} - c_1 \sqrt{d_w}}^2} \int_{0}^{1} \exp\rbr{-c_2 \rbr{\sqrt{n}-c_1\sqrt{d_w}}^2 u^2} \rbr{1 - u}^{-3} d u \\
        &= \frac{1}{\rbr{\sqrt{n} - c_1 \sqrt{d_w}}^2} + \frac{2}{\rbr{\sqrt{n} - c_1 \sqrt{d_w}}^2} \rbr{\int_{0}^{1} \exp\rbr{-\Omega\rbr{u^2}} \rbr{1 - u}^{-3} d u} \\
        &= O\rbr{\frac{1}{\rbr{\sqrt{n} - c_1 \sqrt{d_w}}^2}}.
    \end{align*}
    When $n \ge \Omega(d_w)$, we have $\sqrt{n} - c_1 \sqrt{d_w} \ge \Omega(\sqrt{n})$, and therefore ,
    \begin{align*}
        \E\sbr{\nbr{\rbr{\wt\Gammab_w^\top \wt\Gammab_w}^\dagger}_2}
        \le \E\sbr{\frac{1}{\sigma_{\min}(\wt\Gammab_w^\top \wt\Gammab_w)}} 
        \le O\rbr{\frac{1}{n}}.
    \end{align*}
\end{proof}





\subsection{Proof of \Cref{pro:sft_weak} and \Cref{cor:sft_strong}}\label{apx:pf_sft_weak}
\begin{proof}[Proof of \Cref{pro:sft_weak} and \Cref{cor:sft_strong}]
    The excess risk of the finetuned weak teacher $f_w(\xb) = \phi_w(\xb)^\top \thetab_w$ can be expressed as
    \begin{align*}
        \exrisk(f_w) &= \E_{\xb \sim \Dcal}\sbr{\E_{f_w}\sbr{(f_w(\xb) - f_*(\xb))^2}} \\
        &= \E_{\wt\Scal}\sbr{\frac{1}{n}\nbr{\wt\Phib_w \thetab_w - \wt\fb_*}_2^2},
    \end{align*}
    where $\wt\fb_* = [\fb_*(\wt\xb_1), \ldots, \fb_*(\wt\xb_n)]^\top \in \R^n$; and we recall that $\wt\Phib_w = [\phi_w(\wt\xb_1), \ldots, \phi_w(\wt\xb_n)]^\top$. Notice that the randomness of $\thetab_w$ comes from the SFT samples $\wt\Scal \sim \Dcal(f_*)^n$.

    Observe that the solution of \eqref{eq:sft_weak} as $\alpha_w \to 0$ is given by $\thetab_w = \wt\Phib_w^\dagger \wt\yb$, where $\wt\yb = \wt\fb_* + \wt\zb$ is the noisy label vector with $\wt\zb \sim \Ncal(\b0_n, \sigma^2 \Ib_n)$.
    Therefore, with the randomness over $\wt\Scal \sim \Dcal(f_*)^n$, we have
    \begin{align*}
        \exrisk(f_w) &= \E \sbr{\frac{1}{n}\nbr{\wt\Phib_w \wt\Phib_w^\dagger \wt\yb - \wt\fb_*}_2^2} \\
        &= \E \sbr{\frac{1}{n}\nbr{\wt\Phib_w \wt\Phib_w^\dagger \wt\zb + \rbr{\wt\Phib_w \wt\Phib_w^\dagger \wt\fb_* - \wt\fb_*}}_2^2} \\
        &= \underbrace{\E \sbr{\frac{1}{n}\nbr{\wt\Phib_w \wt\Phib_w^\dagger \wt\zb}_2^2}}_{\vari(f_w)} + \underbrace{\E\sbr{\frac{1}{n}\nbr{\wt\Phib_w \wt\Phib_w^\dagger \wt\fb_* - \wt\fb_*}_2^2}}_{\bias(f_w)}.
    \end{align*}
    
    For bias, by the definition of finetuning capacity (see \Cref{def:ft_est_err}), we have
    \begin{align*}
        \bias(f_w) = \frac{1}{n} \E\sbr{\nbr{\wt\Phib_w \wt\Phib_w^\dagger \wt\fb_* - \wt\fb_*}_2^2} = \frac{\rho_w(n)}{n}.
    \end{align*}
    We observe that $\bias(f_w) \le \rho_w$ by \Cref{lem:low_est_err_ft}.
    Notice that \Cref{lem:low_est_err_ft} also implies $\bias(f_s) = {\rho_s(n)}/{n} \le \rho_s$. 

    For variance, we observe that 
    \begin{align*}
        \vari(f_w) &= \frac{1}{n} \E\sbr{\nbr{\wt\Phib_w \wt\Phib_w^\dagger \wt\zb}_2^2} \\
        &= \frac{1}{n} \E\sbr{\tr\rbr{\wt\Phib_w \wt\Phib_w^\dagger \wt\zb \wt\zb^\top}} \\
        &= \frac{\sigma^2}{n} \E\sbr{\tr\rbr{\wt\Phib_w \wt\Phib_w^\dagger}}.
    \end{align*}
    By \Cref{asm:ft_data}, since $\rank(\wt\Phib_w) = d_w$ almost surely, we have
    \begin{align*}
        \vari(f_w) = \frac{\sigma^2}{n} \E\sbr{\tr\rbr{\wt\Phib_w \wt\Phib_w^\dagger}} = \frac{\sigma^2 d_w}{n}.
    \end{align*}
\end{proof}



\subsection{Proof of \Cref{cor:pgr}}\label{apx:pf_pgr}
\begin{proof}[Proof of \Cref{cor:pgr}]
    Noticing that with $\rank(\wt\Phib_w) = d_w$ and $\rank(\wt\Phib_s) = \rank(\Phib_s) = d_s$ almost surely, the excess risks of $f_w, f_s, f_c$ are characterized exactly in \Cref{pro:sft_weak} and \Cref{cor:sft_strong}, and $\exrisk(f_\wts)$ is upper bounded by \Cref{thm:w2s_ft}.
    Therefore, by directly plugging in the excess risks to the definitions of PGR and OPR, we have
    \begin{align}\label{eq:pgr_lower_tight}
    \begin{split}
        \pgr = &\frac{\exrisk(f_w) - \exrisk(f_\wts)}{\exrisk(f_w) - \exrisk(f_c)} \\
        \ge &\rbr{\sigma^2\ \frac{d_w}{n} + \frac{\rho_w(n)}{n} - \frac{\sigma^2}{n-d_w-1} \rbr{d_{s \wedge w} + \frac{d_s}{N} (d_w-d_{s \wedge w})} - \rbr{\frac{\rho_w(n)}{n} + \frac{\rho_s(N)}{N}}} \\
        &\rbr{\sigma^2\ \frac{d_w}{n} + \frac{\rho_w(n)}{n} - \sigma^2\ \frac{d_s}{N+n} - \frac{\rho_s(N+n)}{N+n}}^{-1} \\
        \ge &\rbr{\sigma^2 \frac{d_w}{n} - \sigma^2 \frac{d_{s \wedge w} + (d_w - d_{s \wedge w}) {d_s}/{N}}{n-d_w-1} - \frac{\rho_s(N)}{N}} \Big/ \rbr{\sigma^2 \frac{d_w}{n} + \frac{\rho_w(n)}{n}}, \\
        \ge &\rbr{\sigma^2 \frac{d_w}{n} - \sigma^2 \frac{d_{s \wedge w} + (d_w - d_{s \wedge w}) {d_s}/{N}}{n-d_w-1} - \rho_s} \Big/ \rbr{\sigma^2 \frac{d_w}{n} + \rho_w},
    \end{split}
    \end{align}
    and 
    \begin{align}\label{eq:opr_lower_tight}
    \begin{split}
        \opr = &\frac{\exrisk(f_s)}{\exrisk(f_\wts)} \\
        \ge &\rbr{\sigma^2\ \frac{d_s}{n} + \frac{\rho_s(n)}{n}} \Big/ \rbr{\sigma^2 \frac{d_{s \wedge w} + (d_w - d_{s \wedge w}) {d_s}/{N}}{n-d_w-1} + \rbr{\frac{\rho_w(n)}{n} + \frac{\rho_s(N)}{N}}} \\
        \ge &\sigma^2 \frac{d_s}{n} \Big/ \rbr{\sigma^2 \frac{d_{s \wedge w} + (d_w - d_{s \wedge w}) {d_s}/{N}}{n-d_w-1} + \rho_w + \rho_s}.
    \end{split}
    \end{align} 

    When taking $n = d_w + q + 1$ for some small constant $q \in \N$, we observe that 
    \begin{align*}
        \pgr &\ge \rbr{\sigma^2 \frac{d_w}{n} - \sigma^2 \frac{d_{s \wedge w} + (d_w - d_{s \wedge w}) {d_s}/{N}}{n-d_w-1} - \rho_s} \Big/ \rbr{\sigma^2 \frac{d_w}{n} + \rho_w} \\
        &\ge \rbr{\frac{d_w}{d_w + q + 1} - \frac{d_{s \wedge w}}{q} - \frac{d_s}{N} \frac{d_w - d_{s \wedge w}}{q} - \frac{\rho_s}{\sigma^2}} \Big/ \rbr{\frac{d_w}{d_w + q + 1} + \frac{\rho_w}{\sigma^2}} \\
        &\ge \rbr{\frac{d_w}{d_w + q + 1} - \frac{d_{s \wedge w}}{q} - \frac{d_s}{N} \frac{d_w - d_{s \wedge w}}{q} - \frac{\rho_s}{\sigma^2} - \frac{\rho_w}{\sigma^2}} \Big/ \rbr{\frac{d_w}{d_w + q + 1} + \frac{\rho_w}{\sigma^2} - \frac{\rho_w}{\sigma^2}} \\
        &= 1 - \frac{n}{d_w} \rbr{\frac{d_{s \wedge w}}{q} + \frac{d_s}{N} \frac{d_w - d_{s \wedge w}}{q} + \frac{\rho_w + \rho_s}{\sigma^2}} \\
        &= 1 - \frac{n}{q}\ {\frac{d_{s \wedge w} + (d_w - d_{s \wedge w}) d_s / N}{d_w}} - \frac{n}{d_w}\ {\frac{\rho_w + \rho_s}{\sigma^2}},
    \end{align*}
    and
    \begin{align*}
        \opr &\ge \sigma^2 \frac{d_s}{n} \Big/ \rbr{\sigma^2 \frac{d_{s \wedge w} + (d_w - d_{s \wedge w}) {d_s}/{N}}{n-d_w-1} + \rho_w + \rho_s} \\
        &= \frac{d_s}{n} \Big/ \rbr{\frac{d_{s \wedge w} + (d_w - d_{s \wedge w}) {d_s}/{N}}{q} + \frac{\rho_w + \rho_s}{\sigma^2}} \\
        &= \rbr{\frac{n}{q}\ \frac{d_{s \wedge w} + (d_w - d_{s \wedge w}) {d_s}/{N}}{d_s} + \frac{n}{d_s}\ \frac{\rho_w + \rho_s}{\sigma^2}}^{-1}.
    \end{align*}
\end{proof}



\subsection{Proof of \Cref{cor:non_monotonic_scaling}}\label{apx:pf_non_monotonic_scaling}
\begin{proof}[Proof of \Cref{cor:non_monotonic_scaling}]
    Recall the notations introduced for conciseness:
    \begin{align*}
        d_\wts(N) = d_{s \wedge w} + (d_w - d_{s \wedge w}) \frac{d_s}{N}, \quad \varrho = \frac{\rho_w + \rho_s}{\sigma^2}.
    \end{align*}
    Then, the lower bounds for $\pgr$ and $\opr$ in \Cref{cor:pgr} can be expressed in terms of $d_\wts(N)$ and $\varrho$ as 
    \begin{align*}
        \pgr \ge 1 - \frac{d_\wts(N)}{d_w} - \frac{d_w + 1}{d_w} \varrho - \frac{d_w + 1}{d_w}\ \frac{d_\wts(N)}{q} - q \frac{\varrho}{d_w},
    \end{align*}
    and 
    \begin{align*}
        \opr \ge \rbr{\frac{d_\wts(N)}{d_s} + \frac{d_w + 1}{d_s} \varrho + \frac{d_\wts(N)}{d_s}\ \frac{d_w + 1}{q} + q \frac{\varrho}{d_s}}^{-1}.
    \end{align*}
    Both lower bounds are maximized when the last two terms in the expressions that involve $q$ are minimized, which is achieved when $q = \sqrt{\rbr{d_w + 1} {d_\wts(N)}/{\varrho}}$. Substituting the optimal $q$ back into the expressions yields the best lower bounds for $\pgr$ and $\opr$:
    \begin{align*}
        \pgr \ge &1 - \frac{d_\wts(N)}{d_w} - \varrho \frac{d_w + 1}{d_w} - 2 \sqrt{\varrho \frac{d_w + 1}{d_w}\ \frac{d_\wts(N)}{d_w}} \\
        = &1 - \rbr{\sqrt{\frac{d_\wts(N)}{d_w}} + \sqrt{\varrho\ \frac{d_w + 1}{d_w}}}^2,
    \end{align*}
    and 
    \begin{align*}
        \opr \ge &\rbr{\frac{d_\wts(N)}{d_s} + \varrho \frac{d_w + 1}{d_s} + 2 \sqrt{\varrho \frac{d_w + 1}{d_s}\ \frac{d_\wts(N)}{d_s}}}^{-1} \\
        = &\rbr{\sqrt{\frac{d_\wts(N)}{d_s}} + \sqrt{\varrho\ \frac{d_w + 1}{d_s}}}^{-2}.
    \end{align*}
\end{proof}




\section{Ridge regression analysis}\label{apx:ridge_regression}
In this section, we investigate the more realistic scenario where the weak and strong feature covariances are not exactly low-rank but admit small numbers of dominating eigenvalues. 

Concretely, we consider the same data distribution $(\xb, y) \sim \Dcal(f_*)$ with $y = f_*(\xb) + z$ for some independent Gaussian label noise $z \sim \Ncal(0, \sigma^2)$ and an unknown ground truth function $f_*: \Xcal \to \R$ as in \Cref{sec:ridgeless_regression}.
Under the same sub-gaussian feature assumption as in \Cref{asm:features}, we adapt \Cref{def:low_intrinsic_dim,def:correlation_dim} to the ridge regression setting as follows.
\begin{assumption}[Data distribution]\label{asm:ridge_regression}
    Let $\phi_s: \Xcal \to \R^d$ and $\phi_w: \Xcal \to \R^d$ be the strong and weak pretrained models that take $\xb \sim \Dcal$ and output pretrained features $\phi_s(\xb), \phi_w(\xb) \in \R^d$, respectively.
    \begin{enumerate}[label=(\roman*)]
        \item \b{Ground truth}: Assume $f_*$ can be expressed as a linear function over an unknown ground truth feature $\phi_*: \Xcal \to \R^d$ such that $f_*(\cdot) = \phi_*(\cdot)^\top \thetab_*$ for some fixed $\thetab_* \in \R^d$.
        \item \b{Sub-gaussian features} (\Cref{asm:features}): Let $\phi_w(\xb)$, $\phi_s(\xb)$, $\phi_*(\xb)$ be zero-mean sub-gaussian random vectors with $\E[\phi_w(\xb)] = \E[\phi_s(\xb)] = \E[\phi_*(\xb)] = \b{0}_d$, and 
        \begin{align*}
            \E[\phi_w(\xb) \phi_w(\xb)^\top] = \Sigmab_w, \quad \E[\phi_s(\xb) \phi_s(\xb)^\top] = \Sigmab_s, \quad \E[\phi_*(\xb) \phi_*(\xb)^\top] = \Sigmab_*.
        \end{align*}
        For conciseness, we assume without loss of generality that these features are roughly normalized, \ie, $\nbr{\Sigmab_w}_2 \asymp 1$, $\nbr{\Sigmab_s}_2 \asymp 1$, and $\nbr{\Sigmab_*}_2 \asymp 1$.
        \item \b{Low intrinsic dimension}: Let $\Sigmab_s$ and $\Sigmab_w$ both be \b{positive-definite} with spectral decompositions $\Sigmab_s = \Vb_s \Lambdab_s \Vb_s^\top$ and $\Sigmab_w = \Vb_w \Lambdab_w \Vb_w^\top$, where $\Lambdab_s, \Lambdab_w \in \R^{d \times d}$ are diagonal matrices with positive eigenvalues in decreasing order; while $\Vb_s \in \R^{d \times d}$ and $\Vb_w \in \R^{d \times d}$ are orthogonal matrices consisting of the corresponding orthonormal eigenvectors. The low intrinsic dimension of FT implies that $\Lambdab_s = \diag(\lambda^s_1,\cdots,\lambda^s_d)$ and $\Lambdab_w = \diag(\lambda^w_1,\cdots,\lambda^w_d)$ consist of a few dominating eigenvalues, while the rest of the eigenvalues are negligible, \ie, there exist $d_s, d_w \ll d$ such that $\sum_{i > d_s} \lambda^s_i \ll \tr(\Sigmab_s)$ and $\sum_{i > d_w} \lambda^w_i \ll \tr(\Sigmab_w)$. Here, 
        \begin{align*}
            \tr(\Sigmab_s) \lesssim d_s \quad \t{and} \quad \tr(\Sigmab_w) \lesssim d_w
        \end{align*}
        effectively measure the intrinsic dimensions of $\phi_s$ and $\phi_w$.
    \end{enumerate}
\end{assumption}

\begin{remark}[Weak-strong similarity]
    In place of correlation dimension (\Cref{def:correlation_dim}) in the ridgeless setting, for the ridge regression analysis, we measure the similarity between the weak and strong models directly through $\tr(\Sigmab_s \Sigmab_w)$. Notice that 
    \begin{align*}
        \tr(\Sigmab_s \Sigmab_w) \le \min\cbr{\tr(\Sigmab_s)\nbr{\Sigmab_w}_2, \tr(\Sigmab_w)\nbr{\Sigmab_s}_2} \lesssim \min\cbr{\tr(\Sigmab_s), \tr(\Sigmab_w)}.
    \end{align*}
    In particular, when $\Sigmab_s$ and $\Sigmab_w$ admit low intrinsic dimensions, $\tr(\Sigmab_s \Sigmab_w)$ can be much smaller than $\min\cbr{\tr(\Sigmab_s), \tr(\Sigmab_w)}$ if their eigenvectors corresponding to the dominating eigenvalues are almost orthogonal.
\end{remark}

\begin{remark}[FT approximation errors]
    It is worth noting that under the ground truth and positive-definite covariance assumptions in \Cref{asm:ridge_regression}(i, iii), the FT approximation errors in \Cref{def:ft_est_err} satisfy
    \begin{align}\label{eq:pf_ridge_ft_approx_err}
    \begin{split}
        &\rho_s = \min_{\thetab \in \R^d} \E_{\xb \sim \Dcal}\sbr{(\phi_s(\xb)^\top \thetab - f_*(\xb))^2} = 0 \quad (\t{when } \thetab = \Sigmab_s^{-1} \Sigmab_* \thetab_*), \\
        &\rho_w = \min_{\thetab \in \R^d} \E_{\xb \sim \Dcal}\sbr{(\phi_w(\xb)^\top \thetab - f_*(\xb))^2} = 0 \quad (\t{when } \thetab = \Sigmab_w^{-1} \Sigmab_* \thetab_*).
    \end{split}
    \end{align}
    In place of \Cref{def:ft_est_err}, with positive-definite covariances in \Cref{asm:ridge_regression}, we measure the alignment between the ground truth feature $\phi_*$ and the weak/strong feature $\phi_w, \phi_s$ through
    \begin{align*}
        \varrho_s = \|\Sigmab_s^{-1/2} \Sigmab_*^{1/2} \thetab_*\|_2^2, \quad \varrho_w = \|\Sigmab_w^{-1/2} \Sigmab_*^{1/2} \thetab_*\|_2^2.
    \end{align*}
    Intuitively, for $\Sigmab_s$ and $\Sigmab_w$ with a few dominating eigenvalues (\Cref{asm:ridge_regression}(iii)), $\varrho_s$ and $\varrho_w$ are small if the eigensubspace associated with non-negligible eigenvalues of $\Sigmab_*$ is fully covered by the eigensubspaces associated with the dominating eigenvalues of $\Sigmab_s$ and $\Sigmab_w$, respectively. 
\end{remark}

The W2S FT under ridge regression consists of two steps.
\begin{enumerate}[label=(\alph*)]
    \item First, the weak teacher $f_w(\xb) = \phi_w(\xb)^\top \thetab_w$ is supervisedly finetuned over $\wt\Scal$: 
    \begin{align}\label{eq:w2s_weak_ridge}
        \thetab_w = \argmin_{\thetab \in \R^d} \frac{1}{n}\nbr{\wt\Phib_w \thetab - \wt\yb}_2^2 + \alpha_w \nbr{\thetab}_2^2, \quad \alpha_w > 0.
    \end{align}
    \item Then, the W2S model $f_\wts(\xb) = \phi_s(\xb)^\top \thetab_\wts$ is finetuned over the strong feature $\phi_s$ through the unlabeled samples $\Scal_x$ and their pseudo-labels generated by the weak teacher model:
    \begin{align}\label{eq:w2s_strong_ridge}
        \thetab_\wts = \argmin_{\thetab \in \R^d} \frac{1}{N}\nbr{\Phib_s \thetab - \Phib_w \thetab_w}_2^2 + \alpha_\wts \nbr{\thetab}_2^2, \quad \alpha_\wts > 0.
    \end{align}
\end{enumerate}

\begin{theorem}[W2S under ridge regression]\label{thm:w2s_ridge}
    Let $\varrho_w = \nbr{\Sigmab_w^{-1/2} \Sigmab_*^{1/2} \thetab_*}_2^2$ and $\varrho_s = \nbr{\Sigmab_s^{-1/2} \Sigmab_*^{1/2} \thetab_*}_2^2$.
    Under \Cref{asm:ridge_regression}, the generalization error of W2S FT via ridge regression with fixed $\alpha_w, \alpha_\wts > 0$, $\exrisk(f_\wts) = \vari(f_\wts) + \bias(f_\wts)$, is upper bounded by
    \begin{align*}
        \vari(f_\wts) \le \frac{\sigma^2 \tr\rbr{\Sigmab_s \Sigmab_w}}{4 (\alpha_w n) (\alpha_\wts N)}, \quad
        \bias(f_\wts) \le \alpha_w \varrho_w + \alpha_\wts \varrho_s.
    \end{align*}
    In particular, when taking  
    \begin{align*}
        \alpha_w = \rbr{\frac{\sigma^2 \tr\rbr{\Sigmab_s \Sigmab_w}}{4 n N}\ \frac{\varrho_s}{\varrho_w^2}}^{1/3}, \quad 
        \alpha_\wts = \rbr{\frac{\sigma^2 \tr\rbr{\Sigmab_s \Sigmab_w}}{4 n N}\ \frac{\varrho_w}{\varrho_s^2}}^{1/3},
    \end{align*}
    the excess risk of W2S FT is upper bounded by
    \begin{align*}
        \exrisk(f_\wts) \le 3 \rbr{\frac{\sigma^2 \tr\rbr{\Sigmab_s \Sigmab_w}}{4 n N}\ \varrho_s \varrho_w}^{1/3}.
    \end{align*}
\end{theorem}

\Cref{thm:w2s_ridge} conveys a similar high-level intuition as in \Cref{thm:w2s_ft} regarding the effect of the weak-strong similarity on the generalization error of W2S FT. In particular, the larger discrepancy between $\phi_s$ and $\phi_w$ (corresponding to the smaller $\tr\rbr{\Sigmab_s \Sigmab_w}$) leads to lower variance and therefore better W2S generalization.

Meanwhile, a key difference in W2S between the ridge and ridgeless settings (\Cref{thm:w2s_ridge} versus \Cref{thm:w2s_ft}) is that the FT approximation errors in \Cref{thm:w2s_ridge}, reflected by $\varrho_s = \|\Sigmab_s^{-1/2} \Sigmab_*^{1/2} \thetab_*\|_2^2$ and $\varrho_w = \|\Sigmab_w^{-1/2} \Sigmab_*^{1/2} \thetab_*\|_2^2$, can be compensated by larger sample sizes $n, N$ and directly affect the sample complexity: 
\begin{align*}
    n N \asymp \sigma^2 \tr\rbr{\Sigmab_s \Sigmab_w} \varrho_s \varrho_w.
\end{align*}
Such difference is a result of optimizing the regularization hyperparameters $\alpha_w, \alpha_\wts$ in ridge regression that control the variance-bias tradeoff.

\begin{proof}[Proof of \Cref{thm:w2s_ridge}]
    We first formalize some useful facts on the features and labels as in \eqref{eq:pf_var_w2s_subgaussian_asm}.
    In particular, the sub-gaussian assumption in \Cref{asm:ridge_regression}(ii) implies that for each $\xb \sim \Dcal$, the corresponding strong/weak feature $\phi_s(\xb), \phi_w(\xb) \in \R^d$, and the ground truth $f_*(\xb) \in \R$ are simultaneously characterized by an independent subgaussian random vector $\gammab \in \R^d$ with $\E[\gammab] = \b0_{d}$ and $\E[\gammab \gammab^\top] = \Ib_{d}$, \ie,
    \begin{align*}
        \phi_s(\xb) = \Sigmab_s^{1/2} \gammab, \quad \phi_w(\xb) = \Sigmab_w^{1/2} \gammab, \quad f_*(\xb) = \phi_*(\xb)^\top \thetab_* = \gammab^\top \Sigmab_*^{1/2} \thetab_*.
    \end{align*}

    Then, for $\Scal$ and $\wt\Scal$, there exist independent random matrices $\Gammab = [\gammab_1, \ldots, \gammab_N]^\top \in \R^{N \times d}$ and $\wt\Gammab = [\wt\gammab_1, \ldots, \wt\gammab_n]^\top \in \R^{n \times d}$ consisting of $\iid$ zero-mean isotropic rows such that
    \begin{align}\label{eq:pf_var_w2s_subgaussian_asm_2}
    \begin{split}
        &\Phib_s = \Gammab \Sigmab_s^{1/2} = \Gammab_s \Lambdab_s^{1/2} \Vb_s^\top, \\
        &\Phib_w = \Gammab \Sigmab_w^{1/2} = \Gammab_w \Lambdab_w^{1/2} \Vb_w^\top, \\
        &\yb = \fb_* + \zb, \quad \fb_* = \Gammab \Sigmab_*^{1/2} \thetab_*, \quad \zb \sim \Ncal(\b0_N, \sigma^2 \Ib_N), \\
        &\wt\Phib_w = \wt\Gammab \Sigmab_w^{1/2} = \wt\Gammab_w \Lambdab_w^{1/2} \Vb_w^\top, \\
        &\wt\yb = \wt\fb_* + \wt\zb, \quad \wt\fb_* = \wt\Gammab \Sigmab_*^{1/2} \thetab_*, \quad \wt\zb \sim \Ncal(\b0_n, \sigma^2 \Ib_n),
    \end{split}
    \end{align}
    where $\Gammab_s = \Gammab \Vb_s$, $\Gammab_w = \Gammab \Vb_w$, and $\wt\Gammab_w = \wt\Gammab \Vb_w$.

    \paragraph{Variance-bias decomposition.}
    Recall that the excess risk of W2S generalization $\exrisk(f_\wts)$ can be decomposed into the variance and bias terms:
    \begin{align*}
        &\vari(f_\wts) = \E_{\xb \sim \Dcal}\sbr{\E_{\Scal_x, \wt\Scal}\sbr{(f_\wts(\xb) - \E_{\Scal_x, \wt\Scal}[f_\wts(\xb)])^2}}, \\
        &\bias(f) = \E_{\xb \sim \Dcal}\sbr{(\E_{\Scal_x, \wt\Scal}[f_\wts(\xb)] - f_*(\xb))^2}.
    \end{align*}
    With $\alpha_w > 0$, \eqref{eq:w2s_weak_ridge} yields a weak teacher model $f_w(\xb) = \phi_w(\xb)^\top \thetab_w$ with 
    \begin{align*}
        \thetab_w = \rbr{\wt\Phib_w^\top \wt\Phib_w + \alpha_w n \Ib_d}^{-1} \wt\Phib_w^\top \rbr{\wt\fb_8 + \wt\zb}.
    \end{align*}
    Then, the W2S model $f_\wts(\xb) = \phi_s(\xb)^\top \thetab_\wts$ is given by \eqref{eq:w2s_strong_ridge} with $\alpha_\wts > 0$:
    \begin{align*}
        \thetab_\wts = &\rbr{\Phib_s^\top \Phib_s + \alpha_\wts N \Ib_d}^{-1} \Phib_s^\top \Phib_w \thetab_w \\
        = &\rbr{\Phib_s^\top \Phib_s + \alpha_\wts N \Ib_d}^{-1} \Phib_s^\top \Phib_w \rbr{\wt\Phib_w^\top \wt\Phib_w + \alpha_w n \Ib_d}^{-1} \wt\Phib_w^\top \rbr{\wt\fb_* + \wt\zb},
    \end{align*}
    which implies
    \begin{align*}
        \E_{\Scal_x, \wt\Scal}[\thetab_\wts] = \rbr{\Phib_s^\top \Phib_s + \alpha_\wts N \Ib_d}^{-1} \Phib_s^\top \Phib_w \rbr{\wt\Phib_w^\top \wt\Phib_w + \alpha_w n \Ib_d}^{-1} \wt\Phib_w^\top \wt\fb_*.
    \end{align*}
    Then, we can concretize the variance and bias terms as:
    \begin{align}\label{eq:pf_ridge_var}
    \begin{split}
        &\vari(f_\wts) = \E_{\xb \sim \Dcal}\sbr{\E_{\Scal_x, \wt\Scal}\sbr{(f_\wts(\xb) - \E_{\Scal_x, \wt\Scal}[f_\wts(\xb)])^2}} \\
        = &\E_{\Scal_x, \wt\Scal}\sbr{\nbr{\Sigmab_s^{1/2} \rbr{\Phib_s^\top \Phib_s + \alpha_\wts N \Ib_d}^{-1} \Phib_s^\top \Phib_w \rbr{\wt\Phib_w^\top \wt\Phib_w + \alpha_w n \Ib_d}^{-1} \wt\Phib_w^\top \wt\zb}_2^2},
    \end{split}
    \end{align}
    and
    \begin{align}\label{eq:pf_ridge_bias}
    \begin{split}
        &\bias(f_\wts) = \E_{\xb \sim \Dcal}\sbr{(\E_{\Scal_x, \wt\Scal}[f_\wts(\xb)] - f_*(\xb))^2} \\
        = &\E_{\Scal_x, \wt\Scal}\sbr{\frac{1}{N} \nbr{\Phib_s \rbr{\Phib_s^\top \Phib_s + \alpha_\wts N \Ib_d}^{-1} \Phib_s^\top \Phib_w \rbr{\wt\Phib_w^\top \wt\Phib_w + \alpha_w n \Ib_d}^{-1} \wt\Phib_w^\top \wt\fb_* - \fb_*}_2^2}.
    \end{split}
    \end{align}
    Now, we are ready to upper bound the variance and bias terms separately.

    \paragraph{Variance.}
    Denote $\zetab = \Lambdab_w^{1/2} \Vb_w^\top \rbr{\wt\Phib_w^\top \wt\Phib_w + \alpha_w n \Ib_d}^{-1} \wt\Phib_w^\top \wt\zb \in \R^d$, whose randomness comes from $\wt\Scal$ only, independent of $\Scal_x$.
    Then, the variance term \eqref{eq:pf_ridge_var} can be expressed as
    \begin{align*}
        &\vari(f_\wts) = \E_{\Scal_x, \wt\Scal}\sbr{\nbr{\Sigmab_s^{1/2} \rbr{\Phib_s^\top \Phib_s + \alpha_\wts N \Ib_d}^{-1} \Phib_s^\top \Phib_w \zetab}_2^2} \\
        = &\tr\rbr{\E_{\Scal_s}\rbr{\Gammab_w^\top \Phib_s \rbr{\Phib_s^\top \Phib_s + \alpha_\wts N \Ib_d}^{-1} \Sigmab_s \rbr{\Phib_s^\top \Phib_s + \alpha_\wts N \Ib_d}^{-1} \Phib_s^\top \Gammab_w} \E_{\wt\Scal}\sbr{\zetab \zetab^\top}} \\
        = &\tr\rbr{\E_{\Scal_s}\rbr{\Gammab_w^\top \Gammab_s \rbr{\Gammab_s^\top \Gammab_s + \alpha_\wts N \Lambdab_s^{-1}}^{-1} \rbr{\Gammab_s^\top \Gammab_s + \alpha_\wts N \Lambdab_s^{-1}}^{-1} \Gammab_s^\top \Gammab_w} \E_{\wt\Scal}\sbr{\zetab \zetab^\top}} \\
        = &\tr\rbr{\E_{\Scal_s}\rbr{\Vb_w^\top \Gammab^\top \Gammab \rbr{\Gammab^\top \Gammab + \alpha_\wts N \Sigmab_s^{-1}}^{-2} \Gammab^\top \Gammab \Vb_w} \E_{\wt\Scal}\sbr{\zetab \zetab^\top}} \\
        = &\tr\rbr{\E_{\Scal_s}\rbr{\Gammab^\top \Gammab \rbr{\Gammab^\top \Gammab + \alpha_\wts N \Sigmab_s^{-1}}^{-2} \Gammab^\top \Gammab} \E_{\wt\Scal}\sbr{\Vb_w \zetab \zetab^\top \Vb_w^\top}}.
    \end{align*}
    Notice that $\rbr{\Gammab^\top \Gammab + \alpha_\wts N \Sigmab_s^{-1}}^{2} \succeq \alpha_\wts N \rbr{\Gammab^\top \Gammab \Sigmab_s^{-1} + \Sigmab_s^{-1} \Gammab^\top \Gammab}$.
    Since matrix inversion is convex, a Jensen-type inequality implies that
    \begin{align*}
        &\Gammab^\top \Gammab \rbr{\Gammab^\top \Gammab + \alpha_\wts N \Sigmab_s^{-1}}^{-2} \Gammab^\top \Gammab \\
        \preceq &\Gammab^\top \Gammab \rbr{\alpha_\wts N \rbr{\Gammab^\top \Gammab \Sigmab_s^{-1} + \Sigmab_s^{-1} \Gammab^\top \Gammab}}^{\dagger} \Gammab^\top \Gammab \\
        = &\frac{1}{2 \alpha_\wts N} \Gammab^\top \Gammab \rbr{\frac{1}{2} \rbr{\Gammab^\top \Gammab \Sigmab_s^{-1} + \Sigmab_s^{-1} \Gammab^\top \Gammab}}^{\dagger} \Gammab^\top \Gammab \\
        \preceq &\frac{1}{4 \alpha_\wts N} \rbr{\Gammab^\top \Gammab \Sigmab_s + \Sigmab_s \Gammab^\top \Gammab}.
    \end{align*}
    Therefore, 
    \begin{align*}
        \E_{\Scal_s}\rbr{\Gammab^\top \Gammab \rbr{\Gammab^\top \Gammab + \alpha_\wts N \Sigmab_s^{-1}}^{-2} \Gammab^\top \Gammab}
        \preceq &\frac{1}{4 \alpha_\wts N} \E_{\Scal_s}\sbr{\Gammab^\top \Gammab \Sigmab_s + \Sigmab_s \Gammab^\top \Gammab} 
        = \frac{1}{2 \alpha_\wts N} \Sigmab_s.
    \end{align*}
    Meanwhile, we observe that
    \begin{align*}
        \E_{\wt\Scal}\sbr{\Vb_w \zetab \zetab^\top \Vb_w^\top} 
        = &\E_{\wt\Scal}\sbr{\Sigmab_w^{1/2} \rbr{\wt\Phib_w^\top \wt\Phib_w + \alpha_w n \Ib_d}^{-1} \wt\Phib_w^\top \wt\zb \wt\zb^\top \wt\Phib_w \rbr{\wt\Phib_w^\top \wt\Phib_w + \alpha_w n \Ib_d}^{-1} \Sigmab_w^{1/2}} \\
        = &\sigma^2 \E_{\wt\Scal}\sbr{\Sigmab_w^{1/2} \rbr{\wt\Phib_w^\top \wt\Phib_w + \alpha_w n \Ib_d}^{-1} \wt\Phib_w^\top \wt\Phib_w \rbr{\wt\Phib_w^\top \wt\Phib_w + \alpha_w n \Ib_d}^{-1} \Sigmab_w^{1/2}},
    \end{align*}
    where 
    \begin{align*}
        \rbr{\wt\Phib_w^\top \wt\Phib_w + \alpha_w n \Ib_d}^{-1} \wt\Phib_w^\top \wt\Phib_w \rbr{\wt\Phib_w^\top \wt\Phib_w + \alpha_w n \Ib_d}^{-1}
        \preceq &\frac{1}{2 \alpha_w n} \Ib_d.
    \end{align*}
    Therefore, we have
    \begin{align*}
        \E_{\wt\Scal}\sbr{\Vb_w \zetab \zetab^\top \Vb_w^\top} 
        \preceq &\sigma^2 \E_{\wt\Scal}\sbr{\Sigmab_w^{1/2} \rbr{\frac{1}{2 \alpha_w n} \Ib_d} \Sigmab_w^{1/2}}
        = \frac{\sigma^2}{2 \alpha_w n} \Sigmab_w.
    \end{align*}
    Overall, the variance of $f_\wts$ can be upper bounded as
    \begin{align}\label{eq:pf_ridge_var_ub}
    \begin{split}
        \vari(f_\wts) 
        = &\tr\rbr{\E_{\Scal_s}\rbr{\Gammab^\top \Gammab \rbr{\Gammab^\top \Gammab + \alpha_\wts N \Sigmab_s^{-1}}^{-2} \Gammab^\top \Gammab} \E_{\wt\Scal}\sbr{\Vb_w \zetab \zetab^\top \Vb_w^\top}} \\
        \le &\frac{\sigma^2 \tr\rbr{\Sigmab_s \Sigmab_w}}{4 (\alpha_w n) (\alpha_\wts N)}.
    \end{split}
    \end{align}

    \paragraph{Bias.}
    Let $\xib = \Sigmab_w^{1/2} \rbr{\wt\Phib_w^\top \wt\Phib_w + \alpha_w n \Ib_d}^{-1} \wt\Phib_w^\top \wt\fb_* \in \R^d$, whose randomness comes from $\wt\Scal$ only, independent of $\Scal_x$.
    Recall from \eqref{eq:pf_ridge_bias}, the bias term \eqref{eq:pf_ridge_bias} can be decomposed as
    \begin{align*}
        &\bias(f_\wts) = \E_{\Scal_x, \wt\Scal}\sbr{\frac{1}{N} \nbr{\Phib_s \rbr{\Phib_s^\top \Phib_s + \alpha_\wts N \Ib_d}^{-1} \Phib_s^\top \Phib_w \rbr{\wt\Phib_w^\top \wt\Phib_w + \alpha_w n \Ib_d}^{-1} \wt\Phib_w^\top \wt\fb_* - \fb_*}_2^2}\\
        &= \E_{\Scal_x, \wt\Scal}\sbr{\frac{1}{N} \rbr{\nbr{\Phib_s \rbr{\Phib_s^\top \Phib_s + \alpha_\wts N \Ib_d}^{-1} \Phib_s^\top \Gammab \xib - \Phib_s \Phib_s^\dagger \fb_*}_2^2 + \nbr{\rbr{\Ib_N - \Phib_s \Phib_s^\dagger} \fb_*}_2^2}},
    \end{align*}
    where by \Cref{lem:low_est_err_ft} and \eqref{eq:pf_ridge_ft_approx_err}
    \begin{align*}
        \E_{\Scal_x}\sbr{\frac{1}{N} \nbr{\rbr{\Ib_N - \Phib_s \Phib_s^\dagger} \fb_*}_2^2}
        = \frac{\rho_s(N)}{N} \le \rho_s = 0.
    \end{align*}
    Therefore, with $\xib = \Sigmab_w^{1/2} \rbr{\wt\Phib_w^\top \wt\Phib_w + \alpha_w n \Ib_d}^{-1} \wt\Phib_w^\top \wt\fb_*$, we have
    \begin{align*}
        \bias(f_\wts) = \E_{\Scal_x, \wt\Scal}\sbr{\frac{1}{N} \nbr{\Phib_s \rbr{\Phib_s^\top \Phib_s + \alpha_\wts N \Ib_d}^{-1} \Phib_s^\top \Gammab \xib - \Phib_s \Phib_s^\dagger \fb_*}_2^2}.
    \end{align*}
    Recall that $\fb_* = \Gammab \Sigmab_*^{1/2} \thetab_*$ and $\Phib_s = \Gammab \Sigmab_s^{1/2} = \Gammab_s \Lambdab_s^{1/2} \Vb_s^\top$.
    Then, we can express the bias term as
    \begin{align*}
        \bias(f_\wts) = &\E_{\Scal_x, \wt\Scal}\sbr{\frac{1}{N} \nbr{\Gammab\rbr{\Gammab^\top \Gammab + \alpha_\wts N \Sigmab_s^{-1}}^{-1} \Gammab^\top \Gammab \xib - \Gammab \Gammab^\dagger \fb_*}_2^2} \\
        = &\E_{\Scal_x, \wt\Scal}\sbr{\frac{1}{N} \nbr{\Gammab \Sigmab_*^{1/2} \thetab_* - \Gammab\rbr{\Gammab^\top \Gammab + \alpha_\wts N \Sigmab_s^{-1}}^{-1} \Gammab^\top \Gammab \xib}_2^2} \\
        = &\E_{\Scal_x, \wt\Scal}\sbr{\frac{1}{N} \nbr{\Gammab \rbr{\Sigmab_*^{1/2} \thetab_* - \xib} + \Gammab \rbr{\Ib_d - \rbr{\Gammab^\top \Gammab + \alpha_\wts N \Sigmab_s^{-1}}^{-1} \Gammab^\top \Gammab} \xib}_2^2} \\
    \end{align*} 
    By Woodbury matrix identity, we have
    \begin{align}\label{eq:pf_ridge_bias_woodbury}
        \Ib_d - \rbr{\Gammab^\top \Gammab + \alpha_\wts N \Sigmab_s^{-1}}^{-1} \Gammab^\top \Gammab
        = \rbr{\Ib_d + \frac{1}{\alpha_\wts N} \Sigmab_s \Gammab^\top \Gammab}^{-1}.
    \end{align}
    Therefore, we have 
    \begin{align}\label{eq:pf_ridge_bias_inter1}
        \bias(f_\wts) = \E_{\Scal_x, \wt\Scal}\Bigg[\frac{1}{N} \Big\|\underbrace{\Gammab \rbr{\Sigmab_*^{1/2} \thetab_* - \xib}}_{\t{Term A}} + \underbrace{\Gammab \rbr{\Ib_d + \frac{1}{\alpha_\wts N} \Sigmab_s \Gammab^\top \Gammab}^{-1} \xib}_{\t{Term B}}\Big\|_2^2\Bigg].
    \end{align}

    For Term A, notice that $\xib = \Sigmab_w^{1/2} \rbr{\wt\Phib_w^\top \wt\Phib_w + \alpha_w n \Ib_d}^{-1} \wt\Phib_w^\top \wt\fb_*$ implies
    \begin{align*}
        \Sigmab_*^{1/2} \thetab_* - \xib 
        = &\Sigmab_*^{1/2} \thetab_* - \Sigmab_w^{1/2} \rbr{\wt\Phib_w^\top \wt\Phib_w + \alpha_w n \Ib_d}^{-1} \wt\Phib_w^\top \wt\fb_* \\
        = &\Sigmab_*^{1/2} \thetab_* - \rbr{\wt\Gammab^\top \wt\Gammab + \alpha_w n \Sigmab_w^{-1}}^{-1} \wt\Gammab^\top \wt\Gammab \Sigmab_*^{1/2} \thetab_* \\
        = &\rbr{\Ib_d - \rbr{\wt\Gammab^\top \wt\Gammab + \alpha_w n \Sigmab_w^{-1}}^{-1} \wt\Gammab^\top \wt\Gammab} \Sigmab_*^{1/2} \thetab_* \\
        = &\rbr{\Ib_d + \frac{1}{\alpha_w n} \Sigmab_w \wt\Gammab^\top \wt\Gammab}^{-1} \Sigmab_*^{1/2} \thetab_*,
    \end{align*}
    where the last equality follows from Woodbury matrix identity as in \eqref{eq:pf_ridge_bias_woodbury}.
    Therefore,
    \begin{align*}
        \E_{\Scal_x, \wt\Scal}\sbr{\frac{1}{N} \nbr{\Gammab \rbr{\Sigmab_*^{1/2} \thetab_* - \xib}}_2^2} 
        = &\E_{\wt\Scal}\sbr{\frac{1}{n} \nbr{\wt\Gammab \rbr{\Sigmab_*^{1/2} \thetab_* - \xib}}_2^2} \\
        = &\E_{\wt\Scal}\sbr{\frac{1}{n} \nbr{\wt\Gammab \rbr{\Ib_d + \frac{1}{\alpha_w n} \Sigmab_w \wt\Gammab^\top \wt\Gammab}^{-1} \Sigmab_*^{1/2} \thetab_*}_2^2}.
    \end{align*}
    Since 
    \begin{align*}
        \rbr{\Ib_d + \frac{1}{\alpha_w n} \Sigmab_w \wt\Gammab^\top \wt\Gammab}^{-1} \wt\Gammab^\top \wt\Gammab \rbr{\Ib_d + \frac{1}{\alpha_w n} \Sigmab_w \wt\Gammab^\top \wt\Gammab}^{-1} \preceq \frac{\alpha_w n}{2} \Sigmab_w^{-1},
    \end{align*}
    we have
    \begin{align}\label{eq:pf_ridge_bias_term1}
    \begin{split}
        \E_{\Scal_x, \wt\Scal}\sbr{\frac{1}{N} \nbr{\Gammab \rbr{\Sigmab_*^{1/2} \thetab_* - \xib}}_2^2} 
        \le &\frac{1}{n} \tr\rbr{\frac{\alpha_w n}{2} \Sigmab_w^{-1} \Sigmab_*^{1/2} \thetab_* \thetab_*^\top \Sigmab_*^{1/2}} \\
        = &\frac{\alpha_w}{2} \nbr{\Sigmab_w^{-1/2} \Sigmab_*^{1/2} \thetab_*}_2^2.
    \end{split}
    \end{align}
    
    For Term B, leveraging Woodbury matrix identity as in \eqref{eq:pf_ridge_bias_woodbury}, we notice that 
    \begin{align*}
        &\E_{\Scal_x, \wt\Scal}\sbr{\frac{1}{N} \nbr{\Gammab \rbr{\Ib_d + \frac{1}{\alpha_\wts N} \Sigmab_s \Gammab^\top \Gammab}^{-1} \xib}_2^2} 
        \le \E_{\Scal_x, \wt\Scal}\sbr{\frac{1}{N} \tr\rbr{\frac{\alpha_\wts N}{2} \Sigmab_s^{-1} \xib \xib^\top}} \\
        = &\frac{\alpha_\wts}{2} \E_{\Scal_x, \wt\Scal}\sbr{\nbr{\Sigmab_s^{-1/2} \Sigmab_w^{1/2} \rbr{\wt\Phib_w^\top \wt\Phib_w + \alpha_w n \Ib_d}^{-1} \wt\Phib_w^\top \wt\fb_*}_2^2} \\
        = &\frac{\alpha_\wts}{2} \E_{\Scal_x, \wt\Scal}\sbr{\nbr{\Sigmab_s^{-1/2} \rbr{\wt\Gammab^\top \wt\Gammab + \alpha_w n \Sigmab_w^{-1}}^{-1} \wt\Gammab^\top \wt\Gammab \Sigmab_*^{1/2} \thetab_*}_2^2}
    \end{align*}
    Since $\rbr{\wt\Gammab^\top \wt\Gammab + \alpha_w n \Sigmab_w^{-1}}^{-1} \wt\Gammab^\top \wt\Gammab \preceq \Ib_d$, we know that
    \begin{align}\label{eq:pf_ridge_bias_term2}
    \begin{split}
        \E_{\Scal_x, \wt\Scal}\sbr{\frac{1}{N} \nbr{\Gammab \rbr{\Ib_d + \frac{1}{\alpha_\wts N} \Sigmab_s \Gammab^\top \Gammab}^{-1} \xib}_2^2} 
        \le \frac{\alpha_\wts}{2} \nbr{\Sigmab_s^{-1/2} \Sigmab_*^{1/2} \thetab_*}_2^2.
    \end{split}
    \end{align}
    Combining \eqref{eq:pf_ridge_bias_inter1}, \eqref{eq:pf_ridge_bias_term1}, and \eqref{eq:pf_ridge_bias_term2}, we can upper bound the bias term as
    \begin{align}\label{eq:pf_ridge_bias_final}
    \begin{split}
        &\bias(f_\wts) = \E_{\Scal_x, \wt\Scal}\Bigg[\frac{1}{N} \Big\|\underbrace{\Gammab \rbr{\Sigmab_*^{1/2} \thetab_* - \xib}}_{\t{Term A}} + \underbrace{\Gammab \rbr{\Ib_d + \frac{1}{\alpha_\wts N} \Sigmab_s \Gammab^\top \Gammab}^{-1} \xib}_{\t{Term B}}\Big\|_2^2\Bigg] \\
        \le &2 \E_{\Scal_x, \wt\Scal}\sbr{\frac{1}{N} \nbr{\Gammab \rbr{\Sigmab_*^{1/2} \thetab_* - \xib}}_2^2} + 2 \E_{\Scal_x, \wt\Scal}\sbr{\frac{1}{N} \nbr{\Gammab \rbr{\Ib_d + \frac{1}{\alpha_\wts N} \Sigmab_s \Gammab^\top \Gammab}^{-1} \xib}_2^2} \\
        \le &\alpha_w \nbr{\Sigmab_w^{-1/2} \Sigmab_*^{1/2} \thetab_*}_2^2 + \alpha_\wts \nbr{\Sigmab_s^{-1/2} \Sigmab_*^{1/2} \thetab_*}_2^2.
    \end{split}
    \end{align}
    
    \paragraph{Variance-bias tradeoff.}
    Overall, by \eqref{eq:pf_ridge_var_ub} and \eqref{eq:pf_ridge_bias_final}, we have
    \begin{align*}
        &\vari(f_\wts) \le \frac{\sigma^2 \tr\rbr{\Sigmab_s \Sigmab_w}}{4 (\alpha_w n) (\alpha_\wts N)}, \\
        &\bias(f_\wts) \le \alpha_w \nbr{\Sigmab_w^{-1/2} \Sigmab_*^{1/2} \thetab_*}_2^2 + \alpha_\wts \nbr{\Sigmab_s^{-1/2} \Sigmab_*^{1/2} \thetab_*}_2^2.
    \end{align*}
    The upper bound the excess risk $\exrisk(f_\wts) = \vari(f_\wts) + \bias(f_\wts)$ is minimized by taking 
    \begin{align*}
        \alpha_w = \rbr{\frac{\sigma^2 \tr\rbr{\Sigmab_s \Sigmab_w}}{4 n N}\ \frac{\nbr{\Sigmab_s^{-1/2} \Sigmab_*^{1/2} \thetab_*}_2^2}{\nbr{\Sigmab_w^{-1/2} \Sigmab_*^{1/2} \thetab_*}_2^4}}^{1/3}, \ 
        \alpha_\wts = \rbr{\frac{\sigma^2 \tr\rbr{\Sigmab_s \Sigmab_w}}{4 n N}\ \frac{\nbr{\Sigmab_w^{-1/2} \Sigmab_*^{1/2} \thetab_*}_2^2}{\nbr{\Sigmab_s^{-1/2} \Sigmab_*^{1/2} \thetab_*}_2^4}}^{1/3},
    \end{align*}
    which leads to the optimal upper bound for the excess risk:
    \begin{align*}
        \exrisk(f_\wts) \le 3 \rbr{\frac{\sigma^2 \tr\rbr{\Sigmab_s \Sigmab_w}}{4 n N}\ \nbr{\Sigmab_s^{-1/2} \Sigmab_*^{1/2} \thetab_*}_2^2 \nbr{\Sigmab_w^{-1/2} \Sigmab_*^{1/2} \thetab_*}_2^2}^{1/3}.
    \end{align*}
\end{proof}






\section{Canonical angles}\label{apx:canonical_angles}
In this section, we review the concept of canonical angles between two subspaces that connect the formal definition of the correlation dimension $d_{s \wedge w} = \nbr{\Vb_s^\top \Vb_w}_F^2$ in \Cref{def:correlation_dim} to the intuitive notion of the alignment between the corresponding subspaces $\Vcal_s$ and $\Vcal_w$ in the introduction: $\sum \cos(\angle(\Vcal_s, \Vcal_w)) = \nbr{\Vb_s^\top \Vb_w}_F^2$.
\begin{definition}[Canonical angles \cite{golub2013matrix}, adapting from \cite{dong2024efficient}]\label{def:canonical_angles}
    Let $\Vcal_s,\Vcal_w \subseteq \R^d$ be two subspaces with dimensions $\dim\rbr{\Vcal_s}=d_s$ and $\dim\rbr{\Vcal_w}=d_w$ (assuming $d_w \geq d_s$ without loss of generality). The canonical angles $\angle\rbr{\Vcal_s,\Vcal_w}=\diag\rbr{\nu_1,\dots,\nu_{d_s}}$ are $d_s$ angles that jointly measure the alignment between $\Vcal_s$ and $\Vcal_w$, defined recursively as follows:
    \begin{align*}
        &\ub_i, \vb_i ~\triangleq~
        \argmax~\ub_i^*\vb_i \\
        \t{s.t.}~
        &\ub_i \in \rbr{\Vcal_s \setminus \spn\cbr{\ub_{\iota}}_{\iota=1}^{i-1}} \cap \SSS^{d-1},\\ 
        &\vb_i \in \rbr{\Vcal_w \setminus \spn\cbr{\vb_{\iota}}_{\iota=1}^{i-1}} \cap \SSS^{d-1}\\
        &\cos (\nu_i) = \ub_i^* \vb_i \quad \forall~ i=1,\dots,k,
    \end{align*}
    such that $0 \leq \nu_1 \leq \dots \leq \nu_k \leq \pi/2$.

    Given two subspaces $\Vcal_s,\Vcal_w \subseteq \R^d$, let $\Vb_s \in \R^{d \times d_s}$ and $\Vb_w \in \R^{d \times d_w}$ be the matrices whose columns form orthonormal bases for $\Vcal_s$ and $\Vcal_w$, respectively. Then, the canonical angles $\angle(\Vcal_s, \Vcal_w)$ are determined by the singular values of $\Vb_s^\top \Vb_w$~\citep[\S 3]{bjorck1973numerical}:
    \begin{align*}
        \cos(\angle_i(\Vcal_s, \Vcal_w)) = \sigma_i(\Vb_s^\top \Vb_w) \quad \forall~ i=1,\dots,d_s,
    \end{align*}
    where $\sigma_i(\Vb_s^\top \Vb_w)$ denotes the $i$-th singular value of $\Vb_s^\top \Vb_w$.
\end{definition}

In particular, since $\Vb_s, \Vb_w$ consist of orthonormal columns, the singular values of $\Vb_s^\top \Vb_w$ fall in $[0,1]$, and therefore,
\begin{align*}
    d_{s \wedge w} = \sum \cos(\angle(\Vcal_s, \Vcal_w)) = \nbr{\Vb_s^\top \Vb_w}_F^2 \in [0, \min\cbr{d_s, d_w}].
\end{align*}




\section{Additional experiments}\label{apx:exp_details}

\subsection{Additional experiments and details on UTKFace regression}\label{apx:exp_img_reg}
This section provides some additional details and results for the UTKFace regression experiments in \Cref{sec:exp_img_reg}. 

\begin{figure}[!h]
    \centering
    \includegraphics[width=\columnwidth]{fig/mse_utkface_resnet18_clipb32.pdf}%\vspace{-2em}
    \caption{Scaling for MSE on UTKFace with \texttt{CLIP-B32} as the strong student and \texttt{ResNet18} as the weak teacher}\label{fig:mse_utkface_resnet18-clip}
\end{figure}

\begin{figure}[!h]
    \centering
    \includegraphics[width=\columnwidth]{fig/mse_utkface_resnet50_clipb32.pdf}%\vspace{-2em}
    \caption{Scaling for MSE on UTKFace with \texttt{CLIP-B32} as the strong student and \texttt{ResNet50} as the weak teacher}\label{fig:mse_utkface_resnet50-clip}
\end{figure}

\begin{figure}[!h]
    \centering
    \includegraphics[width=\columnwidth]{fig/mse_utkface_resnet152_clipb32.pdf}%\vspace{-2em}
    \caption{Scaling for MSE on UTKFace with \texttt{CLIP-B32} as the strong student and \texttt{ResNet152} as the weak teacher}\label{fig:mse_utkface_resnet152-clip}
\end{figure}

We summarize the relevant dimensionality in \Cref{tab:img_reg_dim}. We observe the following:
\begin{itemize}
    \item The intrinsic dimensions of the pretrained features are significantly smaller than the ambiance feature dimensions, which is consistent with our theoretical analysis and the empirical observations in \cite{aghajanyan2020intrinsic}. 
    \item The correlation dimensions $d_{s \wedge w}$ are considerably smaller than the corresponding intrinsic dimensions, indicating that the subspaces spanned by the weak and strong features are not aligned in practice. As shown in \Cref{sec:exp_img_reg}, such discrepancies in the weak and strong features facilitate W2S generalization.
\end{itemize}

\begin{table}[!ht]
    \centering
    \caption{Summary of the pretrained feature dimensions, along with the intrinsic dimensions $d_s, d_w$ and correlation dimensions $d_{s \wedge w}$ (with respect to the strong student \texttt{CLIP-B32}) computed over the entire UTKFace dataset (including training and testing).}\label{tab:img_reg_dim}
    \begin{tabular}{c|ccc}
        \toprule
        Pretrained Model & Feature Dimension & Intrinsic Dimension ($\tau=0.01$) & Correlation Dimension \\
        \midrule
        \texttt{ResNet18} & 512 & 194 & 167.64 \\
        \texttt{ResNet34} & 512 & 150 & 129.97 \\
        \texttt{ResNet50} & 2048 & 522 & 301.06 \\
        \texttt{ResNet101} & 2048 & 615 & 354.52 \\
        \texttt{ResNet152} & 2048 & 589 & 339.90 \\
        \midrule
        \texttt{CLIP-B32} & 768 & 443 & $\times$ \\
        \bottomrule
    \end{tabular}
\end{table}

For reference, we provide the scaling for MSE losses of three representative teacher-student pairs in \Cref{fig:mse_utkface_resnet18-clip,fig:mse_utkface_resnet50-clip,fig:mse_utkface_resnet152-clip}. 
\begin{itemize}
    \item It is worth highlighting that while the MSE loss of $f_\wts$ monotonically decreases with respect to both sample sizes $n,N$, the different rates of convergence compared to $f_w$, $f_s$, and $f_c$ lead to the distinct scaling behavior of the relative W2S performance ($\pgr$ and $\opr$) with respect to $n$ versus $N$ in \Cref{fig:pgr_opr_utkface_resnet-clip,fig:pgr_opr_utkface_vardom_resnet-clip}.
    \item When the strong student has a lower intrinsic dimension than the weak teacher (\cf \Cref{fig:mse_utkface_resnet18-clip} versus \Cref{fig:mse_utkface_resnet50-clip,fig:mse_utkface_resnet152-clip}), $d_s < d_w$, the W2S model $f_\wts$ tends to achieve better generalization in terms of the test MSE. This is consistent with our analysis in \Cref{sec:generalization_errors}.
    \item When $d_s < d_w$, the W2S model $f_\wts$ tends to achieve (slightly) better generalization for (slightly) smaller correlation dimension $d_{s \wedge w}$ (\cf \Cref{fig:mse_utkface_resnet50-clip} versus \Cref{fig:mse_utkface_resnet152-clip}), again coinciding with our analysis in \Cref{sec:generalization_errors}.
    \item W2S generalization generally happens (\ie $f_\wts$ is able to outperform $f_w$) with sufficiently large sample sizes $n, N$. However, as the labeled sample size $n$ increases, the test MSE of $f_\wts$ converges slower than that of the strong baseline and ceiling models, $f_s$ and $f_c$, leading to the inverse scaling for $\pgr$ and $\opr$ with respect to $n$ in \Cref{fig:pgr_opr_utkface_resnet-clip,fig:pgr_opr_utkface_vardom_resnet-clip}. When $n$ is too large, the W2S model $f_\wts$ may not be able to achieve better generalization than the strong baseline $f_s$.
\end{itemize}




\subsection{Experiments on image classification}\label{apx:exp_img_cls}

\paragraph{Dataset.} ColoredMNIST \citep{arjovsky2019invariant} consists of groups of different colors and reassign the label to be binary (digits 0-4 vs 5-9). We pool together the groups to form one dataset. The choice is to bring diversity to the feature space with additional color features and thus potential feature discrepancies. We hold out a test set of 7000 samples and used the rest 63000 samples for training.

\paragraph{Linear probing over pretrained features.} We fix a strong student as DINOv2-s14 \citep{oquab2023dinov2} and vary the weak teacher among the ResNet-d series and ResNet series (ResNet18D, ResNet34D, ResNet101, ResNet152) \citep{he2018resnetd,he2015deepresiduallearningimage}. We replace ResNet18 and ResNet34 used in \Cref{sec:exp_img_reg} to experiment on weak models with similar intrinsic dimensions but different correlation dimensions. We treat the backbone of the models (excluding the classification layer) as $\phi_s$ and $\phi_w$ and finetune them via linear probing. We train the models with cross entropy loss and AdamW optimizer. We tune the training hyperparameters of weak and strong models using a validation set and train them for 800 steps with learning rate 1e-3 and weight decay 1e-6. 

\begin{table}[!ht]
    \centering
    \caption{Summary of the pretrained feature dimensions, along with the intrinsic dimensions $d_s, d_w$ and correlation dimensions $d_{s \wedge w}$ (with respect to the strong student \texttt{DINOv2-S14}) computed over the entire ColoredMNIST dataset (including training and testing).}\label{tab:img_cls_dim_coloredmnist}
    \begin{tabular}{c|ccc}
        \toprule
        Pretrained Model & Feature Dimension & Intrinsic Dimension ($\tau=0.01$) & Correlation Dimension \\
        \midrule
        \texttt{ResNet-18-D} & 512 & 117 & 6.23 \\
        \texttt{ResNet-34-D} & 512 & 127 & 7.07 \\
        \texttt{ResNet101} & 2048 & 121 & 1.74 \\
        \texttt{ResNet152} & 2048 & 128 & 1.88 \\
        \midrule
        \texttt{DINOv2-S14} & 384 & 28 & $\times$ \\
        \bottomrule
    \end{tabular}
\end{table}

\begin{figure}[!h]
    \centering
    \includegraphics[width=\columnwidth]{fig/coloredmnist_lp/coloredmnist_dsw.pdf}%\vspace{-2em}
    \caption{Scaling for $\pgr$ and $\opr$ of different weak teachers with a fixed strong student on ColoredMNIST.}\label{fig:coloredmnist_dscapw}
\end{figure}

\begin{figure}[!h]
    \centering
    \includegraphics[width=\columnwidth]{fig/coloredmnist_lp/coloredmnist_var.pdf}%\vspace{-2em}
    \caption{Scaling for $\pgr$ and $\opr$ of W2S on ColoredMNIST with injected label noise.}\label{fig:coloredmnist_variance}
\end{figure}

\paragraph{Discrepancies lead to better W2S.}
\Cref{fig:coloredmnist_dscapw} shows the scaling of $\pgr$ and $\opr$ with respect to the sample sizes $n, N$ for different weak teachers in the ResNet series with respect to a fixed student, \texttt{CLIP-B32}. 
As in \Cref{sec:exp_img_reg}, we observe that with similar intrinsic dimensions $d_s, d_w$, W2S achieves better relative performance in terms of $\pgr$ and $\opr$ when the correlation dimension $d_{s \wedge w}$ is smaller.

\paragraph{Variance reduction is a key advantage of W2S.}
We inject noise to the labels of the original ColoredMNIST training samples by randomly flipping the ground truth labels with probability $\varsigma \in [0,1]$ (following \cite{arjovsky2019invariant}). 
\Cref{fig:coloredmnist_variance} shows the scaling of $\pgr$ and $\opr$ with respect to $n$ and $N$ when taking DINOv2-S14 as the strong student and ResNet101 as the weak teacher. We observe that the larger artificial label noise $\varsigma$ leads to higher $\pgr$ and $\opr$. 

\subsection{Experiments on sentiment classification}\label{apx:exp_nlp_cls}

\paragraph{Dataset.} The Stanford Sentiment Treebank \citep{socher-etal-2013-sst2} is a corpus with fully labeled parse trees that allows for a complete analysis of the compositional effects of sentiment in language. The corpus is based on the dataset introduced by \citet{pang-lee-2005-sst_original_corpus} and consists of 11,855 single sentences extracted from movie reviews. It was parsed with the Stanford parser and includes a total of 215,154 unique phrases from those parse trees, each annotated by 3 human judges. We conduct binary classification experiments on full sentences (negative or somewhat negative vs somewhat positive or positive with neutral sentences discarded), the so-called SST-2 dataset, and split the dataset into training and testing sets of sizes 63000 and 4349. Generalization errors are estimated with the 0-1 loss over the test set.

\paragraph{Full finetuning.} We fix the strong student as Electra-base-discriminator \citep{clark2020electra} and vary the weak teacher among the Bert series \citep{turc2019bert-tiny} (Bert-Tiny, Bert-Mini, Bert-Small, Bert-Medium). 
With manageable model sizes, we conduct full finetuning experiments following the setup in \cite{burns2023weak}.
We use the standard cross entropy loss for supervised finetuning. 
When training strong students on weak labels (W2S), we use the confidence weighted loss proposed by \cite{burns2023weak}, which is suggested to be able to improve weak-to-strong generalization on many NLP tasks.
All training is conducted via Adam optimizers~\citep{kingma2014adam} with a learning rate of 5e-5, a cosine learning rate schedule, and 40 warmup steps. We train for 3 epochs, which is sufficient for the train and validation loss to stabilize. 

\paragraph{Intrinsic dimension.} The intrinsic dimensions $d_w,d_s$ are measured based on the Structure-Aware Intrinsic Dimension (SAID) method proposed by \cite{aghajanyan2020intrinsic}. We first train the full models on the whole training set, and then train the models with only $d$ trainable parameters based on SAID transformation. The $d_w$ or $d_s$ are the smallest number of parameters under SAID that is necessary to retain 90\% accuracy of the full models. Here, the 90\% accuracy is a common threshold used to estimate intrinsic dimensions in the literature \citep{li2018measuring}.

\begin{figure}[!h]
    \centering
    \includegraphics[width=\columnwidth]{fig/sst2/sst2-dsw.pdf}%\vspace{-2em}
    \caption{Scaling for $\pgr$ and $\opr$ of different weak teachers with a fixed strong student on SST-2.}\label{fig:sst2_dsw}
\end{figure}

\begin{figure}[!h]
    \centering
    \includegraphics[width=\columnwidth]{fig/sst2/sst2-var.pdf}%\vspace{-2em}
    \caption{Scaling for $\pgr$ and $\opr$ of W2S on SST-2 with injected label noise.}\label{fig:sst2_var}
\end{figure}

\paragraph{Correlation Dimension.} 
Let $D_s, D_w \in \N$ be the finetunable parameter counts of the strong and weak models, respectively. For full FT whose dynamics fall in the kernel regime, as explained in \Cref{rmk:lp_to_general_ft}, the strong and weak ``features'' become the gradients\footnote{
    Notice that $f_s, f_w$ are scalar-valued functions for binary classification tasks like SST-2, and thus the gradients $\nabla_{\thetab} f_s$ and $\nabla_{\thetab} f_w$ are row vectors.
    For multi-class classification tasks where $f_s, f_w$ output vectors of logits, a common heuristic to keep $\Phib_s, \Phib_w$ as matrices of manageable sizes (in constrast to tensors) is to replace gradients of the models, $\nabla_{\thetab} f_s$ and $\nabla_{\thetab} f_w$, with gradients of MSE losses at the pretrained initialization. 
    The gradients of MSE can be viewed as a weighted sum of the model gradients for each class.
}, $\Phib_s = \nabla_{\thetab} f_s(\Xb | \theta_s^{(0)}) \in \R^{N \times D_s}$ and $\Phib_w = \nabla_{\thetab} f_w(\Xb | \theta_w^{(0)}) \in \R^{N \times D_w}$, of the respective models at the pretrained initialization, $\theta_s^{(0)} \in \R^{D_s}$ and $\theta_w^{(0)} \in \R^{D_w}$.

A practical challenge is that $D_s, D_w, N$ are all huge for full FT on most NLP tasks, making it infeasible to compute the $D_s \times D_s$ and $D_w \times D_w$ Gram matrices and their spectral decompositions. 
As a remedy, we leverage the significantly lower intrinsic dimensions $d_s \ll D_s, d_w \ll D_w$ (see \Cref{tab:img_cls_dim_coloredmnist}) to accelerate estimation of $d_{s \wedge w}$ via sketching~\citep{halko2011finding,woodruff2014sketching}.
\begin{enumerate}[label=(\roman*)]
    \item We first reduce both $D_s, D_w$ to the same lower dimension $D = 0.01 \min\{D_s, D_w\}$ (with $D \gg \max\{d_s, d_w\}$) by uniform subsampling columns of $\Phib_s, \Phib_w$ to obtain $\Phib_s', \Phib_w' \in \R^{N \times D}$.
    \item Then, we use randomized rangefinder~\citep[Algorithm 4.1]{halko2011finding} to approximate the first $d_s, d_w$ right singular vectors, $\Vb_s \in \R^{D \times d_s}$ and $\Vb_w \in \R^{D \times d_w}$, of $\Phib_s', \Phib_w'$. Taking the evaluation of $\Vb_s$ as an example, we draw a Gaussian random matrix $\Gb_s \in \R^{d_s \times D}$ and compute the orthornormalization $\Vb_s = \ortho(\Phib_s'^\top \Gb_s)$ via QR decomposition.
    \item Finally, we compute the correlation dimension $d_{s \wedge w} = \nbr{\Vb_s^\top \Vb_w}_F^2$.
\end{enumerate}

\begin{table}[!ht]
    \centering
    \caption{Summary of finetunable parameter counts $D_s, D_w$, intrinsic dimensions $d_s, d_w$, and correlation dimensions $d_{s \wedge w}$ (with respect to the strong student \texttt{Electra}) computed over the entire SST-2 dataset (including training and testing).}\label{tab:sst2_dim}
    \begin{tabular}{c|ccc}
        \toprule
        Pretrained Model & $D_s,D_w$ & Intrinsic Dimension ($\tau=0.01$) & Correlation Dimension \\
        \midrule
        \texttt{Bert-Tiny} & 4.4M & 7000 & 81.13 \\
        \texttt{Bert-Mini} & 11.2M & 8500 & 38.67 \\
        \texttt{Bert-Small} & 28.8M & 8000 & 26.19 \\
        \texttt{Bert-Medium} & 41.4M & 4000 & 8.52 \\
        \midrule
        \texttt{Electra} & 109.5M & 700 & $\times$ \\
        \bottomrule
    \end{tabular}
\end{table}

\paragraph{Discrepancies lead to better W2S.}
\Cref{fig:sst2_dsw} shows the scaling of $\pgr$ and $\opr$ with respect to $n$ and $N$ for different $d_{s \wedge w}$. 
As in \Cref{sec:exp_img_reg,apx:exp_img_cls}, we observe the better relative W2S performance in terms of $\pgr$ and $\opr$ when $d_{s \wedge w}/d_w$ is smaller.

\paragraph{Variance reduction is a key advantage of W2S.}
We inject noise to the labels of training samples by randomly flipping labels with probability $\varsigma = 0, 0.1, 0.2, 0.3$. 
\Cref{fig:sst2_var} shows the scaling of $\pgr$ and $\opr$ with respect to $n$ and $N$ when taking \texttt{Electra} as the strong student and \texttt{Bert-Medium} as the weak teacher. We observe that the larger artificial label noise $\varsigma$ leads to higher $\pgr$ and $\opr$. 

\end{document}
\endinput
%%
%% End of file `sample-sigconf-authordraft.tex'.
 