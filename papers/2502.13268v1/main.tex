%%
%% This is file `sample-sigconf-authordraft.tex',
%% generated with the docstrip utility.
%%
%% The original source files were:
%%
%% samples.dtx  (with options: `all,proceedings,bibtex,authordraft')
%% 
%% IMPORTANT NOTICE:
%% 
%% For the copyright see the source file.
%% 
%% Any modified versions of this file must be renamed
%% with new filenames distinct from sample-sigconf-authordraft.tex.
%% 
%% For distribution of the original source see the terms
%% for copying and modification in the file samples.dtx.
%% 
%% This generated file may be distributed as long as the
%% original source files, as listed above, are part of the
%% same distribution. (The sources need not necessarily be
%% in the same archive or directory.)
%%
%%
%% Commands for TeXCount
%TC:macro \cite [option:text,text]
%TC:macro \citep [option:text,text]
%TC:macro \citet [option:text,text]
%TC:envir table 0 1
%TC:envir table* 0 1
%TC:envir tabular [ignore] word
%TC:envir displaymath 0 word
%TC:envir math 0 word
%TC:envir comment 0 0
%%
%%
%% The first command in your LaTeX source must be the \documentclass
%% command.
%%
%% For submission and review of your manuscript please change the
%% command to \documentclass[manuscript, screen, review]{acmart}.
%%
%% When submitting camera ready or to TAPS, please change the command
%% to \documentclass[sigconf]{acmart} or whichever template is required
%% for your publication.
%%
%%
% \documentclass[manuscript,review,anonymous]{acmart}
 % \documentclass[sigconf]{acmart} % for CHI camera ready - also enable ack
\documentclass[manuscript]{acmart} % for arxiv


\usepackage{graphics}
\usepackage{soul}
\usepackage{multirow}

%%
%% \BibTeX command to typeset BibTeX logo in the docs
\AtBeginDocument{%
  \providecommand\BibTeX{{%
    Bib\TeX}}}

%% Rights management information.  This information is sent to you
%% when you complete the rights form.  These commands have SAMPLE
%% values in them; it is your responsibility as an author to replace
%% the commands and values with those provided to you when you
%% complete the rights form.
% \setcopyright{acmlicensed}
% \copyrightyear{2018}
% \acmYear{2018}
% \acmDOI{XXXXXXX.XXXXXXX}
\copyrightyear{2025} 
\acmYear{2025} 
\setcopyright{acmlicensed}\acmConference[CHI '25]{CHI Conference on Human Factors in Computing Systems}{April 26-May 1, 2025}{Yokohama, Japan}
\acmBooktitle{CHI Conference on Human Factors in Computing Systems (CHI '25), April 26-May 1, 2025, Yokohama, Japan}
\acmDOI{10.1145/3706598.3713958}
\acmISBN{979-8-4007-1394-1/25/04}


%% These commands are for a PROCEEDINGS abstract or paper.
% \acmConference[Conference acronym 'XX]{Make sure to enter the correct
%   conference title from your rights confirmation emai}{June 03--05,
%   2018}{Woodstock, NY}
% %%
% %%  Uncomment \acmBooktitle if the title of the proceedings is different
% %%  from ``Proceedings of ...''!
% %%
% %%\acmBooktitle{Woodstock '18: ACM Symposium on Neural Gaze Detection,
% %%  June 03--05, 2018, Woodstock, NY}
% \acmISBN{978-1-4503-XXXX-X/18/06}


%%
%% Submission ID.
%% Use this when submitting an article to a sponsored event. You'll
%% receive a unique submission ID from the organizers
%% of the event, and this ID should be used as the parameter to this command.
%%\acmSubmissionID{123-A56-BU3}

%%
%% For managing citations, it is recommended to use bibliography
%% files in BibTeX format.
%%
%% You can then either use BibTeX with the ACM-Reference-Format style,
%% or BibLaTeX with the acmnumeric or acmauthoryear sytles, that include
%% support for advanced citation of software artefact from the
%% biblatex-software package, also separately available on CTAN.
%%
%% Look at the sample-*-biblatex.tex files for templates showcasing
%% the biblatex styles.
%%

%%
%% The majority of ACM publications use numbered citations and
%% references.  The command \citestyle{authoryear} switches to the
%% "author year" style.
%%
%% If you are preparing content for an event
%% sponsored by ACM SIGGRAPH, you must use the "author year" style of
%% citations and references.
%% Uncommenting
%% the next command will enable that style.
%%\citestyle{acmauthoryear}

\AtBeginDocument{\colorlet{defaultcolor}{.}}

\newif{\ifhidecomments}
% \hidecommentsfalse
\hidecommentstrue
\ifhidecomments
    \newcommand{\rev}[1]{\textcolor{defaultcolor}{#1}}
\else
    \newcommand{\rev}[1]{\textcolor{purple}{#1}}
\fi

%%
%% end of the preamble, start of the body of the document source.
\begin{document}

%%
%% The "title" command has an optional parameter,
%% allowing the author to define a "short title" to be used in page headers.
\title{Talking About the Assumption in the Room}

\author{Ramaravind Kommiya Mothilal}
\email{ram.mothilal@mail.utoronto.ca}
\affiliation{%
  \institution{University of Toronto}
  \country{Canada}
}

\author{Faisal M. Lalani}
\email{faisalmlalani@gmail.com}
\affiliation{%
  \institution{University of Illinois Urbana-Champaign}
  \country{USA}
}

\author{Syed Ishtiaque Ahmed}
\email{ishtiaque@cs.toronto.edu}
\authornotemark[1]
\affiliation{%
  \institution{University of Toronto}
  \country{Canada}
}

\author{Shion Guha}
\email{shion.guha@utoronto.ca}
\authornote{Served as co-supervisors and contributed equally to this work.}
\affiliation{%
  \institution{University of Toronto}
  \country{Canada}
}

\author{Sharifa Sultana}
\email{sharifas@illinois.edu}
\affiliation{%
  \institution{University of Illinois Urbana-Champaign}
  \country{USA}
}
%%
%% The "author" command and its associated commands are used to define
%% the authors and their affiliations.
%% Of note is the shared affiliation of the first two authors, and the
%% "authornote" and "authornotemark" commands
%% used to denote shared contribution to the research.

%%
%% By default, the full list of authors will be used in the page
%% headers. Often, this list is too long, and will overlap
%% other information printed in the page headers. This command allows
%% the author to define a more concise list
%% of authors' names for this purpose.
\renewcommand{\shortauthors}{Kommiya Mothilal et al.}

%%
%% The abstract is a short summary of the work to be presented in the
%% article.

% The presence and controversy of assumptions in technical ecosystems is well-established in responsible AI discourse. What is seldom touched upon is the conceptualization around assumptions, and how AI practitioners identify and handle them throughout their workflows. We present a theoretical framework of how these practitioners generally navigate assumptions and offer recommendations on how to incorporate reflective practices that allow organizations to better utilize these assumptions in avoiding potential harms.

\begin{abstract}
The reference to \textit{assumptions} in how practitioners use or interact with machine learning (ML) systems is ubiquitous in HCI and responsible ML discourse. However, what remains unclear from prior works is the conceptualization of assumptions and how practitioners identify and handle assumptions throughout their workflows. This leads to confusion about what assumptions are and what needs to be done with them. We use the concept of an \textit{argument} from Informal Logic, a branch of Philosophy, to offer a new perspective to understand and explicate the confusions surrounding assumptions. Through semi-structured interviews with 22 ML practitioners, we find what contributes most to these confusions is how \textit{independently} assumptions are constructed, how \textit{reactively} and \textit{reflectively} they are handled, and how \textit{nebulously} they are recorded. Our study brings the peripheral discussion of assumptions in ML to the center and presents recommendations for practitioners to better think about and work with assumptions. 
\end{abstract}

%%
%% The code below is generated by the tool at http://dl.acm.org/ccs.cfm.
%% Please copy and paste the code instead of the example below.
%%
\begin{CCSXML}
<ccs2012>
   <concept>
       <concept_id>10003120.10003121.10011748</concept_id>
       <concept_desc>Human-centered computing~Empirical studies in HCI</concept_desc>
       <concept_significance>500</concept_significance>
       </concept>
 </ccs2012>
\end{CCSXML}

\ccsdesc[500]{Human-centered computing~Empirical studies in HCI}

%%
%% Keywords. The author(s) should pick words that accurately describe
%% the work being presented. Separate the keywords with commas.
\keywords{Assumption, Machine Learning, Responsible ML, Informal Logic, Critical Thinking, ML Practitioners}
%% A "teaser" image appears between the author and affiliation
%% information and the body of the document, and typically spans the
%% page.

% \received{20 February 2007}
% \received[revised]{12 March 2009}
% \received[accepted]{5 June 2009}

%%
%% This command processes the author and affiliation and title
%% information and builds the first part of the formatted document.
\maketitle

 


\section{Introduction}
\IEEEPARstart{I}{n} recent years, flourishing of Artificial Intelligence Generated Content (AIGC) has sparked significant advancements in modalities such as text, image, audio, and even video. 
Among these, AI-Generated Image (AGI) has garnered considerable interest from both researchers and the public.
Plenty of remarkable AGI models and online services, such as StableDiffusion\footnote{\url{https://stability.ai/}}, Midjourney\footnote{\url{https://www.midjourney.com/}}, and FLUX\footnote{\url{https://blackforestlabs.ai/}}, offer users an excellent creative experience.
However, users often remain critical of the quality of the AGI due to image distortions or mismatches with user intentions.
Consequently, methods for assessing the quality of AGI are becoming increasingly crucial to help improve the generative capabilities of these models.

Unlike Natural Scene Image (NSI) quality assessment, which focuses primarily on perception aspects such as sharpness, color, and brightness, AI-Generated Image Quality Assessment (AGIQA) encompasses additional aspects like correspondence and authenticity. 
Since AGI is generated on the basis of user text prompts, it may fail to capture key user intentions, resulting in misalignment with the prompt.
Furthermore, authenticity refers to how closely the generated image resembles real-world artworks, as AGI can sometimes exhibit logical inconsistencies.
While traditional IQA models may effectively evaluate perceptual quality, they are often less capable of adequately assessing aspects such as correspondence and authenticity.

\begin{figure}\label{fig:radar}
    \centering
    \includegraphics[width=1.0\linewidth]{figures/radar_plot.pdf}
    \caption{A comparison on quality, correspondence, and authenticity aspects of AIGCIQA2023~\cite{wang2023aigciqa2023} dataset illustrates the superior performance of our method.}
\end{figure}

Several methods have been proposed specifically for the AGIQA task, including metrics designed to evaluate the authenticity and diversity of generated images~\cite{gulrajani2017improved,heusel2017gans}. 
Nevertheless, these methods tend to compare and evaluate grouped images rather than single instances, which limits their utility for single image assessment.
Beginning with AGIQA-1k~\cite{zhang2023perceptual}, a series of AGIQA databases have been introduced, including AGIQA-3k~\cite{li2023agiqa}, AIGCIQA-20k~\cite{li2024aigiqa}, etc.
Concurrently, there has been a surge in research utilizing deep learning methods~\cite{zhou2024adaptive,peng2024aigc,yu2024sf}, which have significantly benefited from pre-trained models such as CLIP~\cite{radford2021learning}. 
These approaches enhance the analysis by leveraging the correlations between images and their descriptive texts.
While these models are effective in capturing general text-image alignments, they may not effectively detect subtle inconsistencies or mismatches between the generated image content and the detailed nuances of the textual description.
Moreover, as these models are pre-trained on large-scale datasets for broad tasks, they might not fully exploit the textual information pertinent to the specific context of AGIQA without task-specific fine-tuning.
To overcome these limitations, methods that leverage Multimodal Large Language Models (MLLMs)~\cite{wang2024large,wang2024understanding} have been proposed.
These methods aim to fully exploit the synergies of image captioning and textual analysis for AGIQA.
Although they benefit from advanced prompt understanding, instruction following, and generation capabilities, they often do not utilize MLLMs as encoders capable of producing a sequence of logits that integrate both image and text context.

In conclusion, the field of AI-Generated Image Quality Assessment (AGIQA) continues to face significant challenges: 
(1) Developing comprehensive methods to assess AGIs from multiple dimensions, including quality, correspondence, and authenticity; 
(2) Enhancing assessment techniques to more accurately reflect human perception and the nuanced intentions embedded within prompts; 
(3) Optimizing the use of Multimodal Large Language Models (MLLMs) to fully exploit their multimodal encoding capabilities.

To address these challenges, we propose a novel method M3-AGIQA (\textbf{M}ultimodal, \textbf{M}ulti-Round, \textbf{M}ulti-Aspect AI-Generated Image Quality Assessment) which leverages MLLMs as both image and text encoders. 
This approach incorporates an additional network to align human perception and intentions, aiming to enhance assessment accuracy. 
Specially, we distill the rich image captioning capability from online MLLMs into a local MLLM through Low-Rank Adaption (LoRA) fine-tuning, and train this model with human-labeled data. The key contributions of this paper are as follows:
\begin{itemize}
    \item We propose a novel AGIQA method that distills multi-aspect image captioning capabilities to enable comprehensive evaluation. Specifically, we use an online MLLM service to generate aspect-specific image descriptions and fine-tune a local MLLM with these descriptions in a structured two-round conversational format.
    \item We investigate the encoding potential of MLLMs to better align with human perceptual judgments and intentions, uncovering previously underestimated capabilities of MLLMs in the AGIQA domain. To leverage sequential information, we append an xLSTM feature extractor and a regression head to the encoding output.
    \item Extensive experiments across multiple datasets demonstrate that our method achieves superior performance, setting a new state-of-the-art (SOTA) benchmark in AGIQA.
\end{itemize}

In this work, we present related works in Sec.~\ref{sec:related}, followed by the details of our M3-AGIQA method in Sec.~\ref{sec:method}. Sec.~\ref{sec:exp} outlines our experimental design and presents the results. Sec.~\ref{sec:limit},~\ref{sec:ethics} and~\ref{sec:conclusion} discuss the limitations, ethical concerns, future directions and conclusions of our study.
\section{Related Literature}

Making simplifying assumptions to operate at a required abstract level is central to machine learning practice \cite{selbst2019fairness,saitta2013abstraction}. Abstraction inherently involves making assumptions about what is necessary and what is not. For instance, ignoring certain features or choosing a specific representation in data abstracts out certain social contexts and interactions that may be assumed as nonessential. As much as these traits have contributed to the rise of ML applications in diverse domains, the last decade has seen an increasing number of concerns arising from abstraction and assumptions made about human behaviors and characteristics in automated decision-making systems \cite{benjamin2019race,noble2018algorithms,o2017weapons,eubanks2018automating}. While assumptions form an essential constituent of many prior works on how practitioners use ML, in section \ref{rel:periphery}, we review how \rev{prior works in HCI and related disciplines} often place assumptions on the periphery. Then, in section \ref{rel:core}, we discuss how the concept of an \textit{argument} in Informal Logic can offer a new perspective to think about and act on the confusions surrounding assumptions in ML.

\subsection{Assumptions on the Periphery}
\label{rel:periphery}

% \rev{\textbf{Assumptions as Marginal Disruptor.}} 
Prior works in HCI, ML fairness, and AI ethics have extensively looked into how practitioners use ML systems and interact with different phases of an ML workflow (for e.g., \rev{\cite{zhangHowDataScience2020,yang2018investigating,wang2023designing,muller2020interrogating}}.) Many of these works have brief discussions about assumptions or at least mention the word ``assumption'' when analyzing practitioners' interactions with ML-based systems. However, most of these references assign a marginal causal agency to assumptions for disrupting a desired state or chain of actions: some common references to assumptions include phrases such as \textit{``The result is often erroneous assumptions [made by practitioners] about what users would want from AI.''} \rev{\cite[p.~12]{subramonyam2022human}}, \textit{``...they assumed that succinct answers were sufficient.''} (indicating undesired documentation practices) \rev{\cite[p.~17]{heger2022understanding}}, and \textit{``participants who assumed sex was a sensitive feature attempted to mitigate biases in the ML pipeline by simply removing...''} (explaining undesired actions sequence) \rev{\cite[p.~5]{dengExploringHowMachine2022}}.

\rev{In addition to model efficiency,} the desired state or actions discussed in these prior works often revolve around sociotechnical concerns related to fairness, transparency, or collaboration. For instance, some prior works refer to practitioners' assumptions as one of the factors hampering their collaborative efforts with stakeholders of different technical backgrounds \cite{yang2018investigating,varanasi2023currently,wang2023designing}; a few others discuss how assumptions distort practitioners' understanding of ethical or fairness issues \cite{boyd2021datasheets,dengExploringHowMachine2022,aragon2022human,jarrahi2022principles}. 
\rev{In all of these works, assumptions are often discussed peripherally to the main discussion about a desired state or action, such as efficiency or fairness. 
Specifically, along with other factors such as institutional constraints and incentives, assumptions are discussed as one of the factors that affect practitioners' attainment of a goal.}

% Though assumptions are only superficially discussed per se in studying practitioners' usage, 
% \smallskip
% \noindent \textbf{Assumptions as Object of Enumeration.}
\rev{As assumptions are often discussed in terms of their effect,}
their influence in practice is well-appreciated in responsible ML discourse \cite{mitchell2021algorithmic,jarrahi2022principles,aragon2022human,malik2020hierarchy}. 
Consequently, research in HCI and responsible ML has developed numerous toolkits \cite{wong2023seeing} (refer to frameworks, guidelines, etc.) to invoke assumptions in everyday practice \rev{\cite{Sadek2024,Lavin2022,Smith_undated,rismani2023plane}}. There are also a significant number of these toolkits---some very popular such as model cards \cite{mitchellModelCardsModel2019} or datasheets \cite{gebruDatasheetsDatasets2021}---that do not include direct interrogative questions to invoke assumptions, but rather frame questions on various phases of an ML workflow intended to unearth hidden assumptions \rev{\cite{Pushkarna2022,Kawakami2024,Raji2020,elsayed2023responsible}}. 
\rev{In a recent work, \citet{kommiya2024towards} recommend that practitioners deliberately connect, in addition to listing down, the known assumptions to the desired states.}   
However, most of \rev{these} toolkits \rev{and guidelines} are only suggestive, simply prompting the practitioners to list down and document assumptions in some form. While the authors of these toolkits argue that this process intrinsically will help mitigate the undesired consequences of making assumptions, it remains unclear how practitioners actually \textit{conceptualize} and \textit{work through assumptions} when prompted with ``what are the assumptions'' or ``list down the assumptions'' type questions. Though one of the primary intentions of these toolkits is to make practitioners elicit and reflect on the assumptions, the methodology adopted in the toolkits gives meager attention to \rev{the practical steps that practitioners undertake during} assumption articulation, identification, and handling. 

% -- also consequently a substantive number of works disuxss waht assumptions are missed and ignored by practitioner 
% -- consequently enumeration and recording 

Overall, prior works in HCI and responsible ML mostly make peripheral references to assumptions \rev{as objects of enumeration to avoid disrupting a desired state or action.
Several studies provide a review of common assumptions that practitioners ignore or miss during statistical modeling \cite{wang2022against,mitchell2021algorithmic,malik2020hierarchy}.
However, it remains unclear if assumptions in ML can be objectively and uniformly viewed and acted upon, especially when practitioners work with diverse stakeholders to build an ML system.
Hence, to supplement prior works, we examine this fundamental problem of confusion and uncertainties that practitioners might face when conceptualizing and working through assumptions.}
% Overall, prior works in HCI and ML mostly make peripheral references to assumptions in studying practitioners' use of ML systems and in the development of toolkits. 
To our knowledge, there is no work in an ML context that places assumptions in the center stage and investigates the surrounding confusions. In the next section, we introduce \rev{the concept of an \textit{argument}} from philosophical thinking to the HCI community to (re)think about assumptions in ML.

% Finally, we also find several prior works that make no or sparse reference to assumptions  
% With recent focus on improving the regulation on AI systems, governance and risk management framework such as NIST's RMF, 

% - overally it seems like assumptions are on the periphery and not getging enough attneiont

% - how in these tools - while some encourage reflectivess - but do not go deeper,
% - others just do not enocurage reflective practice or gives rise to further assumptions and causes confusion
    
% c. So this sidelining and unclarity in what do people do with assumptions encouraged us to ask our RQ.
% Our findings show that this practitoiners face confusion and unclarity concerning assumptions due to this current setup and infrastrucutre they are in and have

% - last para - many works do not mention assmtipns at all or use other terms but implitly they mean assumptions - so the result is confusion
%     - data workflow is a good example
    
% Prior work has discussed how ML and data practitioners draw on their beliefs and experiences to make subjective decisions at various stages of an ML lifecycle (cite).
% Our findings contribute to and reinterpret this line of work through the assumptions and the surrounding uncertainties and tensions that (a) prompts practitioners to make those subjective decisions, and (b) practitioners assess in others' interpretations and actions.


\subsection{Bringing Assumptions to the Center}
\label{rel:core}
The sub-field of \textit{Informal Logic} in Philosophy emerged as a response to the inefficiency of tools and criteria of formal \textit{Logic}\footnote{The prefix ``informal'' in Informal Logic is contentious and it is sometimes argued as a formal enterprise in a different sense \cite{woods2000philosophical,levi2000defense,johnson1999relation}.} in analyzing and evaluating natural language discursive arguments \cite{blair2012informal,walton2008informal,scriven1980philosophical,scriven1976reasoning}.
Informal Logic appreciates the structural complexity of everyday language use, the formulation of unstated assumptions, and the epistemological questions surrounding argumentation, among others, that analytic and normative tools of Logic ignores or over-simplifies, distorting the meaning of arguments \cite{anthony2015informal,johnson2014rise}. 
Arguments are also extensively studied in \textit{Critical Thinking}, often associated with Informal Logic \cite{weinstein1990towards,johnson2012informal,crews2007critical}, that studies a mode of thinking about an object involving active interpretation, clear articulation and analysis of reasons, assumptions and conclusions, logical evaluation of explanations and evidences, self-regulation, and holding a disposition to use the above-mentioned skills \rev{\cite{fisher1997critical,hitchcock2018critical,emis1962concept,facione1990critical}}.
% While critical thinking can also be about information, communication, and observations, a large overlap between Critical Thinking and Informal Logic lies in their focus on formulating, analyzing, and evaluating \textit{arguments} presented in some form (written, spoken, pictorial, etc. or multi-modal \cite{groarke2015going}.)

In this study, we refer to the conception of an \textit{argument} put forward by \citet{hitchcock2021concept} and review prior works to establish a connection between an assumptions and an argument \cite{kingsbury2002teaching,hitchcock2007informal,govier1992good,goddu2009refining}. By referring to the structuring of arguments as discussed in Informal Logic, we situate assumptions as core elements of arguments that ML practitioners make or engage in \textit{implicitly} or \textit{explicitly}.
We use this assumption-argument paradigm to offer a new perspective to understand and explicate the confusions surrounding assumptions in ML for two reasons. First, assumptions do not exist in a vacuum; they exist as part of arguments that are expressed or implied \rev{\cite{plumer2017presumptions,delin1994assumption,ennis1982identifying}}.
Therefore, critically analyzing the structure of an argument can explain \textit{how} and \textit{why} assumptions are made and what contributes to the confusion. Second, when analyzing an assumption and the surrounding confusion, we are essentially making arguments ourselves to critically think how assumptions shape a resulting argument \cite{berman2001opening,brookfield1992uncovering,delin1994assumption,ennis1982identifying}.

\rev{\citet[p.~105]{hitchcock2021concept}} formulates a simple argument as a premise-conclusion complex as follows:
\begin{quote}
    \textit{A simple argument consists of one or more of the types of expression that can function as reasons, a ``target'' (any type of expression), and an indicator of whether the reasons count for or against the target.}
\end{quote}
Reasons in the above definition refer to the \textit{premises} of an argument that perform a specific function: they commit the author of an argument to \textit{``something's being the case''} either assertively or hypothetically \rev{\cite[p.~10]{hitchcock2007informal,searle1976classification}}. 
% \cite{hitchcock2007informal,searle1976classification}.
In other words, premises constitute the propositions and the accompanying intention (or the illocutionary act \cite{searle1975taxonomy}, in philosophical terminology). 
For instance, when we use ``suppose the data is not representative'' as a premise for an argument, we express the proposition that the data is not representative \textit{hypothetically}.  The \textit{target} or conclusion is also a proposition but can be an illocutionary act type of different kinds, including a directive, a commissive, an expressive, and a declarative \cite{toulmin1958uses,ennis2006probably}. Finally, arguments can also be complex where the premise of one argument can be the target of another and so on \cite{hitchcock2021concept}.

When an ML practitioner makes or uses an assumption, it is often made or used \textit{for} a particular purpose.
For instance, when a feature is assumed to be unnecessary, it is often for realizing a particular objective such as to reduce the complexity of feature space. 
Similarly, when a performance metric is chosen, it is done so that optimized model outcomes are relevant for decision-subjects. The structure of an argument, as described above, then suggests that assumptions can be perceived as premises for attaining a target.
\citet{ennis1982identifying} discuss how these premise-type assumptions back-up or fill gaps for realizing the conclusion, and so the falsity of these premise-type assumptions weakens the support provided for the target. In other words, assumptions now become an essential component of an argument that a practitioner makes or implies. We recognize that there could be other forms of arguments, such as using vivid descriptions to display an identity or marching in protests \cite{jacobs2000rhetoric,hample2015arguing}, but we interpret the actions and expressions of ML practitioners as a premise-conclusion complex in this study and leave further exploration to future studies.
We also recognize the possibilities of interpreting different assumptions in an ML workflow as categories other than premises, such as conclusions and presuppositions, which might require a new lens to analyze \cite{walton2008argumentation,ennis1982identifying}.

Now, there can be situations where a practitioner may \textit{only} be making a premise-type assumption, but an analyst will be the one who is inferring the argument and making a distinction between premise and target. 
For instance, an ML developer can exclude certain text sources from the data and proceed with training their language model, but it is the safety expert in their organization who actually attempts to dissect the reasoning behind the data exclusion assumption\footnote{\rev{Prior works in HCI and responsible ML often do not make a distinction between first- and higher-order assumptions. In other words, assumptions made by the practitioners are not distinguished from those that are interpreted by the authors or someone else. Our point is not to doubt the inferential validity of these works but instead call attention to the complexity of assumptions, which may influence how they are examined and handled \cite{berman2001opening,atkin2017investigating,korzybski1958science}.}}. 
In other situations, a practitioner might need an assumption that they did not explicitly use, but which an analyst could infer. Or practitioners may not need an assumption but could have unintentionally used an assumption in making an argument. It is important to note that arguments are not necessarily localized to what practitioners do or write about in their documentations and reports, instead any (un)intentional action performed by a practitioner, such as choosing a specific algorithm, can be interpreted as an argument whose assumptions could be unearthed by an analyst \cite{kjeldsen2015study,groarke2015going}. Overall, this \textit{premise-target lens of an argument} deconstructs an assumption to understand why it was made (by identifying and relating it to the target) and how it was made (for instance, implicitly or explicitly), and thereby can support us in understanding how and why confusions exist around assumptions in ML practice.

% \textbf{relevance to hci}

% - what is assumtion - thinking
% - analyzing an assumption angle - critical and reflective thinking bit in papers

% - argument not only made but also had or engaged - this in itself has arguments made - implied argument - doc can be seen as that

% - now, conclusions can also be assumptions - but usually straightforward and leae to future work
% - explain why analyzing arguments help us uncover confusions about assumptions

% - taxomony of presmie tyle assumptions
% - how tools of critical thinking helps us as a method and as a theory


% In philosophical language, a premise is a illocutionary act of a proposition, either expressed assertively or hypothetically. 
% Consider two simple arguments, A1 and A2 below, inspired from prior philosophical works on this topic:
% \begin{itemize}
%     \item \textit{A1: Suppose that the data is not representative of the target population. Then it makes sense to add more data points.} 
%     \item \textit{A2: The data is not representative of the target population. So it makes sense to add more data points.}
% \end{itemize}
% Though the premises in both A1 and A2 different, they express the same proposition that the data is not representative.
% The difference lies in the illocutionary act performed by the 

\section{Methodology}

\smallskip
\noindent \textbf{Participant }\rev{\textbf{Recruitment and Demographics.}} Once receiving approval from our institutions' ethics boards, we posted an open call for participants in several AI-oriented online communities \rev{on Slack and LinkedIn}. The call invited practitioners involved in some capacity with the research, development, design, or implementation of ML to participate in in-depth qualitative interviews on how they conceptualize, identify, and handle assumptions within their work. 52 individuals responded to our call, out of which we recruited 22 respondents for remote semi-structured interviews through purposive sampling \cite{sharma2017pros}. While this may not yield a statistically representative sample, it still allowed us to explore rich and unique insights into the experiences of the participants we felt most capable of answering the research questions in our study \cite{sharma2017pros,roy2015sampling}. Those who demonstrated significant experience working on ML projects, either as developers, data scientists, or product managers, as well as individuals closely involved with responsible ML artifacts, were ultimately chosen to participate in our interviews.
\rev{Most of our participants were from Global North locations and identified as males. Table \ref{tab:demographics} provides more details about our participants.}

\begin{table*}[]
\centering
\begin{tabular}{|l|l|}
\hline
\textbf{Dimension} & \textbf{Distribution} \\ \hline
Gender & Male: 16, Female: 6 \\ \hline
Region & Global North: 18, Global South: 4 \\ \hline
Role & ML/Data Engineer: 7, ML/Data Scientist: 6,  Management: 5, \\ 
 & Others (Designer/Ethicist/Academic): 4, \\ \hline
Organization Type & Tech company: 10, NGO/Civil Society: 5, Consulting: 4, \\
& Academia: 2, Government: 1 \\ \hline
\end{tabular}
\caption{\rev{Participant Demographics}}
\label{tab:demographics}
\end{table*}

\smallskip
\noindent \textbf{Interview }\rev{\textbf{Design.}} \citet{brookfield1992uncovering} emphasizes that the key to uncovering assumptions lies in analyzing the lived experience of the assumer in order to embed a specific practice within a realistic context. This motivated how we framed our questions to be reflective, allowing participants to answer in a way that stepped outside their typical frames of reference and assess their assumptions by explicitly thinking about them. The questions were also designed to explore participants' experiences without presuming outcomes while allowing participants to refute our own underlying assumptions \cite{kvale2009interviews}. The downside of this direct approach is that unconscious assumptions---the ones that inform a participant's intuition without them being privy to their existence and persistence---may fall through. 
To remedy this, we offered a second part of the interview in which we extracted specific phrases from model documentation of three popular large language models--- PaLM 2 \cite{anil2023palm}, BLOOM \cite{le2023bloom}, and Llama 2 \cite{touvron2023llama}---and asked participants to vocally analyze them. We chose these models as they varied across different dimensions of openness \cite{liesenfeld2024rethinking}. The sample texts were selected because they most directly offered an argument that follows a typical premise-conclusion structure with deliberate non-technical language that may prompt confusion at first glance. The samples are provided in appendix \ref{appendix}.

Our approach in framing the lifecycle of an assumption by inquiring about how it is conceptualized\footnote{\rev{Some readers may wonder why we focus only on conceptualization and not consider the \textit{operationalization} of an assumption. However, in our experience and based on our interviews with practitioners, ML stakeholders do not operationalize the construct ``assumption'' in practice but operationalize only the content of a specific assumption (e.g., the usage of ``representative'' in the assumption ``this data is representative''). In this view, assumptions function at a meta-level as discussed in prior works in Critical Thinking and Informal Logic (section \ref{rel:core}), and so we focus only on the conceptualization of assumptions in this work to uncover the confusion associated with the practical use of this term in ML. We leave alternative explorations to future work.}}, then identified, then handled aligns with \citet{berman2001opening}'s breakdown of an assumption as a single entity composed of assuming, feeling, thinking, and behaving. By organizing questions through assessing \textit{functions} of assumptions rather than conveying them holistically, we are able to easily distinguish what specific elements contribute to confusion around assumptions, and how participants react to that confusion. Furthermore, following the logic that initial assumptions are likely to predicate how future assumptions are handled, we attempted to frame questions in a way that allowed us to form a narrative of a participant's assumptions.

Questions are also informed by our personal experience in ML ecosystems, aligning with established practices in \textit{reflexive} qualitative research \cite{berger2015now}. The idea of assumptions being present in technical ecosystems and the motivation for the study in assessing their influence is driven by our own observations working within the space and examining it from a critical lens derived from our past and current positions as responsible ML researchers. We make this position explicit to enhance the rigor, credibility, and trustworthiness of the study and allow readers to understand the lens through which we interpreted responses.

\smallskip
\noindent \rev{\textbf{Interview Procedure.}}
\rev{The interview guide was developed by the first and second authors and was thoroughly discussed and approved by all authors. We share our complete interview guide in Appendix \ref{interview}.
We sent our consent letters ahead of the interviews and gave our participants the option of either returning the signed letter or providing verbal consent during the interview.
All our interviews were conducted in English via Zoom. While the first and second authors conducted 9 interviews together, the first author conducted 12 interviews independently, and the second author conducted 1 independently. 
We recorded our calls upon consent and manually took notes of participants who were uncomfortable with recording. 
Our participants were given the option to exit the interviews whenever they needed. 
Our interviews lasted for 60 minutes on average. We compensated participants with 30\$ for their time and contribution.}    

\smallskip
\noindent \rev{\textbf{Data }}\textbf{Analysis.} Our virtual interviews yielded approximately 25 hours of recorded audio, paired with auto-generated transcripts from Zoom. \rev{The interview data and notes were stored in the first author's institutional cloud storage.} 
\rev{As described in our interview design, our questions were broadly framed to extract how assumptions are perceived, identified, handled, and used in practice.}
Given the nature of our more open questions, we employed interpretative and descriptive qualitative analysis \cite{merriam2019qualitative} to decipher \rev{insights} within the transcribed responses. 
\rev{The first and second authors conducted the bulk of the data analysis, and the final themes were discussed and finalized among all authors. The analysis began with multiple readings of the transcripts followed by open-coding on the transcribed data, independently and manually, by the first two authors. They then iteratively went through each other's codes manually, extracted and recorded commonalities, cross-checked with one another for reliability, and finalized the codes after resolving critical disagreements by open discussion.}

\rev{In the next phase, the codes were interpreted through the assumption argument lens (section \ref{rel:core}), mapped, categorized, and structured into themes and sub-themes over multiple iterations.} 
\rev{For instance, several sub-themes such as ``forgotton assumptions'' and ``recording style'' were grouped into one of the main themes, ``informal documentation.'' These sub-themes were created by grouping several codes that revolved around how practitioners noted down their and others' assumptions. 
Further, while some sub-themes, such as ``chained assumptions'' and ``granularity'' had overlapping codes, we categorized these sub-themes into distinct themes (elaborated in sections \ref{subsec:integrate} and \ref{subsec:doc} respectively) as it offers a better frame to understand the confusions around assumptions.
Overall, as key takeaways were found around how participants personally and professionally interacted with assumptions, we were able to form ontological distinctions, procedural inconsistencies, and other confusing elements that helped us craft clear constructions of an assumption, how workflows perpetuate unchecked assumptions, and what practitioners (can) do about it.  
Our findings in section \ref{sec:findings} reflects how we inferred and organized the main themes in our data.
}
% - Answers were categorized in themes - how participants defined and identified assumptions, what givens they had going into the ML process, how they handled assumptions, and how they responded to the case study.


% \smallskip
\noindent \textbf{Limitations.} A study about assumptions will naturally possess a few assumptions itself. First, the premise of the study requires a consensus between the authors and the participants that assumptions in an ML workflow have a significance that needs to be addressed, potentially influencing answers toward having a more proactive stance toward them. Second, the samples chosen in the case study portion of the interviews were pointers extracted from lengthier and more contextualized model documentation; our selection was informed by our own assumptions about what may elicit rich responses. The samples shown were also the same for all participants. While this provided an equal frame of reference, future works could reinforce our findings through comparing similar perceptions in more diverse samples. 
\section{Findings}
\label{findings}

In this section, we first explore \sbma's context-aware capabilities across diverse settings, assessing its effectiveness and constraints in interpreting physical environments, social and stylistic cues, and people's identity. 
% 
Next, we examine its intent-oriented capabilities by evaluating both the strengths and limitations of the system's ability to understand and act on users' intentions. 
% 
For each aspect, our findings unfold in two parts. In \textit{Capabilities of \bma}, we examine how PVI use the system and reveal its capabilities. In \textit{Example}, we illustrate specific scenarios that present both the challenges users faced and their adaptive strategies. 



\subsection{Context Awareness}

This section examines how \bma{} enhances user experiences across diverse environments and interactions. We analyze its roles in enhancing spatial awareness in physical settings, interpreting social and stylistic context, and conveying human identities. 
% 
Our analysis considers both the system's strengths in providing context-rich descriptions and its limitations due to technical constraints and subjective interpretations. 





\subsubsection{Physical Environments}
\label{physical_environments}


Participants used \sbma{} to enhance their perception of indoor and outdoor environments through detailed scene descriptions. 
They explored various settings, from theaters (P2) and room layouts (P3, P7, P9, P13) to holiday decorations (P5) and street views (P11, P12). While the system provides structured visual information that expands spatial awareness, its effectiveness occasionally diminishes due to AI hallucinations or requires participants' guidance for accurate scanning. 




\paragraph{Capabilities of \bma} Seven participants (P2, P3, P7, P9, P11, P13, P14) reported that the AI tool's comprehensive scene descriptions enhanced their spatial awareness. 
These descriptions often uncover spatial details previously unnoticed by users, as noted by P11, \textit{``It gives me unexpected information about things that I didn't even know were there.''} 
% 


% AI hallucinations 
However, this enhanced awareness is compromised by AI hallucinations, instances where AI incorrectly identifies nonexistent objects. 
Such errors can disrupt the user's context understanding, leading to confusion or misinterpretation of the environment. To verify \sbma's descriptions, participants either seek human assistance or rely on their pre-existing mental models of the environment. 



Participants successfully guide \sbma{} using auditory cues, particularly when locating dropped objects. They combine their hearing and spatial memory to adjust camera angles, which enables the tool to scan more accurately. 
% 
This integration of participants' auditory inputs guides and refines the system performance, leading to more accurate and useful descriptions.




\paragraph{Example 1: AI Hallucinations of Adding Non-existent Details}


Be My AI provides detailed descriptions of objects, characterizing their colors, sizes (e.g., ``small,'' ``medium,'' ``large,'' and ``tall''), shapes (e.g., ``round,'' ``square,'' ``fluffy,'' ``wispy,'' ``dense,'' and ``open lattice structure''), and spatial orientations (e.g., ``on the left,'' ``to the right,'' ``in the foreground,'' ``in the middle,'' and ``in the background''). 
% 
However, four participants (P3, P5, P11, P14) encountered AI hallucinations where the tool fabricated details in their environments. 



Participants developed two strategies to verify the accuracy: consulting human assistants and drawing on their spatial memory.
P3's experience illustrates the value of human verification. When \bma{} reported a \textit{``phantom''} object behind her, she first investigated personally, finding nothing. A human assistant then confirmed the absence of any object. 

\begin{quote}
    \textit{``Apparently, Be My AI said there was an object behind me, and there really wasn't. So, then, when I went and asked somebody, `You know what is behind me?' and they said there wasn't anything behind you in the picture.''} (P3)
\end{quote}


P5's experience demonstrates how spatial memory helps identify inaccuracies. When the system incorrectly described objects next to her dogs in her home, her familiarity with the space enabled her to recognize these errors independently. This ability to leverage environmental knowledge allows participants to detect and disregard AI hallucinations effectively. 




%%%%%%%%%%%%
\paragraph{Example 2: Need for User Support in Locating Dropped Objects}

Participants (P4, P12, P13) used \sbma's sequential \textit{``top to bottom, left to right''} descriptions to locate dropped items like earbuds and hair ties. 
% 
These descriptions provided precise locations, such as when P13 learned of \textit{``a picture of a carpet with a hair tie in the upper right hand corner,''} and when P12 discovered \textit{``the earphones are directly in front of you, between your feet.''}
Such sequential descriptions allow participants to pinpoint lost objects with greater accuracy. 



To optimize the tool's scanning effectiveness, participants first used their auditory perception and spatial memory to approximate an object's location before positioning the camera. 
This adjustment allowed the tool to focus on the intended search area, rather than random searching. 
% 
For instance, P4 utilized his \textit{``listening skills''} to detect the location of a fallen object before directing the camera to that specific spot. Similarly, P12 enhanced the camera's view of dropped earbuds by stepping back from his seat to capture a better angle of the floor. 
% 
These examples illustrate how participants synergize their understanding of the environment with the system's technological capabilities to manage tasks that require spatial awareness.








\subsubsection{Social and Stylistic Contexts}
\label{social_stylistic}


Participants leveraged \sbma{} for both social interactions and style-related decisions. 
In social settings, nine participants (P2, P4, P5, P8-11, P13, P14) used it to take leisure pictures of animals like cats, dogs, birds, and horses, to better understand and monitor animal behavior and status. 
% 
In stylistic applications, it aids in identifying clothing colors (P2, P7, P9, P13), recognizing patterns (P2, P9), coordinate outfits by matching clothes with shoes and jewelry (P2, P4, P5, P6, P12), and assessing makeup (P5, P10).
% 
This section examines the system's subjective interpretations within these contexts, highlighting its utility in enriching users’ experiences and the challenges in providing accurate, context-aware descriptions.





\paragraph{Capabilities of \bma}

\sbma{} usually concludes its descriptions by injecting subjective interpretations, which enriches users' understanding of depicted scenarios. 
These interpretations encompass people's or animal's emotional states (e.g., ``a cheerful expression on his face,'' ``smiling slightly,'' ``friendly and approachable,'' ``curious expression''), 
body language (e.g., ``his arms are open as if he is engaging in a conversation or greeting the other man''),
and the ambiance (e.g., ``peaceful and natural,'' ``serene and peaceful atmosphere,'' ``cozy and cheerful holiday vibe,'' ``warm and inviting atmosphere,'' ``the atmosphere seems to be lively and festive''). 




These subjective interpretations help participants better engage in social interactions and understand animal behavior. 
% 
Participants (P5, P9-11) specifically used the system to grasp nuances in animals' facial expressions, activities, and body language (Figure~\ref{human_animal}). P10 highlighted its utility during dog walks: \textit{``Sometimes it's hard for me to know if the dog is peeing or what the dog is doing.''}

\begin{figure}[t!]
\centering
\includegraphics[width=0.4\textwidth]{images/human_animal_1.png}
\caption{\bma's description of a dog, including subjective interpretations of the dog's emotions. This screenshot was provided by P9.}
\Description{Interface of Be My AI. Top of the screen is a photo of a dog. Below the photo is the description of the dog, including subjective interpretations of the dog's emotions. This screenshot was provided by P9. The full description is as follows. The picture features a happy, cream-colored Labrador Retriever sitting in front of a vibrant backdrop of Christmas presents. The dog is wearing a festive red scarf and has a joyful expression with its tongue out. Surrounding the dog are numerous gift boxes wrapped in colorful paper with patterns of snowflakes, Christmas trees, reindeer, and traditional holiday plaid. Some presents have large bows on them. Behind the dog, to the left, is a large teddy bear with a red bow, and to the right, a black bear plush toy dressed in a plaid outfit. The setting gives off a cozy and cheerful holiday vibe.}
\label{human_animal}
\end{figure}
%%%%%%%%%%%%%% fashion
Furthermore, the subjectivity extends to fashion suggestions, with \bma{} recommending stylistically coordinated outfits based on colors and patterns, as well as assessing makeup. 
For instance, P5 employed it to check the color, placement, and overall balance of her makeup, P2 and P9 consulted it to coordinate tops and bottoms, and P12 used it to select a tie that complemented his shirt. 


However, participants noted that BMA's visual interpretations sometimes included excessive subjective elements. 
This subjective input, unverified by human, could potentially undermine the system's ability to provide accurate context awareness.  
As a result, participants preferred their own subjective interpretations or feedback from sighted assistants in contexts involving human-animal interactions and fashion choices. 




%%%%%%%%%%%%%%%%
\paragraph{Example 1: Subjective Interpretations in Human-Animal Interactions}


Participants (P2, P5) identified limitations in how \sbma{} infers animal emotions in its image descriptions, raising concerns about the accuracy of these subjective interpretations. 
For instance, P5 highlighted instances where the tool describes a dog \textit{``appears to be relaxed''} or \textit{``appears to be happy.''} 
Likewise, P2 noted its tendency to include subjective commentary, as in \textit{``That's a white cat curled up on a fuzzy blanket. She looks peaceful and happy and rested.''} 



These interpretations go beyond observable visual elements to make emotional inferences that may not reflect reality. 
% 
As a result, participants expressed a preference for objective, fact-based descriptions. 
P2 articulated a desire for less editorializing, saying, \textit{``Maybe I don't want it to editorialize, you know, maybe I just literally only want the facts of it.''}



Participants (P2, P5) stressed the importance of maintaining human agency in interpreting animal behaviors, preferring their own judgment rather than the system's subjective interpretations. 
P5 particularly valued the ability to modify \sbma's descriptions of her dog's expressions, maintaining control over her interpretation of pet behaviors. 


\begin{quote}
    \textit{``Some blind people think, `How does it know that the dogs are happy? Why does it assume?' Some people don't like that it's making assumptions about the picture. I like having access to that information, but I like to be able to change it if I want.''} (P5)
\end{quote}


In summary, \sbma's subjective inferences can enrich descriptions, yet they risk introducing inaccuracies that undermine the system's reliability. Users therefore value the ability to override these interpretations, preserving their agency in understanding the context. 




\paragraph{Example 2: Subjective Interpretations in Fashion Help}


\sbma's subjective interpretations extend beyond human-animal interactions to fashion help. 
Participants (P6, P7, P10) utilized it to describe colors and patterns but expressed concerns about its subjective fashion suggestions. 
% 
They preferred to make their own style choices or seek human assistance for outfit matching, highlighting the human subjectivity in fashion decisions.



P6 questioned the AI's capacity to authentically replicate human judgment, saying, \textit{``It's interesting how AI is being taught to simulate kind of the human factor of things.''}
% 
She cited experiences where AI-generated responses appeared \textit{``strange''} and \textit{``complete nonsense,''} contrasting these with the nuanced understanding that humans provide.  



\begin{quote}
    \textit{``No, no, no, no, I would never use it to do anything that required human subjectivity... I just don't trust AI with a task that is supposed to be subjective like that, particularly visual like that. Have you ever seen AI weirdness?... I think that just goes to show why I'm not gonna trust AI with my fashion yet.''} (P6)
\end{quote}


P10 explicitly preferred human feedback, stating, \textit{``I'm still more confident asking a sighted person to provide me with the feedback.''} 
This preference underscores how participants maintain their agency by relying on human judgment for fashion decisions rather than deferring to \sbma's suggestions. 







%%%%%%%%%%%%
\subsubsection{Identity Accuracy and Sensitivity}
\label{identity}


Nine participants (P2, P3, P5-7, P9-11, P13) evaluated the system's ability to describe people's identities in images of their families, friends, and social media posts. 
Our analysis examines how the tool handles identity attributes like age and gender, focusing on both its capabilities and limitations in providing sensitive and accurate descriptions. 




\paragraph{Capabilities of \bma}

\sbma{} describes people's identity in terms of gender (e.g., ``woman,'' ``man''), age (e.g., ``old,'' ``late twenties or early thirties''), appearance (e.g., ``long, dark brown hair,'' ``wavy brown hair,'' ``His hair falls past his ears, with a slightly messy but stylish look'') and ethnicity (e.g., ``East Asian''). 
% 
Participants appreciated such detailed descriptions for providing deeper insight into people's visual characteristics. P2 explained: \textit{``I enjoyed hearing, I knew my friend was East Asian when I saw her picture. It was kind of cool to see that because... as a blind person, you don't know that all the time. So, I'd like access to it.''}


However, eight participants (P2, P3, P5-7, P9, P10, P13) encountered challenges with the system's accuracy and sensitivity in describing identity attributes, particularly gender and age. 
% 
These errors reveal the tool's limited ability to interpret contextual cues that extend beyond visual appearance, such as cultural, situational, and personal contexts. 



\paragraph{Example: Inaccurate Identification of People's Gender and Age}


Three participants (P3, P10, P13) reported inaccuracies in \sbma's gender identification. 
P13 described how it misidentified gender by mistaking a hidden ponytail for short hair: \textit{``just looked like short-cropped hair.''} 
% 
Similarly, P10 revealed that the system struggled with gender identification for individuals whose physical attributes do not conform to typical gender norms, noting, \textit{``the person had a short hair, and was a female''}. 
In response, it defaulted to neutral language, \textit{``\bma{} didn't tell me it was a woman or a man, just said a person.''}
% 
These examples illustrate the system's reliance on stereotypical indicators and reveal its challenges in interpreting non-visible details.



Additionally, two participants (P5, P9) identified problems with age description. P9 reported that \bma{} overestimated her daughter's age by one year. 
% 
Likewise, P5 observed sensitivity concerns when the system labeled someone as \textit{``an old woman''} based solely on grey hair. 
% 
To address these age-related inaccuracies, P5 suggested that the tool should adopt more objective, factual descriptions rather than subjective or potentially stigmatizing labels, such as \textit{``a woman with grey hair, instead of an older woman.''}
% 
These examples indicate the system's difficulties with precise age estimation and the potential inaccuracy when AI makes assumptions based on appearance alone.



In summary, these examples illustrate the tool's limited context awareness and its challenge in accurately interpreting people's identities, such as gender and age, due to the reliance on stereotypical visual cues. These misjudgements are rooted in the system's inability to integrate and interpret broader, non-visible contextual elements like cultural norms and personal styling choices. 








%%%%%%%%%%%%%
%%%%%%%%%%%%%
\subsection{Intent-Oriented Capabilities}

In this section, we explore \sbma's limitations \rev{in} comprehending users' goals, providing actionable support to fulfill these objectives, and offering real-time feedback. 
% 
Through analysis of specific examples, we elucidate the extent of the system's intent-oriented capabilities and highlight areas where it falls short in adapting to user needs. 



\begin{figure}[t!]
\centering
\includegraphics[width=0.45\textwidth]{images/cooking_eggshell.png}
\caption{On the left is the original image sent to \bma. On the right is \bma's description of eggs in a frying pan, followed by a question checking for the presence of eggshells. This example was originally drawn from X.}
\Description{On the left is the image of a frying pan on the stove-top. Inside the pan there are 4 eggs, one of whose yolk is broken. On the right is a photo of an user using Be My AI on a smart phone.}
\label{eggshells}
\end{figure}


\subsubsection{Agentic Interaction}
\label{agentic_interaction}

While \bma{} can process visual information, it often struggles to infer users' specific intentions from images. Users compensate for these limitations by guiding the system through targeted prompting. 


\paragraph{Capabilities of \bma}



BMA's ``ask more'' function enables users to explore image details that were not covered in initial descriptions. Through this feature, users can probe for specific details that match their personal needs or interests.
% 
Eleven participants (P1-3, P5, P6, P8-12, P14) used the ``ask more'' function to 
match outfits (P2, P5, P12),
check makeup (P5, P10), 
suggest cooking recipes (P12), 
assist with household appliances (P5, P12), 
examine text or objects (P6, P8, P11, P12, P14),
gain more details about people's facial expressions, attire, or actions (P2, P3, P5, P8, P9, P10, P11), 
check animal status (P3, P5, P11, P14), 
and edit the descriptions for social media posts (P2, P3). 



Seven participants (P1-3, P5, P8-10) valued the flexibility of posing follow-up questions, enhancing their independence by reducing reliance on human assistance. As P3 noted, \textit{``I can ask follow-up questions, so I have a good way to sort of figure out what's in the image independently, which is something I was not able to do before this app came out.''} 


However, participants identified a common issue that the system is unable to discern users' goals in initial response, often resulting in generalized descriptions without knowing which aspects to emphasize. P12 elaborated on this challenge, stating, \textit{``\bma{} loves to make general descriptions, and it doesn't know what to focus on.''} 
To overcome this limitation, users guide the system with targeted prompts that clarify their specific intentions. 





\paragraph{Example 1: Check for the Presence of Eggshells}







In Figure~\ref{eggshells}, BMA's initial response provides a detailed descriptions of what is included in a frying pan. 
The user, likely influenced by prior experiences of inadvertently leaving unwanted items in the pan while cooking, specifically intended to check for such elements. 
% 
However, the system did not infer that the user was looking for unwanted elements like eggshells, rather than seeking a general description of the scene.
% 
To clarify her objective, the user posed a follow-up question, thus directing \sbma{} to recognize and prioritize her specific concerns and intentions. 



In this example, although the AI tool can describe the visual elements accurately and comprehensively (from the condition of the eggs to the surroundings), it failed to identify unusual visual elements that are challenging for PVI to detect but are important to their understanding of the scene. 
% 
User prompting helps refocus the system's attention on these unusual elements, thereby enhancing its ability to interpret and respond to user-specific intents more effectively. 



%%%%%%%%%%
\paragraph{Example 2: Adjust Rotary Control Appliances}
Six participants highlighted the tool's inadequacy in comprehending their goal of adjusting rotary control appliances like washers (P1) and thermostats (P2, P4, P5, P13, P14). This task requires \bma{} to interpret the current settings on these appliances as users modify dials for time, mode, or temperature.
% 
However, the system often offered broad descriptions of visual elements without honing in on the user's specific objectives. 


P13's experience exemplifies this limitation. While the system could recognize a thermostat on the wall, it failed to provide critical information such as the current temperature setting or instructions for adjusting the temperature. 



To overcome the limitation, participants guided the system to better understand their goal through prompting. 
% 
For instance, P5 asked precise questions like \textit{``What is the arrow pointed at right now on the current setting?''} Such specific queries helped redirect \sbma's focus from general scene descriptions to the exact details needed for appliance adjustment. 





\subsubsection{Consistency and Follow-Through}
\label{consistency}


In Section~\ref{agentic_interaction}, we described how users needed to explicitly guide \bma{} when it failed to understand their intentions. 
% 
This section examines the system's difficulties maintaining consistency and following through on tasks even after acknowledging users' goals. 



\paragraph{Capabilities of \bma}


\sbma{} often fails to proactively suggest next steps or support users until they achieve their objectives. 
% 
Effective consistency requires maintaining focus on the user's objective throughout the interaction. \rev{This includes} offering progressively specific and relevant assistance, and ensuring all responses contribute to the user's intended outcome.


Due to these limitations, participants turn to human assistants. \rev{Human assistants} can interpret the context of tasks and goals more dynamically, \rev{providing} conversational guidance and adapting to users' actions to facilitate goal completion. 




%%%%%%%%%%%%
\paragraph{Example 1: Inadequacy in Identifying Central Puzzle Piece}

P14 used \sbma{} to differentiate between various puzzles by describing the images on the boxes like a scene of cats or bears. 
% 
When tasked with identifying the centerpiece of a nine-piece puzzle using the box image, \sbma{} failed to provide the necessary detail for this precise task. 
% 
Initially, the system correctly read text from the puzzle box indicating which piece belonged in the center. 
However, when P14 requested more specific guidance to locate the centerpiece, \sbma{} only noted that it was square -- a characteristic shared by multiple pieces. This response was too vague for successful puzzle assembly. 



\begin{quote}
    \textit{``When \bma{} read me the text of the box on the back of the puzzle that said, you know, this particular piece should be in the center of the puzzle. As a follow-up question I asked, `Could I have more information about the piece in the center?' And it said, `This piece is a square piece,' but I mean, there were many different square pieces, so I could not tell from that.''} (P14)
\end{quote}
% 
This example illustrates the system's limited ability to translate visual understanding into actionable guidance. While it could process the image and comprehend users' general goals, it could not provide the detailed information needed for successful goal achievement. 
As a result, P14 turned to a family member for assistance. 


\begin{figure}[t!]
\centering
\includegraphics[width=0.6\columnwidth]{images/description-of-a-conference-room.png}
\caption{\bma's description of a conference room, with the original image cropped. This example was drawn from X.}
\Description{Screenshot of the Be My AI interface. In the top is the description of a conference room. In the bottom left is the Take Picture button. In the bottom right is the Ask More button. The full description is as follows. The picture shows an indoor conference room with a group of people seated at round tables. The attendees appear to be focused on something at the front of the room, which is not visible in the photo. The room has a wooden floor and a ceiling with white beams, from which stage lights are hanging. The tables are covered with grey tablecloths, and there are some papers and bottles on them. There's a screen visible in the background showing some kind of presentation or logo, but it's not clear what it is. The lighting in the room is bright, and the overall atmosphere seems professional.}
\label{incomplete_info}
\end{figure}
% \footnotetext[2]{\url{https://x.com/Chr1sLew1s/status/1732730506557477052}}

\paragraph{Example 2: Lack of Instruction for Camera Adjustment}
\label{adjust_camera}

A common challenge for participants (P3, P5, P6, P8, P10-12, P14) was aligning the camera properly to capture clear views of their intended areas. 
\bma{} can identify image quality issues, notifying users when pictures were \textit{``cut off''} (P5), \textit{``blurry''} (P6), or incomplete (Figure~\ref{incomplete_info}).  
% 
However, it is unable to provide further actionable guidance on adjusting the camera to improve image clarity. 
% 
This limitation manifested in scenarios where \sbma{} recognized both the user's goal of capturing specific areas and the image deficiencies but could not suggest practical solutions. 
As P6 described, \textit{``It was very frustrating because it said the picture is blurry. Can you please put the label in the frame? But I didn't know how to put the label on the frame.''} 








%%%%%%%%%%%%%
Eight participants (P1-3, P7, P8, P12-14) \rev{resolved this challenge by turning to human assistants, who can offer} adaptive support to achieve their goals, \rev{especially in} tasks that require continuous feedback. P13 shared an example where a human assistant not only understood her goal but also guided her in adjusting the camera and instantly reminded her to turn on the light to achieve her objective.

\begin{quote}
    \textit{``You know, having that ability to communicate and say, `Hey, this is what I'm looking for.' Or one time, I was looking for something and I had the lights off, and they're like, `You need to turn the lights on.' And I go, `Okay,' as opposed to, you know, if I tried using AI for that, it would just say `dark room.'''} (P13)
\end{quote}


This example highlights human assistants' ability to adapt their communication based on the situation and the user's implied needs, providing solutions that directly support achieving the user's goals.
% 
\rev{However}, \bma{} struggles to provide practical assistance despite understanding basic requests. 







\subsubsection{Real-Time Feedback}
\label{realtime_feedback}

We investigate how participants used \sbma{} to support navigation tasks, focusing on location awareness and orientation. 
Our analysis reveals the system's constraints in delivering real-time feedback and comprehensive navigational information from static images, and its inability to facilitate immediate interactions with surroundings.  



\paragraph{Capabilities of \bma}

Participants (P2, P5, P10) employed \sbma{} to aid in localization and orientation while navigating to their destinations. 
% 
For instance, P2 utilized it to identify gate numbers at the airport, P5 employed it to read signage directing toward the airport's transportation area, and P10 used it to recognize her surroundings when disoriented in her neighborhood.  
% 
Despite these benefits, participants encountered challenges when using \sbma{} for navigation. The primary limitations stemmed from the limited camera view and practical mobility issues. 




\paragraph{Example 1: Limited Navigational Information in Static Images}


Six participants (P2, P5, P6, P10, P11, P13) reported that \sbma's reliance on static images rather than real-time videos makes it difficult to capture comprehensive navigational information, such as obstacles and signage, in a single shot. 
% 
As P6 described, users \textit{``have to stand there and keep taking pictures and taking pictures,''} verify the captured content, assess its utility for navigation, and adjust the angle for additional shots. This iterative process can be time-consuming, \textit{``It'll be too much of a task that's supposed to take maybe 10 minutes would probably take like 30.''}



P10 elaborated on the challenge and risk of simultaneously taking pictures and navigating, particularly when attempting to identify and navigate around obstacles. The uncertainty of capturing all potential hazards was a significant concern. 



\begin{quote}
    \textit{``It's still hard to know how to capture, you know, all the obstacles. I think that's the issue. Like, how to know that I captured the right obstacle on my path? I mean, it depends [on] what I'm able to capture with the camera. You know, that's the tricky part for somebody without vision to capture the obstacle.''} (P10)
\end{quote}


Given these limitations, participants preferred human assistance through video-based interactions over \bma{} for real-time navigation. 
Video interactions not only benefit from human adaptability in adjusting guidance (Section~\ref{adjust_camera}) but also offer crucial real-time feedback unavailable in static images. 
Participants (P2, P6, P13) suggested that real-time video interpretation capabilities would significantly enhance the system by eliminating the need for repeated picture-taking. 



\begin{quote}
    \textit{``If you could hold the camera and it could do it in real time, versus having to stop, take a [picture], then assess it... If that were the case, then you could move on to doing things like uploading videos and getting it to describe actual videos and things like that, versus just still images. That would be great. You know, then you could describe more.''} (P2)
\end{quote}
% 
Such real-time video analysis would enable continuous camera movement and immediate feedback, streamlining the navigation process without pausing and reviewing individual images. 



%%%%%%%%%%%%%%
\paragraph{Example 2: Irreplaceable Role of Orientation \& Mobility Skills in Navigation}


Besides the value of real-time visual interpretations from external resources like human assistants, participants (P5, P10, P13, P14) emphasized the indispensable role of real-time feedback through their Orientation and Mobility (O\&M) skills for safe navigation.
% 
P5 and P13 pointed out that even human assistance, though adaptive in guiding PVI away from navigational hazards, cannot substitute for essential O\&M skills required for tasks like street crossing.
Consequently, participants are more cautious about relying on emerging AI tools for navigation.
% 
P14 reinforced this perspective, saying, \textit{``it's not a replacement for our mobility skills or just any skills in general. It can aid and augment the skills but it's not a replacement for them. It shouldn't be.''} 



%%%%%%%
Participants (P5, P13) further delineated vital information provided by O\&M skills or discerned through O\&M tools like white canes or guide dogs, which current AI or human assistance cannot replace. 
First, immediate surroundings. \bma{} can identify obstacles \textit{``that are a little far away''} (P5) but may miss immediate surrounding hazards. 
Second, distance and proximity measurements of obstacles. Although \sbma{} can indicate the presence of obstacles, it lacks the capability to measure their distance.
Third, directional details. Essential navigational information, such as the direction of stairs (ascending or descending) or the presence of railings, are not always detectable through the system. 
% 
With the O\&M skills and tools, users can instantly adapt their movements based on direct interaction with their environment. This level of responsiveness is currently unattainable with AI tools or human assistance. 


\begin{quote}
    \textit{``[\bma] says, you know, `stairs in front,' and it's like, `Okay, that's great, but where are they? How far are they? Are they going up? Are they going down? Is there a railing?' which would be information that the dog or the cane could tell you. So, I would say use it as a tool along with, but definitely not by itself.''} (P13)
\end{quote}


In summary, neither AI systems nor human assistants can replace the essential, real-time feedback and adaptive capabilities offered by O\&M skills and tools. These elements are vital for ensuring safe navigation by allowing users to directly interact with their environment.

\section{Discussion}
\label{sec:discussion}


\begin{figure*}[t!]
\centering
\includegraphics[width=0.85\textwidth]{images/discussion_eggshell_user+app_combined.pdf}
\caption{The top shows the status quo of handoff between the user and \bma. The bottom illustrates our proposed simplified interaction.}
\Description{The top figure illustrates the current state of interactions between the user and Be My AI, where the user submits a photo and Be My AI returns a response in the first time of interaction. Subsequently, the user poses a follow-up question to Be My AI and Be My AI returns a more detailed description in the second time of interaction. The bottom figure is our proposed simplified interaction, where Be My AI learns from previous interactions and returns a detailed description during the first interaction, without user asking follow-up question.}
\label{fig.discussion_eggshell_user+app}
\end{figure*}




In this section, we examine the current state of handoff between users, \bma, and remote sighted assistants, and propose new paradigms to address the challenges identified in our findings. 
Next, we explore how multi-agent systems, both human-human and human-AI interactions, assist visually impaired users, and envision the transition toward AI-AI collaborations for tasks requiring specialized knowledge. Finally, we discuss the potential advantages of real-time video processing in the next generation of AI-powered VQA systems.






\subsection{Handoff Between Users, \bma, and Remote Sighted Assistants}
\label{handoff}

In this study, we illustrated the advantages of the latest LMM-based VQA system in (i) enhancing spatial awareness through detailed scene descriptions of objects' colors, sizes, shapes, and spatial orientations
(Section~\ref{physical_environments}), (ii) enriching users' understanding in social and stylistic contexts by detailing emotional states of people or animals, their body language, ambiance
(Section~\ref{social_stylistic}), and identity recognition (Section~\ref{identity}), 
and (iii) facilitating navigation by interpreting signages (Section~\ref{realtime_feedback}). 


Informed by our findings, despite these various benefits, there are challenges that the system alone cannot overcome.  
\bma{} still requires human intervention, either from the blind user or the remote sighted assistant (RSA), to guide or validate its outputs.  
% 
Users seek confirmation from RSAs or depend on their own spatial memory to overcome AI hallucinations, where \sbma{} inaccurately adds non-existent details to scenes. Users also rely on auditory cues and spatial memory to locate dropped objects and direct the system toward the intended search areas (Section~\ref{physical_environments}).
% 
Moreover, users actively prompt the system to understand their specific objectives, such as checking for eggshells in a frying pan or adjusting appliance dials (Section~\ref{agentic_interaction}). 
% 
There are also instances where users require assistance from RSAs when the system fails to provide adequate support to fulfill users' objectives, such as identifying the centerpiece of puzzles or adjusting the camera angle (Section~\ref{consistency}).
% 
Human assistance or users' O\&M skills are necessary to receive real-time feedback for safe and smooth navigation (Section~\ref{realtime_feedback}). 



Furthermore, our findings revealed that the system might produce inaccurate or controversial interpretations. Users express skepticism towards \sbma's subjective interpretations of animals' emotions and fashion suggestions (Section~\ref{social_stylistic}), and have encountered inaccuracies in \sbma's identification of people's gender and age (Section~\ref{identity}). These instances underline potential areas where human judgment is necessary to corroborate or correct the system's descriptions.





Next, we discuss the handoff~\cite{mulligan2020concept} between users, \bma, and RSAs to mitigate the aforementioned challenges. 




% \begin{figure}[]
% \centering
% \includegraphics[width=0.95\textwidth]{images/discussion_eggshell_user+app.png}
% \caption{Status quo of handoff between the user and \bma.}
% \label{fig.discussion_eggshell_user+app}
% \end{figure}


% \begin{figure}[]
% \centering
% \includegraphics[width=0.95\textwidth]{images/discussion_eggshell_user+app-single-interaction.png}
% \caption{Status quo of handoff between the user and \bma.}
% \label{fig.discussion_eggshell_user+app}
% \end{figure}






%%%%%%%%%%%%%
\subsubsection{Status Quo of Interactions Between Users and \bma}


Through BMA's ``ask more'' function, users are able to request additional details about the images that were not covered in initial descriptions. 
% 
This functionality facilitates a shift in interaction dynamics between users and \bma, even if the system may not accurately understand or answer users' questions in the first attempt. 
In these interactions, users are not merely passive recipients of AI-generated outputs, they actively guide the AI tool with specific prompts to better align AI's responses with their objectives. 


Our findings reported one instance where the AI tool fails to grasp the user's intent to check the presence for eggshells in the beginning (Figure~\ref{fig.discussion_eggshell_user+app} top). First, the user submits an image of eggs in the pan. Respond to the image, the system describes the quantity and object (``Inside the frying pan, there are three eggs'') and states of the yolks and whites (``whites separated'', ``yolk has broken'', ``mixing with the egg white''). Next, the user clarifies her inquiry by asking, ``are there any shells in my eggs?''
This prompts the system to understand the user's goal, reevaluate the image, subsequently confirming the presence (``Yes, there is a small piece of eggshell in the frying pan'') and location of an eggshell to help her remove it (``near the bottom left of the broken egg yolk''). 


This interaction exemplifies the status quo of handoff, where the user and \bma{} engage in a back-and-forth dialogue to refine the descriptions based on the ``ask more'' function and the user's precise prompts. 
% 
\rev{While this iterative process allows the system to eventually understand the users' intent without RSAs' intervention, it places cognitive burden on users who must carefully craft and iteratively refine their prompts. The cognitive load increases as users mentally track what information they've already received, analyze gaps between their needs and the system's responses, and develop increasingly specific queries. 
}






\rev{To reduce users' cognitive load, we propose enabling the system to adopt a mechanism that combines multi-source data input with long-term and short-term memory capabilities~\cite{zhong2024memorybank}. With explicit user consent, future versions of LMM-based VQA systems could integrate data from users' mobile devices (e.g., location information, time data) alongside historical interaction data within the system (e.g., contexts and follow-up questions) to recognize user preferences and common inquiries, and infer their needs, thereby generating responses more effectively in similar contexts. Long-term memory serves as a repository for capturing generalized user preferences, behavior patterns, and aggregated insights across multiple users. This long-term memory is particularly effective for improving system intelligence by identifying common user needs and optimizing general responses~\cite{priyadarshini2023human}. Meanwhile, short-term memory can focus on task-specific optimization within a single interaction session. It retains context from the immediate conversation, such as recent user inputs and system responses, to enhance relevance and coherence in real time. Short-term memory operates dynamically, clearing retained data once the session ends or the task is completed, thereby ensuring privacy and preventing unnecessary data retention.}


% However, this interaction could be further simplified by training \bma{} to learn from previous dialogues. By recognizing user preferences and common inquiries, such as identifying unusual elements in the scenario (eggshells in cooking eggs), we envision that the system could proactively address user concerns more efficiently and thus reduce the need for multiple clarifying prompts (Figure~\ref{fig.discussion_eggshell_user+app} bottom). 
\rev{For example, when identifying unusual elements during cooking (e.g., eggshells in cooking eggs), \bma{} could utilize the user's immediate input while referencing short-term memory from the current session or recent similar interactions. Additionally, by leveraging long-term memory, the system can learn from the user's past questions and query patterns to better match their habits and preferences, i.e., user's typical needs. Furthermore, multi-source data input, such as time or location information, can assist the system in inferring the user's current context, for instance, recognizing that the user is preparing a specific meal at a particular time or place, which allows the system to provide more relevant and context-aware assistance. This approach enables \bma{} to proactively anticipate user intent and deliver targeted responses, reducing the need for multiple clarifying prompts (Figure~\ref{fig.discussion_eggshell_user+app} bottom).} 

\rev{Cognitive Load Theory (CLT) suggests that well-designed interactions can significantly reduce users' extraneous load while enhancing the effective management of germane load~\cite{sweller1988cognitive, chandler1991cognitive}. Following the principles of CLT, we recommend using the above design to enable LMM-based VQA systems to minimize unnecessary clarifying prompts, thereby reducing users' cognitive load. 

}






\begin{figure*}[h!]
\centering
\includegraphics[width=0.8\textwidth]{images/discussion_eggshell_user+app+rsa.png}
\caption{Handoff between the user, \bma, and RSA for identity interpretations.}
\Description{The user submits a photo of people to Be My AI. Be My AI recognizes the requirement for identity interpretations and direct the photo to a remote sighted assistant. The remote sighted assistant returns a description of people's identities to Be My AI. Be My AI learns from the assistant's responses.}
\label{fig.discussion_eggshell_user+app+rsa}
\end{figure*}



%%%%%%%%%%%%%

\subsubsection{AI Deferral Learning for Identity Interpretations} 
\label{sec:deferral_learning}
% 


Our findings elucidated \sbma's capabilities and limitations in interpreting identity attributes. Although the system can describe aspects like gender, age, appearance, and ethnicity, it may make errors due to its reliance on stereotypical indicators and its inability to interpret non-visible details (Section~\ref{identity}).
% 
However, Stangl et al.'s work~\cite{stangl2020person} pointed out that PVI \rev{seek identity interpretations from AI assistants} across various contents, including \rev{browsing} social networking sites where our participants reported using \bma.


% human's ability 
This reveals a tension between PVI's interests in knowing about \rev{identity} attributes and the AI's challenges in providing reliable information\rev{~\cite{hanley2021computer}}. The conflict arises because attributes such as age and gender are not purely perceptual and cannot be accurately identified by visual cues alone. \rev{However}, RSAs \rev{can draw on contextual clues, past interactions, and cultural knowledge to make more nuanced observations about these human traits.} These social strategies are not typically accessible to AI systems. 
% \rev{This suggests a need for human-AI collaboration: certain tasks are best handled by AI assistants, others by human assistants, and still others through coordinated effort between both~\cite{gonzalez2024investigating}.}


% To mitigate these issues, we consider the potential benefits of AI \rev{deferral} learning that involves handoff between the user, \bma, and RSA. The process is illustrated in Figure~\ref{fig.discussion_eggshell_user+app+rsa}.

% To facilitate effective handoff between AI and human assistants,
To mitigate these issues, \rev{we propose adopting a deferral learning architecture~\cite{mozannar2020consistent, raghu2019algorithmic}, where an AI model learns when to defer decisions to humans and when to make decisions by itself. As detailed by Han et al.~\cite{han2024uncovering} and illustrated in Figure \ref{fig.discussion_eggshell_user+app+rsa}, this architecture creates a three-stage information flow:


\begin{itemize}
    \item \textbf{Stage 1}: It begins when users submit image-based queries to \bma. At this stage, the system uses a detection mechanism to identify sensitive contents, focusing particularly on those involving human physical traits. Current large-language models have already incorporated such mechanisms~\cite{perez2022red, bai2022training}; however, they still struggle to interpret human identity with consistent accuracy~\cite{hanley2021computer}.    
    
    % 
    \item \textbf{Stage 2}: Rather than declining sensitive requests outright, \bma{} redirects these queries to RSAs. This maintains the system's helpfulness while ensuring accurate responses. 
    % 
    \item \textbf{Stage 3}: RSAs provide descriptions by leveraging contextual understanding, such as analyzing the users' current environment and cultural background.
    % Additionally, RSAs employ a variety of social strategies by integrating contextual information, historical interactions, and cultural insights to deduce user attributes and preferences, ensuring that responses are not only relevant but also culturally and contextually appropriate. 
    % RSAs provide detailed descriptions by leveraging contextual understanding and social awareness.    
    %\textcolor{red}{RSAs can employ a variety of social strategies to deduce these attributes by leveraging context, prior interactions, and cultural insights.} 
\end{itemize}




In contrast to prior work that addresses stereotypical identity interpretation through purely computational approaches~\cite{wang2019balanced,wang2020towards,ramaswamy2021fair}, our proposed AI deferral learning takes a hybrid human-AI approach that combines AI capabilities with human expertise. 
% 
While previous AI-only solutions have made progress in reducing bias, they still struggle with identity interpretation~\cite{hanley2021computer}.
The challenges arise not only from technical issues but also from the ontological and epistemological limitations of social categories (e.g., the inherent instability of identity categories), as well as from social context and salience (e.g., describing a photograph of the Kennedy assassination merely as ``ten people, car'').
%
Our system leverages RSAs who have got more experience and probably more success in identifying people's identity through their human perception abilities and real-world experience. 
RSAs can interpret subtle contextual cues, understand cultural nuances, and adapt to diverse presentation styles that may challenge AI systems. 
% For example, RSAs can understand cultural markers of identity, and perceive age and ethnicity across different cultural contexts. 
Through the AI deferral learning architecture, AI assistants can learn continuously from human assistants' responses and improve its ability to handle similar situations. The three-way interactions between users, AI assistants, and RSAs generate rich contextual data that can enhance the AI system's identity detection mechanisms.} 

% Such process could also be adapted for requests involving the interpretation of social contexts \rev{like assessing animals' emotions (Section~\ref{social_stylistic})}. It can facilitate \bma{} continuously improve through observation and learning from human expertise. 

% When equipped with memory capabilities and given user consent, the AI assistant can build a knowledge base of individuals whom users frequently encounter. Moreover, the three-way interactions between users, AI assistants, and RSAs generate rich contextual data that can enhance the AI system's identity detection mechanisms.



% compared to prior works attempted to address the issues of stereotypical identity interpretation, how does the proposed AI referral learning could perform better?
% \textcolor{red}{[add prior work on identity interpretation (e.g., CV)]}

% \textcolor{red}{referral, when handoff to human,  challenges of ai referral learning is how to get the dataset. here bma get the dataset for training from rsa.}


% \rev{
% In contrast to prior work that addresses stereotypical identity interpretation through purely computational approaches~\cite{wang2019balanced,wang2020towards,ramaswamy2021fair}, our proposed AI referral learning takes a hybrid human-AI approach that combines AI capabilities with human expertise. 
% % 
% While previous AI-only solutions have made progress in reducing bias, they still struggle with identity interpretation~\cite{hanley2021computer}.
% The challenges arise not only from technical issues but also from the ontological and epistemological limitations of social categories (e.g., the inherent instability of identity categories), as well as from social context and salience (e.g., describing a photograph of the Kennedy assassination merely as ``ten people, car'').
% %
% Our system leverages RSAs who have got more experience and probably more success in identifying people's identity through their human perception abilities and real-world experience. 
% RSAs can interpret subtle contextual cues, understand cultural nuances, and adapt to diverse presentation styles that may challenge AI systems. For example, RSAs can understand cultural markers of identity, and perceive age and ethnicity across different cultural contexts. 
% Through the AI referral learning framework, \bma{} can learn from RSAs' nuanced interpretations in these scenarios, gradually improving its ability to handle similar situations. The system logs RSAs' responses and uses them as training examples, allowing \bma{} to develop more context aware in identity recognition. 
% }
% % 
% Such referral learning processes could also be adapted for requests involving the interpretation of social contexts. It can facilitate \bma{} continuously improve through observation and learning from human expertise. 


\begin{figure*}[t!]
\centering
\includegraphics[width=0.8\textwidth]{images/discussion_eggshell_user+app+rsa-check.png}
\caption{Handoff between the user, \bma, and RSA for fact-checking.}
\Description{The user submit a photo to Be My AI, Be My AI returns a description with possible AI hallucination. Then, the user directs the photo to a remote sighted assistant, who returns a corrected description to the user.}
\label{fig.discussion_eggshell_user+app+rsa-check}
\end{figure*}


%%%%%%%%%%%%%
\subsubsection{Fact-Checking for AI Hallucination Problem}
% ai hallucinations - informed by our findings, cannot be addressed, ask humans 



Our findings highlighted that the AI-generated detailed descriptions helped users understand their physical surroundings. However, there were instances where AI systems hallucinated, i.e. incorrectly added non-existent details to the descriptions, which led to confusion (Section~\ref{physical_environments}). 
% 
\rev{
In fact, hallucinations is a known problem for large language models upon which \bma{} is built~\cite{gonzalez2024investigating}. Current approaches to address this problem include Chain-of-thought (CoT) prompting~\cite{wei2022chain}, self-consistency~\cite{wang2022self}, and retrieval-augmented generation (RAG)~\cite{lewis2020retrieval}.
% 

In CoT prompting~\cite{wei2022chain}, users ask an AI model to show its reasoning steps, like solving a math problem step by step rather than simply giving the final answer. It is similar to the ``think aloud'' protocol in HCI. 
% 
Self-consistency~\cite{wang2022self} is an extension of CoT prompting. Instead of generating just one chain of thought, the model is asked to generate multiple different reasoning paths for the same task. Each reasoning path might arrive at a different answer. The model then takes a ``majority vote'' among these different answers to determine the final response.
% 
In RAG~\cite{lewis2020retrieval}, AI models are provided with relevant information retrieved from a vector storage as ``context'' to reduce factual errors in their responses.
}

\rev{
In light of these techniques, users adopt various strategies that mirror CoT prompting, self-consistency, and RAG. We outline some potential strategies below.
% with a visual (Figure~\ref{fig.discussion_eggshell_user+app+rsa-check}).
\begin{itemize}
    \item \textbf{Part-Whole Prompting}: This strategy parallels the Chain of Thought (CoT) prompting. A user sends an image to \bma{} and requests an initial overall description, followed by a systematic breakdown that justifies this description. For example, users might first ask for a description of the image as a ``whole'', then request to divide the image into smaller ``parts'', like a $3 \times 3$ grid, and describe each grid individually. If the descriptions of the individual parts align coherently, it increases the likelihood that the overall description is accurate. This approach would require processing more information; however, it will provide users with greater confidence in the AI's response, as it enables them to verify consistency between the whole and its constituent parts.  
    % 

    \item \textbf{Prompting from Multiple Perspectives}: This strategy resembles the self-consistency technique. A user sends an image to \bma{} and requests multiple descriptions from different perspectives. For example, users might ask for one description that focuses on the background and another that emphasizes foreground objects. Users can also request descriptions from the viewpoint of objects within the image (e.g., ``How would a person sitting on a chair see this scene?'' and ``How would a person sitting on the floor see this scene?''). While gathering descriptions from multiple perspectives may increase the likelihood of hallucination, it can also help identify common elements that appear consistently across different viewpoints, potentially indicating true features of the image.    
     % 
    \item \textbf{Prompting with Human Knowledge}: This strategy resembles the RAG approach. A user sends an image to \bma{}, provides their current understanding of the image and its context, and requests a description that complements their knowledge. For example, in Figure~\ref{eggshells}, users can specify that someone took the picture in a kitchen environment and that it should show a frying pan containing eggs. Users possess this knowledge through their familiarity with physical environments, self-exploration, spatial memory, and touch~\cite{gonzalez2024investigating}. The user-provided knowledge will help the AI model ground its response in an accurate context~\cite{liu2024coquest}.     
    % 
    \item \textbf{Pairing with Remote Human Assistants}: While the previous three strategies rely on multiple prompting and response aggregation to identify facts, this approach leverages the traditional remote sighted assistance framework.
    This strategy (shown in Figure~\ref{fig.discussion_eggshell_user+app+rsa-check}) differs from the deferral learning framework (Section~\ref{sec:deferral_learning}) in that users forward the AI responses to human assistants, rather than the AI assistant deferring to humans for the response.
    In this strategy, a user first sends an image to \bma{} to receive a description. When users suspect inaccuracies through triangulation~\cite{gonzalez2024investigating}, such as descriptions that conflict with their spatial memory or common sense (e.g., implausible objects like a palm tree in a cold region), they can request a RSA to fact-check the description. The RSA then verifies the description and sends corrected information back to the user. This verification process is likely easier and faster for a RSA than composing a description from scratch, as the RSA's work involves checking rather than creating content.
\end{itemize}
}

\rev{
In summary, AI hallucination presents both challenges and opportunities. By addressing these issues, future work will strengthen the way users, AI models, and human assistants interact with each other.
}















%%%%%%%%%%%%%
%%%%%%%%%%%%%
\subsection{Towards Multi-Agent Systems for Assisting Visually Impaired Users}


%%%%%%%%%%%%
This section examines the transition from human-human interactions to human-AI and AI-AI systems in supporting PVI. We explore how these multi-agent systems, which involve the collaborative efforts of multiple agents (AI or human), are designed to adaptively meet the diverse needs of PVI. 


Lee et al.~\cite{lee2020emerging} identified four contexts in which a professional human-assisted VQA system (Aira) offer support to PVI. The type of information required by PVI is incremental in these contexts. 
First, \textit{scene description} and \textit{object identification} acquire information about ``what is it.''
Second, \textit{navigation} requires description about PVI's surroundings and obstacles (``what is it'') and directional information (``where is it'' and ``how to get to the destination'').
Third, \textit{task performance} like putting on lipstick, cooking, and teaching a class. This context requires description (``what is it'') and domain knowledge on ``how to do it.'' 
% directional information (``where is the intended object'' and ``how to get it''), as well as
Forth, \textit{social engagement} like helping PVI in public spaces or interacting with other people. This needs description (``what is it''), directional information to navigate in social space, and discreet communication (PVI prefer not to disclose their use of VQA systems).


Our study reported how participants used \bma{} for tasks like matching outfits and assessing makeup, fitting under the category of \textit{task performance}. Some participants raised concerns about the accuracy of \sbma's interpretations and suggestions, indicating their preference for human subjectivity in this context. 
% 
Contrasting with this, Lee et al.'s work~\cite{lee2020emerging} highlighted that remote sighted assistants (RSAs), even those professionals RSAs from Aira, sometimes lack the specialized information or domain knowledge required in task performance, thereby they need to collaborate with other RSAs to find solutions. 


Furthering this investigation, Xie et al.~\cite{xie2023two} paired two RSAs to assist one visually impaired user in synchronous sessions, validating the need for RSAs to complement each other's description in task performance like aiding the user in applying makeup and matching outfits. They also explored the challenges in this human-human collaboration, revealing collaboration breakdowns between two opinionated RSAs. 
% 
To address these issues, they proposed a collaboration modality in which one ``silent'' RSA supports the other RSA by researching but not directly communicating. This approach suggested that two RSAs in this multi-agent system should not deliver information simultaneously but have a clear division of labor, designating who takes the lead, to avoid overwhelming PVI with information. 


% AI-RSA handoff (section 5.1) to AI-AI (GPT + expert AI)
Transitioning from human-human to human-AI collaboration, the handoff between the user, \bma{} and RSA (Section~\ref{handoff}) opens up new opportunities for multi-agent systems. 
% 
Our proposed modality of human-AI collaboration integrates the scalable, on-demand capabilities of AI-based visual assistance with the contextual understanding and adaptability of RSAs. 
This multi-agent system involves the AI system recognizing its own limitations and seamlessly handing off tasks to a RSA when appropriate. This collaboration aligns with prior work~\cite{xie2023two}, where AI (\bma) and human (RSA) maintain a clear division of labor, minimizing cumbersome back-and-forth and reducing potential confusion for PVI. 



Looking ahead, we envision the potential for AI-AI collaboration as part of the future multi-agent systems to assist PVI, especially for task performance. 
A domain-specific AI expert can be trained to handle more specialized tasks such as matching outfits, performing mathematical computations, or answering chemistry-related questions. \bma, as the core AI system, can provide general visual descriptions (``what is it'') and delegate more specialized tasks requiring domain knowledge to the domain-specific AI expert. 
% 
This approach is in line with the human-AI collaboration (Section~\ref{handoff}) by ensuring effective handoffs when necessary. By leveraging AI agents with more specialized capabilities, this multi-agent system can better adapt to PVI's needs. 



However, similar to concerns around human-human and human-AI interactions, these AI-AI collaborations must be carefully designed with clear protocols and handoff points for transitioning tasks between AI agents. It is important to make these transitions as seamless and transparent as possible to PVI, thereby avoiding any complexity or confusion. 




%%%%%%%%%%%
%%%%%%%%%%%
\subsection{Towards Real-Time Video Processing in LMM-based VQA Systems}


One of the most significant advantages of \bma{} and other LMM-based assistive tools is their ability to provide contextually relevant and personalized assistance to users. By leveraging machine learning and natural language understanding, these systems can understand and respond to a wide range of user queries. This level of contextual awareness represents a significant advancement over pre-LMM-based assistive technologies, which often fail to adapt to the diverse needs and preferences of individual users.


However, our findings also identified several challenges and limitations associated with the reliance on static images by current LMM-based assistive tools. 
Participants in our study reported frustration with the need to take multiple pictures to capture the desired information, a process they found time-consuming and cognitively demanding (Section~\ref{realtime_feedback}). This iterative process hinders efficiency and also poses safety risks, as participants struggled with taking images while navigating around obstacles.


% \rev{[real-time video process can also facilitate the transition from conveying ``what'' to ``how'' questions]}
To mitigate these issues, integrating real-time video processing capabilities into future LMM-based VQA systems could offer significant benefits. 
% 
Our findings suggest that the dynamic nature of video serves as a foundation for subsequent guidance (Section~\ref{realtime_feedback}), which is currently provided by human assistants through video-based remote sighted assistance.
% 
Shifting to real-time video processing would allow LMM-based VQA systems to transition from identifying objects (answering ``what is it'') to offering practical advice (addressing ``how to do it''), such as how to adjust the camera angle or how to navigate to a destination.
% 
By continuously analyzing the user's surroundings through real-time video feeds, these systems can dynamically interpret changes and provide immediate feedback, thus eliminating the need for static image captures. This capability would enhance the user experience by offering seamless navigation aid in real time. 


The feasibility of real-time video processing is supported by existing technologies demonstrated in commercial products and research prototypes. For instance, systems that utilize sophisticated algorithms for real-time object segmentation in video streams~\cite{wang2021swiftnet} have shown significant potential in other domains. Building on these techniques for video analysis could significantly extend the capabilities of future LMM-based VQA systems. 


Transitioning from static image analysis to real-time video processing can alleviate the burden of iteratively taking pictures and adjusting angles experienced by users. It can also enhance the utility and safety of LMM-based VQA systems, particularly during navigation. 
% 
This progression, driven by ongoing advancements in machine learning and computer vision, is essential for the development of more adaptive and responsive assistive technologies that align with the dynamic nature of real-world environments.




% Our findings highlight a limitation in the capabilities of \bma{} when it comes to providing actionable, goal-oriented guidance to PVI (Section~\ref{appliances}). While \bma{} is good at conveying ``what'' information with most of time accurately describing the visual content of a scene or object, such as identifying a thermostat on the wall. 
% 
% It often struggles with providing ``how'' information, guiding users on the specific actions required to interact with or operate elements in their environment, such as how to adjust the theromstat. This ``what'' and ``how'' divide poses a major challenge to the effectiveness and usability of \bma{}, as PVI rely on it not only for understanding their surroundings but also for completing tasks and achieving their goals.



% To address this limitation and design more goal-oriented AI-powered assistant prosthetics, we propose the following key design implications. 
% % 
% Future AI-powered assistive technologies should be designed with a focus on action-oriented reasoning and task-specific guidance. Although this could be achieved through further user inquiries~\cite{truhn2023large}, integrating knowledge bases~\cite{zhu2014reasoning} and event/behavior reasoning engines~\cite{chen2008using} to enable contextual inference of actions and intentions, and associating visual elements' feedback with reasoning, would greatly reduce the cognitive burden on PVI and enhance the user experience. By leveraging this knowledge, assistive technologies can provide more relevant and actionable guidance to PVI, helping them effectively navigate and interact with their environment to complete desired tasks. Our findings emphasize the importance of human-centered design principles, particularly in the design of assistive technologies, which should be reinforced through a goal-oriented technical roadmap that adapts to users' needs, preferences, and external environments~\cite{fischer2001user,amershi2014power}. By emphasizing action goal-oriented reasoning~\cite{huffman1993goal,letier2002agent}, future AI-powered prosthetics will be optimized, further benefiting PVI.






% reference about real-time image processing:
% By continuously analyzing the user's surroundings and providing relevant information without the need for explicit image capture, these systems could offer a more seamless and efficient user experience. The integration of \textcolor{red}{real-time image processing} into AI-powered VQA systems aligns with the growing availability of commercial products and research prototypes that leverage advanced object detection and text recognition technologies. For example, Microsoft's Seeing AI \cite{SeeingAI2020} and various currency recognition systems \cite{liu2008camera, parlouar2009assistive, paisios2012exchanging} demonstrate the feasibility and potential impact of real-time image processing in assistive technology. By building upon these existing approaches and incorporating state-of-the-art deep learning techniques for object detection \cite{girshick2014rich, girshick2015fast, ren2016faster, krizhevsky2017imagenet} and text recognition \cite{ma2018arbitrary, he2017deep, zhou2017east, yao2016scene, liu2017deep, lyu2018multi}, future AI-powered VQA systems could provide even more robust and reliable assistance to users.







% \subsection{Subjectivity Interpretations}
% % positive, identity
% know gender, race, humans have problem in identifying it, humans also guess but not make public mistake. not purely perceptual. discussion: neutral, conflict. tension between prior work and this one. only objective info and pvi will use social skills. intent in physical social gathering - limitations of prior work, digital interactions
% % https://dl.acm.org/doi/10.1145/3313831.3376404



% % negative, social cues
% [Discussion:] ai, human agency, undermine pvi as people, they can understand the social meanings. compare to human agents, subjectivity comes from ai no verification 








% \subsection{Design Implications for AI-Powered Assistant Prosthetics} %Albert
% % Design Implications: How AI-Powered Visual Question Answering Should Look Like
% As demonstrated by our research, \bma{} has shown significant potential in empowering PVI by providing a more intuitive, user-friendly, and context-aware assistive experience. However, to further enhance the usability and effectiveness of AI-powered VQA systems like \bma{}, several design implications should be considered.
% %lmm: large multi model 
% %
% %Suggest features that could improve \bma's usability, like real-time image processing to reduce the need for multiple pictures.
% %
% %Recommend design changes that might help \bma become more goal-oriented and contextually aware, such as advanced machine learning models trained on diverse environments.



% \subsubsection{Towards Goal-Oriented AI-Powered Assistant Prosthetics}

% %bma good at what information (``what'' is about the visual content) but not ``how'' information
% %
% %(Section~\ref{appliances}) recognize thermostat (what), but not adjust dial (how)

% Our findings highlight a limitation in the capabilities of \bma{} when it comes to providing actionable, goal-oriented guidance to PVI (Section~\ref{appliances}). While \bma{} is good at conveying ``what'' information with most of time accurately describing the visual content of a scene or object, such as identifying a thermostat on the wall. 
% % 
% It often struggles with providing ``how'' information, guiding users on the specific actions required to interact with or operate elements in their environment, such as how to adjust the theromstat. This ``what'' and ``how'' divide poses a major challenge to the effectiveness and usability of \bma{}, as PVI rely on it not only for understanding their surroundings but also for completing tasks and achieving their goals.

% To address this limitation and design more goal-oriented AI-powered assistant prosthetics, we propose the following key design implications. 
% % 
% Future AI-powered assistive technologies should be designed with a focus on action-oriented reasoning and task-specific guidance. Although this could be achieved through further user inquiries~\cite{truhn2023large}, integrating knowledge bases~\cite{zhu2014reasoning} and event/behavior reasoning engines~\cite{chen2008using} to enable contextual inference of actions and intentions, and associating visual elements' feedback with reasoning, would greatly reduce the cognitive burden on PVI and enhance the user experience. By leveraging this knowledge, assistive technologies can provide more relevant and actionable guidance to PVI, helping them effectively navigate and interact with their environment to complete desired tasks. Our findings emphasize the importance of human-centered design principles, particularly in the design of assistive technologies, which should be reinforced through a goal-oriented technical roadmap that adapts to users' needs, preferences, and external environments~\cite{fischer2001user,amershi2014power}. By emphasizing action goal-oriented reasoning~\cite{huffman1993goal,letier2002agent}, future AI-powered prosthetics will be optimized, further benefiting PVI.
 


% \subsubsection{Towards Real-Time Processing AI-Powered Assistant}
% One of the most significant advantages of \bma{} and other LMM-based assistive tools is their ability to provide contextually relevant and personalized assistance to users. By leveraging the power of machine learning and the ability for understanding of natural language, these systems can understand and respond to a wide range of user queries. This level of contextual awareness represents a significant advancement over traditional assistive technologies, which often struggle to adapt to the diverse needs and preferences of individual users.

% However, our findings also identify several challenges and limitations of current LMM-based assistive tools, particularly in terms of their reliance on user-generated images. Participants in our study reported frustration with the need to take multiple pictures to capture the desired information, which can be time-consuming and cognitively demanding (Section~\ref{navigation}). To address this issue, we envision the integration of real-time image processing capabilities into future AI-powered VQA systems. By continuously analyzing the user's surroundings and providing relevant information without the need for explicit image capture, these systems could offer a more seamless and efficient user experience. The integration of real-time image processing into AI-powered VQA systems aligns with the growing availability of commercial products and research prototypes that leverage advanced object detection and text recognition technologies. For example, Microsoft's Seeing AI \cite{SeeingAI2020} and various currency recognition systems \cite{liu2008camera, parlouar2009assistive, paisios2012exchanging} demonstrate the feasibility and potential impact of real-time image processing in assistive technology. By building upon these existing approaches and incorporating state-of-the-art deep learning techniques for object detection \cite{girshick2014rich, girshick2015fast, ren2016faster, krizhevsky2017imagenet} and text recognition \cite{ma2018arbitrary, he2017deep, zhou2017east, yao2016scene, liu2017deep, lyu2018multi}, future AI-powered VQA systems could provide even more robust and reliable assistance to PVI.



% \subsubsection{Towards Reliable AI-Powered Assistant Prosthetics}



% Our findings highlight the significant potential of AI-Powered assistive technologies like \bma{} in enhancing the perception and understanding of surroundings for PVI. However, our study also reveals a notable drawback of AI-powered assistant prosthetics, namely AI hallucinations. These errors, where the artifact inaccurately identifies objects that aren't present, can lead to confusion and mistrust among users. Participants in our study reported instances where \bma{} erroneously added non-existent details to scenes (Section~\ref{scene}). 
% It can be argued that the presence of AI hallucinations poses a major challenge to the reliability and robustness of AI-powered assistive prosthetics. If users cannot trust the information provided by these systems, their effectiveness as cognitive extensions is severely compromised. This issue is particularly critical for PVI, who rely on these technologies to navigate and make sense of their environment.

% To address the problem of AI hallucinations, our participants applied various strategies, such as consulting human assistants for verification or relying on their own knowledge to identify inaccuracies. While these strategies showcase PVI's adaptability and problem-solving skills, they also highlight the need for more reliable and robust AI-powered assistive technologies.

% One potential approach to mitigating AI hallucinations is to incorporate uncertainty estimation and communication into the design of these technologies. By quantifying and conveying the confidence level of predictions, AI-powered assistive systems can help users assess the reliability of the information provided. This approach has been explored in the context of other AI-based systems, such as medical diagnosis~\cite{leibig2017leveraging,begoli2019need} and autonomous vehicles~\cite{michelmore2018evaluating}.

% Another strategy is to develop AI-powered assistive technologies that can learn and adapt to user feedback over time. By allowing PVI or human assistants to correct errors and provide input to improve system performance, which is not only able to continuously improve system's reliability and robustness and also help aware users about possible AI hallucinations. This approach aligns with the principles of interactive machine learning, emphasizing the importance of human-in-the-loop learning for AI systems~\cite{amershi2014power,retzlaff2024human}.

% In addition to these technical solutions, involving PVI in the design and evaluation of AI-powered assistive technologies is crucial. By engaging users as co-designers and co-evaluators, researchers and designers can gain a better understanding of the challenges and requirements of PVI, leading to the development of more reliable and robust systems. This participatory design approach has been widely advocated in the assistive technology domain~\cite{frauenberger2015pursuit,lee2004trust,zhang2023redefining}.




% \subsubsection{Envisioning Multimodal AI-Powered Assistant Prosthetics} 

% %integrate multimodal input, participants can use gestures, touch, haptic feedback to interact with \bma. 

% %llm

% Our research demonstrates the superior ability and performance of \bma{} compared to previous AI-powered systems when handling various tasks. In tasks such as object recognition, processing complex information, and interpreting graphical elements, \bma{} not only completes the tasks but also provides extended explanations and further task collaboration capabilities. 
% This undoubtedly greatly expands the interaction scenarios and user experience of \bma{}. 

% However, as mentioned earlier, the current \bma{} faces challenges, including higher interaction costs due to multiple photo captures and AI hallucinations.
% % previously mentioned, we found that the current \bma{} still has some challenges and limitations in terms of interaction experience, such as the additional interaction costs caused by the need for multiple photo captures and AI hallucinations.
% % 
% Apart from integrating additional advanced technologies to expand \bma{}'s capabilities, we also suggest adopting multimodal interaction. This approach, a key design insight from our research, underscores the value of diverse interactions and feedback in AI-powered assistive tools.
% % apart from integrating other advanced technologies to expand \bma{}'s capabilities, we suggest introducing the concept of multimodal interaction, which is another key design implication derived from our research, emphasizing the importance of multimodal interaction and feedback in artificial intelligence assistive tools. 
% % 
% Although \bma{} primarily relies on voice-based input and output, integrating multimodal interaction methods, such as touch, gestures, and haptic feedback, will greatly enhance the usability and accessibility of LMM-based assistive functions. This perspective of combining multimodal interaction has been widely studied in some previous literature \cite{turk2014multimodal, reeves2004guidelines}.
% % oviat2017handbook
% For example, touch-based gestures can enable users to navigate the system interface more effectively, while haptic feedback can provide additional spatial and contextual information to supplement voice output. These multimodal interactions also allow \bma{} to adapt and personalize to better meet the diverse needs of PVI. Future research should further explore how to design and implement these multimodal interaction technologies and assess their long-term impact on PVI user experience and quality of life.

% %Our research demonstrates the superior ability and performance of \bma{} compared to previous systems and tools when handling various tasks. For instance, in tasks such as object recognition, processing complex information, and interpreting graphical elements, \bma{} not only accomplishes the tasks but also provides \textcolor{red}{multimodal input,} extended explanations, and further task collaboration capabilities. This undoubtedly greatly expands the interaction scenarios and user experience of \bma. Traditional applications and tools, such as those using conventional OCR technology and rapid reading applications like ``Seeing AI,'' are considered effective. \textcolor{red}{[cite:] However, users find it challenging to process a complete task flow using such tools, as their functions are designed for single tasks.} In other words, traditional applications and tools can only complete a specific step in a task flow and cannot connect the context of the entire task. Leveraging the LMM's ability to understand multimodal data, interact using natural language, and be context-sensitive, \bma{} can comprehend more practical needs and provide personalized assistance to users based on their requirements. Furthermore, \bma{} demonstrates its potential to replace previous assistive methods and technologies. \textcolor{red}{This is particularly evident in a series of scenarios where \bma{} substitutes RSA services, such as when participants use \bma{} for reading and control tasks. These tasks, which are considered replaceable by \bma{}, are common needs of PVI in RSA services. }

% %Another key design implication that emerged from our research is the importance of multimodal interaction and feedback in AI-powered assistive tools. While \bma{} primarily relies on voice-based input and output, \textcolor{red}{participants expressed interest in exploring other interaction modalities, such as touch, gestures, and haptic feedback.=> integrate other interaction modalities, such as touch, gestures, and haptic feedback, into \bma} The incorporation of multimodal interaction techniques, which have been extensively studied in the HCI literature \cite{turk2014multimodal, reeves2004guidelines, oviatt2017handbook}, could greatly enhance the usability and accessibility of LMM-based assistive tools. For example, touch-based gestures could enable users to navigate the system's interface more efficiently, while haptic feedback could provide additional spatial and contextual information to supplement voice output.

%\subsubsection{future ai assistant prosthetic}
%from 5.4, about design process? 
%
%integrate new tech: generative ai 
%next iteration




% \subsubsection{Opportunities in Human-AI Collaboration}
% % Evolution of Human-AI Collaboration
% % Future AI Assistant Prosthetic - From Temporal Dimension to Consider Design Implications
% Again, while our findings demonstrate the significant potential of AI-powered assistive technologies in enhancing PVI's capabilities across various contexts, participants also highlighted some limitations of these technologies. However, we argue that these limitations may be part of an adjustment period as PVI learn to interact with and adapt to AI-powered assistive technologies.


% This observation aligns with the findings of~\citet{10.1145/3563657.3595977}, who described how users, when initially using AI-powered assistive technologies, may encounter responses that do not match their expectations. Despite this, most participants were able to quickly learn and adapt to the system, reaching a point where they could collaborate with \bma{} to complete tasks and perceive it as an effective technology.

% This contradictory state may give rise to a form of Human-AI ``confrontation.'' However, this ``confrontation'' appears to diminish as the system undergoes iterative upgrades and PVI update their understanding of the system's capabilities. From a phenomenological perspective, Ihde's theory~\cite{ihde1991instrumental} suggests that the science of tools evolves alongside people's cognition, implying that the relationship between PVI and AI-powered assistive technologies is not static but rather a dynamic process of mutual adaptation and growth.


% As PVI become more familiar with the capabilities and limitations of AI-powered assistive technologies, they develop strategies to leverage these technologies effectively and compensate for the shortcomings. This process of adaptation and learning is a critical aspect of the distributed cognition framework, which emphasizes the role of artifacts and the environment in shaping cognitive processes \cite{hollan2000distributed}.


% Moreover, the iterative nature of AI-powered assistive technologies, with ongoing updates and improvements, allows for a continuous refinement of the Human-AI interaction. As these technologies become more sophisticated and attuned to the needs and preferences of PVI, the initial ``confrontation'' may give way to a more seamless and synergistic collaboration between PVI and AI-powered assistive technologies.

% This perspective highlights the importance of considering the temporal dimension of design implications in the context of AI-powered assistive technologies. Rather than viewing the limitations of these technologies as static barriers, we should recognize the potential for PVI to adapt and develop new cognitive strategies in response to these limitations, and for the technologies themselves to evolve in response to user feedback and needs.


%%%%%%%%%%
% \subsubsection{Ethical Considerations and User Involvement in the Design of AI-Powered Assistive Technologies}
% As we continue to explore the potential of AI-powered assistive technologies for PVI, it is crucial to involve the target users in the design and evaluation process \cite{10.1145/3025453.3025899}. Our research with \bma{} underscores the importance of actively engaging with PVI throughout the design process to ensure that these systems are tailored to their specific needs, preferences, and contexts of use. By involving users as co-designers and co-evaluators, we can create assistive tools that are not only technologically advanced but also truly empowering and inclusive.

% Moreover, the development of AI-powered assistive technologies should be guided by a commitment to ethical and responsible innovation~\cite{floridi2018ai4people,jobin2019global}. As LMM-based systems become increasingly sophisticated and integrated into users' daily lives, it is essential to consider issues of privacy, security, and fairness. Researchers and designers should ensure these technologies are transparent, accountable, and aligned with the values and goals of the communities they serve~\cite{10.1145/3351095.3372873}.

% In conclusion, our research with \bma{} highlights the immense potential of LMM-based assistive tools for PVI while also revealing key design implications for future AI-powered VQA systems. By integrating real-time image processing, supporting multimodal interaction and feedback, and prioritizing user-centered design principles, we can create assistive technologies that are more usable, effective, and empowering. As the field of AI-powered assistive technology continues to evolve, it is essential for HCI researchers and practitioners to collaborate with PVI's communities to ensure that these innovations are developed in an inclusive, ethical, and responsible manner.



%\subsection{\textcolor{red}{Most assistive technologies die after their introduction}}
%
%\textcolor{red}{[combine with design implications, with sub-headings]}

%Assistive technologies for PVI have undergone a rapid evolution in recent years~\cite{10.1145/3597638.3608412}, with the introduction of LLM-based applications like \bma{} marking a significant shift in the landscape. \textcolor{red}{Traditional assistive technologies, such as VQA}~\cite{bigham2010vizwiz_nearly,BeMyEyes2020}, screen readers and braille displays~\cite{muhsin2024review,alves2009assistive}, have long been the primary methods for blind individuals to access the content and navigate the world. However, assistive technology is undergoing a transition from human-driven to AI-driven~\cite{xu2023transitioning, 10.1145/3234695.3239330}, and the emergence of a series of technology-dominated assistive services has challenged the dominance of these traditional technologies, offering a more intuitive, user-friendly, and versatile experience for blind users. Our research further reveals that \bma{} has elevated AI-led assistive technology to new heights, which inevitably leads one to imagine the possibility of a future where assistive technology is fully AI-dominated.
%
%Despite the initial promise and potential of many traditional assistive technologies, a significant number of them gradually lost their appeal after the emergence of applications like \bma{}, and may eventually die out after users fully adopt \bma{}. This phenomenon can be attributed to several factors, including the lack of comprehensive functionality, high complexity of operation, social difficulties and limited usage scenarios~\cite{gori2016devices,manjari2020survey,tapu2020wearable}. These technologies failed to meet the evolving needs and expectations of blind users in one or more aspects, while \bma{} seems to offer a promising solution that integrates and expands the capabilities of various traditional assistive technologies.
%
%
%While \bma{} is not perfect at present, the introduction of \bma{} represents a significant milestone in the development of assistive technologies for blind individuals. These applications offer new possibilities and empower PVI in unprecedented ways. Building on this foundation, it is crucial to approach their development and adoption from a user-centered perspective, ensuring that they are accessible, inclusive, and responsive to the diverse needs of the blind community~\cite{10.1145/3025453.3025895,federici2012assistive}.
%%%%%%%%%%%%%%%%%%%%%%%%%%%%%%%%%%%%%%%%
%this paragraph can be removed.
%The rise and fall of traditional assistive technologies, coupled with the emergence of LLM-based applications \bma{}, offer valuable lessons for the future development and adoption of assistive tools for PVI. First and foremost, user-centered design and continuous adaptation must be at the heart of assistive technology development to ensure that these tools truly meet the evolving needs and expectations of blind users. Second, the success of LLM-based applications highlights the importance of leveraging advanced technologies, such as natural language processing and machine learning, to create more intuitive and versatile assistive tools. Finally, the challenges and concerns surrounding the adoption of LLM-based applications underscore the need for ongoing research, collaboration, and dialogue among researchers, developers, and blind users to ensure that these technologies are developed and deployed in an accessible, inclusive, and responsible manner.


%The introduction of \bma{} represents a significant milestone in the evolution of assistive technologies for blind people. While these applications offer new possibilities and empower blind individuals in unprecedented ways, it is crucial to approach their development and adoption with a user-centered perspective, ensuring that they are accessible, inclusive, and responsive to the diverse needs of the blind community. As the field of assistive technology continues to evolve, researchers and developers must remain committed to creating tools that truly enhance the lives of blind individuals and promote their full participation in the digital world. %The decline of traditional assistive technologies in the face of LLM-based applications serves as a reminder of the importance of continuous innovation, adaptation, and user-centered design in the quest to create a more accessible and inclusive world for all.





% \newpage
\bibliographystyle{ACM-Reference-Format}
\bibliography{references}

\appendix
% \clearpage
\cleardoublepage
\section{Appendix}

\subsection{Details of Evaluation Settings}\label{asec:eval-setting}

\subsubsection{Benchmarks}\label{asec:eval-setting-bench}

Since PLPHP maintains the computational integrity of the LVLMs' Prefilling Stage, its efficiency advantage is primarily reflected in the low decoding latency during the subsequent Decoding Stage. Therefore, we mainly choose benchmarks composed of open-ended VQA and image captioning tasks. The benchmarks we select encompasses both multi-image task benchmarks and single-image task benchmarks.

$\bullet$ \textbf{Multi-Image benchmarks}: The LLaVA-Interleave Bench is a comprehensive benchmark dataset designed to evaluate the performance of LVLMs in multi-image scenarios. It consists of 13 challenging tasks with a total of 17,000 instances. We curated four subsets consisting of open-ended VQA tasks from LLaVA-NeXT-Interleave-Bench: Spot-the-Diff, Image-Edit, Visual-Story-Telling, and Multi-View.

$\bullet$ \textbf{Single-Image benchmarks}: The Flickr30k dataset is a widely used benchmark in the field of image captioning and visual understanding. It consists of 31,783 images collected from the Flickr platform, each paired with five human-annotated captions. The COCO2017 Caption subset contains more than 45,000 images, each annotated with five captions written by human annotators, describing the visual content of the images in detail, including objects, their attributes, and the relationships between them. DetailCaps4870 provides more fine-grained and specific image content descriptions than standard captioning datasets, which is more useful for efficiency analysis. 

\subsubsection{Baselines}\label{asec:eval-setting-baseline}

We select FastV and VTW as our baselines in our experiments. Notably, FastV offers two versions of implementation: one that supports KV cache and one that does not. Since the non-KV-cache implementation introduces substantial computational overhead, we use the version that supports KV cache to ensure a fair comparison. For both of the baselines, we refer to the official open source code \footnote{\url{https://github.com/pkunlp-icler/FastV}} \footnote{\url{https://github.com/lzhxmu/VTW}} and implement them on the models we evaluate.

\subsubsection{Models}\label{asec:eval-setting-impl}

For Qwen2-VL, we set \texttt{max\_pixels} to $1280 \times 28 \times 28$ and \texttt{min\_pixels} to $256 \times 28 \times 28$ according to the official recommendation. The Mantis model that we choose is Mantis-8B-SigLIP-LLaMA3. For LLaVA-OneVision and Mantis, we use the official original versions \footnote{\url{https://huggingface.co/lmms-lab/llava-onevision-qwen2-7b-ov}} \footnote{\url{https://huggingface.co/TIGER-Lab/Mantis-8B-siglip-llama3}}, while using the versions provided by the transformers library \cite{wolf-etal-2020-transformers} for all other models.

\subsection{Case Study}

To showcase the effectiveness of our proposed method, we present a series of case studies in the form of multimodal chatbots, as shown in Figure \ref{fig:case-studies}.

\begin{figure*}[ht]
	\centering
	\subfloat[]{
		\includegraphics[width=0.48\textwidth]{figs/appendix-case3.png}}
        \subfloat[]{
		\includegraphics[width=0.48\textwidth]{figs/appendix-case4.png}}
        \\ \quad \\ \quad \\
	\subfloat[]{
		\includegraphics[width=0.48\textwidth]{figs/appendix-case1.png}}
        \subfloat[]{
		\includegraphics[width=0.48\textwidth]{figs/appendix-case2.png}}
 %        \\
	% \subfloat[]{
	% 	\includegraphics[width=0.9\textwidth]{figs/appendix-case5.png}}
	\caption{\textbf{Multimodal Chatbots with different pruning methods.}}
		\label{fig:case-studies}
\end{figure*}

\end{document}
\endinput
%%
%% End of file `sample-sigconf-authordraft.tex'.
 