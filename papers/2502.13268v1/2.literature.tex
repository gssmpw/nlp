\section{Related Literature}

Making simplifying assumptions to operate at a required abstract level is central to machine learning practice \cite{selbst2019fairness,saitta2013abstraction}. Abstraction inherently involves making assumptions about what is necessary and what is not. For instance, ignoring certain features or choosing a specific representation in data abstracts out certain social contexts and interactions that may be assumed as nonessential. As much as these traits have contributed to the rise of ML applications in diverse domains, the last decade has seen an increasing number of concerns arising from abstraction and assumptions made about human behaviors and characteristics in automated decision-making systems \cite{benjamin2019race,noble2018algorithms,o2017weapons,eubanks2018automating}. While assumptions form an essential constituent of many prior works on how practitioners use ML, in section \ref{rel:periphery}, we review how \rev{prior works in HCI and related disciplines} often place assumptions on the periphery. Then, in section \ref{rel:core}, we discuss how the concept of an \textit{argument} in Informal Logic can offer a new perspective to think about and act on the confusions surrounding assumptions in ML.

\subsection{Assumptions on the Periphery}
\label{rel:periphery}

% \rev{\textbf{Assumptions as Marginal Disruptor.}} 
Prior works in HCI, ML fairness, and AI ethics have extensively looked into how practitioners use ML systems and interact with different phases of an ML workflow (for e.g., \rev{\cite{zhangHowDataScience2020,yang2018investigating,wang2023designing,muller2020interrogating}}.) Many of these works have brief discussions about assumptions or at least mention the word ``assumption'' when analyzing practitioners' interactions with ML-based systems. However, most of these references assign a marginal causal agency to assumptions for disrupting a desired state or chain of actions: some common references to assumptions include phrases such as \textit{``The result is often erroneous assumptions [made by practitioners] about what users would want from AI.''} \rev{\cite[p.~12]{subramonyam2022human}}, \textit{``...they assumed that succinct answers were sufficient.''} (indicating undesired documentation practices) \rev{\cite[p.~17]{heger2022understanding}}, and \textit{``participants who assumed sex was a sensitive feature attempted to mitigate biases in the ML pipeline by simply removing...''} (explaining undesired actions sequence) \rev{\cite[p.~5]{dengExploringHowMachine2022}}.

\rev{In addition to model efficiency,} the desired state or actions discussed in these prior works often revolve around sociotechnical concerns related to fairness, transparency, or collaboration. For instance, some prior works refer to practitioners' assumptions as one of the factors hampering their collaborative efforts with stakeholders of different technical backgrounds \cite{yang2018investigating,varanasi2023currently,wang2023designing}; a few others discuss how assumptions distort practitioners' understanding of ethical or fairness issues \cite{boyd2021datasheets,dengExploringHowMachine2022,aragon2022human,jarrahi2022principles}. 
\rev{In all of these works, assumptions are often discussed peripherally to the main discussion about a desired state or action, such as efficiency or fairness. 
Specifically, along with other factors such as institutional constraints and incentives, assumptions are discussed as one of the factors that affect practitioners' attainment of a goal.}

% Though assumptions are only superficially discussed per se in studying practitioners' usage, 
% \smallskip
% \noindent \textbf{Assumptions as Object of Enumeration.}
\rev{As assumptions are often discussed in terms of their effect,}
their influence in practice is well-appreciated in responsible ML discourse \cite{mitchell2021algorithmic,jarrahi2022principles,aragon2022human,malik2020hierarchy}. 
Consequently, research in HCI and responsible ML has developed numerous toolkits \cite{wong2023seeing} (refer to frameworks, guidelines, etc.) to invoke assumptions in everyday practice \rev{\cite{Sadek2024,Lavin2022,Smith_undated,rismani2023plane}}. There are also a significant number of these toolkits---some very popular such as model cards \cite{mitchellModelCardsModel2019} or datasheets \cite{gebruDatasheetsDatasets2021}---that do not include direct interrogative questions to invoke assumptions, but rather frame questions on various phases of an ML workflow intended to unearth hidden assumptions \rev{\cite{Pushkarna2022,Kawakami2024,Raji2020,elsayed2023responsible}}. 
\rev{In a recent work, \citet{kommiya2024towards} recommend that practitioners deliberately connect, in addition to listing down, the known assumptions to the desired states.}   
However, most of \rev{these} toolkits \rev{and guidelines} are only suggestive, simply prompting the practitioners to list down and document assumptions in some form. While the authors of these toolkits argue that this process intrinsically will help mitigate the undesired consequences of making assumptions, it remains unclear how practitioners actually \textit{conceptualize} and \textit{work through assumptions} when prompted with ``what are the assumptions'' or ``list down the assumptions'' type questions. Though one of the primary intentions of these toolkits is to make practitioners elicit and reflect on the assumptions, the methodology adopted in the toolkits gives meager attention to \rev{the practical steps that practitioners undertake during} assumption articulation, identification, and handling. 

% -- also consequently a substantive number of works disuxss waht assumptions are missed and ignored by practitioner 
% -- consequently enumeration and recording 

Overall, prior works in HCI and responsible ML mostly make peripheral references to assumptions \rev{as objects of enumeration to avoid disrupting a desired state or action.
Several studies provide a review of common assumptions that practitioners ignore or miss during statistical modeling \cite{wang2022against,mitchell2021algorithmic,malik2020hierarchy}.
However, it remains unclear if assumptions in ML can be objectively and uniformly viewed and acted upon, especially when practitioners work with diverse stakeholders to build an ML system.
Hence, to supplement prior works, we examine this fundamental problem of confusion and uncertainties that practitioners might face when conceptualizing and working through assumptions.}
% Overall, prior works in HCI and ML mostly make peripheral references to assumptions in studying practitioners' use of ML systems and in the development of toolkits. 
To our knowledge, there is no work in an ML context that places assumptions in the center stage and investigates the surrounding confusions. In the next section, we introduce \rev{the concept of an \textit{argument}} from philosophical thinking to the HCI community to (re)think about assumptions in ML.

% Finally, we also find several prior works that make no or sparse reference to assumptions  
% With recent focus on improving the regulation on AI systems, governance and risk management framework such as NIST's RMF, 

% - overally it seems like assumptions are on the periphery and not getging enough attneiont

% - how in these tools - while some encourage reflectivess - but do not go deeper,
% - others just do not enocurage reflective practice or gives rise to further assumptions and causes confusion
    
% c. So this sidelining and unclarity in what do people do with assumptions encouraged us to ask our RQ.
% Our findings show that this practitoiners face confusion and unclarity concerning assumptions due to this current setup and infrastrucutre they are in and have

% - last para - many works do not mention assmtipns at all or use other terms but implitly they mean assumptions - so the result is confusion
%     - data workflow is a good example
    
% Prior work has discussed how ML and data practitioners draw on their beliefs and experiences to make subjective decisions at various stages of an ML lifecycle (cite).
% Our findings contribute to and reinterpret this line of work through the assumptions and the surrounding uncertainties and tensions that (a) prompts practitioners to make those subjective decisions, and (b) practitioners assess in others' interpretations and actions.


\subsection{Bringing Assumptions to the Center}
\label{rel:core}
The sub-field of \textit{Informal Logic} in Philosophy emerged as a response to the inefficiency of tools and criteria of formal \textit{Logic}\footnote{The prefix ``informal'' in Informal Logic is contentious and it is sometimes argued as a formal enterprise in a different sense \cite{woods2000philosophical,levi2000defense,johnson1999relation}.} in analyzing and evaluating natural language discursive arguments \cite{blair2012informal,walton2008informal,scriven1980philosophical,scriven1976reasoning}.
Informal Logic appreciates the structural complexity of everyday language use, the formulation of unstated assumptions, and the epistemological questions surrounding argumentation, among others, that analytic and normative tools of Logic ignores or over-simplifies, distorting the meaning of arguments \cite{anthony2015informal,johnson2014rise}. 
Arguments are also extensively studied in \textit{Critical Thinking}, often associated with Informal Logic \cite{weinstein1990towards,johnson2012informal,crews2007critical}, that studies a mode of thinking about an object involving active interpretation, clear articulation and analysis of reasons, assumptions and conclusions, logical evaluation of explanations and evidences, self-regulation, and holding a disposition to use the above-mentioned skills \rev{\cite{fisher1997critical,hitchcock2018critical,emis1962concept,facione1990critical}}.
% While critical thinking can also be about information, communication, and observations, a large overlap between Critical Thinking and Informal Logic lies in their focus on formulating, analyzing, and evaluating \textit{arguments} presented in some form (written, spoken, pictorial, etc. or multi-modal \cite{groarke2015going}.)

In this study, we refer to the conception of an \textit{argument} put forward by \citet{hitchcock2021concept} and review prior works to establish a connection between an assumptions and an argument \cite{kingsbury2002teaching,hitchcock2007informal,govier1992good,goddu2009refining}. By referring to the structuring of arguments as discussed in Informal Logic, we situate assumptions as core elements of arguments that ML practitioners make or engage in \textit{implicitly} or \textit{explicitly}.
We use this assumption-argument paradigm to offer a new perspective to understand and explicate the confusions surrounding assumptions in ML for two reasons. First, assumptions do not exist in a vacuum; they exist as part of arguments that are expressed or implied \rev{\cite{plumer2017presumptions,delin1994assumption,ennis1982identifying}}.
Therefore, critically analyzing the structure of an argument can explain \textit{how} and \textit{why} assumptions are made and what contributes to the confusion. Second, when analyzing an assumption and the surrounding confusion, we are essentially making arguments ourselves to critically think how assumptions shape a resulting argument \cite{berman2001opening,brookfield1992uncovering,delin1994assumption,ennis1982identifying}.

\rev{\citet[p.~105]{hitchcock2021concept}} formulates a simple argument as a premise-conclusion complex as follows:
\begin{quote}
    \textit{A simple argument consists of one or more of the types of expression that can function as reasons, a ``target'' (any type of expression), and an indicator of whether the reasons count for or against the target.}
\end{quote}
Reasons in the above definition refer to the \textit{premises} of an argument that perform a specific function: they commit the author of an argument to \textit{``something's being the case''} either assertively or hypothetically \rev{\cite[p.~10]{hitchcock2007informal,searle1976classification}}. 
% \cite{hitchcock2007informal,searle1976classification}.
In other words, premises constitute the propositions and the accompanying intention (or the illocutionary act \cite{searle1975taxonomy}, in philosophical terminology). 
For instance, when we use ``suppose the data is not representative'' as a premise for an argument, we express the proposition that the data is not representative \textit{hypothetically}.  The \textit{target} or conclusion is also a proposition but can be an illocutionary act type of different kinds, including a directive, a commissive, an expressive, and a declarative \cite{toulmin1958uses,ennis2006probably}. Finally, arguments can also be complex where the premise of one argument can be the target of another and so on \cite{hitchcock2021concept}.

When an ML practitioner makes or uses an assumption, it is often made or used \textit{for} a particular purpose.
For instance, when a feature is assumed to be unnecessary, it is often for realizing a particular objective such as to reduce the complexity of feature space. 
Similarly, when a performance metric is chosen, it is done so that optimized model outcomes are relevant for decision-subjects. The structure of an argument, as described above, then suggests that assumptions can be perceived as premises for attaining a target.
\citet{ennis1982identifying} discuss how these premise-type assumptions back-up or fill gaps for realizing the conclusion, and so the falsity of these premise-type assumptions weakens the support provided for the target. In other words, assumptions now become an essential component of an argument that a practitioner makes or implies. We recognize that there could be other forms of arguments, such as using vivid descriptions to display an identity or marching in protests \cite{jacobs2000rhetoric,hample2015arguing}, but we interpret the actions and expressions of ML practitioners as a premise-conclusion complex in this study and leave further exploration to future studies.
We also recognize the possibilities of interpreting different assumptions in an ML workflow as categories other than premises, such as conclusions and presuppositions, which might require a new lens to analyze \cite{walton2008argumentation,ennis1982identifying}.

Now, there can be situations where a practitioner may \textit{only} be making a premise-type assumption, but an analyst will be the one who is inferring the argument and making a distinction between premise and target. 
For instance, an ML developer can exclude certain text sources from the data and proceed with training their language model, but it is the safety expert in their organization who actually attempts to dissect the reasoning behind the data exclusion assumption\footnote{\rev{Prior works in HCI and responsible ML often do not make a distinction between first- and higher-order assumptions. In other words, assumptions made by the practitioners are not distinguished from those that are interpreted by the authors or someone else. Our point is not to doubt the inferential validity of these works but instead call attention to the complexity of assumptions, which may influence how they are examined and handled \cite{berman2001opening,atkin2017investigating,korzybski1958science}.}}. 
In other situations, a practitioner might need an assumption that they did not explicitly use, but which an analyst could infer. Or practitioners may not need an assumption but could have unintentionally used an assumption in making an argument. It is important to note that arguments are not necessarily localized to what practitioners do or write about in their documentations and reports, instead any (un)intentional action performed by a practitioner, such as choosing a specific algorithm, can be interpreted as an argument whose assumptions could be unearthed by an analyst \cite{kjeldsen2015study,groarke2015going}. Overall, this \textit{premise-target lens of an argument} deconstructs an assumption to understand why it was made (by identifying and relating it to the target) and how it was made (for instance, implicitly or explicitly), and thereby can support us in understanding how and why confusions exist around assumptions in ML practice.

% \textbf{relevance to hci}

% - what is assumtion - thinking
% - analyzing an assumption angle - critical and reflective thinking bit in papers

% - argument not only made but also had or engaged - this in itself has arguments made - implied argument - doc can be seen as that

% - now, conclusions can also be assumptions - but usually straightforward and leae to future work
% - explain why analyzing arguments help us uncover confusions about assumptions

% - taxomony of presmie tyle assumptions
% - how tools of critical thinking helps us as a method and as a theory


% In philosophical language, a premise is a illocutionary act of a proposition, either expressed assertively or hypothetically. 
% Consider two simple arguments, A1 and A2 below, inspired from prior philosophical works on this topic:
% \begin{itemize}
%     \item \textit{A1: Suppose that the data is not representative of the target population. Then it makes sense to add more data points.} 
%     \item \textit{A2: The data is not representative of the target population. So it makes sense to add more data points.}
% \end{itemize}
% Though the premises in both A1 and A2 different, they express the same proposition that the data is not representative.
% The difference lies in the illocutionary act performed by the 
