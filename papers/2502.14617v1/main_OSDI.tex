%%%%%%%%%%%%%%%%%%%%%%%%%%%%%%%%%%%%%%%%%%%%%%%%%%%%%%%%%%%%%%%%%%%%%%%%%%%%%%%%
% Template for USENIX papers.
%
% History:
%
% - TEMPLATE for Usenix papers, specifically to meet requirements of
%   USENIX '05. originally a template for producing IEEE-format
%   articles using LaTeX. written by Matthew Ward, CS Department,
%   Worcester Polytechnic Institute. adapted by David Beazley for his
%   excellent SWIG paper in Proceedings, Tcl 96. turned into a
%   smartass generic template by De Clarke, with thanks to both the
%   above pioneers. Use at your own risk. Complaints to /dev/null.
%   Make it two column with no page numbering, default is 10 point.
%
% - Munged by Fred Douglis <douglis@research.att.com> 10/97 to
%   separate the .sty file from the LaTeX source template, so that
%   people can more easily include the .sty file into an existing
%   document. Also changed to more closely follow the style guidelines
%   as represented by the Word sample file.
%
% - Note that since 2010, USENIX does not require endnotes. If you
%   want foot of page notes, don't include the endnotes package in the
%   usepackage command, below.
% - This version uses the latex2e styles, not the very ancient 2.09
%   stuff.
%
% - Updated July 2018: Text block size changed from 6.5" to 7"
%
% - Updated Dec 2018 for ATC'19:
%
%   * Revised text to pass HotCRP's auto-formatting check, with
%     hotcrp.settings.submission_form.body_font_size=10pt, and
%     hotcrp.settings.submission_form.line_height=12pt
%
%   * Switched from \endnote-s to \footnote-s to match Usenix's policy.
%
%   * \section* => \begin{abstract} ... \end{abstract}
%
%   * Make template self-contained in terms of bibtex entires, to allow
%     this file to be compiled. (And changing refs style to 'plain'.)
%
%   * Make template self-contained in terms of figures, to
%     allow this file to be compiled. 
%
%   * Added packages for hyperref, embedding fonts, and improving
%     appearance.
%   
%   * Removed outdated text.
%
%%%%%%%%%%%%%%%%%%%%%%%%%%%%%%%%%%%%%%%%%%%%%%%%%%%%%%%%%%%%%%%%%%%%%%%%%%%%%%%%

\documentclass[letterpaper,twocolumn,10pt]{article}
\usepackage{OSDI_25/usenix}

% to be able to draw some self-contained figs
\usepackage{tikz}
\usepackage{amsmath}

% inlined bib file
\usepackage{filecontents}

%-------------------------------------------------------------------------------
\begin{filecontents}{\jobname.bib}
%-------------------------------------------------------------------------------
@Book{arpachiDusseau18:osbook,
  author =       {Arpaci-Dusseau, Remzi H. and Arpaci-Dusseau Andrea C.},
  title =        {Operating Systems: Three Easy Pieces},
  publisher =    {Arpaci-Dusseau Books, LLC},
  year =         2015,
  edition =      {1.00},
  note =         {\url{http://pages.cs.wisc.edu/~remzi/OSTEP/}}
}
@InProceedings{waldspurger02,
  author =       {Waldspurger, Carl A.},
  title =        {Memory resource management in {VMware ESX} server},
  booktitle =    {USENIX Symposium on Operating System Design and
                  Implementation (OSDI)},
  year =         2002,
  pages =        {181--194},
  note =         {\url{https://www.usenix.org/legacy/event/osdi02/tech/waldspurger/waldspurger.pdf}}}
\end{filecontents}

%%%%%%%%%%%%%%%%%%%%%%%%%%%%%%%%%%%%%%%%%%%%%%%%%%%%
% *** MATH PACKAGES ***
%
\usepackage{amsmath}
\usepackage{amssymb}
\usepackage{mathtools}
\usepackage{amsthm}
\usepackage{amsfonts}
\usepackage{gensymb}
\usepackage{ulem}
%%%%%%%%%%%%%%%%%%%%%%%%%%%%%%%%%%%%%%%%%%%%%%%%%%%%%%%%%%
% Avoids contiguous empty spaces
\usepackage{xspace}


%%%%%%%%%%%%%%%%%%%%%%%%%%%%%%%%%%%%%%%%%%%%%%%%%%%%
% *** SUBFIGURE PACKAGES ***
\usepackage{subfig} %[caption=false,font=footnotesize,labelfont=sf,textfont=sf]


%%%%%%%%%%%%%%%%%%%%%%%%%%%%%%%%%%%%%%%%%%%%%%%%%%%%%%%%%%
\usepackage{enumitem}
\usepackage{verbatim}
%%%%%%%%%%%%%%%%%%%%%%%%%%%%%%%%%%%%%%%%%%%%%%%%%%%%%%%%%%%%%%%%%%%%%%%%%%%%%%%%%%%%%%%%%%%%%%%%%%%%%
% \usepackage[usenames,dvipsnames,svgnames,x11names]{xcolor}
\usepackage{multirow}

\newcommand{\nbc}[3]{
 {\colorbox{#3}{\bfseries\sffamily\tiny\textcolor{white}{#1}}}
 {\textcolor{#3}{\sf\footnotesize$\blacktriangleright$\textit{#2}~$\blacktriangleleft$}}
}
\newcommand{\sys}{{\scshape SageServe}\xspace} % RaquetServe

\usepackage[most]{tcolorbox}

\newtcolorbox{mytextbox}[1][]{%
  sharp corners,
  enhanced,
  colback=white,
  % height=10cm,
  attach title to upper,
  #1
}


\definecolor{yscolor}{rgb}{0.7,0.3,0.7}
\newcommand{\ysnote}[1]{ {\nbc{YS}{#1}{yscolor}}} % needs a response
\newcommand{\pglen}[1]{\textcolor{red}{\small\textit{#1pgs}}} % needs a response

% Pick your fave color...
\newcommand{\apnote}[1]{ {\nbc{AP}{#1}{orange}}}
\newcommand{\amnote}[1]{ {\nbc{AM}{#1}{orange}}}
\newcommand{\sjnote}[1]{ {\nbc{SJ}{#1}{blue}}}
\newcommand{\kjnote}[1]{ {\nbc{KJ}{#1}{teal}}}

\newcommand{\Note}[1]{\textcolor{red}{#1}} % verify if this is correct

\newcommand{\ysnoted}[1]{ {\textcolor{green} { ***TODO Later: #1 }}} % postpone addressing of comment
% Change tracking for article revisions. Added, Deleted, Replaced, or Modified content.
%
\newcommand{\modc}[1]{{\textcolor{blue}{#1}}}
\newcommand{\addc}[1]{{\textcolor{teal}{#1}}}
%% NOTE: \sout will break if the text has citations with an error that there is an extra "}". Place the citation within mbox: \mbox{foo~\cite{foo}}
\newcommand{\delc}[1]{ {\textcolor{gray} {\sout{#1}} }}
%\newcommand{\delc}[1]{} % uncomment this (and comment above line) to ignore showing deletion
\newcommand{\repc}[2]{ {\textcolor{gray} {\sout{#1}} }{\textcolor{teal} {#2}}}
%\newcommand{\repc}[2]{{\textcolor{teal}{#2}}} % uncomment this (and comment above line) to ignore showing deletion
\newcommand{\para}[1]{{\noindent \bf #1.~}} % RaquetServe



% \renewcommand{\ysnote}[1]{}
\renewcommand{\ysnoted}[1]{}
% \renewcommand{\apnote}[1]{}
\renewcommand{\amnote}[1]{}
\renewcommand{\sjnote}[1]{}
% \renewcommand{\kjnote}[1]{}
\renewcommand{\Note}[1]{#1}
\renewcommand{\delc}[1]{}
\renewcommand{\pglen}[1]{}

\newcommand{\footremember}[2]{%
   \footnote{#2}
    \newcounter{#1}
    \setcounter{#1}{\value{footnote}}%
}
\newcommand{\footrecall}[1]{%
    \footnotemark[\value{#1}]%
} 

%-------------------------------------------------------------------------------
\begin{document}
%-------------------------------------------------------------------------------

%don't want date printed
\date{}

% make title bold and 14 pt font (Latex default is non-bold, 16 pt)
\title{Serving Models, Fast and Slow:\\
Optimizing Heterogeneous LLM Inferencing Workloads at Scale}

%for single author (just remove % characters)
% \author{
% Shashwat Jaiswal\\
% UIUC
% }
% \author{
% Kunal Jain\\
% Microsoft
% } % end author
\author{
{\rm Shashwat Jaiswal$^{1}$\thanks{Equal contribution. Shashwat Jaiswal was an intern at Microsoft}, Kunal Jain$^{2*}$, Yogesh Simmhan$^{3}$, Anjaly Parayil$^{2}$, Ankur Mallick$^{2}$,}\\
{\rm Rujia Wang$^{2}$, Renee St. Amant$^{2}$, Chetan Bansal$^{2}$, Victor Rühle$^{2}$,}\\
{\rm Anoop Kulkarni$^{2}$, Steve Kofsky$^{2}$ and Saravan Rajmohan$^{2}$}\\
$^{1}$University of Illinois Urbana-Champaign\quad$^{2}$Microsoft\quad$^{3}$Indian Institute of Science
} % end author


\maketitle
% \footnotetext[*]{text}
%-------------------------------------------------------------------------------
\begin{abstract}
%-------------------------------------------------------------------------------
Large Language Model (LLM) inference workloads handled by global cloud providers can include both latency-sensitive and insensitive tasks, creating a diverse range of Service Level Agreement (SLA) requirements. Managing these mixed workloads is challenging due to the complexity of the inference stack, which includes multiple LLMs, hardware configurations, and geographic distributions. Current optimization strategies often silo these tasks to ensure that SLAs are met for latency-sensitive tasks, but this leads to significant under-utilization of expensive GPU resources despite the availability of spot and on-demand Virtual Machine (VM) provisioning.
We propose \sys, a comprehensive LLM serving framework that employs adaptive control knobs at varying time scales, ensuring SLA compliance while maximizing the utilization of valuable GPU resources. Short-term optimizations include efficient request routing to data center regions, while long-term strategies involve scaling GPU VMs out/in and redeploying models to existing VMs to align with traffic patterns. These strategies are formulated as an optimization problem for resource allocation and solved using Integer Linear Programming (ILP).
We perform empirical and simulation studies based on production workload traces with over 8M requests using four open-source models deployed across three regions. \sys achieves up to 25\% savings in GPU-hours while maintaining tail latency and satisfying all SLOs, and it reduces the scaling overhead compared to baselines by up to 80\%, confirming the effectiveness of our proposal. In terms of dollar cost, this can save cloud providers up to \$2M over the course of a month.
%{\color{red}\sys achieves significant savings in GPU-hours with a $7\times$ reduction in tail latency and a $3.5\times$ reduction in the scaling overhead with respect to the baseline, confirming the effectiveness of our proposal.} 
\end{abstract}
% \footnote{Equal contribution. Shashwat Jaiswal was an intern at Microsoft}


%-------------------------------------------------------------------------------
\documentclass[../main.tex]{subfiles}
\graphicspath{{../images/}}
\makeatletter
\def\input@path{{../images/}}
\makeatother
\begin{document}
\section{Introduction}
\begin{figure}
\centering
\begin{tikzpicture}
\node[inner sep=0pt] (ws) at (0, 0) {
\includegraphics[height=.4\textwidth, trim={10cm 0 10cm 0},clip]{world_space.png}};
\node[inner sep=0pt] (cs) at (6,0) {\includegraphics[height=.4\textwidth, trim={10cm 1cm 10cm 4cm},clip]{conf_space.png}};
\end{tikzpicture}
\vspace{-5pt}
\label{fig:pbrm_intro}
\caption{\textbf{Left}: Shows world space obstacles as grey spheres. Robots start and goal configuration is colored red and green, respectively. Configurations along the computed path are colored transparent blue. \textbf{Right:} Mapped world space scenario to configuration space. Obstacle region is the grey mesh. Red spheres are collision-free regions computed by the neural SCDF. The optimized shortest path in the convex corridor is the blue curve.}
\vspace{-25pt}
\end{figure}
Motion planning is the problem of finding a collision-free trajectory that connects a given start and goal configuration. The planning takes place in the configuration space of the robot. For single body robots, like mobile robots or drones, the configuration space and the world space are usually the same. This simplifies the planning, since explicit obstacle representations are available which enables geometrical tools like separating hyperplanes, smallest distance to obstacles etc., to be used when designing motion planning algorithms. For multi-body robots like manipulators, the situation is completely different. The world space obstacles are usually mapped to non-convex regions, and to make the problem even harder, the mapping is usually not known. Forming explicit representations of the obstacle region in the configuration space is usually too expensive or intractable. Despite all of this, sampling based planners are used with great success, which mainly is due to their use of implicit representations of the obstacle region. The basic idea is to construct a graph in the configuration space that covers and connects the collision-free region. From this graph, a path can be extracted that connects a given start and goal configuration. The approach is computationally expensive, since the graph is constructed with the smallest geometrical building block available, points, which represents a collision-check. Furthermore, the extracted paths from the graph are non-smooth and jagged due to the stochastic nature of the approach. This adds an additional post-processing step to the process, where the paths are shortcutted and smoothened, before the path can be used for tracking. Clearly a lot of time is invested to form this graph and produce smooth paths. Thus, if the obstacles start to move, then all of this work is done in no use, since all points that make up this graph need to be re-verified, which is simply too time consuming to be done in real time.
\\\\
In this work, we want to address the existing drawbacks of the sampling based planners. Our main contribution is an improved motion planner where each vertex in the graph covers a collision-free region in the form of a sphere instead of a point and where the edges are formed with neighboring intersecting spheres. This representation has the advantage of instead of returning piecewise linear paths, returning a sequence of overlapping spheres, i.e. a convex corridor, that connects a given start and goal configuration, illustrated in Figure \ref{fig:pbrm_intro}. This convex corridor allows us to use convex optimization to produce smooth trajectories, instead of computationally expensive post-processing methods. The representation further allows us to estimate the coverage of the collision-free space, which gives us awareness and feedback in the offline roadmap construction phase. Finally, our representation is simple to adapt to moving obstacles, simply requery for the new radii and recheck for intersections. 
\\\\
The spherical collision-free regions are formed using a signed distance function (SDF), which is a function that returns the smallest distance from an arbitrary point to the boundary of an obstacle. As the name implies, the distance is signed, thus if the point is inside the obstacle it is negative otherwise positive. If the distance is positive, a sphere with radius equal to the distance is guaranteed to cover a collision-free region. Using an SDF in motion planning is not new, but what is novel about our approach is that we express the distance in the configuration space instead of the world space and by doing so allows us to form these convex collision-free regions. We refer to the resulting SDF as a signed configuration distance function (SCDF). Computing an SCDF analytically is non-trivial, our approach is therefore to parameterize the SCDF with a deep neural network and learn the mapping by supervised learning. Our resulting neural SCDF can compute distances for different parameter values of obstacle shapes and we also show how multiple distances can be combined, thus making our approach flexible.
\section{Related work}
Motion planning algorithms can roughly be divided into three families, grid-based, sampling based and optimization based methods. Grid-based methods (GBM) discretize the planning space from which a graph is then compiled. A standard search method is A$^\star$ \citep{a_star}, which is classified as an \textit{informed} search method, since it employs a heuristic function to speed up the search. A$^\star$ guarantees to return an optimal path at the level of discretization used. GBMs usually discretize the planning space by a regular lattice and this limits the GBMs to problems with low dimensionality due to the curse of dimensionality. Thus, GBMs are usually limited to single-body robots where the degrees of freedom (DOF) are low. To overcome the inherent scaling problem with the GBMs, stochastic methods are usually used for multi-body robots. These methods are termed as sampling-based methods (SBM) and core members within this family are the rapidly-exploring random trees (RRT) \citep{rrt} and the probabilistic roadmap (PRM) \citep{prm}. RRT grows a tree from the start configuration and explores the collision-free region in a rapid way until it is able to connect to the goal region. RRT is usually improved by bi-directional planning \citep{rrt_connect}, i.e. an additional tree is grown from the goal configuration and the trees are tested for connection after any tree has been expanded. RRT is a single-query method, thus it searches for a path from scratch each time it is queried. Contrary to this, PRM is a multi-query method, which solves for multiple queries without starting from scratch. PRM does this by creating a roadmap (graph) that covers the collision-free space as an offline step. The graph is then used to solve for multiple queries. PRMs are used in cases where the environment does not change since the extra offline step is too computationally costly and needs to be re-done if the environment is changed. In our work, we address this inherent issue by using a different roadmap representation. Our vertices in the graph cover a collision-free region in the form of spheres and we form the edges by checking for intersecting spheres. If something in the environment changes, we recompute the spheres radii and recheck the intersections, without relying on collision detection. We use a trained neural network to compute the sphere radius, therefore querying for the radius can be done fast, hence our representation enables the PRM for dynamic environments.
\\\\
In the recent decades, optimization based methods (OBM) \citep{chomp, schulman, itomp, stomp} have been introduced as an alternative to SBM for multi-body robots. Like the SBM, the OBMs scale well to higher dimensional problems and produce smoother motion. It is common to use a SDF in the optimization since it is a smooth function, thus enabling gradient-based methods. However, the standard way of expressing the SDF is in world space. The distance therefore needs to be mapped to the configuration space by the forward kinematics. This mapping makes the optimization problem a non-linear program (NLP), which is computationally expensive to solve. Recently, a different approach has been proposed. In \cite{mp_gcs} motion planning is formulated as a convex optimization problem by using the graph of convex sets framework \citep{gcs}. The underlying idea is to decompose the collision-free space into intersecting convex sets from which a convex optimization problem is formulated. In cases where an explicit representation of the obstacles in the configuration space exists, like for single-body robots, creating collision-free convex regions can be done fast \citep{iris}. For multi-body robots, this is non-trivial. Existing work does this successfully \citep{iris_nlp, iris_c} by an optimization based approach, but the methods are still too time consuming to be used in the presence of moving obstacles. Our approach is instead to use deep learning to learn an SDF expressed in the configuration space. With this, we can query for shortest distances to the collision boundary, which allows us to expand spherical regions which are collision-free. Our approach is fast and therefore enables our suggested roadmap planner to be used in dynamic environments.
\\\\
Recent research has focused on learning collision detection \citep{fk_kernel_distance, diffco, graphdistnet} by predicting the signed distance between the robot links and the surrounding obstacles in the world space. The learned SDF is used in trajectory optimization but since the distance is expressed in the world space, the problem becomes an NLP and therefore takes a long time to solve. We take a novel approach and suggest to instead express the signed distance in the configuration space. This allows us to improve the PRM at the same time as it enables convex optimization for trajectory optimization, which runs faster and is more reliable than NLP solvers. In \cite{cspf} a learned signed distance function in the configuration space is proposed similar to our approach. However, their approach is restricted to point cloud representations, while we propose to represent the obstacles as parameterized geometric shapes, e.g. spheres. Furthermore, we also show how to use our learned SCDF to improve an existing roadmap planner.
\section{Problem formulation}
A robot is located in the world space, $\W \subset \R^3 $. The unique location of the robot is given by its configuration $\q \in \C$, where $\C$ is the configuration space. The set of points covered by the robots bodies at a certain configuration is expressed as $\B(\q) \subset \W$. The robot is surrounded by $\NrObst$ obstacles $\O = \bigcup_{i=1}^{\NrObst} \O_i$, where  $\O_i \subset \W$. The representation of the obstacle in the configuration space is the set $\C\O_i = \{\q \in \C \: |\: \B(\q) \cap \O_i \neq \emptyset \}$. The obstacle space is formed as $\Co = \bigcup_{i=1}^{\NrObst} \C \O_i$. The complement is referred to as the free space, $\Cf = \C \setminus \Co$. The path planning problem is a tuple, ($\Cf$, $\qStart$, $\qGoal$), where we want to connect a query pair, consisting of a start, $\qStart$, and goal configuration, $\qGoal$, with a geometric path, $\q(s): [0, 1] \mapsto \Cf$, such that $\q(0)=\qStart$ and $\q(1)=\qGoal$, or report correctly when such a path does not exist.
\end{document}

%-------------------------------------------------------------------------------
\section{Related Work}

\subsection{View-Dependent Control}
View-dependent representations have a long history in computer graphics.
In his pioneering work, Rademacher proposed interpolating between \textit{key viewpoints} and associated \textit{key deformations} to manipulate 3D models~\cite{rademacher1999view}.
Other researchers have extended the idea to create 3D animation systems~\cite{10.1111:j.1467-8659.2004.00772.x}, streamline the modeling process~\cite{DBLP:journals/corr/abs-2103-15472}, and integrate physical simulation\cite{koyama2013view}.
Of particular note, Rivers et al.~\cite{rivers25Dcartoonmodels} introduced \textit{2.5D Cartoon Models}, a combination of planar meshes transformed, based upon view angle, so as to appears three dimensional.
Our work draws upon these works but is, to our knowledge, the first work to attempt to use view-dependent techniques to retarget 3D motion onto 2D characters.   

\subsection{Animation from 2D Images}

% output is still 2D
Many researchers have proposed different methods for creating animations from 2D images. Hornung et al.~\cite{Hornung2007anim2Dpicmotion} presented a method to deform a character from a photograph given user-provided joint annotations.
\textit{Toonsynth}~\cite{Dvoroznak18-SIG} and \textit{Neural Puppet}~\cite{poursaeed2020neural} both present methods to create new images of hand-drawn characters from examples.
% output is 3D model
Other researchers have proposed methods of obtaining 3D geometry from 2D sketches~\cite{igarashi2006teddy, Dvoroznak20-SA} and images~\cite{ArtiSketch,weng2019photo}.
% focus on sketches specifically
A number of works have specifically focused on childlike drawings.
Lingens et al.~\cite{lingens2020towards} proposed an evolutionary algorithm for animating children's drawings. 
\textit{MagicToon}~\cite{feng2017magictoon} creates a 3D model from childlike drawings for AR applications.
Similar to our work, Smith et al.~\cite{SmithHodgins} proposed a method for animating childlike drawings using 3D skeletal motion. 
However, the resulting animations are only suitable for use in 2D applications and the type of motions it supports are limited.

Unlike these previous works, our solution can be used in 3D contexts but does not create a 3D model. We instead relying upon a view-dependent formulation of the animated character.
%%%%%%%%%%%%%%
\section{System and Application Model \pglen{1.75}}
\label{sec:sys-app-model}
%\kjnote{Review bolded paragraph titles.}

Our system and application models described here are motivated by real cloud systems, LLM deployments and workloads in Microsoft, %M365, 
a global public Cloud Service Provider (CSP). {We will open source our trace data upon acceptance.}

%%==================================================
\subsection{Cloud Regions and GPU VMs}
\textbf{Setup. }Our system model for the CSP comprises of multiple data centers (\textit{regions}) with captive GPU server capacity in each. To avoid issues of data sovereignty, we assume that these regions are in the United States (US), e.g., US-West, US-Central, etc. The regions are connected with high bandwidth network, with a typical network latency of $\approx 50ms$. Each region has servers with a mix of GPU generations that can be provisioned as GPU VMs with exclusive access to all underlying server resources, e.g., Azure's ND VM series that have 8 Nvidia V100, A100 or H100 cards, %with infiniband interconnect
or AWS's comparable P5/P4/P3 EC2 instances.
Each region may have 1000s of such GPU VMs (hereafter referred to just as ``VMs'') that can be provisioned within the available capacity to host LLMs.



%%==================================================
\subsection{LLM Instances and Endpoints}
\textbf{LLM Architecture and Types. }There are several standard pre-defined \textit{LLM architectures} that are available, e.g., Llama 2, Bloom, GPT 3.5 turbo, etc.
Further, each model architecture can either use default weights to give a standard behavior, or be fine-tuned for a specific application and have custom weights. The combination of model architecture and weights forms an \textit{LLM type}.

\textbf{Model Instance. }A \textit{model instance} is one instantiation of an \textit{LLM model type} that can serve requests.
Each instance may require one or more VMs,
depending on the size of the LLM and capacity of the VM, e.g., a GPT3 model may require 9 H100 GPUs while a Llama-3 needs 4 H100s~\cite{mei2024helix}. 
Each VM is exclusively used by one LLM instance.
The VM type 
will determine the LLM instance's performance, defined in terms of TPS of throughput achieved at a certain target latency~\cite{mei2024helix}. 
\autoref{fig:tps-real-deployment} shows a boxplot of the TPS achieved on real deployments of the Llama2-70B (\textit{Llama2})~\cite{Llama} and Bloom-176B (\textit{Bloom})~\cite{bloom} models on VMs with $8\times$ Nvidia A100 and H100 GPUs. Whiskers show the $5$--$95\%ile$ range. The TPS drops with model size and improves on newer GPUs. 

\begin{figure}[t!]
    \centering
\includegraphics[width=0.9\columnwidth]{figures/fig3.png}
    \caption{TPS seen for LLMs on VM with $8\times$ GPUs each.
    }
    % \ysnote{add minor grid lines on Y axis; use PDF vector figures. Use 0.5 cols and merge with \ref{fig:tps-simulation} as another 0.5col subfig. X axis should have model on outer, HW on inner.}
    \label{fig:tps-real-deployment}
\includegraphics[width=0.9\columnwidth]{figures/motivation/splitwise_vs_actual.pdf}
    \caption{Comparison of batch execution time predicted by Splitwise vs real model instance.
    % \apnote{can we add dashed line for one of the curves?. }
    }
    \label{fig:splitwise_cs_actual}
\end{figure}

\textbf{Instance Set. }Each LLM type also has an \textit{instance set} with a minimum and maximum range of model instances;
having a minimum instance count provides redundancy in case of a VM failure. The VM type for all instances in the set are identical to ensure similar performance.
Each instance set is exposed through an \textit{endpoint} to receive incoming requests. There can be multiple LLM endpoints for the same LLM type, 
and they all have the identical semantic behavior. 


%%==================================================
\subsection{Heterogeneous LLM Workloads}
\textbf{Workloads. }The CSP needs to serve LLM inferencing requests to support several types of \textit{production workloads}.  One, is to support their own client-facing enterprise products that generate \textit{Interactive Workloads \textbf{(IW)}} for specific LLM model types, with \textit{low latency constraints (O(seconds))}, e.g., to produce and debug code snippets, generate email responses, chatbots, etc. These models require ``fast'' serving. Another is batch or \textit{Non-Interactive Workloads \textbf{(NIW)}} with more \textit{relaxed deadlines (O(hours))}, e.g., 
nightly summaries on documents in an enterprise repository, detailed content generation, etc.
These batch requests expire if not completed within a relaxed deadline.
\ysnoted{Lastly, we have a set of \textit{Opportunistic Workloads \textbf{(OW)}} that typically run on \textit{spot instances}, i.e., LLM instances that are not reserved for production workloads. These have lower priority for completion and are often evicted when capacity is needed for latency-sensitive requests.}
So ``slow'' (or ``no'') serving is acceptable for NIW, in preference to IW.

\begin{figure}[t]
    \centering
\includegraphics[width=\columnwidth]{figures/fig1.pdf}
    \caption{RPS \textit{(solid line)} and TPS \textit{(dashed line)} of IW and NIW requests summed across 4 LLM models and 3 cloud regions for 1 week in Nov, 2024. 
    }
    \label{fig:rps}
    \label{fig:tps}
\end{figure}

\textbf{Interactive Workloads. }For IW, clients for each product/service 
may use one or more pre-defined LLM type, and this inferencing is initiated by the client while using the product. There can be $10,000$s of clients during a day and we observe a diurnal pattern in the requests that are received. This is seen in \autoref{fig:rps}, which shows requests going to all models deployed in a US region, during one week in Nov, 2025. For simplicity, we assume all clients are US-based since the regions are in the US.
These workloads have a \textit{SLA latency limit} that ranges from a second to a minute for the \textit{Time to First Token (TTFT)}, i.e., the time after the prompt request being received to the first response token being generated, for a certain percentile of request, e.g., 95\%ile. Other quality of service factors include the end-to-end (E2E) time for the request to complete generating all output tokens, the number of input and output tokens that can range in the 1000s per request (\autoref{fig:rps}), etc. 

\textbf{Non-Interactive Workloads. }NIW also uses a set of pre-defined LLMs whose architectures often overlap with IW.
The request rate for NIW, however, is lower and not periodic, staying stable through the week (\autoref{fig:rps}). 
Further, given their batch nature, their SLA is a \textit{deadline} for completion before they expire, e.g., $24h$ to complete summarizing a document repository.
\ysnoted{So, the key differences between IW and NIW are primarily in terms of the deadline and on the request rate, potentially forming a continuum as the use of LLMs evolve. \Note{In this work, we assume two discrete SLA deadlines of $10s$ for TTFT for IW and $24h$ for NIW.}\ysnote{if this changes per experiment, we can move this to relevant sections}}

We assume that servicing each IW request within the latency SLA accrues a utility for the CSP, and servicing an NIW request before its deadline expires accrues a (lower) utility.

%%==================================================
\subsection{LLM Endpoint Provisioning and Routing}
\textbf{Scaling Delays. }Creating a new endpoint for an LLM model by deploying an instance set to VMs has several \textit{provisioning costs}
that can vary based on the conditions. 
Allocating VMs to the instance set is an initial cost. Then, if these VMs do not have the LLM already deployed on them, the model architecture and weights need to be copied and instantiated on them. The time taken depends on the model size, and on whether they are available in a local repository in that region, e.g., taking $\approx10mins$, or need to be moved from a remote region, e.g., taking $\approx 2h$.
If the VMs already have the LLM architecture deployed from a prior provisioning but with different weights, only the weights need to be updated and the cost reduces.
When an instance set is being provisioned 
its instances are not available for use. So the model provisioning time constitutes \textit{wasted GPU cycles}. Given that this time can run from mins--hours, frequent re-provisioning is inefficient. {It should be noted that  that in real world datacenter, acquiring a GPU, updating all upstream services such as load balancer, etc. would take time.} 

\textbf{Spot Instances. }The workloads are executed on LLM instances that are provisioned in a private network space.
However, if the endpoints of common LLM models are idle, they can be leased 
to external users as (preemptible) \textit{spot LLM instances} for inferencing at a lower cost, and reclaimed when the internal demand increases. Switching an instance from private to spot, and the reverse, is relatively fast, $\approx 1$ mins.
Typically, the utility benefits of leasing out spot instances is lower than that gained from executing the internal IW and NIW workloads. But this is still better than keeping the VMs idle. During some periods, 25\% of instances in a region may be donated to spot; \textit{this is a lost opportunity cost we aim to fix}.


%%==================================================
\textbf{Routing Mechanisms. }All internal workload requests are directed to a common LLM API service~\cite{BatchAPI}
that redirects a request to one of several LLM endpoints that can service it (\autoref{fig:arch}). 
This \textit{global routing} to one of a configured set of regions 
can be based on network latency, geographical proximity or 
the current load on the region's endpoints. Then, a \textit{region router} sends requests to deployment endpoints for that model in that region, and further to instances within the selected deployment in a round robin manner to balance the load and address token skews.

We assume a managed network and trusted security environment. There are no other security constraints that limit the mapping of instances to VMs, or to internal or spot endpoints.

\subsection{Simulating Datacenters}
%\kjnote{Review}
Experimenting with different scaling mechanisms and verifying their effectiveness with multiple model types and corresponding instances can prove quite costly. Therefore, to alleviate this pressure and enable easier research in this direction, we extend existing SOTA LLM simulator, Splitwise\cite{splitwise}, to simulate a datacenter running multiple models on a variety of hardware.

We begin by independently verifying the accuracy of the Splitwise simulator (\autoref{fig:splitwise_cs_actual}), which reports  $<$3\% MAE on latest models and hardware. We further build upon this to launch multiple instances of these instance simulators and create a unified event queue for them, taking into account routing and iterations through the model. Different components of our simulator are presented in \autoref{fig:simulator}.

%\kjnote{Review}
Our simulator mimics multiple datacenters across the globe and is implemented at a granular level in order to make it flexible for different settings. Thus, it can be used for experimenting and evaluating new routing and batching algorithms as well, in order to study the impact of these changes across the entire inference serving stack. We will open source our work to give service providers holistic simulations and enable work on full stack optimizations.

% \kjnote{Since we can mimic multiple datacenters across the globe with our set up, we can experiment with routing and batching logic as well. Test holistic effect of these changes on our entire stack.}

% We enable easier cost effective research by scaling the Splitwise simulator. \kjnote{refer figure 3}



%%%%%%%%%%%%%%%%%%%%%%%%%%%%%%%%%%%%%%%%%%%%%%%%%%%%%%%%%%%%%%%%%%%%%%%%%%%%%%%%%
%%%%%%%%%%%%%%%%%%%%%%%%%%%%%%%%%%%%%%%%%%%%%%%%%%%%%%%%%%%%%%%%%%%%%%%%%%%%%%%%%
\section{Motivation: Effective Resource Scaling \pglen{1}} \label{sec:empstudy} 

%\apnote{We may introduce the building block of simulator either here or in the previous section and quote Figure 3.} 
%\apnote{We may start with the empirical question we are tackling in this section and quote the insight towards the end of the section.}
We study the effect of scaling the resources allocated to an LLM type, and its effects on the capital efficiency and the SLAs of both IW and NIW workloads. As these VMs are captive and costly resources for the CSP, the goal is not to reduce resource usage but to put them to efficient use to increase the utility -- preferably by processing IW and NIW workloads, or otherwise leasing them as spot instances.

The baseline approach uses a \textit{siloed deployment}. Here, separate pools of LLM instances are maintained for IW and NIW workloads, for each LLM type in a region. Within a pool, instances may be allocated to internal workloads if there is sufficient demand, or released to spot instances otherwise.
This uses a greedy approach for scaling, where we reclaim a spot instance and increase the instance set count for a model when the effective memory utilization (effective memory utilization excludes the memory used for model weights) for the instance set (which is a reliable proxy for the request load) increases above $70\%$, and returns an instance to spot if the utilization drops below $30\%$. These decisions happen on each request arrival. We always remain within the min/max instance counts.
The downside of this is the fragmenting of VMs to captive pools, which can limit their effective utilization.

% \kjnote{Is it right to say we are \textbf{proposing} the reactive heuristic? I believe this is a very common thing?}
As an alternate, we propose a \textit{reactive scaling heuristic} using a single \textit{unified pool} of instances to serve both IW and NIW requests across all models in a region. 
Here, the NIW requests are queued and processed only when the endpoint's utilization falls below a certain threshold utilization by the IW requests (60\%). One or two NIW requests are processed based on the effective utilization. If the value falls below 50\%, two requests are added to the queue.
This has the benefit of sharing the model instances between IW and NIW, and also allowing the pool of VMs to be used to deploy any of the LLMs, thereby improving their use for IW and/or NIW workloads rather than donate to spot.

While switching an LLM instance between spot and internal endpoint take $1min$, switching a VM from hosting one instance to another takes $10mins$. Note that all scaling events are triggered based on effective utilization, which is measured when a request reaches a regional endpoint. Additionally, we allow a cool down period of 15 seconds between any two scaling events.


\ysnoted{intra model is 1mins, itner model is 10mins. so intra is faster.\\
Internval puts a cooldown period before we claim an intra model spot instance for scaling up so that any transient bursts are allowed to smoothen out rather than quickly acquire inter model;}

We evaluate these siloed baseline and unified reacting scaling  heuristic strategies for four LLM models: Bloom, Llama 2, Llama 3.1 and Llama 3.2, deployed in all the three US regions. There are 20 instances per model per region -- all are part of a single pool when managed by the heuristic, and for the siloed approach, we assign 16 for IW and 4 for NIW. The minimum instance set count per endpoint is 2 and maximum is 3. 
We use a realistic simulation harness for \sys which is built on top of Splitwise~\cite{splitwise}, whose results closely match real-world behavior (see Section~\ref{sec:results:simulator}). \autoref{fig:tps-real-deployment} shows the TPS distribution observed when simulating different models using \sys when using 8xA100s per instance. 
% \apnote{missing reference}
We replay one day of workload data from Tuesday of the week in \autoref{fig:rps}. 
%, and set an SLA for IW of $10s$ for TTFT and $24h$ deadline for NIW. 
% We assume a utility gain of 3 for each IW request within SLA, 1 for an NIW request and a utility 5/hr when used as spot instances. We measure the instance counts over time, 95\%ile TTFT and E2E latencies, and utility. 
% \apnote{@kunal, remove this utility numbers after we report the IW and NIW requests served and the instance hours donated to Spot}

%{\autoref{fig:splitwise_cs_actual} verifies the accuracy of splitwise simulator. \apnote{instead of the next line, just add remarks on mean and deviation in the actual and predicted execution time. } We argue that the basic building block of our simulator is replicating a model instance, and since we are able to do that correctly, everything on top of it follows.} \apnote{this paragraph is a little disconnected from other paragraphs.}

\ysnoted{There is no cooldown interval between inter-model scaling.}

\autoref{fig:motivation-instance-hours} shows the instance count at US-Central every 15 mins by the siloed and unified approaches for each model, and the total model instance hours (area under the curve) for the day. 
It can seen that the unified heuristic consumes few instance hours for Bloom and Llama 2, while it is the same for Llama 3.1 and 3.2 since they received few requests and maintained the minimum instance count -- siloed allocates 2 instances each for IW and NIW while unified shares the 2 instances for the IW and NIW requests. This consolidation is reflected in the higher memory utilization for the unified heuristic in \autoref{fig:motivation-utilization}, while not sacrificing the SLA latency for IW.
In both approaches, change in the 95$\%$ile TTFT ranges from 0 to 12\% (c.f., \autoref{table:silo_vs_reactive}). 
% \apnote{pls check if values in this table is accurate.} 
Both approaches process all the queries in the trace (1.4M IW and 0.2M NIW),
% Both approaches  achieve similar utilities for IW and NIW (4.1M and 0.2 M) (\autoref{fig:motivation-utilization}) \apnote{Remove the term, 'utility'. We can just say the number of requests served.}. 
but unified is able to use fewer resources, and donate more to spot, donating 52 instance hours more than the siloed approach.
% \apnote{Change this value to hours donated and report that.}
\begin{table}[t]
    \centering
    \caption{95\%ile of TTFT and end-to-end latencies for serving different models using siloed and unified approach. 
 %   \apnote{pls check this values.}
    %\kjnote{Updated value}
    }
    \label{table:silo_vs_reactive}
    \begin{tabular}{c|c|r|r}
    \hline
        \textbf{Strategy} & \textbf{Model} & \textbf{TTFT (s)} & \textbf{E2E (s)} \\ \hline\hline
        Siloed & Bloom-176B & 14.5 & 55.3 \\ \hline
        Siloed & Llama2-70B & 34.9 & 98.3 \\ \hline
        Siloed & Llama3.1-8B & 1.0 & 10.6 \\ \hline
        Siloed & Llama3.2-3B & 1.0 & 19.2 \\ \hline
        Unified & Bloom-176B & 12.9 & 53.3 \\ \hline
        Unified & Llama2-70B & 34.5 & 99.1 \\ \hline
        Unified & Llama3.1-8B & 1.0 & 10.5 \\ \hline
        Unified & Llama3.2-3B & 1.0 & 18.9 \\ \hline
    \end{tabular}
\end{table}
Also, the memory utilization of Llama 2 is generally less than Bloom, indicating that using the unified pool to allocate across model instances (inter-model scaling) can further help adapt better to complementary demand compared to siloed that does not allow allocation of VMs across models. 

\begin{mytextbox}
%\kjnote{Review}\autoref{table:silo_vs_reactive} and \autoref{fig:motivation-instance-hours} show that  mixing IW and NIW workloads leads to better resource utilization and cost efficiency. This also opens up new optimization avenues for us as NIW workloads can be processed in a flexible manner. 

%As shown in \autoref{fig:ideal_scaling}, reactive scaling mechanisms can easily either harm the SLO requirements of IW requests due to under allocation of resources or increase operating costs due lead to over allocation. This motivates us to look for a proactive scaling approach, which can take advantage of predictable incoming TPS (as shown in \autoref{fig:rps}) and workload mixing.
 Mixing IW and NIW workloads improves resource utilization and cost efficiency, opening new optimization avenues through flexible NIW processing (c.f. \autoref{table:silo_vs_reactive} and \autoref{fig:motivation-instance-hours}). Moreover, reactive scaling can either harm IW SLOs due to insufficient resources or raise costs from overallocation (\autoref{fig:ideal_scaling}). This motivates a proactive scaling approach that leverages predictable TPS (as shown in Fig. \autoref{fig:rps}) and workload mixing.
\end{mytextbox}

%\kjnote{Review}
In the following sections, we first formulate this as an optimization problem and propose an ILP to find the optimal allocation of GPUs (\autoref{sec:optimization}), describe the overall architecture of our resource management systems (\autoref{sec:arch}) and finally, we show the effectiveness of our system(\autoref{sec:results}). 

% \apnote{We may add the insights into a box?}
% \kjnote{Can we add a transition paragraph here saying we've now shown mixing workloads is good but reactive scaling is hurting us a lot (See Figure~\ref{fig:cost-of-scaling}, we paid for X GPU hours without being able to do any inference on them). So, this motivates us to do proactive scaling, and we do this by modelling it as an ILP optimization using ARIMA forecast on the number of tokens we will recieve, which will aim to reduce the GPU cycles wasted in resource allocation}


\begin{figure}[t!]
    \centering
\includegraphics[width=\columnwidth]{figures/fig4-1-new.pdf}  \caption{Model instance count across time and total instance-hours for unified vs. siloed strategies for 1-day of workload in US-Central, with peak 20 instances per model. 
}
    % \ysnote{increase font sizes}
    \begin{comment}
    \ysnote{Rename model names as Llama 2, Bloom, Llama 3.1, Llama 32; }
      \ysnote{
        rename "heuristic" as "unified".Show the AUC value (instance hours) as a text value within the plot for Unified and siloed. Minor grid lines on X and Y axis.}
        \end{comment}
    \label{fig:motivation-instance-hours}
\end{figure}

\begin{figure}[t]
    \centering
\includegraphics[width=0.5\columnwidth]{figures/fig4-2-new1.pdf}
    \caption{
    % \textit{(left)} Utility accrued for IW, NIW, Spot for siloed and unified strategies, across workloads at 3 regions and for 4 models. 
    Memory usage by strategies, for Monday workload at US-West. Both the methods answered 1.4M IW and 0.4M NIW requests during the day. The unified approach used 52 instance hour less than the siloed approach.}
 
  % \ysnote{improvement for total is negligible. Drop total bar as it does not make the case.}
  % \ysnote{move grid lines \textit{behind} the bar. Reduce height by 25\%.}
  % \ysnote{Show markers on right Y axis indicating the 95\%ile TTFT achieved. 
  % Add labels for median util\% and for TTFT markers. Order should be B,A,C,D with model names.}
  % \kjnote{report number of NIW and IW solved}
   \label{fig:motivation-utilization}
\end{figure}


\section{RST in an optimization framework}
\label{subsec:rst_optimization}
Most of trajectory generation and path planning research concentrates in finding an optimal trajectory that minimizes a cost function $J(\cdot)$ under given constraints. Nevertheless, trajectory generation does not necessarily require optimality in the solution. In this section, we present an example of optimization framework built around the RST algorithm and we prove that when the number of waypoints $N+1$ is equal to $2$, RST (or PRST) directly provides the optimal solution in terms of minimum integral of the $p$-th derivative of the position squared, matching the trajectory generated by minimum-snap algorithm \cite{5980409} without any use of quadratic programming.

Sec.~\ref{subsec:rst_results} illustrated the RST algorithm, which is able to generate a polynomial trajectory $x_k(t)$ with minimum degree that satisfies the constraints. However, it is easy to notice that the general set of feasible trajectories is induced by $q(t)$ as follows
\begin{equation}
x_{\text{ext}}(t) = x(t)+\frac{a^{k+1}(t)}{(k+1)!}\cdot q(t),
\end{equation}
where $q(t)$ is a polynomial which introduces extra degrees of freedom needed for an optimization phase and $x(t)=x_k(t)$ is the trajectory generated via RST. 

As an example of an optimization problem, let $p = k+1$ be the order of the derivative of $x_{\text{ext}}(t)$ whose energy has to be minimized. A possible approach finds the solution to
\begin{equation}
\min_{q(t)}{\int_{t_0}^{t_N}{\biggl|\biggl|\frac{d^p}{dt^p}\biggl(x(t)+\frac{a^p(t)}{p!}\cdot q(t)\biggr)\biggr|\biggr|^2 dt}}
\end{equation}
with $q(t)$ polynomial, providing the optimal trajectory as
\begin{equation}
x_{\text{opt}}(t)=x(t)+\frac{a^p(t)}{p!}\cdot q_{\text{opt}}(t).
\end{equation}
Since all the functions inside the functional are polynomials, the coefficients of $q(t)$ can be in principle expressed analytically by integrating polynomials and by solving a system of linear equations. The convexity of the norm squared function guarantees a global minimum. 

When the number of waypoints is equal to $2$, that is $N=1$, the following Lemma asserts the optimality (in terms of energy) of the trajectory $x_k(t)$ generated with RST.

\begin{lemma}
\label{lemma:rst_Lemma5}
Let $x_k(t)$ be the trajectory generated with RST which satisfies the given kinematic constraints $\frac{d^i}{dt^i}x_k(t)\bigr|_{t=t_j}$ for $i=0,1,\dots, k$ and $j=0,1,\dots, N$. If $N=1$ and $p=k+1$, then the solution to
\begin{equation}
\label{prob:rst_functional}
\min_{q(t)}{\int_{t_0}^{t_1}{\biggl|\biggl|\frac{d^p}{dt^p}\biggl(x_{p-1}(t)+\frac{a^p(t)}{p!}\cdot q(t)\biggr)\biggr|\biggr|^2 dt}}
\end{equation}
is $q_{\text{opt}}(t)=0$, therefore the trajectory generated with RST is already the optimal one.
\end{lemma}

\begin{proof}
The proof uses some concepts of calculus of variations. In particular, let $\mathcal{L}$ be a Lagrangian function defined as 
\begin{equation}
\label{eq:rst_Lagrangian}
\mathcal{L} = \biggl(\frac{d^p x_{p-1}(t)}{dt^p}+\frac{d^p f(t)}{dt^p}\biggr)^2,
\end{equation}
with 
\begin{equation}
f(t) = \frac{a^p(t)}{p!}\cdot q(t).
\end{equation}
From calculus of variations theory, solving problem \eqref{prob:rst_functional} is equal to solving the Euler-Lagrange equation
\begin{equation}
\small
\label{eq:rst_EL}
\frac{\partial \mathcal{L}}{\partial f} - \frac{d}{dt}\biggl(\frac{\partial \mathcal{L}}{\partial \dot{f}}\biggr)+ \frac{d^2}{dt^2}\biggl(\frac{\partial \mathcal{L}}{\partial \ddot{f}}\biggr)-\dots+(-1)^{p}\frac{d^{p}}{dt^{p}}\biggl(\frac{\partial \mathcal{L}}{\partial f^{(p)}}\biggr)=0
\end{equation}
and by substituting the Lagrangian defined in~\eqref{eq:rst_Lagrangian} into~\eqref{eq:rst_EL}
it follows that 
\begin{equation}
\frac{d^{2p}}{dt^{2p}}\biggl(x_{p-1}(t)+f(t)\biggr)=0.
\end{equation}
From the considerations in Corollary \ref{corollary:rst_corollary1}, 
\begin{align}
\text{deg}(x_{p-1}(t))&=2p-1, \nonumber \\
\text{deg}(f(t))&= \text{deg}(a(t))+\text{deg}(q(t)) = 2p+Q,
\end{align}
with $Q\geq 0$. But, since $x$ and $f$ are polynomials, each differentiation reduces the degree by one and
\begin{equation}
\text{deg}\Bigg(\frac{d^{2p}}{dt^{2p}}\biggl(x_{p-1}(t)+f(t)\biggr)\Biggr)=Q=0,
\end{equation}
therefore $\text{deg}(q(t))=Q=0$ and in particular $q(t)\equiv 0$. \qedhere
\end{proof}
When the number of blocks $M$ increases, the overall optimal trajectory is obtained by optimizing the trajectories in each block.
When the number of waypoints in a single block is greater than $2$, the intrinsic optimality of the trajectory generated with RST is not guaranteed anymore and the optimization process provides $q(t)\neq 0$.  Next section illustrates trajectories generated via RST and via optimization of the integral of the $p$-th derivative of the position squared, denoted with RST$_{\text{opt}}$. 

\begin{figure*}[t]
\vskip 0.2in
\begin{center}
\centerline{
\includegraphics[width=\textwidth, height=9cm]{figures/architecture_img.pdf}}   
\vspace{-3mm}
\caption{\textbf{Overview of our method at the blending stage. }
% condition
Two input images or concepts are encoded into embeddings, mapped to a shared text space via the Linear Prior Converter from unCLIP~\citep{ramesh2022hierarchical}. These embeddings condition the U-Net: one for downsampling, the other for upsampling.
% module
During the blending stage, a blending latent $L_b$ initialized with Gaussian noise is processed in the Feedback Interpolation Module, conditioned on image embeddings. Noise $\epsilon$ is added to the embeddings to generate initial auxiliary latents, which are interpolated into $L^{(t)}_{b}$ with an increasing weight $p$. The  $L^{(t)}_{a}$ is combined with interpolated latent $L'^{(t)}_{b}$ by proportion $p$. All updated $L'^{(t)}_{a}$ are refined in the auxiliary inference to retain original features using the text prompt for corresponding categories, and $L'^{(t)}_{b}$ is denoised via the blending inference.
% refinement
Finally, the refined $ L_b $ is passed into the VAE decoder to generate the final blending image. 
}
\label{architecture}
\end{center}
% \vskip -0.4in
\vspace{-8mm}
\end{figure*}

\section{Evaluation}
We provide three sets of insights into this section, organised as \textit{findings (F*)}. We quantitatively study the effect of the adversarial and counterfactual perturbations on the performance of informal reasoners and autoformalisation methods. Then, we dive deeper into method variants. Finally, 
we analyse the nature of formalisation errors made by the models.

\subsection{Robustness Analysis}
\paragraph{\textbf{\emph{F1: Noise perturbations have a stronger effect on formalisation methods than informal \ac{LLM} reasoners.}}}
Table~\ref{tab:distraction_k4_formalisation} shows that, on average, the accuracy of both direct and \ac{CoT} informal reasoning remains between $73\%$ and $74\%$ in the face of added noise. While the autoformalisation method performs similarly to informal reasoners on the original dataset, its performance decreases between $4\%$ and $11\%$. The accuracy drops especially with logical (L) and tautological (T) distractions, whose logical language formats trick the \ac{LLM} into formalizing the noisy clauses. On the other hand, the linguistically complex and more natural sentences of encyclopedic distractions show a minor effect, suggesting that \acp{LLM} successfully avoids formalizing the more complicated sentences.

\paragraph{\textbf{\emph{F2: All \ac{LLM}-based reasoning methods suffer a drop for counterfactual perturbations.}}} % influence .}}}
Table~\ref{tab:distraction_k4_formalisation} shows that counterfactual statements cause a significant decrease in performance for both the informal reasoners and autoformalisation methods of between $12\%$ and $13\%$ on average. 
Moreover, this observation also holds for all tested models, i.e., none are robust towards counterfactual perturbations across every evaluated dimension. Even the strongest model, GPT 4o-mini, yields a performance of 63-68\%, which is relatively close to the random performance of 50\%. The high impact of counterfactual statements (the single ``not'' inserted) could be due to the inability of \acp{LLM} to overwrite prior knowledge with explicitly stated information or memorization of the answers. We study the error sources further in §\ref{subsec:errors}.  

\noindent \paragraph{\textbf{\emph{F3: Introducing multiple noise sentences has an effect only for logical distractions.}}}
We show the impact of introducing between one and four sentences for the two top-performing autoformalisation models in Figure~\ref{fig:length_distraction}. The figure shows similar trends with and without counterfactual perturbations.
As additional logical distractions are introduced, the model performance consistently decreases. Tautological (T) distractions lead to a decline in accuracy with a single disruptive sentence, yet adding more noise does not worsen the outcome. 
The tautological corpus introduces truth constants for all sentences as a persistent unseen logical construct. Given that this leads only to a decrease for a single occurrence, we can assume that a model can consistently handle the same unseen logical construct. In contrast, the logical corpus increases the chance of adding text, requiring new, previously unseen reasoning constructs for each added sentence. The impact of encyclopedic noise remains negligible, generalising F1 to $k$ sentences. Similarly, counterfactual perturbations remain much more effective for all settings, generalising F2.

\begin{table}[!t]
\small
\setlength{\modelspacing}{2pt}
\setlength{\tabcolsep}{1.7pt} % Default value: 6pt
\setlength{\belowrulesep}{4pt}
\begin{threeparttable}
    \centering
    \begin{tabular}{cc l r rrr @{\quad} rrrr}
\toprule
\multirow{2}{*}{} & \multirow{2}{*}{} & Reasoning & \multirow{2}{*}{O} & \multicolumn{3}{c}{Distraction} & \multicolumn{4}{c}{Counterfactual} \\
 & & Format & & E& L & T & $\text{O}_C$ & $\text{E}_C$& $\text{L}_C$ & $\text{T}_C$\\
\midrule
\multirow{6}{*}{\rotatebox{90}{Gemma-2}} & \multirow{3}{*}{\rotatebox{90}{9b}}
   & Informal (direct) & \textbf{0.78} & \textbf{0.80} & \textbf{0.79} & \textbf{0.77} & 0.58 & 0.52 & 0.50 & 0.59 \\
 & & Informal (CoT) & 0.72 & 0.78 & 0.73 & 0.76 & 0.61 & \textbf{0.57} & \textbf{0.60} & \textbf{0.66} \\
 & & Formal (FOL) & 0.62 & 0.58 & 0.52 & 0.53 & \textbf{0.63} & 0.52 & 0.46 & 0.46 \\[\modelspacing]
\cmidrule{2-11}
 & \multirow{3}{*}{\rotatebox{90}{27b}} 
   & Informal (direct) & 0.71 & 0.69 & \textbf{0.66} & \textbf{0.68} & 0.59 & 0.51 & 0.54 & 0.59 \\
 & & Informal (CoT) & 0.66 & 0.65 & 0.64 & 0.63 & 0.62 & 0.58 & \textbf{0.62} & \textbf{0.64} \\
 & & Formal (FOL) & \textbf{0.74} & \textbf{0.74} & 0.61 & 0.61 & \underline{\textbf{0.72}} & \underline{\textbf{0.67}} & 0.58 & 0.51 \\[\modelspacing]
\midrule
\multirow{6}{*}{\rotatebox{90}{Mistral}} & \multirow{3}{*}{\rotatebox{90}{7B}} 
   & Informal (direct) & 0.77 & \textbf{0.77} & 0.75 & \textbf{0.79} & \textbf{0.63} & \textbf{0.54} & \textbf{0.54} & \textbf{0.66} \\
 & & Informal (CoT) & \textbf{0.79} & 0.75 & \textbf{0.77} & 0.78 & 0.55 & 0.52 & \textbf{0.54} & 0.58 \\
 & & Formal (FOL) & 0.62 & 0.58 & 0.54 & 0.57 & 0.50 & \textbf{0.54} & 0.51 & 0.52 \\[\modelspacing]
\cmidrule{2-11}
 & \multirow{3}{*}{\rotatebox{90}{Small}} 
   & Informal (direct) & \textbf{0.77} & \textbf{0.76} & \textbf{0.76} & \textbf{0.75} & 0.61 & 0.51 & 0.56 & 0.59 \\
 & & Informal (CoT) & 0.72 & 0.72 & 0.72 & 0.71 & \textbf{0.62} & \textbf{0.59} & \textbf{0.62} & \textbf{0.68} \\
 & & Formal (FOL) & 0.68 & 0.59 & 0.53 & 0.64 & 0.54 & 0.55 & 0.49 & 0.51 \\[\modelspacing]
\midrule
\multirow{6}{*}{\rotatebox{90}{Llama-3.1}} & \multirow{3}{*}{\rotatebox{90}{8B}} 
   & Informal (direct) & 0.63 & 0.61 & 0.64 & 0.66 & 0.61 & \textbf{0.62} & 0.59 & 0.61 \\
 & & Informal (CoT) & 0.73 & \textbf{0.73} & \textbf{0.71} & \textbf{0.72} & \textbf{0.62} & 0.59 & \textbf{0.61} & \textbf{0.65} \\
 & & Formal (FOL) & \textbf{0.77} & 0.71 & 0.63 & 0.52 & 0.60 & 0.58 & 0.55 & 0.52 \\[\modelspacing]
\cmidrule{2-11}
 & \multirow{3}{*}{\rotatebox{90}{70B}} 
   & Informal (direct) & 0.77 & 0.74 & 0.74 & 0.73 & 0.62 & 0.53 & 0.56 & 0.64 \\
 & & Informal (CoT) & \textbf{0.78} & \textbf{0.75} & \textbf{0.76} & \textbf{0.76} & 0.64 & 0.61 & \textbf{0.66} & \underline{\textbf{0.73}} \\
 & & Formal (FOL) & 0.74 & 0.73 & 0.71 & 0.71 & \textbf{0.66} & \textbf{0.62} & 0.59 & 0.57 \\[\modelspacing]
 \midrule
\multirow{3}{*}{\rotatebox{90}{GPT}} & \multirow{3}{*}{\rotatebox{90}{4o-mini}} 
   & Informal (direct) & 0.78 & 0.77 & 0.79 & 0.79 & 0.64 & 0.61 & 0.61 & 0.63 \\
 & & Informal (CoT) & 0.80 & 0.80 & \underline{\textbf{0.81}} & \underline{\textbf{0.82}} & \textbf{0.68} & \textbf{0.63} & \underline{\textbf{0.68}} & \textbf{0.64} \\
 & & Formal (FOL) & \underline{\textbf{0.84}} & \underline{\textbf{0.82}} & 0.73 & 0.79 & 0.63 & 0.62 & 0.57 & 0.54 \\[\modelspacing]
 \midrule
\multicolumn{2}{c}{\multirow{3}{*}{\textbf{Avg}}} 
 & Informal (direct) & 0.74 & 0.73 & 0.73 & 0.73 & 0.61 & 0.55 & 0.56 & 0.62 \\
 & & Informal (CoT) & 0.74 & 0.74 & 0.73 & 0.74 & 0.62 & 0.58 & 0.62 & 0.65 \\
  & & Formal (FOL) & 0.72 & 0.68 &	0.61 & 0.62 & 0.61 & 0.59 & 0.54 & 0.52 \\
\bottomrule
\end{tabular}
\caption{Accuracies of informal and autoformalisation-based deductive reasoners. The best overall model per dataset is underlined; the best model version is marked in bold.}
\label{tab:distraction_k4_formalisation}
\end{threeparttable}
\end{table} 

\begin{figure}[!t]
    \centering
    \scriptsize
    \begin{tikzpicture}
        \begin{axis}[name=gpt,
            title={GPT-4o-mini},
            width=0.6\linewidth,
            height=0.6\linewidth,
            xlabel={\# Noise sentences},
            ylabel={Accuracy},
            xmin=-0.1, xmax=4.1,
            ymin=0.5, ymax=0.9,
            xtick={1,2,4},
            ytick={0.55, 0.6, 0.65, 0.75, 0.8, 0.85},
            title style={yshift=-0.6em},
            legend style={at={(1,-0.15)},
	           anchor=north,legend columns=-1},
            x label style={at={(axis description cs:1,-0.05)},anchor=north},
            y label style={at={(axis description cs:-0.15,0.5)},anchor=south},
            ymajorgrids=true,
            grid style=dashed,
        ]
            \addplot[color=blue, mark=square,]
                coordinates {
                (0,0.848076939582825)(1,0.823076903820038)(2,0.826923072338104)(4,0.821153819561005)
                };
            \addplot[color=red, mark=triangle,]
                coordinates {
                (0,0.848076939582825)(1,0.817307710647583)(2,0.801923096179962)(4,0.759615361690521)
                };
            \addplot[color=green, mark=diamond,] 
                coordinates {
                (0,0.848076939582825)(1,0.767307698726654)(2,0.769230782985687)(4,0.803846180438995)
                };
            \addplot[color=blue, mark=square*] 
                coordinates {
                (0,0.627777755260468)(1,0.622222244739533)(2,0.600000023841858)(4,0.633333325386047)
                };
            \addplot[color=red, mark=triangle*,] 
                coordinates {
                (0,0.627777755260468)(1,0.611111104488373)(2,0.611111104488373)(4,0.594444453716278)
                };
            \addplot[color=green, mark=diamond*,] 
                coordinates {
                (0,0.627777755260468)(1,0.572222232818604)(2,0.538888871669769)(4,0.555555582046509)
                };
                \legend{E,L,T,$\text{E}_C$, $\text{L}_C$ , $\text{T}_C$}
        \end{axis}

        \begin{axis}[name=llama, at={($(gpt.east)+(0.1cm,0)$)},anchor=west,
            title={Llama 3.1 70b},
            width=0.6\linewidth,
            height=0.6\linewidth,
            xmin=-0.1,, xmax=4.1,
            ymin=0.5, ymax=0.9,
            xtick={1,2,4},
            ytick={0.55, 0.6, 0.65, 0.75, 0.8, 0.85},
            title style={yshift=-0.6em},
            yticklabel=\empty,
            ymajorgrids=true,
            grid style=dashed,
        ]
            \addplot[color=blue, mark=square,]
                coordinates {
                (0,0.838461518287659)(1,0.817307710647583)(2,0.805769205093384)(4,0.817307710647583)
                };
            \addplot[color=red, mark=triangle,]
                coordinates {
                (0,0.838461518287659)(1,0.819230794906616)(2,0.803846180438995)(4,0.771153867244721)
                };
            \addplot[color=green, mark=diamond,]
                coordinates {
                (0,0.838461518287659)(1,0.803846180438995)(2,0.807692289352417)(4,0.805769205093384)
                };
            \addplot[color=blue, mark=square*]
                coordinates {
                (0,0.627777755260468)(1,0.622222244739533)(2,0.577777802944183)(4,0.594444453716278)
                };
            \addplot[color=red, mark=triangle*,]
                coordinates {
                (0,0.627777755260468)(1,0.583333313465118)(2,0.561111092567444)(4,0.577777802944183)
                };
            \addplot[color=green, mark=diamond*,]
                coordinates {
                (0,0.627777755260468)(1,0.627777755260468)(2,0.566666662693024)(4,0.577777802944183)
                };
        \end{axis}
    \end{tikzpicture}
    \caption{Influence of the number of noisy sentences for FOL.}
    \label{fig:length_distraction}
\end{figure}



\subsection{Impact of Method Design}
\paragraph{\textbf{\emph{F4: \ac{CoT} prompting is most impactful when both noise and counterfactual perturbations are applied.}}}
The accuracies for the individual \acp{LLM} in Table~\ref{tab:distraction_k4_formalisation} show that the impact of \ac{CoT} is negligible for noise-only datasets (first four columns). Meanwhile, the benefit from \ac{CoT} is most pronounced in the datasets that combine noise and counterfactual perturbations.
The better-performing informal prompting strategy for a model remains stable for all types of distractions. Still, the decline in performance due to counterfactuals leads to a less consistent preference for a specific prompting style.

\paragraph{\textbf{\emph{F5: The best-performing grammar differs per model and is unstable across data versions.}}}

The evaluation of different logical forms for formal \ac{LLM}-based reasoning in Table~\ref{tab:distraction_k4_logical_form} shows the preference of some models for specific syntactic formats.
Llama 3.1 70B has a considerable improvement of $12\%$ with TPTP syntax on the original set, while Llama 3.1 8B benefits from the R-FOL syntax. However, all grammars show a declining accuracy trend and increased syntax errors for noise perturbations, where the best grammar loses its advantage over the rest. 
When comparing the grammars on the counterfactual partitions, we observe that TPTP is consistently more robust than the standard first-order logic grammar. Here, GPT 4o-mini shows a reduction from $O$ to $O_C$ of $20\%$ for FOL and only $12\%$ for the TPTP grammar. Since this does not correlate with fewer syntax errors, the formalisation in TPTP prevents semantical errors for counterfactual premises. 
A positive reading of these results, especially the minor differences between FOL and R-FOL, is that autoformalisation \acp{LLM} can adapt to the grammar syntax prescribed in the prompt without further loss in performance.

\begin{table}[!t]
\small
\setlength{\modelspacing}{2pt}
\setlength{\tabcolsep}{1.7pt} % Default value: 6pt
\setlength{\belowrulesep}{4pt}
\begin{threeparttable}
    \centering
    \begin{tabular}{cc l r rrr @{\quad} rrrr}
\toprule
\multirow{2}{*}{} & \multirow{2}{*}{} & Grammar & \multirow{2}{*}{O} & \multicolumn{3}{c}{Distraction} & \multicolumn{4}{c}{Counterfactual} \\
 & & Syntax & & E& L & T & $\text{O}_C$ & $\text{E}_C$& $\text{L}_C$ & $\text{T}_C$\\
\midrule
\multirow{6}{*}{\rotatebox{90}{Llama-3.1}} & \multirow{3}{*}{\rotatebox{90}{8B}} 
   & FOL & 0.77 & \textbf{0.71} & 0.61 & \textbf{0.53} & 0.58 & \textbf{0.55} & 0.52 & \textbf{0.56} \\
 & & R-FOL & \textbf{0.78} & 0.69 & \textbf{0.62} & \textbf{0.53} & 0.58 & \textbf{0.55} & \textbf{0.54} & 0.52 \\
 & & TPTP & 0.73 & 0.67 & 0.55 & 0.51 & \textbf{0.68} & 0.54 & 0.46 & 0.51 \\[\modelspacing]
\cmidrule{2-11}
 & \multirow{3}{*}{\rotatebox{90}{70B}} 
   & FOL & 0.76 & 0.73 & 0.71 & \textbf{0.72} & 0.67 & 0.57 & 0.63 & 0.56 \\
 & & R-FOL & 0.76 & 0.73 & 0.67 & 0.71 & 0.64 & 0.57 & 0.53 & 0.64 \\
 & & TPTP & \underline{\textbf{0.88}} & \underline{\textbf{0.84}} & \underline{\textbf{0.81}} & \textbf{0.72} & \underline{\textbf{0.81}} & \underline{\textbf{0.68}} & \underline{\textbf{0.67}} & \underline{\textbf{0.68}} \\[\modelspacing]
\midrule
\multirow{3}{*}{\rotatebox{90}{GPT}} & \multirow{3}{*}{\rotatebox{90}{4o-mini}} 
   & FOL & \textbf{0.84} & \textbf{0.82} & \textbf{0.72} & \underline{\textbf{0.78}} & 0.64 & \textbf{0.63} & \textbf{0.61} & 0.51 \\
 & & R-FOL & \textbf{0.84} & 0.77 & 0.70 & \underline{\textbf{0.78}} & \textbf{0.72} & 0.56 & 0.54 & \textbf{0.63} \\
 & & TPTP & 0.83 & \textbf{0.82} & 0.71 & 0.71 & 0.69 & \textbf{0.63} & 0.57 & 0.57 \\
\bottomrule
\end{tabular}
\caption{Accuracies of different formalisation grammars for autoformalisation.}
\label{tab:distraction_k4_logical_form}
\end{threeparttable}
\end{table} 

\paragraph{\textbf{\emph{F6: Feedback does not help \acp{LLM} self-correct to mitigate robustness issues.}}}
\autoref{tab:distraction_k4_feedback} shows the results with different error recovery mechanisms. The results indicate that no feedback strategy emerges as a winner in the different datasets. 
All feedback variants reduce syntax errors for noise perturbations, but given the lack of a consistent increase in accuracy, the corrected formalisations are most likely to contain semantic errors still. 
The type of feedback message only has a minor influence on correcting syntax errors, whereas Llama 3.1 70b and GPT 4o-mini correct slightly more syntax errors with specific error messages. This finding aligns with \cite{huang2023large}, who also found that \acp{LLM} cannot consistently self-correct their reasoning after receiving relevant feedback.

\begin{table}[!ht]
\small
\setlength{\modelspacing}{2pt}
\setlength{\tabcolsep}{1.7pt} % Default value: 6pt
\setlength{\belowrulesep}{4pt}
\begin{threeparttable}
    \centering
    \begin{tabular}{cc l r rrr @{\quad} rrrr}
\toprule
\multirow{2}{*}{} & \multirow{2}{*}{} & \multirow{2}{*}{Feedback} & \multirow{2}{*}{O} & \multicolumn{3}{c}{Distraction} & \multicolumn{4}{c}{Counterfactual} \\
 & & & & E& L & T & $\text{O}_C$ & $\text{E}_C$& $\text{L}_C$ & $\text{T}_C$\\
\midrule
\multirow{8}{*}{\rotatebox{90}{Llama-3.1}} & \multirow{4}{*}{\rotatebox{90}{8B}} 
   & No recovery & 0.77 & \textbf{0.72} & 0.62 & 0.53 & 0.59 & 0.58 & 0.56 & \textbf{0.56} \\
 & & Error type & \textbf{0.79} & 0.71 & 0.63 & \textbf{0.56} & \textbf{0.66} & 0.54 & 0.52 & 0.51 \\
 & & Error message & 0.78 & 0.71 & \textbf{0.67} & 0.55 & 0.59 & 0.53 & \underline{\textbf{0.64}} & 0.49 \\
 & & Warning & 0.74 & 0.66 & 0.58 & 0.55 & 0.55 & \textbf{0.60} & 0.49 & 0.49 \\[\modelspacing]
\cmidrule{2-11}
 & \multirow{4}{*}{\rotatebox{90}{70B}} 
   & No recovery & \textbf{0.77} & \textbf{0.72} & \textbf{0.73} & 0.71 & \textbf{0.64} & 0.59 & \textbf{0.61} & 0.56 \\
 & & Error type & 0.72 & 0.70 & 0.72 & \textbf{0.73} & 0.62 & 0.56 & 0.60 & 0.58 \\
 & & Error message & 0.71 & 0.70 & \textbf{0.73} & 0.71 & \textbf{0.64} & 0.59 & 0.54 & \underline{\textbf{0.64}} \\
 & & Warning & 0.69 & \textbf{0.72} & 0.72 & 0.72 & 0.62 & \underline{\textbf{0.65}} & \textbf{0.61} & 0.63 \\[\modelspacing]
\midrule
\multirow{4}{*}{\rotatebox{90}{GPT}} & \multirow{4}{*}{\rotatebox{90}{4o-mini}} 
   & No recovery & \underline{\textbf{0.84}} & \underline{\textbf{0.82}} & 0.73 & 0.79 & 0.64 & \textbf{0.62} & 0.56 & \textbf{0.56} \\
 & & Error type & 0.83 & 0.79 & 0.74 & 0.76 & 0.67 & 0.57 & 0.56 & \textbf{0.56} \\
 & & Error message & \underline{\textbf{0.84}} & 0.78 & \underline{\textbf{0.77}} & \underline{\textbf{0.80}} & 0.62 & 0.59 & 0.56 & \textbf{0.56} \\
 & & Warning & \underline{\textbf{0.84}} & 0.75 & 0.73 & 0.76 & \underline{\textbf{0.70}} & 0.61 & \textbf{0.61} & 0.55 \\
 \bottomrule
\end{tabular}
\caption{Accuracies of error recovery strategies.}
\label{tab:distraction_k4_feedback}
\end{threeparttable}
\end{table} 

\subsection{Error Analysis}
\label{subsec:errors}
\paragraph{\textbf{\emph{F7: Autoformalisation increases syntax errors for noise perturbations.}}}
The low performance for noise perturbations correlates with more syntax errors for all models and distraction categories (cf. execution rates in Table~\ref{tab:appendix_k4_formalisation_exec}). The three worst-performing models (both Mistral models, Gemma-2 9b) generate, at best, for $37\%$  and, at worst, for only $4\%$ of the samples, a valid logical form.
Gemma-2 9b and Llama3.1 8b produce more syntax errors than the larger counterparts, suggesting that larger models are more robust towards noise perturbations. 
The accuracy of syntactically valid samples is higher than the informal reasoning methods for most distractions (Table~\ref{tab:appendix_k4_formalisation_vacc}), motivating informal reasoning as a backup strategy for formal reasoning. The error message feedback reveals two common syntax errors: 1) errors by models with an initial low execution rate exhibit issues with the template structure, including using incorrect keywords or adding conversational phrases;
2) perturbation-related errors, the most common of which is using undefined truth constants as part of tautological distractions. 

\paragraph{\textbf{\emph{F8: Autoformalisation increases semantic errors for counterfactuals.}}}
Unlike the introduced noise, counterfactual perturbations do not lead to more syntax errors. The execution rate in Table~\ref{tab:appendix_k4_formalisation_exec} is stable or improves for counterfactuals. However, we see a drop in accuracy for the counterfactual column $\text{O}_C$ in Table~\ref{tab:distraction_k4_formalisation} and can conclude that the number of logical forms with semantic errors has to increase. This suggests that the introduced negation is not correctly formalised. Looking at the warnings generated by the feedback mechanism, for GPT 4o-mini, $161$ warning messages are generated on the unperturbed data. $54$ of these were fixed with a single iteration. Not considering predicates and individuals as part of the context is the most frequent warning across all models. 
%

%%%%%%%%%%%%%%%%%%%%%%%%%%%%%%%%%%%%%%%%%%%%%%%%%%%%%%%%%%%%%%%%%%%%%%%%%%%%%%
%%%%%%%%%%%%%%%%%%%%%%%%%%%%%%%%%%%%%%%%%%%%%%%%%%%%%%%%%%%%%%%%%%%%%%%%%%%%%%
\newpage
\section{Long Term Allocation (ILP)} \label{sec:ILP_dummy}
We use an ILP to find the optimal resource allocation on a longer time frame.
\apnote{remove this whole section once we finalize the whole paper.}
\subsection{ILP Formulation}
The ILP will try to minimize the cost of resources whilst servicing all the low-latency workloads. It will take into account re-routing and having multiple kinds of GPUs.
\subsubsection{Input}
\begin{itemize}
    \item $l$: number of different models we will try to serve
    \item $r$: number of regions
    \item $g$: number of GPU architectures we have
    \item $C: [\texttt{int}]_{l\times r\times g}$: $C_{i, j, k}$ is the number of instances of model $i$ at region $j$ running on GPU architecture $k$
    \item $R: [\texttt{int}]_{l\times r}$: $R_{i, j}$ is the \textbf{TPS requested} for model $i$ at region $j$ 
    \item $\bar{\theta}: [\texttt{float}]_{l\times g}$: $\bar{\theta}_{i, k}$ is the TPS of model $i$ on GPU $k$
    \item $G: [\texttt{float}]_{g}$: $G_{k}$ is the cost of acquiring GPU $k$
    \item $s: [\texttt{float}]_{l\times G}$: $s_{i, k}$ is the cost of acquiring model $i$ on GPU $k$
\end{itemize}
\subsubsection{Decision Variable}
Our decision variable $\delta$ will have dimension $l\times r\times g$ where $\delta_{i,j,k}$ will decide the change in number of instances assigned to model $i$ at region $j$ running on GPU $k$. 
\subsubsection{Constraints}
\begin{itemize}
    \item The first constraint will be to not deallocate more models at a particular configuration than there exist: $$\delta_{i,j,k} \geq C_{i, j, k}$$
    \item Second, we should ensure that we have capacity to process all the low latency tokens: 
    $$\theta_{i} = \sum^{j}\sum^{k}(C_{i, j, k} + \delta_{i, j, k}) * \bar{\theta}_{i, k} \text{  }\forall\text{ 
 }i\in\{l\}$$
    $$\theta_{i} \leq \sum^{j}R_{i, j} \text{ 
 }\forall\text{  }i\in \{l\}$$ Because we want to consider re-routing, this condition only checks that for each model, we should have enough capacity over all regions
 \item Next, we want each region $j$ to process atleast $(1-\gamma)R_{i, j}$ and atmost $(1+\gamma)R_{i, j}$ of its tokens for model $i$ in order to limit routing:
 $$\hat{\theta}_{i, j} = \sum^{k}(C_{i, j, k} + \delta_{i, j, k}) * \bar{\theta}_{i, k}$$
 $$(1-\gamma)R_{i, j} \leq \hat{\theta}_{i, j} \text{  }\forall\text{  }(i, j)\in\{l\}\times \{r\}$$
\end{itemize}
\subsubsection{Objective}
Our objective is to minimize the cost for capacity for answering all queries. Cost is affected by allocating new GPUs as well as the start up time for each model. Similarly, it is reduced when you deallocate any GPUs, but the model does not impact that cost:
$$\texttt{GPU cost} = \sum^{k}((\sum^{i, j}\delta_{i, j, k}) * G_{k})$$
$$I_{i, j, k} = \max(0, \delta_{i, j, k})$$
$$\texttt{Model startup cost} = \sum^{k}\sum^{i}(\sum^{j}I_{i, j, k} * s_{i, k})$$
$$\min (\texttt{GPU cost} + \texttt{Model startup cost})$$
\subsection{Load Estimation}
We use ARIMA modelling for estimating the load
\subsection{Buffer Addition using NIW}
We add known non-interactive workloads to the estimated load as buffer
\subsection{Combining Proactive and Reactive Mechanisms}
The proactive mechanism will tell us the final state we should aim for. The reactive mechanism will help us guide there. It will try to converge to the proactive mechanism when there are bursts of requests.


% 
% \begin{figure*}[htpb!]
% \label{}
% \centering

%     {{\label{ROCIowaCedar} \includegraphics[width=\textwidth/3]{figures/IowaCedar_roc.png}}}%
%     \qquad
%     {{\label{ROCIowaDesMoines} \includegraphics[width=\textwidth/3]{figures/IowaDesMoines_roc.png} }%
%   \captionsetup{justification=centering}
%   \caption{\Acf{ROC} curves for \acf{RW} Iowa (CR) and  \acf{RW} Iowa (DM) dataset. Dummy model here represents a model whose output is solely a ``no Flood'' for all pixels.}
%   \label{fig:RW_ROC_Curves}%
% \end{figure*}



\section{Results and Discussions}
\label{sec:Results}

In this section, we aim to answer three main questions. First, we want to validate our hypothesis that \ac{SYN} data is a viable proxy for \ac{RW} data when training ML models for downscaling. Secondly, we seek to assess how much more skillful ML-based downscaling is compared to classical, non-data-driven techniques, such as our baseline methods, \textit{i.e.}, thresholded bicubic and Lanczos interpolation. Finally, we would like to appraise the extent to which data-driven models like ours are transferable (in terms of usefulness) to other regions without major performance degradations.  
To assess the quality of the models, we conduct a multiple comparison test --namely the Holm-Bonferroni procedure \cite{HolmBonferroni1979} -- that is designed to control the \ac{FWER}. We notice that, with a \ac{FWER} of $10^{-3}$, all the differences in model performance are significant. The only exception to this trend was observed in \ac{RW}-GH for whom the pairwise differences between \ac{RCAN} and \ac{ESRT}, Lanczos and Bicubic were not significant with the aforementioned \ac{FWER}. 

%Finally, we aim to find out the factors influencing the transferability of our models from one region to another.

\subsection{Potential of using SYN Data for RW downscaling}

In order to evaluate the utility of synthetic data for training, we compare performances of our candidate models on both \ac{SYN} and \ac{RW} Iowa data whose results are presented in Table \ref{tab:IowaResults}. We notice that 
\textbf{(i)} For the Iowa datasets, there is a drop in performance of all the models when going from \ac{SYN} to \ac{RW} datasets, 
\textbf{(ii)} for the \ac{RW}-IA (CR) as well as \ac{RW}-IA (DM) datasets, both bicubic and Lanczos interpolation have accuracies and MCC up to 70.89\% and 0.42 respectively while the deep learning models have accuracies and MCC up to 73.34\% and 0.46 respectively, 
\textbf{(iii)} There is a roughly 6\% accuracy improvement for the \ac{SYN} data for the deep learning models compared to the bicubic and lanczos models and this improvement drops to about 3\% for \ac{RW} data,  
\textbf{(iv)} the performance of all the models remain consistent across both \ac{RW}-IA datasets and \textbf{(v)} in \figref{fig:RW_ROC_Curves}, we observe that there is a high degree of overlap among the \ac{ROC} curves for the data-driven models.

From (i) and (iv) we can conclude that \ac{SYN} data is more intricate than \ac{RW} data. This implies that the benefits yielded by training with \ac{SYN} dataset, while significant, is not as prominent in the \ac{RW} Iowa datasets. 
% This may be due to sensor noise prevalent in the \ac{RW} Landsat-8 data that can be harder to reproduce in the synthetically generated examples. 
(i), (iii) and (v) implies that while \ac{SYN} data is not an exact replacement for \ac{RW} data, it provides a rather significant edge, which is all the more important when there is insufficient \ac{RW} for training. From (ii) we can conclude that the three proposed data driven models outperform classical super-resolution techniques such as bicubic and lanczos, conclusion supported by the \ac{ROC} curves in Figure \ref{fig:RW_ROC_Curves} for whom the data-driven models, in general, lie above the non-data-driven alternatives. Observation (iv) shows that  for the climatically similar \ac{RW}-Iowa(CR) and \ac{RW}-Iowa(DM) regions, training on \ac{SYN} Iowa data does indeed provide an edge. 

% have similar climate. 

\begin{figure*}[t!]
    \centering
    \begin{subfigure}[t]{0.5\textwidth}
        \centering
        \includegraphics[width=\textwidth/2]{figures/IowaCedar_roc.png}
        \caption{}
    \end{subfigure}%
    ~ 
    \begin{subfigure}[t]{0.5\textwidth}
        \centering
        \includegraphics[width=\textwidth/2]{figures/IowaDesMoines_roc.png}
        \caption{}
    \end{subfigure}
    \vspace*{0.5cm}
    \caption{    \label{fig:RW_ROC_Curves} \Acf{ROC} curves for (a) RW-IA (CR) and (b) RW-IA (DM) dataset. Na\"ive model here represents a model whose output is solely a ``no Flood'' for all pixels. Star here represents the pixel-wise classifier with a threshold of 0.5.}
\end{figure*}


\subsection{Effectiveness of data-driven approaches}

In order to evaluate the effectiveness of ML models in the downscaling task, we compare performances of our candidate models to Lanczos and bicubic interpolation methods by looking at figures of some sample predictions from Iowa (Figure \ref{fig:RWIowaDesMoines}), performance comparison in the region of Iowa in Table \ref{tab:IowaResults} and the ROC curves in Figure \ref{fig:RW_ROC_Curves} for \ac{RW} data. We notice that 
\textbf{(vi)} For RW-IA (DM) samples, the deep learning models maintain a higher degree of spatial continuity in the predicted \ac{FIM}, 
\textbf{(vii)} We observe that  bicubic and Lanczos interpolation produces over-smoothed \ac{FIM} reconstructions, while the plain \ac{RDN}, \ac{RCAN} and \ac{ESRT} models are more detail-inclusive. Similar conclusions can be drawn upon inspecting the \ac{ROC} curves in Figure \ref{fig:RW_ROC_Curves} and 
\textbf{(viii)} For RW-IA (CR), the ML models show a performance improvement of 3.06\% when comparing the best ML model and non-data-driven method and, while for RW-IA (DM) there is a performance improvement of 2.45\%.


Figures \ref{fig:EUSamples} and \ref{fig:RWIowaDesMoines} show the spatial disparity among the models whose details are often obscured in aggregated metrics such as accuracy. (vi) This implies that these data-driven models are better are recognizing an underlying stream network geometry than the classical methods. However, when it comes to narrow river streams, all the models struggle capturing the nuances of the \ac{FIM} resultant from localized high elevation features such as small islands within rivers or man-made structures. (vii) shows a clear advantage of our data-driven approaches over the non-data-driven alternatives. (viii) indicates the benefits of the data-driven models when evaluated over Iowa. 



\subsection{Applicability of our models to external regions}

To evaluate how transferable our models are, we draw conclusions from figures of the sample predictions from Western Europe (Figure \ref{fig:EUSamples}) and Ghana (Figure \ref{fig:GhanaSamples}) as well as the performance comparison in Table \ref{tab:ExternalResults}. We notice that 
\textbf{(ix)} for Ghana all of the models fail to adequately inundate the pixels over separated areas on account of several disconnected regions of inundation in the chosen area,
\textbf{(x)} the ML models outperform non-data driven methods for RW-EU, 
\textbf{(xi)} for the RW-EU dataset, there is an improvement of 4.89\% when comparing the accuracy of the best data- and non-data-driven methods, 
\textbf{(xii)} For RW-RR and RW-GH, there is marginal improvement (up to 0.77\% in accuracy) of the ML methods over the non-data driven methods and 
\textbf{(xiii)} For RW-EU, we notice that the ML models produce more connected streams over the non-data-driven models. 

(x) and (xi) implies that the models are transferable when considering hydroclimaticalogically similar regions since Iowa and the Meuse river in Europe lie within mid temperate zones. Similar to the observation (vi) for RW-IA (DM), (xiii) implies that the benefits of the ML model in identifying underlying network streams is also transferable to hydroclimatologically similar regions. In contrast, (xii) and (ix) both imply that the trained ML models struggle to generalize to RW-RR \& RW-GH. We speculate that this may be due to the significant differences in geography and climate when compared to Iowa. 

% More specifically, we notice that Ghana has a lot of disconnected regions when compared to Iowa and Western Europe, possibly indicating a geomorphological dissimilarity. Additionally, in the case of Red River and Ghana, we also speculate that they include drivers to flood inundation that are different from Iowa and Western Europe, which lie within mild temperate zones. Ghana on the other hand has a tropical (dry and hot) climate.

Our study directly implies that good quality synthetic data can be useful surrogates for downscaling low-resolution \acp{WFM} to high-resolution \acp{FIM} in regions, where such data are hard to come by, even when downscaling by a factor of 10. We noticed that such models were readily transferable to climatically similar regions as the region of training. However, Such derived ML models did not feature significantly different transferability when evaluated over hydroclimatologically dissimilar regions, which we attribute to different flood inundation characteristics, primarily at finer scales. A possible avenue to circumvent such issues is to explore additional training approaches that fall under the general area of domain adaptation. Nevertheless, data-driven models are still advantageous (and, hence, preferable) over non-data-driven alternatives in transfer scenarios like the one we considered here. 


%%%%%%%%%%%%%%%%%%%%%%%%%%%%%%% unused text %%%%%%%%%%%%%%%%%%%%%%%%%%%%%%%%%%%%%%%



% \tabref{tab:AccuracyResults} depicts test accuracies obtained by our models on both \ac{SYN} and \ac{RW} data. For Iowan floods, a comparison of \ac{SYN} and \ac{RW} results shows \textbf{(i)} bicubic and Lanczos interpolations remarkably gaining about $3\%$ in accuracy, as well as \textbf{(ii)} \ac{RDN} and \ac{RCAN} remaining relatively stable, while \textbf{(iii)} topography-aware models loosing $2.7\%$ in performance. From (i) one can conclude that \ac{SYN} data are morphologically slightly more intricate than \ac{RW} data. Also, (i) and (ii) likely imply that \ac{SYN} data, excluding topography, can serve as satisfactory surrogates of \ac{RW} data. However, as implied by (iii), our topography-dependent models seems to be particularly sensitive to distributional shifts of their combined inputs (\acp{WFM} and topographic features). More specifically, the topography-informed models' performance edge, while still statistically significant, is extremely marginal, even when compared to our non-data-driven approaches. Next, when comparing results between the cases of Iowan and Ghanaian \ac{RW} data, one observes that \textbf{(iv)} the accuracy of bicubic and Lanczos interpolations drops by almost $5\%$ due to over-smoothing. This may imply that Ghanaian \acp{FIM} bare a more complex morphology, when compared to Iowan \acp{FIM}. Also, \textbf{(v)} our topography-agnostic, data-driven models' performance degrades more gracefully (by about $2\%$), while \textbf{(vi)} our topography-aware models perform, virtually, as bad as our non-data-driven approaches. Hence, the differences in the data populations of the two regions we considered are significant enough to render our topography-dependent models noncompetitive. 



\section{Conclusions \pglen{0.25}}
\label{sec:conclude}

We present \sys, a holistic system for serving LLM inference requests with a wide range of SLAs, which maintains better GPU utilization, reduces resource fragmentation that occurs in silos, and increases utility by donating surplus instances to Spot instances. 
\sys achieves this through its unique elements, namely, a holistic deployment stack for requests of varying SLAs, its async feed module, and long-term aware proactive scaler logics that capitalize on the underutilized instances of another model in the same region by inter-model redeployment.

Future work includes extending \sys to accomodate workloads with a continuum of SLAs and conducting extensive studies on the benefits of the proposed approach with deployments across heterogeneous hardware types. We plan to open-source our trace data and simulator.


% \input{sections/new_data}

% conference papers do not normally have an appendix
% The Computer Society usually uses the plural form
% \section*{Acknowledgments}
% \ysnote{Thank all your colleagues who helped with the paper. It is good form.}




%----------------------------------------------------------------------------------
% \section*{Acknowledgments}
%-------------------------------------------------------------------------------

%-----------------------------------------------------------------------------
%-------------------------------------------------------------------------------
\bibliographystyle{plain}
%\bibliography{\jobname}
\bibliography{main}

%%%%%%%%%%%%%%%%%%%%%%%%%%%%%%%%%%%%%%%%%%%%%%%%%%%%%%%%%%%%%%%%%%%%%%%%%%%%%%%%
\end{document}
%%%%%%%%%%%%%%%%%%%%%%%%%%%%%%%%%%%%%%%%%%%%%%%%%%%%%%%%%%%%%%%%%%%%%%%%%%%%%%%%

%%  LocalWords:  endnotes includegraphics fread ptr nobj noindent
%%  LocalWords:  pdflatex acks