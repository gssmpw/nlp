\section{Conclusions}
\label{sec:conclusions}

This paper has described a method for creating compelling safety cases. The method seeks to help improve safety case practice in order to address the weaknesses identified in current practice. In particular the method aims to address the issues of confirmation bias, after-the-fact assurance and safety case as paperwork exercise, that have been identified as key problems. Rather than attempting to create new notations and tools to address these issues, we contend that it is improvements in the safety case process that will make the most significant improvement to safety case practice. The new method described in this paper builds upon established approaches and best practice to create an approach that will ensure safety cases are risk-focused, seek to identify ways in which the system may \textbf{\textit{not}} be safe (rather than just assuming it is), drive safe design and operation of the system (influencing the system itself rather than just documenting what's there), are used to support decisions made throughout the life of the system, including system operation and change, and encourage developers and operators to think about and understand why their system is safe (and when it isn't). As systems and technology becomes more complex, such as through increasing use of autonomy and artificial intelligence, increased inter-connectivity and more dynamic and rapidly changing systems, improving safety case practice in this manner becomes even more important to ensure safety cases remain fit for purpose.

We have used a simple example of an infusion pump system to illustrate how the method is applied in practice. Further work to validate the method will focus on application to more complex and complete examples, and assessment of applicability by independent safety case developers.


