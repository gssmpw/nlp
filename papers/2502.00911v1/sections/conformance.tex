\section{Completing the Safety Case: Conformance Arguments}
\label{sec:conform}

The method introduced in this paper has dealt with a number of the different aspects that are expected to be considered within a safety case, namely the risk, confidence and operational arguments for the system. In order to provide a complete safety case for the system there is another aspect that should be considered and that is the issue of conformance (or compliance) with safety standards. Although a compelling safety case can be created based upon risk, confidence and operational arguments, there is often also a requirement to demonstrate that applicable standards are satisfied, and the safety case can provide a mechnaism for achieving this through the provision of a \textbf{\textit{conformance argument}}. 

\textcolor{red}{As described in [conformance args paper Graydon et.al.] a conformance argument is...}

\textcolor{red}{It is important to note that a conformance argument may not always be the most effective way of demonstrating that the requirements of standards are met...}

\textcolor{red}{For those requirements of the standard where a conformance argument is considered to be required there may often be arguments created as part of the risk or confidence arguments that are able to be used to demonstrate conformance, for example... }
Rather than recreate those arguments as part of conformance argument, the conformance argument should refer to those parts of the risk or confidence arguments that are relevant.

Where required, the conformance arguments should be developed alongside the other safety case arguments to ensure that any constraints on the argument strategies adopted eminating from conformance obligations are considered as the safety case is developed, rather than seeking to modify the arguments later to ensure conformance can be demonstrated. 