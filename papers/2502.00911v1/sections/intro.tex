\section{Introduction}
\label{sec:intro}

Safety cases provide a way to communicate a clear, comprehensive and defensible argument that a system is acceptably safe to operate \cite{kelly1999arguing}. They consist of a structured argument, supported by a rigorous body of evidence \cite{0056}. The concept and use of safety cases developed in response to a number of high profile disasters, notably the Windscale reactor fire in 1957 \cite{penney2017report}, the Flixborough factory explosion in 1974 \cite{flix} and the Piper Alpha oil platform fire of 1988 \cite{cullen1993public}, where the requirement for a safety case to explicitly argue the safety of the system was recommended as a way to help prevent similar catastrophes from occurring. The adoption of safefty cases has since become widespread across many different industries including rail \cite{std2019railway}, automotive \cite{26262}, nuclear \cite{onr}, air \cite{icao}, medical \cite{tir382019medical}, oil and gas \cite{hse2006offshore} and defence \cite{mod}.

In order to support the development of compelling safety cases, a number of safety argument notations have been developed to help to create and document clear safety argument structures. The most commonly used safety argument notation is the Goal Structuring Notation (GSN) \cite{group2021a}, with others such as Claims, Argument, Evidence (CAE) \cite{cae} also being used. These notations are also supported by methods that provide guidance on how to use the notation to create safety arguments (such as the 6-step method for GSN \cite{kelly1999arguing}). All safety argument notations build upon the same foundations of inductive argumentation as described by Stephen Toulmin in the 1950s \cite{toulmin2003uses}. Safety arguments presented in safety cases for systems are mostly inductive in nature, where the truth of the safety claims cannot be proven with absolute certainty, and are instead supported by the argument and evidence with a level of probability. Crucially, Toulmin's approach involves assessing the argument being made in order to make it stronger (a theme we return to later). Toulmin introduced the core elements required of any inductive argument, and these elements remain crucial to the arguments used in safety case to this day. Figure \ref{fig:toulmin} shows these elements and their relationships:
\begin{itemize}
    \item Data (D): The evidence, information, etc. which support the truth of the claim. 
    \item Claim (C): The conclusion of the argument. This may be an intermediate step used as data for the next inference.
    \item Warrant (W): The reasons that link the data to the claim. 
    \item Backing (B): Statements that serve as evidence for showing that the warrants are true.
    \item Rebuttals (R): Exceptions to the claim, showing circumstances when the claim might not be true.
    \item Qualifiers (Q): Statements or phrases which reflect the level of probability or truth of the claim.
\end{itemize}


\begin{figure*}
\centering
\includegraphics[width=0.4\textwidth]{images/toulmin.png}
\caption{Toulmin's model of inductive arguments (from \cite{toulmin2003uses}).}
\label{fig:toulmin}
\end{figure*}


Despite the increasing adoption of safety cases across different industries and the establishment of safety argument notations and techniques, it appears that the standard of safety case practice remains mixed. Whilst there are clearly examples of good practice across industry, it is also clear that there are many more examples where safety cases are not being used effectively. This was brought into sharpest focus in the report by Charles Haddon-Cave into the 2006 Nimrod accident in Afghanistan \cite{qc2009nimrod}. Amongst many findings, the report considered the role of the safety case that had been developed for the Nimrod aircraft. Haddon-Cave determined that \textit{``if the Nimrod Safety Case had been prepared with proper skill, care and attention, ... XV230 would not have been lost.''} This supports the assertion that safety cases are crucial to ensuring the safety of systems, but only if they are used effectively. Haddon-Cave goes on to say that, \textit{``Unfortunately, the Nimrod Safety Case was a lamentable job from start to finish. It was riddled with errors. It missed the key dangers. It was essentially a ‘paperwork’ exercise. It was virtually worthless as a safety tool''}.  In particular, he highlights the need to \textit{``...focus attention on the fact that [safety cases] are about managing risk, not assuming safety''}.

There are other sources of criticism of how safety cases are being used in practice \cite{kelly2008safety}, \cite{habli2021safety}. The most high profile of these is probably Nancy Leveson, who has published a number of papers that critique the safety case approach \cite{leveson2020white}, \cite{leveson2011use}. In particular Leveson identifies a number of problems she perceives with the use of safety cases that echo the key problems highlighted by Haddon-Cave. These are:
\begin{itemize}
    \item \textbf{Confirmation bias} - seeking out evidence to support a pre-determined conclusion whilst tending to ignore information that may be contradictory. It is often stated that current safety case practice encourages confirmation bias by starting from the presumption that a system is sufficiently safe. A good safety case requires that focus is also paid to understanding why a system may be unsafe.
    \item \textbf{After-the-fact assurance} - attempting to provide assurance of the safety of a system once development of the system is complete. Current practice can sometimes view the creation of a safety case as an activity to be performed at the end of the lifecycle, thus offering no opportunity for the development of the safety case to influence the design of the system in a timely manner. This practice can also increase the likelihood of confirmation bias in a safety case since it increases pressure to ``show the system is safe''.
    \item \textbf{Paperwork exercise} - viewing the task of creating a safety case as purely one of creating a document. If safety cases are viewed simply as a paperwork exercise (to show compliance with a requirement to produce a document) then they are of little value in terms of ensuring a system is safe to operate. Safety cases treated in this way will not play an active role in the development and assurance of safe systems, and are instead likely to become ``shelf-ware''. They are also likely to produced after-the-fact and subject to confirmation bias. 
\end{itemize}

There have been many papers written discussing how these safety case challenges could be addressed. These often focus on new notations, such as \cite{goodenough2015eliminative}, \cite{holloway2023friendly}, or increased tool support and automation, such as \cite{denney2012advocate}, \cite{matsuno2010dependability}. In contrast, the premise of this paper is that to make a substantive improvement in safety case practice, additional notations and tools are not required, existing notations, building on Toulmin's foundations, provide everything that is needed to create compelling safety cases (as is seen in good practice today). Instead we believe that the solution lies in better application of the existing concepts to address the identified challenges and to better support safety case development for ever more complex systems and emerging classes of system with more rapid development and update cycles. Again we can find support for this in the Nimrod report where it highlights that, \textit{``The basic aims, purpose and underlying philosophy of Safety Cases were clearly defined, but there was limited practical guidance as to how, in fact, to go about constructing a Safety Case''} \cite{qc2009nimrod}.

In this paper we propose a safety case development method that builds on establishes approaches in order to help improve safety case practice. The aim of the method is to address the challenges identified above in order to create safety cases that:
\begin{itemize}
    \item are risk-focussed
    \item seek to identify ways in which the system may \textbf{\textit{not}} be safe (rather than just assuming it is)
    \item drive safe design and operation of the system (influencing the system itself rather than just documenting what's there) \cite{graydon2007assurance}
    \item are used to support decisions made throughout the life of the system, including system operation and change \cite{kelly2001systematic}
    \item encourage developers and operators to think about and understand why their system is safe (and when it isn't)
\end{itemize}

The paper describes the method and illustrates its application through the use of an example.


