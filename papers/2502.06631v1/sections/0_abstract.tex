\begin{abstract}
Human-In-The-Loop (HITL) frameworks are integral to many real-world computer vision systems, enabling human operators to make informed decisions with AI assistance. Conformal Predictions (CP), which provide label sets with rigorous guarantees on ground truth inclusion probabilities, have recently gained traction as a valuable tool in HITL settings. One key application area is video surveillance, closely associated with Human Action Recognition (HAR). This study explores the application of CP on top of state-of-the-art HAR methods that utilize extensively pre-trained Vision-Language Models (VLMs). Our findings reveal that CP can significantly reduce the average number of candidate classes without modifying the underlying VLM. However, these reductions often result in distributions with long tails. To address this, we introduce a method based on tuning the temperature parameter of the VLMs to minimize these tails without requiring additional calibration data. Our code is made available on GitHub at the address https://github.com/tbary/CP4VLM.

\begin{IEEEkeywords}
Conformal predictions, temperature tuning, vision-language models, human action recognition.
\end{IEEEkeywords}

%Many real-world computer vision systems rely on Human-In-The-Loop (HITL) frameworks, where a human operator is assisted by AI for final decisions. Recently, Conformal Predictions (CP), which produce label sets with theoretical guarantees on the ground truth's inclusion probability, have garnered significant attention and proven beneficial in HITL schemes. A notable application is video surveillance, closely linked to Human Action Recognition (HAR). This work investigates the use of CP atop state-of-the-art HAR methods, which rely on extensively pre-trained Vision-Language Models (VLMs). We demonstrate that CP greatly reduces the average number of candidate classes without requiring the adaptation of the underlying VLM, although the distribution of these numbers is typically long-tailed. Following, we propose a straightforward approach to significantly reduce these tails without requiring extra calibration data. Code will be made available after the double blind peer review process.
\end{abstract}
