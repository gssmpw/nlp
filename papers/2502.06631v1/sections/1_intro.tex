\section{Introduction}
Modern Computer Vision (CV) systems offer high performances over a wide variety of tasks, even surpassing human expertise in some cases. However, many applications still rely on Human-In-The-Loop (HITL) frameworks, either to enhance the performance of the underlying CV approach or due to the critical nature of the application, which requires the final decision to be made by a human.
The field of video analysis is no stranger to this dynamic, with HITL frameworks used for video segmentation \cite{oh2019fast}, a notably challenging task, as well as vehicle identification for autonomous driving \cite{li2023human} or Human Action Recognition (HAR) in the context of video surveillance~\cite{stonebraker2020surveillance}. 
%HAR is a prevalent topic in computer vision, with applications including video retrieval, autonomous driving and video surveillance \cite{kong2022human}. 

In this context, Conformal Predictions (CP) have garnered significant interest. CP frameworks provide a reduced label set with a robust guarantee on ground truth coverage from the uncertainty estimates of an underlying model, provided a calibration set. This framework has proven beneficial when used in conjunction with human annotators, particularly in object classification tasks \cite{straitouridesigning, cresswell2024conformal}. This makes CP especially useful in critical applications or when a model's performance on a specific task is suboptimal, complementing existing HITL frameworks for tasks such as HAR. 

Historically, the HAR problem was approached with statistical methods relying on features carefully crafted by experts~\cite{dollar2005behavior}. Later on, the advent large-scale datasets as well as high performing deep-learning frameworks lead researcher to use such networks trained in a supervised manner  \cite{wang2016temporal, feichtenhofer2019slowfast, jiang2019stm, arnab2021vivit}. Although the latter showed a significant leap in accuracy, they are largely unable to handle novel classes at test-time. Current state-of-the art approaches \cite{wang2021actionclip} circumvent this problem by relying on extensively pre-trained Vision Language Models (VLMs), which can use textual descriptions of the classes to generate ad-hoc classifiers with strong \textit{zero-shot} performance~\cite{radford_learning_2021}. Recently, foundation models including VLMs have been shown to be strong conformal predictors on general image classification benchmarks \cite{fillioux2024foundation}.

%Additionallye, conformal predictions (CP) \cite{vovk2005algorithmic, sadinle2019least} have generated significant interest, primarily due to their robust theoretical guarantees. They were shown to be beneficial when used in conjunction with human annotators. It \cite{straitouridesigning}, in particular in the context of object classification \cite{cresswell2024conformal}. CP are especially helpful in the case of critical applications, or when the underlying predictor does not perform well on the current task.  

In this work, we explore the effectiveness of using CPs on top of off-the-shelf VLMs for HAR classification tasks, without any additional fine-tuning. Our results demonstrate that CPs can significantly reduce the number of possible classes for a given video clip, even with high coverage guarantees. We also find that the sizes of the resulting conformal sets typically follow a long-tailed distribution (represented in Figure~\ref{fig:intro_set_sizes}, top). Since human annotation time increases with the number of options available for selection \cite{straitouridesigning, hick1952rate, landauer1985selection}, strategies to shorten this tail and reduce conformal set sizes are valuable, particularly in applications where decision time is constrained, such as live video monitoring. To address this, we highlight the importance of tuning the temperature parameter of the VLM to control the distribution of conformal set sizes. This adjustment can be made using only the calibration set, ensuring no additional data cost for conformal predictor calibration, while preserving the theoretical guarantees of the CP framework.

\begin{figure*}[t]
\begin{center}
\includegraphics[width=1\textwidth]{figures/preliminary_figures/fused_intro_ViT_B_16_density.pdf}
\end{center}
\caption{We show the distribution of conformal sets sizes on three datasets (Kinetics400 with 400 classes, UCF101 with 101 classes and HMDB51 with 51 classes), averaged over 40 folds. The first row shows the distribution of sizes with the baseline, non-tuned temperature parameter. Although the number of classes can be greatly reduced, the distributions are typically long tailed. Comparatively, the second row depicts the results obtained with our temperature tuning strategy. For both rows, the 10\% highest set sizes are shown in the solid color area.}
\label{fig:intro_set_sizes}
\end{figure*}

We summarize our contributions as follow:
\begin{itemize}
\item We demonstrate the significant benefits of applying a conformal predictor on off-the-shelf VLMs for human action recognition tasks.
\item We propose an efficient, actionable approach to control the long-tail distribution of conformal set sizes. This method, specifically designed for VLMs, preserves the theoretical guarantees of the conformal predictor.
\end{itemize}
Our code is available at \href{https://github.com/tbary/CP4VLM}{https://github.com/tbary/CP4VLM}.
