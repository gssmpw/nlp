\section{Conclusion and Future Works}
In this paper, we demonstrated how integrating Conformal Predictors with off-the-shelf Vision-Language Models (VLMs) can significantly reduce the number of possible classes in Human Action Recognition tasks while maintaining coverage guarantees. Our findings highlight the influence of the temperature parameter $\tau$ on the distribution of conformal set sizes, which mediates a trade-off between low mean, long-tailed distributions and higher mean, shorter-tailed ones.

The relationship between annotation time and the number of classes remains an open question. Classical studies~\cite{hick1952rate, landauer1985selection} suggest a logarithmic dependency on decision time, while alternative models propose a sigmoid relationship with task uncertainty~\cite{pavao2016sequence}. Future research could develop models linking annotation speed and accuracy to conformal set sizes, offering practical insights for optimizing workflows.

Finally, we note that Conformal Prediction requires calibration sets, and using these sets to adapt VLMs to specific tasks can violate the exchangeability assumption, weakening CP’s theoretical guarantees. However, recent advancements in few-shot VLM adaptation~\cite{zhang2021tip, zhou2022learning, zanella2024test, wanghard} provide promising avenues for preserving these guarantees. Future work could focus on strategies to leverage calibration sets effectively while ensuring theoretical integrity.

\section{Acknowledgements}

T.~Bary and C.~Fuchs are funded by the MedReSyst project, supported by FEDER and the Walloon Region. Part of the computational resources have been provided by the Consortium des Équipements de Calcul Intensif (CÉCI), funded by the Fonds de la Recherche Scientifique de Belgique (F.R.S.-FNRS) under Grant No. 2.5020.11 and by the Walloon Region.