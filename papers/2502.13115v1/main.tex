
\documentclass{article} 
\usepackage[margin=1in]{geometry}
\usepackage[utf8]{inputenc} %
\usepackage[T1]{fontenc}    %
\usepackage{lmodern}


\usepackage{etoolbox}
\newcommand{\arxiv}[1]{\iftoggle{colt}{}{#1}}
\newcommand{\colt}[1]{\iftoggle{colt}{#1}{}}
\newtoggle{colt}
\global\togglefalse{colt}

%
% --- inline annotations
%
\newcommand{\red}[1]{{\color{red}#1}}
\newcommand{\todo}[1]{{\color{red}#1}}
\newcommand{\TODO}[1]{\textbf{\color{red}[TODO: #1]}}
% --- disable by uncommenting  
% \renewcommand{\TODO}[1]{}
% \renewcommand{\todo}[1]{#1}



\newcommand{\VLM}{LVLM\xspace} 
\newcommand{\ours}{PeKit\xspace}
\newcommand{\yollava}{Yo’LLaVA\xspace}

\newcommand{\thisismy}{This-Is-My-Img\xspace}
\newcommand{\myparagraph}[1]{\noindent\textbf{#1}}
\newcommand{\vdoro}[1]{{\color[rgb]{0.4, 0.18, 0.78} {[V] #1}}}
% --- disable by uncommenting  
% \renewcommand{\TODO}[1]{}
% \renewcommand{\todo}[1]{#1}
\usepackage{slashbox}
% Vectors
\newcommand{\bB}{\mathcal{B}}
\newcommand{\bw}{\mathbf{w}}
\newcommand{\bs}{\mathbf{s}}
\newcommand{\bo}{\mathbf{o}}
\newcommand{\bn}{\mathbf{n}}
\newcommand{\bc}{\mathbf{c}}
\newcommand{\bp}{\mathbf{p}}
\newcommand{\bS}{\mathbf{S}}
\newcommand{\bk}{\mathbf{k}}
\newcommand{\bmu}{\boldsymbol{\mu}}
\newcommand{\bx}{\mathbf{x}}
\newcommand{\bg}{\mathbf{g}}
\newcommand{\be}{\mathbf{e}}
\newcommand{\bX}{\mathbf{X}}
\newcommand{\by}{\mathbf{y}}
\newcommand{\bv}{\mathbf{v}}
\newcommand{\bz}{\mathbf{z}}
\newcommand{\bq}{\mathbf{q}}
\newcommand{\bff}{\mathbf{f}}
\newcommand{\bu}{\mathbf{u}}
\newcommand{\bh}{\mathbf{h}}
\newcommand{\bb}{\mathbf{b}}

\newcommand{\rone}{\textcolor{green}{R1}}
\newcommand{\rtwo}{\textcolor{orange}{R2}}
\newcommand{\rthree}{\textcolor{red}{R3}}
\usepackage{amsmath}
%\usepackage{arydshln}
\DeclareMathOperator{\similarity}{sim}
\DeclareMathOperator{\AvgPool}{AvgPool}

\newcommand{\argmax}{\mathop{\mathrm{argmax}}}     



%
\setlength\unitlength{1mm}
\newcommand{\twodots}{\mathinner {\ldotp \ldotp}}
% bb font symbols
\newcommand{\Rho}{\mathrm{P}}
\newcommand{\Tau}{\mathrm{T}}

\newfont{\bbb}{msbm10 scaled 700}
\newcommand{\CCC}{\mbox{\bbb C}}

\newfont{\bb}{msbm10 scaled 1100}
\newcommand{\CC}{\mbox{\bb C}}
\newcommand{\PP}{\mbox{\bb P}}
\newcommand{\RR}{\mbox{\bb R}}
\newcommand{\QQ}{\mbox{\bb Q}}
\newcommand{\ZZ}{\mbox{\bb Z}}
\newcommand{\FF}{\mbox{\bb F}}
\newcommand{\GG}{\mbox{\bb G}}
\newcommand{\EE}{\mbox{\bb E}}
\newcommand{\NN}{\mbox{\bb N}}
\newcommand{\KK}{\mbox{\bb K}}
\newcommand{\HH}{\mbox{\bb H}}
\newcommand{\SSS}{\mbox{\bb S}}
\newcommand{\UU}{\mbox{\bb U}}
\newcommand{\VV}{\mbox{\bb V}}


\newcommand{\yy}{\mathbbm{y}}
\newcommand{\xx}{\mathbbm{x}}
\newcommand{\zz}{\mathbbm{z}}
\newcommand{\sss}{\mathbbm{s}}
\newcommand{\rr}{\mathbbm{r}}
\newcommand{\pp}{\mathbbm{p}}
\newcommand{\qq}{\mathbbm{q}}
\newcommand{\ww}{\mathbbm{w}}
\newcommand{\hh}{\mathbbm{h}}
\newcommand{\vvv}{\mathbbm{v}}

% Vectors

\newcommand{\av}{{\bf a}}
\newcommand{\bv}{{\bf b}}
\newcommand{\cv}{{\bf c}}
\newcommand{\dv}{{\bf d}}
\newcommand{\ev}{{\bf e}}
\newcommand{\fv}{{\bf f}}
\newcommand{\gv}{{\bf g}}
\newcommand{\hv}{{\bf h}}
\newcommand{\iv}{{\bf i}}
\newcommand{\jv}{{\bf j}}
\newcommand{\kv}{{\bf k}}
\newcommand{\lv}{{\bf l}}
\newcommand{\mv}{{\bf m}}
\newcommand{\nv}{{\bf n}}
\newcommand{\ov}{{\bf o}}
\newcommand{\pv}{{\bf p}}
\newcommand{\qv}{{\bf q}}
\newcommand{\rv}{{\bf r}}
\newcommand{\sv}{{\bf s}}
\newcommand{\tv}{{\bf t}}
\newcommand{\uv}{{\bf u}}
\newcommand{\wv}{{\bf w}}
\newcommand{\vv}{{\bf v}}
\newcommand{\xv}{{\bf x}}
\newcommand{\yv}{{\bf y}}
\newcommand{\zv}{{\bf z}}
\newcommand{\zerov}{{\bf 0}}
\newcommand{\onev}{{\bf 1}}

% Matrices

\newcommand{\Am}{{\bf A}}
\newcommand{\Bm}{{\bf B}}
\newcommand{\Cm}{{\bf C}}
\newcommand{\Dm}{{\bf D}}
\newcommand{\Em}{{\bf E}}
\newcommand{\Fm}{{\bf F}}
\newcommand{\Gm}{{\bf G}}
\newcommand{\Hm}{{\bf H}}
\newcommand{\Id}{{\bf I}}
\newcommand{\Jm}{{\bf J}}
\newcommand{\Km}{{\bf K}}
\newcommand{\Lm}{{\bf L}}
\newcommand{\Mm}{{\bf M}}
\newcommand{\Nm}{{\bf N}}
\newcommand{\Om}{{\bf O}}
\newcommand{\Pm}{{\bf P}}
\newcommand{\Qm}{{\bf Q}}
\newcommand{\Rm}{{\bf R}}
\newcommand{\Sm}{{\bf S}}
\newcommand{\Tm}{{\bf T}}
\newcommand{\Um}{{\bf U}}
\newcommand{\Wm}{{\bf W}}
\newcommand{\Vm}{{\bf V}}
\newcommand{\Xm}{{\bf X}}
\newcommand{\Ym}{{\bf Y}}
\newcommand{\Zm}{{\bf Z}}

% Calligraphic

\newcommand{\Ac}{{\cal A}}
\newcommand{\Bc}{{\cal B}}
\newcommand{\Cc}{{\cal C}}
\newcommand{\Dc}{{\cal D}}
\newcommand{\Ec}{{\cal E}}
\newcommand{\Fc}{{\cal F}}
\newcommand{\Gc}{{\cal G}}
\newcommand{\Hc}{{\cal H}}
\newcommand{\Ic}{{\cal I}}
\newcommand{\Jc}{{\cal J}}
\newcommand{\Kc}{{\cal K}}
\newcommand{\Lc}{{\cal L}}
\newcommand{\Mc}{{\cal M}}
\newcommand{\Nc}{{\cal N}}
\newcommand{\nc}{{\cal n}}
\newcommand{\Oc}{{\cal O}}
\newcommand{\Pc}{{\cal P}}
\newcommand{\Qc}{{\cal Q}}
\newcommand{\Rc}{{\cal R}}
\newcommand{\Sc}{{\cal S}}
\newcommand{\Tc}{{\cal T}}
\newcommand{\Uc}{{\cal U}}
\newcommand{\Wc}{{\cal W}}
\newcommand{\Vc}{{\cal V}}
\newcommand{\Xc}{{\cal X}}
\newcommand{\Yc}{{\cal Y}}
\newcommand{\Zc}{{\cal Z}}

% Bold greek letters

\newcommand{\alphav}{\hbox{\boldmath$\alpha$}}
\newcommand{\betav}{\hbox{\boldmath$\beta$}}
\newcommand{\gammav}{\hbox{\boldmath$\gamma$}}
\newcommand{\deltav}{\hbox{\boldmath$\delta$}}
\newcommand{\etav}{\hbox{\boldmath$\eta$}}
\newcommand{\lambdav}{\hbox{\boldmath$\lambda$}}
\newcommand{\epsilonv}{\hbox{\boldmath$\epsilon$}}
\newcommand{\nuv}{\hbox{\boldmath$\nu$}}
\newcommand{\muv}{\hbox{\boldmath$\mu$}}
\newcommand{\zetav}{\hbox{\boldmath$\zeta$}}
\newcommand{\phiv}{\hbox{\boldmath$\phi$}}
\newcommand{\psiv}{\hbox{\boldmath$\psi$}}
\newcommand{\thetav}{\hbox{\boldmath$\theta$}}
\newcommand{\tauv}{\hbox{\boldmath$\tau$}}
\newcommand{\omegav}{\hbox{\boldmath$\omega$}}
\newcommand{\xiv}{\hbox{\boldmath$\xi$}}
\newcommand{\sigmav}{\hbox{\boldmath$\sigma$}}
\newcommand{\piv}{\hbox{\boldmath$\pi$}}
\newcommand{\rhov}{\hbox{\boldmath$\rho$}}
\newcommand{\upsilonv}{\hbox{\boldmath$\upsilon$}}

\newcommand{\Gammam}{\hbox{\boldmath$\Gamma$}}
\newcommand{\Lambdam}{\hbox{\boldmath$\Lambda$}}
\newcommand{\Deltam}{\hbox{\boldmath$\Delta$}}
\newcommand{\Sigmam}{\hbox{\boldmath$\Sigma$}}
\newcommand{\Phim}{\hbox{\boldmath$\Phi$}}
\newcommand{\Pim}{\hbox{\boldmath$\Pi$}}
\newcommand{\Psim}{\hbox{\boldmath$\Psi$}}
\newcommand{\Thetam}{\hbox{\boldmath$\Theta$}}
\newcommand{\Omegam}{\hbox{\boldmath$\Omega$}}
\newcommand{\Xim}{\hbox{\boldmath$\Xi$}}


% Sans Serif small case

\newcommand{\Gsf}{{\sf G}}

\newcommand{\asf}{{\sf a}}
\newcommand{\bsf}{{\sf b}}
\newcommand{\csf}{{\sf c}}
\newcommand{\dsf}{{\sf d}}
\newcommand{\esf}{{\sf e}}
\newcommand{\fsf}{{\sf f}}
\newcommand{\gsf}{{\sf g}}
\newcommand{\hsf}{{\sf h}}
\newcommand{\isf}{{\sf i}}
\newcommand{\jsf}{{\sf j}}
\newcommand{\ksf}{{\sf k}}
\newcommand{\lsf}{{\sf l}}
\newcommand{\msf}{{\sf m}}
\newcommand{\nsf}{{\sf n}}
\newcommand{\osf}{{\sf o}}
\newcommand{\psf}{{\sf p}}
\newcommand{\qsf}{{\sf q}}
\newcommand{\rsf}{{\sf r}}
\newcommand{\ssf}{{\sf s}}
\newcommand{\tsf}{{\sf t}}
\newcommand{\usf}{{\sf u}}
\newcommand{\wsf}{{\sf w}}
\newcommand{\vsf}{{\sf v}}
\newcommand{\xsf}{{\sf x}}
\newcommand{\ysf}{{\sf y}}
\newcommand{\zsf}{{\sf z}}


% mixed symbols

\newcommand{\sinc}{{\hbox{sinc}}}
\newcommand{\diag}{{\hbox{diag}}}
\renewcommand{\det}{{\hbox{det}}}
\newcommand{\trace}{{\hbox{tr}}}
\newcommand{\sign}{{\hbox{sign}}}
\renewcommand{\arg}{{\hbox{arg}}}
\newcommand{\var}{{\hbox{var}}}
\newcommand{\cov}{{\hbox{cov}}}
\newcommand{\Ei}{{\rm E}_{\rm i}}
\renewcommand{\Re}{{\rm Re}}
\renewcommand{\Im}{{\rm Im}}
\newcommand{\eqdef}{\stackrel{\Delta}{=}}
\newcommand{\defines}{{\,\,\stackrel{\scriptscriptstyle \bigtriangleup}{=}\,\,}}
\newcommand{\<}{\left\langle}
\renewcommand{\>}{\right\rangle}
\newcommand{\herm}{{\sf H}}
\newcommand{\trasp}{{\sf T}}
\newcommand{\transp}{{\sf T}}
\renewcommand{\vec}{{\rm vec}}
\newcommand{\Psf}{{\sf P}}
\newcommand{\SINR}{{\sf SINR}}
\newcommand{\SNR}{{\sf SNR}}
\newcommand{\MMSE}{{\sf MMSE}}
\newcommand{\REF}{{\RED [REF]}}

% Markov chain
\usepackage{stmaryrd} % for \mkv 
\newcommand{\mkv}{-\!\!\!\!\minuso\!\!\!\!-}

% Colors

\newcommand{\RED}{\color[rgb]{1.00,0.10,0.10}}
\newcommand{\BLUE}{\color[rgb]{0,0,0.90}}
\newcommand{\GREEN}{\color[rgb]{0,0.80,0.20}}

%%%%%%%%%%%%%%%%%%%%%%%%%%%%%%%%%%%%%%%%%%
\usepackage{hyperref}
\hypersetup{
    bookmarks=true,         % show bookmarks bar?
    unicode=false,          % non-Latin characters in AcrobatÕs bookmarks
    pdftoolbar=true,        % show AcrobatÕs toolbar?
    pdfmenubar=true,        % show AcrobatÕs menu?
    pdffitwindow=false,     % window fit to page when opened
    pdfstartview={FitH},    % fits the width of the page to the window
%    pdftitle={My title},    % title
%    pdfauthor={Author},     % author
%    pdfsubject={Subject},   % subject of the document
%    pdfcreator={Creator},   % creator of the document
%    pdfproducer={Producer}, % producer of the document
%    pdfkeywords={keyword1} {key2} {key3}, % list of keywords
    pdfnewwindow=true,      % links in new window
    colorlinks=true,       % false: boxed links; true: colored links
    linkcolor=red,          % color of internal links (change box color with linkbordercolor)
    citecolor=green,        % color of links to bibliography
    filecolor=blue,      % color of file links
    urlcolor=blue           % color of external links
}
%%%%%%%%%%%%%%%%%%%%%%%%%%%%%%%%%%%%%%%%%%%



\arxiv{
\floatplacement{algorithm}{t}
}

\usepackage[suppress]{color-edits}
 \addauthor{sr}{red}
 \addauthor{cf}{orange}




\title{Near-Optimal Private Learning in Linear Contextual Bandits}

\colt{
\coltauthor{\Name{Author Name1} \Email{abc@sample.com}\and
 \Name{Author Name2} \Email{xyz@sample.com}\\
 \addr Address}
}

\arxiv{
\author{Fan Chen\\{\small \texttt{fanchen@mit.edu}} \and  Jiachun Li\\{\small \texttt{jiach334@mit.edu}} \and Alexander Rakhlin\\{\small \texttt{rakhlin@mit.edu}} \and David Simchi-Levi\\{\small \texttt{dslevi@mit.edu}} }
}


\begin{document}

\maketitle

\begin{abstract}
We analyze the problem of private learning in generalized linear contextual bandits. Our approach is based on a novel method of \emph{re-weighted} regression, yielding an efficient algorithm with regret of $\tbO{d^2\sqrt{T}+\frac{d^{5/2}}{\alpha}}$ and $\tbO{\sqrt{d^5T}/\alpha}$ in the joint and local model of privacy, respectively. 
\cfedit{Further, we provide near-optimal private procedures that achieve dimension-independent rates in private linear models and linear contextual bandits. In particular, our results imply that joint privacy is almost ``for free'' in all the settings we consider, partially addressing the open problem posed by \citet{azize2024open}.}
\end{abstract}


\section{Introduction}\label{sec:intro}


Contextual bandits provide a natural framework for interactive decision making, applicable to numerous real-world domains. In this setting, the decision maker (or, the algorithm) sequentially observes a context, selects an action, and receives a reward ~\citep{abbasi2011improved, auer2002finite, simchi2020bypassing, foster2020beyond}. The central challenge is to balance exploration (learning the reward structure) and exploitation (maximizing the cumulative rewards). Further, in many applications, there are additional privacy concerns, as the contexts often involve sensitive personal information---such as past purchase histories, credit scores, or physcial data---information not meant for public disclosure ~\citep{dwork2014algorithmic, lei2024privacy, chen2022privacy}. Despite extensive research on interactive decision making, it is not yet well-understood how to achieve the optimal privacy-utility trade-offs even in the fundamental setting of contextual bandits---arguably one of the simplest and most commonly considered models of online learning ~\citep{shariff2018differentially, azize2024open}.



Formally, in the setting of contextual bandits, the learner observes a context $x_t\in\cX$ at each step $t\in[T]$, drawn stochastically as $x_t\sim P$. Based on the context $x_t$ (and the history up to step $t$), the learner selects an action $a_t\in\cA$ and observes a reward $r_t\in[-1,1]$ with expected value $\EE[r_t|x_t,a_t]=\fs(x_t,a_t)$. Here, $\fs:\cX\times\cA\to[-1,1]$ is the underlying mean reward function. The performance of the learner is typically measured by its \emph{regret}, defined as
\begin{align*}
    \Reg\defeq \EE\brac{ \sum_{t=1}^T \max_{\as_t\in\cA} \fs(x_t,\as_t) - \fs(x_t,a_t) },
\end{align*}
which measures the gap between the learner's cumulative rewards and that of an optimal policy with the full knowledge of $\fs$.
In generalized linear contextual bandits, a widely studied model, the ground-truth $\fs$ is assumed to take the form
\begin{align*}
    \fs(x,a)=\nu(\lr \phxa, \ths \rr), \qquad \forall (x,a)\in\cX\times\cA,
\end{align*}
where $\nu:[-1,1]\to[-1,1]$ is a known link function, $\phi:\cX\times\cA\to\Bone$ is a known feature map, and $\ths\in\Bone$ is the unknown underlying parameter. 

Even in linear contextual bandits (where $\nu(t)=t$ is the identity function and $\fs$ is linear), there is limited understanding of how to design privacy-preserving procedures that attain optimal regret.
Under the \emph{joint differential privacy} (JDP) model~\citep{dwork2006calibrating} with privacy parameter $\alpha$, the only known regret upper bound---due to \citet{shariff2018differentially}---scales as $\sqrt{T/\alpha}$. This regret scaling is particularly undesirable in the high-privacy regime (i.e., $\alpha \ll 1$), and \citet{azize2024open} have posed the question of identifying the rate-optimal regret in this setting as an open problem.



\begin{table}
\centering
\renewcommand{\arraystretch}{1.3}
\begin{tabular}{|c|c|c|c|c|}
\hline
Setting & Results & Regret bound & Gen & Adv  \\
\hline
\multirow{4}{*}{Joint DP} & \citet{shariff2018differentially} & $d\sqrt{T}+\frac{d^{3/4}\sqrt{T}}{\sqrt{\alpha}}$ & \yes & \yes \\\cline{2-5}
                         & \cref{thm:regret-upper-JDP} & $\sqrt{d^5T}+\frac{d^{5/2}}{\alpha}$ & \yes & \no \\\cline{2-5}
                         & \cref{thm:regret-upper-JDP-better} ($|\cA|=\bigO{1}$) & $\sqrt{dT}+\frac{d^{3/2}}{\alpha}$ & \no & \no \\\cline{2-5}
                         & Lower bound~\citep{he2022reduction} & $\sqrt{dT\log|\cA|}+\frac{d}{\alpha}$ & / & / \\
\hline
\multirow{6}{*}{Local DP} & \citet{zheng2020locally} & $(dT)^{3/4}/\alpha$ & \yes & \yes \\\cline{2-5}
                         & \citet{han2021generalized}$^\dagger$ & $\sqrt{d\log|\cA|\cdot T}/(\lmins \alpha)$ & \yes & \yes \\\cline{2-5}
                         & \citet{li2024optimal} & $|\cA|^2\log^d(T)\cdot \sqrt{T}/\alpha$ & \no & \no \\\cline{2-5}
                         & \citet{chen2024private} & $ \sqrt{d^3T}/\alpha$ & \yes & \yes \\\cline{2-5}
                         & \cref{thm:regret-upper-LDP} & $\sqrt{d^5T}/\alpha$ & \yes & \no \\\cline{2-5}
                         & Lower bound~\citep{chen2024private} & $\sqrt{d^2T}/\alpha$ & / & / \\
\hline
\end{tabular}
\caption{Summary of the existing results for private learning in (generalized) linear contextual bandits. 
The ``Gen'' column indicates whether similar regret bounds can be obtained under generalized linear contextual bandits.
The ``Adv'' column indicates whether the contexts are allowed to be adversarial (as opposed to stochastic contexts considered in the present paper).
$\dagger$The assumptions of \citet{han2021generalized} are slightly stronger than $\lmins>0$, and their regret bound is always lower bounded by $\sqrt{d\log|\cA|\cdot T}/(\lmins \alpha)$ stated in the table. Note that $\lmins\leq \frac{1}{d}$ always holds.
}
\label{tab:comp}
\end{table}


In the \emph{local differential privacy} (LDP) model~\citep{duchi2013local}, the best-known regret bound until recently was $\sqrt{T}/(\alpha\lmins)$~\citep{han2021generalized}, where \colt{$\lmins\defeq \min_{\pi}\lmin(\EE^{\pi}\phxa\phxa\tp)$}\arxiv{
\begin{align*}
    \lmins\defeq \min_{\pi}\lmin(\EE^{\pi}\phxa\phxa\tp)
\end{align*}
} is the minimum eigenvalue of the covariance matrix over \emph{all} linear policies.\footnote{A policy $\pi$ is \emph{linear} if there exists $\theta\in\R^d$ so that $\pi(x)\in\argmax_{x\in\cX} \lr \theta, \phi(x,a)\rr$. For generalized linear contextual bandits, it is clear that the optimal policy must be linear.}
Assuming $\lmins$ being lower bounded implies that the learner can effectively estimate the ground-truth parameter $\ths$ while executing greedy policies, and hence 
it essentially removes the difficulty of \emph{exploration}. 
\cfedit{Such a condition is typically called ``explorability'' (``diversity'') because it assumes the actions taken by any greedy policy are diverse enough. The quantity $1/\lmins$ can be prohibitively large for most scenarios of interest, e.g., when there exists a direction $v\in\R^d$ that cannot be sufficiently explored, i.e., $\phxa \perp v$ for each $a\in\cA$ with high probability over $x\sim P$.}
However, it is known that for algorithms based on squared loss regression---such as variants of the LinUCB algorithm~\citep{abbasi2011improved}---to achieve the optiaml rates, a dependence on $1/\lmins$ can be unavoidable (cf. \cref{sec:negative}). Therefore, algorithmic innovations are necessary to attain the optimal regret without assuming such an explorability condition.

Towards the optimal regret under LDP, the recent work of %
\citet{li2024optimal} develops an alternative approach based on~\cfedit{\emph{regression with confidence interval ($\Lone$-regression)}.} %
However, as their algorithm involves iteratively performing PCA, the final regret bound scales with $\log^d(T)\sqrt{T}$, i.e., the dependence on $d$ is \emph{exponential}. 
On the other hand, \citet{chen2024private} prove a regret bound of $\sqrt{d^3T}/\alpha$ by controlling the Decision-Estimation Coefficient (DEC)~\citep{foster2021statistical,foster2023tight}, achieved by the Exploration-by-Optimization algorithm~\citep{lattimore2020exploration,foster2022complexity} that operates in at least \emph{exponential-time}.


Building upon these recent advances, we propose a new method called private \emph{re-weighted regression}, which first privately learns a \emph{normalization matrix} $U$ from data, and then performs regression on the loss function, re-weighted according to the matrix $U$. In the generalized linear model, the re-weighted regression provides a near-optimal convergence rate in the \Lone-error. Further, using this method as a subroutine, we propose a computationally efficient algorithm that achieves regret bounds of $\tbO{d^2\sqrt{T}+\frac{d^{5/2}}{\alpha}}$ and $\tbO{\sqrt{d^5T}/\alpha}$ in the joint and local privacy models, respectively. 
In particular, our results \cfedit{imply that the joint privacy is almost ``for free'' in generalized linear contextual bandits, partially resolving} the open problem of \citet{azize2024open}.

\cfedit{Furthermore, under the setting where the dimension $d$ is prohibitively large or even unbounded, we develop private estimation procedures that achieve \emph{dimension-independent} rates in linear models. As application, we provide nearly minimax-optimal dimension-free regret bounds in private linear contextual bandits.}

\paragraph{Existing results}
In \cref{tab:comp}, we summarize the known regret bounds for learning in (generalized) linear contextual bandits under the joint or local model of privacy. For simplicity of presentation, all results are specialized to linear contextual bandits with stochastic contexts, and we omit poly-logarithmic factors. 

\paragraph{Organization}
In \cref{sec:negative}, we discuss the main difficulty of achieving optimal rates while preserving privacy in linear contextual bandits, motivating the approach of $\Lone$-regression. We then present our re-weighted regression method (\cref{sec:l1-reg}) for private generalized linear models. \cfedit{In \cref{sec:cb}, we apply our method to provide private regret guarantees in generalized linear contextual bandits. Further, in \cref{sec:unbounded}, we investigate \emph{dimension-free} private learning in linear models and linear contextual bandits.}
For succinctness, we mostly focus on the JDP setting in the main body of the paper and present the algorithms for LDP setting in the appendices.









\section{Preliminaries}\label{sec:prelim}




Differential privacy (DP) is a widely adopted framework for ensuring privacy in data analysis. In this section, we will introduce the definition of joint DP and local DP algorithms for both offline and interactive settings.  We first review the notion of differentially private (DP) channels.
\newcommand{\pDP}{$\alpha$-DP}
\newcommand{\aDP}{$(\alpha,\beta)$-DP}
\begin{definition}[DP channel]
For the latent observation space $\cZ$ and the noisy observation space $\cO$, a channel $\pr$ is a measurable map  $\cZ\to\DO$. %
A channel $\pr$ is \aDP~if for $z, z'\in\cZ$, any measurable set $E\subseteq \cO$,
\begin{align*}
    \pr(E|z)\leq e^\alpha\pr(E|z')+\beta.
\end{align*}
\cfedit{In this paper, we focus on the regime $\alpha,\beta\in(0,1)$. }
\end{definition}

The Gaussian channel is a standard example of the \aDP~channels (see e.g., \citep{balle2018improving}).
\begin{definition}[Gaussian channel]\label{def:Guassian-channel}
Suppose that $\alpha,\beta\in(0,1)$, and denote $\siga=\frac{2\sqrt{\log(1.25/\beta)}}{\alpha}$. For any given function $F:\cZ\to\R^d$, we let $\Delta(F)\defeq \sup_{z,z'} \nrm{F(z)-F(z')}_2$. Then, for any $\Delta\geq \Delta(F)$, the channel $\pr(\cdot|z)=\normal{ F(z), \siga^2 \Delta^2 }$ is a \aDP~channels.
\end{definition}

In the following, we denote by $\priv[\Delta]{v}=\normal{v, 4\siga^2\Delta^2}$ the Gaussian channel with \emph{sensitivity} $\Delta$. It is guaranteed that for any function $F$ with $\nrm{F(z)}\leq \Delta$, the channel $z\mapsto \priv{F(z)}$ is \aDP. Further, for any symmetric matrix $V\in\Rdd$, we denote $\sympriv{V}$ to be the distribution of $V+Z$, where $Z$ is a symmetric Gaussian random matrix, i.e., $Z_{ij}=Z_{ji}\sim \normal{0,4\siga^2\Delta^2}$ independently. 







\paragraph{Joint DP}
We first recall the definition of (joint) DP algorithms for \emph{non-interactive} problems. In this setting, an \emph{algorithm} maps the dataset $\cD=\set{z_1,\cdots,z_T}\in\cZ^T$ to a distribution over the output decision $\Pi$. The two dataset $\cD=(z_1,\cdots,z_T), \cD'=(z_1',\cdots,z_T')\in \cZ^T$ are neighbored if there is at most one index $i$ such that $z_i\neq z_i'$.

\begin{definition}[JDP for non-interactive algorithms]
An algorithm $\alg$ preserves \aJDP~if for any neighbored dataset $\cD, \cD'$ and any measurable set $E\subseteq \DPi$,
\begin{align*}
    \alg(E|\cD)\leq \ea \alg(E|\cD')+\beta.
\end{align*}
\end{definition}

On the other hand, a $T$-round (interactive) algorithm $\alg$ for contextual bandit problem is specified by a sequence of mapping $\set{ \pi_t(\cdot\mid\cdot) }$, where the $t$-th mapping $\pi\ind{t}(\cdot\mid{}\cH\ind{t-1},x_t)$ specifies the distribution of $a_t\in\cA$ based on the history $\cH\ind{t-1}=\set{ z_s=(x_s,a_s,r_s) }_{s\leq t-1}$ and the current context $x_t$. As the algorithm has to ensure the whole sequence $(a_1,\cdots,a_T)$ protects privacy, it's a more challenging task compared to the non-interactive setting. 
In this work, we consider the following notion of joint DP with interaction.

\begin{definition}[Anticipating JDP]\label{def:JDP-interactive}
An algorithm $\alg$ is said to \emph{preserve \aJDP} (or simply \emph{be \aJDP)} 
\cfreplace{
if for every $t\in[T]$ and any two neighbored trajectory $\cH_T, \cH_T'$ differing only at round $t$, conditioning on the trajectory we have
\begin{align*}
    \PP\sups{\alg}\paren{ (a_{t+1},\cdots,a_T)\in E|\cH_T }\leq \ea\PP\sups{\alg}\paren{ (a_{t+1},\cdots,a_T)\in E|\cH_T' }+\beta, \qquad \forall E\subseteq \cA^{T-t},
\end{align*}
where the probability $\PP\sups{\alg}$ only accounts the randomness of $\alg$, i.e., for $\cH_T=\set{(x_t,a_t,r_t)}_{t\in[T]}$,
\begin{align*}
    \PP\sups{\alg}\paren{ a_{t+1},\cdots,a_T|\cH_T }=\prod_{t'=t+1}^T \pi_{t'}(a_{t'}|(x_s,a_s,r_s)_{s<t'},x_{t'}).
\end{align*}
}{
if for every round $t\in[T]$, any two neighbored history $\cH_t, \cH_t'$ differing only at round $t$, and any sequence of possible future observations $\cD_{t+1:T}=\sset{(x_{t'},r_{t'})}_{t'\in[t+1,T]}$, we have
\begin{align*}
    \PP\sups{\alg}\paren{ (a_{t+1},\cdots,a_T)\in E|\cH_t, \cD_{t+1:T} }\leq \ea\PP\sups{\alg}\paren{ (a_{t+1},\cdots,a_T)\in E|\cH_t', \cD_{t+1:T} }+\beta, \qquad \forall E\subseteq \cA^{T-t},
\end{align*}
where the probability $\PP\sups{\alg}$ only accounts the randomness of $\alg$, i.e., for $\cH_t=\set{(x_t,a_t,r_t)}_{t\in[T]}$,
\begin{align*}
    \PP\sups{\alg}\paren{ a_{t+1},\cdots,a_T|\cH_t, \cD_{t+1:T} }=\prod_{t'=t+1}^T \pi_{t'}(a_{t'}|(x_s,a_s,r_s)_{s<t'},x_{t'}).
\end{align*}
}
\end{definition}



\cref{def:JDP-interactive} is the most widely-adopted definition of privacy-preserving procedures in the literature of contextual bandits~\citep{shariff2018differentially, azize2024open} and episodic RL~\citep{vietri2020private}, and it can be interpreted as following.
Assume that a malicious adversary is trying to identify the private information $(x_t,a_t,r_t)$ of the unit which is treated at round $t\in[T]$, and it can adversarially design the context $x_{s}$ and reward $r_s$ after round $t$. An algorithm is private if the adversary cannot infer the private information  $(x_t,a_t,r_t)$ from the output actions $a_{>t}$ of the algorithm no matter what the history $\cH_{t-1}$ is and how the agent designs the input 
$\{x_{>t},r_{>t}\}$. Since the algorithm is \emph{non-anticipating}, %
the history output $a_{<t}$ will also not be impacted by $(x_t,a_t,r_t)$. In the meanwhile, the output $a_t$ will unavoidably contain information about $x_t$, otherwise no non-trivial regret guarantee could be attained, as proved by~\citet{shariff2018differentially}.


\paragraph{Local DP}
Parallel to the above model of DP, a line of work~\citep{kasiviswanathan2011can,duchi2013local,duchi2016local} studies the \emph{local differential privacy} (LDP) model that provides stronger protection of privacy. It is adopted in scenarios where each individual wants to protect their personal privacy and does not fully trust the data collector, so users locally perturb or add noise to their data before sending it, ensuring that even the collecting party cannot learn the exact original information, but only privatized observation $o \in \cO$. 




For a $T$-round algorithm $\alg$ operating on observation space $\cZ$ and decision space $\Pi$, $\alg$ is said to preserve \aLDP~if it adopts the following protocol for each round $t=1,...,T$:
\begin{itemize}
  \setlength{\parskip}{2pt}
    \item $\alg$ selects a decision $\pi_t\in \Pi$ and a \aDP~channel $\pr_t$ based on the history $\cH_{t-1}=\{\pi_1,o_1 ,\cdots, \pi_{t-1}, o_{t-1}\}$.
    \item The environment generates an observation $z_t\in \cZ$ sampled via $z_t\sim \Mstar(\pi_t)$, where $\Mstar$ is the underlying \emph{model} of the environment. %
    \item $\alg$ receives a noisy observation $o_t\sim \pr_t(\cdot|z_t)$.
\end{itemize}
In other words, an algorithm that preserves local DP never has direct access to the observation $z_t\in\cZ$, and only the ``privatized'' observation $o_t\sim \pr_t(\cdot|z)$ is revealed. Therefore, an LDP algorithm has to adaptively select both the decision $\pi_t$ and also the private channel $\pr_t$ to obtain information from the environment. 







\paragraph{Generalized linear models}
In this paper, we also study the generalized linear models, an important sub-problem of the generalized linear contextual bandits. 
\begin{definition}\label{def:GLM}
In generalized linear models (GLM), a covariate space $\cC\subseteq \Bone$ and a link function $\link:[-1,1]\to[-1,1]$ is given, and the latent observation space is $\cZ=\cC\times [-1,1]$. The ground truth model $\Mstar\in\DZ$ is specified as
\begin{align*}
    (\x,y)\sim \Mstar: \quad \x\sim \pph, ~~
    \EE[y|\x]=\link(\lr \x,\ths\rr),
\end{align*}
where $\pph$ is a covariate distribution over $\cC$, and $\ths\in\Bone$ is the ground truth parameter. 
\end{definition}

Canonical examples of GLM include the \emph{linear models}, where $\link(t)=t$, and the \emph{logistic models}, where $\link(t)=\frac{1}{1+e^{-t}}$. 
In this paper, we assume the link function $\link$ is well-conditioned.
\begin{assumption}\label{asmp:link}
There exists constant $0<\mug\leq \Lipg$ such that $\mug\leq \link'(t)\leq \Lipg$ for all $t\in[-1,1]$, and we denote $\kpg\defeq \frac{\Lipg}{\mug}$ to be the condition number of the link function $\link$.
\end{assumption}


\section{Motivation}\label{sec:negative}

We start by reviewing why most existing private algorithms for linear contextual bandits fail to achieve an optimal regret rate without the strong explorability condition $\lmins>0$. Such insufficiency motivates the confidence intervals-based approach of \citet{li2024optimal} discussed in \cref{ssec:L1-motivation}.

\subsection{Insufficiency of standard regression}





In general, the existing algorithmic principles for learning contextual bandits mostly rely (either explicitly or implicitly) on the regression subroutines that, given a sequence of observation $\set{(x_t,a_t,r_t)}_{t\in[N]}$, produce an reward estimation $\fhat$ %
with bounded mean-square error:
\begin{align*}
    \EE_{(x,a)\sim \cD} (\fhat(x,a)-\fs(x,a) )^2\leq \cE(N)^2.
\end{align*}
For non-private linear contextual bandits, it is well-known that regression-based estimators achieve the optimal rate of $\cE(N)^2\asymp \frac{d}{N}$, and such a convergence guarantee of $N^{-1}$-rate is essential in the regret analysis of the classical LinUCB algorithm and its variants ~\citep{abbasi2011improved,li2019nearly,bastani2020online}.
Further, for contextual bandits with a general reward function class,
the recent regression-oracle based algorithms~\citep{simchi2020bypassing, foster2020beyond} achieve regret bounds scaling with $\tbO{T\cdot \cE(T)}$, %
and hence a $T^{-1}$-rate of convergence under $L_2$-error metric is also crucial
to obtain a regret of order $\widetilde{O}(\sqrt{T})$.






\newcommand{\ellg}{\ell_{\link}}
\newcommand{\Errltwo}[1]{\mathsf{Err}_{2}(#1)}





Therefore, for regression-based algorithms, achieving rate-optimal regret essentially relies on the $L_2$-error guarantee of the regression subroutine.
However, it is known that in linear models, privacy leads to slower convergence under the $L_2$-error if the covariate distribution is ill-conditioned, as the following folklore lemma indicates.

\begin{proposition}[Lower bounds for ill-conditioned linear regression]\label{prop:lower-linear-est}
Suppose that $T\geq 1$, $\alpha\in(0,1]$, $\lambda\in[0,1]$, $d=1$, and the link function $\link(t)=t$ is identity. Let the covariate distribution $p\in\Delta([-1,1])$ be known and given by $p(1)=\lambda, p(0)=1-\lambda$. Then

(1) For any $T$-round \JDP~algorithm $\alg$ with output $\hth$, it holds that
\begin{align*}
    \sup_{\ths\in[-1,1]}\EE\sups{\ths,\alg} \brac{ \Epp{\lr \x, \hth-\ths\rr^2} }\geqsim \min\sset{\frac{1}{\lambda (\alpha+\beta)^2T^2},\lambda}.
\end{align*}

(2) For any $T$-round \aLDP~algorithm $\alg$ with output $\hth$, as long as $\beta\leq \frac{\lambda}{T^2}$, it holds that
\begin{align*}
    \sup_{\ths\in[-1,1]}\EE\sups{\ths,\alg} \brac{ \Epp{\lr \x, \hth-\ths\rr^2} } \geqsim \min\sset{\frac{1}{\lambda \alpha^2T},\lambda}.
\end{align*}
\end{proposition}

Note that in the above construction, $\Ex{\x \x\tp}=\lambda$. Hence, in linear contextual bandits, the oracle-based regret bounds described above will scale with $\tbO{\sqrt{T}+\frac{1}{\lmins \alpha}}$ under joint DP model, and $\tbO{\sqrt{T}/(\lmins \alpha)}$ under the local DP model, where $\lmins$ is the minimum eigenvalue over any policy that the algorithm may play. %
Further, if there is not a lower bound on $\lambda>0$, then \cref{prop:lower-linear-est}
provides the worst-case lower bounds of $\Omega\paren{\frac{1}{(\alpha+\beta)T}}$ and $\Omega\paren{\frac{1}{\alpha\sqrt{T}}}$ for $L_2$-regression under the JDP model and LDP model, respectively, implying significant degradation under privacy.

To sum it up, without a new analysis framework, the existing algorithms that only rely on the regression oracles with $L_2$-error guarantee might not avoid the strong explorability condition $\lmins>0$. 
Therefore, to achieve the optimal regret rates for (generalized) linear contextual bandits, we cannot use the standard linear regression primitives to estimate ground-truth parameter $\ths$. 

\subsection{Alternative approach: Regression with confidence intervals}\label{ssec:L1-motivation}


As an alternative to the standard regression based approach~\citep{foster2018practical,foster2020beyond}, \citet{li2024optimal} propose an action elimination framework based on regression with $\Lone$-error guarantee and the additional confidence interval structures. The key observation is that, while the negative results (\cref{prop:lower-linear-est}) do rule out the regression oracles with $O(1/T)$ convergence rate under $L_2$-error, such oracles are \emph{not} necessary for designing algorithm. 
Specifically, \citet{li2024optimal} consider the regression subroutine with confidence intervals (\emph{$\Lone$-regression} for short), which is defined as following:


\begin{definition}[$\Lone$-regression oracle]\label{def:L1-oracle}
Let $N\geq 1$, $\delta\in(0,1)$.
In contextual bandits, a $\Lone$-regression oracle is a $N$-round algorithm $\alg$ that outputs an estimate of reward function $\fhat: \cX\times \cA \to [-1,1]$ and an confidence bound $\CI: \cX\times \cA \to \R_{\geq 0}$, such that for any fixed policy $\pi:\cX\to \Delta(\cA)$, given data $(x_t,a_t,r_t)$ generated independently as
\begin{align*}
\textstyle
    x_t\sim P, \quad
    a_t\sim \pi(\cdot|x_t), \quad
    \EE[r_t|x_t,a_t]=\fs(x_t,a_t), \qquad t\in[N]
\end{align*}
the following holds \whp:

(1) (Valid confidence interval) $\abs{\fhat(x,a)-\fs(x,a)}\leq \CI(x,a)$ for all $(x,a)\in\cX\times\cA$. 

(2) ($\Lone$-performance bound) $\EE_{x\sim P, a\sim \pi(x)} \brac{  \CI(x,a) }\leq \cE_\delta(N)$.
        

\end{definition}

The above conditions on $\Lone$-regression oracle only imply that the \emph{$\Lone$-error} is bounded as $\EE\absn{\fhat(x,a)-\fs(x,a)}\leq \cE(N)$. This is arguably weaker than the mean-square ($L_2$) convergence $\EE(\fhat(x,a)-\fs(x,a))^2\leq \cE(N)^2$ by the common $L_2$-regression. 

With a private $\Lone$-regression oracle, \citet{li2024optimal} adopt an algorithm based on action elimination that achieves a regret of $\tbO{ T \cdot \cE_{1/T}(T) }$ (for details, see also \cref{appdx:regret-meta}). Therefore, the framework opens the door for a $\sqrt{T}$-regret by developing $\Lone$-regression oracle with $\cE_\delta(T)=\tbOn{1/\sqrt{T}}$. 
However, the $\Lone$-regression oracle of \citet{li2024optimal} is based on iterative private PCA and layered private linear regression, achieving the rate $\cE_{\delta}(T) \leq \tbO{
\frac{\log^d(T)}{\alpha\sqrt{T}}}$ and hence leading to a $\log^d(T)\sqrt{T}$-regret that is exponential of the dimension $d$. This regret bound is meaningful only when the dimension $d=\bigO{1}$ is of constant order. 

On the other hand, \citet{chen2024private} provides a significantly improved regret of $\sqrt{d^3T}/\alpha$. While the Decision-Estimation Coefficient (DEC) approach is much different from the aforementioned ones, the way they upper bound the DEC implicitly utilizes the confidence interval bounds with $\Lone$ guarantee. Indeed, for linear regression, \citet{chen2024private} also provide a near-optimal $T^{-1/2}$-rate under $\Lone$-error. For achieving such guarantees, they propose a novel normalization method based on a re-weighting matrix $U$ (detailed discussion in \cref{sec:l1-reg}). However, this method is introduced purely for upper bounding the DEC, and it is unknown whether it provides a more efficient algorithm.

Inspired by the insights from \citet{li2024optimal} and \citet{chen2024private}, in \cref{sec:l1-reg}, we develop an efficient and near-optimal $\Lone$-regression procedure that applies to both joint DP and local DP settings.
Then, in \cref{sec:cb}, we adopt the proposed $L_1$-regression oracle in the action elimination framework of \citet{li2024optimal} to provide rate-optimal private regret bounds.










\section{Private Reweighted Regression}\label{sec:l1-reg}




In this section, we build upon the techniques of \citet{chen2024private} to provide an optimal estimation guarantee under $L_1$-error. 
For linear regression, \citet{chen2024private} provides a near-optimal convergence rate under \Lone-error, based on the DEC framework and the \ExO~algorithm. For each round, the \ExO~algorithm in some sense estimates a ``worst-case'' context distribution $p$, and \citet{chen2024private} utilize such a context distribution to compute a normalization matrix $U$:
\begin{align}\label{def:U}
    \EE_{\x\sim p}\brac{ \frac{U\x\x\tp U}{\nrm{U\x}} }+\lambda U=\id,
\end{align}
where $\lambda>0$ is a regularization parameter. \citet{chen2024private} then bound the DEC based on the normalization $(U,\lambda)$, hence providing the desired convergence rate through the DEC framework.

However, it is unclear whether the normalization matrix $U$ can be useful \emph{algorithmically}.
As the main motivation of our approach, we first discuss in \cref{ssec:U-intuition} how such a normalization matrix $U$, satisfying the following relaxed equation,
\begin{align}\label{def:U-app}
    \frac12\id\preceq \EE_{\x\sim p}\brac{ \frac{U\x\x\tp U}{\nrm{U\x}} }+\lambda U\preceq 2\id,
\end{align}
can be helpful in $L_1$-regression. We then provide a private procedure (\cref{ssec:U-alg}) that learns a normalization $(U,\lambda)$ satisfying \eqref{def:U-app}, and develop our private reweighted regression method with near-optimal convergence guarantee in \cref{ssec:L1-regression-rates}.


\subsection{Key idea: Reweighting based on normalization matrix}\label{ssec:U-intuition}


For generalized linear models with link function $\link$, we can consider the following loss objective,
\begin{align*}%
\textstyle
    \Lgl(\theta)\defeq \Exy{ \ellg(\lr \x,\theta\rr, y) },
\end{align*}
where the \emph{integral loss} $\ellg$ associated with $\link$ is defined as $\ellg(t,y)\defeq -yt+\int_{0}^t \link(s)ds$. The basic property of $\Lgl$ is that $\nabla \Lgl(\theta)=\Exy{ \paren{\link(\lr \x,\theta \rr) -y }\cdot \x }$,
and hence $\nabla \Lgl(\ths)=0$, i.e., $\ths$ is a global minimizer of $\Lgl$. %


\newcommand{\constrth}{\nrm{\theta}\leq 1}
\newcommand{\constrw}{\nrm{Uw}\leq 1}
\paragraph{Reweighted objective}
Given a normalization matrix $U$ and a parameter together satisfying \eqref{def:U-app}, we may reweigh and regularize the objective function $\Lgl$ according to  
\begin{align*}
    \Lnrm(\theta)\defeq \Exy{ \frac{\ellg(\lr \x,\theta\rr, y)}{\nrm{U\x}} }+\frac{\Lipg\lambda}2\nrm{\theta}^2_{U\iv}.
\end{align*}
By \cref{asmp:link} and \eqref{def:U-app}, we know that $\frac12\mug\cdot  U\iq \preceq \nabla^2 \Lnrm(\theta) \preceq 2\Lipg \cdot U\iq$ for any $\constrth$.
Therefore, the objective $\Lnrm$ is well-conditioned after suitable linear transformation $\theta=Uw$. Specifically, we define
\begin{align}\label{def:Lnew}
    \Lnew(w)\defeq \Exy{ \frac{\ellg(\lr U\x,w\rr, y)}{\nrm{U\x}} }+\frac{\Lipg\lambda}{2}\nrm{w}^2_{U},
\end{align}
and then $\Lnew$ is $(2\Lipg)$-smooth and $(\mug/2)$-strongly convex over the domain $\cW_U\defeq \set{w\in\R^d: \constrw}$. Further, the gradient of $\Lnew$ can be derived as
\begin{align*}
    \nabla \Lnew(w)=\Exy{ \frac{U\x}{\nrm{U\x}}\paren{ \link(\lr U\x, w\rr)-y }} +\lambda \Lipg \cdot Uw.
\end{align*}

The following lemma indicates that, any approximate minimizer of $\Lnew$ is provides a good approximation of the ground truth parameter (under the $\Lone$-error).

\newcommand{\wstar}{w^\star}
\newcommand{\hwst}{\widehat{w}^\star_U}
\begin{lemma}\label{lem:Lnew}
Suppose that $(U,\lambda)$ satisfies \eqref{def:U-app}. We let $\hwst\defeq \argmin_{w: \nrm{Uw}\leq 1} \Lnew(w)$ and $\wstar\defeq U^{-1}\ths$.
Then it holds that $\nrm{\hwst-\wstar}\leq 4\lambda$. Further, the following holds:

(1) Estimation error: for any $w\in\cW$, we have
\begin{align}\label{eq:l1-est-error-bound}
    \errloneg{Uw}\leq \Lipg\errlone{Uw}\leq 2\sqrt{d}\Lipg \nrm{w-\wstar}
\end{align}

(2) Confidence interval: $\forall\x\in\Rd$, $\abs{ \lr \x, Uw\rr - \lr \x,\ths \rr }\leq \nrm{U\x}\cdot \nrm{w-\wstar},$
and $\Ex{\nrm{U\x}}\leq 2d$.
\end{lemma}

Therefore, with $N$ samples, the privatized gradient descent produces a solution $\hw$ with $\nrm{\hw-\hwst}\leq \tbO{N^{-1/2}}$, hence providing the desired convergence rate under $\Lone$-error through \eqref{eq:l1-est-error-bound}.
Note that this bypasses the lower bounds of \cref{prop:lower-linear-est}, because the objective $\Lnew$ is different from the standard squared loss objective. 


\subsection{Learning normalization matrix privately}\label{ssec:U-alg}

We start by describing how we can learn the \um~$U$ satisfying \eqref{def:U-app} from data privately. 
Indeed, even when the covariate distribution $p$ is known, it is not clear how to compute a \um~$U$, and \citet{chen2024private} have to invoke a fixed-point argument (Brouwer's theorem) to prove its existence. 
Our key observation is that the following spectral iterates converge to solution to \eqref{def:U}:
\begin{align}\label{eqn:spectral-exact}
    \cov\kk=\Ep{\uxxu[U\kk]}+\lambda U\kk, \qquad
    U\kp=\sym(\cov\kk\isq U\kk),
\end{align}
with the initial point $U\kz=\id$. Specifically, it holds that
\begin{align*}
    \lmin(\cov\kk)\sq\leq \lmin(\cov\kp)\leq \lmax(\cov\kp)\leq \lmax(\cov\kk)\sq.
\end{align*}
Therefore, if the iteration \cref{eqn:spectral-exact} is exact, the matrix $\cov\kk$ converges to the identity matrix at a quadratic rate, implying that $\bigO{\log\log(1/\lambda)}$ iterations are enough to achieve sufficient accuracy.

\paragraph{Approximate spectral iteration} 
In general, the covariate distribution $p$ is also not known, and we have to approximately implement the update rule \cref{eqn:spectral-exact}. To this end, we propose \cref{alg:U-JDP}, which privately approximates \eqref{eqn:spectral-exact} by \eqref{eq:spectral-approx-JDP} with batched samples.

\newcommand{\dataset}{\cD}
\newcommand{\errpara}{parameter}
\begin{algorithm}
\caption{Subroutine $\JDPLU$ %
}\label{alg:U-JDP}
\begin{algorithmic}
\REQUIRE Dataset $\cD=\sset{(\x_t,y_t)}_{t\in[T]}$, \errpara~$\delta\in(0,1)$.
\REQUIRE Epoch $K\geq 1$, batch size $N=\floor{\frac{T}{K}}$, parameters $\lambda\kz,\cdots,\lambda\kc$. %
\STATE Initialize $U\kz=\id$.
\FOR{$k=0,\cdots,K-1$}
    \STATE Compute the estimate on the split dataset $\dataset\kk=\sset{(\x_t,y_t)}_{t\in[kN+1,(k+1)N]}$:
    \begin{align}\label{eq:spectral-approx-JDP}
        H\kk=\frac1N\sumkn \usqx[U\kk][t].
    \end{align}
    \STATE Privatize $\til H\kk\sim \sympriv[1/N]{ H\kk }$.
    \STATE Update
    \begin{align*}
        \cov\kk=U\kk\sq \til H\kk U\kk\sq+\lambda\kk U\kk, \qquad
        U\kp=\sym(\cov\kk\isq U\kk).
    \end{align*}
\ENDFOR
\ENSURE \Um~$U=U\kc$ and normalization parameter $\lambda=\lambda\kc$.
\end{algorithmic}
\end{algorithm}





\cref{alg:U-JDP} preserves \aJDP~by the composition property of joint DP mechanisms, as proved in \cref{appdx:JDP-verify}. 
\begin{lemma}\label{lem:U-JDP-preserve}
    Subroutine $\JDPLU$~(\cref{alg:U-JDP}) preserves \aJDP.
\end{lemma}
We demonstrate that the iterates of \cref{alg:U-JDP} converge to a solution of \eqref{def:U-app}, as follows. The details are deferred to \cref{appdx:proof-U-JDP}.

\begin{proposition}\label{prop:alg-U-JDP}
Let $T\geq 1, K\geq 1$, $\delta\in(0,1)$, and $\epsN\defeq C_0\paren{ \sqrt{\frac{\log(K/\delta)}{N}}+\frac{\siga\sqrt{d+\log(K/\delta)}}{N} }$, where $C_0$ is an absolute constant chosen according to \cref{lem:spectral-concen-JDP}. Suppose that \cref{alg:U-JDP} is instantiated with parameters $\lambda\kk=(2k+5)\epsN$, and then \whp,
\begin{align*}
    \exp\paren{ -\frac{\log(1/\lambda_0)}{2^{k-1}} }\id \preceq \Ep{ \uxxu[U\kk] }+\lambda\kk U\kk \preceq \exp\paren{ \frac{12}{k} }\id.
\end{align*}
In particular, as long as $K\geq \max\sset{\log\log(N),20}$, \cref{alg:U-JDP} outputs $(U,\lambda)$ that satisfies \eqref{def:U-app} \whp, where $\lambda\defeq \lambda\kc=(2K+5)\epsN$.
\end{proposition}




\subsection{Private regression with reweighting}\label{ssec:L1-regression-rates}


In the following, we present \cref{alg:JDP-L1-regression} for private \Lone-regression, which is based on (1) first learning the normalization $(U,\lambda)$ by the subroutine $\JDPLU$ (\cref{alg:U-JDP}), and then (2) running the private batched SGD subroutine (\cref{alg:JDP-SGD}) on the reweighted objective $\Lnew$ defined in \eqref{def:Lnew}. The batched SGD subroutine is standard and hence deferred to \cref{appdx:JDP-l1-regression}.

\newcommand{\lamgd}{\uline{\boldsymbol{\epsilon}}}
\newcommand{\lamall}{\overline{\boldsymbol{\epsilon}}}

\begin{algorithm}
\caption{$\AlgJDPRegression$ %
}\label{alg:JDP-L1-regression}
\begin{algorithmic}[1]
\REQUIRE Dataset $\dataset=\sset{(\x_t,y_t)}_{t\in[T]}$, \errpara~$\delta\in(0,1)$.
\STATE Split the dataset $\dataset=\dataset_0 \cup \cD_1$ equally.
\STATE Set $(U,\lambda) \leftarrow \JDPLU(\dataset_0,\delta)$.
\STATE Set $\hw\leftarrow \AlgJDPGD(\Lnew,\cD_1)$.
\ENSURE Normalization $(U,\lambda)$, estimator $\hth=U\hw$, and estimation error $\lamall$.
\end{algorithmic}
\end{algorithm}







\begin{theorem}[Generalized linear regression with JDP]\label{thm:JDP-L1-regression}
Let $T\geq 1, \delta\in(0,1)$, and the subroutines of \cref{alg:JDP-L1-regression} are suitably instantiated according to \cref{appdx:JDP-l1-regression}. Then, under generalized linear model, \cref{alg:JDP-L1-regression} preserves \aJDP, and the following holds:

(1) \Whp, the normalization $(U,\lambda)$ satisfies \eqref{def:U-app}, and the returned estimator $\hw$ satisfies $\Lipg \nrm{\hw-\hwst}\leq \lamgd$, where 
\begin{align*}
    \lambda := \lambda(T,\delta)= &~ \tbO{ \sqrt{\frac{\log(1/\delta)}{T}}+ \siga \frac{\sqrt{d\log(1/\delta)}}{T}}, \\
    \lamgd := \lamgd (T,\delta)= &~ \tbO{ \kpg^{3/2}\sqrt{\frac{\log(1/\delta)}{T}} + \siga (\kpg^{3/2}+\kpg\sqrt{d})\frac{\sqrt{\log(1/\delta)}}{T} },
\end{align*}
are defined in \cref{prop:alg-U-JDP} and \cref{prop:JDP-GD},
respectively, and 
$\tbO{\cdot}$ hides polynomial factors of $\log(T)$. The overall estimation error $\lamall$ is defined as $\lamall(T,\delta)\defeq 4\Lipg\lambda(T,\delta)+\lamgd(T,\delta)$.

(2) In particular, under the success event of (1), we have (by \cref{lem:Lnew})
\begin{align*}
    \errloneg{\hth}\leq \Lipg\cdot \errlone{\hth}\leq  2\sqrt{d} \cdot \lamall,
\end{align*}
and for all $\x\in\Rd$, the confidence bound holds: $\absn{ \nu(\lr \x, \hth\rr) - \nu(\lr \x,\ths \rr) }\leq \nrm{U\x} \lamall$.
\end{theorem}

Therefore, \cref{alg:JDP-L1-regression} is an $\Lone$-regression oracle in the sense of \cref{def:L1-oracle}.
In particular, for linear models, we have $\kpg=1$, and hence the convergence rate can further be simplified to $\tbO{ \sqrt{\frac{d}{T}}+\frac{d}{\alpha T} }$. The first term matches the minimax-optimal convergence rate of non-private \Lone-regression (given by the least squares), and the second term (``price-of-privacy'') is of lower order compared to the first term.


\paragraph{Extension: Local DP}
As an LDP extension of \cref{alg:JDP-L1-regression}, we propose \cref{alg:LDP-L1-regression} that preserves local DP while achieving near-optimal \Lone-regression guarantee.
\begin{theorem}[Generalized linear regression with LDP]\label{thm:LDP-L1-regression}
Let $T\geq 1, \delta\in(0,1)$. \cref{alg:LDP-L1-regression} (described in \cref{appdx:LDP-l1-regression}) preserves \aLDP, and it returns estimator $\hth$ such that \whp,
\begin{align*}
    \errloneg{\hth}
    \leq \tbO{\siga (\kpg d+\sqrt{\kpg^3d} ) \sqrt{\frac{\log(1/\delta)}{T}}},
\end{align*}
where $\tbO{\cdot}$ hides polynomial factors of $\log(T)$.
Further, the total number of switches (or, changes) of the deployed private channels is bounded by $\bigO{\kpg \log T}$.
\end{theorem}

For linear models, the upper bound of \cref{thm:LDP-L1-regression} simplifies to $\tbO{\frac{d}{\alpha\sqrt{T}}}$. Such a nearly minimax-optimal convergence rate is first derived by~\citet{chen2024private}, using an algorithm that solves an \emph{exponential-time} optimization problem for each round.







\section{Private Learning in Generalized Linear Contextual Bandits}\label{sec:cb}


In this section, we use our \Lone-regression method from \cref{sec:l1-reg} as a subroutine for the action elimination framework~\citep{li2024optimal}, providing rate-optimal private regret bounds for generalized linear contextual bandits. 

Our algorithm (\cref{alg:batch-cb-JDP}) is epoch-based (with a given epoch schedule $1=T_0<T_1<T_2<\cdots<T_{J}=T$), and it iteratively builds estimations of the ground truth reward function $\fs$ and plans according to the estimations. The algorithm, which consists of an estimation procedure and a planning procedure described as follows, is similar to that of \citet{li2024optimal}. 
\colt{The formal definition is deferred to \cref{appdx:spanner}, and we briefly introduce the main ideas here.}
\arxiv{

\begin{algorithm}
\begin{algorithmic}
\REQUIRE Round $T\geq 1$, parameter $\delta\in(0,1)$, epoch schedule $1=T\epj[0]<T\epj[1]<\cdots<T\epj[J]=T$.
\STATE Initialize $\hft[0]\equiv 0, \CIt[0]\equiv 1$.
\FOR{$j=0,1,\cdots,J-1$}
\STATE Set $\pit[j]\leftarrow\AlgPlan(\set{ (\hft,\CIt)  }_{\tau<j})$ and initialize the dataset $\dataset\epj=\sset{}$.
\FOR{$t=T\epj,\cdots,T\epj[j+1]-1$}
\STATE Observe context $x_t\sim P$, select action $a_t\sim \pit[j](x_t)$, and receive reward $r_t$. 
\STATE Update the dataset $\dataset\epj \leftarrow \dataset\epj \cup \sset{ (\phxa[x_t,a_t], r_t) }$.
\ENDFOR
\STATE Set $(U\epj, \lambda\epj), \hth\epj,\lamall\epj \leftarrow \AlgJDPRegression(\cD\epj,\delta)$.
\STATE Update the estimated reward function and confidence bound as
\begin{align*}
    \hft[j](x,a)=\nu(\lr \phxa, \hth\epj \rr), \qquad
    \CIt[j](x,a)=\lamall \epj \cdot \nrmn{U\epj \phxa}, \qquad \forall (x,a)\in\cX\times\cA.
\end{align*}
\ENDFOR
\end{algorithmic}
\caption{Action Elimination Algorithm with JDP}\label{alg:batch-cb-JDP}
\end{algorithm}

}






\paragraph{Estimation procedure}
For $j$th epoch, the estimate reward function $\hft[j]$ and the confidence radius $\CIt[j]$ are produced by the subroutine $\AlgJDPRegression$. Then, by \cref{thm:JDP-L1-regression}, it holds that with high probability
\begin{align}\label{eq:CI-bound}
\textstyle
    \fs(x,a)\in \brac{ \hft[j](x,a)-\CIt[j](x,a), \hft[j](x,a)+\CIt[j](x,a) }, \qquad \forall x\in\cX, a\in\cA.
\end{align}

\arxiv{
\begin{algorithm}
\begin{algorithmic}
\REQUIRE Confidence interval descriptions $(\hft[0], \CIt[0]),\cdots,(\hft[j-1],\CIt[j-1])$ up to the $j$th epoch.
\FOR{context $x\in\cX$}
\STATE Set $\cAxt[0]=\cA$.
\FOR{$\tau=1,\cdots,j-1$}
\STATE Set
\begin{align}\label{eq:eliminate}
    \cAxt=\sset{ a\in\cAxt[\tau-1]: \hft(x,a)+\CIt(x,a)\geq \max_{a'\in\cA}\hft(x,a')-\CIt(x,a') }.
\end{align}
\ENDFOR
\STATE Compute a spanner $\cAsp[j]$ of $\cAxt[j-1]$ (\cref{def:spanner}).
\STATE Set $\pit[j](x)=\Unif(\cAsp[j])$.
\ENDFOR
\ENSURE Policy $\pit[j]$.
\end{algorithmic}
\caption{Subroutine $\AlgPlan$ %
}\label{alg:plan}
\end{algorithm}


\paragraph{Planning procedure}
The policy $\pit[j]$ of the $j$th epoch is built upon the confidence intervals \cref{eq:CI-bound} given by the estimations $(\hft[0], \CIt[0]),\cdots,(\hft[j-1],\CIt[j-1])$ 
from previous epochs. Given the estimations, subroutine $\AlgPlan$ (\cref{alg:plan}) eliminates the sub-optimal arms for each context $x\in\cX$ according to \eqref{eq:eliminate}, and output the $\pi\epj(x)$ based on a \emph{spanner}
of the remaining actions.\footnote{Defined in \cref{def:spanner}. Particularly, when $|\cA|=\bigO{1}$, we can directly takes $\cAsp[j]=\cAxt[j-1]$ and set $\pi\epj(x)=\Unif(\cAxt[j-1])$.
} This procedure implicitly encourages exploration, as it uses optimistic estimation (UCB) of the value of each arm.
}

\colt{
\paragraph{Planning procedure}
The policy $\pit[j]$ of the $j$th epoch is built upon the confidence intervals \cref{eq:CI-bound} given by the estimations $(\hft[0], \CIt[0]),\cdots,(\hft[j-1],\CIt[j-1])$ 
from previous epochs. Given the estimations, subroutine $\AlgPlan$ (\cref{alg:plan}, detailed in \cref{appdx:spanner}) eliminates the sub-optimal arms for each context $x\in\cX$ according to \eqref{eq:eliminate}, and output the $\pi\epj(x)$ based on a \emph{spanner}
of the remaining actions. This procedure implicitly encourages exploration, as it uses optimistic estimation (UCB) of the value of each arm.
}




\paragraph{Regret guarantee}
We state the upper bound of \cref{alg:batch-cb-JDP} in terms of the dimensionality of the per-context action space: \colt{$\dA\defeq \max_{x\in\cX} \dim\paren{\set{ \phi(x,a): a\in\cA }}$.}
\arxiv{
\begin{align*}
    \dA\defeq \max_{x\in\cX} \dim\paren{\set{ \phi(x,a): a\in\cA }}.
\end{align*}
}
Note that $\dA\leq \min\set{d,|\cA|}$.

\begin{theorem}[Regret upper bound under JDP]\label{thm:regret-upper-JDP}
\cref{alg:batch-cb-JDP} preserves \aJDP. With the epoch schedule be $T\epj=2^j$ for $j=0,1,\cdots$ and $\delta=\frac{1}{T}$, it holds that
\begin{align*}
\textstyle
    \Reg\leq \tbO{\dA d \kpg^{3/2} \sqrt{T} + \siga \dA d(\kpg^{3/2}+\kpg\sqrt{d})}.
\end{align*}
\end{theorem}
\vspace{-10pt}
This provides a regret of order $\sqrt{T}+\frac{1}{\alpha}$ (omitting $\poly(d)$-factors and logarithmic terms), which partially resolves the open problem stated by \citet{azize2024open}.

As a remark, we note that we state \cref{alg:plan} for the sake of clarity: In the implementation of \cref{alg:batch-cb-JDP}, we do \emph{not} need to range over every context $x\in\cX$ to form the policy $\pit[j]$. It is sufficient to compute $\pit[j](x_t)$ for each round $t$ in $j$th epoch (according to \eqref{eq:eliminate}). Note that a spanner can be computed in time $\poly(d,|\cA|)$, and hence
\cref{alg:batch-cb-JDP} can be implemented in time $\poly(d,|\cA|)\cdot T$. Further discussions are deferred to \cref{appdx:spanner}.

\paragraph{Regret for linear contextual bandits}
In linear contextual bandits with a bounded action space $\cA$, we can use a slightly different subroutine for producing confidence intervals (detailed in \cref{appdx:proof-regret-upper-JDP-better}), which provides the following refined upper bound.

\begin{theorem}[Regret upper bound in linear contextual bandits]\label{thm:regret-upper-JDP-better}
\cref{alg:batch-cb-JDP} (with the estimation subroutine replaced by \cref{alg:linear-JDP-better}) preserves \aJDP~and guarantees that
\begin{align*}
\textstyle
    \Reg\leq \tbO{ \dA \sqrt{dT\log|\cA|}+ \siga \dA d^{3/2} }.
\end{align*}
\end{theorem}
\vspace{-10pt}
In particular, when $|\cA|=\bigO{1}$, we obtain a upper bound of $\tbO{\sqrt{dT}+\frac{d^{3/2}}{\alpha}}$. Note that the $\sqrt{dT}$-term matches the optimal non-private regret bound for $|\cA|=\bigO{1}$ (up to logarithmic factors). %
Therefore, in this case, privacy is almost \emph{for free}, as the second term (the ``price of privacy'') grow as $\tbO{1/\alpha}$ and is of lower order as $T\to \infty$.

\paragraph{Extension: Local DP}
By using $\AlgLDPRegression$ (\cref{alg:LDP-L1-regression}) as the estimation subroutine, we can easily adapt \cref{alg:batch-cb-JDP} so that it preserves local DP. We state the corresponding regret bound as follows and defer the details to \cref{appdx:regret-upper-LDP}.
\begin{theorem}[Regret upper bound under LDP]\label{thm:regret-upper-LDP}
A variant of \cref{alg:batch-cb-JDP} with estimation subroutine \cref{alg:LDP-L1-regression} (detailed in \cref{appdx:regret-upper-LDP}) preserves \aLDP~and
\begin{align*}
\textstyle
    \Reg\leq \tbO{\siga \dA d(\kpg\sqrt{d}+\kpg^{3/2}) \sqrt{T} }.
\end{align*}
\end{theorem}
\vspace{-10pt}
Particularly, in linear contextual bandits, \cref{thm:regret-upper-LDP} provides a regret bound of $\sqrt{d^5T}/\alpha$. While this is a $d$-factor worse than the result of \citet{chen2024private}, our algorithm is computationally efficient, and the number of switches (or, changes) of the deployed decision-channel pair $(\pi_t, \pr_t)$ is bounded by $\bigO{\kpg \log^2(T)}$.



\arxiv{
\section{Dimension-free Linear Regression}\label{sec:unbounded}



In \cref{sec:l1-reg} and \cref{sec:cb}, we provided a somewhat satisfactory picture of the optimal rates of private learning in contextual bandits with generalized linear models, when the dimension $d$ is \emph{bounded}. 
In this section, we turn our focus to the setting where the dimension $d$ is prohibitively large or \emph{unbounded}, e.g., when the linear function parametrization is in fact given by a Reproducing Kernel Hilbert Space (RKHS).\footnote{Our approach naturally applies to learning in RKHS. However, to avoid measure-theoretic issues, we only present our algorithms for finite dimensional space.} 


This setting is fundamentally more challenging, as the following lower bounds indicates. The proof can be found in \citep[Appendix C]{chen2024private}. 

\newcommand{\Sd}{\mathbb{S}^{d-1}}
\begin{proposition}\label{prop:unbounded-lower}
Let $d\geq 1$, covariate space $\cC=\Sd$ be the $d$-dimensional unit sphere. 
For each $\theta\in\Sd$, we consider the linear model $M_\theta$:
\begin{align*}
    (\x,y)\sim M_\theta: \qquad \x=\theta, y=1.
\end{align*}
For any parameter $R\in[1,c\sqrt{d}]$ ($c>0$ is a small absolute constant), the following holds:

(a) Suppose that $\alg$ is a $T$-round $(\alpha,0)$-JDP algorithm with output $\nrmn{\hth}\leq R$. Then it holds that
\begin{align*}
    \sup_{\ths\in\Sd}\EE^{M_{\ths},\alg} \absn{\lr \ths,\hth\rr -1 }\geqsim 1, \qquad \text{unless }T\geqsim \frac{d}{R^2\alpha}.
\end{align*}

(b) Suppose that $\alg$ is a $T$-round \aLDP~algorithm with output $\nrmn{\hth}\leq R$. Then it holds that
\begin{align*}
    \sup_{\ths\in\Sd}\EE^{M_{\ths},\alg} \absn{\lr \ths,\hth\rr -1 }\geqsim 1, \qquad \text{unless }T\geqsim \min\sset{\frac{d}{R^2\alpha^2}, \frac{1}{\beta}}.
\end{align*}
\end{proposition}

Note that for each linear model $\PP_\theta$, the covariance matrix $\EE_{\PP_\theta}\x\x\tp=\theta\theta\tp$ is of rank 1. %
Therefore, \cref{prop:unbounded-lower} has two implications for \emph{pure JDP} (and also LDP) regression with unbounded dimension $d$ and unknown covariate distribution:
\begin{itemize}
\item[(1)] Estimating the covariance matrix requires $\Omega(d)$ samples, even when the covariance matrix is known to have rank 1.
\item[(2)] Proper estimator of the parameter $\ths$ also requires $\Omega(d)$ samples to achieve a non-trivial error.
Conversely, any non-trivial estimator $\hth$ must have norm $\nrmn{\hth}\geq \Omega(\sqrt{d/T})$ (with non-trivial probability).
\end{itemize}
Therefore, with unbounded dimension $d$, to achieved estimation guarantees, the estimator in consideration has to be highly \emph{improper}, and it also cannot rely on estimating the covariance matrix.


Based on the observations above,
in this section, we develop improper private procedures with \emph{dimension-free} bounds in private linear regression.
We then apply the proposed methods to provide dimension-free regret bounds in private linear contextual bandits.




\subsection{Private improper batched SGD}\label{ssec:l1-dim-free}


We begin with the joint DP setting. For any non-private estimator $\hth$ with \emph{sensitivity} $s$, it is well-known that the estimator $\hth'=\hth+\zeta$ ensures \aJDP~with noise $\zeta\sim \normal{0,\eps^2\id}$ and parameter $\eps:=s\cdot \siga$. The key idea is that, while $\nrm{\zeta}\asymp \eps\sqrt{d}$, we have $\nrm{\zeta}_{\bSigma}\leqsim \eps$ with high probability, where $\bSigma\defeq \Epp{\x\x\tp}$ is the covariance matrix.
Therefore, to ensure JDP, it is sufficient to privatize a non-private estimator with low sensitivity. 

\newcommand{\pa}[1]{\theta\epk{#1}}
\newcommand{\gd}[1]{g\epk{#1}}
\newcommand{\ogd}[1]{\bar{g}\epk{#1}}
\newcommand{\xt}{\x_t}
\newcommand{\yt}{y_t}
\newcommandx{\clip}[2][1=R]{\mathsf{clip}_{#1}(#2)}
\newcommandx{\nt}[1][1=t]{\zeta_{#1}}
\newcommand{\gt}{g_t}
\newcommand{\tgt}{\Tilde{g}_t}
\newcommandx{\avgtk}[1][1=k]{\frac1N\sum_{t=#1 N+1}^{(#1+1)N}}
\begin{algorithm}
\caption{$\AlgJDPIGD$}\label{alg:JDP-improper-GD}
\begin{algorithmic}[1]
\REQUIRE Dataset $\dataset=\sset{(\x_t,y_t)}_{t\in[T]}$.
\REQUIRE Epoch $K\geq 1$, batch size $N=\floor{\frac{T}{K}}$, stepsize $\eta=1$, parameter $R=2$. %
\STATE Initialize $\theta\kz=\bz$.
\FOR{$k=0,1,\cdots,K-1$}
\STATE Compute gradient estimate
\begin{align*}
    g\kk=\frac1N \sumkn \xt\paren{ \lr \xt, \theta\kk\rr-\yt }.
\end{align*}
\STATE Update
\begin{align*}
    \theta\kp=\Proj_{\BR}\paren{\theta \kk-\eta g\kk}.
\end{align*}
\ENDFOR
\STATE Privatize $\hth\sim \priv[\eta(R+1)/N]{\theta\kc}$
\ENSURE Estimator $\hth$.
\end{algorithmic}
\end{algorithm}

Based on these observations, we consider the projected batched SGD on the standard square-loss:
\begin{align*}
    \Lsq(\theta)=\frac12\Exy{\paren{\lr \x,\theta\rr-y}^2}.
\end{align*}
By directly privatizing its last iterate (\cref{alg:JDP-improper-GD}), we can achieve a near-optimal convergence rate (detailed in \cref{appdx:improper-JDP}).










\begin{theorem}[Dimension-free JDP regression]\label{thm:improper-JDP}
Let $T\geq1, \delta\in(0,1)$. \cref{alg:JDP-improper-GD} preserves \aJDP, and with a suitably chosen parameter $K$, it ensures that
\begin{align*}
    \Epp{\lr \x,\hth-\ths\rr^2}=\nrmn{\hth-\ths}_{\bSigma}^2\leqsim \paren{\frac{\log T\log(1/\delta)}{T}}^{1/2}+\paren{\frac{\siga \sqrt{\log(1/\delta)}}{T}}^{2/3}.
\end{align*}
\end{theorem}

Therefore, the estimator $\hth$ achieves the \emph{dimension-independent} convergence rate of $\frac{1}{\sqrt{T}}+\frac{1}{(\alpha T)^{2/3}}$. In non-private linear models, the $T^{-1/2}$-rate of convergence is known to be minimax-optimal and can be achieved by vanilla gradient descent.

\paragraph{Extension: Local DP}
The situation under the local DP model is much more subtle.
Indeed, one may expect that \cref{alg:JDP-improper-GD} naturally extends to this setting.
However, to ensure local privacy in (batched) SGD, for every step, the gradient estimator
has to be privatized. To this end,
it is typically necessary to add a noise vector $\zeta$ that has norm scaling with $\Omega(\sqrt{d})$ (e.g., when $\zeta$ is the Gaussian noise). In other words, the privatized gradient estimator has norm scaling with $\sqrt{d}$. Hence, after a single step of gradient descent, the iterate falls outside the unit ball $\Bone$, and projection back to $\Bone$ can lead to large bias. Therefore, the method of projected gradient descent may not be applied here.


Instead, we consider performing privatized batch SGD directly, replacing the projection operation with a careful clipping on the gradient estimator (\cref{alg:LDP-improper-GD}). This is based on extending the aforementioned observation under JDP: When $\zeta\sim \normal{0,\eps^2\id}$, while $\nrmn{\zeta}\sim \eps \sqrt{d}$, for the covariate $\x\sim p$ that is independent of $\zeta$, the random variable $\lr \x,\zeta\rr\sim \normal{0,\eps^2\nrm{\x}^2}$ is a zero-mean Gaussian random variable conditional on $\x$. Particularly, it holds that $\abs{\lr \x, \zeta\rr}\leqsim \eps\nrm{\x}$ with high probability with respect to the randomness of the noise $\zeta$ and $\x\sim p$. Using this idea, we can show that \eqref{eq:clip-grad} provides an estimator of $\nabla \Lsq(\theta\kk)$ with small bias for all epochs. Here, the clipping operation is defined as
\begin{align*}
    \clip{v}\defeq \max\sset{\min\sset{v,R},-R}\in[-R,R], \qquad \forall v\in\R.
\end{align*}

\begin{algorithm}
\caption{$\AlgLDPIGD$}\label{alg:LDP-improper-GD}
\begin{algorithmic}[1]
\REQUIRE Round $T\geq 1$.
\REQUIRE Epoch $K\geq 1$, batch size $N=\floor{\frac{T}{K}}$, stepsize $\eta=1$, parameter $R=2$.
\STATE Initialize $\theta\kz=\bz$.
\FOR{$k=0,\cdots,K-1$}
    \FOR{$t=\rangekn$}
        \STATE Observe $(\xt,\yt)\sim p$ and form the gradient estimator
        \begin{align}\label{eq:clip-grad}
            g_t=\xt\paren{ \clip{\lr \theta\kk,\xt \rr}-\yt }
        \end{align}
        \STATE Privatize $\til g_t\sim \priv[R+1]{g_t}$.    
    \ENDFOR
    \STATE Compute batched gradient estimator $\til g\kk=\avgtk \tgt$ and update
\begin{align*}
    \theta\kp=\theta \kk-\eta \til g\kk.
\end{align*}
\ENDFOR
\ENSURE $\hth=\theta\kc$.
\end{algorithmic}
\end{algorithm}

With a careful analysis that bounds the bias introduced by clipping \cref{eq:clip-grad}, we provide the following guarantee for \cref{alg:LDP-improper-GD}.

\begin{theorem}[Dimension-free LDP regression]\label{thm:improper-LDP}
\cref{alg:LDP-improper-GD} preserves \aLDP. For $T\geq 1, \delta\in(0,1)$, with a suitable number of epochs $K\geq 1$, \cref{alg:LDP-improper-GD} returns $\hth$ that \whp,
\begin{align*}
    \Epp{ \lr \x, \hth-\ths \rr^2 }=\nrmn{\hth-\ths}_{\bSigma}^2\leqsim \paren{\frac{\siga\log(T/\delta)}{T}}^{1/3}.
\end{align*}
\end{theorem}

This establishes a convergence rate of $\tbO{(\alpha^2T)^{-1/3}}$ under the square loss. As shown by \citet[Corollary I.8]{chen2024private}, any LDP algorithm must incur an $\Lone$-error of
\begin{align*}
    \Omega\paren{\min\sset{\frac{d}{\sqrt{\alpha^2 T}},\paren{\frac{1}{\alpha^2 T}}^{1/6}}}.
\end{align*}
Therefore, \cref{alg:LDP-improper-GD} achieves the optimal dimension-free convergence rate under $\Lone$-error, and hence it is optimal under the $L_2$-error. Further, \cref{thm:LDP-L1-regression} and \cref{thm:improper-LDP} together provide the near-optimal $\Lone$-estimation error for the full range of $T\in[1,\infty)$ under LDP.

\subsection{Application: Linear contextual bandits with dimension-free regret} 

As an application, we use the dimension-free procedures developed in \cref{ssec:l1-dim-free} as subroutines for learning linear contextual bandits.
We invoke the $\AlgSQCB$ algorithm \citep{abe1999associative,foster2020beyond,simchi2020bypassing}, which has a regret guarantee given any \emph{offline regression oracle} with $L_2$-error bound. In particular, by instantiating the regression oracle as $\AlgJDPIGD$ (\cref{alg:JDP-improper-GD}) or $\AlgLDPIGD$ (\cref{alg:LDP-improper-GD}), we obtain the following private regret bounds. The details are presented in the \cref{appdx:square-cb}.
\begin{theorem}[Dimension-free regret bounds]\label{thm:regret-dim-free}
Let $T\geq 1$ and suppose $\cA$ is finite.

(1) Suppose that $\AlgSQCB$ (\cref{alg:square-cb}) is instantiated by the regression subroutine $\AlgJDPIGD$ (\cref{alg:JDP-improper-GD}). Then $\AlgSQCB$ preserves \aJDP~and it ensures 
\begin{align*}
    \Reg\leq \sqrt{|\cA|}\cdot \tbO{T^{3/4}+\siga^{1/3}T^{2/3}}.
\end{align*}

(2) Suppose that $\AlgSQCB$ (\cref{alg:square-cb}) is instantiated by the regression subroutine $\AlgLDPIGD$ (\cref{alg:LDP-improper-GD}). Then $\AlgSQCB$ preserves \aLDP~and it ensures 
\begin{align*}
    \Reg\leq \sqrt{|\cA|}\cdot \tbO{\siga^{1/3}T^{5/6}}.
\end{align*}
\end{theorem}

To the best of our knowledge, such dimension-free regret bounds are new in private contextual bandits. Under JDP, the regret rate is $\tbO{T^{3/4}+\alpha^{-1/3}T^{2/3}}$, and the first term matches the optimal dimension-free $T^{3/4}$-regret in non-private linear contextual bandits~\citep{abe1999associative,foster2020beyond}, implying that privacy is almost ``for free'' in this setting. Furthermore, the LDP regret bound scales as $\alpha^{-1/3}T^{5/6}$, which nearly matches the minimax lower bound \citep[Corollary I.15]{chen2024private}. 




}

\section*{Conclusion}


In this work, we propose a novel method of private re-weighted regression for private (generalized) linear regression, achieving near-optimal convergence rates under $L_1$-error. Based on this method, we provide efficient algorithms for (generalized) linear contextual bandits with near-optimal regret bounds in both the joint and local model of differential privacy. Furthermore, we also develop the improper private procedures with near-optimal, dimension-independent rates in linear models and linear contextual bandits. 

\subsection*{Acknowledgements} FC and AR acknowledge support from ARO through award W911NF-21-1-0328, as well as Simons Foundation and the NSF through awards DMS-2031883 and PHY-2019786. 

\arxiv{
\bibliographystyle{abbrvnat}
}
\bibliography{ref.bib}

\newpage
\appendix


\section{Technical tools}
\begin{table}[!t]
  \caption{Tools used by the \evaluator}
  \label{tab:tools}
  \centering
  \footnotesize
  \begin{threeparttable}
    \begin{tabular}{p{0.45\linewidth}p{0.4\linewidth}}
    \toprule
    \textbf{Property}     & \textbf{Tool} \\
    \midrule
    Target Binding Affinity & AutoDock Vina~\cite{trott_autodock_2010} \\
    Drug-likeness (QED)     & RDKit~\cite{rdkit}         \\
    Lipinski's Rule         & RDKit~\cite{rdkit}         \\
    Synthetic Accessibility & RDKit~\cite{rdkit}         \\
    Novelty                 & RDKit~\cite{rdkit}         \\
    Diversity               & RDKit~\cite{rdkit}         \\
    \bottomrule
    \end{tabular}
  \end{threeparttable}
  % \vskip -15pt
\end{table}

\section{Proofs from \cref{sec:negative}}\label{appdx:negative}
\subsection{Proof of \cref{prop:lower-linear-est}}\label{appdx:proof-lower-linear}

\cref{prop:lower-linear-est} (2) is a well-known result (see e.g., Lemma 25 and Appendix B.2 of \citep{chen2024private}). In the following, we prove \cref{prop:lower-linear-est} (1) using the same idea. We invoke the following lemma of \citet{karwa2017finite}.
\begin{lemma}\label{lem:TV-JDP}
Fix a non-interactive \aJDP~algorithm $\alg: \cZ^T\to \DPi$. For any model $M\in\DZ$, we let $\PP\sups{M,\alg}\in\DPi$ be the distribution of $\pi\sim \alg(z_1,\cdots,z_T)$ under i.i.d $z_1,\cdots,z_T\sim M$. Then for any models $M, \oM\in\DZ$, it holds that
\begin{align*}
    \PP\sups{M,\alg}(E)\leq e^{\alpha_T}\PP\sups{\oM,\alg}(E)+\beta_T, \qquad \forall E\subseteq \Pi,
\end{align*}
where $\alpha_T=6\alpha T\DTV{M,\oM}$ and $\beta_T=4e^{\alpha_T}T\beta\DTV{M,\oM}$.
\end{lemma}

Now, for any $\theta\in [-1,1]$, we let $M_\theta$ be the distribution of $(x,y)$ under $\x\sim p, y|\x\sim \Rad{\x\theta}$. Then, by definition,
\begin{align*}
    \DTV{M_\theta,M_0}=\EE_{\x\sim p} \frac12 |\x\theta|=\frac{\lambda|\theta|}{2}.
\end{align*}
Let $\theta=\min\sset{ 1, \frac{1}{20\lambda T(\alpha+\beta)} }$. Then, by \cref{lem:TV-JDP}, we have
\begin{align*}
    \PP\sups{M_\theta,\alg}(E)\leq 1.4~\PP\sups{M_0,\alg}(E)+\frac14, \qquad \forall E\subseteq \Pi. 
\end{align*}
In this problem, the decision space is $\Pi=[-1,1]$, and the loss function is $L(\theta,\pi)=\Ep{\abs{\x(\pi-\theta)}^2}=\lambda^2|\pi-\theta|^2$. We consider the event $E=\sset{\pi: \pi>\frac{\theta}{2}}$. By definition,
\begin{align*}
    \EE\sups{M_\theta,\alg} L(\theta,\pi)\geq \frac{\lambda^2\theta^2}{4}\PP\sups{M_\theta,\alg}(E^c), \qquad
    \EE\sups{M_0,\alg} L(0,\pi)\geq \frac{\lambda^2\theta^2}{4}\PP\sups{M_0,\alg}(E).
\end{align*}
Thus,
\begin{align*}
    \max\sset{ \EE\sups{M_\theta,\alg} L(\theta,\pi), \EE\sups{M_0,\alg} L(0,\pi) }\geq&~ \frac{\lambda^2\theta^2}{8}\paren{ \PP\sups{M_\theta,\alg}(E^c)+\PP\sups{M_0,\alg}(E) } \\
    \geq&~\frac{\lambda^2\theta^2}{8}\paren{ \PP\sups{M_\theta,\alg}(E^c)+0.7~\PP\sups{M_\theta,\alg}(E)-\frac{1}{5} } \\
    \geq&~ \frac{\lambda^2\theta^2}{16}
    \geq c_0\min\sset{ \lambda, \frac{1}{\lambda T^2(\alpha+\beta)^2} }.
\end{align*}
This is the desired lower bound.
\qed


\section{Proofs from \cref{sec:l1-reg}}\label{appdx:LG}
\subsection{Proof of \cref{lem:Lnew}}
By definition, $\nabla \Lnew(\wstar)=\lambda \mug \cdot U\wstar$, and hence $\nrm{\nabla \Lnew(\wstar)}\leq \mug \lambda$. Then, because $\Lnew$ is $(\mug/2)$-strongly convex over $w\in\cW$, we have
\begin{align*}
    \Lnew(w)\geq \Lnew(\wstar)+\lr \nabla \Lnew(\wstar), w-\wstar\rr + \frac{\mug}{4}\nrm{w-\wstar}^2, \qquad w\in\cW.
\end{align*}
In particular, using $\Lnew(\hwst)\leq \Lnew(\wstar)$, we have
\begin{align*}
    \frac{\mug}{4}\nrm{\hwst-\wstar}^2\leq -\lr \nabla \Lnew(\wstar), \hwst-\wstar\rr\leq \mug\lambda \nrm{\hwst-\wstar},
\end{align*}
and hence $\nrm{\hwst-\wstar}\leq 4\lambda$.

Next, for any vector $v\in\R^d$, we have
\begin{align*}
    \Ex{ \abs{\lr \x, Uv\rr} }^2
    \leq \Ex{ \nrm{U\x} } \cdot \Ex{\frac{\lr \x, Uv\rr^2}{\nrm{U\x}}}.
\end{align*}
Note that $\Ex{\uxxu}\preceq 2\id$, and hence we have $\Ex{ \nrm{U\x} }\leq 2d$ and $\Ex{\frac{\lr \x, Uv\rr^2}{\nrm{U\x}}}\leq 2\nrm{v}^2$. Therefore, it holds that
\begin{align*}
    \Ex{ \abs{\lr \x, Uv\rr} }\leq 2\sqrt{d}\nrm{v}.
\end{align*}
Substituting $v=w-\wstar$ completes the proof.
\qed



\subsection{Proof of \cref{lem:U-JDP-preserve}}\label{appdx:JDP-verify}




We first note that in \cref{alg:U-JDP}, the dataset $\cD$ is split equally as $\cD=\cD\kz\sqcup \cdots \sqcup \cD\epk{K-1}$, and iteration at the $k$th epoch can be regarded as a random function $(U\kk;\cD\kk)\mapsto U\kp$. Therefore, using the composition property of joint DP (\cref{lem:DP-composition}), we only need to verify that $(U\kk;\cD\kk)\mapsto U\kp$ preserves \aJDP~(with respect to $\cD\kk$).

For the data $(\x_t,y_t)$ in the $k$th epoch, the quantity $\Phi_t=\usqx[U\kk][t]$ can be bounded as $\nrmF{\Phi_t}\leq 1$ (\cref{lem:uxxu}). Thus, $H_{(k)}$ defined in \eqref{eq:spectral-approx-JDP} has sensitivity $\Delta=1/N$ under Frobenius norm. Therefore, by the privacy guarantee of Gaussian channels (\cref{def:Guassian-channel}), the mechanism $(U\kk;\dataset\kk)\mapsto \til H\kk$ preserves \aJDP~with respect to $\dataset\kk$. Consequently, by the post-processing property, $(U\kk;\dataset\kk)\mapsto U\kk$ also preserves \aJDP. Therefore, by applying the composition property (\cref{lem:DP-composition}) inductively, we have shown that \cref{alg:JDP-L1-regression} also preserves \aJDP.




\begin{lemma}[Iterative composition of JDP] \label{lem:DP-composition}
Suppose the algorithm $\alg: \cZ^{N_1+N_2}\to \DPi$ outputs $\pi\sim \alg(z_1,\cdots,z_{N_1+N_2})$ generated as
\begin{align*}
    \pi_1\sim \alg_1(z_1,\cdots,z_{N_1}), \quad
    \pi\sim \alg_2(\pi_1; z_{N_1+1},\cdots,z_{N_1+N_2}),
\end{align*}
where the algorithm $\alg_1:\cZ^{N_1}\to \Delta(\Pi_1)$ preserves \aJDP, and $\alg_2(\pi_1;\cdot):\cZ^{N_2}\to \DPi$ preserves \aJDP~for any $\pi_1\in\Pi_1$. Then $\alg$ preservs \aJDP.
\end{lemma}

\subsubsection{Proof of \cref{lem:DP-composition}}

For ease of presentation, we only consider the case both $\Pi$ and $\Pi_1$ are discrete.
For two neighbored dataset $\cD=\{z_i\}_{i=1}^{N_1+N_2}$ and $\cD'=\{z'_i\}_{i=1}^{N_1+N_2}$, denote 
\begin{align*}
    \cD_1=&~  \{z_1,\cdots, z_{N_1}\}, \qquad 
    \cD_2=\{z_{N_1+1},\cdots, z_{N_1+N_2}\}, \\
    \cD_1'=&~  \{z_1',\cdots, z_{N_1}'\}, \qquad 
    \cD_2'=\{z_{N_1+1}',\cdots, z_{N_1+N_2}'\}.
\end{align*}
Assume that $\cD$ and $\cD'$ differ at index $t\in[N_1+N_2]$, i.e., $z_j=z'_j $ for $j\neq t$. We consider two cases.

\paragraph{Case 1: $t\leq N_1$}
Because $\alg_1$ preserves \aJDP, we have
\begin{align*}
    \alg_1(E_1|\cD_1)\leq \ea \alg_1(E_1|\cD_1)+\beta, \qquad \forall E_1\subseteq \Pi_1,
\end{align*}
and hence, equivalently, it holds that
\begin{align*}
    \sum_{\pi_1\in\Pi_1} \brac{ \alg_1(\pi_1|\cD_1)-\ea \alg_1(\pi_1|\cD_1') }_+ \leq \beta.
\end{align*}
Note that for any $E\subseteq \Pi$,
\begin{align*}
    \alg(E|\cD)=\sum_{\pi_1\in\Pi_1} \alg_1(\pi_1|\cD_1) \cdot \alg_2(E|\pi_1;\cD_2),
\end{align*}
and therefore,
\begin{align*}
    \alg(E|\cD)-\ea\alg(E|\cD')=&~\sum_{\pi_1\in\Pi_1} \brac{ \alg_1(\pi_1|\cD_1) - \ea \alg_1(\pi_1|\cD_1') }\cdot \alg_2(E|\pi_1;\cD_2) \\
    \leq&~ \sum_{\pi_1\in\Pi_1} \brac{ \alg_1(\pi_1|\cD_1) - \ea \alg_1(\pi_1|\cD_1') }_+\leq \beta,
\end{align*}
where we use the fact that $\alg_2(E|\pi_1;\cD_2)\in[0,1]$ for any $\pi_1\in\Pi_1, E\subseteq \Pi$.

\paragraph{Case 2: $t>N_1$}
In this case, because $\alg_2(\pi_1;\cdot)$ preserves \aJDP~for any $\pi_1\in\Pi_1$, for any $E\subseteq \Pi$, we have
\begin{align*}
    \alg(E|\cD)=&~\sum_{\pi_1\in\Pi_1} \alg_1(\pi_1|\cD_1) \cdot \alg_2(E|\pi_1;\cD_2) \\
    \leq&~\sum_{\pi_1\in\Pi_1} \alg_1(\pi_1|\cD_1) \cdot \brac{ \ea\alg_2(E|\pi_1;\cD_2')+\beta }\\
    =&~ \ea \alg(E|\cD')+\beta.
\end{align*}

Combining the inequalities above from both cases completes the proof.
\qed




\subsection{Proof of \cref{prop:alg-U-JDP}}\label{appdx:proof-U-JDP}


The following lemma is a standard concentration result (following from taking union bounds with \cref{lem:Gaussian-concen} and \cref{lem:vec-Hoeffding}). 

\begin{lemma}\label{lem:spectral-concen-JDP}
In \cref{alg:U-JDP}, 
\whp, the following inequalities hold simultaneously for all $k=0,\cdots,K-1$:
\begin{align}
    \nrmF{H\kk-\Ep{\usqx[U\kk]}}\leq C_0\sqrt\frac{\log(K/\delta)}{N}, \\
    \nrmop{\til H\kk- H\kk}\leq C_0\frac{\siga \sqrt{d+\log(K/\delta)}}{N}, 
\end{align}
where $C_0$ is a large absolute constant. In the following, we denote this event as $\cE$.
\end{lemma}


Therefore, we can simplify the iterations in \cref{alg:U-JDP} as follows: $U\kz=\id$, and for $k=0,1,\cdots,K$:
\begin{align}
    \til H\kk=&~\Ep{ \usqx[U\kk] }+E\kk, \label{eq:spec-update-H}\\
    \cov\kk=&~U\kk\sq \til H\kk U\kk\sq+\lambda\kk U\kk, \label{eq:spec-update-cov}\\
    U\kp=&~\sym(\cov\kk\isq U\kk), \label{eq:spec-update-U}
\end{align}
where $\sym(A)=(A\tp A)\sq$, $E\kk$ is a symmetric matrix.
We note that \cref{alg:U-JDP} does not actually compute $(H\kc, \cov\kc)$, and they only appear in our analysis (where we can regard $E\kc=0$).

\begin{proposition}\label{prop:spec-converge}
Suppose that the sequence of matrices $\sset{ (U\kk, H\kk, \cov\kk) }$ is defined recursively by \eqref{eq:spec-update-H} - \cref{eq:spec-update-U}, with $\nrmop{E\kk}\leq \eps$. Suppose that $\lambda\kk=(2k+5)\eps$, and $\eps\leq 0.1$. Then, for any $k\geq 1$, it holds that
\begin{align*}
    \lmin(\cov\kk) \geq \exp\paren{ -\frac{\log(1/\eps)}{2^{k}} }, \qquad \lmax(\cov\kk)\leq \exp\paren{ \frac{12}{k} }.
\end{align*}
In particular, $K\geq \log\log(1/\eps)\vee 20$, we have
\begin{align*}
    \frac12\id\preceq \Ep{ \uxxu[U\kc] }+\lambda\kc U\kc \preceq 2\id.
\end{align*}
\end{proposition}

\paragraph{Proof of \cref{prop:alg-U-JDP}}
By \cref{lem:spectral-concen-JDP}, under the event $\cE$, the matrix $E\kk=\til H\kk -\Ep{ \usqx[U\kk] }$ is bounded as $\nrmop{E\kk}\leq \epsN$. 
Therefore, under $\cE$, we can apply \cref{prop:spec-converge}, which gives the desired results.
\qed



\subsubsection{Proof of \cref{lem:spectral-concen-JDP}}
In \cref{alg:U-JDP}, the matrix $\til H\kk$ is given by $\til H\kk=H\kk+Z\kk$,
\begin{align*}
    H\kk=\frac1N\sumkn \Phi_t, \qquad \Phi_t=\usqx[U\kk][t],
\end{align*}
where $Z\kk$ has i.i.d entries $Z_{ij}=Z_{ji}\sim \normal{0,\frac{4\siga^2}{N}}$, and $\Phi_{kN+1},\cdots, \Phi_{(k+1)N}$ are independent (conditional on $U\kk$). Further, by \cref{lem:uxxu}, we have $\nrmF{\Phi_t}\leq 1$, and $\EE[\Phi_t|U\kk]=\Ep{\usqx[U\kk][]}$. Therefore, using \cref{lem:vec-Hoeffding}, we have \whp,
\begin{align*}
    \nrmF{H\kk-\Ep{\usqx[U\kk][]}}\leq \frac{1+\sqrt{2\log(1/\delta)}}{\sqrt{N}}
\end{align*}
By \cref{lem:Gaussian-concen}, we also have $\nrmop{Z\kk}\leq C\frac{\siga\sqrt{d+\log(1/\delta)}}{N}$ \whp. Taking the union bound over $k=0,1,\cdots,K-1$ and rescaling $\delta\leftarrow \frac{\delta}{2K}$ completes the proof.
\qed


\subsubsection{Proof of \cref{prop:spec-converge}}\label{appdx:proof-spec-converge}
By definition \cref{eq:spec-update-U}, for each $k\geq 1$, there exists an orthogonal matrix $V\kp$ such that $U\kp=V\kp\cov\kk\isq U\kk$. Therefore,
\begin{align*}
    \id=&~V\kp V\kp\tp
    =V\kp \cov\kk\isq \paren{ U\kk\sq \til H\kk U\kk\sq+\lambda\kk U\kk }\cov\kk\isq V\kp\tp \\
    =&~\Ep{ \frac{V\kp\cov\kk\isq U\kk \x\x\tp U\kk \cov\kk\isq V\kp\tp}{\nrm{U\kk\x}} }+V\kp \cov\kk\isq U\kk\sq\paren{ E\kk+\lambda\kk\id }U\kk\sq \cov\kk\isq V\kp\tp \\
    =&~ \Ep{ \frac{U\kp \x\x\tp U\kp}{\nrm{U\kk\x}} }+U\kp U\kk\isq \paren{ E\kk+\lambda\kk\id }U\kk\isq U\kp.
\end{align*}
Therefore, using $-\eps\id\preceq E\kk\preceq \eps\id$ (because $\nrmop{E\kk}\leq \eps$),
\begin{align*}
    \id-(\lambda\kk+\eps)U\kp U\kk^{-1}U\kp \preceq \Ep{ \frac{U\kp \x\x\tp U\kp}{\nrm{U\kk\x}} } \preceq \id-(\lambda\kk-\eps)U\kp U\kk^{-1}U\kp.
\end{align*}
Notice that for any $\x\in\R^d$, it holds that $\nrm{U\kp\x}=\nrm{\cov\kk\isq U\kk\x}$, and hence we have
\begin{align*}
    \sqrt{ \lmin(\cov\kk) }\cdot \Ep{ \frac{U\kp \x\x\tp U\kp}{\nrm{U\kk\x}} } \preceq
    \Ep{ \frac{U\kp \x\x\tp U\kp}{\nrm{U\kp\x}} } \preceq \sqrt{ \lmax(\cov\kk) }\cdot \Ep{ \frac{U\kp \x\x\tp U\kp}{\nrm{U\kk\x}} }.
\end{align*}
Now, combining the inequalities above and using the definition of $\cov\kp$, we can lower bound
\begin{align*}
    \cov\kp=&~\Ep{ \frac{U\kp \x\x\tp U\kp}{\nrm{U\kp\x}} }+U\kp\sq E\kp U\kp\sq+\lambda\kp U\kp \\
    \succeq&~ \sqrt{ \lmin(\cov\kk) }\cdot \Ep{ \frac{U\kp \x\x\tp U\kp}{\nrm{U\kk\x}} } +(\lambda\kp-\eps) U\kp \\
    \succeq&~ \sqrt{ \lmin(\cov\kk) }\paren{ \id-(\lambda\kk+\eps)U\kp U\kk^{-1}U\kp } +(\lambda\kp-\eps) U\kp.
\end{align*}
Notice that $U\kp^2=U\kk\cov\kk^{-1}U\kk\preceq \frac{1}{\lmin(\cov\kk)} U\kk^2$, and the matrix function $U\mapsto -U\isq$ is matrix monotone (cf. \cref{lem:matrix-monotone}), and hence we have
\begin{align}\label{eq:proof-U-iv-k}
    \frac{1}{\sqrt{\lmax(\cov\kk)}} U\kp^{-1} \preceq  U\kk^{-1}\preceq \frac{1}{\sqrt{\lmin(\cov\kk)}} U\kp^{-1}.
\end{align}
In particular, we have proven $\cov\kp\succeq \sqrt{ \lmin(\cov\kk) }\id$, which implies
\begin{align*}
    \lmin(\cov\kk)\geq \lmin(\cov\kz)^{\frac{1}{2^k}}\geq \exp\paren{ -\frac{\log(1/\eps)}{2^k} },
\end{align*}
where we use $\lmin(\cov\kz)\geq \lambda\kz-\eps\geq \eps$.

Similarly,
\begin{align*}
    \cov\kp
    \preceq &~ \sqrt{ \lmax(\cov\kk) }\cdot \Ep{ \frac{U\kp \x\x\tp U\kp}{\nrm{U\kk\x}} } +(\lambda\kp+\eps) U\kp \\
    \preceq &~ \sqrt{ \lmax(\cov\kk) }\paren{ \id-(\lambda\kk-\eps)U\kp U\kk^{-1}U\kp } +(\lambda\kp+\eps) U\kp \\
    \preceq &~ \sqrt{ \lmax(\cov\kk) }\id+4\eps U\kp.
\end{align*}
Note that we have also shown that $\cov\kp\succeq (\lambda\kp-\eps)U\kp$, and hence
\begin{align*}
    U\kp\preceq \frac{1}{\lambda\kp-5\eps}\sqrt{ \lmax(\cov\kk) }\id, \qquad
    \cov\kp\preceq \paren{ 1+ \frac{4\eps}{\lambda\kp-5\eps}}\sqrt{ \lmax(\cov\kk) }\id.
\end{align*}
Therefore, it holds that
\begin{align*}
    \log\lmax(\cov\kp)\leq&~ \frac{4\eps}{\lambda\kp-5\eps}+\frac{1}{2}\log\lmax(\cov\kk)\\
    \leq&~ \sum_{j=0}^{k} \frac{1}{2^j}\cdot \frac{4\eps}{\lambda_{k+1-j}-5\eps}+\frac{\log\lmax(\cov\kz)}{2^{k+1}}.
\end{align*}
Note that $\lmax(\cov\kz)\leq 1+\eps+\lambda\kz$, and we also have
\begin{align*}
    \frac{1}{2^{k+1}}+\sum_{j=0}^{k} \frac{1}{2^j}\cdot \frac{1}{k+1-j}\leq \frac{1}{2^{k+1}}+\frac{2}{k+1}+\sum_{j=0}^{k} \frac{1}{2^j}\cdot \frac{j}{(k+1)(k+1-j)} \leq \frac{6}{k+1}.
\end{align*}
Therefore, we have proven that as long as $\eps\in[0,0.1]$,
\begin{align*}
    \log\lmax(\cov\kp)\leq&~ \frac{12}{k+1}.
\end{align*}
This is the desired result.
\qed



\subsubsection{Tighter rate with a different parameter schedule}

In the following, we show that we can in fact choose $\lambda\kk\asymp \frac{\sqrt{d}\siga}{N}$ in \cref{alg:U-JDP}. This result is useful for getting the refined regret bound in \cref{thm:regret-upper-JDP-better} (cf. \cref{appdx:proof-regret-upper-JDP-better}).


Specifically, our analysis is based on the following concentration result. Its proof is essentially the same as the proof of \cref{lem:spectral-concen-JDP}, except that we apply \cref{lem:cov-concen}.

\newcommand{\Hs}{H^\star}
\begin{lemma}\label{lem:U-JDP-concen-Bern}
Suppose that the sequence $\set{(U\kk, \til H\kk)}$ is generated by \cref{alg:U-JDP}. For each $k=0,1,\cdots,K-1$, we define
\begin{align*}
    \Hs\kk\defeq \Ep{\usqx[U\kk]}.
\end{align*}
Then, for any fixed parameter $c>1$, \whp, the following inequality holds for all $k=0,\cdots,K-1$:
\begin{align}\label{eq:U-JDP-concen-Bern}
    c\iv \Hs\kk-\epsN\id\preceq \til H\kk\preceq c\Hs\kk+\epsN\id,
\end{align}
where $\epsN=C_0\paren{\frac{\log(dK/\delta)}{(c-1)N}+\frac{\siga\sqrt{d+\log(K/\delta)}}{N}}$, and $C_0$ is a large absolute constant. In the following, we denote this event as $\cE$ and condition on $\cE$.
\end{lemma}

Therefore, following the analysis from \cref{prop:spec-converge}, we prove the following result.

\begin{proposition}\label{prop:spec-converge-JDP}
Let $c>1$ be a constant. Suppose that \cref{alg:U-JDP} is instantiated with $\lambda\kk=\lambda=\frac{c^2+1}{c^2-1}\epsN$, where $\epsN$ is defined in \cref{lem:U-JDP-concen-Bern}. Then, under the event $\cE$ of \cref{lem:U-JDP-concen-Bern}, for any $k\geq 1$, it holds that
\begin{align*}
    \lmin(\cov\kk) \geq c^{-4}\exp\paren{ -\frac{\log(1/\epsN)}{2^{k}} }, \qquad \lmax(\cov\kk)\leq c^4 \exp\paren{ \frac{\lambda\kz+\epsN}{2^{k}} }.
\end{align*}
In particular, for $c=1.1$, $\epsN\leq 0.1$, $K\geq \max\sset{\log\log(1/\epsN),10}$, we have
\begin{align*}
    \frac12\id\preceq \Ep{ \uxxu[U\kc] }+\lambda U\kc \preceq 2\id.
\end{align*}
\end{proposition}

\paragraph{Proof of \cref{prop:spec-converge-JDP}}
In the following proof, we abbreviate $\eps\defeq \epsN$.
Recall that we can simplify the iterations in \cref{alg:U-JDP} as follows: $U\kz=\id$, and for $k=0,1,\cdots,K-1$:
\begin{align*}
    \cov\kk=&~U\kk\sq \til H\kk U\kk\sq+\lambda\kk U\kk, \qquad
    U\kp=\sym(\cov\kk\isq U\kk),
\end{align*}
and we regard $\til H\kc=\Hs\kc$.

Then, for each $k\geq 1$, there exists an orthogonal matrix $V\kp$ such that $U\kp=V\kp\cov\kk\isq U\kk$, and hence
\begin{align*}
    \id=&~V\kp V\kp\tp
    =V\kp \cov\kk\isq \paren{ U\kk\sq \til H\kk U\kk\sq+\lambda\kk U\kk }\cov\kk\isq V\kp\tp \\
    =&~ U\kp U\kk\isq (\til H\kk+\lambda\kk\id)U\kk\isq U\kp.
\end{align*}
Using \eqref{eq:U-JDP-concen-Bern}, we have
\begin{align*}
    \id\preceq&~ U\kp U\kk\isq (c\Hs\kk+(\lambda\kk+\epsN)\id)U\kk\isq U\kp \\
    =&~c\cdot \Ep{ \frac{U\kp \x\x\tp U\kp}{\nrm{U\kk\x}} }+(\lambda\kk+\eps)U\kp U\kk\iv U\kp.
\end{align*}
Using \eqref{eq:U-JDP-concen-Bern} again, we can bound
\begin{align*}
    \cov\kp=&~ U\kp\sq \til H\kp U\kp\sq+\lambda\kp U\kp \\
    \succeq&~ c\iv U\kp\sq \Hs\kp U\kp\sq +(\lambda\kp-\eps) U\kp \\
    =&~ c\iv \Ep{ \frac{U\kp \x\x\tp U\kp}{\nrm{U\kp\x}} }+(\lambda\kp-\eps) U\kp \\
    \succeq&~ c\iv\sqrt{ \lmin(\cov\kk) }\cdot \Ep{ \frac{U\kp \x\x\tp U\kp}{\nrm{U\kk\x}} } +(\lambda\kp-\eps) U\kp \\
    \succeq&~ c^{-2} \sqrt{ \lmin(\cov\kk) }\paren{ \id-(\lambda\kk+\eps)U\kp U\kk^{-1}U\kp } +(\lambda\kp-\eps) U\kp \\
    \succeq&~ c^{-2} \sqrt{ \lmin(\cov\kk) },
\end{align*}
where the second inequality follows from $\nrm{U\kp\x}=\nrm{\cov\kk\isq U\kk\x}\leq \sqrt{\lmax(\cov\kk\isq)}\nrm{U\kk\x}=\frac{1}{\sqrt{\lmin(\cov\kk)}}\nrm{U\kk\x}$, and the last inequality uses \eqref{eq:proof-U-iv-k} and the fact that $\lambda\kp-\eps\geq c^{-2}(\lambda\kk+\eps)$. 

Similarly, we have
\begin{align*}
    \cov\kp
    \preceq&~ c U\kp\sq \Hs\kp U\kp\sq +(\lambda\kp+\eps) U\kp \\
    =&~ c\Ep{ \frac{U\kp \x\x\tp U\kp}{\nrm{U\kp\x}} }+(\lambda\kp+\eps) U\kp \\
    \preceq&~ c\sqrt{ \lmax(\cov\kk) }\cdot \Ep{ \frac{U\kp \x\x\tp U\kp}{\nrm{U\kk\x}} } +(\lambda\kp+\eps) U\kp \\
    \preceq&~ c^2 \sqrt{ \lmax(\cov\kk) }\paren{ \id-(\lambda\kk-\eps)U\kp U\kk^{-1}U\kp } +(\lambda\kp+\eps) U\kp \\
    \preceq&~ c^2 \sqrt{ \lmax(\cov\kk) }\id.
\end{align*}
where the last inequality uses \eqref{eq:proof-U-iv-k} and the fact that $\lambda\kp+\eps\leq c^2(\lambda\kk-\eps)$.

Therefore, we have shown
\begin{align*}
    c^{-2} \sqrt{ \lmin(\cov\kk) }\leq \lmin(\cov\kp)\leq \lmax(\cov\kp)\leq c^2 \sqrt{ \lmax(\cov\kk) }.
\end{align*}
Using this inequality recursively, we then have
\begin{align*}
    \lmin(\cov\kk)\geq c^{-4}\lmin(\cov\kz)^{\frac{1}{2^k}}, \qquad
    \lmax(\cov\kk)\leq c^{4}\lmax(\cov\kz)^{\frac{1}{2^k}}.
\end{align*}
The desired conclusion follows by recalling that we regard $\til H\kc=\Hs\kc$ and hence
\begin{align*}
    \cov\kc=\Ep{ \uxxu[U\kc] }+\lambda U\kc.
\end{align*}
\qed




\subsection{Details of \cref{alg:JDP-L1-regression}}\label{appdx:JDP-l1-regression}

In the following, we present the detailed description of the subroutine $\AlgJDPGD$ (\cref{alg:JDP-SGD}).

\begin{algorithm}[H]
\caption{Subroutine $\AlgJDPGD$: Batched SGD under JDP}\label{alg:JDP-SGD}
\begin{algorithmic}
\REQUIRE Loss $\Lnew$ under normalization $(U,\lambda)$, dataset $\dataset=\sset{(\x_t,y_t)}_{t\in[T]}$.
\REQUIRE Epoch $K\geq 1$, batch size $N=\floor{\frac{T}{K}}$, stepsize $\eta=\frac{1}{4\Lipg}$. %
\STATE Initialize $w\kz=\bz$.
\FOR{$k=0,1,\cdots,K-1$}
\STATE Compute gradient estimate
\begin{align*}
    g\kk=\lambda\mug \cdot  U w\kk+\frac1N \sumkn \frac{U\x_t}{\|U\x_t\|}\paren{ \link(\lr U\x_t, w\kk \rr) -y_t }.
\end{align*}
\STATE Privatize $\til g\kk\sim \priv[2/N]{g\kk}$ and update
\begin{align*}
    w\kp=\Proj_{\cW}\paren{ w\kk-\eta \til g\kk }.
\end{align*}
\ENDFOR
\ENSURE Estimator $\hth= U w\kc$.
\end{algorithmic}
\end{algorithm}

By the standard analysis of stochastic approximation, we provide the following guarantee of \cref{alg:JDP-SGD}.
\begin{proposition}[Convergence of batched SGD; JDP]\label{prop:JDP-GD}
Suppose that the input normalization $(U,\lambda)$ satisfies \eqref{def:U-app}, and $\lambda\geq \frac{1}{T}$. Then Algorithm $\AlgJDPGD$ preserves \aJDP~achieves \whp
\begin{align*}
    \nrm{w\kk-\hwst}^2\leq \paren{ 1-\frac{1}{8\kpg} }^k\nrm{\hwst}^2+\bigO{\frac{\log(K/\delta)}{\mug^2N}}+\bigO{\frac{1}{\mug^2}+\frac{d}{\mug\Lipg}}\cdot \frac{\siga^2\log(K/\delta)}{N^2},
\end{align*} 
where we recall that $\hwst=\argmin_{w\in\cW} \Lnew(w)$. 
In particular, when we take $K=16\kpg \log(T)$, we have
\begin{align}\label{eq:def:lamgd}
\begin{aligned}
\MoveEqLeft    \Lipg\cdot \nrm{w\kc-\hwst} \\
\leq&~ C_1\brac{ \kpg^{3/2}\sqrt{\frac{\log(T)\log(\log(T)/\delta)}{T}}+ \siga (\kpg^{3/2}+\kpg\sqrt{d})\frac{ \log(T)\sqrt{\log(\log(T)/\delta)}}{T} }=:\lamgd(T,\delta).
\end{aligned}
\end{align}
\end{proposition}

\paragraph{Proof of \cref{thm:JDP-L1-regression}}
Let the subroutine $\JDPLU$ be instantiated as in \cref{prop:alg-U-JDP}.
By combining \cref{prop:alg-U-JDP} and \cref{prop:JDP-GD}, we know that \whp[2\delta], normalization $(U,\lambda)$ satisfies \eqref{def:U-app}, and the estimator $\hw=\hw\kc$ satisfies $\Lipg\nrm{w\kc-\hwst}\leq \lamgd$. Applying \cref{lem:Lnew} gives \cref{thm:JDP-L1-regression} (2).
\qed

\subsection{Details of the LDP regression algorithm}\label{appdx:LDP-l1-regression}

In this section, we describe the details of $\AlgLDPRegression$ (\cref{alg:LDP-L1-regression}), a LDP analogue of \cref{alg:JDP-L1-regression}.

\begin{algorithm}[H]
\caption{$\AlgLDPRegression$ %
}\label{alg:LDP-L1-regression}
\begin{algorithmic}[1]
\REQUIRE Round $T\geq 1$, \errpara~$\delta\in(0,1)$.
\STATE Run the subroutine $\LDPLU(T/2,\delta)$ for the first $T/2$ rounds and receive $(U,\lambda)$.
\STATE Run the subroutine $\AlgLDPGD(\Lnew,T/2)$ for the remaining $T/2$ rounds and receive $\hw$ and the error bound $\lamgd$ (defined in \eqref{eq:def:lamgd-LDP}).
\ENSURE Normalization $(U,\lambda)$, estimator $\hth=U\hw$, and overall error $\lamall=\lambda+\lamgd$.
\end{algorithmic}
\end{algorithm}

In the following, we specifying the details of the subroutine $\LDPLU$ and $\AlgLDPGD$.


\paragraph{LDP learning normalization}
By adapting \cref{alg:U-JDP}, we can derive a similar algorithm for learning \um~under JDP. 


\begin{algorithm}%
\caption{Subroutine $\LDPLU$}\label{alg:U-LDP}
\begin{algorithmic}
\REQUIRE Round $T\geq 1$, \errpara~$\delta\in(0,1)$.
\REQUIRE Epoch $K\geq 1$, batch size $N=\floor{\frac{T}{K}}$.
\STATE Initialize $U\kz=\id$.
\FOR{$k=0,\cdots,K-1$}
    \FOR{$t=\rangekn$}
        \STATE Observe $\x_t$ and compute $V_t=\usqx[U\kk][t]$.
        \STATE Privatize $\til V_t\sim \sympriv[1]{V_t}$.    
    \ENDFOR
    \STATE Compute $\til H\kk=\frac1N\sumkn \til V_t$.
    \STATE Update
    \begin{align*}
        \cov\kk=U\kk\sq \til H\kk U\kk\sq+\lambda\kk U\kk, \qquad
        U\kp=\sym(\cov\kk\isq U\kk).
    \end{align*}
\ENDFOR
\ENSURE \Um~$U=U\kc$.
\end{algorithmic}
\end{algorithm}

Then, similar to \cref{prop:alg-U-JDP}, we have the following guarantee of \cref{alg:U-LDP}. 

\begin{proposition}\label{prop:alg-U-LDP}
Let $T\geq 1, K\geq 1$, $\delta\in(0,1)$, and $\epsN\defeq C_0\siga\sqrt{\frac{d+\log(K/\delta)}{N}}$, where $C_0$ is an absolute constant chosen according to \cref{lem:spectral-concen-LDP}. Suppose that \cref{alg:U-LDP} is instantiated with parameters $\lambda\kk=(2k+5)\epsN$, and then \whp,
\begin{align*}
    \exp\paren{ -\frac{\log(1/\lambda_0)}{2^{k-1}} }\id \preceq \Ep{ \uxxu[U\kk] }+\lambda\kk U\kk \preceq \exp\paren{ \frac{12}{k} }\id.
\end{align*}
In particular, as long as $K\geq \max\sset{\log\log(N),20}$, \cref{alg:U-LDP} output $(U,\lambda)$ satisfying \eqref{def:U-app} \whp, with
\begin{align*}
    \lambda=(2K+5)\epsN=\tbO{K}\cdot \siga \sqrt{\frac{d+\log(1/\delta)}{N}}.
\end{align*}
\end{proposition}

\paragraph{LDP batched SGD} Similarly, we use the following LDP batched SGD subroutine (\cref{alg:LDP-SGD}).


\begin{algorithm}[H]
\caption{Subroutine $\AlgLDPGD$}\label{alg:LDP-SGD}
\begin{algorithmic}
\REQUIRE Loss $\Lnew$ under normalization $(U,\lambda)$, round $T\geq 1$.
\REQUIRE Epoch $K\geq 1$, batch size $N=\floor{\frac{T}{K}}$, stepsize $\eta=\frac{1}{4\Lipg}$.
\FOR{$k=0,1,\cdots,K-1$}
\FOR{$t=\rangekn$}
\STATE Observe $(\x_t,y_t)$ and form gradient estimator
\begin{align*}
    g_t=\frac{U\x_t}{\|U\x_t\|}\paren{ \link(\lr U\x_t, w\kk \rr) -y_t }+\lambda\mug \cdot  U w\kk.
\end{align*}
\STATE Privatize $\til g_t\sim \priv[2]{g_t}$.
\ENDFOR
\STATE Compute $\til g\kk=\frac1N \sumkn \til g_t$ and update
\begin{align*}
    w\kp=\Proj_{\cW}\paren{ w\kk-\eta \til g\kk} .
\end{align*}
\ENDFOR
\ENSURE $\hw=w\kc$.
\end{algorithmic}
\end{algorithm}

\begin{proposition}\label{prop:LDP-GD}
Suppose that the input normalization $(U,\lambda)$ satisfies \eqref{def:U-app}. Then subroutine $\AlgLDPGD$ (\cref{alg:LDP-SGD}) preserves \aLDP~and
 achieves \whp
\begin{align*}
    \nrm{w\kk-\hwst}^2\leq \paren{ 1-\frac{1}{8\kpg} }^k\nrm{\hwst}^2+\bigO{\frac{1}{\mug^2}+\frac{d}{\mug\Lipg}}\cdot \frac{\siga^2\log(K/\delta)}{N}.
\end{align*} 
In particular, when $\lambda\geq \frac{1}{T}$ and we take $K=16\kpg \log T$, the output of \cref{alg:LDP-SGD} $\hw=w\kc$ satisfies
\begin{align}\label{eq:def:lamgd-LDP}
    \Lipg\cdot \nrm{\hw-\hwst}\leq C_1 \siga (\kpg^{3/2}+\kpg\sqrt{d}) \sqrt{\frac{\log(T)\log(\log(T)/\delta)}{T}}=:\lamgd(T,\delta).
\end{align}
\end{proposition}

\paragraph{Guarantees of $\AlgLDPRegression$} By combining \cref{prop:alg-U-LDP} and \cref{prop:LDP-GD}, we have the following result, as claimed in \cref{thm:LDP-L1-regression}.

\begin{theorem}\label{thm:LDP-L1-regression-full}
Let $T\geq 1, \delta\in(0,1)$, and the subroutines of \cref{alg:LDP-L1-regression} instantiated as in \cref{prop:alg-U-LDP} and \cref{prop:LDP-GD}. Then, \cref{alg:LDP-L1-regression} preserves \aLDP, and the following holds \whp[2\delta]:

(1) The normalization $(U,\lambda)$ satisfies \eqref{def:U-app}, and the estimator $\hw$ satisfies $\Lipg\nrm{\hw-\hwst}\leq \lamgd$, where
\begin{align*}
    \lambda=\lambda(T,\delta)=&~ \tbO{ \siga\sqrt{\frac{d\log(1/\delta)}{T}} }, \\
    \lamgd=\lamgd(T,\delta)=&~ \tbO{\siga (\kpg^{3/2}+\kpg\sqrt{d})\sqrt{\frac{\log(1/\delta)}{T}}},
\end{align*}
are defined in \cref{prop:alg-U-LDP} and \eqref{eq:def:lamgd-LDP}, and $\tbO{\cdot}$ hides polynomial factors of $\log(T)$. The overall error is defined as $\lamall(T,\delta)\defeq 4\Lipg\lambda(T,\delta)+\lamgd(T,\delta)$.

(2) By \cref{lem:Lnew}, the estimator $\hth=U\hw$ satisfies
\begin{align*}
    \errloneg{\hth}\leq \Lipg\cdot \errlone{\hth}\leq  2\sqrt{d} \lamall,
\end{align*}
and for all $\x\in\Rd$, we have the \emph{confidence bound} $\absn{ \nu(\lr \x, \hth \rr) - \nu(\lr \x,\ths \rr) }\leq \nrm{U\x} \lamall$.
\end{theorem}


\subsubsection{Proof of \cref{prop:alg-U-LDP}}


The following lemma can be proven by \cref{lem:Gaussian-concen} and \cref{lem:vec-Hoeffding}, similar to \cref{lem:spectral-concen-JDP}.
\begin{lemma}\label{lem:spectral-concen-LDP}
In \cref{alg:U-LDP}, \whp, the following inequality holds for all $k=0,\cdots,K-1$:
\begin{align}
    \nrmop{\til H\kk-\Ep{\usqx[U\kk]}}\leq C_0\siga\sqrt\frac{d+\log(K/\delta)}{N}=:\epsN,
\end{align}
where $C$ is a large absolute constant. In the following, we denote this event as $\cE$.
\end{lemma}
The proof of \cref{prop:alg-U-LDP} is then completed by combining \cref{lem:spectral-concen-LDP} and \cref{prop:spec-converge}.
\qed










\subsection{Proof of \cref{prop:JDP-GD} and \cref{prop:LDP-GD}}

In the following, we present the analysis of \cref{alg:JDP-SGD} and \cref{alg:LDP-SGD}. We first state the following standard convergence result of batched SGD on a strongly convex function.

\newcommand{\iind}[1]{\epk{#1}}
\newcommand{\us}{w^\star}
\begin{proposition}\label{prop:B-SGD-general}
Suppose that $\cW\subseteq \R^d$ is a closed, convex domain, and $F$ is a smooth convex function over $\cW$ such that $\mu\id\preceq \nabla^2 F(w)\preceq L\id$ for all $w\in\cW$. Consider the following iteration of approximate gradient descent:
\begin{align}\label{eq:B-SGD-general}
    w\iind{k+1}=\Proj_\cW\paren{w\iind{k}-\eta \paren{\nabla F(w\iind{k}) + Z\iind{k}} },
\end{align}
where $Z\iind{k}$ is the noise vector at step $k$ that is $\sigma$-sub-Gaussian (conditional on $w\iind{1:k}$), and the stepsize $\eta\in(0,\frac{1}{2L}]$. Then \whp, it holds that for all $k\in[K]$,
\begin{align*}
    \nrm{w\iind{k}-\us}^2\leq \paren{ 1-\frac{\eta\mu}{2} }^k\nrm{w\iind{0}-\us}^2+\frac{4\eta}{\mu}\max_{j<k}\nrm{Z\iind{j}}^2+\bigO{\frac{\sigma^2\log(K/\delta)}{\mu^2}},
\end{align*}
where $\us\defeq \argmin_{w\in\cW} F(w)$.
\end{proposition}

\paragraph{Proof of \cref{prop:JDP-GD}}
For the $k$th epoch of \cref{alg:JDP-SGD}, we denote
\begin{align*}
    g_t=\frac{U\x_t}{\|U\x_t\|}\paren{ \link(\lr U\x_t, w\kk \rr) -y_t }.
\end{align*}
Then, it is clear that $\EE[g_t|w\kk]=\nabla \Lnew(w\kk)-\lambda\mug\cdot Uw\kk$. 
Therefore, Algorithm $\AlgJDPGD$ (\cref{alg:JDP-SGD}) is an instantiation of \eqref{eq:B-SGD-general} with $F=\Lnew$, $w\iind{0}=0$, $\mu=\frac{\mug}{2}$, $L=2\Lipg$, $\eta=\frac{1}{2L}$. 

As we have $\nrm{g_t}\leq 2$, it is clear that \cref{alg:JDP-SGD} preserves \aJDP.
Furthermore, we can decompose
\begin{align*}
    Z\kk=g\kk-\nabla \Lnew(w\kk)=G\kk+\frac1N \sumkn (g_t-\EE[g_t|w\kk]),
\end{align*}
where $G\kk\in\Rdd$ has i.i.d entries drawn from $\normal{0, \frac{16\siga^2}{N^2}}$.
Therefore, using Hoeffding's bound, $Z\kk$ is $\sigma$-sub-Gaussian with $\sigma^2\leq \bigO{\frac{1}{N}+\frac{\siga^2}{N^2}}$. Using \cref{lem:Gaussian-concen} and \cref{lem:vec-Hoeffding}, we also have $\max_{k}\nrm{Z\iind{k}}^2\leq \bigO{\frac{1}{N}+\frac{d\siga^2}{N^2}}\log(K/\delta)$ \whp. Applying \cref{prop:B-SGD-general} gives the desired upper bounds.
\qed


\paragraph{Proof of \cref{prop:LDP-GD}}
Similarly, for the $k$th epoch of \cref{alg:JDP-SGD}, it is clear that $\nrm{g_t-\lambda\mug\cdot Uw\kk}\leq 2$ and $\EE[g_t|w\kk]=\nabla \Lnew(w\kk)$. Therefore, \cref{alg:JDP-SGD} preserves \aLDP, and it is also an instantiation of \eqref{eq:B-SGD-general} with $F=\Lnew$, $w\iind{0}=0$, $\mu=\frac{\mug}{2}$, $L=2\Lipg$, $\eta=\frac{1}{2L}$. 

To proceed, we decompose
\begin{align*}
    Z\kk=g\kk-\nabla \Lnew(w\kk)=\frac1N \sumkn \brac{ (g_t-\EE[g_t|w\kk])+G_t},
\end{align*}
where $(G_t\in\Rdd)_{t=\rangekn}$ are i.i.d Gaussian random vectors, and each $G_t$ has i.i.d entries drawn from $\normal{0, 16\siga^2}$. Therefore, we denote $G\kk=\frac1N \sumkn G_t$, and $G\kk$ has i.i.d entries drawn from $\normal{0, \frac{16\siga^2}{N} }$.
Therefore, using Hoeffding's bound, $Z\kk$ is $\sigma$-sub-Gaussian with $\sigma^2\leq \bigO{\frac{\siga^2}{N}}$. Using \cref{lem:Gaussian-concen} and \cref{lem:vec-Hoeffding}, we also have $\max_{k}\nrm{Z\iind{k}}^2\leq \bigO{\frac{d\siga^2\log(K/\delta)}{N}}$ \whp. Applying \cref{prop:B-SGD-general} gives the desired upper bounds.
\qed




\subsubsection{Proof of \cref{prop:B-SGD-general}}

By definition, we have $\us=\Proj_\cW(\us-\eta\nabla F(\us))$, and hence
\begin{align*}
    &~\nrm{w\iind{k+1}-\us}^2\\
    =&~
    \nrm{\Proj_\cW\paren{w\iind{k}-\eta \paren{\nabla F(w\iind{k}) + Z\iind{k}} }-\Proj_\cW\paren{\us-\eta\nabla F(\us)}}^2 \\
    \leq&~
    \nrm{\paren{w\iind{k}-\eta \paren{\nabla F(w\iind{k}) + Z\iind{k}}-\us}-\paren{\us-\nabla F(\us)}}^2 \\
    =&~ \nrm{w\iind{k}-\us}^2-2\eta\llr \nabla F(w\iind{k})-\nabla F(\us) + Z\iind{k}, w\iind{k}-\us \rrr +\eta^2\nrm{\nabla F(w\iind{k})-\nabla F(\us) + Z\iind{k}}^2.
\end{align*}
Thus, using the fact that $F$ is $\mu$-strongly-convex and $L$-smooth, we have
\begin{align*}
    \mu\nrm{w-\us}^2\leq \lr \nabla F(w)-\nabla F(\us), w-\us\rr \leq \frac{1}{L}\nrm{\nabla F(w)-\nabla F(\us)}^2.
\end{align*}
Hence, it holds that
\begin{align*}
    &~\nrm{w\iind{k+1}-\us}^2 \\
    \leq &~ \nrm{w\iind{k}-\us}^2-2\eta\llr \nabla F(w\iind{k})-\nabla F(\us) + Z\iind{k}, w\iind{k}-\us \rrr +2\eta^2\nrm{\nabla F(w\iind{k})-\nabla F(\us)}^2+2\eta^2 \nrm{Z\iind{k}}^2 \\
    \leq&~ \nrm{w\iind{k}-\us}^2-2\eta(1-\eta L)\llr \nabla F(w\iind{k})-\nabla F(\us), w\iind{k}-\us \rrr-2\eta \llr Z\iind{k}, w\iind{k}-\us \rrr +2\eta^2 \nrm{Z\iind{k}}^2 \\
    \leq&~ \paren{ 1-\eta\mu }\nrm{w\iind{k}-\us}^2-2\eta \llr Z\iind{k}, w\iind{k}-\us \rrr +2\eta^2 \nrm{Z\iind{k}}^2,
\end{align*}
where we use $\eta\leq \frac{1}{2L}$. 

In the following, we condition on the following event:
\begin{align*}
    \cE\defeq \sset{ \forall k\in[K]: \llr Z\iind{k}, w\iind{k}-\us \rrr\leq Z\nrm{w\iind{k}-\us} },
\end{align*}
where we denote $Z\defeq C\sigma\sqrt{\log(K/\delta)}$, $C$ is a large universal constant. By sub-Gaussian concentration, we know $\PP(\cE)\geq 1-\delta$.

Under $\cE$, we can bound
\begin{align*}
    \nrm{w\iind{k+1}-\us}^2\leq &~
    \paren{ 1-\eta\mu }\nrm{w\iind{k}-\us}^2+2\eta Z\nrm{ w\iind{k}-\us }+2\eta^2 \nrm{Z\iind{k}}^2 \\
    \leq&~ \paren{ 1-\frac{\eta\mu}{2} }\nrm{w\iind{k}-\us}^2 +2\frac{\eta}{\mu}Z^2+2\eta^2 \nrm{Z\iind{k}}^2, \qquad \forall k\in[K],
\end{align*}
where the last step is by Cauchy-Schwarz. Then, we know for any $k\in[K]$,
\begin{align*}
    \nrm{w\iind{k}-\us}^2
    \leq \paren{ 1-\frac{\eta\mu}{2} }^k\nrm{w\iind{0}-\us}^2+\frac{4}{\mu^2}Z^2+\frac{4\eta}{\mu}\max_{j<k}\nrm{Z\iind{j}}^2.
\end{align*}
This is the desired result.
\qed


\section{Proofs from \cref{sec:cb}}\label{appdx:CB}

\colt{
\subsection{Details of the Action Elimation Algorithm}\label{appdx:spanner}


We state our JDP algorithm for generalized linear contextual bandits as follows.

\begin{algorithm}
\begin{algorithmic}
\REQUIRE Round $T\geq 1$, parameter $\delta\in(0,1)$, epoch schedule $1=T\epj[0]<T\epj[1]<\cdots<T\epj[J]=T$.
\STATE Initialize $\hft[0]\equiv 0, \CIt[0]\equiv 1$.
\FOR{$j=0,1,\cdots,J-1$}
\STATE Set $\pit[j]\leftarrow\AlgPlan(\set{ (\hft,\CIt)  }_{\tau<j})$ and initialize the dataset $\dataset\epj=\sset{}$.
\FOR{$t=T\epj,\cdots,T\epj[j+1]-1$}
\STATE Observe context $x_t\sim P$, select action $a_t\sim \pit[j](x_t)$, and receive reward $r_t$. 
\STATE Update the dataset $\dataset\epj \leftarrow \dataset\epj \cup \sset{ (\phxa[x_t,a_t], r_t) }$.
\ENDFOR
\STATE Set $(U\epj, \lambda\epj), \hth\epj,\lamall\epj \leftarrow \AlgJDPRegression(\cD\epj,\delta)$.
\STATE Update the estimated reward function and confidence bound as
\begin{align*}
    \hft[j](x,a)=\nu(\lr \phxa, \hth\epj \rr), \qquad
    \CIt[j](x,a)=\lamall \epj \cdot \nrmn{U\epj \phxa}, \qquad \forall (x,a)\in\cX\times\cA.
\end{align*}
\ENDFOR
\end{algorithmic}
\caption{Action Elimination Algorithm with JDP}\label{alg:batch-cb-JDP}
\end{algorithm}


The batch elimination algorithm \cref{alg:batch-cb-JDP} contains a planning subroutine $\AlgPlan$. Before presenting $\AlgPlan$, we first introduce the notion of \emph{spanner}.

\paragraph{Spanner}
}
\arxiv{
\subsection{Spanner}\label{appdx:spanner}
}
In $\AlgPlan$ (\cref{alg:plan}), we utilize the following notion of the \emph{spanner} of a set of actions.
\begin{definition}\label{def:spanner}
Given a context $x\in\cX$ and a set $\cA_1$ of actions, a subset $\cA'\subseteq \cA_1$ is a \emph{spanner} of $\cA_1$ if for any $a\in\cA$, there exists weights $(\gamma_{a'}\in[-1,1])_{a'\in\cA'}$ such that
\begin{align*}
    \phxa=\sum_{a'\in\cA'} \gamma_{a'} \phxa[x,a'].
\end{align*}
\end{definition}

It is well-known that a spanner with size bounded by the dimension exists, known as the \emph{barycentric spanner}~\citep{awerbuch2008online}.
\begin{lemma}[Barycentric spanner]\label{lem:bary-spanner}
For any context $x\in\cX$ and set $\cA_1$, there exists a spanner of size
\begin{align*}
    \dim(\cA_1,x)\defeq \dim(\sset{a\in\cA_1: \phxa}).
\end{align*}
\end{lemma}
An approximate barycentric spanner can be computed in time $\poly(d,|\cA_1|)$. Further, given a linear optimization oracle over the set $\sset{a\in\cA_1: \phxa}$, the time complexity can further be reduced to $\poly(d)$~\citep{hazan2016volumetric,perchet2016batched}.

\colt{
\paragraph{Planning subroutine}
We state the subroutine $\AlgPlan$ as follows. 
\begin{algorithm}
\begin{algorithmic}
\REQUIRE Confidence interval descriptions $(\hft[0], \CIt[0]),\cdots,(\hft[j-1],\CIt[j-1])$ up to the $j$th epoch.
\FOR{context $x\in\cX$}
\STATE Set $\cAxt[0]=\cA$.
\FOR{$\tau=1,\cdots,j-1$}
\STATE Set
\begin{align}\label{eq:eliminate}
    \cAxt=\sset{ a\in\cAxt[\tau-1]: \hft(x,a)+\CIt(x,a)\geq \max_{a'\in\cA}\hft(x,a')-\CIt(x,a') }.
\end{align}
\ENDFOR
\STATE Compute a spanner $\cAsp[j]$ of $\cAxt[j-1]$ (\cref{def:spanner}).
\STATE Set $\pit[j](x)=\Unif(\cAsp[j])$.
\ENDFOR
\ENSURE Policy $\pit[j]$.
\end{algorithmic}
\caption{Subroutine $\AlgPlan$ %
}\label{alg:plan}
\end{algorithm}

}






\subsection{Proof of \cref{thm:regret-upper-JDP}}\label{appdx:regret-meta}




\newcommandx{\Epi}[2][1=\pi]{\EE^{#1}\brac{#2}}
\newcommand{\Px}[1]{\PP_{x\sim \mu}\paren{#1}}

To provide a unified analysis framework for \cref{alg:batch-cb-JDP} with different private regression subroutines, we first present a general action elimiation algorithm (\cref{alg:batch-cb-meta}) that additionally takes an $\Lone$-regression subroutine $\AlgCIEst$ as input.



\begin{algorithm}
\begin{algorithmic}
\REQUIRE Round $T\geq 1$, epoch schedule $1=T_0<T_1<T_2<\cdots<T_{J}=T$.
\REQUIRE Subroutine $\AlgCIEst$.
\STATE Initialize $\hft[0]\equiv 0, \CIt[0]\equiv 1$.
\FOR{$j=0,1,\cdots,J-1$}
\STATE Set $\pit[j]\leftarrow\AlgPlan(\set{ (\hft,\CIt)  }_{\tau<j})$.
\STATE Initialize the subroutine $\AlgCIEst\epj$ with round $N\epj=T\epj[j+1]-T\epj$.
\FOR{$t=T_j,\cdots,T_{j+1}-1$}
\STATE Receive context $x_t$, take action $a_t\sim \pi\epj(x_t)$, and receive reward $r_t$.
\STATE Feed $(\phxa[x_t,a_t],r_t)$ into $\AlgCIEst\epj$.
\ENDFOR
\STATE Receive estimation $(\hft[j],\CIt[j])$ from $\AlgCIEst\epj$.
\ENDFOR
\end{algorithmic}
\caption{Meta Batch Elimination Algorithm}\label{alg:batch-cb-meta}
\end{algorithm}

Similar to our argument in \cref{appdx:JDP-verify}, we can show that \cref{alg:batch-cb-meta} preserves \aJDP~(\aLDP) if the subroutine $\AlgCIEst$ preserves \aJDP~(\aLDP).
Furthermore, we state the regret guarantee of \cref{alg:batch-cb-meta} under the following assumption on the subroutine $\AlgCIEst$.

\begin{assumption}\label{asmp:EstCI}
For each $j$, the subroutine $\AlgCIEst$ returns $(\hft[j],\CIt[j])$ such that the following holds \whp.

(1) The function $\CIt[j](x,a)$ provides a valid confidence bound:
\begin{align*}
    \Px{ \forall a\in\cA, \abs{ \hft(x,a)-\fs(x,a) }\leq \CIt(x,a) }\geq 1-\delta_0.
\end{align*}

(2) The function $\CIt[j](x,a)=b\epj\paren{\phxa}$ is given by a norm function $b\epj$ over $\Rd$.
\end{assumption}


\begin{theorem}[Meta regret guarantee]\label{thm:regret-upper-meta}
Under \cref{asmp:EstCI}, \cref{alg:batch-cb-meta} ensures that
\begin{align*}
    \Reg\leq \EE\brac{ \sum_{j=0}^{J-2} 4\dA\cdot N\epj[j+1] \Epi[{\pit[j]}]{ \CIt[j](x,a) } } + 2N\epj[0] + 2TJ\delta+ 2T^2\delta_0,
\end{align*}
where $N\epj=T\epj[j+1]-T\epj$ is the batch size of the $j$th epoch. Further, if the subroutine $\AlgCIEst$ preserves \aJDP~(or correspondingly \aLDP), then \cref{alg:batch-cb-meta} preserves \aJDP~(or correspondingly \aLDP).
\end{theorem}

With \cref{thm:regret-upper-meta}, we can prove \cref{thm:regret-upper-JDP} easily from the guarantee of subroutine $\AlgJDPRegression$ (\cref{thm:JDP-L1-regression}).
\paragraph{Proof of \cref{thm:regret-upper-JDP}}
To see how \cref{alg:batch-cb-meta} generalizes \cref{alg:batch-cb-JDP}, we consider instantiate it with the subroutine $\AlgCIEst$ be specified by $\AlgJDPRegression$ (\cref{alg:JDP-L1-regression}), and for the output $(U\epj,\lambda\epj), \hth\epj, \lamall\epj$ of the instance $\AlgJDPRegression\epj$, we set
\begin{align*}
    \hft[j](x,a)=\nu(\lr \phxa, \hth\epj \rr), \qquad
    \CIt[j](x,a)=\lamall \epj \nrm{U\epj \phxa}, \qquad \forall (x,a)\in\cX\times\cA,
\end{align*}
where $\lamall\epj=\lamall(N\epj, \delta)$ is defined in \cref{thm:JDP-L1-regression}.
Then, it is clear that under these specifications, \cref{alg:batch-cb-meta} agrees with \cref{alg:batch-cb-JDP}. 

Further, we have $\Epi[{\pit[j]}]{ \CIt[j](x,a) }\leq 2d\lamall\epj$, and \cref{asmp:EstCI} holds with $\delta_0=0$, because under the success event of \cref{thm:JDP-L1-regression}, we have
\begin{align*}
    \abs{ \hft[j](x,a)-\fs(x,a) }\leq \CIt[j](x,a), \qquad \forall x\in\cX, a\in\cA.
\end{align*}
Then, \cref{thm:regret-upper-meta} yields (with $\delta=\frac1T$)
\begin{align*}
    \Reg\leqsim&~ \dA d\sum_{j=0}^{J-2} N\epj[j+1] \lamall\epj +N\epj[0] \\
    \leq&~ \poly(\log T)\cdot  \dA d\sum_{j=0}^{J-2} \paren{ \kpg^{3/2} \frac{N\epj[j+1]}{\sqrt{N\epj}}+\siga(\kpg^{3/2}+\kpg\sqrt{d})\frac{N\epj[j+1]}{N\epj} } + N\epj[0].
\end{align*}
In particular, under the choice $T\epj=2^{j}$, we have 
\begin{align*}
    \Reg\leq \tbO{\dA d \kpg^{3/2} \sqrt{T} + \siga \dA d(\kpg^{3/2}+\kpg\sqrt{d})}.
\end{align*}
\qed

\subsubsection{Proof of \cref{thm:regret-upper-meta}}
For any policy $\pi:\cX\to\Delta(\cA)$, we define its sub-optimality as
\begin{align*}
    \reg(\pi)=\EE_{x\sim P, a\sim \pi(x)}\brac{ \fs(x,\pis(x))- \fs(x,a) },
\end{align*}
where we recall that $\pis(x)\defeq \argmax_{a\in\cA} \fs(x,a)$ is the optimal policy under $\fs$.
Then, by definition, for \cref{alg:batch-cb-meta},
\begin{align*}
    \Reg=\EE\brac{ \sum_{j=0}^{J-1} \sum_{t=T\epj}^{T\epj[j+1]-1} \reg(\pi\epj)}
    =\EE\brac{ \sum_{j=0}^{J-1} N\epj \cdot \reg(\pi\epj)}.
\end{align*}

In the following, we work with the following quantity:
\begin{align*}
    \Reg^+\defeq \sum_{j=0}^{J-1} N\epj \cdot \reg(\pi\epj),
\end{align*}
which is a random variable measuring the cumulative sub-optimality of the algorithm.


We assume the success event of \cref{asmp:EstCI} and define
\begin{align*}
    \cX\epj\defeq \sset{ x\in\cX: \forall \tau\leq j, a\in\cA,  \abs{ \hft(x,a)-\fs(x,a) }\leq \CIt(x,a) }.
\end{align*}
Then, for any $x\in\cX\epj$ and $\tau\leq j$, we have
\begin{align*}
    \hft(x,\pis(x))+\CIt(x,\pis(x))\geq \fs(x,\pis(x))
    =\max_{a\in\cA} \fs(x,a)
    \geq \max_{a\in\cA} \hft(x,a)-\CIt(x,a),
\end{align*}
and hence
$\pis(x)\in\cAxt[j]$. Further, for any $a\in\cAxt[j]$,
\begin{align*}
    \hft[j](x,a)+\CIt[j](x,a)\geq \hft[j](x,\pis(x))-\CIt[j](x,\pis(x)),
\end{align*}
and hence
\begin{align*}
    \fs(x,\pis(x))-\fs(x,a)
    \leq&~ \hft[j](x,\pis(x))+\CIt[j](x,\pis(x))-\hft[j](x,a)+\CIt[j](x,a)\\
    \leq&~ 2\CIt[j](x,\pis(x))+2\CIt[j](x,a) \\
    \leq&~ 4\max_{a'\in\cAxt[j]}\CIt[j](x,a'), \qquad \forall x\in\cX\epj, a\in\cAxt[j].
\end{align*}
Note that $\pi\epj[j+1](x)$ is supported on $\cAxt[j]$, and hence
\begin{align*}
    \reg(\pi\epj[j+1])=\EE_{x\sim P, a\sim \pi\epj[j+1](x)}\brac{ \fs(x,\pis(x))- \fs(x,a) }
    \leq 4\EE_{x\sim P} \max_{a\in\cAxt[j]}\CIt[j](x,a).
\end{align*}
Further, $\cAsp[j]$ is a barycentric spanner of $\cAxt[j-1]$, and hence for any $a\in\cAxt[j]\subseteq \cAxt[j-1]$, there exists parameters $(\gamma_{x,a,a'}\in[-1,1])_{a'\in\cAsp[j]}$, such that
\begin{align*}
    \phxa=\sum_{a'\in\cAsp[j]} \gamma_{x,a,a'} \phxa[x,a'].
\end{align*}
Hence, by \cref{asmp:EstCI} (2), for any $x\in\cX\epj$, $a\in\cAxt[j]$,
\begin{align*}
    \CIt[j](x,a)=&~ b\epj\paren{ \sum_{a'\in\cAsp[j]} \gamma_{x,a,a'} \phxa[x,a'] } 
    \leq \sum_{a'\in\cAsp[j]} \abs{\gamma_{x,a,a'}} b\epj(\phxa[x,a']) \\
    \leq&~ \sum_{a'\in\cAsp[j]} b\epj(\phxa[x,a'])
    = |\cAsp[j]|\cdot \EE_{a'\sim \pi\epj(x)} \CIt[j](x,a').
\end{align*}
Therefore, for any $x\in\cX\epj$,
\begin{align*}
    \max_{a\in\cAxt[j]}\CIt[j](x,a)\leq \dA \cdot \EE_{a'\sim \pi\epj(x)} \CIt[j](x,a'),
\end{align*}
and thus,
\begin{align*}
    \reg(\pit[j+1])
    \leq 4\dA\cdot \EE_{x\sim P, a\sim \pi\epj(x) }{ \CIt[j](x,a) } + 2P(x\not\in\cX\epj).
\end{align*}
By \cref{asmp:EstCI}, $P(x\not\in\cX\epj)\leq T\delta_0$, and hence taking summation over $j=0,1,\cdots,J-2$ gives
\begin{align*}
    \sum_{j=1}^{J-1} N\epj \cdot \reg(\pi\epj)
    \leq 4\dA\sum_{j=0}^{J-2} N\epj[j+1] \cdot \EE^{\pi\epj[j]}\brac{ \CIt[j](x,a) } +2T^2 \delta_0.
\end{align*}
Note that the above inequality holds under the success event of \cref{asmp:EstCI}, which holds with probability at least $1-J\delta$. Then taking expectation gives the desired upper bound on $\Reg$.
\qed

We also remark that a high-probability upper bound on the regret follows similarly (with an extra step of applying martingale concentration).


\subsection{Proof of \cref{thm:regret-upper-JDP-better}}\label{appdx:proof-regret-upper-JDP-better}

\newcommand{\AlgJDPLR}{\mathsf{JDP\_Reweighted\_Linear\_Regression}}
\newcommand{\delz}{\delta_0}
\newcommand{\lamz}{\lambda_0}

In this section, we present an adaption of \cref{alg:batch-cb-JDP} based on the subroutine $\AlgJDPLR$ (\cref{alg:linear-JDP-better}), which provides a tighter rate of convergence.

\newcommand{\gu}[1]{\frac{U#1}{\|U#1\|}}
\newcommand{\sumnt}{\sum_{t=N+1}^T}
\newcommand{\avgtt}{\frac{1}{N}\sum_{t=N+1}^T}
\newcommand{\sumtt}{\sum_{t=N+1}^T}
\renewcommand{\zt}{\zeta_t}


\begin{algorithm}
\caption{Subroutine $\AlgJDPLR$}\label{alg:linear-JDP-better}
\begin{algorithmic}[1]
\REQUIRE Dataset $\dataset=\sset{(\x_t,y_t)}_{t\in[T]}$ with size $T=2N$, \errpara~$\delta\in(0,1)$, $\delz\in(0,1)$.
\STATE Split $\dataset=\dataset_0\cup \dataset_1$ equally.
\STATE Set $(U,\lambda)\leftarrow \JDPLU(\dataset_0,\delta/2)$
\STATE Compute the following estimates on $\dataset_1$:
\begin{align*}
    \xi=\avgtt \gu{\x_t}y_t, \qquad \Xi=\avgtt \gu{\x_t}\x_t\tp+\lambda\id, \qquad 
    W=\avgtt \frac{U\x_t \x_t\tp U}{\|U\x_t\|^2}+\lambda\id.
\end{align*}
\STATE Privatize $[\til \xi; \til \Xi] \sim \priv[3/N]{ [\xi; \Xi] }$ and $\til W\sim \sympriv[3/N]{W}$.
\ENSURE Estimators $\hth={\til \Xi}\iv{\til \xi}$ and confidence bound
\begin{align*}
    \CI(\x)=16\lambda\nrmn{U\x}+\lamgd\nrmn{{\til \Xi}\itp \x}_{\til W}.
\end{align*}
\end{algorithmic}
\end{algorithm}

The subroutine $\AlgJDPLR$ utilizes the linear structure of the linear models. Instead of running a batched SGD procedure, it directly computes an estimator of $\ths$ by solving a privatized linear equation $\til \Xi \hth=\til \xi$. It is clear that \cref{alg:linear-JDP-better} preserves \aJDP, and we also show that the confidence bound $\abs{\lr \x, \hth-\ths\rr}\leq \CI(\x)$ holds true in a \emph{distributional} sense.

\begin{proposition}\label{prop:linear-JDP-better}
Let $T\geq 1$, $\delta\in(0,1)$, $P'$ be a fixed distribution over $\Bone$, and the subroutine $\JDPLU$ of \cref{alg:linear-JDP-better} be instantiated as in \cref{prop:spec-converge-JDP}. Then \cref{alg:linear-JDP-better} preserves \aJDP~and ensures the following holds \whp:

(1) The normalization $(U,\lambda)$ satisfies \eqref{def:U-app}, with $\lambda$ given by
\begin{align}\label{eq:def:lambda-JDP-better}
    \lambda\defeq \lambda(T,\delta)=C\frac{\siga\sqrt{d+\log(K/\delta)}}{T},
\end{align}
where $C$ is a large absolute constant. 

(2) With $\lamgd\defeq \lamgd(T,\delta)=C_0 \sqrt{\frac{\log(1/(\delta\delz)}{T}}$ for a large absolute constant $C_0$, it holds that 
\begin{align*}
    \PP_{\x'\sim P'}\paren{ \absn{ \lr \x', \hth-\ths \rr }\geq \CI(\x') }\leq \delz,
\end{align*}

(3) It holds that $\Ep{\CI(\x)}\leq 16\sqrt{d}\cdot \lamgd+32d(\lambda+\lamgd^2)$.
\end{proposition}

\cref{thm:regret-upper-JDP-better} then follows from the general guarantee of \cref{thm:regret-upper-meta}.

\paragraph{Proof of \cref{thm:regret-upper-JDP-better}}
We instantiate the subroutine $\AlgCIEst$ to be \\ $\AlgJDPLR$ in \cref{alg:batch-cb-meta}, with parameter $\delz=\frac{1}{\delta T^2|\cA|}$.
We also let $P'$ be the distribution of $\phxa$ under $x\sim P$, $a\sim \Unif(\cA)$.

Then, for the $j$th epoch, we let $\lambda\epj\defeq \lambda(N\epj,\delta)$, $\lamz\epj\defeq \lamz(N\epj,\delta)$. \cref{prop:linear-JDP-better} guarantees that \whp,
\begin{align*}
    \MoveEqLeft \PP_{x\sim P}\paren{ \exists a\in\cA, \abs{\lr \phxa, \hth\epj-\ths\rr}\geq \CI\epj(\phxa) } \\
    \leq&~ |\cA|\cdot \PP_{\x'\sim P'}\paren{  \abs{\lr \x', \hth-\ths \rr }\geq \CI\epj(\x') } \leq |\cA|\delz\leq \frac{\delta}{T},
\end{align*}
and we also have
\begin{align*}
    \EE^{\pi\epj}\brac{\CIt[j](x,a)}\leqsim &~ \sqrt{d}\lamgd\epj+d(\lambda\epj+(\lamgd\epj)^2)\\
    \leqsim&~\sqrt{\frac{d\log(1/\delz)}{N\epj}}+ \frac{\siga d^{3/2}\sqrt{\log(N\epj/\delta)}+d\log(1/\delz)}{N\epj}.
\end{align*}

Therefore, \cref{asmp:EstCI} holds, and \cref{thm:regret-upper-meta} yields
\begin{align*}
    \Reg\leqsim &~ \dA\sqrt{d\log(1/\delz)}\sum_{j=0}^{J-2} \frac{N\epj[j+1]}{\sqrt{N\epj}} + \dA\paren{\siga d^{3/2}\sqrt{\log(T/\delta)}+d\log(1/\delz)} \sum_{j=0}^{J-2} \frac{N\epj[j+1]}{N\epj} \\
    &~+N\epj[0]+TJ\delta.
\end{align*}
In particular, with the choice $T\epj=2^{j+1}$ and $\delta=\frac{1}{T}$, we have
\begin{align*}
    \Reg\leq \tbO{ \dA\sqrt{dT\log|\cA|} +\dA(\siga d^{3/2}+d\log|\cA|)},
\end{align*}
where $\tbO{\cdot}$ hides $\poly(\log T)$ factors.
\qed


\subsubsection{Proof of \cref{prop:linear-JDP-better}}

By \cref{prop:spec-converge-JDP}, the subroutine $\JDPLU$ can be suitably instantiated so that the output $(U,\lambda)$ satisfies \eqref{def:U-app} \whp[\frac\delta2], with epoch size $K=\max\sset{\log\log(T),10}$, and $\lambda$ given by \eqref{eq:def:lambda-JDP-better}. 


In the following, we proceed to prove (2).
By definition, we have
\begin{align*}
    \til \xi = \xi+Z_\xi, \qquad
    \til \Xi=\Xi+Z_\Xi, \qquad
    \til W = W+Z_W,
\end{align*}
where entries of $Z=[Z_\xi; Z_\Xi; Z_W]$ are independent zero-mean Gaussian random variable with variance $\frac{36\siga^2}{N^2}$. Further, we also have
\begin{align*}
    y_t=\lr \x_t, \ths\rr+\zt, \qquad \EE[\zt|\x_t]=0,
\end{align*}
and conditional on the random variables $U, (\x_t)_{t\in[N+1,T]}$ and $Z$, the random variables $(\zt)_{t\in[N+1,T]}$ are independent.

In particular, the following event $\cE_1$ holds \whp[\frac{\delta}{6}]:
\begin{align*}
    \cE_1:\quad \max\sset{ \nrm{Z_\xi}, \nrmop{Z_\Xi}, \nrmop{Z_W}  }\leq \frac{\lambda}{16}.
\end{align*}
Further, by \cref{lem:cov-concen}, the following event $\cE_2$ holds \whp[\frac{\delta}{6}]:
\begin{align*}
    \cE_2: \quad \avgtt \usqx[U][t]+\frac12\lambda\id\succeq \frac12 \Epp{\usqx[U]}.
\end{align*}
In the following, we condition on $\cE_1\cap \cE_2$.

Under $\cE_2$, we have
\begin{align*}
    D:=\avgtt \uxxu[U][t]+\lambda U \succeq \frac14\id.
\end{align*}
Note that $\Xi=DU\iv$. For any vector $v\in\Rd$, we can bound
\begin{align*}
    \nrm{U{\til \Xi}\tp v}
    =\nrm{ (D+U Z_\Xi)v  } 
    \geq \nrm{Dv}- \nrm{U Z_\Xi v}
    \geq \frac{1}{4}\nrm{v}-\nrmopn{U}\nrmop{Z_\Xi}\nrm{v}\geq \frac18\nrm{v},
\end{align*}
where we use $\nrmopn{U}\leq \frac{2}{\lambda}$ and $\nrmop{Z_\Xi}\leq \frac{\lambda}{16}$. Therefore, we have $\nrmop{{\til \Xi}\itp U\iv}\leq 8$.

Now, we decompose
\begin{align*}
    \til \xi=\avgtt \gu{\x_t}\x_t\tp \ths + \avgtt \gu{\x_t}\zt+Z_\xi,
    \qquad
    \til \Xi=\avgtt \gu{\x_t}\x_t\tp+\lambda\id+Z_\Xi.
\end{align*}
Hence, we can re-write
\begin{align*}
    \til \xi=\paren{ \til \Xi-Z_\Xi-\lambda\id }\ths + \avgtt \gu{\x_t}\zt+Z_\xi,
\end{align*}
and we denote $e=\avgtt \gu{\x_t}\zt$.
Note that under $\cE_1$, we have $\nrm{ \til \xi -\til \Xi \ths- e} \leq 2\lambda$, and hence for any vector $\x\in\Rd$,
\begin{align*}
\abs{\lr \x, \hth-\ths \rr}\leq \abs{\lr {\til \Xi}\itp \x, \til \xi -\til \Xi \ths- e \rr}  + \abs{\lr \x, \Xi\iv e \rr } \leq 16\lambda\nrm{U \x}+\abs{\lr \x, \Xi\iv e \rr },
\end{align*}
where we also use $\nrmop{{\til \Xi}\itp U\iv}\leq 8$.

It remains to upper bound $\abs{\lr \x', \Xi\iv e \rr }$ under the fixed distribution $\x'\sim P'$.
The following lemma asserts $e$ is a sub-Gaussian random vector, conditional on $\x_{N+1},\cdots,\x_T$. The proof is a direct corollary of Hoeffding's bound and is deferred to the end of this section.
\begin{lemma}\label{lem:sub-Gaussian-empirical}
There is an absolute constant $c_0>0$ such that for any $v\in\Bone$,
\begin{align*}
    \EE_\zeta\cond{ \exp\paren{ c_0N\frac{\lr v,e\rr^2}{\nrm{v}_{W_0}^2} } }{U,\x_{N+1},\cdots,\x_T,Z}\leq 2,
\end{align*}
where $W_0=\avgtt \frac{U\x_t \x_t\tp U}{\|U\x_t\|^2}$.
\end{lemma}

In particular, taking expectation over $v={\til \Xi}\itp \x'$, with $\x'\sim P$, we have
\begin{align*}
    \EE_{\x'\sim P}\EE_\zeta\brac{ \exp\paren{ c_0N\frac{\lr v,{\til \Xi}\iv e\rr^2}{\nrmn{{\til \Xi}\itp \x'}_{W_0}^2} } }\leq 2.
\end{align*}
By Markov's inequality, the following event $\cE_3$ holds \whp:
\begin{align*}
    \cE_3:\quad \EE_{\x'\sim P}\brac{ \exp\paren{ c_0N\frac{\lr \x',{\til \Xi}\iv e\rr^2}{\nrmn{{\til \Xi}\itp \x'}_{W_0}^2} } } \leq \frac{6}{\delta}.
\end{align*}
Note that under $\cE_3$, using Markov's inequality again, we have
\begin{align*}
    \PP_{\x'\sim P'}\paren{ \frac{\absn{\lr \x',{\til \Xi}\iv e\rr}}{\nrmn{{\til \Xi}\itp \x'}_{W_0}}\geq \sqrt{\frac{\log(6/(\delta\delz))}{c_0N}} }\leq \delz.
\end{align*}
Note that under $\cE_1$, we have $\til W=W+Z_W=W_0+\lambda\id+Z_W\succeq W_0$.
Therefore, we can conclude that
\begin{align*}
    \PP_{\x'\sim P'}\paren{ \absn{\lr \x',\hth-\ths\rr}\geq 16\lambda\nrm{U\x'}+\sqrt{\frac{\log(6/(\delta\delz))}{c_0N}}\nrmn{{\til \Xi}\itp \x'}_{\til W} }\leq \delz.
\end{align*}

It remains to prove (3). by Cauchy inequality,
\begin{align*}
    \paren{ \Epp{ \nrm{{\til \Xi}\itp \x}_{\til W} } }^2
    \leq&~ \Epp{\nrmn{U \x}} \cdot \Ep{ \frac{\nrmn{{\til \Xi}\itp \x}_{\til W}^2}{\nrmn{U \x}} } \\
    \leq&~ 2d\cdot \llr \til W, \Epp{ \frac{{\til \Xi}\itp \x\x\tp {\til \Xi}\iv}{\nrmn{U \x}} }\rrr \\
    \leq&~ 4d\nrmop{{\til \Xi}\itp U\iv}^2\tr(\til W),
\end{align*}
where we use the fact that $\Ep{\uxxu}\preceq 2\id$. Using $\tr(\til W)=\tr(W_0)+\lambda d+\tr(Z_W)\leq 1+2\lambda d$, we have $\Epp{ \nrm{{\til \Xi}\itp \x}_{\til W} }\leq 16\sqrt{d(1+2\lambda d)}$. Combining the inequalities above completes the proof.
\qed

\paragraph{Proof of \cref{lem:sub-Gaussian-empirical}}
In the following, we condition on $(U,\x_{N+1},\cdots,\x_T,Z)$.
By Hoeffding's bound, for any $\lambda\in\R$,
\begin{align*}
    \EE_\zeta\brac{ \exp\paren{ \lambda \lr v,Ne\rr } }
    =&~ \EE_\zeta\brac{ \exp\paren{ \sumtt \lambda \lr v,\gu{\x_t}\rr \zt } } \\
    \leq&~ \exp\paren{ \frac{\lambda^2}{2}\sumtt \lr v,\gu{\x_t}\rr^2 } \\
    =&~ \exp\paren{ \frac{N\lambda^2}{2}\nrm{v}_{W_0}^2 }.
\end{align*}
Therefore, $\lr v, Ne\rr$ is a $\sigma_v$-sub-Gaussian random variable with $\sigma^2_v\leq N\nrm{v}_{W_0}^2$. Hence, there is an absolute constant $c_0>0$ such that for any $v\in\R^d$,
\begin{align*}
    \EE_\zeta\brac{ \exp\paren{ c_0N\frac{\lr v,e\rr^2}{\nrm{v}_{W_0}^2} } }\leq 2.
\end{align*}
\qed


\subsection{Proof of \cref{thm:regret-upper-LDP}}\label{appdx:regret-upper-LDP}


In \cref{alg:batch-cb-meta}, we instantiate the subroutine $\AlgCIEst$ with $\AlgLDPRegression$ (\cref{alg:LDP-L1-regression}). For the $j$th epoch, we let $(U\epj,\lambda\epj), \hth\epj, \lamall\epj$ be the output of the subroutine $\AlgLDPRegression$, and we let
\begin{align*}
    \hft[j](x,a)=\nu(\lr \phxa, \hth\epj \rr), \qquad
    \CIt[j](x,a)=\lamall \epj \cdot \nrmn{U\epj \phxa}, \qquad \forall (x,a)\in\cX\times\cA,
\end{align*}
following \cref{alg:batch-cb-JDP}. Recall that for the subroutine $\AlgLDPRegression$, we have
\begin{align*}
    \lamall\epj\defeq \lamall(N\epj,\delta)=\tbO{\siga (\kpg^{3/2}+\kpg\sqrt{d})\sqrt{\frac{\log(1/\delta)}{T}}},
\end{align*}
which is defined in \cref{thm:LDP-L1-regression-full}. 

Note that \cref{asmp:EstCI} holds with $\delta_0=0$ under the above specifications, and we also have $\EE^{\pi\epj}\brac{\CIt[j](x,a)}\leq 2d\lamall\epj$. Therefore, applying \cref{thm:regret-upper-meta} yields
\begin{align*}
    \Reg\leqsim&~ \dA d\sum_{j=0}^{J-2} N\epj[j+1] \cdot \lamall\epj + N\epj[0]+TJ\delta.
\end{align*}
In particular, with the epoch schedule $T\epj=2^{j}$ and $\delta=\frac1T$, we have
\begin{align*}
    \Reg\leq \tbO{ \dA d(\kpg^{3/2}+\kpg\sqrt{d})\sqrt{T} }.
\end{align*}
This is the desired upper bound.
\qed


\arxiv{
\section{Proofs from \cref{sec:unbounded}}\label{appdx:unbouned}

\subsection{Proof of \cref{thm:improper-JDP}}\label{appdx:improper-JDP}

\newcommand{\cDsub}{\cD_{\mathsf{sub}}}

\paragraph{Privacy guarantee}
To make the presentation clear, we re-write the iteration of \cref{alg:JDP-improper-GD} as follows. For any sub-dataset $\cDsub=\sset{(\x_{t},y_{t})}$, we denote
\begin{align*}
    g(\theta;\cDsub)\defeq&~ \frac1{|\cDsub|} \sum_{(\x,y)\in\cDsub}\x\paren{ \lr \x, \theta\rr-y }, \\
    F(\theta;\cDsub)\defeq&~ \Proj_{\BR}\paren{ \theta-\eta g(\theta;\cDsub) }.
\end{align*}
Then, the iteration of \cref{alg:JDP-improper-GD} can be re-written as follows:
$\theta\kz=\bz$, and for $k=0,1,\cdots,K-1$, 
\begin{align*}
    \theta\kp(\cD\epk{0:k})\defeq F\paren{\theta \kk(\cD\epk{0:k-1});\cD\kk)}.
\end{align*}
Note that for any $\cDsub$,
\begin{align*}
    g(\theta;\cDsub)=\nabla \paren{ \frac{1}{2|\cDsub|}\sum_{(\x,y)\in\cDsub} (\lr \x,\theta\rr-y)^2 }
\end{align*}
is the gradient of a 1-Lipschitz convex function, and hence $\theta\mapsto F(\theta;\cD)$ is a contraction under $\nrm{\cdot}=\nrm{\cdot}_2$:
\begin{align*}
    \nrm{\theta-\theta'}\leq \nrm{F(\theta;\cD)-F(\theta';\cD)}, \qquad \forall \theta,\theta'.
\end{align*}
Further note that, for neighbored sub-dataset $\cDsub$ and $\cDsub'$ of size $N$, we have
\begin{align*}
    \nrm{g(\theta;\cDsub)-g(\theta;\cDsub')}\leq \frac{2(R+1)}{N}, \qquad \forall \nrm{\theta}\leq R,
\end{align*}
and hence 
\begin{align*}
    \nrm{F(\theta;\cDsub)-F(\theta;\cDsub')}\leq \frac{2(R+1)\eta}{N}.
\end{align*}
Therefore, for any neighbored dataset $\cD$ and $\cD'$, using the inequalities above, we have
\begin{align*}
    \nrm{\theta\kc(\cD)-\theta\kc(\cD')}\leq \frac{2(R+1)\eta}{N}.
\end{align*}
This immediately show that \cref{alg:JDP-improper-GD} preserves \aJDP~by the privacy of Gaussian channels (\cref{def:Guassian-channel}).

\newcommand{\og}{\Bar{g}}
\newcommand{\gerr}[1]{\mathsf{err}\epk{#1}}
\paragraph{Convergence guarantee}
We state the following convergence guarantee.
\begin{proposition}\label{prop:improper-JDP}
Let $K,N\geq 2, \delta\in(0,1)$. We denote $B_\delta\defeq 10(R+1)\sqrt{\frac{K\log(K/\delta)}{N}}$. Suppose that the parameters $(\eta, R)$ are chosen so that
\begin{align}\label{eq:unbounded-cond-JDP}
    R\geq 1+B_\delta \eta.
\end{align}
Then it holds that \whp
\begin{align*}
    \nrm{\hth-\ths}_{\bSigma}\leqsim \frac{1}{\sqrt{\eta K}}+\sqrt{\frac{\eta\log K\log(1/\delta)}{ N}}+\frac{\eta (R+1)\siga\sqrt{\log(1/\delta)}}{N}.
\end{align*}
\end{proposition}

In particular, we may choose $R=2$, $\eta=1$, and
\begin{align*}
    K=\min\sset{ c\sqrt{\frac{T}{\log T \log(1/\delta)}} , \paren{\frac{T}{\siga \sqrt{\log(1/\delta)}}}^{2/3}}\vee 1, \qquad
    N=\frac{T}{K}, 
\end{align*}
where $c>0$ is a small absolute constant so that \eqref{eq:unbounded-cond-JDP} holds. Then, it holds that
\begin{align*}
    \nrm{\hth-\ths}_{\bSigma}\leqsim \paren{\frac{\log T\log(1/\delta)}{T}}^{1/4}+\paren{\frac{\siga \sqrt{\log(1/\delta)}}{T}}^{1/3}.
\end{align*}
This completes the proof of \cref{thm:improper-JDP}.
\qed

\subsubsection{Proof of \cref{prop:improper-JDP}}
We denote
\begin{align*}
    \og(\theta)\defeq \Exy{\x(\lr \x,\theta\rr-y)}=\bSigma (\theta-\ths).
\end{align*}
We also write
\begin{align*}
    \gerr{k}\defeq&~ g(\theta\kk;\cD\kk)-\og(\theta\kk), \\
    E\kk\defeq&~ \sum_{i=0}^{k} (\id-\eta\bSigma)^{k-i} \gerr{i}.
\end{align*}

We first invoke the following concentration result.
\begin{lemma}\label{lem:improper-JDP-concen}
The following holds \whp:
For all $k=0,1,\cdots,K-1$:
\begin{align*}
    \nrm{E\kk}\leq 5(R+1)\sqrt{\frac{K\log(K/\delta)}{N}}\defeq B_\delta,
\end{align*}
and
\begin{align*}
    \nrm{E\kc}_{\bSigma}\leq 10(R+1)\sqrt{\frac{(\eta^{-1}\log K+2)\log(2/\delta)}{N}}=:\epsN.
\end{align*}
\end{lemma}

\paragraph{Proof of \cref{lem:improper-JDP-concen}}
By definition,
\begin{align*}
    N\cdot E\kk=&~N\sum_{i=0}^{k} (\id-\eta\bSigma)^{k-i}\gerr{i} \\
    =&~ \sum_{i=0}^{k}\sum_{t=iN+1}^{(i+1)N} (\id-\eta\bSigma)^{k-i}(g_t-\og(\theta\epk{i})),
\end{align*}
where $g_t:=\phi_t(\lr \x_t,\theta_{(i)}\rr-y_t)$ is the gradient at $t \in [iN+1, (i+1)N]$. Note that $\EE[g_t|\theta\epk{i}]=\og(\theta\epk{i})$ for round $t$ in $i$th epoch. Thus, we denote $Z_t\defeq (\id-\eta\bSigma)^{k-i}(g_t-\og(\theta\epk{i}))$ for $t\in[iN+1,(i+1)N]$. Then, the sequence $Z_t$ is a martingale difference sequence with respect to the filtration $\cF_t=\sigma(\sset{(\x_s,y_s)}_{s\leq t})$.
Further, note that $\nrm{Z_t}\leq 2(R+1)$, and hence using \cref{lem:vec-Hoeffding} with a union bound, we have \whp~for all $k\in[K]$,
\begin{align*}
    N\nrm{E\kk}=\nrm{\sum_{t=1}^{(k+1)N} Z_t}\leq 2(R+1)\sqrt{NK}\cdot \paren{1+\sqrt{2\log(K/\delta)}}.
\end{align*}
Further, by \cref{lem:cov-k-converge}, we also have
\begin{align*}
    \nrm{\bSigma\sq Z_t}\leq 2(R+1)\nrmop{\bSigma\sq(\id-\eta\bSigma)^{k-i}}\leq 5(R+1)\min\sset{\frac{1}{\sqrt{\eta(k-i)}},1}.
\end{align*}
Therefore, using \cref{lem:vec-Hoeffding}, we have \whp
\begin{align*}
    N\nrm{E\epk{K-1}}_{\bSigma}=\nrm{\sum_{t=1}^{KN} (\bSigma\sq Z_t)}\leq 5(R+1)\sqrt{N\cdot \paren{\frac{\log K}{\eta}+2}}\cdot \paren{1+\sqrt{2\log(1/\delta)}}.
\end{align*}
Taking the union bound again and rescaling $\delta\leftarrow \frac{\delta}{2}$ completes the proof.
\qed


We denote $\cE_1$ to be the success event of \cref{lem:improper-JDP-concen}. 
In the following, we work under $\cE_1$.

We inductively show that for all $i<K$:
\begin{align}\label{eq:improper-JDP-rec}
    \theta\epk{i+1}=\theta\epk{i}-\eta g\epk{i}.
\end{align}
The base case is trivial: \eqref{eq:improper-JDP-rec} holds for all $i<0$. 

Now, we assume that for some $k<K$, \eqref{eq:improper-JDP-rec} holds for all $i<k$. Then, we only need to prove \eqref{eq:improper-JDP-rec} for the case $i=k$. We denote $\theta\kp^+=\theta\kk-\eta g\kk$. Using $\og(\theta)=\bSigma (\theta-\ths)$ and \eqref{eq:improper-JDP-rec} recursively for $i<k$, we know
\begin{align*}
    \theta\kp^+-\ths=(\id-\eta\bSigma)^{k+1}(\theta\kz-\ths)-\eta \sum_{i=0}^{k} (\id-\eta\bSigma)^{k-i} \gerr{i}.
\end{align*}
Therefore,
\begin{align*}
    \nrm{\theta\kp^+}\leq \nrm{\paren{\id-(\id-\eta\bSigma)^{k+1}}\ths}+\eta \nrm{E\kk}\leq 1+\eta B_\delta\leq R.
\end{align*}
Therefore, $\theta\kp^+\in\BR$, and hence
\begin{align*}
    \theta\kp=\Proj_{\BR}(\theta\kp^+)=\theta\kp^+.
\end{align*}
This completes the proof of case $i=k$. 

Therefore, by induction, \eqref{eq:improper-JDP-rec} indeed holds for all $j<K$. In particular, we have
\begin{align*}
    \theta\kc-\ths=&~(\id-\eta\bSigma)^{K}(\theta\kz-\ths)-\eta \sum_{i=0}^{K-1} (\id-\eta\bSigma)^{K-1-i} \gerr{i} \\
    =&~ (\id-\eta\bSigma)^{K}(\theta\kz-\ths)-\eta E\epk{K-1}.
\end{align*}
Hence, it holds that (by \cref{lem:cov-k-converge} and under $\cE_1$):
\begin{align*}
    \nrm{\theta\kc-\ths}_{\bSigma}\leq \nrm{(\id-\eta\bSigma)^{K}\ths}_{\bSigma}+\eta\nrm{E\epk{K-1}}_{\bSigma}
    \leqsim \frac{1}{\sqrt{\eta K}}+\sqrt{\frac{\eta\log K\log(1/\delta)}{ N}}.
\end{align*}
Finally, we know $\hth=\theta\kc+Z$, where $Z\sim \normal{0,\frac{4(R+1)^2\siga^2}{N^2}}$ is a Gaussian random vector. Therefore, \whp, $\nrm{Z}_{\bSigma}\leqsim \frac{(R+1)\siga\sqrt{\log(1/\delta)}}{N}$. Combining the inequalities above and rescaling $\delta\leftarrow \frac{\delta}{3}$ complete the proof.
\qed

\subsection{Proof of \cref{thm:improper-LDP}}





\newcommand{\sigsp}{\sigma_{N}}



We prove the following convergence rate of \cref{alg:LDP-improper-GD}, under general choice of $(\eta,R)$.
\begin{proposition}\label{prop:unbounded-converge}
Let $K,N\geq 2, \delta\in(0,1)$. We denote $B_\delta\defeq 6(R+1)\sqrt{\frac{K\log(K/\delta)}{N}}$ and $\epsN=(R+1)\sqrt{\frac{K\log(K/\delta)}{N}}$. Suppose that the parameters $(\eta, R)$ are chosen so that
\begin{align}\label{eq:unbounded-cond}
    R\geq 1+\eta\cdot \paren{ B_\delta+4\epsN }.
\end{align}
Then it holds that \whp
\begin{align*}
    \nrm{\theta\kc-\ths}_{\bSigma}\leqsim \frac{1}{\sqrt{\eta K}}+R\eta\siga \sqrt{\frac{K\log(K/\delta)}{N}}.
\end{align*}
\end{proposition}

\paragraph{Proof of \cref{thm:improper-LDP}}
For \cref{alg:LDP-improper-GD}, we choose $\eta=1$, $R=2$, and
\begin{align*}
    K=c\paren{\frac{T}{\siga^2 \log (T/\delta)}}^{1/3}\vee 1, \qquad
    N=\frac{T}{K}, 
\end{align*}
where $c>0$ is an absolute constant so that \eqref{eq:unbounded-cond} holds. Then, by \cref{prop:unbounded-converge}, \cref{alg:JDP-improper-GD} achieves
\begin{align*}
    \nrm{\theta\kc-\ths}_{\bSigma}\leqsim \paren{\frac{\siga^2\log(T/\delta)}{T}}^{1/6}.
\end{align*}
This is the desired upper bound.

\subsubsection{Proof of \cref{prop:unbounded-converge}}

In \cref{alg:LDP-improper-GD}, for epoch $k=0,1,\cdots,K-1$, we have
\begin{align*}
    \til g\kk=\zeta\kk+\avgtk g_t, \qquad
    \theta\kp=\theta\kk-\eta \til g\kk,
\end{align*}
where $\set{\zeta\kz,\cdots,\zeta\epk{K-1}}$ are i.i.d samples from $\normal{0,\sigsp^2}$ with $\sigsp=\frac{(R+1)\siga}{\sqrt{N}}$ and independent of the dataset $\sset{(\x_t,y_t)}_{t\in[T]}$. 

\newcommand{\err}[1]{\mathsf{err}^{(#1)}}
To begin with, we denote $\bSigma\defeq \Ep{\x\x}$ (the covariance matrix), 
\begin{align*}
    \ogd{k}\defeq&~ \EE_{(x,y)\sim p}\brac{ x\paren{ \clip{\lr \pa{k},x \rr}-y }},
\end{align*}
and define the error vectors
\begin{align*}
    \err{k}_0\defeq \nabla \Lsq(\pa{k})-\ogd{k} \qquad
    \err{k}_1\defeq \ogd{k}-\avgtk g^t, \qquad
    \err{k}_2=-\zeta\kk.
\end{align*}
Then, we can decompose the error of the estimator $\gd{k}$ as
\begin{align*}
    \err{k}\defeq \nabla \Lsq(\pa{k}) - \gd{k} = \err{k}_0 + \err{k}_1 +\err{k}_2.
\end{align*}

Notice that by definition, we have
\begin{align*}
    \Lsq(\theta\kk)=\frac12\nrm{\theta\kk-\ths}_{\bSigma}^2, \qquad
    \nabla \Lsq(\pa{k})=\bSigma(\pa{k}-\ths).
\end{align*}
Therefore, recursively using $\gd{k}=\nabla \Lsq(\pa{k})-\err{k}$ and $\pa{k+1}=\pa{k}-\eta \gd{k}$, we have
\begin{align}\label{eqn:proof-unbounded-decomp}
    \pa{k}-\ths=(\id-\eta\bSigma)^k(\pa{0}-\ths)+\eta\sum_{i=0}^{k-1} (\id-\eta\bSigma)^{k-i-1}\err{i}.
\end{align}
We bound the three types of error separately: For each $j\in\set{0,1,2}$, denote
\newcommandx{\Ek}[1][1=k]{E^{(#1)}}
\begin{align*}
    \Ek_j\defeq \sum_{i=0}^{k-1} (\id-\eta\bSigma)^{k-i-1}\err{i}_j.
\end{align*}

\newcommand{\epsNi}[1]{\eps_{N,#1}}





\begin{lemma}\label{lem:unbounded-E1}
\Whp, the following holds:
For all $k\in[K]$, it holds that
\begin{align*}
    \nrm{ \Ek_1 }\leq 6(R+1)\sqrt{\frac{K\log(K/\delta)}{N}}=:B_\delta.
\end{align*}

\end{lemma}

The proof of \cref{lem:unbounded-E1} is essentially similar to that of \cref{lem:improper-JDP-concen}, and hence we omit it for succintness.


\begin{lemma}\label{lem:unbounded-E2}
Denote $\epsN\defeq \sigsp\sqrt{K\log(2K/\delta)}$. \Whp, for all $k=0,1,\cdots,K-1$, it holds that
\begin{align*}
    \Pp{ \absn{\lr \Ek_2, \x\rr}> 3\epsN}\leq \frac{1}{K^6}, \qquad
    \Epp\lr \Ek_2, \x\rr^2\leq 4\epsN^2.
\end{align*}
where $C_2$ is an absolute constant. We denote this success event at $\cE_2$.
\end{lemma}

\paragraph{Proof of \cref{lem:unbounded-E2}}
Fix a $k\in[K]$. Then by definition
\begin{align*}
    \Ek_2=\sum_{i=0}^{k} (\id-\eta\bSigma)^{k-i}\zeta\epk{i},
\end{align*}
where $\zeta\epk{i}\sim \normal{0,\sigsp^2\id}$. Therefore, because $\zeta=(\zeta\kz,\cdots,\zeta\kc)$ is a sequence of independent Gaussian random variables, we have
\begin{align*}
    \Ek_2\sim \normal{0, C_k}, \qquad C_k=\sigsp^2\sum_{i=0}^{k}(\id-\eta\bSigma)^{2(k-i)}\preceq K\sigsp^2 \id.
\end{align*}
Therefore, for any fixed $\x\in\Bone$, $\lr \Ek_2,\x\rr\sim \normal{0,\nrm{\x}^2_{C_k}}$ is a zero-mean Gaussian random variable with variance $\nrm{\x}^2_{C_k}\leq K\sigsp^2$. This immediately implies that
\begin{align*}
    \forall \x\in\Bone, \qquad 
    \EE_\zeta\brac{ \exp\paren{ \frac{\absn{\lr \Ek_2,\x\rr}^2 }{4K\sigsp^2} } }\leq 2,
\end{align*}
where the expectation is taken over the sequence $\zeta=(\zeta\kz,\cdots,\zeta\kc)$ of independent Gaussian random vectors. Therefore, we have
\begin{align*}
    \EE_\zeta\brac{ \Epp{\exp\paren{ \frac{\absn{\lr \Ek_2,\x\rr}^2 }{4K\sigsp^2} }} } \leq 2, \quad\forall k.
\end{align*}
Therefore, by Markov's inequality and taking the union bound, we have
\begin{align*}
    \PP_\zeta\paren{ \forall k\in[K]: \Epp{\exp\paren{ \frac{\absn{\lr \Ek_2,\x\rr}^2 }{4K\sigsp^2} } }\leq \frac{2K}{\delta} }\geq 1-\delta.
\end{align*}
Let this event be $\cE_2$. Then, under $\cE_2$, 
using Markov inequality's again, we have
\begin{align*}
    \Pp{\absn{\lr \Ek_2,\x\rr}\geq 3\sigsp\sqrt{K\log(K/\delta)}}\leq \frac{1}{K^6}, \qquad \forall k.
\end{align*}
Similarly, under $\cE_2$, using Jensen's inequality, we also have
\begin{align*}
    \Epp{\absn{\lr \Ek_2,\x\rr}^2}\leq  4\sigsp^2 K\log(2K/\delta).
\end{align*}
This is the desired result.
\qed


\begin{lemma}\label{lem:unbounded-E0}
Under the success event of \cref{lem:unbounded-E2} and assuming that $K\geq 2$ and
\begin{align*}
    R\geq 1+\eta\cdot \paren{ B_\delta+4\epsN }.
\end{align*}
Then, we have for all $k=0,1,\cdots,K-1$:
\begin{align}\label{lem:unbouned-err-0}
    \nrm{\err{k}_0}\leq \frac{2\eta\epsN}{K^3}.
\end{align}
In particular, it holds that $\nrm{\Ek_0}\leq \frac{2\eta\epsN}{K^2}$.
\end{lemma}

\paragraph{Proof of \cref{lem:unbounded-E2}}
We prove by induction. The base case $k=0$ is trivial because $\err{0}_0=0$.

Now, suppose that \eqref{lem:unbouned-err-0} holds for all $k'\leq k$. In the following, we proceed to prove \eqref{lem:unbouned-err-0} for $k+1$.

By \eqref{eqn:proof-unbounded-decomp}, we have
\begin{align*}
    \pa{k+1}=\paren{\id-(\id-\eta\bSigma)^k}\ths+\eta\paren{\Ek_0+\Ek_1+\Ek_2},
\end{align*}
and under $\cE_1\cap \cE_2$, we have $\nrm{\cE_1}\leq C_1\epsN$, 
\begin{align*}
    \Pp{ \absn{\lr \Ek_2, \x\rr}> 3\epsN }\leq \frac{1}{K^6}.
\end{align*}
By induction hypothesis, 
\begin{align*}
    \nrm{\Ek_0}=\nrm{\sum_{i=0}^{k} (\id-\eta\bSigma)^{k-i}\err{i}_0}
    \leq \sum_{i=0}^{k} \nrm{\err{i}_0}\leq \frac{2\epsN}{K^2}.
\end{align*}
Combining all the equations above, we have
\begin{align*}
    \abs{\lr\x, \pa{k+1}\rr}
    \leq 1+\eta\paren{\frac{2\eta\epsN}{K^2}+B_\delta+\abs{\lr \Ek_2, \x\rr}}.
\end{align*}
By our assumption on $R$, we know that $\abs{\lr\x, \pa{k+1}\rr}-R\leq \eta\paren{\absn{\lr \Ek_2, \x\rr}-3\epsN}$, and hence
\begin{align*}
    \Pp{ \absn{\lr\x, \pa{k+1}\rr} \geq R}
    \leq \Pp{ \absn{\lr \Ek_2, \x\rr}> 3\epsN } \leq \frac{1}{K^6}.
\end{align*}
Finally, by the definition of $\err{k+1}_0$, we have
\begin{align*}
    \err{k+1}_0=\nabla \Lsq(\pa{k+1})-\ogd{k+1}
    =\Ep{ \x\paren{ \lr \pa{k+1},\x \rr-\clip{\lr \pa{k+1},\x \rr} }}.
\end{align*}
Hence,
\begin{align*}
    \nrm{\err{k+1}_0}\leq&~ \Ep{\indic{ \absn{\lr \pa{k+1},\x \rr}>R }\cdot \paren{ \absn{\lr \pa{k+1},\x \rr} - R } } \\
    \leq&~ \sqrt{\Pp{ \absn{\lr\x, \pa{k+1}\rr} \geq R} \cdot \Epp{\paren{\absn{\lr \pa{k+1},\x \rr}-R}_+^2  }} \\
    \leq&~ \sqrt{\frac{\eta^2}{K^6}\Epp{\lr \Ek_2,\x \rr^2 } }
    \leq \frac{2\eta\epsN }{K^3}.
\end{align*}
This completes the proof of the step $k+1$.
\qed

Now, we prove \cref{prop:improper-JDP} by combining the lemmas above. By definition,
\begin{align*}
    \nrm{\theta\kc-\ths}_{\bSigma}\leq \nrm{(\id-\eta\bSigma)^K\ths}+\eta\nrm{\Ek[K]_0}_{\bSigma}+\eta\nrm{\Ek[K]_1}_{\bSigma}+\eta\nrm{\Ek[K]_2}_{\bSigma}.
\end{align*}
Under the success event of \cref{lem:unbounded-E1} and \cref{lem:unbounded-E2}, we have 
\begin{align*}
    \nrm{\Ek[K]_1}_{\bSigma}\leq \nrm{\Ek[K]_1}\leq B_\delta\leqsim \epsN, \qquad
    \nrm{\Ek[K]_2}_{\bSigma}\leq 2\epsN,
\end{align*}
and $\nrm{\Ek[K]_0}\leq \frac{2\eta\epsN}{K^2}$. Further, by \cref{lem:cov-k-converge}, we also have 
\begin{align*}
    \nrm{(\id-\eta\bSigma)^K\ths}\leq \sqrt{\frac{2e}{\eta K}}.
\end{align*}
Combining the inequalities above gives
\begin{align*}
    \nrm{\theta\kc-\ths}_{\bSigma}\leqsim \frac{1}{\sqrt{\eta K}}+\eta \epsN.
\end{align*}
This is the desired upper bound.
\qed

\subsection{Algorithm SquareCB}\label{appdx:square-cb}


\newcommand{\AlgRegression}{\mathsf{Regression}}
\begin{algorithm}
\begin{algorithmic}
\REQUIRE Round $T\geq 1$, epoch schedule $1=T_0<T_1<T_2<\cdots<T_{J}=T$.
\REQUIRE Oracle $\AlgRegression$, parameter $\delta\in(0,1)$.
\STATE Initialize $\hft[0]\equiv 0$.
\FOR{$j=0,1,\cdots,J-1$}
\STATE Initialize the subroutine $\AlgRegression\epj$ with round $N\epj=T\epj[j+1]-T\epj$ and confidence $\delta'=\frac{\delta}{2J^2}$.
\FOR{$t=T_j,\cdots,T_{j+1}-1$}
\STATE Receive context $x_t$.
\STATE Let $\hat{a}_t\defeq \argmax_{a\in\cA} \hft[j](x_t,a)$, and set
\begin{align*}
    p_t(a)\defeq \begin{cases}
        \frac{1}{|\cA|+\gamma\epj\paren{\hft[j](x_t,a_t)-\hft[j](x_t,a)}}, & a\neq \hat{a}_t, \\
        1-\sum_{a\neq \hat{a}_t} p_t(a), & a=\hat{a}_t.
    \end{cases}
\end{align*}
\STATE Take action $a_t\sim p_t$, and receive reward $r_t$.
\STATE Feed $(\phxa[x_t,a_t],r_t)$ into $\AlgRegression\epj$.
\ENDFOR
\STATE Receive estimation $\hft[j+1]$ from $\AlgRegression\epj$.
\ENDFOR
\end{algorithmic}
\caption{$\AlgSQCB$~\citep{foster2020beyond,simchi2020bypassing}}\label{alg:square-cb}
\end{algorithm}

\begin{assumption}\label{asmp:L2-regression}
For any policy $\pi:\cX\to \Delta(\cA)$ and any linear reward function $\fs$, given $N$ independent samples $\sset{(x_t,a_t,r_t)}$ generated as
\begin{align*}
    x_t\sim P,\quad
    a_t\sim \pi(x_t), \quad
    \EE[r_t|x_t,a_t]=\fs(x_t,a_t),
\end{align*}
the regression oracle $\AlgRegression$ (initialized with round $N$ and confidence $\delta$) returns an estimated mean function $\hf:\cX\times\cA\to\R$ such that \whp,
\begin{align*}
    \EE_{x\sim P, a\sim \pi(x)} \paren{\hf(x,a)-\fs(x,a)}^2\leq \cE_{\delta}(N)^2,
\end{align*}
where $\cE_\delta$ is a non-increasing function of $N$. 
\end{assumption}
\begin{theorem}[Guarantee of $\AlgSQCB$]\label{thm:square-cb}
Suppose that \cref{asmp:L2-regression} holds. Then, with parameters
\begin{align*}
    \gamma\epj[0]=1, \qquad \gamma\epj\defeq \frac{\sqrt{|\cA|}}{\cE_{\delta'}(N\epj[j-1])}, \quad j=1,\cdots,J-1,
\end{align*}
$\AlgSQCB$ (\cref{alg:square-cb}) achieves
\begin{align*}
    \Reg\leq C\sqrt{|\cA|}\sum_{j=0}^{J-1} \cE_{\delta'}(N\epj)\cdot N\epj[j+1]+N\epj[0]+T\delta.
\end{align*}
\end{theorem}

Furthermore, the privacy guarantee of \cref{alg:square-cb} can be implied by the privacy guarantee of the regression oracle (similar to \cref{alg:batch-cb-meta}).
\begin{lemma}
If the oracle $\AlgRegression$ preserves \aJDP~(or correspondingly \aLDP), then \cref{alg:square-cb} preserves \aJDP~(or correspondingly \aLDP)
\end{lemma}

\paragraph{Proof of \cref{thm:regret-dim-free} (1)}
For JDP learning, we consider instantiating the regression oracle $\AlgRegression$ with the algorithm $\AlgJDPIGD$ (\cref{alg:JDP-improper-GD}). For the output $\hth$ of $\AlgJDPIGD$ given a dataset of size $N$, we consider the estimated mean function $\hf(x,a)\defeq \lr \hth,\phxa\rr$ for all $(x,a)\in\cX\times\cA$. Then, by \cref{thm:improper-JDP}, \cref{asmp:L2-regression} holds with 
\begin{align*}
    \cE_\delta(N)\leqsim \paren{\frac{\log N \log(1/\delta)}{N}}^{1/4}+\paren{\frac{\siga\log(1/\delta)}{N}}^{1/3}.
\end{align*}
Therefore, we choose $T\epj=2^j$ and $\delta=\frac{1}{T}$, and \cref{thm:square-cb} provides the following regret bound:
\begin{align*}
    \Reg\leq \sqrt{|\cA|}\cdot \tbO{T^{3/4}+\siga^{1/3}T^{2/3}}.
\end{align*}
This is the desired result.
\qed


\paragraph{Proof of \cref{thm:regret-dim-free} (2)}
Similarly, for LDP learning, we consider instantiating the regression oracle $\AlgRegression$ with the algorithm $\AlgLDPIGD$ (\cref{alg:LDP-improper-GD}). Then, by \cref{thm:improper-LDP}, \cref{asmp:L2-regression} holds with 
\begin{align*}
    \cE_\delta(N)\leqsim \paren{\frac{\siga^2\log(N/\delta)}{N}}^{1/6}.
\end{align*}
Therefore, we choose $T\epj=2^j$ and $\delta=\frac{1}{T}$, and \cref{thm:square-cb} provides the following regret bound:
\begin{align*}
    \Reg\leq \sqrt{|\cA|}\cdot \tbO{\siga^{1/3}T^{5/6}}.
\end{align*}
This is the desired regret bound.
\qed

}


\end{document}
