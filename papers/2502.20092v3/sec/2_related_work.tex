\section{Related Works} ~\label{Sec:related_works}
\begin{table*}
	\caption{Comparison between WalnutData and other aerial datasets.}
	\label{tab:Table_1}
	\setlength{\tabcolsep}{12pt}
	\centering
	\begin{tabular}{lccccccc}
		\toprule
		Dataset & Environment & Image Widths & Altitude Range & Angle Range & Images & Instances & Classes \\
		\midrule
		VisDrone~\cite{zhu2021detection} & Traffic & 960-2,000 & 5-200m & 0-90° & 8,599 & 540,000 & 10 \\
		UAVDT~\cite{du2018unmanned} & Traffic & 1,024 & 5-200m & 0-90° & 80,000 & 840,000 & 3 \\
		SeaDronesSee~\cite{varga2022seadronessee} & Maritime & 3,840-5,456 & 5-260m & 0-90° & 8,295 & - & 6 \\
		SDS-ODv2~\cite{kiefer20231st} & Maritime & 3,840-5,456 & 5-260m & 0-90° & 14,227 & 403,192 & 6 \\
		\textbf{WalnutData(our)} & Agriculture & 1,024 & 12-30m & 90° & 30,240 & 706,208 & 4 \\
		\bottomrule
	\end{tabular}
\end{table*}

\begin{table*}
	\caption{Compare WalnutData with the annotated datasets in agriculture in recent years.}
	\label{tab:Table_2}
	\centering
	\begin{tabular}{lcccccc}
		\toprule
		Dataset & Object & Device & Images & Classes & Image Widths & Label Type \\
		\midrule
		Apeinans I et al.~\cite{apeinans2024cherry} & Cherry & - & 2283 & 1 & 640 & YOLO \\
		Wu Z et al.~\cite{wu2023dataset} & Tomato & DSLR camera and mobile phone & 520 & 2 & 4,000/4,032 & VOC \\
		Hani N et al.~\cite{hani2020minneapple} & Apple & Mobile phone & 1,001 & 1 & 1280 & - \\
		Bargoti S et al.~\cite{bargoti2017deep} & Apple,mango, and almond & UGV and DSLR camera & 2,750 & 3 & 202/300/500 & COCO \\
		Stein M et al.~\cite{stein2016image} & Mango & UGV & 1,500 & 1 & 3,296 & - \\
		Santos T et al.~\cite{santos2021methodology} & Apple & UAV & 1,139 & 1 & 256 & COCO \\
		Butte S et al.~\cite{butte2021potato} & Potato & UAV & 360 & 2 & 1,500 & VOC \\
		\textbf{WalnutData(our)} & Walnut & UAV & 30,240 & 4 & 1,024 & VOC, COCO, YOLO \\
		\bottomrule
	\end{tabular}
\end{table*}
In this section, we review the main annotated datasets that can be used for supervised learning models in the field of UAV vision and agricultural scenarios.
\subsection{Annotated Image Datasets Collected by UAV}
In recent years, most of the mainstream annotated datasets collected by UAV are used to describe data of traffic roads or marine environments, such as VisDrone ~\cite{zhu2021detection}, UAVDT~\cite{du2018unmanned}, SeaDronesSee~\cite{varga2022seadronessee}, and SDS-ODv2~\cite{kiefer20231st}. As can be seen from Table~\ref{tab:Table_1}, the images captured by these UAVs have a height range of 5-260 meters, a shooting angle range of 0-90°, and an image width range of 960-5,456 pixel. These datasets cover various scenarios such as cities, villages, and oceans, mainly focusing on road traffic environment analysis and maritime rescue, but lacking coverage of agricultural target scenarios. WalnutData is an agricultural scenario dataset different from the traffic and maritime fields. The UAV images are collected in the range of 12-30m, with an aerial shooting angle of -90°. The width of the dataset images is 1,024 pixel after the original images are segmented. WalnutData has a significant advantage over other datasets in terms of the number of images and instances.

\subsection{Annotated Datasets in Agriculture}
With the rapid popularization of deep learning technology and the urgent needs of precision agriculture, more and more datasets in the field of computer vision for agriculture have been constructed and made public. The main objects of study in these datasets listed in Table~\ref{tab:Table_2} include apple, potato, tomato, mango, etc. However, there is still a lack of research on green walnut targets.

In studies such as tomato~\cite{wu2023dataset}, apple~\cite{bargoti2017deep}, and mango~\cite{stein2016image}, the image data are mainly collected by DSLR camera, UGV, or mobile phone. These shooting methods are affected by the ground environment. Moreover, the planting terrains of crops such as tomato, cherry, or apple are relatively flat, which is quite different from the growing terrain of walnut trees. In addition, in the research on agricultural UAV-related datasets, the apple trees studied by Santos T et al.~\cite{santos2021methodology} have a neat interval, which is very conducive to collecting relatively regular data information. Thus, effective algorithms can be used for apple detection, tracking, and positioning. Butte S et al.~\cite{butte2021potato} proposed a potato dataset. Through the model they designed, it is possible to accurately identify healthy or drought-stressed potatoes, providing a new idea for precision agriculture.

In Yunnan Province, China, most walnut trees are planted in mountainous areas with large altitude differences and complex terrains, and the fruit trees are unevenly distributed~\cite{wang2025ow}. Therefore, in this study, UAV aerial photography is used for data collection to obtain WalnutData. Compared with other datasets, WalnutData has a more detailed division of crop characteristic states and a larger amount of data, which can provide a more solid foundation for model design. In addition, WalnutData provides three types of labels (VOC, COCO, and YOLO), which are suitable for many current mainstream object detection models and offer multiple choices for researchers in related fields.

