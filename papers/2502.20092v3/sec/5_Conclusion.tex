\section{Conclusion}
This research aims to address the computer vision challenges of walnut fruit detection from a drone perspective, such as the impacts of lighting variations and occlusions on the algorithms. To this end, we have constructed a fine-grained agricultural drone dataset for walnut detection, which is the first large-scale dataset in the field of smart walnut farming. The dataset’s scale and fine-grained feature segmentation give it significant research value and engineering application potential in the field of agricultural computer vision.In addition, by conducting benchmark evaluations of WalnutData using a series of mainstream object detection models, we hope to drive the development of precision agriculture and the smart walnut sector.

In the future, research based on WalnutData can further advance the application of automated harvesting robots and precision management systems in smart agriculture. By optimizing existing object detection algorithms and integrating more agricultural data, more efficient and accurate crop monitoring and yield prediction can be achieved.


\section*{ACKNOWLEDGMENT}
This study is supported by the Yunnan Province Applied Basic Research Program Key Project (202401AS070034) and the Yunnan Province Forest and Grassland Science and Technology Innovation Joint Project (202404CB090002). We thank Haoyu Wang, Shuangyao Liu, Tingfeng Li, Shuyi Wan, Haotian Feng, Luhao Fang, Songfan Shi, Shiyu Du and all the others who involved in the annotations of WalnutData.