\section{Related work}
\label{sec:related}

Process mining methods are covered by a vast number of works. 
Most aspects of process mining are covered in the books \citep{Aalst16,Burattin15,MansAV15} and in the subsequent handbook \citep{van2022process}. These works provide a solid foundation for the area with all the necessary definitions, methods, and algorithms.

The standard DFG discovery method is known to be problematic in some cases \citep{Van_der_Aalst2019-sp}. The work describes the problem with cycles that appear in process models containing parallel events or due to event names happening in different orders, even if in observable behaviour all event names appear at most once per trace. That implies that users should be aware of the features of the DFG discovery algorithm to avoid misleading interpretations.

BPMN models with repeated event nodes were considered in \cite{Lieben2018-st}, where the authors describe a discovery algorithm for exploratory data analysis. By allowing vertices to repeat in the models, the resulting models can be simplified. Repeating event nodes is also an alternative way to visualize parallel events. This creates less visual clutter and results in clearer and more accurate models.

The search for similarities and differences in process model patterns is present in many fields. Works that examine the area of concept drift focus on changes in the process over time \citep{Bose2011-hg, Bose2014-by}. This can also be described as finding the parts where the processes are the same and where they are different. Another area is the repair of process models \citep{FahlandA15,MitsyukLA17,PolyvyanyyAHW17}, where the goal is to improve the quality of a model by changing as few parts as possible. During the model repair process, the model is decomposed into parts \citep{Aalst13}, which are analysed separately. 
Bad parts can be repaired, and then all the parts are composed into a repaired model \citep{MitsyukLA17}. 
In addition, process models can be decomposed into parts to make them easier to compare more accurately \citep{BrockhoffGUA24}.

Assessment of process models is covered by conformance checking techniques \citep{CarmonaDSW18,Munoz-Gama16}. Two main ways to assess a model and an event log are alignments and token-based replay. The token-based replay approach was first mentioned in the work of \cite{Rozinat2008-bd}. The trace alignment approach was first mentioned in the work of \cite{Van_der_Aalst2012-kp}. With the help of these algorithms, we can obtain the precision and fitness of a model based on a given event log. Conformance checking algorithms also relate to the question of similarities and differences between process models and event logs. One of the works \citep{Artamonov2019-vq} described the conformance checking technique that can detect differences in model fragments through event relations. 

%------------------------------------------------------------------------------