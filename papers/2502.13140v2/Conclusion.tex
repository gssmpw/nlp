\section{Conclusion}~\label{sec:conclusion}
Tensor decomposition methods are highly effective in analyzing multi-modal and multidimensional data such as biomedical data~\cite{cichocki2016tensor}. In this paper we reviewed the state-of-the-art algorithms in TD and analyzed the best practices for implementing them in solving various tasks in biomedical imaging, multi-omics data, and spatial transcriptomics. We designed an unsupervised approach using topic modeling leveraging the transformer architecture with BERTopic~\cite{grootendorst2022bertopic} and identified 536 documents from PubMed in the last decade with applications of TD in biomedicine. We obtained several relevant topics such as on brain EEG, MRI denoising, single-cell omics, transcriptomics, etc. after analyzing and visualizing the literature from PubMed. To further investigate on each of these topics, we performed a systematic review on each sub-domain of biomedical imaging, multi-omics analysis, and spatial transcriptomics, and revealed how TD methods are applied and which tasks are more likely to be solved by tensorizing and obtaining latent factors. For each sub-domain, we also investigated the challenges and limitations of TD, along with a thorough understanding of computational complexity and hardness of the problem. Specifically, we sought to understand the time and space complexity with respect to the growing size of the data to make TD amenable for biobank-scale data. 

Armed with deeper understanding of challenges in applications of TD in biomedicine, we proposed an adaptation of recent advances in quantum computing to develop algorithms for tensor decomposition~\cite{hastings2020classical,zhou2024statistical} for current quantum hardware implementation, with a future outlook on fault-tolerant quantum devices. We proposed a framework (Figure~\ref{fig:qtd_framework}) to apply quantum tensor decomposition and analyze biomedical data to perform various downstream tasks. We acknowledge that there are various challenges in implementing quantum algorithms in PFTQD, due to the parametric nature of the variational quantum algorithms, which need extensive hyperparameter tuning. Furthermore, current quantum computers suffer from hardware noise that can impact computations, leading to instabilities. However, with recent advances in noise mitigation and suppression techniques, we aim to obtain high-quality and robust results from quantum computers in performing the QTD task, at scale. We performed a preliminary resource estimation analysis of implementing QTD in PFTQD using QPE algorithms (Figure~\ref{fig:qubits}) and provide a reaistic roadmap of applying QTD in biomedical data. Integrating quantum algorithms for tensor decompositions with their classical counterparts not only aims to enhance the efficiency of performing TD at scale, but also improve the quality and generalizability of the latent tensor factors in lower dimensions. Quantum computing is an emerging technology which is advancing rapidly with theoretical and practical developments. This comprehensive review of tensor decomposition methods, combined with the proposed development of a quantum tensor decomposition framework, may serve as a model for how quantum computing can influence and drive impact in biomedicine. 

\section*{Acknowledgement}
The authors like to thank Brian Quanz, Vaibhaw Kumar, Charles Chung, Gopal Karemore, and Travis Scholten from IBM Quantum for their insightful feedback.

