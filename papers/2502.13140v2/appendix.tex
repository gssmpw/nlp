\appendix

\section{ PARAFAC2 as a special case of Tucker decomposition}
    Following example shows that PARAFAC2 decomposition is a specialization of Tucker decomposition. Consider a 3-order tensor \( \mathcal{X} \in \mathbb{R}^{I \times J_k \times K} \), where the size of the second mode (\( J_k \)) varies across slices \( k \). For example:
    \[
    \mathcal{X}(:,:,1) \in \mathbb{R}^{3 \times 2 \times 1}, \quad
    \mathcal{X}(:,:,2) \in \mathbb{R}^{3 \times 3 \times 1}.
    \]
    Here:
    \begin{itemize}
        \item \( I = 3 \) (fixed first dimension),
        \item \( J_k \) varies (\( J_1 = 2, J_2 = 3 \)),
        \item \( K = 2 \) (fixed third dimension).
    \end{itemize}
    
    This varying second mode makes \textbf{PARAFAC2 applicable}, while Tucker decomposition remains a general alternative.
    
    PARAFAC2 models this tensor as:
    \[
    \mathcal{X}_k = \mathbf{A} \, \text{diag}(\mathbf{c}_k) \, \mathbf{B}_k^\top,
    \]
    where:
    \begin{itemize}
        \item \( \mathbf{A} \in \mathbb{R}^{I \times R} \) (shared across slices),
        \item \( \mathbf{c}_k \in \mathbb{R}^{R} \) (slice-specific scaling),
        \item \( \mathbf{B}_k \in \mathbb{R}^{J_k \times R} \) (slice-specific factor matrix),
    \end{itemize}
    and \( \mathbf{B}_k^\top \mathbf{B}_k = \mathbf{B}_j^\top \mathbf{B}_j \) for all \( k, j \). 
    
    % For example:
    % \[
    % \mathbf{A} = 
    % \begin{bmatrix}
    % 1 & 0.5 \\
    % 0.8 & 0.3 \\
    % 0.6 & 0.1
    % \end{bmatrix}, \quad
    % \mathbf{B}_1 = 
    % \begin{bmatrix}
    % 1 & 0 \\
    % 0 & 1
    % \end{bmatrix}, \quad
    % \mathbf{B}_2 = 
    % \begin{bmatrix}
    % 1 & 0 \\
    % 0.7 & 0.7 \\
    % 0 & 1
    % \end{bmatrix}.
    % \]
    
    
    % Tucker decomposition represents the tensor as:
    % \[
    % \mathcal{X} = \mathcal{G} \times_1 U^{(1)} \times_2 U^{(2)} \times_3 U^{(3)},
    % \]
    % where:
    % \begin{itemize}
    %     \item \( \mathcal{G} \) is the core tensor (fixed size),
    %     \item \( U^{(q)} \) are factor matrices for each mode (\( q = 1, 2, 3 \)).
    % \end{itemize}
    
    
    % For the varying \( J_k \), Tucker decomposition can still apply by "padding" or treating the slices separately within \( U^{(2)} \), storing these variations in the core tensor \( \mathcal{G} \), which serves as a general model allowing for the representation of tensors with flexible core structures and mode-specific factors. PARAFAC2 tensor decomposition, a subset of Tucker decomposition, introduces flexibility in handling tensors with variations across one mode while maintaining structural consistency across others.

\section{DEDICOM as a Special Case of PARAFAC}
\label{sec:dedicom-to-parafac}

The DEDICOM model can be seen as a constrained version of the PARAFAC decomposition where:
\begin{enumerate}
    \item The same factor matrix \( A \) is used in both the first and second modes.
    \item An asymmetric relationship matrix \( R \) governs interactions between components.
    \item Diagonal matrices \( D_k \) provide mode-3 scaling.
\end{enumerate}

PARAFAC Decomposition (Section 3 of \cite{kolda2009tensor}):

The PARAFAC (or CP) decomposition for a third-order tensor \( \mathcal{X} \in \mathbb{R}^{I \times J \times K} \) is expressed as:
\begin{equation}
    \mathcal{X} \approx \sum_{r=1}^{R} a_r \circ b_r \circ c_r,
\end{equation}
where \( a_r \in \mathbb{R}^I \), \( b_r \in \mathbb{R}^J \), and \( c_r \in \mathbb{R}^K \) are factor vectors for each mode, and \( \circ \) denotes the outer product.

In elementwise form:
\begin{equation}
    x_{ijk} \approx \sum_{r=1}^{R} a_{ir} b_{jr} c_{kr}.
\end{equation}

DEDICOM Decomposition (Section 5.4 of \cite{kolda2009tensor}):

The three-way DEDICOM model for a tensor \( \mathcal{X} \in \mathbb{R}^{I \times I \times K} \), assuming square slices in mode-1 and mode-2, is given by:
\begin{equation}
    \mathcal{X}_k \approx A D_k R D_k A^\top, \quad \text{for } k = 1, \dots, K,
\end{equation}
where:
\begin{itemize}
    \item \( A \in \mathbb{R}^{I \times R} \) is the factor matrix describing latent components.
    \item \( R \in \mathbb{R}^{R \times R} \) captures asymmetric relationships between components.
    \item \( D_k \in \mathbb{R}^{R \times R} \) is a diagonal matrix representing component weights for slice \( k \).
\end{itemize}

Rewriting the DEDICOM model in elementwise form:
\begin{equation}
    x_{ijk} = \sum_{p=1}^{R} \sum_{q=1}^{R} a_{ip} (D_k)_{pp} r_{pq} (D_k)_{qq} a_{jq}.
\end{equation}

Define \( c_{kr} = (D_k)_{rr} \), the diagonal entries of \( D_k \). Then:
\begin{equation}
    x_{ijk} = \sum_{p=1}^{R} \sum_{q=1}^{R} a_{ip} r_{pq} a_{jq} c_{kp} c_{kq}.
\end{equation}

This structure resembles the PARAFAC model but with:
\begin{itemize}
    \item Shared Factors: The first two modes share the same factor matrix \( A \).
    \item Asymmetry: The relationship matrix \( R \) introduces asymmetry, which is not typically present in PARAFAC.
    \item Scaling: The diagonal matrices \( D_k \) impose additional structure through \( c_{kr} \).
\end{itemize}


\section{INDSCAL as a special case of CANDELINC}
\label{sec:indscal-to-candelinc}
INDSCAL Model (Section 5.1 of \cite{kolda2009tensor}):

   \begin{equation}
       \mathcal{X} \approx \langle A, A, C \rangle = \sum_{r=1}^{R} a_r \circ a_r \circ c_r,
   \end{equation}
   where:

   \begin{enumerate}
       \item $A \in \mathbb{R}^{I \times R}$ is the factor matrix shared across the first two modes (symmetry is enforced).
       \item $C \in \mathbb{R}^{K \times R}$ represents the third mode's variation.
   \end{enumerate}

The symmetry implies $x_{ijk} = x_{jik}$ for all $i$, $j$, and $k$.

CANDELINC Model (Section 5.3 of \cite{kolda2009tensor}):
   \begin{equation}
       \mathcal{X} \approx \langle \Phi_A A, \Phi_B B, \Phi_C C \rangle,
   \end{equation}
   where:

   \begin{enumerate}
       \item $\Phi_A, \Phi_B, \Phi_C$ are constraint matrices.
       \item For example, $\Phi_A A$ represents that the factor matrix $A$ is constrained to lie in the subspace defined by $\Phi_A$.
   \end{enumerate}




The mapping of INDSCAL onto CANDELINC is given below.
\begin{enumerate}
    \item  In the INDSCAL model, the symmetry between the first two modes enforces that $B = A$.
    \item This can be written as a **linear constraint** in the CANDELINC framework: $\Phi_B = \Phi_A = I, \quad B = A$.
  \item Substituting these constraints into the CANDELINC model gives:
  \begin{equation}
      \mathcal{X} \approx \langle \Phi_A A, \Phi_A A, \Phi_C C \rangle.
  \end{equation}
\end{enumerate}


In CANDELINC, the linear constraints are applied via matrices $\Phi_A, \Phi_B, \Phi_C$:

\begin{enumerate}
    \item Setting $\Phi_B = \Phi_A = I$ ensures that $A$ is shared across the first two modes.
    \item The matrix $\Phi_C$ allows $C$ to remain unconstrained, capturing the individual differences specific to the third mode.
\end{enumerate}

CANDELINC accommodates a broader range of linear constraints through $\Phi_A, \Phi_B, \Phi_C$. By specifying: $\Phi_B = \Phi_A = I \quad \text{and enforcing symmetry on } A$,
the INDSCAL model is obtained. Thus, INDSCAL is a specialized case of CANDELINC with symmetry and shared structure constraints.






\section{Topic Modeling Details}

\setcounter{table}{1}
\begin{table}[h!]
\centering
 \begin{tabular}{||c | c||} 
 \hline
 \textbf{Search Term} & \textbf{No. of Documents} \\ 
 \hline\hline
genomics & 46 \\ 
 \hline
transcriptomics & 31 \\ 
 \hline
proteomics & 4 \\ 
 \hline
metabolomics & 3 \\ 
 \hline
epigenomics & 6 \\ 
 \hline
microbiomics & 3 \\ 
 \hline
multiomics & 27 \\ 
 \hline
cancer & 81 \\ 
 \hline
cardiovascular disease & 21 \\ 
 \hline
diabetes & 8 \\ 
 \hline
alzheimer's disease & 20 \\ 
 \hline
neurological disorder & 55 \\ 
 \hline
autoimmune disease & 2 \\ 
 \hline
kidney disease & 4 \\ 
 \hline
obesity & 4 \\ 
 \hline
medical imaging & 218 \\ 
 \hline
 \end{tabular}
 \caption{Number of documents extracted from PubMed for each search term. Each term was used along with ``\texttt{AND} tensor decomposition".}
\end{table}

\begin{table}[h!]
\centering
 \begin{tabular}{||c | c||} 
 \hline
 \textbf{Topic Representation} & \textbf{No. of Documents} \\ 
 \hline\hline
cell\_omics\_expression\_single & 144 \\
\hline
diffusion\_images\_imaging\_noise & 129 \\
\hline
brain\_connectivity\_functional\_eeg	& 108 \\
\hline
segmentation\_dti\_imaging\_95 & 42 \\
\hline
imaging\_bayesian\_high\_neuroimaging & 17 \\
\hline
ecg\_avf\_cardiac\_flow	& 16 \\
\hline
tissue\_mrsi\_matrix\_ncpd & 14 \\
\hline
pd\_ad\_task\_progression & 13 \\
\hline
kidney\_expression\_disease\_cell & 11 \\
 \hline
 \end{tabular}
 \caption{Number of documents per topic embedding.}
\end{table}
