\documentclass{article}
\usepackage{graphicx} % Required for inserting images
\usepackage{authblk}
\usepackage[margin=1in]{geometry}
\usepackage{amsmath, amssymb}
\usepackage{dirtytalk}
\usepackage{hyperref}
\usepackage[english]{babel}
\usepackage[toc,page]{appendix}
\usepackage{subcaption}
\usepackage[x11names, svgnames, dvipsnames]{xcolor}
\usepackage{wrapfig}
\usepackage{dirtytalk}
\usepackage{lineno}
\usepackage[T1]{fontenc}
\usepackage{tikz}
\usepackage{cite}
\usepackage{float}
\usepackage{enumitem}
\usetikzlibrary{trees}


%\linenumbers
\newtheorem{definition}{Definition}
\newcommand{\AB}[1]{\textcolor{cyan}{#1}} 

\newcommand{\os}[1]{\textcolor{red}{#1}} 
\newcommand{\hd}[1]{\textcolor{blue}{#1}} 


\title{Towards Quantum Tensor Decomposition in  Biomedical Applications}

\author[1]{Myson Burch}
\author[2,3]{Jiasen Zhang}
\author[4]{Gideon Idumah}
\author[5]{Hakan Doga}
\author[3,6]{Richard Lartey}
\author[7]{Lamis Yehia}
\author[3,6]{Mingrui Yang}
\author[8]{Murat Yildirim}
\author[7,10]{Mihriban Karaayvaz}
\author[9]{Omar Shehab}
\author[2]{Weihong Guo}
\author[4,10,11]{Ying Ni}
\author[1]{Laxmi Parida}
\author[3,6,12]{Xiaojuan Li}
\author[1]{Aritra Bose\thanks{Correspondence: a.bose@ibm.com}}

% \author[1]{}
\affil[1]{IBM Research, Yorktown Heights, NY}
\affil[2]{Department of Mathematics, Case Western Reserve University, Cleveland, OH}
\affil[3]{Program for Advanced Musculoskeletal Imaging (PAMI), Cleveland Clinic, Cleveland, OH}
\affil[4]{Center for Immunotherapy and Precision Immuno-Oncology, Lerner Research Institute, Cleveland Clinic, Cleveland, OH}
\affil[5]{IBM Quantum, Almaden Research Center, San Jose, CA}
\affil[6]{Department of Biomedical Engineering, Lerner Research Institute, Cleveland Clinic, Cleveland, OH}
\affil[7]{Genomic Medicine Institute, Lerner Research Institute, Cleveland Clinic, Cleveland, OH}
\affil[8]{Department of Neurosciences, Lerner Research Institute, Cleveland Clinic, Cleveland, OH}
\affil[9]{IBM Quantum, Yorktown Heights, NY, USA}
\affil[10]{Department of Molecular Medicine, Cleveland Clinic Lerner College of Medicine, Case Western Reserve University, Cleveland, OH}
\affil[11]{Case Comprehensive Cancer Center, Case Western Reserve University, Cleveland, OH}
\affil[12]{Department of Biomedical Engineering, Case Western Reserve University, Cleveland, OH}


\date{}

\begin{document}

\maketitle



\begin{abstract}
   Tensor decomposition has emerged as a powerful framework for feature extraction in multi-modal biomedical data. In this review, we present a comprehensive analysis of tensor decomposition methods such as Tucker, CANDECOMP/PARAFAC, spiked tensor decomposition, etc. and their diverse applications across biomedical domains such as imaging, multi-omics, and spatial transcriptomics. To systematically investigate the literature, we applied a topic modeling-based approach that identifies and groups distinct thematic sub-areas in biomedicine where tensor decomposition has been used, thereby revealing key trends and research directions. We evaluated challenges related to the scalability of latent spaces along with obtaining the optimal rank of the tensor, which often hinder the extraction of meaningful features from increasingly large and complex datasets. Additionally, we discuss recent advances in quantum algorithms for tensor decomposition, exploring how quantum computing can be leveraged to address these challenges. Our study includes a preliminary resource estimation analysis for quantum computing platforms and examines the feasibility of implementing quantum-enhanced tensor decomposition methods on near-term quantum devices. Collectively, this review not only synthesizes current applications and challenges of tensor decomposition in biomedical analyses but also outlines promising quantum computing strategies to enhance its impact on deriving actionable insights from complex biomedical data. 
\end{abstract}

%\tableofcontents

\section{Introduction}

Large language models (LLMs) have achieved remarkable success in automated math problem solving, particularly through code-generation capabilities integrated with proof assistants~\citep{lean,isabelle,POT,autoformalization,MATH}. Although LLMs excel at generating solution steps and correct answers in algebra and calculus~\citep{math_solving}, their unimodal nature limits performance in plane geometry, where solution depends on both diagram and text~\citep{math_solving}. 

Specialized vision-language models (VLMs) have accordingly been developed for plane geometry problem solving (PGPS)~\citep{geoqa,unigeo,intergps,pgps,GOLD,LANS,geox}. Yet, it remains unclear whether these models genuinely leverage diagrams or rely almost exclusively on textual features. This ambiguity arises because existing PGPS datasets typically embed sufficient geometric details within problem statements, potentially making the vision encoder unnecessary~\citep{GOLD}. \cref{fig:pgps_examples} illustrates example questions from GeoQA and PGPS9K, where solutions can be derived without referencing the diagrams.

\begin{figure}
    \centering
    \begin{subfigure}[t]{.49\linewidth}
        \centering
        \includegraphics[width=\linewidth]{latex/figures/images/geoqa_example.pdf}
        \caption{GeoQA}
        \label{fig:geoqa_example}
    \end{subfigure}
    \begin{subfigure}[t]{.48\linewidth}
        \centering
        \includegraphics[width=\linewidth]{latex/figures/images/pgps_example.pdf}
        \caption{PGPS9K}
        \label{fig:pgps9k_example}
    \end{subfigure}
    \caption{
    Examples of diagram-caption pairs and their solution steps written in formal languages from GeoQA and PGPS9k datasets. In the problem description, the visual geometric premises and numerical variables are highlighted in green and red, respectively. A significant difference in the style of the diagram and formal language can be observable. %, along with the differences in formal languages supported by the corresponding datasets.
    \label{fig:pgps_examples}
    }
\end{figure}



We propose a new benchmark created via a synthetic data engine, which systematically evaluates the ability of VLM vision encoders to recognize geometric premises. Our empirical findings reveal that previously suggested self-supervised learning (SSL) approaches, e.g., vector quantized variataional auto-encoder (VQ-VAE)~\citep{unimath} and masked auto-encoder (MAE)~\citep{scagps,geox}, and widely adopted encoders, e.g., OpenCLIP~\citep{clip} and DinoV2~\citep{dinov2}, struggle to detect geometric features such as perpendicularity and degrees. 

To this end, we propose \geoclip{}, a model pre-trained on a large corpus of synthetic diagram–caption pairs. By varying diagram styles (e.g., color, font size, resolution, line width), \geoclip{} learns robust geometric representations and outperforms prior SSL-based methods on our benchmark. Building on \geoclip{}, we introduce a few-shot domain adaptation technique that efficiently transfers the recognition ability to real-world diagrams. We further combine this domain-adapted GeoCLIP with an LLM, forming a domain-agnostic VLM for solving PGPS tasks in MathVerse~\citep{mathverse}. 
%To accommodate diverse diagram styles and solution formats, we unify the solution program languages across multiple PGPS datasets, ensuring comprehensive evaluation. 

In our experiments on MathVerse~\citep{mathverse}, which encompasses diverse plane geometry tasks and diagram styles, our VLM with a domain-adapted \geoclip{} consistently outperforms both task-specific PGPS models and generalist VLMs. 
% In particular, it achieves higher accuracy on tasks requiring geometric-feature recognition, even when critical numerical measurements are moved from text to diagrams. 
Ablation studies confirm the effectiveness of our domain adaptation strategy, showing improvements in optical character recognition (OCR)-based tasks and robust diagram embeddings across different styles. 
% By unifying the solution program languages of existing datasets and incorporating OCR capability, we enable a single VLM, named \geovlm{}, to handle a broad class of plane geometry problems.

% Contributions
We summarize the contributions as follows:
We propose a novel benchmark for systematically assessing how well vision encoders recognize geometric premises in plane geometry diagrams~(\cref{sec:visual_feature}); We introduce \geoclip{}, a vision encoder capable of accurately detecting visual geometric premises~(\cref{sec:geoclip}), and a few-shot domain adaptation technique that efficiently transfers this capability across different diagram styles (\cref{sec:domain_adaptation});
We show that our VLM, incorporating domain-adapted GeoCLIP, surpasses existing specialized PGPS VLMs and generalist VLMs on the MathVerse benchmark~(\cref{sec:experiments}) and effectively interprets diverse diagram styles~(\cref{sec:abl}).

\iffalse
\begin{itemize}
    \item We propose a novel benchmark for systematically assessing how well vision encoders recognize geometric premises, e.g., perpendicularity and angle measures, in plane geometry diagrams.
	\item We introduce \geoclip{}, a vision encoder capable of accurately detecting visual geometric premises, and a few-shot domain adaptation technique that efficiently transfers this capability across different diagram styles.
	\item We show that our final VLM, incorporating GeoCLIP-DA, effectively interprets diverse diagram styles and achieves state-of-the-art performance on the MathVerse benchmark, surpassing existing specialized PGPS models and generalist VLM models.
\end{itemize}
\fi

\iffalse

Large language models (LLMs) have made significant strides in automated math word problem solving. In particular, their code-generation capabilities combined with proof assistants~\citep{lean,isabelle} help minimize computational errors~\citep{POT}, improve solution precision~\citep{autoformalization}, and offer rigorous feedback and evaluation~\citep{MATH}. Although LLMs excel in generating solution steps and correct answers for algebra and calculus~\citep{math_solving}, their uni-modal nature limits performance in domains like plane geometry, where both diagrams and text are vital.

Plane geometry problem solving (PGPS) tasks typically include diagrams and textual descriptions, requiring solvers to interpret premises from both sources. To facilitate automated solutions for these problems, several studies have introduced formal languages tailored for plane geometry to represent solution steps as a program with training datasets composed of diagrams, textual descriptions, and solution programs~\citep{geoqa,unigeo,intergps,pgps}. Building on these datasets, a number of PGPS specialized vision-language models (VLMs) have been developed so far~\citep{GOLD, LANS, geox}.

Most existing VLMs, however, fail to use diagrams when solving geometry problems. Well-known PGPS datasets such as GeoQA~\citep{geoqa}, UniGeo~\citep{unigeo}, and PGPS9K~\citep{pgps}, can be solved without accessing diagrams, as their problem descriptions often contain all geometric information. \cref{fig:pgps_examples} shows an example from GeoQA and PGPS9K datasets, where one can deduce the solution steps without knowing the diagrams. 
As a result, models trained on these datasets rely almost exclusively on textual information, leaving the vision encoder under-utilized~\citep{GOLD}. 
Consequently, the VLMs trained on these datasets cannot solve the plane geometry problem when necessary geometric properties or relations are excluded from the problem statement.

Some studies seek to enhance the recognition of geometric premises from a diagram by directly predicting the premises from the diagram~\citep{GOLD, intergps} or as an auxiliary task for vision encoders~\citep{geoqa,geoqa-plus}. However, these approaches remain highly domain-specific because the labels for training are difficult to obtain, thus limiting generalization across different domains. While self-supervised learning (SSL) methods that depend exclusively on geometric diagrams, e.g., vector quantized variational auto-encoder (VQ-VAE)~\citep{unimath} and masked auto-encoder (MAE)~\citep{scagps,geox}, have also been explored, the effectiveness of the SSL approaches on recognizing geometric features has not been thoroughly investigated.

We introduce a benchmark constructed with a synthetic data engine to evaluate the effectiveness of SSL approaches in recognizing geometric premises from diagrams. Our empirical results with the proposed benchmark show that the vision encoders trained with SSL methods fail to capture visual \geofeat{}s such as perpendicularity between two lines and angle measure.
Furthermore, we find that the pre-trained vision encoders often used in general-purpose VLMs, e.g., OpenCLIP~\citep{clip} and DinoV2~\citep{dinov2}, fail to recognize geometric premises from diagrams.

To improve the vision encoder for PGPS, we propose \geoclip{}, a model trained with a massive amount of diagram-caption pairs.
Since the amount of diagram-caption pairs in existing benchmarks is often limited, we develop a plane diagram generator that can randomly sample plane geometry problems with the help of existing proof assistant~\citep{alphageometry}.
To make \geoclip{} robust against different styles, we vary the visual properties of diagrams, such as color, font size, resolution, and line width.
We show that \geoclip{} performs better than the other SSL approaches and commonly used vision encoders on the newly proposed benchmark.

Another major challenge in PGPS is developing a domain-agnostic VLM capable of handling multiple PGPS benchmarks. As shown in \cref{fig:pgps_examples}, the main difficulties arise from variations in diagram styles. 
To address the issue, we propose a few-shot domain adaptation technique for \geoclip{} which transfers its visual \geofeat{} perception from the synthetic diagrams to the real-world diagrams efficiently. 

We study the efficacy of the domain adapted \geoclip{} on PGPS when equipped with the language model. To be specific, we compare the VLM with the previous PGPS models on MathVerse~\citep{mathverse}, which is designed to evaluate both the PGPS and visual \geofeat{} perception performance on various domains.
While previous PGPS models are inapplicable to certain types of MathVerse problems, we modify the prediction target and unify the solution program languages of the existing PGPS training data to make our VLM applicable to all types of MathVerse problems.
Results on MathVerse demonstrate that our VLM more effectively integrates diagrammatic information and remains robust under conditions of various diagram styles.

\begin{itemize}
    \item We propose a benchmark to measure the visual \geofeat{} recognition performance of different vision encoders.
    % \item \sh{We introduce geometric CLIP (\geoclip{} and train the VLM equipped with \geoclip{} to predict both solution steps and the numerical measurements of the problem.}
    \item We introduce \geoclip{}, a vision encoder which can accurately recognize visual \geofeat{}s and a few-shot domain adaptation technique which can transfer such ability to different domains efficiently. 
    % \item \sh{We develop our final PGPS model, \geovlm{}, by adapting \geoclip{} to different domains and training with unified languages of solution program data.}
    % We develop a domain-agnostic VLM, namely \geovlm{}, by applying a simple yet effective domain adaptation method to \geoclip{} and training on the refined training data.
    \item We demonstrate our VLM equipped with GeoCLIP-DA effectively interprets diverse diagram styles, achieving superior performance on MathVerse compared to the existing PGPS models.
\end{itemize}

\fi 


\section{A Primer on Tensor Decomposition}~\label{sec:td_def}
Mathematically, tensors are defined as multilinear functions on the Cartesian product of vector spaces \cite{spivak2018calculus}; however, it is helpful to think of tensors as a generalization of matrices to higher dimensions. As a result, tensor decompositions can be viewed as a generalization of matrix decomposition to higher-orders. The \textit{order} of a tensor is usually defined as the number of its dimensions, or equivalently the number of vectors spaces included in the product. As decomposing matrices to lower rank is generally non-unique~\cite{rabanser2017introduction} and only unique under certain constraints, such as imposed by the singular value decomposition (SVD)~\cite{kolda2009tensor}, tensor decompositions in higher-order are shown to be unique with minimal constraints and hence more general.

We will start by defining notation and preliminary properties of tensors (for a more detailed introduction to tensor decomposition methods, we refer to~\cite{kolda2009tensor}). Scalars are denoted in lowercase letters, e.g., $a$. We denote vectors (tensors of order one) in bold lowercase such as $\mathbf{u}$, with its $i^{th}$ element being $u_i$. Matrices (order-two tensors) are denoted in bold uppercase, e.g. $\mathbf{X}$ and element $(i,j)$ is denoted as $x_{ij}$. The $n^{th}$ element in a sequence is denoted as $\textbf{u}^{(n)}$, such that it is the $n^{th}$ vector in a sequence of vectors. Higher-order tensors (order three or more) are defined in Euler script, e.g., $\mathcal{T}$. 

Next, we define important properties of tensors such as its \textit{mode}, which is another name for the \textit{order} of the tensor (also called \textit{way} in some cases).  It is a measure of how many indices exist in the tensor. A zero tensor of any order is defined as a tensor with all its elements equal to zero. A third-order (or 3-way) tensor $\mathcal{T}$ has three indices $T \in \mathbb{R}^{I\times J\times K}$, where $I,J,K$ are the three indices respectively. 
%Analogous to rows and columns, tensors have \textit{fibers}. 
Another important property of tensors are \textit{slices}, which are two-dimensional sections of a tensor, defined by fixing all but two indices. For a third-order tensor $\mathcal{T}$, we will have horizontal, lateral, and frontal slices denoted by $\mathcal{T}_{i::}$, $\mathcal{T}_{:j:}$, and $\mathcal{T}_{::k}$, respectively~\cite{kolda2009tensor}. Moreover, we use the notation ``$\circ$'' to the denote outer product of vectors, and adopt ``$\otimes$'' for the tensor product operation. While these two operations are different and yield different mathematical objects, they are related in the sense that the result of a tensor product can be viewed as the vectorization of an outer product output. Next, we define rank-one tensors and the rank of a tensor before defining different types of tensor decompositions. 

% \begin{definition}~\label{def:rankone}
%    \textbf{Rank-one tensors.} A $d$-way tensor $\mathcal{T} \in \mathbb{R}^{N_1 \times N_2 \times \cdots \times N_d}$ is rank $one$ if it can be written as the outer product of $d$ vectors, 
%     $$
%         \mathcal{T} = \mathbf{u}^{(1)} \otimes \mathbf{u}^{(2)} \otimes \cdots \otimes \mathbf{u}^{(d)} 
%     $$
%     where $\otimes$ is the outer product of factor vectors $\mathbf{u}^{(d)} \in \mathbb{R}^{N_d}$. Each element of the tensor is the product of the corresponding vector elements $t_{n_1,n_2,\cdots,n_d} = u_{i_1}^{(1)}u_{n_2}^{(2)}\cdots u_{n_d}^{(d)}$ $ \forall$  $1 \leq n_d \leq N_d$
% \end{definition}

%{\color{red} Laxmi's def}
\begin{definition}~\label{def:rankone}
   \textbf{(Rank-one tensor).} A $d$-way non-zero tensor $\mathcal{T} \in \mathbb{R}^{N_1 \times N_2 \times \cdots \times N_d}$ is of rank $one$ if and only if it can be written as 
    $$
        \mathcal{T} = \mathbf{u}^{(1)} \circ \mathbf{u}^{(2)} \circ \cdots \circ \mathbf{u}^{(d)}, 
    $$
    where $\mathbf{u}^{(i)} \in \mathbb{R}^{N_i}$ and $\circ$ denotes the outer product. (In other words, each entry $t_{n_1,n_2,\cdots,n_d}$ of the tensor is the product, $u_{n_1}^{(1)}u_{n_2}^{(2)}\cdots u_{n_d}^{(d)}$, of the corresponding vector entries).
\end{definition}



\begin{definition}~\label{def:rank}
    \textbf{(Tensor rank).} The $rank$ of a tensor $\mathcal{T}$, denoted as \texttt{rank}($\mathcal{T}$) is defined as the smallest integer $r$ such that there exist $r$ rank one tensors $\mathcal{T}_1, ..., \mathcal{T}_r$ with
    $$
        \mathcal{T} = \sum_{i=1}^{r} 
        \mathcal{T}_i.
        %% \mathbf{u}_i^{(1)} \otimes \mathbf{u}_i^{(2)} \otimes \dots \otimes \mathbf{u}_i^{(d)}
    $$
    %% where 
    %% each $\mathbf{u}_i^{(k)} \in \mathbb{R}^{n_k}$.
    %% each $\mathcal{T}_i$ is a tensor of rank one. 
\end{definition}
%% The tensor rank can also be defined as the smallest number of components needed in a Canonical polyadic decomposition  to obtain equality in Equation~\ref{eq:cp}~\cite{kolda2009tensor}.

The idea of decomposing a tensor into its polyadic form, i.e. expressing it as the sum of a finite number of rank-one tensor has existed for over a century~\cite{hitchcock1927expression}. It has been re-introduced in the literature as CANDECOMP (canonical decomposition) and PARAFAC (parallel factors) later and collectively known as the CP decomposition~\cite{harshman1970foundations, harshman1972parafac2}. It factorizes a tensor into a sum of component rank-one tensor as defined in Definition~\ref{def:rankone}. 
\begin{figure}[!htbp]
    \centering
    \includegraphics[width=0.8\linewidth]{figures/cp_tensor_v2.png}
    \caption{CP decomposition of a third-order tensor into $r$ rank-one tensors}
    \label{fig:cp}
\end{figure}
%% \begin{definition}\label{def:cp}    \textbf{(CP decomposition).} 
    
    As an example, a third-order CP decomposition (Figure~\ref{fig:cp}) can be written as 
    \begin{equation}\label{eq:cp}
        \mathcal{T} \approx \sum_{i=1}^r \lambda_i \, \mathbf{u}_i^{(1)} \circ \mathbf{u}_i^{(2)} \circ \cdots \circ \mathbf{u}_i^{(d)},
    \end{equation}
    where $\mathbf{u}_i^{(n)} \in \mathbb{R}^{N_d}$ are the factor vectors for order $d$, $\lambda_i$ is the scaling weight, and  $r$ is the tensor rank. 
%% \end{definition}    

\subsection{Hierarchy of tensor decompositions}
The hierarchy of TD methods \cite{kolda2009tensor} reveals a nested structure in which broader frameworks encompass more specialized approaches (Figure~\ref{fig:hierarchy}). Therefore, any computational advantage for a specific tensor decomposition applies to all of its subtypes. At the top of this hierarchy is Tensor Train (TT) decomposition. It represents a higher-order tensor as a sequential \say{train} of lower-order tensors (TT cores), each connected through shared dimensions known as TT ranks~\cite{oseledets2011tensor}. 
\begin{definition}\label{def:tt}
    \textbf{(Tensor Train decomposition).}  Given a $d$-order tensor $\mathcal{T} \in \mathbb{R}^{N_1 \times N_2 \times \dots \times N_d}$, TT decomposition factorizes it into a sequence of \textit{3-way} core tensors $G^{(n)}$ (except for the first and last cores, which are matrices), connected in a train-like structure as a product of $d$ core entries 
    %% ($G^{(n)}(.,.,.)$) 
    as follows,
    \begin{equation}\label{eq:tt}
        \mathcal{T}(n_1, n_2, \cdots, n_d) = \sum_{r_0, r_1, \dots, r_{d-1},r_d}G^{(1)}(r_0,n_1,r_1) G^{(2)}(r_1,n_2,r_2) \cdots  G^{(d)}(r_{d-1},n_d,r_d) 
    \end{equation}
    where $G^{(n)} \in \mathbb{R}^{R_{n-1}\times N_n \times R_n}$ for $1 \leq n \leq d$. $G^{(1)} \in \mathbb{R}^{N_1 \times R_1}$ is a matrix in the first mode and $G^{(d)} \in \mathbb{R}^{R_{d-1} \times N_d}$ is a matrix for the last mode. Here $R_n$ are \texttt{TT-ranks} that control the compression and approximation quality. 
\end{definition}

% Given a \( Q \)-order tensor \( \mathcal{X} \in \mathbb{R}^{N_1 \times N_2 \times \cdots \times N_Q} \), TT decomposition expresses it as a product of \( Q \) cores: \( G_1 \in \mathbb{R}^{N_1 \times R_1}, G_2 \in \mathbb{R}^{R_1 \times N_2 \times R_2}, \ldots, G_Q \in \mathbb{R}^{R_{Q-1} \times N_Q} \), where \( R_1, \dots, R_{Q-1} \) are the TT ranks. This decomposition mitigates the "curse of dimensionality" by reducing the storage complexity from \( O(N^Q) \) to \( O(QNR^2) \), where \( R = \max(R_1, \dots, R_{Q-1}) \) and \( N = \max(N_1, \dots, N_Q) \). 

The computational complexity for constructing the TT decomposition using methods like TT-SVD is $\mathcal{O}(dN^d)$ for unstructured tensors. However, it becomes manageable in practice for low TT ranks and structured tensors, making it highly effective for large-scale, high-dimensional data applications \cite{oseledets2011tensor, Zhang2020trillion}. 


A special case of TT decomposition is Tucker Decomposition~\cite{tucker1966some, zniyed2020high}. It is a tensor factorization method that expresses a high-order tensor as a core tensor multiplied by factor matrices along each mode. 

\begin{definition}\label{def:tucker}
    \textbf{(Tucker decomposition).} Given a $d$-order tensor $\mathcal{T} \in \mathbb{R}^{N_1 \times N_2 \times \cdots \times N_d}$, the decomposition is represented as, 
    \begin{equation}\label{eq:tucker}
        \mathcal{T} = \mathcal{G} \times_1 U^{(1)} \times_2 U^{(2)} \cdots \times_d U^{(d)},
    \end{equation}
     $\mathcal{G} \in \mathbb{R}^{T_1 \times T_2 \times \cdots \times T_d}$ is the core tensor, and $U^{(n)} \in \mathbb{R}^{N_n \times T_n}$ are factor matrices and $\times_n$ denotes $n$-mode matrix multiplication. The multilinear ranks $T_n$ determine the dimensionality of the core tensor along each mode.
\end{definition}

Tucker decomposition is often computed using algorithms like High-Order Singular Value Decomposition (HOSVD). The storage cost is $\mathcal{O}(dNT + T^d)$, where $N = \max(N_n)$, $T = \max(T_n)$, and $T^d$ reflects the size of the core tensor. The computational complexity of HOSVD is $\mathcal{O}(dN^{d-1}T)$, dominated by the SVD of the mode unfoldings of the tensor. Tucker decomposition is widely used for dimensionality reduction, compression, and exploratory analysis of multi-way data~\cite{kolda2009tensor}.

\begin{wrapfigure}{l}{0.5\textwidth}
      \centering
        \includegraphics[width=0.48\textwidth]{figures/td_hierarchy.png}
    \caption{Hierarchy of tensor decompositions. (See Sections ~\ref{sec:indscal-to-candelinc} for mapping of INDSCAL onto CANDELINC, ~\ref{sec:dedicom-to-parafac} for mapping of DEDICOM onto PARAFAC.)}
    \label{fig:hierarchy}
\end{wrapfigure}


Unlike Tucker decomposition, CP imposes a more constrained structure, making it unique under mild conditions and suitable for identifying latent components. The storage complexity is $\mathcal{O}(rdN)$, where $N = \max(N_d)$, the order $d \ll N$, and rank $r$. Computational complexity depends on iterative algorithms like Alternating Least Squares (ALS), requiring $\mathcal{O}(rN^d)$ operations per iteration for unstructured tensors, but can be reduced with sparse or structured data. CP is extensively used in fields like chemometrics, psychometrics, and machine learning for data interpretation and factorization~\cite{kolda2009tensor}. 

% \textit{PARAFAC Decomposition}, which is another name for \textit{Canonical polyadic decomposition} (see Figure ~\ref{fig:hierarchy}) represents a tensor as a sum of rank-one tensors \cite{harshman1970foundations}. Given a \( Q \)-order tensor \( \mathcal{X} \in \mathbb{R}^{N_1 \times N_2 \times \cdots \times N_Q} \), PARAFAC expresses it as \( \mathcal{X} \approx \sum_{r=1}^R \lambda_r \, \mathbf{a}_r^{(1)} \otimes \mathbf{a}_r^{(2)} \otimes \cdots \otimes \mathbf{a}_r^{(Q)} \), where \( \mathbf{a}_r^{(q)} \in \mathbb{R}^{N_q} \) are the factor vectors for mode \( q \), \( \lambda_r \) is the scaling weight, and \( R \) is the tensor rank. Unlike Tucker decomposition, PARAFAC imposes a more constrained structure, making it unique under mild conditions and suitable for identifying latent components. The storage complexity is \( O(QNR) \), where \( N = \max(N_q) \) and \( R \) is typically much smaller than \( N \). Computational complexity depends on iterative algorithms like Alternating Least Squares (ALS), requiring \( O(RN^Q) \) operations per iteration for unstructured tensors, but can be reduced with sparse or structured data. PARAFAC is widely used in chemometrics, psychometrics, and machine learning for data interpretation and factorization.

PARAFAC2 is a generalization of CP decomposition that relaxes the constraint of fixed factors and is applicable to a collection of matrices where each have the same number of columns but, varying number of rows. PARAFAC2 applies the same factor across one mode while allowing the other factor matrix to vary, rendering it suitable for real-world scenarios where one mode varies in size or structure across slices \cite{harshman1972parafac2}. Like CP, it expresses a tensor as a sum of rank-one components. However, PARAFAC2 introduces slice-specific factor matrices \( \mathbf{B}_k \) for the varying mode, with the constraint \( \mathbf{B}_k^\top \mathbf{B}_k = \mathbf{B}_j^\top \mathbf{B}_j \) for all \( k, j \). 
\begin{definition}\label{def:parafac2}
    \textbf{PARAFAC2 decomposition.}  The decomposition is given by
    $$
    \mathcal{X}_k = \mathbf{A} \, \text{diag}(\mathbf{c}_k) \, \mathbf{B}_k^\top,
    $$
    where $\mathbf{A}$ is shared across slices, $\mathbf{c}_k$ contains slice-specific weights, and $\mathbf{B}_k$ adapts to the varying dimensions.  
\end{definition}

 The storage complexity of PARAFAC2 is $\mathcal{O}(rN + r\sum_k J_k )$, where $N$ is the size of the first mode, $J_k$ the size of the varying mode, and $r$ the rank. Computational complexity depends on iterative methods like ALS, typically requiring $\mathcal{O}(rN \max(J_k) K)$ operations per iteration for $K$ slices. PARAFAC2 is widely used for longitudinal or time-varying data in applications like recommender systems and signal processing. PARAFAC2 decomposition is a specialization of Tucker decomposition (See Appendix for a detailed example). 

 Within PARAFAC2, the CP decomposition emerges as a more constrained subset, representing tensors as a sum of rank-one components (Definition~\ref{def:rankone}). Furthermore, spiked tensor decomposition, CANDELINC, and DEDICOM tensor decompositions are subsets of CP decomposition, designed for specific applications or structural assumptions. Finally, INDSCAL tensor decomposition is a refined version of CANDELINC, incorporating individual differences scaling to model individual variation in tensor components. 
 
 The spiked tensor decomposition requires structural assumptions on the data such as the existence of a low-rank signal tensor in presence of noise. The assumption is that of a \say{spiked} structure, which separates the underlying signal from the noise. It is a specific case of CP decomposition that, like CP, recovers a low-rank structure of the tensor. However, when the signal-to-noise ratio (SNR) is low, it resembles the CP decomposition. 
 \begin{definition}\label{def:st}
     \textbf{Spiked Tensor decomposition.} An underlying statistical model called the \say{spiked tensor model}, introduced in \cite{richard2014statistical}, is considered, where a $N$-dimensional real or complex signal vector $v_{sig}$ is randomly chosen to form 
     \begin{equation}\label{eq:st}
          T_0 = \lambda v_{sig}^{\otimes p} +G
     \end{equation} 
       
    In this spiked tensor formulation, $\lambda$ represents the signal strength or SNR and $G$ is Gaussian noise. The task is either to approximate the signal vector $v_{sig}$ or to determine whether the signal can be detected above some predetermined threshold.
 \end{definition}


The hierarchical framework of tensor decompositions, based on their computational complexity classes (Figure~\ref{fig:hierarchy}), highlights the interrelatedness of TD methods while showcasing their suitability for distinct scenarios and data characteristics.

\subsection{Phase transition of computational hardness}~\label{sec:hardness}
A computational problem is said to exhibit a \textit{phase transition} if there exists a sharp threshold in some control parameter, such as signal-to-noise ratio or rank, at which the problem changes from being computationally feasible (solvable in polynomial time) to infeasible (requiring super-polynomial or exponential time). Mathematically, let $P(n)$ be a problem instance of size $n$ with a control parameter $\theta(n)$. The computational phase transition occurs at some critical threshold $\theta_c(n)$, such that:

\[
\lim_{n \to \infty} \mathbb{P}(\text{efficient algorithm solves } P(n)) = \begin{cases} 1, & \theta(n) < \theta_c(n) \\ 0, & \theta(n) > \theta_c(n) \end{cases}
\]

where \say{efficient algorithm} refers to an algorithm with polynomial runtime in $n$.

Various phase transition parameters are available to study the computational hardness in tensor decomposition. A commonly used hardness metric to quantify computational tractability is \textit{Minimum Mean Squared Error} (MMSE) in low-degree polynomials~\cite{wein2023average}. In a random low-rank tensor decomposition, if the tensor rank satisfies $r \ll n^{k/2}$, then $\text{MMSE} \to 0$ and the decomposition is considered computationally feasible with efficient signal recovery . Otherwise, if it is $r \gg n^{k/2}$, then $\text{MMSE} \to 1$ and the decomposition is infeasible without efficient recovery~\cite{wein2023average}. This threshold applies to general order-$k$ tensors and provides a tight characterization of computational hardness.
% establishes a computational phase transition for random low-rank tensor decomposition, showing that: 

% \begin{itemize}
%     \item If the tensor rank satisfies $r \ll n^{k/2}$, decomposition is computationally feasible.
%     \item If $r \gg n^{k/2}$, decomposition is computationally infeasible, even though statistically possible.
% \end{itemize}

Signal strength is also used as a parameter for phase transition. In higher-order tensor clustering, the decay rate of the signal, $\beta$ controls the phase transition in the form of $\tilde{\Theta}(n^{-\beta})$~\cite{luo2022tensor}. Specific values of $\beta$ indicates computationally easy and hard regimes. 
% Luo and Zhang (2022) \cite{luo2022tensor} define phase transitions in high-order tensor clustering by introducing signal strength parameters:
% \[
% \lambda = \tilde{\Theta}(n^{-\beta}), \quad \mu = \tilde{\Theta}(n^{-\beta}), 
% \]
%are used, with $\beta$ controlling the decay rate of the signal~\cite{luo2022tensor}. The phase transition occurs at specific values of $\beta$, distinguishing computationally easy and hard regimes.
Similarly, for sparse tensor decomposition, sparsity is a crucial parameter. It is defined as 
%Luo \& Zhang (2022) \cite{luo2022tensor}  define:

\[
k = \tilde{\Theta}(n^{\alpha})
\]

where $\alpha$ represents the sparsity in the form of the fraction of non-zero entries in the tensor. Lower values of $\alpha$ makes the tensor decomposition computationally harder~\cite{luo2022tensor}. The Signal-to-Noise Ratio (SNR) is another commonly used parameter for measuring hardness. In a low-rank Bernoulli model

% Wang \& Li (2022) \cite{wang2020learning} introduce a probabilistic model for binary multiway data, defining a low-rank Bernoulli model:

\[
Y | \Theta \sim \text{Bernoulli}(f(\Theta)), 
\]

where $\Theta$ is a low-rank continuous-valued tensor, three  distinct phases of learnability were derived~\cite{wang2020learning}: 
\begin{itemize}
    \item \textit{High SNR (Noise Helps):} Moderate noise improves estimation due to a dithering effect.
    \item \textit{Intermediate SNR (Noise Hurts):} Standard behavior where noise degrades performance.
    \item \textit{Low SNR (Impossible Phase):} Learning is infeasible below a critical threshold.
\end{itemize}
Similarly, in a tensor completion problem, the phase transition in recovering missing values is based on \textit{Inverse SNR}, which governs the feasibility of recovery~\cite{stephan2024non}. The study of the phase transition of computational hardness  allows us to systematically investigate potential use cases for a new algorithm where the state-of-the-art approaches may not be performant.
%Stephan \& Zhu (2024) \cite{stephan2024non} study \textit{tensor completion} using a non-backtracking spectral method and identify a phase transition in recoverability based on:
% \begin{itemize}
%     \item \textit{Inverse Signal-to-Noise Ratio ($\tau_i = \theta / \nu_i$):} Governs the feasibility of recovery.
%     \item \textit{Detection Threshold ($\theta$):} Determines when singular values of the observed tensor are detectable.
%     \item \textit{Aspect Ratio ($\eta$):} Characterizes the difficulty of matrix unfolding in tensor completion.
%     \item \textit{Sample Complexity Threshold ($O(n^{k/2})$):} The number of observed tensor entries required for feasible completion.
% \end{itemize}
% A commonly used hardness metric to quantify computational tractability is \textit{Minimum Mean Squared Error} (MMSE) in low-degree polynomials~\cite{wein2023average}. 
% The \textit{Degree-$D$ MMSE} is defined as:
% \[
% \text{MMSE}_{\leq D} = \inf_{f, \deg f \leq D} \mathbb{E}_a [ ( f(T) - a_{11} )^2 ]
% \

% where $f(T)$ is an estimator using polynomials of degree $D$. 
% The key result is:

% \begin{itemize}
%     \item If $r \ll n^{k/2}$, $\text{MMSE} \to 0$ (efficient recovery is possible).
%     \item If $r \gg n^{k/2}$, $\text{MMSE} \to 1$ (efficient recovery is impossible).
% \end{itemize}


\section{Applications in Biomedical Analysis}
~\label{sec:td_biomed}
Tensor decomposition extracts latent structures from high dimensional data and preserves interpretability of the features. This is critical in analyzing biomedical data to ensure the insights drawn from lower-dimensional factors are reliable and actionable, leading to biomarker discovery, prognosis, and developing therapeutics for complex diseases. 
% Preserving this interpretability in biological data is of critical importance to ensure the insights drawn from these analyses are reliable and actionable, which is essential when these results directly impact health and scientific discovery. Given the increasing volume and complexity of biomedical data, ranging from multi-omics profiles to high-resolution imaging and spatial transcriptomics, tensor decomposition offers a unique approach to uncover latent biological signals, integrate diverse datasets, and enhance interpretability in data-driven research. 

% \paragraph{Multi-omics} Tensor decomposition helps integrate heterogeneous biological data (e.g., genomics, transcriptomics, proteomics) by identifying shared and distinct molecular signatures across different biological layers. This improves biomarker discovery, patient stratification, and disease subtyping, leading to more personalized medicine approaches.  

% \paragraph{Medical imaging} For MRI, fMRI, and CT scans, tensor methods aid in denoising, feature extraction, and identifying disease-related patterns by decomposing images into meaningful components. This improves diagnostic accuracy and allows for better interpretation of underlying pathophysiological processes.  

% \paragraph{Spatial transcriptomics} Tensor decomposition is used to analyze high-dimensional spatial gene expression data, revealing underlying cellular structures and tissue organization. By capturing spatially varying gene expression patterns, it enhances our understanding of cell-cell interactions and disease microenvironments, such as in cancer or neurodegenerative disorders.  

%\paragraph{Topic modeling} 
To further explore the role of tensor decomposition in biomedical research, we used a topic modeling-based approach by employing BERTopic~\cite{grootendorst2022bertopic} to conduct an unsupervised analysis of the literature on tensor decomposition and biological data in PubMed. By leveraging transformer-based embeddings and topic clustering, we systematically identified key themes and research trends related to TD and its applications in biomedicine. %multi-omics, medical imaging, and spatial transcriptomics. 
The terms we searched were \say{tensor decomposition} in combination with (\texttt{AND}) the following terms: \say{genomics},
\say{transcriptomics},
\say{proteomics},
\say{metabolomics},
\say{epigenomics},
\say{microbiomics},
\say{multiomics} \texttt{OR} \say{multi-omics},
\say{cancer},
\say{cardiovascular disease},
\say{diabetes},
\say{alzheimer's disease},
\say{neurological disorder},
\say{autoimmune disease},
\say{kidney disease},
\say{obesity},
\say{medical imaging}. From those 16 terms, 536 documents were extracted from PubMed in the past ten years. This approach enabled us to identify dominant research areas, emerging applications, and potential gaps in the field, providing a structured overview of how tensor decomposition is being utilized across different biomedical domains.

Our analysis revealed a well-established use of TD in medical imaging, where it has been applied for tasks such as noise reduction, feature extraction, and disease pattern recognition~\cite{bustin2019high,Brender2019,Li2021,Zhang2017,Chatzichristos2019}(Figure~\ref{fig:sidebyside}A), with 218 articles cited over 1,400 times (Supplementary Table 1). This aligns with the long history of tensor methods in signal processing and image analysis. This is followed by applications in complex diseases such as cardiovascular disease~\cite{zhao2019detecting}, with TD methods used to factor ECG signals~\cite{liu2020automated} and X-rays~\cite{sedighin2024tensor}, among other applications. Similarly, TD methods have been used to analyze EEG~\cite{cong2015tensor} and neuroimaging data for neurological diseases such as Alzheimer's disease~\cite{durusoy2018multi}, Parkinson's disease~\cite{pham2017tensor}, stroke~\cite{zhou2024tensor}, glioblastoma~\cite{yang2014discrimination}, etc. More recently, we observed a slight increase in the application of TD to multi-omics data, reflecting the growing need for integrative approaches in systems biology (Figure~\ref{fig:sidebyside}A). This trend highlights the increasing recognition of tensor methods as powerful tools for extracting meaningful patterns from high-dimensional, heterogeneous biological datasets.
\begin{figure}[htbp]
    \centering
    \includegraphics[width=\linewidth]{figures/fig3_pubmed.png}
    \caption{Tensor decomposition applications in biomedical data in the literature. \textbf{A.} Stacked bar chart of the number of papers from PubMed published with the topic \protect\say{tensor decomposition} \texttt{AND} categories listed in the legend in the past decade. \protect\textbf{B.} UMAP of the BERTopic embeddings visualizing the different topic clusters. Each data point is a research paper. Applications of \protect\say{spike tensors} are highlighted with a star symbol.}
   \label{fig:sidebyside}
\end{figure}

% \begin{figure}[htbp] 
%     \centering
%     \begin{subfigure}[b]{0.45\textwidth}
%         \centering
%         \includegraphics[width=\linewidth]{figures/doc_trend.png}
%         \caption{Stacked bar chart of the number of papers from PubMed published with the topic \say{tensor decomposition} \texttt{AND} categories listed in the legend in the past decade. }
%         \label{fig:fig1}
%     \end{subfigure}
%     \hfill
%     \begin{subfigure}[b]{0.45\textwidth}
%         \centering
%         \includegraphics[width=\linewidth]{figures/umap_w_legend.png}
%         \caption{UMAP of the BERTopic embeddings visualizing the different topics found for tensor decomposition applications in the literature across past ten years. Documents relating to \say{spike tensors} are highlighted with a star symbol. Every data point is a research paper.}
%         \label{fig:fig2}
%     \end{subfigure}
%     \caption{Tensor decomposition application in biomedical data across the past decade in the literature. }
%     \label{fig:sidebyside}
% \end{figure}

To further interpret the structure of the identified topics, we visualized the document embeddings using UMAP, with topics color-coded for clarity (Figure~\ref{fig:sidebyside}B). This visualization revealed distinct topic clusters, highlighting distinct areas of research focus. The most represented topics were \say{\texttt{cell\_omics\_expression}}, \say{\texttt{diffusion\_imaging\_noise}}, and \say{\texttt{brain\_connectivity\_eeg}}, reflecting the strong presence of tensor decomposition in imaging-related applications. The \say{\texttt{cell\_omics\_expression}} topic encompassed studies leveraging tensor methods for multi-omics and spatial transcriptomics to analyze gene expression across cells and tissues~\cite{hore2016tensor,Omberg2007,Ramdhani2020,Wang2019}. The \say{\texttt{brain\_connectivity\_eeg}} topic captured research applying TD to neurological and electrophysiological data, particularly EEG and fMRI, for brain signal processing and cognitive studies~\cite{Li2021,Chatzichristos2019}. The \say{\texttt{diffusion\_imaging\_noise}} topic represented works focused on image denoising, enhancement, and reconstruction in medical imaging, further reinforcing the dominant role of tensor methods in this domain~\cite{bustin2019high,Brender2019,Li2021,Zhang2017,Chatzichristos2019}. More information on the other topic clusters are included in the Supplementary Tables 1-3.

This exploration highlights the significant role of TD in biomedicine, demonstrating its widespread application in medical imaging, multi-omics, and neuroscience. The increasing adoption of these methods emphasizes their value in extracting meaningful insights from complex biomedical data. Tensors are unique in their applications in multi-modal, multi-omics data because they have the ability to integrate these data sets, find linear and non-linear ineteractions between them, and extract meaningful latent representations which can be then used to perform downstream tasks~\cite{kolda2009tensor, luo2017tensor}. We provide a framework for integrating TD methods, including emerging methods such as quantum tensor decomposition in Figure~\ref{fig:qtd_framework}. Although insightful, this investigation is far from exhaustive and does not provide a holistic view on the literature with specific sub-areas such as genomics, MRI images, electronic health records (EHR), spatial transcriptomics, etc. having more applications of TD~\cite{hore2016tensor, olesen2023tensor, sedighin2024tensor, ho2014marble, song2023gntd}. Hence, we will focus on each of these sub-areas broadly categorized into biomedical imaging, multi-omics analysis, and spatial transcriptomics. 
%Looking ahead, we aim to explore the potential of quantum computing to further enhance tensor decomposition techniques, as quantum approaches may offer new computational efficiencies and capabilities for handling the ever-growing scale and complexity of biomedical datasets.




\subsection{Biomedical Imaging}\label{sec:td_imaging}
%\subsubsection{Applications}

Tensor decomposition methods have emerged as powerful tools for analyzing high-dimensional data in biomedical imaging~\cite{sedighin2024tensor}.
%By leveraging the ability to disentangle high-dimensional data into lower-dimensional components, these techniques can reduce computational complexity while preserving meaningful structural information, which is often obscured in raw data. 
Applications span a wide range of domains, including medical image denoising, super resolution, reconstruction, and so on, where tensors effectively model spatial, temporal, and spectral relationships in MRI and hyperspectral images. Beyond MRI and conventional imaging modalities, TD techniques have shown promise in analyzing high-dimensional datasets generated by multiphoton microscopy, including fluorescence-based and label-free modalities such as second harmonic generation (SHG) and third harmonic generation (THG) imaging~\cite{jamesdarian2021recent}.  In this section, we will review these applications from a methods perspective by discussing how each tensor decomposition technique is used in biomedical imaging analysis. 

\paragraph{CP decomposition.} CP decomposition is widely used in real-world data due to its simplicity, interpretability, and ability to efficiently represent multi-dimensional data. It decomposes the data tensor into a sum of rank-one components (Equation~\ref{eq:cp}) to uncover latent structures, which is particularly useful in data with spatial, temporal, and spectral dimensions like MRI data and hyperspectral image. CP decomposition has been used to reconstruct images from under-sampled data like brain \cite{li2023learned, hou2020matrix}, cardiac \cite{wu2018multiple} MRI and CT scans \cite{zhang2016tensor}. Using the low-rank nature of medical images it achieves high-quality reconstructions with reduced scan times. In magnetic resonance fingerprinting (MRF) reconstruction, CP decomposition is used as a component in a neural network to learn low-rank tensor priors~\cite{li2023learned}, avoiding computationally expensive SVD on high-dimensional data.  

CP decomposition can help reduce redundancy and has been used for brain MRI denoising~\cite{cao2016tensor, cui2020multidimensional}. By selecting the dominant components in CP decomposition, noise and artifacts in medical images can be effectively removed, preserving only the significant structures or patterns. Variants such as Bayesian CP decomposition have been proposed to recover the de-noised tensor from a noisy tensor by estimating rank-one tensors through a variational Bayesian inference strategy~\cite{cao2016tensor}. CP decomposition has been used in a dental CT super resolution task producing high SNR with robust segmentation quality~\cite{hatvani2018tensor}. It is also used in hyperspectral image unmixing with applications in fluorescence microscopy of retina~\cite{dey2019tensor}. In this work, different excitation wavelengths were assembled as a tensor and a non-negative CP decomposition was performed to obtain low-rank factors. Other applications include cardiac $T_1$ mapping \cite{yaman2019low} and feature extraction for multi-modal data, where a robust coupled CP decomposition was used with an assumption that the tensors shared the first factor marix~\cite{zhao2023robust}. In multiphoton microscopy, CP decomposition has been utilized to extract meaningful components from 3D distribution in disease models such as cancer and fibrosis~\cite{uckermann2020label}.
%In Cao \textit{et al.}'s work \cite{cao2016tensor}, a Bayesian CP decomposition method is proposed to recover the clean tensor from its noisy observation, in which the rank-one tensors are estimated through a variational Bayesian inference strategy. 

% For it providing simple representations for high dimensional tensors, CP decomposition has also been used for various tasks such as dental CT super resolution\cite{hatvani2018tensor}, hyperspectral image unmixing for fluorescence microscopy of retina \cite{dey2019tensor}, cardiac $T_1$ mapping \cite{yaman2019low}, and feature extraction for multimodal data \cite{zhao2023robust}. In Karahan \textit{et al.}'s work \cite{dey2019tensor}, images from different excitation wavelengths are assembled as a tensor, and non-negative CP decomposition is applied to the tensor to perform low-rank decomposition. In Zhao \textit{et al.}'s work \cite{zhao2023robust}, to extract common features from multiple tensors, a robust coupled CP decomposition is proposed with the assumption that these tensors share the first factor matrix of CP decomposition.


\paragraph{Tucker decomposition.}
%\os{(Shehab: In this section, every paragraph starts with the same phrase. Maybe some of them should be rephrased? Jiasen: They has been rephrased)}
Tucker decomposition expresses a tensor as a core tensor multiplied by factor matrices along each mode. Compared with HOSVD and CP decomposition, Tucker decomposition allows for arbitrary forms of factor matrices, providing a more general representation. It has been widely used for predicting missing information in high-dimensional images, with applications in super-resolution \cite{gui2017brain, jia2022nonconvex, hatvaniy2021single} and reconstruction \cite{wu2018multiple, wang2021spectral, li2018efficient}. The general model can be summarized as $\mathcal{T} = \mathcal{A}\mathcal{X} $ where $T$ is the observed tensor, $X$ is the original image to be predicted and $X$ is a linear operator. By representing $X$ with Tucker decomposition, the operation for each mode and their interactions can be separated, providing interpretable insight into the data. For example, in super-resolution of dental CT images, the high-resolution image is represented by a Tucker decomposition~\cite{hatvaniy2021single}. Applications of Tucker decomposition also includes 
%regarding  the high resolution image to be predicted is represented with Tucker decomposition. The blurring operation is modeled with three 2D matrices multiplied with the factor matrices.
%Tucker decomposition has also been used to help improve 
$T_1$ mapping \cite{liu2021accelerating, yaman2019low}. Accelerating the acquisition of 3D $T{1\rho}$ mapping of cartilage, the image tensor is defined as the sum of a sparse component and a low rank tensor. Tucker decomposition is used for tensor low-rank regularization~\cite{liu2021accelerating}. It can also be used for approximating higher quality images and improving $T_1$ mapping performance~\cite{yaman2019low}.  

%In Liu \textit{et al.}'s work \cite{liu2021accelerating} regarding accelerating the acquisition of 3D $T{1\rho}$ mapping of cartilage, the image tensor is defined as the sum of a sparse component and a low-rank tensor, and the Tucker decomposition is used for tensor low-rank regularization. In Yaman \textit{et al.}'s work \cite{yaman2019low}, a CP or Tucker decomposition based method is proposed to approximate the images with higher quality and then improve $T_1$ mapping performance.


Another application of Tucker decomposition is in medical image fusion \cite{yin2018tensor, zhang2023tdfusion, chen2024multi}. In these methods, original or processed medical images from diverse modalities are represented with Tucker decomposition sharing the same core tensor or factor matrices. Sparsity regularization is introduced to the core tensors. For example, Yin \textit{et al.}~\cite{yin2018tensor} tackled image fusion by first learning a dictionary from training images. They represented each image as a tensor split into unique sparse core tensors and shared factor matrices. When fusing images, they used these same factor matrices to compute the new tensor representations, letting the core tensors dictate the fusion weights. Beyond MRI, Tucker decomposition has been leveraged to fuse multiphoton imaging data with complementary modalities, such as Raman or fluorescence lifetime imaging, allowing integrative tissue characterization in tumor microenvironments~\cite{karahan2015tensor}. It enables robust feature extraction and pattern recognition, facilitating automated classification of healthy versus diseased tissue regions. 
% In Yin \textit{et al.}'s work \cite{yin2018tensor} about image fusion through dictionary learning, tensor sparse representations of a set of training images are obtained, with different sparse core tensors and sharing factor matrices for each mode. In the fusion step, the tensor sparse representations of the fused images are computed similarly with the learned factor matrices. Finally the fusion weights are decided by the core tensors of the fused images.
Besides, Tucker decomposition has shown the potential to be used for medical image segmentation \cite{khan2022deep, weber2025posttraining}, usually as a component of a deep neural network. In knee cartilage MRI segmentation, the original input images are reconstructed by Tucker decomposition with low rank~\cite{khan2022deep}. 
%Khan \textit{et al.}'s work \cite{khan2022deep} about  
Both the original and reconstructed inputs are fed into the neural network to learn the inter-dimensional tissue structures. In another work~\cite{weber2025posttraining}, the Tucker-decomposed convolution is proposed to replace the conventional 3D convolution and improve computational efficiency of the neural network. 

The application of Tucker decomposition in MRI and multiphoton brain imaging data extends to optogenetics and imaging studies, where simultaneous stimulation and recording of neuronal populations require advanced computational techniqus to disentangle overlapping signals. It separates the evoked responses from spontaneous activity in high-dimensional optogenetic data, aiding the identification of circuit-level changes in neurodevelopmental and neurodegenerative disorders~\cite{erol2022tensors}. 


\paragraph{HOSVD.} The applications of HOSVD, a generalization of the matrix SVD, are ubiquitous in biomedical image analysis. As a constraint case of Tucker decomposition, the factor matrices and core tensor of HOSVD are orthogonal. Therefore, HOSVD is less flexible than Tucker decomposition but can ensure decorrelation between components along each mode. For its ability to use the redundancy in high-dimensional data, HOSVD has been used for brain MRI denoising \cite{fu20163d, zhang2017denoise, bustin2019high, wang2020modified, kim2021denoising, olesen2023tensor}. Specifically,
%a  In Olesen \textit{et al.}'s work \cite{olesen2023tensor}, 
Marchenko-Pastur principal component analysis (MPPCA), a matrix-based method for denoising, is generalized to tensor MPPCA based on HOSVD so that it can directly process multidimensional MRI data~\cite{olesen2023tensor}. A patch-based HOSVD method for denoising has also been proposed, combined with a global HOSVD method to mitigate stripe artifacts in the outputs~\cite{zhang2017denoise}.  
Like Tucker decomposition, HOSVD has been used for medical image reconstruction~\cite{liu2021calibrationless, yi2021joint, roohi2016dynamic, zhang2022compressed} of MRI data by exploiting sparsity and low rank properties. It has also been applied in low-rank tensor approximation of the multi-slice image tensor in parallel MRI reconstruction~\cite{liu2021calibrationless}. HOSVD, analogous to the Tucker decomposition, enables the computation of a core tensor and the factorization of three-way tensors into three matrices. This capability has proven particularly effective for feature extraction, as demonstrated in a blood cell recognition task and wavelet transform in colored pupil images, where they are naturally represented as three-way tensors~\cite{le2023tensor,giap2023adaptive}. 

HOSVD has found utility in analyzing complex spatiotemporal patterns in brain imaging data acquired through single-photon and multiphoton techniques such as two photon and three-photon calcium imaging~\cite{grewe2010high}. It has been applied to denoise neuronal activity maps while preserving underlying spatiotemporal dynamics by reducing motion artificats in calcium imaging datasets. It improved the detection of neuronal ensembles in awake behaving animals~\cite{cho2023robust}. 


% HOSVD has also been used for feature extraction\cite{le2023tensor, giap2023adaptive}. In Le \textit{et al.}'s work \cite{le2023tensor} about feature extraction for blood cell recognition, the color image is represented as a 3D tensor. Through HOSVD, the core tensor is computed and split into three matrices. While the middle one is kept unchanged, the other two are replaced with all-zero matrices and then the feature is obtained by reconstructing the image with the new core tensor. In Giap \textit{et al.}'s work \cite{giap2023adaptive} for feature extraction of color pupil image, a 3D tensor is constructed from the color pupil image based on wavelet transform. HOSVD is applied to eliminate redundant information in the tensor and estimate the feature information.


\paragraph{t-SVD.}
    Unlike HOSVD, which is based on tensor-matrix product, tensor-SVD (t-SVD) represents a 3-way tensor as the \say{t-product} of three 3-way tensors. The t-SVD, a relatively recent and nuanced tensor decomposition technique, has been effectively employed in a range of medical imaging applications, particularly for enforcing low-rank regularization constraints. It has been used for MRI reconstruction \cite{jiang2020improved, liu2025dynamic, ai2018dynamic, liu2023low}. Specifically, it has been used to approximate the rank minimization problem, which is NP-hard to a tensor nuclear norm minimization, with applications in low-rank tensor regularization in a dynamic MRI reconstruction model~\cite{liu2025dynamic}. Similar to other tensor decomposition methods, t-SVD has been used in denoising MRI images~\cite{khaleel2018denoising, kong2017new, khaleel2018denoising2} as well as in image segmentation tasks~\cite{shi2021multi}. A low-rank approximation of the noisy image is obtained by thresholding the t-SVD coefficients in the Fourier domain to perform denoising~\cite{khaleel2018denoising2}. For image segmentation of pathological liver CT, a low-rank tensor decomposition is performed based on t-SVD~\cite{shi2021multi}. Specifically, it is used to recover the underlying low-rank structure of the 3D images and generate tumor-free liver atlases.  
    t-SVD based low-rank approximations have also been used to model neural network connectivity and extract functional components from large-scale neural activity data~\cite{williams2018unsupervised}. 


As observed in several tensor decomposition methods applied in biomedical imaging, it is often used together with deep learning techniques such as convolutional neural networks (CNNs)~\cite{yaman2019low,oymak2021learning, khan2022deep, li2023learned}. Deep learning techniques have shown empirical success to process and extract features from images, or to reconstruct them. However, their effectiveness is still a mystery with their heavy dependence on several parameters and their sensitivity to noise along with lack of generalization to out-of-distribution data, and overfitting in limited training samples~\cite{chen2023deep}. Tensor decompositions play a crucial role here extracting meaningful features which increase generalizability of neural network models. Tensor decomposition methods have shown to learn kernels from training data and increase performance of CNNs~\cite{oymak2021learning}. They are also used to reduce training parameters of CNNs and reduce model size, leading to effective training~\cite{liu2023tensor}. Hence, TD methods can be used in conjunction with deep learning methods to enhance the overall throughput and performance of downstream tasks such as prediction, segmentation, reconstruction, super-resolution, etc. in analyzing biomedical images. 

    %For example, in Liu \textit{et al.}'s work \cite{liu2025dynamic}, a multi-directional low-rank tensor regularization is applied to the dynamic MRI reconstruction model. The NP-hard rank-minimization problem is then approximated to a tensor nuclear norm minimization, which can be represented with the singular tensors obtained through t-SVD. 

% t-SVD has also been used for brain MRI denoising \cite{khaleel2018denoising, kong2017new, khaleel2018denoising2}. In Khaleel \textit{et al.}'s work \cite{khaleel2018denoising, khaleel2018denoising2}, low rank approximation of the noisy image is obtained by thresholding the t-SVD coefficients in Fourier domain.


% Besides, t-SVD shows potential to be applied in medical image segmentation. In Shi \textit{et al.}'s work \cite{shi2021multi} for pathological liver CT segmentation, low-rank tensor decompisition (LRTD) method based on an improved t-SVD is applied to different steps of the multi-atlas segmentation framework. Specifically, it is used to recover the underlying low-rank structure of the 3D images and generate tumor free liver atlases to benefit liver segmentation.

% \paragraph{Tensor Decomposition in Tissue Imaging with Multiphoton Microscopy}
% Beyond MRI and conventional imaging modalities, TD techniques have shown promise in analyzing high-dimensional datasets generated by multiphoton microscopy, including fluorescence-based and label-free modalities such as second harmonic generation (SHG) and third harmonic generation (THG) imaging~\cite{jamesdarian2021recent}. These imaging techniques offer high spatial resolution and deep tissue penetration, making them ideal for studying tissue architecture, extracellular matrix composition, and cellular interactions in both healthy and pathological conditions~\cite{campagnola2003second}.
% However, the large-scale, multi-channel datasets acquired from multiphoton imaging pose computational challenges, necessitating advanced decomposition strategies for efficient analysis, denoising, and feature extraction~\cite{vinegoni2020fluorescence}.

% In SHG and THG imaging, tensor decomposition methods such as CP decomposition and Tucker decomposition can be employed to separate intrinsic signal components from noise and enhance structural details in fibrous tissues. For example, CP decomposition has been utilized to extract meaningful components from 3D SHG datasets of collagen-rich tissues, improving segmentation and quantitative analysis of collagen density distribution in disease models such as cancer and fibrosis~\cite{uckermann2020label}. Similarly, Tucker decomposition has been leveraged to fuse multiphoton imaging data with complementary modalities, such as Raman or fluorescence lifetime imaging, allowing for integrative tissue characterization in tumor microenvironments~\cite{karahan2015tensor}. These techniques enable robust feature extraction and pattern recognition, facilitating automated classification of healthy versus diseased tissue regions.

% \paragraph{Tensor Decomposition for Single and Multiphoton Brain Imaging}
% Multiphoton microscopy is widely used in neuroscience to investigate brain activity at the cellular and network levels. Tensor decomposition approaches provide powerful means to analyze the complex spatiotemporal patterns present in brain imaging data acquired through single-photon and multiphoton techniques, such as two-photon and three-photon calcium imaging~\cite{grewe2010high}. In these applications, HOSVD and t-SVD have been applied to denoise and reconstruct neuronal activity maps while preserving the underlying spatiotemporal dynamics. For instance, HOSVD has been used to reduce motion artifacts and enhance neural signal extraction from calcium imaging datasets, improving the detection of neuronal ensembles in awake behaving animals~\cite{cho2023robust}. Similarly, t-SVD-based low-rank approximations have been employed to model neural network connectivity and extract functional components from large-scale recordings~\cite{cai2022review}.

% The application of tensor decomposition in multiphoton brain imaging extends to optogenetics and functional imaging studies, where simultaneous stimulation and recording of neuronal populations require advanced computational techniques to disentangle overlapping signals. Tucker decomposition has been used to analyze high-dimensional optogenetic data, allowing for the separation of evoked responses from spontaneous activity and aiding in the identification of circuit-level changes in neurodevelopmental and neurodegenerative disorders~\cite{erol2022tensors}. As multiphoton imaging technology continues to advance, tensor decomposition methods will play a crucial role in handling the increasing data complexity, improving image reconstruction, and extracting biologically relevant information from large-scale neural imaging experiments~\cite{uckermann2020label}.
% \paragraph{Tensor train}

% \paragraph{Tensor ring}

% \paragraph{Reconstruction}
% Image reconstruction in biomedical image analysis refers to the process of recovering under-sampled image to get into high-quality and interpretable images. It is critical for visualizing anatomical structures, assessing physiological functions, and aiding in diagnosis and treatment planning. Tensor decomposition methods have been applied in this task especially for MRI or dynamic MRI reconstruction, because MRI data usually involves data with three or more dimensions. Besides, such methods can be naturally combined with compressive sensing, a technique overwhelmingly used in image reconstruction. Treating the high dimensional MRI data as tensors rather than matrices help reduce redundancy and achieve better reconstruction quality. 

% \paragraph{Brain, knee, blood cell, eye}
% Brain image analysis is an important issue in biomedical imaging. The objectives of brain image analysis include structral and functional analysis, pathology detection and cognitive studies. The workflow involves brain image denoising, segmentation, feature extraction and so on. 


% \subsubsection{Challenges}
% % \AB{Phase transition/computational hardness discussions with plots following from the notebook shared with you}

% % Tensor based modeling starts with the idea of preserving the high dimensional structures (geometries) of biomedical data during the formulation of the problems. The computation of numerical results often converts the tensor based models into  equivalent matrix/vector based ones and it often requires formulation of large matrices coming from Kronecker operators. State of the art traditional computing hardware is not able to handle the vectorization of the entire biomedical data and one often needs to break the computation into small overlapping patches. However, not all biomedical application problems can be split into smaller patches. For instance, for super resolution of MRI images, the available low resolution image is a global Fourier transformation of the entire underlying high resolution image and thus can't be computed using patches which take only part of the Fourier information. 

% Tensor-based modeling in medical imaging seeks to preserve the intrinsic high-dimensional structure of biomedical images. However, computation of numerical results from tensors often requires converting these models into matrix or vector forms. This leads to the construction of large matrices involving Kronecker products, imposing significant computational burdens on current hardware. Consequently, computations are typically divided into small, overlapping patches, an approach that is not universally applicable, especially in medical imaging. For example, in MRI super-resolution, the low-resolution image is derived from a global Fourier transform of the entire high-resolution image, making patch-based processing infeasible. 

% Another general challenge in tensor decomposition is rank selection. For example, in Tucker decomposition, selecting the size of the core tensor, which is also the rank, is hard. It is significant since it determines the level of dimensionality reduction and the trade-off between performance and computational efficiency. If the rank is too low, capturing the latent structures in the data may be weakened, while a high rank increases computational complexity and memory usage significantly. As we discuss in Section~\ref{sec:hardness}, understanding the optimal rank of the tensor leads to a low MMSE, yielding efficient signal recovery. This phenomenon is true for all tensor decomposition techniques and their applications in medical imaging ranging from MRI to neural activity patterns and calcium imaging. This is demonstrated in Figure ~\ref{fig:mri_tensor_decomposition} using multi-slice MRI data with image size $512\times784\times912$ and voxel size $0.2\times0.18\times0.18$~mm, 
% %\os{(Shehab: People from quantum background may have no clue what it means. Are these voxels?)}, 
% where the dimensions represent the number of slices (512), and the in-plane resolution  ($784\times912$) of the imaging sequence. 
% %\os{(Shehab: For uninitiated readers like me, can we explain why the memory usage and execution time are bad? For example memory is cheap and 1.5 GB is nothing. It seems multi-volume MRI data could take 100+ GB when Voxel size: 0.5 $\times$ 0.5 $\times$ 0.5 mm, Matrix size: 512 $\times$ 512 $\times$ 512, Single 3D volume: $\sim$256 MB, Multi-subject dataset (400+ subjects, multiple sequences): 100+ GB. Maybe we can scale up the problem such that the memory requirement looks really bad, in hundreds of GBs? Same is true for execution time. Why 30 second is bad? )}.

% \begin{figure}[ht]
%     \centering
%     \subfloat{\includegraphics[width=0.45\textwidth]{figures/mri_memory_usage.png}}
%     \hspace{0.5cm} % Adds some spacing between images
%     \subfloat{\includegraphics[width=0.45\textwidth]{figures/mri_execution_time.png}}
%     \caption{Memory usage and execution time for tensor decomposition of 4D MRI data of size (384, 384, 28, 7). \os{(Can we increase the font size of the axis labels?)}}
%     \label{fig:mri_tensor_decomposition}
% \end{figure}
% Besides, real-world tensors in medical imaging, often contain noise or artifacts that reduce performance and increase computational requirements for robust solutions. Performance reduction when adding noise has been reported in different methods using CP decomposition \cite{zhao2023robust}, Tucker decomposition \cite{prevost2020hyperspectral}, t-SVD \cite{liu2025dynamic}. 

% Applications of TD methods in tissue and brain imaging also leads to significant challenges, particularly,  in the context of high-speed multiphoton microscopy and large-scale neural recordings. These challenges stem from the high-dimensional and multi-modal nature of the data acquired from multiphoton imaging techniques, such as second and third harmonic generation microscopy, two-photon and three-photon imaging, and functional calcium imaging. These datasets are inherently large, spanning spatial, temporal, and spectral domains, making tensor-based approaches computationally demanding~\cite{kolda2009tensor}. 
% %Existing tensor decomposition methods often struggle with the sheer volume of data, requiring advanced computational strategies to efficiently process and store high-dimensional image stacks without loss of information \os{(Shehab: Does this paragraph need some citations?)}.

% Another critical challenge is motion artifacts and signal contamination in live brain imaging. In awake behaving animals, head movements, heartbeat, and breathing introduce artifacts in calcium imaging and optogenetic recordings, complicating the extraction of meaningful neural signals. While tensor-based motion correction techniques have shown promise, they often require large computational resources and can introduce biases if the decomposition fails to properly separate artifacts from genuine neural activity~\cite{kara2024facilitating}. 
% % Additionally, rank selection and model generalization remain significant issues, particularly for dynamic imaging applications. Selecting the optimal rank for CP, Tucker, or t-SVD decomposition in functional brain imaging is challenging, as an improper choice can either oversimplify neural activity patterns or introduce excessive computational burden. The variability in neural signals across subjects and experimental conditions further complicates the application of a single decomposition model, necessitating adaptive approaches that can generalize across different imaging modalities and experimental paradigms. \os{(Shehab: Does this paragraph need some citations?)}
% Furthermore, biological noise and tissue heterogeneity present difficulties in tensor-based models applied to tissue imaging. In label-free multiphoton microscopy, where contrast arises from intrinsic tissue properties rather than fluorescent markers, non-uniform signal intensities, variations in optical scattering, and background noise can degrade the performance of decomposition techniques~\cite{borile2021label}. Robust pre-processing and artifact rejection methods are essential to improve the reliability of tensor-based segmentation, classification, and feature extraction in histopathological and live-tissue imaging. Finally, hardware limitations pose another barrier. While parallel computing and GPU acceleration have improved computational feasibility, real-time applications, such as closed-loop optogenetics or high-speed volumetric imaging, require further advancements in tensor decomposition algorithms to achieve near-instantaneous data processing and decision-making. Addressing these challenges will be crucial for fully harnessing tensor decomposition in large-scale tissue and brain imaging studies.

\subsection{Analyzing Multi-omics Data}\label{sec:td_multiomics}

%\subsubsection{Applications}
% \AB{General applications of TD in multi-omics data. Mostly focused on integrating multi-omics data such as MONTI~\cite{jung2021monti}, etc. or this recent paper~\cite{mitchel2024coordinated}.
% We might have to look into approximate TD methods such as probabilistic TD, etc. as well. 
% We should also review this paper~\cite{hore2016tensor}. We need to also look at mono-omic applications for TD such as captured clinical records~\cite{luo2017tensor}, gene expression~\cite{taguchi2022tensor}.} 

The application of tensor decomposition in multi-omics data analysis has significantly advanced in recent years, offering scalable and interpretable solutions for integrating complex biological datasets~\cite{hore2016tensor, amin2023tensor, lee2018gift, taguchi2022adapted, leistico2021epigenomic, taguchi2023tensor, wang2023probabilistic, tsuyuzaki2023sctensor, jung2021monti, taguchi2018tensor, fang2019tightly, taguchi2020tensor}. 
Compared to traditional approaches (e.g., clustering, PCA, correlation analysis), which typically analyze relationships between only one or two variables at a time, TD simultaneously captures complex interactions across multiple omics layers, preserving high-order biological structures~\cite{jung2021monti}. As multi-omics studies continue to generate increasingly high-dimensional and heterogenous data, TD techniques have evolved to address challenges related to dimensionality reduction, data sparsity, and the extraction of biologically meaningful patterns. 

One major development in tensor decomposition for multi-omics datasets has been the integration of probabilistic models to address sparsity and noise. Traditional TD methods assume homogenous data distributions, which limits their ability to model multiomics data with intrinsic variability. Probabilistic TD methods, like \texttt{SCOIT} (Single Cell Multiomics Data Integration with Tensor decomposition)~\cite{10.1093/nar/gkad570}, leverage statistical distributions such as Gaussian, Poisson, and negative binomial models to better capture variability across different omics layers. These models enhance downstream analyses such as cell clustering, gene expression integration, and regulatory network inference, making them particularly valuable for single-cell transcriptomics and epigenomics. To improve biological interpretability, non-negative TD methods enforce non-negativity constraints on the factorized components, ensuring that all contributions to latent factors remain positive. This is particularly useful in biomedical research, where feature contributions must be biologically meaningful. A prime example is \texttt{MONTI} (Multi-Omics Non-negative Tensor Decomposition for Integrative Analysis)~\cite{jung2021monti}, which selects key molecular features by identifying non-negative latent factors that drive disease progression. By preserving additive relationships in multi-omics features, \texttt{MONTI} has been effective in distinguishing molecular signatures associated with cancer subtypes, enhancing patient stratification.

Structural constraints can also be incorporated in TD to improve the separability of multi-omics data while preserving both local and global values. \texttt{BioSTD} (Strong Complementarity Tensor Decomposition Model)~\cite{gao2023biostd} uses t-SVD to factor multi-omics data from the Cancer Genome Atlas (TCGA) and enforces a strong complementarity constraint to maintain high-dimensional spatial relationships across omics layers. This improves coordination between different types of omics data and ensures that redundant features do not obscure biologically significant patterns. Such an approach is particularly beneficial when analyzing high-dimensional datasets in precision oncology, where feature redundancy can compromise the interpretability of predictive biomarkers. Multi-omics integration often faces challenges due to the unequal feature dimensions across omics layers, where transcriptomics datasets typically contain tens of thousands of features, while proteomics and metabolomics datasets are comparatively sparse. \texttt{GSTRPCA} (Irregular Tensor Singular Value Decomposition Model)~\cite{Cui2024GSTRPCA} overcomes this limitation by maintaining the original data structure while incorporating low-rank and sparsity constraints using a tensor-PCA based model. This enables effective feature selection and clustering without requiring aggressive pre-processing or feature imputation, thereby preserving biologically relevant information.

Handling missing values in multi-omics datasets remains a fundamental challenge, particularly in epigenomics, where experimental limitations often result in incomplete data across different cell types. \texttt{PREDICTD} (Parallel Epigenomics Data Imputation with Cloud-Based Tensor Decomposition)
~\cite{Durham2018PREDICTD} addresses this issue by constructing a tensor where dimensions represent cell types, genomic regions, and epigenomic marks, allowing the decomposition process to identify latent factors that capture underlying regulatory patterns. By applying CP decomposition, the method imputes missing values using latent factor reconstruction. This cloud-based framework has been validated in large-scale projects such as ENCODE and the Roadmap Epigenomics Project, highlighting its utility in functional genomics. Although tensor decomposition effectively reduces dimensionality, one of its key challenges is ensuring biological interpretability. Guided and Interpretable Factorization for Tensors~\cite{lee2018gift} (\texttt{GIFT}) enhances interpretability by incorporating functional gene set information as a regularization term within the decomposition process. It adapts Tucker decomposition to factorize with prior biological knowledge, \texttt{GIFT} ensures that extracted components correspond to meaningful gene regulatory patterns. Applied to datasets such as the PanCan12 dataset from TCGA, \texttt{GIFT} has successfully identified relationships between gene expression, DNA methylation, and copy number variations across cancer subtypes, outperforming traditional tensor decomposition approaches in terms of accuracy and scalability. 

In terms of scalability, TD has evolved to accommodate large-scale genomic data through parallelized optimization techniques. An example is \texttt{SNeCT} (Scalable Network Constrained Tucker Decomposition)~\cite{Choi2020SNeCT} which extends traditional Tucker decomposition by incorporating prior biological networks into the decomposition process. This method constructs a tensor from multi-platform genomic data while integrating pathway-based constraints to improve feature selection and classification performance. By leveraging parallel stochastic gradient descent, \texttt{SNeCT} efficiently processes large-scale datasets such as TCGA’s PanCan12 dataset. Its ability to integrate gene association networks ensures that the resulting factor matrices maintain biological relevance, making it particularly effective for cancer subtype classification and patient stratification. Standard TD approaches often treat time as a discrete variable, limiting their ability to model continuous temporal changes in biological processes. Temporal Tensor Decomposition~\cite{Shi2024TEMPTED} (\texttt{TEMPTED}) overcomes this limitation by integrating time as a continuous variable within the decomposition framework. This method enables the characterization of dynamic transcriptional and epigenomic changes across different time points, making it particularly useful for analyzing developmental processes and disease progression. \texttt{TEMPTED} also accounts for non-uniform temporal sampling and missing data, providing a more comprehensive approach to longitudinal multi-omics analysis. More recently, Mitchel \textit{et al.} have introduced a computational approach called single-cell interpretable tensor decomposition (\texttt{scITD}), that utilizes tensor decomposition to analyze single-cell gene expression data~\cite{mitchel2024coordinated}. This approach enables the identification of coordinated transcriptional variations across multiple cell types, facilitating the discovery of common patterns among individuals, in turn facilitating patient stratification.

As tensor decomposition methodologies for multi-omics datasets continue to evolve, future advancements will likely focus on improving real-time data analysis capabilities, refining computational efficiency, and expanding the applicability of TD in emerging fields such as spatial transcriptomics and single-cell multi-omics. The continued refinement of these methods will be critical for advancing precision medicine and uncovering novel insights into complex biological systems.


% Precision medicine has been revolutionized by genetic analysis, enabling clinicians to identify individuals at high risk of disease and implement targeted, genetically informed interventions. However, multi-omics data, encompassing genomics, phenomics, transcriptomics, proteomics, metabolomics, etc. present unique challenges: they are inherently noisy, sparse, high-dimensional, and plagued by batch effects. Although there are emerging methods for integrating these diverse modalities, there is a lack of a standardized, user-friendly approach for multi-omics integration~\cite{tarazona2021undisclosed}. Moreover, current computational models still struggle to accurately predict phenotype outcomes from genotype data due to the complex, multi-layered interactions among genetic variants, epigenetic modifications, and environmental influences~\cite{ritchie2015methods}. Tensor decompositions is particularly well-suited for analyzing such data because it captures the multidimensional relationships within these datasets effectively. A tensor is a multi-dimensional array that can naturally represent relationships across different omics layers. Instead of analyzing two-dimensional matrices (e.g., samples × genes), tensors allow us to model higher-order interactions~\cite{chang2021gene}. In comparison to traditional approaches (e.g., clustering, PCA, correlation analysis), which typically analyze relationships between only one or two variables at a time, TD, however, simultaneously captures complex interactions across multiple omics layers, preserving high-order biological structures~\cite{jung2021monti}.  Hence, TD methods have been widely applied in both single-omic analysis such as in genomics~\cite{hore2016tensor, amin2023tensor}, transcriptomics~\cite{lee2018gift, taguchi2022adapted}, epigenomics~\cite{leistico2021epigenomic}, single-cell analysis~\cite{taguchi2023tensor, wang2023probabilistic, tsuyuzaki2023sctensor}, as well as in multi-omics studies~\cite{jung2021monti, taguchi2018tensor, fang2019tightly, taguchi2020tensor}. 


% The triumph of utilizing genetics in precision medicine is the ability to flag individuals at very high risk of disease, with the ability to enact specific gene-informed medical management. Despite their promise in identifying biologically relevant outputs, omics data are noisy, sparse (i.e., contain many zeros), high-dimensional, and contain batch effects. Particularly, there are no standard or user-friendly tools that exist to integrate or link data from all these modalities. The area under the curve (AUC - a measure of the predictive power) for many genetic risk scores has plateaued below 0.7 \os{(Shehab: Citations?)}, demonstrating the limits in linking genomes and outcomes based on classical computational methods. Multi-omics datasets, which integrate data from various omics layers such as genomics, transcriptomics, proteomics, and metabolomics, are inherently high-dimensional and complex. Tensor decomposition is particularly well-suited for analyzing such data because it captures the multidimensional relationships within these datasets effectively. A tensor is a multi-dimensional array that can naturally represent relationships across different omics layers. Instead of analyzing two-dimensional matrices (e.g., samples × genes), tensors allow us to model higher-order interactions. In comparison to traditional approaches (e.g., clustering, PCA, correlation analysis), which typically analyze relationships between only one or two variables at a time, tensor decomposition, however, simultaneously captures complex interactions across multiple omics layers, preserving high-order biological structures. 


% Tensor decomposition has been applied in multi-omics data processing, including integrative analyzes, disease or phenotype subtyping, feature extraction, biomarker discovery, gene regulation studies, as well as temporal or longitudinal data analyses \cite{luo2017tensor}. Hore et al. introduced a Bayesian method designed to analyze multi-tissue gene expression data, with a focus on uncovering gene networks associated with genetic variation by decomposing the three-dimensional matrix (tensor) of gene expression measurements into latent components \cite{hore2016tensor}. The study uncovered several gene networks related to genetic variation, providing insights into the genetic architecture of gene expression across multiple tissues, thus highlighting the potential of tensor decomposition methods in elucidating complex gene regulatory mechanisms. Jung et al. developed MONTI (Multi-Omics Non-negative Tensor Decomposition for Integrative Analysis), a multi-omics non-negative tensor decomposition framework tailored for integrative analysis of multi-omics data \cite{jung2021monti}. Through integrating multi-omics data in a gene-centric manner, MONTI enabled the detection of cancer subtype-specific features and other clinical features, which in turn facilitates the identification of biologically meaningful associations. In another study, Wang et al. introduced SCOIT (Single Cell Multiomics Data Integration with Tensor decomposition), a probabilistic tensor decomposition framework designed to extract embeddings from single cell multiomics data \cite{10.1093/nar/gkad570}. The framework was applied to eight single-cell multi-omics datasets from various sequencing protocols, covering DNA methylation data, RNA expression data, proteomics data, and chromatin accessibility data. SCOIT achieved superior clustering accuracy compared to nine state-of-the-art methods on heterogeneous datasets. More recently, Mitchel et al. have introduced a computational approach called single-cell interpretable tensor decomposition (scITD), a computational method that utilizes tensor decomposition to analyze single-cell gene expression data \cite{mitchel2024coordinated}. This approach enables the identification of coordinated transcriptional variations across multiple cell types, facilitating the discovery of common patterns among individuals, in turn facilitating patient stratification.


% \subsubsection{Challenges}
% % \AB{Phase transition/computational hardness discussions with plots following from the notebook shared with you}

% Multi-omics data analysis presents several challenges, particularly in handling, integration, and interpretation. One of the major difficulties lies in the computational complexities associated with large-scale multi-omics datasets. These datasets are often high-dimensional high dimensions due to the large number of features across multiple omics layers. Tensor decomposition methods, widely used for multi-omics integration, often have difficulties with scalability due to increasing computational demands. Therefore, efficient algorithms and scalable frameworks are required to handle these datasets effectively, as tensor decomposition can be both computationally intensive and time-consuming. 

% To illustrate these computational challenges, we performed tensor decomposition on a multi-omics dataset with a tensor size of (264, 10772, 31), where the dimensions represent the number of samples (264), genomic features (10,772), and metabolomic features (31). Figure~\ref{fig:tensor_decomposition} presents the results of this analysis. The first plot shows execution time (in seconds) as the decomposition rank varies from 2 to 30, revealing an initial increase followed by a decline after rank 14. The second plot illustrates memory usage (in MB), which steadily rises with increasing decomposition rank. These results highlight a critical trade-off between rank selection and computational efficiency, emphasizing the need for optimized feature selection strategies to balance computational feasibility and biological relevance when applying tensor decomposition to multi-omics data. Another key challenge is the heterogeneity of multi-omics data. These datasets comprise diverse data types, including continuous, binary, and categorical variables, complicating integration efforts. This heterogeneity complicates the integration process, as traditional tensor decomposition methods such as Tucker and CANDECOMP / PARAFAC (CP) may not adequately model complex interactions and different data types \cite{xu2015bayesian}. Beyond computational and integration challenges, interpreting the latent factors extracted from tensor decomposition in a biologically relevant context can be challenging. Moreover, multi-omics data are often noisy and contain outliers. This can distort the results of the tensor decomposition. However, there are new methods such as SCOIT \cite{10.1093/nar/gkad570} which attempt to address this challenge by incorporating various distributions to model noise and sparsity in the data. 

% Finally, multi-omics data often contain missing values or sparse entries, which can significantly impact the performance of tensor decomposition methods. Several traditional approaches require complete data or extensive preprocessing to handle missing data, which can be time-consuming and may introduce biases \cite{xu2015bayesian, taguchi2021tensor}. While tensor decomposition provides a powerful framework for multi-omics integration,  overcoming computational constraints, handling data heterogeneity, improving interpretability, mitigating noise, and addressing missing data challenges remain critical areas for future research and methodological advancements.

% \begin{figure}[ht]
%     \centering
%     \subfloat{\includegraphics[width=0.45\textwidth]{figures/memory_usage.png}}
%     \hspace{0.5cm} % Adds some spacing between images
%     \subfloat{\includegraphics[width=0.45\textwidth]{figures/execution_time.png}}
%     \caption{Memory usage and execution time for tensor decomposition for random dataset. \os{(Shehab: Can we add more details? How the dataset was generated?)}}
%     \label{fig:tensor_decomposition}
% \end{figure}




\begin{figure*}[t]
\begin{minipage}{0.98\textwidth}
    \begin{subfigure}[b]{0.3\linewidth}
        \centering
        \includegraphics[width=0.95\linewidth]{figures/files/vanilla_rope.pdf}
        \caption{3D visualization for Vanilla RoPE.}
        \label{fig:vanilla_rope}
    \end{subfigure}
    \hfill
    \begin{subfigure}[b]{0.3\linewidth}
        \centering
        \includegraphics[width=0.95\linewidth]{figures/files/m_rope.pdf}
        \caption{3D visualization for M-RoPE.}
        \label{fig:m_rope}
    \end{subfigure}
    \hfill
    \begin{subfigure}[b]{0.3\linewidth}
        \centering
        \includegraphics[width=0.95\linewidth]{figures/files/m_modify_rope.pdf}
        \caption{3D visualization for \methodname.}
        \label{fig:video_rope}
    \end{subfigure}
    \hfill
    \vspace{-6pt}
    \caption{\footnotesize The 3D visualization for different position embedding. \textbf{(a)} The vanilla 1D RoPE~\cite{su2024roformer} does not incorporate spatial modeling.
    \textbf{(b)} M-RoPE~\cite{wang2024qwen2}, while have the 3D structure, introduces a discrepancy in index growth for visual tokens across frames, with some indices remaining constant.
    \textbf{(c)} In contrast, our \methodname achieves the desired balance, maintaining the consistent index growth pattern of vanilla RoPE while simultaneously incorporating spatial modeling. 
    }
    %3D visualization of different position embeddings. \textbf{(a)} The vanilla 1D RoPE~\cite{su2024roformer} lacks spatial modeling. \textbf{(b)} M-RoPE~\cite{wang2024qwen2}, while have the 3D structure, introduces a discrepancy in index growth for visual tokens across frames, with some indices remaining constant. \textbf{(c)} Our \methodname balances the index growth of vanilla RoPE while incorporating spatial modeling. For more details on the index, see Appendix \ref{app:supp_explain_modules}.
    \vspace{-12pt}
    \label{fig:spatial}
\end{minipage}
\end{figure*} 

\section{Problem Statement} \label{sec: statement}


\subsection{Deploying GNN locally Causes Vulnerabilities} \label{ps: gnn inference}
Deploying GNNs on local devices requires access to graph data in addition to the trained model, which introduces unique security and privacy challenges. 
Similar to DNN deployment, the IP of the well-performed local model, including its trained weights and biases, is valuable asset that must be protected against model extraction attacks.
Beyond the model IP, local GNN inference raises additional privacy concerns due to the nature of GNN architecture. 
Specifically, during the message-passing phase of GNN inference, target nodes aggregate information from neighboring nodes to update their embeddings. 
This process involves accessing sensitive edge data, such as user-product interactions in recommender systems.
In our work, we will address the GNN IP infringement and edge data breach vulnerability during GNN deployment.

\subsection{Edge Privacy is Valuable} \label{motivation: edge importance}

\begin{figure}[t]
    \centering
    \includegraphics[width=0.95\linewidth]{imgs/scenario.pdf}
    \caption{\textbf{Motivation Example:} Alice (victim) builds a graph of products and trains a GNN RS. She deploys both edge data and RS on local devices. Bob (attacker) accesses this device and steals the edge data and model parameters.}
    \label{fig: problem-statement}
\end{figure}

Membership inference attack is the most common data privacy threat to machine learning models~\cite{shokri2017membership}, where the goal is to determine whether a given data point belongs to the training set. 
However, in the context of GNNs, edge data raises additional privacy concerns. 
Link stealing attacks~\cite{he2021stealing, ding2023vertexserum} aim to infer the connectivity between any pair of given nodes. 
In this work, we focus on the adjacency information (edges), while considering the node features as public.
A real-world example is illustrated in Fig.~\ref{fig: problem-statement}, where Alice (victim) deploys a recommender system (RS) on local edge devices. 
In such a product graph, the node features are public attributes of the products—such as price, user reviews, or categories—that are available to any user. 
However, the internal relationships between products require intensive learning from user behavior data, which is valuable IP for the model vendor. 
Therefore, safeguarding the node connectivity information during GNN local inference is of great importance.


\subsection{TEE Has Memory Restrictions} \label{ps: TEE}
The introduction of TEE greatly enhances data security and privacy with secure compartments. 
However, TEE platforms face significant memory limitations, a critical constraint for secure computation. 
For instance, for Intel SGX trusted enclaves, the physical reserved memory (PRM) is limited to 128MB, with 96 MB of it allocated to the Enclave Page Cache (EPC)~\cite{intel2017sgx}. 
Excessive memory allocation will lead to frequent page swapping between the unprotected main memory and the protected enclave, which can cause high overhead and additional encryption/decryption to ensure data integrity~\cite{costan2016intel}.
This memory constraint poses a significant challenge for deploying GNN models and the entire graph (including node features and adjacency information) within the secure enclave, which often far exceed the PRM limitation of enclaves.



\section{Quantum Algorithm for Tensor Decomposition}~\label{sec:td_quantum} 
Quantum computing leverages principles like superposition, entanglement, measurement, and destructive interference among solution probabilities to perform computational tasks more efficiently than classical computing (See Section 2 of \cite{basu2023towards} for a short primer on quantum computing and \cite{nielsen2010quantum} for a detailed introduction). In the past decade, quantum algorithms for matrix and tensor decomposition have emerged as powerful tools with significant applications in machine learning, data compression, quantum chemistry, and more. A notable advancement is the quantum singular value decomposition (QSVD), which expresses a matrix $A$ as $A = U \Sigma V^\dagger$, where $U$ and $V$ are unitary matrices and $\Sigma$ is a diagonal matrix of singular values. A quantum algorithm for the singular value decomposition of nonsparse low-rank matrices, providing exponential speedup over classical algorithms under certain conditions was proposed by Rebenstrot \textit{et al.}~\cite{rebentrost2018quantum}. This algorithm employs quantum state preparation techniques and Quantum Phase Estimation (QPE) to efficiently extract singular values and corresponding singular vectors.

Building on this, variational quantum singular value decomposition (VQSVD) algorithm was introduced~\cite{wang2021variational}. Designed for PFTQD, VQSVD uses a hybrid quantum-classical optimization loop to approximate singular values and vectors by minimizing a cost function. This makes it practical for current quantum hardware and effective for matrices where only a few singular values are significant.
Another significant development is quantum principal component analysis (QPCA)~\cite{lloyd2014quantum} introduced a quantum algorithm for PCA that can extract principal components exponentially faster than classical algorithms under certain conditions . The QPCA algorithm uses quantum algorithms for Hamiltonian simulation and phase estimation to find the eigenvalues and eigenvectors of the covariance matrix, processing large datasets encoded in quantum states and making it promising for big data analysis. Foundational to many quantum linear algebra algorithms is the quantum algorithm for solving linear systems of equations developed by Harrow, Hassidim, and Lloyd~\cite{harrow2009quantum}, also known as the HHL algorithm. The HHL algorithm solves systems of the form $Ax = b$ in logarithmic time relative to the size of the system under conditions such as sparsity and a low condition number of matrix $A$. It is crucial for quantum matrix computations, including inversions and decompositions, with applications spanning machine learning and optimization.

Extending these concepts to tensors, which generalize matrices to higher dimensions, offers significant potential in processing multidimensional data. Hastings~\cite{hastings2020classical} introduced classical and quantum algorithms for tensor principal component analysis (Tensor PCA), providing insights into the computational complexity of the problem and demonstrating potential quantum advantages for the problem of spiked tensor decomposition (Equation~\ref{eq:st}). The quantum algorithm presented in \cite{hastings2020classical} proves a quartic speedup over the best classical spectral algorithm. This quantum algorithm takes advantage of QPE and amplitude amplification, coupled with a suitable choice of the input state initialization to achieve the speedup. The quantum Hamiltonian constructed from the initial tensor $T_0$ operates on a set of $n_{bos}$ qudits of dimension $N$. More precisely, for a given tensor $T$ of order $p$ and dimension $N$, the Hamiltonian is a linear operator on the vector space $(\mathbb{R}^N)^{\otimes n_{bos}}$ or $(\mathbb{C}^N)^{\otimes n_{bos}}$ where $n_{bos} \geq \frac{p}{2}$. The bosonic Hamiltonian on the full Hilbert space has basis elements of the form $|\mu_1\rangle \otimes |\mu_2\rangle \otimes \dots \otimes |\mu_{n_{bos}} \rangle$ where each $\mu_i \in \{0, 1, \dots , N-1\}$ and can be written as,

\begin{equation}\label{eq:hamiltonian}
H(T) = \frac{1}{2} \sum_{i_1, \dots , i_{p/2}} \left( \sum_{\mu_1, \dots , \mu_p} T_{\mu_1, \mu_2, \dots, \mu_p} |\mu_1\rangle _{i_1} \langle \mu_{1+p/2} |\otimes |\mu_2\rangle _{i_2} \langle \mu_{2+p/2} |\otimes \cdots \otimes |\mu_{p/2}\rangle _{i_{p/2}} \langle \mu_{p}| + h.c.  \right)
\end{equation}

where the first sum iterates over unique set of qudits $i_1, i_2, \dots, i_p$, $T_{\mu_1, \mu_2, \dots, \mu_p}$ are the corresponding elements from the tensor $T$ and $h.c.$ refers to adding Hermitian conjugates for the terms. Outer products $|\mu_n\rangle _{k} \langle\mu_{m}|$ are performed on the corresponding qudit $k$. Then the quantum algorithm in \cite{hastings2020classical} recovers the leading eigenvalue and the corresponding eigenvector, assuming the detection is successful, and outputs a vector that has large normalized overlap with the signal vector $v_{sig}$. 

The Hamiltonian $H(T_0)$, inherits its spectral properties, which in turn dictate the sample complexity of the quantum algorithm. A key factor is the spectral gap $\Delta$, which determines how well the ground state can be distinguished from excited states. Since the largest eigenvalue of $H(T_0)$ is driven by the signal term $\lambda v_{\text{sig}}^{\otimes p}$, a higher SNR (larger $\lambda$) results in a larger spectral gap, making it easier to resolve the ground state with quantum phase estimation and reducing sample complexity, which scales as $\mathcal{O}(1/\Delta^2)$. However, when the SNR is low, the signal eigenvalue becomes buried in the noise spectrum, causing the spectral gap to shrink and significantly increasing the number of measurements required. Additionally, the operator norm $\|\hat{H}\|$, which grows with the tensor order $p$ and dimension $N$, influences the spread of energy levels and dictates the precision needed in phase estimation, affecting the total measurement overhead. While the quantum algorithm achieves a quartic speedup over classical methods, its efficiency is still constrained by the interplay between SNR, tensor order, and spectral properties - if the spectral gap is too small due to low SNR or high $p$, the measurement complexity increases, limiting the quantum advantage.

In QPE, ancilla qubits store the binary representation of the estimated eigenvalue, determining the precision of the approximation. It also determines the depth of the quantum circuit. Quantum Amplitude Amplification (the second step in Hastings' algorithm~\cite{hastings2020classical}) improves the success probability of measuring the correct eigenvalue but does not alter the role or requirement of ancilla qubits in QPE. The number of ancilla qubits, as well as the number of times the controlled unitary is applied to the main register, is dictated by the precision requirement of the problem. Specifically, to achieve 2, 4, 6, and 8 decimal digit precision, the required number of ancilla qubits is 7, 14, 20, and 27, respectively. It also means that the controlled unitary on the main register will be applied $\sim 2$, $\sim2^3$, $\sim 2^5$, and $\sim 2^7$  times respectively. Such depth makes an underlying quantum error correcting layer necessary for the algorithm. In practice, when applied on biomedical data, up to four decimal digit precision is often sufficient to qualitatively evaluate tasks such as clustering, prediction, classification, or feature extraction. Hence, in realistic terms, a QPE when applied on real-world data would approximately require upto 14 ancilla qubits. However, this estimate is dependent on other factors such as number of qubits, size and order of the tensor, etc.


The quantum algorithmic resources needed to implement this algorithm is high due to QPE yielding high circuit depths, however a more recent work \cite{zhou2024statistical} provides numerical estimations and analyzes the asymptotic behavior of this model using a $p$-step quantum approximate optimization algorithm (QAOA), making it more amenable in near-term utility-scale quantum computers. 


While explicit quantum algorithms for CP (CANDECOMP/PARAFAC) and Tucker decompositions are still under development, the principles of quantum linear algebra can be extended to these tensor factorizations. This extension potentially offers speedups in processing high-dimensional data, which is crucial in fields like computer vision, chemometrics, neuroscience, and quantum chemistry. As quantum computing technology advances, these algorithms pave the way for more efficient data processing techniques and inspire further research into quantum algorithms for complex data structures.



\section{Quantum Tensor Decomposition for Biomedical Data}~\label{sec:td_usecase} 
% \AB{In this section, we will focus on how quantum tensor decomposition can be applied to specific problems emanating from the three focus areas we have discussed. We will discuss the size of the problems, number of qubits required to implement the algorithm, and how to achieve quantum utility. 
% Discussion on quantum advantage?! - phase estimation and use cases. 
% }
\begin{figure}
    \centering
    \includegraphics[width=\linewidth]{figures/td_use_case_reviewpaper.png}
    \caption{Proposed framework of quantum tensor decomposition methods for analyzing biomedical data for various downstream tasks.}
    \label{fig:qtd_framework}
\end{figure}
The applications of quantum tensor decomposition for analyzing biomedical data must integrate the recent advances in QTD algorithms~\cite{hastings2020classical,zhou2024statistical} with existing approaches in classical algorithms in spiked tensor decomposition or CP decomposition. The QTD algorithm should be compared against these classical approaches for quantitative and qualitative benchmarking and assessing the problems that are more suitable for the quantum algorithm. We propose a framework to implement QTD in biomedical data analysis (Figure~\ref{fig:qtd_framework}) and perform a host of downstream tasks targeted towards the nature and modality of the data.  

The framework ingests multi-modal data such as imaging or multi-omics including spatial transcriptomics. It tensorizes them into higher-order tensors (with orders greater than two) to produce the tensor $T_0$ in the form of a spike tensor problem (Equation~\ref{eq:st}). Then we take a multi-pronged approach to find the underlying signal $v_{sig}$. We form the Hamiltonian, $H(T_0)$ from $T_0$ as per Equation~\ref{eq:hamiltonian} and perform a spectral decomposition, as shown in previous work~\cite{hastings2020classical}, which proposed both classical and quantum algorithms for spectral decomposition. We decompose $H(T_0)$ by solving a QPE problem~\cite{kitaev1995quantum} in FTQD, to efficiently extract eigenvalues and corresponding eigenvectors. The QPE circuit is shown in the inset of Figure~\ref{fig:qtd_framework}'s quantum approach. Once we obtain the leading eigenvalue with its corresponding eigenvector, we can recover $\bar{v}_{sig}$ up to some accuracy. Simultaneously, we can also apply a CP decomposition to the original $T_0$ and obtain the $v_{sig}$. Then, we need to compare the correlation between $\bar{v}_{sig}$ and the $v_{sig}$, to assess the quality of the signal obtained from the quantum algorithm. 

\begin{figure}[H]
    \centering
    \includegraphics[width=0.9\textwidth]{figures/qubit_scaling.png}
    \caption{Figure shows the qubit scaling in the spiked tensor Hamiltonian as the tensor order $p$ and dimension $N$ (in log scale) increases. The horizontal planes represent the cut-off points for quantum computers, from IBM Quantum's roadmap, with corresponding number of qubits. In particular, Heron and Flamingo QPUs contain 133 and 156 qubits. With the modular structures where multiple QPUs are coupled, the qubit numbers increase to 399 and 1092 respectively. It is important to note that, our qubit counts are problem qubit counts while the qubit numbers on IBM Quantum roadmap are physical qubits. Depending on the type of algorithm, they may not map to each other on a 1-to-1 basis. Further information can be found at \href{https://www.ibm.com/quantum/technology}{IBM Quantum Technology Roadmap}~\cite{gambetta2020ibm}.}
    \label{fig:qubits}
\end{figure}

Furthermore, to understand the complexity of the problem and how it maps to a quantum computer, we also analyze the number of qubits required to map a given tensor to a qubit Hamiltonian (see Figure \ref{fig:qubits}). Note that the Hamiltonian from Equation \ref{eq:hamiltonian} is formulated for a bosonic system, i.e. an $N$-level qudit system. In Figure \ref{fig:qubits}, we show the final scaling of how this qudit system maps to qubits, since the quantum computers we consider operate natively with qubits. We are aware that there many other parameters to consider in this case, such as the circuit depth (in particular 2-qubit gate depth), overhead from error mitigation/suppression in utility-scale quantum computer or the overhead of error correction for QPE, the trade-off between accuracy and number of repetitions in QPE, the strength of the signal-to-noise ratio in the spiked tensor and the number of bosonic modes required to successfully recover the signal vector etc. However, these require a deeper analysis and are beyond the scope of this paper.  

With this recovered signal, we can perform several downstream tasks in biomedical data analysis. For example, QTD can perform tasks such as denoising, reconstruction, and segmentation of biomedical images. On the other hand, QTD can produce latent representations by integrating multi-modal data in lower dimensions which can be used as input to deep neural networks for downstream analysis. Similarly, these lower-dimensional embeddings can be directly used to discover biomarkers, prognosis, and perform classification or regression tasks with multi-omics data.



We present RiskHarvester, a risk-based tool to compute a security risk score based on the value of the asset and ease of attack on a database. We calculated the value of asset by identifying the sensitive data categories present in a database from the database keywords. We utilized data flow analysis, SQL, and Object Relational Mapper (ORM) parsing to identify the database keywords. To calculate the ease of attack, we utilized passive network analysis to retrieve the database host information. To evaluate RiskHarvester, we curated RiskBench, a benchmark of 1,791 database secret-asset pairs with sensitive data categories and host information manually retrieved from 188 GitHub repositories. RiskHarvester demonstrates precision of (95\%) and recall (90\%) in detecting database keywords for the value of asset and precision of (96\%) and recall (94\%) in detecting valid hosts for ease of attack. Finally, we conducted an online survey to understand whether developers prioritize secret removal based on security risk score. We found that 86\% of the developers prioritized the secrets for removal with descending security risk scores.

\newpage 
\bibliographystyle{unsrt}
% \bibliographystyle{plain} % We choose the "plain" reference style
\bibliography{references}

\newpage
\subsection{Lloyd-Max Algorithm}
\label{subsec:Lloyd-Max}
For a given quantization bitwidth $B$ and an operand $\bm{X}$, the Lloyd-Max algorithm finds $2^B$ quantization levels $\{\hat{x}_i\}_{i=1}^{2^B}$ such that quantizing $\bm{X}$ by rounding each scalar in $\bm{X}$ to the nearest quantization level minimizes the quantization MSE. 

The algorithm starts with an initial guess of quantization levels and then iteratively computes quantization thresholds $\{\tau_i\}_{i=1}^{2^B-1}$ and updates quantization levels $\{\hat{x}_i\}_{i=1}^{2^B}$. Specifically, at iteration $n$, thresholds are set to the midpoints of the previous iteration's levels:
\begin{align*}
    \tau_i^{(n)}=\frac{\hat{x}_i^{(n-1)}+\hat{x}_{i+1}^{(n-1)}}2 \text{ for } i=1\ldots 2^B-1
\end{align*}
Subsequently, the quantization levels are re-computed as conditional means of the data regions defined by the new thresholds:
\begin{align*}
    \hat{x}_i^{(n)}=\mathbb{E}\left[ \bm{X} \big| \bm{X}\in [\tau_{i-1}^{(n)},\tau_i^{(n)}] \right] \text{ for } i=1\ldots 2^B
\end{align*}
where to satisfy boundary conditions we have $\tau_0=-\infty$ and $\tau_{2^B}=\infty$. The algorithm iterates the above steps until convergence.

Figure \ref{fig:lm_quant} compares the quantization levels of a $7$-bit floating point (E3M3) quantizer (left) to a $7$-bit Lloyd-Max quantizer (right) when quantizing a layer of weights from the GPT3-126M model at a per-tensor granularity. As shown, the Lloyd-Max quantizer achieves substantially lower quantization MSE. Further, Table \ref{tab:FP7_vs_LM7} shows the superior perplexity achieved by Lloyd-Max quantizers for bitwidths of $7$, $6$ and $5$. The difference between the quantizers is clear at 5 bits, where per-tensor FP quantization incurs a drastic and unacceptable increase in perplexity, while Lloyd-Max quantization incurs a much smaller increase. Nevertheless, we note that even the optimal Lloyd-Max quantizer incurs a notable ($\sim 1.5$) increase in perplexity due to the coarse granularity of quantization. 

\begin{figure}[h]
  \centering
  \includegraphics[width=0.7\linewidth]{sections/figures/LM7_FP7.pdf}
  \caption{\small Quantization levels and the corresponding quantization MSE of Floating Point (left) vs Lloyd-Max (right) Quantizers for a layer of weights in the GPT3-126M model.}
  \label{fig:lm_quant}
\end{figure}

\begin{table}[h]\scriptsize
\begin{center}
\caption{\label{tab:FP7_vs_LM7} \small Comparing perplexity (lower is better) achieved by floating point quantizers and Lloyd-Max quantizers on a GPT3-126M model for the Wikitext-103 dataset.}
\begin{tabular}{c|cc|c}
\hline
 \multirow{2}{*}{\textbf{Bitwidth}} & \multicolumn{2}{|c|}{\textbf{Floating-Point Quantizer}} & \textbf{Lloyd-Max Quantizer} \\
 & Best Format & Wikitext-103 Perplexity & Wikitext-103 Perplexity \\
\hline
7 & E3M3 & 18.32 & 18.27 \\
6 & E3M2 & 19.07 & 18.51 \\
5 & E4M0 & 43.89 & 19.71 \\
\hline
\end{tabular}
\end{center}
\end{table}

\subsection{Proof of Local Optimality of LO-BCQ}
\label{subsec:lobcq_opt_proof}
For a given block $\bm{b}_j$, the quantization MSE during LO-BCQ can be empirically evaluated as $\frac{1}{L_b}\lVert \bm{b}_j- \bm{\hat{b}}_j\rVert^2_2$ where $\bm{\hat{b}}_j$ is computed from equation (\ref{eq:clustered_quantization_definition}) as $C_{f(\bm{b}_j)}(\bm{b}_j)$. Further, for a given block cluster $\mathcal{B}_i$, we compute the quantization MSE as $\frac{1}{|\mathcal{B}_{i}|}\sum_{\bm{b} \in \mathcal{B}_{i}} \frac{1}{L_b}\lVert \bm{b}- C_i^{(n)}(\bm{b})\rVert^2_2$. Therefore, at the end of iteration $n$, we evaluate the overall quantization MSE $J^{(n)}$ for a given operand $\bm{X}$ composed of $N_c$ block clusters as:
\begin{align*}
    \label{eq:mse_iter_n}
    J^{(n)} = \frac{1}{N_c} \sum_{i=1}^{N_c} \frac{1}{|\mathcal{B}_{i}^{(n)}|}\sum_{\bm{v} \in \mathcal{B}_{i}^{(n)}} \frac{1}{L_b}\lVert \bm{b}- B_i^{(n)}(\bm{b})\rVert^2_2
\end{align*}

At the end of iteration $n$, the codebooks are updated from $\mathcal{C}^{(n-1)}$ to $\mathcal{C}^{(n)}$. However, the mapping of a given vector $\bm{b}_j$ to quantizers $\mathcal{C}^{(n)}$ remains as  $f^{(n)}(\bm{b}_j)$. At the next iteration, during the vector clustering step, $f^{(n+1)}(\bm{b}_j)$ finds new mapping of $\bm{b}_j$ to updated codebooks $\mathcal{C}^{(n)}$ such that the quantization MSE over the candidate codebooks is minimized. Therefore, we obtain the following result for $\bm{b}_j$:
\begin{align*}
\frac{1}{L_b}\lVert \bm{b}_j - C_{f^{(n+1)}(\bm{b}_j)}^{(n)}(\bm{b}_j)\rVert^2_2 \le \frac{1}{L_b}\lVert \bm{b}_j - C_{f^{(n)}(\bm{b}_j)}^{(n)}(\bm{b}_j)\rVert^2_2
\end{align*}

That is, quantizing $\bm{b}_j$ at the end of the block clustering step of iteration $n+1$ results in lower quantization MSE compared to quantizing at the end of iteration $n$. Since this is true for all $\bm{b} \in \bm{X}$, we assert the following:
\begin{equation}
\begin{split}
\label{eq:mse_ineq_1}
    \tilde{J}^{(n+1)} &= \frac{1}{N_c} \sum_{i=1}^{N_c} \frac{1}{|\mathcal{B}_{i}^{(n+1)}|}\sum_{\bm{b} \in \mathcal{B}_{i}^{(n+1)}} \frac{1}{L_b}\lVert \bm{b} - C_i^{(n)}(b)\rVert^2_2 \le J^{(n)}
\end{split}
\end{equation}
where $\tilde{J}^{(n+1)}$ is the the quantization MSE after the vector clustering step at iteration $n+1$.

Next, during the codebook update step (\ref{eq:quantizers_update}) at iteration $n+1$, the per-cluster codebooks $\mathcal{C}^{(n)}$ are updated to $\mathcal{C}^{(n+1)}$ by invoking the Lloyd-Max algorithm \citep{Lloyd}. We know that for any given value distribution, the Lloyd-Max algorithm minimizes the quantization MSE. Therefore, for a given vector cluster $\mathcal{B}_i$ we obtain the following result:

\begin{equation}
    \frac{1}{|\mathcal{B}_{i}^{(n+1)}|}\sum_{\bm{b} \in \mathcal{B}_{i}^{(n+1)}} \frac{1}{L_b}\lVert \bm{b}- C_i^{(n+1)}(\bm{b})\rVert^2_2 \le \frac{1}{|\mathcal{B}_{i}^{(n+1)}|}\sum_{\bm{b} \in \mathcal{B}_{i}^{(n+1)}} \frac{1}{L_b}\lVert \bm{b}- C_i^{(n)}(\bm{b})\rVert^2_2
\end{equation}

The above equation states that quantizing the given block cluster $\mathcal{B}_i$ after updating the associated codebook from $C_i^{(n)}$ to $C_i^{(n+1)}$ results in lower quantization MSE. Since this is true for all the block clusters, we derive the following result: 
\begin{equation}
\begin{split}
\label{eq:mse_ineq_2}
     J^{(n+1)} &= \frac{1}{N_c} \sum_{i=1}^{N_c} \frac{1}{|\mathcal{B}_{i}^{(n+1)}|}\sum_{\bm{b} \in \mathcal{B}_{i}^{(n+1)}} \frac{1}{L_b}\lVert \bm{b}- C_i^{(n+1)}(\bm{b})\rVert^2_2  \le \tilde{J}^{(n+1)}   
\end{split}
\end{equation}

Following (\ref{eq:mse_ineq_1}) and (\ref{eq:mse_ineq_2}), we find that the quantization MSE is non-increasing for each iteration, that is, $J^{(1)} \ge J^{(2)} \ge J^{(3)} \ge \ldots \ge J^{(M)}$ where $M$ is the maximum number of iterations. 
%Therefore, we can say that if the algorithm converges, then it must be that it has converged to a local minimum. 
\hfill $\blacksquare$


\begin{figure}
    \begin{center}
    \includegraphics[width=0.5\textwidth]{sections//figures/mse_vs_iter.pdf}
    \end{center}
    \caption{\small NMSE vs iterations during LO-BCQ compared to other block quantization proposals}
    \label{fig:nmse_vs_iter}
\end{figure}

Figure \ref{fig:nmse_vs_iter} shows the empirical convergence of LO-BCQ across several block lengths and number of codebooks. Also, the MSE achieved by LO-BCQ is compared to baselines such as MXFP and VSQ. As shown, LO-BCQ converges to a lower MSE than the baselines. Further, we achieve better convergence for larger number of codebooks ($N_c$) and for a smaller block length ($L_b$), both of which increase the bitwidth of BCQ (see Eq \ref{eq:bitwidth_bcq}).


\subsection{Additional Accuracy Results}
%Table \ref{tab:lobcq_config} lists the various LOBCQ configurations and their corresponding bitwidths.
\begin{table}
\setlength{\tabcolsep}{4.75pt}
\begin{center}
\caption{\label{tab:lobcq_config} Various LO-BCQ configurations and their bitwidths.}
\begin{tabular}{|c||c|c|c|c||c|c||c|} 
\hline
 & \multicolumn{4}{|c||}{$L_b=8$} & \multicolumn{2}{|c||}{$L_b=4$} & $L_b=2$ \\
 \hline
 \backslashbox{$L_A$\kern-1em}{\kern-1em$N_c$} & 2 & 4 & 8 & 16 & 2 & 4 & 2 \\
 \hline
 64 & 4.25 & 4.375 & 4.5 & 4.625 & 4.375 & 4.625 & 4.625\\
 \hline
 32 & 4.375 & 4.5 & 4.625& 4.75 & 4.5 & 4.75 & 4.75 \\
 \hline
 16 & 4.625 & 4.75& 4.875 & 5 & 4.75 & 5 & 5 \\
 \hline
\end{tabular}
\end{center}
\end{table}

%\subsection{Perplexity achieved by various LO-BCQ configurations on Wikitext-103 dataset}

\begin{table} \centering
\begin{tabular}{|c||c|c|c|c||c|c||c|} 
\hline
 $L_b \rightarrow$& \multicolumn{4}{c||}{8} & \multicolumn{2}{c||}{4} & 2\\
 \hline
 \backslashbox{$L_A$\kern-1em}{\kern-1em$N_c$} & 2 & 4 & 8 & 16 & 2 & 4 & 2  \\
 %$N_c \rightarrow$ & 2 & 4 & 8 & 16 & 2 & 4 & 2 \\
 \hline
 \hline
 \multicolumn{8}{c}{GPT3-1.3B (FP32 PPL = 9.98)} \\ 
 \hline
 \hline
 64 & 10.40 & 10.23 & 10.17 & 10.15 &  10.28 & 10.18 & 10.19 \\
 \hline
 32 & 10.25 & 10.20 & 10.15 & 10.12 &  10.23 & 10.17 & 10.17 \\
 \hline
 16 & 10.22 & 10.16 & 10.10 & 10.09 &  10.21 & 10.14 & 10.16 \\
 \hline
  \hline
 \multicolumn{8}{c}{GPT3-8B (FP32 PPL = 7.38)} \\ 
 \hline
 \hline
 64 & 7.61 & 7.52 & 7.48 &  7.47 &  7.55 &  7.49 & 7.50 \\
 \hline
 32 & 7.52 & 7.50 & 7.46 &  7.45 &  7.52 &  7.48 & 7.48  \\
 \hline
 16 & 7.51 & 7.48 & 7.44 &  7.44 &  7.51 &  7.49 & 7.47  \\
 \hline
\end{tabular}
\caption{\label{tab:ppl_gpt3_abalation} Wikitext-103 perplexity across GPT3-1.3B and 8B models.}
\end{table}

\begin{table} \centering
\begin{tabular}{|c||c|c|c|c||} 
\hline
 $L_b \rightarrow$& \multicolumn{4}{c||}{8}\\
 \hline
 \backslashbox{$L_A$\kern-1em}{\kern-1em$N_c$} & 2 & 4 & 8 & 16 \\
 %$N_c \rightarrow$ & 2 & 4 & 8 & 16 & 2 & 4 & 2 \\
 \hline
 \hline
 \multicolumn{5}{|c|}{Llama2-7B (FP32 PPL = 5.06)} \\ 
 \hline
 \hline
 64 & 5.31 & 5.26 & 5.19 & 5.18  \\
 \hline
 32 & 5.23 & 5.25 & 5.18 & 5.15  \\
 \hline
 16 & 5.23 & 5.19 & 5.16 & 5.14  \\
 \hline
 \multicolumn{5}{|c|}{Nemotron4-15B (FP32 PPL = 5.87)} \\ 
 \hline
 \hline
 64  & 6.3 & 6.20 & 6.13 & 6.08  \\
 \hline
 32  & 6.24 & 6.12 & 6.07 & 6.03  \\
 \hline
 16  & 6.12 & 6.14 & 6.04 & 6.02  \\
 \hline
 \multicolumn{5}{|c|}{Nemotron4-340B (FP32 PPL = 3.48)} \\ 
 \hline
 \hline
 64 & 3.67 & 3.62 & 3.60 & 3.59 \\
 \hline
 32 & 3.63 & 3.61 & 3.59 & 3.56 \\
 \hline
 16 & 3.61 & 3.58 & 3.57 & 3.55 \\
 \hline
\end{tabular}
\caption{\label{tab:ppl_llama7B_nemo15B} Wikitext-103 perplexity compared to FP32 baseline in Llama2-7B and Nemotron4-15B, 340B models}
\end{table}

%\subsection{Perplexity achieved by various LO-BCQ configurations on MMLU dataset}


\begin{table} \centering
\begin{tabular}{|c||c|c|c|c||c|c|c|c|} 
\hline
 $L_b \rightarrow$& \multicolumn{4}{c||}{8} & \multicolumn{4}{c||}{8}\\
 \hline
 \backslashbox{$L_A$\kern-1em}{\kern-1em$N_c$} & 2 & 4 & 8 & 16 & 2 & 4 & 8 & 16  \\
 %$N_c \rightarrow$ & 2 & 4 & 8 & 16 & 2 & 4 & 2 \\
 \hline
 \hline
 \multicolumn{5}{|c|}{Llama2-7B (FP32 Accuracy = 45.8\%)} & \multicolumn{4}{|c|}{Llama2-70B (FP32 Accuracy = 69.12\%)} \\ 
 \hline
 \hline
 64 & 43.9 & 43.4 & 43.9 & 44.9 & 68.07 & 68.27 & 68.17 & 68.75 \\
 \hline
 32 & 44.5 & 43.8 & 44.9 & 44.5 & 68.37 & 68.51 & 68.35 & 68.27  \\
 \hline
 16 & 43.9 & 42.7 & 44.9 & 45 & 68.12 & 68.77 & 68.31 & 68.59  \\
 \hline
 \hline
 \multicolumn{5}{|c|}{GPT3-22B (FP32 Accuracy = 38.75\%)} & \multicolumn{4}{|c|}{Nemotron4-15B (FP32 Accuracy = 64.3\%)} \\ 
 \hline
 \hline
 64 & 36.71 & 38.85 & 38.13 & 38.92 & 63.17 & 62.36 & 63.72 & 64.09 \\
 \hline
 32 & 37.95 & 38.69 & 39.45 & 38.34 & 64.05 & 62.30 & 63.8 & 64.33  \\
 \hline
 16 & 38.88 & 38.80 & 38.31 & 38.92 & 63.22 & 63.51 & 63.93 & 64.43  \\
 \hline
\end{tabular}
\caption{\label{tab:mmlu_abalation} Accuracy on MMLU dataset across GPT3-22B, Llama2-7B, 70B and Nemotron4-15B models.}
\end{table}


%\subsection{Perplexity achieved by various LO-BCQ configurations on LM evaluation harness}

\begin{table} \centering
\begin{tabular}{|c||c|c|c|c||c|c|c|c|} 
\hline
 $L_b \rightarrow$& \multicolumn{4}{c||}{8} & \multicolumn{4}{c||}{8}\\
 \hline
 \backslashbox{$L_A$\kern-1em}{\kern-1em$N_c$} & 2 & 4 & 8 & 16 & 2 & 4 & 8 & 16  \\
 %$N_c \rightarrow$ & 2 & 4 & 8 & 16 & 2 & 4 & 2 \\
 \hline
 \hline
 \multicolumn{5}{|c|}{Race (FP32 Accuracy = 37.51\%)} & \multicolumn{4}{|c|}{Boolq (FP32 Accuracy = 64.62\%)} \\ 
 \hline
 \hline
 64 & 36.94 & 37.13 & 36.27 & 37.13 & 63.73 & 62.26 & 63.49 & 63.36 \\
 \hline
 32 & 37.03 & 36.36 & 36.08 & 37.03 & 62.54 & 63.51 & 63.49 & 63.55  \\
 \hline
 16 & 37.03 & 37.03 & 36.46 & 37.03 & 61.1 & 63.79 & 63.58 & 63.33  \\
 \hline
 \hline
 \multicolumn{5}{|c|}{Winogrande (FP32 Accuracy = 58.01\%)} & \multicolumn{4}{|c|}{Piqa (FP32 Accuracy = 74.21\%)} \\ 
 \hline
 \hline
 64 & 58.17 & 57.22 & 57.85 & 58.33 & 73.01 & 73.07 & 73.07 & 72.80 \\
 \hline
 32 & 59.12 & 58.09 & 57.85 & 58.41 & 73.01 & 73.94 & 72.74 & 73.18  \\
 \hline
 16 & 57.93 & 58.88 & 57.93 & 58.56 & 73.94 & 72.80 & 73.01 & 73.94  \\
 \hline
\end{tabular}
\caption{\label{tab:mmlu_abalation} Accuracy on LM evaluation harness tasks on GPT3-1.3B model.}
\end{table}

\begin{table} \centering
\begin{tabular}{|c||c|c|c|c||c|c|c|c|} 
\hline
 $L_b \rightarrow$& \multicolumn{4}{c||}{8} & \multicolumn{4}{c||}{8}\\
 \hline
 \backslashbox{$L_A$\kern-1em}{\kern-1em$N_c$} & 2 & 4 & 8 & 16 & 2 & 4 & 8 & 16  \\
 %$N_c \rightarrow$ & 2 & 4 & 8 & 16 & 2 & 4 & 2 \\
 \hline
 \hline
 \multicolumn{5}{|c|}{Race (FP32 Accuracy = 41.34\%)} & \multicolumn{4}{|c|}{Boolq (FP32 Accuracy = 68.32\%)} \\ 
 \hline
 \hline
 64 & 40.48 & 40.10 & 39.43 & 39.90 & 69.20 & 68.41 & 69.45 & 68.56 \\
 \hline
 32 & 39.52 & 39.52 & 40.77 & 39.62 & 68.32 & 67.43 & 68.17 & 69.30  \\
 \hline
 16 & 39.81 & 39.71 & 39.90 & 40.38 & 68.10 & 66.33 & 69.51 & 69.42  \\
 \hline
 \hline
 \multicolumn{5}{|c|}{Winogrande (FP32 Accuracy = 67.88\%)} & \multicolumn{4}{|c|}{Piqa (FP32 Accuracy = 78.78\%)} \\ 
 \hline
 \hline
 64 & 66.85 & 66.61 & 67.72 & 67.88 & 77.31 & 77.42 & 77.75 & 77.64 \\
 \hline
 32 & 67.25 & 67.72 & 67.72 & 67.00 & 77.31 & 77.04 & 77.80 & 77.37  \\
 \hline
 16 & 68.11 & 68.90 & 67.88 & 67.48 & 77.37 & 78.13 & 78.13 & 77.69  \\
 \hline
\end{tabular}
\caption{\label{tab:mmlu_abalation} Accuracy on LM evaluation harness tasks on GPT3-8B model.}
\end{table}

\begin{table} \centering
\begin{tabular}{|c||c|c|c|c||c|c|c|c|} 
\hline
 $L_b \rightarrow$& \multicolumn{4}{c||}{8} & \multicolumn{4}{c||}{8}\\
 \hline
 \backslashbox{$L_A$\kern-1em}{\kern-1em$N_c$} & 2 & 4 & 8 & 16 & 2 & 4 & 8 & 16  \\
 %$N_c \rightarrow$ & 2 & 4 & 8 & 16 & 2 & 4 & 2 \\
 \hline
 \hline
 \multicolumn{5}{|c|}{Race (FP32 Accuracy = 40.67\%)} & \multicolumn{4}{|c|}{Boolq (FP32 Accuracy = 76.54\%)} \\ 
 \hline
 \hline
 64 & 40.48 & 40.10 & 39.43 & 39.90 & 75.41 & 75.11 & 77.09 & 75.66 \\
 \hline
 32 & 39.52 & 39.52 & 40.77 & 39.62 & 76.02 & 76.02 & 75.96 & 75.35  \\
 \hline
 16 & 39.81 & 39.71 & 39.90 & 40.38 & 75.05 & 73.82 & 75.72 & 76.09  \\
 \hline
 \hline
 \multicolumn{5}{|c|}{Winogrande (FP32 Accuracy = 70.64\%)} & \multicolumn{4}{|c|}{Piqa (FP32 Accuracy = 79.16\%)} \\ 
 \hline
 \hline
 64 & 69.14 & 70.17 & 70.17 & 70.56 & 78.24 & 79.00 & 78.62 & 78.73 \\
 \hline
 32 & 70.96 & 69.69 & 71.27 & 69.30 & 78.56 & 79.49 & 79.16 & 78.89  \\
 \hline
 16 & 71.03 & 69.53 & 69.69 & 70.40 & 78.13 & 79.16 & 79.00 & 79.00  \\
 \hline
\end{tabular}
\caption{\label{tab:mmlu_abalation} Accuracy on LM evaluation harness tasks on GPT3-22B model.}
\end{table}

\begin{table} \centering
\begin{tabular}{|c||c|c|c|c||c|c|c|c|} 
\hline
 $L_b \rightarrow$& \multicolumn{4}{c||}{8} & \multicolumn{4}{c||}{8}\\
 \hline
 \backslashbox{$L_A$\kern-1em}{\kern-1em$N_c$} & 2 & 4 & 8 & 16 & 2 & 4 & 8 & 16  \\
 %$N_c \rightarrow$ & 2 & 4 & 8 & 16 & 2 & 4 & 2 \\
 \hline
 \hline
 \multicolumn{5}{|c|}{Race (FP32 Accuracy = 44.4\%)} & \multicolumn{4}{|c|}{Boolq (FP32 Accuracy = 79.29\%)} \\ 
 \hline
 \hline
 64 & 42.49 & 42.51 & 42.58 & 43.45 & 77.58 & 77.37 & 77.43 & 78.1 \\
 \hline
 32 & 43.35 & 42.49 & 43.64 & 43.73 & 77.86 & 75.32 & 77.28 & 77.86  \\
 \hline
 16 & 44.21 & 44.21 & 43.64 & 42.97 & 78.65 & 77 & 76.94 & 77.98  \\
 \hline
 \hline
 \multicolumn{5}{|c|}{Winogrande (FP32 Accuracy = 69.38\%)} & \multicolumn{4}{|c|}{Piqa (FP32 Accuracy = 78.07\%)} \\ 
 \hline
 \hline
 64 & 68.9 & 68.43 & 69.77 & 68.19 & 77.09 & 76.82 & 77.09 & 77.86 \\
 \hline
 32 & 69.38 & 68.51 & 68.82 & 68.90 & 78.07 & 76.71 & 78.07 & 77.86  \\
 \hline
 16 & 69.53 & 67.09 & 69.38 & 68.90 & 77.37 & 77.8 & 77.91 & 77.69  \\
 \hline
\end{tabular}
\caption{\label{tab:mmlu_abalation} Accuracy on LM evaluation harness tasks on Llama2-7B model.}
\end{table}

\begin{table} \centering
\begin{tabular}{|c||c|c|c|c||c|c|c|c|} 
\hline
 $L_b \rightarrow$& \multicolumn{4}{c||}{8} & \multicolumn{4}{c||}{8}\\
 \hline
 \backslashbox{$L_A$\kern-1em}{\kern-1em$N_c$} & 2 & 4 & 8 & 16 & 2 & 4 & 8 & 16  \\
 %$N_c \rightarrow$ & 2 & 4 & 8 & 16 & 2 & 4 & 2 \\
 \hline
 \hline
 \multicolumn{5}{|c|}{Race (FP32 Accuracy = 48.8\%)} & \multicolumn{4}{|c|}{Boolq (FP32 Accuracy = 85.23\%)} \\ 
 \hline
 \hline
 64 & 49.00 & 49.00 & 49.28 & 48.71 & 82.82 & 84.28 & 84.03 & 84.25 \\
 \hline
 32 & 49.57 & 48.52 & 48.33 & 49.28 & 83.85 & 84.46 & 84.31 & 84.93  \\
 \hline
 16 & 49.85 & 49.09 & 49.28 & 48.99 & 85.11 & 84.46 & 84.61 & 83.94  \\
 \hline
 \hline
 \multicolumn{5}{|c|}{Winogrande (FP32 Accuracy = 79.95\%)} & \multicolumn{4}{|c|}{Piqa (FP32 Accuracy = 81.56\%)} \\ 
 \hline
 \hline
 64 & 78.77 & 78.45 & 78.37 & 79.16 & 81.45 & 80.69 & 81.45 & 81.5 \\
 \hline
 32 & 78.45 & 79.01 & 78.69 & 80.66 & 81.56 & 80.58 & 81.18 & 81.34  \\
 \hline
 16 & 79.95 & 79.56 & 79.79 & 79.72 & 81.28 & 81.66 & 81.28 & 80.96  \\
 \hline
\end{tabular}
\caption{\label{tab:mmlu_abalation} Accuracy on LM evaluation harness tasks on Llama2-70B model.}
\end{table}

%\section{MSE Studies}
%\textcolor{red}{TODO}


\subsection{Number Formats and Quantization Method}
\label{subsec:numFormats_quantMethod}
\subsubsection{Integer Format}
An $n$-bit signed integer (INT) is typically represented with a 2s-complement format \citep{yao2022zeroquant,xiao2023smoothquant,dai2021vsq}, where the most significant bit denotes the sign.

\subsubsection{Floating Point Format}
An $n$-bit signed floating point (FP) number $x$ comprises of a 1-bit sign ($x_{\mathrm{sign}}$), $B_m$-bit mantissa ($x_{\mathrm{mant}}$) and $B_e$-bit exponent ($x_{\mathrm{exp}}$) such that $B_m+B_e=n-1$. The associated constant exponent bias ($E_{\mathrm{bias}}$) is computed as $(2^{{B_e}-1}-1)$. We denote this format as $E_{B_e}M_{B_m}$.  

\subsubsection{Quantization Scheme}
\label{subsec:quant_method}
A quantization scheme dictates how a given unquantized tensor is converted to its quantized representation. We consider FP formats for the purpose of illustration. Given an unquantized tensor $\bm{X}$ and an FP format $E_{B_e}M_{B_m}$, we first, we compute the quantization scale factor $s_X$ that maps the maximum absolute value of $\bm{X}$ to the maximum quantization level of the $E_{B_e}M_{B_m}$ format as follows:
\begin{align}
\label{eq:sf}
    s_X = \frac{\mathrm{max}(|\bm{X}|)}{\mathrm{max}(E_{B_e}M_{B_m})}
\end{align}
In the above equation, $|\cdot|$ denotes the absolute value function.

Next, we scale $\bm{X}$ by $s_X$ and quantize it to $\hat{\bm{X}}$ by rounding it to the nearest quantization level of $E_{B_e}M_{B_m}$ as:

\begin{align}
\label{eq:tensor_quant}
    \hat{\bm{X}} = \text{round-to-nearest}\left(\frac{\bm{X}}{s_X}, E_{B_e}M_{B_m}\right)
\end{align}

We perform dynamic max-scaled quantization \citep{wu2020integer}, where the scale factor $s$ for activations is dynamically computed during runtime.

\subsection{Vector Scaled Quantization}
\begin{wrapfigure}{r}{0.35\linewidth}
  \centering
  \includegraphics[width=\linewidth]{sections/figures/vsquant.jpg}
  \caption{\small Vectorwise decomposition for per-vector scaled quantization (VSQ \citep{dai2021vsq}).}
  \label{fig:vsquant}
\end{wrapfigure}
During VSQ \citep{dai2021vsq}, the operand tensors are decomposed into 1D vectors in a hardware friendly manner as shown in Figure \ref{fig:vsquant}. Since the decomposed tensors are used as operands in matrix multiplications during inference, it is beneficial to perform this decomposition along the reduction dimension of the multiplication. The vectorwise quantization is performed similar to tensorwise quantization described in Equations \ref{eq:sf} and \ref{eq:tensor_quant}, where a scale factor $s_v$ is required for each vector $\bm{v}$ that maps the maximum absolute value of that vector to the maximum quantization level. While smaller vector lengths can lead to larger accuracy gains, the associated memory and computational overheads due to the per-vector scale factors increases. To alleviate these overheads, VSQ \citep{dai2021vsq} proposed a second level quantization of the per-vector scale factors to unsigned integers, while MX \citep{rouhani2023shared} quantizes them to integer powers of 2 (denoted as $2^{INT}$).

\subsubsection{MX Format}
The MX format proposed in \citep{rouhani2023microscaling} introduces the concept of sub-block shifting. For every two scalar elements of $b$-bits each, there is a shared exponent bit. The value of this exponent bit is determined through an empirical analysis that targets minimizing quantization MSE. We note that the FP format $E_{1}M_{b}$ is strictly better than MX from an accuracy perspective since it allocates a dedicated exponent bit to each scalar as opposed to sharing it across two scalars. Therefore, we conservatively bound the accuracy of a $b+2$-bit signed MX format with that of a $E_{1}M_{b}$ format in our comparisons. For instance, we use E1M2 format as a proxy for MX4.

\begin{figure}
    \centering
    \includegraphics[width=1\linewidth]{sections//figures/BlockFormats.pdf}
    \caption{\small Comparing LO-BCQ to MX format.}
    \label{fig:block_formats}
\end{figure}

Figure \ref{fig:block_formats} compares our $4$-bit LO-BCQ block format to MX \citep{rouhani2023microscaling}. As shown, both LO-BCQ and MX decompose a given operand tensor into block arrays and each block array into blocks. Similar to MX, we find that per-block quantization ($L_b < L_A$) leads to better accuracy due to increased flexibility. While MX achieves this through per-block $1$-bit micro-scales, we associate a dedicated codebook to each block through a per-block codebook selector. Further, MX quantizes the per-block array scale-factor to E8M0 format without per-tensor scaling. In contrast during LO-BCQ, we find that per-tensor scaling combined with quantization of per-block array scale-factor to E4M3 format results in superior inference accuracy across models. 


\end{document}

% \section{Quantum matrix decomposition}
% \begin{enumerate}
%     \item Quantum Singular Value Decomposition (QSVD) \cite{rebentrost2018quantum}
%     \item Quantum Phase Estimation (QPE) for Eigenvalue Problems
%     \item Quantum Principal Component Analysis (QPCA)
%     \item HHL Algorithm for Solving Linear Systems
%     \item Quantum Eigenvalue Estimation (QEE)
%     \item Quantum Fourier Transform (QFT) in Matrix Decomposition
%     \item Quantum Randomized Algorithms for Matrix Factorization
% \end{enumerate}

% \section{Quantum tensor decomposition}


%\newpage