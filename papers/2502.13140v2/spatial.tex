\subsection{Spatial Transcriptomics}\label{sec:td_spatial}
%\subsubsection{Applications}
% \AB{General applications of TD in spatial transcriptomics. Mostly focused on integrating and analyzing single-cell spatial transcriptomic data~\cite{broadbent2024deciphering, song2023gntd, liu2017characterizing, armingol2022context} by tensor decomposition or extracting latent embeddings. We need to motivate usage of tensor decomposition in integrating spatial transcriptomics by regions of interest. }

Spatial transcriptomics enables the mapping of gene expression patterns within tissue architecture, providing critical insights into cellular organization and function. However, analyzing such high-dimensional data remains a computational challenge, especially when integrating multiple regions of interest (ROIs) to uncover spatially organized molecular patterns. As discussed above, tensor decomposition is a powerful approach to extract meaningful latent structures from spatial transcriptomic datasets, facilitating deconvolution of cell–cell interactions, spatial clustering of gene expression, and integration of multimodal spatial omics data. Armingol et al.~\cite{armingol2022context} developed \texttt{Tensor-cell2cell}, a non-negative tensor component analysis method, which is an extension of NMF to higher-order tensors—to analyze cell–cell communication while incorporating spatial context. In their work, a 4D tensor representing cell–cell interactions was reconstructed, and a comprehensive error analysis was subsequently performed. Traditional approaches often rely on bulk transcriptomic deconvolution, which disregards spatial positioning, but \texttt{Tensor-cell2cell} leverages tensor decomposition to preserve high-order interactions between signaling pathways across different tissue regions. By integrating spatially resolved transcriptomic profiles, this method enhances the identification of key cell–cell communication networks that drive tissue function and disease progression. In another study~\cite{broadbent2024deciphering}, a graph-guided Tucker decomposition method was used to decipher high-order structures in spatial transcriptomic data. This approach combines TD with graph-based priors to account for spatial dependencies between neighboring regions of interest. Unlike conventional decomposition techniques, which treat gene expression as independent across space, this method embeds structural tissue information into the decomposition process, leading to improved reconstruction of spatial transcriptomes and more biologically meaningful clustering of tissue subregions. Low-rank TD methods have also been utilized in characterizing the spatiotemporal transcriptome of the human brain~\cite{liu2017characterizing}. By applying tensor factorization, the study extracted low-dimensional representations of gene expression across different brain regions and developmental time points, enabling the identification of major gene expression modules that define functional brain architecture. This method provides a scalable framework for integrating multi-region transcriptomic datasets, offering insights into both spatial organization and temporal dynamics of gene regulation in the brain. Tensor-based spatial transcriptomics analysis was further advanced by the introduction of Graph-Guided Neural Tensor Decomposition (\texttt{GNTD})~\cite{song2023gntd}.  This model integrates spatial and functional relationships by embedding gene co-expression networks into the TD framework. Unlike traditional tensor methods that primarily focus on spatial structure, \texttt{GNTD} combines neural network-based embeddings with graph-guided TD, allowing for the reconstruction of missing transcriptomic signals while preserving biologically relevant spatial interactions. This approach enhances the accuracy of transcriptomic reconstructions in sparsely sampled regions, improving the detection of functional tissue microdomains.

Collectively, these studies highlight the power of TD in integrating and analyzing spatial transcriptomic data across different regions of interest. Hence, tensor-based techniques can uncover latent gene expression patterns, reconstruct missing transcriptomic information, and improve spatially aware clustering of cells. As spatial transcriptomics technologies continue to evolve, tensor decomposition will play a crucial role in enabling more precise and scalable analyses of high-dimensional spatial omics datasets.

% \subsubsection{Challenges}
% %\AB{Phase transition/computational hardness discussions with plots following from the notebook shared with you}

% Despite the advantages of tensor decomposition in spatial transcriptomics, several challenges remain in its application to high-dimensional and spatially heterogeneous datasets. One key challenge is the complexity of spatial dependencies, as gene expression varies not only across different tissue regions but also within microenvironments, making it difficult to define an optimal tensor structure that captures both local and global expression patterns. Traditional tensor decomposition methods often assume a predefined rank or spatial resolution, which may not be suitable for highly dynamic and heterogeneous tissue architectures. Additionally, sparsity in spatial transcriptomic data—where many genes are not detected in all spatial locations—limits the accuracy of decomposition models and can lead to biased reconstructions, especially when missing data is not randomly distributed but influenced by biological or technical factors. Another challenge is computational scalability, as large-scale spatial transcriptomics datasets with thousands of genes and spatial positions require substantial memory and processing power, making traditional tensor-based approaches computationally expensive. Furthermore, biological interpretability of tensor decomposition outputs remains a challenge, as extracted latent components may not always correspond to clear biological pathways or cell-type-specific interactions, necessitating additional validation with external datasets or functional assays. Finally, integration of multimodal spatial omics data poses a challenge, as combining spatial transcriptomics with proteomics or epigenomic data using tensor methods requires designing decomposition strategies that can accommodate different data modalities, resolutions, and measurement biases. Addressing these challenges will be crucial for unlocking the full potential of tensor decomposition in spatial transcriptomics and improving its utility for biological discovery.