\section{Applications in Biomedical Analysis}
~\label{sec:td_biomed}
Tensor decomposition extracts latent structures from high dimensional data and preserves interpretability of the features. This is critical in analyzing biomedical data to ensure the insights drawn from lower-dimensional factors are reliable and actionable, leading to biomarker discovery, prognosis, and developing therapeutics for complex diseases. 
% Preserving this interpretability in biological data is of critical importance to ensure the insights drawn from these analyses are reliable and actionable, which is essential when these results directly impact health and scientific discovery. Given the increasing volume and complexity of biomedical data, ranging from multi-omics profiles to high-resolution imaging and spatial transcriptomics, tensor decomposition offers a unique approach to uncover latent biological signals, integrate diverse datasets, and enhance interpretability in data-driven research. 

% \paragraph{Multi-omics} Tensor decomposition helps integrate heterogeneous biological data (e.g., genomics, transcriptomics, proteomics) by identifying shared and distinct molecular signatures across different biological layers. This improves biomarker discovery, patient stratification, and disease subtyping, leading to more personalized medicine approaches.  

% \paragraph{Medical imaging} For MRI, fMRI, and CT scans, tensor methods aid in denoising, feature extraction, and identifying disease-related patterns by decomposing images into meaningful components. This improves diagnostic accuracy and allows for better interpretation of underlying pathophysiological processes.  

% \paragraph{Spatial transcriptomics} Tensor decomposition is used to analyze high-dimensional spatial gene expression data, revealing underlying cellular structures and tissue organization. By capturing spatially varying gene expression patterns, it enhances our understanding of cell-cell interactions and disease microenvironments, such as in cancer or neurodegenerative disorders.  

%\paragraph{Topic modeling} 
To further explore the role of tensor decomposition in biomedical research, we used a topic modeling-based approach by employing BERTopic~\cite{grootendorst2022bertopic} to conduct an unsupervised analysis of the literature on tensor decomposition and biological data in PubMed. By leveraging transformer-based embeddings and topic clustering, we systematically identified key themes and research trends related to TD and its applications in biomedicine. %multi-omics, medical imaging, and spatial transcriptomics. 
The terms we searched were \say{tensor decomposition} in combination with (\texttt{AND}) the following terms: \say{genomics},
\say{transcriptomics},
\say{proteomics},
\say{metabolomics},
\say{epigenomics},
\say{microbiomics},
\say{multiomics} \texttt{OR} \say{multi-omics},
\say{cancer},
\say{cardiovascular disease},
\say{diabetes},
\say{alzheimer's disease},
\say{neurological disorder},
\say{autoimmune disease},
\say{kidney disease},
\say{obesity},
\say{medical imaging}. From those 16 terms, 536 documents were extracted from PubMed in the past ten years. This approach enabled us to identify dominant research areas, emerging applications, and potential gaps in the field, providing a structured overview of how tensor decomposition is being utilized across different biomedical domains.

Our analysis revealed a well-established use of TD in medical imaging, where it has been applied for tasks such as noise reduction, feature extraction, and disease pattern recognition~\cite{bustin2019high,Brender2019,Li2021,Zhang2017,Chatzichristos2019}(Figure~\ref{fig:sidebyside}A), with 218 articles cited over 1,400 times (Supplementary Table 1). This aligns with the long history of tensor methods in signal processing and image analysis. This is followed by applications in complex diseases such as cardiovascular disease~\cite{zhao2019detecting}, with TD methods used to factor ECG signals~\cite{liu2020automated} and X-rays~\cite{sedighin2024tensor}, among other applications. Similarly, TD methods have been used to analyze EEG~\cite{cong2015tensor} and neuroimaging data for neurological diseases such as Alzheimer's disease~\cite{durusoy2018multi}, Parkinson's disease~\cite{pham2017tensor}, stroke~\cite{zhou2024tensor}, glioblastoma~\cite{yang2014discrimination}, etc. More recently, we observed a slight increase in the application of TD to multi-omics data, reflecting the growing need for integrative approaches in systems biology (Figure~\ref{fig:sidebyside}A). This trend highlights the increasing recognition of tensor methods as powerful tools for extracting meaningful patterns from high-dimensional, heterogeneous biological datasets.
\begin{figure}[htbp]
    \centering
    \includegraphics[width=\linewidth]{figures/fig3_pubmed.png}
    \caption{Tensor decomposition applications in biomedical data in the literature. \textbf{A.} Stacked bar chart of the number of papers from PubMed published with the topic \protect\say{tensor decomposition} \texttt{AND} categories listed in the legend in the past decade. \protect\textbf{B.} UMAP of the BERTopic embeddings visualizing the different topic clusters. Each data point is a research paper. Applications of \protect\say{spike tensors} are highlighted with a star symbol.}
   \label{fig:sidebyside}
\end{figure}

% \begin{figure}[htbp] 
%     \centering
%     \begin{subfigure}[b]{0.45\textwidth}
%         \centering
%         \includegraphics[width=\linewidth]{figures/doc_trend.png}
%         \caption{Stacked bar chart of the number of papers from PubMed published with the topic \say{tensor decomposition} \texttt{AND} categories listed in the legend in the past decade. }
%         \label{fig:fig1}
%     \end{subfigure}
%     \hfill
%     \begin{subfigure}[b]{0.45\textwidth}
%         \centering
%         \includegraphics[width=\linewidth]{figures/umap_w_legend.png}
%         \caption{UMAP of the BERTopic embeddings visualizing the different topics found for tensor decomposition applications in the literature across past ten years. Documents relating to \say{spike tensors} are highlighted with a star symbol. Every data point is a research paper.}
%         \label{fig:fig2}
%     \end{subfigure}
%     \caption{Tensor decomposition application in biomedical data across the past decade in the literature. }
%     \label{fig:sidebyside}
% \end{figure}

To further interpret the structure of the identified topics, we visualized the document embeddings using UMAP, with topics color-coded for clarity (Figure~\ref{fig:sidebyside}B). This visualization revealed distinct topic clusters, highlighting distinct areas of research focus. The most represented topics were \say{\texttt{cell\_omics\_expression}}, \say{\texttt{diffusion\_imaging\_noise}}, and \say{\texttt{brain\_connectivity\_eeg}}, reflecting the strong presence of tensor decomposition in imaging-related applications. The \say{\texttt{cell\_omics\_expression}} topic encompassed studies leveraging tensor methods for multi-omics and spatial transcriptomics to analyze gene expression across cells and tissues~\cite{hore2016tensor,Omberg2007,Ramdhani2020,Wang2019}. The \say{\texttt{brain\_connectivity\_eeg}} topic captured research applying TD to neurological and electrophysiological data, particularly EEG and fMRI, for brain signal processing and cognitive studies~\cite{Li2021,Chatzichristos2019}. The \say{\texttt{diffusion\_imaging\_noise}} topic represented works focused on image denoising, enhancement, and reconstruction in medical imaging, further reinforcing the dominant role of tensor methods in this domain~\cite{bustin2019high,Brender2019,Li2021,Zhang2017,Chatzichristos2019}. More information on the other topic clusters are included in the Supplementary Tables 1-3.

This exploration highlights the significant role of TD in biomedicine, demonstrating its widespread application in medical imaging, multi-omics, and neuroscience. The increasing adoption of these methods emphasizes their value in extracting meaningful insights from complex biomedical data. Tensors are unique in their applications in multi-modal, multi-omics data because they have the ability to integrate these data sets, find linear and non-linear ineteractions between them, and extract meaningful latent representations which can be then used to perform downstream tasks~\cite{kolda2009tensor, luo2017tensor}. We provide a framework for integrating TD methods, including emerging methods such as quantum tensor decomposition in Figure~\ref{fig:qtd_framework}. Although insightful, this investigation is far from exhaustive and does not provide a holistic view on the literature with specific sub-areas such as genomics, MRI images, electronic health records (EHR), spatial transcriptomics, etc. having more applications of TD~\cite{hore2016tensor, olesen2023tensor, sedighin2024tensor, ho2014marble, song2023gntd}. Hence, we will focus on each of these sub-areas broadly categorized into biomedical imaging, multi-omics analysis, and spatial transcriptomics. 
%Looking ahead, we aim to explore the potential of quantum computing to further enhance tensor decomposition techniques, as quantum approaches may offer new computational efficiencies and capabilities for handling the ever-growing scale and complexity of biomedical datasets.




\subsection{Biomedical Imaging}\label{sec:td_imaging}
%\subsubsection{Applications}

Tensor decomposition methods have emerged as powerful tools for analyzing high-dimensional data in biomedical imaging~\cite{sedighin2024tensor}.
%By leveraging the ability to disentangle high-dimensional data into lower-dimensional components, these techniques can reduce computational complexity while preserving meaningful structural information, which is often obscured in raw data. 
Applications span a wide range of domains, including medical image denoising, super resolution, reconstruction, and so on, where tensors effectively model spatial, temporal, and spectral relationships in MRI and hyperspectral images. Beyond MRI and conventional imaging modalities, TD techniques have shown promise in analyzing high-dimensional datasets generated by multiphoton microscopy, including fluorescence-based and label-free modalities such as second harmonic generation (SHG) and third harmonic generation (THG) imaging~\cite{jamesdarian2021recent}.  In this section, we will review these applications from a methods perspective by discussing how each tensor decomposition technique is used in biomedical imaging analysis. 

\paragraph{CP decomposition.} CP decomposition is widely used in real-world data due to its simplicity, interpretability, and ability to efficiently represent multi-dimensional data. It decomposes the data tensor into a sum of rank-one components (Equation~\ref{eq:cp}) to uncover latent structures, which is particularly useful in data with spatial, temporal, and spectral dimensions like MRI data and hyperspectral image. CP decomposition has been used to reconstruct images from under-sampled data like brain \cite{li2023learned, hou2020matrix}, cardiac \cite{wu2018multiple} MRI and CT scans \cite{zhang2016tensor}. Using the low-rank nature of medical images it achieves high-quality reconstructions with reduced scan times. In magnetic resonance fingerprinting (MRF) reconstruction, CP decomposition is used as a component in a neural network to learn low-rank tensor priors~\cite{li2023learned}, avoiding computationally expensive SVD on high-dimensional data.  

CP decomposition can help reduce redundancy and has been used for brain MRI denoising~\cite{cao2016tensor, cui2020multidimensional}. By selecting the dominant components in CP decomposition, noise and artifacts in medical images can be effectively removed, preserving only the significant structures or patterns. Variants such as Bayesian CP decomposition have been proposed to recover the de-noised tensor from a noisy tensor by estimating rank-one tensors through a variational Bayesian inference strategy~\cite{cao2016tensor}. CP decomposition has been used in a dental CT super resolution task producing high SNR with robust segmentation quality~\cite{hatvani2018tensor}. It is also used in hyperspectral image unmixing with applications in fluorescence microscopy of retina~\cite{dey2019tensor}. In this work, different excitation wavelengths were assembled as a tensor and a non-negative CP decomposition was performed to obtain low-rank factors. Other applications include cardiac $T_1$ mapping \cite{yaman2019low} and feature extraction for multi-modal data, where a robust coupled CP decomposition was used with an assumption that the tensors shared the first factor marix~\cite{zhao2023robust}. In multiphoton microscopy, CP decomposition has been utilized to extract meaningful components from 3D distribution in disease models such as cancer and fibrosis~\cite{uckermann2020label}.
%In Cao \textit{et al.}'s work \cite{cao2016tensor}, a Bayesian CP decomposition method is proposed to recover the clean tensor from its noisy observation, in which the rank-one tensors are estimated through a variational Bayesian inference strategy. 

% For it providing simple representations for high dimensional tensors, CP decomposition has also been used for various tasks such as dental CT super resolution\cite{hatvani2018tensor}, hyperspectral image unmixing for fluorescence microscopy of retina \cite{dey2019tensor}, cardiac $T_1$ mapping \cite{yaman2019low}, and feature extraction for multimodal data \cite{zhao2023robust}. In Karahan \textit{et al.}'s work \cite{dey2019tensor}, images from different excitation wavelengths are assembled as a tensor, and non-negative CP decomposition is applied to the tensor to perform low-rank decomposition. In Zhao \textit{et al.}'s work \cite{zhao2023robust}, to extract common features from multiple tensors, a robust coupled CP decomposition is proposed with the assumption that these tensors share the first factor matrix of CP decomposition.


\paragraph{Tucker decomposition.}
%\os{(Shehab: In this section, every paragraph starts with the same phrase. Maybe some of them should be rephrased? Jiasen: They has been rephrased)}
Tucker decomposition expresses a tensor as a core tensor multiplied by factor matrices along each mode. Compared with HOSVD and CP decomposition, Tucker decomposition allows for arbitrary forms of factor matrices, providing a more general representation. It has been widely used for predicting missing information in high-dimensional images, with applications in super-resolution \cite{gui2017brain, jia2022nonconvex, hatvaniy2021single} and reconstruction \cite{wu2018multiple, wang2021spectral, li2018efficient}. The general model can be summarized as $\mathcal{T} = \mathcal{A}\mathcal{X} $ where $T$ is the observed tensor, $X$ is the original image to be predicted and $X$ is a linear operator. By representing $X$ with Tucker decomposition, the operation for each mode and their interactions can be separated, providing interpretable insight into the data. For example, in super-resolution of dental CT images, the high-resolution image is represented by a Tucker decomposition~\cite{hatvaniy2021single}. Applications of Tucker decomposition also includes 
%regarding  the high resolution image to be predicted is represented with Tucker decomposition. The blurring operation is modeled with three 2D matrices multiplied with the factor matrices.
%Tucker decomposition has also been used to help improve 
$T_1$ mapping \cite{liu2021accelerating, yaman2019low}. Accelerating the acquisition of 3D $T{1\rho}$ mapping of cartilage, the image tensor is defined as the sum of a sparse component and a low rank tensor. Tucker decomposition is used for tensor low-rank regularization~\cite{liu2021accelerating}. It can also be used for approximating higher quality images and improving $T_1$ mapping performance~\cite{yaman2019low}.  

%In Liu \textit{et al.}'s work \cite{liu2021accelerating} regarding accelerating the acquisition of 3D $T{1\rho}$ mapping of cartilage, the image tensor is defined as the sum of a sparse component and a low-rank tensor, and the Tucker decomposition is used for tensor low-rank regularization. In Yaman \textit{et al.}'s work \cite{yaman2019low}, a CP or Tucker decomposition based method is proposed to approximate the images with higher quality and then improve $T_1$ mapping performance.


Another application of Tucker decomposition is in medical image fusion \cite{yin2018tensor, zhang2023tdfusion, chen2024multi}. In these methods, original or processed medical images from diverse modalities are represented with Tucker decomposition sharing the same core tensor or factor matrices. Sparsity regularization is introduced to the core tensors. For example, Yin \textit{et al.}~\cite{yin2018tensor} tackled image fusion by first learning a dictionary from training images. They represented each image as a tensor split into unique sparse core tensors and shared factor matrices. When fusing images, they used these same factor matrices to compute the new tensor representations, letting the core tensors dictate the fusion weights. Beyond MRI, Tucker decomposition has been leveraged to fuse multiphoton imaging data with complementary modalities, such as Raman or fluorescence lifetime imaging, allowing integrative tissue characterization in tumor microenvironments~\cite{karahan2015tensor}. It enables robust feature extraction and pattern recognition, facilitating automated classification of healthy versus diseased tissue regions. 
% In Yin \textit{et al.}'s work \cite{yin2018tensor} about image fusion through dictionary learning, tensor sparse representations of a set of training images are obtained, with different sparse core tensors and sharing factor matrices for each mode. In the fusion step, the tensor sparse representations of the fused images are computed similarly with the learned factor matrices. Finally the fusion weights are decided by the core tensors of the fused images.
Besides, Tucker decomposition has shown the potential to be used for medical image segmentation \cite{khan2022deep, weber2025posttraining}, usually as a component of a deep neural network. In knee cartilage MRI segmentation, the original input images are reconstructed by Tucker decomposition with low rank~\cite{khan2022deep}. 
%Khan \textit{et al.}'s work \cite{khan2022deep} about  
Both the original and reconstructed inputs are fed into the neural network to learn the inter-dimensional tissue structures. In another work~\cite{weber2025posttraining}, the Tucker-decomposed convolution is proposed to replace the conventional 3D convolution and improve computational efficiency of the neural network. 

The application of Tucker decomposition in MRI and multiphoton brain imaging data extends to optogenetics and imaging studies, where simultaneous stimulation and recording of neuronal populations require advanced computational techniqus to disentangle overlapping signals. It separates the evoked responses from spontaneous activity in high-dimensional optogenetic data, aiding the identification of circuit-level changes in neurodevelopmental and neurodegenerative disorders~\cite{erol2022tensors}. 


\paragraph{HOSVD.} The applications of HOSVD, a generalization of the matrix SVD, are ubiquitous in biomedical image analysis. As a constraint case of Tucker decomposition, the factor matrices and core tensor of HOSVD are orthogonal. Therefore, HOSVD is less flexible than Tucker decomposition but can ensure decorrelation between components along each mode. For its ability to use the redundancy in high-dimensional data, HOSVD has been used for brain MRI denoising \cite{fu20163d, zhang2017denoise, bustin2019high, wang2020modified, kim2021denoising, olesen2023tensor}. Specifically,
%a  In Olesen \textit{et al.}'s work \cite{olesen2023tensor}, 
Marchenko-Pastur principal component analysis (MPPCA), a matrix-based method for denoising, is generalized to tensor MPPCA based on HOSVD so that it can directly process multidimensional MRI data~\cite{olesen2023tensor}. A patch-based HOSVD method for denoising has also been proposed, combined with a global HOSVD method to mitigate stripe artifacts in the outputs~\cite{zhang2017denoise}.  
Like Tucker decomposition, HOSVD has been used for medical image reconstruction~\cite{liu2021calibrationless, yi2021joint, roohi2016dynamic, zhang2022compressed} of MRI data by exploiting sparsity and low rank properties. It has also been applied in low-rank tensor approximation of the multi-slice image tensor in parallel MRI reconstruction~\cite{liu2021calibrationless}. HOSVD, analogous to the Tucker decomposition, enables the computation of a core tensor and the factorization of three-way tensors into three matrices. This capability has proven particularly effective for feature extraction, as demonstrated in a blood cell recognition task and wavelet transform in colored pupil images, where they are naturally represented as three-way tensors~\cite{le2023tensor,giap2023adaptive}. 

HOSVD has found utility in analyzing complex spatiotemporal patterns in brain imaging data acquired through single-photon and multiphoton techniques such as two photon and three-photon calcium imaging~\cite{grewe2010high}. It has been applied to denoise neuronal activity maps while preserving underlying spatiotemporal dynamics by reducing motion artificats in calcium imaging datasets. It improved the detection of neuronal ensembles in awake behaving animals~\cite{cho2023robust}. 


% HOSVD has also been used for feature extraction\cite{le2023tensor, giap2023adaptive}. In Le \textit{et al.}'s work \cite{le2023tensor} about feature extraction for blood cell recognition, the color image is represented as a 3D tensor. Through HOSVD, the core tensor is computed and split into three matrices. While the middle one is kept unchanged, the other two are replaced with all-zero matrices and then the feature is obtained by reconstructing the image with the new core tensor. In Giap \textit{et al.}'s work \cite{giap2023adaptive} for feature extraction of color pupil image, a 3D tensor is constructed from the color pupil image based on wavelet transform. HOSVD is applied to eliminate redundant information in the tensor and estimate the feature information.


\paragraph{t-SVD.}
    Unlike HOSVD, which is based on tensor-matrix product, tensor-SVD (t-SVD) represents a 3-way tensor as the \say{t-product} of three 3-way tensors. The t-SVD, a relatively recent and nuanced tensor decomposition technique, has been effectively employed in a range of medical imaging applications, particularly for enforcing low-rank regularization constraints. It has been used for MRI reconstruction \cite{jiang2020improved, liu2025dynamic, ai2018dynamic, liu2023low}. Specifically, it has been used to approximate the rank minimization problem, which is NP-hard to a tensor nuclear norm minimization, with applications in low-rank tensor regularization in a dynamic MRI reconstruction model~\cite{liu2025dynamic}. Similar to other tensor decomposition methods, t-SVD has been used in denoising MRI images~\cite{khaleel2018denoising, kong2017new, khaleel2018denoising2} as well as in image segmentation tasks~\cite{shi2021multi}. A low-rank approximation of the noisy image is obtained by thresholding the t-SVD coefficients in the Fourier domain to perform denoising~\cite{khaleel2018denoising2}. For image segmentation of pathological liver CT, a low-rank tensor decomposition is performed based on t-SVD~\cite{shi2021multi}. Specifically, it is used to recover the underlying low-rank structure of the 3D images and generate tumor-free liver atlases.  
    t-SVD based low-rank approximations have also been used to model neural network connectivity and extract functional components from large-scale neural activity data~\cite{williams2018unsupervised}. 


As observed in several tensor decomposition methods applied in biomedical imaging, it is often used together with deep learning techniques such as convolutional neural networks (CNNs)~\cite{yaman2019low,oymak2021learning, khan2022deep, li2023learned}. Deep learning techniques have shown empirical success to process and extract features from images, or to reconstruct them. However, their effectiveness is still a mystery with their heavy dependence on several parameters and their sensitivity to noise along with lack of generalization to out-of-distribution data, and overfitting in limited training samples~\cite{chen2023deep}. Tensor decompositions play a crucial role here extracting meaningful features which increase generalizability of neural network models. Tensor decomposition methods have shown to learn kernels from training data and increase performance of CNNs~\cite{oymak2021learning}. They are also used to reduce training parameters of CNNs and reduce model size, leading to effective training~\cite{liu2023tensor}. Hence, TD methods can be used in conjunction with deep learning methods to enhance the overall throughput and performance of downstream tasks such as prediction, segmentation, reconstruction, super-resolution, etc. in analyzing biomedical images. 

    %For example, in Liu \textit{et al.}'s work \cite{liu2025dynamic}, a multi-directional low-rank tensor regularization is applied to the dynamic MRI reconstruction model. The NP-hard rank-minimization problem is then approximated to a tensor nuclear norm minimization, which can be represented with the singular tensors obtained through t-SVD. 

% t-SVD has also been used for brain MRI denoising \cite{khaleel2018denoising, kong2017new, khaleel2018denoising2}. In Khaleel \textit{et al.}'s work \cite{khaleel2018denoising, khaleel2018denoising2}, low rank approximation of the noisy image is obtained by thresholding the t-SVD coefficients in Fourier domain.


% Besides, t-SVD shows potential to be applied in medical image segmentation. In Shi \textit{et al.}'s work \cite{shi2021multi} for pathological liver CT segmentation, low-rank tensor decompisition (LRTD) method based on an improved t-SVD is applied to different steps of the multi-atlas segmentation framework. Specifically, it is used to recover the underlying low-rank structure of the 3D images and generate tumor free liver atlases to benefit liver segmentation.

% \paragraph{Tensor Decomposition in Tissue Imaging with Multiphoton Microscopy}
% Beyond MRI and conventional imaging modalities, TD techniques have shown promise in analyzing high-dimensional datasets generated by multiphoton microscopy, including fluorescence-based and label-free modalities such as second harmonic generation (SHG) and third harmonic generation (THG) imaging~\cite{jamesdarian2021recent}. These imaging techniques offer high spatial resolution and deep tissue penetration, making them ideal for studying tissue architecture, extracellular matrix composition, and cellular interactions in both healthy and pathological conditions~\cite{campagnola2003second}.
% However, the large-scale, multi-channel datasets acquired from multiphoton imaging pose computational challenges, necessitating advanced decomposition strategies for efficient analysis, denoising, and feature extraction~\cite{vinegoni2020fluorescence}.

% In SHG and THG imaging, tensor decomposition methods such as CP decomposition and Tucker decomposition can be employed to separate intrinsic signal components from noise and enhance structural details in fibrous tissues. For example, CP decomposition has been utilized to extract meaningful components from 3D SHG datasets of collagen-rich tissues, improving segmentation and quantitative analysis of collagen density distribution in disease models such as cancer and fibrosis~\cite{uckermann2020label}. Similarly, Tucker decomposition has been leveraged to fuse multiphoton imaging data with complementary modalities, such as Raman or fluorescence lifetime imaging, allowing for integrative tissue characterization in tumor microenvironments~\cite{karahan2015tensor}. These techniques enable robust feature extraction and pattern recognition, facilitating automated classification of healthy versus diseased tissue regions.

% \paragraph{Tensor Decomposition for Single and Multiphoton Brain Imaging}
% Multiphoton microscopy is widely used in neuroscience to investigate brain activity at the cellular and network levels. Tensor decomposition approaches provide powerful means to analyze the complex spatiotemporal patterns present in brain imaging data acquired through single-photon and multiphoton techniques, such as two-photon and three-photon calcium imaging~\cite{grewe2010high}. In these applications, HOSVD and t-SVD have been applied to denoise and reconstruct neuronal activity maps while preserving the underlying spatiotemporal dynamics. For instance, HOSVD has been used to reduce motion artifacts and enhance neural signal extraction from calcium imaging datasets, improving the detection of neuronal ensembles in awake behaving animals~\cite{cho2023robust}. Similarly, t-SVD-based low-rank approximations have been employed to model neural network connectivity and extract functional components from large-scale recordings~\cite{cai2022review}.

% The application of tensor decomposition in multiphoton brain imaging extends to optogenetics and functional imaging studies, where simultaneous stimulation and recording of neuronal populations require advanced computational techniques to disentangle overlapping signals. Tucker decomposition has been used to analyze high-dimensional optogenetic data, allowing for the separation of evoked responses from spontaneous activity and aiding in the identification of circuit-level changes in neurodevelopmental and neurodegenerative disorders~\cite{erol2022tensors}. As multiphoton imaging technology continues to advance, tensor decomposition methods will play a crucial role in handling the increasing data complexity, improving image reconstruction, and extracting biologically relevant information from large-scale neural imaging experiments~\cite{uckermann2020label}.
% \paragraph{Tensor train}

% \paragraph{Tensor ring}

% \paragraph{Reconstruction}
% Image reconstruction in biomedical image analysis refers to the process of recovering under-sampled image to get into high-quality and interpretable images. It is critical for visualizing anatomical structures, assessing physiological functions, and aiding in diagnosis and treatment planning. Tensor decomposition methods have been applied in this task especially for MRI or dynamic MRI reconstruction, because MRI data usually involves data with three or more dimensions. Besides, such methods can be naturally combined with compressive sensing, a technique overwhelmingly used in image reconstruction. Treating the high dimensional MRI data as tensors rather than matrices help reduce redundancy and achieve better reconstruction quality. 

% \paragraph{Brain, knee, blood cell, eye}
% Brain image analysis is an important issue in biomedical imaging. The objectives of brain image analysis include structral and functional analysis, pathology detection and cognitive studies. The workflow involves brain image denoising, segmentation, feature extraction and so on. 


% \subsubsection{Challenges}
% % \AB{Phase transition/computational hardness discussions with plots following from the notebook shared with you}

% % Tensor based modeling starts with the idea of preserving the high dimensional structures (geometries) of biomedical data during the formulation of the problems. The computation of numerical results often converts the tensor based models into  equivalent matrix/vector based ones and it often requires formulation of large matrices coming from Kronecker operators. State of the art traditional computing hardware is not able to handle the vectorization of the entire biomedical data and one often needs to break the computation into small overlapping patches. However, not all biomedical application problems can be split into smaller patches. For instance, for super resolution of MRI images, the available low resolution image is a global Fourier transformation of the entire underlying high resolution image and thus can't be computed using patches which take only part of the Fourier information. 

% Tensor-based modeling in medical imaging seeks to preserve the intrinsic high-dimensional structure of biomedical images. However, computation of numerical results from tensors often requires converting these models into matrix or vector forms. This leads to the construction of large matrices involving Kronecker products, imposing significant computational burdens on current hardware. Consequently, computations are typically divided into small, overlapping patches, an approach that is not universally applicable, especially in medical imaging. For example, in MRI super-resolution, the low-resolution image is derived from a global Fourier transform of the entire high-resolution image, making patch-based processing infeasible. 

% Another general challenge in tensor decomposition is rank selection. For example, in Tucker decomposition, selecting the size of the core tensor, which is also the rank, is hard. It is significant since it determines the level of dimensionality reduction and the trade-off between performance and computational efficiency. If the rank is too low, capturing the latent structures in the data may be weakened, while a high rank increases computational complexity and memory usage significantly. As we discuss in Section~\ref{sec:hardness}, understanding the optimal rank of the tensor leads to a low MMSE, yielding efficient signal recovery. This phenomenon is true for all tensor decomposition techniques and their applications in medical imaging ranging from MRI to neural activity patterns and calcium imaging. This is demonstrated in Figure ~\ref{fig:mri_tensor_decomposition} using multi-slice MRI data with image size $512\times784\times912$ and voxel size $0.2\times0.18\times0.18$~mm, 
% %\os{(Shehab: People from quantum background may have no clue what it means. Are these voxels?)}, 
% where the dimensions represent the number of slices (512), and the in-plane resolution  ($784\times912$) of the imaging sequence. 
% %\os{(Shehab: For uninitiated readers like me, can we explain why the memory usage and execution time are bad? For example memory is cheap and 1.5 GB is nothing. It seems multi-volume MRI data could take 100+ GB when Voxel size: 0.5 $\times$ 0.5 $\times$ 0.5 mm, Matrix size: 512 $\times$ 512 $\times$ 512, Single 3D volume: $\sim$256 MB, Multi-subject dataset (400+ subjects, multiple sequences): 100+ GB. Maybe we can scale up the problem such that the memory requirement looks really bad, in hundreds of GBs? Same is true for execution time. Why 30 second is bad? )}.

% \begin{figure}[ht]
%     \centering
%     \subfloat{\includegraphics[width=0.45\textwidth]{figures/mri_memory_usage.png}}
%     \hspace{0.5cm} % Adds some spacing between images
%     \subfloat{\includegraphics[width=0.45\textwidth]{figures/mri_execution_time.png}}
%     \caption{Memory usage and execution time for tensor decomposition of 4D MRI data of size (384, 384, 28, 7). \os{(Can we increase the font size of the axis labels?)}}
%     \label{fig:mri_tensor_decomposition}
% \end{figure}
% Besides, real-world tensors in medical imaging, often contain noise or artifacts that reduce performance and increase computational requirements for robust solutions. Performance reduction when adding noise has been reported in different methods using CP decomposition \cite{zhao2023robust}, Tucker decomposition \cite{prevost2020hyperspectral}, t-SVD \cite{liu2025dynamic}. 

% Applications of TD methods in tissue and brain imaging also leads to significant challenges, particularly,  in the context of high-speed multiphoton microscopy and large-scale neural recordings. These challenges stem from the high-dimensional and multi-modal nature of the data acquired from multiphoton imaging techniques, such as second and third harmonic generation microscopy, two-photon and three-photon imaging, and functional calcium imaging. These datasets are inherently large, spanning spatial, temporal, and spectral domains, making tensor-based approaches computationally demanding~\cite{kolda2009tensor}. 
% %Existing tensor decomposition methods often struggle with the sheer volume of data, requiring advanced computational strategies to efficiently process and store high-dimensional image stacks without loss of information \os{(Shehab: Does this paragraph need some citations?)}.

% Another critical challenge is motion artifacts and signal contamination in live brain imaging. In awake behaving animals, head movements, heartbeat, and breathing introduce artifacts in calcium imaging and optogenetic recordings, complicating the extraction of meaningful neural signals. While tensor-based motion correction techniques have shown promise, they often require large computational resources and can introduce biases if the decomposition fails to properly separate artifacts from genuine neural activity~\cite{kara2024facilitating}. 
% % Additionally, rank selection and model generalization remain significant issues, particularly for dynamic imaging applications. Selecting the optimal rank for CP, Tucker, or t-SVD decomposition in functional brain imaging is challenging, as an improper choice can either oversimplify neural activity patterns or introduce excessive computational burden. The variability in neural signals across subjects and experimental conditions further complicates the application of a single decomposition model, necessitating adaptive approaches that can generalize across different imaging modalities and experimental paradigms. \os{(Shehab: Does this paragraph need some citations?)}
% Furthermore, biological noise and tissue heterogeneity present difficulties in tensor-based models applied to tissue imaging. In label-free multiphoton microscopy, where contrast arises from intrinsic tissue properties rather than fluorescent markers, non-uniform signal intensities, variations in optical scattering, and background noise can degrade the performance of decomposition techniques~\cite{borile2021label}. Robust pre-processing and artifact rejection methods are essential to improve the reliability of tensor-based segmentation, classification, and feature extraction in histopathological and live-tissue imaging. Finally, hardware limitations pose another barrier. While parallel computing and GPU acceleration have improved computational feasibility, real-time applications, such as closed-loop optogenetics or high-speed volumetric imaging, require further advancements in tensor decomposition algorithms to achieve near-instantaneous data processing and decision-making. Addressing these challenges will be crucial for fully harnessing tensor decomposition in large-scale tissue and brain imaging studies.

\subsection{Analyzing Multi-omics Data}\label{sec:td_multiomics}

%\subsubsection{Applications}
% \AB{General applications of TD in multi-omics data. Mostly focused on integrating multi-omics data such as MONTI~\cite{jung2021monti}, etc. or this recent paper~\cite{mitchel2024coordinated}.
% We might have to look into approximate TD methods such as probabilistic TD, etc. as well. 
% We should also review this paper~\cite{hore2016tensor}. We need to also look at mono-omic applications for TD such as captured clinical records~\cite{luo2017tensor}, gene expression~\cite{taguchi2022tensor}.} 

The application of tensor decomposition in multi-omics data analysis has significantly advanced in recent years, offering scalable and interpretable solutions for integrating complex biological datasets~\cite{hore2016tensor, amin2023tensor, lee2018gift, taguchi2022adapted, leistico2021epigenomic, taguchi2023tensor, wang2023probabilistic, tsuyuzaki2023sctensor, jung2021monti, taguchi2018tensor, fang2019tightly, taguchi2020tensor}. 
Compared to traditional approaches (e.g., clustering, PCA, correlation analysis), which typically analyze relationships between only one or two variables at a time, TD simultaneously captures complex interactions across multiple omics layers, preserving high-order biological structures~\cite{jung2021monti}. As multi-omics studies continue to generate increasingly high-dimensional and heterogenous data, TD techniques have evolved to address challenges related to dimensionality reduction, data sparsity, and the extraction of biologically meaningful patterns. 

One major development in tensor decomposition for multi-omics datasets has been the integration of probabilistic models to address sparsity and noise. Traditional TD methods assume homogenous data distributions, which limits their ability to model multiomics data with intrinsic variability. Probabilistic TD methods, like \texttt{SCOIT} (Single Cell Multiomics Data Integration with Tensor decomposition)~\cite{10.1093/nar/gkad570}, leverage statistical distributions such as Gaussian, Poisson, and negative binomial models to better capture variability across different omics layers. These models enhance downstream analyses such as cell clustering, gene expression integration, and regulatory network inference, making them particularly valuable for single-cell transcriptomics and epigenomics. To improve biological interpretability, non-negative TD methods enforce non-negativity constraints on the factorized components, ensuring that all contributions to latent factors remain positive. This is particularly useful in biomedical research, where feature contributions must be biologically meaningful. A prime example is \texttt{MONTI} (Multi-Omics Non-negative Tensor Decomposition for Integrative Analysis)~\cite{jung2021monti}, which selects key molecular features by identifying non-negative latent factors that drive disease progression. By preserving additive relationships in multi-omics features, \texttt{MONTI} has been effective in distinguishing molecular signatures associated with cancer subtypes, enhancing patient stratification.

Structural constraints can also be incorporated in TD to improve the separability of multi-omics data while preserving both local and global values. \texttt{BioSTD} (Strong Complementarity Tensor Decomposition Model)~\cite{gao2023biostd} uses t-SVD to factor multi-omics data from the Cancer Genome Atlas (TCGA) and enforces a strong complementarity constraint to maintain high-dimensional spatial relationships across omics layers. This improves coordination between different types of omics data and ensures that redundant features do not obscure biologically significant patterns. Such an approach is particularly beneficial when analyzing high-dimensional datasets in precision oncology, where feature redundancy can compromise the interpretability of predictive biomarkers. Multi-omics integration often faces challenges due to the unequal feature dimensions across omics layers, where transcriptomics datasets typically contain tens of thousands of features, while proteomics and metabolomics datasets are comparatively sparse. \texttt{GSTRPCA} (Irregular Tensor Singular Value Decomposition Model)~\cite{Cui2024GSTRPCA} overcomes this limitation by maintaining the original data structure while incorporating low-rank and sparsity constraints using a tensor-PCA based model. This enables effective feature selection and clustering without requiring aggressive pre-processing or feature imputation, thereby preserving biologically relevant information.

Handling missing values in multi-omics datasets remains a fundamental challenge, particularly in epigenomics, where experimental limitations often result in incomplete data across different cell types. \texttt{PREDICTD} (Parallel Epigenomics Data Imputation with Cloud-Based Tensor Decomposition)
~\cite{Durham2018PREDICTD} addresses this issue by constructing a tensor where dimensions represent cell types, genomic regions, and epigenomic marks, allowing the decomposition process to identify latent factors that capture underlying regulatory patterns. By applying CP decomposition, the method imputes missing values using latent factor reconstruction. This cloud-based framework has been validated in large-scale projects such as ENCODE and the Roadmap Epigenomics Project, highlighting its utility in functional genomics. Although tensor decomposition effectively reduces dimensionality, one of its key challenges is ensuring biological interpretability. Guided and Interpretable Factorization for Tensors~\cite{lee2018gift} (\texttt{GIFT}) enhances interpretability by incorporating functional gene set information as a regularization term within the decomposition process. It adapts Tucker decomposition to factorize with prior biological knowledge, \texttt{GIFT} ensures that extracted components correspond to meaningful gene regulatory patterns. Applied to datasets such as the PanCan12 dataset from TCGA, \texttt{GIFT} has successfully identified relationships between gene expression, DNA methylation, and copy number variations across cancer subtypes, outperforming traditional tensor decomposition approaches in terms of accuracy and scalability. 

In terms of scalability, TD has evolved to accommodate large-scale genomic data through parallelized optimization techniques. An example is \texttt{SNeCT} (Scalable Network Constrained Tucker Decomposition)~\cite{Choi2020SNeCT} which extends traditional Tucker decomposition by incorporating prior biological networks into the decomposition process. This method constructs a tensor from multi-platform genomic data while integrating pathway-based constraints to improve feature selection and classification performance. By leveraging parallel stochastic gradient descent, \texttt{SNeCT} efficiently processes large-scale datasets such as TCGA’s PanCan12 dataset. Its ability to integrate gene association networks ensures that the resulting factor matrices maintain biological relevance, making it particularly effective for cancer subtype classification and patient stratification. Standard TD approaches often treat time as a discrete variable, limiting their ability to model continuous temporal changes in biological processes. Temporal Tensor Decomposition~\cite{Shi2024TEMPTED} (\texttt{TEMPTED}) overcomes this limitation by integrating time as a continuous variable within the decomposition framework. This method enables the characterization of dynamic transcriptional and epigenomic changes across different time points, making it particularly useful for analyzing developmental processes and disease progression. \texttt{TEMPTED} also accounts for non-uniform temporal sampling and missing data, providing a more comprehensive approach to longitudinal multi-omics analysis. More recently, Mitchel \textit{et al.} have introduced a computational approach called single-cell interpretable tensor decomposition (\texttt{scITD}), that utilizes tensor decomposition to analyze single-cell gene expression data~\cite{mitchel2024coordinated}. This approach enables the identification of coordinated transcriptional variations across multiple cell types, facilitating the discovery of common patterns among individuals, in turn facilitating patient stratification.

As tensor decomposition methodologies for multi-omics datasets continue to evolve, future advancements will likely focus on improving real-time data analysis capabilities, refining computational efficiency, and expanding the applicability of TD in emerging fields such as spatial transcriptomics and single-cell multi-omics. The continued refinement of these methods will be critical for advancing precision medicine and uncovering novel insights into complex biological systems.


% Precision medicine has been revolutionized by genetic analysis, enabling clinicians to identify individuals at high risk of disease and implement targeted, genetically informed interventions. However, multi-omics data, encompassing genomics, phenomics, transcriptomics, proteomics, metabolomics, etc. present unique challenges: they are inherently noisy, sparse, high-dimensional, and plagued by batch effects. Although there are emerging methods for integrating these diverse modalities, there is a lack of a standardized, user-friendly approach for multi-omics integration~\cite{tarazona2021undisclosed}. Moreover, current computational models still struggle to accurately predict phenotype outcomes from genotype data due to the complex, multi-layered interactions among genetic variants, epigenetic modifications, and environmental influences~\cite{ritchie2015methods}. Tensor decompositions is particularly well-suited for analyzing such data because it captures the multidimensional relationships within these datasets effectively. A tensor is a multi-dimensional array that can naturally represent relationships across different omics layers. Instead of analyzing two-dimensional matrices (e.g., samples × genes), tensors allow us to model higher-order interactions~\cite{chang2021gene}. In comparison to traditional approaches (e.g., clustering, PCA, correlation analysis), which typically analyze relationships between only one or two variables at a time, TD, however, simultaneously captures complex interactions across multiple omics layers, preserving high-order biological structures~\cite{jung2021monti}.  Hence, TD methods have been widely applied in both single-omic analysis such as in genomics~\cite{hore2016tensor, amin2023tensor}, transcriptomics~\cite{lee2018gift, taguchi2022adapted}, epigenomics~\cite{leistico2021epigenomic}, single-cell analysis~\cite{taguchi2023tensor, wang2023probabilistic, tsuyuzaki2023sctensor}, as well as in multi-omics studies~\cite{jung2021monti, taguchi2018tensor, fang2019tightly, taguchi2020tensor}. 


% The triumph of utilizing genetics in precision medicine is the ability to flag individuals at very high risk of disease, with the ability to enact specific gene-informed medical management. Despite their promise in identifying biologically relevant outputs, omics data are noisy, sparse (i.e., contain many zeros), high-dimensional, and contain batch effects. Particularly, there are no standard or user-friendly tools that exist to integrate or link data from all these modalities. The area under the curve (AUC - a measure of the predictive power) for many genetic risk scores has plateaued below 0.7 \os{(Shehab: Citations?)}, demonstrating the limits in linking genomes and outcomes based on classical computational methods. Multi-omics datasets, which integrate data from various omics layers such as genomics, transcriptomics, proteomics, and metabolomics, are inherently high-dimensional and complex. Tensor decomposition is particularly well-suited for analyzing such data because it captures the multidimensional relationships within these datasets effectively. A tensor is a multi-dimensional array that can naturally represent relationships across different omics layers. Instead of analyzing two-dimensional matrices (e.g., samples × genes), tensors allow us to model higher-order interactions. In comparison to traditional approaches (e.g., clustering, PCA, correlation analysis), which typically analyze relationships between only one or two variables at a time, tensor decomposition, however, simultaneously captures complex interactions across multiple omics layers, preserving high-order biological structures. 


% Tensor decomposition has been applied in multi-omics data processing, including integrative analyzes, disease or phenotype subtyping, feature extraction, biomarker discovery, gene regulation studies, as well as temporal or longitudinal data analyses \cite{luo2017tensor}. Hore et al. introduced a Bayesian method designed to analyze multi-tissue gene expression data, with a focus on uncovering gene networks associated with genetic variation by decomposing the three-dimensional matrix (tensor) of gene expression measurements into latent components \cite{hore2016tensor}. The study uncovered several gene networks related to genetic variation, providing insights into the genetic architecture of gene expression across multiple tissues, thus highlighting the potential of tensor decomposition methods in elucidating complex gene regulatory mechanisms. Jung et al. developed MONTI (Multi-Omics Non-negative Tensor Decomposition for Integrative Analysis), a multi-omics non-negative tensor decomposition framework tailored for integrative analysis of multi-omics data \cite{jung2021monti}. Through integrating multi-omics data in a gene-centric manner, MONTI enabled the detection of cancer subtype-specific features and other clinical features, which in turn facilitates the identification of biologically meaningful associations. In another study, Wang et al. introduced SCOIT (Single Cell Multiomics Data Integration with Tensor decomposition), a probabilistic tensor decomposition framework designed to extract embeddings from single cell multiomics data \cite{10.1093/nar/gkad570}. The framework was applied to eight single-cell multi-omics datasets from various sequencing protocols, covering DNA methylation data, RNA expression data, proteomics data, and chromatin accessibility data. SCOIT achieved superior clustering accuracy compared to nine state-of-the-art methods on heterogeneous datasets. More recently, Mitchel et al. have introduced a computational approach called single-cell interpretable tensor decomposition (scITD), a computational method that utilizes tensor decomposition to analyze single-cell gene expression data \cite{mitchel2024coordinated}. This approach enables the identification of coordinated transcriptional variations across multiple cell types, facilitating the discovery of common patterns among individuals, in turn facilitating patient stratification.


% \subsubsection{Challenges}
% % \AB{Phase transition/computational hardness discussions with plots following from the notebook shared with you}

% Multi-omics data analysis presents several challenges, particularly in handling, integration, and interpretation. One of the major difficulties lies in the computational complexities associated with large-scale multi-omics datasets. These datasets are often high-dimensional high dimensions due to the large number of features across multiple omics layers. Tensor decomposition methods, widely used for multi-omics integration, often have difficulties with scalability due to increasing computational demands. Therefore, efficient algorithms and scalable frameworks are required to handle these datasets effectively, as tensor decomposition can be both computationally intensive and time-consuming. 

% To illustrate these computational challenges, we performed tensor decomposition on a multi-omics dataset with a tensor size of (264, 10772, 31), where the dimensions represent the number of samples (264), genomic features (10,772), and metabolomic features (31). Figure~\ref{fig:tensor_decomposition} presents the results of this analysis. The first plot shows execution time (in seconds) as the decomposition rank varies from 2 to 30, revealing an initial increase followed by a decline after rank 14. The second plot illustrates memory usage (in MB), which steadily rises with increasing decomposition rank. These results highlight a critical trade-off between rank selection and computational efficiency, emphasizing the need for optimized feature selection strategies to balance computational feasibility and biological relevance when applying tensor decomposition to multi-omics data. Another key challenge is the heterogeneity of multi-omics data. These datasets comprise diverse data types, including continuous, binary, and categorical variables, complicating integration efforts. This heterogeneity complicates the integration process, as traditional tensor decomposition methods such as Tucker and CANDECOMP / PARAFAC (CP) may not adequately model complex interactions and different data types \cite{xu2015bayesian}. Beyond computational and integration challenges, interpreting the latent factors extracted from tensor decomposition in a biologically relevant context can be challenging. Moreover, multi-omics data are often noisy and contain outliers. This can distort the results of the tensor decomposition. However, there are new methods such as SCOIT \cite{10.1093/nar/gkad570} which attempt to address this challenge by incorporating various distributions to model noise and sparsity in the data. 

% Finally, multi-omics data often contain missing values or sparse entries, which can significantly impact the performance of tensor decomposition methods. Several traditional approaches require complete data or extensive preprocessing to handle missing data, which can be time-consuming and may introduce biases \cite{xu2015bayesian, taguchi2021tensor}. While tensor decomposition provides a powerful framework for multi-omics integration,  overcoming computational constraints, handling data heterogeneity, improving interpretability, mitigating noise, and addressing missing data challenges remain critical areas for future research and methodological advancements.

% \begin{figure}[ht]
%     \centering
%     \subfloat{\includegraphics[width=0.45\textwidth]{figures/memory_usage.png}}
%     \hspace{0.5cm} % Adds some spacing between images
%     \subfloat{\includegraphics[width=0.45\textwidth]{figures/execution_time.png}}
%     \caption{Memory usage and execution time for tensor decomposition for random dataset. \os{(Shehab: Can we add more details? How the dataset was generated?)}}
%     \label{fig:tensor_decomposition}
% \end{figure}




\begin{figure*}[t]
\begin{minipage}{0.98\textwidth}
    \begin{subfigure}[b]{0.3\linewidth}
        \centering
        \includegraphics[width=0.95\linewidth]{figures/files/vanilla_rope.pdf}
        \caption{3D visualization for Vanilla RoPE.}
        \label{fig:vanilla_rope}
    \end{subfigure}
    \hfill
    \begin{subfigure}[b]{0.3\linewidth}
        \centering
        \includegraphics[width=0.95\linewidth]{figures/files/m_rope.pdf}
        \caption{3D visualization for M-RoPE.}
        \label{fig:m_rope}
    \end{subfigure}
    \hfill
    \begin{subfigure}[b]{0.3\linewidth}
        \centering
        \includegraphics[width=0.95\linewidth]{figures/files/m_modify_rope.pdf}
        \caption{3D visualization for \methodname.}
        \label{fig:video_rope}
    \end{subfigure}
    \hfill
    \vspace{-6pt}
    \caption{\footnotesize The 3D visualization for different position embedding. \textbf{(a)} The vanilla 1D RoPE~\cite{su2024roformer} does not incorporate spatial modeling.
    \textbf{(b)} M-RoPE~\cite{wang2024qwen2}, while have the 3D structure, introduces a discrepancy in index growth for visual tokens across frames, with some indices remaining constant.
    \textbf{(c)} In contrast, our \methodname achieves the desired balance, maintaining the consistent index growth pattern of vanilla RoPE while simultaneously incorporating spatial modeling. 
    }
    %3D visualization of different position embeddings. \textbf{(a)} The vanilla 1D RoPE~\cite{su2024roformer} lacks spatial modeling. \textbf{(b)} M-RoPE~\cite{wang2024qwen2}, while have the 3D structure, introduces a discrepancy in index growth for visual tokens across frames, with some indices remaining constant. \textbf{(c)} Our \methodname balances the index growth of vanilla RoPE while incorporating spatial modeling. For more details on the index, see Appendix \ref{app:supp_explain_modules}.
    \vspace{-12pt}
    \label{fig:spatial}
\end{minipage}
\end{figure*} 