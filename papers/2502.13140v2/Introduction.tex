\section{Introduction} 

Biomedical data encompasses a diverse set of multi-modal and multi-omics data from healthcare and life sciences providing critical insights into the complexity of disease etiology and biological functions. A comprehensive understanding of biomedical data is required to advance precision medicine and treatment strategies for complex diseases. Innovations in technologies like genomics, proteomics, medical imaging, such as magnetic resonance imaging (MRI), computed tomography (CT), etc. have resulted in high-quality data that enable personalized clinical decision making. This opportunity also presents us with significant challenges in data management and analysis~\cite{lambin2017radiomics}, such as representing the different modalities of biomedical data and finding higher-order relationships between them. With the increasing size and complexity of biomedical data, tensors  are emerging as a powerful method for the integration and analysis of data ~\cite{ho2014marble,luo2017tensor,wang2021variational, jung2021monti, taguchi2022novel, gao2023biostd, liu2023multiomics, zhou2013tensor}.

Tensors are multidimensional arrays rendered as generalizations of vector spaces~\cite{kolda2009tensor}. A vector is a first-order tensor, a matrix is a second-order tensor, and tensors of order three or higher are called higher-order tensors. Tensor decompositions, although originally proposed about a century ago~\cite{hitchcock1927expression}, have found renewed interest in several applications in signal processing, machine learning~\cite{sidiropoulos2017tensor}, data mining~\cite{papalexakis2016tensors}, computer vision~\cite{shashua2005non}, neuroscience~\cite{beckmann2005tensorial}, etc. in the past two decades, with the ability to represent complex and multimodal data. Tensors have the ability to represent each biomedical modality in a different dimension. Hence, they can be used to integrate multi-modal biomedical data spanning multi-omics variables. Tensor factorization or decomposition allows us to obtain a latent structure from a higher-dimensional space spanned by all the multi-modal variables and identify interactions between them~\cite{kolda2009tensor, luo2017tensor}. Tensor decomposition (TD) has widespread application in biomedical data analysis ranging from analyzing MRI images and electroencephalography (EEG) signals~\cite{sedighin2024tensor}, finding genotype-phenotype relationships~\cite{kessler2014learning}, quantifying expression quantitative trait loci (eQTL)~\cite{hore2016tensor}, integrating multi-omics data~\cite{jung2021monti}, to spatial transcriptomics~\cite{broadbent2024deciphering}. There have been attempts to review the applications of TD in biomedical data but they were focused on specific applications such as imaging analysis~\cite{sedighin2024tensor}, precision medicine~\cite{luo2017tensor} or, on specific diseases such as cancer~\cite{movahed2024tensor}, and often limited in scope. 

This review examines recent advances in tensor decomposition for biomedical data analysis. We review its application across diverse biomedical domains over the past decade, highlighting state-of-the-art practices, the inherent advantages of these methods, and the scalability challenges that remain. Moreover, we discuss a unique perspective of adapting emerging technologies such as quantum computing (QC) and strategies for developing quantum algorithms for TD to ameliorate the challenges arising in classical TD applications. QC has the potential to accelerate computing in biomedicine~\cite{durant2024primer, flother2024quantum, emani2021quantum, nalkecz2024quantum}, with possible applications in single-cell analysis~\cite{basu2023towards,utro2024perspective}, predictive modeling in multi-omics analysis~\cite{bose2024quantum}, clinical trials~\cite{doga2024can}, breast cancer subtyping using histopathological images~\cite{ray2024hybrid}, protein structure prediction~\cite{doga2024perspective, raubenolt2023quantum}, among others. We explore development of a quantum tensor decomposition (QTD) algorithm and provide a framework of how it can be implemented in present-day pre-fault tolerant quantum devices (PFTQD) to solve problems in biomedicine, as well as discussing future implementations in fault-tolerant quantum devices.

This review is structured as follows. We introduce different types of TD approaches in Section~\ref{sec:td_def} and discuss methods to evaluate computational hardness in Section~\ref{sec:hardness}. Next, we continue to review the current literature for TD applications in biomedicine using an unsupervised technique in Section~\ref{sec:td_biomed} and subsequently discuss the main findings on imaging (Section~\ref{sec:td_imaging}), multi-omics (Section~\ref{sec:td_multiomics}), and spatial transcriptomics (Section~\ref{sec:td_spatial}). We also analyze the hardness of tensor decomposition in real-world MRI and multi-omics data, highlighting the factors impacting its implementation in classical systems in Section~\ref{sec:challenges}. Thereafter, we review the literature on quantum algorithms for TD (Section~\ref{sec:td_quantum}) and discuss particular use cases where quantum tensor decomposition can be applied in biomedical data (Section~\ref{sec:td_usecase}). Finally, we summarize our review and future perspective in Section~\ref{sec:conclusion}.  
