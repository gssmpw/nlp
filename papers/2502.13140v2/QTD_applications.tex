\section{Quantum Tensor Decomposition for Biomedical Data}~\label{sec:td_usecase} 
% \AB{In this section, we will focus on how quantum tensor decomposition can be applied to specific problems emanating from the three focus areas we have discussed. We will discuss the size of the problems, number of qubits required to implement the algorithm, and how to achieve quantum utility. 
% Discussion on quantum advantage?! - phase estimation and use cases. 
% }
\begin{figure}
    \centering
    \includegraphics[width=\linewidth]{figures/td_use_case_reviewpaper.png}
    \caption{Proposed framework of quantum tensor decomposition methods for analyzing biomedical data for various downstream tasks.}
    \label{fig:qtd_framework}
\end{figure}
The applications of quantum tensor decomposition for analyzing biomedical data must integrate the recent advances in QTD algorithms~\cite{hastings2020classical,zhou2024statistical} with existing approaches in classical algorithms in spiked tensor decomposition or CP decomposition. The QTD algorithm should be compared against these classical approaches for quantitative and qualitative benchmarking and assessing the problems that are more suitable for the quantum algorithm. We propose a framework to implement QTD in biomedical data analysis (Figure~\ref{fig:qtd_framework}) and perform a host of downstream tasks targeted towards the nature and modality of the data.  

The framework ingests multi-modal data such as imaging or multi-omics including spatial transcriptomics. It tensorizes them into higher-order tensors (with orders greater than two) to produce the tensor $T_0$ in the form of a spike tensor problem (Equation~\ref{eq:st}). Then we take a multi-pronged approach to find the underlying signal $v_{sig}$. We form the Hamiltonian, $H(T_0)$ from $T_0$ as per Equation~\ref{eq:hamiltonian} and perform a spectral decomposition, as shown in previous work~\cite{hastings2020classical}, which proposed both classical and quantum algorithms for spectral decomposition. We decompose $H(T_0)$ by solving a QPE problem~\cite{kitaev1995quantum} in FTQD, to efficiently extract eigenvalues and corresponding eigenvectors. The QPE circuit is shown in the inset of Figure~\ref{fig:qtd_framework}'s quantum approach. Once we obtain the leading eigenvalue with its corresponding eigenvector, we can recover $\bar{v}_{sig}$ up to some accuracy. Simultaneously, we can also apply a CP decomposition to the original $T_0$ and obtain the $v_{sig}$. Then, we need to compare the correlation between $\bar{v}_{sig}$ and the $v_{sig}$, to assess the quality of the signal obtained from the quantum algorithm. 

\begin{figure}[H]
    \centering
    \includegraphics[width=0.9\textwidth]{figures/qubit_scaling.png}
    \caption{Figure shows the qubit scaling in the spiked tensor Hamiltonian as the tensor order $p$ and dimension $N$ (in log scale) increases. The horizontal planes represent the cut-off points for quantum computers, from IBM Quantum's roadmap, with corresponding number of qubits. In particular, Heron and Flamingo QPUs contain 133 and 156 qubits. With the modular structures where multiple QPUs are coupled, the qubit numbers increase to 399 and 1092 respectively. It is important to note that, our qubit counts are problem qubit counts while the qubit numbers on IBM Quantum roadmap are physical qubits. Depending on the type of algorithm, they may not map to each other on a 1-to-1 basis. Further information can be found at \href{https://www.ibm.com/quantum/technology}{IBM Quantum Technology Roadmap}~\cite{gambetta2020ibm}.}
    \label{fig:qubits}
\end{figure}

Furthermore, to understand the complexity of the problem and how it maps to a quantum computer, we also analyze the number of qubits required to map a given tensor to a qubit Hamiltonian (see Figure \ref{fig:qubits}). Note that the Hamiltonian from Equation \ref{eq:hamiltonian} is formulated for a bosonic system, i.e. an $N$-level qudit system. In Figure \ref{fig:qubits}, we show the final scaling of how this qudit system maps to qubits, since the quantum computers we consider operate natively with qubits. We are aware that there many other parameters to consider in this case, such as the circuit depth (in particular 2-qubit gate depth), overhead from error mitigation/suppression in utility-scale quantum computer or the overhead of error correction for QPE, the trade-off between accuracy and number of repetitions in QPE, the strength of the signal-to-noise ratio in the spiked tensor and the number of bosonic modes required to successfully recover the signal vector etc. However, these require a deeper analysis and are beyond the scope of this paper.  

With this recovered signal, we can perform several downstream tasks in biomedical data analysis. For example, QTD can perform tasks such as denoising, reconstruction, and segmentation of biomedical images. On the other hand, QTD can produce latent representations by integrating multi-modal data in lower dimensions which can be used as input to deep neural networks for downstream analysis. Similarly, these lower-dimensional embeddings can be directly used to discover biomarkers, prognosis, and perform classification or regression tasks with multi-omics data.

