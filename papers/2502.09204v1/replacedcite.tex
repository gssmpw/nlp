\section{Related work and conclusion}
\label{sec:related-conclusion}

Existing systems for legal analysis employ various techniques, including text classification, machine learning, and rule-based reasoning____. However, these approaches often have limitations in handling nuanced legal reasoning. 

Text classification systems such as Legal-Document-Classifier ____ and LexNLP ____  can categorize legal documents based on keywords and named entities but lack the ability to perform comprehensive legal reasoning. 

Machine learning-based systems like Kira ____ and LawGeex ____ can extract key terms and identify potential issues in contracts, and models like Lex Machina ____ can predict legal outcomes with some accuracy. However, these systems often operate as black boxes, raising concerns about transparency and fairness. They may also be limited by data quality and biases.

Rule-based legal reasoning systems like PROLEG____ offer support for judges in civil litigation by incorporating predefined rules and handling uncertainty. However, their complexity can pose challenges for users.

In contrast, we successfully developed a system for analyzing landlord-tenant disputes in New York State by leveraging Large Language Models (LLMs) for information extraction and Prolog for legal reasoning. Achieving high accuracy and efficiency, the system offers several advantages over existing LLM-based legal analysis systems.

By separating information extraction from legal reasoning, the system achieves greater transparency and control over the legal logic applied to each case. Additionally, the use of Prolog enables the implementation of defeasible logic ____, allowing the system to handle nuanced legal reasoning, such as resolving conflicting legal principles and dealing with uncertain or incomplete information____.


In conclusion, this work demonstrates the potential of combining Large Language Models (LLMs) and logic-based reasoning to create innovative tools for legal analysis. By addressing the limitations of existing approaches, LogicLease paves the way for more sophisticated and transparent systems in the field of legal technology.

Future work includes expanding the system's capabilities by employing techniques for caching frequently accessed legal information. Additionally, improving the testing process and adding more test cases will ensure better coverage and reliability of the system. Open-sourcing the code could encourage further development and broader adoption within the legal domain.

\nocite{*}
\bibliographystyle{eptcs}
\bibliography{generic}