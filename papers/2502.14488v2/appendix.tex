\section{Further Details on Tested Indexes and the $\tau$ Values}\label{sec:app-tau}

Here we explain in more detail how each of the indexes was used.

\parag{The \method{libsais} Suffix Array}
For the plain input, including for DNA, we use a $\tau=8$ byte encoding.
For the sketched text $S$, this depends on $c$. When $\log_2(c)\leq 8$, we use one-byte
encoding and call the \texttt{libsais} function. When $\log_2(c)\leq 16$, we use
two-byte encoding and call \texttt{libsais16}. For larger $c$, we first remap
all IDs to values starting at $0$, as recommended by the library authors, and then call the function \texttt{libsais\_int} on 32-bit
input values.

\parag{The SDSL-lite FM-index}
For the plain text, we use the Huffman-shaped wavelet tree with default parameters, i.e., the class
\verb|csa_wt<wt_huff<rrr_vector<63>>,32,64>|.
For the {\uindex} counterpart, we instead use
\verb|csa_wt<wt_int<rrr_vector<63>>,32,64>|, a wavelet-tree over a variable-width integer alphabet.
Both these indexes use a sampling factor of
$32$ to sample the suffix array. Also, for both versions we remap
$k$-mers to ids starting at $1$ instead of $0$, since SDSL does not support input values of $0$.
For this reason, we do not have a \texttt{-H} no-remap variant of the SDSL {\fmindex}.

\parag{The AWRY FM-index}
The AWRY {\fmindex} only supports DNA and protein alphabets. For consistency, we
only use the DNA version. This means that we consistently use $\tau=2$. Thus,
IDs with $\log_2(c)$ bits are encoded into $\lceil\log_2(c)\rceil$ DNA bases,
which are passed to AWRY as 8-bit \texttt{ACGT} characters. Similarly, plain
text protein and English input is encoded into 4 underlying characters.
This index also uses a suffix array sampling factor of $32$.

\parag{The \sindex}
The \sindex consists of an array of 32-bit integers indicating text positions.
It is constructed using the Rust standard library
\verb|Vec<u32>::sort_unstable_by_key| function that compares text indices
by comparing the corresponding suffixes.

% Put title on the same page as the figures.
\clearpage
\section{Experimental Results on Proteins and English Datasets}\label{sec:app}

\begin{figure*}[h]
\centering
\includegraphics[width=\textwidth]{plots/plot-proteins.pdf}%
\caption{Results on $200$ MiB of protein sequences.
Refer to the caption of \cref{fig:plot-v2}.
% \giulio{Is it funny that in the \cref{sec:evaluation} we refer to the Appendix and here we point back again to the caption of \cref{fig:plot-v2} from \cref{sec:evaluation}? :D}
}
\label{fig:plot-proteins}
\end{figure*}

\begin{figure*}[h]
\centering
\includegraphics[width=\textwidth]{plots/plot-english.pdf}%
\caption{Results on $200$ MiB of English text.
Refer to the caption of \cref{fig:plot-v2}.}
\label{fig:plot-english}
\vspace{-14em} % Prevent figure from wrapping to the next page.
\end{figure*}
