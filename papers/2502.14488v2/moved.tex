\section{Moved Text}

\parag{Construction}
We initialize two integers $c := 0$ and $A:=0$.
We initialize a hash table $H$ with integer keys and integer values.
We initialize two vectors $R$ and $V$ of integer elements.
In addition, let $\tau\in[2, n]$ be a parameter that is set by the user.
We compute the set $\Minimizers(T)$ of the positions of the minimizers of $T$
by scanning $T$ from left to right in $\cO(n)$ time~\cite{DBLP:journals/tkde/LoukidesPS23} using a rolling hash function $\phi : \Sigma^k \to \mathbb{N}$, e.g., Karp-Rabin (KR) fingerprints~\cite{DBLP:journals/ibmrd/KarpR87}, to determine the order between {\kmer}s.

Let us now provide an intuition about the role of each variable during the construction.

\begin{enumerate}
    \item $c$ is an integer variable that counts the number of distinct $k$-mers selected as minimizers in $T$.
    \item $A$ is an integer variable that counts the number of minimizers in $T$.
    \item $H$ is a hash table that stores key-value pairs: a key is the KR
      fingerprint of a $k$-mer selected as a minimizer; and its value is a
      unique integer from $[0,c)$ viewed as the id of the minimizer.
        \ragnar{There's no need to index kmers by their hash -- we can index
          them by their kmer value directly. I maintain the point that the hash
          function used for minimizers should be independent from the one used
          for indexing purposes, especially since for minimizers we can get away
        with using a relatively weak random hash, while that is not good for indexing.}
        \ragnar{I only realized later that we need to store hashes to save space
          when $k$ is large. We should explicitly remark this.}
        \giulio{Exactly. I was about to make the same remark but $k$ can be very large.}
    \item $R$ is an integer vector that stores the sequence of id's of the selected minimizers from left to right; i.e., $R[A]=c$ denotes that the $A$th selected minimizer has id $c$.
    \item $V$ is an integer vector that stores the positions of the selected minimizers from left to right; i.e., $V[A]=i$ denotes that the $A$th selected minimizer occurs at position $i$ of $T$.
\end{enumerate}

\rob{Are all elements of this index essential? If we have 5, and the original text, then why do we need 4? If we have th positions of the minimizers, than we can recall each of them directly by referring to them in the original text.}

The construction processes the elements of $\Minimizers(T)$ in increasing order and uses the above variables to construct the \emph{sketch} of $T$, that is,
a string $S$ over alphabet $[0,\tau)$. For each $i \in \Minimizers(T)$,
we lookup $\phi(T[i\dd i+k-1])$ in $H$. If it is not there, we add $H[\phi(T[i\dd i+k-1])] = c$, set $R[A]=c$, and increment $c$ by one. If it is there, then we set $R[A] = H[\phi(T[i\dd i+k-1])]$.
We also set $V[A] = i$ and increase $A$ by one.
We repeat this process for the next position of a minimizer $i'> i$ found, until we reach the end of $T$, when we will have found all minimizers and fully constructed $H$, $R$, and $V$. At this point, we have $A=|\Minimizers(T)|\geq c$.
We initialize an empty vector $S$ of $\lceil \log \tau \rceil$-bit integers.
We read $R$ from left to right and append $R[i]$ to $S$ by expressing it as
$\lceil\log c/\log \tau\rceil$ integers \ragnar{the denominator needs a ceil
  here and elsewhere}.
Note the purpose of the parameter $\tau$: it lets the user control the size of the alphabet of $S$ as something that lies in $[2,n]$.
This is a useful feature because some compressed full-text indexes, like the FM-index~\cite{10.1145/1082036.1082039} and the $r$-index~\cite{10.1145/3375890}, take advantage of the repetitiveness of the text to improve its compression.

Finally we construct the $\INDEX$ of $S$, which is over alphabet $[0,\tau)$ and is of length $A\lceil\log c/\log \tau\rceil\leq A\lceil\log A/\log \tau\rceil$.\footnote{In the unlikely event of $A\lceil\log c/\log \tau\rceil > n$, we can either increase $\tau$ to have $A\lceil\log c/\log \tau\rceil \leq n$ or simply set $S:=T$.} We also delete $R$.

To conclude, our framework assumes read-only random access to $T$, takes parameters $\ell,k$ and $\tau$ as input, and constructs an index on top of $T$ that consists only of $H$, $V$, and the $\INDEX$ of $S$ over $\tau$.

\giulio{I'd suggest to separate the mere construction part (i.e., how one obtains the list of minimizers) from the encoding part, i.e., how we encode $S$ over the alphabet $[\tau]$. It is much cleaner. Currently, we anticipate what $\tau$ is without using it until the end.}