\begin{abstract}

\parag{Motivation}
Text indexing is a fundamental and well-studied problem.
Classic solutions to this problem either replace the original text with a compressed representation, e.g., the FM-index and its variants, or keep it uncompressed but attach some redundancy --- an index --- to accelerate matching, e.g., the suffix array. The former solutions thus retain excellent compressed space, but are practically slow to construct and query. The latter approaches, instead, sacrifice space efficiency but are typically faster; for example, the suffix array takes much more space than the text itself for commonly used alphabets, like ASCII or DNA, but it is fast to construct and query.

\parag{Methods}
In this paper, we show that efficient text indexing
can be achieved using just a small extra space on top of the original text, provided that the query patterns are sufficiently long.
More specifically, we develop a new indexing paradigm in which a \emph{sketch} of a query pattern is first matched against a \emph{sketch} of the text. Once \emph{candidate} matches are retrieved, they are verified using the original text.
This paradigm is thus \textit{universal} in the sense that it allows us to use
\textit{any} solution to index the sketched text, like a suffix array, FM-index, or r-index.

\parag{Results}
We explore both the theory and the practice of this universal framework. With an extensive experimental analysis, we show that, surprisingly, universal indexes can be constructed much faster than their unsketched counterparts and take a fraction of the space, as a direct consequence of (i) having a lower bound on the length of patterns and (ii) working in sketch space. 
Furthermore, these data structures have the potential of retaining or even improving query time, because matching against the sketched text is faster and verifying candidates can be theoretically done in constant time per occurrence (or, in practice, by short and cache-friendly scans of the text).
%\giulio{To be enriched with more precise bounds and experimental results.}
%SPP: Giulio, I think the abstract is already too long, and many explanations need to be added to include bounds.
%\ragnar{I think the 'retaining' is mostly OK, but could be considered at odds with our results.}

Finally, we discuss some important applications of this novel indexing paradigm to computational biology. We hypothesize that such indexes will be particularly effective when the queries are sufficiently long, and so demonstrate applications in long-read mapping.
%\giulio{say something more concrete when we decide what to implement as applications.}
%\giulio{We only ``discuss'' potential applications, but do not include a comparison with bioinformatics software tools for, e.g., read mapping, etc., right?}
%\ragnar{Yeah; I'd consider that out of scope for now -- at least if we're still aiming for SEA.}

\end{abstract}
