\section{Conclusions and Future Work}\label{sec:conclusions}

In this work, we introduced the {\uindex} --- a universal framework to enhance the performance of any off-the-shelf text index, provided that the patterns to match are sufficiently long.
This is achieved, in short, by sketching the text and using any desired index
for the sketched text.
Intuitively, this saves resources at building time and considerably reduces the
final index size, simply
because the sketched text is shorter than the input text.
Our experiments indeed confirm that the \uindex 
has excellent performance when used in combination with the suffix array, as
it significantly improves index size and construction while not slowing down queries too much. When paired with the FM-index, the savings are more modest but still significant.

The \textit{sparse} suffix array index by Grabowski and Raniszewski~\cite{DBLP:journals/spe/GrabowskiR17} remains a great solution in
this regime, having smaller size and significantly faster queries than the
\uindex around suffix arrays (which have, however, faster construction time).
For example, albeit somewhat larger than the SDSL FM-index, the sparse suffix array is over $100\times$ faster to query.
However, the benefit of the {\uindex} framework lies in its universality and usability.
We remark that the primary objective of this work is to highlight
these two important properties.

We anticipate that the \uindex may be especially useful around the r-index~\cite{10.1145/3375890} when used on highly repetitive data,
but we leave this as future work.
The sparse suffix array will \emph{by design} be unable to take advantage of the underlying repetitiveness. Hence, other than universality, another important virtue of our framework is that it preserves string similarity: for any two highly similar texts $T_1$ and $T_2$, it will be the case that $S_1=\Sketch(T_1)$ and $S_2=\Sketch(T_2)$ are also highly similar (assuming the sketches are based on minimizers).
% As the r-index is designed to exploit the repetitiveness of the input text, we believe the \uindex around the r-index should offer drastic savings in the size of the index. We leave this for future work.

In terms of theory, it would make sense to bound
$\Count(Q,S)$ as a function of $\Count(P,T)$ and the sketching parameters.
Such a function would bound the number of false positives.
