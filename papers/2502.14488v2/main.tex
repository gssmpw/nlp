% \documentclass{article}
\documentclass[a4paper, UKenglish, cleveref, autoref, thm-restate, unicode, colorlinks=true, allcolors=blue]{lipics-v2021}

\nolinenumbers{}

% Language setting
% Replace `english' with e.g. `spanish' to change the document language
% \usepackage[english]{babel}

% Set page size and margins
% Replace `letterpaper' with `a4paper' for UK/EU standard size
% \usepackage[letterpaper,top=2cm,bottom=2cm,left=3cm,right=3cm,marginparwidth=1.75cm]{geometry}

% Useful packages
\usepackage{amsmath,amssymb}
\usepackage{graphicx}
% \usepackage[colorlinks=true, allcolors=blue]{hyperref}
% \usepackage[capitalize]{cleveref}
\usepackage{xspace}
\usepackage{xcolor}
\usepackage{thmtools}
\usepackage{thm-restate}
\usepackage{url}
\usepackage{svg}
\usepackage[normalem]{ulem}
\usepackage[T1]{fontenc}


%
\setlength\unitlength{1mm}
\newcommand{\twodots}{\mathinner {\ldotp \ldotp}}
% bb font symbols
\newcommand{\Rho}{\mathrm{P}}
\newcommand{\Tau}{\mathrm{T}}

\newfont{\bbb}{msbm10 scaled 700}
\newcommand{\CCC}{\mbox{\bbb C}}

\newfont{\bb}{msbm10 scaled 1100}
\newcommand{\CC}{\mbox{\bb C}}
\newcommand{\PP}{\mbox{\bb P}}
\newcommand{\RR}{\mbox{\bb R}}
\newcommand{\QQ}{\mbox{\bb Q}}
\newcommand{\ZZ}{\mbox{\bb Z}}
\newcommand{\FF}{\mbox{\bb F}}
\newcommand{\GG}{\mbox{\bb G}}
\newcommand{\EE}{\mbox{\bb E}}
\newcommand{\NN}{\mbox{\bb N}}
\newcommand{\KK}{\mbox{\bb K}}
\newcommand{\HH}{\mbox{\bb H}}
\newcommand{\SSS}{\mbox{\bb S}}
\newcommand{\UU}{\mbox{\bb U}}
\newcommand{\VV}{\mbox{\bb V}}


\newcommand{\yy}{\mathbbm{y}}
\newcommand{\xx}{\mathbbm{x}}
\newcommand{\zz}{\mathbbm{z}}
\newcommand{\sss}{\mathbbm{s}}
\newcommand{\rr}{\mathbbm{r}}
\newcommand{\pp}{\mathbbm{p}}
\newcommand{\qq}{\mathbbm{q}}
\newcommand{\ww}{\mathbbm{w}}
\newcommand{\hh}{\mathbbm{h}}
\newcommand{\vvv}{\mathbbm{v}}

% Vectors

\newcommand{\av}{{\bf a}}
\newcommand{\bv}{{\bf b}}
\newcommand{\cv}{{\bf c}}
\newcommand{\dv}{{\bf d}}
\newcommand{\ev}{{\bf e}}
\newcommand{\fv}{{\bf f}}
\newcommand{\gv}{{\bf g}}
\newcommand{\hv}{{\bf h}}
\newcommand{\iv}{{\bf i}}
\newcommand{\jv}{{\bf j}}
\newcommand{\kv}{{\bf k}}
\newcommand{\lv}{{\bf l}}
\newcommand{\mv}{{\bf m}}
\newcommand{\nv}{{\bf n}}
\newcommand{\ov}{{\bf o}}
\newcommand{\pv}{{\bf p}}
\newcommand{\qv}{{\bf q}}
\newcommand{\rv}{{\bf r}}
\newcommand{\sv}{{\bf s}}
\newcommand{\tv}{{\bf t}}
\newcommand{\uv}{{\bf u}}
\newcommand{\wv}{{\bf w}}
\newcommand{\vv}{{\bf v}}
\newcommand{\xv}{{\bf x}}
\newcommand{\yv}{{\bf y}}
\newcommand{\zv}{{\bf z}}
\newcommand{\zerov}{{\bf 0}}
\newcommand{\onev}{{\bf 1}}

% Matrices

\newcommand{\Am}{{\bf A}}
\newcommand{\Bm}{{\bf B}}
\newcommand{\Cm}{{\bf C}}
\newcommand{\Dm}{{\bf D}}
\newcommand{\Em}{{\bf E}}
\newcommand{\Fm}{{\bf F}}
\newcommand{\Gm}{{\bf G}}
\newcommand{\Hm}{{\bf H}}
\newcommand{\Id}{{\bf I}}
\newcommand{\Jm}{{\bf J}}
\newcommand{\Km}{{\bf K}}
\newcommand{\Lm}{{\bf L}}
\newcommand{\Mm}{{\bf M}}
\newcommand{\Nm}{{\bf N}}
\newcommand{\Om}{{\bf O}}
\newcommand{\Pm}{{\bf P}}
\newcommand{\Qm}{{\bf Q}}
\newcommand{\Rm}{{\bf R}}
\newcommand{\Sm}{{\bf S}}
\newcommand{\Tm}{{\bf T}}
\newcommand{\Um}{{\bf U}}
\newcommand{\Wm}{{\bf W}}
\newcommand{\Vm}{{\bf V}}
\newcommand{\Xm}{{\bf X}}
\newcommand{\Ym}{{\bf Y}}
\newcommand{\Zm}{{\bf Z}}

% Calligraphic

\newcommand{\Ac}{{\cal A}}
\newcommand{\Bc}{{\cal B}}
\newcommand{\Cc}{{\cal C}}
\newcommand{\Dc}{{\cal D}}
\newcommand{\Ec}{{\cal E}}
\newcommand{\Fc}{{\cal F}}
\newcommand{\Gc}{{\cal G}}
\newcommand{\Hc}{{\cal H}}
\newcommand{\Ic}{{\cal I}}
\newcommand{\Jc}{{\cal J}}
\newcommand{\Kc}{{\cal K}}
\newcommand{\Lc}{{\cal L}}
\newcommand{\Mc}{{\cal M}}
\newcommand{\Nc}{{\cal N}}
\newcommand{\nc}{{\cal n}}
\newcommand{\Oc}{{\cal O}}
\newcommand{\Pc}{{\cal P}}
\newcommand{\Qc}{{\cal Q}}
\newcommand{\Rc}{{\cal R}}
\newcommand{\Sc}{{\cal S}}
\newcommand{\Tc}{{\cal T}}
\newcommand{\Uc}{{\cal U}}
\newcommand{\Wc}{{\cal W}}
\newcommand{\Vc}{{\cal V}}
\newcommand{\Xc}{{\cal X}}
\newcommand{\Yc}{{\cal Y}}
\newcommand{\Zc}{{\cal Z}}

% Bold greek letters

\newcommand{\alphav}{\hbox{\boldmath$\alpha$}}
\newcommand{\betav}{\hbox{\boldmath$\beta$}}
\newcommand{\gammav}{\hbox{\boldmath$\gamma$}}
\newcommand{\deltav}{\hbox{\boldmath$\delta$}}
\newcommand{\etav}{\hbox{\boldmath$\eta$}}
\newcommand{\lambdav}{\hbox{\boldmath$\lambda$}}
\newcommand{\epsilonv}{\hbox{\boldmath$\epsilon$}}
\newcommand{\nuv}{\hbox{\boldmath$\nu$}}
\newcommand{\muv}{\hbox{\boldmath$\mu$}}
\newcommand{\zetav}{\hbox{\boldmath$\zeta$}}
\newcommand{\phiv}{\hbox{\boldmath$\phi$}}
\newcommand{\psiv}{\hbox{\boldmath$\psi$}}
\newcommand{\thetav}{\hbox{\boldmath$\theta$}}
\newcommand{\tauv}{\hbox{\boldmath$\tau$}}
\newcommand{\omegav}{\hbox{\boldmath$\omega$}}
\newcommand{\xiv}{\hbox{\boldmath$\xi$}}
\newcommand{\sigmav}{\hbox{\boldmath$\sigma$}}
\newcommand{\piv}{\hbox{\boldmath$\pi$}}
\newcommand{\rhov}{\hbox{\boldmath$\rho$}}
\newcommand{\upsilonv}{\hbox{\boldmath$\upsilon$}}

\newcommand{\Gammam}{\hbox{\boldmath$\Gamma$}}
\newcommand{\Lambdam}{\hbox{\boldmath$\Lambda$}}
\newcommand{\Deltam}{\hbox{\boldmath$\Delta$}}
\newcommand{\Sigmam}{\hbox{\boldmath$\Sigma$}}
\newcommand{\Phim}{\hbox{\boldmath$\Phi$}}
\newcommand{\Pim}{\hbox{\boldmath$\Pi$}}
\newcommand{\Psim}{\hbox{\boldmath$\Psi$}}
\newcommand{\Thetam}{\hbox{\boldmath$\Theta$}}
\newcommand{\Omegam}{\hbox{\boldmath$\Omega$}}
\newcommand{\Xim}{\hbox{\boldmath$\Xi$}}


% Sans Serif small case

\newcommand{\Gsf}{{\sf G}}

\newcommand{\asf}{{\sf a}}
\newcommand{\bsf}{{\sf b}}
\newcommand{\csf}{{\sf c}}
\newcommand{\dsf}{{\sf d}}
\newcommand{\esf}{{\sf e}}
\newcommand{\fsf}{{\sf f}}
\newcommand{\gsf}{{\sf g}}
\newcommand{\hsf}{{\sf h}}
\newcommand{\isf}{{\sf i}}
\newcommand{\jsf}{{\sf j}}
\newcommand{\ksf}{{\sf k}}
\newcommand{\lsf}{{\sf l}}
\newcommand{\msf}{{\sf m}}
\newcommand{\nsf}{{\sf n}}
\newcommand{\osf}{{\sf o}}
\newcommand{\psf}{{\sf p}}
\newcommand{\qsf}{{\sf q}}
\newcommand{\rsf}{{\sf r}}
\newcommand{\ssf}{{\sf s}}
\newcommand{\tsf}{{\sf t}}
\newcommand{\usf}{{\sf u}}
\newcommand{\wsf}{{\sf w}}
\newcommand{\vsf}{{\sf v}}
\newcommand{\xsf}{{\sf x}}
\newcommand{\ysf}{{\sf y}}
\newcommand{\zsf}{{\sf z}}


% mixed symbols

\newcommand{\sinc}{{\hbox{sinc}}}
\newcommand{\diag}{{\hbox{diag}}}
\renewcommand{\det}{{\hbox{det}}}
\newcommand{\trace}{{\hbox{tr}}}
\newcommand{\sign}{{\hbox{sign}}}
\renewcommand{\arg}{{\hbox{arg}}}
\newcommand{\var}{{\hbox{var}}}
\newcommand{\cov}{{\hbox{cov}}}
\newcommand{\Ei}{{\rm E}_{\rm i}}
\renewcommand{\Re}{{\rm Re}}
\renewcommand{\Im}{{\rm Im}}
\newcommand{\eqdef}{\stackrel{\Delta}{=}}
\newcommand{\defines}{{\,\,\stackrel{\scriptscriptstyle \bigtriangleup}{=}\,\,}}
\newcommand{\<}{\left\langle}
\renewcommand{\>}{\right\rangle}
\newcommand{\herm}{{\sf H}}
\newcommand{\trasp}{{\sf T}}
\newcommand{\transp}{{\sf T}}
\renewcommand{\vec}{{\rm vec}}
\newcommand{\Psf}{{\sf P}}
\newcommand{\SINR}{{\sf SINR}}
\newcommand{\SNR}{{\sf SNR}}
\newcommand{\MMSE}{{\sf MMSE}}
\newcommand{\REF}{{\RED [REF]}}

% Markov chain
\usepackage{stmaryrd} % for \mkv 
\newcommand{\mkv}{-\!\!\!\!\minuso\!\!\!\!-}

% Colors

\newcommand{\RED}{\color[rgb]{1.00,0.10,0.10}}
\newcommand{\BLUE}{\color[rgb]{0,0,0.90}}
\newcommand{\GREEN}{\color[rgb]{0,0.80,0.20}}

%%%%%%%%%%%%%%%%%%%%%%%%%%%%%%%%%%%%%%%%%%
\usepackage{hyperref}
\hypersetup{
    bookmarks=true,         % show bookmarks bar?
    unicode=false,          % non-Latin characters in AcrobatÕs bookmarks
    pdftoolbar=true,        % show AcrobatÕs toolbar?
    pdfmenubar=true,        % show AcrobatÕs menu?
    pdffitwindow=false,     % window fit to page when opened
    pdfstartview={FitH},    % fits the width of the page to the window
%    pdftitle={My title},    % title
%    pdfauthor={Author},     % author
%    pdfsubject={Subject},   % subject of the document
%    pdfcreator={Creator},   % creator of the document
%    pdfproducer={Producer}, % producer of the document
%    pdfkeywords={keyword1} {key2} {key3}, % list of keywords
    pdfnewwindow=true,      % links in new window
    colorlinks=true,       % false: boxed links; true: colored links
    linkcolor=red,          % color of internal links (change box color with linkbordercolor)
    citecolor=green,        % color of links to bibliography
    filecolor=blue,      % color of file links
    urlcolor=blue           % color of external links
}
%%%%%%%%%%%%%%%%%%%%%%%%%%%%%%%%%%%%%%%%%%%



\title{U-index: A Universal Indexing Framework for Matching Long Patterns}%: Theory, Practice, and Applications}

\authorrunning{L. A. K. Ayad et al.}

\author{Lorraine A. K. Ayad}
       {Brunel University London, London, UK}
       {lorraine.ayad@brunel.ac.uk}
       {orcid.org/0000-0003-0846-2616}
       {}
\author{Gabriele Fici}
       {Dipartimento di Matematica e Informatica, Università di Palermo, Italy}
       {gabriele.fici@unipa.it}
       {https://orcid.org/0000-0002-3536-327X}
       {Supported by MIUR project PRIN 2022 APML – 20229BCXNW}
\author{Ragnar Groot Koerkamp}
       {ETH Zurich, Zurich, Switzerland}
       {ragnar.grootkoerkamp@inf.ethz.ch}
       {https://orcid.org/0000-0002-2091-1237}
       {ETH Research Grant ETH-1721-1 to Gunnar Rätsch.}
\author{Grigorios Loukides}
       {King's College London, London, UK}
       {grigorios.loukides@kcl.ac.uk}
       {https://orcid.org/0000-0003-0888-5061}
       {}
\author{Rob Patro}
       {University of Maryland, College Park, MD, USA}
       {rob@cs.umd.edu}
       {https://orcid.org/0000-0001-8463-1675}
       {NIH grant award number R01HG009937, NSF award CNS-1763680 and grants
         252586 and 2024342821 from the Chan Zuckerberg Initiative DAF, an
         advised fund of Silicon Valley Community Foundation.
         RP is a co-founder of Ocean Genomics, Inc.
       }
\author{Giulio Ermanno Pibiri}
       {Ca' Foscari University of Venice, Venice, Italy\\ISTI-CNR, Pisa, Italy}
       {giulioermanno.pibiri@unive.it}
       {https://orcid.org/0000-0003-0724-7092}
       {European Union’s Horizon Europe research and innovation programme (EFRA project, Grant Agreement Number 101093026). This work was also partially supported by DAIS – Ca’ Foscari University of Venice within the IRIDE program.}
\author{Solon P. Pissis}
       {CWI, Amsterdam, The Netherlands\\Vrije Universiteit, Amsterdam, The Netherlands}
       {solon.pissis@cwi.nl}
       {https://orcid.org/0000-0002-1445-1932}
       {Supported by the PANGAIA and ALPACA projects that have received funding from the European Union’s Horizon 2020 research and innovation programme under the Marie Skłodowska-Curie grant agreements No 872539 and 956229, respectively.}

\ccsdesc{Theory of computation~Sketching and sampling}
\ccsdesc{Applied computing~Bioinformatics}

\supplementdetails[subcategory={Rust}, linktext={github.com/u-index/u-index-rs}]{Software}{
https://github.com/u-index/u-index-rs
}

\keywords{Sketching; Hashing; Minimizers; Text Indexing}

\hideLIPIcs

% \EventEditors{}
% \EventNoEds{}
% \EventLongTitle{}
% \EventShortTitle{}
% \EventAcronym{}
% \EventYear{}
% \EventDate{}
% \EventLocation{}
% \EventLogo{}
% \SeriesVolume{}
% \ArticleNo{}

\begin{document}
\maketitle

\begin{abstract}  
Test time scaling is currently one of the most active research areas that shows promise after training time scaling has reached its limits.
Deep-thinking (DT) models are a class of recurrent models that can perform easy-to-hard generalization by assigning more compute to harder test samples.
However, due to their inability to determine the complexity of a test sample, DT models have to use a large amount of computation for both easy and hard test samples.
Excessive test time computation is wasteful and can cause the ``overthinking'' problem where more test time computation leads to worse results.
In this paper, we introduce a test time training method for determining the optimal amount of computation needed for each sample during test time.
We also propose Conv-LiGRU, a novel recurrent architecture for efficient and robust visual reasoning. 
Extensive experiments demonstrate that Conv-LiGRU is more stable than DT, effectively mitigates the ``overthinking'' phenomenon, and achieves superior accuracy.
\end{abstract}  

%% \newpage
\setcounter{page}{1}

\section{Introduction}
\label{sec:introduction}
The business processes of organizations are experiencing ever-increasing complexity due to the large amount of data, high number of users, and high-tech devices involved \cite{martin2021pmopportunitieschallenges, beerepoot2023biggestbpmproblems}. This complexity may cause business processes to deviate from normal control flow due to unforeseen and disruptive anomalies \cite{adams2023proceddsriftdetection}. These control-flow anomalies manifest as unknown, skipped, and wrongly-ordered activities in the traces of event logs monitored from the execution of business processes \cite{ko2023adsystematicreview}. For the sake of clarity, let us consider an illustrative example of such anomalies. Figure \ref{FP_ANOMALIES} shows a so-called event log footprint, which captures the control flow relations of four activities of a hypothetical event log. In particular, this footprint captures the control-flow relations between activities \texttt{a}, \texttt{b}, \texttt{c} and \texttt{d}. These are the causal ($\rightarrow$) relation, concurrent ($\parallel$) relation, and other ($\#$) relations such as exclusivity or non-local dependency \cite{aalst2022pmhandbook}. In addition, on the right are six traces, of which five exhibit skipped, wrongly-ordered and unknown control-flow anomalies. For example, $\langle$\texttt{a b d}$\rangle$ has a skipped activity, which is \texttt{c}. Because of this skipped activity, the control-flow relation \texttt{b}$\,\#\,$\texttt{d} is violated, since \texttt{d} directly follows \texttt{b} in the anomalous trace.
\begin{figure}[!t]
\centering
\includegraphics[width=0.9\columnwidth]{images/FP_ANOMALIES.png}
\caption{An example event log footprint with six traces, of which five exhibit control-flow anomalies.}
\label{FP_ANOMALIES}
\end{figure}

\subsection{Control-flow anomaly detection}
Control-flow anomaly detection techniques aim to characterize the normal control flow from event logs and verify whether these deviations occur in new event logs \cite{ko2023adsystematicreview}. To develop control-flow anomaly detection techniques, \revision{process mining} has seen widespread adoption owing to process discovery and \revision{conformance checking}. On the one hand, process discovery is a set of algorithms that encode control-flow relations as a set of model elements and constraints according to a given modeling formalism \cite{aalst2022pmhandbook}; hereafter, we refer to the Petri net, a widespread modeling formalism. On the other hand, \revision{conformance checking} is an explainable set of algorithms that allows linking any deviations with the reference Petri net and providing the fitness measure, namely a measure of how much the Petri net fits the new event log \cite{aalst2022pmhandbook}. Many control-flow anomaly detection techniques based on \revision{conformance checking} (hereafter, \revision{conformance checking}-based techniques) use the fitness measure to determine whether an event log is anomalous \cite{bezerra2009pmad, bezerra2013adlogspais, myers2018icsadpm, pecchia2020applicationfailuresanalysispm}. 

The scientific literature also includes many \revision{conformance checking}-independent techniques for control-flow anomaly detection that combine specific types of trace encodings with machine/deep learning \cite{ko2023adsystematicreview, tavares2023pmtraceencoding}. Whereas these techniques are very effective, their explainability is challenging due to both the type of trace encoding employed and the machine/deep learning model used \cite{rawal2022trustworthyaiadvances,li2023explainablead}. Hence, in the following, we focus on the shortcomings of \revision{conformance checking}-based techniques to investigate whether it is possible to support the development of competitive control-flow anomaly detection techniques while maintaining the explainable nature of \revision{conformance checking}.
\begin{figure}[!t]
\centering
\includegraphics[width=\columnwidth]{images/HIGH_LEVEL_VIEW.png}
\caption{A high-level view of the proposed framework for combining \revision{process mining}-based feature extraction with dimensionality reduction for control-flow anomaly detection.}
\label{HIGH_LEVEL_VIEW}
\end{figure}

\subsection{Shortcomings of \revision{conformance checking}-based techniques}
Unfortunately, the detection effectiveness of \revision{conformance checking}-based techniques is affected by noisy data and low-quality Petri nets, which may be due to human errors in the modeling process or representational bias of process discovery algorithms \cite{bezerra2013adlogspais, pecchia2020applicationfailuresanalysispm, aalst2016pm}. Specifically, on the one hand, noisy data may introduce infrequent and deceptive control-flow relations that may result in inconsistent fitness measures, whereas, on the other hand, checking event logs against a low-quality Petri net could lead to an unreliable distribution of fitness measures. Nonetheless, such Petri nets can still be used as references to obtain insightful information for \revision{process mining}-based feature extraction, supporting the development of competitive and explainable \revision{conformance checking}-based techniques for control-flow anomaly detection despite the problems above. For example, a few works outline that token-based \revision{conformance checking} can be used for \revision{process mining}-based feature extraction to build tabular data and develop effective \revision{conformance checking}-based techniques for control-flow anomaly detection \cite{singh2022lapmsh, debenedictis2023dtadiiot}. However, to the best of our knowledge, the scientific literature lacks a structured proposal for \revision{process mining}-based feature extraction using the state-of-the-art \revision{conformance checking} variant, namely alignment-based \revision{conformance checking}.

\subsection{Contributions}
We propose a novel \revision{process mining}-based feature extraction approach with alignment-based \revision{conformance checking}. This variant aligns the deviating control flow with a reference Petri net; the resulting alignment can be inspected to extract additional statistics such as the number of times a given activity caused mismatches \cite{aalst2022pmhandbook}. We integrate this approach into a flexible and explainable framework for developing techniques for control-flow anomaly detection. The framework combines \revision{process mining}-based feature extraction and dimensionality reduction to handle high-dimensional feature sets, achieve detection effectiveness, and support explainability. Notably, in addition to our proposed \revision{process mining}-based feature extraction approach, the framework allows employing other approaches, enabling a fair comparison of multiple \revision{conformance checking}-based and \revision{conformance checking}-independent techniques for control-flow anomaly detection. Figure \ref{HIGH_LEVEL_VIEW} shows a high-level view of the framework. Business processes are monitored, and event logs obtained from the database of information systems. Subsequently, \revision{process mining}-based feature extraction is applied to these event logs and tabular data input to dimensionality reduction to identify control-flow anomalies. We apply several \revision{conformance checking}-based and \revision{conformance checking}-independent framework techniques to publicly available datasets, simulated data of a case study from railways, and real-world data of a case study from healthcare. We show that the framework techniques implementing our approach outperform the baseline \revision{conformance checking}-based techniques while maintaining the explainable nature of \revision{conformance checking}.

In summary, the contributions of this paper are as follows.
\begin{itemize}
    \item{
        A novel \revision{process mining}-based feature extraction approach to support the development of competitive and explainable \revision{conformance checking}-based techniques for control-flow anomaly detection.
    }
    \item{
        A flexible and explainable framework for developing techniques for control-flow anomaly detection using \revision{process mining}-based feature extraction and dimensionality reduction.
    }
    \item{
        Application to synthetic and real-world datasets of several \revision{conformance checking}-based and \revision{conformance checking}-independent framework techniques, evaluating their detection effectiveness and explainability.
    }
\end{itemize}

The rest of the paper is organized as follows.
\begin{itemize}
    \item Section \ref{sec:related_work} reviews the existing techniques for control-flow anomaly detection, categorizing them into \revision{conformance checking}-based and \revision{conformance checking}-independent techniques.
    \item Section \ref{sec:abccfe} provides the preliminaries of \revision{process mining} to establish the notation used throughout the paper, and delves into the details of the proposed \revision{process mining}-based feature extraction approach with alignment-based \revision{conformance checking}.
    \item Section \ref{sec:framework} describes the framework for developing \revision{conformance checking}-based and \revision{conformance checking}-independent techniques for control-flow anomaly detection that combine \revision{process mining}-based feature extraction and dimensionality reduction.
    \item Section \ref{sec:evaluation} presents the experiments conducted with multiple framework and baseline techniques using data from publicly available datasets and case studies.
    \item Section \ref{sec:conclusions} draws the conclusions and presents future work.
\end{itemize}
\section{Background}\label{sec:backgrnd}

\subsection{Cold Start Latency and Mitigation Techniques}

Traditional FaaS platforms mitigate cold starts through snapshotting, lightweight virtualization, and warm-state management. Snapshot-based methods like \textbf{REAP} and \textbf{Catalyzer} reduce initialization time by preloading or restoring container states but require significant memory and I/O resources, limiting scalability~\cite{dong_catalyzer_2020, ustiugov_benchmarking_2021}. Lightweight virtualization solutions, such as \textbf{Firecracker} microVMs, achieve fast startup times with strong isolation but depend on robust infrastructure, making them less adaptable to fluctuating workloads~\cite{agache_firecracker_2020}. Warm-state management techniques like \textbf{Faa\$T}~\cite{romero_faa_2021} and \textbf{Kraken}~\cite{vivek_kraken_2021} keep frequently invoked containers ready, balancing readiness and cost efficiency under predictable workloads but incurring overhead when demand is erratic~\cite{romero_faa_2021, vivek_kraken_2021}. While these methods perform well in resource-rich cloud environments, their resource intensity challenges applicability in edge settings.

\subsubsection{Edge FaaS Perspective}

In edge environments, cold start mitigation emphasizes lightweight designs, resource sharing, and hybrid task distribution. Lightweight execution environments like unikernels~\cite{edward_sock_2018} and \textbf{Firecracker}~\cite{agache_firecracker_2020}, as used by \textbf{TinyFaaS}~\cite{pfandzelter_tinyfaas_2020}, minimize resource usage and initialization delays but require careful orchestration to avoid resource contention. Function co-location, demonstrated by \textbf{Photons}~\cite{v_dukic_photons_2020}, reduces redundant initializations by sharing runtime resources among related functions, though this complicates isolation in multi-tenant setups~\cite{v_dukic_photons_2020}. Hybrid offloading frameworks like \textbf{GeoFaaS}~\cite{malekabbasi_geofaas_2024} balance edge-cloud workloads by offloading latency-tolerant tasks to the cloud and reserving edge resources for real-time operations, requiring reliable connectivity and efficient task management. These edge-specific strategies address cold starts effectively but introduce challenges in scalability and orchestration.

\subsection{Predictive Scaling and Caching Techniques}

Efficient resource allocation is vital for maintaining low latency and high availability in serverless platforms. Predictive scaling and caching techniques dynamically provision resources and reduce cold start latency by leveraging workload prediction and state retention.
Traditional FaaS platforms use predictive scaling and caching to optimize resources, employing techniques (OFC, FaasCache) to reduce cold starts. However, these methods rely on centralized orchestration and workload predictability, limiting their effectiveness in dynamic, resource-constrained edge environments.



\subsubsection{Edge FaaS Perspective}

Edge FaaS platforms adapt predictive scaling and caching techniques to constrain resources and heterogeneous environments. \textbf{EDGE-Cache}~\cite{kim_delay-aware_2022} uses traffic profiling to selectively retain high-priority functions, reducing memory overhead while maintaining readiness for frequent requests. Hybrid frameworks like \textbf{GeoFaaS}~\cite{malekabbasi_geofaas_2024} implement distributed caching to balance resources between edge and cloud nodes, enabling low-latency processing for critical tasks while offloading less critical workloads. Machine learning methods, such as clustering-based workload predictors~\cite{gao_machine_2020} and GRU-based models~\cite{guo_applying_2018}, enhance resource provisioning in edge systems by efficiently forecasting workload spikes. These innovations effectively address cold start challenges in edge environments, though their dependency on accurate predictions and robust orchestration poses scalability challenges.

\subsection{Decentralized Orchestration, Function Placement, and Scheduling}

Efficient orchestration in serverless platforms involves workload distribution, resource optimization, and performance assurance. While traditional FaaS platforms rely on centralized control, edge environments require decentralized and adaptive strategies to address unique challenges such as resource constraints and heterogeneous hardware.



\subsubsection{Edge FaaS Perspective}

Edge FaaS platforms adopt decentralized and adaptive orchestration frameworks to meet the demands of resource-constrained environments. Systems like \textbf{Wukong} distribute scheduling across edge nodes, enhancing data locality and scalability while reducing network latency. Lightweight frameworks such as \textbf{OpenWhisk Lite}~\cite{kravchenko_kpavelopenwhisk-light_2024} optimize resource allocation by decentralizing scheduling policies, minimizing cold starts and latency in edge setups~\cite{benjamin_wukong_2020}. Hybrid solutions like \textbf{OpenFaaS}~\cite{noauthor_openfaasfaas_2024} and \textbf{EdgeMatrix}~\cite{shen_edgematrix_2023} combine edge-cloud orchestration to balance resource utilization, retaining latency-sensitive functions at the edge while offloading non-critical workloads to the cloud. While these approaches improve flexibility, they face challenges in maintaining coordination and ensuring consistent performance across distributed nodes.



%\section{Related Work}
%\label{sec:related-work}

%\subsection{Background}

%Defect detection is critical to ensure the yield of integrated circuit manufacturing lines and reduce faults. Previous research has primarily focused on wafer map data, which engineers produce by marking faulty chips with different colors based on test results. The specific spatial distribution of defects on a wafer can provide insights into the causes, thereby helping to determine which stage of the manufacturing process is responsible for the issues. Although such research is relatively mature, the continual miniaturization of integrated circuits and the increasing complexity and density of chip components have made chip-level detection more challenging, leading to potential risks\cite{ma2023review}. Consequently, there is a need to combine this approach with magnified imaging of the wafer surface using scanning electron microscopes (SEMs) to detect, classify, and analyze specific microscopic defects, thus helping to identify the particular process steps where defects originate.

%Previously, wafer surface defect classification and detection were primarily conducted by experienced engineers. However, this method relies heavily on the engineers' expertise and involves significant time expenditure and subjectivity, lacking uniform standards. With the ongoing development of artificial intelligence, deep learning methods using multi-layer neural networks to extract and learn target features have proven highly effective for this task\cite{gao2022review}.

%In the task of defect classification, it is typical to use a model structure that initially extracts features through convolutional and pooling layers, followed by classification via fully connected layers. Researchers have recently developed numerous classification model structures tailored to specific problems. These models primarily focus on how to extract defect features effectively. For instance, Chen et al. presented a defect recognition and classification algorithm rooted in PCA and classification SVM\cite{chen2008defect}. Chang et al. utilized SVM, drawing on features like smoothness and texture intricacy, for classifying high-intensity defect images\cite{chang2013hybrid}. The classification of defect images requires the formulation of numerous classifiers tailored for myriad inspection steps and an Abundance of accurately labeled data, making data acquisition challenging. Cheon et al. proposed a single CNN model adept at feature extraction\cite{cheon2019convolutional}. They achieved a granular classification of wafer surface defects by recognizing misclassified images and employing a k-nearest neighbors (k-NN) classifier algorithm to gauge the aggregate squared distance between each image feature vector and its k-neighbors within the same category. However, when applied to new or unseen defects, such models necessitate retraining, incurring computational overheads. Moreover, with escalating CNN complexity, the computational demands surge.

%Segmentation of defects is necessary to locate defect positions and gather information such as the size of defects. Unlike classification networks, segmentation networks often use classic encoder-decoder structures such as UNet\cite{ronneberger2015u} and SegNet\cite{badrinarayanan2017segnet}, which focus on effectively leveraging both local and global feature information. Han Hui et al. proposed integrating a Region Proposal Network (RPN) with a UNet architecture to suggest defect areas before conducting defect segmentation \cite{han2020polycrystalline}. This approach enables the segmentation of various defects in wafers with only a limited set of roughly labeled images, enhancing the efficiency of training and application in environments where detailed annotations are scarce. Subhrajit Nag et al. introduced a new network structure, WaferSegClassNet, which extracts multi-scale local features in the encoder and performs classification and segmentation tasks in the decoder \cite{nag2022wafersegclassnet}. This model represents the first detection system capable of simultaneously classifying and segmenting surface defects on wafers. However, it relies on extensive data training and annotation for high accuracy and reliability. 

%Recently, Vic De Ridder et al. introduced a novel approach for defect segmentation using diffusion models\cite{de2023semi}. This approach treats the instance segmentation task as a denoising process from noise to a filter, utilizing diffusion models to predict and reconstruct instance masks for semiconductor defects. This method achieves high precision and improved defect classification and segmentation detection performance. However, the complex network structure and the computational process of the diffusion model require substantial computational resources. Moreover, the performance of this model heavily relies on high-quality and large amounts of training data. These issues make it less suitable for industrial applications. Additionally, the model has only been applied to detecting and segmenting a single type of defect(bridges) following a specific manufacturing process step, limiting its practical utility in diverse industrial scenarios.

%\subsection{Few-shot Anomaly Detection}
%Traditional anomaly detection techniques typically rely on extensive training data to train models for identifying and locating anomalies. However, these methods often face limitations in rapidly changing production environments and diverse anomaly types. Recent research has started exploring effective anomaly detection using few or zero samples to address these challenges.

%Huang et al. developed the anomaly detection method RegAD, based on image registration technology. This method pre-trains an object-agnostic registration network with various images to establish the normality of unseen objects. It identifies anomalies by aligning image features and has achieved promising results. Despite these advancements, implementing few-shot settings in anomaly detection remains an area ripe for further exploration. Recent studies show that pre-trained vision-language models such as CLIP and MiniGPT can significantly enhance performance in anomaly detection tasks.

%Dong et al. introduced the MaskCLIP framework, which employs masked self-distillation to enhance contrastive language-image pretraining\cite{zhou2022maskclip}. This approach strengthens the visual encoder's learning of local image patches and uses indirect language supervision to enhance semantic understanding. It significantly improves transferability and pretraining outcomes across various visual tasks, although it requires substantial computational resources.
%Jeong et al. crafted the WinCLIP framework by integrating state words and prompt templates to characterize normal and anomalous states more accurately\cite{Jeong_2023_CVPR}. This framework introduces a novel window-based technique for extracting and aggregating multi-scale spatial features, significantly boosting the anomaly detection performance of the pre-trained CLIP model.
%Subsequently, Li et al. have further contributed to the field by creating a new expansive multimodal model named Myriad\cite{li2023myriad}. This model, which incorporates a pre-trained Industrial Anomaly Detection (IAD) model to act as a vision expert, embeds anomaly images as tokens interpretable by the language model, thus providing both detailed descriptions and accurate anomaly detection capabilities.
%Recently, Chen et al. introduced CLIP-AD\cite{chen2023clip}, and Li et al. proposed PromptAD\cite{li2024promptad}, both employing language-guided, tiered dual-path model structures and feature manipulation strategies. These approaches effectively address issues encountered when directly calculating anomaly maps using the CLIP model, such as reversed predictions and highlighting irrelevant areas. Specifically, CLIP-AD optimizes the utilization of multi-layer features, corrects feature misalignment, and enhances model performance through additional linear layer fine-tuning. PromptAD connects normal prompts with anomaly suffixes to form anomaly prompts, enabling contrastive learning in a single-class setting.

%These studies extend the boundaries of traditional anomaly detection techniques and demonstrate how to effectively address rapidly changing and sample-scarce production environments through the synergy of few-shot learning and deep learning models. Building on this foundation, our research further explores wafer surface defect detection based on the CLIP model, especially focusing on achieving efficient and accurate anomaly detection in the highly specialized and variable semiconductor manufacturing process using a minimal amount of labeled data.

\begin{figure*}[ht]
    \centering
    \includegraphics[width=\textwidth, trim=79 280 93 123, clip]{figures/framework_img.pdf}
    \caption{The pipeline of the \ENDow{} framework 
    %where each component is specified in a given configuration. 
    which yields a downstream task score and a WER score of the transcript set input to the task. The pipeline is executed for several severeties of noising and types of cleaning techniques. %Acoustic noising is applied at $k$ intensities, providing $k+1$ audio versions (including the non-noised version), eventually producing $k+2$ transcript versions (including the source transcript). Applying transcript cleaning reveals the effect of \textit{types} of noise. 
    Resulting scores are plotted on a graph for the analyses, as in, e.g., \autoref{fig_cleaning_graphs}.}
    %The pipeline is executed on $k+1$ intensities of acoustic noising (including the non-noised version), producing $k+2$ scores for the downstream task (including execution on the source transcripts). This process eventually describes the effect of the \textit{intensity} of transcript noise on the downstream task. The process is repeated for $m$ cleaning techniques ($m+1$ when including no cleaning), to analyze the benefit of a cleaning approach and the effect of the \textit{types} of transcript noise.}
    \label{fig_framework}
\end{figure*}
\section{Experiments}
\label{sec:experiments}
The experiments are designed to address two key research questions.
First, \textbf{RQ1} evaluates whether the average $L_2$-norm of the counterfactual perturbation vectors ($\overline{||\perturb||}$) decreases as the model overfits the data, thereby providing further empirical validation for our hypothesis.
Second, \textbf{RQ2} evaluates the ability of the proposed counterfactual regularized loss, as defined in (\ref{eq:regularized_loss2}), to mitigate overfitting when compared to existing regularization techniques.

% The experiments are designed to address three key research questions. First, \textbf{RQ1} investigates whether the mean perturbation vector norm decreases as the model overfits the data, aiming to further validate our intuition. Second, \textbf{RQ2} explores whether the mean perturbation vector norm can be effectively leveraged as a regularization term during training, offering insights into its potential role in mitigating overfitting. Finally, \textbf{RQ3} examines whether our counterfactual regularizer enables the model to achieve superior performance compared to existing regularization methods, thus highlighting its practical advantage.

\subsection{Experimental Setup}
\textbf{\textit{Datasets, Models, and Tasks.}}
The experiments are conducted on three datasets: \textit{Water Potability}~\cite{kadiwal2020waterpotability}, \textit{Phomene}~\cite{phomene}, and \textit{CIFAR-10}~\cite{krizhevsky2009learning}. For \textit{Water Potability} and \textit{Phomene}, we randomly select $80\%$ of the samples for the training set, and the remaining $20\%$ for the test set, \textit{CIFAR-10} comes already split. Furthermore, we consider the following models: Logistic Regression, Multi-Layer Perceptron (MLP) with 100 and 30 neurons on each hidden layer, and PreactResNet-18~\cite{he2016cvecvv} as a Convolutional Neural Network (CNN) architecture.
We focus on binary classification tasks and leave the extension to multiclass scenarios for future work. However, for datasets that are inherently multiclass, we transform the problem into a binary classification task by selecting two classes, aligning with our assumption.

\smallskip
\noindent\textbf{\textit{Evaluation Measures.}} To characterize the degree of overfitting, we use the test loss, as it serves as a reliable indicator of the model's generalization capability to unseen data. Additionally, we evaluate the predictive performance of each model using the test accuracy.

\smallskip
\noindent\textbf{\textit{Baselines.}} We compare CF-Reg with the following regularization techniques: L1 (``Lasso''), L2 (``Ridge''), and Dropout.

\smallskip
\noindent\textbf{\textit{Configurations.}}
For each model, we adopt specific configurations as follows.
\begin{itemize}
\item \textit{Logistic Regression:} To induce overfitting in the model, we artificially increase the dimensionality of the data beyond the number of training samples by applying a polynomial feature expansion. This approach ensures that the model has enough capacity to overfit the training data, allowing us to analyze the impact of our counterfactual regularizer. The degree of the polynomial is chosen as the smallest degree that makes the number of features greater than the number of data.
\item \textit{Neural Networks (MLP and CNN):} To take advantage of the closed-form solution for computing the optimal perturbation vector as defined in (\ref{eq:opt-delta}), we use a local linear approximation of the neural network models. Hence, given an instance $\inst_i$, we consider the (optimal) counterfactual not with respect to $\model$ but with respect to:
\begin{equation}
\label{eq:taylor}
    \model^{lin}(\inst) = \model(\inst_i) + \nabla_{\inst}\model(\inst_i)(\inst - \inst_i),
\end{equation}
where $\model^{lin}$ represents the first-order Taylor approximation of $\model$ at $\inst_i$.
Note that this step is unnecessary for Logistic Regression, as it is inherently a linear model.
\end{itemize}

\smallskip
\noindent \textbf{\textit{Implementation Details.}} We run all experiments on a machine equipped with an AMD Ryzen 9 7900 12-Core Processor and an NVIDIA GeForce RTX 4090 GPU. Our implementation is based on the PyTorch Lightning framework. We use stochastic gradient descent as the optimizer with a learning rate of $\eta = 0.001$ and no weight decay. We use a batch size of $128$. The training and test steps are conducted for $6000$ epochs on the \textit{Water Potability} and \textit{Phoneme} datasets, while for the \textit{CIFAR-10} dataset, they are performed for $200$ epochs.
Finally, the contribution $w_i^{\varepsilon}$ of each training point $\inst_i$ is uniformly set as $w_i^{\varepsilon} = 1~\forall i\in \{1,\ldots,m\}$.

The source code implementation for our experiments is available at the following GitHub repository: \url{https://anonymous.4open.science/r/COCE-80B4/README.md} 

\subsection{RQ1: Counterfactual Perturbation vs. Overfitting}
To address \textbf{RQ1}, we analyze the relationship between the test loss and the average $L_2$-norm of the counterfactual perturbation vectors ($\overline{||\perturb||}$) over training epochs.

In particular, Figure~\ref{fig:delta_loss_epochs} depicts the evolution of $\overline{||\perturb||}$ alongside the test loss for an MLP trained \textit{without} regularization on the \textit{Water Potability} dataset. 
\begin{figure}[ht]
    \centering
    \includegraphics[width=0.85\linewidth]{img/delta_loss_epochs.png}
    \caption{The average counterfactual perturbation vector $\overline{||\perturb||}$ (left $y$-axis) and the cross-entropy test loss (right $y$-axis) over training epochs ($x$-axis) for an MLP trained on the \textit{Water Potability} dataset \textit{without} regularization.}
    \label{fig:delta_loss_epochs}
\end{figure}

The plot shows a clear trend as the model starts to overfit the data (evidenced by an increase in test loss). 
Notably, $\overline{||\perturb||}$ begins to decrease, which aligns with the hypothesis that the average distance to the optimal counterfactual example gets smaller as the model's decision boundary becomes increasingly adherent to the training data.

It is worth noting that this trend is heavily influenced by the choice of the counterfactual generator model. In particular, the relationship between $\overline{||\perturb||}$ and the degree of overfitting may become even more pronounced when leveraging more accurate counterfactual generators. However, these models often come at the cost of higher computational complexity, and their exploration is left to future work.

Nonetheless, we expect that $\overline{||\perturb||}$ will eventually stabilize at a plateau, as the average $L_2$-norm of the optimal counterfactual perturbations cannot vanish to zero.

% Additionally, the choice of employing the score-based counterfactual explanation framework to generate counterfactuals was driven to promote computational efficiency.

% Future enhancements to the framework may involve adopting models capable of generating more precise counterfactuals. While such approaches may yield to performance improvements, they are likely to come at the cost of increased computational complexity.


\subsection{RQ2: Counterfactual Regularization Performance}
To answer \textbf{RQ2}, we evaluate the effectiveness of the proposed counterfactual regularization (CF-Reg) by comparing its performance against existing baselines: unregularized training loss (No-Reg), L1 regularization (L1-Reg), L2 regularization (L2-Reg), and Dropout.
Specifically, for each model and dataset combination, Table~\ref{tab:regularization_comparison} presents the mean value and standard deviation of test accuracy achieved by each method across 5 random initialization. 

The table illustrates that our regularization technique consistently delivers better results than existing methods across all evaluated scenarios, except for one case -- i.e., Logistic Regression on the \textit{Phomene} dataset. 
However, this setting exhibits an unusual pattern, as the highest model accuracy is achieved without any regularization. Even in this case, CF-Reg still surpasses other regularization baselines.

From the results above, we derive the following key insights. First, CF-Reg proves to be effective across various model types, ranging from simple linear models (Logistic Regression) to deep architectures like MLPs and CNNs, and across diverse datasets, including both tabular and image data. 
Second, CF-Reg's strong performance on the \textit{Water} dataset with Logistic Regression suggests that its benefits may be more pronounced when applied to simpler models. However, the unexpected outcome on the \textit{Phoneme} dataset calls for further investigation into this phenomenon.


\begin{table*}[h!]
    \centering
    \caption{Mean value and standard deviation of test accuracy across 5 random initializations for different model, dataset, and regularization method. The best results are highlighted in \textbf{bold}.}
    \label{tab:regularization_comparison}
    \begin{tabular}{|c|c|c|c|c|c|c|}
        \hline
        \textbf{Model} & \textbf{Dataset} & \textbf{No-Reg} & \textbf{L1-Reg} & \textbf{L2-Reg} & \textbf{Dropout} & \textbf{CF-Reg (ours)} \\ \hline
        Logistic Regression   & \textit{Water}   & $0.6595 \pm 0.0038$   & $0.6729 \pm 0.0056$   & $0.6756 \pm 0.0046$  & N/A    & $\mathbf{0.6918 \pm 0.0036}$                     \\ \hline
        MLP   & \textit{Water}   & $0.6756 \pm 0.0042$   & $0.6790 \pm 0.0058$   & $0.6790 \pm 0.0023$  & $0.6750 \pm 0.0036$    & $\mathbf{0.6802 \pm 0.0046}$                    \\ \hline
%        MLP   & \textit{Adult}   & $0.8404 \pm 0.0010$   & $\mathbf{0.8495 \pm 0.0007}$   & $0.8489 \pm 0.0014$  & $\mathbf{0.8495 \pm 0.0016}$     & $0.8449 \pm 0.0019$                    \\ \hline
        Logistic Regression   & \textit{Phomene}   & $\mathbf{0.8148 \pm 0.0020}$   & $0.8041 \pm 0.0028$   & $0.7835 \pm 0.0176$  & N/A    & $0.8098 \pm 0.0055$                     \\ \hline
        MLP   & \textit{Phomene}   & $0.8677 \pm 0.0033$   & $0.8374 \pm 0.0080$   & $0.8673 \pm 0.0045$  & $0.8672 \pm 0.0042$     & $\mathbf{0.8718 \pm 0.0040}$                    \\ \hline
        CNN   & \textit{CIFAR-10} & $0.6670 \pm 0.0233$   & $0.6229 \pm 0.0850$   & $0.7348 \pm 0.0365$   & N/A    & $\mathbf{0.7427 \pm 0.0571}$                     \\ \hline
    \end{tabular}
\end{table*}

\begin{table*}[htb!]
    \centering
    \caption{Hyperparameter configurations utilized for the generation of Table \ref{tab:regularization_comparison}. For our regularization the hyperparameters are reported as $\mathbf{\alpha/\beta}$.}
    \label{tab:performance_parameters}
    \begin{tabular}{|c|c|c|c|c|c|c|}
        \hline
        \textbf{Model} & \textbf{Dataset} & \textbf{No-Reg} & \textbf{L1-Reg} & \textbf{L2-Reg} & \textbf{Dropout} & \textbf{CF-Reg (ours)} \\ \hline
        Logistic Regression   & \textit{Water}   & N/A   & $0.0093$   & $0.6927$  & N/A    & $0.3791/1.0355$                     \\ \hline
        MLP   & \textit{Water}   & N/A   & $0.0007$   & $0.0022$  & $0.0002$    & $0.2567/1.9775$                    \\ \hline
        Logistic Regression   &
        \textit{Phomene}   & N/A   & $0.0097$   & $0.7979$  & N/A    & $0.0571/1.8516$                     \\ \hline
        MLP   & \textit{Phomene}   & N/A   & $0.0007$   & $4.24\cdot10^{-5}$  & $0.0015$    & $0.0516/2.2700$                    \\ \hline
       % MLP   & \textit{Adult}   & N/A   & $0.0018$   & $0.0018$  & $0.0601$     & $0.0764/2.2068$                    \\ \hline
        CNN   & \textit{CIFAR-10} & N/A   & $0.0050$   & $0.0864$ & N/A    & $0.3018/
        2.1502$                     \\ \hline
    \end{tabular}
\end{table*}

\begin{table*}[htb!]
    \centering
    \caption{Mean value and standard deviation of training time across 5 different runs. The reported time (in seconds) corresponds to the generation of each entry in Table \ref{tab:regularization_comparison}. Times are }
    \label{tab:times}
    \begin{tabular}{|c|c|c|c|c|c|c|}
        \hline
        \textbf{Model} & \textbf{Dataset} & \textbf{No-Reg} & \textbf{L1-Reg} & \textbf{L2-Reg} & \textbf{Dropout} & \textbf{CF-Reg (ours)} \\ \hline
        Logistic Regression   & \textit{Water}   & $222.98 \pm 1.07$   & $239.94 \pm 2.59$   & $241.60 \pm 1.88$  & N/A    & $251.50 \pm 1.93$                     \\ \hline
        MLP   & \textit{Water}   & $225.71 \pm 3.85$   & $250.13 \pm 4.44$   & $255.78 \pm 2.38$  & $237.83 \pm 3.45$    & $266.48 \pm 3.46$                    \\ \hline
        Logistic Regression   & \textit{Phomene}   & $266.39 \pm 0.82$ & $367.52 \pm 6.85$   & $361.69 \pm 4.04$  & N/A   & $310.48 \pm 0.76$                    \\ \hline
        MLP   &
        \textit{Phomene} & $335.62 \pm 1.77$   & $390.86 \pm 2.11$   & $393.96 \pm 1.95$ & $363.51 \pm 5.07$    & $403.14 \pm 1.92$                     \\ \hline
       % MLP   & \textit{Adult}   & N/A   & $0.0018$   & $0.0018$  & $0.0601$     & $0.0764/2.2068$                    \\ \hline
        CNN   & \textit{CIFAR-10} & $370.09 \pm 0.18$   & $395.71 \pm 0.55$   & $401.38 \pm 0.16$ & N/A    & $1287.8 \pm 0.26$                     \\ \hline
    \end{tabular}
\end{table*}

\subsection{Feasibility of our Method}
A crucial requirement for any regularization technique is that it should impose minimal impact on the overall training process.
In this respect, CF-Reg introduces an overhead that depends on the time required to find the optimal counterfactual example for each training instance. 
As such, the more sophisticated the counterfactual generator model probed during training the higher would be the time required. However, a more advanced counterfactual generator might provide a more effective regularization. We discuss this trade-off in more details in Section~\ref{sec:discussion}.

Table~\ref{tab:times} presents the average training time ($\pm$ standard deviation) for each model and dataset combination listed in Table~\ref{tab:regularization_comparison}.
We can observe that the higher accuracy achieved by CF-Reg using the score-based counterfactual generator comes with only minimal overhead. However, when applied to deep neural networks with many hidden layers, such as \textit{PreactResNet-18}, the forward derivative computation required for the linearization of the network introduces a more noticeable computational cost, explaining the longer training times in the table.

\subsection{Hyperparameter Sensitivity Analysis}
The proposed counterfactual regularization technique relies on two key hyperparameters: $\alpha$ and $\beta$. The former is intrinsic to the loss formulation defined in (\ref{eq:cf-train}), while the latter is closely tied to the choice of the score-based counterfactual explanation method used.

Figure~\ref{fig:test_alpha_beta} illustrates how the test accuracy of an MLP trained on the \textit{Water Potability} dataset changes for different combinations of $\alpha$ and $\beta$.

\begin{figure}[ht]
    \centering
    \includegraphics[width=0.85\linewidth]{img/test_acc_alpha_beta.png}
    \caption{The test accuracy of an MLP trained on the \textit{Water Potability} dataset, evaluated while varying the weight of our counterfactual regularizer ($\alpha$) for different values of $\beta$.}
    \label{fig:test_alpha_beta}
\end{figure}

We observe that, for a fixed $\beta$, increasing the weight of our counterfactual regularizer ($\alpha$) can slightly improve test accuracy until a sudden drop is noticed for $\alpha > 0.1$.
This behavior was expected, as the impact of our penalty, like any regularization term, can be disruptive if not properly controlled.

Moreover, this finding further demonstrates that our regularization method, CF-Reg, is inherently data-driven. Therefore, it requires specific fine-tuning based on the combination of the model and dataset at hand.
%Placeholder for general introduction of the Accelerators and Applications theme sections. Themes of applications include machine-learning, bioinformatics, space applications, radio astronomy and weather simulations. Some of these references will overlap with other sections, e.g. when contributions are made on applying effective distributed computing for the purpose of weather forecasting.

FPGAs have emerged as powerful accelerators for a wide range of applications. In this section, we discuss FPGA-based solutions in machine learning (Section~\ref{sec:ml}), astronomy (Section~\ref{sec:astr}), particle physics experiments (Section~\ref{sec:phys}), quantum computing (Section~\ref{sec:quant}), space applications (Section~\ref{sec:space}), and bioinformatics (Section~\ref{sec:bio}).

\subsection{Machine learning}
\label{sec:ml}
% Three main parts, adapting an existing ML approach to hardware, designing hardware to accelerate an existing ML approach, (co-)design hardware for exotic ML approach.
% Main categories of evaluation are throughput, power, hardware area / resources, accuracy.
% \begin{itemize}
%     \item Accelerating existing ML models with new hardware design
%         \begin{itemize}
%             \item CNN acceleration (5)
%             \item TPU (1)
%             \item Benchmarking FPGAs (1)
%     \item Co-design existing ML models to hardware accelerate
%         \begin{itemize}
%             \item Pruning
%             \item Quantization / fixed point
%             \item Weight sharing
%             \item NAS adaptive to hardware
%         \end{itemize}
%     \item Design new hardware for exotic ML model
%         \begin{itemize}
%             \item Spiking / neuromorphic (7)
%             \item Bayesian (1)
%             \item Oscillating (2)
%         \end{itemize}
%     \end{itemize}
%     \item Hardware for 
% \end{itemize}
% \subsubsection{Background}
In the field of machine learning, and in particular deep learning, hardware acceleration plays a vital role. GPUs are the predominant method for hardware acceleration due to their high parallelism, but FPGA research is showing promising results. FPGAs enable inference at greater speed and better power efficiency when compared to GPUs \cite{hw-efficiency-compare} by designing model-specific accelerated pipelines \cite{ml-energy-efficient-cnn}. Through the co-design of machine learning models and machine learning hardware on FPGAs, models are accelerated without compromising on performance metrics and utilizing limited FPGA resources. In addition, the flexibility of the FPGA's architecture enables the realization of unconventional deep learning technology, such as Spiking Neural Networks (SNNs). 
%These networks can operate on a fraction of the power required by conventional networks on CPU or GPU.

%\subsubsection{Research topics}
\paragraph{Hardware acceleration} Ample research on hardware acceleration focuses on accelerating existing neural network architectures. One common class of architectures is convolutional neural networks (CNNs), which learn image filters in order to identify abstract image features. CNNs are often deployed in embedded applications which require real-time image processing and low energy consumption, making FPGAs a suitable candidate for CNN acceleration. \citet{ml-energy-efficient-cnn} propose an implementation of the LeNet architecture using Vitis HLS, pipelining the CNN layers, and outperforms other FPGA based implementations at a processing time of $70 \mu s$. One downside to this approach is the inflexibility of designing a specific model architecture in HLS which can be resolved by using partial reconfiguration \cite{ml-cnn-acclr-part-reconf}. To increase CNN throughput, further parallelization can be exploited, and in combination with the use of the high bandwidth OpenCAPI interface, can achieve a latency of less than $10 \mu s$ on the LeNet-5 model, streaming data from an HDMI interface \cite{ml-FPQNet}. In each of these implementations, fully pipelined CNNs are possible due to the limited number of parameters in small CNNs. As larger pipelined networks are deployed on FPGAs, parallelization puts strain on the available resources, and in particular the amount of on-chip-memory becomes a bottleneck. A proposed solution to this is using Frequency Compensated Memory Packing \cite{ml-mem-efficient-df-inf}.

In addition to CNN acceleration, general neural network acceleration has been developed by means of a programmable Tensor Processing Unit (TPU) as an overlay on an FPGA accelerator \cite{ml-agile-tuned-tpu}. Deep learning acceleration using FPGAs is also relevant to space technology research. Since the reprogrammability of FPGAs make them a suitable contender for deployment on space missions, FPGA implementations of existing deep learning models are being benchmarked for space applications \cite{ml-myriad-2-space-cnn} \cite{ml-mem-efficient-df-inf}.

\paragraph{Spiking neural networks} Spiking Neural Networks (SNNs) are computational models formed using spiking neuronal units that operate in parallel and mimic the basic operational principles of biological systems. These features endow SNNs with potentially richer dynamics than traditional artificial neural network models based on the McCulloch-Pitts point neurons or simple ReLU activation functions that do not incorporate timing information. Thus, SNNs excel in handling temporal information streams and are well-suited for innovative non-von-Neumann computer architectures, which differ from traditional sequential processing systems. SNNs are particularly well-suited for implementation in FPGAs due to their massive parallelism and requirement for significant on-chip memories with high-memory bandwidth for storing neuron states and synaptic weights. Additionally, SNNs use sparse binary communication, which is beneficial for low-latency operations because both computing and memory updates are triggered by events. FPGAs' inherent flexibility allows for reprogramming and customization, which enable reprogrammable SNNs in FPGAs, resulting in flexible, efficient, and low-latency systems~\cite{Corradi2021Gyro:Analytics,Irmak2021ADesigns,SankaranAnInference}. \citet{corradi2024accelerated} demonstrated the application of a Spiking Convolutional Neural Network (SCNN) to population genomics. The SCNN architecture achieved comparable classification accuracy to state-of-the-art CNNs while processing only about 59.9\% of the input data, reaching 97.6\% of CNN accuracy for classifying selective-sweep and recombination-hotspot genomic regions. This was enabled by % success is attributed to 
the SCNN's capability to temporize genetic information, allowing it to produce classification outputs without processing the entire genomic input sequence. Additionally, when implemented on FPGA hardware, the SCNN model exhibited over three times higher throughput and more than 100 times greater energy efficiency than a GPU implementation, markedly enhancing the processing of large-scale population genomics datasets.


\paragraph{Model/hardware co-design} Previous examples demonstrate that existing deep neural network models can be accelerated using FPGAs. Typically, research in this area focuses on designing an optimal hardware solution for an existing model. A more effective approach, however, is to co-design the model and the hardware accelerator simultaneously. However, simultaneous co-design of DNN models and accelerators is challenging. DNN designers often need more specialized knowledge to consider hardware constraints, while hardware designers may need help to maintain the quality and accuracy of DNN models. Furthermore, efficiently exploring the extensive co-design space is a significant challenge. This co-design methodology leads to better performance, leveraging FPGAs' flexibility and rapid prototyping capabilities. For example, \citet{Rocha2020BinaryWrist-PPG}, by co-designing the bCorNET framework, which combines binary CNNs and LSTMs, they were able to create an efficient hardware accelerator that processes HR estimation from PPG signals in real-time. The pipelined architecture and quantization strategies employed allowed for significant reductions in memory footprint and computational complexity, enabling real-time processing with low latency.

In SNNs, encoding information in spike streams is a crucial co-design aspect. SNNs primarily use two encoding strategies: rate-coding and time-to-first-spike (TTFS) coding. Rate coding is common in SNN models, encoding information based on the instantaneous frequency of spike streams. Higher spike frequencies result in higher precision but at the cost of increased energy consumption due to frequent spiking. While rate coding offers accuracy, it reduces sparsity. In FPGA implementations, rate coding is often used for its robustness, simplicity, ease of training through the conversion of analog neural networks to spiking neural networks, and practicality in multi-sensor data fusion, where it helps represent real values from various sensors (radars, cameras) even in the presence of jitter or imperfect synchronization~\cite{Corradi2021Gyro:Analytics}.
Conversely, TTFS coding has been demonstrated in SNNs implemented on FPGAs to enhance sparsity and has the potential of reducing energy consumption by encoding information in spike timing. For instance, Pes et al.~\cite{Pes2024ActiveNetworks} introduced a novel SNN model with active dendrites to address catastrophic forgetting in sequential learning tasks. Active dendrites enable the SNN to dynamically select different sub-networks for different tasks, improving continual learning and mitigating catastrophic forgetting. This model was implemented on a Xilinx Zynq-7020 SoC FPGA, demonstrating practical viability with a high accuracy of 80\% and an average inference time of 37.3 ms, indicating significant potential for real-world deployment in edge devices.

%To overcome this challenges, Cong et al in \textcolor{red}{\textcolor{red}{~\cite{FPGA/DNN Co-Design: An Efficient Design Methodology for IoT
%Intelligence on the Edge}}} introduced a co-design methodology for FPGAs and DNNs that integrates both bottom-up and top-down approaches, in which a bottom-up search for DNN models that prioritize high accuracy is paired with a top-down design of FPGA accelerators tailored to the specific characteristics of DNNs.
%Other methods leverage an automatic toolchain comprising  auto-DNN engine for hardware-aware DNN model optimization and an auto-HLS engine to generate FPGA-suitable synthesizable code, or hardware-aware Neural Architecture Search (NAS). When co-design is applied, it typicaly produces DNN models and FPGA accelerators that outperform state-of-the-art FPGA designs in various metrics, including accuracy, speed, power consumption, and energy efficiency \textcolor{red}{~\cite{When Neural Architecture Search Meets Hardware Implementation: from Hardware Awareness to Co-Design}.

%\textcolor{blue}{Co-design is critical in developing FPGA-based systems, merging hardware and software engineering from the initial design stages. This integrated method is essential for optimizing system performance, functionality, and cost-effectiveness. Co-design leverages the adaptable nature of FPGAs, tailoring the computing workload to meet specific hardware needs and adjusting the hardware to suit software demands. This synergy results in improved system performance and greater energy efficiency.}
%\textcolor{blue}{Many co-design examples exists in literature that demonstrate how clever distributed memory layouts can results in increased performances~\cite{}. }

%\paragraph{Novel hardware architecture} 

%\textcolor{blue}{Modern co-design methodologies allow the generation of hardware architectures and applications for advanced Reconfigurable Acceleration Devices (RAD) that go beyond traditional FPGA capabilities. These devices integrate FPGA fabric with other components like general-purpose processors, specialized accelerators, and high-performance networks-on-chip (NoCs) within a system-in-package framework. This integration enables complex data center applications to be handled more efficiently than conventional FPGAs. In particular, Boutrous et al in \cite{Architecture and Application Co-Design for Beyond-FPGA Reconfigurable Acceleration Devices} introduce RAD-Sim, an architecture simulator, to aid in the design space exploration of RADs. This allows for the study of interactions between different system components. Notably, they demonstrated mapping deep learning FPGA overlays to different RAD configurations, demonstrating how RAD-Sim can guide the adaptation of applications to exploit the novel features of RADs effectively.}


\subsection{Astronomy}
\label{sec:astr}
%\subsubsection{Introduction}

Astronomy is the study of everything in the universe beyond our Earth's atmosphere. Observations are done at different modalities and wavelengths, such as detection of a range of different particles (e.g., Cherenkov detector based systems such as KM3NeT \cite{KM3NeT:2009xxi}), gravitational waves, optical observations, gamma and x-ray observations and radio (e.g., WSRT \cite{van_Cappellen_2022}, LOFAR \cite{van_Haarlem_2013}, SKA \cite{book-SKA}). Observations can be done from space or from earth; in this section, we limit the scope to ground-based astronomy. A common denominator for instruments required for observation of the different modalities and different wave lengths is that the systems need to be very sensitive in order to observe very faint signals from outside the Earth's atmosphere. Instruments are typically large and/or distributed over a large area %in order 
to achieve %reach 
good sensitivity and resolution. Different modalities and wavelengths require distinct types of sensors, cameras, or antennas to convert observed phenomena into electrical signals. Each system is tailored to its specific modality and wavelength, necessitating specialized components to accurately capture and translate the data. %At different modalities and different wave lengths, the systems each require different kinds of sensors, camera's or antenna's that convert the observed phenomenon to an electrical signal. 
%The electrical signal is at some point in the signal chain converted to the digital domain and processed in various stages into an end product used by scientists. 
At a certain stage in the signal chain, the electrical signal is converted into the digital domain, where it undergoes multiple processing stages. This processed signal ultimately results in an end product that can be utilized by scientists for analysis and research purposes.
Systems can roughly be split into two parts, a front-end and a back-end. The front-end requires interfacing with and processing of data from the sensor; electronics commonly deployed in the front-end are constrained in space (size), temperature, power, cost, RFI, environmental conditions and serviceability. The back-end processes data produced by the front-end(s) either in an online or offline fashion, which is usually %typically be 
done with server infrastructure in a data center. % environment, either on site or centralized. 
In the back-end, the main challenges are the high data bandwidth and large data size coming from the front-ends. Although the environment is more flexible, systems are still constrained in space, power, and cost.

%\subsubsection{Background}

FPGAs have been used in astronomy instrumentation for quite some time, as they 
%FPGAs have since long found applications in astronomy instrumentation. 
%Typically FPGA's 
are %very 
efficient in interfacing with Analog to Digital Converters (ADCs), and well suited to the conditions faced in instrumentation front-ends (e.g. NCLE \cite{karapakula2024ncle}). Moreover, FPGA are also used further down the processing stages for various signal processing operations, both in the front-ends (e.g., Uniboard2 in LOFAR \cite{doi:10.1142/S225117171950003X}) as well as in the back-ends of systems (e.g., MeerKAT \cite{2022JATIS...8a1006V} and SKA \cite{SKA-CBF}). GPUs represent a good alternative in back-end processing (e.g., LOFAR's system COBALT \cite{Broekema_2018}) as well. The work by Veenboer et al. \cite{10.1007/978-3-030-29400-7_36} describes a trade-off between using a GPU and an FPGA accelerator in the implementation of an image processing operation in a radio telescope back-end.

%\subsubsection{Research topics}
%Dutch academia has contributed to several astronomy instruments:
%Often large international consortia, not immidiately clear what the role of the Dutch partners was. But also some work which is mainly done by Dutch institutes
\paragraph{Hardware Development for the Radio Neutrino Observatory in Greenland (RNO-G)}
The RNO-G \cite{Smith2022HardwareRNO-G} is a radio detection array for neutrinos. It consists of 35 autonomous stations deployed over a $5 \times 6$ km grid near the NSF Summit Station
in Greenland. Each station includes an FPGA-based phased trigger. The station has to operate in a 25 W power envelope. The implementation on FPGA seems to be favorable due to environmental conditions and operation constraints.

\paragraph{Implementation of a Correlator onto a Hardware Beam-Former to Calculate Beam-Weights}
The Apertif Phased Array Feed (PAF) \cite{van_Cappellen_2022} is a radio telescope front-end used in the WSRT system in the Netherlands. FPGAs are used for antenna read out as well as signal processing close to the antenna. Schoonderbeek et al. \cite{Schoonderbeek2020ImplementationBeam-Weights} describe the transformation and implementation of a beamformer algorithm on FPGA in order to build a more efficient system.

\paragraph{Near Memory Acceleration and Reduced-Precision Acceleration for High Resolution Radio Astronomy Imaging}
\citet{Corda2020NearImaging} describe the implementation of a 2D FFT on FPGA, leveraging Near-Memory Computing. The 2D FFT is applied to an image processing implementation on FPGA in the back-end of a radio telescope and compared with implementations on CPU and GPU. \citet{Corda2022Reduced-PrecisionHardware} explore %the concept of 
reduced-precision computation on an FPGA %is explored 
for the same image processing application. %They propose an implementation on an FPGA accelerator and compare with an implementation on CPU and GPU.

\paragraph{The MUSCAT Readout Electronics Backend: Design and Pre-deployment Performance}
The MUSCAT is a large single dish radio telescope with 1458 receives in the focal plane. The system uses FPGA based electronics to read out and pre-process the data from the receivers \cite{Rowe2023ThePerformance}. %\emph{Electronics Backend} in this case relates to the electronics close to the antenna, referred to as front-end in our description here.

% Small contribution by NL through SRON

\paragraph{Cherenkov Telescope Arrays}
Three different contributions have been made to three different Cherenkov based Telescope Arrays.
%\paragraph{A NECTAr-based upgrade for the Cherenkov cameras of the H.E.S.S. 12-meter telescopes}
Ashton et al.~\cite{Ashton2020ATelescopes} describe a system for the High Energy Stereoscopic System (H.E.S.S.) where a custom board with ARM CPU and an FPGA is used to read out and pre-process a custom designed NECTAr digitizer chip in the front-end of the system. After pre-processing, the data is distributed to a back-end over Ethernet.
%Anton Pannekoek Institute for Astronomy
%\paragraph{A White Rabbit-Synchronized Accurate Time-Stamping Solution for the Small-Sized Cameras of the Cherenkov Telescope Array}
Sánchez-Garrido et al.~\cite{Sanchez-Garrido2021AArray} present the design of a Zynq FPGA SoC based platform for White Rabbit time synchronization in the ZEN-CTA telescope array front-ends. Data captured and pre-processed at the front-ends is distributed over Ethernet to the back-end including the time stamp.
%\paragraph{Architecture and performance of the KM3NeT front-end firmware}
Aiello et al.~\cite{Aiello2021ArchitectureFirmware} outline the architecture and performance of the KM3Net front-end firmware. The KM3NeT telescope consists of two deep-sea three-dimensional sensor grids being deployed in the Mediterranean Sea. A central logic board with FPGA in the front-end serves as a Time to Digital Converter to record events and time at the sensors; the data is transmitted and further processed in a back-end on shore.
%S. Aiello et al., “KM3NeT front-end and readout electronics system:
%hardware, firmware, and software,” J. Astronomical Telescopes Inst.
%Syst., vol. 5, no. 4, pp. 1–15, 2019.

%\subsubsection{Future direction}

\vspace{0.4cm}
%In the works included in this survey, 
FPGA are mainly used for front-end sensor interfacing and pre-processing. \citet{Corda2020NearImaging, Corda2022Reduced-PrecisionHardware} underline that FPGAs are also still relevant in the back-end, providing improved performance over CPU and on-par performance with GPU accelerators. FPGA are expected to remain the dominant choice for platforms in astronomy instrumentation front-ends due to the strong interfacing capabilities and the adaptability and suitability to the constraints imposed by instrumentation front-ends. In the back-end, FPGAs provide a viable solution to application acceleration, but will have to compete with other accelerator architectures, e.g., GPUs~\cite{10.1007/978-3-030-29400-7_36}. 
An emerging new technology are the Artificial Intelligence Engines in the AMD Versal Adaptive SoC. The work from \citet{Versal-ACAP} evaluated the AI Engines for a signal processing application in radio astronomy. The flexibility and programmability of the AI Engines, combined with the interfacing capabilities of the FPGA can lead to a powerful platform for telescope front-ends.

\subsection{Particle physics experiments}
\label{sec:phys}
The Large Hadron Collider (LHC) features various particle accelerators to facilitate particle physics experiments. Experiments performed using particle accelerators can produce massive amounts of data that needs to be propagated and preprocessed at high speeds before the reduced relevant data is recorded for offline storage. FPGAs are widely employed throughout systems LHC particle accelerators, such as ATLAS and LHCb, for their high-bandwidth capabilities, and the flexibility that reconfigurable hardware offers without requiring hardware alterations to the system. Recently both the ATLAS and LHCb particle accelerators have been commissioned for upgrades. The Dutch Institute for Subatomic Physics (Nikhef) is one of the collaborating institutes working on the LHC accelerators.

%\subsubsection{Research topics}
%\paragraph{TODO - revise text into research topics}

LHCb is a particle accelerator that specializes in experiments that study the bottom quark. Major upgrades to the LHCb that enable handling a higher collision rate require new front-end and back-end electronics. To facilitate the back-end of the upgrade, the LHCb implements the custom PCIe40 board, which features an Intel Arria 10 FPGA. Four PCIe40 boards are dedicated for controlling part of the LHCb system, and 52 PCIe40 boards are used to read out each of the detector’s slices, producing an aggregated data rate of 2.85 Tb/s \cite{FernandezPrieto2020PhaseExperiment}.

ATLAS is one of the general particle accelerators of the LHC. ATLAS uses two trigger stages in order to record only the particle interactions of interest. In an upgrade to the ATLAS accelerator, ASIC-based calorimeter trigger preprocessor boards are replaced by FPGA-based hardware. Using FPGAs for this purpose allows implementing enhanced signal processing methods \cite{Aad2020PerformanceTrigger}. After the two trigger stages, FPGAs are deployed to process the triggers for tracking particles \cite{Aad2021TheSystem}.

Ongoing upgrades to the LHC particle accelerators, referred to as High Luminosity LHC (HL-LHC), will facilitate higher energy collisions. HL-LHC will produce increased background rates. To reduce false triggers due to background, the New Small Wheel checks for coinciding hits. Each trigger processor features Virtex-7, Kintex Ultrascale and Zynq FPGAs \cite{Iakovidis2023TheElectronics}. Interaction to and from front-end hardware is done through Front-End Link eXchange (FELiX) boards. As part of the HL-LHC upgrades, each FELiX board must facilitate a maximum throughput of 200 Gbps. To enable this, Remote Direct Memory Access (RDMA) over Converged Ethernet (RoCE) as part of the FELiX FPGA system is proposed \cite{Vasile2022FPGALHC, Vasile2023IntegrationLHC}. The performance of the FELiX upgrade in combination with an upgraded Software ReadOut Driver (SW ROD) satisfies the data transfer requirements of the upgraded ATLAS system \cite{Gottardo2020FEliXSystem}.




\subsection{Quantum computing}
\label{sec:quant}
Quantum computing promises to help solving many global challenges of our time such as quantum chemistry problems to design new medicines, the prediction of material properties for efficient energy storage, and the handling of big data needed for complex climate physics~\cite{Gibney-nat-2014}. The most promising quantum algorithms demand systems comprising thousands to millions of quantum bits~\cite{Meter-2013}, the quantum counterpart of a classical bit. A quantum processor comprising up to 50 qubits has been realized using solid-state superconducting qubits~\cite{Arute-nat-2019}, but its operation requires a combination of cryogenic temperatures below ~100 mK and hundreds of coaxial lines for qubit control and readout. %Furthermore, 
While in systems with a few qubits, this can be controlled using off-the-shelf electronic equipment, such approach becomes infeasible when scaling qubit systems toward thousands or millions of qubits that are required for a practical quantum computer. 

%\subsubsection{Research topics}

A means to tackle the foreseeable bottleneck in scaling the operation of qubit systems is to integrate FPGA technology in the control and readout of solid-state qubits. FPGAs have been used to generate highly-stable waveforms suitable for the control of quantum bits with latency significantly lower than software alternatives~\cite{Ireland-2020}. In systems of semiconductor spin qubits, FPGAs have provided in-hardware syncing of quantum dot control voltages with the signal acquisition and buffering and thus enabled the observation of real-time charge-tunneling events~\cite{Hartman-2023}. FPGAs have also been used to configure and synchronize a cryo-controller with an arbitrary waveform generator required to generate complex pulse shapes and perform quantum operations~\cite{Xue-nat-2021}. Such setup has enabled the demonstration of universal control of a quantum processor hosting six semiconductor spin qubits~\cite{Philips-nat-2022}. FPGAs have proven to be essential for implementing quantum error correction algorithms, which are critical for mitigating the effects of dephasing and decoherence in solid-state qubits. %FPGAs have also been shown to essential for the implementation of quantum error correction algorithms needed to mitigate the effects of dephasing and decoherence in solid-state qubits. 
In qubit systems based on superconducting quantum circuits, the first efficient demonstration of quantum error correction was made possible by a FPGA-controlled data acquisition system which provided dynamic real-time feedback on the evolution of the quantum system~\cite{Ofek-nat-2016}. It has been further predicted that FPGA can enable highly-efficient quantum error correction based on neural-network decoders~\cite{Overwater-2022}.

%\subsubsection{Future directions}

FPGA technology has proven invaluable in the development of the emerging research field of quantum computing.
However, the complexity of programming FPGA circuits hinders their implementation in quantum computing systems. Commercial efforts have been done toward providing graphical tools for designing FPGA programs, namely the Quantum Researchers Toolkit by Keysight Technologies and the FPGA-based multi-instrument platform Moku-Pro developed by Liquid Instruments. These tools are essential for implementing customized algorithms without the need for dedicated expertise in hardware description languages. Future research is also needed in integrating FPGAs in cryogenic platforms required to operate qubit systems. Such capability has already been demonstrated; commercial FPGAs can operate at temperatures below 4 K and be integrated in a cryogenic platform for qubit control~\cite{Homulle-2017}. These efforts provide evidence that FPGA technology is of great interest for enabling a scalable and practically applicable quantum computer. 


\subsection{Space}
\label{sec:space}
The flexibility of FPGA technology makes it a suitable platform for many applications on-board space missions. The European Space Research and Technology Centre (ESTEC), as part of the European Space Agency (ESA) actively explores FPGA technology for space applications, and has an extensive portfolio of FPGA Intellectual Property (IP) Cores~\cite{esa_ip}.

%\subsubsection{Research topics}

FPGAs can flexibly route its input and output ports, and can be configured to support many different communication protocols. This makes FPGAs good contenders as devices that communicate with the various hardware platforms and sensors on a space mission. FPGAs and have been implemented as interface devices in novel on-board machine learning and digital signal processing  implementations~\cite{Leon2021ImprovingSoC, Leon2021FPGABenchmarks, karapakula2024ncle}. 

An on-board task for which FPGAs are used is hyperspectral imaging. This type of on-board imaging produces vast amounts of data. To reduce transmission bandwidth requirements when transmitting the sensory data to earth, real-time on-board compression handling high data rates is required. FPGAs are well-suited for such tasks, and research has been done on using space-grade radiation-hardened FPGAs \cite{Barrios2020SHyLoCMissions} as well as commercial off-the-shelf (COTS) FPGAs \cite{Rodriguez2019ScalableCompression} for on-board hyperspectral image compression. COTS devices are generally cheaper than space-grade devices, but the higher susceptibility of these devices to radiation-induced effects makes them challenging to employ.

Communication between on-board systems often requires high data-rates and is susceptible to radiation induced effects. To deal with the unique constraints of space applications, dedicated communication protocols such as SpaceWire, and its successor, SpaceFibre have been developed. These protocols are available as FPGA IP implementations, and testing environments of SpaceFibre have been developed \cite{MystkowskaSimulationSpaceFibre, AnSection}. SpaceWire can interface with the common AXI4 protocol using a dedicated bridge \cite{RubattuASystems}, enabling its integration with SpaceWire interfaces. Direct Memory Access (DMA) allows peripherals to transfer data to and from an FPGA without going through a CPU. The application of DMA in space is being investigated, however its application as of now is limited since DMA is susceptibility to radiation-induced effects \cite{Portaluri2022Radiation-inducedDevices}.



\subsection{Bioinformatics}
\label{sec:bio}
FPGA technology has been extensively explored for accelerating Bioinformatics kernels. Bioinformatics is an interdisciplinary scientific field that combines biology, computer science, mathematics, and statistics to analyze and interpret biological data. The field primarily focuses on the development and application of methods, algorithms, and tools to handle, process, and analyze large sets of biological data, such as DNA sequences, protein structures, and gene expression patterns.Continuous advances in DNA sequencing technologies~\cite{hu2021next} have led to the rapid accumulation of biological data, creating an urgent need for high-performance computational solutions capable of efficiently managing increasingly larger datasets.

\citet{Shahroodi2022KrakenOnMem:Profiling} describe a hardware/software co-designed framework to accelerate and improve energy consumption of taxonomic profiling. In metagenomics, the main goal is to understand the role of each organism in our environment in order to
improve our quality of life, and taxonomic profiling involves the identification and categorization of the various types of organisms present in a biological sample by analyzing DNA or protein sequences from the sample to determine which species or taxa are represented. The study focuses on boosting performance of table lookup, which is the primary bottleneck in taxonomic profilers, by proposing a processing-in-memory hardware accelerator. Using large-scale simulations, the authors report an average of 63.1\% faster execution and orders of magnitude higher energy efficiency than the  widely used metagenomic analysis tool Kraken2~\cite{wood2019improved} executed on a 128-core server with AMD EPYC 7742 processors  operating at 2.25 GHz. An FPGA was used for prototyping and emulation purposes.

\citet{Corts2022AcceleratedFPGAs} employ FPGAs to accelerate the detection of traces of positive natural selection in genomes. The authors designed a hardware accelerator for the $\omega$ statistic~\cite{kim2004linkage}, which is extensively used in population genetics as an indicator of positive selection. In comparison with a single CPU core,
the FPGA accelerator can deliver up to $57.1\times$ faster
computation of the $\omega$ statistic, using the OmegaPlus~\cite{alachiotis2012omegaplus} software implementation as reference.


%\citet{Ahmad2022Communication-EfficientFlight}



\citet{Malakonakis2020ExploringRAxML} use FPGAs to accelerate the widely used phylogenetics software tool RAxML~\cite{stamatakis2014raxml}. The study implements the Phylogenetic Likelihood Function (PLF), which is used for evaluating phylogenetic trees, on a Xilinx ZCU102 development board and a cloud-based Amazon AWS EC2 F1 instance. The first system (ZCU102) can deploy two accelerator instances, operating at 250MHz, and delivers up to $7.7\times$ faster executions than sequential software execution on a AWS EC2 F1 instance. %Xeon processors. 
The AWS-based accelerated system is $5.2\times$ faster than the same software implementation. %In comparison with previous work by Alachiotis et al.~\cite{alachiotis2009exploring[7_12]}, the implementation on the Xilinx development board is about 2.35x faster. %, but the older technology should certainly be taken into consideration. 



\citet{Alachiotis2021AcceleratingCloud} also target the PLF implementation in RAxML, and propose an optimization method for data movement in PCI-attached accelerators using tree-search algorithms. They developed a software cache controller that leverages data dependencies between consecutive PLF calls to cache data on the accelerator card. In combination with double buffering over PCIe, this approach led to nearly $4\times$ improvement in the performance of an FPGA-based PLF accelerator. Executing the complete RAxML algorithm on an AWS EC2 F1 instance, the authors observed up to $9.2\times$ faster processing of protein data than a $2.7$ GHz Xeon processor in the same cloud environment.

With genomic datasets continuing to expand, bioinformatics analyses are likely to increasingly rely on cloud computing in the future. This shift will be supported by new programming models and frameworks designed to address the data-movement challenges posed by cloud-based hardware accelerators. These accelerators, such as FPGAs and GPUs, need data transfers from the host processor, which can significantly impact execution times and negate gains from computation improvements. Fortunately, similar data-movement concerns exist for both FPGAs and GPUs, and ongoing engineering efforts are likely to converge on common solutions~\cite{Corts2023AGenetics}. This will help bring optimized, hardware-accelerate processing techniques into more widespread use among computational biologists and bioinformaticians in the future.





\vspace{-0.2cm}
\section{Impact: Why Free Scientific Knowledge?}
\vspace{-0.1cm}

Historically, making knowledge widely available has driven transformative progress. Gutenberg’s printing press broke medieval monopolies on information, increasing literacy and contributing to the Renaissance and Scientific Revolution. In today's world, open source projects such as GNU/Linux and Wikipedia show that freely accessible and modifiable knowledge fosters innovation while ensuring creators are credited through copyleft licenses. These examples highlight a key idea: \textit{access to essential knowledge supports overall advancement.} 

This aligns with the arguments made by Prabhakaran et al. \cite{humanrightsbasedapproachresponsible}, who specifically highlight the \textbf{ human right to participate in scientific advancement} as enshrined in the Universal Declaration of Human Rights. They emphasize that this right underscores the importance of \textit{ equal access to the benefits of scientific progress for all}, a principle directly supported by our proposal for Knowledge Units. The UN Special Rapporteur on Cultural Rights further reinforces this, advocating for the expansion of copyright exceptions to broaden access to scientific knowledge as a crucial component of the right to science and culture \cite{scienceright}. 

However, current intellectual property regimes often create ``patently unfair" barriers to this knowledge, preventing innovation and access, especially in areas critical to human rights, as Hale compellingly argues \cite{patentlyunfair}. Finding a solution requires carefully balancing the imperative of open access with the legitimate rights of authors. As Austin and Ginsburg remind us, authors' rights are also human rights, necessitating robust protection \cite{authorhumanrights}. Shareable knowledge entities like Knowledge Units offer a potential mechanism to achieve this delicate balance in the scientific domain, enabling wider dissemination of research findings while respecting authors' fundamental rights.

\vspace{-0.2cm}
\subsection{Impact Across Sectors}

\textbf{Researchers:} Collaboration across different fields becomes easier when knowledge is shared openly. For instance, combining machine learning with biology or applying quantum principles to cryptography can lead to important breakthroughs. Removing copyright restrictions allows researchers to freely use data and methods, speeding up discoveries while respecting original contributions.

\textbf{Practitioners:} Professionals, especially in healthcare, benefit from immediate access to the latest research. Quick access to newer insights on the effectiveness of drugs, and alternative treatments speeds up adoption and awareness, potentially saving lives. Additionally, open knowledge helps developing countries gain access to health innovations.

\textbf{Education:} Education becomes more accessible when teachers use the latest research to create up-to-date curricula without prohibitive costs. Students can access high-quality research materials and use LM assistance to better understand complex topics, enhancing their learning experience and making high-quality education more accessible.

\textbf{Public Trust:} When information is transparent and accessible, the public can better understand and trust decision-making processes. Open access to government policies and industry practices allows people to review and verify information, helping to reduce misinformation. This transparency encourages critical thinking and builds trust in scientific and governmental institutions.

Overall, making scientific knowledge accessible supports global fairness. By viewing knowledge as a common resource rather than a product to be sold, we can speed up innovation, encourage critical thinking, and empower communities to address important challenges.

\vspace{-0.2cm}
\section{Open Problems}
\vspace{-0.1cm}

Moving forward, we identify key research directions to further exploit the potential of converting original texts into shareable knowledge entities such as demonstrated by the conversion into Knowledge Units in this work:


\textbf{1. Enhancing Factual Accuracy and Reliability:}  Refining KUs through cross-referencing with source texts and incorporating community-driven correction mechanisms, similar to Wikipedia, can minimize hallucinations and ensure the long-term accuracy of knowledge-based datasets at scale.

\textbf{2. Developing Applications for Education and Research:}  Using KU-based conversion for datasets to be employed in practical tools, such as search interfaces and learning platforms, can ensure rapid dissemination of any new knowledge into shareable downstream resources, significantly improving the accessibility, spread, and impact of KUs.

\textbf{3. Establishing Standards for Knowledge Interoperability and Reuse:}  Future research should focus on defining standardized formats for entities like KU and knowledge graph layouts \citep{lenat1990cyc}. These standards are essential to unlock seamless interoperability, facilitate reuse across diverse platforms, and foster a vibrant ecosystem of open scientific knowledge. 

\textbf{4. Interconnecting Shareable Knowledge for Scientific Workflow Assistance and Automation:} There might be further potential in constructing a semantic web that interconnects publicly shared knowledge, together with mechanisms that continually update and validate all shareable knowledge units. This can be starting point for a platform that uses all collected knowledge to assist scientific workflows, for instance by feeding such a semantic web into recently developed reasoning models equipped with retrieval augmented generation. Such assistance could assemble knowledge across multiple scientific papers, guiding scientists more efficiently through vast research landscapes. Given further progress in model capabilities, validation, self-repair and evolving new knowledge from already existing vast collection in the semantic web can lead to automation of scientific discovery, assuming that knowledge data in the semantic web can be freely shared.

We open-source our code and encourage collaboration to improve extraction pipelines, enhance Knowledge Unit capabilities, and expand coverage to additional fields.

\vspace{-0.2cm}
\section{Conclusion}
\vspace{-0.1cm}

In this paper, we highlight the potential of systematically separating factual scientific knowledge from protected artistic or stylistic expression. By representing scientific insights as structured facts and relationships, prototypes like Knowledge Units (KUs) offer a pathway to broaden access to scientific knowledge without infringing copyright, aligning with legal principles like German \S 24(1) UrhG and U.S. fair use standards. Extensive testing across a range of domains and models shows evidence that Knowledge Units (KUs) can feasibly retain core information. These findings offer a promising way forward for openly disseminating scientific information while respecting copyright constraints.

\section*{Author Contributions}

Christoph conceived the project and led organization. Christoph and Gollam led all the experiments. Nick and Huu led the legal aspects. Tawsif led the data collection. Ameya and Andreas led the manuscript writing. Ludwig, Sören, Robert, Jenia and Matthias provided feedback. advice and scientific supervision throughout the project. 

\section*{Acknowledgements}

The authors would like to thank (in alphabetical order): Sebastian Dziadzio, Kristof Meding, Tea Mustać, Shantanu Prabhat for insightful feedback and suggestions. Special thanks to Andrej Radonjic for help in scaling up data collection. GR and SA acknowledge financial support by the German Research Foundation (DFG) for the NFDI4DataScience Initiative (project number 460234259). AP and MB acknowledge financial support by the Federal Ministry of Education and Research (BMBF), FKZ: 011524085B and Open Philanthropy Foundation funded by the Good Ventures Foundation. AH acknowledges financial support by the Federal Ministry of Education and Research (BMBF), FKZ: 01IS24079A and the Carl Zeiss Foundation through the project "Certification and Foundations of Safe ML Systems" as well as the support from the International Max Planck Research School for Intelligent Systems (IMPRS-IS). JJ acknowledges funding by the Federal Ministry of Education and Research of Germany (BMBF) under grant no. 01IS22094B (WestAI - AI Service Center West), under grant no. 01IS24085C (OPENHAFM) and under the grant DE002571 (MINERVA), as well as co-funding by EU from EuroHPC Joint Undertaking programm under grant no. 101182737 (MINERVA) and from Digital Europe Programme under grant no. 101195233 (openEuroLLM) 

\bibliographystyle{plain}
\bibliography{bibliography}

\newpage
\appendix
\subsection{Lloyd-Max Algorithm}
\label{subsec:Lloyd-Max}
For a given quantization bitwidth $B$ and an operand $\bm{X}$, the Lloyd-Max algorithm finds $2^B$ quantization levels $\{\hat{x}_i\}_{i=1}^{2^B}$ such that quantizing $\bm{X}$ by rounding each scalar in $\bm{X}$ to the nearest quantization level minimizes the quantization MSE. 

The algorithm starts with an initial guess of quantization levels and then iteratively computes quantization thresholds $\{\tau_i\}_{i=1}^{2^B-1}$ and updates quantization levels $\{\hat{x}_i\}_{i=1}^{2^B}$. Specifically, at iteration $n$, thresholds are set to the midpoints of the previous iteration's levels:
\begin{align*}
    \tau_i^{(n)}=\frac{\hat{x}_i^{(n-1)}+\hat{x}_{i+1}^{(n-1)}}2 \text{ for } i=1\ldots 2^B-1
\end{align*}
Subsequently, the quantization levels are re-computed as conditional means of the data regions defined by the new thresholds:
\begin{align*}
    \hat{x}_i^{(n)}=\mathbb{E}\left[ \bm{X} \big| \bm{X}\in [\tau_{i-1}^{(n)},\tau_i^{(n)}] \right] \text{ for } i=1\ldots 2^B
\end{align*}
where to satisfy boundary conditions we have $\tau_0=-\infty$ and $\tau_{2^B}=\infty$. The algorithm iterates the above steps until convergence.

Figure \ref{fig:lm_quant} compares the quantization levels of a $7$-bit floating point (E3M3) quantizer (left) to a $7$-bit Lloyd-Max quantizer (right) when quantizing a layer of weights from the GPT3-126M model at a per-tensor granularity. As shown, the Lloyd-Max quantizer achieves substantially lower quantization MSE. Further, Table \ref{tab:FP7_vs_LM7} shows the superior perplexity achieved by Lloyd-Max quantizers for bitwidths of $7$, $6$ and $5$. The difference between the quantizers is clear at 5 bits, where per-tensor FP quantization incurs a drastic and unacceptable increase in perplexity, while Lloyd-Max quantization incurs a much smaller increase. Nevertheless, we note that even the optimal Lloyd-Max quantizer incurs a notable ($\sim 1.5$) increase in perplexity due to the coarse granularity of quantization. 

\begin{figure}[h]
  \centering
  \includegraphics[width=0.7\linewidth]{sections/figures/LM7_FP7.pdf}
  \caption{\small Quantization levels and the corresponding quantization MSE of Floating Point (left) vs Lloyd-Max (right) Quantizers for a layer of weights in the GPT3-126M model.}
  \label{fig:lm_quant}
\end{figure}

\begin{table}[h]\scriptsize
\begin{center}
\caption{\label{tab:FP7_vs_LM7} \small Comparing perplexity (lower is better) achieved by floating point quantizers and Lloyd-Max quantizers on a GPT3-126M model for the Wikitext-103 dataset.}
\begin{tabular}{c|cc|c}
\hline
 \multirow{2}{*}{\textbf{Bitwidth}} & \multicolumn{2}{|c|}{\textbf{Floating-Point Quantizer}} & \textbf{Lloyd-Max Quantizer} \\
 & Best Format & Wikitext-103 Perplexity & Wikitext-103 Perplexity \\
\hline
7 & E3M3 & 18.32 & 18.27 \\
6 & E3M2 & 19.07 & 18.51 \\
5 & E4M0 & 43.89 & 19.71 \\
\hline
\end{tabular}
\end{center}
\end{table}

\subsection{Proof of Local Optimality of LO-BCQ}
\label{subsec:lobcq_opt_proof}
For a given block $\bm{b}_j$, the quantization MSE during LO-BCQ can be empirically evaluated as $\frac{1}{L_b}\lVert \bm{b}_j- \bm{\hat{b}}_j\rVert^2_2$ where $\bm{\hat{b}}_j$ is computed from equation (\ref{eq:clustered_quantization_definition}) as $C_{f(\bm{b}_j)}(\bm{b}_j)$. Further, for a given block cluster $\mathcal{B}_i$, we compute the quantization MSE as $\frac{1}{|\mathcal{B}_{i}|}\sum_{\bm{b} \in \mathcal{B}_{i}} \frac{1}{L_b}\lVert \bm{b}- C_i^{(n)}(\bm{b})\rVert^2_2$. Therefore, at the end of iteration $n$, we evaluate the overall quantization MSE $J^{(n)}$ for a given operand $\bm{X}$ composed of $N_c$ block clusters as:
\begin{align*}
    \label{eq:mse_iter_n}
    J^{(n)} = \frac{1}{N_c} \sum_{i=1}^{N_c} \frac{1}{|\mathcal{B}_{i}^{(n)}|}\sum_{\bm{v} \in \mathcal{B}_{i}^{(n)}} \frac{1}{L_b}\lVert \bm{b}- B_i^{(n)}(\bm{b})\rVert^2_2
\end{align*}

At the end of iteration $n$, the codebooks are updated from $\mathcal{C}^{(n-1)}$ to $\mathcal{C}^{(n)}$. However, the mapping of a given vector $\bm{b}_j$ to quantizers $\mathcal{C}^{(n)}$ remains as  $f^{(n)}(\bm{b}_j)$. At the next iteration, during the vector clustering step, $f^{(n+1)}(\bm{b}_j)$ finds new mapping of $\bm{b}_j$ to updated codebooks $\mathcal{C}^{(n)}$ such that the quantization MSE over the candidate codebooks is minimized. Therefore, we obtain the following result for $\bm{b}_j$:
\begin{align*}
\frac{1}{L_b}\lVert \bm{b}_j - C_{f^{(n+1)}(\bm{b}_j)}^{(n)}(\bm{b}_j)\rVert^2_2 \le \frac{1}{L_b}\lVert \bm{b}_j - C_{f^{(n)}(\bm{b}_j)}^{(n)}(\bm{b}_j)\rVert^2_2
\end{align*}

That is, quantizing $\bm{b}_j$ at the end of the block clustering step of iteration $n+1$ results in lower quantization MSE compared to quantizing at the end of iteration $n$. Since this is true for all $\bm{b} \in \bm{X}$, we assert the following:
\begin{equation}
\begin{split}
\label{eq:mse_ineq_1}
    \tilde{J}^{(n+1)} &= \frac{1}{N_c} \sum_{i=1}^{N_c} \frac{1}{|\mathcal{B}_{i}^{(n+1)}|}\sum_{\bm{b} \in \mathcal{B}_{i}^{(n+1)}} \frac{1}{L_b}\lVert \bm{b} - C_i^{(n)}(b)\rVert^2_2 \le J^{(n)}
\end{split}
\end{equation}
where $\tilde{J}^{(n+1)}$ is the the quantization MSE after the vector clustering step at iteration $n+1$.

Next, during the codebook update step (\ref{eq:quantizers_update}) at iteration $n+1$, the per-cluster codebooks $\mathcal{C}^{(n)}$ are updated to $\mathcal{C}^{(n+1)}$ by invoking the Lloyd-Max algorithm \citep{Lloyd}. We know that for any given value distribution, the Lloyd-Max algorithm minimizes the quantization MSE. Therefore, for a given vector cluster $\mathcal{B}_i$ we obtain the following result:

\begin{equation}
    \frac{1}{|\mathcal{B}_{i}^{(n+1)}|}\sum_{\bm{b} \in \mathcal{B}_{i}^{(n+1)}} \frac{1}{L_b}\lVert \bm{b}- C_i^{(n+1)}(\bm{b})\rVert^2_2 \le \frac{1}{|\mathcal{B}_{i}^{(n+1)}|}\sum_{\bm{b} \in \mathcal{B}_{i}^{(n+1)}} \frac{1}{L_b}\lVert \bm{b}- C_i^{(n)}(\bm{b})\rVert^2_2
\end{equation}

The above equation states that quantizing the given block cluster $\mathcal{B}_i$ after updating the associated codebook from $C_i^{(n)}$ to $C_i^{(n+1)}$ results in lower quantization MSE. Since this is true for all the block clusters, we derive the following result: 
\begin{equation}
\begin{split}
\label{eq:mse_ineq_2}
     J^{(n+1)} &= \frac{1}{N_c} \sum_{i=1}^{N_c} \frac{1}{|\mathcal{B}_{i}^{(n+1)}|}\sum_{\bm{b} \in \mathcal{B}_{i}^{(n+1)}} \frac{1}{L_b}\lVert \bm{b}- C_i^{(n+1)}(\bm{b})\rVert^2_2  \le \tilde{J}^{(n+1)}   
\end{split}
\end{equation}

Following (\ref{eq:mse_ineq_1}) and (\ref{eq:mse_ineq_2}), we find that the quantization MSE is non-increasing for each iteration, that is, $J^{(1)} \ge J^{(2)} \ge J^{(3)} \ge \ldots \ge J^{(M)}$ where $M$ is the maximum number of iterations. 
%Therefore, we can say that if the algorithm converges, then it must be that it has converged to a local minimum. 
\hfill $\blacksquare$


\begin{figure}
    \begin{center}
    \includegraphics[width=0.5\textwidth]{sections//figures/mse_vs_iter.pdf}
    \end{center}
    \caption{\small NMSE vs iterations during LO-BCQ compared to other block quantization proposals}
    \label{fig:nmse_vs_iter}
\end{figure}

Figure \ref{fig:nmse_vs_iter} shows the empirical convergence of LO-BCQ across several block lengths and number of codebooks. Also, the MSE achieved by LO-BCQ is compared to baselines such as MXFP and VSQ. As shown, LO-BCQ converges to a lower MSE than the baselines. Further, we achieve better convergence for larger number of codebooks ($N_c$) and for a smaller block length ($L_b$), both of which increase the bitwidth of BCQ (see Eq \ref{eq:bitwidth_bcq}).


\subsection{Additional Accuracy Results}
%Table \ref{tab:lobcq_config} lists the various LOBCQ configurations and their corresponding bitwidths.
\begin{table}
\setlength{\tabcolsep}{4.75pt}
\begin{center}
\caption{\label{tab:lobcq_config} Various LO-BCQ configurations and their bitwidths.}
\begin{tabular}{|c||c|c|c|c||c|c||c|} 
\hline
 & \multicolumn{4}{|c||}{$L_b=8$} & \multicolumn{2}{|c||}{$L_b=4$} & $L_b=2$ \\
 \hline
 \backslashbox{$L_A$\kern-1em}{\kern-1em$N_c$} & 2 & 4 & 8 & 16 & 2 & 4 & 2 \\
 \hline
 64 & 4.25 & 4.375 & 4.5 & 4.625 & 4.375 & 4.625 & 4.625\\
 \hline
 32 & 4.375 & 4.5 & 4.625& 4.75 & 4.5 & 4.75 & 4.75 \\
 \hline
 16 & 4.625 & 4.75& 4.875 & 5 & 4.75 & 5 & 5 \\
 \hline
\end{tabular}
\end{center}
\end{table}

%\subsection{Perplexity achieved by various LO-BCQ configurations on Wikitext-103 dataset}

\begin{table} \centering
\begin{tabular}{|c||c|c|c|c||c|c||c|} 
\hline
 $L_b \rightarrow$& \multicolumn{4}{c||}{8} & \multicolumn{2}{c||}{4} & 2\\
 \hline
 \backslashbox{$L_A$\kern-1em}{\kern-1em$N_c$} & 2 & 4 & 8 & 16 & 2 & 4 & 2  \\
 %$N_c \rightarrow$ & 2 & 4 & 8 & 16 & 2 & 4 & 2 \\
 \hline
 \hline
 \multicolumn{8}{c}{GPT3-1.3B (FP32 PPL = 9.98)} \\ 
 \hline
 \hline
 64 & 10.40 & 10.23 & 10.17 & 10.15 &  10.28 & 10.18 & 10.19 \\
 \hline
 32 & 10.25 & 10.20 & 10.15 & 10.12 &  10.23 & 10.17 & 10.17 \\
 \hline
 16 & 10.22 & 10.16 & 10.10 & 10.09 &  10.21 & 10.14 & 10.16 \\
 \hline
  \hline
 \multicolumn{8}{c}{GPT3-8B (FP32 PPL = 7.38)} \\ 
 \hline
 \hline
 64 & 7.61 & 7.52 & 7.48 &  7.47 &  7.55 &  7.49 & 7.50 \\
 \hline
 32 & 7.52 & 7.50 & 7.46 &  7.45 &  7.52 &  7.48 & 7.48  \\
 \hline
 16 & 7.51 & 7.48 & 7.44 &  7.44 &  7.51 &  7.49 & 7.47  \\
 \hline
\end{tabular}
\caption{\label{tab:ppl_gpt3_abalation} Wikitext-103 perplexity across GPT3-1.3B and 8B models.}
\end{table}

\begin{table} \centering
\begin{tabular}{|c||c|c|c|c||} 
\hline
 $L_b \rightarrow$& \multicolumn{4}{c||}{8}\\
 \hline
 \backslashbox{$L_A$\kern-1em}{\kern-1em$N_c$} & 2 & 4 & 8 & 16 \\
 %$N_c \rightarrow$ & 2 & 4 & 8 & 16 & 2 & 4 & 2 \\
 \hline
 \hline
 \multicolumn{5}{|c|}{Llama2-7B (FP32 PPL = 5.06)} \\ 
 \hline
 \hline
 64 & 5.31 & 5.26 & 5.19 & 5.18  \\
 \hline
 32 & 5.23 & 5.25 & 5.18 & 5.15  \\
 \hline
 16 & 5.23 & 5.19 & 5.16 & 5.14  \\
 \hline
 \multicolumn{5}{|c|}{Nemotron4-15B (FP32 PPL = 5.87)} \\ 
 \hline
 \hline
 64  & 6.3 & 6.20 & 6.13 & 6.08  \\
 \hline
 32  & 6.24 & 6.12 & 6.07 & 6.03  \\
 \hline
 16  & 6.12 & 6.14 & 6.04 & 6.02  \\
 \hline
 \multicolumn{5}{|c|}{Nemotron4-340B (FP32 PPL = 3.48)} \\ 
 \hline
 \hline
 64 & 3.67 & 3.62 & 3.60 & 3.59 \\
 \hline
 32 & 3.63 & 3.61 & 3.59 & 3.56 \\
 \hline
 16 & 3.61 & 3.58 & 3.57 & 3.55 \\
 \hline
\end{tabular}
\caption{\label{tab:ppl_llama7B_nemo15B} Wikitext-103 perplexity compared to FP32 baseline in Llama2-7B and Nemotron4-15B, 340B models}
\end{table}

%\subsection{Perplexity achieved by various LO-BCQ configurations on MMLU dataset}


\begin{table} \centering
\begin{tabular}{|c||c|c|c|c||c|c|c|c|} 
\hline
 $L_b \rightarrow$& \multicolumn{4}{c||}{8} & \multicolumn{4}{c||}{8}\\
 \hline
 \backslashbox{$L_A$\kern-1em}{\kern-1em$N_c$} & 2 & 4 & 8 & 16 & 2 & 4 & 8 & 16  \\
 %$N_c \rightarrow$ & 2 & 4 & 8 & 16 & 2 & 4 & 2 \\
 \hline
 \hline
 \multicolumn{5}{|c|}{Llama2-7B (FP32 Accuracy = 45.8\%)} & \multicolumn{4}{|c|}{Llama2-70B (FP32 Accuracy = 69.12\%)} \\ 
 \hline
 \hline
 64 & 43.9 & 43.4 & 43.9 & 44.9 & 68.07 & 68.27 & 68.17 & 68.75 \\
 \hline
 32 & 44.5 & 43.8 & 44.9 & 44.5 & 68.37 & 68.51 & 68.35 & 68.27  \\
 \hline
 16 & 43.9 & 42.7 & 44.9 & 45 & 68.12 & 68.77 & 68.31 & 68.59  \\
 \hline
 \hline
 \multicolumn{5}{|c|}{GPT3-22B (FP32 Accuracy = 38.75\%)} & \multicolumn{4}{|c|}{Nemotron4-15B (FP32 Accuracy = 64.3\%)} \\ 
 \hline
 \hline
 64 & 36.71 & 38.85 & 38.13 & 38.92 & 63.17 & 62.36 & 63.72 & 64.09 \\
 \hline
 32 & 37.95 & 38.69 & 39.45 & 38.34 & 64.05 & 62.30 & 63.8 & 64.33  \\
 \hline
 16 & 38.88 & 38.80 & 38.31 & 38.92 & 63.22 & 63.51 & 63.93 & 64.43  \\
 \hline
\end{tabular}
\caption{\label{tab:mmlu_abalation} Accuracy on MMLU dataset across GPT3-22B, Llama2-7B, 70B and Nemotron4-15B models.}
\end{table}


%\subsection{Perplexity achieved by various LO-BCQ configurations on LM evaluation harness}

\begin{table} \centering
\begin{tabular}{|c||c|c|c|c||c|c|c|c|} 
\hline
 $L_b \rightarrow$& \multicolumn{4}{c||}{8} & \multicolumn{4}{c||}{8}\\
 \hline
 \backslashbox{$L_A$\kern-1em}{\kern-1em$N_c$} & 2 & 4 & 8 & 16 & 2 & 4 & 8 & 16  \\
 %$N_c \rightarrow$ & 2 & 4 & 8 & 16 & 2 & 4 & 2 \\
 \hline
 \hline
 \multicolumn{5}{|c|}{Race (FP32 Accuracy = 37.51\%)} & \multicolumn{4}{|c|}{Boolq (FP32 Accuracy = 64.62\%)} \\ 
 \hline
 \hline
 64 & 36.94 & 37.13 & 36.27 & 37.13 & 63.73 & 62.26 & 63.49 & 63.36 \\
 \hline
 32 & 37.03 & 36.36 & 36.08 & 37.03 & 62.54 & 63.51 & 63.49 & 63.55  \\
 \hline
 16 & 37.03 & 37.03 & 36.46 & 37.03 & 61.1 & 63.79 & 63.58 & 63.33  \\
 \hline
 \hline
 \multicolumn{5}{|c|}{Winogrande (FP32 Accuracy = 58.01\%)} & \multicolumn{4}{|c|}{Piqa (FP32 Accuracy = 74.21\%)} \\ 
 \hline
 \hline
 64 & 58.17 & 57.22 & 57.85 & 58.33 & 73.01 & 73.07 & 73.07 & 72.80 \\
 \hline
 32 & 59.12 & 58.09 & 57.85 & 58.41 & 73.01 & 73.94 & 72.74 & 73.18  \\
 \hline
 16 & 57.93 & 58.88 & 57.93 & 58.56 & 73.94 & 72.80 & 73.01 & 73.94  \\
 \hline
\end{tabular}
\caption{\label{tab:mmlu_abalation} Accuracy on LM evaluation harness tasks on GPT3-1.3B model.}
\end{table}

\begin{table} \centering
\begin{tabular}{|c||c|c|c|c||c|c|c|c|} 
\hline
 $L_b \rightarrow$& \multicolumn{4}{c||}{8} & \multicolumn{4}{c||}{8}\\
 \hline
 \backslashbox{$L_A$\kern-1em}{\kern-1em$N_c$} & 2 & 4 & 8 & 16 & 2 & 4 & 8 & 16  \\
 %$N_c \rightarrow$ & 2 & 4 & 8 & 16 & 2 & 4 & 2 \\
 \hline
 \hline
 \multicolumn{5}{|c|}{Race (FP32 Accuracy = 41.34\%)} & \multicolumn{4}{|c|}{Boolq (FP32 Accuracy = 68.32\%)} \\ 
 \hline
 \hline
 64 & 40.48 & 40.10 & 39.43 & 39.90 & 69.20 & 68.41 & 69.45 & 68.56 \\
 \hline
 32 & 39.52 & 39.52 & 40.77 & 39.62 & 68.32 & 67.43 & 68.17 & 69.30  \\
 \hline
 16 & 39.81 & 39.71 & 39.90 & 40.38 & 68.10 & 66.33 & 69.51 & 69.42  \\
 \hline
 \hline
 \multicolumn{5}{|c|}{Winogrande (FP32 Accuracy = 67.88\%)} & \multicolumn{4}{|c|}{Piqa (FP32 Accuracy = 78.78\%)} \\ 
 \hline
 \hline
 64 & 66.85 & 66.61 & 67.72 & 67.88 & 77.31 & 77.42 & 77.75 & 77.64 \\
 \hline
 32 & 67.25 & 67.72 & 67.72 & 67.00 & 77.31 & 77.04 & 77.80 & 77.37  \\
 \hline
 16 & 68.11 & 68.90 & 67.88 & 67.48 & 77.37 & 78.13 & 78.13 & 77.69  \\
 \hline
\end{tabular}
\caption{\label{tab:mmlu_abalation} Accuracy on LM evaluation harness tasks on GPT3-8B model.}
\end{table}

\begin{table} \centering
\begin{tabular}{|c||c|c|c|c||c|c|c|c|} 
\hline
 $L_b \rightarrow$& \multicolumn{4}{c||}{8} & \multicolumn{4}{c||}{8}\\
 \hline
 \backslashbox{$L_A$\kern-1em}{\kern-1em$N_c$} & 2 & 4 & 8 & 16 & 2 & 4 & 8 & 16  \\
 %$N_c \rightarrow$ & 2 & 4 & 8 & 16 & 2 & 4 & 2 \\
 \hline
 \hline
 \multicolumn{5}{|c|}{Race (FP32 Accuracy = 40.67\%)} & \multicolumn{4}{|c|}{Boolq (FP32 Accuracy = 76.54\%)} \\ 
 \hline
 \hline
 64 & 40.48 & 40.10 & 39.43 & 39.90 & 75.41 & 75.11 & 77.09 & 75.66 \\
 \hline
 32 & 39.52 & 39.52 & 40.77 & 39.62 & 76.02 & 76.02 & 75.96 & 75.35  \\
 \hline
 16 & 39.81 & 39.71 & 39.90 & 40.38 & 75.05 & 73.82 & 75.72 & 76.09  \\
 \hline
 \hline
 \multicolumn{5}{|c|}{Winogrande (FP32 Accuracy = 70.64\%)} & \multicolumn{4}{|c|}{Piqa (FP32 Accuracy = 79.16\%)} \\ 
 \hline
 \hline
 64 & 69.14 & 70.17 & 70.17 & 70.56 & 78.24 & 79.00 & 78.62 & 78.73 \\
 \hline
 32 & 70.96 & 69.69 & 71.27 & 69.30 & 78.56 & 79.49 & 79.16 & 78.89  \\
 \hline
 16 & 71.03 & 69.53 & 69.69 & 70.40 & 78.13 & 79.16 & 79.00 & 79.00  \\
 \hline
\end{tabular}
\caption{\label{tab:mmlu_abalation} Accuracy on LM evaluation harness tasks on GPT3-22B model.}
\end{table}

\begin{table} \centering
\begin{tabular}{|c||c|c|c|c||c|c|c|c|} 
\hline
 $L_b \rightarrow$& \multicolumn{4}{c||}{8} & \multicolumn{4}{c||}{8}\\
 \hline
 \backslashbox{$L_A$\kern-1em}{\kern-1em$N_c$} & 2 & 4 & 8 & 16 & 2 & 4 & 8 & 16  \\
 %$N_c \rightarrow$ & 2 & 4 & 8 & 16 & 2 & 4 & 2 \\
 \hline
 \hline
 \multicolumn{5}{|c|}{Race (FP32 Accuracy = 44.4\%)} & \multicolumn{4}{|c|}{Boolq (FP32 Accuracy = 79.29\%)} \\ 
 \hline
 \hline
 64 & 42.49 & 42.51 & 42.58 & 43.45 & 77.58 & 77.37 & 77.43 & 78.1 \\
 \hline
 32 & 43.35 & 42.49 & 43.64 & 43.73 & 77.86 & 75.32 & 77.28 & 77.86  \\
 \hline
 16 & 44.21 & 44.21 & 43.64 & 42.97 & 78.65 & 77 & 76.94 & 77.98  \\
 \hline
 \hline
 \multicolumn{5}{|c|}{Winogrande (FP32 Accuracy = 69.38\%)} & \multicolumn{4}{|c|}{Piqa (FP32 Accuracy = 78.07\%)} \\ 
 \hline
 \hline
 64 & 68.9 & 68.43 & 69.77 & 68.19 & 77.09 & 76.82 & 77.09 & 77.86 \\
 \hline
 32 & 69.38 & 68.51 & 68.82 & 68.90 & 78.07 & 76.71 & 78.07 & 77.86  \\
 \hline
 16 & 69.53 & 67.09 & 69.38 & 68.90 & 77.37 & 77.8 & 77.91 & 77.69  \\
 \hline
\end{tabular}
\caption{\label{tab:mmlu_abalation} Accuracy on LM evaluation harness tasks on Llama2-7B model.}
\end{table}

\begin{table} \centering
\begin{tabular}{|c||c|c|c|c||c|c|c|c|} 
\hline
 $L_b \rightarrow$& \multicolumn{4}{c||}{8} & \multicolumn{4}{c||}{8}\\
 \hline
 \backslashbox{$L_A$\kern-1em}{\kern-1em$N_c$} & 2 & 4 & 8 & 16 & 2 & 4 & 8 & 16  \\
 %$N_c \rightarrow$ & 2 & 4 & 8 & 16 & 2 & 4 & 2 \\
 \hline
 \hline
 \multicolumn{5}{|c|}{Race (FP32 Accuracy = 48.8\%)} & \multicolumn{4}{|c|}{Boolq (FP32 Accuracy = 85.23\%)} \\ 
 \hline
 \hline
 64 & 49.00 & 49.00 & 49.28 & 48.71 & 82.82 & 84.28 & 84.03 & 84.25 \\
 \hline
 32 & 49.57 & 48.52 & 48.33 & 49.28 & 83.85 & 84.46 & 84.31 & 84.93  \\
 \hline
 16 & 49.85 & 49.09 & 49.28 & 48.99 & 85.11 & 84.46 & 84.61 & 83.94  \\
 \hline
 \hline
 \multicolumn{5}{|c|}{Winogrande (FP32 Accuracy = 79.95\%)} & \multicolumn{4}{|c|}{Piqa (FP32 Accuracy = 81.56\%)} \\ 
 \hline
 \hline
 64 & 78.77 & 78.45 & 78.37 & 79.16 & 81.45 & 80.69 & 81.45 & 81.5 \\
 \hline
 32 & 78.45 & 79.01 & 78.69 & 80.66 & 81.56 & 80.58 & 81.18 & 81.34  \\
 \hline
 16 & 79.95 & 79.56 & 79.79 & 79.72 & 81.28 & 81.66 & 81.28 & 80.96  \\
 \hline
\end{tabular}
\caption{\label{tab:mmlu_abalation} Accuracy on LM evaluation harness tasks on Llama2-70B model.}
\end{table}

%\section{MSE Studies}
%\textcolor{red}{TODO}


\subsection{Number Formats and Quantization Method}
\label{subsec:numFormats_quantMethod}
\subsubsection{Integer Format}
An $n$-bit signed integer (INT) is typically represented with a 2s-complement format \citep{yao2022zeroquant,xiao2023smoothquant,dai2021vsq}, where the most significant bit denotes the sign.

\subsubsection{Floating Point Format}
An $n$-bit signed floating point (FP) number $x$ comprises of a 1-bit sign ($x_{\mathrm{sign}}$), $B_m$-bit mantissa ($x_{\mathrm{mant}}$) and $B_e$-bit exponent ($x_{\mathrm{exp}}$) such that $B_m+B_e=n-1$. The associated constant exponent bias ($E_{\mathrm{bias}}$) is computed as $(2^{{B_e}-1}-1)$. We denote this format as $E_{B_e}M_{B_m}$.  

\subsubsection{Quantization Scheme}
\label{subsec:quant_method}
A quantization scheme dictates how a given unquantized tensor is converted to its quantized representation. We consider FP formats for the purpose of illustration. Given an unquantized tensor $\bm{X}$ and an FP format $E_{B_e}M_{B_m}$, we first, we compute the quantization scale factor $s_X$ that maps the maximum absolute value of $\bm{X}$ to the maximum quantization level of the $E_{B_e}M_{B_m}$ format as follows:
\begin{align}
\label{eq:sf}
    s_X = \frac{\mathrm{max}(|\bm{X}|)}{\mathrm{max}(E_{B_e}M_{B_m})}
\end{align}
In the above equation, $|\cdot|$ denotes the absolute value function.

Next, we scale $\bm{X}$ by $s_X$ and quantize it to $\hat{\bm{X}}$ by rounding it to the nearest quantization level of $E_{B_e}M_{B_m}$ as:

\begin{align}
\label{eq:tensor_quant}
    \hat{\bm{X}} = \text{round-to-nearest}\left(\frac{\bm{X}}{s_X}, E_{B_e}M_{B_m}\right)
\end{align}

We perform dynamic max-scaled quantization \citep{wu2020integer}, where the scale factor $s$ for activations is dynamically computed during runtime.

\subsection{Vector Scaled Quantization}
\begin{wrapfigure}{r}{0.35\linewidth}
  \centering
  \includegraphics[width=\linewidth]{sections/figures/vsquant.jpg}
  \caption{\small Vectorwise decomposition for per-vector scaled quantization (VSQ \citep{dai2021vsq}).}
  \label{fig:vsquant}
\end{wrapfigure}
During VSQ \citep{dai2021vsq}, the operand tensors are decomposed into 1D vectors in a hardware friendly manner as shown in Figure \ref{fig:vsquant}. Since the decomposed tensors are used as operands in matrix multiplications during inference, it is beneficial to perform this decomposition along the reduction dimension of the multiplication. The vectorwise quantization is performed similar to tensorwise quantization described in Equations \ref{eq:sf} and \ref{eq:tensor_quant}, where a scale factor $s_v$ is required for each vector $\bm{v}$ that maps the maximum absolute value of that vector to the maximum quantization level. While smaller vector lengths can lead to larger accuracy gains, the associated memory and computational overheads due to the per-vector scale factors increases. To alleviate these overheads, VSQ \citep{dai2021vsq} proposed a second level quantization of the per-vector scale factors to unsigned integers, while MX \citep{rouhani2023shared} quantizes them to integer powers of 2 (denoted as $2^{INT}$).

\subsubsection{MX Format}
The MX format proposed in \citep{rouhani2023microscaling} introduces the concept of sub-block shifting. For every two scalar elements of $b$-bits each, there is a shared exponent bit. The value of this exponent bit is determined through an empirical analysis that targets minimizing quantization MSE. We note that the FP format $E_{1}M_{b}$ is strictly better than MX from an accuracy perspective since it allocates a dedicated exponent bit to each scalar as opposed to sharing it across two scalars. Therefore, we conservatively bound the accuracy of a $b+2$-bit signed MX format with that of a $E_{1}M_{b}$ format in our comparisons. For instance, we use E1M2 format as a proxy for MX4.

\begin{figure}
    \centering
    \includegraphics[width=1\linewidth]{sections//figures/BlockFormats.pdf}
    \caption{\small Comparing LO-BCQ to MX format.}
    \label{fig:block_formats}
\end{figure}

Figure \ref{fig:block_formats} compares our $4$-bit LO-BCQ block format to MX \citep{rouhani2023microscaling}. As shown, both LO-BCQ and MX decompose a given operand tensor into block arrays and each block array into blocks. Similar to MX, we find that per-block quantization ($L_b < L_A$) leads to better accuracy due to increased flexibility. While MX achieves this through per-block $1$-bit micro-scales, we associate a dedicated codebook to each block through a per-block codebook selector. Further, MX quantizes the per-block array scale-factor to E8M0 format without per-tensor scaling. In contrast during LO-BCQ, we find that per-tensor scaling combined with quantization of per-block array scale-factor to E4M3 format results in superior inference accuracy across models. 


%\newpage
%\section{Moved Text}

\parag{Construction}
We initialize two integers $c := 0$ and $A:=0$.
We initialize a hash table $H$ with integer keys and integer values.
We initialize two vectors $R$ and $V$ of integer elements.
In addition, let $\tau\in[2, n]$ be a parameter that is set by the user.
We compute the set $\Minimizers(T)$ of the positions of the minimizers of $T$
by scanning $T$ from left to right in $\cO(n)$ time~\cite{DBLP:journals/tkde/LoukidesPS23} using a rolling hash function $\phi : \Sigma^k \to \mathbb{N}$, e.g., Karp-Rabin (KR) fingerprints~\cite{DBLP:journals/ibmrd/KarpR87}, to determine the order between {\kmer}s.

Let us now provide an intuition about the role of each variable during the construction.

\begin{enumerate}
    \item $c$ is an integer variable that counts the number of distinct $k$-mers selected as minimizers in $T$.
    \item $A$ is an integer variable that counts the number of minimizers in $T$.
    \item $H$ is a hash table that stores key-value pairs: a key is the KR
      fingerprint of a $k$-mer selected as a minimizer; and its value is a
      unique integer from $[0,c)$ viewed as the id of the minimizer.
        \ragnar{There's no need to index kmers by their hash -- we can index
          them by their kmer value directly. I maintain the point that the hash
          function used for minimizers should be independent from the one used
          for indexing purposes, especially since for minimizers we can get away
        with using a relatively weak random hash, while that is not good for indexing.}
        \ragnar{I only realized later that we need to store hashes to save space
          when $k$ is large. We should explicitly remark this.}
        \giulio{Exactly. I was about to make the same remark but $k$ can be very large.}
    \item $R$ is an integer vector that stores the sequence of id's of the selected minimizers from left to right; i.e., $R[A]=c$ denotes that the $A$th selected minimizer has id $c$.
    \item $V$ is an integer vector that stores the positions of the selected minimizers from left to right; i.e., $V[A]=i$ denotes that the $A$th selected minimizer occurs at position $i$ of $T$.
\end{enumerate}

\rob{Are all elements of this index essential? If we have 5, and the original text, then why do we need 4? If we have th positions of the minimizers, than we can recall each of them directly by referring to them in the original text.}

The construction processes the elements of $\Minimizers(T)$ in increasing order and uses the above variables to construct the \emph{sketch} of $T$, that is,
a string $S$ over alphabet $[0,\tau)$. For each $i \in \Minimizers(T)$,
we lookup $\phi(T[i\dd i+k-1])$ in $H$. If it is not there, we add $H[\phi(T[i\dd i+k-1])] = c$, set $R[A]=c$, and increment $c$ by one. If it is there, then we set $R[A] = H[\phi(T[i\dd i+k-1])]$.
We also set $V[A] = i$ and increase $A$ by one.
We repeat this process for the next position of a minimizer $i'> i$ found, until we reach the end of $T$, when we will have found all minimizers and fully constructed $H$, $R$, and $V$. At this point, we have $A=|\Minimizers(T)|\geq c$.
We initialize an empty vector $S$ of $\lceil \log \tau \rceil$-bit integers.
We read $R$ from left to right and append $R[i]$ to $S$ by expressing it as
$\lceil\log c/\log \tau\rceil$ integers \ragnar{the denominator needs a ceil
  here and elsewhere}.
Note the purpose of the parameter $\tau$: it lets the user control the size of the alphabet of $S$ as something that lies in $[2,n]$.
This is a useful feature because some compressed full-text indexes, like the FM-index~\cite{10.1145/1082036.1082039} and the $r$-index~\cite{10.1145/3375890}, take advantage of the repetitiveness of the text to improve its compression.

Finally we construct the $\INDEX$ of $S$, which is over alphabet $[0,\tau)$ and is of length $A\lceil\log c/\log \tau\rceil\leq A\lceil\log A/\log \tau\rceil$.\footnote{In the unlikely event of $A\lceil\log c/\log \tau\rceil > n$, we can either increase $\tau$ to have $A\lceil\log c/\log \tau\rceil \leq n$ or simply set $S:=T$.} We also delete $R$.

To conclude, our framework assumes read-only random access to $T$, takes parameters $\ell,k$ and $\tau$ as input, and constructs an index on top of $T$ that consists only of $H$, $V$, and the $\INDEX$ of $S$ over $\tau$.

\giulio{I'd suggest to separate the mere construction part (i.e., how one obtains the list of minimizers) from the encoding part, i.e., how we encode $S$ over the alphabet $[\tau]$. It is much cleaner. Currently, we anticipate what $\tau$ is without using it until the end.}

\end{document}

\section{Implementation}

A few important remarks with respect to a C\texttt{++} implementation.

\begin{remark}
The simple construction algorithm (without $F$ and the tries) uses a single pass over $T$.
We never store $\Minimizers$. We compute the elements of $\Minimizers$ from left to right and construct $H,V,$ and $R$.
It may be the case that for long patterns, we need to implement vector $F$ (but not the tries).
\end{remark}

\begin{remark}
The value of $\tau$ can for example be set to $8$ so that the alphabet is $[0,256)$.
Thus if we have an implementation of \INDEX that works with \texttt{unsigned char}, it will work with $S$ over $[0,256)$.
\end{remark}

\begin{remark}\label{rem:V}
Vector $V$ is monotonically increasing and so it can be compressed using Elias-Fano coding~\cite{DBLP:journals/jacm/Elias74}; see \url{https://simongog.github.io/assets/data/sdsl-cheatsheet.pdf} (\texttt{enc\_vector}). Specifically, we can use \texttt{sdsl::enc\_vector<sdsl::coder::elias\_delta, 128> enc\_delta(V)}.
\end{remark}

\begin{remark}\label{rem:H}
Hash table $H$ can also be compressed as follows.
Sort the keys of $H$ in increasing order and store them in a vector $K$. 
We rename vector $R$ according to the following: $K[i]$ is the fingerprint for letter $i$.
Since $K$ is monotonically increasing we compress it using Elias-Fano coding.
For pattern matching queries, we binary search on $K$ to find the letter representing a fingerprint.
\end{remark}

\cref{rem:V} and \cref{rem:H} can be tested first to see if they give any improvements.
