%%%%%%%% ICML 2025 EXAMPLE LATEX SUBMISSION FILE %%%%%%%%%%%%%%%%%

\documentclass{article}

% Recommended, but optional, packages for figures and better typesetting:
\usepackage{microtype}
\usepackage{graphicx}
\usepackage{subfigure}
\usepackage{booktabs} % for professional tables

% hyperref makes hyperlinks in the resulting PDF.
% If your build breaks (sometimes temporarily if a hyperlink spans a page)
% please comment out the following usepackage line and replace
% \usepackage{icml2025} with \usepackage[nohyperref]{icml2025} above.
\usepackage{hyperref}


% Attempt to make hyperref and algorithmic work together better:
\newcommand{\theHalgorithm}{\arabic{algorithm}}

% Use the following line for the initial blind version submitted for review:
% \usepackage{icml2025}

% If accepted, instead use the following line for the camera-ready submission:
\usepackage[accepted]{icml2025}

% For theorems and such
\usepackage{amsmath}
\usepackage{amssymb}
\usepackage{mathtools}
\usepackage{amsthm}

% jiayi add
\usepackage{pifont}
% \usepackage{algpseudocode}
\usepackage{multirow}
\usepackage{array}
\usepackage{booktabs}
\usepackage{colortbl}
\usepackage{xcolor}
\usepackage{color}
\usepackage{enumitem}
\usepackage{tcolorbox}    
% \usepackage{minted}      
\usepackage[frozencache,cachedir=minted-cache]{minted}
% \usepackage[finalizecache,cachedir=minted-cache]{minted}
\usepackage{xcolor}
\tcbuselibrary{breakable} 
\usepackage{ulem}
% if you use cleveref..
\usepackage[capitalize,noabbrev]{cleveref}

%%%%%%%%%%%%%%%%%%%%%%%%%%%%%%%%
% THEOREMS
%%%%%%%%%%%%%%%%%%%%%%%%%%%%%%%%
\theoremstyle{plain}
\newtheorem{theorem}{Theorem}[section]
\newtheorem{proposition}[theorem]{Proposition}
\newtheorem{lemma}[theorem]{Lemma}
\newtheorem{corollary}[theorem]{Corollary}
\theoremstyle{definition}
\newtheorem{definition}[theorem]{Definition}
\newtheorem{assumption}[theorem]{Assumption}
\theoremstyle{remark}
\newtheorem{remark}[theorem]{Remark}

% Todonotes is useful during development; simply uncomment the next line
%    and comment out the line below the next line to turn off comments
%\usepackage[disable,textsize=tiny]{todonotes}
\usepackage[textsize=tiny]{todonotes}


% The \icmltitle you define below is probably too long as a header.
% Therefore, a short form for the running title is supplied here:
\icmltitlerunning{Self-Supervised Prompt Optimization}

%%%%%%%%%%%---SETME-----%%%%%%%%%%%%%
%replace @@ with the submission number submission site.
\newcommand{\thiswork}{INF$^2$\xspace}
%%%%%%%%%%%%%%%%%%%%%%%%%%%%%%%%%%%%


%\newcommand{\rev}[1]{{\color{olivegreen}#1}}
\newcommand{\rev}[1]{{#1}}


\newcommand{\JL}[1]{{\color{cyan}[\textbf{\sc JLee}: \textit{#1}]}}
\newcommand{\JW}[1]{{\color{orange}[\textbf{\sc JJung}: \textit{#1}]}}
\newcommand{\JY}[1]{{\color{blue(ncs)}[\textbf{\sc JSong}: \textit{#1}]}}
\newcommand{\HS}[1]{{\color{magenta}[\textbf{\sc HJang}: \textit{#1}]}}
\newcommand{\CS}[1]{{\color{navy}[\textbf{\sc CShin}: \textit{#1}]}}
\newcommand{\SN}[1]{{\color{olive}[\textbf{\sc SNoh}: \textit{#1}]}}

%\def\final{}   % uncomment this for the submission version
\ifdefined\final
\renewcommand{\JL}[1]{}
\renewcommand{\JW}[1]{}
\renewcommand{\JY}[1]{}
\renewcommand{\HS}[1]{}
\renewcommand{\CS}[1]{}
\renewcommand{\SN}[1]{}
\fi

%%% Notion for baseline approaches %%% 
\newcommand{\baseline}{offloading-based batched inference\xspace}
\newcommand{\Baseline}{Offloading-based batched inference\xspace}


\newcommand{\ans}{attention-near storage\xspace}
\newcommand{\Ans}{Attention-near storage\xspace}
\newcommand{\ANS}{Attention-Near Storage\xspace}

\newcommand{\wb}{delayed KV cache writeback\xspace}
\newcommand{\Wb}{Delayed KV cache writeback\xspace}
\newcommand{\WB}{Delayed KV Cache Writeback\xspace}

\newcommand{\xcache}{X-cache\xspace}
\newcommand{\XCACHE}{X-Cache\xspace}


%%% Notions for our methods %%%
\newcommand{\schemea}{\textbf{Expanding supported maximum sequence length with optimized performance}\xspace}
\newcommand{\Schemea}{\textbf{Expanding supported maximum sequence length with optimized performance}\xspace}

\newcommand{\schemeb}{\textbf{Optimizing the storage device performance}\xspace}
\newcommand{\Schemeb}{\textbf{Optimizing the storage device performance}\xspace}

\newcommand{\schemec}{\textbf{Orthogonally supporting Compression Techniques}\xspace}
\newcommand{\Schemec}{\textbf{Orthogonally supporting Compression Techniques}\xspace}



% Circular numbers
\usepackage{tikz}
\newcommand*\circled[1]{\tikz[baseline=(char.base)]{
            \node[shape=circle,draw,inner sep=0.4pt] (char) {#1};}}

\newcommand*\bcircled[1]{\tikz[baseline=(char.base)]{
            \node[shape=circle,draw,inner sep=0.4pt, fill=black, text=white] (char) {#1};}}

\begin{document}

\twocolumn[
\icmltitle{Self-Supervised Prompt Optimization}

% \hspace{-1em}\raisebox{-1ex}{\includegraphics[height=2em]{images/figure-logo-2.pdf}}\hspace{1em}

% It is OKAY to include author information, even for blind
% submissions: the style file will automatically remove it for you
% unless you've provided the [accepted] option to the icml2025
% package.

% List of affiliations: The first argument should be a (short)
% identifier you will use later to specify author affiliations
% Academic affiliations should list Department, University, City, Region, Country
% Industry affiliations should list Company, City, Region, Country

% You can specify symbols, otherwise they are numbered in order.
% Ideally, you should not use this facility. Affiliations will be numbered
% in order of appearance and this is the preferred way.
\icmlsetsymbol{equal}{*}
\icmlsetsymbol{co}{$\dagger$}

\begin{icmlauthorlist}
\icmlauthor{Jinyu Xiang}{1,equal}
\icmlauthor{Jiayi Zhang}{2,equal}
\icmlauthor{Zhaoyang Yu}{3}
\icmlauthor{Fengwei Teng}{3}
\icmlauthor{Jinhao Tu}{4}
\\
\icmlauthor{Xinbing Liang}{1}
\icmlauthor{Sirui Hong}{1}
\icmlauthor{Chenglin Wu}{1,co}
\icmlauthor{Yuyu Luo}{2,co}

% \\Huawei Noah's Ark Lab
\end{icmlauthorlist}


\icmlaffiliation{1}{DeepWisdom}
\icmlaffiliation{2}{The Hong Kong University of Science and Technology (Guangzhou)}
\icmlaffiliation{3}{Renmin University of China}
\icmlaffiliation{4}{Independent Researcher}


\icmlcorrespondingauthor{Chenglin Wu}{alexanderwu@deepwisdom.ai}
\icmlcorrespondingauthor{Yuyu Luo}{yuyuluo@hkust-gz.edu.cn}

% You may provide any keywords that you
% find helpful for describing your paper; these are used to populate
% the "keywords" metadata in the PDF but will not be shown in the document
% \icmlkeywords{Machine Learning, ICML}

\vskip 0.3in
]

% this must go after the closing bracket ] following \twocolumn[ ...

% This command actually creates the footnote in the first column
% listing the affiliations and the copyright notice.
% The command takes one argument, which is text to display at the start of the footnote.
% The \icmlEqualContribution command is standard text for equal contribution.
% Remove it (just {}) if you do not need this facility.

%\printAffiliationsAndNotice{}  % leave blank if no need to mention equal contribution
\printAffiliationsAndNotice{\icmlEqualContribution} % otherwise use the standard text.

% Large language model (LLM)-based agents have shown promise in tackling complex tasks by interacting dynamically with the environment. 
Existing work primarily focuses on behavior cloning from expert demonstrations and preference learning through exploratory trajectory sampling. However, these methods often struggle in long-horizon tasks, where suboptimal actions accumulate step by step, causing agents to deviate from correct task trajectories.
To address this, we highlight the importance of \textit{timely calibration} and the need to automatically construct calibration trajectories for training agents. We propose \textbf{S}tep-Level \textbf{T}raj\textbf{e}ctory \textbf{Ca}libration (\textbf{\model}), a novel framework for LLM agent learning. 
Specifically, \model identifies suboptimal actions through a step-level reward comparison during exploration. It constructs calibrated trajectories using LLM-driven reflection, enabling agents to learn from improved decision-making processes. These calibrated trajectories, together with successful trajectory data, are utilized for reinforced training.
Extensive experiments demonstrate that \model significantly outperforms existing methods. Further analysis highlights that step-level calibration enables agents to complete tasks with greater robustness. 
Our code and data are available at \url{https://github.com/WangHanLinHenry/STeCa}.
\begin{abstract}
Well-designed prompts are crucial for enhancing Large language models' (LLMs) reasoning capabilities while aligning their outputs with task requirements across diverse domains. However, manually designed prompts require expertise and iterative experimentation. 
While existing prompt optimization methods aim to automate this process, they rely heavily on external references such as ground truth or by humans, limiting their applicability in real-world scenarios where such data is unavailable or costly to obtain. 
To address this, we propose \textbf{S}elf-Supervised \textbf{P}rompt \textbf{O}ptimization (\ours), a cost-efficient framework that discovers effective prompts for both closed and open-ended tasks without requiring external reference.
Motivated by the observations that prompt quality manifests directly in LLM outputs and LLMs can effectively assess adherence to task requirements, we derive evaluation and optimization signals purely from output comparisons.
Specifically, \ours selects superior prompts through pairwise output comparisons evaluated by an LLM evaluator, followed by an LLM optimizer that aligns outputs with task requirements.
Extensive experiments demonstrate that \ours outperforms state-of-the-art prompt optimization methods, achieving comparable or superior results with significantly lower costs (e.g., \textbf{1.1\% to 5.6\%} of existing methods) and fewer samples (e.g., three samples).  The code is available at \href{https://github.com/geekan/MetaGPT}{https://github.com/geekan/MetaGPT}.
\end{abstract}
% \section{Introduction}

Despite the remarkable capabilities of large language models (LLMs)~\cite{DBLP:conf/emnlp/QinZ0CYY23,DBLP:journals/corr/abs-2307-09288}, they often inevitably exhibit hallucinations due to incorrect or outdated knowledge embedded in their parameters~\cite{DBLP:journals/corr/abs-2309-01219, DBLP:journals/corr/abs-2302-12813, DBLP:journals/csur/JiLFYSXIBMF23}.
Given the significant time and expense required to retrain LLMs, there has been growing interest in \emph{model editing} (a.k.a., \emph{knowledge editing})~\cite{DBLP:conf/iclr/SinitsinPPPB20, DBLP:journals/corr/abs-2012-00363, DBLP:conf/acl/DaiDHSCW22, DBLP:conf/icml/MitchellLBMF22, DBLP:conf/nips/MengBAB22, DBLP:conf/iclr/MengSABB23, DBLP:conf/emnlp/YaoWT0LDC023, DBLP:conf/emnlp/ZhongWMPC23, DBLP:conf/icml/MaL0G24, DBLP:journals/corr/abs-2401-04700}, 
which aims to update the knowledge of LLMs cost-effectively.
Some existing methods of model editing achieve this by modifying model parameters, which can be generally divided into two categories~\cite{DBLP:journals/corr/abs-2308-07269, DBLP:conf/emnlp/YaoWT0LDC023}.
Specifically, one type is based on \emph{Meta-Learning}~\cite{DBLP:conf/emnlp/CaoAT21, DBLP:conf/acl/DaiDHSCW22}, while the other is based on \emph{Locate-then-Edit}~\cite{DBLP:conf/acl/DaiDHSCW22, DBLP:conf/nips/MengBAB22, DBLP:conf/iclr/MengSABB23}. This paper primarily focuses on the latter.

\begin{figure}[t]
  \centering
  \includegraphics[width=0.48\textwidth]{figures/demonstration.pdf}
  \vspace{-4mm}
  \caption{(a) Comparison of regular model editing and EAC. EAC compresses the editing information into the dimensions where the editing anchors are located. Here, we utilize the gradients generated during training and the magnitude of the updated knowledge vector to identify anchors. (b) Comparison of general downstream task performance before editing, after regular editing, and after constrained editing by EAC.}
  \vspace{-3mm}
  \label{demo}
\end{figure}

\emph{Sequential} model editing~\cite{DBLP:conf/emnlp/YaoWT0LDC023} can expedite the continual learning of LLMs where a series of consecutive edits are conducted.
This is very important in real-world scenarios because new knowledge continually appears, requiring the model to retain previous knowledge while conducting new edits. 
Some studies have experimentally revealed that in sequential editing, existing methods lead to a decrease in the general abilities of the model across downstream tasks~\cite{DBLP:journals/corr/abs-2401-04700, DBLP:conf/acl/GuptaRA24, DBLP:conf/acl/Yang0MLYC24, DBLP:conf/acl/HuC00024}. 
Besides, \citet{ma2024perturbation} have performed a theoretical analysis to elucidate the bottleneck of the general abilities during sequential editing.
However, previous work has not introduced an effective method that maintains editing performance while preserving general abilities in sequential editing.
This impacts model scalability and presents major challenges for continuous learning in LLMs.

In this paper, a statistical analysis is first conducted to help understand how the model is affected during sequential editing using two popular editing methods, including ROME~\cite{DBLP:conf/nips/MengBAB22} and MEMIT~\cite{DBLP:conf/iclr/MengSABB23}.
Matrix norms, particularly the L1 norm, have been shown to be effective indicators of matrix properties such as sparsity, stability, and conditioning, as evidenced by several theoretical works~\cite{kahan2013tutorial}. In our analysis of matrix norms, we observe significant deviations in the parameter matrix after sequential editing.
Besides, the semantic differences between the facts before and after editing are also visualized, and we find that the differences become larger as the deviation of the parameter matrix after editing increases.
Therefore, we assume that each edit during sequential editing not only updates the editing fact as expected but also unintentionally introduces non-trivial noise that can cause the edited model to deviate from its original semantics space.
Furthermore, the accumulation of non-trivial noise can amplify the negative impact on the general abilities of LLMs.

Inspired by these findings, a framework termed \textbf{E}diting \textbf{A}nchor \textbf{C}ompression (EAC) is proposed to constrain the deviation of the parameter matrix during sequential editing by reducing the norm of the update matrix at each step. 
As shown in Figure~\ref{demo}, EAC first selects a subset of dimension with a high product of gradient and magnitude values, namely editing anchors, that are considered crucial for encoding the new relation through a weighted gradient saliency map.
Retraining is then performed on the dimensions where these important editing anchors are located, effectively compressing the editing information.
By compressing information only in certain dimensions and leaving other dimensions unmodified, the deviation of the parameter matrix after editing is constrained. 
To further regulate changes in the L1 norm of the edited matrix to constrain the deviation, we incorporate a scored elastic net ~\cite{zou2005regularization} into the retraining process, optimizing the previously selected editing anchors.

To validate the effectiveness of the proposed EAC, experiments of applying EAC to \textbf{two popular editing methods} including ROME and MEMIT are conducted.
In addition, \textbf{three LLMs of varying sizes} including GPT2-XL~\cite{radford2019language}, LLaMA-3 (8B)~\cite{llama3} and LLaMA-2 (13B)~\cite{DBLP:journals/corr/abs-2307-09288} and \textbf{four representative tasks} including 
natural language inference~\cite{DBLP:conf/mlcw/DaganGM05}, 
summarization~\cite{gliwa-etal-2019-samsum},
open-domain question-answering~\cite{DBLP:journals/tacl/KwiatkowskiPRCP19},  
and sentiment analysis~\cite{DBLP:conf/emnlp/SocherPWCMNP13} are selected to extensively demonstrate the impact of model editing on the general abilities of LLMs. 
Experimental results demonstrate that in sequential editing, EAC can effectively preserve over 70\% of the general abilities of the model across downstream tasks and better retain the edited knowledge.

In summary, our contributions to this paper are three-fold:
(1) This paper statistically elucidates how deviations in the parameter matrix after editing are responsible for the decreased general abilities of the model across downstream tasks after sequential editing.
(2) A framework termed EAC is proposed, which ultimately aims to constrain the deviation of the parameter matrix after editing by compressing the editing information into editing anchors. 
(3) It is discovered that on models like GPT2-XL and LLaMA-3 (8B), EAC significantly preserves over 70\% of the general abilities across downstream tasks and retains the edited knowledge better.
\section{Introduction}

\begin{figure}[t!]
	\centering
\includegraphics[width=\linewidth]{images/figure-compare.pdf}
        \vspace{-1em}
        \caption{\textbf{Comparison of Prompt Optimization Methods.} (a) illustrates the traditional prompt optimization process with external reference, where feedback from the ground truth of humans is used to iteratively improve the best prompt. (b) presents our proposed self-supervised prompt optimization, which utilizes pairwise comparisons of LLM's own outputs to optimize prompts without relying on external reference.}
	\label{fig:contrast}
\end{figure}

\begin{figure*}[t!]
	\centering
	\includegraphics[width=\linewidth]{images/figure-performance.pdf}
        \vspace{-1em}
	\caption{\textbf{Comparison of Performance ($y$-axis) and Optimization Costs in Dollars ($x$-axis) across Six Prompt Optimization Methods.}
     \ours demonstrates competitive performance, consistently ranking among the top two methods while maintaining significantly lower costs (ranging from 1.1\% to 5.6\% of the costs incurred by other methods) across all datasets.
     }
    \label{fig:performance-cost}

\end{figure*}


As large language models (LLMs) continue to advance, well-designed prompts have become critical for maximizing their reasoning capabilities~\cite{wei2022COT, hua2024step, deng2023rephrase} and ensuring alignment with diverse task requirements~\cite{sirui2024meta, liu2024surveynl2sqllargelanguage, zhang2024mobileexperts, hong2024data}.
However, creating effective prompts often requires substantial trial-and-error experimentation and deep task-specific knowledge.


To address this challenge, researchers have explored Prompt Optimization (PO) methods that use LLMs' own capabilities to automatically improve prompts. PO advances beyond traditional prompt engineering, by providing a more systematic and efficient approach to prompt design. As shown in Figure~\ref{fig:contrast}(a), these methods typically involve an iterative process of prompt optimization, execution, and evaluation. The design choices for these components significantly influence optimization effectiveness and efficiency.
Existing approaches have demonstrated success with both numerical evaluation mechanisms~\cite{xin2024pa, yang2023opro, chris2024pb} and textual ``gradient'' optimization strategies~\cite{wang2024semantic, mert2024textgrad}. Through these innovations, PO methods have shown promise in reducing manual effort while enhancing task performance~\cite{reid2023protegi, zhang2024aflow, zhou2024zepo}.


Despite their potential, existing PO methods face significant challenges in real-world scenarios, as discussed below. First, current methods often depend heavily on external references for evaluation. Methods using ground truth for evaluation~\cite{yang2023opro, chris2024pb, mert2024textgrad,reid2023protegi} require large amounts of annotated data to assess prompt quality, yet such standard answers are often unavailable in many practical applications, especially for open-ended tasks. Similarly, methods relying on human~\cite{yong2024promst, lin2024apohf} require manual evaluations or human-designed rules to generate feedback, which is time-consuming and contradicts the goal of automation.
Second, existing methods typically require evaluating prompts on numerous samples to obtain reliable feedback, leading to substantial computational overhead~\cite{xin2024pa, chris2024pb}.


At the core of these challenges lies the absence of reliable and efficient reference-free methods for assessing prompt quality. Analysis of LLM behavior reveals two key insights that inform our approach. First, prompt quality inherently manifests in model outputs, as evidenced by how different prompting strategies significantly influence both reasoning paths~\cite{wei2022COT, deng2023rephrase} and response features~\cite{lei2024character, schmidgall2025agentlaboratory}. Second, extensive studies on LLM-as-a-judge have demonstrated their effectiveness in evaluating output adherence to task requirements~\cite{lianmin2023mtbench, dawei2024laajsurvey}. These observations suggest that by leveraging LLMs' inherent ability to assess outputs that naturally reflect prompt quality, reference-free prompt optimization becomes feasible.



Motivated by these insights, we propose a cost-efficient framework that generates evaluation and optimization signals purely from LLM outputs, similar to how self-supervised learning derives training signals from data. We term this approach \textbf{S}elf-Supervised \textbf{P}rompt \textbf{O}ptimization (\ours). As shown in Figure \ref{fig:contrast}, \ours builds upon the fundamental Optimize-Execute-Evaluate loop while introducing several innovative mechanisms: 

(1) \textbf{\textit{Output as Pairwise Evaluation Reference}}: At its core, \ours employs a pairwise comparison approach that assesses the relative quality of outputs from different prompts. This evaluation mechanism leverages LLM's inherent capability to understand task requirements, validating optimization effectiveness without external references.

(2) \textbf{\textit{Output as Optimization Guidance}}: \ours optimizes prompts through LLM's understanding of better solutions for the current best output. Rather than relying on explicit optimization signals, this process naturally aligns prompt modifications with the model's comprehension of optimal task solutions.

\textbf{Contributions.}
Our main contributions are as follows:

(1) \textbf{Self-Supervised Prompt Optimization Framework.} We introduce \ours, a novel framework that leverages pairwise comparisons of LLM's outputs to guide prompt optimization without requiring external reference.

(2) \textbf{Cost-effective Optimization.} 
\ours optimizes prompts with minimal computational overhead (\$0.15 per dataset) and sample requirements (3 samples), significantly reducing resource demands.

(3) \textbf{Extensive Evaluation.} As shown in Figure~\ref{fig:performance-cost}, \ours requires only \textbf{1.1\% to 5.6\%} of the cost of state-of-the-art methods while maintaining superior performance across both closed and open-ended tasks.




% \input{files/3-1-formulation}
\section{Preliminary}
\label{sec:pre}

\subsection{Problem Definition} 
\label{sec:problem-formulation}

\textbf{Prompt Optimization} aims to automatically enhance the effectiveness of a prompt for a given task. Formally, let $T = (Q, G_t)$ represent a task, where $Q$ denotes the input question and $G_t$ is the optional ground truth.
The goal is to generate a task-specific prompt $P_t^*$ that maximizes performance on task $T$. This optimization objective can be formally expressed as:

\begin{equation}
P_t^* = \argmax_{P_t \in \mathcal{P}} \mathbb{E}_{T \sim D}[\phi_{eval}(\phi_{exe}(Q, P_t))],
\end{equation}


where $\mathcal{P}$ represents the space of all possible prompts. As illustrated in Figure~\ref{fig:contrast}, this optimization process typically involves three fundamental functions:
(1) Optimization function ($\phi_{opt}$): generates a revised prompt based on the candidate prompt; 
(2) Execution function ($\phi_{exe}$): applies the revised prompt with an LLM to produce outputs $O$, consisting of a reasoning path and a final answer; 
(3) Evaluation function ($\phi_{eval}$):  assesses the quality of $O$ and provides feedback $F$ to guide further optimization, refining the candidate prompts iteratively.

Among these functions, the evaluation function plays a pivotal role as its output (feedback $F$) guides the assessment and improvement of prompts.
We will discuss the evaluation framework for prompt optimization in Section~\ref{sec:feedback}.



\subsection{Evaluation Framework in  Prompt Optimization}
\label{sec:feedback}

This section outlines our evaluation framework for prompt optimization, covering three key components: evaluation sources, evaluation methods, and feedback types, as shown in Figure~\ref{fig:components}. We conclude by introducing our selected evaluation framework for \ours.


\textbf{Evaluation Sources}
As shown in Figure~\ref{fig:components}(a), two primary sources can be used for evaluation: LLM-generated outputs and task-specific ground truth. These sources provide the basis for assessing prompt performance.


\textbf{Evaluation Methods}
The evaluation method defines how the evaluation sources are assessed and the associated costs. Three common methods are used:
(1) \textit{Benchmark} relies on predefined metrics~\cite{mirac2023bbh, david2023gpqa} or rules~\cite{yong2024promst}. 
(2) \textit{LLM-as-a-judge}~\cite{lianmin2023mtbench} leverage LLMs capability to understand and assess outputs based on task requirements. 
(3) \textit{Human Feedback}~\cite{lin2024apohf} provides the most comprehensive evaluation through direct human assessment of outputs.

While Human Feedback offers the most thorough evaluation by capturing human preferences and task-specific needs, it incurs substantially higher costs than Benchmark or LLM-as-a-judge methods, creating a trade-off between evaluation quality and feasibility.


\textbf{Feedback Types}
Feedback produced by evaluation methods typically take three forms:
(1) \textit{Numerical Feedback} provides quantitative performance measures across the dataset. However, it requires substantial samples for stable evaluation and may overlook instance-specific details~\cite{zhang2024aflow}. 
(2) \textit{Textual Feedback} offers rich, instance-specific guidance through analysis and suggestions, directly generating optimization signals~\cite{mert2024textgrad}.
(3) \textit{Ranking or Selection Feedback}~\cite{yin2024pair} establishes relative quality ordering among outputs through either complete ranking or pairwise comparisons, providing clear optimization direction without requiring absolute quality measures.

\begin{figure}[t!]
	\centering
	\includegraphics[width=\linewidth]{images/figure-components.pdf}
        \vspace{-2em}
	\caption{Components of the Evaluation Framework for Prompt Optimization. (a) Evaluation Sources: Compares different outputs, including ground truth and model-generated outputs, to assess quality.
(b) Evaluation Methods: Showcases various evaluation techniques, including benchmark comparisons, LLM-as-a-Judge, and human feedback.
(c) Feedback Types: Showcases a range of feedback.
The \textit{\textcolor{bleudefrance}{\textbf{blue}}} in (a), (b), and (c) indicate the specific evaluation approach selected for \ours.
}
	\label{fig:components}
\end{figure}

\paragraph{Evaluation Framework}
Building on the previous discussion on evaluation's sources, methods, and feedback types, the evaluation framework determines how sources are compared and assessed within the context of prompt optimization. Specifically, we derive two evaluation frameworks to generate feedback $F$ for prompt optimization:


(1) \textbf{Output \textit{vs.} Ground Truth (OvG):} 
    Feedback is generated by comparing outputs $O$ with ground truth $G_T$:
    
    \begin{equation}
    \small
        f_{OvG}(O_i, G_i) = \phi_{eval}(\phi_{exe}(Q_i, T_{p_i}), G_i)
    \end{equation}
    
    Although this approach allows for a direct quality assessment through an external reference, it requires well-defined ground truth, making it unsuitable for open-ended tasks where ground truth may not always be available or practical to define.
    
(2) \textbf{Output \textit{vs.} Output (OvO):}
    When ground truth is unavailable, we turn to direct output comparison. The core idea behind OvO is that even in the absence of perfect ground truth, comparing outputs generated by different prompts can offer valuable signals about their relative quality. This method removes the dependency on external references and is particularly useful for open-ended tasks where multiple answers may be valid. It can be formally expressed as:
    
    \begin{equation}\small
        f_{OvO}(O_1, ..., O_k) = \phi_{eval}(\{\phi_{exe}(Q_i, P_{t_i})\}_{i=1}^k)
    \end{equation}

After introducing the \textbf{OvG} and \textbf{OvO} evaluation frameworks, we emphasize that \textbf{OvO} serves as the core method in Self-Supervised Prompt Optimization (\ours). By comparing outputs generated by different prompts, \textbf{OvO} provides valuable feedback on their relative quality without relying on external reference. This approach aligns with our objective of generating feedback directly from the outputs themselves, thus facilitating iterative optimization in both closed and open-ended tasks.

% 

\subsection{Sample Weight Averaging}

To mitigate the issue of low effective sample size and high variance in previous independence-based sample reweighting methods, we turn to bagging for inspiration. 
As a conventional ensemble learning strategy, bagging can decrease the estimation variance by averaging models trained on bootstrap-sampled data from the original dataset. 
Thus we consider designing a similar ensemble procedure. 
In order to generate diverse weighting functions, we note that in DWR, since the sample size, i.e. the number of parameters for sample weight learning, is much larger than the feature dimension, it could bear resemblance to the overparameterization characteristics of neural networks, e.g. one may anticipate the existence of multiple local minima when optimizing with gradient descent. The same is true for SRDO since the number of MLP parameters is much larger than the feature dimension. 
Consequently, we are likely to obtain diverse solutions even applying the same algorithm, as long as we vary elements of randomness like initialization. 
We theoretically substantiate this intuition in \Cref{prop:dwr} and \ref{prop:srdo}, and empirically confirm it in \Cref{fig:dist-comp} and \ref{fig:sim-comp} of \Cref{sec:synthetic}. 



Thus we propose SAmple Weight Averaging (SAWA) to improve covariate-shift generalization ability of independence-based sample reweighting algorithms. It learns multiple sets of sample weights by varying the random initialization of parameters $\boldtheta$ in weight learning. 
For DWR, we adopt standard normal distribution to initialize sample weights. For SRDO, we use Xavier Glorot Initialization \citep{glorot2010understanding} for the MLP-structured weighting function. 
Then we directly average the set of sample weights to yield the ensemble result. The entire procedure is described in \Cref{alg:sawa}.


\begin{algorithm}[t]
\caption{SAmple Weight Averaging (SAWA)} \label{alg:sawa}
\begin{algorithmic}
    \STATE {\bfseries Input:} 
    \item Dataset $[\boldsymbol{x}, \boldsymbol{y}]$, where $\boldsymbol{x}\in \mathbb{R}^{n\times p}, \boldsymbol{y}\in \mathbb{R}^{n\times 1}$. 
    \item Weight learning algorithm $\mathbb{A}$.  
    \item Number of averaged sets of sample weights $K$. 
    
    \STATE {\bfseries Output:} Sample weights $\bar{\boldsymbol{W}}$. 
    \STATE Initialize $\tilde{\boldsymbol{W}}$ as an empty list.
    \FOR {$k=1$ to $K$}
        \item Generate a random initialization $\boldtheta_0^{(k)}$. 
        \item Execute the weight learning algorithm to get the weighting function $w^{(k)}=\mathbb{A}(\boldsymbol{x}, \boldtheta_0^{(k)})$. 
        \item Calculate discrete sample weights $\boldsymbol{W}^{(k)}=w^{(k)}(\boldsymbol{x})$. 
        \item Add $\boldsymbol{W}^{(k)}$ to $\tilde{\boldsymbol{W}}$. 
    \ENDFOR
    \STATE Average sets of sample weights in $\tilde{\boldsymbol{W}}$ for the ensemble result $\bar{\boldsymbol{W}}$. 
    \STATE {\bfseries return: $\bar{\boldsymbol{W}}$} 
\end{algorithmic}
\end{algorithm}



Since the learning processes of these sets of sample weights can be easily parallelized, SAWA exhibits a low time cost in contrast to SVI \citep{yu2023stable}, which incurs a high time cost due to its iterative framework that can hardly be parallelized. 
Meanwhile, this strategy does not require information from outcome labels \citep{yu2023stable} or environment labels \citep{shen2020stable2}, and can be flexibly incorporated into any existing independence-based sample reweighting methods, since the weight learning algorithm $\mathbb{A}$ in Algorithm \ref{alg:sawa} can be DWR, SRDO, SVI or any other ones. 
Next, we provide theoretical results from two perspectives. Detailed proofs can be referred to in Appendix. 

\subsubsection{Validity of averaged sample weights}
\label{sec:validity}
We prove that the average of possible solutions is also a valid solution for weight learning. 

\begin{proposition}
\label{prop:dwr}
For a stronger version of DWR that constrains both weighted covariance and weighted mean equal to zero, when $n>\frac{p(p+1)}{2}+1$, it will have infinite solutions if solvable. Furthermore, the solution space is a convex set. 
\end{proposition}

\begin{proposition}
\label{prop:srdo}
For SRDO, when using the LSIF loss $\mathbb{E}_{\tilde{P}}[-w(\boldx)]+\mathbb{E}_{P}[w(\boldx)^2/2]$ to directly learn the density ratio, i.e. the weighting function $w$, if restricting $w$ coming from the linear parameterized weighting function family $\mathcal{W}_{\rm lin}=\{w_{\boldtheta}(\boldx)=a(\boldx)^T \boldtheta+b(\boldx)\ | \ a:\mathcal{X}\mapsto \mathbb{R}^p,b:\mathcal{X}\mapsto \mathbb{R}\}$, then the minima constitute a convex set. 
\end{proposition}

For \Cref{prop:srdo}, the weighting function family $\mathcal{W}_{\rm lin}$ is rich because functions $a$ and $b$ can arbitrarily change and the dimension of $\boldtheta$ can be very high. In our implementation of SRDO, we use MLP-structured weighting functions. With proper assumptions, we can use Neural Tangent Kernel (NTK) approximation \citep{lee2019wide} to include wide MLPs into $\mathcal{W}_{\rm lin}$. 
As the possible solutions constitute a convex set for both DWR and SRDO, the average of multiple optimization results also belongs to the set, thus also a possible optimization outcome of the corresponding weight learning algorithm. So we confirm the validity and rationality of sample weight averaging. Note that other reweighting algorithms like SVI are based on DWR and SRDO, to which \Cref{prop:dwr} and \ref{prop:srdo} can also be applied. 

\subsubsection{Benefits of decreasing error of weight learning and model parameter estimation}
\label{sec:benefits}
Following the theory of bagging \citep{ghojogh2019theory} and existing theoretical analyses for model parameter averaging \citep{rame2022diverse}, we come up with the following proposition. 
\begin{proposition}    
Denote $w$ as some desired weighting function in $\mathcal{W}_{\perp}$. 
Denote $w^E$ as the expected weighting function outputted by a single weight learning procedure over the joint distribution $P^g$ of training data $\boldsymbol{x}$ and random initialization $\boldtheta_0$, calculated as $w^E(\boldx)=\expect{g\sim P^g}{w^g(\boldx)}$, where $g=(\boldsymbol{x}, \boldtheta_0)$. 
Denote $\bar{w}$ as the average of the $K$ learned weighting functions, calculated as $\bar{w}(\boldx)=\frac1K \sum_{k=1}^K w^{(k)}(\boldx)$, where $w^{(k)}=\mathbb{A}(g^{(k}))$, $\{g^{(k)}\}_{k=1}^K$ are identically sampled from $P^g$, and all pairs of elements in $\{g^{(k)}\}_{k=1}^K$ shares the same covariance. 
Then expected estimation error of the averaged weighting function over $P^{te}$ and $P^g$ can be decomposed into the following three parts:
\begin{equation} \label{eq:decomp}
\small
\begin{aligned}
&\expect{\left\{g^{(k)}\right\}_{k=1}^K}{\expect{\boldx\sim P^{te}}{(\bar{w}(\boldx)-w(\boldx))^2}}\\
=& \mathop{\mathbb{E}}\limits_{\boldx\sim P^{te}}\Bigg[\left(w^E(\boldx)-w(\boldx)\right)^2\\
+&\frac1K \expect{g^{(k)}}{\left(w^{(k)}(\boldx)-w^E(\boldx)\right)^2}\\
+&\frac{K-1}{K}\expect{g^{(l)},g^{(m)} 
\atop 
l\neq m}{\left(w^{(l)}(\boldx)-w^E(\boldx)\right)\left(w^{(m)}(\boldx)-w^E(\boldx)\right)}\Bigg]
\end{aligned}
\end{equation}
\label{prop:decomp}
\end{proposition}

The first term of the right-hand side is $(w^E(\boldx)-w(\boldx))^2$, the squared bias of weight learning, solely related to the weight learning algorithm itself. It remains constant irrespective of the averaging strategy. 
The second term can be interpreted as the variance of weight learning, inversely proportional to $K$. Therefore, when we apply SAWA, an increase of $K$ results in a reduction of this variance component. 
The third term characterizes the degree of diversity present in sample weights. It depicts the correlation between two distinct weighting functions. By enhancing the diversity among weighting functions used for averaging, this term can be mitigated. This can elucidate the superiority of averaging sample weights from different initializations, as compared with moving average from the same initialization, which is popular in current DG research \citep{cha2021swad,arpit2022ensemble}. This finding aligns with the conclusion drawn by~\citet{rame2022diverse}. Relevant empirical analyses are in \Cref{fig:sim-comp} of \Cref{sec:synthetic}. 



Finally, following~\citet{xu2021stable}, we connect weight learning error with regression coefficient estimation error. 
\begin{proposition}
    Denote $\hat{\boldb}_{\bar{w}}$ as the model coefficient estimated by WLS using $\bar{w}$ with sample size $n$. 
    Denote $\boldb_w$ as the model coefficient estimated by WLS using some $w\in \mathcal{W}_{\perp}$ with infinite samples. 
    Denote $\Lambda_w$ as the smallest eigenvalue of the population-level weighted covariance matrix. 
    Then with mild assumptions, we have:
    \begin{equation}
        \left\|\hat{\boldb}_{\bar{w}}-\boldb_w\right\|\leq \frac{4\epsilon^2 M_w}{\left(\Lambda_w-\epsilon\sqrt{\expect{}{\|\boldx\|_2^4}}\right)^2}+O\left(\frac1n\right)
    \end{equation}
    where $\epsilon^2=\expect{\boldx\sim P^{tr}}{(\bar{w}(\boldx)-w(\boldx))^2}$ is the weight learning error, $M_w$ is a term only related to $w$. 
\label{prop:error}
\end{proposition}
\Cref{prop:error} reveals that as $n$ grows large enough, the dominated term in the bound of coefficient estimation error is positively related to the weight learning error. 
Notably, coefficients associated with unstable variables $\boldv$ in $\boldb_w$ are almost surely zero almost according to~\citet{xu2021stable}. 
Consequently, by refining the learning of sample weights, we can achieve improved estimations for the model coefficients on $\bolds$ and drive coefficients on $\boldv$ towards zero. Such refinement will lead to a stronger covariate-shift generalization ability and more stable prediction. 



\section{Self-Supervised Prompt Optimization}
\label{sec:method}

In this section, we first overview our method (Section~\ref{sub:overview}) and then analyze its effectiveness (Section~\ref{sec:method_essence}).

\begin{figure*}[t!]
  \centering
\includegraphics[width=\linewidth]{images/figure-method.pdf}
  \vspace{-2em}
  \caption{A Running Example of \ours Framework: Through pairwise evaluation on \highlight{output}{output}, \ours extract labels indicate which \highlight{prompt}{prompt} is \highlight{response}{better} and guide optimization. Furthermore, using a case from MT-bench, we show the complete process of \ours's $\phi_{opt}$, $\phi_{exe}$, and $\phi_{eval}$ and corresponding \highlight{metaprompt}{prompt}.}
  \label{fig:main_method}
  \vspace{-1em}
\end{figure*}

\subsection{An Overview of SPO}
\label{sub:overview}

A core challenge in reference-free prompt optimization is how to construct effective evaluation and optimization signals. We propose Self-Supervised Prompt Optimization (\ours), a simple yet effective framework that retains the basic Optimize-Execute-Evaluate loop while enabling reference-free optimization by leveraging only model outputs as both evaluation sources and optimization guidance.

As shown in Algorithm~\ref{alg:concise-algo-pipo}, \ours operates through three key components and the corresponding prompts are shown in Appendix \ref{appendix:prompt}: 

\begin{itemize}
\item Optimization function ($\phi_{opt}$): Generates new prompts by analyzing the current best prompt and its corresponding outputs.
\item Execution function ($\phi_{exe}$): Applies the generated prompts to obtain outputs.
\item Evaluation function ($\phi_{eval}$): Uses an LLM to compare outputs and determine the superior prompt through pairwise comparisons.
\end{itemize}

This iterative process begins with a basic prompt template (\eg Chain-of-Thought~\cite{wei2022COT}) and a small question set
sampled from the dataset. In each iteration, \ours generates new prompts, executes them, and performs pairwise evaluations of outputs to assess their adherence to task requirements.

The prompt associated with the superior output is selected as the best candidate for the next iteration. The process continues until a predefined maximum number of iterations is reached.

\textbf{A Running Example}
As illustrated in Figure~\ref{fig:main_method}, \ours achieves high efficiency, requiring only 8 LLM calls per iteration with three samples, significantly lower than existing methods~\cite{xin2024pa, chris2024pb, mert2024textgrad, 10720675, yong2023ape}. Despite its simplicity, \ours demonstrates superior performance across a range of tasks, as detailed in Section~\ref{sec:exp}. In the following section, we analyze the theoretical foundations of its effectiveness.


\begin{algorithm}[t!]
\small
\caption{An Overview of \ours.}
\label{alg:concise-algo-pipo}
\begin{algorithmic}[1]
\REQUIRE Dataset $D$
\ENSURE Optimized Prompt $P^*$
\STATE Initialize $P_0$; Sample 3 Questions $Q$ from $D$
\STATE $\text{Best Prompt } P^* \gets P_0$
\STATE $\text{Best Answer } A^* \gets \phi_{exe}(Q, P^*)$
\FOR{$iteration \gets 1$ to $N_{max}$}
    \STATE $P' \gets \phi_{opt}(P^*, A^*)$
    \STATE $A' \gets \phi_{exe}(Q, P')$
    \STATE $optimizationSuccess \gets \phi_{eval}(Q, A', A^*)$
    \IF{$optimizationSuccess$}
        \STATE $P^* \gets P'$
        \STATE $A^* \gets A'$
    \ENDIF
\ENDFOR
\STATE \textbf{return} $P^*$
\end{algorithmic}
\end{algorithm}


\subsection{Understanding the Effectiveness of \ours}
\label{sec:method_essence}


The theoretical foundation of \ours is built upon two key observations:

First, the outputs of LLMs inherently contain rich quality information that directly reflects prompt effectiveness, as evidenced by how step-by-step reasoning paths demonstrate the success of Chain-of-thought prompting~\cite{wei2022COT}. Second, LLMs exhibit human-like task comprehension, enabling them to assess answer quality and identify superior solutions based on task requirements. These complementary capabilities allow SPO to perform prompt evaluation and optimization without external references. These two aspects of utilizing model outputs work together to enable effective prompt optimization:

\textbf{Output as Optimization Guidance}
In terms of $\phi_{opt}$ design, unlike other methods that introduce explicit optimization signals \cite{chris2024pb, mert2024textgrad, reid2023protegi}, $\phi_{opt}$ optimizes directly based on the prompt and its corresponding outputs. The optimization signal stems from the LLMs' inherent ability to assess output quality, while the optimization behavior is guided by its understanding of what constitutes superior solutions. Therefore, even without explicit optimization signals, \ours's optimization essentially guides prompts toward the LLM's optimal understanding of the task.

\textbf{Output as Pairwise Evaluation Reference} 
Regarding $\phi_{eval}$ design, by employing the evaluation model to perform pairwise selection, we are effectively leveraging the evaluation model's inherent preference understanding of tasks. This internal signal can be obtained through simple pairwise comparisons of outputs, avoiding the need for large sample sizes to ensure scoring stability, which is typically required in score-based feedback methods.

 While we mitigate potential biases through four rounds of randomized evaluation, these biases cannot be completely eliminated~\cite{zhou2024zepo}. However, these biases do not affect the overall optimization trend because eval's feedback merely serves as a reference for the next round of optimization. The overall optimization process naturally aligns with the optimization model's task understanding, with the eval mechanism serving to validate the effectiveness of each iteration.




\section{Simulations and Experiment}
In this section, we conduct comprehensive experiments in both simulation and the real-world robot to address the following questions:
\begin{itemize}[leftmargin=*]
    \item \textbf{Q1(Sim)}: How does the \our policy perform in tracking across different commands?
    \item  \textbf{Q2(Sim)}: How to reasonably combine various commands in the general command space? % Command Analysis
    \item \textbf{Q3(Sim)}: How does large-scale noise intervention training help in policy robustness? % Ablation Study
    \item \textbf{Q4(Real)}: How does \our behave in the real world? % Real World Demo
\end{itemize}

\noindent\textbf{Robot and Simulator.} 
Our main experiments in this paper are conducted on the Unitree H1 robot, which has 19 Degrees of Freedom (DOF) in total, including 
two 3-DOF shoulder joints, two elbow joints, one waist joint, two 3-DOF hip joints, two knee joints, and two ankle joints.
The simulation training is based on the NVIDIA IsaacGym simulator~\citep{makoviychuk2021isaac}. It takes 16 hours on a single RTX 4090 GPU to train one policy.

\noindent\textbf{Command analysis principle and metric.}
One of the main contributions of this paper is an extended and general command space for humanoid robots. Therefore, we pay much attention to command analysis (regarding Q1 and Q2). This includes analysis of single command tracking errors, along with the combination of different commands under different gaits.
% we categorize the commands into three groups: \emph{movement}, \emph{foot}, and \emph{posture}. The \emph{movement} commands include the linear velocity and angular velocity, forming the foundational locomotion commands and are considered the most critical aspect of the tasks. The \emph{foot} commands include the gait frequency and foot swing height, representing the mode of leg movement. The \emph{posture} commands include body height, body pitch and waist yaw, which determine the desired body posture.
For analysis, we evaluate the averaged episodic command tracking error (denoted as $E_\text{cmd}$), which measures the discrepancy between the actual robot states and the command space using $L_1$ norm.
% The tracking error is measured in units of $m/s$, $rad/s$, $Hz$, $m$, and $rad$, corresponding to linear velocity, angular velocity, frequency, position, and rotation, respectively.
All commands are uniformly sampled within a pre-defined command range, as shown in \tb{tab:commands}\footnote{Note that the hopping gait keeps a different command range, due to its asymmetric type of motion. More details can be referred to \ap{ap:Hopping}.}.

%%%%%%%%%%%---SETME-----%%%%%%%%%%%%%
%replace @@ with the submission number submission site.
\newcommand{\thiswork}{INF$^2$\xspace}
%%%%%%%%%%%%%%%%%%%%%%%%%%%%%%%%%%%%


%\newcommand{\rev}[1]{{\color{olivegreen}#1}}
\newcommand{\rev}[1]{{#1}}


\newcommand{\JL}[1]{{\color{cyan}[\textbf{\sc JLee}: \textit{#1}]}}
\newcommand{\JW}[1]{{\color{orange}[\textbf{\sc JJung}: \textit{#1}]}}
\newcommand{\JY}[1]{{\color{blue(ncs)}[\textbf{\sc JSong}: \textit{#1}]}}
\newcommand{\HS}[1]{{\color{magenta}[\textbf{\sc HJang}: \textit{#1}]}}
\newcommand{\CS}[1]{{\color{navy}[\textbf{\sc CShin}: \textit{#1}]}}
\newcommand{\SN}[1]{{\color{olive}[\textbf{\sc SNoh}: \textit{#1}]}}

%\def\final{}   % uncomment this for the submission version
\ifdefined\final
\renewcommand{\JL}[1]{}
\renewcommand{\JW}[1]{}
\renewcommand{\JY}[1]{}
\renewcommand{\HS}[1]{}
\renewcommand{\CS}[1]{}
\renewcommand{\SN}[1]{}
\fi

%%% Notion for baseline approaches %%% 
\newcommand{\baseline}{offloading-based batched inference\xspace}
\newcommand{\Baseline}{Offloading-based batched inference\xspace}


\newcommand{\ans}{attention-near storage\xspace}
\newcommand{\Ans}{Attention-near storage\xspace}
\newcommand{\ANS}{Attention-Near Storage\xspace}

\newcommand{\wb}{delayed KV cache writeback\xspace}
\newcommand{\Wb}{Delayed KV cache writeback\xspace}
\newcommand{\WB}{Delayed KV Cache Writeback\xspace}

\newcommand{\xcache}{X-cache\xspace}
\newcommand{\XCACHE}{X-Cache\xspace}


%%% Notions for our methods %%%
\newcommand{\schemea}{\textbf{Expanding supported maximum sequence length with optimized performance}\xspace}
\newcommand{\Schemea}{\textbf{Expanding supported maximum sequence length with optimized performance}\xspace}

\newcommand{\schemeb}{\textbf{Optimizing the storage device performance}\xspace}
\newcommand{\Schemeb}{\textbf{Optimizing the storage device performance}\xspace}

\newcommand{\schemec}{\textbf{Orthogonally supporting Compression Techniques}\xspace}
\newcommand{\Schemec}{\textbf{Orthogonally supporting Compression Techniques}\xspace}



% Circular numbers
\usepackage{tikz}
\newcommand*\circled[1]{\tikz[baseline=(char.base)]{
            \node[shape=circle,draw,inner sep=0.4pt] (char) {#1};}}

\newcommand*\bcircled[1]{\tikz[baseline=(char.base)]{
            \node[shape=circle,draw,inner sep=0.4pt, fill=black, text=white] (char) {#1};}}

\subsection{Single Command Tracking}
We first analyze each command separately while keeping all other commands held at their default values. The results are shown in \tb{tab:Single commands}.
It is easily observed that the tracking errors in the walking and standing gaits are significantly lower than those in the jumping and hopping, with hopping exhibiting the largest tracking errors.
For hopping gaits, the robot may fall during the tracking of specific commands, like high-speed tracking, body pitch, and waist-yaw control.
This can be attributed to the fact that hopping requires rather high stability. Moreover, the complex postures and motions further exacerbate the risk of instability. Consequently, the policy prioritizes learning to maintain the balance, which, to some extent, compromises the accuracy of command tracking.

We conclude that the tracking accuracy of each gait aligns with the training difficulty of that gait in simulation. For example, the walking and standing patterns can be learned first during training, while the jumping and hopping gaits appear later and require an extended training period for the robot to acquire proficiency.
Similarly, the tracking accuracy of robots under low velocity is significantly better than those under high velocity, since 1) the locomotion skills under low velocity are much easier to master, and 2) the dynamic stability of the robot decreases at high speeds, leading to a trade-off with tracking accuracy.

We also found that the tracking accuracy for longitudinal velocity commands $v_x$ surpasses that of horizontal velocity commands $v_y$, which is due to the limitation of the hardware configuration of the selected Unitree H1 robots. In addition, the {foot swing height} $l$ is the least accurately tracked.
Furthermore, the tracking reward related to foot placement outperforms the tracking performance associated with posture control, since adjusting posture introduces greater challenges to stability. In response, the policy adopts more conservative actions to mitigate balance-threatening postural changes.
% In contrast, the influence of foot placement on stability is comparatively less pronounced, allowing for more precise tracking.

\begin{table}[t]
\setlength{\abovecaptionskip}{0.cm}
\setlength{\belowcaptionskip}{-0.cm}
\centering
\caption{\small \textbf{Single command tracking error.} The tracking errors for foot commands are calculated over a complete gait cycle, and the remaining ones are over one environmental step. For standing gait, we only tested the body height, body pitch, and waist yaw tracking error. $E^\text{high}$ and $E^\text{low}$ represents high-speed ($v_x > 1m/s$) and low-speed ($v_x \le 1m/s$) modes categorized by the linear velocity $v$. 
The tracking error is computed by sampling each command in a predefined range (\tb{tab:commands}) while keeping all other commands held at their default values.}
\label{tab:Single commands}
\resizebox{\columnwidth}{!}{
\begin{tabular}{@{}c|cccc|cc|ccc@{}}
\toprule
\multirow{3}{*}{Gait} & \multicolumn{4}{c|}{Movement} & \multicolumn{2}{c|}{Foot} & \multicolumn{3}{c}{Posture} \\
\cmidrule(l){2-5} \cmidrule{6-7} \cmidrule{8-10} 
& \multirow{2}{*}{\makecell{$E_{v_x}^\text{low}$\\($m/s$)}} & \multirow{2}{*}{\makecell{$E_{v_x}^\text{high}$\\($m/s$)}} & \multirow{2}{*}{\makecell{$E_{v_y}$\\($m/s$)}} & \multirow{2}{*}{\makecell{$E_{\omega}$\\$rad/s$}} & \multirow{2}{*}{\makecell{$E_{f}$\\($HZ$)}} & \multirow{2}{*}{\makecell{$E_{l}$\\($m$)}} & \multirow{2}{*}{\makecell{$E_{h}$\\($m$)}}  & \multirow{2}{*}{\makecell{$E_{p}$\\($rad$)}} & \multirow{2}{*}{\makecell{$E_{w}$\\($rad$)}}   \\ 
&  &  &  &  &  &  &  &  &    \\ 
\midrule
Standing  & - & - & - & - & - & - & 0.035 & 0.047 & 0.022  \\
Walking   & 0.030 & 0.216 & 0.085 & 0.054 & 0.028 & 0.011 & 0.064 & 0.038 & 0.075  \\
Jumping  & 0.090 & 0.532 & 0.069 & 0.077 & 0.027 & 0.012 & 0.058 & 0.048 & 0.022 \\
Hopping   & 0.033 & - & 0.046 & 0.078 & - & - & 0.103 & - & - \\
\bottomrule
\end{tabular}}
\end{table}



\begin{table*}[t]
\setlength{\abovecaptionskip}{0.cm}
\setlength{\belowcaptionskip}{-0.cm}
\centering
\caption{\small \textbf{Tracking errors with different intervention strategies under the walking gait}. We evaluate three upper-body intervention training strategies: Noise (\our), the AMASS dataset, and no intervention at all. The tracking errors across various task and behavior commands reflect the intervention tolerance, \textit{i.e.}, the ability of precise locomotion control under external intervention.}
\label{tab:Intervetion Tracking Error}
\begin{tabular}{c|c|ccc|cc|ccc}
\toprule
\multirow{3}{*}{Training Strategy} & \multirow{3}{*}{Intervention Task} & \multicolumn{3}{c|}{Task Commands}                        & \multicolumn{5}{c}{Behavior Commands}\\ \cmidrule{3-10}
 & & \multicolumn{3}{c|}{Movement}                        & \multicolumn{2}{c|}{Foot}          & \multicolumn{3}{c}{Posture}                         \\ \cmidrule{3-10}
                                      &                                      &$E_{v_x}$ ($m/s$)     & $E_{v_y}$ ($m/s$)   & $E_{\omega}$ ($rad/s$)    & $E_{f}$ ($Hz$)         & $E_{l}$ ($m$)         & $E_{h}$ ($m$)        & $E_{p}$ ($rad$)     & $E_{w}$ ($rad$)         \\ \midrule
\multirow{3}{*}{\makecell{Noise Curriculum\\(\our)}}        & Noise                        & \textbf{0.0483} & \textbf{0.0962} & \textbf{0.1879} & \textbf{0.0471} & \textbf{0.0542} & \textbf{0.0402} & \textbf{0.0432} & \textbf{0.0552} \\
                                      & AMASS                                & \textbf{0.0391} & \textbf{0.0920} & \textbf{0.1039} & \textbf{0.0464} & \textbf{0.0543} & \textbf{0.0387} & \textbf{0.0364} & \textbf{0.0540} \\
                                      & None                                 & \textbf{0.0264} & \textbf{0.0863} & \textbf{0.0543} & \textbf{0.0447} & \textbf{0.0522} & 0.0372          & 0.0375          & 0.0475          \\ \cmidrule{1-10}
\multirow{3}{*}{AMASS}                & Noise                        & 0.1697          & 0.1055          & 0.2156          & 0.0621          & 0.0542          & 0.0620          & 0.0812          & 0.0694          \\
                                      & AMASS                                & 0.0567          & 0.0965          & 0.1593          & 0.0466          & 0.0555          & 0.0579          & 0.0458          & 0.0554          \\
                                      & None                                 & 0.0645          & 0.0916          & 0.0802          & 0.0460          & 0.0531          & 0.0577          & 0.0455          & 0.0568          \\ \cmidrule{1-10}
\multirow{3}{*}{No Intervention}                 & Noise                        & 0.8658          & 0.7511          & 0.9116          & 0.1930          & 0.1913          & 0.1658          & 0.3622          & 0.2241          \\
                                      & AMASS                                & 0.6299          & 0.4026          & 0.5758          & 0.2245          & 0.2527          & 0.1305          & 0.2367          & 0.1112          \\
                                      & None                                 & 0.0755          & 0.1076          & 0.1151          & 0.0450          & 0.0678          & \textbf{0.0255} & \textbf{0.0211} & \textbf{0.0380} \\ \bottomrule
\end{tabular}
\end{table*}



\begin{table}[t]
\setlength{\abovecaptionskip}{0.cm}
\setlength{\belowcaptionskip}{-0.cm}
\centering
\caption{ \small
\textbf{Averaged foot displacement under intervention}. We compare foot displacement $D_\text{cmd}$ of different training strategies under various intervention tasks, which computes the total movement of both feet in one episode with sampled posture behavior commands.
}
\label{tab:Intervention Mean Foot Movement}
\resizebox{\linewidth}{!}{
\begin{tabular}{ccccc}
\toprule
Training Strategy                 & Intervention Task     & $D_{h}$ ($m/s$)                  & $D_{p}$ ($m/s$)      & $D_{w}$ ($m/s$)       \\ \midrule
\multirow{3}{*}{\makecell{Noise Curriculum\\(\our)}}  & Noise & \textbf{0.0339}             & \textbf{0.0892} & \textbf{0.0199} \\
                       & AMASS         & \textbf{0.0454}             & \textbf{0.0728} & \textbf{0.0196} \\
                       & None          & \textbf{0.0003}             & \textbf{0.0016} & \textbf{0.0007} \\ \midrule
\multirow{3}{*}{AMASS only} & Noise         & 2.0815                      & 2.8978          & 3.2630          \\
                       & AMASS         & 0.0536                      & 0.1743          & 0.0396          \\
                       & None          & 0.0139                      & 0.0160          & 0.0013          \\ \midrule
\multirow{3}{*}{No Intervention}  & Noise         & 17.5358                     & 17.9732         & 25.7132         \\
                       & AMASS         & 25.3802 & 26.3496         & 21.3078         \\
                       & None          & 0.0159  & 1.7065          & 1.7152          \\ \bottomrule
\end{tabular}}
\end{table}

\subsection{Command Combination Analysis}
To provide an in-depth analysis of the command space and to 
reveal the underlying interaction of various commands under different gaits.
Here, we aim to analyze the \emph{orthogonality} of commands based on the interference or conflict between the tracking errors of these commands across their reasonable ranges. For instance, when we say that a set of commands are \emph{orthogonal}, each command does not significantly affect the tracking performance of each other in its range. To this end, we plot the tracking error $E_\text{cmd}$ as heat maps, generated by systematically scanning the command values for each pair of parameters, revealing the correlation of each command.
We leave the full heat maps at \ap{ap:heatmaps}, and conclude our main observation for all gaits.

\noindent\textbf{Walking.} Walking is the most basic gait, which preserves the best performance of the robot hardware.
\begin{itemize}[leftmargin=*]
    \item The {linear velocity} $v_x$, the {angular velocity yaw} $\omega$, the {body height} $h$, and the {waist yaw} $w$ are orthogonal during walking.
    \item When the {linear velocity} $v_x$ exceeds $1.5m/s$, the orthogonality between $v_x$ and other commands decreases due to reduced dynamic stability and the robot's need to maintain body stability over tracking accuracy.
    \item The {gait frequency} $f$ shows discrete orthogonality, with optimal tracking performance at frequencies of 1.5 or 2. High-frequency gait conditions reduce tracking accuracy.
    \item The {linear velocity} $v_y$, the {foot swing height} $l$, and the {body pitch} $p$ are orthogonal to other commands only within a narrow range.
\end{itemize}

\noindent\textbf{Jumping.} The command orthogonality in jumping is similar to walking, but the overall orthogonal range is smaller, due to the increased challenge of the jumping gait, especially in high-speed movement modes.
During each gait cycle, the robot must leap forward significantly to maintain its speed. To execute this complex jumping action continuously, the robot must adopt an optimal posture at the beginning of each cycle. Both legs exert substantial torque to propel the body forward. Upon landing, the robot must quickly readjust its posture to maintain stability and repeat the actions. Consequently, during movement, the robot can only execute other commands within a relatively narrow range.

\noindent\textbf{Hopping.}
The hopping gait introduces more instability, and the robot's control system must focus more on maintaining balance, making it difficult to simultaneously handle complex, multi-dimensional commands.
\begin{itemize}[leftmargin=*]
    \item Hopping gait commands lack clear orthogonal relationships.
    \item Effective tracking is limited to the x-axis {linear velocity} $v_x$, the y-axis {linear velocity} $v_y$, the {angular velocity yaw} $\omega$, and the {body height} $h$.
    \item Adjustments to $h$ can be understood that a lower body height improves dynamic stability, therefore, it plays a positive role in maintaining the target body posture.
    % enhancing the robot's hopping performance.
\end{itemize}

\noindent\textbf{Standing.} As for the standing gait, we tested the tracking errors of commands related to posture. The results showed that the tracking errors were similar to those observed during walking with zero velocity.

\begin{itemize}[leftmargin=*]
    \item The {waist yaw} $w$ command is almost orthogonal to the other two commands.
    \item As the range of commands increases, orthogonality between the {body height} $h$ and the {body pitch} $p$ decreases. This is because the H1 robot has only one degree of freedom at the waist, limiting posture adjustments to the hip pitch joint.
    \item A 0.3 m decrease of the body height relative to the default height reduces the range of motion of the hip pitch joint to almost zero, hindering precise tracking of body pitch.
\end{itemize}

Furthermore, we conclude that {gait frequency} $f$ highly affects the tracking accuracy of \emph{movement} commands when it is excessively high and low; the \emph{posture} commands can significantly impact the tracking errors of other commands, especially when they are near the range limits.
% We categorize the commands into three groups: \emph{movement}, \emph{foot}, and \emph{posture}. 1) The \emph{movement} commands include the linear velocity $v_x, v_y$ and angular velocity $\omega$, forming the foundational locomotion commands, and are considered the most critical aspect of the tasks. 2) The \emph{foot} commands include the {foot swing height} $l$, which is the least accurately tracked; and the {gait frequency} $f$, which can affect the tracking accuracy of \emph{movement} commands when it is excessively high and low. 3) The \emph{posture} commands, which include body height $h$, the body pitch $p$, and waist yaw $w$, determine the desired body posture, and can significantly impact the tracking errors of other commands, especially when the command is challenging. 
For different gaits, the orthogonality range between commands is greatest in the walking gait and smallest in the hopping gait.

\subsection{Ablation on Intervention Training Strategy}
\label{sec:InterventionExp}
% The three policies use the same random seeds and training time.
To validate the effectiveness of the intervention training strategy on the policy robustness when external upper-body intervention is involved, we compare the policies trained with different strategies, including noise curriculum (\our), filtered AMASS data~\citep{he2024omnih2o}, and no intervention. We test the tracking errors under two different intervention tasks, \textit{i.e.}, uniform noise, AAMAS dataset, along with a no-intervention setup. The results under the walking gait are shown in \tb{tab:Intervetion Tracking Error}, and we leave other gaits in \ap{ap:SingleCommandsTracking-REMAIN}. 
It is obvious that the noise curriculum strategy of \our achieved the best performance under almost all test cases, except the posture-related tracking with no intervention. 
In particular, \our showed less of a decrease in tracking accuracy with various interventions, indicating our noise curriculum intervention strategy enables the control policy to handle a large range of arm movements, making it very useful and supportive for loco-manipulation tasks.
In comparison, the policy trained with AMASS data shows a significant decrease in the tracking accuracy when intervening with uniform noise, due to the limited motion in the training data. The policy trained without any intervention only performs well without external upper-body control.

It is worth noting that when intervention training is involved, the tracking error related to the movement and foot is also better than those of the policy trained without intervention, and \our provides the most accurate tracking. This shows that intervention training also contributes to the robustness of the policy. During our real robot experiments, we further observed that the robot behaves with a harder force when in contact with the floor, indicating a possible trade-off between motion regularization and tracking accuracy when involving intervention.

\noindent\textbf{Stability under standing gait.}
Adjusting posture in the standing state introduces additional requirements for stability, since the robot pacing to maintain balance may increase the difficulty of achieving manipulation tasks that require stand still. To investigate the necessity of noise curriculum for manipulation, we further measured the averaged foot displacement (in meters) under the standing gait, which computes the total movement of both feet in one episode (20 seconds) while tracking the posture behavior commands. Results in \tb{tab:Intervention Mean Foot Movement} show that \our exhibits minimal foot displacement. On the contrary, the strategy trained on AMASS data requires frequent small steps to adjust the posture and maintain stability for noise interventions. 
Without intervention training, the policy tends to tip over when involving intervention, leading to failure of the entire task.

%  鲁棒性测试的结果分析
\begin{figure}[t]
    \centering
    \includegraphics[width=\linewidth]{imgs/radar_chart_V2.pdf}
    \vspace{-13pt}
    \caption{\small \textbf{External disturbance tolerance}. Left: A constant and continuous force is applied to the robot. Right: A one-second force is exerted on the robot. The experiment is conducted under a standing gait with default commands. If the robot's survival ratio exceeds $98\%$, it is deemed capable of tolerating such external disturbance. 
    The survival ratio computes the trajectory ratio of non-termination (ends of timeout) during 4096 rollouts.}
    \label{fig:Robust}
    \vspace{-12pt}
\end{figure}
\noindent\textbf{Robustness for external disturbance.}
Finally, we test the contribution of intervention training and noise curriculum to the robustness of external disturbance. In particular, we evaluated the robot's maximum tolerance to external disturbance forces in eight directions and compared the policy trained without intervention. Results illustrated in \fig{fig:Robust} demonstrate that \our preserves greater tolerance for external disturbances in both pushing and loading scenarios across most of the directions. The reason behind this is that the intervention brings the robot exposed to various disturbances originating from its upper body, and thereby enhances the overall stability by dynamically adjusting leg strength.

% \our has a significantly higher tolerance for external disturbance forces in almost all directions compared to the strategy without intervention training.
% This is attributed to the fact that, during large-scale noise intervention training, the robot effectively explored a wide range of extreme scenarios and learned to enhance body stability by adjusting leg movements.

\subsection{Real-World Experiments}
We deploy \our on a real-world robot to verify its effectiveness. In \fig{fig:teaser}, we illustrate the humanoid capabilities supported by \our, showing the versatile behavior of the Unitree H1 robot. In particular, we demonstrate the intriguing potential of the comprehensive task range that \our is able to achieve, with a flexible combination of commands in high dynamics. To qualitatively analyze the performance of \our, we estimate the tracking error of two pose parameters (body pitch $p$ and waist rotation $w$ from the motor readings) on real robots, since other commands are hard to measure without a highly accurate motion capture system. The results are shown in \tb{tb:track-real}, where $E^{\text{real}}_{\text{cmd}}$ illustrates the tracking error of the posture command.
We observe that the tracking error in real-world experiments is slightly higher than in simulation environments, primarily due to sensor noise and the wear of the robot's hardware. Among different gaits, the tracking error for the waist rotation $w$ is smaller compared to that for the body pitch $p$, as waist control has less impact on the robot’s overall stability. In both error tests, the jumping gait exhibited the smallest $E_{cmd}$, while the walking gait showed slightly higher errors, consistent with the findings observed in the simulation environment.

\begin{table}[t]
\centering
\caption{\small \textbf{Tracking error in real world.} We conducted five tests to measure the tracking error for each command under three gaits. The tracking error for each command was calculated during each control step. The tested commands gradually increased from the minimum to the maximum values within a predefined range, while the remaining commands were kept at their default values.} % To account for the impact of communication delays on the actual tracking error, we introduced a 0.1-second delay in the command execution.
\label{tb:track-real}
\begin{tabular}{c|cc} \toprule
Gait     & $E_p^{\text{real}}$ & $E_w^{\text{real}}$ \\ \midrule
Standing & 0.0712 $\pm$ 0.0425 & 0.0718 $\pm$ 0.0614 \\
Walking  & 0.1006 $\pm$ 0.0581  & 0.0571 $\pm$ 0.0489 \\
Jumping  & 0.0674 $\pm$ 0.0569  & 0.0552 $\pm$ 0.0469 \\ \bottomrule
\end{tabular}
\end{table}



\section{Related Work}
Our work draws on and contributes to research in mobility aids and the built environment, online image-based survey for urban assessment, personalized routing applications and accessibility maps.

\subsection{Mobility Aids and the Built Environment}
People who use mobility aids (\textit{e.g.,} canes, walkers, mobility scooters, manual wheelchairs and motorized wheelchairs) face significant challenges navigating their communities.
Studies have repeatedly found that sidewalk conditions can significantly impede mobility among these users~\cite{bigonnesse_role_2018,fomiatti_experience_2014,f_bromley_city_2007,rosenberg_outdoor_2013, harris_physical_2015,korotchenko_power_2014}. 
In a review of the physical environment's role in mobility, \citet{bigonnesse_role_2018} summarized factors affecting mobility aid users, including uneven or narrow sidewalks (\textit{e.g.,}~\cite{fomiatti_experience_2014,f_bromley_city_2007}), rough pavements (\textit{e.g.,}~\cite{fomiatti_experience_2014,f_bromley_city_2007}), absent or poorly designed curb ramps (\textit{e.g.,}~\cite{rosenberg_outdoor_2013, f_bromley_city_2007, korotchenko_power_2014}), lack of crosswalks (\textit{e.g.,}~\cite{harris_physical_2015}), and various temporary obstacles (\textit{e.g.,}~\cite{harris_physical_2015}).

Though most research on mobility disability and the built environment has focused on wheelchair users~\cite{bigonnesse_role_2018}, mobility challenges are not experienced uniformly across different user populations~\cite{prescott_factors_2020, bigonnesse_role_2018}. 
For example, crutch users could overcome a specific physical barrier (such as two stairs down to a street), whereas motorized wheelchair users could not (without a ramp)~\cite{bigonnesse_role_2018}. 
Such variability demonstrates how person-environment interaction can differ based on mobility aids and environmental factors~\cite{sakakibara_rasch_2018,smith_review_2016}.
Further, mobility aids such as canes, crutches, or walkers are more commonly used than wheelchairs in the U.S.~\cite{taylor_americans_2014, firestine_travel_2024}: in 2022, approximately 4.7 million adults used a cane, crutches, or a walker, compared to 1.7 million who used a wheelchair~\cite{firestine_travel_2024}.
This underscores the importance of considering a diverse range of mobility aid users in urban accessibility research.
For example, \citet{prescott_factors_2020} explored the daily path areas of users of manual wheelchairs, motorized wheelchairs, scooters, walkers, canes, and crutches and found that the type of mobility device had a strong association with users' daily path area size.
Our study aims to further advance knowledge of how different mobility aid users perceive sidewalk barriers, with a more inclusive understanding of urban accessibility.

\begin{figure*}
    \centering
    \includegraphics[width=1\linewidth]{figures/figure-tutorial.png}
    \caption{Survey Part 2.1 showed all 52 images and asked participants to rate their passability based on their lived experience and use of their mobility aid. Above is the interactive tutorial we showed at the beginning of this part.}
    \Description{This figure shows a screenshot from the online survey. In survey part 2.1, participants were presented with 52 images and were asked to rate their passibility based on their lived experience and use of their mobility aid. The screenshot shows the interactive tutorial shown before this section.}
    \label{fig:survey-part2-instructions}
\end{figure*}

\subsection{Online Image-Based Survey for Urban Assessment}
Sidewalk barriers hinder individuals with mobility impairments not just by preventing particular travel paths but also by reducing confidence in self-navigating and decreasing one's willingness to travel to areas that might be physically challenging or unsafe~\cite{vasudevan_exploration_2016,clarke_mobility_2008}.
Prior work in this area traditionally uses three main study methods: in-person interviews (\textit{e.g}.~\cite{rosenberg_outdoor_2013,castrodale_mobilizing_2018}), GPS-based activity studies (\textit{e.g.,}~\cite{prescott_exploration_2021, prescott_factors_2020,rosenberg_outdoor_2013}), and online-questionnaires (\textit{e.g.,}~\cite{carlson_wheelchair_2002}). 
In-person interviews, while providing detailed and nuanced information, are limited by small sample sizes~\cite{rosenberg_outdoor_2013}. GPS-based activity studies involve tracking mobility aids user activity over a period of time, offering insights into movement patterns and activity space; however, these studies are constrained by geographical location~\cite{prescott_exploration_2021}. In contrast, online questionnaires can reach much larger populations and cover broader geographical regions, but they often yield high-level information that lacks the depth and nuance of the other approaches~\cite{carlson_wheelchair_2002}.
Our study aims to strike a balance between these approaches, capturing nuanced perspectives of mobility aid users about the built environment while maintaining a sufficiently large enough sample size for robust statistical analysis. 
Building on~\citet{bigonnesse_role_2018}'s work, we explore not only the types of factors considered to be barriers, but the \textit{intensity} of these barriers and their differential impacts.

Visual assessment of environmental features has long been employed by researchers across diverse fields, including human well-being~\cite{humpel_environmental_2002}, ecosystem sustainability~\cite{gobster_shared_2007}, and public policy~\cite{dobbie_public_2013}. 
These studies examine the relationship between images and the reactions they provoke in respondents or compare differences in reactions between groups.
Over the past decade, online visual preference surveys have gained popularity (\textit{e.g.,}~\cite{evans-cowley_streetseen_2014, salesses_collaborative_2013, goodspeed_research_2017}), where respondents are asked to make pairwise comparisons between randomly selected images.
Using this approach has two advantages: it adheres to the law of comparative judgment~\cite{thurstone_law_2017} by allowing respondents to make direct comparisons, and it prevents inter-rater inconsistency possible with scale ratings~\cite{goodspeed_research_2017}.
Additionally, online surveys generally offer advantages of increased sample sizes, reduced costs, and greater flexibility~\cite{wherrett_issues_1999}.
For people with disabilities, online surveys can be particularly beneficial. They help reach hidden or difficult-to-access populations~\cite{cook_challenges_2007,wright_researching_2005} and are believed to encourage more honest answers to sensitive questions~\cite{eckhardt_research_2007} by providing a higher level of anonymity and confidentiality~\cite{cook_challenges_2007, wright_researching_2005}.

\begin{figure*}
    \centering
    \includegraphics[width=1\linewidth]{figures/figure-comaprison-screenshot.png}
    \caption{In survey Part 2.2, participants were asked to perform a series of pairwise comparisons based on their 2.1 responses.}
    \Description{This figure shows a screenshot from the online survey. In Survey Part 2.2, participants were asked to perform a series of pairwise comparisons based on their 2.1 responses.}
    \label{fig:survey-part2b-pairwise}
\end{figure*}

\subsection{Personalized Routing Applications and Accessibility Maps}
Navigation challenges faced by mobility aid users can be mitigated through the provision of routes and directions that guide them to destinations safely, accurately, and efficiently~\cite{kasemsuppakorn_understanding_2015}. However, current commercial routing applications (\textit{e.g.}, \textit{Google Maps}) do not provide sufficient guidance for mobility aid users.
To address this gap, significant research has focused on routing systems for this population over the past two decades~\cite{barczyszyn_collaborative_2018, karimanzira_application_2006, matthews_modelling_2003, kasemsuppakorn_understanding_2015, volkel_routecheckr_2008, holone_people_2008, wheeler_personalized_2020, gharebaghi_user-specific_2021, ding_design_2007}.
One early, well-known prototype system is \textit{MAGUS}~\cite{matthews_modelling_2003}, which computes optimal routes for wheelchair users based on shortest distance, minimum barriers, fewest slopes, and limits on road crossings and challenging surfaces.
\textit{U-Access}~\cite{sobek_u-access_2006} provides the shortest route for people with three accessibility levels: unaided mobility, aided mobility (using crutch, cane, or walker), and wheelchair users.
However, U-Access only considers distance and ignores other
important factors for mobility aid users~\cite{barczyszyn_collaborative_2018}.
A series of projects by Kasemsuppakorn \textit{et al}.~\cite{kasemsuppakorn_personalised_2009, kasemsuppakorn_understanding_2015} attempted to create personalized routes for wheelchair users using fuzzy logic and \textit{Analytic Hierarchy Process} (AHP).

While influential, many personalized routing prototypes face limited adoption due to a scarcity of accessibility data for the built environment. 
Geo-crowdsourcing~\cite{karimi_personalized_2014}, a.k.a. volunteered geographic information (VGI)~\cite{goodchild_citizens_2007}, has emerged as an effective solution~\cite{karimi_personalized_2014, wheeler_personalized_2020}.
In this approach, users annotate maps with specific criteria or share personal experiences of locations, typically using web applications based on Google Maps or \textit{OpenStreetMap} (OSM)~\cite{karimi_personalized_2014}.
Examples include \textit{Wheelmap}~\cite{mobasheri_wheelmap_2017}, \textit{CAP4Access}~\cite{cap4access_cap4access_2014}, \textit{AXS Map}~\cite{axs_map_axs_2012}, and \textit{Project Sidewalk}~\cite{saha_project_2019}.
Recent research demonstrated the potential of using crowdsourced geodata for personalized routing~\cite{goldberg_interactive_2016, bolten_accessmap_2019,menkens_easywheel_2011, neis_measuring_2015}.
For example, \textit{EasyWheel}~\cite{menkens_easywheel_2011}, a mobile social navigation system based on OSM, provides wheelchair users with optimized routing, accessibility information for points of interest, and a social community for reporting barriers. 
\textit{AccessMap}~\cite{bolten_accessmap_2019} offers routing information tailored to users of canes, manual wheelchairs, or powered wheelchairs, calculating routes based on OSM data that includes slope, curbs, stairs and landmarks. 
Our work builds on the above by gathering perceptions of sidewalk obstacles from different mobility aid users to create generalizable profiles based on mobility aid type. We envision that these profiles can provide starting points in tools like Google Maps for personalized routing but can be further customized by the end user to specify additional needs (\textit{e.g.}, ability to navigate hills, \textit{etc.})

Beyond routing applications, our study data can contribute to modeling and visualizing higher-level abstractions of accessibility. 
Similar to \textit{AccessScore}~\cite{li_interactively_2018}, data from our survey can provide personalizable and interactive visual analytics of city-wide accessibility. By identifying both differences between mobility groups and common barriers within groups, we can develop analytical tools to prioritize barriers and assess the impact of their mitigation or removal, potentially benefiting the broadest range of mobility group users. Incorporating perceptions of passibility into urban planning processes provides a new dimension for urban planners' toolkits, which are often narrowly focused on compliance with ADA standards.




\section{Conclusion}

In this paper, we propose a sample weight averaging strategy to address variance inflation of previous independence-based sample reweighting algorithms. 
We prove its validity and benefits with theoretical analyses. 
Extensive experiments across synthetic and multiple real-world datasets demonstrate its superiority in mitigating variance inflation and improving covariate-shift generalization.  


% \clearpage
\section*{Impact Statement} 
\ours offers significant advancements in prompt engineering for LLMs, offering benefits such as democratized access, reduced costs, and improved performance across various tasks. However, it also carries risks, including potential bias amplification, misuse of harmful content generation, and over-reliance on LLMs. 


\bibliography{cited}
\bibliographystyle{icml2025}

\clearpage
\appendix
\onecolumn

\section{Appendix}
\section{Additional Related Works}
\label{sec:app-add-rel-works}
\subsection{Training Data Selection}

\begin{figure*}[!ht]
    \centering
    \includegraphics[width=\textwidth]{figs/per-token-loss-diff.pdf}
    \caption{Histograms of MIA signal of tokens. Each figure depicts a sample. Blue means the member samples while orange represents the non-member samples. We limited the y-axis range to -3 to 3 for better visibility, so it can result in missing several non-significant outliers.}
    \label{fig:add-per-token-loss}
\end{figure*}

Training data selection are methods that filter high-quality data from noisy big data \textit{before training} to improve the model utility and training efficiency. There are several works leveraging reference models~\cite{Coleman2020Selection, xie2023doremi}, prompting LLMs~\cite{li-etal-2024-one}, deduplication~\cite{lee2022deduplicating, kandpal2022deduplicating}, and distribution matching~\cite{kang2024get}. However, we do not aim to cover this data selection approach, as it is orthogonal and can be combined with ours.


\subsection{Selective Training}
Selective training refers to methods that \textit{dynamically choose} specific samples or tokens \textit{during training}. Selective training methods are the most relevant to our work. Generally, sample selection has been widely studied in the context of traditional classification models via online batch selection~\cite{loshchilov2016o, Angelosonl, pmlr-v108-kawaguchi20a}. These batch selection methods replace the naive random mini-batch sampling with mechanisms that consider the importance of each sample mainly via their loss values. ~\citet{2022PrioritizedTraining} indeed choose highly important samples from regular random batches by utilizing a reference model. However, due to the sequential nature of LLMs, which makes the training significantly different from the traditional classification ML, sample-level selection is not effective for language modeling~\cite{kaddour2023no}. \citet{lin2024not} extend the reference model-based framework to select meaningful tokens within batches. All of the previous methods for selective training aim to improve the training performance and compute efficiency. Our work is the first looking at this aspect for defending against MIAs.

\section{Token-level membership inference risk analysis}
Figures~\ref{fig:add-per-token-loss} and~\ref{fig:add-per-token-dynamics} present the analysis for additional samples. Generally, the trends are consistent with the one presented in Section~\ref{sec:analysis}.

\begin{figure*}[!ht]
    \centering
    \includegraphics[width=0.28\textwidth]{figs/mia-ranking_1.png}
    \includegraphics[width=0.28\textwidth]{figs/mia-ranking_2.png}
    \includegraphics[width=0.3\textwidth]{figs/mia-ranking_3.png}    
    \caption{MIA signal ranking of tokens during training. Each figure illustrates a sample.}
    \label{fig:add-per-token-dynamics}
\end{figure*}

\label{sec:app-analysis}

\section{Experiment settings}
\subsection{Implementation details}
\label{sec:app-implementation}
$\bullet$ \textbf{FT}. We implement the conventional fine tuning using Huggingface Trainer. We manually tune the learning rate to make sure no significant underfitting or overfitting. The batch size is selected appropriately to fit the physical memory and comparable with the other methods'.

\noindent $\bullet$ \textbf{Goldfish}. Goldfish is also implemented with Huggingface Trainer, where we custom the \texttt{compute\_loss} function. We implement the deterministic masking version rather than the random masking to make sure the same tokens are masked over epochs, potentially leading to better preventing memorization. The learning rate is also manually tuned, we noticed that the optimal Goldfish learning rate is usually slightly greater than FT's. This can be the gradients of two methods are almost similar, Goldfish just removes some tokens' contribution to the loss calculation. The batch size of FT can set as the same as FT, as Goldfish does not have significant overhead on memory.

\noindent $\bullet$ \textbf{DPSGD}. DPSGD is implemented by FastDP~\cite{bu2023zero}. We implement DPSGD with fastDP~\cite{bu2023zero} which offers state-of-the-art efficiency in terms of memory and training speed. We also use automatic clipping~\cite{bu2023automatic} and a mixed optimization strategy~\cite{mixclipping} between per-layer and per-sample clipping for robust performance and stability.

\noindent $\bullet$ \textbf{\methodname}. We implement \methodname using Huggingface Trainer, same as FT and Goldfish. The learning is reused from FT. The batch size of \methodname is usually smaller than FT and Goldfish when the model becomes large such as Pythia and Llama 2 due to the reference model, which consumes some memory.

For a fair comparison, we aim to implement the same batch size for all methods if feasible. In case of OOM (out of memory), we perform gradient accumulation, so all the methods can have comparable batch sizes. We provide the hyper-parameters of method for GPT2 in Table~\ref{tab:hyperparameter}. For Pythia and Llama 2, the learning rate, batch size, and number of epochs are tuned again, but the hyper-parameters regarding the privacy mechanisms remain the same. To make sure there is no naive overfitting, we evaluate the methods by selecting the best models on a validation set. Moreover, the testing and attack datasets remains identical for evaluating all methods. Additionally, we balance the number of member and non-member samples for MIA evaluation. It is worth noting that for the ablation study and analysis, if not state, the default model architecture and dataset are GPT2 and CC-news.

\begin{table*}[!ht]
    \centering
    \begin{tabular}{c|clc}
    \textbf{LLM} & \textbf{Method} & \textbf{Hyper-parameter} & \textbf{Value}  \\ \hline
     \multirow{22}{*}{\textbf{GPT2}}  &  \multirow{4}{*}{FT} &  Learning rate & 1.75e-5 \\ 
     & & Batch size & 96 \\
     & & Gradient accumulation steps & 1 \\
     & & Number of epochs & 20 \\ \cline{2-4}
       &  \multirow{5}{*}{Goldfish} &  Learning rate & 2e-5 \\ 
     & & Batch size & 96 \\
     & & Grad accumulation steps & 1 \\
     & & Number of epochs & 20 \\
     & & Masking Rate & 25\% \\ \cline{2-4}
           &  \multirow{6}{*}{DPSGD} &  Learning rate & 1.5e-3 \\ 
     & & Batch size & 96 \\
     & & Grad accumulation steps & 1 \\
     & & Number of epochs & 10 \\
     & & Clipping & automatic clipping \\ 
     & & Privacy budget & (8, 1e-5)-DP \\ \cline{2-4}
           &  \multirow{6}{*}{DuoLearn} &  Learning rate & 1.75e-3 \\ 
     & & Batch size & 96 \\
     & & Grad accumulation steps & 1 \\
     & & Number of epochs & 20 \\
     & & $K_h$ & 60\% \\ 
     & & $K_m$ & 20\% \\
     & & $\tau$ & 0 \\
     & & $\alpha$ & 0.8 \\ \hline
    \end{tabular}
    \caption{Hyper-parameters of the methods for GPT2.}
    \label{tab:hyperparameter}
\end{table*}


\section{Additional Results}
\label{sec:app-add-res}

\begin{figure}[!ht]
    \centering
    \includegraphics[width=0.8\linewidth]{figs/add_loss_vs_steps_ft_duolearn.pdf}
    \caption{Breakdown to the cross entropy loss values of FT on the testing set and \methodname on the training set during training.}
    \label{fig:add-overlap-breakdown}
\end{figure}

\subsection{Overall Evaluation}
% \begin{table*}[htp]
%     \centering
%     \begin{tabular}{cl|ccccc|ccccc}
%      \multirow{3}{*}{\textbf{LLM}}  & \multirow{3}{*}{\textbf{Method}} &  \multicolumn{5}{c|}{\textbf{CCNews}} & \multicolumn{5}{c}{\textbf{Wikipedia}} \\ \cmidrule(lr){3-7}  \cmidrule(lr){8-12}
%       &  & PPL & Loss & Ref & min-k & \multicolumn{1}{c|}{zlib} & PPL & Loss & Ref & min-k & zlib \\ \midrule
%       \multirow{4}{*}{GPT2} & \textit{Base} & \textit{29.442} & \textit{0.018} & \textit{0.002} & \textit{0.022} & \textit{0.006} & \textit{34.429} & \textit{0.002} & \textit{0.014} & \textit{0.010} & \textit{0.002} \\ 
%       \multirow{4}{*}{124M} & FT & \textbf{21.861} & 0.030 & 0.026 & 0.016 & 0.016 & \textbf{12.729} & 0.018 & 0.574 & 0.016 & 0.014 \\
%       & Goldfish & 21.902 & 0.030 & 0.024 & 0.028 & 0.016 & 12.853 & 0.018 & 0.632 & 0.016 & 0.010 \\
%       & DPSGD & 26.022 & \textbf{0.018} & \textbf{0.004} & \textbf{0.018} & 0.008 & 18.523 & \textbf{0.004} & 0.036 & 0.018 & 0.006 \\
%       & \methodname & 23.733 & 0.030 & 0.022 & 0.026 & \textbf{0.006} & 13.628 & 0.014 & \textbf{0.010} & \textbf{0.014} & \textbf{0.004} \\ \midrule
      
%       \multirow{4}{*}{Pythia} & \textit{Base} & \textit{13.973} & \textit{0.002} & \textit{0.008} & \textit{0.020} & \textit{0.014} & \textit{10.287} & \textit{0.002} & \textit{0.014} & \textit{0.006} & \textit{0.008} \\ 
%       \multirow{4}{*}{1.4B} & FT & 11.922 & 0.014 & 0.008 & 0.022 & 0.020 & \textbf{6.439} & 0.020 & 0.440 & 0.010 & 0.020 \\
%       & Goldfish & \textbf{11.903} & 0.014 & 0.008 & 0.024 & 0.018 & 6.465 & 0.016 & 0.412 & 0.010 & 0.020 \\
%       & DPSGD & 13.286 & \textbf{0.002} & \textbf{0.004} & \textbf{0.018} & \textbf{0.014} & 7.751 & \textbf{0.004} & \textbf{0.016} & {0.010} & \textbf{0.004} \\
%       & \methodname & 12.670 & 0.004 & 0.020 & \textbf{0.018} & 0.016 & 6.553 & 0.008 & 0.030 & \textbf{0.006} & 0.006 \\ \midrule
      
%       \multirow{4}{*}{Llama-2} & \textit{Base} & \textit{9.364} & \textit{0.006} & \textit{0.006} & \textit{0.024} & \textit{0.006} & \textit{7.014} & \textit{0.006} & \textit{0.016} & \textit{0.016} & \textit{0.010} \\ 
%       \multirow{4}{*}{7B} & FT & \textbf{6.261} & 0.002 & 0.018 & 0.002 & 0.002 & \textbf{3.830} & 0.028 & 0.170 & 0.030 & 0.028 \\
%       & Goldfish & 6.280 & 0.002 & 0.018 & 0.002 & 0.006 & 3.839 & 0.028 & 0.198 & 0.028 & 0.028 \\
%       & DPSGD & 6.777 & 0.008 & 0.026 & 0.016 & 0.010 & 4.490 & \textbf{0.006} & 0.014 & \textbf{0.020} & \textbf{0.010} \\
%       & \methodname & 6.395 & \textbf{0.002} & \textbf{0.020} & \textbf{0.004} & \textbf{0.002} & 4.006 & 0.010 & \textbf{0.002} & 0.028 & 0.012 \\ 
%     \end{tabular}
%     \caption{TPR at FPR of 1\% \textcolor{red}{TODO: check consistency with the main table of MIA AUC scores}}
%     \label{tab:tpr}
% \end{table*}


\begin{table*}[!ht]
  \centering
  \resizebox{\textwidth}{!}{\begin{tabular}{cl|ccccc|ccccc}
   \multirow{3}{*}{\textbf{LLM}}  & \multirow{3}{*}{\textbf{Method}} &  \multicolumn{5}{c|}{\textbf{Wikipedia}} & \multicolumn{5}{c}{\textbf{CC-news}} \\ \cmidrule(lr){3-7}  \cmidrule(lr){8-12}
    &  & PPL & Loss & Ref & min-k & \multicolumn{1}{c|}{zlib} & PPL & Loss & Ref & min-k & zlib \\ \midrule
    \multirow{4}{*}{GPT2} & \textit{Base} & \textit{34.429} & \textit{0.002} & \textit{0.014} & \textit{0.010} & \textit{0.002} & \textit{29.442} & \textit{0.018} & \textit{0.002} & \textit{0.022} & \textit{0.006} \\ 
    \multirow{4}{*}{124M} & FT & \textbf{12.729} & 0.018 & 0.574 & 0.016 & 0.014 & \textbf{21.861} & 0.030 & 0.026 & 0.016 & 0.016 \\
    & Goldfish & 12.853 & 0.018 & 0.632 & 0.016 & 0.010 & 21.902 & 0.030 & 0.024 & 0.028 & 0.016 \\
    & DPSGD & 18.523 & \textbf{0.004} & 0.036 & 0.018 & 0.006 & 26.022 & \textbf{0.018} & \textbf{0.004} & \textbf{0.018} & 0.008 \\
    & \methodname & 13.628 & 0.014 & \textbf{0.010} & \textbf{0.014} & \textbf{0.004} & 23.733 & 0.030 & 0.022 & 0.026 & \textbf{0.006} \\ \midrule
    
    \multirow{4}{*}{Pythia} & \textit{Base} & \textit{10.287} & \textit{0.002} & \textit{0.014} & \textit{0.006} & \textit{0.008} & \textit{13.973} & \textit{0.002} & \textit{0.008} & \textit{0.020} & \textit{0.014} \\ 
    \multirow{4}{*}{1.4B} & FT & \textbf{6.439} & 0.020 & 0.440 & 0.010 & 0.020 & 11.922 & 0.014 & 0.008 & 0.022 & 0.020 \\
    & Goldfish & 6.465 & 0.016 & 0.412 & 0.010 & 0.020 & \textbf{11.903} & 0.014 & 0.008 & 0.024 & 0.018 \\
    & DPSGD & 7.751 & \textbf{0.004} & \textbf{0.016} & {0.010} & \textbf{0.004} & 13.286 & \textbf{0.002} & \textbf{0.004} & \textbf{0.018} & \textbf{0.014} \\
    & \methodname & 6.553 & 0.008 & 0.030 & \textbf{0.006} & 0.006 & 12.670 & 0.004 & 0.020 & \textbf{0.018} & 0.016 \\ \midrule
    
    \multirow{4}{*}{Llama-2} & \textit{Base} & \textit{7.014} & \textit{0.006} & \textit{0.016} & \textit{0.016} & \textit{0.010} & \textit{9.364} & \textit{0.006} & \textit{0.006} & \textit{0.024} & \textit{0.006} \\ 
    \multirow{4}{*}{7B} & FT & \textbf{3.830} & 0.028 & 0.170 & 0.030 & 0.028 & \textbf{6.261} & 0.002 & 0.018 & 0.002 & 0.002 \\
    & Goldfish & 3.839 & 0.028 & 0.198 & 0.028 & 0.028 & 6.280 & 0.002 & 0.018 & 0.002 & 0.006 \\
    & DPSGD & 4.490 & \textbf{0.006} & 0.014 & \textbf{0.020} & \textbf{0.010} & 6.777 & 0.008 & 0.026 & 0.016 & 0.010 \\
    & \methodname & 4.006 & 0.010 & \textbf{0.002} & 0.028 & 0.012 & 6.395 & \textbf{0.002} & \textbf{0.020} & \textbf{0.004} & \textbf{0.002} \\ 
  \end{tabular}}
  \caption{Overall Evaluation: Perplexity (PPL) and TPR at FPR of 1\% scores of the MIAs with different signals (Loss/Ref/Min-k/Zlib). For all metrics, the lower the value, the better the result.}
  \label{tab:tpr}
\end{table*}
Table~\ref{tab:tpr} provides the True Positive Rate (TPR) at low False Positive Rate (FPR) of the overall evaluation. Generally, compared to CC-news, Wikipedia poses a significant higher risk at low FPR. For example, the reference-based attack can achieve a score of 0.57~ on GPT2 if no protection. In general, Goldfish fails to mitigate the risk in this scenario, while both DPSGD and \methodname offer robust protection.

\subsection{Auxiliary dataset}
We investigate the size of the auxiliary dataset which is disjoint with the training data of the target model and the attack model. In this experiment, the other methods are trained with 3K samples. Figure~\ref{fig:aux_size} presents the language modeling performance while varying the auxiliary dataset's size. The result demonstrates that the better reference model, the better language modeling performance. It is worth noting that even with a very small number of samples, \methodname can still outperform DPSGD. Additionally, there is only a little benefit when increasing from 1000 to 3000, this indicates that the reference model is not needed to be perfect, as it just serves as a calibration factor. This phenomena is consistent with previous selective training works~\cite{lin2024not, 2022PrioritizedTraining}.
\begin{figure}
    \centering
    \includegraphics[width=0.8\linewidth]{figs/auxiliary_size.pdf}
    \caption{Language modeling performance while varying the auxiliary dataset's size. Note that the results of FT and Goldfish are significantly overlapping.}
    \label{fig:aux_size}
\end{figure}

\subsection{Training time}
We report the training time for full fine-tuning Pythia 1.4B. We manually increase the batch size that could fit into the GPU's physical memory. As a results, FT and Goldfish can run with a batch size of 48, while DPSGD and \methodname can reach the batch size of 32. We also implement gradient accumulation, so all the methods can have the same virtual batch size.

\begin{table}[!ht]
    \centering
    \begin{tabular}{c|c}
        \textbf{Training Time} & \textbf{\textbf{1 epoch}} (in minutes) \\ \hline
        {FT} & 2.10 \\ 
        {Goldfish} & 2.10 \\
        % {RelaxLoss} & 2.10 \\        
        {DPSGD} & 3.19 \\ 
        {DuoLearn} & 2.85 
    \end{tabular}
    \caption{Training time for one epoch of (full) Pythia 1.4B on a single H100 GPU}
    \label{tab:training-time}
\end{table}

Table~\ref{tab:training-time} presents the training time for one epoch. Goldfish has little to zero overhead compared to FT. DPSGD and \methodname have a slightly higher training time due to the additional computation of the privacy mechanism. In particular, DPSGD has the highest overhead due to the clipping and noise addition mechanisms. Meanwhile, \methodname requires an additional forward pass on the reference model to select the learning and unlearning tokens. \methodname is also feasible to work at scale that has been demonstrated in the pretraining settings of the previous work~\cite{lin2024not}.

\section{Limitations}
The main limitation of our work is the small-scale experiment setting due to the limited computing resources. However, we believe \methodname can be directly applied to large-scale pretraining without requiring any modifications, as done in previous selective pretraining work~\cite{lin2024not}. Another limitation is the reference model, which may be restrictive in highly sensitive or domain-limited settings~\cite{tramr2024position}. From a technical perspective, while we show that \methodname performs well across different datasets and architectures, there is room for improvement. The current approach selects a fixed number of tokens, which may not be optimal since selected tokens contribute unequally. Future work could explore adaptive selection or weighted tokens' contribution. At a high-level, compared to DPSGD, \methodname has not been supported by theoretical guarantees. Future work can investigate the convergence and overfitting analysis.
\end{document}


