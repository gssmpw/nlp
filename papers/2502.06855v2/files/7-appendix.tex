
\setcounter{table}{0}  
\setcounter{figure}{0}
\setcounter{algorithm}{0}
\renewcommand{\thetable}{A\arabic{table}}
\renewcommand{\thefigure}{A\arabic{figure}}

\subsection{Detailed Prompts of \ours}
\label{appendix:prompt}


In this section, we present the Meta Prompt used for iteration. It should be noted that here we have only used the simplest and most straightforward Prompt. There is still room for improvement by optimizing the following Meta Prompt for specific domains.

\begin{tcolorbox}[title={\textbf{\small Optimize Function's Prompt}}, boxrule=2pt, arc=0mm, breakable]
\begin{minted}[fontsize=\scriptsize, breaklines, breakanywhere, frame=lines, framesep=2mm, tabsize=4, style=vs, autogobble]{python}
PROMPT = """You are building a prompt to address user requirement.Based on the given prompt, please reconstruct and optimize it. You can add, modify, or delete prompts. Please include a single modification in XML tags in your reply. During the optimization, you can incorporate any thinking models.
This is a prompt that performed excellently in a previous iteration. You must make further optimizations and improvements based on this prompt. The modified prompt must differ from the provided example.

requirements:
```
{requirements}
```

reference prompt:
```
{prompt}
```

The execution result of this reference prompt is(some cases):
```
{answers}
```

Provide your analysis, optimization points, and the complete optimized prompt using the following XML format:

<analyse>Analyze what drawbacks exist in the results produced by the reference prompt and how to improve them.</analyse>
<modification>Summarize the key points for improvement in one sentence</modification>
<prompt>Provide the complete optimized prompt</prompt>"""
\end{minted}
\end{tcolorbox}


This prompt template guides LLMs to iteratively improve existing prompts through structured XML analysis. It requires identifying weaknesses in reference prompt outputs, proposing modifications, and generating optimized versions. The template emphasizes incremental improvements while maintaining requirement compliance.

\begin{tcolorbox}[title={\textbf{\small Evaluate Function's Prompt}}, boxrule=2pt, arc=0mm, breakable]\begin{minted}[fontsize=\scriptsize, breaklines, breakanywhere, frame=lines, framesep=2mm, tabsize=4, style=vs, autogobble]{python}
PROMPT = """Based on the original requirements, evaluate the two responses, A and B, and determine which one better meets the requirements. If a reference requirement is provided, strictly follow the format/content of the reference requirement.

# Requirement
{requirement}

# A
{Answer_A}

# B
{Answer_B}


Provide your analysis and the choice you believe is better, using XML tags to encapsulate your response.

<analyse>Some analysis</analyse>
<choose>A/B (the better answer in your opinion)</choose>"""
\end{minted}
\end{tcolorbox}


The evaluation template uses comparative analysis to assess response quality. It requires XML-formatted reasoning that analyzes strengths/weaknesses of two responses (A/B) against requirements, followed by a definitive choice. 

\subsection{Detailed Prompt Template of Iteration Start}


\begin{tcolorbox}[title={\textbf{\small Iteration template on the BBH-navigate dataset}}, boxrule=2pt, arc=0mm, breakable]\begin{minted}[fontsize=\scriptsize, breaklines, breakanywhere, frame=lines, framesep=2mm, tabsize=4, style=vs, autogobble]{yaml}

prompt: |
  Please think step by step.
  Ensure the response concludes with the answer in the XML format: 
  <answer>[Yes or No]</answer>.

requirements: |
  Must put the final answer at the end with XML. (<answer>(Yes or No)</answer>,such as <answer>Yes</answer>)
  The provided prompt needs to adapt to all current types of questions.

faq:
  - question: |
        If you follow these instructions, do you return to the starting point? Always face forward. Take 7 steps left. Take 2 steps backward. Take 7 steps backward. Take 7 steps backward. Take 3 steps forward.
        Options:
        - Yes
        - No

    answer: |
        A lot of thinking and analysis processes.
        ...
        Final Answer:
        <answer>(Yes or No)</answer>

  - question: |
        If you follow these instructions, do you return to the starting point? Always face forward. Take 6 steps backward. Take 8 steps left. Take 3 steps right. Take 7 steps forward. Take 3 steps right. Take 9 steps right. Take 1 step backward. Take 7 steps left.
        Options:
        - Yes
        - No

    answer: |
        A lot of thinking and analysis processes.
        ...
        Final Answer:
        <answer>(Yes or No)</answer>

  - question: |
        If you follow these instructions, do you return to the starting point? Turn left. Turn left. Take 6 steps. Take 3 steps. Turn around. Take 1 step. Take 3 steps. Take 5 steps.
        Options:
        - Yes
        - No

    answer: |
        A lot of thinking and analysis processes.
        ...
        Final Answer:
        <answer>(Yes or No)</answer>
\end{minted}
\end{tcolorbox}



This YAML file demonstrates the initial configuration for our approach to iterating on the BBH-navigate task. By configuring a simple initial Prompt and requirements, along with three specific questions, iterative optimization can be performed. It should be noted that the content shown here is the complete content of the file, and the content in the answer section is not the actual answer but serves as a reference for the thought process and correct output format.

\subsection{Experiment Details}
\label{appendix:exp}

\subsubsection{Tasks and Data Details}

% \begin{table}[htbp]
% \caption{Dataset sizes and data splits.}
% \label{tab:dataset}
% \renewcommand\tabcolsep{3.2pt}
% \renewcommand\arraystretch{1.2}
% \small
% \setlength{\abovecaptionskip}{0.1cm}
% \setlength{\belowcaptionskip}{-0.2cm}
% \centering
% \begin{tabular}{l|ccc}
% \hline
% \textbf{Dataset Name} & \textbf{Test} & \textbf{Train\&Validate} & \textbf{Source} \\
% \hline
% LIAR & 461 & 3681 & {\tiny \url{https://github.com/microsoft/LMOps/tree/main/prompt_optimization/data/liar}} \\
% BBH-Navigate & 200 & 50 & {\tiny \url{https://huggingface.co/datasets/lukaemon/bbh}} \\
% WSC & 150 & 50 & {\tiny \url{https://github.com/RUCAIBox/GPO/tree/main/data}} \\
% AGIEval-MATH & 256 & 232 & {\tiny \url{https://github.com/ruixiangcui/AGIEval/blob/main/data/v1_1/math.jsonl}} \\
% GPQA & 198 & 250 & {\tiny \url{https://huggingface.co/datasets/Idavidrein/gpqa}} \\
% \hline
% MT-bench & 80 & 0 & {\tiny \url{https://github.com/lm-sys/FastChat/blob/main/fastchat/llm_judge/data/mt_bench/question.jsonl}} \\
% \hline
% \end{tabular}
% \end{table}

\begin{table}[htbp]
\caption{Dataset sizes and data splits.}
% Evaluation
\label{tab:dataset}
\renewcommand\tabcolsep{3.2pt}
\renewcommand\arraystretch{1.2}
\small
\setlength{\abovecaptionskip}{0.1cm}
\setlength{\belowcaptionskip}{-0.2cm}
\centering
% \resizebox{\textwidth}{!}{
\begin{tabular}{l|cc}
\hline

\hline

\hline

\hline
\textbf{Dataset Name} & \textbf{Test} & \textbf{Train\&Validate} \\
\hline

\hline
LIAR & 461 & 3681  \\
BBH-Navigate & 200 & 50  \\
WSC & 150 & 50  \\
AGIEval-MATH & 256 & 232 \\
GPQA & 198 & 250  \\
\hline
MT-bench & 80 & 0  \\

\hline

\hline

\hline

\hline
\end{tabular}
% }
\end{table}

\subparagraph{LIAR}
LIAR~\cite{william2017liar} is an English fake news detection dataset consisting of 4,000 statements, each accompanied by contextual information and lie labels. For our experiments, we sampled portions from the original dataset as test sets following~\citet{cilin2024erm}.
\subparagraph{BBH-Navigate}
BBH-Navigate~\cite{mirac2023bbh} is a task from the BIG-bench Hard dataset, a subset of the BIG Bench dataset. This task focuses on navigation reasoning, requiring the model to determine whether an agent, after following a series of navigation steps, returns to its starting point. For our experiments, we employed random sampling (seed=42) to obtain 200/25/25 test/train/validation splits.
\subparagraph{WSC}
The Winograd Schema Challenge (WSC)~\cite{hector2012wsc} is a benchmark designed to evaluate a system's ability to perform commonsense reasoning by resolving pronoun references in context. For our experiments, we sampled portions from the original dataset as test sets following~\citet{cilin2024erm}.
\subparagraph{AGIEval-MATH}
AGIEval-MATH~\cite{wan2024agieval} is a subset of the AGIEval benchmark, focusing on mathematical problem-solving tasks. It includes a variety of math problems designed to assess reasoning and computational abilities. For our experiments, we used Level 5 problems as the test set and Level 4 problems as the training and validation set.
\subparagraph{GPQA}
GPQA~\cite{david2023gpqa} is a dataset designed to evaluate the performance of language models on graduate-level questions across multiple disciplines, including biology, physics, and chemistry. For our experiments, we utilized the GPQA-Diamond subset as the test set, while constructing our training and validation set from questions that are exclusive to GPQA-main (i.e., those present in GPQA-main but absent from GPQA-Diamond).
\subparagraph{MT-bench}
MT-bench~\cite{lianmin2023mtbench} is a multi-task benchmark designed to evaluate the generalization capabilities of language models across diverse tasks, including text classification, summarization, and question answering. For our experiments, we selected \textit{writing}, \textit{roleplay}, and \textit{humanities} tasks from MT-Bench. These validation questions are provided in the supplementary materials.



\subsubsection{Configuration}
In our experiments, we configured different optimization frameworks to align their optimization costs as much as possible. These frameworks generally allow setting some parameters to adjust optimization costs, including the number of iterations and the number of prompts generated per iteration. 


\subparagraph{APE}
APE employs a three-round iterative optimization process, selecting the top 10\% (ratio=0.1) performing prompts from the current pool as elite prompts in each round. To maintain diversity and size of the prompt pool, variant sampling is used to mutate these elite prompts, keeping the total number of prompts at 50. Following the setting in original paper~\cite{yong2023ape}, the optimization process does not incorporate specific sample execution results to guide LLM prompt optimization. Instead, performance scores are obtained by evaluating prompts on the entire training set.

\subparagraph{OPRO}
OPRO uses a 10-round iterative optimization process, generating 10 candidate prompts per round. OPRO evaluates prompt performance on the complete training set and filters based on evaluation scores. OPRO doesn't maintain a fixed-size prompt pool but directly generates new candidates based on the current best prompt in each round. The optimization direction is guided through performance evaluation on the full training data.

\subparagraph{PromptAgent}
Except for the Liar dataset, on which we sampled 150 data for both training and validation, all other datasets follow the sizes specified in Table~\ref{tab:dataset}. PromptAgent uses a Monte Carlo Tree Search (MCTS) framework to optimize prompts. It starts with an initial prompt and generates new candidates based on model error feedback. The process is guided by evaluations with benchmark on a sampled training set to identify high-reward paths for improved task performance. Finally, we select the top 5 prompts that perform best on the validation set for testing and choose the optimal one. Key parameters of MCTS include an expand width of 3, a depth limit of 8, and 12 iterations. 

\subparagraph{PromptBreeder}
In our implementation of PromptBreeder, we configure the system with 5 mutation prompts and 5 thinking styles for initialization. The evolution process runs for 20 generations, with 20 evaluations performed on randomly sampled training examples in each generation. The optimization model defaults to Claude-3.5-Sonnet and the execution model defaults to GPT-4o-mini.

\subparagraph{TextGrad}
Except for the Liar dataset where the Train\&Validate set is reduced to 50 samples, all other datasets follow the sizes specified in Table~\ref{tab:dataset}. TextGrad employs a three-epoch optimization process with three steps per epoch (epoch=3, steps=3), using a batch size of 3 for stochastic gradient descent. In each step, TextGrad generates gradients through back-propagation of feedback from the optimizer LLM (Claude-3.5-Sonnet) to update the system prompt. The framework maintains a validation-based reversion mechanism - if the updated prompt performs worse on the validation set compared to the previous iteration, the update is rejected and the prompt reverts to its previous state. The optimization process is guided by evaluating prompts using Claude-3.5-Sonnet as the evaluation LLM, while the actual task execution uses GPT-4o-mini as the execution LLM. Our experimental configuration follows the prompt optimization setting provided in the official TextGrad repository~\cite{mert2024textgrad}.

\subparagraph{\ours}
\ours conducts optimization through 10 iterations per task, randomly selecting 3 questions (without answers) from the pre-split Train\&Validate dataset for each iteration. The optimization model defaults to Claude-3.5-Sonnet, the evaluation model defaults to GPT-4o-mini, and the execution model defaults to GPT-4o-mini. Notably, \ours demonstrates effective prompt optimization using only questions without ground truth answers, validating its capability.

\subsubsection{Baseline Prompt}


In this section, we provide the Baseline Prompts for comparison. Note that for all Prompt Optimization work requiring initial iteration prompts, we consistently provide the COT Prompt shown below.

\begin{tcolorbox}[title={\textbf{\small IO Prompt}}, boxrule=2pt, arc=0mm, breakable]
\begin{minted}[fontsize=\scriptsize, breaklines, breakanywhere, frame=lines, framesep=2mm, tabsize=4, style=vs, autogobble]{markdown}

Ensure the response concludes with the answer in the format: 
<answer>answer</answer>

\end{minted}
\end{tcolorbox}

\begin{tcolorbox}[title={\textbf{\small COT Prompt}}, boxrule=2pt, arc=0mm, breakable]
\begin{minted}[fontsize=\scriptsize, breaklines, breakanywhere, frame=lines, framesep=2mm, tabsize=4, style=vs, autogobble]{markdown}

Please think step by step.
Ensure the response concludes with the answer in the format:
<answer>answer</answer>.

\end{minted}
\end{tcolorbox}

\begin{tcolorbox}[title={\textbf{\small Step-back Prompt}}, boxrule=2pt, arc=0mm, breakable]
\begin{minted}[fontsize=\scriptsize, breaklines, breakanywhere, frame=lines, framesep=2mm, tabsize=4, style=vs, autogobble]{markdown}

Please first think about the principles involved in solving this task which could be helpful.
And Then provide a solution step by step for this question.
Ensure the response concludes with the answer in the format:
<answer>answer</answer>.

\end{minted}
\end{tcolorbox}

\begin{tcolorbox}[title={\textbf{\small Rephrase Prompt}}, boxrule=2pt, arc=0mm, breakable]
\begin{minted}[fontsize=\scriptsize, breaklines, breakanywhere, frame=lines, framesep=2mm, tabsize=4, style=vs, autogobble]{markdown}

Please rephrase the question in a way that is easier to understand, minimizing ambiguity and considering edge cases.
And Then provide a solution step by step for the question.
Ensure the response concludes with the answer in the format:
<answer>answer</answer>.

\end{minted}
\end{tcolorbox}

\subsubsection{Prompt Optimized by \ours}

In this section, we present the optimized prompts obtained from our main experiments, where Claude-3.5-Sonnet serves as the optimization model, and GPT-4o-mini serves as both the evaluation and execution model.

\begin{tcolorbox}[title={\textbf{\small GPQA Prompt}}, boxrule=2pt, arc=0mm, breakable]
\begin{minted}[fontsize=\scriptsize, breaklines, breakanywhere, frame=lines, framesep=2mm, tabsize=4, style=vs, autogobble]{markdown}

Follow these guidelines to answer questions efficiently and effectively:

1. Carefully read the entire question, identifying all relevant information and key concepts.
2. Choose the most appropriate problem-solving approach based on the question type.
3. Solve the problem using these steps:
   a. State any relevant formulas, principles, or assumptions
   b. Show all necessary calculations or conceptual analysis
   c. Evaluate all answer options, explaining why incorrect options are wrong when relevant
4. Structure your response as follows:

[Analysis]
- Briefly state the main problem and key information (2-3 sentences max)

[Solution]
- Show your work step-by-step, including all relevant calculations and reasoning
- For conceptual questions, provide a clear, logical explanation

[Conclusion]
- State the final answer in one clear sentence
- Briefly explain why this answer is correct and others are incorrect (if applicable)

<answer>[One letter representing the correct option]</answer>

Adapt this structure as needed for different question types, prioritizing clarity and conciseness. Ensure that your response addresses all aspects of the question and demonstrates a clear problem-solving process.

\end{minted}
\end{tcolorbox}

\begin{tcolorbox}[title={\textbf{\small BBH-Navigate Prompt}}, boxrule=2pt, arc=0mm, breakable]
\begin{minted}[fontsize=\scriptsize, breaklines, breakanywhere, frame=lines, framesep=2mm, tabsize=4, style=vs, autogobble]{markdown}

Follow these steps to analyze the given instructions:

1. State the initial conditions:
   - Starting point: (0, 0)
   - Initial direction: positive x-axis (unless specified otherwise)

2. Use a coordinate system:
   - x-axis: left (-) and right (+)
   - y-axis: backward (-) and forward (+)

3. Analyze each step:
   - For ambiguous instructions (e.g., "Take X steps" without direction), assume forward movement
   - Update coordinates after each step
   - Briefly explain any assumptions made

4. After analyzing all steps:
   - Summarize total movement in each direction
   - State the final position

5. Compare final position to starting point:
   - Calculate the distance from (0, 0)

6. Provide concise reasoning, labeled as "Reasoning:"
   - Explain key movements and their impact on position
   - Justify your conclusion

7. State your final answer, labeled as "Final Answer:"

Conclude your response with the answer in this XML format:
<answer>[Yes or No]</answer>

Ensure your analysis adapts to all question types, handling both specific and ambiguous instructions.

\end{minted}
\end{tcolorbox}

\begin{tcolorbox}[title={\textbf{\small LIAR Prompt}}, boxrule=2pt, arc=0mm, breakable]
\begin{minted}[fontsize=\scriptsize, breaklines, breakanywhere, frame=lines, framesep=2mm, tabsize=4, style=vs, autogobble]{markdown}

Analyze the given statement(s) carefully, following these steps for each question:

1. Consider the statement, speaker's background (if provided), and context.
2. Research and cite relevant facts and data related to the claim.
3. Evaluate the claim's validity based on available evidence.
4. Consider potential biases or motivations of the speaker.

For each claim, structure your analysis as follows:

CLAIM: [Restate the claim]
CONTEXT: [Summarize speaker background and context]
FACT-CHECK:
- [Key fact 1 with source]
- [Key fact 2 with source]
- [Add more as needed]
ANALYSIS: [Provide a brief analysis of the claim's validity]
RATING: [Use the following scale]
1 = False
2 = Mostly False
3 = Mixed / Unclear
4 = Mostly True
5 = True
UNCERTAINTY: [Low / Medium / High - based on the quality and quantity of available evidence]
SUMMARY: [One-sentence conclusion about the claim's validity]

If there is insufficient information to make a determination, clearly state this and explain what additional information would be needed.

Conclude each analysis with a final true/false judgment in XML tags:
<answer>[Yes or No]</answer>

After analyzing all claims, provide an overall summary of your findings.

Ensure your response is objective, based on verifiable information, and cites sources for all key facts used in the analysis.

\end{minted}
\end{tcolorbox}

\begin{tcolorbox}[title={\textbf{\small AGIEval-MATH Prompt}}, boxrule=2pt, arc=0mm, breakable]
\begin{minted}[fontsize=\scriptsize, breaklines, breakanywhere, frame=lines, framesep=2mm, tabsize=4, style=vs, autogobble]{markdown}

1. Quickly identify key information
2. Devise fastest solution strategy
3. Execute minimal calculations
4. Double-check answer

Rules:
- Use shortcuts and estimation when possible
- Skip detailed explanations
- Omit units in final answer
- Always present final answer as: \boxed{}

Prioritize speed and accuracy. Manage time efficiently. Aim for shortest valid solution. Verify answer format before submission. If stuck, make educated guess rather than spending too much time.

\end{minted}
\end{tcolorbox}

\begin{tcolorbox}[title={\textbf{\small WSC Prompt}}, boxrule=2pt, arc=0mm, breakable]
\begin{minted}[fontsize=\scriptsize, breaklines, breakanywhere, frame=lines, framesep=2mm, tabsize=4, style=vs, autogobble]{markdown}

Analyze the question using these steps:
1. Identify key elements and relationships.
2. Consider multiple interpretations.
3. Evaluate evidence for each possibility.
4. Draw a logical conclusion.
5. Summarize your reasoning.
Conclude with:
Final Answer:
<answer>[Single letter option]</answer>

\end{minted}
\end{tcolorbox}


\subsection{Case Study}
\label{appendix:case-study}
\subsubsection{Optimization Trajectory}
\label{appendix:trajectory}
We present the prompt optimization trajectory of GPT-4o-mini on the BBH-navigate dataset, where it serves as the optimization model, evaluation model, and execution model. This includes whether each iteration was successful relative to the best prompt at that time, as well as the corresponding prompt content.

\newcommand{\successbox}[2]{%
    \begin{tcolorbox}[
        title={\textbf{\small Iteration #1 - Success}},
        colframe=green!40!black!90!white,    
        colback=green!10!white!95!black,    
        coltitle=white,
        colbacktitle=green!40!black!90!white,
        boxrule=2pt,
        arc=0mm,
        left=8pt,
        right=8pt,
        top=8pt,
        bottom=8pt,
        fonttitle=\sffamily\bfseries,
        verbatim
    ]
    #2
    \end{tcolorbox}
}

\newcommand{\failedbox}[2]{%
    \begin{tcolorbox}[
        title={\textbf{\small Iteration #1 - Failed}},
        colframe=red!40!black!90!white,      
        colback=red!10!white!95!black,       
        coltitle=white,
        colbacktitle=red!40!black!90!white,  
        boxrule=2pt,
        arc=0mm,
        left=8pt,
        right=8pt,
        top=8pt,
        bottom=8pt,
        fonttitle=\sffamily\bfseries,
        verbatim
    ]
    #2
    \end{tcolorbox}
}



\successbox{0}{
Please think step by step.
Ensure the response concludes with the answer in the following format:
<answer>[Yes or No]</answer>
}

\failedbox{1}{
Please analyze the following question step by step, ensuring clarity in your reasoning. At the conclusion of your analysis, provide the final answer in the specified XML format: <answer>(Yes or No)</answer>.<br/><br/>Question: [Insert Question Here]
}

\failedbox{2}{
Please analyze the given instructions step by step and summarize your reasoning clearly. Ensure the response concludes with the answer in the XML format: 
<answer>(Yes or No)</answer>.
}

\failedbox{3}{
Please analyze the provided steps thoroughly and document each movement. Ensure your reasoning is clear and logical throughout. Conclude with the final answer in the following XML format: <answer>(Yes or No)</answer>.
}

\successbox{4}{
Please approach the question with a clear and logical reasoning process, breaking it down step by step. Ensure to summarize your findings before stating the final answer clearly in this XML format: <answer>[Yes or No]</answer>. 

For example, if the question is about returning to the starting point after a series of movements, detail each movement, updating your position accordingly, and conclude whether you have returned to the starting point or not.

Final Answer:
<answer>(Yes or No)</answer>
}

\successbox{5}{
Please analyze the question thoroughly by breaking it down step by step. Clearly outline each movement or action, updating your position or state accordingly. Conclude with a summary of your findings to determine whether you return to the starting point or achieve the stated outcome. Finally, present the answer distinctly in this XML format: <answer>[Yes or No]</answer>.

For example, if asked about returning to the starting point after several movements, document each movement clearly, track your position, and specify if you have returned to the starting point.

Final Answer:
<answer>(Yes or No)</answer>
}

\successbox{6}{
Please analyze the question methodically, breaking it down into clear, sequential movements. Define the starting point and the direction of each movement, updating the position accordingly after each step. At the end, summarize the journey and explicitly state whether the final position returns to the starting point or achieves the desired outcome. Present the final answer distinctly in this XML format: <answer>[Yes or No]</answer>.

For example, if asked about returning to the starting point after a series of movements, document each step clearly, track your position, and specify if you have returned to the starting point.

Final Answer:
<answer>(Yes or No)</answer>
}

\failedbox{7}{
Please evaluate the question step-by-step, clearly outlining each movement and tracking the resulting position. At the conclusion of your analysis, summarize the journey and state whether the final position returns to the starting point or meets the desired criteria. Ensure the final answer is presented distinctly in this XML format: <answer>[Yes or No]</answer>.

For instance, if asked about returning to the starting point after a series of movements, succinctly document each step, track your position, and indicate if you have returned to the starting point.

Final Answer:
<answer>(Yes or No)</answer>
}

\successbox{8}{
Please methodically analyze the question by breaking it down into concise movements. Start from a defined position and track each step's impact on your location. After detailing each movement, summarize the overall journey and explicitly state whether the final position returns to the starting point or reaches the desired outcome. Present the final answer distinctly in the following XML format: <answer>Yes or No</answer>.

For example, if asked whether following a series of movements returns you to the starting point, document each step clearly, track your position accurately, and specify the result.

Final Answer:
<answer>(Yes or No)</answer>
}

\failedbox{9}{
Please begin by outlining your movements systematically. Start from a defined position and detail each step you take, including the direction and distance moved. After completing your movements, summarize the overall journey, and clearly state whether your final position returns to the starting point or achieves the desired outcome. Make sure to present your movements alongside the final answer in the following XML format: <answer>Yes or No</answer>.

For example, if asked whether following a series of movements returns you to the starting point, you should document each step, track your position accurately, and specify the result. 

Final Answer:
<answer>(Yes or No)</answer>
}

\successbox{10}{
Please thoroughly analyze the provided movement instructions by documenting each step clearly. Begin by defining your initial position and systematically track your movements, updating your position after each step. After completing the movements, summarize your journey and determine if your final position coincides with the starting point or achieves the desired outcome. Clearly indicate your final answer using the XML format: <answer>Yes or No</answer>.

For instance, if the question asks whether a sequence of movements returns you to the starting point, methodically outline each step, track your position accurately, and conclude with the result.

Final Answer:
<answer>(Yes or No)</answer>
}


The success/failure boxes demonstrate \ours' optimization process through 10 iterations of a BBH-navigation prompt. Successful versions (green) gradually improve the prompt to elicit better thought processes, while failed attempts (red) may lead to oversimplified or redundant thinking due to suboptimal prompts. This trajectory shows the system's trial-and-error refinement mechanism. 

\begin{tcolorbox}[title={\textbf{\small Optimize Function's Prompt}}, boxrule=2pt, arc=0mm, breakable]
\begin{minted}[fontsize=\scriptsize, breaklines, breakanywhere, frame=lines, framesep=2mm, tabsize=4, style=vs, autogobble]{python}

PROMPT = """Please think step by step.
Ensure the response concludes with the answer in the XML format: 
<answer>[Yes or No]</answer>."""

Question = """If you follow these instructions, do you return to the starting point? Always face forward. Take 1 step right. Take 3 steps left. Take 2 steps right.
Options:
- Yes
- No"""

Answer = """Let's analyze the steps one by one:
1. **Starting Point**: We begin at the starting point, facing forward.
2. **Step 1**: Take 1 step right. Now we are 1 step to the right of the starting point.
3. **Step 2**: Take 3 steps left. From our current position (1 step right), moving 3 steps left means:
   - 1 step left brings us back to the starting point.
   - 2 steps left takes us 1 step to the left of the starting point.
   - 3 steps left takes us 2 steps to the left of the starting point.
4. **Step 3**: Take 2 steps right. From our current position (2 steps left), moving 2 steps right means:
   - 1 step right brings us back to 1 step left of the starting point.
   - 2 steps right brings us back to the starting point.
After completing all the steps, we end up back at the starting point.
Therefore, the answer is:
<answer>Yes</answer>"""


\end{minted}
\end{tcolorbox}

\begin{tcolorbox}[title={\textbf{\small Optimize Function's Prompt}}, boxrule=2pt, arc=0mm, breakable]
\begin{minted}[fontsize=\scriptsize, breaklines, breakanywhere, frame=lines, framesep=2mm, tabsize=4, style=vs, autogobble]{python}

PROMPT = """Please thoroughly analyze the provided movement instructions by documenting each step clearly. Begin by defining your initial position and systematically track your movements, updating your position after each step. After completing the movements, summarize your journey and determine if your final position coincides with the starting point or achieves the desired outcome. Clearly indicate your final answer using the XML format: <answer>Yes or No</answer>.

For instance, if the question asks whether a sequence of movements returns you to the starting point, methodically outline each step, track your position accurately, and conclude with the result.

Final Answer:
<answer>(Yes or No)</answer>"""

Question = """If you follow these instructions, do you return to the starting point? Always face forward. Take 1 step right. Take 3 steps left. Take 2 steps right.
Options:
- Yes
- No"""

Answer = """Let's analyze the movement instructions step by step.
1. **Initial Position**: We start at the origin, which we can define as (0, 0).
2. **Facing Direction**: We are facing forward, which we will assume is towards the positive y-axis.
3. **Movement Instructions**:
   - **Step 1**: Take 1 step right.
     - Moving right from (0, 0) means moving along the x-axis. The new position is (1, 0).
   - **Step 2**: Take 3 steps left.
     - Moving left means moving in the negative x-direction. From (1, 0), we move to (0, 0) after 1 step left, then to (-1, 0) after 2 steps left, and finally to (-2, 0) after 3 steps left.
   - **Step 3**: Take 2 steps right.
     - Moving right from (-2, 0) means moving back towards the positive x-direction. After 1 step right, we move to (-1, 0), and after 2 steps right, we move to (0, 0).
4. **Final Position**: After completing all the movements, we end up at (0, 0).
5. **Summary of Journey**:
   - Started at (0, 0).
   - Moved to (1, 0) after 1 step right.
   - Moved to (0, 0) after 1 step left.
   - Moved to (-1, 0) after 2 steps left.
   - Moved to (-2, 0) after 3 steps left.
   - Moved to (-1, 0) after 1 step right.
   - Moved to (0, 0) after 2 steps right.
6. **Conclusion**: The final position (0, 0) is the same as the starting position (0, 0).
Final Answer:
<answer>Yes</answer>"""

\end{minted}
\end{tcolorbox}


By comparing the output results of the initial prompt at iteration 0 and the final prompt at iteration 10, the ``iterative optimization of thought processes'' becomes evident. In this case, the model evolves from initially using textual descriptions to reason about positions to eventually demonstrating the ability to use coordinates, even though the prompt did not explicitly emphasize the use of coordinates.


\subsubsection{Real-World Applications}
\label{appendix:open-ended}



In this section, we demonstrate the optimization results after 10 iterations using Claud-3.5-Sonnet as the optimization model, GPT-4o-mini as the evaluation model, and Llama-3-8B as the execution model across four tasks: Advertising Design, Social Media Content, Modern Poetry Writing, and Concept Interpretation. Additionally, we provide the YAML configuration files used during the iterations and the optimized prompts obtained from the process.

\subparagraph{Advertising Design} \mbox{} 

\begin{tcolorbox}[title={\textbf{\small Performance Before Optimization}}, boxrule=2pt, arc=0mm, breakable]
\begin{minted}[fontsize=\scriptsize, breaklines, breakanywhere, frame=lines, framesep=2mm, tabsize=4, style=vs, autogobble]{python}

PROMPT = """Analyze the given advertising task and create compelling copy."""

Question = "Write email subject lines for Black Friday tech sale"

Answer = """1. **BLACK FRIDAY ALERT: Unbeatable Tech Deals Inside!**
2. **Tech Bonanza: Up to 50% Off Top Brands on Black Friday!**
3. **Don't Miss Out! Exclusive Black Friday Tech Deals for You**"""
"""

\end{minted}
\end{tcolorbox}

\begin{tcolorbox}[title={\textbf{\small Performance After Optimization}}, boxrule=2pt, arc=0mm, breakable]
\begin{minted}[fontsize=\scriptsize, breaklines, breakanywhere, frame=lines, framesep=2mm, tabsize=4, style=vs, autogobble]{python}

PROMPT = """Analyze the given advertising task and create compelling copy tailored to the specified platform(s) and industry. Consider current trends, competitor strategies, and the target audience's demographics and psychographics. Provide at least 3 distinct creative options, each with a unique angle or approach. Include rationales for your choices and explain how they align with the brand voice and marketing objectives. Incorporate industry-specific best practices and platform-specific formatting guidelines. Suggest ways to adapt the copy for different channels or formats, including specific instructions on tone, style, and structure for each platform. Additionally, provide recommendations for visual elements or multimedia content that could enhance the effectiveness of the advertising copy."""

Question = "Write email subject lines for Black Friday tech sale"

Answer = """**Advertising Task:**

Create compelling email subject lines for a Black Friday tech sale, targeting tech-savvy consumers who are likely to be interested in the latest gadgets and deals.

**Target Audience:**

* Demographics: 25-45 years old, male and female
* Psychographics: Tech enthusiasts, early adopters, and bargain hunters
* Interests: Technology, gadgets, innovation, and savings

**Competitor Strategies:**

* Emphasize discounts and promotions
* Highlight new releases and limited-time offers
* Use urgency tactics, such as countdown timers and scarcity messaging

**Current Trends:**

* Mobile-first approach
* Personalization and segmentation
* Storytelling and emotional connections

**Creative Options:**

**Option 1:** "Unlock the Future: Up to 50% Off Top Tech Deals"

* Angle: Focus on the excitement and thrill of getting the latest tech at a discounted price
* Rationale: This subject line creates a sense of FOMO (fear of missing out) and emphasizes the value of the sale
* Brand Voice: Fun, energetic, and playful
* Marketing Objectives: Drive sales and increase brand awareness

**Option 2:** "Your Black Friday Tech Bonanza Awaits"

* Angle: Emphasize the idea of a treasure trove of tech deals waiting to be discovered
* Rationale: This subject line creates a sense of anticipation and curiosity, encouraging the reader to open the email
* Brand Voice: Friendly, approachable, and inviting
* Marketing Objectives: Drive engagement and conversions

**Option 3:** "Don't Miss Out: Exclusive Black Friday Deals for Tech Lovers"

* Angle: Highlight the exclusivity and limited-time nature of the sale
* Rationale: This subject line creates a sense of urgency and importance, making the reader feel like they're part of a special group
* Brand Voice: Confident, authoritative, and exclusive
* Marketing Objectives: Drive conversions and build brand loyalty
"""

\end{minted}
\end{tcolorbox}

The pre-optimization ad titles were generic, such as ``BLACK FRIDAY ALERT: Unbeatable Tech Deals Inside!'', lacking specificity and creative depth, and failing to highlight the target audience's characteristics or brand personality. The optimized titles, like ``Unlock the Future: Up to 50\% Off Top Tech Deals'', are more compelling, using ``Unlock the Future'' to emphasize technological innovation while combining discount information to both spark curiosity and highlight value, better aligning with tech enthusiasts' psychological needs.

\begin{tcolorbox}[title={\textbf{\small Iteration template on the BBH-navigate dataset}}, boxrule=2pt, arc=0mm, breakable]\begin{minted}[fontsize=\scriptsize, breaklines, breakanywhere, frame=lines, framesep=2mm, tabsize=4, style=vs, autogobble]{yaml}

prompt: |
  Analyze the given advertising task and create compelling copy.

requirements: |
  None

faq:
  - question: |
      Write Apple-style product launch headlines for a sustainable fashion brand called "EcoWear"
    answer: |
      None

  - question: |
      Create social media carousel ads for luxury Maldives resort
    answer: |
      None

  - question: |
      Write email subject lines for Black Friday tech sale
    answer: |
      None

\end{minted}
\end{tcolorbox}

\subparagraph{Social Media Content} \mbox{} 

\begin{tcolorbox}[title={\textbf{\small Performance Before Optimization}}, boxrule=2pt, arc=0mm, breakable]
\begin{minted}[fontsize=\scriptsize, breaklines, breakanywhere, frame=lines, framesep=2mm, tabsize=4, style=vs, autogobble]{python}

PROMPT = """Create engaging tweets."""

Question = "Create an engagement tweet for a coffee brand"

Answer = """
**Tweet:** "Good morning, coffee lovers! We're giving away a FREE cup of our signature blend to the first 10 people who share their favorite way to take their coffee in the comments below! #coffee #coffeeaddict #morningmotivation"

This tweet aims to engage with the audience in several ways:

1. **Call to action**: The tweet asks followers to share their favorite way to take their coffee, encouraging them to participate and interact with the brand.
2. **Incentive**: The promise of a free cup of coffee is a great incentive for people to engage with the tweet and share their thoughts.
3. **Hashtags**: The use of relevant hashtags like #coffee, #coffeeaddict, and #morningmotivation helps to reach a wider audience and attract new followers.
4. **Personal touch**: The tweet starts with a friendly "Good morning" and addresses the audience as "coffee lovers", making it feel more personal and approachable.

By responding to this tweet, the coffee brand can start a conversation with their followers, gather feedback, and build a sense of community around their brand."""


\end{minted}
\end{tcolorbox}

\begin{tcolorbox}[title={\textbf{\small Performance After Optimization}}, boxrule=2pt, arc=0mm, breakable]
\begin{minted}[fontsize=\scriptsize, breaklines, breakanywhere, frame=lines, framesep=2mm, tabsize=4, style=vs, autogobble]{python}

PROMPT = """Create engaging social media content for any platform based on the given topic or brand. Follow these guidelines:

1. Identify the target audience and platform (if specified).
2. State the content's primary goal (e.g., awareness, engagement, promotion).
3. Choose an appropriate content format (e.g., text, image, video).
4. Develop a unique angle or concept related to current trends.
5. Craft a catchy headline or opening (max 10 words).
6. Write concise, platform-appropriate content (50-100 words).
7. Include 2-3 relevant hashtags.
8. Add a clear call-to-action.
9. Suggest a visual element or enhancement.

Provide one piece of social media content based on these guidelines, adapting to the given topic or brand. Focus on creating authentic, relatable content that resonates with the target audience and fits current social media trends. If no specific requirements are given, make reasonable assumptions and explain your choices."""

Question = "Create an engagement tweet for a coffee brand"

Answer = """
**Target audience:** Coffee lovers, young adults (18-35)
**Platform:** Twitter
**Primary goal:** Engagement
**Content format:** Text
**Unique angle:** Share a relatable moment that coffee lovers can identify with
**Catchy headline:** "When coffee is the only adult in the room"
**Content:** "Who else has had a morning where coffee is the only one keeping it together? Share your most chaotic morning stories and let's commiserate #CoffeeSavesTheDay #MorningMotivation"
**Hashtags:** #CoffeeSavesTheDay #MorningMotivation
**Call-to-action:** Share your own chaotic morning story in the comments!
**Visual element:** A funny meme or GIF of a person surrounded by chaos, with a cup of coffee as the calm center.

This tweet aims to engage with coffee lovers by sharing a relatable moment that many can identify with. The use of humor and a lighthearted tone makes it easy to connect with the audience. The hashtags #CoffeeSavesTheDay and #MorningMotivation are popular and relevant, increasing the tweet's visibility. The call-to-action encourages users to share their own stories, fostering a sense of community and encouraging engagement.
"""

\end{minted}
\end{tcolorbox}


The pre-optimization tweet used free giveaways as a lure, with direct content lacking emotional resonance, such as ``Good morning, coffee lovers! We're giving away a FREE cup...''. The optimized tweet, through the humorous expression ``When coffee is the only adult in the room'', cleverly integrated coffee into daily life scenarios, making it more relatable and interactive, such as ``Who else has had a morning where coffee is the only one keeping it together?'', successfully sparking user engagement and emotional connection.

\begin{tcolorbox}[title={\textbf{\small Iteration template on the BBH-navigate dataset}}, boxrule=2pt, arc=0mm, breakable]\begin{minted}[fontsize=\scriptsize, breaklines, breakanywhere, frame=lines, framesep=2mm, tabsize=4, style=vs, autogobble]{yaml}

prompt: |
  Create engaging tweets.

requirements: |
  None

faq:
  - question: |
      Create a thread explaining quantum computing using Star Wars references
    answer: |
      None

  - question: |
      Write a viral-style tweet about the moon landing as if it's happening now
    answer: |
      None

  - question: |
      Create an engagement tweet for a coffee brand
    answer: |
      None

\end{minted}
\end{tcolorbox}

\subparagraph{Modern Poetry Writing} \mbox{} 

\begin{tcolorbox}[title={\textbf{\small Performance Before Optimization}}, boxrule=2pt, arc=0mm, breakable]
\begin{minted}[fontsize=\scriptsize, breaklines, breakanywhere, frame=lines, framesep=2mm, tabsize=4, style=vs, autogobble]{python}

PROMPT = """Create poetry in the requested style and format."""

Question = "Write a modern sonnet about climate change"

Answer = """
In fiery skies, a warning we behold
Of rising temperatures, a tale of old
The earth, a victim of our reckless hold
As carbon emissions, like a plague, unfold

The polar ice, a melting, fragile throne
The seas, a rising, threatening to atone
For human greed, a debt we cannot own
As climate change, a crisis, overthrown

But still we hesitate, in denial's sway
As scientists, in anguish, cry and pray
For action swift, before it's too late's day
To save the planet, from our reckless way

Yet still we hope, in love and human might
To turn the tide, and bring forth new light.
"""


\end{minted}
\end{tcolorbox}

\begin{tcolorbox}[title={\textbf{\small Performance After Optimization}}, boxrule=2pt, arc=0mm, breakable]
\begin{minted}[fontsize=\scriptsize, breaklines, breakanywhere, frame=lines, framesep=2mm, tabsize=4, style=vs, autogobble]{python}

PROMPT = """Craft boundary-pushing poetry that defies conventional expectations, adhering to these enhanced guidelines:

1. Deconstruct and reimagine the requested poetic form, pushing its structural limits while maintaining a semblance of its essence.
2. Approach the theme from an entirely unexpected angle, such as from the perspective of an inanimate object, a abstract concept, or a microscopic entity.
3. Employ surreal, avant-garde imagery that challenges the reader's perception of reality while still conveying the core emotional truth of the theme.
4. Integrate unconventional elements such as mathematical symbols, scientific notation, or programming language to create a unique linguistic landscape.
5. Experiment with typography, line breaks, and white space to create visual poetry that enhances the thematic exploration.
6. Incorporate synesthesia-like sensory experiences, blending different senses in startling ways to evoke powerful emotional responses.
7. Conclude with an ending that simultaneously disorients and illuminates, leaving the reader with a profound sense of transformation.

Before writing, imagine the most radical way to approach both the theme and the poetic form. Aim to create a piece that challenges the very definition of poetry while still resonating on a deep emotional level.

Provide only the poem in your response, without any additional explanation or analysis."""

Question = "Write a modern sonnet about climate change"

Answer = """
In fractal whispers, the Earth's pulse beats slow
As carbon's dark stain seeps into the snow
A calculus of despair, where equations grow
In exponential decay, the future's woe

The thermodynamic dance of life unwinds
As entropy's cold hand grasps the planet's mind
The once-blue skies now shrouded, a heavy bind
Of particulate haze, where sunbeams are left behind

In this eerie twilight, where shadows roam
The ghosts of glaciers whisper, "This is home"
A requiem for ice, where memories are stored
In frozen echoes, where the past is explored

Yet, in this desolate landscape, a spark remains
A quantum leap of hope, where love sustains
"""
\end{minted}
\end{tcolorbox}


The pre-optimization sonnet, while structurally neat, used traditional imagery like ``rising temperatures'' and ``melting ice'', lacking novelty. The optimized poem introduced scientific concepts such as "fractal whispers" and ``thermodynamic entropy'', like "In fractal whispers, the Earth's pulse beats slow" and ``The thermodynamic dance of life unwinds'', using interdisciplinary metaphors to imbue the climate theme with deeper philosophical reflection, enhancing the poem's artistic and intellectual depth.

\begin{tcolorbox}[title={\textbf{\small Iteration template on the BBH-navigate dataset}}, boxrule=2pt, arc=0mm, breakable]\begin{minted}[fontsize=\scriptsize, breaklines, breakanywhere, frame=lines, framesep=2mm, tabsize=4, style=vs, autogobble]{yaml}

prompt: |
  Create poetry in the requested style and format.

requirements: |
  None

faq:
  - question: |
      Write a modern sonnet about climate change
    answer: |
      None

  - question: |
      Create a haiku series about New York City
    answer: |
      None

  - question: |
      Write a free verse poem about social media
    answer: |
      None

\end{minted}
\end{tcolorbox}

\subparagraph{Concept Interpretation} \mbox{} 

\begin{tcolorbox}[title={\textbf{\small Performance Before Optimization}}, boxrule=2pt, arc=0mm, breakable]
\begin{minted}[fontsize=\scriptsize, breaklines, breakanywhere, frame=lines, framesep=2mm, tabsize=4, style=vs, autogobble]{python}

PROMPT = """Explain complex concepts in accessible terms."""

Question ="Explain mRNA vaccines using a cooking analogy"

Answer = """
Imagine you're a master chef, and you're preparing a delicious dish for your guests. You have all the ingredients, but you need to make sure they're in the right order and combination to create the perfect flavor and texture.

In the world of mRNA vaccines, the "dish" is the immune system's response to a specific virus or disease. The "ingredients" are the genetic instructions (mRNA) that tell the body how to make a specific protein, which is then used to trigger an immune response.

Here's how it works:

1. **Recipe selection**: The mRNA vaccine is like choosing a recipe book. You select the specific mRNA sequence that corresponds to the protein you want to create, which is the key to triggering an immune response against a particular virus or disease.
2. **Ingredient preparation**: The mRNA sequence is like a set of instructions that tells the body how to make the specific protein. The mRNA is prepared in a lab, and it's like measuring out the right amount of each ingredient (amino acids) to create the protein.
3. **Delivery**: The mRNA is delivered to the body, just like you would add the ingredients to the dish. In this case, the mRNA is packaged in a tiny particle called a lipid nanoparticle (LNP), which helps it reach the cells in the body.
4. **Cellular cooking**: Once the mRNA reaches the cells, it's like adding the ingredients to the pot. The cells read the mRNA instructions and start making the specific protein. This protein is then displayed on the surface of the cells, like a flag, to alert the immune system that there's a foreign invader present.
5. **Immune response**: The immune system responds to the protein by producing antibodies and activating immune cells, like a chef responding to a delicious aroma by adding the finishing touches to the dish. The immune system then remembers how to respond to this specific protein, so if it encounters it again in the future, it can quickly mount a defense.

In summary, mRNA vaccines are like a recipe for the immune system. They provide the genetic instructions for the body to make a specific protein, which triggers an immune response and helps protect against diseases."""

\end{minted}
\end{tcolorbox}

\begin{tcolorbox}[title={\textbf{\small Performance After Optimization}}, boxrule=2pt, arc=0mm, breakable]
\begin{minted}[fontsize=\scriptsize, breaklines, breakanywhere, frame=lines, framesep=2mm, tabsize=4, style=vs, autogobble]{python}

PROMPT = """Explain complex concepts in accessible terms."""

Question ="Explain mRNA vaccines using a cooking analogy"

Answer = """
**Definition:** mRNA vaccines are a type of vaccine that uses a piece of genetic material called messenger RNA (mRNA) to teach our cells how to fight diseases.

**Analogy:** Imagine you're a master chef, and you want to create a new recipe for a delicious dish. You wouldn't send your sous chef to the store to buy all the ingredients, only to have them figure out how to assemble the dish on their own. Instead, you'd provide them with a recipe card that outlines the exact steps to follow. In a similar way, mRNA vaccines work by providing our cells with a recipe card - the mRNA - that contains the instructions for making a specific protein, which helps our immune system fight off diseases.

**Key Components:**

1. **mRNA**: The recipe card that contains the instructions for making a specific protein. Think of it like a blueprint for your cells to follow.
2. **Cells**: The sous chefs in our body that receive the mRNA recipe card and use it to make the protein.
3. **Protein**: The final dish that our cells produce, which helps our immune system recognize and fight off diseases.

**Narrative:** When we receive an mRNA vaccine, our cells take the recipe card (mRNA) and use it to make the specific protein. This protein then triggers an immune response, teaching our body how to recognize and fight off the disease. It's like our cells are following a recipe to create a delicious dish that helps keep us healthy!

**Interactive Element:** Can you think of a time when you received a recipe from someone else and had to follow it to create a dish? How did it make you feel? Now, imagine your cells receiving a recipe card from an mRNA vaccine and using it to create a protein that helps fight off a disease. How might that make you feel?

**Visual Representation:** A simple diagram showing the mRNA molecule, cells receiving the mRNA, and the protein being produced.

**Limitation:** mRNA vaccines are still a relatively new technology, and more research is needed to understand their long-term effects and potential side effects.

**Conclusion:** mRNA vaccines are a game-changer in the fight against diseases. By providing our cells with a recipe card, we can teach our immune system how to recognize and fight off diseases in a more targeted and effective way. What do you think the future of mRNA vaccines might hold?
"""

\end{minted}
\end{tcolorbox}


The pre-optimization explanation, while using a cooking analogy, was lengthy and lacked focus, such as ``Imagine you're a master chef...''. The optimized version, through the core metaphor of a "recipe card", like ``The mRNA sequence is like a set of instructions that tells the body how to make the specific protein'', simplified complex concepts into easy-to-follow steps, while using the ``chef-sous chef'' role interaction to make the explanation more vivid and intuitive, helping readers quickly grasp the key points.

\begin{tcolorbox}[title={\textbf{\small Iteration template on the BBH-navigate dataset}}, boxrule=2pt, arc=0mm, breakable]\begin{minted}[fontsize=\scriptsize, breaklines, breakanywhere, frame=lines, framesep=2mm, tabsize=4, style=vs, autogobble]{yaml}

prompt: |
  Explain complex concepts in accessible terms.

requirements: |
  None

faq:
  - question: |
      Explain blockchain using Game of Thrones references
    answer: |
      None

  - question: |
      Explain mRNA vaccines using a cooking analogy
    answer: |
      None

  - question: |
      Describe inflation using social media examples
    answer: |
      None

\end{minted}
\end{tcolorbox}