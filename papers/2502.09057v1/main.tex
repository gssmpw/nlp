\pdfoutput=1

\documentclass[a4paper,twoside]{article}

\usepackage{epsfig}
\usepackage{subcaption}
\usepackage{calc}
\usepackage{amssymb}
\usepackage{amstext}
\usepackage{amsmath}
\usepackage{amsthm}
\usepackage{multicol}
\usepackage{pslatex}
\usepackage{apalike}
\usepackage{algorithm2e}
\usepackage[bottom]{footmisc}
\usepackage{SCITEPRESS}     % Please add other packages that you may need BEFORE the SCITEPRESS.sty package.
\usepackage{graphicx}
\usepackage{pdfpages}
\usepackage{booktabs}
\usepackage{array}
\usepackage[accsupp]{axessibility}

\begin{document}

\title{Vision-Language In-Context Learning Driven Few-Shot Visual Inspection Model}

\author{\authorname{
Shiryu Ueno\sup{1}\orcidAuthor{0009-0006-0842-1362}, 
Yoshikazu Hayashi\sup{1},
Shunsuke Nakatsuka\sup{1}\orcidAuthor{0000-0000-0000-0000}},
Yusei Yamada\sup{1} ,
Hiroaki Aizawa\sup{2}\orcidAuthor{0000-0002-6241-3973}, 
Kunihito Kato\sup{1},
\affiliation{\sup{1}Faculty. of Engineering, Gifu University,  1-1 Yanagido, Gifu, Japan}
\affiliation{\sup{2}Graduate School of Advanced Science and Engineering, Hiroshima University, Higashihiroshima, Hiroshima 739-8527, Japan}
\email{\{ueno, hayashi, nakatsuka, yyamada\}@cv.info.gifu-u.ac.jp, hiroaki-aizawa@hiroshima-u.ac.jp}
}

\keywords{Visual Inspection, Vision-Language Model, In-Context Learning}

\abstract{
We propose general visual inspection model using Vision-Language Model~(VLM) with few-shot images of non-defective or defective products, along with explanatory texts that serve as inspection criteria. 
% 
Although existing VLM exhibit high performance across various tasks, they are not trained on specific tasks such as visual inspection. 
% 
Thus, we construct a dataset consisting of diverse images of non-defective and defective products collected from the web, along with unified formatted output text, and fine-tune VLM.
% 
For new products, our method employs In-Context Learning, which allows the model to perform inspections with an example of non-defective or defective image and the corresponding explanatory texts with visual prompts.
% 
This approach eliminates the need to collect a large number of training samples and re-train the model for each product. 
% 
The experimental results show that our method achieves high performance, with MCC of 0.804 and F1-score of 0.950 on MVTec AD in a one-shot manner. 
% 
Our code is available at~https://github.com/ia-gu/Vision-Language-In-Context-Learning-Driven-Few-Shot-Visual-Inspection-Model.
}

\onecolumn \maketitle \normalsize \setcounter{footnote}{0} \vfill

\section{Introduction}
\label{sec:intro}
% \definecolor{myblue}{RGB}{132, 180, 213}
% \definecolor{myred}{RGB}{213, 123, 101}
\definecolor{myblue}{RGB}{0, 0, 0}
\definecolor{myred}{RGB}{0, 0, 0}
\begin{figure*}[tp]
\centering
\begin{tabular}{ccc}
\begin{minipage}[t]{0.3\linewidth}\includegraphics[width=1\linewidth]{figures/fig_ablation_t_afhq_rebuttal/afhq_metric_L1.png}\end{minipage} &
\begin{minipage}[t]{0.3\linewidth}\includegraphics[width=1\linewidth]{figures/fig_ablation_t_afhq_rebuttal/afhq_metric_L2.png}\end{minipage} &
\begin{minipage}[t]{0.3\linewidth}\includegraphics[width=1\linewidth]{figures/fig_ablation_t_afhq_rebuttal/afhq_metric_SSIM.png}\end{minipage} \\
(a) $L_1$ & (b) $L_2$ & (c) SSIM \\
\begin{minipage}[t]{0.3\linewidth}\includegraphics[width=1\linewidth]{figures/fig_ablation_t_afhq_rebuttal/afhq_metric_PSNR.png}\end{minipage} &
\begin{minipage}[t]{0.3\linewidth}\includegraphics[width=1\linewidth]{figures/fig_ablation_t_afhq_rebuttal/afhq_metric_FID.png}\end{minipage} &
\begin{minipage}[t]{0.3\linewidth}\includegraphics[width=1\linewidth]{figures/fig_ablation_t_afhq_rebuttal/afhq_metric_LPIPS.png}\end{minipage} \\
(d) PSNR & (e) FID & (f) LPIPS
\end{tabular}
\caption{
    \textbf{Analysis on the preset timestep, $t$.}
    %
    We introduce 6 distinct metrics, \textit{i.e.}, $L_1$, $L_2$, SSIM, PSNR, FID, and LPIPS.
    %
    We plot the distance between $(x_t,y_t)$ and $(x_0, x_t)$ at different timesteps on AFHQ dataset, which are shown in red and blue curves, respectively.
}
\label{fig:ablation_t_afhq}
\vspace{-10pt}
\end{figure*}



In this study, we propose a method that can detect defective locations in new product images by using Vision-Language Model~(VLM)~\cite{SurveyMultimodalLargeLanguageModelsa}~\cite{MMBenchYourMultimodalModelAllaroundPlayer} and In-Context Learning~(ICL)~\cite{SurveyIncontextLearning}~\cite{VLICLBenchDevilDetailsBenchmarkingMultimodalInContextLearning}.

\figone

With the advancements in deep learning technology, the automation of visual inspection has become increasingly common in recent years. 
% 
However, current visual inspection models inspect specific products by collecting a large number of images of the target product and training the model. 
% 
Thus, these models are only applicable to the target products on which they have been trained, and re-training is necessary for new products.
% 
Although some methods can inspect multiple products with a single model, they still require hyperparameter tuning or additional training for each product. 
% 
In this study, we propose a general visual inspection model that leverages VLM and ICL allowing the inspection of new products without any hyperparameter tuning or model training.

Many of the current VLMs~\cite{ImprovedBaselinesVisualInstructionTuning}~\cite{ShikraUnleashingMultimodalLLMReferentialDialogueMagic} leverage Large Language Model~(LLM) to align visual and language features, demonstrating excellent performance in a wide range of tasks.
% 
These tasks range from basic image recognition tasks, such as classification, to advanced vision-language tasks, such as Visual Question Answering~(VQA). 
% 
However, these VLMs are not trained on specific tasks such as visual inspection.

In this study, we propose a general visual inspection model that can detect defective locations in new products without any hyperparameter tuning or model re-training, using VLM and ICL.
% 
The framework of our proposed method is shown in Fig.~\ref{fig1}. 
First, we fine-tune the VLM for general visual inspection with a dataset constructed from a diverse set of non-defective and defective product images collected from the web.
% 
In this study, we use ViP-LLaVA~\cite{ViPLLaVAMakingLargeMultimodalModelsUnderstandArbitraryVisualPrompts}, which has been trained on visual prompt recognition, as the foundation of our VLM, and fine-tune it with our dataset.
% 
In addition, in typical visual inspection processes by humans, inspectors use inspection standards for the target products.
% 
To emulate this inspection process by human, we use ICL during the evaluation to provide an example of non-defective or defective product image along with explanatory texts that serve as inspection criteria.
% 
ICL is a method the model learns from few-shot input-output examples as prompts, without parameter update.
% 
Using ICL during the inference of new products, we provide VLM with inspection criteria, enabling specific inspection of target products.
% 
Since ICL performance varies significantly based on the provided examples, we propose an algorithm that can select high-quality example based on the distance in Euclidean space. 
% 
Consequently, our proposed method does not need to collect a large number of images or to re-train the model for each target product.

In summary, our main contributions are:

\begin{itemize}
\item We propose a general visual inspection model capable of inspections and detecting defective locations for new products using VLM and ICL with only an example.
% 
In our proposed method, fine-tune VLM on visual inspection and utilize ICL enabling the inspection of specific products.
% 
\item We construct a new dataset consisting of diverse non-defective and defective products collected from the web, along with unified formatted output, for fine-tuning.
% 
Also, our dataset includes coordinates of defective locations for defective products, ensuring the explainability of the model.
% 
\item To empirically verify the proposed methodology, we evaluate on MVTec AD~\cite{MVTecADComprehensiveRealWorldDatasetUnsupervisedAnomalyDetection} and VisA~\cite{SPottheDifferenceSelfSupervisedPretrainingAnomalyDetectionSegmentation}. 
% 
Our method achieves MCC~\cite{advantagesMatthewscorrelationcoefficientMCCF1scoreaccuracybinaryclassificationevaluation} of 0.804 and F1-score~\cite{AccuracyFScoreROCFamilyDiscriminantMeasuresPerformanceEvaluation} of 0.950 on the MVTec AD dataset in a one-shot manner.
\end{itemize}


\section{Related Work}
\label{sec:relatedwork}

\subsection{Visual Inspection}
\label{sec:vi}
Many visual inspection methods based on deep learning are trained only on non-defective images~\cite{PatchSVDDPatchlevelSVDDAnomalyDetectionSegmentation}~\cite{Padimpatchdistributionmodelingframeworkanomalydetectionlocalization}.
% 
Thus, such methods require the collection of training samples and the re-training of the model for each target product. 
% 
Consequently, it is challenging to apply the same model to different products without re-training.

Recently, visual inspection methods combining vision and language have been proposed.
% 
AnomalyGPT~\cite{AnomalyGPTDetectingIndustrialAnomaliesUsingLargeVisionLanguageModelsa} can detect defective locations by learning an image decoder from non-defective and pseudo-defective images. 
However, AnomalyGPT utilizes PaDiM or PatchCore~\cite{totalrecallindustrialanomalydetection} for anomaly maps, and these methods need re-training for each products.
% 
WinCLIP~\cite{WinCLIPZeroFewShotAnomalyClassificationSegmentation} calculates the similarity between images and texts of non-defective and defective images using CLIP~\cite{LearningTransferableVisualModelsNaturalLanguageSupervisiona} and can detect defective locations by using relative anomaly scores.
% 
However, WinCLIP only assigns anomaly scores to test samples during inference. 
To inspect correctly, it is necessary to experimentally determine the optimal threshold on test samples.
Thus, these existing approaches cannot be considered general visual inspection models.



\subsection{Vision-Language Model}
\label{sec:VLM}
VLMs leverage LLM to align visual and language features, demonstrating excellent performance across a wide range of tasks, from basic image recognition tasks such as classification, to advanced vision-language tasks, such as VQA. 
% 
For example, LLaVA~\cite{VisualInstructionTuning} inputs the vision embedding vectors and language embedding vectors into the LLM decoder to learn the alignment between vision and language.
% 
LLaVA has spawned many derivative methods, among which ViP-LLaVA focuses on visual prompt recognition by utilizing a dataset where arrows or visual cues are directly embedded in the input images, thereby strengthening the alignment between low-level image details and language. 
% 
However, these VLMs have not been trained on visual inspection tasks and thus lack the general knowledge for visual inspection~\cite{MMBenchYourMultimodalModelAllaroundPlayer}.


\model


\subsection{In-Context Learning}
\label{sec:icl}
ICL is a method that the model learns from few-shot input-output examples as prompts, without updating model parameters.
% 
For instance, given the input ``Example input: (4, 2), Example output: 6, Question: (5, 6),'' the model infers from the provided example that the task is addition and can answer ``11.''
% 
In multi-modal ICL, the model makes inferences based on images, prompts, and their examples.
% 
Many VLMs are trained on diverse image-text pairs, enabling them to acquire ICL capabilities~\cite{UnderstandingImprovingInContextLearningVisionlanguageModels}.

Some VLMs are explicitly built to enhance ICL capabilities.
% 
Otter~\cite{OtterMultiModalModelInContextInstructionTuninga} enhances ICL capabilities by fine-tuning Open Flamingo~\cite{OpenFlamingoOpenSourceFrameworkTrainingLargeAutoregressiveVisionLanguageModelsa} on MIMIC-IT~\cite{MIMICITMultiModalInContextInstructionTuninga}, which is in an ICL and Instruction Tuning format. 
% 
At the same time, not to forget the knowledge of Open Flamingo, Otter only update parameters of Perceiver Resampler and Cross Attention Layer in language model.
% 
Similarly, LCL~\cite{LinkContextLearningMultimodalLLMsa} proposes a new evaluation dataset, ISEKAI, which includes new concepts in the examples, making it challenging without seeing the examples.
% 
To address ISEKAI, LCL enhances its ICL capability by fully fine-tuning Shikra~\cite{ShikraUnleashingMultimodalLLMReferentialDialogueMagic} on a custom dataset based on ImageNet~\cite{Imagenetlargescalehierarchicalimagedatabase}.
% 
However, in practice, these VLM explicitly designed to enhance ICL capabilities do not necessarily outperform regular VLM~\cite{UnderstandingImprovingInContextLearningVisionlanguageModels}~\cite{VLICLBenchDevilDetailsBenchmarkingMultimodalInContextLearning}.



\section{Proposed Method}

\subsection{Overview}
\label{sec:overview}
In this study, we propose a general visual inspection model that combines VLM and ICL, enabling the specific inspection of new products without parameter optimization.
% 
In addition, by constructing unified output format dataset for fine-tuning, we enable quantitative evaluation of visual inspections using VLM. 
% 
An overview of the proposed method is shown in Fig.~\ref{fig1}.


\subsection{Model}
\label{sec:model}
In this study, we use ViP-LLaVA~\cite{ViPLLaVAMakingLargeMultimodalModelsUnderstandArbitraryVisualPrompts} as the foundational VLM.
% 
ViP-LLaVA is a model that improves recognition capabilities for visual prompts by fine-tuning LLaVA 1.5~\cite{ImprovedBaselinesVisualInstructionTuning} on a dataset where red circles or arrows are overlaid on the original images. 
% 
In addition to this, ViP-LLaVA utilizes the multi-level visual features to address the tendency of CLIP’s deeper features to overlook low-level details.
% 
These methodologies improves the recognition capability for low-level details, which is especially needed for visual inspection.
% 
ViP-LLaVA has not been trained on visual inspection tasks.

The model architecture of ViP-LLaVA is shown in Fig.~\ref{model}. 
% 
ViP-LLaVA consists of a vision encoder to extract visual features, LayerNorm~\cite{LayerNormalization} and an MLP to tokenize visual features, a tokenizer to tokenize the language, and an LLM to generate text from these tokens.
% 
The vision encoder is CLIP-ViT-L/14\cite{LearningTransferableVisualModelsNaturalLanguageSupervisiona}, and the LLM is LLaMA2~\cite{Llama2OpenFoundationFineTunedChatModels} 
% 
During fine-tuning, we update the parameters of the LayerNorm, MLP, and LLM in accordance with the ViP-LLaVA procedure.

\subsection{Dataset for Fine-tune}
\label{sec:dataset}
To enhance the general knowledge of existing VLM for visual inspection, we collect images of non-defective and defective products from the web. The image collection process consists of five main steps:

\begin{enumerate}
\item Generate product names and inspection-related keywords~(e.g., ``disk'', ``broken disk'', ``discolored disk'') by GPT-4~\cite{GPT4TechnicalReport}.

\item Expand the keywords into eight languages: English, Chinese, Spanish, French, Portuguese, German, Italian, and Japanese.

\item Perform image searches using the expanded keywords and collect images with selenium.

\item Remove duplicate or unclear images.

\item Annotate the defective location coordinates for the defective images in the remaining set.
\end{enumerate}

\difficulty

Through this procedure, we collect images of various products. Each product category includes images of non-defective and defective products with up to five types of defects. 
% 
Finally, we obtained a final set of 941 images of 84 categories.

After collecting the images, we construct a dataset for fine-tuning.
The format of the dataset is based on VQA~(i.e., a pair of question and answer for each images).
Question is ``This is an image of \{product\}. Does this \{product\} in the image have any defects? If yes, please provide the anomaly mode and the bounding box coordinate of the region where the defect is located. If no, please say None.'', answer is coordinates of the defective location for defective image, ``None'' for non-defective image.


\subsection{In-Context Learning Driven Visual Inspection}
\label{sec:iclevaluation}
It is challenging to inspect new products from a single image. 
% 
An example is shown in Fig.~\ref{difficulty}. 
% 
As shown in Fig.~\ref{difficulty}, some products need their specific inspection criteria for accurate visual inspection.
% 
Thus, in this study, we utilize ICL to provide an example of non-defective or defective image along with explanatory texts that serve as inspection criteria. 
% 
Based on the example, model precisely inspects new products.

In addition, in multi-modal ICL, example images significantly influence the output of VLM~\cite{WhatMakesMultimodalInContextLearningWork}.
% 
RICES~\cite{EmpiricalStudyGPT3FewShotKnowledgeBasedVQA} is an existing algorithm for selecting examples in ICL, it uses cosine similarity of features.
% 
However, cosine similarity can yield high values when the scales of features differ or when the feature dimensions are large, failing to accurately evaluate similarity~\cite{CosineSimilarityEmbeddingsReallySimilarity}.
% 
Thus, in this study, we propose a new selection algorithm. Our proposed method is shown in Eq.~(\ref{eq:ours}).

\begin{equation}
    \operatorname{argmin} \left( \min \left\| f(x_i) - f(x_q) \right\|_2 \right)
  \label{eq:ours}
\end{equation}

Where $x$ denotes the image, $f$ denotes the vision encoder~(pre-trained ResNet50~\cite{DeepResidualLearningImageRecognition}), and $q$ denotes the index of the test image for inference, $i$ denotes the index of the image except for $q$.
% 
Eq.~(\ref{eq:ours}) is an algorithm that selects neighboring image of the test image as example based on Euclidean distance.
% 
By this, Eq.~(\ref{eq:ours}) takes into account the scale and dimensions of the features, and expected to select better example compared to RICES.


\section{Experiment}
\subsection{Settings}
\label{sec:settings}

The dataset used for fine-tuning was collected and created as described in Sec.~\ref{sec:dataset}. 
% 
We perform one-stage fine-tuning for 300 epochs using ZeRO2 with XTuner~\cite{XTunerToolkitEfficientlyFinetuningLLM} on 8 NVIDIA 6000Ada-48GB GPUs. The batch size is set to 4, thus the global batch size is set to 32.
We utilized the AdamW~\cite{DecoupledWeightDecayRegularizationa} optimizer and 1e-4 learning rate, with the warm-up ratio set to 0.03. We also apply cosine decay to the learning rate.

To evaluate the performance, we used MVTec AD and VisA. These datasets were not used at all during training.
% 
During the evaluation, one example image is given by using ICL~(i.e., one-shot manner). The method for selecting the example image follows the procedure described in Eq.~(\ref{eq:ours}). 
% 
Also, to evaluate the effectiveness of the proposed example image selection algorithm, we compare its performance without example image and with that of RICES.
Finally, framework of the evaluation is shown in Fig.~\ref{prompttest}.

\prompttest
\begin{table*}[!tb]
\caption{Result of MVTec-AD. 'N/A' means that zero division occurred. Bold means the highest performance.}
\label{tab:result_mvtec}
\centering
\begin{tabular}{lcccccccc}
\toprule
\multicolumn{1}{c}{\textbf{Settings}} & \multicolumn{2}{c}{\textbf{Vanilla}} & \multicolumn{2}{c}{\textbf{w/o ICL}} & \multicolumn{2}{c}{\textbf{ICL (RICES)}} & \multicolumn{2}{c}{\textbf{ICL (Ours)}} \\
\cmidrule(r){1-9}
\multicolumn{1}{c}{Product Name} & F1-score & MCC & F1-score & MCC & F1-score & MCC & F1-score & MCC \\
\midrule
Bottle & N/A & N/A & 0.863 & N/A & 0.892 & 0.510 & 0.917 & \textbf{0.610} \\
Cable & N/A & N/A & 0.400 & 0.338 & 0.795 & 0.564 & 0.899 & \textbf{0.754} \\
Capsule & N/A & N/A & 0.750 & 0.426 & 0.912 & 0.384 & 0.946 & \textbf{0.658} \\
Carpet & 0.044 & 0.074 & 1.000 & \textbf{1.000} & 0.983 & 0.929 & 1.000 & \textbf{1.000}  \\
Grid & N/A & N/A & 0.973 & 0.910 & 0.884 & 0.476 & 0.982 & \textbf{0.935} \\
Hazelnut & 0.228 & 0.226 & 0.780 & N/A & 0.795 & 0.257 & 0.800 & \textbf{0.289} \\
Leather & 0.043 & 0.076 & 1.000 & \textbf{1.000} & 1.000 & \textbf{1.000} & 1.000 & \textbf{1.000} \\
Metal Nut & N/A & N/A & 0.832 & 0.540 & 0.912 & 0.468 & 0.989 & \textbf{0.947} \\
Pill & N/A & N/A & 0.838 & 0.402 & 0.922 & 0.368 & 0.968 & \textbf{0.814} \\
Screw & 0.209 & 0.140 & 0.851 & 0.244 & 0.903 & 0.506 & 0.925 & \textbf{0.673} \\
Tile & N/A & N/A & 0.957 & 0.870 & 0.977 & \textbf{0.916} & 0.977 & \textbf{0.916} \\
Toothbrush & N/A & N/A & 0.906 & 0.633 & 0.866 & 0.418 & 0.921 & \textbf{0.697} \\
Transistor & N/A & N/A & 0.762 & 0.592 & 0.871 & 0.780 & 0.894 & \textbf{0.821} \\
Wood & N/A & N/A & 0.976 & 0.752 & 0.992 & \textbf{0.965} & 0.992 & \textbf{0.965} \\
Zipper & N/A & N/A & 0.741 & 0.541 & 0.975 & 0.879 & 0.987 & \textbf{0.941} \\
\cmidrule(r){1-9}
All category & 0.042 & 0.068 & 0.860 & 0.519 & 0.917 & 0.665 & 0.950 & \textbf{0.804} \\
\bottomrule
\end{tabular}
\end{table*}


We use F1-score~\cite{AccuracyFScoreROCFamilyDiscriminantMeasuresPerformanceEvaluation} and Matthews Correlation Coefficient~(MCC)~\cite{advantagesMatthewscorrelationcoefficientMCCF1scoreaccuracybinaryclassificationevaluation} for the evaluation. 
% 
F1-score, as shown in Eq.~(\ref{eq:f1}) , is a common metric for binary classification.

\begin{equation}
    \text{F1-score} = \frac{2 \times \text{TP}}{2 \times \text{TP} + \text{FP} + \text{FN}}
  \label{eq:f1}
\end{equation}

As shown in Eq.~(\ref{eq:f1}), F1-score does not use prediction of true negative. 
Thus, when there is a large number of positive samples during inference, the performance can be significantly inflated by predicting all samples as positive.
% 
For instance, in MVTec AD, with 1,258 positive samples and 467 negative samples in the test data, F1-score shows a high value of 0.844. This shows F1-score is suspicious when there is a bias in the test data.
% 
Thus, we use not only F1-score but also MCC as evaluation metrics. 
MCC is reported to be adequate for binary classification, particularly for better consistency and less variance~\cite{MetricsMultiClassClassificationOverview}~\cite{GoodClassificationMeasuresHowFindThem}. 
% 
MCC is shown in Eq.~(\ref{eq:mcc}).

\begin{equation}
    \text{MCC} = \frac{\text{TP} \times \text{TN} - \text{FP} \times \text{FN}}{\sqrt{(\text{TP} + \text{FP})(\text{TP} + \text{FN})(\text{TN} + \text{FP})(\text{TN} + \text{FN})}}
  \label{eq:mcc}
\end{equation}

MCC ranges from -1 to 1, where 1 indicates perfect prediction of all samples, -1 indicates incorrect prediction of all samples, and 0 indicates random prediction. 
% 
In the previously mentioned example, MCC cannot be calculated because the denominator becomes zero.
% 
Thus, in this study, we use both F1-score and MCC for evaluation.
The evaluation methods include assessing the performance for each product within each dataset, as well as the overall performance across the entire dataset.


\subsection{Evaluation of Results}
\subsubsection{Result of MVTec AD}
The results for MVTec AD are shown in Tab.~\ref{tab:result_mvtec}. 
% 
The settings are as follows: ``Vanilla'' for ViP-LLaVA before fine-tune, ``w/o ICL'' for ViP-LLaVA after fine-tune without using an example during inference, ``ICL (RICES)'' for using a selected example image with the RICES algorithm during inference, and ``ICL (Ours)'' for using a selected example image with Eq.~(\ref{eq:ours}) during inference.
% 
In each settings, results of F1-score and MCC are in a row.
% 
From the table, we confirm that providing an example significantly improves performance.
% 
This demonstrates the effectiveness of our framework. 
% 
Additionally, compared to RICES, our selection algorithm achieves improvement in performance with an increase in MCC, demonstrating the effectiveness of our algorithm.


Next, for qualitative evaluation, the visualization of the model prediction is shown in Fig.~\ref{mvtecbb2}.
% 
As shown in the figure, our approach can roughly detect defective locations, which means the model recognizes the defects in the image.
% 
However, the model cannot detect multiple defects or logical defects, such as those in ``Cable''. This is due to the lack of variety in the training dataset.
% 
Thus, further image collection and an enlarged training dataset are required for performance improvement.
% 

Also, for some products like ``Hazelnut'', while our approach improved the performance, it is still insufficient for real-world conditions.
For ``Hazelnut'', the model detected thin parts as defective, indicating that the model does not fully leverage ICL.
Thus, providing detailed inspection criteria is necessary. 
% 
It has been reported that increasing the number of examples improves ICL performance~\cite{ManyShotInContextLearning}~\cite{InContextLearningLongContextModelsInDepthExploration}.
% 
Alternatively, further performance improvement is expected by proposing an optimal selection algorithm that selects multiple example images~(based on the query strategies, including those from Deep Active Learning~\cite{SurveyDeepActiveLearning}~\cite{BenchmarkingQueryStrategiesFutureDeepActiveLearning}).
% 
\mvtecbb

Additionally, for all products, although coordinates are output, their positions deviate from the actual defective locations.
% 
Indeed, pixel-level AUROC was 0.730, which is very low compared to the existing methods.
% 
This is because the CrossEntropyLoss used for training uniformly calculates the loss for differences in token values.
% 
For example, when the ground truth of the starting x-coordinate is 100, the loss is the same when the model outputs 101 and 900 (assuming the prediction probabilities are equal).
% 
Thus, CrossEntropyLoss is not optimal for tasks requiring specific numerical outputs like coordinates. 
% 
However, existing VLMs are trained with CrossEntropyLoss, meaning their outputs are text-based cannot be safely converted to floats~(with gradient flow intact), thus performance improvement is expected by constructing a multi-head VLM for defect detection and modifying the loss function to alternatives like Mean Squared Error or GIoU Loss.
% 
\comparison

While ``Bottle'' has the same product in the training dataset, their performance is lower compared to ``Wood'', which also has the same product in the training dataset.
% 
This is likely due to the significant differences in appearance between the images in the training data and those in MVTec AD, as shown in Fig.~\ref{comparison}.
% 
However, despite the differences in appearance, ``Tile'' shows high performance, confirming the generalization capability for some products.
% 
Also, to prevent the forgetting of knowledge acquired during pre-training when fine-tuning, it is necessary to use Parameter Efficient Fine-Tuning methods, such as Low-Rank Adaptation~\cite{LoRALowRankAdaptationLargeLanguageModels}, which forget less than fine-tuning~\cite{LoRALearnsLessForgetsLess}.

\begin{table*}[!tb]
\caption{Result of VisA.}
\label{tab:result_visa}
\centering
\begin{tabular}{lcccccccc}
\toprule
\multicolumn{1}{c}{\textbf{Settings}} & \multicolumn{2}{c}{\textbf{Vanilla}} & \multicolumn{2}{c}{\textbf{w/o ICL}} & \multicolumn{2}{c}{\textbf{ICL (RICES)}} & \multicolumn{2}{c}{\textbf{ICL (Ours)}} \\
\cmidrule(r){1-9}
\multicolumn{1}{c}{Product Name} & F1-score & MCC & F1-score & MCC & F1-score & MCC & F1-score & MCC \\
\midrule
Candle & N/A & N/A & 0.635 & \textbf{0.539} & 0.692 & 0.241 & 0.694 & 0.253  \\
Capsules & N/A & N/A & 0.599 & 0.415 & 0.841 & \textbf{0.513} & 0.809 & 0.389 \\
Cashew & N/A & N/A & 0.814 & 0.623 & 0.890 & 0.670 & 0.889 & \textbf{0.674} \\
Chewinggum & N/A & N/A & 0.921 & 0.758 & 0.921 & 0.758 & 0.935 & \textbf{0.804} \\
Fryum & N/A & N/A & 0.867 & 0.699 & 0.917 & \textbf{0.741} & 0.888 & 0.648 \\
Macaroni1 & N/A & N/A & 0.760 & \textbf{0.502} & 0.685 & 0.204 & 0.683 & 0.190 \\
Macaroni2 & N/A & N/A & 0.669 & \textbf{0.071} & 0.667 & N/A & 0.667 & N/A \\
PCB1 & N/A & N/A & 0.131 & 0.190 & 0.891 & \textbf{0.792} & 0.875 & 0.762 \\
PCB2 & N/A & N/A & 0.347 & 0.343 & 0.772 & \textbf{0.493} & 0.763 & 0.471 \\
PCB3 & N/A & N/A & 0.243 & 0.248 & 0.747 & 0.503 & 0.751 & \textbf{0.513} \\
PCB4 & N/A & N/A & 0.622 & 0.516 & 0.801 & 0.594 & 0.817 & \textbf{0.610} \\
Pipe Fryum & N/A & N/A & 0.870 & 0.726 & 0.920 & 0.744 & 0.929 & \textbf{0.774} \\
\cmidrule(r){1-9}
All category & N/A & N/A & 0.671 & 0.429 & 0.800 & \textbf{0.492} & 0.795 & 0.479 \\
\bottomrule
\end{tabular}
\end{table*}
\subsubsection{Result of VisA}

The results for VisA are shown in Tab.~\ref{tab:result_visa}. 
The table follows the same format as Tab.~\ref{tab:result_mvtec}.
% 
From the table, it can be confirmed that the performance improves by using ICL in VisA as well, demonstrating the effectiveness of the proposed framework.
% 
However, compared to RICES, our selection algorithm does not show significant improvement. 
% 
This is because both RICES and our selection algorithm are based on similarity, which depends on the data distribution. 
Most of the products in VisA are too widely distributed~(e.g., ``Macaroni'', ``PCB''). 
% 
Thus, proposing a more distribution-robust selection algorithm could potentially improve performance.
% 
Also, it can be seen that the performance does not improve regardless of the presence of ICL when there are two or more products in the image, especially if those products are not aligned.
% 
In fact, ``Macaroni1'', which is neatly aligned, shows higher qualitative and quantitative performance compared to ``Macaroni2'', which is randomly arranged.
% 
This is likely due to the lack of training dataset that considers differences in product positions and orientations. 
% 
Thus, performance improvement is expected by collecting fine-tuning data and performing data augmentation, such as rotation and flipping.
% 
Simultaneously, it should be noted that for some products, positional shifts or orientation differences may be defined as defects.

For qualitative evaluation, the visualization of the model prediction is shown in Fig.~\ref{visa_bb}.
% 
As shown in Fig.~\ref{visa_bb}, for products that have multiple objects like ``Candle'' or ``Capsules'', the model prediction gets worse.
% 
As mentioned, our dataset is still insufficient for generalization because there are limited products and they are mostly single object. 
% 
In addition, images with multiple objects are highly distributed compared to the images with single object, which influences the performance of ICL because the selection algorithms depend on the distribution.
\visabb

\section{Conclusion}
In this study, we propose a general visual inspection model based on a few images of non-defective or defective products along with explanatory texts serving as inspection criteria. 
% 
For future work, further performance improvement is expected by collecting more images for fine-tuning. 
% 
In this study, we enabled visual inspection using VLM by training on a dataset consisting of only 941 images, which is very small compared to the pre-training dataset of VLM. 
% 
Another consideration is to construct the multi-head VLM and change of the loss function.
% 
Furthermore, introducing the example image selection algorithm is another way for improvement. 
% 
Specifically, existing algorithms are for selecting one example image for the inspection, so proposing an optimal selection algorithm for many example images improves model performance.
% 
In addition, the proposed method is based on VLM, so by adding the rationale statements for the decision in the response, model explainability is expected to improve, and performance could be enhanced through multitasking.
% 
Finally, in our study, we evaluate only on MVTec AD and VisA. However, more comprehensive benchmark such as MMAD~\cite{jiang2024mmad} is required.


\bibliographystyle{apalike}
{\small
\bibliography{main}}

\newpage
\subsection{Lloyd-Max Algorithm}
\label{subsec:Lloyd-Max}
For a given quantization bitwidth $B$ and an operand $\bm{X}$, the Lloyd-Max algorithm finds $2^B$ quantization levels $\{\hat{x}_i\}_{i=1}^{2^B}$ such that quantizing $\bm{X}$ by rounding each scalar in $\bm{X}$ to the nearest quantization level minimizes the quantization MSE. 

The algorithm starts with an initial guess of quantization levels and then iteratively computes quantization thresholds $\{\tau_i\}_{i=1}^{2^B-1}$ and updates quantization levels $\{\hat{x}_i\}_{i=1}^{2^B}$. Specifically, at iteration $n$, thresholds are set to the midpoints of the previous iteration's levels:
\begin{align*}
    \tau_i^{(n)}=\frac{\hat{x}_i^{(n-1)}+\hat{x}_{i+1}^{(n-1)}}2 \text{ for } i=1\ldots 2^B-1
\end{align*}
Subsequently, the quantization levels are re-computed as conditional means of the data regions defined by the new thresholds:
\begin{align*}
    \hat{x}_i^{(n)}=\mathbb{E}\left[ \bm{X} \big| \bm{X}\in [\tau_{i-1}^{(n)},\tau_i^{(n)}] \right] \text{ for } i=1\ldots 2^B
\end{align*}
where to satisfy boundary conditions we have $\tau_0=-\infty$ and $\tau_{2^B}=\infty$. The algorithm iterates the above steps until convergence.

Figure \ref{fig:lm_quant} compares the quantization levels of a $7$-bit floating point (E3M3) quantizer (left) to a $7$-bit Lloyd-Max quantizer (right) when quantizing a layer of weights from the GPT3-126M model at a per-tensor granularity. As shown, the Lloyd-Max quantizer achieves substantially lower quantization MSE. Further, Table \ref{tab:FP7_vs_LM7} shows the superior perplexity achieved by Lloyd-Max quantizers for bitwidths of $7$, $6$ and $5$. The difference between the quantizers is clear at 5 bits, where per-tensor FP quantization incurs a drastic and unacceptable increase in perplexity, while Lloyd-Max quantization incurs a much smaller increase. Nevertheless, we note that even the optimal Lloyd-Max quantizer incurs a notable ($\sim 1.5$) increase in perplexity due to the coarse granularity of quantization. 

\begin{figure}[h]
  \centering
  \includegraphics[width=0.7\linewidth]{sections/figures/LM7_FP7.pdf}
  \caption{\small Quantization levels and the corresponding quantization MSE of Floating Point (left) vs Lloyd-Max (right) Quantizers for a layer of weights in the GPT3-126M model.}
  \label{fig:lm_quant}
\end{figure}

\begin{table}[h]\scriptsize
\begin{center}
\caption{\label{tab:FP7_vs_LM7} \small Comparing perplexity (lower is better) achieved by floating point quantizers and Lloyd-Max quantizers on a GPT3-126M model for the Wikitext-103 dataset.}
\begin{tabular}{c|cc|c}
\hline
 \multirow{2}{*}{\textbf{Bitwidth}} & \multicolumn{2}{|c|}{\textbf{Floating-Point Quantizer}} & \textbf{Lloyd-Max Quantizer} \\
 & Best Format & Wikitext-103 Perplexity & Wikitext-103 Perplexity \\
\hline
7 & E3M3 & 18.32 & 18.27 \\
6 & E3M2 & 19.07 & 18.51 \\
5 & E4M0 & 43.89 & 19.71 \\
\hline
\end{tabular}
\end{center}
\end{table}

\subsection{Proof of Local Optimality of LO-BCQ}
\label{subsec:lobcq_opt_proof}
For a given block $\bm{b}_j$, the quantization MSE during LO-BCQ can be empirically evaluated as $\frac{1}{L_b}\lVert \bm{b}_j- \bm{\hat{b}}_j\rVert^2_2$ where $\bm{\hat{b}}_j$ is computed from equation (\ref{eq:clustered_quantization_definition}) as $C_{f(\bm{b}_j)}(\bm{b}_j)$. Further, for a given block cluster $\mathcal{B}_i$, we compute the quantization MSE as $\frac{1}{|\mathcal{B}_{i}|}\sum_{\bm{b} \in \mathcal{B}_{i}} \frac{1}{L_b}\lVert \bm{b}- C_i^{(n)}(\bm{b})\rVert^2_2$. Therefore, at the end of iteration $n$, we evaluate the overall quantization MSE $J^{(n)}$ for a given operand $\bm{X}$ composed of $N_c$ block clusters as:
\begin{align*}
    \label{eq:mse_iter_n}
    J^{(n)} = \frac{1}{N_c} \sum_{i=1}^{N_c} \frac{1}{|\mathcal{B}_{i}^{(n)}|}\sum_{\bm{v} \in \mathcal{B}_{i}^{(n)}} \frac{1}{L_b}\lVert \bm{b}- B_i^{(n)}(\bm{b})\rVert^2_2
\end{align*}

At the end of iteration $n$, the codebooks are updated from $\mathcal{C}^{(n-1)}$ to $\mathcal{C}^{(n)}$. However, the mapping of a given vector $\bm{b}_j$ to quantizers $\mathcal{C}^{(n)}$ remains as  $f^{(n)}(\bm{b}_j)$. At the next iteration, during the vector clustering step, $f^{(n+1)}(\bm{b}_j)$ finds new mapping of $\bm{b}_j$ to updated codebooks $\mathcal{C}^{(n)}$ such that the quantization MSE over the candidate codebooks is minimized. Therefore, we obtain the following result for $\bm{b}_j$:
\begin{align*}
\frac{1}{L_b}\lVert \bm{b}_j - C_{f^{(n+1)}(\bm{b}_j)}^{(n)}(\bm{b}_j)\rVert^2_2 \le \frac{1}{L_b}\lVert \bm{b}_j - C_{f^{(n)}(\bm{b}_j)}^{(n)}(\bm{b}_j)\rVert^2_2
\end{align*}

That is, quantizing $\bm{b}_j$ at the end of the block clustering step of iteration $n+1$ results in lower quantization MSE compared to quantizing at the end of iteration $n$. Since this is true for all $\bm{b} \in \bm{X}$, we assert the following:
\begin{equation}
\begin{split}
\label{eq:mse_ineq_1}
    \tilde{J}^{(n+1)} &= \frac{1}{N_c} \sum_{i=1}^{N_c} \frac{1}{|\mathcal{B}_{i}^{(n+1)}|}\sum_{\bm{b} \in \mathcal{B}_{i}^{(n+1)}} \frac{1}{L_b}\lVert \bm{b} - C_i^{(n)}(b)\rVert^2_2 \le J^{(n)}
\end{split}
\end{equation}
where $\tilde{J}^{(n+1)}$ is the the quantization MSE after the vector clustering step at iteration $n+1$.

Next, during the codebook update step (\ref{eq:quantizers_update}) at iteration $n+1$, the per-cluster codebooks $\mathcal{C}^{(n)}$ are updated to $\mathcal{C}^{(n+1)}$ by invoking the Lloyd-Max algorithm \citep{Lloyd}. We know that for any given value distribution, the Lloyd-Max algorithm minimizes the quantization MSE. Therefore, for a given vector cluster $\mathcal{B}_i$ we obtain the following result:

\begin{equation}
    \frac{1}{|\mathcal{B}_{i}^{(n+1)}|}\sum_{\bm{b} \in \mathcal{B}_{i}^{(n+1)}} \frac{1}{L_b}\lVert \bm{b}- C_i^{(n+1)}(\bm{b})\rVert^2_2 \le \frac{1}{|\mathcal{B}_{i}^{(n+1)}|}\sum_{\bm{b} \in \mathcal{B}_{i}^{(n+1)}} \frac{1}{L_b}\lVert \bm{b}- C_i^{(n)}(\bm{b})\rVert^2_2
\end{equation}

The above equation states that quantizing the given block cluster $\mathcal{B}_i$ after updating the associated codebook from $C_i^{(n)}$ to $C_i^{(n+1)}$ results in lower quantization MSE. Since this is true for all the block clusters, we derive the following result: 
\begin{equation}
\begin{split}
\label{eq:mse_ineq_2}
     J^{(n+1)} &= \frac{1}{N_c} \sum_{i=1}^{N_c} \frac{1}{|\mathcal{B}_{i}^{(n+1)}|}\sum_{\bm{b} \in \mathcal{B}_{i}^{(n+1)}} \frac{1}{L_b}\lVert \bm{b}- C_i^{(n+1)}(\bm{b})\rVert^2_2  \le \tilde{J}^{(n+1)}   
\end{split}
\end{equation}

Following (\ref{eq:mse_ineq_1}) and (\ref{eq:mse_ineq_2}), we find that the quantization MSE is non-increasing for each iteration, that is, $J^{(1)} \ge J^{(2)} \ge J^{(3)} \ge \ldots \ge J^{(M)}$ where $M$ is the maximum number of iterations. 
%Therefore, we can say that if the algorithm converges, then it must be that it has converged to a local minimum. 
\hfill $\blacksquare$


\begin{figure}
    \begin{center}
    \includegraphics[width=0.5\textwidth]{sections//figures/mse_vs_iter.pdf}
    \end{center}
    \caption{\small NMSE vs iterations during LO-BCQ compared to other block quantization proposals}
    \label{fig:nmse_vs_iter}
\end{figure}

Figure \ref{fig:nmse_vs_iter} shows the empirical convergence of LO-BCQ across several block lengths and number of codebooks. Also, the MSE achieved by LO-BCQ is compared to baselines such as MXFP and VSQ. As shown, LO-BCQ converges to a lower MSE than the baselines. Further, we achieve better convergence for larger number of codebooks ($N_c$) and for a smaller block length ($L_b$), both of which increase the bitwidth of BCQ (see Eq \ref{eq:bitwidth_bcq}).


\subsection{Additional Accuracy Results}
%Table \ref{tab:lobcq_config} lists the various LOBCQ configurations and their corresponding bitwidths.
\begin{table}
\setlength{\tabcolsep}{4.75pt}
\begin{center}
\caption{\label{tab:lobcq_config} Various LO-BCQ configurations and their bitwidths.}
\begin{tabular}{|c||c|c|c|c||c|c||c|} 
\hline
 & \multicolumn{4}{|c||}{$L_b=8$} & \multicolumn{2}{|c||}{$L_b=4$} & $L_b=2$ \\
 \hline
 \backslashbox{$L_A$\kern-1em}{\kern-1em$N_c$} & 2 & 4 & 8 & 16 & 2 & 4 & 2 \\
 \hline
 64 & 4.25 & 4.375 & 4.5 & 4.625 & 4.375 & 4.625 & 4.625\\
 \hline
 32 & 4.375 & 4.5 & 4.625& 4.75 & 4.5 & 4.75 & 4.75 \\
 \hline
 16 & 4.625 & 4.75& 4.875 & 5 & 4.75 & 5 & 5 \\
 \hline
\end{tabular}
\end{center}
\end{table}

%\subsection{Perplexity achieved by various LO-BCQ configurations on Wikitext-103 dataset}

\begin{table} \centering
\begin{tabular}{|c||c|c|c|c||c|c||c|} 
\hline
 $L_b \rightarrow$& \multicolumn{4}{c||}{8} & \multicolumn{2}{c||}{4} & 2\\
 \hline
 \backslashbox{$L_A$\kern-1em}{\kern-1em$N_c$} & 2 & 4 & 8 & 16 & 2 & 4 & 2  \\
 %$N_c \rightarrow$ & 2 & 4 & 8 & 16 & 2 & 4 & 2 \\
 \hline
 \hline
 \multicolumn{8}{c}{GPT3-1.3B (FP32 PPL = 9.98)} \\ 
 \hline
 \hline
 64 & 10.40 & 10.23 & 10.17 & 10.15 &  10.28 & 10.18 & 10.19 \\
 \hline
 32 & 10.25 & 10.20 & 10.15 & 10.12 &  10.23 & 10.17 & 10.17 \\
 \hline
 16 & 10.22 & 10.16 & 10.10 & 10.09 &  10.21 & 10.14 & 10.16 \\
 \hline
  \hline
 \multicolumn{8}{c}{GPT3-8B (FP32 PPL = 7.38)} \\ 
 \hline
 \hline
 64 & 7.61 & 7.52 & 7.48 &  7.47 &  7.55 &  7.49 & 7.50 \\
 \hline
 32 & 7.52 & 7.50 & 7.46 &  7.45 &  7.52 &  7.48 & 7.48  \\
 \hline
 16 & 7.51 & 7.48 & 7.44 &  7.44 &  7.51 &  7.49 & 7.47  \\
 \hline
\end{tabular}
\caption{\label{tab:ppl_gpt3_abalation} Wikitext-103 perplexity across GPT3-1.3B and 8B models.}
\end{table}

\begin{table} \centering
\begin{tabular}{|c||c|c|c|c||} 
\hline
 $L_b \rightarrow$& \multicolumn{4}{c||}{8}\\
 \hline
 \backslashbox{$L_A$\kern-1em}{\kern-1em$N_c$} & 2 & 4 & 8 & 16 \\
 %$N_c \rightarrow$ & 2 & 4 & 8 & 16 & 2 & 4 & 2 \\
 \hline
 \hline
 \multicolumn{5}{|c|}{Llama2-7B (FP32 PPL = 5.06)} \\ 
 \hline
 \hline
 64 & 5.31 & 5.26 & 5.19 & 5.18  \\
 \hline
 32 & 5.23 & 5.25 & 5.18 & 5.15  \\
 \hline
 16 & 5.23 & 5.19 & 5.16 & 5.14  \\
 \hline
 \multicolumn{5}{|c|}{Nemotron4-15B (FP32 PPL = 5.87)} \\ 
 \hline
 \hline
 64  & 6.3 & 6.20 & 6.13 & 6.08  \\
 \hline
 32  & 6.24 & 6.12 & 6.07 & 6.03  \\
 \hline
 16  & 6.12 & 6.14 & 6.04 & 6.02  \\
 \hline
 \multicolumn{5}{|c|}{Nemotron4-340B (FP32 PPL = 3.48)} \\ 
 \hline
 \hline
 64 & 3.67 & 3.62 & 3.60 & 3.59 \\
 \hline
 32 & 3.63 & 3.61 & 3.59 & 3.56 \\
 \hline
 16 & 3.61 & 3.58 & 3.57 & 3.55 \\
 \hline
\end{tabular}
\caption{\label{tab:ppl_llama7B_nemo15B} Wikitext-103 perplexity compared to FP32 baseline in Llama2-7B and Nemotron4-15B, 340B models}
\end{table}

%\subsection{Perplexity achieved by various LO-BCQ configurations on MMLU dataset}


\begin{table} \centering
\begin{tabular}{|c||c|c|c|c||c|c|c|c|} 
\hline
 $L_b \rightarrow$& \multicolumn{4}{c||}{8} & \multicolumn{4}{c||}{8}\\
 \hline
 \backslashbox{$L_A$\kern-1em}{\kern-1em$N_c$} & 2 & 4 & 8 & 16 & 2 & 4 & 8 & 16  \\
 %$N_c \rightarrow$ & 2 & 4 & 8 & 16 & 2 & 4 & 2 \\
 \hline
 \hline
 \multicolumn{5}{|c|}{Llama2-7B (FP32 Accuracy = 45.8\%)} & \multicolumn{4}{|c|}{Llama2-70B (FP32 Accuracy = 69.12\%)} \\ 
 \hline
 \hline
 64 & 43.9 & 43.4 & 43.9 & 44.9 & 68.07 & 68.27 & 68.17 & 68.75 \\
 \hline
 32 & 44.5 & 43.8 & 44.9 & 44.5 & 68.37 & 68.51 & 68.35 & 68.27  \\
 \hline
 16 & 43.9 & 42.7 & 44.9 & 45 & 68.12 & 68.77 & 68.31 & 68.59  \\
 \hline
 \hline
 \multicolumn{5}{|c|}{GPT3-22B (FP32 Accuracy = 38.75\%)} & \multicolumn{4}{|c|}{Nemotron4-15B (FP32 Accuracy = 64.3\%)} \\ 
 \hline
 \hline
 64 & 36.71 & 38.85 & 38.13 & 38.92 & 63.17 & 62.36 & 63.72 & 64.09 \\
 \hline
 32 & 37.95 & 38.69 & 39.45 & 38.34 & 64.05 & 62.30 & 63.8 & 64.33  \\
 \hline
 16 & 38.88 & 38.80 & 38.31 & 38.92 & 63.22 & 63.51 & 63.93 & 64.43  \\
 \hline
\end{tabular}
\caption{\label{tab:mmlu_abalation} Accuracy on MMLU dataset across GPT3-22B, Llama2-7B, 70B and Nemotron4-15B models.}
\end{table}


%\subsection{Perplexity achieved by various LO-BCQ configurations on LM evaluation harness}

\begin{table} \centering
\begin{tabular}{|c||c|c|c|c||c|c|c|c|} 
\hline
 $L_b \rightarrow$& \multicolumn{4}{c||}{8} & \multicolumn{4}{c||}{8}\\
 \hline
 \backslashbox{$L_A$\kern-1em}{\kern-1em$N_c$} & 2 & 4 & 8 & 16 & 2 & 4 & 8 & 16  \\
 %$N_c \rightarrow$ & 2 & 4 & 8 & 16 & 2 & 4 & 2 \\
 \hline
 \hline
 \multicolumn{5}{|c|}{Race (FP32 Accuracy = 37.51\%)} & \multicolumn{4}{|c|}{Boolq (FP32 Accuracy = 64.62\%)} \\ 
 \hline
 \hline
 64 & 36.94 & 37.13 & 36.27 & 37.13 & 63.73 & 62.26 & 63.49 & 63.36 \\
 \hline
 32 & 37.03 & 36.36 & 36.08 & 37.03 & 62.54 & 63.51 & 63.49 & 63.55  \\
 \hline
 16 & 37.03 & 37.03 & 36.46 & 37.03 & 61.1 & 63.79 & 63.58 & 63.33  \\
 \hline
 \hline
 \multicolumn{5}{|c|}{Winogrande (FP32 Accuracy = 58.01\%)} & \multicolumn{4}{|c|}{Piqa (FP32 Accuracy = 74.21\%)} \\ 
 \hline
 \hline
 64 & 58.17 & 57.22 & 57.85 & 58.33 & 73.01 & 73.07 & 73.07 & 72.80 \\
 \hline
 32 & 59.12 & 58.09 & 57.85 & 58.41 & 73.01 & 73.94 & 72.74 & 73.18  \\
 \hline
 16 & 57.93 & 58.88 & 57.93 & 58.56 & 73.94 & 72.80 & 73.01 & 73.94  \\
 \hline
\end{tabular}
\caption{\label{tab:mmlu_abalation} Accuracy on LM evaluation harness tasks on GPT3-1.3B model.}
\end{table}

\begin{table} \centering
\begin{tabular}{|c||c|c|c|c||c|c|c|c|} 
\hline
 $L_b \rightarrow$& \multicolumn{4}{c||}{8} & \multicolumn{4}{c||}{8}\\
 \hline
 \backslashbox{$L_A$\kern-1em}{\kern-1em$N_c$} & 2 & 4 & 8 & 16 & 2 & 4 & 8 & 16  \\
 %$N_c \rightarrow$ & 2 & 4 & 8 & 16 & 2 & 4 & 2 \\
 \hline
 \hline
 \multicolumn{5}{|c|}{Race (FP32 Accuracy = 41.34\%)} & \multicolumn{4}{|c|}{Boolq (FP32 Accuracy = 68.32\%)} \\ 
 \hline
 \hline
 64 & 40.48 & 40.10 & 39.43 & 39.90 & 69.20 & 68.41 & 69.45 & 68.56 \\
 \hline
 32 & 39.52 & 39.52 & 40.77 & 39.62 & 68.32 & 67.43 & 68.17 & 69.30  \\
 \hline
 16 & 39.81 & 39.71 & 39.90 & 40.38 & 68.10 & 66.33 & 69.51 & 69.42  \\
 \hline
 \hline
 \multicolumn{5}{|c|}{Winogrande (FP32 Accuracy = 67.88\%)} & \multicolumn{4}{|c|}{Piqa (FP32 Accuracy = 78.78\%)} \\ 
 \hline
 \hline
 64 & 66.85 & 66.61 & 67.72 & 67.88 & 77.31 & 77.42 & 77.75 & 77.64 \\
 \hline
 32 & 67.25 & 67.72 & 67.72 & 67.00 & 77.31 & 77.04 & 77.80 & 77.37  \\
 \hline
 16 & 68.11 & 68.90 & 67.88 & 67.48 & 77.37 & 78.13 & 78.13 & 77.69  \\
 \hline
\end{tabular}
\caption{\label{tab:mmlu_abalation} Accuracy on LM evaluation harness tasks on GPT3-8B model.}
\end{table}

\begin{table} \centering
\begin{tabular}{|c||c|c|c|c||c|c|c|c|} 
\hline
 $L_b \rightarrow$& \multicolumn{4}{c||}{8} & \multicolumn{4}{c||}{8}\\
 \hline
 \backslashbox{$L_A$\kern-1em}{\kern-1em$N_c$} & 2 & 4 & 8 & 16 & 2 & 4 & 8 & 16  \\
 %$N_c \rightarrow$ & 2 & 4 & 8 & 16 & 2 & 4 & 2 \\
 \hline
 \hline
 \multicolumn{5}{|c|}{Race (FP32 Accuracy = 40.67\%)} & \multicolumn{4}{|c|}{Boolq (FP32 Accuracy = 76.54\%)} \\ 
 \hline
 \hline
 64 & 40.48 & 40.10 & 39.43 & 39.90 & 75.41 & 75.11 & 77.09 & 75.66 \\
 \hline
 32 & 39.52 & 39.52 & 40.77 & 39.62 & 76.02 & 76.02 & 75.96 & 75.35  \\
 \hline
 16 & 39.81 & 39.71 & 39.90 & 40.38 & 75.05 & 73.82 & 75.72 & 76.09  \\
 \hline
 \hline
 \multicolumn{5}{|c|}{Winogrande (FP32 Accuracy = 70.64\%)} & \multicolumn{4}{|c|}{Piqa (FP32 Accuracy = 79.16\%)} \\ 
 \hline
 \hline
 64 & 69.14 & 70.17 & 70.17 & 70.56 & 78.24 & 79.00 & 78.62 & 78.73 \\
 \hline
 32 & 70.96 & 69.69 & 71.27 & 69.30 & 78.56 & 79.49 & 79.16 & 78.89  \\
 \hline
 16 & 71.03 & 69.53 & 69.69 & 70.40 & 78.13 & 79.16 & 79.00 & 79.00  \\
 \hline
\end{tabular}
\caption{\label{tab:mmlu_abalation} Accuracy on LM evaluation harness tasks on GPT3-22B model.}
\end{table}

\begin{table} \centering
\begin{tabular}{|c||c|c|c|c||c|c|c|c|} 
\hline
 $L_b \rightarrow$& \multicolumn{4}{c||}{8} & \multicolumn{4}{c||}{8}\\
 \hline
 \backslashbox{$L_A$\kern-1em}{\kern-1em$N_c$} & 2 & 4 & 8 & 16 & 2 & 4 & 8 & 16  \\
 %$N_c \rightarrow$ & 2 & 4 & 8 & 16 & 2 & 4 & 2 \\
 \hline
 \hline
 \multicolumn{5}{|c|}{Race (FP32 Accuracy = 44.4\%)} & \multicolumn{4}{|c|}{Boolq (FP32 Accuracy = 79.29\%)} \\ 
 \hline
 \hline
 64 & 42.49 & 42.51 & 42.58 & 43.45 & 77.58 & 77.37 & 77.43 & 78.1 \\
 \hline
 32 & 43.35 & 42.49 & 43.64 & 43.73 & 77.86 & 75.32 & 77.28 & 77.86  \\
 \hline
 16 & 44.21 & 44.21 & 43.64 & 42.97 & 78.65 & 77 & 76.94 & 77.98  \\
 \hline
 \hline
 \multicolumn{5}{|c|}{Winogrande (FP32 Accuracy = 69.38\%)} & \multicolumn{4}{|c|}{Piqa (FP32 Accuracy = 78.07\%)} \\ 
 \hline
 \hline
 64 & 68.9 & 68.43 & 69.77 & 68.19 & 77.09 & 76.82 & 77.09 & 77.86 \\
 \hline
 32 & 69.38 & 68.51 & 68.82 & 68.90 & 78.07 & 76.71 & 78.07 & 77.86  \\
 \hline
 16 & 69.53 & 67.09 & 69.38 & 68.90 & 77.37 & 77.8 & 77.91 & 77.69  \\
 \hline
\end{tabular}
\caption{\label{tab:mmlu_abalation} Accuracy on LM evaluation harness tasks on Llama2-7B model.}
\end{table}

\begin{table} \centering
\begin{tabular}{|c||c|c|c|c||c|c|c|c|} 
\hline
 $L_b \rightarrow$& \multicolumn{4}{c||}{8} & \multicolumn{4}{c||}{8}\\
 \hline
 \backslashbox{$L_A$\kern-1em}{\kern-1em$N_c$} & 2 & 4 & 8 & 16 & 2 & 4 & 8 & 16  \\
 %$N_c \rightarrow$ & 2 & 4 & 8 & 16 & 2 & 4 & 2 \\
 \hline
 \hline
 \multicolumn{5}{|c|}{Race (FP32 Accuracy = 48.8\%)} & \multicolumn{4}{|c|}{Boolq (FP32 Accuracy = 85.23\%)} \\ 
 \hline
 \hline
 64 & 49.00 & 49.00 & 49.28 & 48.71 & 82.82 & 84.28 & 84.03 & 84.25 \\
 \hline
 32 & 49.57 & 48.52 & 48.33 & 49.28 & 83.85 & 84.46 & 84.31 & 84.93  \\
 \hline
 16 & 49.85 & 49.09 & 49.28 & 48.99 & 85.11 & 84.46 & 84.61 & 83.94  \\
 \hline
 \hline
 \multicolumn{5}{|c|}{Winogrande (FP32 Accuracy = 79.95\%)} & \multicolumn{4}{|c|}{Piqa (FP32 Accuracy = 81.56\%)} \\ 
 \hline
 \hline
 64 & 78.77 & 78.45 & 78.37 & 79.16 & 81.45 & 80.69 & 81.45 & 81.5 \\
 \hline
 32 & 78.45 & 79.01 & 78.69 & 80.66 & 81.56 & 80.58 & 81.18 & 81.34  \\
 \hline
 16 & 79.95 & 79.56 & 79.79 & 79.72 & 81.28 & 81.66 & 81.28 & 80.96  \\
 \hline
\end{tabular}
\caption{\label{tab:mmlu_abalation} Accuracy on LM evaluation harness tasks on Llama2-70B model.}
\end{table}

%\section{MSE Studies}
%\textcolor{red}{TODO}


\subsection{Number Formats and Quantization Method}
\label{subsec:numFormats_quantMethod}
\subsubsection{Integer Format}
An $n$-bit signed integer (INT) is typically represented with a 2s-complement format \citep{yao2022zeroquant,xiao2023smoothquant,dai2021vsq}, where the most significant bit denotes the sign.

\subsubsection{Floating Point Format}
An $n$-bit signed floating point (FP) number $x$ comprises of a 1-bit sign ($x_{\mathrm{sign}}$), $B_m$-bit mantissa ($x_{\mathrm{mant}}$) and $B_e$-bit exponent ($x_{\mathrm{exp}}$) such that $B_m+B_e=n-1$. The associated constant exponent bias ($E_{\mathrm{bias}}$) is computed as $(2^{{B_e}-1}-1)$. We denote this format as $E_{B_e}M_{B_m}$.  

\subsubsection{Quantization Scheme}
\label{subsec:quant_method}
A quantization scheme dictates how a given unquantized tensor is converted to its quantized representation. We consider FP formats for the purpose of illustration. Given an unquantized tensor $\bm{X}$ and an FP format $E_{B_e}M_{B_m}$, we first, we compute the quantization scale factor $s_X$ that maps the maximum absolute value of $\bm{X}$ to the maximum quantization level of the $E_{B_e}M_{B_m}$ format as follows:
\begin{align}
\label{eq:sf}
    s_X = \frac{\mathrm{max}(|\bm{X}|)}{\mathrm{max}(E_{B_e}M_{B_m})}
\end{align}
In the above equation, $|\cdot|$ denotes the absolute value function.

Next, we scale $\bm{X}$ by $s_X$ and quantize it to $\hat{\bm{X}}$ by rounding it to the nearest quantization level of $E_{B_e}M_{B_m}$ as:

\begin{align}
\label{eq:tensor_quant}
    \hat{\bm{X}} = \text{round-to-nearest}\left(\frac{\bm{X}}{s_X}, E_{B_e}M_{B_m}\right)
\end{align}

We perform dynamic max-scaled quantization \citep{wu2020integer}, where the scale factor $s$ for activations is dynamically computed during runtime.

\subsection{Vector Scaled Quantization}
\begin{wrapfigure}{r}{0.35\linewidth}
  \centering
  \includegraphics[width=\linewidth]{sections/figures/vsquant.jpg}
  \caption{\small Vectorwise decomposition for per-vector scaled quantization (VSQ \citep{dai2021vsq}).}
  \label{fig:vsquant}
\end{wrapfigure}
During VSQ \citep{dai2021vsq}, the operand tensors are decomposed into 1D vectors in a hardware friendly manner as shown in Figure \ref{fig:vsquant}. Since the decomposed tensors are used as operands in matrix multiplications during inference, it is beneficial to perform this decomposition along the reduction dimension of the multiplication. The vectorwise quantization is performed similar to tensorwise quantization described in Equations \ref{eq:sf} and \ref{eq:tensor_quant}, where a scale factor $s_v$ is required for each vector $\bm{v}$ that maps the maximum absolute value of that vector to the maximum quantization level. While smaller vector lengths can lead to larger accuracy gains, the associated memory and computational overheads due to the per-vector scale factors increases. To alleviate these overheads, VSQ \citep{dai2021vsq} proposed a second level quantization of the per-vector scale factors to unsigned integers, while MX \citep{rouhani2023shared} quantizes them to integer powers of 2 (denoted as $2^{INT}$).

\subsubsection{MX Format}
The MX format proposed in \citep{rouhani2023microscaling} introduces the concept of sub-block shifting. For every two scalar elements of $b$-bits each, there is a shared exponent bit. The value of this exponent bit is determined through an empirical analysis that targets minimizing quantization MSE. We note that the FP format $E_{1}M_{b}$ is strictly better than MX from an accuracy perspective since it allocates a dedicated exponent bit to each scalar as opposed to sharing it across two scalars. Therefore, we conservatively bound the accuracy of a $b+2$-bit signed MX format with that of a $E_{1}M_{b}$ format in our comparisons. For instance, we use E1M2 format as a proxy for MX4.

\begin{figure}
    \centering
    \includegraphics[width=1\linewidth]{sections//figures/BlockFormats.pdf}
    \caption{\small Comparing LO-BCQ to MX format.}
    \label{fig:block_formats}
\end{figure}

Figure \ref{fig:block_formats} compares our $4$-bit LO-BCQ block format to MX \citep{rouhani2023microscaling}. As shown, both LO-BCQ and MX decompose a given operand tensor into block arrays and each block array into blocks. Similar to MX, we find that per-block quantization ($L_b < L_A$) leads to better accuracy due to increased flexibility. While MX achieves this through per-block $1$-bit micro-scales, we associate a dedicated codebook to each block through a per-block codebook selector. Further, MX quantizes the per-block array scale-factor to E8M0 format without per-tensor scaling. In contrast during LO-BCQ, we find that per-tensor scaling combined with quantization of per-block array scale-factor to E4M3 format results in superior inference accuracy across models. 

\end{document}

