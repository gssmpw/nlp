\newcommand{\figone}{
\begin{figure*}[!tb]
    \begin{center}
    \includegraphics[keepaspectratio, width=\linewidth]{image/fig1.pdf}
    \caption{Framework of our proposed method. We utilize ICL for multiple image inputs to give VLM the inspection criteria of new products. Our framework gives the coordinates of the defective location, which helps the user understand the model's decision. In addition, it is easy to address by replacing the foundational model when a better VLM is proposed.}
    \label{fig1}
    \end{center}
\end{figure*}
}

\newcommand{\model}{
\begin{figure}[!tb]
    \begin{center}
    \includegraphics[keepaspectratio, width=\linewidth]{image/model.pdf}
    \caption{Architecture of ViP-LLaVA. After providing an image and the corresponding text, the image is tokenized by CLIP ViT, LayerNorm, and MLP layers, while the text is tokenized by tokenizer. Then the visual tokens and the text tokens are given to the LLM to generate the answer.}
    \label{model}
    \end{center}
\end{figure}
}


\newcommand{\collectedimages}{
\begin{figure*}[!tb]
    \begin{center}
    \includegraphics[keepaspectratio, width=\linewidth]{image/collected_images.pdf}
    \caption{Examples of collected images and removal of non-defective and defective products. Images with a black cross mark are removed. 
    We remove the duplicated images and images that are too difficult for visual inspection. For instance, in ``painting'' images, there are ``fading'' or ``warp'' images, but these are considered non-defective if they are part of the image's style. Thus, we remove these kinds of images.}
    \label{collectedimages}
    \end{center}
\end{figure*}
}

\newcommand{\difficulty}{
\begin{figure}[!tb]
    \begin{center}
    \includegraphics[keepaspectratio, width=\linewidth]{image/difficulty_v2.pdf}
    \caption{Examples of non-defective and defective images of ``Pill'' in MVTec AD and ``Capsules'' in VisA. For ``Pill'', the non-defective image also contains red spots, making it difficult to inspect Similarly, for ``Capsules'', the non-defective image also contains brown stains.}
    \label{difficulty}
    \end{center}
\end{figure}
}

\newcommand{\prompttrain}{
\begin{figure*}[!tb]
    \begin{center}
    \includegraphics[keepaspectratio, width=\linewidth]{image/prompt_train.pdf}
    \caption{Prompts with unified output format. For the upper one, non-defective image, the answer is ``None''. For the lower one, defective image, the answer is the coordinates of the defective location.}
    \label{prompttrain}
    \end{center}
\end{figure*}
}

\newcommand{\prompttest}{
\begin{figure}[!tb]
    \begin{center}
    \includegraphics[keepaspectratio, width=\linewidth]{image/prompt_test.pdf}
    \caption{Framework of evaluation. First, select the example based on Eq. (1), then infer the test image with ICL.}
    \label{prompttest}
    \end{center}
\end{figure}
}

\newcommand{\mvtecbb}{
\begin{figure}[!tb]
    \begin{center}
    \includegraphics[keepaspectratio, width=\linewidth]{image/mvtec_bb_v2.pdf}
    \caption{Visualize the model prediction for MVTec AD.}
    \label{mvtecbb2}
    \end{center}
\end{figure}
}

\newcommand{\visabb}{
\begin{figure}[!tb]
    \begin{center}
    \includegraphics[keepaspectratio, width=\linewidth]{image/visa_bb_v2.pdf}
    \caption{Visualize the model prediction for VisA.}
    \label{visa_bb}
    \end{center}
\end{figure}
}

\newcommand{\comparison}{
\begin{figure}[!tb]
    \begin{center}
    \includegraphics[keepaspectratio, width=\linewidth]{image/images_comparison.pdf}
    \caption{Examples of the images of ``Bottle'', and ``Tile'' from the collected images and MVTec AD.}
    \label{comparison}
    \end{center}
\end{figure}
}

\newcommand{\category}{
\begin{figure}[!tb]
    \begin{center}
    \includegraphics[keepaspectratio, width=\linewidth]{image/category.pdf}
    \caption{Product category of our dataset.}
    \label{category}
    \end{center}
\end{figure}
}

\newcommand{\resultwoft}{
\begin{figure*}[!tb]
    \begin{center}
    \includegraphics[keepaspectratio, width=\linewidth]{image/result_wo_ft.pdf}
    \caption{Result of the ViP-LLaVA before fine-tuning.}
    \label{resultwoft}
    \end{center}
\end{figure*}
}

\newcommand{\mvtecbbfull}{
\begin{figure*}[!tb]
    \begin{center}
    \includegraphics[keepaspectratio, width=0.6\linewidth]{image/mvtec_bb_full.pdf}
    \caption{Visualize the model prediction of all products for MVTec AD.}
    \label{mvtecbbfull}
    \end{center}
\end{figure*}
}

\newcommand{\visabbfull}{
\begin{figure*}[!tb]
    \begin{center}
    \includegraphics[keepaspectratio, width=0.7\linewidth]{image/visa_bb_full.pdf}
    \caption{Visualize the model prediction of all products for VisA.}
    \label{visabbfull}
    \end{center}
\end{figure*}
}