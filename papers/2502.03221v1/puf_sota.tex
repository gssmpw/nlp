\section{State of the Art}\label{sec:sota}

While some silicon PUFs such as the SRAM or arbiter PUF directly output digital information, other silicon PUFs, like the Ring-Oscillator or TERO PUF, and in particular non-silicon PUFs output analog, or finely quantized digital data. They all have in common, that they undergo a processing chain involving helper data and error correction, until a cryptographic key is output (see Section~\ref{subsec:HDA}).

\subsection{PUF-Based Tamper Protection}\label{subsec:puf-basedtampprot}

In the past, tamper protection was implemented through a continuously powered detection mechanism, e.g. in tamper-responsive envelopes and covers that wrap or cover the structure to be protected \cite{IMFC13,OI18b}. Within the protective structure, a physical measure such as electrical resistance is measured and triggers an alarm as soon as the measured value exceeds a threshold to erase sensitive information, stored e.g. in battery-backed memory. While the device is only active during a fraction of the time, the protective measures need to be active for the entire life-time of the device, after it leaves a trusted manufacturing site. PUF-based tamper protection promises to increase sensitivity and to ease the handling during operation of the devices, as the device can be fully powered off. Thus PUFs, can contribute to an easy-to-handle and future-proof tamper protection technology.

Over the last 20 years, different measures were taken to combine PUFs and tamper protection. An early type is the coating PUF, where a protective coating is spread over an IC and evaluated from its inside to detect physical tampering when the coating is penetrated \cite{SMKT06,TSS+06}. In addition to electrical measurements, also optical approaches \cite{esbach2012new,vai2015secure}, mechanical properties \cite{GS22} or propagation behavior of radio waves \cite{STZP22} were proposed.

In this work, we use PUF-based tamper protection realized by foils and covers, as also proposed in \cite{IOK+18,Imm19,ION+19,GOFK21} as reference. This scenario is depicted in Figure~\ref{fig:pufenv}. The foil is wrapped around a PCB with mounted electronics components to be protected and consists of two structured electrode layers (blue and red) and two electrical shielding layers (black). The capacitive coupling between the electrode layers is measured to obtain the PUF response \cite{OIHS18}. In addition, faster mechanisms for run-time tamper protection are included.

\begin{figure}
	\centering
	\includegraphics[width=0.5\linewidth]{puffigs/schematic_leer-eps-converted-to.pdf}
	\caption{Sketch of PUF-based envelope \cite{IOK+18} with two conducting electrode layers in red and blue, and two shielding layers in black}
	\label{fig:pufenv}
\end{figure}

The capacitance values are subject to measurement noise, temperature and other environmental effects, as discussed based on measured values, e.g. in \cite{GXKF22}. A broad range of deterministic effects can be compensated with linear or also higher order reference points of fits \cite{OIHS18,GXKF22,riehm2023structured,gunlu2015reliable}, whereas Gaussian measurement noise remains in all cases. We will focus on this fundamental noise issue using a wiretap scenario in this work and refer to the compound case for including multiple environmental conditions \cite{liang2009compound,boche2013secret}.

\subsection{Quantization}

Analog PUFs evaluated on embedded devices need to quantize the digitized PUF data into a finite alphabet that is input into the error correcting code (see Section~\ref{subsec:hda_analog}). Typically, public helper data that references the distance and direction to the next interval border is stored to move a measured data point away from decision borders right into the middle of the quantization interval. So far, work on PUFs typically either uses equiprobable or equidistant quantization \cite{IHKS16}. As shown in Section~\ref{subsec:zero_leakage_hds} it is possible though to use arbitrary input quantization while still generating helper data that is not leaking any information about the secret (zero-leakage helper data).

\paragraph{Equiprobable Quantization} The range of possible output values of the PUF is split into $d$ intervals with the same probability such that all indices as sampled with the same probability. This is favorable from a security point of view, as iid PUF values are mapped into uniform data in the finite alphabet. However, this comes with two downsides: 

First, the common helper data generation bringing the expected value during reconstruction into the center of the interval leaks information about the secret, as the helper data pointers of different intervals have different distributions \cite{IHKS16}. Later, we will introduce a method to obtain zero leakage helper data such that this problem is mitigated. 

Second, the decision borders in the center are rather narrow, so that the values in this intervals are subject to a higher error rate. The wide intervals in the tails of the Gaussian distribution considered in our model also have a decreased sensitivity for physical tampering.

Equiprobable quantization can be applied if the requirement for uniformity is more substantial than the requirement for tamper detection sensitivity \cite{GXKF22}.

\paragraph{Equidistant Quantization} In contrast, equidistant quantization samples the analog values by mapping them to intervals of the same size. This has the advantage that additive noise effects all values in the same way and error probabilities are constant with respect to the PUF values. Also, only a negligible leakage can be observed \cite{IHKS16} through the aforementioned helper data pointers. With the later on introduced zero leakage helper data algorithms, this is less of a factor. As a downside, this comes at the expense of a heavy bias as the intervals differ in probability. This can be mitigated through variable-length encoding at the expense of a limited selection of code classes for the later ECC \cite{IHL+18}.
As shown in \cite{BDH+10}, also higher-dimensional structures can be used for embedding secret data, which adds more degrees of freedom.

\subsection{Error Correction}

Aside from the helper data used to perform better output quantization during reconstruction (denoted by $W^n$ later in Section~\ref{subsec:HDA}) additional helper data $\widetilde{W}$ is generated to link the PUF response to a codeword of an error correcting code. 
This can be done either by linear schemes that generate the link through linear dependencies such as syndromes or XORs or pointers that refer to parts of the PUF response \cite{HKS20}.

In any case, an ECC is used to reduce the key error probability e.g. down to $10^{-6}$ or $10^{-9}$ to generate reliable keys. This can be achieved with standard codes such as BCH, Reed-Solomon or Convolutional codes \cite{MS77}, or newer code classes such as limited magnitude codes \cite{IU19} or Polar codes \cite{CIW+17,GXKF22}.

In addition, wiretap codes can be used either for leakage prevention \cite{HO17} or attack prevention \cite{GXKF22}.
