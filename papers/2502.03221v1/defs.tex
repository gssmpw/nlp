%\usepackage{lipsum}


% General Setup

% Color
%\usepackage[dvipsnames]{xcolor}

% Blautöne:
\definecolor{TUMBlau}{RGB}{0,101,189} % Pantone 300 (0, 0.396, 0.7412)
\definecolor{TUMBlauDunkel}{RGB}{0,82,147} % Pantone 301 (0, 0.3216,0.5765)
\definecolor{TUMBlauHell}{RGB}{152,198,234} % Pantone 283 (0.596,0.776,0.918)
\definecolor{TUMBlauMittel}{RGB}{100,160,200} % Pantone 542 (0.392, 0.627, 0.784)

% Hervorhebung:
\definecolor{TUMElfenbein}{RGB}{218,215,203} % Pantone 7527 -Elfenbein
\definecolor{TUMGruen}{RGB}{162,173,0} % Pantone 383 - Grün (0.6353,0.6784,0)
\definecolor{TUMOrange}{RGB}{227,114,34} % Pantone 158 - Orange (0.8902, 0.4471, 0.133)
\definecolor{TUMGrau}{gray}{0.6} % Grau 60%

% More colors, not sure if super official
\definecolor{TUMGruenDunkel}{RGB}{0,124,48} % (0,0.4863,0.1882)
\definecolor{TUMRot}{RGB}{196,7,27} % (0.7686,0.02745,0.10588)

% English colors
\definecolor{TUMBlue}{RGB}{0,101,189} % Pantone 300 (0, 0.396, 0.7412)
\definecolor{TUMBlueDark}{RGB}{0,82,147} % Pantone 301 (0, 0.3216,0.5765)
\definecolor{TUMBlueLight}{RGB}{152,198,234} % Pantone 283 (0.596,0.776,0.918)
\definecolor{TUMBlueMedium}{RGB}{100,160,200} % Pantone 542 (0.392, 0.627, 0.784)

% Hervorhebung:
\definecolor{TUMIvory}{RGB}{218,215,203} % Pantone 7527 -Elfenbein
\definecolor{TUMGreen}{RGB}{162,173,0} % Pantone 383 - Grün (0.6353,0.6784,0)
\definecolor{TUMGray}{gray}{0.6} % Grau 60%
\definecolor{TUMGrayDark}{gray}{0.3} % Grau 80%

% More colors, not sure if super official
\definecolor{TUMGreenDark}{RGB}{0,124,48} % (0,0.4863,0.1882)
\definecolor{TUMRed}{RGB}{196,7,27} % (0.7686,0.02745,0.10588)

% Plot colors
\definecolor{plotColor1}{RGB}{0,101,189} % TUMBlue, Pantone 300 (0, 0.396, 0.7412)
\definecolor{plotColor2}{RGB}{0,124,48} % TUMGreenDark, (0,0.4863,0.1882)
\definecolor{plotColor3}{RGB}{196,7,27} % TUMRed, (0.7686,0.02745,0.10588)
\definecolor{plotColor4}{RGB}{227,114,34} % TUMOrange, Pantone 158 - Orange (0.8902, 0.4471, 0.133)
\definecolor{plotColor5}{RGB}{0,82,147} % TUMBlueDark, Pantone 7527 -Elfenbein
\definecolor{plotColor6}{RGB}{162,173,0} % TUMGreen, Pantone 283 (0.596,0.776,0.918)
\definecolor{plotColor7}{gray}{0.3} % TUMGrayDark, Grau 80%

% Plot colors
\definecolor{plotColorIA1}{RGB}{0,124,48} % TUMGreenDark, (0,0.4863,0.1882)
\definecolor{plotColorIA2}{RGB}{0,82,147} % TUMBlueDark, Pantone 7527 -Elfenbein
\definecolor{plotColorIA3}{RGB}{100,160,200} % TUMBlueMedium, Pantone 542 (0.392, 0.627, 0.784)
\definecolor{plotColorIA4}{RGB}{152,198,234} % TUMBlueLight, Pantone 283 (0.596,0.776,0.918)
\definecolor{plotColorIA5}{gray}{0.3} % TUMGrayDark, Grau 80%
\definecolor{plotColorIA6}{RGB}{227,114,34} % TUMOrange, Pantone 158 - Orange (0.8902, 0.4471, 0.133)
\definecolor{plotColorIA7}{RGB}{196,7,27} % TUMRed, (0.7686,0.02745,0.10588)


%%% Comments
\newif\ifshowComment
 % \showCommenttrue
 \showCommentfalse
\newcommand{\todo}[1]{\ifshowComment {\color{TUMBlue}[#1]} \fi}
\newcommand{\maybe}[1]{\ifshowComment {\color{TUMOrange}[#1]} \fi}
\newcommand{\critical}[1]{\ifshowComment {\color{TUMRed}[#1]} \fi}
\newcommand{\notation}[1]{\ifshowComment {\color{TUMGreen}[#1]} \fi}

% Highlight changes
% \newif\ifhighlightChanges
% \highlightChangestrue
% \highlightChangesfalse
% \newcommand{\draft}[1]{{\ifhighlightChanges \color{TUMGrayDark} \fi #1}}



% Math packages
\usepackage{amsmath}
%\numberwithin{equation}{chapter}
\usepackage{amssymb,mathtools,amsbsy}
\usepackage{bbm}
\usepackage{epsfig}
\usepackage{mathdots}
\setcounter{MaxMatrixCols}{20}

% Font
\renewcommand{\rm}{\normalfont \rmfamily}

% Tables
% \usepackage{colortbl}
\usepackage{booktabs}
\usepackage{tabularx}
\usepackage{longtable} % Table with pagebreaks, must be before "arydshln"

% Arrays
\usepackage{array}
\usepackage{multirow}
\usepackage{arydshln}
\newcolumntype{C}{>{$}c<{$}} % math-mode version of "c" column type

% Links in PDF
\usepackage[hidelinks]{hyperref}

% Algorithm
\usepackage[noend]{algpseudocode}
\usepackage[linesnumbered,ruled,vlined,titlenumbered]{algorithm2e}
\usepackage{algorithmicx}

% Environments
% TiKz/pgfplot setup
\usepackage{pgfplots}
\pgfplotsset{compat=newest}
\usepgfplotslibrary{fillbetween}
\usepackage{tikz}
\usetikzlibrary{matrix,arrows,arrows.meta,positioning,calc,shapes}
\usepackage{circuitikz}


% Citation

% Allows for more control over bibliography
% \usepackage[backend=bibtex,
% %style=numeric,
% style=alphabetic,
% minalphanames=3,
% maxalphanames=4,
% % bibencoding=ascii,
% % style=reading,
% % sorting=nty,
% maxbibnames=99
% ]{biblatex}

%\renewcommand*{\labelalphaothers}{\textsuperscript{+}} % Make + sign in citations superscript

% Captions
\usepackage{caption,subcaption}

% Itemize
\usepackage{enumitem}




% Miscellaneous
\usepackage{stackengine} % Two-lined text
\def\stackalignment{l}
% Reference same footnotes several times in the text
\usepackage{footnote}
\makeatletter
\newcommand{\footnoteref}[1]{\protected@xdef\@thefnmark{\ref{#1}}\@footnotemark}
\makeatother

\usepackage{nicefrac} % nicer in-line fractions


% Theorem environments
\usepackage{amsthm}
\usepackage{thm-restate}

% References
\usepackage[capitalise]{cleveref}
% \labelformat{section}{\thechapter.#1}% give chapter number when referencing sections - doesn't work well, looks like referencing a subsection

% \newtheorem{theorem}{Theorem}[chapter]
% \declaretheorem[name=Theorem,sibling=theorem]{mythm}
% \newtheorem{definition}{Definition}[chapter]
% \newtheorem{remark}{Remark}[chapter]
% \newtheorem{lemma}{Lemma}[chapter]
% \newtheorem{corollary}{Corollary}[chapter]
% \newtheorem{construction}{Construction}[chapter]
% \newtheorem{example}{Example}[chapter]
% \newtheorem{conjecture}{Conjecture}[chapter]
% \newtheorem{proposition}{Proposition}[chapter]
% \newtheorem{question}{Question}[chapter]
% \newtheorem{problem}{Problem}[chapter]
% \newtheorem{openproblem}{Open Research Problem}[chapter]
% \newtheorem{assumption}{Assumption}


%%% Notation macros %%%

%%% Letters %%%

% Caligraphic
\newcommand{\cA}{\mathcal{A}}
\newcommand{\cB}{\mathcal{B}}
\newcommand{\cC}{\mathcal{C}}
\newcommand{\cD}{\mathcal{D}}
\newcommand{\cE}{\mathcal{E}}
\newcommand{\cF}{\mathcal{F}}
\newcommand{\cG}{\mathcal{G}}
\newcommand{\cH}{\mathcal{H}}
\newcommand{\cI}{\mathcal{I}}
\newcommand{\cJ}{\mathcal{J}}
\newcommand{\cK}{\mathcal{K}}
\newcommand{\cL}{\mathcal{L}}
\newcommand{\cM}{\mathcal{M}}
\newcommand{\cN}{\mathcal{N}}
\newcommand{\cO}{\mathcal{O}}
\newcommand{\cP}{\mathcal{P}}
\newcommand{\cQ}{\mathcal{Q}}
\newcommand{\cR}{\mathcal{R}}
\newcommand{\cS}{\mathcal{S}}
\newcommand{\cT}{\mathcal{T}}
\newcommand{\cU}{\mathcal{U}}
\newcommand{\cV}{\mathcal{V}}
\newcommand{\cW}{\mathcal{W}}
\newcommand{\cX}{\mathcal{X}}
\newcommand{\cY}{\mathcal{Y}}
\newcommand{\cZ}{\mathcal{Z}}

% Bold
\newcommand{\bA}{\mathbf{A}}
\newcommand{\bB}{\mathbf{B}}
\newcommand{\bC}{\mathbf{C}}
\newcommand{\bD}{\mathbf{D}}
\newcommand{\bE}{\mathbf{E}}
\newcommand{\bF}{\mathbf{F}}
\newcommand{\bG}{\mathbf{G}}
\newcommand{\bH}{\mathbf{H}}
\newcommand{\bI}{\mathbf{I}}
\newcommand{\bJ}{\mathbf{J}}
\newcommand{\bK}{\mathbf{K}}
\newcommand{\bL}{\mathbf{L}}
\newcommand{\bM}{\mathbf{M}}
\newcommand{\bN}{\mathbf{N}}
\newcommand{\bO}{\mathbf{O}}
\newcommand{\bP}{\mathbf{P}}
\newcommand{\bQ}{\mathbf{Q}}
\newcommand{\bR}{\mathbf{R}}
\newcommand{\bS}{\mathbf{S}}
\newcommand{\bT}{\mathbf{T}}
\newcommand{\bU}{\mathbf{U}}
\newcommand{\bV}{\mathbf{V}}
\newcommand{\bW}{\mathbf{W}}
\newcommand{\bX}{\mathbf{X}}
\newcommand{\bY}{\mathbf{Y}}
\newcommand{\bZ}{\mathbf{Z}}

\newcommand{\ba}{\mathbf{a}}
\newcommand{\bb}{\mathbf{b}}
\newcommand{\bc}{\mathbf{c}}
\newcommand{\bd}{\mathbf{d}}
\newcommand{\be}{\mathbf{e}}
\renewcommand{\bf}{\mathbf{f}}
\newcommand{\bg}{\mathbf{g}}
\newcommand{\bh}{\mathbf{h}}
\newcommand{\bi}{\mathbf{i}}
\newcommand{\bj}{\mathbf{j}}
\newcommand{\bk}{\mathbf{k}}
\newcommand{\bl}{\mathbf{l}}
\newcommand{\bm}{\mathbf{m}}
\newcommand{\bn}{\mathbf{n}}
\newcommand{\bo}{\mathbf{o}}
\newcommand{\bp}{\mathbf{p}}
\newcommand{\bq}{\mathbf{q}}
\newcommand{\br}{\mathbf{r}}
\newcommand{\bs}{\mathbf{s}}
\newcommand{\bt}{\mathbf{t}}
\newcommand{\bu}{\mathbf{u}}
\newcommand{\bv}{\mathbf{v}}
\newcommand{\bw}{\mathbf{w}}
\newcommand{\bx}{\mathbf{x}}
\newcommand{\by}{\mathbf{y}}
\newcommand{\bz}{\mathbf{z}}

\newcommand{\0}{\mathbf{0}}
\newcommand{\1}{\mathbf{1}}

% bb letters
\newcommand{\bbA}{\mathbb{A}}
\newcommand{\bbB}{\mathbb{B}}
\newcommand{\bbC}{\mathbb{C}}
\newcommand{\bbD}{\mathbb{D}}
\newcommand{\bbE}{\mathbb{E}}
\newcommand{\bbF}{\mathbb{F}}
\newcommand{\bbG}{\mathbb{G}}
\newcommand{\bbH}{\mathbb{H}}
\newcommand{\bbI}{\mathbb{I}}
\newcommand{\bbJ}{\mathbb{J}}
\newcommand{\bbK}{\mathbb{K}}
\newcommand{\bbL}{\mathbb{L}}
\newcommand{\bbM}{\mathbb{M}}
\newcommand{\bbN}{\mathbb{N}}
\newcommand{\bbO}{\mathbb{O}}
\newcommand{\bbP}{\mathbb{P}}
\newcommand{\bbQ}{\mathbb{Q}}
\newcommand{\bbR}{\mathbb{R}}
\newcommand{\bbS}{\mathbb{S}}
\newcommand{\bbT}{\mathbb{T}}
\newcommand{\bbU}{\mathbb{U}}
\newcommand{\bbV}{\mathbb{V}}
\newcommand{\bbW}{\mathbb{W}}
\newcommand{\bbX}{\mathbb{X}}
\newcommand{\bbY}{\mathbb{Y}}
\newcommand{\bbZ}{\mathbb{Z}}

% sf letters
\newcommand{\sfA}{\mathsf{A}}
\newcommand{\sfB}{\mathsf{B}}
\newcommand{\sfC}{\mathsf{C}}
\newcommand{\sfD}{\mathsf{D}}
\newcommand{\sfE}{\mathsf{E}}
\newcommand{\sfF}{\mathsf{F}}
\newcommand{\sfG}{\mathsf{G}}
\newcommand{\sfH}{\mathsf{H}}
\newcommand{\sfI}{\mathsf{I}}
\newcommand{\sfJ}{\mathsf{J}}
\newcommand{\sfK}{\mathsf{K}}
\newcommand{\sfL}{\mathsf{L}}
\newcommand{\sfM}{\mathsf{M}}
\newcommand{\sfN}{\mathsf{N}}
\newcommand{\sfO}{\mathsf{O}}
\newcommand{\sfP}{\mathsf{P}}
\newcommand{\sfQ}{\mathsf{Q}}
\newcommand{\sfR}{\mathsf{R}}
\newcommand{\sfS}{\mathsf{S}}
\newcommand{\sfT}{\mathsf{T}}
\newcommand{\sfU}{\mathsf{U}}
\newcommand{\sfV}{\mathsf{V}}
\newcommand{\sfW}{\mathsf{W}}
\newcommand{\sfX}{\mathsf{X}}
\newcommand{\sfY}{\mathsf{Y}}
\newcommand{\sfZ}{\mathsf{Z}}

\newcommand{\sfa}{\mathsf{a}}
%\renewcommand{\sfb}{\mathsf{b}}
\newcommand{\sfc}{\mathsf{c}}
\newcommand{\sfd}{\mathsf{d}}
\newcommand{\sfe}{\mathsf{e}}
\newcommand{\sff}{\mathsf{f}}
\newcommand{\sfg}{\mathsf{g}}
\newcommand{\sfh}{\mathsf{h}}
\newcommand{\sfi}{\mathsf{i}}
\newcommand{\sfj}{\mathsf{j}}
\newcommand{\sfk}{\mathsf{k}}
\newcommand{\sfl}{\mathsf{l}}
\newcommand{\sfm}{\mathsf{m}}
\newcommand{\sfn}{\mathsf{n}}
\newcommand{\sfo}{\mathsf{o}}
\newcommand{\sfp}{\mathsf{p}}
\newcommand{\sfq}{\mathsf{q}}
\newcommand{\sfr}{\mathsf{r}}
\newcommand{\sfs}{\mathsf{s}}
\newcommand{\sft}{\mathsf{t}}
\newcommand{\sfu}{\mathsf{u}}
\newcommand{\sfv}{\mathsf{v}}
\newcommand{\sfw}{\mathsf{w}}
\newcommand{\sfx}{\mathsf{x}}
\newcommand{\sfy}{\mathsf{y}}
\newcommand{\sfz}{\mathsf{z}}

% Gothic letters

\newcommand{\fD}{\mathfrak{D}}

% Fields

\newcommand{\F}{\mathbb{F}}
\newcommand{\Fq}{\mathbb{F}_q}
\newcommand{\Fqm}{\mathbb{F}_{q^m}}
\newcommand{\FqM}{\mathbb{F}_{q^M}}
\newcommand{\Fqml}{\mathbb{F}_{q^{m\ell}}}
\newcommand{\Fqmu}{\mathbb{F}_{q^{m\mu}}}
\newcommand{\Lfield}{\ensuremath{\mathbb{L}}}
\newcommand{\GF}[2]{\ensuremath{\mathbb{#1}_{#2}}}

% Math Operators
\DeclareMathOperator{\diag}{diag}
\DeclareMathOperator{\wt}{wt}
\DeclareMathOperator{\supp}{supp}
\DeclareMathOperator{\colsupp}{colsupp}
\DeclareMathOperator{\rank}{rank}
\DeclareMathOperator{\short}{short}
\DeclareMathOperator{\punct}{punct}
\DeclareMathOperator{\extsmallfield}{ext}
\DeclareMathOperator{\Tr}{Tr}
\DeclareMathOperator{\Gr}{Gr}
\DeclareMathOperator{\ev}{ev}
\DeclareMathOperator{\card}{card}
\DeclareMathOperator{\order}{order}
\DeclareMathOperator*{\argmax}{arg\,max}
\DeclareMathOperator*{\argmin}{arg\,min}
\DeclareMathOperator{\proj}{Proj}
\DeclareMathOperator{\poly}{poly}


\DeclareMathOperator{\dHam}{d_{\sfH}}
\newcommand{\dt}[2]{\dHam(#1 , #2)}

\newcommand{\myspan}[1]{\left\langle #1 \right\rangle}
\newcommand{\myspanBig}[1]{\left\langle #1 \right\rangle}
\newcommand{\myspanRow}[1]{\left\langle #1 \right\rangle_{\mathsf{row}}}
\newcommand{\myspanCol}[1]{\left\langle #1 \right\rangle_{\mathsf{col}}}

\newcommand{\ceil}[1]{\ensuremath{\left\lceil #1 \right\rceil}}
\newcommand{\floor}[1]{\ensuremath{\left\lfloor #1 \right\rfloor}}

\newcommand*\diff{\mathop{}\!\mathrm{d}}

\newcommand{\abstna}[6]{#1}
\newcommand{\abstnk}[6]{#2}
\newcommand{\abstti}[6]{#3}
\newcommand{\abstor}[6]{#4}
\newcommand{\abstad}[6]{#5}
\newcommand{\abstab}[6]{#6}

\newcommand{\gm}[1]{{\color{olive}[gm: #1]}}

\def\NN{{\mathbb N}}
\def\QQ{{\mathbb Q}}
\def\RR{{\mathbb R}}
\def\ZZ{{\mathbb Z}}

\tikzset{%
	dots/.style args={#1per #2}{%
		line cap=round,
		dash pattern=on 0 off #2/#1
	}
}
\definecolor{light-gray}{gray}{0.7}
\definecolor{dark-gray}{gray}{0.3}
\usetikzlibrary{spy}
\usetikzlibrary{backgrounds}
\usetikzlibrary{decorations}
\usetikzlibrary{patterns}
\usetikzlibrary{arrows,matrix,positioning,calc}

%\newtheorem*{theorem*}{Theorem}
%\newtheorem{theorem}{Theorem}
%\newtheorem{proposition}{Proposition}
%\newtheorem{corollary}{Corollary}
%\newtheorem{lemma}[theorem]{Lemma}
%\newtheorem{conjecture}{Conjecture}
%\newtheorem{example}{Example}

%\newcommand{\remove}[1]{}

%\theoremstyle{definition}
%\newtheorem{definition}{Definition}
%
%\theoremstyle{remark}
%\newtheorem{remark}{Remark}


% \commentfalse % Uncomment to not show any comments
\newcommand{\cross}{crossing}

\newcommand{\C}{\mathcal{C}}
\newcommand{\x}{\boldsymbol{x}}
\renewcommand{\c}{\boldsymbol{c}}
\newcommand{\e}{\boldsymbol{e}}
\renewcommand{\v}{\boldsymbol{v}}
\newcommand{\z}{\boldsymbol{z}}
\newcommand{\w}{\boldsymbol{w}}
\newcommand{\y}{\boldsymbol{y}}
\newcommand{\E}{\mathfrak{E}}
\renewcommand{\epsilon}{\varepsilon}
\definecolor{darkgray}{RGB}{64,64,64}
\definecolor{litegray}{RGB}{192,192,192}
\tikzstyle{block}=[draw, rectangle, minimum height=1cm, text width=1.5cm, text centered, draw=darkgray, font=\small]
\tikzstyle{block_medium}=[draw, rectangle, minimum height=1.5cm, text width=2cm, text centered, draw=darkgray, font=\small]
\tikzstyle{block_large}=[draw, rectangle, minimum height=1.75cm, text width=3cm, text centered, draw=darkgray, font=\small]
\tikzstyle{line} = [draw, -latex]

\usepackage{bm}
%\usepackage{pifont}
%\usepackage[nofillcomment]{algorithm2e}
%%\RestyleAlgo{ruled}
%\usepackage{amsfonts}
%\usepackage{bbm}
%\usepackage{mathtools}
%\usepackage{setspace}
%\usepackage{multicol}
%\newcommand{\repeatthanks}{\textsuperscript{\thefootnote}}
%\newcommand{\at}{\makeatletter @\makeatother}
%\usepackage{float}

%\newcommand{\repeatthanks}{\textsuperscript{\thefootnote}}
%\newcommand{\at}{\makeatletter @\makeatother}
%\usepackage{float}
%\usepackage{cite}

\definecolor{mycolor1}{rgb}{0.00000,0.44700,0.74100}
\definecolor{mycolor2}{rgb}{0.85000,0.32500,0.09800}
\definecolor{mycolor3}{rgb}{0.4660, 0.6740, 0.1880}

\def\qed{\hfill $\blacksquare$}
\usepackage{color}

\newcommand{\RQ}{\mathcal{R}_q}
\newcommand{\alsize}{Q}
\newcommand{\coefffailprob}{\textup{Pr}(\mathcal{E})}
\newcommand{\decfailprob}{\text{DFR}}
\newcommand{\compnoisev}{c_{N_v}}
\newcommand{\compnoiseu}{c_{N_u}}
\newcommand{\compnoise}{\bm{change}}
\newcommand{\encmap}{\textsf{Encode/Map}}
\newcommand{\demapdec}{\textsf{Demap/Decode}}
%\DeclareMathOperator{\supp}{supp}
\newcommand{\decomp}{\textsf{decomp}_{q}}
\newcommand{\comp}{\textsf{comp}_{q}}

\newcommand\inner[2]{\langle #1, #2 \rangle}

%\usetikzlibrary{external}
%\tikzexternalize[prefix=./]
%\tikzexternalize

\usepackage{stackengine} 
\newcommand\oast{\stackMath\mathbin{\stackinset{c}{0ex}{c}{0ex}{\ast}{\bigcirc}}}
\newcommand\mygeqa{\stackrel{\mathclap{(a)}}{\geq}}
\newcommand\mygeqb{\stackrel{\mathclap{(b)}}{\geq}}

%\newcommand{\remove}[1]{}

\newcommand{\myalgorithm}[1]{%
	\begingroup
	\removelatexerror
	\begin{algorithm*}[H]
		#1
	\end{algorithm*}
	\endgroup}

\newcommand{\p}{\bm{p}}

\newcommand{\N}{{\mathbb N}}
\newcommand{\R}{{\mathbb R}}
\newcommand{\M}{{\mathcal M}}
\newcommand{\X}{{\mathcal X}}
\newcommand{\Y}{{\mathcal Y}}
\newcommand{\va}{{\textbf{\textit{a}}}}
\newcommand{\vb}{{\textbf{\textit{b}}}}
\newcommand{\vj}{{\textbf{\textit{j}}}}
\newcommand{\vx}{{\textbf{\textit{x}}}}

\renewcommand{\epsilon}{\varepsilon}
\newcommand{\al}{\alpha}
\newcommand{\ta}{\theta}
\newcommand{\la}{\lambda}
\newcommand{\A}{\mathcal{A}}
\newcommand{\pr}[1]{\operatorname{Pr}\left({#1}\right)}
\newcommand{\cpr}[2]{\operatorname{Pr}\left({#1}\mid{#2}\right)}
\newcommand{\expect}[1]{\operatorname{E}{#1}}

\tikzstyle{line} = [draw, -latex]
\newcommand{\abs}[1]{\left\lvert {#1}\right\rvert}
\newcommand{\dset}[2]{\left\{#1 : #2\right\}}
\newcommand{\sset}[1]{\left\{#1\right\}}
\newcommand{\from}{\colon}
\usepackage{mathtools}
\newcommand{\defeq}{\vcentcolon=}
\newcommand{\eqdef}{=\vcentcolon}

\def\qed{\hfill $\blacksquare$}

%\tikzstyle{block}=[draw, rectangle, minimum height=1cm, text width=1.5cm, text centered, draw=darkgray, font=\small]
\tikzstyle{block_medium}=[draw, rectangle, minimum height=1.5cm, text width=2cm, text centered, draw=darkgray, font=\small]
\tikzstyle{block_large}=[draw, rectangle, minimum height=2cm, text width=2cm, text centered, draw=darkgray, font=\small]

\usepackage[ruled,linesnumbered]{algorithm2e}
\usetikzlibrary{shapes.geometric,arrows,positioning}
\tikzstyle{decision} = [diamond, draw, fill=blue!20, 
text width=5em, text badly centered, node distance=4cm, inner sep=0pt]
\tikzstyle{block} = [rectangle, draw, fill=blue!20,  text centered, rounded corners, minimum height=4em]
\tikzstyle{line} = [draw, -latex']
\tikzstyle{cloud} = [draw, ellipse,fill=red!20, node distance=6.6cm,
minimum height=1em]
\tikzstyle{algorithm} = [rectangle, draw, fill=green!20,  text centered, rounded corners, minimum height=4em, minimum width =6em]
\tikzstyle{initialization} = [rectangle, draw,   text centered, minimum height=4em, minimum width =6em]

\usetikzlibrary{spy}
\usetikzlibrary{backgrounds}
\usetikzlibrary{decorations}
\usetikzlibrary{patterns}
\usetikzlibrary{arrows,matrix,positioning,calc}

%\DeclareMathOperator{\supp}{supp}
\DeclareMathOperator*{\Var}{Var}
\DeclareMathOperator{\erfcx_1}{erfcx_1}
\DeclareMathOperator{\erfc}{erfc}

\newcommand*\fooA{\mathrel{-\mkern-3mu{\circ}\mkern-3mu-}}

\usepackage{appendix}
\AtBeginEnvironment{subappendices}{%
	\chapter*{Appendix}
	\addcontentsline{toc}{chapter}{Appendices}
	\counterwithin{figure}{section}
	\counterwithin{table}{section}
}