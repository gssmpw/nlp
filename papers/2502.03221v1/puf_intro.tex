\section{Introduction}

Physical Unclonable Functions (PUFs) evaluate physical properties of devices to obtain unique identifiers of electronic devices and provide physical roots of trust for cryptographic keys. Furthermore, PUFs can serve as a foundation for tamper protection technology that facilitates to validate the physical integrity of an embedded system after its power-up. All approaches have in common that minuscule manufacturing variations within physical objects, or mostly electronic components, are evaluated to generate an internal device-unique output. While there are several works on the assessment of the entropy or randomness of PUFs \cite{maiti2013systematic,wilde2018spatial,frisch2023practical} and the leakage through the published helper data necessary within the reconstruction phase, e.g. \cite{DGSV15,DGV+16}, we are currently lacking a theoretical model to quantify the security in the light of a physical attacker who destroys parts of the PUF response to read out the remainder.

Going from silicon PUFs, e.g. SRAM, ring oscillator or arbiter PUF \cite{HYKD14}, to system-level PUFs facilitates to incorporate tamper protection capabilities to protect an entire embedded device with components that cannot resist advanced physical attacks on their own, such as processors, FPGAs, or external memories and their communication, or discrete components that are susceptible e.g. to side-channel attacks. This reduces the attack surface from several individual vulnerabilities, e.g., against laser or EM fault injection, EM or optical side channel analysis, and analysis of digital communication interfaces between components to the attack resistance of the surrounding barrier. We also consider it infeasible to perform power side-channels since the attacker can only access the power supply of the entire printed circuit board and has no direct access to the power supply of the chip or individual discrete components.

For the remainder of this work, we will refer to the PUF-based tamper protection foil proposed by Immler \emph{et al.} \cite{IOK+18,ION+19}. However, the generic results of this work can be adapted and applied to other PUF types as well. Examples are the coating PUF \cite{tuyls2006read} and the polymer waveguide PUF \cite{vai2016secure,geis2017}. The tamper protection is based on a foil that is wrapped around an entire Printed Circuit Board (PCB), or a cover that is placed on top and bottom. This foil consists of a mesh of electrodes, leading to a large number of capacitances that can be measured by a mixed-signal circuit from within the protected area. It evaluates the capacitive coupling between electrodes to derive the cryptographic key and to validate the physical integrity of the system, and performs run-time tamper detection during operation to protect the system.

One of the critical attack vectors, considered during the evaluation of hardware devices with security boxes  \cite{jilHDSB}, is that an attacker penetrates the foil with a small needle or drill and accesses internal signals. If the required drill diameter is sufficiently large, major changes occur in the capacitance measurements  of a significant portion of the foil, leading to an incorrectly reconstructed PUF response during the reconstruction phase. Therefore, the secret cannot be uncovered by an attacker, as discussed, e.g., in \cite{GXKF22}. As the foil's PUF values may also change over time due to noise, aging, and varying environmental conditions such as temperature or humidity, an error-correcting code is implemented in the system to compensate for those effects to ensure that the correct cryptographic key is derived with a probability $> 1-10^{-6}$ or even $>1-10^{-9}$ so that the PUF does not have significant impact on the reliability of the overall system.

Our goal is to analyze the resulting wiretap channel \cite{wyner1975wire,csiszar1978broadcast} between the enrollment and reconstruction phase of the legitimate user as well as the reconstruction phase of the attacker from an information theoretical point of view. We establish lower bounds on the secrecy capacities of the resulting channels as well as finite blocklength achievability and converse bounds on the maximal achievable secrecy rate, making our results relevant in practice as they provide benchmarks for implementations by quantifying the distance of a practical implementation to the theoretical limit.

\subsection{Related Works}
For a survey on information and coding theoretic techniques covering enrollment and reconstruction without tamper protection see \cite{gunlu2020optimality}. We also mention literature in the context of biometric secrecy as this field is closely related to PUFs. \cite{gunlu2019code} for example code constructions for both biometric secrecy systems as well as PUFs are given. Achievable rate regions of biometric secrecy systems under security and privacy constraints are presented in \cite{ignatenko2009biometric}. Approaches to achieve biometric secrecy using Slepian-Wolf distributed source coding techniques are presented in \cite{vetro2009securing,draper2007using}.

\subsection{Main Results}

The main results presented in this work are:

\begin{itemize}
	\item Information theoretical channel model including zero leakage helper data generation for physical tampering with PUFs
	\item Asymptotic results for lower bounds on the channel capacity of the resulting PUF-channel under different attack scenarios
	\item Finite blocklength achievability and converse results on the number of required capacitances to achieve a predefined security level in two attack scenarios
	\item Proof that previously used helper data schemes do not achieve required security levels without leaking information about the secret via the helper data
	\item Quantitative results that demonstrate that a $128$ bit security level is achievable with $1400$ PUF cells for $18\%$ and $36\%$ erasure probability for digital and analog attacker, respectively
        \item Proof that an existing converse bound on finite blocklength wiretap codes cannot be tight for all channels
\end{itemize}

\subsection{Outline}

Section~\ref{sec:sota} gives a brief overview over related work. Section~\ref{sec:preliminaries} introduces the notation used throughout this work and recaps known results in the field of information theory, in particular for finite blocklength that are used to proof the main results presented in this work. Furthermore, helper data algorithms with an emphasis on zero leakage helper data are recapped and their connection between secret sharing using common randomness is examined. Section~\ref{sec:channel_model} gives some background information on the foil PUF and introduces the resulting channel model. In Section~\ref{sec:securing_foil}, we obtain results on the secret key capacity of the HDA for digital and analog attacker. Section~\ref{sec:secret_sharing_oneway} investigates secret sharing using common randomness using one-way communication for finite blocklength. It serves as a foundation to analyze the HDA performance with respect to the required amount of capacitive PUF cells for the foil PUF presented in Section~\ref{sec:finite}. In Section~\ref{sec:converse_application} we use the converse result on the necessary amount of PUF cells to show that a given implementation has to either leak via the helper data or be insecure by other means. Section~\ref{sec:conclusion_pufs} sums up the results and states open problems.
