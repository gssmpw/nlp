\section{Literature Review}
Sandhya et al.~\cite{mishra_2020_smishing} suggested a smishing detection approach using the Naive Bayes algorithm through analysis of SMS content, while Ankit Sandhya et al.~\cite{jain_2022_a} suggested employing KNN, ETC and RF based on URL behavior analysis. 

Rubaiath et al.~\cite{rubaiatheulfath_2021_detecting} implemented a technique for identifying phishing attempts in SMS messages using NLP-based extraction of features and selection, achieving a remarkable 98.39\% accuracy by primarily targeting SMS containing phishing URLs. The utilization of a dataset comprising both legitimate and smishing messages, employing SVM and RF algorithms. The methodology and machine learning classifiers used in previous studies \cite{balim_2019_automatic, mdfarhanshahriyar_2024_phishing} have achieved a satisfactory level of accuracy in detecting phishing and SMiShing frauds. This shows their effectiveness in providing reliable solutions to protect victims from financial and psychological threats \cite{boukari_2021_machine}.

Conventional feature extraction techniques and machine learning algorithms frequently fall short of effectively capturing the nuanced features present in complex and unstructured SMS texts, which hampers phishing SMS detection. To address this limitation, integrating advanced deep learning techniques, such as CNN\cite{ramilyarullin_2021_bert,nizojamanshohan_2024_enhancing}, can significantly improve identifying phishing attempts in text messages. Followed by Phishing Detection framework merges a pre-trained transformer model, MPNet, using Bi-directional Gated Recurrent Units (GRU) and supervised ConvNets (CNN) is designed to detect complex patterns in unstructured short phishing text messages effectively by Rubaiath et al.~\cite{rubaiatheulfath_2021_hybrid}.

The "SpotSpam" study~\cite{oswald_2022_spotspam} discovered SMS spam detection by using BERT embeddings and intention analysis, addressing the limitations of static textual methods. This approach achieves 98.07\% accuracy, improving adaptability and stability in dynamic keyword environments.

H. Nam ~\cite{nguyen_2023_transformerbased} suggested a novel multi-class intrusion detection system (IDS) for the in-vehicle Controller Area Network (CAN) bus using an attention network built on transformers (TAN). The major finding of this study is that transformers could enhance intrusion detection by transferring knowledge from a source model to a target model.

This study~\cite{sultanzavrak_2023_email} introduced an email spam detection method using a blend of CNNs, and GRUs, with attention mechanisms. The technique enhances feature extraction through hierarchical convolution layers and improves detection accuracy by integrating attention mechanisms at both sentence and word levels. The cross-dataset evaluation shows that the proposed FT+HAN architecture, which integrates CNNs with attention mechanisms, surpasses existing methods as attention layers with unique text processing efficiency~\cite{vaswani_2017_attention}.