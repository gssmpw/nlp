\section{Solvable Instances} 
\label{sec: solvable}
In Sections~\ref{sec: upper bound} and~\ref{sec: lower bound unquantized}, we provided nearly matching upper and lower bounds for instances with positive gap.
In this section, we study the ``(un)solvabality'' of bandit instances with zero gap, and show that essentially all bandit instances that are ``solvable'' have positive gap, as long as parameter~$c$ is large enough (see Remark~\ref{rem: picking large enough c}).
To formalize this idea, we define the following class of bandit instances.



\begin{definition}[Solvable instances]
\label{def:solvable}
     Let $\A, \epsilon,$ and $q$ be fixed. 
     We say that an instance $\nu \in \cE$ is $\epsilon$-\emph{solvable} if for each $\delta \in (0, 1)$, there exists an algorithm that is $(\epsilon, \delta)$-reliable and it holds under instance $\nu$ that\footnote{We could require that $\PP_{\nu}[\tau < \infty] = 1$ in this case and the subsequent analysis and conclusions would be essentially unchanged.  Recall also that $\PP_{\nu}[\cdot]$ denotes probability under instance $\nu$.}
    \begin{equation}
        \PP_{\nu}[\tau < \infty \cap \hat{k}\in\Ac_{\epsilon}]\ge1-\delta.
    \end{equation}
    If no such algorithm exists, we say that $\nu$ is $\epsilon$-\emph{unsolvable}.
\end{definition}

\begin{remark}
\label{rem: solvable inclusion}
    Fix $0 < \epsilon_1 \le \epsilon_2$. If an instance $\nu$ is
    $\epsilon_1$-solvable, then it is $\epsilon_2$-solvable.
    This follows directly from $\A_{\epsilon_1}(\nu) \subseteq \A_{\epsilon_2}(\nu)$.
\end{remark}

From Corollary~\ref{cor: combined guarantee}, we deduce that any instance with a positive gap is solvable. 
 
\begin{corollary}[Positive gap is solvable]
\label{cor: positive gap is solvable}
    Let $\A, \lambda, \epsilon, q,$ and $c$ be fixed. 
    Suppose an instance $\nu$ satisfies
    $ \Delta > 0$, where $\Delta =  \Delta(\nu, \lambda, \epsilon, c, q) $ is as defined in Theorem~\ref{theorem: upper bound}.
    Then $\nu$ is $\epsilon$-solvable. 
\end{corollary}
The main result of this section is that the reverse inclusion nearly holds, in the following sense.
 \begin{theorem}[Zero gap is unsolvable]
 \label{thm: zero gap is unsolvable}
    Let $\lambda, \epsilon, c,$ and $q$ be fixed, 
    and let $\tilde{\epsilon} = \tilde{\epsilon}(\lambda, \epsilon, c)$ be as defined in 
    Algorithm~\ref{alg: main}.
    Suppose an instance $\nu \in \cE$ satisfies $\Delta(\nu, \lambda, \epsilon, c, q) = 0$.
    If we assume for $\nu$ that there exists some sufficiently small $\eta_0 > 0$ such that 
    $0 \le Q_k^+(q-\eta_0) \le Q_k(q+\eta_0) \le \lambda$, then $\nu$ is $c\tilde{\epsilon}$-unsolvable.
 \end{theorem}
\begin{proof}
    See Appendix~\ref{sec: appendix solvable instance}.
\end{proof}


\begin{remark}[Removing the additional assumption]
\label{rem: remove additional assumption}
    The additional assumption involving $\eta_0$ 
    % in Theorem~\ref{thm: zero gap is unsolvable} and Corollary~\ref{cor: zero gap is unsolvable} 
    is mild; it is trivially satisfied by instances with all reward distributions supported on $[0, \lambda]$, and also holds significantly more generally since $\eta_0$ can be arbitrarily small.
    % The additional assumption is added to simplify the proof of Lemma~\ref{lem:two_instances}. Specifically, it helps ensuring both $(G_k)^{-1}(q)$ and $(G_a)^{-1}(q)$ as defined in \eqref{eq: shifted q quantiles} are in $[0, \lambda]$ so that the constructed instance $\nu'$ satisfies $\nu' \in \cE$.
    Moreover, in Appendix~\ref{sec: assumption removal}, we show that
    this assumption is unnecessary if we use the modified gap (see Remark~\ref{rem: further improvement}) instead of $\Delta$.
    % and Definition~\ref{def: modified gap}) in Theorem~\ref{thm: zero gap is unsolvable}: if the modified gap is zero for an instance~$\nu$, then $\nu$
    % is $c\tilde{\epsilon}$-unsolvable.
\end{remark}


 
\begin{remark}
\label{rem: picking large enough c}
 For each $\theta \in (0, 1)$, picking $c = \lceil 2\theta / (1-\theta)\rceil$ yields
\begin{equation}
       \nu \text{ is } \theta\epsilon\text{-solvable} 
    \implies
    \nu \text{ is } c\tilde{\epsilon}\text{-solvable}
    \implies
     \Delta(\nu, \lambda, \epsilon, c, q) > 0
     \implies 
     \nu \text{ is } \epsilon\text{-solvable},
\end{equation}
where the last two implications follow from Theorem~\ref{thm: zero gap is unsolvable} and Corollary~\ref{cor: positive gap is solvable}, and the first implication follows from Remark~\ref{rem: solvable inclusion} and the following inequality:
\begin{equation}
    c \tilde{\epsilon} 
    = 
    \frac{c \lambda}{ \left\lceil (c+1) \lambda/\epsilon \right\rceil}
    \ge
     % \frac{c \lambda}{  (c+1) \lambda/\epsilon  + \lambda/\epsilon }
     % =
     \frac{c \lambda}{  (c+2) \lambda / \epsilon }
     = 
     \left(1 - \frac{2}{c+2} \right) \epsilon
     \ge
     \left(1 - \frac{2}{\frac{2\theta+2-2\theta}{1-\theta}} \right) \epsilon
     = \theta \epsilon.
\end{equation} 
Since $\theta$ can be arbitrarily close to $1$,
% $\lim\limits_{c \to \infty} c \tilde{\epsilon} = \epsilon$, 
we have $\Delta(\nu, \lambda, \epsilon, c, q) > 0$
    % we can use Algorithm~\ref{alg: main} to solve 
     for essentially all $\epsilon$-solvable instances by picking a sufficiently large $c$.
 \end{remark}

The proof of Theorem~\ref{thm: zero gap is unsolvable} will turn out to directly extend to a ``limiting'' version in which we replace $c\tilde{\epsilon}$ by $\lim\limits_{c \to \infty} c\tilde{\epsilon} = \epsilon$ and $\Delta(\nu, \lambda, \epsilon, c, q)$ by $\lim\limits_{c \to \infty} \Delta(\nu, \lambda, \epsilon, c, q)$, giving the following corollary.


 \begin{corollary}
 \label{cor: zero gap is unsolvable}
    Let $\lambda, \epsilon$, and $q$ be fixed.
    Let $\Delta_{k}(\nu, \epsilon, q)$ be the gap defined in Definition~\ref{def: our gap} with $c \to \infty$ (see~\eqref{eq: gap k infinite c} for the explicit form).    
    Suppose an instance $\nu \in \cE$ satisfies $\Delta(\nu, \epsilon, q) = \max\limits_{k \in \A_{\epsilon(\nu)}} \Delta_k(\nu, \epsilon, q) = 0$.
    If we assume for $\nu$ that there exists some sufficiently small $\eta_0 > 0$ such that 
    $0 \le Q_k^+(q-\eta_0) \le Q_k(q+\eta_0) \le \lambda$, then $\nu$ is $\epsilon$-unsolvable.
 \end{corollary}
 \begin{proof}
    See Appendix~\ref{sec: appendix solvable instance}.
\end{proof}