\section{Details on Remark~\ref{rem: gap generalization} (Comparison to Existing Gap Definitions)}
\label{sec: appendix gap definition generalization}


We first recall some existing arm gap definitions for the exact quantile bandit problem (i.e., $\epsilon = 0$) in the setting of unquantized rewards.
In \cite[Definition 2]{nikolakakis2021quantile}, the authors defined the gap $\Delta_k^{\mathrm{NKSS}} $ for each suboptimal arm $k \ne k^*$ by
\begin{equation}
\label{eq: gap NKSS}
    \Delta_k^{\mathrm{NKSS}} \coloneqq
     \sup
    \{
         \Delta \in \left[0, \min(q, 1-q) \right]
        : 
        Q_k(q + \Delta)
        \le
        Q_{k^*}(q - \Delta)
    \}.
\end{equation}
While the authors did not define the arm gap for $k^*$, we can take it to be the same as the gap of the ``best'' suboptimal arm, as their algorithm terminates only when all suboptimal arms are eliminated.
On the other hand, the arm gap defined in
\cite[(Eq. (27)]{howard2022sequential}
is given by
\begin{equation}
\label{eq: gap HR}
    \Delta_k^{\mathrm{HR}} \coloneqq
    \begin{cases}
     \sup
    \{
         \Delta \in \left[0, \min(q, 1-q) \right]
        : 
        Q_k(q + \Delta)
        \le
        \max\limits_{a \in \A}
        Q_a(q - \Delta)
    \}
    & \text{if } k \ne k^*
    \\
     \sup
    \{
        \Delta 
        \in \left[0, q \right] :
        Q_k(q - \Delta)
        \ge
        \max\limits_{a \neq k}
        Q_{a}\big(q + \Delta_a^{\mathrm{HR}}
        \big)
    \}
    & \text{if } k = k^* 
    \end{cases}.
\end{equation}
Similar to our arm gap definition (Definition~\ref{def: our gap}), the gaps $\Delta_k^{\mathrm{HR}}$ for suboptimal arms $k \ne k^*$ are not defined based on the quantile function of $k^*$. It follows that $\Delta_a^{\mathrm{HR}} \ge \Delta_a^{\mathrm{NKSS}}$ for all arms $a \in \A$.

We now study the effect of taking $c \to \infty$ in our gap, which is given below in~\eqref{eq: gap k infinite c}. From~\eqref{eq: gap k infinite c}, it is straightforward to verify that~\eqref{eq: gap HR} is recovered from our gap (Definition~\ref{def: our gap}) by using only lower quantile functions and taking $S = \A$ and $c \to \infty$. 

\textbf{Effect of parameter $c$ in the gap definition.}
For any $1 \le c_1 \le c_2$, let 
\begin{equation}
\label{eq: tilde eps 1 and 2}
    \tilde{\epsilon_1} = \frac{\lambda} {\left\lceil (c_1+1) \lambda/\epsilon \right\rceil}
    \quad 
    \text{and}
    \quad
    \tilde{\epsilon_2} =  \frac{\lambda} {\left\lceil (c_2+1) \lambda/\epsilon \right\rceil}
\end{equation} 
be as defined 
using Lines~\ref{line: number of points}--\ref{line: tilde epsilon} of 
in Algorithm~\ref{alg: main}. It can readily be verified that 
\begin{equation}
\label{eq: c1 tilde eps 1 and 2}
    \tilde{\epsilon_1} \ge \tilde{\epsilon_2}
    \quad \text{and} \quad
    c_1 \tilde{\epsilon_1} \le c_2 \tilde{\epsilon_2} \le \epsilon
    \quad \text{and} \quad
    \Delta_{k}(\nu, \lambda, \epsilon, c_1, q)
\le \Delta_{k}(\nu, \lambda, \epsilon, c_2, q).
\end{equation}
Since $\lim\limits_{c \to \infty}  \tilde{\epsilon} = 0$
and $\lim\limits_{c \to \infty} c \tilde{\epsilon} = \epsilon$,
the gap as defined in Definition~\ref{def: our gap} converges to a quantity $\Delta_{k} \coloneqq \Delta_{k}(\nu, \epsilon, 
    q) = \lim\limits_{c \to \infty}
    \Delta_{k}(\nu, \lambda, \epsilon, c, q)$, given by
\begin{equation}
\begin{aligned}
\label{eq: gap k infinite c}
   \Delta_{k} =
    \begin{cases}
    \sup
    \left\{
        \Delta \in \left[0, \min(q, 1-q) \right]
        \colon
        Q_k(q + \Delta) 
        \le
        \max\limits_{a \in \A  }
        Q^+_{a}(q - \Delta) 
        \right\}
    &\hspace{-2mm} \text{if }  k \not\in \A_{\epsilon} \\
   \max\limits_{\A_{\epsilon} \subseteq S }
   \left\{
        \sup
    \Big\{
        \Delta \in 
       \Big[0, \min\limits_{a \not\in S} \Delta_{a}  \Big]
        % \ \middle\vert\
        \colon
        Q^+_k(q - \Delta) 
        \ge
        \max\limits_{ a \in S \setminus \{k\}} 
        Q_{a}(q + \Delta) - \epsilon
        \Big\}
        \right\}
    &\hspace{-2mm} \text{if } k \in \A_{\epsilon} 
    \end{cases}.
\end{aligned}
\end{equation}
    Note that $\Delta_k$ is independent of $c$ and $\lambda$.
    

\begin{remark}[Use of upper quantile function]
    \label{rem: upper quantile}
    To our knowledge, we are the first to incorporate upper quantile functions in the gap definition.
    This may lead to a potentially larger arm gap as compared to defining using only lower quantile functions (e.g., changing $Q_a^+(\cdot)$ and $Q_k^+(\cdot)$ in~\eqref{eq: our gap} and~\eqref{eq: Delta k^S} to $Q_a(\cdot)$ and $Q_k(\cdot)$ respectively), and hence a tighter upper bound.
\end{remark}

\begin{remark}[Dependency on $Q_{k^*}(q-\Delta)$]
    Existing papers using an elimination-based algorithm have their arm gaps defined according to $Q_{k^*}(q-\Delta)$; see~\eqref{eq: gap NKSS} for an example.
    In contrast, we remove this dependency and define using $\max\limits_{a \in \A  }
        Q^+_{a}(q - \Delta)$, which may lead to a tighter upper bound.  
    The resulting analysis required is more challenging --
    see the discussion in Remark~\ref{rem: elim suboptimal}. 
\end{remark}



Since our gap definitions generalizes existing gap definitions, we expect that their gaps being positive on an instance $\nu$ would imply our gap being positive on $\nu$. That is, their gaps being positive is a sufficient condition for Algorithm~\ref{alg: main} to return a satisfying arm with high-probability (see Corollary~\ref{cor: combined guarantee}).

\begin{proposition}
\label{prop: generalized formulation}
     Fix an instance $\nu \in \cE$ that has a unique arm $k^*$ with the highest $q$-quantile. Let $\Delta_a^{\mathrm{NKSS}}$ and 
$\Delta_a^{\mathrm{HR}}$ be as defined in~\eqref{eq: gap NKSS} and \eqref{eq: gap HR} for each $a \in \A$.
If $\min\limits_{a \in \A} 
\left\{ \Delta_a^{\mathrm{NKSS}} \right\} > 0$
 or
 $\min\limits_{a \in \A} 
\left\{ \Delta_a^{\mathrm{HR}} \right\} > 
 0$, then $\Delta =  \Delta(\nu, \lambda, \epsilon, c, q) $ as defined in Theorem~\ref{theorem: upper bound} is also positive.
\end{proposition}

\begin{proof}
    It suffices to consider the case
    $ \min\limits_{a \in \A}  
    \left\{ \Delta_a^{\mathrm{HR}} \right\} > 0$,
    since $\Delta_a^{\mathrm{HR}} \ge \Delta_a^{\mathrm{NKSS}}$ for all arms $a \in \A$.
    Let $\eta = \frac{1}{2} \min\limits_{a \in \A}  \Delta_a^{\mathrm{HR}} > 0$.
    Then we have
    \begin{equation}
    \label{eq: positive HR implies positive gap}
        Q_{k^*}^+(q - \eta)
        \ge 
         Q_{k^*}(q - \eta)
        \ge 
        \max\limits_{a \neq k}
        Q_{a}\big(q + \Delta_a^{\mathrm{HR}}\big)
        \ge
        \max\limits_{a \in \A \setminus \{k^*\} } Q_a(q + \eta) - c \tilde{\epsilon},
    \end{equation}
    where the second inequality follows from~\eqref{eq: gap HR} 
    and $\tilde{\epsilon} = \tilde{\epsilon}(\lambda, \epsilon, c)$ is as defined in Algorithm~\ref{alg: main}.
    Combining~\eqref{eq: positive HR implies positive gap} and~\eqref{eq: Delta k^S} of our gap definition,  we have
    \begin{equation}
        \max\limits_{a \in \A_{\epsilon}} \Delta_{a} 
        \ge 
        \Delta_{k^*}  = \max\limits_{\A_{\epsilon} \subseteq S \subseteq \A}
        \Delta_{k^*}^{(S)} 
        \ge
        \Delta_{k^*}^{(\A)} \ge \eta > 0
    \end{equation}
    as desired.
\end{proof}



We now show that the converse is not true in general. 
In other words, there exists an instance $\nu \in \cE$ where no algorithm can distinguish which arm has a higher quantile using a finite number of arm pulls (see \cite[Theorem 2]{nikolakakis2021quantile}), but Algorithm~\ref{alg: main} is capable of returning an $\epsilon$-satisfying arm with high probability.

\begin{proposition}
\label{prop: converse zero gap not true}
    Fix $\lambda \ge \epsilon > 0$ and $\delta \in (0, 0.5)$.
    There exists a two-arm bandit instance $\nu \in \cE$ that has a unique arm $k^*$ with the highest median such that
    $\Delta =  \Delta(\nu, \lambda, \epsilon, c, q) $ as defined in Theorem~\ref{theorem: upper bound} is positive for $c \ge 2$,
    but $\min\limits_{a \in \A} \big\{ \Delta_a^{\mathrm{NKSS}}  \big\} 
= \min\limits_{a \in \A} \big\{ \Delta_a^{\mathrm{HR}}  \big\} = 0$.
\end{proposition}

\begin{proof}
    Consider two arms $\A = \{1, 2\}$ with the following CDFs:
\begin{equation}
    F_1(x) = 
    \begin{cases}
        0  & \text{ for } x < 0
        \\
        \frac{x}{2m_1} & \text{ for } 0 \le x < 2m_1 \\
        1 & \text{ for } x \ge 2m_1
    \end{cases}
    \quad \text{and} \quad
    F_2(x) = 
    \begin{cases}
        0 & \text{ for } x < m_2 \\
        0.5 & \text{ for } m_2 \le x < 2m_1 \\
        1 & \text{ for } x \ge 2 m_1
    \end{cases},
\end{equation}
where $ m_2 \in (m_1 - \epsilon/2, m_1)$
such that both arms are $\epsilon$-optimal, with arm 1 being the unique best arm. 
Note that for each $\eta > 0$, we have
\begin{equation}
    Q_2(0.5 + \eta) = 2 m_1
    > m_1 = Q_1(0.5) \ge Q_1(0.5 - \eta),
\end{equation}
and so $\Delta_2^{\mathrm{NKSS}} = \Delta_2^{\mathrm{HR}} = 0$.
However, under our gap definition (Definition~\ref{def: our gap}) with $\A_{\epsilon}(\nu) = \{1, 2\} = \A$ and any $c \ge 2$, we have
\begin{align}
    \Delta \ge \Delta_2 
    \ge \Delta_{2}^{(\{1,2\})}
    &=
    \sup
    \left\{
        \Delta \in [0, 0.5]
        :
        Q^+_2(0.5 - \Delta) 
        \ge
        Q_{1}(0.5 + \Delta) - c\tilde{\epsilon}
        \right\} \\
    &\ge
    \sup
    \left\{
        \Delta  \in [0, 0.5]
        :
        Q^+_2(0.5 - \Delta) 
        \ge
        Q_{1}(0.5 + \Delta) - \frac{\epsilon}{2}
        \right\} \\    
    &=
    \sup
    \left\{
        \Delta  \in [0, 0.5]
        :
        m_2
        \ge
        (1+2 \Delta) m_1 - \frac{\epsilon}{2}
        \right\} \\
     &= \min\left\{0.5, \frac{m_2 - (m_1 - \epsilon/2)}{2m_1} \right\} >0,
\end{align}
where the second inequality follows from the calculation in Remark~\ref{rem: picking large enough c}, and the last inequality follows from the assumptions that $m_1 > 0$ and $m_2 > m_1 - \epsilon/2$.
\end{proof}


