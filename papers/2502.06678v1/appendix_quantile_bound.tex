\section{Proof of Lemma~\ref{lem: quantile anytime bound} (Anytime Quantile Bounds)}
\label{sec: appendix anytime quantile bounds}
We first present a useful auxiliary lemma.
\begin{lemma}
    Under the setup of Lemma~\ref{lem:  quantile anytime bound} (including Event $E$ from Lemma~\ref{lem: good events} holding), we have the following bounds:
    \begin{equation}
    \label{eq: xltk upper bound}
        x_{l_{t, k}} < Q_k(q)
    \end{equation}
    \begin{equation}
    \label{eq: xltk+1 lower bound}
        Q^+_k\big( q -  \Delta^{(t)} \big)
        \le x_{l_{t, k} + 1}
    \end{equation}
    \begin{equation}
    \label{eq: xutk upper bound}
        x_{u_{t, k}} < Q_k\big( q +  \Delta^{(t)} \big)
    \end{equation}
    \begin{equation}
    \label{eq: xutk+1 lower bound}
        Q^+_k(q)
        \le x_{u_{t, k} + 1}
    \end{equation}
    for each round $t \ge  1$ and arm $k \in \A_t$.
\end{lemma}
\begin{proof}
We will prove only~\eqref{eq: xltk upper bound} and~\eqref{eq: xltk+1 lower bound}
for an arbitrary $t  \ge 1$ and  $k \in \A_t$
in detail, as~\eqref{eq: xutk upper bound} and~\eqref{eq: xutk+1 lower bound} can be proved similarly.
Observe that, under event $E_{t,k,l} \subset E$ (see \eqref{eq: event Etkl}), we have 
\begin{equation}
\label{eq: Fk_xltk bound}
    F_k(x_{l_{t, k}}) < q
    \quad \text{and} \quad
    q -  \Delta^{(t)} < F_k(x_{l_{t, k}+1})
\end{equation}
respectively,
as otherwise the interval $[F_k(x_{l_{t, k}}), F_k(x_{l_{t, k}+1})]$
would fall on the right and the left, respectively, of the interval $\left( q - \Delta^{(t)}, q   \right)$. A similar argument through the event $E_{t,k,u} \subset E$ (see \eqref{eq: event Etku}) yields
\begin{equation}
\label{eq: Fk_xutk bound}
    F_k(x_{u_{t, k}}) < q + \Delta^{(t)}
    \quad \text{and} \quad
    q  < F_k(x_{u_{t, k}+1}).
\end{equation}

We now prove~\eqref{eq: xltk upper bound} using~\eqref{eq: Fk_xltk bound}; the inequality~\eqref{eq: xutk upper bound} can be proved similarly through~\eqref{eq: Fk_xutk bound}. If $x_{l_{t, k}} = - \infty$, then~\eqref{eq: xltk upper bound} holds trivially.
Therefore, we proceed on the assumption that $x_{l_{t, k}} \in \R$.
Then, using standard properties of quantile functions (see, e.g., \cite[4.3 Theorem]{dufour1995distribution}), we have $x_{l_{t, k}} < Q_k(q)$ as desired.

We now prove~\eqref{eq: xltk+1 lower bound} using~\eqref{eq: Fk_xltk bound}; the inequality~\eqref{eq: xutk+1 lower bound} can be proved similarly through~\eqref{eq: Fk_xutk bound}. 
If $x_{l_{t, k}+1} = \infty$, then~\eqref{eq: xltk+1 lower bound} holds trivially.
Therefore, we proceed on the assumption that $x_{l_{t, k}+1} \in \R$.
In this case, it is a finite upper bound on the values in the set
    $\{ z \in \R: F_k(z) \le q -  \Delta^{(t)}  \}$, and so this set has a finite supremum. It follows that 
    \begin{equation}
        x_{l_{t, k}+1} \ge 
        \sup \{ z \in \R : F_k(z) \le q -  \Delta^{(t)}  \} =
        Q^+_k\big( q -  \Delta^{(t)} \big)
    \end{equation}
    as desired.
\end{proof}


\begin{proof}[Proof of Lemma~\ref{lem:  quantile anytime bound}]
We break down the bounds into  inequalities as follows:
\begin{multicols}{2}
\begin{enumerate}[label=(\roman*)]

    \item  $\mathrm{LCB}_{\tau}(k) \le \mathrm{LCB}_t(k)$
    
    \item $\mathrm{LCB}_t(k)  < Q_k(q)$
    
    % \item $Q_k(q) \le  Q^+_k(q)$
    
    \item $Q_k(q) \le \mathrm{UCB}_t(k)$
    
    \item $\mathrm{UCB}_t(k)  \le \mathrm{UCB}_{\tau}(k)$
    
    \item $Q^+_k\big(q -  \Delta^{(t)} \big)  \le \mathrm{LCB}_t(k) + \tilde{\epsilon}$

    \item $\mathrm{UCB}_t(k) - \tilde{\epsilon}
        % \le x_{u_{t, k} } 
        <
         Q_k\big(q + \Delta^{(t)} \big)$
\end{enumerate}
\end{multicols}
We will prove only inequalities (i), (ii), and (iv)
for an arbitrary $t  > \tau \ge 0$ and  $k \in \A_t$
in detail, as all the other inequalities can be proved similarly.

Inequality (i) follows immediately from Line~\ref{LCB definition} of Algorithm~\ref{alg: main} and induction. Likewise, we can show (iv) using Line~\ref{UCB definition} of Algorithm~\ref{alg: main}.
 
We now show inequality (ii) by induction on $t$;
inequality (iii) can be proved similarly.
    For the base case $t = 1,$
    we have
    \begin{equation}
        \mathrm{LCB}_1(k) =
        \max \left( x_{l_{t, k}}, \mathrm{LCB}_{0}(k) \right) =
        \max \left( x_{l_{t, k}}, 0 \right) =
        x_{l_{t, k}} < Q_k(q),
    \end{equation}
    where the last inequality follows from~\eqref{eq: xltk upper bound}.   
    For the inductive step, suppose that $\mathrm{LCB}_t(k) < Q_k(q)$ for a fixed $t \ge 1$. Since $x_{l_{t, k}} < Q_k(q)$, we have
    \begin{equation}
        \mathrm{LCB}_{t+1}(k) =
            \max
            \left(
            x_{l_{t, k}},
            \mathrm{LCB}_{t}(k)
            \right)
            < Q_k(q)
    \end{equation}
    as desired.
    

    We now show inequality (v) using~\eqref{eq: xltk+1 lower bound}; inequality (vi)
         can be shown using a similar argument through~\eqref{eq: xutk upper bound}.
        We consider three cases for the index $l_{t, k}$:
        \begin{itemize}
            \item ($l_{t, k} = 0$) In this case, we have 
                    $x_{l_{t, k} + 1} = x_{1} =  0 = \mathrm{LCB}_0(k)$, and so
                    \begin{equation}
                       Q^+_k\big( q -  \Delta^{(t)} \big)
                        \le 
                        x_{l_{t, k} + 1} =
                        \mathrm{LCB}_0(k)
                        \le 
                        \mathrm{LCB}_t(k)
                        < \mathrm{LCB}_t(k) + \tilde{\epsilon},
                    \end{equation}
                    where the first inequality follows from~\eqref{eq: xltk+1 lower bound} and the second inequality follows from inequality~(i).
            
            \item ($1 \le l_{t, k} \le n$) In this case, we have
                    \begin{equation}
                   Q^+_k\big( q -  \Delta^{(t)} \big)
                    \le x_{l_{t, k} + 1}
                    = x_{l_{t, k}} + \tilde{\epsilon}
                    \le \mathrm{LCB}_t(k) + \tilde{\epsilon},
                \end{equation}
                where the first inequality follows from~\eqref{eq: xltk+1 lower bound}, the equality follows from distance between
                consecutive points set in Line~\ref{line: list of points} of Algorithm~\ref{alg: main}, 
                and the last inequality follows from Line~\ref{LCB definition} of Algorithm~\ref{alg: main}.

           \item  ($l_{t, k} = n+1$) In this case, we have  $x_{l_{t, k}} = x_{n+1} = \lambda \ge Q_k(q) \ge  Q^+_k\big( q -  \Delta^{(t)} \big)$,
           and so
                    \begin{equation}
                       Q^+_k\big( q -  \Delta^{(t)} \big) \le 
                        x_{l_{t, k}}  \le 
                        \mathrm{LCB}_t(k)
                        < \mathrm{LCB}_t(k) + \tilde{\epsilon},
                    \end{equation}
                    where the second inequality follows from Line~\ref{LCB definition} of Algorithm~\ref{alg: main}.
        \end{itemize}
    Combining all three cases, we have
    $Q^+_k\big( q -  \Delta^{(t)} \big)
        \le \mathrm{LCB}_t(k) + \tilde{\epsilon}$ as desired.
\end{proof}


