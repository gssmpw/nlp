\documentclass[final,12pt]{alt2025} % Anonymized submission
\input{common-header.sty}


\makeatletter
\newcommand\footnoteref[1]{\protected@xdef\@thefnmark{\ref{#1}}\@footnotemark}
\makeatother

\title[Quantile Multi-Armed Bandits with 1-bit Feedback]{Quantile Multi-Armed Bandits with 1-bit Feedback}

% Two authors with the same address
\altauthor{
    \Name{Ivan Lau} 
    \Email{ivan.lau@u.nus.edu} \\
    % \and
    % \Name{Jonathan Scarlett} 
    % \Email{scarlett@comp.nus.edu.sg}\\
 \addr National University of Singapore \\
 \Name{Jonathan Scarlett} 
    \Email{scarlett@comp.nus.edu.sg}\\
 \addr National University of Singapore
 }



\renewcommand{\cite}{\citep}


\begin{document}
\maketitle

\begin{abstract}
   In this paper, we study a variant of best-arm identification involving elements of risk sensitivity and communication constraints. Specifically, the goal of the learner is to identify the arm with the highest quantile reward, while the communication from an agent (who observes rewards) and the learner (who chooses actions) is restricted to only one bit of feedback per arm pull. We propose an algorithm that utilizes noisy binary search as a subroutine, allowing the learner to estimate quantile rewards through 1-bit feedback. We derive an instance-dependent upper bound on the sample complexity of our algorithm and provide an algorithm-independent lower bound for specific instances, with the two matching to within logarithmic factors under mild conditions, or even to within constant factors in certain low error probability scaling regimes. The lower bound is applicable even in the absence of communication constraints, and thus we conclude that restricting to 1-bit feedback has a minimal impact on the scaling of the sample complexity.
\end{abstract}

\begin{keywords}%
  Best-Arm identification, quantile bandits, 1-bit quantization
\end{keywords}

\section{Introduction}
\label{sec:intro}

\begin{figure*}[tb]
    \centering
    \includegraphics[width=0.848\linewidth]{figs/circuitnn.pdf} 
    \caption{Illustration of differentiable CircuitNN. CircuitNN is designed based on differentiable NAND gates. After DAS is guided by PI and PO pairs of the truth table, CircuitNN can get the precise circuit architecture logic equivalent to the truth table.}
    \label{fig:circuitnn}
\end{figure*}

% 1. Describe the importance of logic synthesis
% 2. Existing Problems
% (a) Neural Architecture Search: Unstable, Predefined Setting, etc.
% (b) Circuit Generation: Probabilistic Model, Logic Equivalence

With the rapid advancement of technology, the scale of integrated circuits (ICs) has expanded exponentially. 
This expansion has introduced significant challenges in chip manufacturing, particularly concerning power and area metrics.
A primary objective in IC design is achieving the same circuit function with fewer transistors, thereby reducing power usage and area occupancy.

Logic synthesis~\cite{hachtel2005logicsynth}, a critical step in electronic design automation (EDA), transforms behavioral-level circuit designs into optimized gate-level circuits, ultimately yielding the final IC layout. 
The primary goal of logic synthesis is to identify the physical implementation with the fewest gates for a given circuit function. 
This task constitutes a challenging NP-hard combinatorial optimization problem. 
Current logic synthesis tools~\cite{brayton2010abc, wolf2013yosys} rely on human-designed heuristics, often leading to sub-optimal outcomes.

Differentiable architecture search (DAS) techniques~\cite{liu2018darts, chu2020darts} offer novel perspectives on addressing challenges in this problem.
Circuit functions can be represented through truth tables, which map binary inputs to their corresponding outputs. 
Truth tables provide a precise representation of input-output relationships, ensuring the design of functionally equivalent circuits.
Inspired by this, researchers~\cite{deepmind2024ai4sys, wang2024tnet} have begun exploring the application of DAS to synthesize circuits directly from truth tables.
Specifically, \citet{deepmind2024ai4sys} proposed CircuitNN, a framework that learns differentiable connection structures with logic gates, enabling the automatic generation of logic circuits from truth tables.
This approach significantly reduces the complexity of traditional circuit generation. 
Building on this, \citet{wang2024tnet} introduced T-Net, a triangle-shaped variant of CircuitNN, incorporating regularization techniques to enhance the efficiency of DAS.

Despite these advancements, several challenges remain. 
The computational complexity of DAS grows quadratically with the number of gates, posing scalability issues.
Although triangle-shaped architecture~\cite{wang2024tnet} partially mitigates this problem, redundancy persists. 
%Additionally, DAS is susceptible to converging to local optima, limiting the ability to search architectures that satisfy the given truth tables~\cite{liu2018darts}. 
%Furthermore, hyperparameters (network depth and layer width) require extensive searches, introducing complexity and prolonging the synthesis process. 
Additionally, DAS is susceptible to converging to local optima~\cite{liu2018darts} and hyperparameters (network depth and layer width) require extensive searches. 
The challenges arise from the vast search space in DAS. 
% Even with predefined settings for CircuitNN, finding a configuration that meets the truth table requires extensive trial and error during the DAS process. 
Intuitively, limiting the search space through predefined parameters (network depth, gates per layer, and connection probabilities) can significantly reduce the complexity.

Recent advances~\cite{openai2023gpt4, abramson2024alphafold3, esser2024sd3, li2024mar} in conditional generative models have demonstrated remarkable performance across language, vision, and graph generation tasks. 
Motivated by these developments, we propose a novel approach to circuit generation that generates preliminary circuit structures to guide DAS in generating refined circuits matching specified truth tables. 
Firstly, we introduce CircuitVQ, a tokenizer with a discrete codebook for circuit tokenization. 
Built upon our Circuit AutoEncoder framework~\cite{hou2022graphmae,li2023maskgae,wu2025mgvga}, CircuitVQ is trained through a circuit reconstruction task. 
Specifically, the CircuitVQ encoder encodes input circuits into discrete tokens using a learnable codebook, while the decoder reconstructs the circuit adjacency matrix based on these tokens.
Subsequently, the CircuitVQ encoder serves as a circuit tokenizer for CircuitAR pretraining, which employs a masked autoregressive modeling paradigm~\cite{chang2022maskgit, li2023mage}. 
In this process, the discrete codes function as supervision signals. 
After training, CircuitAR can generate discrete tokens progressively, which can be decoded into initial circuit structures by the decoder of the CircuitVQ. 
These prior insights can guide DAS in producing refined circuits that match the target truth tables precisely.

Our key contributions can be summarized as follows:
\begin{itemize}
\item We introduce CircuitVQ, a circuit tokenizer that facilitates graph autoregressive modeling for circuit generation, based on our Circuit AutoEncoder framework;
\item Develop CircuitAR, a model trained using masked autoregressive modeling, which generates initial circuit structures conditioned on given truth tables;
\item Propose a refinement framework that integrates differentiable architecture search to produce functionally equivalent circuits guided by target truth tables;
\item Comprehensive experiments demonstrating the scalability and capability emergence of our CircuitAR and the superior performance of the proposed circuit generation approach.
\end{itemize}

% Motivation
% (a) Diffusion (Vision, Graph), Autoregressive (Language, Vision)
% (b) Circuit Generation for Predefined Setting
% (c) Neural Architecture Search for Strict Logic Equivalence

% Contribution
% (a) Circuit Tokenizer (new transformer arch, training strategy)
% (b) CircuitAR (train and gen strategies, post-ar strategy)
% (c) Extensive Evaluation including BitD (Bit Distance) for Scalability

\section{Problem Setup and Contributions}
\subsection{Problem Setup}
\label{sec: setup}
We study the following variant of fixed-confidence best arm identification for quantile bandits. 

\textbf{Arms and quantile rewards.}
The learner is given a set of arms $\A = \{1, 2, \dots, K\}$ with a stochastic reward setting. That is, for each arm $k \in \A$, the observations/realizations of its reward are i.i.d. random variables from some fixed but unknown reward distribution with CDF~$F_k$. This defines a (lower) quantile function $Q_k \colon [0,1] \to \R$ for each $k \in \A$ as follows:\footnote{The equality follows from the right-continuity of $F_k$.}
\begin{equation}
    Q_k(p) \coloneqq \sup \{ x  \in \R : F_k(x) < p \} 
    =
    \inf \{ x \in \R : F_k(x) \ge p \}.
\end{equation} 
The learner is interested in identifying an arm $\hat{k}$ with the highest $q$-quantile.
% , i.e., 
% an arm $\hat{k}$ satisfying
% \begin{equation}
% % \label{def: performance def}
%      Q_{\hat{k}}(q)
%         =
%         \max_{a \in \A}
%         Q_{a}(q).
% \end{equation}
While the reward of each arm is allowed to be unbounded, 
we assume the $q$-quantile of each arm to be bounded in a known range $[0, \lambda]$.\footnote{We note that setting the lower limit to 0 is without loss of generality, and regarding the interval length $\lambda$, even a crude upper bound is reasonable since the sample complexity will only have logarithmic dependence; see Theorem~\ref{theorem: upper bound}.}
%\footnote{Any finite interval of length $\lambda$ can be shifted to this range.}
% We treat $q \in (0, 1)$ as a fixed constant (e.g., $q = 1/2$ for the median) throughout the paper, meaning its dependence may be omitted in $O(\cdot)$ notation.
We let $\gP = \gP(q, \lambda)$ denote the collection of all distributions with $q$-quantile in $[0, \lambda]$, and let $\cE \coloneqq \gP
^K$ be the collection of all possible instances the learner could face.  We will sometimes write $\PP_{\nu}[\cdot]$ and $\EE_{\nu}[\cdot]$ to explicitly denote probabilities and expectations under an instance $\nu \in \cE$.

\textbf{1-bit communication constraint.} We frame the problem as having a single learner that makes decisions, and a single agent that observes rewards and sends information on them to the learner. In Remark \ref{rem:assump} below, we discuss how this can also have a multi-agent interpretation. 
% 
With a single agent, the following occurs at each iteration/time~$t \ge 1$ indexing the number of arm pulls:
\begin{enumerate}[topsep=0pt, itemsep=0pt]
    \item The learner asks the agent
    to pull an arm $a_t \in \A$, and sends the agent some side information~$S_t$.
    % based on the history of the game until time $t-1$.

    \item The agent pulls $a_t$ and observes a random reward $r_{a_t, t}$
    distributed according to CDF $F_{a_t}$.
    
    \item The agent transmits a 1-bit message to the learner, where the message is based on $r_{a_t, t}$ and $S_t$.

    \item The learner decides on arm $a_{t+1} \in \A$ and side information $S_{t+1}$,
    based on arms and the 1-bit information
    received in iterations $1, \ldots, t$.
   
\end{enumerate}
We will focus on the \emph{threshold query model}, where at iteration $t$, the side information $S_t$ is a query of the form 
``Is $r_{a_t, t} \le \gamma_t$?'' and the 1-bit message is the corresponding binary feedback $\boldsymbol{1}\{ r_{a_t,t} \le \gamma_t \}$.
The learner will only use such queries as side information in our algorithm, though the problem itself is of interest for both threshold queries and general 1-bit quantization methods (possibly having different forms of side information).


\begin{remark} \label{rem:assump}
    We do not impose any (downlink) communication constraint from the learner to the agent, as this cost is typically not expensive.  While we framed the problem as having a single agent for clarity, we are motivated by settings where the agent at each time instant could potentially correspond to a different user/device.  For this reason, and also motivated by settings where agents are low-memory sensors, we assume that the agent is `memoryless', meaning the 1-bit message transmitted cannot be dependent on rewards observed from previous arm pulls.  The preceding assumptions were similarly adopted in some of the most related previous works \cite{hanna2022solving, mitra2023linear, mayekar2023communication}.
    % This is motivated by settings where agents observing rewards are low memory sensors,  or where the agent at any given iteration may differ from those at the previous iterations.
\end{remark}


\textbf{$\epsilon$-relaxation.}
Fix a QMAB instance $\nu  \in \cE$,
and let $k^* \in \A$ be an arm with the largest $q$-quantile for the instance $\nu$.
Instead of insisting on identifying an arm with the exact highest quantile, we relax the task by only requiring the identified arm $\hat{k}$ to be at most $\epsilon$-suboptimal in the following sense:
    \begin{equation}
    \label{def: performance def}
    \hat{k} \in
    \A_{\epsilon}(\nu) \coloneqq 
    \Big\{ k \in \A 
    \ \Big\vert\
     Q_k(q)
        \ge
        Q_{k^*}(q)
        - \epsilon
        \Big\}.
\end{equation}
This allows us to limit the effort on distinguishing arms whose $q$-quantile rewards are very close to each other; analogous relaxations are common in the BAI literature.
This relaxation is also motivated by the threshold query model mentioned above; specifically, we will see in Section \ref{sec: log lambda epsilon dependence} that achieving~\eqref{def: performance def} under the threshold query model requires
$\Omega(\log(\lambda/ \epsilon))$ arm pulls even in the case of \textit{deterministic} two-arm bandits.
Our goal is to design an algorithm to identify an arm satisfying~\eqref{def: performance def} with high probability while using as few arm pulls as possible.



\subsection{Summary of Contributions.}
\label{sec: contributions}
With the problem setup now in place, we summarize our main contributions as follows: 


\begin{itemize}[topsep=0pt, itemsep=0pt]
    \item We provide an algorithm (Algorithm~\ref{alg: main}) for our setup, with the uplink communication satisfying the 1-bit constraint. Unlike standard bandit algorithms that compute empirical statistics using lossless observations of rewards, we use a noisy binary search subroutine for the learner to estimate the quantile rewards (see Appendix~\ref{sec: appendix QuantEst}).
    
    \item  We introduce fundamental arm gaps~$\Delta_k$ (Definition~\ref{def: our gap}) that generalize those proposed in prior work (see Remark~\ref{rem: gap generalization}). These gaps capture the
    difficulty of our problem setup in the sense that the problems with positive gaps essentially coincide with the set of problems that are solvable; see Theorem~\ref{thm: zero gap is unsolvable} and Remark \ref{rem: picking large enough c} for precise statements.

    \item We provide an instance-dependent upper bound on the number of arm pulls to guarantee~\eqref{def: performance def} with high probability (Corollary~\ref{cor: combined guarantee}), expressed in terms of $\lambda, \epsilon$, and fundamental arm gap~$\Delta_k$.
    Our upper bound scales logarithmically with $\lambda/\epsilon$, which contrasts with the existing upper bound for mean-based bandits with 1-bit quantization scaling linearly with $\lambda$~\cite{vial2020one, hanna2022solving}.
    

    \item  We also derive a worst-case lower bound (Theorem~\ref{thm: lower bound unquantized}) showing that our upper bound is tight to within logarithmic factors under mild conditions, and can even be tight to within constant factors when the target error probability $\delta$ decays to zero fast enough.  We additionally provide a lower bound (Theorem \ref{thm: log lambda/epsilon dependence}) showing that $\Omega(\log(\lambda/ \epsilon))$ dependence is unavoidable under threshold queries in arbitrary scaling regimes.  
    The former lower bound is applicable even in the absence of communication constraints, so we can conclude that restricting to 1-bit feedback has a minimal impact on the sample complexity, at least in terms of scaling laws.

\end{itemize}
\section{The general case: Proof of \texorpdfstring{\Cref{thm:main-decomp}}{Theorem 1.6}}\label{sec:algo}

First, we show that data structure of \Cref{l:max_min_query} can be used to compute distances witnessed by shortest paths that pass through a constant-size separator.

\begin{lemma}\label{l:single_adhesion}
Fix a constant $k \in \mathbb{N}$. There exists an algorithm which as the input receives an edge-weighted graph $G$ on $n$ vertices and $m$ edges together with a partition of its vertices into three sets $A, B, C$ such that $|B| \leq k$ and there are no edges between $A$ and $C$, and as the output computes $\max_{c \in C} \dist(a, c)$ for every $a \in A$. The running time is $\Oh(m \log n + n \log^{k - 1} n)$.
\end{lemma}

\begin{proof}
Let $B = \{b_1, \ldots, b_k\}$. For any $a \in A, c \in C$, we have $\dist(a, c) = \min_{i \in [k]} \dist(a, b_i) + \dist(c, b_i)$. First, we run Dijkstra's algorithm from every vertex in $B$ to find $\dist(v, b_i)$ for every $v \in V(G)$ and $i \in [k]$. Next, we use \Cref{l:max_min_query} to construct a data structure $\mathbb{D}$ for the point set $\{(\dist(c, b_1), \dots, \dist(c, b_k))\colon c\in C\}\subseteq \mathbb{R}^k$. Now, the value $\max_{c \in C} \dist(a, c)$ for any given $a$ is equal to the answer of $\mathbb{D}$ to the query with argument $(\dist(a, b_1), \dots, \dist(a, b_k))$.
\end{proof}

After computing the distances over a constant-size separator, we will use the following observation to simplify one of the sides of the separation.

\begin{lemma}\label{l:inserting_paths}
Let $G$ be a edge-weighted connected graph and let $A, B, C$ be a partition of its vertices such that there are no edges between $A$ and $C$. For every pair of vertices $u, v \in B$, let $P_{u, v}$ be any shortest path from $u$ to $v$ with all internal vertices in $C$ (assuming such a path exists).

Let $G'$ denote a graph obtained from $G[A \cup B]$ by adding an edge from $u$ to $v$ of weight equal to the length of $P_{u, v}$, for all $u, v \in B$ for which $P_{u, v}$ exists. Then,  $$\dist_G(s, t) = \dist_{G'}(s, t)\qquad\textrm{for all }s,t\in A\cup B.$$
\end{lemma}
\begin{proof}
Let $G''$ be the graph obtained by adding new edges of $G'$ to $G$.
Fix any $s, t \in A \cup B$ and let $P$ denote the shortest path from $s$ to $t$ in $G''$ which minimizes the number of vertices from $C$ visited. Naturally, the weight of $P$ is equal $\dist_G(s, t)$. Assume that such path visits at least one vertex of $C$. Then, the path $P$ is of the form $s \xrightarrow{P_1} x \xrightarrow{P_2} y \xrightarrow{P_3} t$, where $x, y \in B$ and all the internal vertices of $P_2$ are in $C$. By the construction of $G'$, $P_2$ can be replaced with a direct edge from $x$ to $y$ of the same weight. We obtain a same weight path with a smaller number of vertices of $C$ visited, which is a contradiction. Therefore, $P$ is entirely contained in $A \cup B$, hence it exists in $G'$. This shows that $\dist_G(s, t) = \dist_{G'}(s, t)$.
\end{proof}


The next lemma encapsulates the main algorithmic content of the proof of \Cref{thm:main-decomp}. The algorithm will split the tree decomposition provided on input into smaller parts for which the eccentricities are easier to calculate. We use the following lemma to handle a single such part.
\begin{lemma}\label{l:star}
Fix constants $k, g \in \mathbb{N}, 0 < \delta < \frac{1}{54}$. Assume we are given $n \in \mathbb{N}$, an edge-weighted graph $G$ on at most $n$ vertices with a weight function $w \colon E(G) \to \mathbb{N}$, a vertex subset $A$ and a collection of non-empty vertex subsets $V_0, V_1, \dots, V_\ell$ satisfying the following conditions:
\begin{itemize}[nosep]
	\item The sum of weights of all the edges in $G$ is bounded by $\Oh(n)$.
	\item $V(G) \setminus A = V_0 \cup V_1 \cup \dots \cup V_\ell$.
	\item $|A| \leq k$.
	\item For every $i \in [\ell]$, $G[V_i \setminus V_0]$ is connected, $N_G(V_i \setminus V_0) = V_i \cap V_0$, $|V_i| = \Oh(n^\delta)$, and $|V_0 \cap V_i| \leq 4$.
	\item For all $i, j \in [\ell], i \neq j$, $V_i \setminus V_0$ and $V_j \setminus V_0$ are disjoint and non-adjacent in $G$.
	\item Every edge $uv \in E(G)$ with $u, v \not\in A$ is contained in $G[V_i]$ for some $i\in \{0,1,\ldots,\ell\}$.
	\item The graph obtained by taking $G[V_0]$ and adding a clique on $V_0 \cap V_i$ for every $i \in [\ell]$ has Euler genus bounded by $g$.
\end{itemize}
Then, we can compute the eccentricity of every vertex of $G$ in time $\Oh \left( n^{1 + \frac{150 + 54 \delta}{151}} \log^k n \right)$.
\end{lemma}

\begin{proof}
Fix $\delta' = \frac{1 + 97 \delta}{151}$; we have $\delta' - \delta = \frac{1 - 54\delta}{151} > 0$.
Let $E_i$ denote the set of edges with one endpoint in $V_i$ and the other endpoint in $V_i \setminus V_0$. For $i \in [\ell]$, we shall say that $V_i$ is {\em{heavy}} if the sum of weights of $E_i$ is larger than $n^{\delta'}$. Since the sets $E_i$ are pairwise disjoint and the total sum of weights of all the edges is bounded by $\Oh(n)$, the number of heavy subsets is bounded by $\Oh(n^{1 - \delta'})$. Without loss of generality, we may assume that $V_{\ell' + 1}, \dots, V_\ell$ are heavy and $V_1, \dots, V_{\ell'}$ are not, for some $\ell'\in \{0,\ldots,\ell\}$.


For any source vertex $s$, we can calculate distances from $s$ to every vertex of $G$  using breadth first search in time $\Oh(\sum_{e \in E(G)} w(e)) = \Oh(n)$.
In particular, for every $\ell' < i \leq \ell$, we can compute the distances from every vertex of $V_i$ to every vertex of $G$ in total time $\Oh(n^{2 - \delta' + \delta})$, because $$|V_{\ell'+1}\cup \ldots\cup V_{\ell}|\leq n^{1-\delta'}\cdot \Oh(n^\delta)=\Oh(n^{1-\delta'+
\delta}).$$
Additionally, we calculate distances $\dist_G(a, v)$ for every $a \in A, v \in V(G)$ in time $O(n)$.

For every $i \in [\ell]$ and $u,v \in V_0 \cap V_i$, there exists a shortest path $P_{i,u,v}$ from $u$ to $v$ with all internal vertices belonging to $V_i - V_0$ due to the assumption that $G[V_i - V_0]$ is connected and $N_G(V_i - V_0) = V_i \cap V_0$. Therefore, the distance from $u$ to $v$ is bounded by the sum of weights of edges in $E_i$. In particular, for $i \in [\ell']$, $\dist_G(u, v) \leq n^{\delta'}$.

We define $\widetilde{G}$ to be the graph obtained by taking $G[A \cup V_0 \cup \dots \cup V_{\ell'}]$ and applying the following operation for every $i \in \{\ell' + 1, \dots, \ell\}$:
for each pair of vertices $u, v \in A \cup (V_0 \cap V_i)$, add an edge in $\widetilde{G}$ between $u$ and $v$ with weight equal to the total weight of $P_{i,u,v}$. For a fixed $i, u$, we can find $P_{i, u, v}$ for all $v$ using breadth first search in time $\Oh(n)$. Taking a sum over all $i, u$, we get that $\tilde{G}$ can be computed in total time $\Oh(n^{2 - \delta'})$.


\begin{claim}\label{cl:wG}
The sum of the edge weights in $\widetilde{G}$ is $\Oh(n)$. Moreover, for all $u, v \in V(\widetilde{G})$, we have $\dist_{\widetilde{G}}(u, v) = \dist_{G}(u, v)$.
\end{claim}

\begin{proof}
Consider $i \in \{\ell' + 1, \dots, \ell\}$ and any $u, v \in A \cup (V_0 \cap V_i)$ for which we added an edge. Its weight is bounded by the sum of weights of edges in $E_i$. Therefore, the total weight of all edges added is at most
$$
\sum_{i \in \{\ell' + 1, \dots, \ell\}} \left( |A \cup (V_0 \cap V_i)|^2 \sum_{e \in E_i} w(e) \right) \leq (4 + k)^2 \sum_{e \in E(G)} w(e) = \Oh(n).
$$
This proves the first part of the claim.

For the second part of the claim, consider any $i \in \{\ell' + 1, \dots, \ell \}$ and observe that by our assumptions, $A \cup (V_0 \cap V_i)$ separates $(V_0 \cup \dots \cup V_{\ell'} \cup V_{i + 1} \cup \dots \cup V_\ell) \setminus V_i$ from $V_i \setminus V_0$. Hence it suffices to repeatedly apply \Cref{l:inserting_paths}.
\end{proof}

For every $u \in V(\widetilde{G})$, we have $\ecc_G(u) = \max(\ecc_{\widetilde{G}}(v), \max_{v \in V(G) \setminus V(\widetilde{G})} \dist_G(u, v))$. Note, that we already know all the distances $\dist_G(u, v)$ for $v \in V(G) \setminus V(\widetilde{G})$. Similarly, we can already compute $\ecc_G(u)$ for every $u \in V(G) \setminus V(\widetilde{G})$. Therefore, it remains to compute $\ecc_{\widetilde{G}}(v)$ for each $v \in V(\widetilde{G})$. Our goal is to show that this can be done efficiently using \Cref{l:main_ecc}.

Now, let $G'$ be the graph obtained from $\tilde{G}$ by replacing every edge $e$ non-indicent to $A$ with $w(e)\geq 2$ with a path of length $w(e)$ consisting of unit-weight edges. This operation again preserves the distances. Since the sum of edge weights in $\tilde{G}$ is of $\Oh(n)$, the total number of vertices in $G'$ is of $\Oh(n)$. For $0 \leq i \leq \ell'$, we write $V'_i$ to denote the set $V_i$ together with all the vertices added as a part of a path between two endpoints in $V_i$.
As $V_i$ is not heavy for each $i\in [\ell']$, we have
$$
|V'_i \setminus V'_0| \leq |V_i| + \sum_{e \in E_i} w(e) = \Oh(n^{\delta'})\qquad \textrm{for all }i\in [\ell'].
$$

Let $G_0$ denote the graph $G'[V'_0]$ and let $G_0^*$ denote the graph $G'- A$ with $V'_i - V'_0$ contracted to a single vertex $v_i^*$, for each $i \in [\ell']$; note that, all edges of $G_0$ and $G_0^*$ have unit weight.

\begin{claim}
	The graph $G_0^*$ is does not contain $K_{t}$ as a minor, where $t = \Oh(\sqrt{g})$.
\end{claim}

\begin{proof}
Let $\bar{G}_0$ denote the graph obtained by taking $G_0$ and adding a clique on $V_0 \cap V_i$ for every $i \in [\ell']$.
By lemma assumptions and the fact that subdividing edges does not increase the Euler genus, $\bar{G}_0$ has Euler genus at most $g$. In particular, $\bar{G}_0$ is $K_{t'}$-minor-free for some $t' = \Oh(\sqrt{g})$, because the Euler genus of $K_{t'}$ is $\Omega({t'}^2)$.

Similarly, let $\bar{G}_0^*$ be the graph obtained by taking $G_0^*$ and adding a clique on each $V_0 \cap V_i$.
Note, that $\bar{G}_0^* - \{v_1^*, \dots, v_{\ell'}^*\}$ is precisely $\bar{G}_0$. Let $t = \max(t', 6)$.
Recall that a minor model of a clique $K_t$ consists of $t$ pairwise vertex-disjoint connected subgraphs, called
branch sets, such that there is at least one edge between each pair of the branch sets.
Consider a minor model $\varphi$ of $K_{t}$ in $\bar{G}^*_0$.
Note that $\varphi$ cannot contain any singleton branch set of the form $\{v^*_i\}$, for the degree of $v^*_i$ in $\bar{G}^*_0$ is at most $4 < t - 1$. Furthermore, since $N_{\bar{G}^*_0}(v^*_i) = V_0 \cap V_i$, any branch set containing $v^*_i$ and at least one other vertex contains some $u \in V_0 \cap V_i$, and $N_{\bar{G}^*_0}(v^*_i)\subseteq N_{\bar{G}^*_0}(u)$, hence removing $v^*_i$ from this branch set preserves the model. Therefore, we can assume without loss of generality that all branch sets of $\varphi$ are disjoint from $\{v^*_1, \dots, v^*_{\ell'}\}$, hence $\varphi$ is a minor model of $K_{t}$ in $\bar{G}_0$. This is a contradiction, as $t \geq t'$ and $\bar{G}_0$ is $K_{t'}$-minor-free. Therefore, $\bar{G}_0^*$ is $K_t$-minor-free, hence $G_0^*$ also.
\end{proof}

Let $\rho' = \frac{2 - 108 \delta}{151} > 0$. The graph $G^*_0$ is a unit-weight graph and is $K_{t}$-minor-free.
Hence, by applying \Cref{t:r_division} to $G^*_0$ (with $\varepsilon = \rho'/2$)
we obtain an $n^{\rho'}$-division $\mathcal{R}_0$ in time $\Oh(n^{1 + \rho'})$.
We extend it to $G' - A$ by mapping every contracted vertex $v^*_i$ to $N_{G' - A}[V'_i - V'_0] = (V'_i - V'_0) \cup (V_0 \cap V_i)$. Formally, we put $V''_i \coloneqq N_{G' - A}[V'_i - V'_0]$ and 
$$
\mathcal{R} \coloneqq \left\{ (R_0 \cap V'_0) \cup \bigcup_{i \colon v^*_i \in R_0} V''_i \colon R_0 \in \mathcal{R}_0 \right\}.
$$

Now, we argue that $\mathcal{R}$ is a reasonable division of $G' - A$. Clearly, all sets $R \in \mathcal{R}$ are connected in $G' - A$. Pick any $R \in \mathcal{R}$ and let $R_0$ be its corresponding set in $\mathcal{R}_0$.
Every vertex $v^*_i$ is mapped to a set of size $\Oh(n^{\delta'})$, therefore
$$|R| \leq |R_0| \cdot \Oh(n^{\delta'}) = \Oh(n^{\rho' + \delta'}).$$

By our construction, for every $i \in [\ell']$, $R$ is either disjoint from $V'_i - V'_0$ or contains whole $N_{G' - A}[V'_i - V'_0]$. This means that no vertex belonging to any $V'_i - V'_0$ can be in $\partial R$, hence $\partial R \subseteq V'_0$.

Pick any $u \in \partial R \cap R_0$. Assume that $u \not\in \partial R_0$. Then every vertex of $N_{G_0^*}(u)$ must be in $R_0$, hence $N_{G - A'}(u) \subseteq R$, which is a contradiction. This means that $\partial R \cap R_0 \subseteq \partial R_0$.

Pick any $u \in \partial R - R_0$. Then, $u \in V_0 \cap V_i$ for some $i \in [\ell']$ such that $v_i^* \in R_0$. Moreover, $v_i^* \in \partial R_0$ and is adjacent to $u$ in $G_0^*$. The number of such $u$ is bounded by $4 |\partial R_0 \cap \{ v_1^*, \dots, v_{\ell'}^* \}|$.

Putting two cases together, we obtain:
$$
\sum_{R \in \mathcal{R}} |\partial R| = \sum_{R \in \mathcal{R}} \left(|\partial R \cap R_0| + |\partial R - R_0|\right) \leq \sum_{R_0 \in \mathcal{R}_0} \left(|\partial R_0| + 4 |\partial R_0 \cap \{ v_1^*, \dots, v_{\ell'}^* \}|\right) = \Oh(n^{1 - \frac{1}{2}\rho'}).
$$

It remains to show the following claim.

\begin{claim}
Pick any $R \in \mathcal{R}, s_R \in R$. The number of different distance profiles on $R$ relative to $s_R$ in $G' - A$ is of $\Oh(n^{48\rho' + 54\delta'})$.
\end{claim}
\begin{proof}
We look at every vertex $v \in V(G') \setminus A$ and consider three cases: $v \in R$, $v \in V'_0$, and $v \in V'_i \setminus (V'_0 \cup R)$ for some $i \in [\ell']$. By our construction, $R \cap V'_0$ is non-empty, hence w.l.o.g. we can assume that $s_R \in V'_0$ as whether two vertices have the same profile on $R$ is independent of the choice of the pivot vertex.

In the first case, there are at most $|R| = \Oh(n^{\rho' + \delta'})$ such vertices, hence they realise at most that many profiles.

In the second case, we want to observe that profile of any vertex $v \in V'_0$ on $R$ depends only on its profile on $R \cap V'_0$ (relative to $s_R$). Pick any $t \in R - V'_0$. Then $t \in V'_i - V'_0$ for some $i \in [\ell']$, $V_i \cap V_0 \subseteq R \cap V'_0$, and every path from $v$ to $t$ intersects $V_i \cap V_0$. In particular, distances from $v$ to vertices of $V_i \cap V_0$ determine its distance to $t$, which proves the observation.

Let $\tilde{G}_0$ denote the graph obtained by taking $G'[V'_0]$ and for every $i \in [\ell'], u, v \in V_0 \cap V_i$ adding a disjoint path from $u$ to $v$ of length $\dist(u, v)$. Let $P_i$ denote the vertex set of paths added between $V_0 \cap V_i$. For every $t \in V'_0$ we have $\dist_{G' - A}(v, t) = \dist_{\tilde{G}_0}(v, t)$, so it suffices to bound the number of profiles on $R \cap V'_0$ in $\tilde{G}_0$. By our assumptions, $\tilde{G}_0$ has Euler genus bounded by $g$ and all $P_i$ are of size $\Oh(n^{\delta'})$.

Let $R_0$ be the set of $\mathcal{R}_0$ corresponding to $R$. Let $\tilde{R}_0$ denote the set $(R \cap V'_0) \cup \bigcup_{i : v^*_i \in R_0} P_i$. Such set is connected in $\tilde{G}_0$. Moreover, similarly to $R$, its size is $\Oh(n^{\rho' + \delta'})$. Applying \Cref{thm:distprofiles}, we get that the number of distance profiles on $\tilde{R}_0$ in $\tilde{G}_0$ is $\Oh(n^{12(\rho' + \delta')})$, which also bounds the number of profiles on $R$ in $G' - A$ realised by $V'_0$.

For the third case, assume $v \in V'_i \setminus (V'_0 \cup R)$ for some $i\in [\ell']$. Every path from $v$ to any vertex of $R$ in $G' - A$ intersects $V_i \cap V_0$. Let $v_1, \dots v_p$ be the vertices of $V_i \cap V_0$, where $p \leq 4$. The profile of $v$ on $R$ is then determined by the following:
\begin{itemize}[nosep]
 \item[(a)] the profile of each $v_j$ on $R$,
 \item[(b)] $\dist_{G' - A}(v, v_j) - \dist_{G' - A}(v, v_1)$ for each $2 \leq j \leq p$, and
 \item[(c)] $\dist_{G' - A}(s_R, v_j) - \dist_{G' - A}(s_R, v_1)$ for each $2 \leq j \leq p$ where $s_R$ is some pivot vertex of $R$.
\end{itemize}
By the previous case, the number of distance profiles of each $v_j$ is $\Oh(n^{12(\rho' + \delta')})$. The distances between $v$ and $v_j$ are bounded by $|V'_i|$, hence each quantity described in (b) can take $\Oh(n^{\delta'})$ different possible values. Similarly, since $v_1$ and $v_j$ are connected via $V'_i$, $|\dist_{G' - A}(s_R, v_j) - \dist_{G' - A}(s_R, v_1)| \leq \Oh(n^{\delta'})$. The number of different possible profiles of such $v$ is therefore bounded by $\Oh(n^{48(\rho' + \delta') + 6\delta'}) = \Oh(n^{48\rho' + 54\delta'})$. This finishes the proof of the claim.
\end{proof}

Now we can apply \Cref{l:main_ecc} to graph $G'$ with apex set $A$, $X = V(\widetilde{G})$, and the following constants: $$\rho = \rho' + \delta',\qquad \gamma = 1 - \frac{1}{2}\rho',\quad \textrm{and}\quad \alpha = 48\rho' + 54 \delta'.$$ This allows us to calculate all $V(\widetilde{G})$-eccentricities in $G'$ in time
$$
\Oh \left( \left(
	n^{ 2 - \frac{1}{2} \rho' } +
	n^{ 1 + 48\rho' + 54 \delta' }
\right) \log^k n \right) =
\Oh \left( n^{1 + \frac{150 + 54 \delta}{151}} \log^k n \right).
$$
Since for each $v\in V(\widetilde{G})$ we have $\ecc_{\widetilde{G}}(v) = \max_{u \in V(\widetilde{G})} \dist_{\widetilde{G}}(v, u) = \max_{u \in V(\widetilde{G})} \dist_{G'}(v, u)$, this means that we have successfully computed all the eccentricities in $\widetilde{G}$; and as we argued, this is enough to compute all the eccentricities in $G$ as well.

Finally, the total running time of the algorithm is
$$
\Oh \left( n^{1 + \frac{150 + 54 \delta}{151}} \log^k n + n^{2 - \delta' + \delta} \right) =
\Oh \left( n^{1 + \frac{150 + 54 \delta}{151}} \log^k n \right).
$$\qedhere
\end{proof}


\begin{lemma}\label{l:star2}
Fix constants $k, g \in \mathbb{N}, 0 < \delta < \frac{1}{54}$. Assume we are given $n \in \mathbb{N}$, an edge-weighted graph $G$ on at most $n$ vertices with a weight function $w \colon E(G) \to \mathbb{N}$, a vertex subset $A$ and a collection of non-empty vertex subsets $V_0, V_1, \dots, V_\ell$ satisfying the same conditions as in \Cref{l:star} with the following differences:
\begin{itemize}
	\item we don't require $G[V_i - V_0]$ to be connected and $V_i - V_0$ to be adjacent to whole $V_i \cap V_0$;
	\item instead of $|V_0 \cap V_i| \leq 4$, we require $|V_0 \cap V_i| \leq k$.
\end{itemize}
Then, we can compute the eccentricity of every vertex of $G$ in time $\Oh \left( n^{1 + \frac{150 + 54 \delta}{151}} \log^{k + 5g} n \right)$.
\end{lemma}

\begin{proof}
We will reduce our input to one which will satisfy the conditions of \Cref{l:star}. We start by addressing the adhesions $V_0 \cap V_i$ containing too many vertices.

Let $G_0$ denote the graph $G[V_0]$ with cliques placed at $V_0 \cap V_i$ for every $i \in [\ell]$.
For every $i \in [\ell]$ we repeat the following procedure: while $|V_0 \cap V_i| > 4$,
remove arbitrary $5$ vertices from $V_0 \cap V_i$. Since $|V_0 \cap V_i| \leq k$ for each $i\in [\ell]$,
this procedure can be implemented in total time $\Oh(n)$. As a result, at the end we have $|V_0 \cap V_i| \leq 4$ for all $i \in [\ell]$. Let $M$ be the set of all the removed vertices. By our assumptions, $G_0$ has Euler genus bounded by $g$, hence it cannot contain $g + 1$ pairwise disjoint copies of $K_5$
(as the Euler genus of a graph is the sum of the Euler genera of its 2-connected components~\cite{StahlB77} and $K_5$ is not planar). Each removed quintiple of vertices induces a $K_5$ in $G_0$, hence we have $|M| \leq 5g$. We set $A' = A \cup M$ and may thus assume that $V_i$ is disjoint from $A'$ for all $0 \leq i \leq \ell$.

Now, fix $i \in [\ell]$. Let $C^i_1, \dots, C^i_{r_i}$ denote the connected components of $V_i - V_0$ in $G - A'$. We define $W^i_j := N_{G - A'}[C^i_j]$ for every $j \in [r_i]$. Clearly, all $W^i_j$ induce a connected subgraph of $G$ and satisfy $N_{G - A'}(W^i_j - V_0) = W^i_j \cap V_0$. We put $V'_0 := V_0$ and enumerate
$$
\{V'_1, V'_2, \dots V'_{\ell'}\} := \{ W^i_j \colon i \in [\ell], j \in [r_i] \}.
$$
It is easy to verify that the sets $A'$ and $V'_0, V'_1, \dots, V'_{\ell'}$ satisfy the conditions of \Cref{l:star}. We apply said lemma to calculate the eccentricity of every vertex of $G$ in the desired time.
\end{proof}



The next statement is a reformulation of \Cref{thm:main-decomp}.

\begin{theorem}
Fix constants $k, g \in \mathbb{N}$. Assume we are given a graph $G$ on $n$ vertices together with its tree decomposition $(T, \beta)$ and a set of private apices $A_t \subseteq \beta(t)$ for each node $t\in V(T)$ such that the following conditions hold:
\begin{itemize}[nosep]
 \item For every node $t \in V(T)$, we have $|A_t| \leq k$.
 \item For every edge $st \in E(T)$,  we have $|\beta(v) \cap \beta(u)|\leq k$.
 \item For every node $t \in V(T)$, graph obtained by taking $G[\beta(t)] - A_t$ and turning  $(\beta(t) \cap \beta(s))\setminus A_t$ into a clique for every edge $st \in E(T)$ has Euler genus bounded by $g$.
\end{itemize}
Then, we can compute the eccentricity of every vertex of $G$ in time $\Oh \left( n^{1 + \frac{355}{356}} \log^{k + 5g} n \right)$.
\end{theorem}

\begin{proof}
We may assume that $|V(T)|\leq n$, for every tree decomposition with no two bags comparable by inclusion has this property; and adjacent comparable bags can be merged by contracting the edge between them.

For a node $t\in V(T)$, by the {\em{weight}} of $t$ we mean the size of the corresponding bag, that is, $|\beta(t)|$. For any subset of nodes $S \subseteq V(T)$, we define $\beta(S) \coloneqq \bigcup_{t \in S} \beta(t)$ By the {\em{weight}} of $S$, we mean the total weight of the elements of $S$, that is, $\sum_{t\in S} |\beta(t)|$. 

\begin{claim}\label{cl:weight-T}
The weight of $V(T)$ is of $\Oh(n)$.
\end{claim}

\begin{proof}
The sets $\beta'(t) := \beta(t) - \bigcup_{s \in N_T(t)} \beta(s)$ are pairwise disjoint. We have
$$
\sum_{t \in V(T)} |\beta(t)| =
\sum_{t \in V(T)} |\beta'(t)| + 2 \cdot \sum_{st \in E(T)} |\beta(s) \cap \beta(t)| \leq
|V(T)| + 2k|E(T)| = \Oh(n).
$$
\end{proof}

Since every bag induces a graph of bounded Euler genus, the number of edges contained in a bag is linear in its size. In particular, this implies that the total number of edges of $G$ is also bounded by $\Oh(n)$.

We set $$\delta \coloneqq \frac{1}{356}\qquad\textrm{and}\qquad \Delta \coloneqq \frac{355}{356}.$$ Root the tree $T$ in an arbitrarily chosen node; this naturally imposes an ancestor-descendant relation in $T$ (for convenience, every node is considered its own ancestor and descendant).

We start by partitioning $T$ into connected subtrees using the following procedure.
We proceed bottom-up over $T$, processing nodes in any order so that a node is processed after all its strict descendants have been processed. Along the way, we mark some nodes and split the edges of $T$ into heavy and light. Let $t \in V(T)$ be the currently processed non-root node of $T$ and let $e \in E(T)$ be the edge connecting $t$ with its parent. If the total weight of all the unmarked nodes that are descendants of $t$ is at least $n^\delta$ (recall that this includes $t$ itself as well), then we declare $e$ heavy and mark all the descendants of $t$ that were unmarked so far. Otherwise, the edge $e$ is declared light and the procedure proceeds to further nodes of $T$.

Observe that
removing all heavy edges splits $T$ into connected subtrees, say $T'_1, \cdots T'_m$. All of the subtrees, except for possibly the subtree containing the root node, are of weight at least $n^\delta$. In particular, the number of subtrees $m$, and therefore the number of heavy edges, is  bounded by $\Oh(n^{1 - \delta})$. Moreover, in every subtree $T'_i$, removing the node closest to the root splits $T'_i$ into smaller components, each of weight less than $n^\delta$.

Fix a heavy edge $e$ and let $T^e_1$ and $T^e_2$ be the two subtrees into which $T$ splits after removing~$e$. Let $X^e_i = \beta(T^e_i)$ for $i \in \{1, 2\}$. Put $A_e = X^e_1 \setminus X^e_2$, $C_e = X^e_2 \setminus X^e_1$, and $B_e = X^e_1 \cap X^e_2$. By the properties of tree decompositions, such choice of $A_e, B_e, C_e$ satisfies the conditions of \Cref{l:single_adhesion}, hence in time $\Oh(n \log^{k - 1} n)$ we can compute $\max_{v \in X^e_2} \dist_G(u,v)$ for every $u \in X^e_1$, and $\max_{u \in X^e_1} \dist_G(u,v)$ for every $v \in X^e_2$. Computing this for every heavy edge $e$ takes total time $\Oh(n^{2 - \delta} \log^{k - 1} n)$.

Fix any subtree $T'=T'_j$. Let $e_1 = t^{e_1}_1t^{e_1}_2, e_2 = t^{e_2}_1 t^{e_2}_2, \dots, e_\ell = t^{e_\ell}_1 t^{e_\ell}_2$ denote the heavy edges incident to $T'$, where $t^{e_i}_1 \in V(T')$ and $V(T') \subseteq V(T_1^{e_i})$ for every $i \in [\ell]$.
For a vertex $v \in \beta(T')$, let
$$d_0(v) = \max_{u \in \beta(T')} \dist_G(v, u)\qquad\textrm{and}\qquad d_i(v) = \max_{u \in X_2^{e_i}}\dist_G(v,u),\quad\textrm{for } i \in [\ell].$$ We have $\ecc(v) = \max \{ d_i(v)\colon i\in \{0,1,\ldots,\ell\}\}$.The values of $d_i(v)$ are already calculated for all $i\in [\ell]$, hence it remains to compute $d_0(v)$.

For every $i \in [\ell]$ and every pair of vertices $u, v \in \beta(t^{e_i}_1) \cap \beta(t^{e_i}_2)$ we find a shortest path between $u$ and $v$ with all internal vertices inside $X^{e_i}_2$ (or determine that it doesn't exist). For a fixed $u, v$ this can be done in time $\Oh(n)$. Since in total we perform this step at most $2k^2$ times per heavy edge, it takes $\Oh(n^{2 - \delta})$ time in total. Let $P_{i, u, v}$ denote such path, assuming it exists.

Let $G'$ denote the graph obtained from $G[\beta(T')]$ by taking every $i, u, v$ for which $P_{i, u, v}$ exists and adding an edge between $u$ and $v$ of weight equal to the total weight of $P_{i, u, v}$.
The weight of every edge inserted in $\beta(t^{e_i}_1) \cap \beta(t^{e_i}_2)$ is bounded by $|X^{e_i}_2|+1$. The total weight of all edges inserted is therefore at most
$$
\sum_{i \in [\ell]} |\beta(t^{e_i}_1) \cap \beta(t^{e_i}_2)|^2 \cdot (|X^{e_i}_2|+1) \leq
k^2 \sum_{i \in [\ell]} (|X^{e_i}_2|+1) = \Oh(n),
$$
where the last equality follows from the fact that all the trees $T^{e_i}_2$ are pairwise disjoint.
By \Cref{l:inserting_paths}, we have $\dist_{G'}(u, v) = \dist_G(u, v)$ for each $u, v \in \beta(T')$. Hence, computing $d_0(v)$ for every $v \in \beta(T')$ is equivalent to computing the eccentricity of every vertex in $G'$.

If the size of $\beta(T')$ is smaller than $n^\Delta$, we compute the eccentricities naively in time $\Oh(|\beta(T')|^2)$, 
noting that $G'$ has $\Oh(|\beta(T')|)$ edges (thanks to Claim~\ref{cl:weight-T} and bounded genus assumption 
of the last bullet of the theorem statement). Otherwise, we argue that we can use the algorithm in \Cref{l:star} as follows.

Let $t$ be the node of $T'$ closest to the root. Let $s_1, \dots, s_p$ be the children of $t$ in $T$ and let $T''_i$ denote the connected component of $T' - \{t\}$ containing $s_i$. Set $V_0 = \beta(t)$ and $V_i = \beta(T''_i)$ for $i \in [p]$.

It is now easy to verify that $G'$ and sets $A, \{V_i\colon 0\leq i\leq p\}$ selected this way satisfy the assumptions of \Cref{l:star2}. This allows us to use it to compute the eccentricities in $G'$ in time
$$
\Oh \left( n^{1 + \frac{150 + 54\delta}{151}} \log^{k + 5g} n \right) =
\Oh \left( n^{1 + \frac{354}{356}} \log^{k + 5g} n \right).
$$
As we argued, from these eccentricities, we may easily compute all the eccentricities in $G$.

Now, let us analyse the total running time of the whole algorithm. We invoke \Cref{l:star} $\Oh(n^{1 - \Delta})$ times, since we apply it only to subtrees $T'_i$ of size at least $n^\Delta$. The total running time of those applications is hence
$$
\Oh \left( n^{2 + \frac{354}{356} - \Delta} \log^{k + 5g} n \right) =
\Oh \left( n^{1 + \frac{355}{356}} \log^{k + 5g} n \right).
$$
We compute the eccentricities naively for subtrees smaller than $n^\Delta$, hence the total running time of this computation is
$$
\sum_{i \in [m] \colon |\beta(T'_i)| \leq n^\Delta} |\beta(T'_i)|^2 \leq
n^\Delta \cdot \sum_{i \in m} |\beta(T'_i)| = \Oh(n^{1 + \Delta})=\Oh\left(n^{1+\frac{355}{356}}\right).
$$
The rest of computation can be done in $\Oh(n^{2 - \delta} \log^k n)$. Therefore, the whole algorithm runs in time $\Oh \left( n^{1 + \frac{355}{356}} \log^{k + 5g} n \right)$.
\end{proof}

\subsection{Lower bounds on sample complexity}\label{sec:sample_compexity}
We establish a lower bound for generalized linear measurements using standard information-theoretic arguments based on Fano's inequality. While the upper bound in Theorem~\ref{thm:alg_general} is derived for the maximum probability of error over all  $k$-sparse vectors, the lower bound applies even in the weaker setting of the average probability of error, where 
$\bx$ is chosen uniformly at random.
\begin{theorem}[Lower bound for GLMs]\label{thm: lower_bdglm} Consider any  sensing matrix $\vecA$.
For a uniformly chosen $k$-sparse vector $\bx$, an algorithm $\phi$ satisfies $$\bbP\inp{\phi(\vecA, \by) \neq \bx}\leq \delta$$   only if the number of measurements $$m\geq \frac{k\log\inp{\frac{n}{k}}}{I}\inp{1 - \frac{h_2(\delta) + \delta k\log{n}}{k\log{n/k}}}$$ for some $I$ such that $I\geq {I(y_i; \bx|\vecA)}, \, i\in [m]$. In particular, when $y\in \inb{-1, 1}$, we have $\bbE\insq{\inp{g(\vecA_i^T\bx)}^2} \geq I(y_i, \bx|\vecA)$ where the expectation is over the randomness of $\vecA$ and $\bx$.
\end{theorem}
The lower bound can be interpreted in terms of a communication problem, where the input message $\bx$ is encoded to $\vecA\bx$. The decoding function takes in as input the encoding map $\vecA$ and the output vector $\by$ in order to recover $\bx$ with high probability. For optimal recovery, one needs at least $\frac{\text{message entropy}}{\text{capacity}}$ number of measurements (follows from noisy channel coding theorem~\cite{thomas2006elements}). In Theorem~\ref{thm: lower_bdglm}, the entropy of the message set $\log{n \choose k}\approx k\log{n/k}$ and the proxy for capacity is the upper bound on mutual information $I$. We provide a detailed proof of the theorem in  Section~\ref{sec:proofs}.


We first present lower bounds for \bcs\  and \logreg. The lower bound for \bcs\ is given for any sensing matrix $\vecA$ which satisfies the power constraint given by \eqref{eq:power_constraint}, whereas the one for \logreg\ is only for the special case when each entry of the sensing matrix is iid $\cN(0,1)$. Recall that \eqref{eq:power_constraint} holds in this case.  For \bcs\ (and \logreg\ respectively), we can use the upper bound of $\bbE\insq{\inp{g(\vecA_i^T\bx)}^2}$ on the mutual information term. The dependence of $\sigma^2$ (and $1/\beta^2$ respectively) requires careful bounding of this term, which is done in the formal proofs in Appendix~\ref{proof:sec:lower_bd}.


As mentioned earlier, we need at least $k\log\inp{n/k}$ measurements for \bcs and \logreg. This is because the entropy of a randomly chosen $k$-sparse vector is approximately $k\log\inp{n/k}$ and we learn at most one bit with each measurement. However, due to corruption with noise, we learn less than a bit of information about the unknown signal with each measurement. The information gain gets worse as the noise level increases. 
Our lower bounds make this reasoning explicit.  
\begin{corollary}[\bcs\ lower bound]\label{thm: lower_bd_bcs} Suppose, each row $\vecA_i, \, i\in [1:m]$ of the sensing matrix $\vecA$ satisfies the power constraint~\eqref{eq:power_constraint}.
For a uniformly chosen $k$-sparse vector $\bx$, an algorithm $\phi$ satisfies $$\bbP\inp{\phi(\vecA, {\by}) \neq \bx}\leq \delta$$ for the problem of $\bcs$ only if the number of measurements $$m\geq \frac{k+\sigma^2}{2}\log\inp{\frac{n}{k}}\inp{1 - \frac{h_2(\delta) + \delta k\log{n}}{k\log{n/k}}}.$$ 
\end{corollary}

\begin{corollary}[\logreg\ lower bound]\label{thm: lower_bd_log_reg} Consider a Gaussian  sensing matrix $\vecA$ where each entry is chosen iid $N(0,1)$.
For a uniformly chosen $k$-sparse vector $\bx$, an algorithm $\phi$ satisfies $$\bbP\inp{\phi(\vecA, \bw) \neq \bx}\leq \delta$$ for the problem of $\logreg$ only if the number of measurements $$m\geq \frac{1}{2}\inp{k+\frac{1}{\beta^2}}\log\inp{\frac{n}{k}}\inp{1 - \frac{h_2(\delta) + \delta k\log{n}}{k\log{n/k}}}.$$ 
\end{corollary}



Theorem~\ref{thm: lower_bdglm} also implies an information theoretic lower bound for \spl, which is presented below and proved in Appendix~\ref{proof:sec:lower_bd}. Note that the denominator term in the bound $\frac{1}{2}\log\inp{1+\frac{k}{\sigma^2}}$ is the capacity of a Gaussian channel with power constraint $k$ and noise variance $\sigma^2$. 
\begin{corollary}[\spl\ lower bound]\label{thm: spl_lower_bd_1}
Under the average power constraint \eqref{eq:power_constraint} on  $\vecA$, for a uniformly chosen $k$-sparse vector $\bx$, an algorithm $\phi$ satisfies $$\bbP\inp{\phi(\vecA, {\by}) \neq \bx}\leq \delta$$ only if the number of measurements
$$m\geq \frac{k\log\inp{\frac{n}{k}}-\inp{h_2(\delta) + \delta k\log{n}}}{\frac{1}{2}\log\inp{1+\frac{k}{\sigma^2}}}.$$
\end{corollary} 

\subsection{Tighter upper and lower bounds for \spl}\label{sec:tighter_bounds_spl}
We present information theoretic upper and lower bounds for \spl\ in this section. Similar to Section~\ref{sec:alg}, our upper bound is for the maximum probability of error, while the lower bounds hold even for the weaker criterion of average probability of error.

We first present an upper bound based on the maximum likelihood estimator (MLE) where  we  decode to $\hat{\bx}$ if, on output $\by$, 
\begin{align*}
\hat{\bx} = \argmax_{\stackrel{\bx\in \inb{0,1}^n}{\wh{\bx} = k}}\,\, p(\by|{\bx})
\end{align*} where $p(\by|{\bx})$ denotes the probability density function of $\by$ on input $\bx$.
\begin{theorem}[MLE upper bound for \spl]\label{thm:upper_bd_mle} Suppose  entries of the measurement matrix $\vecA$ are i.i.d. $\cN(0,1).$
The MLE  is correct with high probability if 
\begin{align}m\geq \max_{l\in[1:k]}  \frac{nN(l)}{\frac{1}{2}\log\inp{\frac{ l}{2\sigma^2}+1}}\label{eq:upper_bd_mle}
\end{align}where  $N(l):=  \frac{k}{n} h_2\inp{\frac{l}{k}} + (1-\frac{k}{n})h_2\inp{\frac{l}{n-k}}$. 
\end{theorem}
We prove the theorem in Appendix~\ref{proof:MLE}. The main proof idea involves analysing the probability that the output of the MLE is $2l$ Hamming distance away from the unknown signal $\bx$ for different values of $l\in [1:k]$ (assuming $k\leq n/2$). This depends on the number of such vectors (approximately $2^{nN(l)}$) and the probability that the MLE outputs a vector which is $2l$ Hamming distance away from $\bx$. 

Note that when $l = k\inp{1-\frac{k}{n}}$, $nN(l) = nh_2(k/n)\approx k\log{\frac{n}{k}}$ and $\log\inp{\frac{k\inp{1-k/n}}{2\sigma^2}+1}\leq \log\inp{\frac{k}{2\sigma^2}+1}$.
Thus, $m$ is at least $\frac{2k\log{n/k}}{\log\inp{\frac{k}{2\sigma^2}+1}}$ (see the bound for Corollary~\ref{thm: spl_lower_bd_1}). It is not immediately clear if this value of $l= k\inp{1-\frac{k}{n}}$ is the optimizer. However, for large $n$, this appears to be the case numerically as shown in Plot~\ref{plot:1}.

\begin{figure}[t]
\includegraphics[width=7cm]{Unknown2.png}
\centering
\caption{The figure shows the plot of the MLE upper bound \eqref{eq:upper_bd_mle} (given by m1) for different values of $k$. This is displayed in blue color. A plot of $\frac{2nN(l)}{\log\inp{\frac{ l}{2\sigma^2}+1}}$ is also presented for $l = k\inp{1-\frac{k}{n}}$ in orange color, given by m2. A part of the plot is zoomed in to emphasize the closeness between the lines. In these plots,  $\sigma^2$ is set to 1,  $n$ is 50000 and $k$ ranges from 1000 to 25000 $(n/2)$. }\label{plot:1}
\end{figure}


Inspired by the MLE analysis, we derive a lower bound with the same structure as \eqref{eq:upper_bd_mle}. We generate the unknown signal $\bx$ using the following distribution: A vector $\tilde{\bx}$ is chosen uniformly at random from the set of all $k$-sparse vectors. Given $\tilde{\bx}$, the unknown input signal $\bx$ is chosen uniformly from the set of all $k$-sparse vector which are at a Hamming distance $2l$ from $\bx$. 
The lower bound is then obtained by computing upper and lower bounds on $I(\vecA, \by;\bx|\tilde{\bx})$.
We show this lower bound only for random matrices where each entry is chosen iid $\cN(0,1)$.
\begin{theorem}[\spl\ lower bound]\label{thm:lower_bd_spl}
If each entry of $\vecA$ is chosen iid $\cN(0,1)$, then for a uniformly chosen $k$-sparse vector $\bx$, an algorithm $\phi$ satisfies 
\begin{align}
    \bbP\inp{\phi(\vecA, {\by}) \neq \bx}\leq \delta\label{eq:spl_lower_bd_l}
\end{align}  only if the number of measurements $$m\geq \max_l\frac{nN(l) - 2\log{n}- h_2(\delta) - \delta k\log{n}}{\frac{1}{2}\log\inp{1+\frac{l}{\sigma^2}\inp{2-\frac{l}{k}}}} .$$
\end{theorem} The proof of Theorem~\ref{thm:lower_bd_spl} is given in Appendix~\ref{proof:MLE}.

If we choose $l = k\inp{1-\frac{k}{n}}$ in Theorem~\ref{thm:lower_bd_spl}, we recover corollary~\ref{thm: spl_lower_bd_1} for the special case of Gaussian design.
% \begin{corollary}\label{corollary2:lower_bd_spl}
% If  each entry of $\vecA$ is chosen iid $\cN(0,1)$, then for a uniformly chosen $k$-sparse vector $\bx$, an algorithm $\phi$ satisfies 
% $$\bbP\inp{\phi(\vecA, {\by}) \neq \bx}\leq \delta$$
% only if the number of measurements 
% $$m\geq \frac{k\log\inp{\frac{n}{k}} - 2\log{n}- h_2(\delta) - \delta k\log{n}}{\log\inp{1+\frac{k}{\sigma^2}}} .$$
% \end{corollary}

% Corollary~\ref{corollary2:lower_bd_spl} can also be proved directly for any sensing matrix $\vecA$ which satisfies \eqref{eq:power_constraint} (non-necessarily a Gaussian design). 


% \begin{figure}[t]
% \includegraphics[width=8cm]{plot.png}
% \centering
% \caption{The figure shows the plot of the MLE upper bound \eqref{eq:upper_bd_mle} (given by m1) for different values of $n$. This is displayed in blue color. A plot of $\frac{2nN(l)}{\log\inp{\frac{ l}{2\sigma^2}+1}}$ is also presented for $l = k\inp{1-\frac{k}{n}}$ in orange color, given by m2. In these plots,  $\sigma^2$ is set to 1 and $k$ is $0.2n$. }\label{plot:1}
% \end{figure}


\section{Solvable Instances} 
\label{sec: solvable}
In Sections~\ref{sec: upper bound} and~\ref{sec: lower bound unquantized}, we provided nearly matching upper and lower bounds for instances with positive gap.
In this section, we study the ``(un)solvabality'' of bandit instances with zero gap, and show that essentially all bandit instances that are ``solvable'' have positive gap, as long as parameter~$c$ is large enough (see Remark~\ref{rem: picking large enough c}).
To formalize this idea, we define the following class of bandit instances.



\begin{definition}[Solvable instances]
\label{def:solvable}
     Let $\A, \epsilon,$ and $q$ be fixed. 
     We say that an instance $\nu \in \cE$ is $\epsilon$-\emph{solvable} if for each $\delta \in (0, 1)$, there exists an algorithm that is $(\epsilon, \delta)$-reliable and it holds under instance $\nu$ that\footnote{We could require that $\PP_{\nu}[\tau < \infty] = 1$ in this case and the subsequent analysis and conclusions would be essentially unchanged.  Recall also that $\PP_{\nu}[\cdot]$ denotes probability under instance $\nu$.}
    \begin{equation}
        \PP_{\nu}[\tau < \infty \cap \hat{k}\in\Ac_{\epsilon}]\ge1-\delta.
    \end{equation}
    If no such algorithm exists, we say that $\nu$ is $\epsilon$-\emph{unsolvable}.
\end{definition}

\begin{remark}
\label{rem: solvable inclusion}
    Fix $0 < \epsilon_1 \le \epsilon_2$. If an instance $\nu$ is
    $\epsilon_1$-solvable, then it is $\epsilon_2$-solvable.
    This follows directly from $\A_{\epsilon_1}(\nu) \subseteq \A_{\epsilon_2}(\nu)$.
\end{remark}

From Corollary~\ref{cor: combined guarantee}, we deduce that any instance with a positive gap is solvable. 
 
\begin{corollary}[Positive gap is solvable]
\label{cor: positive gap is solvable}
    Let $\A, \lambda, \epsilon, q,$ and $c$ be fixed. 
    Suppose an instance $\nu$ satisfies
    $ \Delta > 0$, where $\Delta =  \Delta(\nu, \lambda, \epsilon, c, q) $ is as defined in Theorem~\ref{theorem: upper bound}.
    Then $\nu$ is $\epsilon$-solvable. 
\end{corollary}
The main result of this section is that the reverse inclusion nearly holds, in the following sense.
 \begin{theorem}[Zero gap is unsolvable]
 \label{thm: zero gap is unsolvable}
    Let $\lambda, \epsilon, c,$ and $q$ be fixed, 
    and let $\tilde{\epsilon} = \tilde{\epsilon}(\lambda, \epsilon, c)$ be as defined in 
    Algorithm~\ref{alg: main}.
    Suppose an instance $\nu \in \cE$ satisfies $\Delta(\nu, \lambda, \epsilon, c, q) = 0$.
    If we assume for $\nu$ that there exists some sufficiently small $\eta_0 > 0$ such that 
    $0 \le Q_k^+(q-\eta_0) \le Q_k(q+\eta_0) \le \lambda$, then $\nu$ is $c\tilde{\epsilon}$-unsolvable.
 \end{theorem}
\begin{proof}
    See Appendix~\ref{sec: appendix solvable instance}.
\end{proof}


\begin{remark}[Removing the additional assumption]
\label{rem: remove additional assumption}
    The additional assumption involving $\eta_0$ 
    % in Theorem~\ref{thm: zero gap is unsolvable} and Corollary~\ref{cor: zero gap is unsolvable} 
    is mild; it is trivially satisfied by instances with all reward distributions supported on $[0, \lambda]$, and also holds significantly more generally since $\eta_0$ can be arbitrarily small.
    % The additional assumption is added to simplify the proof of Lemma~\ref{lem:two_instances}. Specifically, it helps ensuring both $(G_k)^{-1}(q)$ and $(G_a)^{-1}(q)$ as defined in \eqref{eq: shifted q quantiles} are in $[0, \lambda]$ so that the constructed instance $\nu'$ satisfies $\nu' \in \cE$.
    Moreover, in Appendix~\ref{sec: assumption removal}, we show that
    this assumption is unnecessary if we use the modified gap (see Remark~\ref{rem: further improvement}) instead of $\Delta$.
    % and Definition~\ref{def: modified gap}) in Theorem~\ref{thm: zero gap is unsolvable}: if the modified gap is zero for an instance~$\nu$, then $\nu$
    % is $c\tilde{\epsilon}$-unsolvable.
\end{remark}


 
\begin{remark}
\label{rem: picking large enough c}
 For each $\theta \in (0, 1)$, picking $c = \lceil 2\theta / (1-\theta)\rceil$ yields
\begin{equation}
       \nu \text{ is } \theta\epsilon\text{-solvable} 
    \implies
    \nu \text{ is } c\tilde{\epsilon}\text{-solvable}
    \implies
     \Delta(\nu, \lambda, \epsilon, c, q) > 0
     \implies 
     \nu \text{ is } \epsilon\text{-solvable},
\end{equation}
where the last two implications follow from Theorem~\ref{thm: zero gap is unsolvable} and Corollary~\ref{cor: positive gap is solvable}, and the first implication follows from Remark~\ref{rem: solvable inclusion} and the following inequality:
\begin{equation}
    c \tilde{\epsilon} 
    = 
    \frac{c \lambda}{ \left\lceil (c+1) \lambda/\epsilon \right\rceil}
    \ge
     % \frac{c \lambda}{  (c+1) \lambda/\epsilon  + \lambda/\epsilon }
     % =
     \frac{c \lambda}{  (c+2) \lambda / \epsilon }
     = 
     \left(1 - \frac{2}{c+2} \right) \epsilon
     \ge
     \left(1 - \frac{2}{\frac{2\theta+2-2\theta}{1-\theta}} \right) \epsilon
     = \theta \epsilon.
\end{equation} 
Since $\theta$ can be arbitrarily close to $1$,
% $\lim\limits_{c \to \infty} c \tilde{\epsilon} = \epsilon$, 
we have $\Delta(\nu, \lambda, \epsilon, c, q) > 0$
    % we can use Algorithm~\ref{alg: main} to solve 
     for essentially all $\epsilon$-solvable instances by picking a sufficiently large $c$.
 \end{remark}

The proof of Theorem~\ref{thm: zero gap is unsolvable} will turn out to directly extend to a ``limiting'' version in which we replace $c\tilde{\epsilon}$ by $\lim\limits_{c \to \infty} c\tilde{\epsilon} = \epsilon$ and $\Delta(\nu, \lambda, \epsilon, c, q)$ by $\lim\limits_{c \to \infty} \Delta(\nu, \lambda, \epsilon, c, q)$, giving the following corollary.


 \begin{corollary}
 \label{cor: zero gap is unsolvable}
    Let $\lambda, \epsilon$, and $q$ be fixed.
    Let $\Delta_{k}(\nu, \epsilon, q)$ be the gap defined in Definition~\ref{def: our gap} with $c \to \infty$ (see~\eqref{eq: gap k infinite c} for the explicit form).    
    Suppose an instance $\nu \in \cE$ satisfies $\Delta(\nu, \epsilon, q) = \max\limits_{k \in \A_{\epsilon(\nu)}} \Delta_k(\nu, \epsilon, q) = 0$.
    If we assume for $\nu$ that there exists some sufficiently small $\eta_0 > 0$ such that 
    $0 \le Q_k^+(q-\eta_0) \le Q_k(q+\eta_0) \le \lambda$, then $\nu$ is $\epsilon$-unsolvable.
 \end{corollary}
 \begin{proof}
    See Appendix~\ref{sec: appendix solvable instance}.
\end{proof}


\acks{This work was supported by the Singapore National Research Foundation under its AI Visiting Professorship programme.}


\newpage
\bibliography{bibliography}

\newpage
\appendix

\section{Quantile Estimation Subroutine}
\label{sec: appendix QuantEst}

\subsection{Noisy Binary Search}
\label{sec: appendix MNBS reformulation}
We first momentarily depart from MAB and discuss the monotonic noisy binary search (MNBS) problem of~\cite{karp2007noisy}; see also the end of Appendix~\ref{sec: appendix quant est related work} for a summary of some related work on noisy binary search.
The original problem formulation was stated in terms of finding a special coin $i$ among $n$ coins, but this can be restated as follows: 
We have a random variable~$R$ with an unknown CDF $F$ and a list of $n$ points $x_1  \le \cdots \le x_n$ such that $Q(\tau) \in [x_1, x_n]$, and the goal is to find an index $i$  satisfying 
\begin{equation}
\label{eq: MNBS guarantee}
    [F(x_i), F(x_{i+1})] \cap  (\tau - \Delta, \tau + \Delta) \ne 
    \varnothing
\end{equation}
via adaptive queries of the form $\mathbf{1}(R \le x_j)$. Note that each query $\mathbf{1}(R \le x_j)$ is an independent Bernoulli random variable with parameter $F(x_j)$.
We will make use of the following main result from~\cite[Theorem 1.1]{gretta2023sharp}.

\begin{proposition}[Noisy binary search guarantee]
\label{prop: MNBS guarantee}
For any $\delta \in (0, 1)$ and relaxation parameter $\Delta \le \min(\tau, 1-\tau)$, the MNBS algorithm in \cite{gretta2023sharp}
output an index $i$ after at most
$O \big( \frac{1}{\Delta^2}  \log  \frac{n}{\delta} \big)$
queries\footnote{\label{footnote: constants MNBS}The expression for the number of iterations in \cite{gretta2023sharp} is more complicated because it has some terms with explicit constant factors, but in $O(\cdot)$ notation it simplifies to $O \big( \frac{1}{\Delta^2}  \log  \frac{n}{\delta} \big)$. We do not specify the exact number of loops in Algorithm~\ref{alg: quantile interval}, as doing so is somewhat cumbersome and the focus of our work is on the scaling laws.} and $i$ satisfies~\eqref{eq: MNBS guarantee} with probability at least $1- \delta$. 
\end{proposition}
The bulk of the MNBS algorithm in \cite{gretta2023sharp} is based on Bayesian multiplicative weight updates: Start with a uniform prior over which of the $n$ intervals crosses quantile $\tau$, make the query at $x_j$ whose $F(x_j)$ is nearest to $\tau$ under current distribution, update the posterior by multiplying intervals on one side of the query by $1 + c \Delta$ and the other side by $1 - c \Delta$ for some fixed constant $c$, and repeat. Other MNBS algorithms such as those in \cite{karp2007noisy}, or even a naive binary search with repetitions (see \cite[\S 1.2]{karp2007noisy}), could also be used to solve the MNBS problem, but we choose  \cite{gretta2023sharp} since it has the best known scaling of the query complexity. Further comparisons of the relevant theoretical guarantees and practical performance can be found in \cite{gretta2023sharp}.


\subsection{Quantile Estimation with 1-bit Feedback}
\label{sec: appendix quant est related work}
The MNBS algorithm can be implemented under our 1-bit communication-constrained setup. Specifically, the learner decides which arm $k$ to query as well as the point $x_j$  to query, and then sends a threshold
query ``Is $R_k \le x_j$?'' as side information to the agent, where $R_k$ is the random reward (variable) of the arm~$k$ with CDF $F_k$. The agent will then pull arm $k$ and reply with a 1-bit binary feedback corresponding to the observation. Note that 
while the $O \big( \frac{1}{\Delta^2}  \log  \frac{n}{\delta} \big)$ queries for a given arm are done in an adaptive manner, the queries themselves can be requested at different time steps without any requirement of agent memory. 
A high-level description of the implementation for a fixed arm is given in Algorithm~\ref{alg: quantile interval}. This gives us the following guarantee, which is a simple consequence of Proposition~\ref{prop: MNBS guarantee}.




\begin{algorithm}
    \caption{Communication-constrained quantile estimation subroutine ($\mathtt{QuantEst}$ in
Algorithm~\ref{alg: main})}
    \label{alg: quantile interval}

   \textbf{Input}: 
    Arm with reward $R$ distributed according to CDF $F$,
    a list $\mathbf{x}$ of $n$ points $x_1 \le  \cdots \le x_n$,
    quantile $\tau \in (0, 1)$ satisfying $Q(\tau) \in [x_1, x_n]$,
    approximation parameter~$\Delta \le \min(\tau, 1-\tau)$,
    error probability $\delta \in (0,1)$
    
    \textbf{Output} Index $i \in \{1, \dots, n-1\}$    

\begin{algorithmic}[1]
      
        \For{$t = 1$ to $t_{\max}$ (with\footref{footnote: constants MNBS} $t_{\max} = O \big( \frac{1}{\Delta^2}  \log  \frac{n}{\delta} \big)$)}

          \State \textbf{At Learner:}
          
          \State~~~~Pick index $j$
          % $j \in \{1,  \ldots, n-1\}$
          according to Bayesian weight update as in~\cite{gretta2023sharp}
          
          \State~~~~Send threshold query 
          ``Is $R \le x_j$?'' to the agent

            \State \textbf{At Agent:}
          
          \State~~~~Pull arm and observe reward $r$
          
          \State~~~~Send 1-bit feedback $\mathbf{1}(r \le x_j)$ to the learner

        
        \EndFor

    \State Return index $i$  according to~\cite{gretta2023sharp}

\end{algorithmic}    
\end{algorithm}


\begin{corollary}[$\mathtt{QuantEst}$ guarantee]
\label{cor: QuantEst guarantee}
Let $(F, \mathbf{x}, \tau, \Delta, \delta)$ be a valid input of Algorithm~\ref{alg: quantile interval}, and let~$n$ be the number of points in $\mathbf{x}$. 
Then the algorithm outputs  an index $i$ after at most
$O \big( \frac{1}{\Delta^2}  \log  \frac{n}{\delta} \big)$
queries and $i$  satisfies
    $\mathbb{P}
    \left([F(x_i), F(x_{i+1})] \cap  (\tau - \Delta, \tau + \Delta) = \varnothing \right) < \delta$.
\end{corollary}


\textbf{Related work on noisy binary search and quantile estimation.}
We briefly recap the original MNBS problem in \cite{karp2007noisy, gretta2023sharp}:
There are $n$ coins whose unknown probabilities $p_j \in [0, 1]$ are sorted in nondecreasing order, where flipping coin $j$ results in head with probability $p_j$. The goal is to identify a coin $i$ such that the interval $[p_i, p_{i+1}]$ has a nonempty intersection with $(\tau - \Delta, \tau + \Delta)$. This model subsumes noisy binary search with a fixed noise level \cite{burnashev1974interval, ben2008bayesian, dereniowski2021noisy, gu2023optimal} (where $p_j = \frac{1}{2} - \Delta$ for $j \leq i$ and $p_j = \frac{1}{2} + \Delta$ otherwise) as well as regular binary search (where $p_j \in \{0, 1\}$). As we discussed in Appendix~\ref{sec: appendix MNBS reformulation}, this problem can be reformulated into the problem of estimating (the quantile of) a distribution using threshold/comparison queries, where the noise in the feedback is stochastic.  This quantile estimation problem has been generalized to a non-stochastic noise setting \cite{meister2021learning, okoroafor2023non}, and was also studied in the context of online dynamic pricing and auctions \cite{kleinberg2003value, leme2023pricing, leme2023description}. 
In particular, \cite[Algorithm~1]{leme2023pricing} is similar to Algorithm~\ref{alg: quantile interval} (or equivalently subroutine $\mathtt{QuantEst}$ used on Lines~\ref{ltk def} and~\ref{utk def} of Algorithm~\ref{alg: main}), in the sense that both use noisy binary search to identify the quantile of a \textit{single} distribution. However, they use the naive binary search with repetitions to form confidence intervals containing the quantile, which has a suboptimal complexity $O \big( \frac{1}{\Delta^2} \log n \log  \frac{\log n}{\delta} \big)$; see \cite[\S 1.2]{karp2007noisy} for details.  Overall, while ideas from the existing literature on quantile estimation of a \textit{single} distribution with threshold queries may provide useful context, they do not readily translate into Algorithm~\ref{alg: main} or the analysis that led to our main contributions.








\subsection{Proof of Lemma~\ref{lem: good events} (Bounding the Probability of Event E) }
\label{sec: proof event E}
\begin{proof}[Proof of Theorem~\ref{lem: good events}]
    For a fixed $t \ge 1$ and a fixed $k \in \mathcal{A}_t$,
    we have
    \begin{equation}
        \Pr{
        \overline{E_{t, k, l}}
        }
        \le  \frac{\delta \cdot \Delta^{(t)}}{2 |\mathcal{A}_t|}
        \quad
        \text{and}
        \quad
         \Pr{
        \overline{E_{t, k, u}}
        }
        \le  \frac{\delta \cdot \Delta^{(t)} }{2 |\mathcal{A}_t|}
        \quad
    \end{equation}
    by the guarantee of the $\mathtt{QuantEst}$ (see Corollary~\ref{cor: QuantEst guarantee}).
    Applying the union bound, we obtain
    \begin{equation}
        \Pr{\overline{E}}
        \le 
        \sum_{t \ge 1}
        \sum_{k \in \mathcal{A}_t} 
        \frac{\delta \cdot \Delta^{(t)}}{ |\mathcal{A}_t|}
        \le 
        \sum_{t \ge 1}
        |\mathcal{A}_t| \cdot 
        \frac{\delta \cdot \Delta^{(t)}}{ |\mathcal{A}_t|}
        =
        \delta
         \sum_{t \ge 1}
         \Delta^{(t)}
         =
         \delta
         \sum_{t \ge 1}
         2^{-t} 
         \le \delta
    \end{equation}
    as desired. The number of arm pulls~\eqref{eq: QuantEst arm pulls} follows immediately from the guarantee of $\mathtt{QuantEst}$ from Corollary \ref{cor: QuantEst guarantee}, $|\A_t| \le |\A| = K$, 
    and the number of points $n = \Theta(c\lambda/\epsilon)$.
\end{proof}
\section{Proof of Lemma~\ref{lem: quantile anytime bound} (Anytime Quantile Bounds)}
\label{sec: appendix anytime quantile bounds}
We first present a useful auxiliary lemma.
\begin{lemma}
    Under the setup of Lemma~\ref{lem:  quantile anytime bound} (including Event $E$ from Lemma~\ref{lem: good events} holding), we have the following bounds:
    \begin{equation}
    \label{eq: xltk upper bound}
        x_{l_{t, k}} < Q_k(q)
    \end{equation}
    \begin{equation}
    \label{eq: xltk+1 lower bound}
        Q^+_k\big( q -  \Delta^{(t)} \big)
        \le x_{l_{t, k} + 1}
    \end{equation}
    \begin{equation}
    \label{eq: xutk upper bound}
        x_{u_{t, k}} < Q_k\big( q +  \Delta^{(t)} \big)
    \end{equation}
    \begin{equation}
    \label{eq: xutk+1 lower bound}
        Q^+_k(q)
        \le x_{u_{t, k} + 1}
    \end{equation}
    for each round $t \ge  1$ and arm $k \in \A_t$.
\end{lemma}
\begin{proof}
We will prove only~\eqref{eq: xltk upper bound} and~\eqref{eq: xltk+1 lower bound}
for an arbitrary $t  \ge 1$ and  $k \in \A_t$
in detail, as~\eqref{eq: xutk upper bound} and~\eqref{eq: xutk+1 lower bound} can be proved similarly.
Observe that, under event $E_{t,k,l} \subset E$ (see \eqref{eq: event Etkl}), we have 
\begin{equation}
\label{eq: Fk_xltk bound}
    F_k(x_{l_{t, k}}) < q
    \quad \text{and} \quad
    q -  \Delta^{(t)} < F_k(x_{l_{t, k}+1})
\end{equation}
respectively,
as otherwise the interval $[F_k(x_{l_{t, k}}), F_k(x_{l_{t, k}+1})]$
would fall on the right and the left, respectively, of the interval $\left( q - \Delta^{(t)}, q   \right)$. A similar argument through the event $E_{t,k,u} \subset E$ (see \eqref{eq: event Etku}) yields
\begin{equation}
\label{eq: Fk_xutk bound}
    F_k(x_{u_{t, k}}) < q + \Delta^{(t)}
    \quad \text{and} \quad
    q  < F_k(x_{u_{t, k}+1}).
\end{equation}

We now prove~\eqref{eq: xltk upper bound} using~\eqref{eq: Fk_xltk bound}; the inequality~\eqref{eq: xutk upper bound} can be proved similarly through~\eqref{eq: Fk_xutk bound}. If $x_{l_{t, k}} = - \infty$, then~\eqref{eq: xltk upper bound} holds trivially.
Therefore, we proceed on the assumption that $x_{l_{t, k}} \in \R$.
Then, using standard properties of quantile functions (see, e.g., \cite[4.3 Theorem]{dufour1995distribution}), we have $x_{l_{t, k}} < Q_k(q)$ as desired.

We now prove~\eqref{eq: xltk+1 lower bound} using~\eqref{eq: Fk_xltk bound}; the inequality~\eqref{eq: xutk+1 lower bound} can be proved similarly through~\eqref{eq: Fk_xutk bound}. 
If $x_{l_{t, k}+1} = \infty$, then~\eqref{eq: xltk+1 lower bound} holds trivially.
Therefore, we proceed on the assumption that $x_{l_{t, k}+1} \in \R$.
In this case, it is a finite upper bound on the values in the set
    $\{ z \in \R: F_k(z) \le q -  \Delta^{(t)}  \}$, and so this set has a finite supremum. It follows that 
    \begin{equation}
        x_{l_{t, k}+1} \ge 
        \sup \{ z \in \R : F_k(z) \le q -  \Delta^{(t)}  \} =
        Q^+_k\big( q -  \Delta^{(t)} \big)
    \end{equation}
    as desired.
\end{proof}


\begin{proof}[Proof of Lemma~\ref{lem:  quantile anytime bound}]
We break down the bounds into  inequalities as follows:
\begin{multicols}{2}
\begin{enumerate}[label=(\roman*)]

    \item  $\mathrm{LCB}_{\tau}(k) \le \mathrm{LCB}_t(k)$
    
    \item $\mathrm{LCB}_t(k)  < Q_k(q)$
    
    % \item $Q_k(q) \le  Q^+_k(q)$
    
    \item $Q_k(q) \le \mathrm{UCB}_t(k)$
    
    \item $\mathrm{UCB}_t(k)  \le \mathrm{UCB}_{\tau}(k)$
    
    \item $Q^+_k\big(q -  \Delta^{(t)} \big)  \le \mathrm{LCB}_t(k) + \tilde{\epsilon}$

    \item $\mathrm{UCB}_t(k) - \tilde{\epsilon}
        % \le x_{u_{t, k} } 
        <
         Q_k\big(q + \Delta^{(t)} \big)$
\end{enumerate}
\end{multicols}
We will prove only inequalities (i), (ii), and (iv)
for an arbitrary $t  > \tau \ge 0$ and  $k \in \A_t$
in detail, as all the other inequalities can be proved similarly.

Inequality (i) follows immediately from Line~\ref{LCB definition} of Algorithm~\ref{alg: main} and induction. Likewise, we can show (iv) using Line~\ref{UCB definition} of Algorithm~\ref{alg: main}.
 
We now show inequality (ii) by induction on $t$;
inequality (iii) can be proved similarly.
    For the base case $t = 1,$
    we have
    \begin{equation}
        \mathrm{LCB}_1(k) =
        \max \left( x_{l_{t, k}}, \mathrm{LCB}_{0}(k) \right) =
        \max \left( x_{l_{t, k}}, 0 \right) =
        x_{l_{t, k}} < Q_k(q),
    \end{equation}
    where the last inequality follows from~\eqref{eq: xltk upper bound}.   
    For the inductive step, suppose that $\mathrm{LCB}_t(k) < Q_k(q)$ for a fixed $t \ge 1$. Since $x_{l_{t, k}} < Q_k(q)$, we have
    \begin{equation}
        \mathrm{LCB}_{t+1}(k) =
            \max
            \left(
            x_{l_{t, k}},
            \mathrm{LCB}_{t}(k)
            \right)
            < Q_k(q)
    \end{equation}
    as desired.
    

    We now show inequality (v) using~\eqref{eq: xltk+1 lower bound}; inequality (vi)
         can be shown using a similar argument through~\eqref{eq: xutk upper bound}.
        We consider three cases for the index $l_{t, k}$:
        \begin{itemize}
            \item ($l_{t, k} = 0$) In this case, we have 
                    $x_{l_{t, k} + 1} = x_{1} =  0 = \mathrm{LCB}_0(k)$, and so
                    \begin{equation}
                       Q^+_k\big( q -  \Delta^{(t)} \big)
                        \le 
                        x_{l_{t, k} + 1} =
                        \mathrm{LCB}_0(k)
                        \le 
                        \mathrm{LCB}_t(k)
                        < \mathrm{LCB}_t(k) + \tilde{\epsilon},
                    \end{equation}
                    where the first inequality follows from~\eqref{eq: xltk+1 lower bound} and the second inequality follows from inequality~(i).
            
            \item ($1 \le l_{t, k} \le n$) In this case, we have
                    \begin{equation}
                   Q^+_k\big( q -  \Delta^{(t)} \big)
                    \le x_{l_{t, k} + 1}
                    = x_{l_{t, k}} + \tilde{\epsilon}
                    \le \mathrm{LCB}_t(k) + \tilde{\epsilon},
                \end{equation}
                where the first inequality follows from~\eqref{eq: xltk+1 lower bound}, the equality follows from distance between
                consecutive points set in Line~\ref{line: list of points} of Algorithm~\ref{alg: main}, 
                and the last inequality follows from Line~\ref{LCB definition} of Algorithm~\ref{alg: main}.

           \item  ($l_{t, k} = n+1$) In this case, we have  $x_{l_{t, k}} = x_{n+1} = \lambda \ge Q_k(q) \ge  Q^+_k\big( q -  \Delta^{(t)} \big)$,
           and so
                    \begin{equation}
                       Q^+_k\big( q -  \Delta^{(t)} \big) \le 
                        x_{l_{t, k}}  \le 
                        \mathrm{LCB}_t(k)
                        < \mathrm{LCB}_t(k) + \tilde{\epsilon},
                    \end{equation}
                    where the second inequality follows from Line~\ref{LCB definition} of Algorithm~\ref{alg: main}.
        \end{itemize}
    Combining all three cases, we have
    $Q^+_k\big( q -  \Delta^{(t)} \big)
        \le \mathrm{LCB}_t(k) + \tilde{\epsilon}$ as desired.
\end{proof}



\section{Proof of Theorem~\ref{thm: correctness} (Reliability of Algorithm~\ref{alg: main})}
\label{sec: appendix correctness}
\begin{proof}[Proof of Theorem~\ref{thm: correctness}]
    We first show by induction that an optimal arm $k^*$ of instance $\nu$ (i.e., one having the highest $ q$-quantile) will always be active, i.e., $k^* \in \A_t$ for each round $t \ge 1$. 
        For the base case $t = 1$, we have $k^* \in \{1, \dots, K\} = \A_1$ trivially. We now show the inductive step: if $k^* \in \A_t$ holds, then $k^* \in \A_{t+1}$. For all arms $a \in \A_t$, we have
    \begin{equation}
        \mathrm{UCB}_t(k^*)
        \ge
        Q_{k^*}(q) 
        \ge Q_{a}(q)  
        > 
        \mathrm{LCB}_t(a),
    \end{equation}
    where the second inequality follows from the optimality of  arm $k^*$,
    while the other two inequalities follow from the anytime quantile bounds (Lemma~\ref{lem: quantile anytime bound}).
    It follows that
    $\mathrm{UCB}_t(k^*) >
            \max\limits_{a \in \mathcal{A}_{t}} \mathrm{LCB}_t(a)$,
    and so $k^* \in \A_{t+1}$ by definition (see Line~\ref{line: active arm} of Algorithm~\ref{alg: main}).


    We now argue that if Algorithm~\ref{alg: main} terminates, then the returned arm~$\hat{k}$ satisfies~\eqref{def: performance def}. 
    If Algorithm~\ref{alg: main} terminates,
    then the while-loop (Lines~\ref{line: start while loop}--\ref{line: end while loop}) must have terminated and therefore the returned arm $\hat{k}$
    satisfies the condition
    \begin{equation}
    \label{eq: k condition}
    \mathrm{LCB}_t(\hat{k})  \ge
                \max\limits_{a \in \A_t \setminus \{ \hat{k} \} }  
                \mathrm{UCB}_t(a) -  (c+1)\tilde{\epsilon}
                 \ge
          \max\limits_{a \in \A_t \setminus \{ \hat{k} \} }  
                \mathrm{UCB}_t(a)  - \epsilon,
    \end{equation}
    where the second inequality follows from Lines~\ref{line: number of points}--\ref{line: tilde epsilon} of Algorithm~\ref{alg: main}: $\tilde{\epsilon} \le \lambda \cdot \epsilon/((c+1) \lambda) = \epsilon/(c+1)$.
    If $\hat{k} = k^*$, then the returned arm satisfies~\eqref{def: performance def} trivially. Therefore, we assume that $\hat{k} \ne k^*$ for the rest of the proof. In this case, we have
    \begin{equation}
        Q_{\hat{k}}(q) >
        \mathrm{LCB}_t(\hat{k})  \ge
        \max\limits_{a \in \A_t \setminus \{ \hat{k} \} } 
        \mathrm{UCB}_t(a) -  
        \epsilon
        \ge 
        \mathrm{UCB}_t(k^*) -  \epsilon
        \ge
        \max\limits_{a \in \A_t \setminus \{ \hat{k} \} }
        Q_a(q) -  \epsilon.
    \end{equation}
    where the first and the last inequalities follow from the anytime quantile bounds (see Lemma~\ref{lem: quantile anytime bound}), while the second inequality follows from the condition~\eqref{eq: k condition} and the third inequality follows from $k^* \in \A_t$ (see above) and the assumption that $\hat{k} \ne k^*$.
\end{proof}
\section{Details on Remark~\ref{rem: gap generalization} (Comparison to Existing Gap Definitions)}
\label{sec: appendix gap definition generalization}


We first recall some existing arm gap definitions for the exact quantile bandit problem (i.e., $\epsilon = 0$) in the setting of unquantized rewards.
In \cite[Definition 2]{nikolakakis2021quantile}, the authors defined the gap $\Delta_k^{\mathrm{NKSS}} $ for each suboptimal arm $k \ne k^*$ by
\begin{equation}
\label{eq: gap NKSS}
    \Delta_k^{\mathrm{NKSS}} \coloneqq
     \sup
    \{
         \Delta \in \left[0, \min(q, 1-q) \right]
        : 
        Q_k(q + \Delta)
        \le
        Q_{k^*}(q - \Delta)
    \}.
\end{equation}
While the authors did not define the arm gap for $k^*$, we can take it to be the same as the gap of the ``best'' suboptimal arm, as their algorithm terminates only when all suboptimal arms are eliminated.
On the other hand, the arm gap defined in
\cite[(Eq. (27)]{howard2022sequential}
is given by
\begin{equation}
\label{eq: gap HR}
    \Delta_k^{\mathrm{HR}} \coloneqq
    \begin{cases}
     \sup
    \{
         \Delta \in \left[0, \min(q, 1-q) \right]
        : 
        Q_k(q + \Delta)
        \le
        \max\limits_{a \in \A}
        Q_a(q - \Delta)
    \}
    & \text{if } k \ne k^*
    \\
     \sup
    \{
        \Delta 
        \in \left[0, q \right] :
        Q_k(q - \Delta)
        \ge
        \max\limits_{a \neq k}
        Q_{a}\big(q + \Delta_a^{\mathrm{HR}}
        \big)
    \}
    & \text{if } k = k^* 
    \end{cases}.
\end{equation}
Similar to our arm gap definition (Definition~\ref{def: our gap}), the gaps $\Delta_k^{\mathrm{HR}}$ for suboptimal arms $k \ne k^*$ are not defined based on the quantile function of $k^*$. It follows that $\Delta_a^{\mathrm{HR}} \ge \Delta_a^{\mathrm{NKSS}}$ for all arms $a \in \A$.

We now study the effect of taking $c \to \infty$ in our gap, which is given below in~\eqref{eq: gap k infinite c}. From~\eqref{eq: gap k infinite c}, it is straightforward to verify that~\eqref{eq: gap HR} is recovered from our gap (Definition~\ref{def: our gap}) by using only lower quantile functions and taking $S = \A$ and $c \to \infty$. 

\textbf{Effect of parameter $c$ in the gap definition.}
For any $1 \le c_1 \le c_2$, let 
\begin{equation}
\label{eq: tilde eps 1 and 2}
    \tilde{\epsilon_1} = \frac{\lambda} {\left\lceil (c_1+1) \lambda/\epsilon \right\rceil}
    \quad 
    \text{and}
    \quad
    \tilde{\epsilon_2} =  \frac{\lambda} {\left\lceil (c_2+1) \lambda/\epsilon \right\rceil}
\end{equation} 
be as defined 
using Lines~\ref{line: number of points}--\ref{line: tilde epsilon} of 
in Algorithm~\ref{alg: main}. It can readily be verified that 
\begin{equation}
\label{eq: c1 tilde eps 1 and 2}
    \tilde{\epsilon_1} \ge \tilde{\epsilon_2}
    \quad \text{and} \quad
    c_1 \tilde{\epsilon_1} \le c_2 \tilde{\epsilon_2} \le \epsilon
    \quad \text{and} \quad
    \Delta_{k}(\nu, \lambda, \epsilon, c_1, q)
\le \Delta_{k}(\nu, \lambda, \epsilon, c_2, q).
\end{equation}
Since $\lim\limits_{c \to \infty}  \tilde{\epsilon} = 0$
and $\lim\limits_{c \to \infty} c \tilde{\epsilon} = \epsilon$,
the gap as defined in Definition~\ref{def: our gap} converges to a quantity $\Delta_{k} \coloneqq \Delta_{k}(\nu, \epsilon, 
    q) = \lim\limits_{c \to \infty}
    \Delta_{k}(\nu, \lambda, \epsilon, c, q)$, given by
\begin{equation}
\begin{aligned}
\label{eq: gap k infinite c}
   \Delta_{k} =
    \begin{cases}
    \sup
    \left\{
        \Delta \in \left[0, \min(q, 1-q) \right]
        \colon
        Q_k(q + \Delta) 
        \le
        \max\limits_{a \in \A  }
        Q^+_{a}(q - \Delta) 
        \right\}
    &\hspace{-2mm} \text{if }  k \not\in \A_{\epsilon} \\
   \max\limits_{\A_{\epsilon} \subseteq S }
   \left\{
        \sup
    \Big\{
        \Delta \in 
       \Big[0, \min\limits_{a \not\in S} \Delta_{a}  \Big]
        % \ \middle\vert\
        \colon
        Q^+_k(q - \Delta) 
        \ge
        \max\limits_{ a \in S \setminus \{k\}} 
        Q_{a}(q + \Delta) - \epsilon
        \Big\}
        \right\}
    &\hspace{-2mm} \text{if } k \in \A_{\epsilon} 
    \end{cases}.
\end{aligned}
\end{equation}
    Note that $\Delta_k$ is independent of $c$ and $\lambda$.
    

\begin{remark}[Use of upper quantile function]
    \label{rem: upper quantile}
    To our knowledge, we are the first to incorporate upper quantile functions in the gap definition.
    This may lead to a potentially larger arm gap as compared to defining using only lower quantile functions (e.g., changing $Q_a^+(\cdot)$ and $Q_k^+(\cdot)$ in~\eqref{eq: our gap} and~\eqref{eq: Delta k^S} to $Q_a(\cdot)$ and $Q_k(\cdot)$ respectively), and hence a tighter upper bound.
\end{remark}

\begin{remark}[Dependency on $Q_{k^*}(q-\Delta)$]
    Existing papers using an elimination-based algorithm have their arm gaps defined according to $Q_{k^*}(q-\Delta)$; see~\eqref{eq: gap NKSS} for an example.
    In contrast, we remove this dependency and define using $\max\limits_{a \in \A  }
        Q^+_{a}(q - \Delta)$, which may lead to a tighter upper bound.  
    The resulting analysis required is more challenging --
    see the discussion in Remark~\ref{rem: elim suboptimal}. 
\end{remark}



Since our gap definitions generalizes existing gap definitions, we expect that their gaps being positive on an instance $\nu$ would imply our gap being positive on $\nu$. That is, their gaps being positive is a sufficient condition for Algorithm~\ref{alg: main} to return a satisfying arm with high-probability (see Corollary~\ref{cor: combined guarantee}).

\begin{proposition}
\label{prop: generalized formulation}
     Fix an instance $\nu \in \cE$ that has a unique arm $k^*$ with the highest $q$-quantile. Let $\Delta_a^{\mathrm{NKSS}}$ and 
$\Delta_a^{\mathrm{HR}}$ be as defined in~\eqref{eq: gap NKSS} and \eqref{eq: gap HR} for each $a \in \A$.
If $\min\limits_{a \in \A} 
\left\{ \Delta_a^{\mathrm{NKSS}} \right\} > 0$
 or
 $\min\limits_{a \in \A} 
\left\{ \Delta_a^{\mathrm{HR}} \right\} > 
 0$, then $\Delta =  \Delta(\nu, \lambda, \epsilon, c, q) $ as defined in Theorem~\ref{theorem: upper bound} is also positive.
\end{proposition}

\begin{proof}
    It suffices to consider the case
    $ \min\limits_{a \in \A}  
    \left\{ \Delta_a^{\mathrm{HR}} \right\} > 0$,
    since $\Delta_a^{\mathrm{HR}} \ge \Delta_a^{\mathrm{NKSS}}$ for all arms $a \in \A$.
    Let $\eta = \frac{1}{2} \min\limits_{a \in \A}  \Delta_a^{\mathrm{HR}} > 0$.
    Then we have
    \begin{equation}
    \label{eq: positive HR implies positive gap}
        Q_{k^*}^+(q - \eta)
        \ge 
         Q_{k^*}(q - \eta)
        \ge 
        \max\limits_{a \neq k}
        Q_{a}\big(q + \Delta_a^{\mathrm{HR}}\big)
        \ge
        \max\limits_{a \in \A \setminus \{k^*\} } Q_a(q + \eta) - c \tilde{\epsilon},
    \end{equation}
    where the second inequality follows from~\eqref{eq: gap HR} 
    and $\tilde{\epsilon} = \tilde{\epsilon}(\lambda, \epsilon, c)$ is as defined in Algorithm~\ref{alg: main}.
    Combining~\eqref{eq: positive HR implies positive gap} and~\eqref{eq: Delta k^S} of our gap definition,  we have
    \begin{equation}
        \max\limits_{a \in \A_{\epsilon}} \Delta_{a} 
        \ge 
        \Delta_{k^*}  = \max\limits_{\A_{\epsilon} \subseteq S \subseteq \A}
        \Delta_{k^*}^{(S)} 
        \ge
        \Delta_{k^*}^{(\A)} \ge \eta > 0
    \end{equation}
    as desired.
\end{proof}



We now show that the converse is not true in general. 
In other words, there exists an instance $\nu \in \cE$ where no algorithm can distinguish which arm has a higher quantile using a finite number of arm pulls (see \cite[Theorem 2]{nikolakakis2021quantile}), but Algorithm~\ref{alg: main} is capable of returning an $\epsilon$-satisfying arm with high probability.

\begin{proposition}
\label{prop: converse zero gap not true}
    Fix $\lambda \ge \epsilon > 0$ and $\delta \in (0, 0.5)$.
    There exists a two-arm bandit instance $\nu \in \cE$ that has a unique arm $k^*$ with the highest median such that
    $\Delta =  \Delta(\nu, \lambda, \epsilon, c, q) $ as defined in Theorem~\ref{theorem: upper bound} is positive for $c \ge 2$,
    but $\min\limits_{a \in \A} \big\{ \Delta_a^{\mathrm{NKSS}}  \big\} 
= \min\limits_{a \in \A} \big\{ \Delta_a^{\mathrm{HR}}  \big\} = 0$.
\end{proposition}

\begin{proof}
    Consider two arms $\A = \{1, 2\}$ with the following CDFs:
\begin{equation}
    F_1(x) = 
    \begin{cases}
        0  & \text{ for } x < 0
        \\
        \frac{x}{2m_1} & \text{ for } 0 \le x < 2m_1 \\
        1 & \text{ for } x \ge 2m_1
    \end{cases}
    \quad \text{and} \quad
    F_2(x) = 
    \begin{cases}
        0 & \text{ for } x < m_2 \\
        0.5 & \text{ for } m_2 \le x < 2m_1 \\
        1 & \text{ for } x \ge 2 m_1
    \end{cases},
\end{equation}
where $ m_2 \in (m_1 - \epsilon/2, m_1)$
such that both arms are $\epsilon$-optimal, with arm 1 being the unique best arm. 
Note that for each $\eta > 0$, we have
\begin{equation}
    Q_2(0.5 + \eta) = 2 m_1
    > m_1 = Q_1(0.5) \ge Q_1(0.5 - \eta),
\end{equation}
and so $\Delta_2^{\mathrm{NKSS}} = \Delta_2^{\mathrm{HR}} = 0$.
However, under our gap definition (Definition~\ref{def: our gap}) with $\A_{\epsilon}(\nu) = \{1, 2\} = \A$ and any $c \ge 2$, we have
\begin{align}
    \Delta \ge \Delta_2 
    \ge \Delta_{2}^{(\{1,2\})}
    &=
    \sup
    \left\{
        \Delta \in [0, 0.5]
        :
        Q^+_2(0.5 - \Delta) 
        \ge
        Q_{1}(0.5 + \Delta) - c\tilde{\epsilon}
        \right\} \\
    &\ge
    \sup
    \left\{
        \Delta  \in [0, 0.5]
        :
        Q^+_2(0.5 - \Delta) 
        \ge
        Q_{1}(0.5 + \Delta) - \frac{\epsilon}{2}
        \right\} \\    
    &=
    \sup
    \left\{
        \Delta  \in [0, 0.5]
        :
        m_2
        \ge
        (1+2 \Delta) m_1 - \frac{\epsilon}{2}
        \right\} \\
     &= \min\left\{0.5, \frac{m_2 - (m_1 - \epsilon/2)}{2m_1} \right\} >0,
\end{align}
where the second inequality follows from the calculation in Remark~\ref{rem: picking large enough c}, and the last inequality follows from the assumptions that $m_1 > 0$ and $m_2 > m_1 - \epsilon/2$.
\end{proof}



\section{Proof of Theorem~\ref{theorem: upper bound} (Upper Bound of Algorithm~\ref{alg: main})}
\label{sec: appendix upper bound}
We break down the upper bound on the number of arm pulls used by Algorithm~\ref{alg: main} as follows. We bound the number of rounds required for a non-satisfying arm $k \not\in \A_{\epsilon}(\nu)$ to be eliminated in Lemma~\ref{lem: elim suboptimal}. Then in Lemma~\ref{lem: termination}, we bound the number of rounds each non-eliminated arm has gone through when the termination condition of the while-loop is triggered. Combining these lemmas with the number of arm pulls used by $\mathtt{QuantEst}$ for each round index $t \ge 1$ and active arm $k \in \A_t$ as stated in~\eqref{eq: QuantEst arm pulls}
gives us an upper bound on the total number of arm pulls.

We first present a useful lemma that will be used in the proofs of the two subsequent lemmas. 

\begin{lemma}[$\max \mathrm{LCB}$ is non-decreasing]
\label{lem: max LCB increasing}
    Under Event $E$ as defined in Lemma~\ref{lem: good events}, we have 
    \begin{equation}
    \max\limits_{a \in \mathcal{A}_{t}} 
            \mathrm{LCB}_{t}(a) \ge 
    \max\limits_{a \in \mathcal{A}_{\tau}} 
            \mathrm{LCB}_{\tau}(a).
    \end{equation}
    for all rounds $t > \tau  \ge 1$.
\end{lemma}
\begin{proof}
    Let round index $\tau \ge 1$ be arbitrary
    and let $k \in \argmax\limits_{a \in \A_{\tau}} 
            \mathrm{LCB}_{\tau}(a).$
    We have $k \in \A_{\tau+1}$ since 
    $\mathrm{UCB}_{\tau}(k) > \mathrm{LCB}_{\tau}(k) 
    = \max\limits_{a \in \A_{\tau}} 
            \mathrm{LCB}_{\tau}(a)$ by~\eqref{eq:  quantile anytime bound} 
    of the anytime quantile bounds.        
    It then follows that
    \begin{equation}
        \max\limits_{a \in \mathcal{A}_{\tau+1}} 
            \mathrm{LCB}_{\tau+1}(a) 
        \ge \mathrm{LCB}_{\tau+1}(k) 
        \ge \mathrm{LCB}_{\tau}(k) = 
        \max\limits_{a \in \A_{\tau}} 
            \mathrm{LCB}_{\tau}(j),
    \end{equation}    
    where the second inequality follows from~\eqref{eq:  quantile anytime bound} 
    of the anytime quantile bounds. 
    Applying the argument repeatedly yields the claim for all $t > \tau.$
\end{proof}

\begin{lemma}[Elimination of non-satisfying arms]
\label{lem: elim suboptimal}
     Fix an instance $\nu \in \cE$, and suppose Algorithm~\ref{alg: main} is run with input $(\A, \lambda, \epsilon, q, \delta)$ and parameter $c \ge 1$.
    Let $\A_{\epsilon} = \A_{\epsilon}(\nu) $ be as defined in~\eqref{def: performance def} and let the gap $\Delta_{k} = \Delta_{k}(\nu, \lambda, \epsilon, c, q)$ be as defined in Definition~\ref{def: our gap} 
    for each arm $k \in \A$.
    Consider an arm $k \not\in \A_{\epsilon}$.
    Under Event $E$ as defined in Lemma~\ref{lem: good events}, when the round index~$t$
    of Algorithm~\ref{alg: main} satisfies $\Delta^{(t)}  \le \frac{1}{2} \Delta_k$, we have  $k \not\in \A_{t+1}$.
\end{lemma}
\begin{proof}
    If $k \not\in \A_t$, then $k \not\in \A_{t+1}$ trivially. Therefore, we assume for the rest of the proof that $k \in \A_t$, and we will show that
    \begin{equation}
    \label{eq: eliminate condition}
        \mathrm{UCB}_t(k) \le \max\limits_{a \in \mathcal{A}_{t}} \mathrm{LCB}_t(a)
    \end{equation}
    or equivalently
    \begin{equation}
    \label{eq: eliminate condition equivalent}
        \mathrm{UCB}_t(k) < \max\limits_{a \in \mathcal{A}_{t}} \mathrm{LCB}_t(a) + \tilde{\epsilon},
    \end{equation}
    where $\tilde{\epsilon} = \tilde{\epsilon}(\lambda, \epsilon, c)$ is as defined in Lines~\ref{line: number of points} and~\ref{line: tilde epsilon} of Algorithm~\ref{alg: main}.
    Note that these conditions are equivalent because both
    $\mathrm{UCB}_t(k)$ and
    $\max\limits_{a \in \A_t } \mathrm{LCB}_t(a)$ 
    are elements of 
    \begin{equation}
        \left[ 0, 
        \tilde{\epsilon}, 
        2\tilde{\epsilon}, \cdots,
        (n-1) \tilde{\epsilon}, \lambda\right],
    \end{equation}
   which follows from Lines~\ref{line: list of points},~\ref{eq: initiate default conf interval}, and~\ref{ltk def}--\ref{UCB definition} of Algorithm~\ref{alg: main}.

    
    Since $k \not\in \A_{\epsilon}$, when the round index $t$ satisfies~$\Delta^{(t)} \le \frac{1}{2} \Delta_k $ we have
    \begin{equation}
    \label{eq: gap k realized with arm j}
        \mathrm{UCB}_t(k)
        < Q_k \big( q + \Delta^{(t)} \big)  + \tilde{\epsilon} 
        \le Q^+_{j}\big(q - \Delta^{(t)} \big) 
    \end{equation}
    for some arm $j \in \A$  by~\eqref{eq: upper approx quantile anytime bound} of the anytime quantile bounds
    and Definition~\ref{def: our gap}. 
    We now consider two cases: (i) $j \in \A_t$ and (ii) $j \not\in \A_t$.

    
    If $j \in \A_t$, we have
    \begin{equation}
    \label{eq: j in At}
        Q^+_{j}\big(q - \Delta^{(t)}\big) 
        \le \mathrm{LCB}_t(j) + \tilde{\epsilon} 
        \le \max\limits_{a \in \mathcal{A}_{t}} \mathrm{LCB}_t(a) + \tilde{\epsilon} 
    \end{equation}
    by~\eqref{eq: lower approx quantile anytime bound} of the anytime quantile bounds and the assumption that $j \in \A_t$. Combining~\eqref{eq: gap k realized with arm j} and~\eqref{eq: j in At} gives us condition~\eqref{eq: eliminate condition equivalent} as desired.
    
    If $j \not\in \A_t$, then it is eliminated at some round $\tau < t$, i.e., 
    $j \in \A_{\tau}$ but $j \not\in \A_{\tau + 1}$.
    By \eqref{eq: quantile anytime bound} of the anytime quantile bounds, the definition of active arm set (Line~\ref{line: active arm} of Algorithm~\ref{alg: main}) applied to $\A_{\tau + 1}$,
    and the fact that max LCB is non-decreasing (Lemma~\ref{lem: max LCB increasing}), 
    we have 
    \begin{equation}
    \label{eq: j not in At}
    Q_{j}(q) 
    \le
    \mathrm{UCB}_{\tau}(j) 
    \le
            \max\limits_{a \in \mathcal{A}_{\tau}} 
            \mathrm{LCB}_{\tau}(a)     
        \le
        \max\limits_{a \in \mathcal{A}_{t}} 
            \mathrm{LCB}_{t}(a). 
    \end{equation}
    Combining~\eqref{eq: gap k realized with arm j}, the trivial inequality 
    $Q^+_{j}\big(q - \Delta^{(t)}\big) 
        \le Q_{j}(q) $, and~\eqref{eq: j not in At} yields
    condition~\eqref{eq: eliminate condition} as desired.
\end{proof}

\begin{remark}
    \label{rem: elim suboptimal}
    As seen in the analysis for the case  $j \not\in \A_t$ above, the property that $\max \mathrm{LCB}$ is non-decreasing (Lemma~\ref{lem: max LCB increasing}) is crucial in establishing~\eqref{eq: j not in At}. We will see below that the same argument is used again in establishing~\eqref{eq: LCB_t(k) > Fj}. This property of Lemma~\ref{lem: max LCB increasing} itself is a consequence of ensuring $\mathrm{LCB}_t(k)$ is non-decreasing in $t$; see Remark~\ref{rem: LCB non decreasing}.
\end{remark}



\begin{lemma}[While-loop termination]
\label{lem: termination}
     Fix an instance $\nu \in \cE$, and suppose Algorithm~\ref{alg: main} is run with input $(\A, \lambda, \epsilon, q, \delta)$ and parameter $c \ge 1$.
    Let $\A_{\epsilon} = \A_{\epsilon}(\nu) $ be as defined in~\eqref{def: performance def} and let the gap $\Delta_{k} = \Delta_{k}(\nu, \lambda, \epsilon, c, q)$ be as defined in Definition~\ref{def: our gap} 
    for each arm $k \in \A$.
    Under Event $E$, when the round index~$t$
    of Algorithm~\ref{alg: main} satisfies $\Delta^{(t)} \le \frac{1}{2} \max \limits_{a \in \A_{\epsilon}} \Delta_a$, Algorithm~\ref{alg: main} will terminate in round $t+1$.
\end{lemma}

\begin{proof}
     If $\A_{t+1} = \{ k^* \}$, then
      \begin{equation}
          \max\limits_{a \in \A_{t+1} \setminus \{k^*\} }                 \mathrm{UCB}_t(a) - (c+1)\tilde{\epsilon}
      = -\infty \le \mathrm{LCB}_{t}(k^*),
      \end{equation}
     and so the algorithm will terminate and return arm $k^*$ in round $t+1$.
     Therefore, we assume for the rest of the proof that
     there exists another arm $a \ne k^*$ such that $a \in \A_{t+1}$. 

     We first show that the following condition is sufficient to trigger the termination condition of the while-loop (Lines~\ref{line: start while loop}--\ref{line: end while loop}) of Algorithm~\ref{alg: main}: There exists an arm $k \in \A_{t+1}$
     satisfying
    \begin{equation}
    \label{eq: suf cond trigger termination}
          \mathrm{LCB}_t(k)  
          \ge
          \max\limits_{a \in \A_{t+1} \setminus \{k\} }
        Q_{a}\big(q + \Delta^{(t)}\big) -  (c+1)\tilde{\epsilon} .
    \end{equation}
    Using~\eqref{eq: upper approx quantile anytime bound} of the anytime quantile bound, 
    condition~\eqref{eq: suf cond trigger termination} implies that
    \begin{equation}
    \label{eq: termination condition strict equality}
        \mathrm{LCB}_t(k)  
        >
          \max\limits_{a \in \mathcal{A}_{t+1} \setminus \{k\} } \mathrm{UCB}_t(a)
          - (c+2)\tilde{\epsilon},
    \end{equation}
    which is equivalent to the termination condition
    \begin{equation}
          \mathrm{LCB}_t(k)  
            \ge
          \max\limits_{a \in \mathcal{A}_{t+1} \setminus \{k\} } \mathrm{UCB}_t(a) - (c+1)\tilde{\epsilon},
    \end{equation}
    where the equivalence follows from an argument similar to the equivalence between~\eqref{eq: eliminate condition} and 
        \eqref{eq: eliminate condition equivalent}.

    It remains to pick an arm $k \in \A_{t+1}$ satisfying condition~\eqref{eq: suf cond trigger termination}.
    Let arm $j \in \argmax\limits_{a \in \A_{\epsilon}} \Delta_a$ and consider the following two cases: (i) $j \in \A_{t+1}$ and (ii) $j \not\in \A_{t+1}$.

    If $j \in \A_{t+1}$, we pick $k = j$. We also pick~$
    T \in \argmax\limits_{\A_{\epsilon} \subseteq S \subseteq \A}
        \Delta_{k}^{(S)}    
    $ 
    to be the set associated to $\Delta_k$ (see Definition~\ref{def: our gap}).
    Note that every arm that is not in $T$
    is a non-satisfying arm since 
    $\A_{\epsilon} \subseteq T$.
    Furthermore, every non-satisfying arm that is not in $T$, hence every arm that is not in $T$, is eliminated,
    which follows from Lemma~\ref{lem: elim suboptimal} and
    \begin{equation}
        \Delta^{(t)} 
        \le \frac{1}{2} \max \limits_{a \in \A_{\epsilon}} \Delta_a
        = 
        \frac{1}{2} \Delta_k \le \frac{1}{2} \min\limits_{a \not\in T} \Delta_a,
    \end{equation} 
    where the last inequality follows from applying~\eqref{eq: Delta k^S} to $k$ and $T$.
    Therefore, we have
    $\A_{t+1} \subseteq T$.
    It follows that
    \begin{align}
         \mathrm{LCB}_t(k)
          &\ge
          Q^+_k\big(q - \Delta^{(t)}\big) - \tilde{\epsilon} \\
          &\ge
        \max\limits_{a \in T \setminus \{k\} }
        Q_{a}\big(q + \Delta^{(t)}\big) - (c+1)\tilde{\epsilon} \\
        &\ge
      \max\limits_{a \in \A_{t+1} \setminus \{k\} }
        Q_{a}\big(q + \Delta^{(t)}\big) - (c+1)\tilde{\epsilon},
    \end{align}
    where the first inequality follows from~\eqref{eq: lower approx quantile anytime bound} of the anytime quantile bound, the second inequality follows from applying~\eqref{eq: Delta k^S} to $k$ and $T$,
    and the last inequality follows from  $\A_{t+1} \subseteq T$.

    If $j \not\in \A_{t+1}$,
    we pick an arm 
    $k \in \argmax\limits_{a \in \mathcal{A}_{t+1}} 
    \mathrm{LCB}_{t}(a)$ arbitrarily.
    We also pick $T \in \argmax\limits_{\A_{\epsilon} \subseteq S \subseteq \A} \Delta_{k}^{(S)}$ and  
    we have $\A_{t+1} \subseteq T$ as in the case above.
    Furthermore, since $j \not\in \A_{t+1}$,
    we have
    \begin{equation}
    \label{eq: LCB_t(k) > Fj}
        Q^+_j \big(q - \Delta^{(t)}\big)
    \le 
    Q_{j}(q) 
    \le
    \max\limits_{a \in \mathcal{A}_{t+1}} 
    \mathrm{LCB}_{t}(a) 
    = \mathrm{LCB}_{t}(k),
    \end{equation}
    where the second inequality follows from an argument similar to~\eqref{eq: j not in At}.
    It follows that
    \begin{align}
         \mathrm{LCB}_t(k) 
        &\ge Q^+_j \big(q - \Delta^{(t)}\big)  \\
        &\ge
        \max\limits_{a \in T \setminus \{j\} }
        Q_{a}\big(q + \Delta^{(t)}\big)   - c \tilde{\epsilon} \\
        &\ge
        \max\limits_{a \in \A_{t+1} \setminus \{k\} }
        Q_{a}\big(q + \Delta^{(t)}\big) - (c+1) \tilde{\epsilon},
    \end{align}
    where the first inequality follows from~\eqref{eq: LCB_t(k) > Fj}, the second inequality follows from applying~\eqref{eq: Delta k^S} to $j$ and $T$,
    and the last inequality follows from  $\A_{t+1} \subseteq T$.
\end{proof}

\section{Lower bound}


In this section, we provide the proofs of the two lower bounds.

\subsection{White noise case (Theorem~\ref{lower bound white noise})}\label{appendix subsection white noise loss}
We start by the representation of our loss function as a quadratic form.
\begin{proposition} \label{prop white noise loss}
    In the white noise case, the correlation factor $\rho$ is null. The loss $\mathcal{L}_\text{time}(c, d)$ writes 
    \[
    \mathcal{L}_\text{time}(c, d)= 1 + \sum_{k=0}^{+\infty}\vert c_k\vert^2 - 2\textnormal{Re}\big(\sum_{k=0}^{+\infty}c_kd_k\big),
    \]
    where $c_k=\sum_{s=1}^Sa_s^kb_s$. Therefore, the loss writes 
    \[
    \mathcal{L}_\text{time}(c, d) = 1 + \sum_{s, s'}^S\frac{b_s\bar{b}_{s'}}{1-a_s\bar{a}_{s'}} - 2\textnormal{Re}\big(\sum_{s=1}^Sb_sa_s^K\big).
    \]
\end{proposition}

\begin{proof}
    On the one hand, 
    \begin{align*}
        \sum_{k=0}^{+\infty}\vert c_k\vert^2 &= \sum_{k=0}^{+\infty}\big\vert\sum_{s=1}^Sa_s^kb_s\big\vert 
        =\sum_{k=0}^{+\infty}\sum_{s=1}^S\sum_{s'=1}^Sa_s^k\bar{a}_{s'}^kb_sb_{s'} \\
        &= \sum_{s=1}^S\sum_{s'=1}^Sb_sb_{s'}\sum_{k=0}^{+\infty}a_s^k\bar{a}_{s'}^k =\sum_{s=1}^S\sum_{s'=1}^Sb_sb_{s'}\frac{1}{1-a_s\bar{a}_{s'}}.
    \end{align*}
    On the other hand,
    \begin{align*}
        \textnormal{Re}\big(\sum_{k=0}^{+\infty}c_kd_k\big) &= c_Kd_K
        =\sum_{s=1}^Sb_sa_s^K.
    \end{align*}
    Hence the result.
\end{proof}

\begin{proposition}[Performance criterion]
    Minimizing the loss in Proposition \ref{prop white noise loss} boils down to maximizing the following performance criterion
    \[
    F_K = \sum_{s,s'=1}^S\bar{a}_s^K(C^{-1})_{ss'}a_{s'}^K,
    \]
    where $C_{ss'} = \frac{1}{1-a_s\bar{a}_{s'}}$.
\end{proposition}
\begin{proof}
    The loss $\mathcal{L}_\text{time}$ writes 
    \[
    1 + \langle\bar{b}, C\bar{b}\rangle - \langle\bar{b}, a^K\rangle -\langle a^K, \bar{b}\rangle.
    \]
    We thus want to maximize with respect to $a_s$ and $b_s$ the quantity
    \[
    \langle\bar{b}, a^K\rangle +\langle a^K, \bar{b}\rangle - \langle\bar{b}, C\bar{b}\rangle.
    \]
    This is convex and quadratic with respect to $b$, and the minimizer $\bar{b}^*$ is $C^{-1}a^K$, leading to the performance criterion
    \[
    F_K = \langle a^K, C^{-1}a^K\rangle = \sum_{s, s'=1}^S\bar{a}_s^K(C^{-1})_{ss'}a_{s'}^K.
    \]
\end{proof}

We can now move to the proof of Theorem \ref{lower bound white noise}, by first analyzing properties of the matrix $C$.

\paragraph{Linear algebra preview.} We use the similarities with Cauchy matrices and their so-called displacement structure~\citep{yang2003generalized,calvetti1996solution}.

Starting from
$$
C - \Diag( {a}) C \Diag(\bar{a})  = 1_S 1_S^\top,
$$
we get by  post multiplying by $\Diag(\bar{a})^{-1}$,
$$
C\Diag(\bar{a})^{-1}  - \Diag( {a}) C  = 1_S 1_S^\top\Diag(\bar{a})^{-1}
$$
and thus, by pre and post multiplying by $C^{-1}$:
$$
\Diag(\bar{a})^{-1}C^{-1}  - C^{-1}\Diag( {a})    = C^{-1}1_S 1_S^\top\Diag(\bar{a})^{-1}C^{-1},
$$
leading to
$$
\Diag(\bar{a})^{-1}C^{-1}\Diag( {a}) ^{-1}   - C^{-1} = C^{-1}1_S 1_S^\top\Diag(\bar{a})^{-1}C^{-1}\Diag( {a})^{-1}
= u v^*,
$$
with $ u = C^{-1} 1_S$ and $v = \Diag( \bar{a})^{-1}  C^{-1} \Diag(a)^{-1} 1_S$. This leads to a closed form expression for the inverse:
$$
(C^{-1})_{ss'} \big( \frac{1}{ \bar{a}_s a_{s'}}-1 \big) = u_s \bar{v}_{s'}. 
$$
We get
$$
v-u = \big[ \Diag( \bar{a})^{-1}  C^{-1} \Diag(a)^{-1} - C^{-1} \big] 1_S
= u v^\ast 1_S = u 1_S^\top\Diag(\bar{a})^{-1}C^{-1}\Diag( {a})^{-1}1_S,
$$
which leads to $ \ds v  = u ( 1 +1_S^\top\Diag(\bar{a})^{-1}C^{-1}\Diag( {a})^{-1}1_S )$. Moreover we can write
\BEAS
1_S^\top\Diag(\bar{a})^{-1}C^{-1}\Diag( {a})^{-1}1_S 
& = & 1_S^\top ( C^{-1} + u v^\ast) 1_S=  1_S^\top ( C^{-1} + C^{-1} 1_S v^\ast) 1_S\\
& = & 1_S^\top C^{-1} 1_S \cdot (1 + v^\ast 1_S )
\\
& = &  1_S^\top C^{-1} 1_S \cdot (1 + 1_S^\top\Diag(\bar{a})^{-1}C^{-1}\Diag( {a})^{-1}1_S ) , \EEAS
which leads to
$1_S^\top\Diag(\bar{a})^{-1}C^{-1}\Diag( {a})^{-1}1_S   = \frac{ 1_S^\top C^{-1} 1_S }{1- 1_S^\top C^{-1} 1_S }$, and thus
 $$ 1 +1_S^\top\Diag(\bar{a})^{-1}C^{-1}\Diag( {a})^{-1}1_S  = \frac{1}{1-1_S^\top C^{-1} 1_S} =
\frac{1}{1 - u^\top 1_S}.$$
Moreover, we have for any $z \in \mathbb{C}$, if all $a_s$ are distinct:
$$
\sum_{s'=1}^S \frac{ u_{s'}}{1 - z \bar{a}_{s'}} = 1 - \prod_{s'=1}^S \bar{a}_{s'} \prod_{s'=1}^S \frac{ a_{s'} - z}{1 - z \bar{a}_{s'}}
$$
(the two rational functions have the same degrees, the same poles and are equal for $z=a_1,\dots,a_S$),
which leads to for $z=0$,
$$
\sum_{s'=1}^S  { u_{s'}}=1_S^\top C^{-1} 1_S = \sum_{s'=1}^S   u_{s'} = 1 - \prod_{s'=1}^S | {a}_{s'}|^2,
$$
and thus $\ds 1 - 1_S^\top C^{-1} 1 = \prod_{s'=1}^S | {a}_{s'}|^2$.

We have, if $|z|=1$,
$$
\Big|\prod_{s'=1}^S \frac{ a_{s'} - z}{1 - z \bar{a}_{s'}}\Big|
= 1,
$$
which will be used in the bound (such expressions are typically referred to as Blaschke products~\citep{baratchart2016minimax}, and are known to have unit magnitude).


\paragraph{Proof of the lower bound (by upper bounding $F_K$).}
We have, using our linear algebra preview,
$$
F_K = \langle a^K, C^{-1} a^K \rangle
= \sum_{s,s'=1}^S \bar{a}_s^K (C^{-1})_{ss'} a_{s'}^K
= \sum_{s,s'=1}^S (\bar{a}_s  a_{s'})^{K+1} \frac{u_s \bar{v}_{s'} }{1 -  \bar{a}_s a_{s'}}.
$$
We get, using our linear algebra results,
$$
F_K - F_{K+1} = 
\sum_{s,s'=1}^S (\bar{a}_s  a_{s'})^{K+1} (1 -  \bar{a}_s a_{s'})\frac{u_s \bar{v}_{s'} }{1 -  \bar{a}_s a_{s'}}
=  \frac{1}{\prod_{s'=1}^S | {a}_{s'}|^2} \Big| \sum_{s=1}^S \bar{a}_s^{K+1}  u_s \Big|^2.
$$
This leads to 
\BEAS
F_K & = &  \sum_{L=K}^{+\infty}
( F_L - F_{L+1}) = 
\sum_{L=K+1}^{+\infty} \Big| \sum_{s=1}^S \bar{a}_s^L  u_s \Big|^2   \frac{1}{\prod_{s'=1}^S | {a}_{s'}|^2} 
.\EEAS
We have:
\BEAS
\sum_{L=K+1}^{+\infty} \Big| \sum_{s=1}^S \bar{a}_s^L  u_s \Big|^2 
& \leqslant &  \frac{1}{K+1} 
\sum_{L=0}^{+\infty} L \Big| \sum_{s'=1}^S \bar{a}_{s'}^L  u_{s'} \Big|^2 \mbox{ since } 1_{L \geqslant K+1} \leqslant \frac{L}{K+1}.
 \EEAS
 We consider the sequence $\ds w_L =  \sum_{s=1}^S \bar{a}_s^L  u_s$, with Fourier series
 $$
 W(\omega) = \sum_{L=0}^{+\infty} w_L e^{-i \omega L} 
 =  \sum_{s=1}^S \frac {u_s}{1-\bar{a}_s e^{-i \omega }}   =
  1 - \prod_{s'=1}^S \bar{a}_{s'} \prod_{s'=1}^S \frac{ a_{s'} - e^{-i\omega}}{1 -  e^{-i\omega} \bar{a}_{s'}}.
 $$
 We then use Proposition \ref{proposition semi parseval} to write:
 $$
 \sum_{L = 0 }^{+\infty}
 L | w_L|^2 =   \frac{i}{2\pi} \int_0^{2\pi} W'(\omega) \overline{W(\omega)}d\omega
, $$
 leading to
 \BEAS
&&\sum_{L=K+1}^{+\infty} \Big| \sum_{s=1}^S \bar{a}_s^L  u_s \Big|^2\\
& \leqslant &  \frac{1}{K+1} 
  \frac{i}{2\pi   }  \int_0^{2\pi} 
\frac{d}{d\omega} \Big[    - \prod_{s=1}^S \bar{a}_{s} \prod_{s=1}^S \frac{ a_{s} - e^{i\omega}}{1 - e^{i\omega} \bar{a}_{s}}
 \Big]
\overline{ \Big(
 1 - \prod_{s'=1}^S \bar{a}_{s'} \prod_{s'=1}^S \frac{ a_{s'} - e^{i\omega}}{1 - e^{i\omega} \bar{a}_{s'}}
 \Big)}
      d\omega
\\
& = &  \frac{1}{K+1} 
  \frac{i}{2\pi   }  \int_0^{2\pi} 
\frac{d}{d\omega} \Big[      \prod_{s=1}^S \bar{a}_{s} \prod_{s=1}^S \frac{ a_{s} - e^{i\omega}}{1 - e^{i\omega} \bar{a}_{s}}
 \Big]
\overline{ \Big(
    \prod_{s'=1}^S \bar{a}_{s'} \prod_{s'=1}^S \frac{ a_{s'} - e^{i\omega}}{1 - e^{i\omega} \bar{a}_{s'}}
 \Big)}
      d\omega.
\\
      \EEAS
      We now have, by taking derivatives of the product:
     \BEAS
     \frac{d}{d\omega} \Big[       \prod_{s=1}^S \frac{ a_{s} - e^{i\omega}}{1 - e^{i\omega} \bar{a}_{s}}
 \Big] & = & 
  \prod_{s=1}^S \frac{ a_{s} - e^{i\omega}}{1 - e^{i\omega} \bar{a}_{s}}
  \sum_{s=1}^S \frac{1 - e^{i\omega} \bar{a}_{s}}{ a_{s} - e^{i\omega}} 
   \frac{d}{d\omega} \Big[      \frac{ a_{s} - e^{i\omega}}{1 - e^{i\omega} \bar{a}_{s}}\Big]
\\
 & = & 
  \prod_{s=1}^S \frac{ a_{s} - e^{i\omega}}{1 - e^{i\omega} \bar{a}_{s}}
  \sum_{s=1}^S \frac{1 - e^{i\omega} \bar{a}_{s}}{ a_{s} - e^{i\omega}} 
   \frac{d}{d\omega} \Big[     \frac{1}{\bar{a}_s} +   \frac{a_s - \frac{1}{\bar{a}_s}}{1 - e^{i\omega} \bar{a}_{s}}\Big]
\\
 & = & 
  \prod_{s=1}^S \frac{ a_{s} - e^{i\omega}}{1 - e^{i\omega} \bar{a}_{s}}
  \sum_{s=1}^S \frac{1 - e^{i\omega} \bar{a}_{s}}{ a_{s} - e^{i\omega}} 
  \Big[
(1-|a_s|^2) \frac{ -i e^{i \omega}}{ (1 - e^{i\omega} \bar{a}_{s})^2}\Big]
\\
 & = & 
  \prod_{s=1}^S \frac{ a_{s} - e^{i\omega}}{1 - e^{i\omega} \bar{a}_{s}}
  \sum_{s=1}^S (1-|a_s|^2) \frac{-i  }{ |e^{-i\omega}  - \bar{a}_{s}|^2}
. \EEAS 
      This leads to, using the unit magnitude of $\frac{ a_{s} - e^{i\omega}}{1 - e^{i\omega} \bar{a}_{s}}$,
\BEAS
      F_K
      & \leqslant &  \frac{1}{K+1} 
  \frac{1}{2\pi   } 
   \sum_{s=1}^S ( 1- |a_s|^2) 
  \int_0^{2\pi}  \frac{1}{|a_s - e^{i\omega}|^2}
      d\omega =  \frac{S}{K+1},
\EEAS
using an explicit integration $\ds \frac{1}{2\pi   } 
    \int_0^{2\pi}  \frac{1}{|a_s - e^{i\omega}|^2}
      d\omega = \frac{1}{1-|a_s|^2}$.
    
 The approximation error $\mathcal{L}_\text{time}(c, d)$ is  thus 
$
1 - F_K$, 
which leads to the desired result.

\subsection{Autocorrelated case (Theorem~\ref{theorem 
autocorrelated lower bound})}
\label{proof auto}
We follow the same proof technique as for Theorem~\ref{lower bound white noise}, and compute first an explicit expression of the loss, this time, by introducing a new $a_s$, equal to $\rho$, with the introduction of new weights $w_s = b_s a_s / ( a_s - \rho)$ for $s \in \{1,\dots,S\}$, the weight $w_{S+1}$ being determined by the linear constraint.
\begin{lemma}
    In the autocorrelated case ($\rho \neq 0$), $\mathcal{L}_\text{time}(c, d)$ as in Eq.~\eqref{correlated time domain loss} writes 
    \begin{equation}
    1 - 2(1-\rho^2)\textnormal{Re}\big(\sum_{s=1}^{S+1}\frac{w_sa_s^k}{1-a_s\rho}\big) + (1-\rho^2)\sum_{s, s'}^{S+1}\frac{w_s\bar{w}_{s'}}{1-a_s\bar{a}_{s'}},
    \label{appendix constrained autocorrelated loss}
    \end{equation}    
    where $a_{S+1}=\rho$ and the constraint $\sum_{s=1}^{S+1}w_sa_s^{-1}=0$ holds. 
    \end{lemma}
\begin{proof}
   We aim to minimize 
    \[
    \sum_{k, k'}(c_k-d_k)(c_{k'}-d_{k'})\gamma(k-k'),
    \]
    where $\gamma(k-k')=\rho^{\vert k-k'\vert}$.
    Denoting $C(e^{i\omega}), D(e^{i\omega})$ and $\Gamma(e^{i\omega})$ the Fourier transforms of $(c_n), (d_n)$ and $(\gamma_n)$ respectively, Parseval's theorem yields 
    \[
    \sum_{k, k'}(c_k-d_k)(c_{k'}-d_{k'})\gamma(k-k') = \frac{1}{2\pi}\int_{-\pi}^\pi\big\vert C(e^{i\omega}) - D(e^{i\omega})\big\vert^2\Gamma(e^{i\omega})d\omega.
    \]

    We have $D(e^{i\omega})=e^{-iK\omega}$ (Fourier transform of a shifted Dirac at timestep K), and 
    \begin{align*}
        C(e^{i\omega}) &= \sum_{k=0}^{+\infty}\sum_{s=1}^Sb_sa_s^ke^{-i\omega k} = \sum_{s=1}^S\frac{b_s}{1-a_se^{-i\omega}},\\
        \Gamma(e^{i\omega})&=\sum_{k=-\infty}^{+\infty}\gamma(k)e^{-i\omega k} = \frac{1}{1 - \rho e^{-i\omega}}\frac{1-\rho^2}{1 - \rho e^{i\omega}}.
    \end{align*}
    The criterion becomes (with an error of $1$ if $C=0$):
\BEAS
&&\frac{1}{2\pi} \int_0^{2\pi} | D(e^{i\omega}) - C(e^{i\omega})|^2 \Gamma(e^{i\omega}) d\omega\\
& = & 
\frac{ 1-\rho^2}{2\pi} \int_0^{2\pi} \Big| D(e^{i\omega})\frac{1}{1 - \rho e^{-i\omega}} - C(e^{i\omega}) \frac{1}{1 - \rho e^{-i\omega}}\Big|^2  d\omega \\
& = & 1 - 
\frac{1-\rho^2}{2\pi} 2 {\textnormal{ Re}} \Big(\int_0^{2\pi} 
\overline{D(e^{i\omega})\frac{1}{1 - \rho e^{-i\omega}}}
C(e^{i\omega}) \frac{1}{1 - \rho e^{-i\omega}}
\Big) d\omega   \\
& & \hspace*{2cm} + \frac{1-\rho^2}{2\pi}  \int_0^{2\pi} \Big|C(e^{i\omega}) \frac{1}{1 - \rho e^{-i\omega}}\Big|^2  d\omega.
\EEAS 
We have 
$$
\frac{1}{1- a_se^{-i\omega}}\frac{1}{1 - \rho e^{-i\omega}}
= \frac{1}{a_s-\rho} \Big( \frac{a_s}{1 - a_s e^{-i\omega}} - \frac{\rho}{1 - \rho e^{-i\omega}} \Big),
$$
and thus
\BEAS
C(e^{i\omega}) \frac{1}{1 - \rho e^{-i\omega}}  & = &  \sum_{s=1}^{S}  
\frac{b_s}{a_s-\rho} \Big( \frac{a_s}{1 - a_s e^{-i\omega}} - \frac{\rho}{1 - \rho e^{-i\omega}} \Big) \\
 & = & \sum_{s=1}^{S+1} \frac{w_s  }{1- a_se ^{-i\omega}},
\EEAS
with $w_s = b_s a_s / ( a_s - \rho)$, $a_{S+1} = \rho$, and the constraint $\ds \sum_{s=1}^{S+1} w_sa_s^{-1}  = 0$.
The criterion becomes
\BEAS
& & 1 - 
(1-\rho^2)\sum_{s=1}^{S+1} 2 {\textnormal{Re}} \Big( \frac{w_s a_s^K}{1 - a_s \rho}
 \Big)   +(1-\rho^2) \sum_{s,s'=1}^{S+1} \frac{\bar{w}_s w_{s'}  }{1- a_s \bar{a}_s'},
\EEAS 
after straightforward computations.
\end{proof}

\paragraph{Proof of Theorem \ref{theorem autocorrelated lower bound}.}

The minimum with respect to $w$ in Eq.~\eqref{appendix constrained autocorrelated loss} with the constraint is greater than the unconstrained minimizer, equal to
\BEAS
H_K & =  & 1 - (1-\rho^2)\sum_{s,s'=1}^{S+1} 
\frac{ \bar{a}_s^K}{1 - \bar{a}_s \rho}\frac{ a_{s'}^K}{1 - a_{s'} \rho} (C^{-1})_{ss'},
\EEAS
where we recall that $C_{ss'} = \frac{1}{1-a_s\bar{a}_{s'}}$.

Using linear algebra properties from above with $S+1$ zeros and poles, we get
\BEAS
H_K&= & 1 -  (1-\rho^2)\sum_{s,s'=1}^{S+1}
\frac{ 1}{1 - \bar{a}_s \rho}\frac{1}{1 - a_{s'} \rho}  \frac{(\bar{a}_s a_{s'})^{K+1}u_s \bar{v}_{s'}}{1 - \bar{a}_s a_{s'}} 
\\
& = & 1 - (1-\rho^2)\frac{1}{\prod_{s=1}^{S+1} |a_s|^2 }
\sum_{s,s'=1}^{S+1} 
\frac{ 1}{1 - \bar{a}_s \rho}\frac{1}{1 - a_{s'} \rho}  \frac{(\bar{a}_s a_{s'})^{K+1}u_s \bar{u}_{s'}}{1 - \bar{a}_s a_{s'}} ,
\EEAS
where we recall that $u = C^{-1}1_S$ and $v = \text{Diag}(\bar{a})^{-1}C^{-1}\text{Diag}(a)^{-1}1_S$.

We have
\BEAS
H_{K+1} - H_K 
& = & \frac{1-\rho^2}{\prod_{s=1}^{S+1} |a_s|^2 }
\sum_{s,s'=1}^{S+1} 
\frac{ 1}{1 - \bar{a}_s \rho}\frac{1}{1 - a_{s'} \rho}   (\bar{a}_s a_{s'})^{K+1}u_s \bar{u}_{s'} 
\\
& = & \frac{1-\rho^2}{\prod_{s=1}^{S+1} |a_s|^2 }
\Big| \sum_{s=1}^{S+1}
\frac{ 1}{1 - \bar{a}_s \rho}   \bar{a}_s  ^{K+1}u_s  
\Big|^2,
 \EEAS
 leading to
 \BEAS
 H_K & = & \sum_{L=K}^{+\infty} ( H_L - H_{L+1} ) + 1 \\
 & = & 1 - \frac{1-\rho^2}{\prod_{s=1}^{S+1} |a_s|^2 }
 \sum_{L = K}^{+\infty}
 \Big| \sum_{s=1}^{S+1} 
\frac{ 1}{1 - \bar{a}_s \rho}   \bar{a}_s  ^{L} \bar{a}_su_s  
\Big|^2 \\
& \geqslant & 
1 - \frac{1}{K} \frac{1-\rho^2}{\prod_{s=1}^{S+1} |a_s|^2 }
 \sum_{L = 0 }^{+\infty}
 L\Big| \sum_{s=1}^{S+1}
\frac{ 1}{1 - \bar{a}_s \rho}   \bar{a}_s  ^{L} \bar{a}_s u_s  
\Big|^2 ,
 \EEAS
 using $1_{L \geqslant K} \leqslant \frac{L}{K}$.
 
 The sequence $\ds w_L = \sum_{s=1}^{S+1} 
\frac{ 1}{1 - \bar{a}_s \rho}   \bar{a}_s  ^{L} \bar{a}_su_s  $, has Fourier series
\BEAS
W(\omega) & = &  \sum_{L=0}^{+\infty} w_L e^{-i\omega L}
= \sum_{L=0}^{+\infty}   e^{-i\omega L}\sum_{s=1}^{S+1} 
\frac{ 1}{1 - \bar{a}_s \rho}   \bar{a}_s  ^{L}\bar{a}_s u_s \\
& = & \sum_{s=1}^{S+1}
\frac{ 1}{1 - \bar{a}_s \rho}   \frac{\bar{a}_s u_s}{1 - \bar{a}_s e^{-i\omega}} 
= \sum_{s=1}^{S+1} u_s \Big( 
\frac{ 1}{1 - \bar{a}_s \rho}   - \frac{1}{1 - \bar{a}_s e^{-i\omega}} \Big)\frac{1}{  \rho - e^{-i\omega}} \\
& = & 
\frac{1}{  \rho - e^{-i\omega}} \Big( \prod_{s=1}^{S+1} \bar{a}_s\Big) \Big( \prod_{s=1}^{S+1} \frac{a_s - e^{-i\omega}}{1-e^{-i\omega} \bar{a}_s}
-\prod_{s=1}^{S+1} \frac{a_s - \rho }{1- \rho \bar{a}_s}
\Big)
\\
 & = & 
\frac{1}{  \rho - e^{-i\omega}} \Big(\prod_{s=1}^{S+1} \bar{a}_s \Big) \prod_{s=1}^{S+1} \frac{a_s - e^{-i\omega}}{1-e^{-i\omega} \bar{a}_s},
 \EEAS
 because of the link between $u,C$ and rational functions.
 
We have:
\BEAS
 1 - H_K
 & \leqslant & 
 \frac{1-\rho^2}{K} \frac{1}{\prod_{s=1}^{S+1} |a_s|^2 }
 \sum_{L = 0 }^{+\infty}
 L | w_L|^2 \\
 & = & \frac{1-\rho^2}{K} \frac{1}{\prod_{s=1}^{S+1} |a_s|^2 }
\frac{i}{2\pi} \int_0^{2\pi} W'(\omega) \overline{W(\omega)}d\omega
\ \mbox{ using properties of Fourier Series,} \\
 & = & \frac{1-\rho^2}{K}  
\frac{i}{2\pi} \int_0^{2\pi} \frac{d}{d\omega} \Big(
\frac{1}{  \rho - e^{-i\omega}}   \prod_{s=1}^{S+1} \frac{a_s - e^{-i\omega}}{1-e^{-i\omega} \bar{a}_s}
\Big)
\overline{\frac{1}{  \rho - e^{-i\omega}}   \prod_{s=1}^{S+1} \frac{a_s - e^{-i\omega}}{1-e^{-i\omega} \bar{a}_s}}d\omega
\\
 & = & \frac{1-\rho^2}{K}  
\frac{i}{2\pi} \int_0^{2\pi}  
\frac{-i e^{-i\omega}}{  (\rho - e^{-i\omega})^2}   \prod_{s=1}^{S+1} \frac{a_s - e^{-i\omega}}{1-e^{-i\omega} \bar{a}_s}
\overline{\frac{1}{  \rho - e^{-i\omega}}   \prod_{s=1}^{S+1} \frac{a_s - e^{-i\omega}}{1-e^{-i\omega} \bar{a}_s}}d\omega
\\
& & \hspace*{1cm} + 
\frac{1-\rho^2}{K}  
\frac{i}{2\pi} \int_0^{2\pi} 
\frac{1}{  \rho - e^{-i\omega}} \frac{d}{d\omega} \Big(   \prod_{s=1}^{S+1} \frac{a_s - e^{-i\omega}}{1-e^{-i\omega} \bar{a}_s}
\Big)
\overline{\frac{1}{  \rho - e^{-i\omega}}   \prod_{s=1}^{S+1} \frac{a_s - e^{-i\omega}}{1-e^{-i\omega} \bar{a}_s}}d\omega.
\EEAS
Using the following identities,
\BEAS
 \frac{a_s - e^{-i\omega}}{1-e^{-i\omega} \bar{a}_s}
& = & \frac{1}{\bar{a}_s} + \frac{a_s - 1/ \bar{a}_s}{1-e^{-i\omega} \bar{a}_s}, \\
\frac{d}{d\omega} \Big( \frac{a_s - e^{-i\omega}}{1-e^{-i\omega} \bar{a}_s}
\Big) & = & \frac{a_s - 1/ \bar{a}_s}{(1-e^{-i\omega} \bar{a}_s)^2} \bar{a}_s (-i e^{-i\omega})
=  i e^{-i\omega}\frac{1-|a_s|^2 }{(1-e^{-i\omega} \bar{a}_s)^2}, \\
\Big| \frac{a_s - e^{-i\omega}}{1-e^{-i\omega} \bar{a}_s}
\Big|  & = & 1, \EEAS
we get
\BEAS
 1 - H_K & \leqslant & \frac{1-\rho^2}{K}  
\frac{1}{2\pi} \int_0^{2\pi}  
\frac{e^{-i\omega}}{  (\rho - e^{-i\omega})^2 (\rho - e^{i\omega})}     d\omega
\\
& & \hspace*{4cm} +
\frac{1-\rho^2}{K}  
\sum_{s=1}^{S+1}  ( 1- |a_s|^2) 
\frac{1}{2\pi} \int_0^{2\pi} 
\frac{1}{  |\rho - e^{-i\omega}|^2}  
 \frac{1}{|a_s - e^{-i\omega}|^2} d\omega
\\
& = & \frac{1-\rho^2}{K}  
\frac{1}{2\pi} \int_0^{2\pi}  
\frac{e^{-i\omega}}{  (\rho - e^{-i\omega})^2 (\rho - e^{i\omega})}     d\omega
\\
& & \hspace*{-1cm}
+
\frac{(1-\rho^2)^2}{K}  
\frac{1}{2\pi} \int_0^{2\pi} 
\frac{1}{  |\rho - e^{-i\omega}|^4}  d\omega
 +
\frac{1-\rho^2}{K}  
\sum_{s=1}^{S}  ( 1- |a_s|^2) 
\frac{1}{2\pi} \int_0^{2\pi} 
\frac{1}{  |\rho - e^{-i\omega}|^2}  
 \frac{1}{|a_s - e^{-i\omega}|^2} d\omega
\\
& = & - \frac{1}{K} \frac{1}{1-\rho^2}   +
 \frac{1}{K} \frac{1+\rho^2}{1-\rho^2}   + 
\frac{1-\rho^2}{K}  
\sum_{s=1}^{S}  ( 1- |a_s|^2) 
\frac{1}{2\pi} \int_0^{2\pi} 
\frac{1}{  |\rho - e^{-i\omega}|^2}  
 \frac{1}{|a_s - e^{-i\omega}|^2} d\omega  
 \EEAS
 by exact integration. Then, using $\ds\frac{1}{  |\rho - e^{-i\omega}|^2}   \leqslant \frac{1}{(1-\rho)^2}$, 
 and $\ds\frac{1}{2\pi} \int_0^{2\pi} 
  \frac{1}{|a_s - e^{-i\omega}|^2} d\omega  = \frac{1}{1-|a_s|^2}$, we get
  \BEAS
1-H_K
& \leqslant &  
 \frac{1}{K} \frac{\rho^2}{1-\rho^2}   + 
\frac{1+\rho}{1-\rho}\frac{S}{K}  \leqslant  \frac{1}{K} \frac{\rho}{1-\rho}   + 
\frac{2}{1-\rho}\frac{S}{K}
= \frac{1}{K} \frac{1}{1-\rho} ( \rho + 2 S).
 \EEAS
Thus, we get an approximation error greater than
$\displaystyle
\Big( 1 - \frac{1}{K} \frac{3S}{1-\rho} \Big)_+.
$ (since it is always nonnegative).

\section{Proof of Theorem \ref{thm: zero gap is unsolvable} and Corollary~\ref{cor: zero gap is unsolvable} (Solvable Instances)}
% \section{Solvable Instances}
\label{sec: appendix solvable instance}

We first state a useful lemma for Theorem \ref{thm: zero gap is unsolvable}.

\begin{lemma} 
\label{lem:two_instances}
        Let $\lambda, \epsilon, c,$ and $q$ be given, 
        and let $\tilde{\epsilon} = \tilde{\epsilon}(\lambda, \epsilon, c)$ be as defined in 
        % Lines~\ref{line: number of points} and~\ref{line: tilde epsilon} of 
        Algorithm~\ref{alg: main}.
        Suppose that $\nu \in \cE$ is an instance with gap $\Delta(\nu, \lambda, \epsilon, c, q) = 0 $ and let $\eta_0 = \eta_0(\nu) > 0$ be the constant given in the assumption in Theorem \ref{thm: zero gap is unsolvable}.  
        Then, for each arm $k \in \A_{c \tilde{\epsilon}}(\nu)$ and each $\eta \in  (0, \eta_0) $, there exists another instance $\nu' \in \cE$ satisfying the following:
    \begin{itemize}[topsep=0pt, itemsep=0pt]
        \item There exists an arm $a \in \Ac \setminus \{k\}$ such that instances $\nu$ and $\nu'$ are identical for all arms in $\Ac \setminus \{a,k\}$;
        
        \item $\dTV(F_a,G_a) \le \eta$ and $\dTV(F_k,G_k) \le \eta$, where $F_{(\cdot)}$ and $G_{(\cdot)}$ represent the arm distributions for instances $\nu$ and $\nu'$ respectively;
        
        \item $k \notin \Ac_{c \tilde{\epsilon} }(\nu')$, i.e., 
        under relaxation parameter $c \tilde{\epsilon}$, arm $k$ is not a satisfying arm for instance $\nu'$.
    \end{itemize}
\end{lemma}

     
 \begin{proof}   
    Let $\nu \in \cE$ be an instance with gap $\Delta(\nu, \lambda, \epsilon, c, q) = 0$. For each arm $k \in \A_{\epsilon}(\nu)$, we have $\Delta_{k}^{(\A)} = 0$ by Definition~\ref{def: our gap} since 
    $0 \le \Delta_{k}^{(\A)}  \le \Delta_{k}  \le \Delta  = 0$.
    Applying \eqref{eq: Delta k^S} with set $S = \A$ yields:
     \begin{equation}
     \label{eq: arm a positive eta}
        \text{for each } k \in \A_{\epsilon}(\nu)
        \text{ and each } \eta > 0, 
        \text{ there exists } a \ne k
        \text{ such that }
         Q^+_{k}(q - \eta) 
        <
        % \max\limits_{ a \ne k} 
        Q_{a}(q + \eta) - c\tilde{\epsilon}.
     \end{equation}
    Fix an arm $k \in \A_{c \tilde{\epsilon}}(\nu)$ and $\eta \in (0, \eta_0)$.
    Since $c \tilde{\epsilon} \le \epsilon$ (see calculation in~\eqref{eq: tilde eps 1 and 2}--\eqref{eq: c1 tilde eps 1 and 2}), we have 
    $\A_{c \tilde{\epsilon}}(\nu) \subseteq \A_{\epsilon}(\nu)$, and hence $k \in \A_{\epsilon}(\nu)$. 
    It follows from~\eqref{eq: arm a positive eta} that there exists some arm $a \ne k$ that
    \begin{equation}
     \label{eq: arm a positive eta c tilde epsilon}
         Q^+_{k}(q - \eta) 
        <
        % \max\limits_{ a \ne k} 
         Q_{a}(q + \eta) - c\tilde{\epsilon}.
     \end{equation}
    We now construct instance $\nu'$ such that $\nu$ and $\nu'$
    have identical distributions for all arms in $\A \setminus \{a, k\}$, 
    while $F_a$ and $F_k$ are being replaced with $G_a$ and $G_k$ defined as follows:
    \begin{enumerate}[topsep=0pt, itemsep=0pt]
        \item 
        $G_a$ is any distribution obtained by moving $\eta$-probability mass from the interval $(-\infty, Q_a(q))$ to the point $Q_a(q+2\eta)$; 
        
        \item 
        $G_k$ is any distribution obtained by moving $\eta$-probability mass from the interval $(Q_k(q), \infty)$ to the point $ Q_k(q-2\eta)$.
    \end{enumerate}     
     Under these definitions and the assumption on $\eta_0$ in Theorem~\ref{thm: zero gap is unsolvable}, we can readily verify that
     \begin{equation}
     \label{eq: shifted q quantiles}
         (G_k)^{-1}(q) =  Q_k(q-\eta) 
         % \le Q^+_{k}(q - \eta) 
         \in [0, \lambda]
         \quad 
         \text{and}
         \quad  
         (G_a)^{-1}(q) = Q_a(q+\eta) \in   [0, \lambda]
     \end{equation}
     and
     \begin{equation}
         d_{\mathrm{TV}}(F_k, G_k) =  d_{\mathrm{TV}}(F_a, G_a) = \eta.
     \end{equation}
    
     Finally, combining~\eqref{eq: arm a positive eta c tilde epsilon} and~\eqref{eq: shifted q quantiles} yields
     \begin{equation}
     \label{eq: G_k unsatisfying}
          (G_k)^{-1}(q) 
         < 
         (G_a)^{-1}(q) - c\tilde{\epsilon},
     \end{equation}
     which implies $k \notin \Ac_{c \tilde{\epsilon} }(\nu')$. 
     By construction, $\nu'$ satisfies all three properties as desired.
\end{proof}


\begin{remark}
\label{rem: limit version of two instance lemma}
     We can obtain a ``limiting'' version of Lemma~\ref{lem:two_instances} in which we replace the gap $\Delta(\nu, \lambda, \epsilon, c, q)$ by $\Delta(\nu, \epsilon, q)$ as defined in Corollary~\ref{cor: zero gap is unsolvable} and the satisfying arm set $\A_{c \tilde{\epsilon}}(\cdot)$
     by $\A_{\epsilon}(\cdot)$.
     The proof is essentially identical.
     We construct instance $\nu'$ in a similar manner as above to satisfy the first two properties in the statement of Lemma~\ref{lem:two_instances}.
     The last property $(k \not\in \A_{\epsilon}(\nu'))$ then follows from the definition of the limit gap $\Delta_{k}(\nu, \epsilon, 
    q)$ as defined in~\eqref{eq: gap k infinite c}, which allows us to replace the $c\tilde{\epsilon}$ terms in~\eqref{eq: arm a positive eta},~\eqref{eq: arm a positive eta c tilde epsilon}, and~\eqref{eq: G_k unsatisfying} by $\epsilon$.    
\end{remark}


We proceed to prove Theorem \ref{thm: zero gap is unsolvable}.  
\begin{proof}[Proof of Theorem~\ref{thm: zero gap is unsolvable}]
Assume for contradiction that there exists some instance $\nu \in \cE$ satisfies $\Delta(\nu, \lambda, \epsilon, c, q) = 0 $, but is $c\tilde{\epsilon}$-solvable. 
Fix a $\delta \in (0, 1)$ satisfying
\begin{equation}
    \label{eq: delta very small}
    \delta < \frac{1}{2+2|\A|}.
\end{equation}
By Definition \ref{def:solvable}, there exists a $(c\tilde{\epsilon}, \delta)$-reliable algorithm such that
\begin{equation}
    \PP_{\nu}[\tau < \infty \cap \hat{k} \in \Ac_{c\tilde{\epsilon}}(\nu)] \ge 1-\delta.
\end{equation}
In general the condition $\tau < \infty$ may not imply a \emph{uniform} upper bound on $\tau$; we handle this by relaxing the probability from $1-\delta$ to $1-2\delta$, such that there exists some $\tau_{\max} < \infty$ satisfying
\begin{equation}
    \PP_{\nu}[\hat{k} \in \Ac_{c\tilde{\epsilon}}(\nu) \cap \tau \le \tau_{\max}] \ge 1-2\delta. \label{eq:tau_max}
\end{equation}
From this, we claim that there exists an arm $k_{\nu} \in \Ac_{c\tilde{\epsilon}}(\nu)$ such that
\begin{equation}
    \PP_{\nu}[\hat{k} = k_{\nu} \cap \tau \le \tau_{\max}] \ge \frac{1-2\delta}{|\Ac|}. 
    \label{eq:success_nu}
\end{equation}
Indeed, if this were not the case, then summing these probabilities over elements in $\Ac_{\epsilon}(\nu)$ would produce a total below $1-2\delta$, which would contradict \eqref{eq:tau_max}.


Let $P_{\tau_{\max}}^{(\nu)}$ be the joint distribution on the $|\Ac| \times \tau_{\max}$ matrix of unquantized rewards:
the $(i,j)$-th entry of this matrix contains the $j$-th unquantized reward for arm $i$ under instance $\nu$. Under the event $\tau \le \tau_{\max}$, the algorithm's output does not depend on any rewards beyond those appearing in this matrix.  In other words, the output $\hat{k}$ is a (possibly randomized) function of this matrix.

By picking $\eta > 0$ to be sufficiently small in Lemma \ref{lem:two_instances}, we can find an instance $\nu' \in \cE$ such that $k_{\nu} \notin \Ac_{c\tilde{\epsilon}}(\nu')$ and
\begin{equation}
    \dTV\big( P_{\tau_{\max}}^{(\nu)}, P_{\tau_{\max}}^{(\nu')} \big) \le \delta.
\end{equation}
Here, $P_{\tau_{\max}}^{(\nu')}$ is defined similarly to $P_{\tau_{\max}}^{(\nu)}$, but for instance $\nu'$.
Since the output $\hat{k}$ is a (possibly randomized) function of the matrix defining $P_{\tau_{\max}}^{(\cdot)}$, we have
 \begin{equation}
    \label{eq: DPI}
     \dTV\big(  \PP_{\nu},  \PP_{\nu'} \big) \le \dTV\big( P_{\tau_{\max}}^{(\nu)}, P_{\tau_{\max}}^{(\nu')} \big) \le \delta
 \end{equation}
 by the data processing inequality for $f$-divergence~\cite[Theorem 7.4]{polyanskiy2024information}.
Using the definition $\dTV(P,Q) = \sup_{A} |P(A) - Q(A)|$, and applying~\eqref{eq: DPI},~\eqref{eq:success_nu},~\eqref{eq: delta very small}, we obtain
\begin{equation}
    \PP_{\nu'}[\hat{k} = k_{\nu} \cap \tau \le \tau_{\max}] \ge 
    \PP_{\nu}[\hat{k} = k_{\nu} \cap \tau \le \tau_{\max}] -
    \dTV\big(  \PP_{\nu},  \PP_{\nu'} \big)  
    % &\ge \PP_{\nu}[\hat{k} = k_{\nu} \cap \tau \le \tau_{\max}] -
    % \dTV\big( P_{\tau_{\max}}^{(\nu)}, P_{\tau_{\max}}^{(\nu')} \big)  \\
    \ge
    \frac{1-2\delta}{|\Ac|} - \delta > \delta.
    \label{eq:failure_nu'}
\end{equation}
Since $k_{\nu} \notin \Ac_{c \tilde{\epsilon} }(\nu')$, this means that the algorithm is \emph{not} $(c\tilde{\epsilon}, \delta)$-reliable (see Definition~\ref{def: reliable}), we have arrived at the desired contradiction.
\end{proof}

Corollary~\ref{cor: zero gap is unsolvable} can be proved similarly by using the ``limiting'' version of Lemma~\ref{lem:two_instances} (see Remark~\ref{rem: limit version of two instance lemma}).



\section{Details on Remark~\ref{rem: further improvement} (Improved Gap Definition)}
\label{sec: appendix potential improvement}

\subsection{Modified Arm Gaps}
We first state the modified gap definition explicitly by replacing $Q^+_{(\cdot)}(q - \Delta)$ and $Q_{(\cdot)}(q + \Delta)$ in Definition~\ref{def: our gap}
    with $\max\big\{0, Q^+_{(\cdot)}(q - \Delta)\big\}$ and $\min\big\{\lambda, Q_{(\cdot)}(q + \Delta)\big\}$ respectively, and provide an instance that has a positive modified gap but zero gap under the original definition.
    
\begin{definition}[Modified arm gaps]
\label{def: modified gap}
     Fix an instance $\nu \in \cE$.
     Let $\tilde{\epsilon}$ and $\A_{\epsilon}$ be as in Definition~\ref{def: our gap}.
    For each arm $k \in \A$, we define the improved gap $\tilde{\Delta}_{k} =
    \tilde{\Delta}_{k}(\nu, \lambda, \epsilon, c, q) \in \left[0, \min(q, 1-q) \right]$ as follows: 
    \begin{itemize}
        \item  
        
        If $k \not\in \A_{\epsilon}$, then $\tilde{\Delta}_{k}$ is defined as
        \begin{equation}
            \sup
            \left\{
                \Delta 
                \in \left[0, \min(q, 1-q) \right]
                \colon
               \min\{\lambda, Q_k(q + \Delta)   \}
                \le
                 \max\limits_{a \in \A  }
                 \left\{
                 \max\left\{0,  Q^+_{a}(q - \Delta) \right\}  - \tilde{\epsilon} 
                 \right\}
                \right\}
        \end{equation}

        \item

             If $k \in \A_{\epsilon}$, then we define $\tilde{\Delta}_{k} = \max\limits_{\A_{\epsilon} \subseteq S \subseteq \A}
        \tilde{\Delta}_{k}^{(S)}$, where 
        \begin{equation}
        \label{eq: improved Delta k^S}
            \tilde{\Delta}_{k}^{(S)} =
           \sup
            \Big\{
                \Delta \in 
               \Big[0, \min_{a \not\in S} \tilde{\Delta}_{a}  \Big]
                % \ \middle\vert\
                :
                \max\{0, Q^+_{k}(q - \Delta)\}
                \ge 
                \max\limits_{ a \in S \setminus \{k\}} 
                \min\{\lambda, Q_{a}(q + \Delta)\} - c \tilde{\epsilon}
                \Big\}
        \end{equation}
        for each subset $S$ satisfying $\A_{\epsilon} \subseteq S \subseteq \A$.
                    
    \end{itemize}
We use the convention that the minimum  (resp. maximum) of an empty set is $\infty$ (resp. $- \infty$).
\end{definition}

\begin{remark}[Intuition on the modified arm gap]
    Fix an instance $\nu = (F_k) \in \cE$.
     An interpretation of this modified gap is that
    $\tilde{\Delta}_{k}(\nu, \lambda, \epsilon, c, q) =
    \Delta_{k}(\mathrm{clipped}(\nu), \lambda, \epsilon, c, q)$,
    where $\mathrm{clipped}(\nu) = (\tilde{F}_k) \in \cE$
    is the instance with all distributions supported on $[0, \lambda]$ defined by
    \begin{equation}
        \tilde{F}_k(x)  =
    \begin{cases}
        0 & \text{ for } x < 0 \\
        F_k(x) & \text{ for } 0 \le x < \lambda \\
        1 & \text{ for } x > \lambda 
    \end{cases}
    \quad 
    \text{for each } k \in \A.
    \end{equation}
    That is, $\tilde{F}_k$ is obtained from $F_k$ by moving all mass below 0 to 0, and all mass above $\lambda$ to $\lambda$.  Note that an algorithm could be designed to clip rewards in this way, but our improved upper bound in Theorem \ref{theorem: modified upper bound} below applies even when Algorithm \ref{alg: main} is run without change.
\end{remark}

It is straightforward to verify that the modified gap is at least as large as the unmodified gap (Definition~\ref{def: our gap}), i.e., $\tilde{\Delta} \ge \Delta$. We provide an example of bandit instance that has positive gap under the modified definition but is zero using the unmodified definition. Consider $q = 1/2$, let $\lambda \ge 2 \epsilon > 0$, and consider two arms $\A = \{1, 2\}$ with an identical CDF as follows:
\begin{equation}
    F_1(x) = 
    F_2(x) =
    \begin{cases}
        0 & \text{ for } x < \lambda - \epsilon/3 \\
        0.5 & \text{ for } \lambda - \epsilon/3 \le x < 2 \lambda \\
        1 & \text{ for } x \ge 2 \lambda 
    \end{cases},
\end{equation}
and so both arms are satisfying, i.e., $\A_{\epsilon} = \A$.
Note that for any $\Delta > 0$, we have
\begin{equation}
    Q^+_2(0.5 - \Delta)  =
    \lambda - \epsilon/3 <
    2\lambda - \epsilon \le
     2\lambda - c\tilde{\epsilon} =
    Q_1(0.5 + \Delta) - c\tilde{\epsilon},
\end{equation}
where the second inequality follows from the discussion in~\eqref{eq: tilde eps 1 and 2}--\eqref{eq: c1 tilde eps 1 and 2}. It follows that
\begin{equation}
    \Delta_2 
    = \Delta_{2}^{\A}
    =
    \sup
    \left\{
        \Delta \in [0,0.5]
        :
        Q^+_2(0.5 - \Delta) 
        \ge
        Q_{1}(0.5 + \Delta) - c\tilde{\epsilon}
        \right\} 
    = 0
\end{equation}
under the original gap definition. By symmetry, we also have $\Delta_1 = 0$.
However, under the modified definition, we have
\begin{align}
     \tilde{\Delta}_2 
    = \tilde{\Delta}_{2}^{\A}
    &= \sup
    \left\{
        \Delta \in [0, 0.5]
        :
        \max\{0,  \lambda - \epsilon/2 \}
        \ge
        \min\{\lambda, 2 \lambda\} - c\tilde{\epsilon}
        \right\} \\
    &= \sup
    \left\{
        \Delta \in [0, 0.5]
        :
          \lambda - \epsilon/3 
        \ge
        \lambda  - c\tilde{\epsilon}
        \right\} \\
        &= 0.5,
\end{align}
    where the last inequality follows since 
    $c\tilde{\epsilon} \ge \epsilon/3 $ for any $c \ge 1$ 
    (see the calculation in Remark~\ref{rem: picking large enough c}).

\subsection{Improved Upper Bound}
With the modified gap definition, we obtain the following improved upper bound.

\begin{theorem}[Improved upper bound]
\label{theorem: modified upper bound}
   Fix an instance $\nu \in \cE$, and suppose Algorithm~\ref{alg: main} is run with input $(\A, \lambda, \epsilon, q, \delta)$ and parameter $c \ge 1$.
    Let $\A_{\epsilon}(\nu) $ be as defined in~\eqref{def: performance def} and let the gap $\tilde{\Delta}_{k} = \tilde{\Delta}_{k}(\nu, \lambda, \epsilon, c, q)$ be as defined in Definition~\ref{def: modified gap} 
    for each arm $k \in \A$.
    Under Event~$E$ as defined in Lemma~\ref{lem: good events},
    the total number of arm pulls is upper bounded~by    
    \begin{equation}
        O
        \left(
        \left(
        \sum_{ k \in \A }
        \dfrac{1}{ \max\big( \tilde{\Delta}_{k},  \tilde{\Delta}  \big)^2} \cdot 
        \left( 
         \log \left(\frac{1}{ \delta } \right) +
         \log \left(\frac{1}{ \max\big( \tilde{\Delta}_{k},  \tilde{\Delta}  \big)}\right) +
         \log \left(\frac{c \lambda K}{ \epsilon } \right)    
        \right)
        \right)
        \right),
    \end{equation}
    where $\tilde{\Delta}  =  \tilde{\Delta}(\nu, \lambda, \epsilon, c, q) = \max\limits_{a \in \A_{\epsilon}(\nu)} \tilde{\Delta}_{a}$.
\end{theorem}

The proof is essentially identical to the proof of Theorem~\ref{theorem: upper bound}, but requires tightening of~\eqref{eq: lower approx quantile anytime bound} and~\eqref{eq: upper approx quantile anytime bound}
of anytime quantile bound to
    \begin{equation} 
    \label{eq: modified lower approx quantile anytime bound}
       \max\{0, Q^+_k\big(q -  \Delta^{(t)} \big) \}
        \le \mathrm{LCB}_t(k) + \tilde{\epsilon}
    \end{equation}
    and
    \begin{equation} 
    \label{eq: modified upper approx quantile anytime bound}
        \mathrm{UCB}_t(k) 
        <
         \min\{\lambda, Q_k\big(q + \Delta^{(t)} \big)\} + \tilde{\epsilon}
    \end{equation}
respectively.
Note that the two new bounds \eqref{eq: modified lower approx quantile anytime bound} and~\eqref{eq: modified upper approx quantile anytime bound} can be verified easily using the properties that $ \mathrm{LCB}_t(k) \ge 0$ and $ \mathrm{UCB}_t(k) \le \lambda$ (see Lines~\ref{eq: initiate default conf interval},~\ref{LCB definition}, and~\ref{UCB definition} of Algorithm~\ref{alg: main}), as well as the established bounds~\eqref{eq: lower approx quantile anytime bound} and~\eqref{eq: upper approx quantile anytime bound}.



\subsection{Removing the Assumption in Theorem~\ref{thm: zero gap is unsolvable} (Unsolvability)}
\label{sec: assumption removal}

The assumption involving $\eta_0$ in Theorem~\ref{thm: zero gap is unsolvable} is included to ensure that both $(G_k)^{-1}(q) =  Q_k(q-\eta) $ and $(G_a)^{-1}(q) =  Q_a(q+\eta) $ are in $[0, \lambda]$ in the proof of Lemma~\ref{lem:two_instances}, so that the constructed instance $\nu'$ satisfies $\nu' \in \cE$. As mentioned in Remark~\ref{rem: remove additional assumption}, the assumption can be removed if we use the modified gap instead; formally, we have the following.


 \begin{theorem}[Zero gap is unsolvable -- assumption-free version]
 \label{thm: modified zero gap is unsolvable}
    Let $\lambda, \epsilon, c,$ and $q$ be fixed, 
    and let $\tilde{\epsilon} = \tilde{\epsilon}(\lambda, \epsilon, c)$ be as defined in 
    % Lines~\ref{line: number of points} and~\ref{line: tilde epsilon} of 
    Algorithm~\ref{alg: main}.
    Let $\tilde{\Delta} = \tilde{\Delta}(\nu, \lambda, \epsilon, c, q)$ be as defined in Theorem \ref{theorem: modified upper bound}.    
    If an instance $\nu \in \cE$ satisfies $\tilde{\Delta} = 0 $, then $\nu$ is $c\tilde{\epsilon}$-unsolvable.
 \end{theorem}


  The proof is essentially identical to the proof of Theorem~\ref{thm: zero gap is unsolvable}, and requires only some straightforward modifications in Lemma~\ref{lem:two_instances}. Specifically, under the new gap definition,~\eqref{eq: arm a positive eta c tilde epsilon} would be replaced by
     \begin{equation}
          \max\{0, Q^+_{k}(q - \eta)\}
       <
        \min\{\lambda, Q_{a}(q + \eta)\} - c \tilde{\epsilon}
        \Big\}
     \end{equation}
    We then construct instance $\nu'$ in a similar manner to the proof of Lemma~\ref{lem:two_instances}, but the definitions of $G_a$ and $G_k$ modified to include clipping:
    \begin{enumerate}[topsep=0pt, itemsep=0pt]
        \item 
        $G_a$ is any distribution obtained by moving $\eta$-probability mass from the interval $(-\infty, Q_a(q))$ to the point $\min\{\lambda, Q_{a}(q + 2\eta)\}$;
    
        \item 
        $G_k$ is any distribution obtained by moving $\eta$-probability mass from the interval $(Q_k(q), \infty)$ to the point $\max\{0, Q_{k}(q - 2\eta)\}$.    
    \end{enumerate}     
    It now follows that
     \begin{equation}
         (G_k)^{-1}(q) = \max\{0, Q_{k}(q - \eta)\} 
         \in [0, \lambda]
     \end{equation}
     and
      \begin{equation}
         (G_a)^{-1}(q) =  \min\{\lambda, Q_{a}(q + \eta)\} \in [0, \lambda],
     \end{equation}
     and hence $\nu' \in \cE$ as desired.


\end{document}
