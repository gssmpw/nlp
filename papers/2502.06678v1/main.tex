\documentclass[final,12pt]{alt2025} % Anonymized submission
\input{common-header.sty}


\makeatletter
\newcommand\footnoteref[1]{\protected@xdef\@thefnmark{\ref{#1}}\@footnotemark}
\makeatother

\title[Quantile Multi-Armed Bandits with 1-bit Feedback]{Quantile Multi-Armed Bandits with 1-bit Feedback}

% Two authors with the same address
\altauthor{
    \Name{Ivan Lau} 
    \Email{ivan.lau@u.nus.edu} \\
    % \and
    % \Name{Jonathan Scarlett} 
    % \Email{scarlett@comp.nus.edu.sg}\\
 \addr National University of Singapore \\
 \Name{Jonathan Scarlett} 
    \Email{scarlett@comp.nus.edu.sg}\\
 \addr National University of Singapore
 }



\renewcommand{\cite}{\citep}


\begin{document}
\maketitle

\begin{abstract}
   In this paper, we study a variant of best-arm identification involving elements of risk sensitivity and communication constraints. Specifically, the goal of the learner is to identify the arm with the highest quantile reward, while the communication from an agent (who observes rewards) and the learner (who chooses actions) is restricted to only one bit of feedback per arm pull. We propose an algorithm that utilizes noisy binary search as a subroutine, allowing the learner to estimate quantile rewards through 1-bit feedback. We derive an instance-dependent upper bound on the sample complexity of our algorithm and provide an algorithm-independent lower bound for specific instances, with the two matching to within logarithmic factors under mild conditions, or even to within constant factors in certain low error probability scaling regimes. The lower bound is applicable even in the absence of communication constraints, and thus we conclude that restricting to 1-bit feedback has a minimal impact on the scaling of the sample complexity.
\end{abstract}

\begin{keywords}%
  Best-Arm identification, quantile bandits, 1-bit quantization
\end{keywords}

\section{Introduction}


\begin{figure}[t]
\centering
\includegraphics[width=0.6\columnwidth]{figures/evaluation_desiderata_V5.pdf}
\vspace{-0.5cm}
\caption{\systemName is a platform for conducting realistic evaluations of code LLMs, collecting human preferences of coding models with real users, real tasks, and in realistic environments, aimed at addressing the limitations of existing evaluations.
}
\label{fig:motivation}
\end{figure}

\begin{figure*}[t]
\centering
\includegraphics[width=\textwidth]{figures/system_design_v2.png}
\caption{We introduce \systemName, a VSCode extension to collect human preferences of code directly in a developer's IDE. \systemName enables developers to use code completions from various models. The system comprises a) the interface in the user's IDE which presents paired completions to users (left), b) a sampling strategy that picks model pairs to reduce latency (right, top), and c) a prompting scheme that allows diverse LLMs to perform code completions with high fidelity.
Users can select between the top completion (green box) using \texttt{tab} or the bottom completion (blue box) using \texttt{shift+tab}.}
\label{fig:overview}
\end{figure*}

As model capabilities improve, large language models (LLMs) are increasingly integrated into user environments and workflows.
For example, software developers code with AI in integrated developer environments (IDEs)~\citep{peng2023impact}, doctors rely on notes generated through ambient listening~\citep{oberst2024science}, and lawyers consider case evidence identified by electronic discovery systems~\citep{yang2024beyond}.
Increasing deployment of models in productivity tools demands evaluation that more closely reflects real-world circumstances~\citep{hutchinson2022evaluation, saxon2024benchmarks, kapoor2024ai}.
While newer benchmarks and live platforms incorporate human feedback to capture real-world usage, they almost exclusively focus on evaluating LLMs in chat conversations~\citep{zheng2023judging,dubois2023alpacafarm,chiang2024chatbot, kirk2024the}.
Model evaluation must move beyond chat-based interactions and into specialized user environments.



 

In this work, we focus on evaluating LLM-based coding assistants. 
Despite the popularity of these tools---millions of developers use Github Copilot~\citep{Copilot}---existing
evaluations of the coding capabilities of new models exhibit multiple limitations (Figure~\ref{fig:motivation}, bottom).
Traditional ML benchmarks evaluate LLM capabilities by measuring how well a model can complete static, interview-style coding tasks~\citep{chen2021evaluating,austin2021program,jain2024livecodebench, white2024livebench} and lack \emph{real users}. 
User studies recruit real users to evaluate the effectiveness of LLMs as coding assistants, but are often limited to simple programming tasks as opposed to \emph{real tasks}~\citep{vaithilingam2022expectation,ross2023programmer, mozannar2024realhumaneval}.
Recent efforts to collect human feedback such as Chatbot Arena~\citep{chiang2024chatbot} are still removed from a \emph{realistic environment}, resulting in users and data that deviate from typical software development processes.
We introduce \systemName to address these limitations (Figure~\ref{fig:motivation}, top), and we describe our three main contributions below.


\textbf{We deploy \systemName in-the-wild to collect human preferences on code.} 
\systemName is a Visual Studio Code extension, collecting preferences directly in a developer's IDE within their actual workflow (Figure~\ref{fig:overview}).
\systemName provides developers with code completions, akin to the type of support provided by Github Copilot~\citep{Copilot}. 
Over the past 3 months, \systemName has served over~\completions suggestions from 10 state-of-the-art LLMs, 
gathering \sampleCount~votes from \userCount~users.
To collect user preferences,
\systemName presents a novel interface that shows users paired code completions from two different LLMs, which are determined based on a sampling strategy that aims to 
mitigate latency while preserving coverage across model comparisons.
Additionally, we devise a prompting scheme that allows a diverse set of models to perform code completions with high fidelity.
See Section~\ref{sec:system} and Section~\ref{sec:deployment} for details about system design and deployment respectively.



\textbf{We construct a leaderboard of user preferences and find notable differences from existing static benchmarks and human preference leaderboards.}
In general, we observe that smaller models seem to overperform in static benchmarks compared to our leaderboard, while performance among larger models is mixed (Section~\ref{sec:leaderboard_calculation}).
We attribute these differences to the fact that \systemName is exposed to users and tasks that differ drastically from code evaluations in the past. 
Our data spans 103 programming languages and 24 natural languages as well as a variety of real-world applications and code structures, while static benchmarks tend to focus on a specific programming and natural language and task (e.g. coding competition problems).
Additionally, while all of \systemName interactions contain code contexts and the majority involve infilling tasks, a much smaller fraction of Chatbot Arena's coding tasks contain code context, with infilling tasks appearing even more rarely. 
We analyze our data in depth in Section~\ref{subsec:comparison}.



\textbf{We derive new insights into user preferences of code by analyzing \systemName's diverse and distinct data distribution.}
We compare user preferences across different stratifications of input data (e.g., common versus rare languages) and observe which affect observed preferences most (Section~\ref{sec:analysis}).
For example, while user preferences stay relatively consistent across various programming languages, they differ drastically between different task categories (e.g. frontend/backend versus algorithm design).
We also observe variations in user preference due to different features related to code structure 
(e.g., context length and completion patterns).
We open-source \systemName and release a curated subset of code contexts.
Altogether, our results highlight the necessity of model evaluation in realistic and domain-specific settings.





\section{Problem Setup and Contributions}
\subsection{Problem Setup}
\label{sec: setup}
We study the following variant of fixed-confidence best arm identification for quantile bandits. 

\textbf{Arms and quantile rewards.}
The learner is given a set of arms $\A = \{1, 2, \dots, K\}$ with a stochastic reward setting. That is, for each arm $k \in \A$, the observations/realizations of its reward are i.i.d. random variables from some fixed but unknown reward distribution with CDF~$F_k$. This defines a (lower) quantile function $Q_k \colon [0,1] \to \R$ for each $k \in \A$ as follows:\footnote{The equality follows from the right-continuity of $F_k$.}
\begin{equation}
    Q_k(p) \coloneqq \sup \{ x  \in \R : F_k(x) < p \} 
    =
    \inf \{ x \in \R : F_k(x) \ge p \}.
\end{equation} 
The learner is interested in identifying an arm $\hat{k}$ with the highest $q$-quantile.
% , i.e., 
% an arm $\hat{k}$ satisfying
% \begin{equation}
% % \label{def: performance def}
%      Q_{\hat{k}}(q)
%         =
%         \max_{a \in \A}
%         Q_{a}(q).
% \end{equation}
While the reward of each arm is allowed to be unbounded, 
we assume the $q$-quantile of each arm to be bounded in a known range $[0, \lambda]$.\footnote{We note that setting the lower limit to 0 is without loss of generality, and regarding the interval length $\lambda$, even a crude upper bound is reasonable since the sample complexity will only have logarithmic dependence; see Theorem~\ref{theorem: upper bound}.}
%\footnote{Any finite interval of length $\lambda$ can be shifted to this range.}
% We treat $q \in (0, 1)$ as a fixed constant (e.g., $q = 1/2$ for the median) throughout the paper, meaning its dependence may be omitted in $O(\cdot)$ notation.
We let $\gP = \gP(q, \lambda)$ denote the collection of all distributions with $q$-quantile in $[0, \lambda]$, and let $\cE \coloneqq \gP
^K$ be the collection of all possible instances the learner could face.  We will sometimes write $\PP_{\nu}[\cdot]$ and $\EE_{\nu}[\cdot]$ to explicitly denote probabilities and expectations under an instance $\nu \in \cE$.

\textbf{1-bit communication constraint.} We frame the problem as having a single learner that makes decisions, and a single agent that observes rewards and sends information on them to the learner. In Remark \ref{rem:assump} below, we discuss how this can also have a multi-agent interpretation. 
% 
With a single agent, the following occurs at each iteration/time~$t \ge 1$ indexing the number of arm pulls:
\begin{enumerate}[topsep=0pt, itemsep=0pt]
    \item The learner asks the agent
    to pull an arm $a_t \in \A$, and sends the agent some side information~$S_t$.
    % based on the history of the game until time $t-1$.

    \item The agent pulls $a_t$ and observes a random reward $r_{a_t, t}$
    distributed according to CDF $F_{a_t}$.
    
    \item The agent transmits a 1-bit message to the learner, where the message is based on $r_{a_t, t}$ and $S_t$.

    \item The learner decides on arm $a_{t+1} \in \A$ and side information $S_{t+1}$,
    based on arms and the 1-bit information
    received in iterations $1, \ldots, t$.
   
\end{enumerate}
We will focus on the \emph{threshold query model}, where at iteration $t$, the side information $S_t$ is a query of the form 
``Is $r_{a_t, t} \le \gamma_t$?'' and the 1-bit message is the corresponding binary feedback $\boldsymbol{1}\{ r_{a_t,t} \le \gamma_t \}$.
The learner will only use such queries as side information in our algorithm, though the problem itself is of interest for both threshold queries and general 1-bit quantization methods (possibly having different forms of side information).


\begin{remark} \label{rem:assump}
    We do not impose any (downlink) communication constraint from the learner to the agent, as this cost is typically not expensive.  While we framed the problem as having a single agent for clarity, we are motivated by settings where the agent at each time instant could potentially correspond to a different user/device.  For this reason, and also motivated by settings where agents are low-memory sensors, we assume that the agent is `memoryless', meaning the 1-bit message transmitted cannot be dependent on rewards observed from previous arm pulls.  The preceding assumptions were similarly adopted in some of the most related previous works \cite{hanna2022solving, mitra2023linear, mayekar2023communication}.
    % This is motivated by settings where agents observing rewards are low memory sensors,  or where the agent at any given iteration may differ from those at the previous iterations.
\end{remark}


\textbf{$\epsilon$-relaxation.}
Fix a QMAB instance $\nu  \in \cE$,
and let $k^* \in \A$ be an arm with the largest $q$-quantile for the instance $\nu$.
Instead of insisting on identifying an arm with the exact highest quantile, we relax the task by only requiring the identified arm $\hat{k}$ to be at most $\epsilon$-suboptimal in the following sense:
    \begin{equation}
    \label{def: performance def}
    \hat{k} \in
    \A_{\epsilon}(\nu) \coloneqq 
    \Big\{ k \in \A 
    \ \Big\vert\
     Q_k(q)
        \ge
        Q_{k^*}(q)
        - \epsilon
        \Big\}.
\end{equation}
This allows us to limit the effort on distinguishing arms whose $q$-quantile rewards are very close to each other; analogous relaxations are common in the BAI literature.
This relaxation is also motivated by the threshold query model mentioned above; specifically, we will see in Section \ref{sec: log lambda epsilon dependence} that achieving~\eqref{def: performance def} under the threshold query model requires
$\Omega(\log(\lambda/ \epsilon))$ arm pulls even in the case of \textit{deterministic} two-arm bandits.
Our goal is to design an algorithm to identify an arm satisfying~\eqref{def: performance def} with high probability while using as few arm pulls as possible.



\subsection{Summary of Contributions.}
\label{sec: contributions}
With the problem setup now in place, we summarize our main contributions as follows: 


\begin{itemize}[topsep=0pt, itemsep=0pt]
    \item We provide an algorithm (Algorithm~\ref{alg: main}) for our setup, with the uplink communication satisfying the 1-bit constraint. Unlike standard bandit algorithms that compute empirical statistics using lossless observations of rewards, we use a noisy binary search subroutine for the learner to estimate the quantile rewards (see Appendix~\ref{sec: appendix QuantEst}).
    
    \item  We introduce fundamental arm gaps~$\Delta_k$ (Definition~\ref{def: our gap}) that generalize those proposed in prior work (see Remark~\ref{rem: gap generalization}). These gaps capture the
    difficulty of our problem setup in the sense that the problems with positive gaps essentially coincide with the set of problems that are solvable; see Theorem~\ref{thm: zero gap is unsolvable} and Remark \ref{rem: picking large enough c} for precise statements.

    \item We provide an instance-dependent upper bound on the number of arm pulls to guarantee~\eqref{def: performance def} with high probability (Corollary~\ref{cor: combined guarantee}), expressed in terms of $\lambda, \epsilon$, and fundamental arm gap~$\Delta_k$.
    Our upper bound scales logarithmically with $\lambda/\epsilon$, which contrasts with the existing upper bound for mean-based bandits with 1-bit quantization scaling linearly with $\lambda$~\cite{vial2020one, hanna2022solving}.
    

    \item  We also derive a worst-case lower bound (Theorem~\ref{thm: lower bound unquantized}) showing that our upper bound is tight to within logarithmic factors under mild conditions, and can even be tight to within constant factors when the target error probability $\delta$ decays to zero fast enough.  We additionally provide a lower bound (Theorem \ref{thm: log lambda/epsilon dependence}) showing that $\Omega(\log(\lambda/ \epsilon))$ dependence is unavoidable under threshold queries in arbitrary scaling regimes.  
    The former lower bound is applicable even in the absence of communication constraints, so we can conclude that restricting to 1-bit feedback has a minimal impact on the sample complexity, at least in terms of scaling laws.

\end{itemize}
\begin{algorithm}[ht!]
\caption{\textit{NovelSelect}}
\label{alg:novelselect}
\begin{algorithmic}[1]
\State \textbf{Input:} Data pool $\mathcal{X}^{all}$, data budget $n$
\State Initialize an empty dataset, $\mathcal{X} \gets \emptyset$
\While{$|\mathcal{X}| < n$}
    \State $x^{new} \gets \arg\max_{x \in \mathcal{X}^{all}} v(x)$
    \State $\mathcal{X} \gets \mathcal{X} \cup \{x^{new}\}$
    \State $\mathcal{X}^{all} \gets \mathcal{X}^{all} \setminus \{x^{new}\}$
\EndWhile
\State \textbf{return} $\mathcal{X}$
\end{algorithmic}
\end{algorithm}

\subsection{Lower bounds on sample complexity}\label{sec:sample_compexity}
We establish a lower bound for generalized linear measurements using standard information-theoretic arguments based on Fano's inequality. While the upper bound in Theorem~\ref{thm:alg_general} is derived for the maximum probability of error over all  $k$-sparse vectors, the lower bound applies even in the weaker setting of the average probability of error, where 
$\bx$ is chosen uniformly at random.
\begin{theorem}[Lower bound for GLMs]\label{thm: lower_bdglm} Consider any  sensing matrix $\vecA$.
For a uniformly chosen $k$-sparse vector $\bx$, an algorithm $\phi$ satisfies $$\bbP\inp{\phi(\vecA, \by) \neq \bx}\leq \delta$$   only if the number of measurements $$m\geq \frac{k\log\inp{\frac{n}{k}}}{I}\inp{1 - \frac{h_2(\delta) + \delta k\log{n}}{k\log{n/k}}}$$ for some $I$ such that $I\geq {I(y_i; \bx|\vecA)}, \, i\in [m]$. In particular, when $y\in \inb{-1, 1}$, we have $\bbE\insq{\inp{g(\vecA_i^T\bx)}^2} \geq I(y_i, \bx|\vecA)$ where the expectation is over the randomness of $\vecA$ and $\bx$.
\end{theorem}
The lower bound can be interpreted in terms of a communication problem, where the input message $\bx$ is encoded to $\vecA\bx$. The decoding function takes in as input the encoding map $\vecA$ and the output vector $\by$ in order to recover $\bx$ with high probability. For optimal recovery, one needs at least $\frac{\text{message entropy}}{\text{capacity}}$ number of measurements (follows from noisy channel coding theorem~\cite{thomas2006elements}). In Theorem~\ref{thm: lower_bdglm}, the entropy of the message set $\log{n \choose k}\approx k\log{n/k}$ and the proxy for capacity is the upper bound on mutual information $I$. We provide a detailed proof of the theorem in  Section~\ref{sec:proofs}.


We first present lower bounds for \bcs\  and \logreg. The lower bound for \bcs\ is given for any sensing matrix $\vecA$ which satisfies the power constraint given by \eqref{eq:power_constraint}, whereas the one for \logreg\ is only for the special case when each entry of the sensing matrix is iid $\cN(0,1)$. Recall that \eqref{eq:power_constraint} holds in this case.  For \bcs\ (and \logreg\ respectively), we can use the upper bound of $\bbE\insq{\inp{g(\vecA_i^T\bx)}^2}$ on the mutual information term. The dependence of $\sigma^2$ (and $1/\beta^2$ respectively) requires careful bounding of this term, which is done in the formal proofs in Appendix~\ref{proof:sec:lower_bd}.


As mentioned earlier, we need at least $k\log\inp{n/k}$ measurements for \bcs and \logreg. This is because the entropy of a randomly chosen $k$-sparse vector is approximately $k\log\inp{n/k}$ and we learn at most one bit with each measurement. However, due to corruption with noise, we learn less than a bit of information about the unknown signal with each measurement. The information gain gets worse as the noise level increases. 
Our lower bounds make this reasoning explicit.  
\begin{corollary}[\bcs\ lower bound]\label{thm: lower_bd_bcs} Suppose, each row $\vecA_i, \, i\in [1:m]$ of the sensing matrix $\vecA$ satisfies the power constraint~\eqref{eq:power_constraint}.
For a uniformly chosen $k$-sparse vector $\bx$, an algorithm $\phi$ satisfies $$\bbP\inp{\phi(\vecA, {\by}) \neq \bx}\leq \delta$$ for the problem of $\bcs$ only if the number of measurements $$m\geq \frac{k+\sigma^2}{2}\log\inp{\frac{n}{k}}\inp{1 - \frac{h_2(\delta) + \delta k\log{n}}{k\log{n/k}}}.$$ 
\end{corollary}

\begin{corollary}[\logreg\ lower bound]\label{thm: lower_bd_log_reg} Consider a Gaussian  sensing matrix $\vecA$ where each entry is chosen iid $N(0,1)$.
For a uniformly chosen $k$-sparse vector $\bx$, an algorithm $\phi$ satisfies $$\bbP\inp{\phi(\vecA, \bw) \neq \bx}\leq \delta$$ for the problem of $\logreg$ only if the number of measurements $$m\geq \frac{1}{2}\inp{k+\frac{1}{\beta^2}}\log\inp{\frac{n}{k}}\inp{1 - \frac{h_2(\delta) + \delta k\log{n}}{k\log{n/k}}}.$$ 
\end{corollary}



Theorem~\ref{thm: lower_bdglm} also implies an information theoretic lower bound for \spl, which is presented below and proved in Appendix~\ref{proof:sec:lower_bd}. Note that the denominator term in the bound $\frac{1}{2}\log\inp{1+\frac{k}{\sigma^2}}$ is the capacity of a Gaussian channel with power constraint $k$ and noise variance $\sigma^2$. 
\begin{corollary}[\spl\ lower bound]\label{thm: spl_lower_bd_1}
Under the average power constraint \eqref{eq:power_constraint} on  $\vecA$, for a uniformly chosen $k$-sparse vector $\bx$, an algorithm $\phi$ satisfies $$\bbP\inp{\phi(\vecA, {\by}) \neq \bx}\leq \delta$$ only if the number of measurements
$$m\geq \frac{k\log\inp{\frac{n}{k}}-\inp{h_2(\delta) + \delta k\log{n}}}{\frac{1}{2}\log\inp{1+\frac{k}{\sigma^2}}}.$$
\end{corollary} 

\subsection{Tighter upper and lower bounds for \spl}\label{sec:tighter_bounds_spl}
We present information theoretic upper and lower bounds for \spl\ in this section. Similar to Section~\ref{sec:alg}, our upper bound is for the maximum probability of error, while the lower bounds hold even for the weaker criterion of average probability of error.

We first present an upper bound based on the maximum likelihood estimator (MLE) where  we  decode to $\hat{\bx}$ if, on output $\by$, 
\begin{align*}
\hat{\bx} = \argmax_{\stackrel{\bx\in \inb{0,1}^n}{\wh{\bx} = k}}\,\, p(\by|{\bx})
\end{align*} where $p(\by|{\bx})$ denotes the probability density function of $\by$ on input $\bx$.
\begin{theorem}[MLE upper bound for \spl]\label{thm:upper_bd_mle} Suppose  entries of the measurement matrix $\vecA$ are i.i.d. $\cN(0,1).$
The MLE  is correct with high probability if 
\begin{align}m\geq \max_{l\in[1:k]}  \frac{nN(l)}{\frac{1}{2}\log\inp{\frac{ l}{2\sigma^2}+1}}\label{eq:upper_bd_mle}
\end{align}where  $N(l):=  \frac{k}{n} h_2\inp{\frac{l}{k}} + (1-\frac{k}{n})h_2\inp{\frac{l}{n-k}}$. 
\end{theorem}
We prove the theorem in Appendix~\ref{proof:MLE}. The main proof idea involves analysing the probability that the output of the MLE is $2l$ Hamming distance away from the unknown signal $\bx$ for different values of $l\in [1:k]$ (assuming $k\leq n/2$). This depends on the number of such vectors (approximately $2^{nN(l)}$) and the probability that the MLE outputs a vector which is $2l$ Hamming distance away from $\bx$. 

Note that when $l = k\inp{1-\frac{k}{n}}$, $nN(l) = nh_2(k/n)\approx k\log{\frac{n}{k}}$ and $\log\inp{\frac{k\inp{1-k/n}}{2\sigma^2}+1}\leq \log\inp{\frac{k}{2\sigma^2}+1}$.
Thus, $m$ is at least $\frac{2k\log{n/k}}{\log\inp{\frac{k}{2\sigma^2}+1}}$ (see the bound for Corollary~\ref{thm: spl_lower_bd_1}). It is not immediately clear if this value of $l= k\inp{1-\frac{k}{n}}$ is the optimizer. However, for large $n$, this appears to be the case numerically as shown in Plot~\ref{plot:1}.

\begin{figure}[t]
\includegraphics[width=7cm]{Unknown2.png}
\centering
\caption{The figure shows the plot of the MLE upper bound \eqref{eq:upper_bd_mle} (given by m1) for different values of $k$. This is displayed in blue color. A plot of $\frac{2nN(l)}{\log\inp{\frac{ l}{2\sigma^2}+1}}$ is also presented for $l = k\inp{1-\frac{k}{n}}$ in orange color, given by m2. A part of the plot is zoomed in to emphasize the closeness between the lines. In these plots,  $\sigma^2$ is set to 1,  $n$ is 50000 and $k$ ranges from 1000 to 25000 $(n/2)$. }\label{plot:1}
\end{figure}


Inspired by the MLE analysis, we derive a lower bound with the same structure as \eqref{eq:upper_bd_mle}. We generate the unknown signal $\bx$ using the following distribution: A vector $\tilde{\bx}$ is chosen uniformly at random from the set of all $k$-sparse vectors. Given $\tilde{\bx}$, the unknown input signal $\bx$ is chosen uniformly from the set of all $k$-sparse vector which are at a Hamming distance $2l$ from $\bx$. 
The lower bound is then obtained by computing upper and lower bounds on $I(\vecA, \by;\bx|\tilde{\bx})$.
We show this lower bound only for random matrices where each entry is chosen iid $\cN(0,1)$.
\begin{theorem}[\spl\ lower bound]\label{thm:lower_bd_spl}
If each entry of $\vecA$ is chosen iid $\cN(0,1)$, then for a uniformly chosen $k$-sparse vector $\bx$, an algorithm $\phi$ satisfies 
\begin{align}
    \bbP\inp{\phi(\vecA, {\by}) \neq \bx}\leq \delta\label{eq:spl_lower_bd_l}
\end{align}  only if the number of measurements $$m\geq \max_l\frac{nN(l) - 2\log{n}- h_2(\delta) - \delta k\log{n}}{\frac{1}{2}\log\inp{1+\frac{l}{\sigma^2}\inp{2-\frac{l}{k}}}} .$$
\end{theorem} The proof of Theorem~\ref{thm:lower_bd_spl} is given in Appendix~\ref{proof:MLE}.

If we choose $l = k\inp{1-\frac{k}{n}}$ in Theorem~\ref{thm:lower_bd_spl}, we recover corollary~\ref{thm: spl_lower_bd_1} for the special case of Gaussian design.
% \begin{corollary}\label{corollary2:lower_bd_spl}
% If  each entry of $\vecA$ is chosen iid $\cN(0,1)$, then for a uniformly chosen $k$-sparse vector $\bx$, an algorithm $\phi$ satisfies 
% $$\bbP\inp{\phi(\vecA, {\by}) \neq \bx}\leq \delta$$
% only if the number of measurements 
% $$m\geq \frac{k\log\inp{\frac{n}{k}} - 2\log{n}- h_2(\delta) - \delta k\log{n}}{\log\inp{1+\frac{k}{\sigma^2}}} .$$
% \end{corollary}

% Corollary~\ref{corollary2:lower_bd_spl} can also be proved directly for any sensing matrix $\vecA$ which satisfies \eqref{eq:power_constraint} (non-necessarily a Gaussian design). 


% \begin{figure}[t]
% \includegraphics[width=8cm]{plot.png}
% \centering
% \caption{The figure shows the plot of the MLE upper bound \eqref{eq:upper_bd_mle} (given by m1) for different values of $n$. This is displayed in blue color. A plot of $\frac{2nN(l)}{\log\inp{\frac{ l}{2\sigma^2}+1}}$ is also presented for $l = k\inp{1-\frac{k}{n}}$ in orange color, given by m2. In these plots,  $\sigma^2$ is set to 1 and $k$ is $0.2n$. }\label{plot:1}
% \end{figure}


\section{Solvable Instances} 
\label{sec: solvable}
In Sections~\ref{sec: upper bound} and~\ref{sec: lower bound unquantized}, we provided nearly matching upper and lower bounds for instances with positive gap.
In this section, we study the ``(un)solvabality'' of bandit instances with zero gap, and show that essentially all bandit instances that are ``solvable'' have positive gap, as long as parameter~$c$ is large enough (see Remark~\ref{rem: picking large enough c}).
To formalize this idea, we define the following class of bandit instances.



\begin{definition}[Solvable instances]
\label{def:solvable}
     Let $\A, \epsilon,$ and $q$ be fixed. 
     We say that an instance $\nu \in \cE$ is $\epsilon$-\emph{solvable} if for each $\delta \in (0, 1)$, there exists an algorithm that is $(\epsilon, \delta)$-reliable and it holds under instance $\nu$ that\footnote{We could require that $\PP_{\nu}[\tau < \infty] = 1$ in this case and the subsequent analysis and conclusions would be essentially unchanged.  Recall also that $\PP_{\nu}[\cdot]$ denotes probability under instance $\nu$.}
    \begin{equation}
        \PP_{\nu}[\tau < \infty \cap \hat{k}\in\Ac_{\epsilon}]\ge1-\delta.
    \end{equation}
    If no such algorithm exists, we say that $\nu$ is $\epsilon$-\emph{unsolvable}.
\end{definition}

\begin{remark}
\label{rem: solvable inclusion}
    Fix $0 < \epsilon_1 \le \epsilon_2$. If an instance $\nu$ is
    $\epsilon_1$-solvable, then it is $\epsilon_2$-solvable.
    This follows directly from $\A_{\epsilon_1}(\nu) \subseteq \A_{\epsilon_2}(\nu)$.
\end{remark}

From Corollary~\ref{cor: combined guarantee}, we deduce that any instance with a positive gap is solvable. 
 
\begin{corollary}[Positive gap is solvable]
\label{cor: positive gap is solvable}
    Let $\A, \lambda, \epsilon, q,$ and $c$ be fixed. 
    Suppose an instance $\nu$ satisfies
    $ \Delta > 0$, where $\Delta =  \Delta(\nu, \lambda, \epsilon, c, q) $ is as defined in Theorem~\ref{theorem: upper bound}.
    Then $\nu$ is $\epsilon$-solvable. 
\end{corollary}
The main result of this section is that the reverse inclusion nearly holds, in the following sense.
 \begin{theorem}[Zero gap is unsolvable]
 \label{thm: zero gap is unsolvable}
    Let $\lambda, \epsilon, c,$ and $q$ be fixed, 
    and let $\tilde{\epsilon} = \tilde{\epsilon}(\lambda, \epsilon, c)$ be as defined in 
    Algorithm~\ref{alg: main}.
    Suppose an instance $\nu \in \cE$ satisfies $\Delta(\nu, \lambda, \epsilon, c, q) = 0$.
    If we assume for $\nu$ that there exists some sufficiently small $\eta_0 > 0$ such that 
    $0 \le Q_k^+(q-\eta_0) \le Q_k(q+\eta_0) \le \lambda$, then $\nu$ is $c\tilde{\epsilon}$-unsolvable.
 \end{theorem}
\begin{proof}
    See Appendix~\ref{sec: appendix solvable instance}.
\end{proof}


\begin{remark}[Removing the additional assumption]
\label{rem: remove additional assumption}
    The additional assumption involving $\eta_0$ 
    % in Theorem~\ref{thm: zero gap is unsolvable} and Corollary~\ref{cor: zero gap is unsolvable} 
    is mild; it is trivially satisfied by instances with all reward distributions supported on $[0, \lambda]$, and also holds significantly more generally since $\eta_0$ can be arbitrarily small.
    % The additional assumption is added to simplify the proof of Lemma~\ref{lem:two_instances}. Specifically, it helps ensuring both $(G_k)^{-1}(q)$ and $(G_a)^{-1}(q)$ as defined in \eqref{eq: shifted q quantiles} are in $[0, \lambda]$ so that the constructed instance $\nu'$ satisfies $\nu' \in \cE$.
    Moreover, in Appendix~\ref{sec: assumption removal}, we show that
    this assumption is unnecessary if we use the modified gap (see Remark~\ref{rem: further improvement}) instead of $\Delta$.
    % and Definition~\ref{def: modified gap}) in Theorem~\ref{thm: zero gap is unsolvable}: if the modified gap is zero for an instance~$\nu$, then $\nu$
    % is $c\tilde{\epsilon}$-unsolvable.
\end{remark}


 
\begin{remark}
\label{rem: picking large enough c}
 For each $\theta \in (0, 1)$, picking $c = \lceil 2\theta / (1-\theta)\rceil$ yields
\begin{equation}
       \nu \text{ is } \theta\epsilon\text{-solvable} 
    \implies
    \nu \text{ is } c\tilde{\epsilon}\text{-solvable}
    \implies
     \Delta(\nu, \lambda, \epsilon, c, q) > 0
     \implies 
     \nu \text{ is } \epsilon\text{-solvable},
\end{equation}
where the last two implications follow from Theorem~\ref{thm: zero gap is unsolvable} and Corollary~\ref{cor: positive gap is solvable}, and the first implication follows from Remark~\ref{rem: solvable inclusion} and the following inequality:
\begin{equation}
    c \tilde{\epsilon} 
    = 
    \frac{c \lambda}{ \left\lceil (c+1) \lambda/\epsilon \right\rceil}
    \ge
     % \frac{c \lambda}{  (c+1) \lambda/\epsilon  + \lambda/\epsilon }
     % =
     \frac{c \lambda}{  (c+2) \lambda / \epsilon }
     = 
     \left(1 - \frac{2}{c+2} \right) \epsilon
     \ge
     \left(1 - \frac{2}{\frac{2\theta+2-2\theta}{1-\theta}} \right) \epsilon
     = \theta \epsilon.
\end{equation} 
Since $\theta$ can be arbitrarily close to $1$,
% $\lim\limits_{c \to \infty} c \tilde{\epsilon} = \epsilon$, 
we have $\Delta(\nu, \lambda, \epsilon, c, q) > 0$
    % we can use Algorithm~\ref{alg: main} to solve 
     for essentially all $\epsilon$-solvable instances by picking a sufficiently large $c$.
 \end{remark}

The proof of Theorem~\ref{thm: zero gap is unsolvable} will turn out to directly extend to a ``limiting'' version in which we replace $c\tilde{\epsilon}$ by $\lim\limits_{c \to \infty} c\tilde{\epsilon} = \epsilon$ and $\Delta(\nu, \lambda, \epsilon, c, q)$ by $\lim\limits_{c \to \infty} \Delta(\nu, \lambda, \epsilon, c, q)$, giving the following corollary.


 \begin{corollary}
 \label{cor: zero gap is unsolvable}
    Let $\lambda, \epsilon$, and $q$ be fixed.
    Let $\Delta_{k}(\nu, \epsilon, q)$ be the gap defined in Definition~\ref{def: our gap} with $c \to \infty$ (see~\eqref{eq: gap k infinite c} for the explicit form).    
    Suppose an instance $\nu \in \cE$ satisfies $\Delta(\nu, \epsilon, q) = \max\limits_{k \in \A_{\epsilon(\nu)}} \Delta_k(\nu, \epsilon, q) = 0$.
    If we assume for $\nu$ that there exists some sufficiently small $\eta_0 > 0$ such that 
    $0 \le Q_k^+(q-\eta_0) \le Q_k(q+\eta_0) \le \lambda$, then $\nu$ is $\epsilon$-unsolvable.
 \end{corollary}
 \begin{proof}
    See Appendix~\ref{sec: appendix solvable instance}.
\end{proof}


\acks{This work was supported by the Singapore National Research Foundation under its AI Visiting Professorship programme.}


\newpage
\bibliography{bibliography}

\newpage
\appendix

\section{Quantile Estimation Subroutine}
\label{sec: appendix QuantEst}

\subsection{Noisy Binary Search}
\label{sec: appendix MNBS reformulation}
We first momentarily depart from MAB and discuss the monotonic noisy binary search (MNBS) problem of~\cite{karp2007noisy}; see also the end of Appendix~\ref{sec: appendix quant est related work} for a summary of some related work on noisy binary search.
The original problem formulation was stated in terms of finding a special coin $i$ among $n$ coins, but this can be restated as follows: 
We have a random variable~$R$ with an unknown CDF $F$ and a list of $n$ points $x_1  \le \cdots \le x_n$ such that $Q(\tau) \in [x_1, x_n]$, and the goal is to find an index $i$  satisfying 
\begin{equation}
\label{eq: MNBS guarantee}
    [F(x_i), F(x_{i+1})] \cap  (\tau - \Delta, \tau + \Delta) \ne 
    \varnothing
\end{equation}
via adaptive queries of the form $\mathbf{1}(R \le x_j)$. Note that each query $\mathbf{1}(R \le x_j)$ is an independent Bernoulli random variable with parameter $F(x_j)$.
We will make use of the following main result from~\cite[Theorem 1.1]{gretta2023sharp}.

\begin{proposition}[Noisy binary search guarantee]
\label{prop: MNBS guarantee}
For any $\delta \in (0, 1)$ and relaxation parameter $\Delta \le \min(\tau, 1-\tau)$, the MNBS algorithm in \cite{gretta2023sharp}
output an index $i$ after at most
$O \big( \frac{1}{\Delta^2}  \log  \frac{n}{\delta} \big)$
queries\footnote{\label{footnote: constants MNBS}The expression for the number of iterations in \cite{gretta2023sharp} is more complicated because it has some terms with explicit constant factors, but in $O(\cdot)$ notation it simplifies to $O \big( \frac{1}{\Delta^2}  \log  \frac{n}{\delta} \big)$. We do not specify the exact number of loops in Algorithm~\ref{alg: quantile interval}, as doing so is somewhat cumbersome and the focus of our work is on the scaling laws.} and $i$ satisfies~\eqref{eq: MNBS guarantee} with probability at least $1- \delta$. 
\end{proposition}
The bulk of the MNBS algorithm in \cite{gretta2023sharp} is based on Bayesian multiplicative weight updates: Start with a uniform prior over which of the $n$ intervals crosses quantile $\tau$, make the query at $x_j$ whose $F(x_j)$ is nearest to $\tau$ under current distribution, update the posterior by multiplying intervals on one side of the query by $1 + c \Delta$ and the other side by $1 - c \Delta$ for some fixed constant $c$, and repeat. Other MNBS algorithms such as those in \cite{karp2007noisy}, or even a naive binary search with repetitions (see \cite[\S 1.2]{karp2007noisy}), could also be used to solve the MNBS problem, but we choose  \cite{gretta2023sharp} since it has the best known scaling of the query complexity. Further comparisons of the relevant theoretical guarantees and practical performance can be found in \cite{gretta2023sharp}.


\subsection{Quantile Estimation with 1-bit Feedback}
\label{sec: appendix quant est related work}
The MNBS algorithm can be implemented under our 1-bit communication-constrained setup. Specifically, the learner decides which arm $k$ to query as well as the point $x_j$  to query, and then sends a threshold
query ``Is $R_k \le x_j$?'' as side information to the agent, where $R_k$ is the random reward (variable) of the arm~$k$ with CDF $F_k$. The agent will then pull arm $k$ and reply with a 1-bit binary feedback corresponding to the observation. Note that 
while the $O \big( \frac{1}{\Delta^2}  \log  \frac{n}{\delta} \big)$ queries for a given arm are done in an adaptive manner, the queries themselves can be requested at different time steps without any requirement of agent memory. 
A high-level description of the implementation for a fixed arm is given in Algorithm~\ref{alg: quantile interval}. This gives us the following guarantee, which is a simple consequence of Proposition~\ref{prop: MNBS guarantee}.




\begin{algorithm}
    \caption{Communication-constrained quantile estimation subroutine ($\mathtt{QuantEst}$ in
Algorithm~\ref{alg: main})}
    \label{alg: quantile interval}

   \textbf{Input}: 
    Arm with reward $R$ distributed according to CDF $F$,
    a list $\mathbf{x}$ of $n$ points $x_1 \le  \cdots \le x_n$,
    quantile $\tau \in (0, 1)$ satisfying $Q(\tau) \in [x_1, x_n]$,
    approximation parameter~$\Delta \le \min(\tau, 1-\tau)$,
    error probability $\delta \in (0,1)$
    
    \textbf{Output} Index $i \in \{1, \dots, n-1\}$    

\begin{algorithmic}[1]
      
        \For{$t = 1$ to $t_{\max}$ (with\footref{footnote: constants MNBS} $t_{\max} = O \big( \frac{1}{\Delta^2}  \log  \frac{n}{\delta} \big)$)}

          \State \textbf{At Learner:}
          
          \State~~~~Pick index $j$
          % $j \in \{1,  \ldots, n-1\}$
          according to Bayesian weight update as in~\cite{gretta2023sharp}
          
          \State~~~~Send threshold query 
          ``Is $R \le x_j$?'' to the agent

            \State \textbf{At Agent:}
          
          \State~~~~Pull arm and observe reward $r$
          
          \State~~~~Send 1-bit feedback $\mathbf{1}(r \le x_j)$ to the learner

        
        \EndFor

    \State Return index $i$  according to~\cite{gretta2023sharp}

\end{algorithmic}    
\end{algorithm}


\begin{corollary}[$\mathtt{QuantEst}$ guarantee]
\label{cor: QuantEst guarantee}
Let $(F, \mathbf{x}, \tau, \Delta, \delta)$ be a valid input of Algorithm~\ref{alg: quantile interval}, and let~$n$ be the number of points in $\mathbf{x}$. 
Then the algorithm outputs  an index $i$ after at most
$O \big( \frac{1}{\Delta^2}  \log  \frac{n}{\delta} \big)$
queries and $i$  satisfies
    $\mathbb{P}
    \left([F(x_i), F(x_{i+1})] \cap  (\tau - \Delta, \tau + \Delta) = \varnothing \right) < \delta$.
\end{corollary}


\textbf{Related work on noisy binary search and quantile estimation.}
We briefly recap the original MNBS problem in \cite{karp2007noisy, gretta2023sharp}:
There are $n$ coins whose unknown probabilities $p_j \in [0, 1]$ are sorted in nondecreasing order, where flipping coin $j$ results in head with probability $p_j$. The goal is to identify a coin $i$ such that the interval $[p_i, p_{i+1}]$ has a nonempty intersection with $(\tau - \Delta, \tau + \Delta)$. This model subsumes noisy binary search with a fixed noise level \cite{burnashev1974interval, ben2008bayesian, dereniowski2021noisy, gu2023optimal} (where $p_j = \frac{1}{2} - \Delta$ for $j \leq i$ and $p_j = \frac{1}{2} + \Delta$ otherwise) as well as regular binary search (where $p_j \in \{0, 1\}$). As we discussed in Appendix~\ref{sec: appendix MNBS reformulation}, this problem can be reformulated into the problem of estimating (the quantile of) a distribution using threshold/comparison queries, where the noise in the feedback is stochastic.  This quantile estimation problem has been generalized to a non-stochastic noise setting \cite{meister2021learning, okoroafor2023non}, and was also studied in the context of online dynamic pricing and auctions \cite{kleinberg2003value, leme2023pricing, leme2023description}. 
In particular, \cite[Algorithm~1]{leme2023pricing} is similar to Algorithm~\ref{alg: quantile interval} (or equivalently subroutine $\mathtt{QuantEst}$ used on Lines~\ref{ltk def} and~\ref{utk def} of Algorithm~\ref{alg: main}), in the sense that both use noisy binary search to identify the quantile of a \textit{single} distribution. However, they use the naive binary search with repetitions to form confidence intervals containing the quantile, which has a suboptimal complexity $O \big( \frac{1}{\Delta^2} \log n \log  \frac{\log n}{\delta} \big)$; see \cite[\S 1.2]{karp2007noisy} for details.  Overall, while ideas from the existing literature on quantile estimation of a \textit{single} distribution with threshold queries may provide useful context, they do not readily translate into Algorithm~\ref{alg: main} or the analysis that led to our main contributions.








\subsection{Proof of Lemma~\ref{lem: good events} (Bounding the Probability of Event E) }
\label{sec: proof event E}
\begin{proof}[Proof of Theorem~\ref{lem: good events}]
    For a fixed $t \ge 1$ and a fixed $k \in \mathcal{A}_t$,
    we have
    \begin{equation}
        \Pr{
        \overline{E_{t, k, l}}
        }
        \le  \frac{\delta \cdot \Delta^{(t)}}{2 |\mathcal{A}_t|}
        \quad
        \text{and}
        \quad
         \Pr{
        \overline{E_{t, k, u}}
        }
        \le  \frac{\delta \cdot \Delta^{(t)} }{2 |\mathcal{A}_t|}
        \quad
    \end{equation}
    by the guarantee of the $\mathtt{QuantEst}$ (see Corollary~\ref{cor: QuantEst guarantee}).
    Applying the union bound, we obtain
    \begin{equation}
        \Pr{\overline{E}}
        \le 
        \sum_{t \ge 1}
        \sum_{k \in \mathcal{A}_t} 
        \frac{\delta \cdot \Delta^{(t)}}{ |\mathcal{A}_t|}
        \le 
        \sum_{t \ge 1}
        |\mathcal{A}_t| \cdot 
        \frac{\delta \cdot \Delta^{(t)}}{ |\mathcal{A}_t|}
        =
        \delta
         \sum_{t \ge 1}
         \Delta^{(t)}
         =
         \delta
         \sum_{t \ge 1}
         2^{-t} 
         \le \delta
    \end{equation}
    as desired. The number of arm pulls~\eqref{eq: QuantEst arm pulls} follows immediately from the guarantee of $\mathtt{QuantEst}$ from Corollary \ref{cor: QuantEst guarantee}, $|\A_t| \le |\A| = K$, 
    and the number of points $n = \Theta(c\lambda/\epsilon)$.
\end{proof}
\section{Proof of Lemma~\ref{lem: quantile anytime bound} (Anytime Quantile Bounds)}
\label{sec: appendix anytime quantile bounds}
We first present a useful auxiliary lemma.
\begin{lemma}
    Under the setup of Lemma~\ref{lem:  quantile anytime bound} (including Event $E$ from Lemma~\ref{lem: good events} holding), we have the following bounds:
    \begin{equation}
    \label{eq: xltk upper bound}
        x_{l_{t, k}} < Q_k(q)
    \end{equation}
    \begin{equation}
    \label{eq: xltk+1 lower bound}
        Q^+_k\big( q -  \Delta^{(t)} \big)
        \le x_{l_{t, k} + 1}
    \end{equation}
    \begin{equation}
    \label{eq: xutk upper bound}
        x_{u_{t, k}} < Q_k\big( q +  \Delta^{(t)} \big)
    \end{equation}
    \begin{equation}
    \label{eq: xutk+1 lower bound}
        Q^+_k(q)
        \le x_{u_{t, k} + 1}
    \end{equation}
    for each round $t \ge  1$ and arm $k \in \A_t$.
\end{lemma}
\begin{proof}
We will prove only~\eqref{eq: xltk upper bound} and~\eqref{eq: xltk+1 lower bound}
for an arbitrary $t  \ge 1$ and  $k \in \A_t$
in detail, as~\eqref{eq: xutk upper bound} and~\eqref{eq: xutk+1 lower bound} can be proved similarly.
Observe that, under event $E_{t,k,l} \subset E$ (see \eqref{eq: event Etkl}), we have 
\begin{equation}
\label{eq: Fk_xltk bound}
    F_k(x_{l_{t, k}}) < q
    \quad \text{and} \quad
    q -  \Delta^{(t)} < F_k(x_{l_{t, k}+1})
\end{equation}
respectively,
as otherwise the interval $[F_k(x_{l_{t, k}}), F_k(x_{l_{t, k}+1})]$
would fall on the right and the left, respectively, of the interval $\left( q - \Delta^{(t)}, q   \right)$. A similar argument through the event $E_{t,k,u} \subset E$ (see \eqref{eq: event Etku}) yields
\begin{equation}
\label{eq: Fk_xutk bound}
    F_k(x_{u_{t, k}}) < q + \Delta^{(t)}
    \quad \text{and} \quad
    q  < F_k(x_{u_{t, k}+1}).
\end{equation}

We now prove~\eqref{eq: xltk upper bound} using~\eqref{eq: Fk_xltk bound}; the inequality~\eqref{eq: xutk upper bound} can be proved similarly through~\eqref{eq: Fk_xutk bound}. If $x_{l_{t, k}} = - \infty$, then~\eqref{eq: xltk upper bound} holds trivially.
Therefore, we proceed on the assumption that $x_{l_{t, k}} \in \R$.
Then, using standard properties of quantile functions (see, e.g., \cite[4.3 Theorem]{dufour1995distribution}), we have $x_{l_{t, k}} < Q_k(q)$ as desired.

We now prove~\eqref{eq: xltk+1 lower bound} using~\eqref{eq: Fk_xltk bound}; the inequality~\eqref{eq: xutk+1 lower bound} can be proved similarly through~\eqref{eq: Fk_xutk bound}. 
If $x_{l_{t, k}+1} = \infty$, then~\eqref{eq: xltk+1 lower bound} holds trivially.
Therefore, we proceed on the assumption that $x_{l_{t, k}+1} \in \R$.
In this case, it is a finite upper bound on the values in the set
    $\{ z \in \R: F_k(z) \le q -  \Delta^{(t)}  \}$, and so this set has a finite supremum. It follows that 
    \begin{equation}
        x_{l_{t, k}+1} \ge 
        \sup \{ z \in \R : F_k(z) \le q -  \Delta^{(t)}  \} =
        Q^+_k\big( q -  \Delta^{(t)} \big)
    \end{equation}
    as desired.
\end{proof}


\begin{proof}[Proof of Lemma~\ref{lem:  quantile anytime bound}]
We break down the bounds into  inequalities as follows:
\begin{multicols}{2}
\begin{enumerate}[label=(\roman*)]

    \item  $\mathrm{LCB}_{\tau}(k) \le \mathrm{LCB}_t(k)$
    
    \item $\mathrm{LCB}_t(k)  < Q_k(q)$
    
    % \item $Q_k(q) \le  Q^+_k(q)$
    
    \item $Q_k(q) \le \mathrm{UCB}_t(k)$
    
    \item $\mathrm{UCB}_t(k)  \le \mathrm{UCB}_{\tau}(k)$
    
    \item $Q^+_k\big(q -  \Delta^{(t)} \big)  \le \mathrm{LCB}_t(k) + \tilde{\epsilon}$

    \item $\mathrm{UCB}_t(k) - \tilde{\epsilon}
        % \le x_{u_{t, k} } 
        <
         Q_k\big(q + \Delta^{(t)} \big)$
\end{enumerate}
\end{multicols}
We will prove only inequalities (i), (ii), and (iv)
for an arbitrary $t  > \tau \ge 0$ and  $k \in \A_t$
in detail, as all the other inequalities can be proved similarly.

Inequality (i) follows immediately from Line~\ref{LCB definition} of Algorithm~\ref{alg: main} and induction. Likewise, we can show (iv) using Line~\ref{UCB definition} of Algorithm~\ref{alg: main}.
 
We now show inequality (ii) by induction on $t$;
inequality (iii) can be proved similarly.
    For the base case $t = 1,$
    we have
    \begin{equation}
        \mathrm{LCB}_1(k) =
        \max \left( x_{l_{t, k}}, \mathrm{LCB}_{0}(k) \right) =
        \max \left( x_{l_{t, k}}, 0 \right) =
        x_{l_{t, k}} < Q_k(q),
    \end{equation}
    where the last inequality follows from~\eqref{eq: xltk upper bound}.   
    For the inductive step, suppose that $\mathrm{LCB}_t(k) < Q_k(q)$ for a fixed $t \ge 1$. Since $x_{l_{t, k}} < Q_k(q)$, we have
    \begin{equation}
        \mathrm{LCB}_{t+1}(k) =
            \max
            \left(
            x_{l_{t, k}},
            \mathrm{LCB}_{t}(k)
            \right)
            < Q_k(q)
    \end{equation}
    as desired.
    

    We now show inequality (v) using~\eqref{eq: xltk+1 lower bound}; inequality (vi)
         can be shown using a similar argument through~\eqref{eq: xutk upper bound}.
        We consider three cases for the index $l_{t, k}$:
        \begin{itemize}
            \item ($l_{t, k} = 0$) In this case, we have 
                    $x_{l_{t, k} + 1} = x_{1} =  0 = \mathrm{LCB}_0(k)$, and so
                    \begin{equation}
                       Q^+_k\big( q -  \Delta^{(t)} \big)
                        \le 
                        x_{l_{t, k} + 1} =
                        \mathrm{LCB}_0(k)
                        \le 
                        \mathrm{LCB}_t(k)
                        < \mathrm{LCB}_t(k) + \tilde{\epsilon},
                    \end{equation}
                    where the first inequality follows from~\eqref{eq: xltk+1 lower bound} and the second inequality follows from inequality~(i).
            
            \item ($1 \le l_{t, k} \le n$) In this case, we have
                    \begin{equation}
                   Q^+_k\big( q -  \Delta^{(t)} \big)
                    \le x_{l_{t, k} + 1}
                    = x_{l_{t, k}} + \tilde{\epsilon}
                    \le \mathrm{LCB}_t(k) + \tilde{\epsilon},
                \end{equation}
                where the first inequality follows from~\eqref{eq: xltk+1 lower bound}, the equality follows from distance between
                consecutive points set in Line~\ref{line: list of points} of Algorithm~\ref{alg: main}, 
                and the last inequality follows from Line~\ref{LCB definition} of Algorithm~\ref{alg: main}.

           \item  ($l_{t, k} = n+1$) In this case, we have  $x_{l_{t, k}} = x_{n+1} = \lambda \ge Q_k(q) \ge  Q^+_k\big( q -  \Delta^{(t)} \big)$,
           and so
                    \begin{equation}
                       Q^+_k\big( q -  \Delta^{(t)} \big) \le 
                        x_{l_{t, k}}  \le 
                        \mathrm{LCB}_t(k)
                        < \mathrm{LCB}_t(k) + \tilde{\epsilon},
                    \end{equation}
                    where the second inequality follows from Line~\ref{LCB definition} of Algorithm~\ref{alg: main}.
        \end{itemize}
    Combining all three cases, we have
    $Q^+_k\big( q -  \Delta^{(t)} \big)
        \le \mathrm{LCB}_t(k) + \tilde{\epsilon}$ as desired.
\end{proof}



\section{Proof of Theorem~\ref{thm: correctness} (Reliability of Algorithm~\ref{alg: main})}
\label{sec: appendix correctness}
\begin{proof}[Proof of Theorem~\ref{thm: correctness}]
    We first show by induction that an optimal arm $k^*$ of instance $\nu$ (i.e., one having the highest $ q$-quantile) will always be active, i.e., $k^* \in \A_t$ for each round $t \ge 1$. 
        For the base case $t = 1$, we have $k^* \in \{1, \dots, K\} = \A_1$ trivially. We now show the inductive step: if $k^* \in \A_t$ holds, then $k^* \in \A_{t+1}$. For all arms $a \in \A_t$, we have
    \begin{equation}
        \mathrm{UCB}_t(k^*)
        \ge
        Q_{k^*}(q) 
        \ge Q_{a}(q)  
        > 
        \mathrm{LCB}_t(a),
    \end{equation}
    where the second inequality follows from the optimality of  arm $k^*$,
    while the other two inequalities follow from the anytime quantile bounds (Lemma~\ref{lem: quantile anytime bound}).
    It follows that
    $\mathrm{UCB}_t(k^*) >
            \max\limits_{a \in \mathcal{A}_{t}} \mathrm{LCB}_t(a)$,
    and so $k^* \in \A_{t+1}$ by definition (see Line~\ref{line: active arm} of Algorithm~\ref{alg: main}).


    We now argue that if Algorithm~\ref{alg: main} terminates, then the returned arm~$\hat{k}$ satisfies~\eqref{def: performance def}. 
    If Algorithm~\ref{alg: main} terminates,
    then the while-loop (Lines~\ref{line: start while loop}--\ref{line: end while loop}) must have terminated and therefore the returned arm $\hat{k}$
    satisfies the condition
    \begin{equation}
    \label{eq: k condition}
    \mathrm{LCB}_t(\hat{k})  \ge
                \max\limits_{a \in \A_t \setminus \{ \hat{k} \} }  
                \mathrm{UCB}_t(a) -  (c+1)\tilde{\epsilon}
                 \ge
          \max\limits_{a \in \A_t \setminus \{ \hat{k} \} }  
                \mathrm{UCB}_t(a)  - \epsilon,
    \end{equation}
    where the second inequality follows from Lines~\ref{line: number of points}--\ref{line: tilde epsilon} of Algorithm~\ref{alg: main}: $\tilde{\epsilon} \le \lambda \cdot \epsilon/((c+1) \lambda) = \epsilon/(c+1)$.
    If $\hat{k} = k^*$, then the returned arm satisfies~\eqref{def: performance def} trivially. Therefore, we assume that $\hat{k} \ne k^*$ for the rest of the proof. In this case, we have
    \begin{equation}
        Q_{\hat{k}}(q) >
        \mathrm{LCB}_t(\hat{k})  \ge
        \max\limits_{a \in \A_t \setminus \{ \hat{k} \} } 
        \mathrm{UCB}_t(a) -  
        \epsilon
        \ge 
        \mathrm{UCB}_t(k^*) -  \epsilon
        \ge
        \max\limits_{a \in \A_t \setminus \{ \hat{k} \} }
        Q_a(q) -  \epsilon.
    \end{equation}
    where the first and the last inequalities follow from the anytime quantile bounds (see Lemma~\ref{lem: quantile anytime bound}), while the second inequality follows from the condition~\eqref{eq: k condition} and the third inequality follows from $k^* \in \A_t$ (see above) and the assumption that $\hat{k} \ne k^*$.
\end{proof}
\section{Details on Remark~\ref{rem: gap generalization} (Comparison to Existing Gap Definitions)}
\label{sec: appendix gap definition generalization}


We first recall some existing arm gap definitions for the exact quantile bandit problem (i.e., $\epsilon = 0$) in the setting of unquantized rewards.
In \cite[Definition 2]{nikolakakis2021quantile}, the authors defined the gap $\Delta_k^{\mathrm{NKSS}} $ for each suboptimal arm $k \ne k^*$ by
\begin{equation}
\label{eq: gap NKSS}
    \Delta_k^{\mathrm{NKSS}} \coloneqq
     \sup
    \{
         \Delta \in \left[0, \min(q, 1-q) \right]
        : 
        Q_k(q + \Delta)
        \le
        Q_{k^*}(q - \Delta)
    \}.
\end{equation}
While the authors did not define the arm gap for $k^*$, we can take it to be the same as the gap of the ``best'' suboptimal arm, as their algorithm terminates only when all suboptimal arms are eliminated.
On the other hand, the arm gap defined in
\cite[(Eq. (27)]{howard2022sequential}
is given by
\begin{equation}
\label{eq: gap HR}
    \Delta_k^{\mathrm{HR}} \coloneqq
    \begin{cases}
     \sup
    \{
         \Delta \in \left[0, \min(q, 1-q) \right]
        : 
        Q_k(q + \Delta)
        \le
        \max\limits_{a \in \A}
        Q_a(q - \Delta)
    \}
    & \text{if } k \ne k^*
    \\
     \sup
    \{
        \Delta 
        \in \left[0, q \right] :
        Q_k(q - \Delta)
        \ge
        \max\limits_{a \neq k}
        Q_{a}\big(q + \Delta_a^{\mathrm{HR}}
        \big)
    \}
    & \text{if } k = k^* 
    \end{cases}.
\end{equation}
Similar to our arm gap definition (Definition~\ref{def: our gap}), the gaps $\Delta_k^{\mathrm{HR}}$ for suboptimal arms $k \ne k^*$ are not defined based on the quantile function of $k^*$. It follows that $\Delta_a^{\mathrm{HR}} \ge \Delta_a^{\mathrm{NKSS}}$ for all arms $a \in \A$.

We now study the effect of taking $c \to \infty$ in our gap, which is given below in~\eqref{eq: gap k infinite c}. From~\eqref{eq: gap k infinite c}, it is straightforward to verify that~\eqref{eq: gap HR} is recovered from our gap (Definition~\ref{def: our gap}) by using only lower quantile functions and taking $S = \A$ and $c \to \infty$. 

\textbf{Effect of parameter $c$ in the gap definition.}
For any $1 \le c_1 \le c_2$, let 
\begin{equation}
\label{eq: tilde eps 1 and 2}
    \tilde{\epsilon_1} = \frac{\lambda} {\left\lceil (c_1+1) \lambda/\epsilon \right\rceil}
    \quad 
    \text{and}
    \quad
    \tilde{\epsilon_2} =  \frac{\lambda} {\left\lceil (c_2+1) \lambda/\epsilon \right\rceil}
\end{equation} 
be as defined 
using Lines~\ref{line: number of points}--\ref{line: tilde epsilon} of 
in Algorithm~\ref{alg: main}. It can readily be verified that 
\begin{equation}
\label{eq: c1 tilde eps 1 and 2}
    \tilde{\epsilon_1} \ge \tilde{\epsilon_2}
    \quad \text{and} \quad
    c_1 \tilde{\epsilon_1} \le c_2 \tilde{\epsilon_2} \le \epsilon
    \quad \text{and} \quad
    \Delta_{k}(\nu, \lambda, \epsilon, c_1, q)
\le \Delta_{k}(\nu, \lambda, \epsilon, c_2, q).
\end{equation}
Since $\lim\limits_{c \to \infty}  \tilde{\epsilon} = 0$
and $\lim\limits_{c \to \infty} c \tilde{\epsilon} = \epsilon$,
the gap as defined in Definition~\ref{def: our gap} converges to a quantity $\Delta_{k} \coloneqq \Delta_{k}(\nu, \epsilon, 
    q) = \lim\limits_{c \to \infty}
    \Delta_{k}(\nu, \lambda, \epsilon, c, q)$, given by
\begin{equation}
\begin{aligned}
\label{eq: gap k infinite c}
   \Delta_{k} =
    \begin{cases}
    \sup
    \left\{
        \Delta \in \left[0, \min(q, 1-q) \right]
        \colon
        Q_k(q + \Delta) 
        \le
        \max\limits_{a \in \A  }
        Q^+_{a}(q - \Delta) 
        \right\}
    &\hspace{-2mm} \text{if }  k \not\in \A_{\epsilon} \\
   \max\limits_{\A_{\epsilon} \subseteq S }
   \left\{
        \sup
    \Big\{
        \Delta \in 
       \Big[0, \min\limits_{a \not\in S} \Delta_{a}  \Big]
        % \ \middle\vert\
        \colon
        Q^+_k(q - \Delta) 
        \ge
        \max\limits_{ a \in S \setminus \{k\}} 
        Q_{a}(q + \Delta) - \epsilon
        \Big\}
        \right\}
    &\hspace{-2mm} \text{if } k \in \A_{\epsilon} 
    \end{cases}.
\end{aligned}
\end{equation}
    Note that $\Delta_k$ is independent of $c$ and $\lambda$.
    

\begin{remark}[Use of upper quantile function]
    \label{rem: upper quantile}
    To our knowledge, we are the first to incorporate upper quantile functions in the gap definition.
    This may lead to a potentially larger arm gap as compared to defining using only lower quantile functions (e.g., changing $Q_a^+(\cdot)$ and $Q_k^+(\cdot)$ in~\eqref{eq: our gap} and~\eqref{eq: Delta k^S} to $Q_a(\cdot)$ and $Q_k(\cdot)$ respectively), and hence a tighter upper bound.
\end{remark}

\begin{remark}[Dependency on $Q_{k^*}(q-\Delta)$]
    Existing papers using an elimination-based algorithm have their arm gaps defined according to $Q_{k^*}(q-\Delta)$; see~\eqref{eq: gap NKSS} for an example.
    In contrast, we remove this dependency and define using $\max\limits_{a \in \A  }
        Q^+_{a}(q - \Delta)$, which may lead to a tighter upper bound.  
    The resulting analysis required is more challenging --
    see the discussion in Remark~\ref{rem: elim suboptimal}. 
\end{remark}



Since our gap definitions generalizes existing gap definitions, we expect that their gaps being positive on an instance $\nu$ would imply our gap being positive on $\nu$. That is, their gaps being positive is a sufficient condition for Algorithm~\ref{alg: main} to return a satisfying arm with high-probability (see Corollary~\ref{cor: combined guarantee}).

\begin{proposition}
\label{prop: generalized formulation}
     Fix an instance $\nu \in \cE$ that has a unique arm $k^*$ with the highest $q$-quantile. Let $\Delta_a^{\mathrm{NKSS}}$ and 
$\Delta_a^{\mathrm{HR}}$ be as defined in~\eqref{eq: gap NKSS} and \eqref{eq: gap HR} for each $a \in \A$.
If $\min\limits_{a \in \A} 
\left\{ \Delta_a^{\mathrm{NKSS}} \right\} > 0$
 or
 $\min\limits_{a \in \A} 
\left\{ \Delta_a^{\mathrm{HR}} \right\} > 
 0$, then $\Delta =  \Delta(\nu, \lambda, \epsilon, c, q) $ as defined in Theorem~\ref{theorem: upper bound} is also positive.
\end{proposition}

\begin{proof}
    It suffices to consider the case
    $ \min\limits_{a \in \A}  
    \left\{ \Delta_a^{\mathrm{HR}} \right\} > 0$,
    since $\Delta_a^{\mathrm{HR}} \ge \Delta_a^{\mathrm{NKSS}}$ for all arms $a \in \A$.
    Let $\eta = \frac{1}{2} \min\limits_{a \in \A}  \Delta_a^{\mathrm{HR}} > 0$.
    Then we have
    \begin{equation}
    \label{eq: positive HR implies positive gap}
        Q_{k^*}^+(q - \eta)
        \ge 
         Q_{k^*}(q - \eta)
        \ge 
        \max\limits_{a \neq k}
        Q_{a}\big(q + \Delta_a^{\mathrm{HR}}\big)
        \ge
        \max\limits_{a \in \A \setminus \{k^*\} } Q_a(q + \eta) - c \tilde{\epsilon},
    \end{equation}
    where the second inequality follows from~\eqref{eq: gap HR} 
    and $\tilde{\epsilon} = \tilde{\epsilon}(\lambda, \epsilon, c)$ is as defined in Algorithm~\ref{alg: main}.
    Combining~\eqref{eq: positive HR implies positive gap} and~\eqref{eq: Delta k^S} of our gap definition,  we have
    \begin{equation}
        \max\limits_{a \in \A_{\epsilon}} \Delta_{a} 
        \ge 
        \Delta_{k^*}  = \max\limits_{\A_{\epsilon} \subseteq S \subseteq \A}
        \Delta_{k^*}^{(S)} 
        \ge
        \Delta_{k^*}^{(\A)} \ge \eta > 0
    \end{equation}
    as desired.
\end{proof}



We now show that the converse is not true in general. 
In other words, there exists an instance $\nu \in \cE$ where no algorithm can distinguish which arm has a higher quantile using a finite number of arm pulls (see \cite[Theorem 2]{nikolakakis2021quantile}), but Algorithm~\ref{alg: main} is capable of returning an $\epsilon$-satisfying arm with high probability.

\begin{proposition}
\label{prop: converse zero gap not true}
    Fix $\lambda \ge \epsilon > 0$ and $\delta \in (0, 0.5)$.
    There exists a two-arm bandit instance $\nu \in \cE$ that has a unique arm $k^*$ with the highest median such that
    $\Delta =  \Delta(\nu, \lambda, \epsilon, c, q) $ as defined in Theorem~\ref{theorem: upper bound} is positive for $c \ge 2$,
    but $\min\limits_{a \in \A} \big\{ \Delta_a^{\mathrm{NKSS}}  \big\} 
= \min\limits_{a \in \A} \big\{ \Delta_a^{\mathrm{HR}}  \big\} = 0$.
\end{proposition}

\begin{proof}
    Consider two arms $\A = \{1, 2\}$ with the following CDFs:
\begin{equation}
    F_1(x) = 
    \begin{cases}
        0  & \text{ for } x < 0
        \\
        \frac{x}{2m_1} & \text{ for } 0 \le x < 2m_1 \\
        1 & \text{ for } x \ge 2m_1
    \end{cases}
    \quad \text{and} \quad
    F_2(x) = 
    \begin{cases}
        0 & \text{ for } x < m_2 \\
        0.5 & \text{ for } m_2 \le x < 2m_1 \\
        1 & \text{ for } x \ge 2 m_1
    \end{cases},
\end{equation}
where $ m_2 \in (m_1 - \epsilon/2, m_1)$
such that both arms are $\epsilon$-optimal, with arm 1 being the unique best arm. 
Note that for each $\eta > 0$, we have
\begin{equation}
    Q_2(0.5 + \eta) = 2 m_1
    > m_1 = Q_1(0.5) \ge Q_1(0.5 - \eta),
\end{equation}
and so $\Delta_2^{\mathrm{NKSS}} = \Delta_2^{\mathrm{HR}} = 0$.
However, under our gap definition (Definition~\ref{def: our gap}) with $\A_{\epsilon}(\nu) = \{1, 2\} = \A$ and any $c \ge 2$, we have
\begin{align}
    \Delta \ge \Delta_2 
    \ge \Delta_{2}^{(\{1,2\})}
    &=
    \sup
    \left\{
        \Delta \in [0, 0.5]
        :
        Q^+_2(0.5 - \Delta) 
        \ge
        Q_{1}(0.5 + \Delta) - c\tilde{\epsilon}
        \right\} \\
    &\ge
    \sup
    \left\{
        \Delta  \in [0, 0.5]
        :
        Q^+_2(0.5 - \Delta) 
        \ge
        Q_{1}(0.5 + \Delta) - \frac{\epsilon}{2}
        \right\} \\    
    &=
    \sup
    \left\{
        \Delta  \in [0, 0.5]
        :
        m_2
        \ge
        (1+2 \Delta) m_1 - \frac{\epsilon}{2}
        \right\} \\
     &= \min\left\{0.5, \frac{m_2 - (m_1 - \epsilon/2)}{2m_1} \right\} >0,
\end{align}
where the second inequality follows from the calculation in Remark~\ref{rem: picking large enough c}, and the last inequality follows from the assumptions that $m_1 > 0$ and $m_2 > m_1 - \epsilon/2$.
\end{proof}



\section{Proof of Theorem~\ref{theorem: upper bound} (Upper Bound of Algorithm~\ref{alg: main})}
\label{sec: appendix upper bound}
We break down the upper bound on the number of arm pulls used by Algorithm~\ref{alg: main} as follows. We bound the number of rounds required for a non-satisfying arm $k \not\in \A_{\epsilon}(\nu)$ to be eliminated in Lemma~\ref{lem: elim suboptimal}. Then in Lemma~\ref{lem: termination}, we bound the number of rounds each non-eliminated arm has gone through when the termination condition of the while-loop is triggered. Combining these lemmas with the number of arm pulls used by $\mathtt{QuantEst}$ for each round index $t \ge 1$ and active arm $k \in \A_t$ as stated in~\eqref{eq: QuantEst arm pulls}
gives us an upper bound on the total number of arm pulls.

We first present a useful lemma that will be used in the proofs of the two subsequent lemmas. 

\begin{lemma}[$\max \mathrm{LCB}$ is non-decreasing]
\label{lem: max LCB increasing}
    Under Event $E$ as defined in Lemma~\ref{lem: good events}, we have 
    \begin{equation}
    \max\limits_{a \in \mathcal{A}_{t}} 
            \mathrm{LCB}_{t}(a) \ge 
    \max\limits_{a \in \mathcal{A}_{\tau}} 
            \mathrm{LCB}_{\tau}(a).
    \end{equation}
    for all rounds $t > \tau  \ge 1$.
\end{lemma}
\begin{proof}
    Let round index $\tau \ge 1$ be arbitrary
    and let $k \in \argmax\limits_{a \in \A_{\tau}} 
            \mathrm{LCB}_{\tau}(a).$
    We have $k \in \A_{\tau+1}$ since 
    $\mathrm{UCB}_{\tau}(k) > \mathrm{LCB}_{\tau}(k) 
    = \max\limits_{a \in \A_{\tau}} 
            \mathrm{LCB}_{\tau}(a)$ by~\eqref{eq:  quantile anytime bound} 
    of the anytime quantile bounds.        
    It then follows that
    \begin{equation}
        \max\limits_{a \in \mathcal{A}_{\tau+1}} 
            \mathrm{LCB}_{\tau+1}(a) 
        \ge \mathrm{LCB}_{\tau+1}(k) 
        \ge \mathrm{LCB}_{\tau}(k) = 
        \max\limits_{a \in \A_{\tau}} 
            \mathrm{LCB}_{\tau}(j),
    \end{equation}    
    where the second inequality follows from~\eqref{eq:  quantile anytime bound} 
    of the anytime quantile bounds. 
    Applying the argument repeatedly yields the claim for all $t > \tau.$
\end{proof}

\begin{lemma}[Elimination of non-satisfying arms]
\label{lem: elim suboptimal}
     Fix an instance $\nu \in \cE$, and suppose Algorithm~\ref{alg: main} is run with input $(\A, \lambda, \epsilon, q, \delta)$ and parameter $c \ge 1$.
    Let $\A_{\epsilon} = \A_{\epsilon}(\nu) $ be as defined in~\eqref{def: performance def} and let the gap $\Delta_{k} = \Delta_{k}(\nu, \lambda, \epsilon, c, q)$ be as defined in Definition~\ref{def: our gap} 
    for each arm $k \in \A$.
    Consider an arm $k \not\in \A_{\epsilon}$.
    Under Event $E$ as defined in Lemma~\ref{lem: good events}, when the round index~$t$
    of Algorithm~\ref{alg: main} satisfies $\Delta^{(t)}  \le \frac{1}{2} \Delta_k$, we have  $k \not\in \A_{t+1}$.
\end{lemma}
\begin{proof}
    If $k \not\in \A_t$, then $k \not\in \A_{t+1}$ trivially. Therefore, we assume for the rest of the proof that $k \in \A_t$, and we will show that
    \begin{equation}
    \label{eq: eliminate condition}
        \mathrm{UCB}_t(k) \le \max\limits_{a \in \mathcal{A}_{t}} \mathrm{LCB}_t(a)
    \end{equation}
    or equivalently
    \begin{equation}
    \label{eq: eliminate condition equivalent}
        \mathrm{UCB}_t(k) < \max\limits_{a \in \mathcal{A}_{t}} \mathrm{LCB}_t(a) + \tilde{\epsilon},
    \end{equation}
    where $\tilde{\epsilon} = \tilde{\epsilon}(\lambda, \epsilon, c)$ is as defined in Lines~\ref{line: number of points} and~\ref{line: tilde epsilon} of Algorithm~\ref{alg: main}.
    Note that these conditions are equivalent because both
    $\mathrm{UCB}_t(k)$ and
    $\max\limits_{a \in \A_t } \mathrm{LCB}_t(a)$ 
    are elements of 
    \begin{equation}
        \left[ 0, 
        \tilde{\epsilon}, 
        2\tilde{\epsilon}, \cdots,
        (n-1) \tilde{\epsilon}, \lambda\right],
    \end{equation}
   which follows from Lines~\ref{line: list of points},~\ref{eq: initiate default conf interval}, and~\ref{ltk def}--\ref{UCB definition} of Algorithm~\ref{alg: main}.

    
    Since $k \not\in \A_{\epsilon}$, when the round index $t$ satisfies~$\Delta^{(t)} \le \frac{1}{2} \Delta_k $ we have
    \begin{equation}
    \label{eq: gap k realized with arm j}
        \mathrm{UCB}_t(k)
        < Q_k \big( q + \Delta^{(t)} \big)  + \tilde{\epsilon} 
        \le Q^+_{j}\big(q - \Delta^{(t)} \big) 
    \end{equation}
    for some arm $j \in \A$  by~\eqref{eq: upper approx quantile anytime bound} of the anytime quantile bounds
    and Definition~\ref{def: our gap}. 
    We now consider two cases: (i) $j \in \A_t$ and (ii) $j \not\in \A_t$.

    
    If $j \in \A_t$, we have
    \begin{equation}
    \label{eq: j in At}
        Q^+_{j}\big(q - \Delta^{(t)}\big) 
        \le \mathrm{LCB}_t(j) + \tilde{\epsilon} 
        \le \max\limits_{a \in \mathcal{A}_{t}} \mathrm{LCB}_t(a) + \tilde{\epsilon} 
    \end{equation}
    by~\eqref{eq: lower approx quantile anytime bound} of the anytime quantile bounds and the assumption that $j \in \A_t$. Combining~\eqref{eq: gap k realized with arm j} and~\eqref{eq: j in At} gives us condition~\eqref{eq: eliminate condition equivalent} as desired.
    
    If $j \not\in \A_t$, then it is eliminated at some round $\tau < t$, i.e., 
    $j \in \A_{\tau}$ but $j \not\in \A_{\tau + 1}$.
    By \eqref{eq: quantile anytime bound} of the anytime quantile bounds, the definition of active arm set (Line~\ref{line: active arm} of Algorithm~\ref{alg: main}) applied to $\A_{\tau + 1}$,
    and the fact that max LCB is non-decreasing (Lemma~\ref{lem: max LCB increasing}), 
    we have 
    \begin{equation}
    \label{eq: j not in At}
    Q_{j}(q) 
    \le
    \mathrm{UCB}_{\tau}(j) 
    \le
            \max\limits_{a \in \mathcal{A}_{\tau}} 
            \mathrm{LCB}_{\tau}(a)     
        \le
        \max\limits_{a \in \mathcal{A}_{t}} 
            \mathrm{LCB}_{t}(a). 
    \end{equation}
    Combining~\eqref{eq: gap k realized with arm j}, the trivial inequality 
    $Q^+_{j}\big(q - \Delta^{(t)}\big) 
        \le Q_{j}(q) $, and~\eqref{eq: j not in At} yields
    condition~\eqref{eq: eliminate condition} as desired.
\end{proof}

\begin{remark}
    \label{rem: elim suboptimal}
    As seen in the analysis for the case  $j \not\in \A_t$ above, the property that $\max \mathrm{LCB}$ is non-decreasing (Lemma~\ref{lem: max LCB increasing}) is crucial in establishing~\eqref{eq: j not in At}. We will see below that the same argument is used again in establishing~\eqref{eq: LCB_t(k) > Fj}. This property of Lemma~\ref{lem: max LCB increasing} itself is a consequence of ensuring $\mathrm{LCB}_t(k)$ is non-decreasing in $t$; see Remark~\ref{rem: LCB non decreasing}.
\end{remark}



\begin{lemma}[While-loop termination]
\label{lem: termination}
     Fix an instance $\nu \in \cE$, and suppose Algorithm~\ref{alg: main} is run with input $(\A, \lambda, \epsilon, q, \delta)$ and parameter $c \ge 1$.
    Let $\A_{\epsilon} = \A_{\epsilon}(\nu) $ be as defined in~\eqref{def: performance def} and let the gap $\Delta_{k} = \Delta_{k}(\nu, \lambda, \epsilon, c, q)$ be as defined in Definition~\ref{def: our gap} 
    for each arm $k \in \A$.
    Under Event $E$, when the round index~$t$
    of Algorithm~\ref{alg: main} satisfies $\Delta^{(t)} \le \frac{1}{2} \max \limits_{a \in \A_{\epsilon}} \Delta_a$, Algorithm~\ref{alg: main} will terminate in round $t+1$.
\end{lemma}

\begin{proof}
     If $\A_{t+1} = \{ k^* \}$, then
      \begin{equation}
          \max\limits_{a \in \A_{t+1} \setminus \{k^*\} }                 \mathrm{UCB}_t(a) - (c+1)\tilde{\epsilon}
      = -\infty \le \mathrm{LCB}_{t}(k^*),
      \end{equation}
     and so the algorithm will terminate and return arm $k^*$ in round $t+1$.
     Therefore, we assume for the rest of the proof that
     there exists another arm $a \ne k^*$ such that $a \in \A_{t+1}$. 

     We first show that the following condition is sufficient to trigger the termination condition of the while-loop (Lines~\ref{line: start while loop}--\ref{line: end while loop}) of Algorithm~\ref{alg: main}: There exists an arm $k \in \A_{t+1}$
     satisfying
    \begin{equation}
    \label{eq: suf cond trigger termination}
          \mathrm{LCB}_t(k)  
          \ge
          \max\limits_{a \in \A_{t+1} \setminus \{k\} }
        Q_{a}\big(q + \Delta^{(t)}\big) -  (c+1)\tilde{\epsilon} .
    \end{equation}
    Using~\eqref{eq: upper approx quantile anytime bound} of the anytime quantile bound, 
    condition~\eqref{eq: suf cond trigger termination} implies that
    \begin{equation}
    \label{eq: termination condition strict equality}
        \mathrm{LCB}_t(k)  
        >
          \max\limits_{a \in \mathcal{A}_{t+1} \setminus \{k\} } \mathrm{UCB}_t(a)
          - (c+2)\tilde{\epsilon},
    \end{equation}
    which is equivalent to the termination condition
    \begin{equation}
          \mathrm{LCB}_t(k)  
            \ge
          \max\limits_{a \in \mathcal{A}_{t+1} \setminus \{k\} } \mathrm{UCB}_t(a) - (c+1)\tilde{\epsilon},
    \end{equation}
    where the equivalence follows from an argument similar to the equivalence between~\eqref{eq: eliminate condition} and 
        \eqref{eq: eliminate condition equivalent}.

    It remains to pick an arm $k \in \A_{t+1}$ satisfying condition~\eqref{eq: suf cond trigger termination}.
    Let arm $j \in \argmax\limits_{a \in \A_{\epsilon}} \Delta_a$ and consider the following two cases: (i) $j \in \A_{t+1}$ and (ii) $j \not\in \A_{t+1}$.

    If $j \in \A_{t+1}$, we pick $k = j$. We also pick~$
    T \in \argmax\limits_{\A_{\epsilon} \subseteq S \subseteq \A}
        \Delta_{k}^{(S)}    
    $ 
    to be the set associated to $\Delta_k$ (see Definition~\ref{def: our gap}).
    Note that every arm that is not in $T$
    is a non-satisfying arm since 
    $\A_{\epsilon} \subseteq T$.
    Furthermore, every non-satisfying arm that is not in $T$, hence every arm that is not in $T$, is eliminated,
    which follows from Lemma~\ref{lem: elim suboptimal} and
    \begin{equation}
        \Delta^{(t)} 
        \le \frac{1}{2} \max \limits_{a \in \A_{\epsilon}} \Delta_a
        = 
        \frac{1}{2} \Delta_k \le \frac{1}{2} \min\limits_{a \not\in T} \Delta_a,
    \end{equation} 
    where the last inequality follows from applying~\eqref{eq: Delta k^S} to $k$ and $T$.
    Therefore, we have
    $\A_{t+1} \subseteq T$.
    It follows that
    \begin{align}
         \mathrm{LCB}_t(k)
          &\ge
          Q^+_k\big(q - \Delta^{(t)}\big) - \tilde{\epsilon} \\
          &\ge
        \max\limits_{a \in T \setminus \{k\} }
        Q_{a}\big(q + \Delta^{(t)}\big) - (c+1)\tilde{\epsilon} \\
        &\ge
      \max\limits_{a \in \A_{t+1} \setminus \{k\} }
        Q_{a}\big(q + \Delta^{(t)}\big) - (c+1)\tilde{\epsilon},
    \end{align}
    where the first inequality follows from~\eqref{eq: lower approx quantile anytime bound} of the anytime quantile bound, the second inequality follows from applying~\eqref{eq: Delta k^S} to $k$ and $T$,
    and the last inequality follows from  $\A_{t+1} \subseteq T$.

    If $j \not\in \A_{t+1}$,
    we pick an arm 
    $k \in \argmax\limits_{a \in \mathcal{A}_{t+1}} 
    \mathrm{LCB}_{t}(a)$ arbitrarily.
    We also pick $T \in \argmax\limits_{\A_{\epsilon} \subseteq S \subseteq \A} \Delta_{k}^{(S)}$ and  
    we have $\A_{t+1} \subseteq T$ as in the case above.
    Furthermore, since $j \not\in \A_{t+1}$,
    we have
    \begin{equation}
    \label{eq: LCB_t(k) > Fj}
        Q^+_j \big(q - \Delta^{(t)}\big)
    \le 
    Q_{j}(q) 
    \le
    \max\limits_{a \in \mathcal{A}_{t+1}} 
    \mathrm{LCB}_{t}(a) 
    = \mathrm{LCB}_{t}(k),
    \end{equation}
    where the second inequality follows from an argument similar to~\eqref{eq: j not in At}.
    It follows that
    \begin{align}
         \mathrm{LCB}_t(k) 
        &\ge Q^+_j \big(q - \Delta^{(t)}\big)  \\
        &\ge
        \max\limits_{a \in T \setminus \{j\} }
        Q_{a}\big(q + \Delta^{(t)}\big)   - c \tilde{\epsilon} \\
        &\ge
        \max\limits_{a \in \A_{t+1} \setminus \{k\} }
        Q_{a}\big(q + \Delta^{(t)}\big) - (c+1) \tilde{\epsilon},
    \end{align}
    where the first inequality follows from~\eqref{eq: LCB_t(k) > Fj}, the second inequality follows from applying~\eqref{eq: Delta k^S} to $j$ and $T$,
    and the last inequality follows from  $\A_{t+1} \subseteq T$.
\end{proof}

\section{Omitted Details in \pref{sec: lower_bound}}\label{app:lower_bound}
In this section, we show omitted proofs in \pref{sec: lower_bound}. We first prove \pref{thm:hedge_lower_bound}, which shows that Hedge suffers $\Omega(T^{\frac{1}{3}})$ alternating regret in the expert problem.
\begin{proof}[Proof of \pref{thm:hedge_lower_bound}]
We prove the lower bound by constructing two environments and showing that if the Hedge algorithm achieves $\order(T^{1/3})$ alternating regret for one environment, then it will suffer $\Omega(T^{1/3})$ alternating regret for the other one.

\textbf{Environment 1}: We consider the time horizon to be $3T$ and $3$ actions with the loss vector cycling between the three basis vectors in $\mathbb{R}^3: (1,0,0), (0,1,0), (0,0,1)$ and we have $\min_{i\in[3]} \sum_{t=1}^{3T} \ell_{t,i} = T$. Direct calculation shows that Hedge with learning rate $\eta>0$ predict the $p_t$ sequence as follows: $p_{3t-2}=\rbr{\frac{1}{3},\frac{1}{3},\frac{1}{3}}, p_{3t-1}=\rbr{\frac{e^{-\eta}}{2+e^{-\eta}},\frac{1}{2+e^{-\eta}},\frac{1}{2+e^{-\eta}}}, p_{3t}=\rbr{\frac{e^{-\eta}}{1+2e^{-\eta}},\frac{e^{-\eta}}{1+2e^{-\eta}},\frac{1}{1+2e^{-\eta}}}$, $t\in[T]$. Thus, we can bound the alternating regret as follows:
\begin{align*}
   & \RegAlt = T\rbr{\frac{2}{3}+\frac{1+e^{-\eta}}{2+e^{-\eta}}+\frac{1+e^{-\eta}}{1+2e^{-\eta}}} - 2T= \frac{T(1-e^{-\eta})^2}{3(2+e^{-\eta})(1+2e^{-\eta})}\ge \frac{T(1-e^{-\eta})^2}{27}.
\end{align*}
To proceed, note that when $\eta \geq 1$, the above inequality already means that $\RegAlt=\Omega(T)$. Therefore, we only consider the case when $\eta\leq 1$. Using the fact that $e^{-\eta} \le 1-\eta+\frac{\eta^2}{2}$ for $\eta\geq 0$, we can further lower bound $\frac{T(1-e^{-\eta})^2}{27}$ by $
    \RegAlt \ge \frac{T(\eta-\frac{{\eta}^2}{2}^2)}{27}\ge \frac{\eta^2T}{108},$
where the last inequality is due to $\eta\leq 1$. Therefore, we know that $\RegAlt = \Omega(\eta^2T)$.

\textbf{Environment 2}: We consider the time horizon to be $T$ and $3$ actions with the loss vector $\ell_t$ being $(1,0,0)$ for all rounds. Here, the benchmark $\min_{i\in[3]} \sum_{t=1}^T \ell_{t,i} $ is $0$, and $p_{t,1}=\frac{e^{-\eta T}}{2+e^{-\eta T}}$ for all $t\in[T]$. In this case, if $\eta\leq \frac{2}{T}$, we know that $p_{t,1}\geq \frac{e^{-2}}{2+e^{-2}}$ for all $t\in [T]$ and the algorithm will be suffering $\Omega(T)$ regret. When $1\geq \eta\geq \frac{2}{T}$, we have:
\begin{align*}
    \RegAlt &= \sum_{t=0}^T \frac{2e^{-\eta t}}{2+e^{-\eta t}} - \frac{1}{3} -\frac{e^{-\eta T}}{2+e^{-\eta T}} \ge \sum_{t=1}^{T-1} \frac{2}{1+2e^{\eta t}}\ge \int_{1}^{T} \frac{2}{3e^{\eta t}}dt= \frac{2}{3\eta}\left[-e^{-\eta t}\right]\Big|_{1}^{T}\ge\frac{e^{-1}}{3\eta},
\end{align*}
where the second inequality uses $e^{\eta t}\geq 1$ and the last inequality uses $\eta\leq 1$.
Therefore, we have $\RegAlt = \Omega\Big(\frac{1}{\eta}\Big)$. Combining both environments, we know that Hedge with learning rate $\eta$ suffers a $\Omega(\max\{\frac{1}{\eta},\eta^2 T\})$, leadning to a $\Omega(T^{\frac{1}{3}})$ lower bound.
\end{proof}


Next, we show that \PRM suffers $\Omega(\sqrt{T})$ alternating regret in the adversarial environment.

\begin{proof}[Proof of \pref{thm: PRM+}]
The procedure of \PRM is as follows: Let $\wh{R}_1 = {R}_1 = {r}_{0} = \bm{0}$, where $\bm{0}$ is a zero vector with all components equal to zero,
and for $t\ge 1$, \PRM selects $p_t$ to be $\hat{{R}}_t/\norm{\hat{{R}}_t}_1$ where $\hat{{R}}_t = [{R}_t+{r}_{t-1}]^+$ and update ${R}_{t+1}$ to be $[{R}_t+{r}_t]^+$, where ${r}_t = \langle {p}_t, {\ell}_t\rangle \bm{1}_d - {\ell}_t$.
Here, $\bm{1}_d$ is a vector with all components equal to one.
We use the following loss sequence to show the lower bound: for $k\ge 0$, consider
    \begin{align*}
    {\ell}_{2k}=
    \begin{bmatrix}
        1\\
        0
    \end{bmatrix},\quad
    {\ell}_{2k+1}=
    \begin{bmatrix}
        -0.5\\
        0
    \end{bmatrix}.
\end{align*}
To simplify notation, denote $\alpha_k=R_{2k+1,2}$ for $k\in [\frac{T}{2}-1]$. \pref{lem: PRM+} shows that for $k\ge 5$, $\alpha_k$ follows the following recurrence relation: $\alpha_{k+1}=\alpha_k+\frac{1}{1+\alpha_k}$. We use this recurrence to compute the values of $p_t$ for all rounds $t>10$.


Thus, using \pref{lem: PRM+}, the alternating loss (standard loss + cheating loss) for the rounds $t=2k+1$ and $t=2k+2$ can be calculated as
\begin{align*}
    \inner{{p}_{2k+1},\ell_{2k}+\ell_{2k+1}}+\inner{{p}_{2k+2},\ell_{2k+1}+\ell_{2k+2}}= \frac{1}{2(1+\alpha_k)}.
\end{align*}
Since the action $2$ always has a loss of 0, the benchmark here is 0.
Therefore, the alternating regret is:
\begin{equation}\label{eqn:prm+-regalt}
    \RegAlt = C+\frac{1}{2}\sum_{k=5}^{\frac{T}{2}}\frac{1}{1+\alpha_k},
\end{equation}
where $C$ is a constant bounding the regret for the first $10$ rounds.
To estimate the quantity above, we prove $\alpha_k\le 2\sqrt{k}-1$ by induction. The base case $k=5$ can be verified by direct calculation. Now, let us assume that the claim holds for $k$. Then, for $k+1$, 
\begin{align*}
\alpha_{k+1}&=\alpha_k +\frac{1}{1+\alpha_k}\le 2\sqrt{k}-1+\frac{1}{2\sqrt{k}}\le\frac{4k+1}{2\sqrt{k}}-1\le 2\sqrt{k+1}-1.
\end{align*}
The first inequality comes from the fact that the function $f(x)=x+\frac{1}{1+x}$ is monotonically increasing for $x\ge 0$.
Substituting it in \pref{eqn:prm+-regalt}, we get
\begin{align*}
\RegAlt\ge C+\frac{1}{4}\sum_{k=5}^{\frac{T}{2}} \frac{1}{\sqrt{k}}=\Theta(\sqrt{T}).
\end{align*}
Therefore, $\RegAlt = \Omega(\sqrt{T})$ for the loss sequence proposed.
\end{proof}

\begin{lemma}\label{lem: PRM+}
   Suppose that the loss vector sequence satisfies that $\ell_{2k}=\begin{bmatrix}
    1\\ 0
\end{bmatrix},{\ell}_{2k+1}=\begin{bmatrix}
    -0.5\\ 0
\end{bmatrix}$ for $k\geq 0$. Then, \PRM guarantees that for $k\geq 5$
\begin{align*}
&{p}_{2k+1}=\begin{bmatrix}
    0\\ 1
\end{bmatrix},{p}_{2k+2}=\begin{bmatrix}
    \frac{1}{1+\alpha_k}\\ \frac{\alpha_k}{1+\alpha_k}
\end{bmatrix},\\
&{R}_{2k+2}=\begin{bmatrix}
    0.5\\ \alpha_k
\end{bmatrix},R_{2k+3}=\begin{bmatrix}
    0\\ \alpha_k+\frac{1}{1+\alpha_k}
\end{bmatrix},\\
&\wh{R}_{2k+2}=\begin{bmatrix}
    1\\ \alpha_k
\end{bmatrix},\wh{R}_{2k+3}=\begin{bmatrix}
    0\\ \alpha_k+\frac{2}{1+\alpha_k}
\end{bmatrix},
\end{align*}
where $\alpha_5>2$ is certain constant and $\alpha_{k+1}=\alpha_k+\frac{1}{1+\alpha_k}$ for $k\geq 5$.
\end{lemma}

\begin{proof}
It can be verified that \pref{lem: PRM+} holds true when $k=5$. For $k>5$, we prove by induction. Suppose \pref{lem: PRM+} holds for $k$. Then, for $k+1$, we have,
\begin{align*}
    &{p}_{2k+3} =\frac{\wh{R}_{2k+3}}{\norm{\wh{R}_{2k+3}}_1} =\begin{bmatrix}
        0\\ 1
    \end{bmatrix},  \\
    &R_{2k+4} = [R_{2k+3}+r_{2k+3}]^+ = \begin{bmatrix}
        0.5\\
        \alpha_k+\frac{1}{1+\alpha_k}
    \end{bmatrix} =
    \begin{bmatrix}
        0.5\\
        \alpha_{k+1}
    \end{bmatrix},
    \\
    &r_{2k+3} = \inner{p_{2k+3},\ell_{2k+3}}\mathbf{1}_d - \ell_{2k+3} = \begin{bmatrix}
        0.5\\
        0
    \end{bmatrix},\\
    &\hat{R}_{2k+4} = [R_{2k+4}+r_{2k+3}]^+ = 
    \begin{bmatrix}
        1\\
        \alpha_{k+1} 
    \end{bmatrix}.
\end{align*}
Since $\alpha_k\ge 2$, we have $\alpha_{k+1}\ge 2$ as well. Using this, we can see that
\begin{align*}
    &{p}_{2k+4} =
    \frac{\hat{R}_{2k+4}}{\|\hat{R}_{2k+4}\|_1} = 
    \begin{bmatrix}
         \frac{1}{1+\alpha_{k+1}}\\
         \frac{\alpha_{k+1}}{1+\alpha_{k+1}}
    \end{bmatrix}, \\
    &r_{2k+4} = \inner{p_{2k+4},\ell_{2k+4}}\mathbf{1}_d - \ell_{2k+4} = \begin{bmatrix}
        -\frac{\alpha_{k+1}}{1+\alpha_{k+1}}\\
        \frac{1}{1+\alpha_{k+1}}
    \end{bmatrix}, \\
    &R_{2k+5} = [R_{2k+4}+r_{2k+4}]^+ = 
    \begin{bmatrix}
        0\\
        \alpha_{k+1}+\frac{1}{1+\alpha_{k+1}}
    \end{bmatrix}, \\
    &\hat{R}_{2k+5} = [R_{2k+5}+r_{2k+4}]^+ = 
    \begin{bmatrix}
        0\\
        \alpha_{k+1}+\frac{2}{1+\alpha_{k+1}}
    \end{bmatrix},
\end{align*}
where the third equality uses the fact that $\frac{\alpha_{k+1}}{1+\alpha_{k+1}}\geq \frac{2}{3}$. Thus, the claim is true for $k+1$, and hence it holds for all $k\ge 5$.
\end{proof}
\section{Proof of Theorem \ref{thm: zero gap is unsolvable} and Corollary~\ref{cor: zero gap is unsolvable} (Solvable Instances)}
% \section{Solvable Instances}
\label{sec: appendix solvable instance}

We first state a useful lemma for Theorem \ref{thm: zero gap is unsolvable}.

\begin{lemma} 
\label{lem:two_instances}
        Let $\lambda, \epsilon, c,$ and $q$ be given, 
        and let $\tilde{\epsilon} = \tilde{\epsilon}(\lambda, \epsilon, c)$ be as defined in 
        % Lines~\ref{line: number of points} and~\ref{line: tilde epsilon} of 
        Algorithm~\ref{alg: main}.
        Suppose that $\nu \in \cE$ is an instance with gap $\Delta(\nu, \lambda, \epsilon, c, q) = 0 $ and let $\eta_0 = \eta_0(\nu) > 0$ be the constant given in the assumption in Theorem \ref{thm: zero gap is unsolvable}.  
        Then, for each arm $k \in \A_{c \tilde{\epsilon}}(\nu)$ and each $\eta \in  (0, \eta_0) $, there exists another instance $\nu' \in \cE$ satisfying the following:
    \begin{itemize}[topsep=0pt, itemsep=0pt]
        \item There exists an arm $a \in \Ac \setminus \{k\}$ such that instances $\nu$ and $\nu'$ are identical for all arms in $\Ac \setminus \{a,k\}$;
        
        \item $\dTV(F_a,G_a) \le \eta$ and $\dTV(F_k,G_k) \le \eta$, where $F_{(\cdot)}$ and $G_{(\cdot)}$ represent the arm distributions for instances $\nu$ and $\nu'$ respectively;
        
        \item $k \notin \Ac_{c \tilde{\epsilon} }(\nu')$, i.e., 
        under relaxation parameter $c \tilde{\epsilon}$, arm $k$ is not a satisfying arm for instance $\nu'$.
    \end{itemize}
\end{lemma}

     
 \begin{proof}   
    Let $\nu \in \cE$ be an instance with gap $\Delta(\nu, \lambda, \epsilon, c, q) = 0$. For each arm $k \in \A_{\epsilon}(\nu)$, we have $\Delta_{k}^{(\A)} = 0$ by Definition~\ref{def: our gap} since 
    $0 \le \Delta_{k}^{(\A)}  \le \Delta_{k}  \le \Delta  = 0$.
    Applying \eqref{eq: Delta k^S} with set $S = \A$ yields:
     \begin{equation}
     \label{eq: arm a positive eta}
        \text{for each } k \in \A_{\epsilon}(\nu)
        \text{ and each } \eta > 0, 
        \text{ there exists } a \ne k
        \text{ such that }
         Q^+_{k}(q - \eta) 
        <
        % \max\limits_{ a \ne k} 
        Q_{a}(q + \eta) - c\tilde{\epsilon}.
     \end{equation}
    Fix an arm $k \in \A_{c \tilde{\epsilon}}(\nu)$ and $\eta \in (0, \eta_0)$.
    Since $c \tilde{\epsilon} \le \epsilon$ (see calculation in~\eqref{eq: tilde eps 1 and 2}--\eqref{eq: c1 tilde eps 1 and 2}), we have 
    $\A_{c \tilde{\epsilon}}(\nu) \subseteq \A_{\epsilon}(\nu)$, and hence $k \in \A_{\epsilon}(\nu)$. 
    It follows from~\eqref{eq: arm a positive eta} that there exists some arm $a \ne k$ that
    \begin{equation}
     \label{eq: arm a positive eta c tilde epsilon}
         Q^+_{k}(q - \eta) 
        <
        % \max\limits_{ a \ne k} 
         Q_{a}(q + \eta) - c\tilde{\epsilon}.
     \end{equation}
    We now construct instance $\nu'$ such that $\nu$ and $\nu'$
    have identical distributions for all arms in $\A \setminus \{a, k\}$, 
    while $F_a$ and $F_k$ are being replaced with $G_a$ and $G_k$ defined as follows:
    \begin{enumerate}[topsep=0pt, itemsep=0pt]
        \item 
        $G_a$ is any distribution obtained by moving $\eta$-probability mass from the interval $(-\infty, Q_a(q))$ to the point $Q_a(q+2\eta)$; 
        
        \item 
        $G_k$ is any distribution obtained by moving $\eta$-probability mass from the interval $(Q_k(q), \infty)$ to the point $ Q_k(q-2\eta)$.
    \end{enumerate}     
     Under these definitions and the assumption on $\eta_0$ in Theorem~\ref{thm: zero gap is unsolvable}, we can readily verify that
     \begin{equation}
     \label{eq: shifted q quantiles}
         (G_k)^{-1}(q) =  Q_k(q-\eta) 
         % \le Q^+_{k}(q - \eta) 
         \in [0, \lambda]
         \quad 
         \text{and}
         \quad  
         (G_a)^{-1}(q) = Q_a(q+\eta) \in   [0, \lambda]
     \end{equation}
     and
     \begin{equation}
         d_{\mathrm{TV}}(F_k, G_k) =  d_{\mathrm{TV}}(F_a, G_a) = \eta.
     \end{equation}
    
     Finally, combining~\eqref{eq: arm a positive eta c tilde epsilon} and~\eqref{eq: shifted q quantiles} yields
     \begin{equation}
     \label{eq: G_k unsatisfying}
          (G_k)^{-1}(q) 
         < 
         (G_a)^{-1}(q) - c\tilde{\epsilon},
     \end{equation}
     which implies $k \notin \Ac_{c \tilde{\epsilon} }(\nu')$. 
     By construction, $\nu'$ satisfies all three properties as desired.
\end{proof}


\begin{remark}
\label{rem: limit version of two instance lemma}
     We can obtain a ``limiting'' version of Lemma~\ref{lem:two_instances} in which we replace the gap $\Delta(\nu, \lambda, \epsilon, c, q)$ by $\Delta(\nu, \epsilon, q)$ as defined in Corollary~\ref{cor: zero gap is unsolvable} and the satisfying arm set $\A_{c \tilde{\epsilon}}(\cdot)$
     by $\A_{\epsilon}(\cdot)$.
     The proof is essentially identical.
     We construct instance $\nu'$ in a similar manner as above to satisfy the first two properties in the statement of Lemma~\ref{lem:two_instances}.
     The last property $(k \not\in \A_{\epsilon}(\nu'))$ then follows from the definition of the limit gap $\Delta_{k}(\nu, \epsilon, 
    q)$ as defined in~\eqref{eq: gap k infinite c}, which allows us to replace the $c\tilde{\epsilon}$ terms in~\eqref{eq: arm a positive eta},~\eqref{eq: arm a positive eta c tilde epsilon}, and~\eqref{eq: G_k unsatisfying} by $\epsilon$.    
\end{remark}


We proceed to prove Theorem \ref{thm: zero gap is unsolvable}.  
\begin{proof}[Proof of Theorem~\ref{thm: zero gap is unsolvable}]
Assume for contradiction that there exists some instance $\nu \in \cE$ satisfies $\Delta(\nu, \lambda, \epsilon, c, q) = 0 $, but is $c\tilde{\epsilon}$-solvable. 
Fix a $\delta \in (0, 1)$ satisfying
\begin{equation}
    \label{eq: delta very small}
    \delta < \frac{1}{2+2|\A|}.
\end{equation}
By Definition \ref{def:solvable}, there exists a $(c\tilde{\epsilon}, \delta)$-reliable algorithm such that
\begin{equation}
    \PP_{\nu}[\tau < \infty \cap \hat{k} \in \Ac_{c\tilde{\epsilon}}(\nu)] \ge 1-\delta.
\end{equation}
In general the condition $\tau < \infty$ may not imply a \emph{uniform} upper bound on $\tau$; we handle this by relaxing the probability from $1-\delta$ to $1-2\delta$, such that there exists some $\tau_{\max} < \infty$ satisfying
\begin{equation}
    \PP_{\nu}[\hat{k} \in \Ac_{c\tilde{\epsilon}}(\nu) \cap \tau \le \tau_{\max}] \ge 1-2\delta. \label{eq:tau_max}
\end{equation}
From this, we claim that there exists an arm $k_{\nu} \in \Ac_{c\tilde{\epsilon}}(\nu)$ such that
\begin{equation}
    \PP_{\nu}[\hat{k} = k_{\nu} \cap \tau \le \tau_{\max}] \ge \frac{1-2\delta}{|\Ac|}. 
    \label{eq:success_nu}
\end{equation}
Indeed, if this were not the case, then summing these probabilities over elements in $\Ac_{\epsilon}(\nu)$ would produce a total below $1-2\delta$, which would contradict \eqref{eq:tau_max}.


Let $P_{\tau_{\max}}^{(\nu)}$ be the joint distribution on the $|\Ac| \times \tau_{\max}$ matrix of unquantized rewards:
the $(i,j)$-th entry of this matrix contains the $j$-th unquantized reward for arm $i$ under instance $\nu$. Under the event $\tau \le \tau_{\max}$, the algorithm's output does not depend on any rewards beyond those appearing in this matrix.  In other words, the output $\hat{k}$ is a (possibly randomized) function of this matrix.

By picking $\eta > 0$ to be sufficiently small in Lemma \ref{lem:two_instances}, we can find an instance $\nu' \in \cE$ such that $k_{\nu} \notin \Ac_{c\tilde{\epsilon}}(\nu')$ and
\begin{equation}
    \dTV\big( P_{\tau_{\max}}^{(\nu)}, P_{\tau_{\max}}^{(\nu')} \big) \le \delta.
\end{equation}
Here, $P_{\tau_{\max}}^{(\nu')}$ is defined similarly to $P_{\tau_{\max}}^{(\nu)}$, but for instance $\nu'$.
Since the output $\hat{k}$ is a (possibly randomized) function of the matrix defining $P_{\tau_{\max}}^{(\cdot)}$, we have
 \begin{equation}
    \label{eq: DPI}
     \dTV\big(  \PP_{\nu},  \PP_{\nu'} \big) \le \dTV\big( P_{\tau_{\max}}^{(\nu)}, P_{\tau_{\max}}^{(\nu')} \big) \le \delta
 \end{equation}
 by the data processing inequality for $f$-divergence~\cite[Theorem 7.4]{polyanskiy2024information}.
Using the definition $\dTV(P,Q) = \sup_{A} |P(A) - Q(A)|$, and applying~\eqref{eq: DPI},~\eqref{eq:success_nu},~\eqref{eq: delta very small}, we obtain
\begin{equation}
    \PP_{\nu'}[\hat{k} = k_{\nu} \cap \tau \le \tau_{\max}] \ge 
    \PP_{\nu}[\hat{k} = k_{\nu} \cap \tau \le \tau_{\max}] -
    \dTV\big(  \PP_{\nu},  \PP_{\nu'} \big)  
    % &\ge \PP_{\nu}[\hat{k} = k_{\nu} \cap \tau \le \tau_{\max}] -
    % \dTV\big( P_{\tau_{\max}}^{(\nu)}, P_{\tau_{\max}}^{(\nu')} \big)  \\
    \ge
    \frac{1-2\delta}{|\Ac|} - \delta > \delta.
    \label{eq:failure_nu'}
\end{equation}
Since $k_{\nu} \notin \Ac_{c \tilde{\epsilon} }(\nu')$, this means that the algorithm is \emph{not} $(c\tilde{\epsilon}, \delta)$-reliable (see Definition~\ref{def: reliable}), we have arrived at the desired contradiction.
\end{proof}

Corollary~\ref{cor: zero gap is unsolvable} can be proved similarly by using the ``limiting'' version of Lemma~\ref{lem:two_instances} (see Remark~\ref{rem: limit version of two instance lemma}).



\section{Details on Remark~\ref{rem: further improvement} (Improved Gap Definition)}
\label{sec: appendix potential improvement}

\subsection{Modified Arm Gaps}
We first state the modified gap definition explicitly by replacing $Q^+_{(\cdot)}(q - \Delta)$ and $Q_{(\cdot)}(q + \Delta)$ in Definition~\ref{def: our gap}
    with $\max\big\{0, Q^+_{(\cdot)}(q - \Delta)\big\}$ and $\min\big\{\lambda, Q_{(\cdot)}(q + \Delta)\big\}$ respectively, and provide an instance that has a positive modified gap but zero gap under the original definition.
    
\begin{definition}[Modified arm gaps]
\label{def: modified gap}
     Fix an instance $\nu \in \cE$.
     Let $\tilde{\epsilon}$ and $\A_{\epsilon}$ be as in Definition~\ref{def: our gap}.
    For each arm $k \in \A$, we define the improved gap $\tilde{\Delta}_{k} =
    \tilde{\Delta}_{k}(\nu, \lambda, \epsilon, c, q) \in \left[0, \min(q, 1-q) \right]$ as follows: 
    \begin{itemize}
        \item  
        
        If $k \not\in \A_{\epsilon}$, then $\tilde{\Delta}_{k}$ is defined as
        \begin{equation}
            \sup
            \left\{
                \Delta 
                \in \left[0, \min(q, 1-q) \right]
                \colon
               \min\{\lambda, Q_k(q + \Delta)   \}
                \le
                 \max\limits_{a \in \A  }
                 \left\{
                 \max\left\{0,  Q^+_{a}(q - \Delta) \right\}  - \tilde{\epsilon} 
                 \right\}
                \right\}
        \end{equation}

        \item

             If $k \in \A_{\epsilon}$, then we define $\tilde{\Delta}_{k} = \max\limits_{\A_{\epsilon} \subseteq S \subseteq \A}
        \tilde{\Delta}_{k}^{(S)}$, where 
        \begin{equation}
        \label{eq: improved Delta k^S}
            \tilde{\Delta}_{k}^{(S)} =
           \sup
            \Big\{
                \Delta \in 
               \Big[0, \min_{a \not\in S} \tilde{\Delta}_{a}  \Big]
                % \ \middle\vert\
                :
                \max\{0, Q^+_{k}(q - \Delta)\}
                \ge 
                \max\limits_{ a \in S \setminus \{k\}} 
                \min\{\lambda, Q_{a}(q + \Delta)\} - c \tilde{\epsilon}
                \Big\}
        \end{equation}
        for each subset $S$ satisfying $\A_{\epsilon} \subseteq S \subseteq \A$.
                    
    \end{itemize}
We use the convention that the minimum  (resp. maximum) of an empty set is $\infty$ (resp. $- \infty$).
\end{definition}

\begin{remark}[Intuition on the modified arm gap]
    Fix an instance $\nu = (F_k) \in \cE$.
     An interpretation of this modified gap is that
    $\tilde{\Delta}_{k}(\nu, \lambda, \epsilon, c, q) =
    \Delta_{k}(\mathrm{clipped}(\nu), \lambda, \epsilon, c, q)$,
    where $\mathrm{clipped}(\nu) = (\tilde{F}_k) \in \cE$
    is the instance with all distributions supported on $[0, \lambda]$ defined by
    \begin{equation}
        \tilde{F}_k(x)  =
    \begin{cases}
        0 & \text{ for } x < 0 \\
        F_k(x) & \text{ for } 0 \le x < \lambda \\
        1 & \text{ for } x > \lambda 
    \end{cases}
    \quad 
    \text{for each } k \in \A.
    \end{equation}
    That is, $\tilde{F}_k$ is obtained from $F_k$ by moving all mass below 0 to 0, and all mass above $\lambda$ to $\lambda$.  Note that an algorithm could be designed to clip rewards in this way, but our improved upper bound in Theorem \ref{theorem: modified upper bound} below applies even when Algorithm \ref{alg: main} is run without change.
\end{remark}

It is straightforward to verify that the modified gap is at least as large as the unmodified gap (Definition~\ref{def: our gap}), i.e., $\tilde{\Delta} \ge \Delta$. We provide an example of bandit instance that has positive gap under the modified definition but is zero using the unmodified definition. Consider $q = 1/2$, let $\lambda \ge 2 \epsilon > 0$, and consider two arms $\A = \{1, 2\}$ with an identical CDF as follows:
\begin{equation}
    F_1(x) = 
    F_2(x) =
    \begin{cases}
        0 & \text{ for } x < \lambda - \epsilon/3 \\
        0.5 & \text{ for } \lambda - \epsilon/3 \le x < 2 \lambda \\
        1 & \text{ for } x \ge 2 \lambda 
    \end{cases},
\end{equation}
and so both arms are satisfying, i.e., $\A_{\epsilon} = \A$.
Note that for any $\Delta > 0$, we have
\begin{equation}
    Q^+_2(0.5 - \Delta)  =
    \lambda - \epsilon/3 <
    2\lambda - \epsilon \le
     2\lambda - c\tilde{\epsilon} =
    Q_1(0.5 + \Delta) - c\tilde{\epsilon},
\end{equation}
where the second inequality follows from the discussion in~\eqref{eq: tilde eps 1 and 2}--\eqref{eq: c1 tilde eps 1 and 2}. It follows that
\begin{equation}
    \Delta_2 
    = \Delta_{2}^{\A}
    =
    \sup
    \left\{
        \Delta \in [0,0.5]
        :
        Q^+_2(0.5 - \Delta) 
        \ge
        Q_{1}(0.5 + \Delta) - c\tilde{\epsilon}
        \right\} 
    = 0
\end{equation}
under the original gap definition. By symmetry, we also have $\Delta_1 = 0$.
However, under the modified definition, we have
\begin{align}
     \tilde{\Delta}_2 
    = \tilde{\Delta}_{2}^{\A}
    &= \sup
    \left\{
        \Delta \in [0, 0.5]
        :
        \max\{0,  \lambda - \epsilon/2 \}
        \ge
        \min\{\lambda, 2 \lambda\} - c\tilde{\epsilon}
        \right\} \\
    &= \sup
    \left\{
        \Delta \in [0, 0.5]
        :
          \lambda - \epsilon/3 
        \ge
        \lambda  - c\tilde{\epsilon}
        \right\} \\
        &= 0.5,
\end{align}
    where the last inequality follows since 
    $c\tilde{\epsilon} \ge \epsilon/3 $ for any $c \ge 1$ 
    (see the calculation in Remark~\ref{rem: picking large enough c}).

\subsection{Improved Upper Bound}
With the modified gap definition, we obtain the following improved upper bound.

\begin{theorem}[Improved upper bound]
\label{theorem: modified upper bound}
   Fix an instance $\nu \in \cE$, and suppose Algorithm~\ref{alg: main} is run with input $(\A, \lambda, \epsilon, q, \delta)$ and parameter $c \ge 1$.
    Let $\A_{\epsilon}(\nu) $ be as defined in~\eqref{def: performance def} and let the gap $\tilde{\Delta}_{k} = \tilde{\Delta}_{k}(\nu, \lambda, \epsilon, c, q)$ be as defined in Definition~\ref{def: modified gap} 
    for each arm $k \in \A$.
    Under Event~$E$ as defined in Lemma~\ref{lem: good events},
    the total number of arm pulls is upper bounded~by    
    \begin{equation}
        O
        \left(
        \left(
        \sum_{ k \in \A }
        \dfrac{1}{ \max\big( \tilde{\Delta}_{k},  \tilde{\Delta}  \big)^2} \cdot 
        \left( 
         \log \left(\frac{1}{ \delta } \right) +
         \log \left(\frac{1}{ \max\big( \tilde{\Delta}_{k},  \tilde{\Delta}  \big)}\right) +
         \log \left(\frac{c \lambda K}{ \epsilon } \right)    
        \right)
        \right)
        \right),
    \end{equation}
    where $\tilde{\Delta}  =  \tilde{\Delta}(\nu, \lambda, \epsilon, c, q) = \max\limits_{a \in \A_{\epsilon}(\nu)} \tilde{\Delta}_{a}$.
\end{theorem}

The proof is essentially identical to the proof of Theorem~\ref{theorem: upper bound}, but requires tightening of~\eqref{eq: lower approx quantile anytime bound} and~\eqref{eq: upper approx quantile anytime bound}
of anytime quantile bound to
    \begin{equation} 
    \label{eq: modified lower approx quantile anytime bound}
       \max\{0, Q^+_k\big(q -  \Delta^{(t)} \big) \}
        \le \mathrm{LCB}_t(k) + \tilde{\epsilon}
    \end{equation}
    and
    \begin{equation} 
    \label{eq: modified upper approx quantile anytime bound}
        \mathrm{UCB}_t(k) 
        <
         \min\{\lambda, Q_k\big(q + \Delta^{(t)} \big)\} + \tilde{\epsilon}
    \end{equation}
respectively.
Note that the two new bounds \eqref{eq: modified lower approx quantile anytime bound} and~\eqref{eq: modified upper approx quantile anytime bound} can be verified easily using the properties that $ \mathrm{LCB}_t(k) \ge 0$ and $ \mathrm{UCB}_t(k) \le \lambda$ (see Lines~\ref{eq: initiate default conf interval},~\ref{LCB definition}, and~\ref{UCB definition} of Algorithm~\ref{alg: main}), as well as the established bounds~\eqref{eq: lower approx quantile anytime bound} and~\eqref{eq: upper approx quantile anytime bound}.



\subsection{Removing the Assumption in Theorem~\ref{thm: zero gap is unsolvable} (Unsolvability)}
\label{sec: assumption removal}

The assumption involving $\eta_0$ in Theorem~\ref{thm: zero gap is unsolvable} is included to ensure that both $(G_k)^{-1}(q) =  Q_k(q-\eta) $ and $(G_a)^{-1}(q) =  Q_a(q+\eta) $ are in $[0, \lambda]$ in the proof of Lemma~\ref{lem:two_instances}, so that the constructed instance $\nu'$ satisfies $\nu' \in \cE$. As mentioned in Remark~\ref{rem: remove additional assumption}, the assumption can be removed if we use the modified gap instead; formally, we have the following.


 \begin{theorem}[Zero gap is unsolvable -- assumption-free version]
 \label{thm: modified zero gap is unsolvable}
    Let $\lambda, \epsilon, c,$ and $q$ be fixed, 
    and let $\tilde{\epsilon} = \tilde{\epsilon}(\lambda, \epsilon, c)$ be as defined in 
    % Lines~\ref{line: number of points} and~\ref{line: tilde epsilon} of 
    Algorithm~\ref{alg: main}.
    Let $\tilde{\Delta} = \tilde{\Delta}(\nu, \lambda, \epsilon, c, q)$ be as defined in Theorem \ref{theorem: modified upper bound}.    
    If an instance $\nu \in \cE$ satisfies $\tilde{\Delta} = 0 $, then $\nu$ is $c\tilde{\epsilon}$-unsolvable.
 \end{theorem}


  The proof is essentially identical to the proof of Theorem~\ref{thm: zero gap is unsolvable}, and requires only some straightforward modifications in Lemma~\ref{lem:two_instances}. Specifically, under the new gap definition,~\eqref{eq: arm a positive eta c tilde epsilon} would be replaced by
     \begin{equation}
          \max\{0, Q^+_{k}(q - \eta)\}
       <
        \min\{\lambda, Q_{a}(q + \eta)\} - c \tilde{\epsilon}
        \Big\}
     \end{equation}
    We then construct instance $\nu'$ in a similar manner to the proof of Lemma~\ref{lem:two_instances}, but the definitions of $G_a$ and $G_k$ modified to include clipping:
    \begin{enumerate}[topsep=0pt, itemsep=0pt]
        \item 
        $G_a$ is any distribution obtained by moving $\eta$-probability mass from the interval $(-\infty, Q_a(q))$ to the point $\min\{\lambda, Q_{a}(q + 2\eta)\}$;
    
        \item 
        $G_k$ is any distribution obtained by moving $\eta$-probability mass from the interval $(Q_k(q), \infty)$ to the point $\max\{0, Q_{k}(q - 2\eta)\}$.    
    \end{enumerate}     
    It now follows that
     \begin{equation}
         (G_k)^{-1}(q) = \max\{0, Q_{k}(q - \eta)\} 
         \in [0, \lambda]
     \end{equation}
     and
      \begin{equation}
         (G_a)^{-1}(q) =  \min\{\lambda, Q_{a}(q + \eta)\} \in [0, \lambda],
     \end{equation}
     and hence $\nu' \in \cE$ as desired.


\end{document}
