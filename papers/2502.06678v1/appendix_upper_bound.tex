\section{Proof of Theorem~\ref{theorem: upper bound} (Upper Bound of Algorithm~\ref{alg: main})}
\label{sec: appendix upper bound}
We break down the upper bound on the number of arm pulls used by Algorithm~\ref{alg: main} as follows. We bound the number of rounds required for a non-satisfying arm $k \not\in \A_{\epsilon}(\nu)$ to be eliminated in Lemma~\ref{lem: elim suboptimal}. Then in Lemma~\ref{lem: termination}, we bound the number of rounds each non-eliminated arm has gone through when the termination condition of the while-loop is triggered. Combining these lemmas with the number of arm pulls used by $\mathtt{QuantEst}$ for each round index $t \ge 1$ and active arm $k \in \A_t$ as stated in~\eqref{eq: QuantEst arm pulls}
gives us an upper bound on the total number of arm pulls.

We first present a useful lemma that will be used in the proofs of the two subsequent lemmas. 

\begin{lemma}[$\max \mathrm{LCB}$ is non-decreasing]
\label{lem: max LCB increasing}
    Under Event $E$ as defined in Lemma~\ref{lem: good events}, we have 
    \begin{equation}
    \max\limits_{a \in \mathcal{A}_{t}} 
            \mathrm{LCB}_{t}(a) \ge 
    \max\limits_{a \in \mathcal{A}_{\tau}} 
            \mathrm{LCB}_{\tau}(a).
    \end{equation}
    for all rounds $t > \tau  \ge 1$.
\end{lemma}
\begin{proof}
    Let round index $\tau \ge 1$ be arbitrary
    and let $k \in \argmax\limits_{a \in \A_{\tau}} 
            \mathrm{LCB}_{\tau}(a).$
    We have $k \in \A_{\tau+1}$ since 
    $\mathrm{UCB}_{\tau}(k) > \mathrm{LCB}_{\tau}(k) 
    = \max\limits_{a \in \A_{\tau}} 
            \mathrm{LCB}_{\tau}(a)$ by~\eqref{eq:  quantile anytime bound} 
    of the anytime quantile bounds.        
    It then follows that
    \begin{equation}
        \max\limits_{a \in \mathcal{A}_{\tau+1}} 
            \mathrm{LCB}_{\tau+1}(a) 
        \ge \mathrm{LCB}_{\tau+1}(k) 
        \ge \mathrm{LCB}_{\tau}(k) = 
        \max\limits_{a \in \A_{\tau}} 
            \mathrm{LCB}_{\tau}(j),
    \end{equation}    
    where the second inequality follows from~\eqref{eq:  quantile anytime bound} 
    of the anytime quantile bounds. 
    Applying the argument repeatedly yields the claim for all $t > \tau.$
\end{proof}

\begin{lemma}[Elimination of non-satisfying arms]
\label{lem: elim suboptimal}
     Fix an instance $\nu \in \cE$, and suppose Algorithm~\ref{alg: main} is run with input $(\A, \lambda, \epsilon, q, \delta)$ and parameter $c \ge 1$.
    Let $\A_{\epsilon} = \A_{\epsilon}(\nu) $ be as defined in~\eqref{def: performance def} and let the gap $\Delta_{k} = \Delta_{k}(\nu, \lambda, \epsilon, c, q)$ be as defined in Definition~\ref{def: our gap} 
    for each arm $k \in \A$.
    Consider an arm $k \not\in \A_{\epsilon}$.
    Under Event $E$ as defined in Lemma~\ref{lem: good events}, when the round index~$t$
    of Algorithm~\ref{alg: main} satisfies $\Delta^{(t)}  \le \frac{1}{2} \Delta_k$, we have  $k \not\in \A_{t+1}$.
\end{lemma}
\begin{proof}
    If $k \not\in \A_t$, then $k \not\in \A_{t+1}$ trivially. Therefore, we assume for the rest of the proof that $k \in \A_t$, and we will show that
    \begin{equation}
    \label{eq: eliminate condition}
        \mathrm{UCB}_t(k) \le \max\limits_{a \in \mathcal{A}_{t}} \mathrm{LCB}_t(a)
    \end{equation}
    or equivalently
    \begin{equation}
    \label{eq: eliminate condition equivalent}
        \mathrm{UCB}_t(k) < \max\limits_{a \in \mathcal{A}_{t}} \mathrm{LCB}_t(a) + \tilde{\epsilon},
    \end{equation}
    where $\tilde{\epsilon} = \tilde{\epsilon}(\lambda, \epsilon, c)$ is as defined in Lines~\ref{line: number of points} and~\ref{line: tilde epsilon} of Algorithm~\ref{alg: main}.
    Note that these conditions are equivalent because both
    $\mathrm{UCB}_t(k)$ and
    $\max\limits_{a \in \A_t } \mathrm{LCB}_t(a)$ 
    are elements of 
    \begin{equation}
        \left[ 0, 
        \tilde{\epsilon}, 
        2\tilde{\epsilon}, \cdots,
        (n-1) \tilde{\epsilon}, \lambda\right],
    \end{equation}
   which follows from Lines~\ref{line: list of points},~\ref{eq: initiate default conf interval}, and~\ref{ltk def}--\ref{UCB definition} of Algorithm~\ref{alg: main}.

    
    Since $k \not\in \A_{\epsilon}$, when the round index $t$ satisfies~$\Delta^{(t)} \le \frac{1}{2} \Delta_k $ we have
    \begin{equation}
    \label{eq: gap k realized with arm j}
        \mathrm{UCB}_t(k)
        < Q_k \big( q + \Delta^{(t)} \big)  + \tilde{\epsilon} 
        \le Q^+_{j}\big(q - \Delta^{(t)} \big) 
    \end{equation}
    for some arm $j \in \A$  by~\eqref{eq: upper approx quantile anytime bound} of the anytime quantile bounds
    and Definition~\ref{def: our gap}. 
    We now consider two cases: (i) $j \in \A_t$ and (ii) $j \not\in \A_t$.

    
    If $j \in \A_t$, we have
    \begin{equation}
    \label{eq: j in At}
        Q^+_{j}\big(q - \Delta^{(t)}\big) 
        \le \mathrm{LCB}_t(j) + \tilde{\epsilon} 
        \le \max\limits_{a \in \mathcal{A}_{t}} \mathrm{LCB}_t(a) + \tilde{\epsilon} 
    \end{equation}
    by~\eqref{eq: lower approx quantile anytime bound} of the anytime quantile bounds and the assumption that $j \in \A_t$. Combining~\eqref{eq: gap k realized with arm j} and~\eqref{eq: j in At} gives us condition~\eqref{eq: eliminate condition equivalent} as desired.
    
    If $j \not\in \A_t$, then it is eliminated at some round $\tau < t$, i.e., 
    $j \in \A_{\tau}$ but $j \not\in \A_{\tau + 1}$.
    By \eqref{eq: quantile anytime bound} of the anytime quantile bounds, the definition of active arm set (Line~\ref{line: active arm} of Algorithm~\ref{alg: main}) applied to $\A_{\tau + 1}$,
    and the fact that max LCB is non-decreasing (Lemma~\ref{lem: max LCB increasing}), 
    we have 
    \begin{equation}
    \label{eq: j not in At}
    Q_{j}(q) 
    \le
    \mathrm{UCB}_{\tau}(j) 
    \le
            \max\limits_{a \in \mathcal{A}_{\tau}} 
            \mathrm{LCB}_{\tau}(a)     
        \le
        \max\limits_{a \in \mathcal{A}_{t}} 
            \mathrm{LCB}_{t}(a). 
    \end{equation}
    Combining~\eqref{eq: gap k realized with arm j}, the trivial inequality 
    $Q^+_{j}\big(q - \Delta^{(t)}\big) 
        \le Q_{j}(q) $, and~\eqref{eq: j not in At} yields
    condition~\eqref{eq: eliminate condition} as desired.
\end{proof}

\begin{remark}
    \label{rem: elim suboptimal}
    As seen in the analysis for the case  $j \not\in \A_t$ above, the property that $\max \mathrm{LCB}$ is non-decreasing (Lemma~\ref{lem: max LCB increasing}) is crucial in establishing~\eqref{eq: j not in At}. We will see below that the same argument is used again in establishing~\eqref{eq: LCB_t(k) > Fj}. This property of Lemma~\ref{lem: max LCB increasing} itself is a consequence of ensuring $\mathrm{LCB}_t(k)$ is non-decreasing in $t$; see Remark~\ref{rem: LCB non decreasing}.
\end{remark}



\begin{lemma}[While-loop termination]
\label{lem: termination}
     Fix an instance $\nu \in \cE$, and suppose Algorithm~\ref{alg: main} is run with input $(\A, \lambda, \epsilon, q, \delta)$ and parameter $c \ge 1$.
    Let $\A_{\epsilon} = \A_{\epsilon}(\nu) $ be as defined in~\eqref{def: performance def} and let the gap $\Delta_{k} = \Delta_{k}(\nu, \lambda, \epsilon, c, q)$ be as defined in Definition~\ref{def: our gap} 
    for each arm $k \in \A$.
    Under Event $E$, when the round index~$t$
    of Algorithm~\ref{alg: main} satisfies $\Delta^{(t)} \le \frac{1}{2} \max \limits_{a \in \A_{\epsilon}} \Delta_a$, Algorithm~\ref{alg: main} will terminate in round $t+1$.
\end{lemma}

\begin{proof}
     If $\A_{t+1} = \{ k^* \}$, then
      \begin{equation}
          \max\limits_{a \in \A_{t+1} \setminus \{k^*\} }                 \mathrm{UCB}_t(a) - (c+1)\tilde{\epsilon}
      = -\infty \le \mathrm{LCB}_{t}(k^*),
      \end{equation}
     and so the algorithm will terminate and return arm $k^*$ in round $t+1$.
     Therefore, we assume for the rest of the proof that
     there exists another arm $a \ne k^*$ such that $a \in \A_{t+1}$. 

     We first show that the following condition is sufficient to trigger the termination condition of the while-loop (Lines~\ref{line: start while loop}--\ref{line: end while loop}) of Algorithm~\ref{alg: main}: There exists an arm $k \in \A_{t+1}$
     satisfying
    \begin{equation}
    \label{eq: suf cond trigger termination}
          \mathrm{LCB}_t(k)  
          \ge
          \max\limits_{a \in \A_{t+1} \setminus \{k\} }
        Q_{a}\big(q + \Delta^{(t)}\big) -  (c+1)\tilde{\epsilon} .
    \end{equation}
    Using~\eqref{eq: upper approx quantile anytime bound} of the anytime quantile bound, 
    condition~\eqref{eq: suf cond trigger termination} implies that
    \begin{equation}
    \label{eq: termination condition strict equality}
        \mathrm{LCB}_t(k)  
        >
          \max\limits_{a \in \mathcal{A}_{t+1} \setminus \{k\} } \mathrm{UCB}_t(a)
          - (c+2)\tilde{\epsilon},
    \end{equation}
    which is equivalent to the termination condition
    \begin{equation}
          \mathrm{LCB}_t(k)  
            \ge
          \max\limits_{a \in \mathcal{A}_{t+1} \setminus \{k\} } \mathrm{UCB}_t(a) - (c+1)\tilde{\epsilon},
    \end{equation}
    where the equivalence follows from an argument similar to the equivalence between~\eqref{eq: eliminate condition} and 
        \eqref{eq: eliminate condition equivalent}.

    It remains to pick an arm $k \in \A_{t+1}$ satisfying condition~\eqref{eq: suf cond trigger termination}.
    Let arm $j \in \argmax\limits_{a \in \A_{\epsilon}} \Delta_a$ and consider the following two cases: (i) $j \in \A_{t+1}$ and (ii) $j \not\in \A_{t+1}$.

    If $j \in \A_{t+1}$, we pick $k = j$. We also pick~$
    T \in \argmax\limits_{\A_{\epsilon} \subseteq S \subseteq \A}
        \Delta_{k}^{(S)}    
    $ 
    to be the set associated to $\Delta_k$ (see Definition~\ref{def: our gap}).
    Note that every arm that is not in $T$
    is a non-satisfying arm since 
    $\A_{\epsilon} \subseteq T$.
    Furthermore, every non-satisfying arm that is not in $T$, hence every arm that is not in $T$, is eliminated,
    which follows from Lemma~\ref{lem: elim suboptimal} and
    \begin{equation}
        \Delta^{(t)} 
        \le \frac{1}{2} \max \limits_{a \in \A_{\epsilon}} \Delta_a
        = 
        \frac{1}{2} \Delta_k \le \frac{1}{2} \min\limits_{a \not\in T} \Delta_a,
    \end{equation} 
    where the last inequality follows from applying~\eqref{eq: Delta k^S} to $k$ and $T$.
    Therefore, we have
    $\A_{t+1} \subseteq T$.
    It follows that
    \begin{align}
         \mathrm{LCB}_t(k)
          &\ge
          Q^+_k\big(q - \Delta^{(t)}\big) - \tilde{\epsilon} \\
          &\ge
        \max\limits_{a \in T \setminus \{k\} }
        Q_{a}\big(q + \Delta^{(t)}\big) - (c+1)\tilde{\epsilon} \\
        &\ge
      \max\limits_{a \in \A_{t+1} \setminus \{k\} }
        Q_{a}\big(q + \Delta^{(t)}\big) - (c+1)\tilde{\epsilon},
    \end{align}
    where the first inequality follows from~\eqref{eq: lower approx quantile anytime bound} of the anytime quantile bound, the second inequality follows from applying~\eqref{eq: Delta k^S} to $k$ and $T$,
    and the last inequality follows from  $\A_{t+1} \subseteq T$.

    If $j \not\in \A_{t+1}$,
    we pick an arm 
    $k \in \argmax\limits_{a \in \mathcal{A}_{t+1}} 
    \mathrm{LCB}_{t}(a)$ arbitrarily.
    We also pick $T \in \argmax\limits_{\A_{\epsilon} \subseteq S \subseteq \A} \Delta_{k}^{(S)}$ and  
    we have $\A_{t+1} \subseteq T$ as in the case above.
    Furthermore, since $j \not\in \A_{t+1}$,
    we have
    \begin{equation}
    \label{eq: LCB_t(k) > Fj}
        Q^+_j \big(q - \Delta^{(t)}\big)
    \le 
    Q_{j}(q) 
    \le
    \max\limits_{a \in \mathcal{A}_{t+1}} 
    \mathrm{LCB}_{t}(a) 
    = \mathrm{LCB}_{t}(k),
    \end{equation}
    where the second inequality follows from an argument similar to~\eqref{eq: j not in At}.
    It follows that
    \begin{align}
         \mathrm{LCB}_t(k) 
        &\ge Q^+_j \big(q - \Delta^{(t)}\big)  \\
        &\ge
        \max\limits_{a \in T \setminus \{j\} }
        Q_{a}\big(q + \Delta^{(t)}\big)   - c \tilde{\epsilon} \\
        &\ge
        \max\limits_{a \in \A_{t+1} \setminus \{k\} }
        Q_{a}\big(q + \Delta^{(t)}\big) - (c+1) \tilde{\epsilon},
    \end{align}
    where the first inequality follows from~\eqref{eq: LCB_t(k) > Fj}, the second inequality follows from applying~\eqref{eq: Delta k^S} to $j$ and $T$,
    and the last inequality follows from  $\A_{t+1} \subseteq T$.
\end{proof}
