\section{Details on Remark~\ref{rem: further improvement} (Improved Gap Definition)}
\label{sec: appendix potential improvement}

\subsection{Modified Arm Gaps}
We first state the modified gap definition explicitly by replacing $Q^+_{(\cdot)}(q - \Delta)$ and $Q_{(\cdot)}(q + \Delta)$ in Definition~\ref{def: our gap}
    with $\max\big\{0, Q^+_{(\cdot)}(q - \Delta)\big\}$ and $\min\big\{\lambda, Q_{(\cdot)}(q + \Delta)\big\}$ respectively, and provide an instance that has a positive modified gap but zero gap under the original definition.
    
\begin{definition}[Modified arm gaps]
\label{def: modified gap}
     Fix an instance $\nu \in \cE$.
     Let $\tilde{\epsilon}$ and $\A_{\epsilon}$ be as in Definition~\ref{def: our gap}.
    For each arm $k \in \A$, we define the improved gap $\tilde{\Delta}_{k} =
    \tilde{\Delta}_{k}(\nu, \lambda, \epsilon, c, q) \in \left[0, \min(q, 1-q) \right]$ as follows: 
    \begin{itemize}
        \item  
        
        If $k \not\in \A_{\epsilon}$, then $\tilde{\Delta}_{k}$ is defined as
        \begin{equation}
            \sup
            \left\{
                \Delta 
                \in \left[0, \min(q, 1-q) \right]
                \colon
               \min\{\lambda, Q_k(q + \Delta)   \}
                \le
                 \max\limits_{a \in \A  }
                 \left\{
                 \max\left\{0,  Q^+_{a}(q - \Delta) \right\}  - \tilde{\epsilon} 
                 \right\}
                \right\}
        \end{equation}

        \item

             If $k \in \A_{\epsilon}$, then we define $\tilde{\Delta}_{k} = \max\limits_{\A_{\epsilon} \subseteq S \subseteq \A}
        \tilde{\Delta}_{k}^{(S)}$, where 
        \begin{equation}
        \label{eq: improved Delta k^S}
            \tilde{\Delta}_{k}^{(S)} =
           \sup
            \Big\{
                \Delta \in 
               \Big[0, \min_{a \not\in S} \tilde{\Delta}_{a}  \Big]
                % \ \middle\vert\
                :
                \max\{0, Q^+_{k}(q - \Delta)\}
                \ge 
                \max\limits_{ a \in S \setminus \{k\}} 
                \min\{\lambda, Q_{a}(q + \Delta)\} - c \tilde{\epsilon}
                \Big\}
        \end{equation}
        for each subset $S$ satisfying $\A_{\epsilon} \subseteq S \subseteq \A$.
                    
    \end{itemize}
We use the convention that the minimum  (resp. maximum) of an empty set is $\infty$ (resp. $- \infty$).
\end{definition}

\begin{remark}[Intuition on the modified arm gap]
    Fix an instance $\nu = (F_k) \in \cE$.
     An interpretation of this modified gap is that
    $\tilde{\Delta}_{k}(\nu, \lambda, \epsilon, c, q) =
    \Delta_{k}(\mathrm{clipped}(\nu), \lambda, \epsilon, c, q)$,
    where $\mathrm{clipped}(\nu) = (\tilde{F}_k) \in \cE$
    is the instance with all distributions supported on $[0, \lambda]$ defined by
    \begin{equation}
        \tilde{F}_k(x)  =
    \begin{cases}
        0 & \text{ for } x < 0 \\
        F_k(x) & \text{ for } 0 \le x < \lambda \\
        1 & \text{ for } x > \lambda 
    \end{cases}
    \quad 
    \text{for each } k \in \A.
    \end{equation}
    That is, $\tilde{F}_k$ is obtained from $F_k$ by moving all mass below 0 to 0, and all mass above $\lambda$ to $\lambda$.  Note that an algorithm could be designed to clip rewards in this way, but our improved upper bound in Theorem \ref{theorem: modified upper bound} below applies even when Algorithm \ref{alg: main} is run without change.
\end{remark}

It is straightforward to verify that the modified gap is at least as large as the unmodified gap (Definition~\ref{def: our gap}), i.e., $\tilde{\Delta} \ge \Delta$. We provide an example of bandit instance that has positive gap under the modified definition but is zero using the unmodified definition. Consider $q = 1/2$, let $\lambda \ge 2 \epsilon > 0$, and consider two arms $\A = \{1, 2\}$ with an identical CDF as follows:
\begin{equation}
    F_1(x) = 
    F_2(x) =
    \begin{cases}
        0 & \text{ for } x < \lambda - \epsilon/3 \\
        0.5 & \text{ for } \lambda - \epsilon/3 \le x < 2 \lambda \\
        1 & \text{ for } x \ge 2 \lambda 
    \end{cases},
\end{equation}
and so both arms are satisfying, i.e., $\A_{\epsilon} = \A$.
Note that for any $\Delta > 0$, we have
\begin{equation}
    Q^+_2(0.5 - \Delta)  =
    \lambda - \epsilon/3 <
    2\lambda - \epsilon \le
     2\lambda - c\tilde{\epsilon} =
    Q_1(0.5 + \Delta) - c\tilde{\epsilon},
\end{equation}
where the second inequality follows from the discussion in~\eqref{eq: tilde eps 1 and 2}--\eqref{eq: c1 tilde eps 1 and 2}. It follows that
\begin{equation}
    \Delta_2 
    = \Delta_{2}^{\A}
    =
    \sup
    \left\{
        \Delta \in [0,0.5]
        :
        Q^+_2(0.5 - \Delta) 
        \ge
        Q_{1}(0.5 + \Delta) - c\tilde{\epsilon}
        \right\} 
    = 0
\end{equation}
under the original gap definition. By symmetry, we also have $\Delta_1 = 0$.
However, under the modified definition, we have
\begin{align}
     \tilde{\Delta}_2 
    = \tilde{\Delta}_{2}^{\A}
    &= \sup
    \left\{
        \Delta \in [0, 0.5]
        :
        \max\{0,  \lambda - \epsilon/2 \}
        \ge
        \min\{\lambda, 2 \lambda\} - c\tilde{\epsilon}
        \right\} \\
    &= \sup
    \left\{
        \Delta \in [0, 0.5]
        :
          \lambda - \epsilon/3 
        \ge
        \lambda  - c\tilde{\epsilon}
        \right\} \\
        &= 0.5,
\end{align}
    where the last inequality follows since 
    $c\tilde{\epsilon} \ge \epsilon/3 $ for any $c \ge 1$ 
    (see the calculation in Remark~\ref{rem: picking large enough c}).

\subsection{Improved Upper Bound}
With the modified gap definition, we obtain the following improved upper bound.

\begin{theorem}[Improved upper bound]
\label{theorem: modified upper bound}
   Fix an instance $\nu \in \cE$, and suppose Algorithm~\ref{alg: main} is run with input $(\A, \lambda, \epsilon, q, \delta)$ and parameter $c \ge 1$.
    Let $\A_{\epsilon}(\nu) $ be as defined in~\eqref{def: performance def} and let the gap $\tilde{\Delta}_{k} = \tilde{\Delta}_{k}(\nu, \lambda, \epsilon, c, q)$ be as defined in Definition~\ref{def: modified gap} 
    for each arm $k \in \A$.
    Under Event~$E$ as defined in Lemma~\ref{lem: good events},
    the total number of arm pulls is upper bounded~by    
    \begin{equation}
        O
        \left(
        \left(
        \sum_{ k \in \A }
        \dfrac{1}{ \max\big( \tilde{\Delta}_{k},  \tilde{\Delta}  \big)^2} \cdot 
        \left( 
         \log \left(\frac{1}{ \delta } \right) +
         \log \left(\frac{1}{ \max\big( \tilde{\Delta}_{k},  \tilde{\Delta}  \big)}\right) +
         \log \left(\frac{c \lambda K}{ \epsilon } \right)    
        \right)
        \right)
        \right),
    \end{equation}
    where $\tilde{\Delta}  =  \tilde{\Delta}(\nu, \lambda, \epsilon, c, q) = \max\limits_{a \in \A_{\epsilon}(\nu)} \tilde{\Delta}_{a}$.
\end{theorem}

The proof is essentially identical to the proof of Theorem~\ref{theorem: upper bound}, but requires tightening of~\eqref{eq: lower approx quantile anytime bound} and~\eqref{eq: upper approx quantile anytime bound}
of anytime quantile bound to
    \begin{equation} 
    \label{eq: modified lower approx quantile anytime bound}
       \max\{0, Q^+_k\big(q -  \Delta^{(t)} \big) \}
        \le \mathrm{LCB}_t(k) + \tilde{\epsilon}
    \end{equation}
    and
    \begin{equation} 
    \label{eq: modified upper approx quantile anytime bound}
        \mathrm{UCB}_t(k) 
        <
         \min\{\lambda, Q_k\big(q + \Delta^{(t)} \big)\} + \tilde{\epsilon}
    \end{equation}
respectively.
Note that the two new bounds \eqref{eq: modified lower approx quantile anytime bound} and~\eqref{eq: modified upper approx quantile anytime bound} can be verified easily using the properties that $ \mathrm{LCB}_t(k) \ge 0$ and $ \mathrm{UCB}_t(k) \le \lambda$ (see Lines~\ref{eq: initiate default conf interval},~\ref{LCB definition}, and~\ref{UCB definition} of Algorithm~\ref{alg: main}), as well as the established bounds~\eqref{eq: lower approx quantile anytime bound} and~\eqref{eq: upper approx quantile anytime bound}.



\subsection{Removing the Assumption in Theorem~\ref{thm: zero gap is unsolvable} (Unsolvability)}
\label{sec: assumption removal}

The assumption involving $\eta_0$ in Theorem~\ref{thm: zero gap is unsolvable} is included to ensure that both $(G_k)^{-1}(q) =  Q_k(q-\eta) $ and $(G_a)^{-1}(q) =  Q_a(q+\eta) $ are in $[0, \lambda]$ in the proof of Lemma~\ref{lem:two_instances}, so that the constructed instance $\nu'$ satisfies $\nu' \in \cE$. As mentioned in Remark~\ref{rem: remove additional assumption}, the assumption can be removed if we use the modified gap instead; formally, we have the following.


 \begin{theorem}[Zero gap is unsolvable -- assumption-free version]
 \label{thm: modified zero gap is unsolvable}
    Let $\lambda, \epsilon, c,$ and $q$ be fixed, 
    and let $\tilde{\epsilon} = \tilde{\epsilon}(\lambda, \epsilon, c)$ be as defined in 
    % Lines~\ref{line: number of points} and~\ref{line: tilde epsilon} of 
    Algorithm~\ref{alg: main}.
    Let $\tilde{\Delta} = \tilde{\Delta}(\nu, \lambda, \epsilon, c, q)$ be as defined in Theorem \ref{theorem: modified upper bound}.    
    If an instance $\nu \in \cE$ satisfies $\tilde{\Delta} = 0 $, then $\nu$ is $c\tilde{\epsilon}$-unsolvable.
 \end{theorem}


  The proof is essentially identical to the proof of Theorem~\ref{thm: zero gap is unsolvable}, and requires only some straightforward modifications in Lemma~\ref{lem:two_instances}. Specifically, under the new gap definition,~\eqref{eq: arm a positive eta c tilde epsilon} would be replaced by
     \begin{equation}
          \max\{0, Q^+_{k}(q - \eta)\}
       <
        \min\{\lambda, Q_{a}(q + \eta)\} - c \tilde{\epsilon}
        \Big\}
     \end{equation}
    We then construct instance $\nu'$ in a similar manner to the proof of Lemma~\ref{lem:two_instances}, but the definitions of $G_a$ and $G_k$ modified to include clipping:
    \begin{enumerate}[topsep=0pt, itemsep=0pt]
        \item 
        $G_a$ is any distribution obtained by moving $\eta$-probability mass from the interval $(-\infty, Q_a(q))$ to the point $\min\{\lambda, Q_{a}(q + 2\eta)\}$;
    
        \item 
        $G_k$ is any distribution obtained by moving $\eta$-probability mass from the interval $(Q_k(q), \infty)$ to the point $\max\{0, Q_{k}(q - 2\eta)\}$.    
    \end{enumerate}     
    It now follows that
     \begin{equation}
         (G_k)^{-1}(q) = \max\{0, Q_{k}(q - \eta)\} 
         \in [0, \lambda]
     \end{equation}
     and
      \begin{equation}
         (G_a)^{-1}(q) =  \min\{\lambda, Q_{a}(q + \eta)\} \in [0, \lambda],
     \end{equation}
     and hence $\nu' \in \cE$ as desired.