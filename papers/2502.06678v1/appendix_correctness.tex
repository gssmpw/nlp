\section{Proof of Theorem~\ref{thm: correctness} (Reliability of Algorithm~\ref{alg: main})}
\label{sec: appendix correctness}
\begin{proof}[Proof of Theorem~\ref{thm: correctness}]
    We first show by induction that an optimal arm $k^*$ of instance $\nu$ (i.e., one having the highest $ q$-quantile) will always be active, i.e., $k^* \in \A_t$ for each round $t \ge 1$. 
        For the base case $t = 1$, we have $k^* \in \{1, \dots, K\} = \A_1$ trivially. We now show the inductive step: if $k^* \in \A_t$ holds, then $k^* \in \A_{t+1}$. For all arms $a \in \A_t$, we have
    \begin{equation}
        \mathrm{UCB}_t(k^*)
        \ge
        Q_{k^*}(q) 
        \ge Q_{a}(q)  
        > 
        \mathrm{LCB}_t(a),
    \end{equation}
    where the second inequality follows from the optimality of  arm $k^*$,
    while the other two inequalities follow from the anytime quantile bounds (Lemma~\ref{lem: quantile anytime bound}).
    It follows that
    $\mathrm{UCB}_t(k^*) >
            \max\limits_{a \in \mathcal{A}_{t}} \mathrm{LCB}_t(a)$,
    and so $k^* \in \A_{t+1}$ by definition (see Line~\ref{line: active arm} of Algorithm~\ref{alg: main}).


    We now argue that if Algorithm~\ref{alg: main} terminates, then the returned arm~$\hat{k}$ satisfies~\eqref{def: performance def}. 
    If Algorithm~\ref{alg: main} terminates,
    then the while-loop (Lines~\ref{line: start while loop}--\ref{line: end while loop}) must have terminated and therefore the returned arm $\hat{k}$
    satisfies the condition
    \begin{equation}
    \label{eq: k condition}
    \mathrm{LCB}_t(\hat{k})  \ge
                \max\limits_{a \in \A_t \setminus \{ \hat{k} \} }  
                \mathrm{UCB}_t(a) -  (c+1)\tilde{\epsilon}
                 \ge
          \max\limits_{a \in \A_t \setminus \{ \hat{k} \} }  
                \mathrm{UCB}_t(a)  - \epsilon,
    \end{equation}
    where the second inequality follows from Lines~\ref{line: number of points}--\ref{line: tilde epsilon} of Algorithm~\ref{alg: main}: $\tilde{\epsilon} \le \lambda \cdot \epsilon/((c+1) \lambda) = \epsilon/(c+1)$.
    If $\hat{k} = k^*$, then the returned arm satisfies~\eqref{def: performance def} trivially. Therefore, we assume that $\hat{k} \ne k^*$ for the rest of the proof. In this case, we have
    \begin{equation}
        Q_{\hat{k}}(q) >
        \mathrm{LCB}_t(\hat{k})  \ge
        \max\limits_{a \in \A_t \setminus \{ \hat{k} \} } 
        \mathrm{UCB}_t(a) -  
        \epsilon
        \ge 
        \mathrm{UCB}_t(k^*) -  \epsilon
        \ge
        \max\limits_{a \in \A_t \setminus \{ \hat{k} \} }
        Q_a(q) -  \epsilon.
    \end{equation}
    where the first and the last inequalities follow from the anytime quantile bounds (see Lemma~\ref{lem: quantile anytime bound}), while the second inequality follows from the condition~\eqref{eq: k condition} and the third inequality follows from $k^* \in \A_t$ (see above) and the assumption that $\hat{k} \ne k^*$.
\end{proof}