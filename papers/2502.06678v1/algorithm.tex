\section{Algorithm and Upper Bound}


In this section, we introduce our main algorithm and provide its performance guarantee. 

\subsection{Description of the Algorithm}
Our algorithm (Algorithm~\ref{alg: main}) is based on successive elimination, which is well-studied in the standard BAI problem and has also been adapted for other variations.  The algorithm pulls arms in \emph{rounds}, where each round consists of multiple pulls (namely, pulling all non-eliminated arms).  
For each arm~$k$ that is active at round~$t$,\footnote{We slightly abuse notation and use $t$ to index ``rounds'', each consisting of several arm pulls; it will be clear from the context whether $t$ is indexing a round or indexing the number of pulls so far.  We still use the \emph{total} number of arm pulls to characterize the performance of the algorithm.} the algorithm computes a confidence interval 
$\left[ \mathrm{LCB}_t(k), \mathrm{UCB}_t(k)\right]$ that
contains the $q$-quantile $Q_k(q)$ with high probability (see Lines~\ref{LCB definition} and~\ref{UCB definition}). Based on the confidence intervals, the algorithm eliminates arms that are suboptimal (see Line~\ref{line: active arm}). When the algorithm identifies that some arm satisfies~\eqref{def: performance def} based on the confidence bounds, it terminates and returns that arm (see Lines~\ref{line: start while loop} and~\ref{line: return}).

This high-level idea was also used in~\cite{szorenyi2015qualitative, nikolakakis2021quantile} for the quantile bandit problem with no communication constraint, but the procedures to obtain the confidence intervals are very different.
Their confidence intervals are computed using empirical quantiles of the (direct) observed rewards, which our learner does not have the luxury of accessing. Instead, we discretize the continuous interval $[0, \lambda]$ to a discrete interval $\left[0, \tilde{\epsilon}, 2 \tilde{\epsilon},  \ldots,  \lambda\right]$,\footnote{We use an input parameter $c > 0$ to control how finely the continuous interval is discretized; see Remark~\ref{rem: input c}.} and use a quantile estimation algorithm $\mathtt{QuantEst}$
to help us find $\mathrm{LCB}_t(k)$ and $\mathrm{UCB}_t(k)$ from the discretized interval; see Lines~\ref{line: number of points}--\ref{line: tilde epsilon} and Lines~\ref{ltk def}--\ref{UCB definition} respectively.
$\mathtt{QuantEst}$ can be implemented in our problem setup while respecting the 1-bit uplink communication constraint: the learner sends threshold queries in the form ``Is $r_{a_t, t} \le \gamma_t$?'' to the agent and receives 1-bit comparison feedback $\mathbf{1}(r_{a_t, t} \le \gamma_t)$.
Based on the feedback received, the learner then uses a noisy binary search strategy to compute $\mathrm{LCB}_t(k)$ and $\mathrm{UCB}_t(k)$. 
The details of $\mathtt{QuantEst}$ are deferred to Algorithm~\ref{alg: quantile interval} in Appendix~\ref{sec: appendix QuantEst}. 
 For now, we only need to treat $\mathtt{QuantEst}$ as a ``black box" with the following guarantee: Given input CDF $F$, non-decreasing list $\mathbf{x} = [x_1, \ldots, x_n]$, quantile of interest $\tau \in (0, 1)$, approximation parameter $\Delta \le \min(\tau, 1-\tau)$ and probability parameter $\delta$, $\mathtt{QuantEst}(F, \mathbf{x}, \tau, \Delta, \delta)$ will use at most
$O \big( \frac{1}{\Delta^2}  \log  \frac{n}{\delta} \big)$
threshold queries and output an index $i$ satisfying
    $\mathbb{P}
    \left([F(x_i), F(x_{i+1})] \cap  (\tau - \Delta, \tau + \Delta) = \varnothing \right) < \delta$. 
% For now, we only need to treat this as a ``black box" with guarantees on
 Formally, the guarantees on its outputs $l_{t, k}$ and $u_{t, k}$ (see Lines~\ref{ltk def} and~\ref{utk def}) as well as the number of arm pulls used are stated as follows.
\begin{algorithm}[!t]
    \caption{Main Algorithm}
      \hspace*{\algorithmicindent} \textbf{Input}:
      Arms $\A = \{1, \dots, K\}$,
        and 
        $\lambda, \epsilon, q, \delta$,
        where
        $\lambda > \epsilon$ 
        and $q, \delta \in (0,1)$ 
        
        \hspace*{\algorithmicindent} \textbf{Parameter}: $c \in \mathbb{Z}^+$
    \label{alg: main}
    \begin{algorithmic}[1]
        \State Set 
        $n \coloneqq 
        \left\lceil (c+1) \lambda/\epsilon \right\rceil
        $
        \label{line: number of points}
        
        \State Set 
        $\tilde{\epsilon} \coloneqq    
        \lambda / n
        $  \footnote{The distance between $x_i$ and $x_{i+1}$ for $1 \le i \le n$ is exactly $\tilde{\epsilon}$, which is approximately $\epsilon/(c+1)$ (up to the impact of rounding in Line 1).
    We choose the spacings to be equal for ease of analysis.}
        \label{line: tilde epsilon}
        
        

        \State Initiate a list $\mathbf{x} 
        = [x_0, x_1, \ldots, 
        x_n, x_{n+1}, x_{n+2} ]
        = \left[
        -\infty, 0, 
         \tilde{\epsilon}, 2 \tilde{\epsilon}, 
       \ldots,
        (n-1)  \tilde{\epsilon}, \lambda,
        \infty
        \right]$
        \footnote{
        We add $\pm \infty$ to the ends of the list $\mathbf{x}$ to handle the edge cases
    $F_k(0) = q$ and $F_k(\lambda) = q$. Without this, Lemma~\ref{lem: good events} may not be satisfied:
    $[F_k(0), F_k( \tilde{\epsilon })] \cap  
        \big( q - \Delta^{(t)}, q   \big) = \varnothing$
    and 
    $[F_k(\lambda - \tilde{\epsilon}), F_k(\lambda)] \cap  
        \big( q, q+\Delta^{(t)}  \big) = \varnothing$.}
        \label{line: list of points}

        \State Initiate round index $t = 1$

        \State Initiate the set of active arms $\mathcal{A}_t = \A = \{1, \dots, K \}$

        \For{arm $k \in \mathcal{A}_t$}
            \State  $\mathrm{LCB}_0(k) = x_1 = 0; \
            \mathrm{UCB}_0(k) = x_{n+1} = \lambda$
            \label{eq: initiate default conf interval}
        \EndFor
                
        \While {$\mathrm{LCB}_{t-1}(k) < 
                \max\limits_{a \in \mathcal{A}_{t} \setminus \{k\} }  
                \mathrm{UCB}_{t-1}(a) -  (c+1)\tilde{\epsilon}$ for all arm $k \in \A_t$}
                \footnote{We use the convention that the maximum of an empty set is $- \infty$, so the while-loop termination condition is trivially satisfied when $|\A_t|= 1$.}

        \label{line: start while loop}
        

           
            
            \State $\Delta^{(t)} \leftarrow 2^{-t+1} \cdot \min(q, 1-q)$
            \label{def: Delta_t}
            
            \For{arm $k \in \mathcal{A}_t$}

                \State  
                \label{ltk def}
                \parbox[t]{\dimexpr\linewidth-3em}{%
          Run $\mathtt{QuantEst}$ (Algorithm~\ref{alg: quantile interval}) with input $\big(F_k, \mathbf{x}, q - \frac{\Delta^{(t)}}{2}, 
                \frac{\Delta^{(t)}}{2}, 
                \frac{\delta \cdot \Delta^{(t)}}{2 |\mathcal{A}_t|}\big)$
                to obtain an index~$l_{t, k} 
                 \in \{0, \dots, n+1\}$
        }
        
                
                % $
                % l_{t, k} =
                % \mathtt{QuantEst}
                % \left(F_k, \mathbf{x}, q - \dfrac{\Delta^{(t)}}{2}, 
                % \dfrac{\Delta^{(t)}}{2}, 
                % \dfrac{\delta \cdot \Delta^{(t)}}{2 |\mathcal{A}_t|}\right)
                %  \in \{0, \dots, n+1\}
                % $
    
                \State
                \label{LCB definition}
                $\mathrm{LCB}_t(k) =
                \max
                \left(
                x_{l_{t, k}},
                \mathrm{LCB}_{t-1}(k)
                \right) 
                $
                
                \State
                \label{utk def}
                \parbox[t]{\dimexpr\linewidth-3em}{%
          Run $\mathtt{QuantEst}$ (Algorithm~\ref{alg: quantile interval}) with input $\big(F_k, \mathbf{x}, q + \frac{\Delta^{(t)}}{2}, 
                \frac{\Delta^{(t)}}{2}, 
                \frac{\delta \cdot \Delta^{(t)}}{2 |\mathcal{A}_t|}\big)$
                to obtain an index~$u_{t, k}  \in \{0, \dots, n+1\}$
        }
                
                % \State $
                % u_{t, k} = 
                % \mathtt{QuantEst}\left(F_k, \mathbf{x}, q + \dfrac{\Delta^{(t)}}{2}, 
                % \dfrac{\Delta^{(t)}}{2}, 
                % \dfrac{\delta \cdot \Delta^{(t)}}{2 |\mathcal{A}_t|}\right)
                % \in \{0, \dots, n+1\}
                % $
    
                \State
                $\mathrm{UCB}_t(k) =
                \min
                \left(
                x_{u_{t, k} + 1},
                \mathrm{UCB}_{t-1}(k)
                \right)$
                \label{UCB definition}
            \EndFor

            
            
            \State Update $\mathcal{A}_{t+1} =
            \big\{
            k \in \mathcal{A}_{t}:
            \mathrm{UCB}_t(k) >
            \max\limits_{a \in \mathcal{A}_{t}} 
            \mathrm{LCB}_t(a)
            \big\} 
            $
            \label{line: active arm}
            
             \State  Increment round index $t \leftarrow t+1$
             
            \label{line: end while loop}
        \EndWhile

        \State \Return any arm $\hat{k} \in \A_t$ 
        satisfying $\mathrm{LCB}_t(\hat{k})  \ge
                \max\limits_{a \in \A_t \setminus \{ \hat{k} \} }  
                \mathrm{UCB}_t(a) -  (c+1)\tilde{\epsilon}$ 
        \label{line: return}

    \end{algorithmic}
\end{algorithm}

 
\begin{lemma}[Good event]
\label{lem: good events}
    Fix an instance $\nu \in \cE$, and suppose Algorithm~\ref{alg: main} is run with input $(\A, \lambda, \epsilon, q, \delta)$ and parameter $c \ge 1$.
     Let $\Delta^{(t)}$, $\mathcal{A}_t$, $l_{t, k}$, $u_{t, k}$ be as defined in Algorithm~\ref{alg: main} for each round index $t \ge 1$ and each arm $k \in \A_t$.
     Define events $E_{t, k, l}$ and events $E_{u, k, l}$ respectively by
    \begin{equation}
    \label{eq: event Etkl}
        E_{t, k, l} \coloneqq
        \left\{
        [F_k(x_{l_{t, k}}), F_k(x_{l_{t, k}+1})] \cap  
        \big( q - \Delta^{(t)}, q   \big) 
        \text{ is non-empty}
        \right\}
    \end{equation}
    and
     \begin{equation}
      \label{eq: event Etku}
        E_{t, k, u} \coloneqq
        \left\{
        [F_k(x_{u_{t, k}}), F_k(x_{u_{t, k}+1})] \cap  
        \big( q, q + \Delta^{(t)}  \big) 
        \text{ is non-empty}
        \right\}.
    \end{equation}
    Then the Event~$E$ defined by
    \begin{equation}
        E \coloneqq 
        \bigcap_{t \ge 1}
        \bigcap_{k \in \mathcal{A}_t} 
        \left(
        E_{t, k, l}
        \cap
        E_{t, k, u}
        \right)
    \end{equation}    
    occurs with probability at least $1 - \delta$.
    Furthermore, for each $t$ and $k \in \A_t$, the number of arm pulls used by $\mathtt{QuantEst}$ to output $l_{t,k}$ and $u_{t, k}$ scales as 
    \begin{equation}
    \label{eq: QuantEst arm pulls}
        O\left(
    \frac{1}{(\Delta^{(t)})^2} 
    \log 
    \left(
    \frac{2n  |\A_t| }{ \delta \Delta^{(t)}}
    \right)
    \right)
    =
    O\left(
    \frac{1}{(\Delta^{(t)})^2} 
    \cdot
    \left( 
     \log \left(\frac{1}{ \delta } \right) +
     \log \left(\frac{1}{ \Delta^{(t)}}\right) +
     \log \left(\frac{c \lambda K}{ \epsilon } \right)
    \right)
    \right),
    \end{equation}
    where $n =\left\lceil (c+1) \lambda/\epsilon \right\rceil$ and $\Delta^{(t)}= 2^{-t+1} \cdot \min(q, 1-q)$ as stated in Lines~\ref{line: number of points} and~\ref{def: Delta_t} of Algorithm~\ref{alg: main}.
\end{lemma}
\begin{proof}
    See Appendix~\ref{sec: appendix QuantEst} for the details, in which we make use of a noisy binary search subroutine from \cite{gretta2023sharp}.
\end{proof}
\begin{remark}
    \label{rem: input c}
    We note that the parameter $c \ge 1$ in the algorithm controls how finely the continuous interval $[0,\lambda]$ is discretized; one can think of $c=1$ for simplicity to have roughly $n = 2\lambda/\epsilon$ discretization points spaced by roughly $\epsilon/2$, but we will see in Section~\ref{sec: solvable} that picking a larger value of~$c$ can be beneficial.
\end{remark}




\subsection{Anytime Quantile Bounds}
Under Event $E$  as defined in Lemma~\ref{lem: good events}, we obtain the following anytime bounds for the quantiles when running Algorithm~\ref{alg: main}. 
These bounds will be used in the proofs of the correctness of Algorithm~\ref{alg: main} (Theorem~\ref{thm: correctness}) and the upper bound on the number of arm pulls (Theorem~\ref{theorem: upper bound}).

\begin{lemma}[Anytime quantile bounds]
\label{lem: quantile anytime bound}
    Fix an instance $\nu \in \cE$, and suppose Algorithm~\ref{alg: main} is run with input $(\A, \lambda, \epsilon, q, \delta)$ and parameter $c \ge 1$.
    Let~$\tilde{\epsilon} = \tilde{\epsilon}(\lambda, \epsilon, c)$,
    and $\Delta^{(t)}$, $\A_t$, $\mathrm{LCB}_t(k)$, and $\mathrm{UCB}_t(k)$ be as defined in Algorithm~\ref{alg: main} for each round index $t \ge 1$ and each arm $k \in \A_t$.    
    Under Event~$E$ as defined in Lemma~\ref{lem: good events}, we have the following bounds for the 
    arms' lower quantile functions $Q_k(\cdot )$ and upper quantile functions
     $Q^+_k(p) \coloneqq \sup \{ x \mid F_k(x) \le p \} $:
    \begin{equation}
    \label{eq: quantile anytime bound}
           \mathrm{LCB}_{\tau}(k)
        \le \mathrm{LCB}_t(k)
        < Q_k(q) \le  Q^+_k(q)
        \le \mathrm{UCB}_t(k)
        \le \mathrm{UCB}_{\tau}(k)
    \end{equation}
    \begin{equation}
    \label{eq: lower approx quantile anytime bound}
       Q^+_k\big(q -  \Delta^{(t)} \big)
        \le \mathrm{LCB}_t(k) + \tilde{\epsilon}
    \end{equation}
    \begin{equation}
    \label{eq: upper approx quantile anytime bound}
        \mathrm{UCB}_t(k) 
        <
         Q_k\big(q + \Delta^{(t)} \big) + \tilde{\epsilon}
    \end{equation}
    for all rounds $t  > \tau \ge 0$ and each arm $k \in \A_t$.
\end{lemma}
\begin{proof}
    This follows from applying properties of quantile functions to events $E_{t, k, l}$ and $E_{t, k, u}$ defined in~\eqref{eq: event Etkl} and \eqref{eq: event Etku}; see Appendix~\ref{sec: appendix anytime quantile bounds} for the details.
\end{proof}

\begin{remark}
\label{rem: LCB non decreasing}
    % This property itself is a consequence of 
    % ensuring $\mathrm{LCB}_t(k)$ is non-decreasing in $t$ (see Line~\ref{LCB definition} of Algorithm~\ref{alg: main}).
    The property that $ \mathrm{LCB}_t(k)$ is non-decreasing 
    in $t$, i.e., the first inequality of \eqref{eq: quantile anytime bound}, may appear to be enforced ``artificially'' by Line~\ref{LCB definition} of Algorithm~\ref{alg: main}. 
    % (see Line~\ref{LCB definition} and~\ref{UCB definition}  of Algorithm~\ref{alg: main}).
    It will turn out that this property is crucial in proving Lemma~\ref{lem: max LCB increasing}, which 
    in turn is important for the analysis in upper bounding the number of arm pulls -- see Remark~\ref{rem: elim suboptimal}.
\end{remark}

\subsection{Correctness}
In this section, we give the performance guarantee of Algorithm~\ref{alg: main} using the anytime quantile bounds (Lemma~\ref{lem: quantile anytime bound}).
We first formalize the notion of an algorithm returning an \textit{incorrect} output with at most a small error probability $\delta$.
% , except with some small error probability $\delta$.  

\begin{definition}[$(\epsilon, \delta)$-reliable.]
     Consider an algorithm $\pi$ for the QMAB problem with quantized or  unquantized rewards that takes $(\Ac, \lambda, \epsilon, q, \delta)$ as input and operates on instances $\nu \in \cE$. 
     Then, we say~$\pi$ is $(\epsilon, \delta)$-reliable if for each instance $\nu \in \cE$, it returns an \emph{incorrect} output with probability at most $\delta$, i.e.,     
    \begin{equation}
    \label{def: reliable}
        \text{for each } \nu \in \cE, \quad 
        \PP_{\nu}[ \tau < \infty \cap \hat{k} \notin \Ac_{\epsilon}(\nu) ] \le \delta,
        % \textcolor{red}{\text{Do we need to define } \PP_{\nu}?}
    \end{equation}
    where $\tau = \tau(\nu) \le \infty$ is the random stopping time of $\pi$ on instance $\nu$,
     % (possibly with $\tau = \infty$), 
    arm $\hat{k}$ is the output upon termination, and~$\Ac_{\epsilon}(\nu)$ is as defined in~\eqref{def: performance def}.
\end{definition}

\begin{remark}
    \label{rem: PAC}
    This definition is related to the notion of being $(\epsilon, \delta)$-PAC (see~\cite{EvenDar2002PACBF}). It can be seen as a relaxation of $(\epsilon, \delta)$-PAC since an $(\epsilon, \delta)$-reliable algorithm is allowed to be non-terminating on some instances -- a high probability of correctness is needed only on instances it terminates on.
    As we will see in Section~\ref{sec: solvable}, this relaxation is required when considering every possible $\nu \in \cE$,
    as there are instances that are not ``solvable'' for any finite number of arm pulls.
\end{remark}


\begin{theorem}[Reliability of Algorithm~\ref{alg: main}]
\label{thm: correctness}
    Fix an instance $\nu \in \cE$, and suppose Algorithm~\ref{alg: main} is run with input $(\A, \lambda, \epsilon, q, \delta)$ and parameter $c \ge 1$.
    Under Event $E$ as defined in Lemma~\ref{lem: good events}, if Algorithm~\ref{alg: main} terminates, then
    it returns an arm~$\hat{k}$ satisfying~\eqref{def: performance def}.
\end{theorem}
Since Event $E$ occurs with probability at least $1-\delta$ (Lemma~\ref{lem: good events}), we conclude the following.
\begin{corollary}
\label{cor: main alg reliable}
    Algorithm~\ref{alg: main} is $(\epsilon, \delta)$-reliable.
\end{corollary}

The proof details of Theorem~\ref{thm: correctness} are given in Appendix~\ref{sec: appendix correctness}, and we provide a sketch here.
    Combining the guarantee from Line~\ref{line: return}, inequalities~\eqref{eq: quantile anytime bound}, and the choice of $\tilde{\epsilon} \le \lambda \cdot \epsilon/((c+1) \lambda) = \epsilon/(c+1)$ yields
    \begin{equation}
    Q_{\hat{k}}(q) >
    \mathrm{LCB}_t(\hat{k})  \ge
    \max\limits_{a \in \A_t \setminus \{ \hat{k} \} }  
    \mathrm{UCB}_t(a) -  (c+1)\tilde{\epsilon}    
    \ge 
    \max\limits_{a \in \A_t \setminus \{ \hat{k} \} }
    Q_a(q) -  \epsilon.  
    \end{equation}
    It remains to show that the optimal arm $k^*$ lies in $\A_t$ for all $t$, which we defer to Appendix~\ref{sec: appendix correctness}.


\subsection{Upper Bound}
\label{sec: upper bound}
In this section, we bound the number of arm pulls for a given instance $\nu \in \cE$. To characterize the number of arm pulls, we define the gap of each arm as follows.
\begin{definition}[Arm gaps]
\label{def: our gap}
     Fix an instance $\nu \in \cE$.
     Let $\tilde{\epsilon} = \tilde{\epsilon}(\lambda, \epsilon, c)$ and $\A_{\epsilon} = \A_{\epsilon}(\nu)$ be as defined in Algorithm~\ref{alg: main} and~\eqref{def: performance def} respectively.
    For each arm $k \in \A$, we define the gap  $\Delta_{k} =
    \Delta_{k}(\nu, \lambda, \epsilon, c, q)$ as follows:
\begin{equation}
    \label{eq: our gap}
    \Delta_{k}
    \coloneqq
    \begin{cases}
    \sup
    \Big\{
        \Delta \in \left[0, \min(q, 1-q) \right]
        :
        Q_k(q + \Delta) 
        \le
        \max\limits_{a \in \A  }
        Q^+_{a}(q - \Delta) - \tilde{\epsilon}
        \Big\}
    & \text{if }  k \not\in \A_{\epsilon} \\
   \max\limits_{\A_{\epsilon} \subseteq S \subseteq \A}
        \Delta_{k}^{(S)}
    & \text{if } k \in \A_{\epsilon} 
    \end{cases},
\end{equation}
where $Q^+_k(p)$ is the upper quantile function defined in Lemma~\ref{lem: quantile anytime bound},  
and
\begin{equation}
\label{eq: Delta k^S}
    \Delta_{k}^{(S)} \coloneqq
   \sup
    \Big\{
        \Delta \in 
       \Big[0, \min_{a \not\in S} \Delta_{a}  \Big]
        % \ \middle\vert\
        :
        Q^+_k(q - \Delta) 
        \ge 
        \max\limits_{ a \in S \setminus \{k\}} 
        Q_{a}(q + \Delta) - c \tilde{\epsilon}
        \Big\}
\end{equation}
for each subset $S$ satisfying $\A_{\epsilon} \subseteq S \subseteq \A$. We use the convention that the minimum (resp. maximum) of an empty set is $\infty$ (resp. $- \infty$).
\end{definition}

\begin{remark}[Intuition on arm gaps]
    We provide some intuition for the gap definitions:
\begin{itemize}[topsep=0pt,itemsep=0pt]
    \item 
    For an arm $k \not\in \A_{\epsilon}$, the gap
    $\Delta_{k}$ captures how much worse $k$ is than some other arm $a$. 
    When $k$ is sufficiently pulled relative to $1/\Delta_{k}$, 
    we can establish that
    $\mathrm{UCB}_t(k) \le \mathrm{LCB}_t(a)$, 
    which implies that $k$ is suboptimal, and we can stop pulling it. The details and derivation are given in Lemma~\ref{lem: elim suboptimal} and its proof.
    
    \item 

     To understand the gap $\Delta_{k} = \max\limits_{\A_{\epsilon} \subseteq S \subseteq \A}
        \Delta_{k}^{(S)}$ for a satisfying arm $k \in \A_{\epsilon}$,
    we first consider $\Delta_{k}^{(S)}$ for a fixed subset $S \supseteq \A_{\epsilon}$.
     This captures how much better arm $k$ is than the ``best'' arm $a \in S$ (up to $\epsilon$). 
     When arm $k$ is sufficiently pulled relative to $1/\Delta_{k}^{(S)}$, we can establish that 
    the termination condition is satisfied.
    Since $S \supseteq \A_{\epsilon}$ is arbitrary, we define $\Delta_{k}$ based on the set $S$ giving the highest $\Delta_{k}^{(S)}$.
     The details and derivation are given in Lemma~\ref{lem: termination} and its proof.
    
    \item When some arm $k^*$ is the only satisfying arm (i.e., $\A_{\epsilon} = \{k^*\}$), we have 
    \begin{equation}
    \label{eq: lower bound k* arm gap}
    \Delta_{k^*}  
     \ge  \Delta_{k^*}^{(\A_{\epsilon})}
     = \sup  \Big\{   \Delta \in 
       \Big[0, \min_{a \not\in \A_{\epsilon}} \Delta_a \Big] :
        Q^+_k(q - \Delta)  \ge  -\infty
        \Big\} 
    = \min\limits_{a \not\in \A_{\epsilon}} \Delta_a
    = \min\limits_{a \neq k^*} \Delta_a.
    \end{equation}
    This indicates that $k^*$ is pulled at most as many times as the smallest $\Delta_a$ value would dictate, and possibly fewer (if the while-loop terminates before $|\A_t| = 1$).    
\end{itemize}
\end{remark}


    
\begin{remark}[Generalization and improvement over existing arm gap]
    \label{rem: gap generalization}
    Our gap definitions were developed with the view of ensuring that we can solve essentially all solvable instances, and we will establish results of this type in Section~\ref{sec: solvable}.  Achieving this goal required several subtle choices in our gap definition, including the parameter $c$ and the optimization over $S$.
    We generalize existing gaps for the QMAB problem in the sense that those are recovered by considering $c \to \infty$, $S = \A$, and using only lower quantile functions. In Appendix~\ref{sec: appendix gap definition generalization}, we provide more details about these choices and give an instance where the gap is positive under our definition but is zero using existing definitions.
\end{remark}

\begin{remark}[Further improvement]
    \label{rem: further improvement}
     Due to the assumption that the $q$-quantile of each arm is in $[0, \lambda]$, we can improve our gap definition by replacing the terms $Q^+_{(\cdot)}(q - \Delta)$ and  $Q_{(\cdot)}(q + \Delta)$
    with $\max\big\{0, Q^+_{(\cdot)}(q - \Delta)\big\}$
    and $\min\big\{\lambda, Q_{(\cdot)}(q + \Delta)\big\}$ respectively.  
    % The idea is that in cases where taking $\max$ (resp. $\min$) makes a difference, then 0 (resp. $\lambda$) would be a better lower (resp. upper) bound for the $q$-quantile. 
    % This modified gap can be used to remove the additional assumption made in Theorem~\ref{thm: zero gap is unsolvable} - see Remark~\ref{rem: remove additional assumption}.
    We adopt Definition~\ref{def: our gap} to avoid further complicating the gap definition and subsequent analysis, but we will provide detailed discussion of this modified gap in Appendix~\ref{sec: appendix potential improvement}.
    % and provide an instance that is positive under the modified gap but is zero under the unmodified gap.
\end{remark}




Having defined the arm gaps, we now state an upper bound on the total number of arm pulls by Algorithm~\ref{alg: main}.

\begin{theorem}[Upper bound]
\label{theorem: upper bound}
   Fix an instance $\nu \in \cE$, and suppose Algorithm~\ref{alg: main} is run with input $(\A, \lambda, \epsilon, q, \delta)$ and parameter $c \ge 1$.
    Let $\A_{\epsilon}(\nu) $ be as defined in~\eqref{def: performance def} and let the gap $\Delta_{k} = \Delta_{k}(\nu, \lambda, \epsilon, c, q)$ be as defined in Definition~\ref{def: our gap} for each arm $k \in \A$.
    Under Event~$E$ as defined in Lemma~\ref{lem: good events},
    the total number of arm pulls is upper bounded~by    
    \begin{equation}
    \label{eq: upper bound}
        O
        \left(
        \left(
        \sum_{ k \in \A }
        \dfrac{1}{ \max\big( \Delta_{k},  \Delta  \big)^2} \cdot 
        \left( 
         \log \left(\frac{1}{ \delta } \right) +
         \log \left(\frac{1}{ \max\big( \Delta_{k},  \Delta  \big)}\right) +
         \log \left(\frac{c \lambda K}{ \epsilon } \right)    
        \right)
        \right)
        \right),
    \end{equation}
    where $\Delta  =  \Delta(\nu, \lambda, \epsilon, c, q) = \max\limits_{a \in \A_{\epsilon}(\nu)} \Delta_{a}$.
\end{theorem}


Combining Lemma~\ref{lem: good events}, Theorem~\ref{thm: correctness}, and Theorem~\ref{theorem: upper bound}, we obtain
the high-probability correctness for instances with positive gap.
\begin{corollary}
\label{cor: combined guarantee}
    Fix an instance $\nu \in \cE$ and suppose
    $\Delta =  \Delta(\nu, \lambda, \epsilon, c, q) $ as defined in Theorem~\ref{theorem: upper bound} is positive. Then, with probability at least $1-\delta$, Algorithm~\ref{alg: main} returns an arm~$\hat{k}$ satisfying~\eqref{def: performance def} and uses a total number of arm pulls satisfying~\eqref{eq: upper bound}.
\end{corollary}

We will provide near-matching lower bounds in the next section, and an impossibility result for the instances with zero gap in Section~\ref{sec: solvable}. The proof details of Theorem~\ref{theorem: upper bound} are given in Appendix~\ref{sec: appendix upper bound}, and we provide a sketch here.

\begin{proof}[Proof outline for Theorem~\ref{theorem: upper bound}]
    Under Event $E$, the while-loop of Algorithm~\ref{alg: main} terminates
    when the round index~$t$ is large enough to satisfy
    $\Delta^{(t)} \le \frac{1}{2} \Delta $,
    which happens when $t = \log_2 (1/\Delta ) + \Theta(1)$ since $\Delta^{(t)} = 2^{-t+1} \dot \min(q,1-q)$.
    Summing through the number of arm pulls $\widetilde{O}\big( \left(\Delta^{(t)}\right)^{-2}\big)$ given in~\eqref{eq: QuantEst arm pulls} for $t = 1, \ldots, \log_2 (1/\Delta ) + \Theta(1) $
    yields the upper bound $\widetilde{O}\left( \Delta ^{-2}\right)$ for each arm $k \in \A$.
    However, it is also possible that some arms are eliminated before the while-loop terminates.
    Specifically, each non-satisfying arm $k \not\in \A_{\epsilon}(\nu)$ is eliminated when the index~$t$ satisfies $\Delta^{(t)} \le \frac{1}{2} \Delta_{k}$,  which yields the upper bound $\widetilde{O}\left(\Delta_{k}^{-2}\right)$. Taking the minimum between these two gives~\eqref{eq: upper bound}.
\end{proof}

