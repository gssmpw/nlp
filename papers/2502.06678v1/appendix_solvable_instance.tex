\section{Proof of Theorem \ref{thm: zero gap is unsolvable} and Corollary~\ref{cor: zero gap is unsolvable} (Solvable Instances)}
% \section{Solvable Instances}
\label{sec: appendix solvable instance}

We first state a useful lemma for Theorem \ref{thm: zero gap is unsolvable}.

\begin{lemma} 
\label{lem:two_instances}
        Let $\lambda, \epsilon, c,$ and $q$ be given, 
        and let $\tilde{\epsilon} = \tilde{\epsilon}(\lambda, \epsilon, c)$ be as defined in 
        % Lines~\ref{line: number of points} and~\ref{line: tilde epsilon} of 
        Algorithm~\ref{alg: main}.
        Suppose that $\nu \in \cE$ is an instance with gap $\Delta(\nu, \lambda, \epsilon, c, q) = 0 $ and let $\eta_0 = \eta_0(\nu) > 0$ be the constant given in the assumption in Theorem \ref{thm: zero gap is unsolvable}.  
        Then, for each arm $k \in \A_{c \tilde{\epsilon}}(\nu)$ and each $\eta \in  (0, \eta_0) $, there exists another instance $\nu' \in \cE$ satisfying the following:
    \begin{itemize}[topsep=0pt, itemsep=0pt]
        \item There exists an arm $a \in \Ac \setminus \{k\}$ such that instances $\nu$ and $\nu'$ are identical for all arms in $\Ac \setminus \{a,k\}$;
        
        \item $\dTV(F_a,G_a) \le \eta$ and $\dTV(F_k,G_k) \le \eta$, where $F_{(\cdot)}$ and $G_{(\cdot)}$ represent the arm distributions for instances $\nu$ and $\nu'$ respectively;
        
        \item $k \notin \Ac_{c \tilde{\epsilon} }(\nu')$, i.e., 
        under relaxation parameter $c \tilde{\epsilon}$, arm $k$ is not a satisfying arm for instance $\nu'$.
    \end{itemize}
\end{lemma}

     
 \begin{proof}   
    Let $\nu \in \cE$ be an instance with gap $\Delta(\nu, \lambda, \epsilon, c, q) = 0$. For each arm $k \in \A_{\epsilon}(\nu)$, we have $\Delta_{k}^{(\A)} = 0$ by Definition~\ref{def: our gap} since 
    $0 \le \Delta_{k}^{(\A)}  \le \Delta_{k}  \le \Delta  = 0$.
    Applying \eqref{eq: Delta k^S} with set $S = \A$ yields:
     \begin{equation}
     \label{eq: arm a positive eta}
        \text{for each } k \in \A_{\epsilon}(\nu)
        \text{ and each } \eta > 0, 
        \text{ there exists } a \ne k
        \text{ such that }
         Q^+_{k}(q - \eta) 
        <
        % \max\limits_{ a \ne k} 
        Q_{a}(q + \eta) - c\tilde{\epsilon}.
     \end{equation}
    Fix an arm $k \in \A_{c \tilde{\epsilon}}(\nu)$ and $\eta \in (0, \eta_0)$.
    Since $c \tilde{\epsilon} \le \epsilon$ (see calculation in~\eqref{eq: tilde eps 1 and 2}--\eqref{eq: c1 tilde eps 1 and 2}), we have 
    $\A_{c \tilde{\epsilon}}(\nu) \subseteq \A_{\epsilon}(\nu)$, and hence $k \in \A_{\epsilon}(\nu)$. 
    It follows from~\eqref{eq: arm a positive eta} that there exists some arm $a \ne k$ that
    \begin{equation}
     \label{eq: arm a positive eta c tilde epsilon}
         Q^+_{k}(q - \eta) 
        <
        % \max\limits_{ a \ne k} 
         Q_{a}(q + \eta) - c\tilde{\epsilon}.
     \end{equation}
    We now construct instance $\nu'$ such that $\nu$ and $\nu'$
    have identical distributions for all arms in $\A \setminus \{a, k\}$, 
    while $F_a$ and $F_k$ are being replaced with $G_a$ and $G_k$ defined as follows:
    \begin{enumerate}[topsep=0pt, itemsep=0pt]
        \item 
        $G_a$ is any distribution obtained by moving $\eta$-probability mass from the interval $(-\infty, Q_a(q))$ to the point $Q_a(q+2\eta)$; 
        
        \item 
        $G_k$ is any distribution obtained by moving $\eta$-probability mass from the interval $(Q_k(q), \infty)$ to the point $ Q_k(q-2\eta)$.
    \end{enumerate}     
     Under these definitions and the assumption on $\eta_0$ in Theorem~\ref{thm: zero gap is unsolvable}, we can readily verify that
     \begin{equation}
     \label{eq: shifted q quantiles}
         (G_k)^{-1}(q) =  Q_k(q-\eta) 
         % \le Q^+_{k}(q - \eta) 
         \in [0, \lambda]
         \quad 
         \text{and}
         \quad  
         (G_a)^{-1}(q) = Q_a(q+\eta) \in   [0, \lambda]
     \end{equation}
     and
     \begin{equation}
         d_{\mathrm{TV}}(F_k, G_k) =  d_{\mathrm{TV}}(F_a, G_a) = \eta.
     \end{equation}
    
     Finally, combining~\eqref{eq: arm a positive eta c tilde epsilon} and~\eqref{eq: shifted q quantiles} yields
     \begin{equation}
     \label{eq: G_k unsatisfying}
          (G_k)^{-1}(q) 
         < 
         (G_a)^{-1}(q) - c\tilde{\epsilon},
     \end{equation}
     which implies $k \notin \Ac_{c \tilde{\epsilon} }(\nu')$. 
     By construction, $\nu'$ satisfies all three properties as desired.
\end{proof}


\begin{remark}
\label{rem: limit version of two instance lemma}
     We can obtain a ``limiting'' version of Lemma~\ref{lem:two_instances} in which we replace the gap $\Delta(\nu, \lambda, \epsilon, c, q)$ by $\Delta(\nu, \epsilon, q)$ as defined in Corollary~\ref{cor: zero gap is unsolvable} and the satisfying arm set $\A_{c \tilde{\epsilon}}(\cdot)$
     by $\A_{\epsilon}(\cdot)$.
     The proof is essentially identical.
     We construct instance $\nu'$ in a similar manner as above to satisfy the first two properties in the statement of Lemma~\ref{lem:two_instances}.
     The last property $(k \not\in \A_{\epsilon}(\nu'))$ then follows from the definition of the limit gap $\Delta_{k}(\nu, \epsilon, 
    q)$ as defined in~\eqref{eq: gap k infinite c}, which allows us to replace the $c\tilde{\epsilon}$ terms in~\eqref{eq: arm a positive eta},~\eqref{eq: arm a positive eta c tilde epsilon}, and~\eqref{eq: G_k unsatisfying} by $\epsilon$.    
\end{remark}


We proceed to prove Theorem \ref{thm: zero gap is unsolvable}.  
\begin{proof}[Proof of Theorem~\ref{thm: zero gap is unsolvable}]
Assume for contradiction that there exists some instance $\nu \in \cE$ satisfies $\Delta(\nu, \lambda, \epsilon, c, q) = 0 $, but is $c\tilde{\epsilon}$-solvable. 
Fix a $\delta \in (0, 1)$ satisfying
\begin{equation}
    \label{eq: delta very small}
    \delta < \frac{1}{2+2|\A|}.
\end{equation}
By Definition \ref{def:solvable}, there exists a $(c\tilde{\epsilon}, \delta)$-reliable algorithm such that
\begin{equation}
    \PP_{\nu}[\tau < \infty \cap \hat{k} \in \Ac_{c\tilde{\epsilon}}(\nu)] \ge 1-\delta.
\end{equation}
In general the condition $\tau < \infty$ may not imply a \emph{uniform} upper bound on $\tau$; we handle this by relaxing the probability from $1-\delta$ to $1-2\delta$, such that there exists some $\tau_{\max} < \infty$ satisfying
\begin{equation}
    \PP_{\nu}[\hat{k} \in \Ac_{c\tilde{\epsilon}}(\nu) \cap \tau \le \tau_{\max}] \ge 1-2\delta. \label{eq:tau_max}
\end{equation}
From this, we claim that there exists an arm $k_{\nu} \in \Ac_{c\tilde{\epsilon}}(\nu)$ such that
\begin{equation}
    \PP_{\nu}[\hat{k} = k_{\nu} \cap \tau \le \tau_{\max}] \ge \frac{1-2\delta}{|\Ac|}. 
    \label{eq:success_nu}
\end{equation}
Indeed, if this were not the case, then summing these probabilities over elements in $\Ac_{\epsilon}(\nu)$ would produce a total below $1-2\delta$, which would contradict \eqref{eq:tau_max}.


Let $P_{\tau_{\max}}^{(\nu)}$ be the joint distribution on the $|\Ac| \times \tau_{\max}$ matrix of unquantized rewards:
the $(i,j)$-th entry of this matrix contains the $j$-th unquantized reward for arm $i$ under instance $\nu$. Under the event $\tau \le \tau_{\max}$, the algorithm's output does not depend on any rewards beyond those appearing in this matrix.  In other words, the output $\hat{k}$ is a (possibly randomized) function of this matrix.

By picking $\eta > 0$ to be sufficiently small in Lemma \ref{lem:two_instances}, we can find an instance $\nu' \in \cE$ such that $k_{\nu} \notin \Ac_{c\tilde{\epsilon}}(\nu')$ and
\begin{equation}
    \dTV\big( P_{\tau_{\max}}^{(\nu)}, P_{\tau_{\max}}^{(\nu')} \big) \le \delta.
\end{equation}
Here, $P_{\tau_{\max}}^{(\nu')}$ is defined similarly to $P_{\tau_{\max}}^{(\nu)}$, but for instance $\nu'$.
Since the output $\hat{k}$ is a (possibly randomized) function of the matrix defining $P_{\tau_{\max}}^{(\cdot)}$, we have
 \begin{equation}
    \label{eq: DPI}
     \dTV\big(  \PP_{\nu},  \PP_{\nu'} \big) \le \dTV\big( P_{\tau_{\max}}^{(\nu)}, P_{\tau_{\max}}^{(\nu')} \big) \le \delta
 \end{equation}
 by the data processing inequality for $f$-divergence~\cite[Theorem 7.4]{polyanskiy2024information}.
Using the definition $\dTV(P,Q) = \sup_{A} |P(A) - Q(A)|$, and applying~\eqref{eq: DPI},~\eqref{eq:success_nu},~\eqref{eq: delta very small}, we obtain
\begin{equation}
    \PP_{\nu'}[\hat{k} = k_{\nu} \cap \tau \le \tau_{\max}] \ge 
    \PP_{\nu}[\hat{k} = k_{\nu} \cap \tau \le \tau_{\max}] -
    \dTV\big(  \PP_{\nu},  \PP_{\nu'} \big)  
    % &\ge \PP_{\nu}[\hat{k} = k_{\nu} \cap \tau \le \tau_{\max}] -
    % \dTV\big( P_{\tau_{\max}}^{(\nu)}, P_{\tau_{\max}}^{(\nu')} \big)  \\
    \ge
    \frac{1-2\delta}{|\Ac|} - \delta > \delta.
    \label{eq:failure_nu'}
\end{equation}
Since $k_{\nu} \notin \Ac_{c \tilde{\epsilon} }(\nu')$, this means that the algorithm is \emph{not} $(c\tilde{\epsilon}, \delta)$-reliable (see Definition~\ref{def: reliable}), we have arrived at the desired contradiction.
\end{proof}

Corollary~\ref{cor: zero gap is unsolvable} can be proved similarly by using the ``limiting'' version of Lemma~\ref{lem:two_instances} (see Remark~\ref{rem: limit version of two instance lemma}).


