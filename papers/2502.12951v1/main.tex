% This is samplepaper.tex, a sample chapter demonstrating the
% LLNCS macro package for Springer Computer Science proceedings;
% Version 2.21 of 2022/01/12
%
\documentclass[runningheads]{llncs}
%
\usepackage[T1]{fontenc}
% T1 fonts will be used to generate the final print and online PDFs,
% so please use T1 fonts in your manuscript whenever possible.
% Other font encondings may result in incorrect characters.
%
\usepackage{graphicx}
% Used for displaying a sample figure. If possible, figure files should
% be included in EPS format.
%
% If you use the hyperref package, please uncomment the following two lines
% to display URLs in blue roman font according to Springer's eBook style:
%\usepackage{color}
%\renewcommand\UrlFont{\color{blue}\rmfamily}
%\urlstyle{rm}
%
\usepackage{amsmath}
\usepackage{amssymb}
\usepackage{array}
\usepackage{wrapfig}
\usepackage{float}
\usepackage{placeins}
\begin{document}
%
\title{Guaranteed Conditional Diffusion: 3D Block-based Models for Scientific Data Compression}
%
\titlerunning{Guaranteed Conditional Diffusion Models}
% If the paper title is too long for the running head, you can set
% an abbreviated paper title here
%
% \author{Anonymous Author(s)}
\author{Jaemoon Lee \and
Xiao Li \and
Liangji Zhu \and
Sanjay Ranka 
\and Anand Rangarajan}
\authorrunning{J. Lee et al.}

% First names are abbreviated in the running head.
% If there are more than two authors, 'et al.' is used.
%
% \institute{}
\institute{University of Florida, Gainesville FL 32611, USA}
% \email{\{j.lee1,xiao.li,zhu.liangji\}@ufl.edu, \{ranka,anand\}@cise.ufl.edu}}
%
\maketitle              % typeset the header of the contribution
%
\begin{abstract}
%Lossy scientific data compression has recently emerged as an important area due to the tremendous amounts of data generated by scientific applications in high-performance computing (HPC).
This paper proposes a new compression paradigm---Guaranteed Conditional Diffusion with Tensor Correction (GCDTC)---for lossy scientific data compression. The framework is based on recent conditional diffusion (CD) generative models, and it consists of a conditional diffusion model, tensor correction, and error guarantee. Our diffusion model is a mixture of 3D conditioning and 2D denoising U-Net. The approach leverages a 3D block-based compressing module to address spatiotemporal correlations in structured scientific data. Then, the reverse diffusion process for 2D spatial data is conditioned on the ``slices'' of content latent variables produced by the compressing module. After training, the denoising decoder reconstructs the data with zero noise and content latent variables, and thus it is entirely deterministic.
The reconstructed outputs of the CD model are further post-processed by our tensor correction and error guarantee steps to control and ensure a maximum error distortion, which is an inevitable requirement in lossy scientific data compression. %In the tensor correction step, the reconstructed data of the CD model is enhanced at a tensor level by an overcomplete feedforward neural network, which is a learned mapping from the reconstructed tensors to the original tensors. Then, the error guarantee step ensures a maximum reconstruction error. We apply principal component analysis (PCA) to the residual between the original data and reconstructed data, which yields a basis matrix. The basis is then used to project the residual of each instance whose distortion ``exceeds'' a specified error bound. The optimal subset of the projected coefficients is selected for bounded error distortion.
Our experiments involving two datasets generated by climate and chemical combustion simulations show that our framework outperforms standard convolutional autoencoders and yields competitive compression quality with an existing scientific data compression algorithm.


\keywords{Conditional Diffusion \and Tensor Correction  \and Error Guarantee \and Scientific Data Compression.}
\end{abstract}
%
%
%

\section{Introduction}

Lossy scientific data compression has emerged as a vitally important area in the past decade. The volume and velocity of scientific data heighten the urgency of the requirement of good data compression algorithms, specifically methods that can provide performance guarantees in terms of error bounds on the primary data (PD) of interest. The concomitant rise of machine learning has seen the flowering of different learning-based compression paradigms. The primary ones are super-resolution, transform-based, and more recently methods based on generative AI. We first examine these paradigms before turning to the relatively new approaches based on generative models---the paradigm adopted in the present work.

Lossy compression based on \textbf{super-resolution} \cite{Khani2021,Conde2022swin2sr} is based on the premise that the data of interest can be faithfully reconstructed from a small set of ``true'' samples. Machine learning methods based on this paradigm attempt data reconstruction (of the original tensor) from this sample set. \textbf{Transform-based} methods have traditionally been the most popular paradigm with discrete cosine transforms (DCT), wavelets, principal component analysis (PCA) and dictionary-based methods leading the way. More recently, autoencoders (AE) which transform the data into a compact and quantized latent space from which learned decoders reconstruct the original tensor have been the paradigm of choice among ML practitioners. However, these paradigms do not leverage recent advances in generative AI. In this newer approach---termed \textbf{conditional diffusion (CD)} \cite{Yang2023cd}---the original tensor is first gradually converted into zero mean, Gaussian noise. Then, a decoder is learned which gradually denoises the tensor through stages to finally produce a tensor approximately drawn from the probability distribution of the original images. A latent space embedding is used to guide the diffusion process. We propose to work within this paradigm but in the context of scientific data compression.

%\begin{figure}
    %\centering
    %\includegraphics[width=\textwidth]{Figures/Overview.pdf}
    %\caption{Overview of our conditional diffusion model for compression. The $\boldsymbol{x}_i$ denotes the $i^\mathrm{th}$ slice of a 3D block. We compress 3D blocks to capture spatiotemporal correlations in scientific datasets. The compressed (latent) codecs guide a 2D denoising diffusion process. Our diffusion model reconstructs each of the 2D slices in 3D blocks based on its corresponding latent data $\boldsymbol{z}_i$. This enables us to keep a relatively simple U-Net architecture while getting effective latent variables via 3D block compression.}\label{fig:overview}
%\end{figure}

\begin{wrapfigure}{r}{0.5\textwidth} 
\vspace{-0.4cm}
    \centering
    \includegraphics[width=\linewidth]{Figures/Overview.pdf}
    %\hspace{0.25cm}
  \caption{Overview of our conditional diffusion model for compression. We compress 3D blocks to capture spatiotemporal correlations in scientific datasets. The latent variables guide a 2D denoising diffusion process. Our denoising decoder reconstructs each of the 2D slices in 3D blocks based on its corresponding latent data $\boldsymbol{z}_i$. This enables us to keep a relatively simple U-Net architecture while getting effective conditioning via 3D block compression.}
    \label{fig:overview}
    \vspace{-0.4cm}
\end{wrapfigure}
 
Our approach to scientific data compression, situated within the conditional diffusion paradigm is now described. Figure~\ref{fig:overview} illustrates the overview of our proposed conditional diffusion models. Our model is a mixture of 3D block conditioning and 2D denoising diffusion. We first divide the original data into blocks of 3D tensors. 3D tensors are encoded into latent variables, which results in compressed codecs. We construct 3D latent embeddings using these codecs and they act as the conditioning information in CD. Unlike latent embeddings, we learn the denoising decoder in 2D space. Each 2D slice $\boldsymbol{x}_i$ of the 3D tensors is gradually converted into white noise in a stovepiped manner as described above. The denoising decoder estimates the noise of the 2D slice at the diffusion stage $t$. Using an embedding for the diffusion stage index, we learn the denoising decoder and the latent space embeddings in an end-to-end fashion. After training the machine, the original 3D tensor blocks are reconstructed with zero noise at the input (so that the decoder is entirely deterministic). The reconstructed primary data are examined to see if error bounds are violated and if so, we correct for the PD to be within pre-specified error bounds (using PCA or via a separate error bounding neural network). The main contributions are:
\begin{itemize}
    \item We propose a CD model for lossy scientific data compression. We divide the entire data into 3D tensors and map them into compressed codecs to capture spatiotemporal correlations in scientific datasets.
    \item The proposed CD model is a mixture of 3D conditioning and 2D diffusion. We prevent a complexity increase of our denoising decoder by avoiding a 3D diffusion process.
    \item To the best of our knowledge, this application of CD to scientific data compression with error guarantees---termed guaranteed conditional diffusion (GCD)---is a new contribution within a relatively new data compression paradigm.
\end{itemize}




\section{Related Works}

\subsection{Knowledge-Handling in LLMs}

Prior work has explored different strategies to balance parametric and external knowledge when generating responses. 
For instance, \citet{li-etal-2023-large} trains LLMs to generate context-based responses when external knowledge is informative and parametric-based responses when it is uninformative.
Building on this approach, our framework incorporates parametric knowledge and integrates abstention as a key behavior to enhance reliability.


Similarly, \citet{neeman-etal-2023-disentqa} introduces an approach that allows LLMs to provide both context-based and parametric-based responses simultaneously. 
While their approach predicts two separate answers, we expect LLMs to internally select a single response based on the presence of parametric knowledge and the informativeness of external knowledge.


\subsection{Knowledge Conflict}

There have been efforts to resolve knowledge conflicts to improve the model's ability to incorporate external information \citep{zhou-etal-2023-context, park2024toward, shi-etal-2024-trusting}.
Some studies indicate that LLMs tend to adhere to their parametric knowledge \citep{longpre-etal-2021-entity, jin-etal-2024-tug}. 


They have primarily viewed knowledge conflict through the lens of external knowledge, focusing on factors such as temporal shifts \citep{NEURIPS2023_9941624e, dhingra-etal-2022-time}, synthetically updated facts \citep{longpre-etal-2021-entity}, and contextual plausibility \citep{xie2023adaptive, tan-etal-2024-blinded}.
We extend perspective to the model's parametric knowledge by considering whether the model possesses the knowledge to answer a question, enabling a broader range of analyses beyond external knowledge alone. 



\subsection{Uninformative Context}

Retrieval-augmented language models (RALMs) are prone to reduced performance due to distraction from uninformative external information \citep{yoran2023making, shen-etal-2024-assessing}. 
To mitigate the performance drop, researchers have explored methods to encourage LLMs to rely on parametric knowledge when external information is uninformative at the inference-time \citep{yu-etal-2024-chain, park2024enhancing, baek-etal-2023-knowledge-augmented-language} or through training \citep{yoran2023making, asai2024selfrag, xia2024improving, luo-etal-2023-search}.


Another line of research handles abstention behavior when presented with uninformative contexts \citep{wen-etal-2024-characterizing}.
We extend this perspective to cases where LLMs lack correct or complete internal knowledge to address a query \citep{feng-etal-2024-dont, zhang-etal-2024-r, wen2024know}. 
Our work bridges these two abstention criteria by jointly considering both parametric and external knowledge in refusal decisions.



\input{Narrative/03Methodology}
\section{Experiments}
This section presents the experimental results of our compression framework---Guaranteed Conditional Diffusion with Tensor Correction (GCDTC). We utilize two scientific datasets generated by E3SM and S3D applications. All the experiments are conducted using an Nvidia K80 GPU in OLCF's Andes.

\subsection{Datasets, Metrics, and Baselines}

%\paragraph{\textbf{E3SM Dataset.}}
\subsubsection{E3SM Dataset.}
We use the dataset generated by the E3SM (Energy Exascale Earth System Model) \cite{e3sm} that simulates Earth's climate system. For each 30-day period, the E3SM simulates climate variables of $240\times 240$ resolution with float32 data points in 6 regions with 720 timesteps. We use the cropped sea-level pressure (PSL) climate variable for 3 months, which results in a dataset with dimensions of $6\times 2160\times 192 \times 192$. We construct 3D blocks across temporal and spatial spaces in each region.
\vspace{-0.2cm}
\subsubsection{S3D Dataset.}
The dataset is generated by Sandia’s compressible reacting direct numerical simulation (DNS) code, S3D \cite{s3d}. The S3D simulates chemically reacting flow, involving detailed chemical mechanisms across numerous species. We use three species' $640\times 640$ mass fraction with float64 data points, collected over 288 timesteps, and create 3D blocks for each species.

\vspace{-0.2cm}
\subsubsection{Metrics.}
We use the \textit{NRMSE} and \textit{compression ratio} for the compression quality evaluation. NRMSE is the normalized RMSE, where RMSE is divided by a data range. In the computation of compression ratios, we consider all the sizes of models and dictionaries for entropy coding as scientific compression techniques are applied to a specific scientific application. Otherwise, we might use an extremely large machine, even bigger than a dataset, which can produce ideal compression results using overfitting.

\vspace{-0.2cm}
\subsubsection{Baselines.}
We compare our method to SZ3 \cite{SZ3} and a standard convolutional autoencoder (AE). SZ is one of the most dominant error-bounded lossy compressors for lossy scientific data compression. It is a prediction-based method, where a data point is estimated by its neighbors. The prediction accuracy is affected by a specified point-wise error bound. For the autoencoder, we incorporate 3D convolutional layers. The architecture is almost the same as the encoder and embedder structures of the conditional diffusion model, described in the Appendix. The only difference is the number of the output channels in the last unit. The model is trained to minimize the MSE loss between the original and reconstructed data. We then apply the error guarantee process in Section~\ref{subsec:tc_eg} to the output of the autoencoder.

\subsection{Implementation Details}
%\paragraph{\textbf{Training.}}
\subsubsection{Training.}
We don't split the datasets into training and test sets as error-bounded lossy compressors including SZ are not learned models. We divide each dataset into a set of $16\times 64\times 64$ blocks. The model is trained using the Adam optimizer \cite{adam} with the learning rate of $1\times 10^{-3}$ and 100 epochs. The number of diffusion steps is 1000 and the linear noise scheduling method of the DDPM is incorporated. The maximum and minimum $\beta$ are set to $5\times 10^{-3}$ and $1\times 10^{-5}$ respectively. Our framework is implemented using Pytorch \cite{pytorch}. The architecture detail of our conditional diffusion model is illustrated in the Appendix. We use the tensor correction network in \cite{JL-S3D_arxiv}. The inputs are 60-dimensional and 48-dimensional tensors across temporal space in E3SM and S3D respectively. In the error guarantee process in Section~\ref{subsec:tc_eg}, we correct $4\times 4\times 4$ blocks for both E3SM and S3D. The quantization factor $b$ and bin size $a$ are set to 1000 and 16 respectively for both latent variables $\boldsymbol{z}$ and selected coefficients $\boldsymbol{c}_s$.

\subsection{Results and Discussion}
We compress and reconstruct the datasets, denoted as PD. We vary the error bounds, the maximum point-wise distortion (SZ) and region-wise distortion (Ours), to get various compression results as shown in Figure~\ref{fig:compress_result}. Our experiments reveal that our proposed GCDTC outperforms the standard convolutional autoencoder in both the E3SM and S3D datasets. In the conditional diffusion model, latent variables convey content information and the denoising decoder processes further details. Both of them contribute reconstructions, which results in better compression quality compared to the autoencoder. Compared to SZ, GCDTC achieves at least twice larger compression ratios above $1\times 10^{-4}$ NRMSE in E3SM, while yielding competitive compression results in S3D. In lossy scientific data compression, $1\times 10^{-3}$ NRMSE is usually used as a target or an acceptable compression quality level for post-analysis, and GCDTC shows decent amounts of data reduction at this NRMSE level. Figure~\ref{fig:compress_example} shows reconstruction examples at the compression ratio 100. We zoom into a small region to check the details. Our framework can capture the details of the original data.

\begin{figure}[H]  % 使用 H 强制固定位置
    \centering
    \includegraphics[width=0.4\textwidth]{Figures/E3SM_PSL_PD_NRMSE.pdf}
    \hfill
    \includegraphics[width=0.4\textwidth]{Figures/S3D_PD_NRMSE.pdf}
    \caption{Reconstruction quality vs. compression ratio evaluation on E3SM (left) and S3D (right) datasets. GCDTC and GCAE denote Guaranteed Conditional Diffusion with Tensor Correction and Guaranteed Convolutional AutoEncoder. Note that the NRMSE results (y-axis) are plotted on a log scale. The result shows that our GCDTC outperforms GCAE, while yielding competitive performance with SZ.}
    \label{fig:compress_result}
\end{figure}

\FloatBarrier  % 确保 Figure 3 先被放置,Figure 4 不会跑到前面去

\begin{figure}[H]  % 继续使用 H 让 Figure 4 紧随 Figure 3
    \centering
    \includegraphics[width=\textwidth]{Figures/Results_example.pdf}
    \caption{Visualization of reconstructions in E3SM and S3D at compression ratio 100.}
    \label{fig:compress_example}
\end{figure}

\FloatBarrier  % 确保 Table 不会被浮动到奇怪的位置

We also evaluate the complexity by comparing the number of model parameters and decoding time. The machine sizes are included in the compression ratio computation. GCDTC suffers from slow decoding speed due to the iterative denoising process. We will address this limitation by incorporating progressive distillation \cite{Salimans2022progressive} that reduces the number of iterations in our future work. Despite this limitation, our conditional diffusion model proves its effectiveness and introduces the recent paradigm in generative AI in the context of lossy scientific data compression.

\begin{table}[H]  % 让 Table 也紧跟前面的内容,而不会漂移到其他页面
\centering
\caption{Model complexity and decoding time. Acronyms are in Figure~\ref{fig:compress_result}.}
\label{tab:complexity}
\begin{tabular}{| >{\centering}m{3.5cm}| >{\centering}m{2.5cm}| >{\centering}m{2.5cm}|>{\centering\arraybackslash}m{2.5cm}|}
\hline
\textbf{Model} & \textbf{GCDTC (Ours)} & \textbf{GCAE} & \textbf{SZ} \\
\hline
Number of Parameters & 1.5 M & 1.0 M & - \\
\hline
Decoding Time (sec) & 978.2 & 2.2 & 12.2 \\
\hline
\end{tabular}
\end{table}

\section{Conclusions}

In this work, we have shown the efficacy of a generative AI model---guaranteed conditional diffusion (GCD)---for scientific data compression. In contrast to traditional video, scientific data comprise blocks of tensors and therefore we designed GCD to work in this setting. GCD has three modules: (i) the standard 2D diffusion architecture with U-Net to learn each denoising stage, (ii) an encoder which compresses 3D blocks, produces latent variables, and acts as the conditioning information, and (iii) a post-processing tensor correction network which provides error bound guarantees at each compression ratio. Results on two applications---a climate dataset (E3SM) and a CFD dataset (S3D)---demonstrate that GCD is better or competitive with SZ---a standard lossy scientific data compression method and convolutional autoencoders. Future work will center on four aspects which are expected to further improve GCD: (i) a full adaptation to blocks and hyper-blocks of tensors, (ii) the use of a scale hyperprior approach for quantization, (iii) incorporation of attention within U-Net, and (iv) fast decoding via distillation. We eventually expect GCD and its variants to be a popular third paradigm for data compression, taking its place alongside super-resolution and transform-based paradigms.




\begin{credits}
\subsubsection{\ackname} This work was partially supported by DOE RAPIDS2 DE-SC0021320 and DOE DE-SC0022265.

\subsubsection{\discintname}
The authors have no competing interests to declare that are relevant to the content of this article.
%It is now necessary to declare any competing interests or to specifically state that the authors have no competing interests. Please place the statement with a bold run-in heading in small font size beneath the (optional) acknowledgments\footnote{If EquinOCS, our proceedings submission system, is used, then the disclaimer can be provided directly in the system.}, for example: The authors have no competing interests to declare that are relevant to the content of this article. Or: Author A has received research grants from Company W. Author B has received a speaker honorarium from Company X and owns stock in Company Y. Author C is a member of committee Z.
\end{credits}

\section*{\appendixname}

\paragraph{\textbf{Architectures.}}
We follow a U-Net design in our denoising decoder with simple units to reduce the number of parameters. The key idea is to produce a 3D latent embedding $\boldsymbol{z}_i^e$ and use 2D slices for conditioning. To do so, the first embedding unit, denoted as $\textrm{EU}^1$ in Figure~\ref{fig:cd_archit}, constructs $\boldsymbol{z}_{i}^{e_1}$ such that the number of 2D slices in $\boldsymbol{z}_{i}^{e_1}$ becomes equal to that of  2D slices in $\boldsymbol{x}_0$. In this way, the 2D slices of latent embeddings can be used to condition the 2D diffusion process.

\begin{figure}[ht]
    \vspace{-0.4cm}
    \centering
    \includegraphics[width=\textwidth]{Figures/Architecture.pdf}
    \caption{Illustration of our conditional diffusion model architecture. Numbers above or below the units indicate output channels.}\label{fig:cd_archit}
    \vspace{-0.8cm}
\end{figure}


%
% ---- Bibliography ----
%
% BibTeX users should specify bibliography style 'splncs04'.
% References will then be sorted and formatted in the correct style.
%
\bibliographystyle{splncs04}
\bibliography{ref}
%

\end{document}
