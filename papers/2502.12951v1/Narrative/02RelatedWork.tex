\section{Related Work}

\subsubsection{Diffusion Models for Image Compression.}

Diffusion models \cite{Sohl2015,Song2019,Ho2020,Song2021scorebased,Nichol2021,Dhariwal2021,yu2024pet} have emerged as a dominant class of generative models in machine learning, demonstrating remarkable success across various domains, including image generation and natural language processing. Diffusion models also have recently been explored for image compression tasks. These works demonstrate how diffusion-based generative models can be adapted to achieve state-of-the-art performance in lossy image compression while maintaining high reconstruction quality.

Hoogeboom et al. proposed an autoregressive diffusion model (ADM) \cite{Hoogeboom2022autoregressive} to generate arbitrary-order data with order-agnostic autoregressive models and discrete diffusion models. They showed that ADM could be used for lossless compression tasks. Yang et al. \cite{Yang2023cd} proposed a lossy image compression framework using conditional diffusion models. Their approach transformed an image $\boldsymbol{x}_0$ into latent variables and produced embeddings used to guide a diffusion process of $\boldsymbol{x}_0$. Li et al. \cite{Li2024extreme} introduced a two-stage compression framework combining VAE-based encoding with pre-trained diffusion models for image reconstruction. The VAE-based module produced latent features that guided diffusion in latent space. The denoised latent features were processed by a decoder for reconstruction. Careil et al. \cite{Careil2024towards} employed iterative diffusion models for ultra-low bitrate image compression. Their two-branch architecture balances global structure and local texture, achieving state-of-the-art results in perceptual metrics like FID and KID. Relic et al. \cite{Relic2024lossy} reformulated quantization error removal as a denoising task, leveraging diffusion for latent recovery with minimal computation. Their codec outperforms traditional methods in realism metrics, maintaining high user preference even at reduced bitrates. Ma et al. \cite{ma2024correcting} developed a privileged end-to-end decoder using diffusion models, combining theoretical insights with convolutional decoders to enhance score function approximation. Their method achieves superior perceptual and distortion metrics compared to prior approaches.

These studies demonstrate the adaptability and promise of diffusion models for advancing image compression, setting the stage for future innovations. However, there is no work that leverages diffusion models in lossy scientific data compression.
