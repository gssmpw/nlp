\section*{\appendixname}

\paragraph{\textbf{Architectures.}}
We follow a U-Net design in our denoising decoder with simple units to reduce the number of parameters. The key idea is to produce a 3D latent embedding $\boldsymbol{z}_i^e$ and use 2D slices for conditioning. To do so, the first embedding unit, denoted as $\textrm{EU}^1$ in Figure~\ref{fig:cd_archit}, constructs $\boldsymbol{z}_{i}^{e_1}$ such that the number of 2D slices in $\boldsymbol{z}_{i}^{e_1}$ becomes equal to that of  2D slices in $\boldsymbol{x}_0$. In this way, the 2D slices of latent embeddings can be used to condition the 2D diffusion process.

\begin{figure}[ht]
    \vspace{-0.4cm}
    \centering
    \includegraphics[width=\textwidth]{Figures/Architecture.pdf}
    \caption{Illustration of our conditional diffusion model architecture. Numbers above or below the units indicate output channels.}\label{fig:cd_archit}
    \vspace{-0.8cm}
\end{figure}