\newpage
\appendix
\onecolumn

\section{More In-depth Explanation of Concept Attention}

We show pseudo-code depicting the difference between a vanilla multi-modal attention mechanism and a multi-modal attention mechanism with concept attention added to it. 

\begin{figure*}[h!]
    \centering
    \includegraphics[width=\linewidth]{figures/CodeDifference.pdf}
    \vspace{-0.3in}
    \caption{\textbf{Pseudo-code depicting the (a) multi-modal attention operation used by Flux DiTs and (b) our \tool{} operation.} We leverage the parameters of a multi-modal attention layer to construct a set of contextualized concept embeddings. The concepts query the image tokens (cross-attention) and other concept tokens (self-attention) in an attention operation. The updated concept embeddings are returned in addition to the image and text embeddings. }
    % \caption{\textbf{Psuedo-code depicting the difference between the default multi-modal attention operation and our modified concept attention. } Our concept attention approach can be easily integrated into the multi-modal attention layers of a Flux model. We perform the same multi-modal attention operation between the prompt and image tokens. We then perform a combined cross and self attention operation between the concept and image tokens. The concepts then query the image tokens (cross attention) and other concept tokens (self-attention) in an attention operation. The updated concept embeddings are returned in addition to the image and text embeddings.  }
    \label{fig:concept_attention_code}
\end{figure*}

\newpage
\section{More Qualitative Results}
\label{QualitativeAppendix}

Here we show a variety of qualitative results for our method that we could not fit into the original paper. 

\begin{figure*}[h!]
    \centering
    \includegraphics[width=\linewidth]{figures/supplemental_imagenet_segmentations/QualitativeComparisonFigure.png}
    \vspace{-0.2in}
    \caption{A qualitative comparison between our method and several others. }
    \label{fig:enter-label}
\end{figure*}

\begin{figure*}[h!]
    \centering
    \includegraphics[width=\linewidth]{figures/supplemental_imagenet_segmentations/QualitativeComparisonFigure2.pdf}
    \vspace{-0.2in}
    \caption{A qualitative comparison between our method and several others. }
    \label{fig:enter-label}
\end{figure*}

\begin{figure*}[h!]
    \centering
    \includegraphics[width=\linewidth]{figures/supplemental_imagenet_segmentations/QualitativeComparisonFigure3.pdf}
    \vspace{-0.2in}
    \caption{A qualitative comparison between our method and several others. }
    \label{fig:enter-label}
\end{figure*}




% \begin{figure*}[h!]
%     \centering
%     \includegraphics[width=\linewidth]{figures/QualitativeComparisonFigure2.pdf}
%     \vspace{-0.2in}
%     \caption{A qualitative comparison between our method and several others. }
%     \label{fig:enter-label}
% \end{figure*}

% \begin{figure*}[h!]
%     \centering
%     \includegraphics[width=\linewidth]{figures/QualitativeComparisonFigure2.pdf}
%     \vspace{-0.2in}
%     \caption{A qualitative comparison between our method and several others. }
%     \label{fig:enter-label}
% \end{figure*}



% \begin{figure*}
%     \centering
%     \includegraphics[width=\linewidth]{figures/supplemental_imagenet_segmentations/supplemental_7.png}
%     \caption{A qualitative comparison between numerous baselines on ImageNet Segmentation Images. The top row shows the soft predictions of each method and the bottom shows the binarized segmentation predictions. }
%     \label{fig:enter-label}
% \end{figure*}


\begin{figure*}
    \centering
    \includegraphics[width=\linewidth]{figures/supplemental_imagenet_segmentations/supplemental_6.png}
    \caption{A qualitative comparison between numerous baselines on ImageNet Segmentation Images. The top row shows the soft predictions of each method and the bottom shows the binarized segmentation predictions. }
    \label{fig:enter-label}
\end{figure*}

\begin{figure*}
    \centering
    \includegraphics[width=\linewidth]{figures/supplemental_imagenet_segmentations/supplemental_3.png}
    \caption{A qualitative comparison between numerous baselines on ImageNet Segmentation Images. The top row shows the soft predictions of each method and the bottom shows the binarized segmentation predictions. }
    \label{fig:enter-label}
\end{figure*}


\begin{figure*}
    \centering
    \includegraphics[width=\linewidth]{figures/supplemental_imagenet_segmentations/supplemental_4.png}
    \caption{A qualitative comparison between numerous baselines on ImageNet Segmentation Images. The top row shows the soft predictions of each method and the bottom shows the binarized segmentation predictions. }
    \label{fig:enter-label}
\end{figure*}


% \begin{figure*}
%     \centering
%     \includegraphics[width=\linewidth]{figures/supplemental_imagenet_segmentations/supplemental_5.png}
%     \caption{A qualitative comparison between numerous baselines on ImageNet Segmentation Images. The top row shows the soft predictions of each method and the bottom shows the binarized segmentation predictions. }
%     \label{fig:enter-label}
% \end{figure*}
