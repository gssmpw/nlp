% This must be in the first 5 lines to tell arXiv to use pdfLaTeX, which is strongly recommended.
\pdfoutput=1
% In particular, the hyperref package requires pdfLaTeX in order to break URLs across lines.

\documentclass[11pt]{article}

% Change "review" to "final" to generate the final (sometimes called camera-ready) version.
% Change to "preprint" to generate a non-anonymous version with page numbers.
\usepackage[preprint]{acl}

% Standard package includes
\usepackage{times}
\usepackage{latexsym}

% For proper rendering and hyphenation of words containing Latin characters (including in bib files)
\usepackage[T1]{fontenc}
% For Vietnamese characters
% \usepackage[T5]{fontenc}
% See https://www.latex-project.org/help/documentation/encguide.pdf for other character sets

% This assumes your files are encoded as UTF8
\usepackage[utf8]{inputenc}

% This is not strictly necessary, and may be commented out,
% but it will improve the layout of the manuscript,
% and will typically save some space.
\usepackage{microtype}

% This is also not strictly necessary, and may be commented out.
% However, it will improve the aesthetics of text in
% the typewriter font.
\usepackage{inconsolata}

%Including images in your LaTeX document requires adding
%additional package(s)
\usepackage{graphicx}
\usepackage{amsmath}
\usepackage{enumitem}
\usepackage{xcolor}
\usepackage{todonotes}

\usepackage{multirow} 
\usepackage{hyperref}
\usepackage{graphicx}
\usepackage{array}
\usepackage{booktabs}

% If the title and author information does not fit in the area allocated, uncomment the following
%
%\setlength\titlebox{<dim>}
%
% and set <dim> to something 5cm or larger.

\newcommand{\fys}[1]{{\color{orange}{[fys: #1]}}}

\title{IPAD: Inverse Prompt for AI Detection - A Robust and Explainable LLM-Generated Text Detector}

% Author information can be set in various styles:
% For several authors from the same institution:
% \author{Author 1 \and ... \and Author n \\
%         Address line \\ ... \\ Address line}
% if the names do not fit well on one line use
%         Author 1 \\ {\bf Author 2} \\ ... \\ {\bf Author n} \\
% For authors from different institutions:
% \author{Author 1 \\ Address line \\  ... \\ Address line
%         \And  ... \And
%         Author n \\ Address line \\ ... \\ Address line}
% To start a separate ``row'' of authors use \AND, as in
% \author{Author 1 \\ Address line \\  ... \\ Address line
%         \AND
%         Author 2 \\ Address line \\ ... \\ Address line \And
%         Author 3 \\ Address line \\ ... \\ Address line}


\author{%
  \textbf{Zheng Chen\textsuperscript{1}\thanks{Authors contributed equally to this work.}},%
  \textbf{Yushi Feng\textsuperscript{2}\footnotemark[1]},%
  \textbf{Changyang He\textsuperscript{3}},%
  \textbf{Yue Deng\textsuperscript{1}},%
  \textbf{Hongxi Pu\textsuperscript{4}},%
  \textbf{Bo Li\textsuperscript{1}}\\[1ex]
  \textsuperscript{1}Hong Kong University of Science and Technology, Hong Kong\\
  \textsuperscript{2}The University of Hong Kong, Hong Kong\\
  \textsuperscript{3}Max Planck Institute, Germany\\
  \textsuperscript{4}University of Michigan, United States\\[1ex]
  \small{\textbf{Correspondence:} \href{mailto:zchenin@connect.ust.hk}{zchenin@connect.ust.hk}}
}


%\author{
%  \textbf{First Author\textsuperscript{1}},
%  \textbf{Second Author\textsuperscript{1,2}},
%  \textbf{Third T. Author\textsuperscript{1}},
%  \textbf{Fourth Author\textsuperscript{1}},
%\\
%  \textbf{Fifth Author\textsuperscript{1,2}},
%  \textbf{Sixth Author\textsuperscript{1}},
%  \textbf{Seventh Author\textsuperscript{1}},
%  \textbf{Eighth Author \textsuperscript{1,2,3,4}},
%\\
%  \textbf{Ninth Author\textsuperscript{1}},
%  \textbf{Tenth Author\textsuperscript{1}},
%  \textbf{Eleventh E. Author\textsuperscript{1,2,3,4,5}},
%  \textbf{Twelfth Author\textsuperscript{1}},
%\\
%  \textbf{Thirteenth Author\textsuperscript{3}},
%  \textbf{Fourteenth F. Author\textsuperscript{2,4}},
%  \textbf{Fifteenth Author\textsuperscript{1}},
%  \textbf{Sixteenth Author\textsuperscript{1}},
%\\
%  \textbf{Seventeenth S. Author\textsuperscript{4,5}},
%  \textbf{Eighteenth Author\textsuperscript{3,4}},
%  \textbf{Nineteenth N. Author\textsuperscript{2,5}},
%  \textbf{Twentieth Author\textsuperscript{1}}
%\\
%\\
%  \textsuperscript{1}Affiliation 1,
%  \textsuperscript{2}Affiliation 2,
%  \textsuperscript{3}Affiliation 3,
%  \textsuperscript{4}Affiliation 4,
%  \textsuperscript{5}Affiliation 5
%\\
%  \small{
%    \textbf{Correspondence:} \href{mailto:email@domain}{email@domain}
%  }
%}

\begin{document}
\maketitle
\begin{abstract}
Large Language Models (LLMs) have attained human-level fluency in text generation, which complicates the distinguishing between human-written and LLM-generated texts. This increases the risk of misuse and highlights the need for reliable detectors. Yet, existing detectors exhibit poor robustness on out-of-distribution (OOD) data and attacked data, which is critical for real-world scenarios. Also, they struggle to provide explainable evidence to support their decisions, thus undermining the reliability. In light of these challenges, we propose ~\textbf{IPAD (Inverse Prompt for AI Detection)}, a novel framework consisting of a ~\textbf{Prompt Inverter} that identifies predicted prompts that could have generated the input text, and a ~\textbf{Distinguisher} that examines how well the input texts align with the predicted prompts. We develop and examine two versions of ~\textbf{Distinguishers}. Empirical evaluations demonstrate that both ~\textbf{Distinguishers} perform significantly better than the baseline methods, with version2 outperforming baselines by 9.73\% on in-distribution data (F1-score) and 12.65\% on OOD data (AUROC). Furthermore, a user study is conducted to illustrate that IPAD enhances the AI detection trustworthiness by allowing users to directly examine the decision-making evidence, which provides interpretable support for its state-of-the-art detection results.


%: the ~\textit{Prompt-Text Consistency Verifier} that verifies whether the predicted prompt could plausibly generate the input text, and the ~\textit{Regeneration Comparator} that evaluates the similarity between the regenerated texts and the input texts. 

% a novel framework for detecting AI-generated text by identifying the most likely prompts that could have produced it. IPAD experiments in two settings: \textbf{Setting1} verifies whether the predicted prompt could plausibly generate the input text, and \textbf{Setting2} further assesses the similarity between the re-generated texts and the input texts. Empirical evaluations demonstrate that both settings significantly surpass baseline detectors, with Setting2 outperforming baselines by 11.3\% on in-distribution data and 12.65\% on out-of-distribution (ood) data. Additionally, Setting2 exhibits superior robustness on ood data and attacked data. These results highlight the enhanced reliability, robustness, and explainability of IPAD. Furthermore, a user study indicates that IPAD enhances the AI detection experience by providing more explainable and reliable results.
%也许太长不放在abstract里. If the result is affirmative, it suggests that the possible prompt is likely to have generated the user input, implying that the content is probably AI-generated. Conversely, if the result is negative, it indicates that even the most probable prompt cannot generate the user input, suggesting that it is most likely human-written.
%说AI DETECTOR还是LLM DETECTOR也许还要斟酌一下。

\end{abstract}

\section{Introduction}\label{sec:Introduction}
\section{Introduction}

Tutoring has long been recognized as one of the most effective methods for enhancing human learning outcomes and addressing educational disparities~\citep{hill2005effects}. 
By providing personalized guidance to students, intelligent tutoring systems (ITS) have proven to be nearly as effective as human tutors in fostering deep understanding and skill acquisition, with research showing comparable learning gains~\citep{vanlehn2011relative,rus2013recent}.
More recently, the advancement of large language models (LLMs) has offered unprecedented opportunities to replicate these benefits in tutoring agents~\citep{dan2023educhat,jin2024teach,chen2024empowering}, unlocking the enormous potential to solve knowledge-intensive tasks such as answering complex questions or clarifying concepts.


\begin{figure}[t!]
\centering
\includegraphics[width=1.0\linewidth]{Figs/Fig.intro.pdf}
\caption{An illustration of coding tutoring, where a tutor aims to proactively guide students toward completing a target coding task while adapting to students' varying levels of background knowledge. \vspace{-5pt}}
\label{fig:example}
\end{figure}

\begin{figure}[t!]
\centering
\includegraphics[width=1.0\linewidth]{Figs/Fig.scaling.pdf}
\caption{\textsc{Traver} with the trained verifier shows inference-time scaling for coding tutoring (detailed in \S\ref{sec:scaling_analysis}). \textbf{Left}: Performance vs. sampled candidate utterances per turn. \textbf{Right}: Performance vs. total tokens consumed per tutoring session. \vspace{-15pt}}
\label{fig:scale}
\end{figure}


Previous research has extensively explored tutoring in educational fields, including language learning~\cite{swartz2012intelligent,stasaski-etal-2020-cima}, math reasoning~\cite{demszky-hill-2023-ncte,macina-etal-2023-mathdial}, and scientific concept education~\cite{yuan-etal-2024-boosting,yang2024leveraging}. 
Most aim to enhance students' understanding of target knowledge by employing pedagogical strategies such as recommending exercises~\cite{deng2023towards} or selecting teaching examples~\cite{ross-andreas-2024-toward}. 
However, these approaches fall short in broader situations requiring both understanding and practical application of specific pieces of knowledge to solve real-world, goal-driven problems. 
Such scenarios demand tutors to proactively guide people toward completing targeted tasks (e.g., coding).
Furthermore, the tutoring outcomes are challenging to assess since targeted tasks can often be completed by open-ended solutions.



To bridge this gap, we introduce \textbf{coding tutoring}, a promising yet underexplored task for LLM agents.
As illustrated in Figure~\ref{fig:example}, the tutor is provided with a target coding task and task-specific knowledge (e.g., cross-file dependencies and reference solutions), while the student is given only the coding task. The tutor does not know the student's prior knowledge about the task.
Coding tutoring requires the tutor to proactively guide the student toward completing the target task through dialogue.
This is inherently a goal-oriented process where tutors guide students using task-specific knowledge to achieve predefined objectives. 
Effective tutoring requires personalization, as tutors must adapt their guidance and communication style to students with varying levels of prior knowledge. 


Developing effective tutoring agents is challenging because off-the-shelf LLMs lack grounding to task-specific knowledge and interaction context.
Specifically, tutoring requires \textit{epistemic grounding}~\citep{tsai2016concept}, where domain expertise and assessment can vary significantly, and \textit{communicative grounding}~\citep{chai2018language}, necessary for proactively adapting communications to students' current knowledge.
To address these challenges, we propose the \textbf{Tra}ce-and-\textbf{Ver}ify (\textbf{\model}) agent workflow for building effective LLM-powered coding tutors. 
Leveraging knowledge tracing (KT)~\citep{corbett1994knowledge,scarlatos2024exploring}, \model explicitly estimates a student's knowledge state at each turn, which drives the tutor agents to adapt their language to fill the gaps in task-specific knowledge during utterance generation. 
Drawing inspiration from value-guided search mechanisms~\citep{lightman2023let,wang2024math,zhang2024rest}, \model incorporates a turn-by-turn reward model as a verifier to rank candidate utterances. 
By sampling more candidate tutor utterances during inference (see Figure~\ref{fig:scale}), \model ensures the selection of optimal utterances that prioritize goal-driven guidance and advance the tutoring progression effectively. 
Furthermore, we present \textbf{Di}alogue for \textbf{C}oding \textbf{T}utoring (\textbf{\eval}), an automatic protocol designed to assess the performance of tutoring agents. 
\eval employs code generation tests and simulated students with varying levels of programming expertise for evaluation. While human evaluation remains the gold standard for assessing tutoring agents, its reliance on time-intensive and costly processes often hinders rapid iteration during development. 
By leveraging simulated students, \eval serves as an efficient and scalable proxy, enabling reproducible assessments and accelerated agent improvement prior to final human validation. 



Through extensive experiments, we show that agents developed by \model consistently demonstrate higher success rates in guiding students to complete target coding tasks compared to baseline methods. We present detailed ablation studies, human evaluations, and an inference time scaling analysis, highlighting the transferability and scalability of our tutoring agent workflow.

\section{Methodology}\label{sec:Method}
\section{Method}

We conduct a systematic literature review to address our research question. Following prior method~\cite{nightingale2009guide}, we aim to identify relevant research papers on RPAs and provide a comprehensive summary of the literature. We selected four widely used academic databases: Google Scholar, ACM Digital Library, IEEE Xplore, and ACL Anthology. These databases encompass a broad spectrum of research across AI, human-computer interaction, and computational linguistics. Given the rapid advancements in LLM research, we included both peer-reviewed and preprint studies (e.g., from arXiv) to capture the latest developments. Below, we detail our paper selection and annotation process.

\begin{table*}[t]
\small

\caption{Definition and examples of six agent attributes.}
% \vspace{-0.5em}
\resizebox{\textwidth}{!}{%
\begin{tabular}{@{}p{0.21\textwidth}p{0.45\textwidth}p{0.33\textwidth}@{}}
\toprule
\textbf{Agent attributes}     & \textbf{Definition}     & \textbf{Examples} \\ 
\midrule
Activity History        & A record of past actions, behaviors, and engagements, including schedules, browsing history, and lifestyle choices. & Backstory, plot, weekly schedule, browsing history, social media posts, lifestyle       \\ 
Belief and Value        & The principles, attitudes, and ideological stances that shape an individual's perspectives and decisions.           & Stances, beliefs, attitudes, values, political leaning, religion                            \\ 
Demographic Information & Personal identifying details such as name, age, education, career, and location.                                    & Name, appearance, gender, age, date of birth, education, location, career, household income \\ 
Psychological Traits    & Characteristics related to personality, emotions, interests, and cognitive tendencies.                              & Personality, hobby and interest, emotional                                                  \\ 
Skill and Expertise     & The knowledge level, proficiency, and capability in specific domains or technologies.                             & Knowledge level, technology proficiency, skills                                            \\ 
Social Relationships & The nature and dynamics of interactions with others, including roles, connections, and communication styles.        & Parenting styles, interactions with players                                                \\ 
\bottomrule
\end{tabular}
}
\vspace{-1em}
\label{attr_def}
\end{table*}

\subsection{Literature Search and Screening Method}

\begin{figure}
    \includegraphics[width=\linewidth]{Figures/simple-PRISMA-1.png}
    % \vspace{-1em}
    \caption{Screening process of literature review. We initially retrieved $1,676$ papers published between 2021 and 2024, and narrowed down to $122$ final selections.}
    \vspace{-1em}
    \label{fig:prisma}
\end{figure}

Our literature review focuses on LLM agents that role-play human behaviors, such as decision-making, reasoning, and deliberate actions. We specifically focus on studies where LLM agents demonstrate the ability to simulate human-like cognitive processes in their objectives, methodologies, or evaluation techniques. To ensure methodological rigor, we define explicit inclusion and exclusion criteria (Tab.~\ref{tab:criteria} in Appendix~\ref{tab: inclusion and exclusion criteria}). 

The inclusion criteria require that an LLM agent in the study exhibits human-like behavior, engages in cognitive activities such as decision-making or reasoning, and operates in an open-ended task environment. We excluded studies where LLM agents primarily serve as chatbots, task-specific assistants, evaluators, or agents operating within predefined and finite action spaces. Additionally, studies focusing solely on perception-based tasks (e.g., computer vision or sensor-based autonomous driving) without cognitive simulation were also excluded.

Using this scope, we searched four databases using the query string provided in Appendix~\ref{query string}, retrieving $1,676$ papers published between January 2021 to December 2024. After removing duplicates, $1,573$ unique papers remained. Two authors independently screened the paper titles and abstracts based on the inclusion criteria. If at least one author deemed a paper relevant, it proceeded to full-text screening, where two authors reviewed the paper in detail and resolved any disagreements through discussion (Fig.~\ref{fig:prisma}). The final set of selected studies comprised $122$ publications.


\subsection{Paper Annotation Method}
Our team followed established open coding procedures \cite{brod2009qualitative} to conduct an inductive coding process to identify key themes. Three co-authors with extensive experience in LLM agents (``annotators,'' hereinafter) collaboratively annotated the papers on three dimensions: \textbf{agent attributes}, \textbf{task attributes}, and \textbf{evaluation metrics}. 

To ensure consistency, two annotators independently annotated the same 20\% of articles and then held a meeting to discuss and refine an initial set of categories for the three dimensions. After reaching a consensus, each annotator annotated half of the remaining papers and cross-validated the other half annotated by the other annotator. Once the annotations were completed, a third annotator reviewed the coded data and identified potential discrepancies. 
Any discrepancies were discussed among the annotators to ensure consistency until disagreements were resolved, ensuring reliability and validity through an iterative refinement process.
\section{Experiments}\label{sec:Evaluation}
We investigate the following questions through our experiments:  

\begin{itemize}[itemsep=1pt, topsep=1pt]

\item[$\bullet$]Assess the robustness of IPAD (using various LLMs as generators, comparing with other detectors, and evaluating on out-of-distribution (OOD) datasets).  
\item[$\bullet$]Independently analyze the necessity and effectiveness of the \textbf{Prompt Inverter} and the \textbf{Distinguishers}.  
\item[$\bullet$]Explore the explainability of IPAD (through a user study and analysis of linguistic differences between prompts generated by HWT and LGT).
\end{itemize}

\subsection{Robustness of IPAD}
\subsubsection{Evaluation Baselines and Metrics}
The in-distribution experiments refer to the testing results presented in ~\cite{r3}, where the data aligns with the training data used for the IPAD ~\textbf{Distinguishers}, thereby serving as our baseline. The OOD experiments refer to the DetectRL baseline ~\cite{r58}, which is a comprehensive benchmark consisting of academic abstracts from the arXiv Archive (covering the years 2002 to 2017)\footnote{http://kaggle.com/datasets/spsayakpaul/arxiv-paper-abstracts/data}, news articles from the XSum dataset ~\cite{r59}, creative stories from Writing Prompts ~\cite{r60}, and social reviews from Yelp Reviews~\cite{r61}. It also employs three attack methods to simulate complex real-world detection scenarios, which includes the prompt attacks, paraphrase attacks, and perturbation attacks~\cite{r58}. All the testing sets have 1,000 samples in our experiments.

The ~\textbf{Area Under Receiver Operating Characteristic curve (AUROC)} is widely used for assessing detection method ~\cite{r55} because it considers the True Positive Rate (TPR) and False
Positive Rate (FPR) across different classification thresholds. Since our models predicts binary labels, we follow the ~\textit{Wilcoxon-Mann-Whitney} statistic~\cite{r56}, and the formula is shown in appendix ~\ref{sec:AUROC formula}. 
The ~\textbf{AvgRec} is the average of ~\textbf{HumanRec} and ~\textbf{MachineRec}. In our evaluation, ~\textbf{HumanRec} is the recall for detecting Human-written texts, and ~\textbf{MachineRec} is the recall for detecting LLM-generated texts~\cite{r57}.  The ~\textbf{F1 Score} provides a comprehensive evaluation of detector capabilities by balancing the model’s Precision and Recall. We use ~\textbf{AvgRec} and ~\textbf{F1} on in-distribution data, and we use ~\textbf{AUROC} for OOD data to align the test benchmarks for the same dataset.


\subsubsection{Robustness across different LLMs}
The results of IPAD for detecting the dataset OUTFOX~\cite{r3} across LLMs are presented in Table \ref{tab:performance_metrics_setting1} and Table \ref{tab:performance_metrics_setting2}, respectively. They show that both versions are highly robust across various LLMs, while ~\textit{Regeneration Comparator} is a bit more efficient. 

As for ~\textit{Regeneration Comparator}, when the original generator and re-generator are the same model, the performance is optimal. However, even when the re-generator is different from the original generator, the results remain impressive with ChatGPT used as the re-generator. These results imply that, in practical applications, it is possible to use a common set of LLMs as re-generators. If one or more correponding ~\textbf{Distinguishers} from different LLMs classify the results as 'yes', it can be inferred that the text is likely to be LGT, whereas if all ~\textbf{Distinguishers} classify the results as 'no', the text is more likely to be HWT. Furthermore, for applications aiming to save computational resources and improve efficiency, using ChatGPT as the sole re-generator still yields robust performance across all tested models.

\begin{table}[ht!]
  \centering
  \resizebox{0.5\textwidth}{!}{%
    \begin{tabular}{ccccc}
      \hline
      \multirow{2}{*}{\textbf{Original Generator}} &  \multicolumn{4}{c}{\textbf{Metrics (\%)}} \\
      \cline{2-5}
      & \textbf{HumanRec} & \textbf{MachineRec} & \textbf{AvgRec} & \textbf{F1} \\
      \hline
      ChatGPT     & 98.00\% & 99.80\%  & 98.90\% & 98.89\% \\
      \hline
      GPT-3.5     & 97.20\% & 99.90\%  & 98.55\% & 98.53\% \\
      \hline
      Qwen-turbo  & 98.00\% & 98.10\%  & 98.05\% & 98.05\% \\
      \hline
      Llama-3-70B & 98.00\% & 100.00\% & 99.00\% & 98.99\% \\
      \hline
    \end{tabular}%
  }
  \caption{IPAD with ~\textit{Prompt-Text Consistency Verifier} performance on different LLMs}
  \label{tab:performance_metrics_setting1}
\end{table}

\begin{table}[ht!]
  \centering
  \resizebox{0.5\textwidth}{!}{
    \begin{tabular}{cccccc}
      \hline
      \multirow{2}{*}{\textbf{Original Generator}} & \multirow{2}{*}{\textbf{Re-Generator}} & \multicolumn{4}{c}{\textbf{Metrics (\%)}} \\
      \cline{3-6}
      & & \textbf{HumanRec} & \textbf{MachineRec} & \textbf{AvgRec} & \textbf{F1} \\
      \hline
      ChatGPT & ChatGPT & 99.70\% & 100.00\% & \textbf{99.85\%} & \textbf{99.85\%} \\
      \hline
      GPT-3.5 & GPT-3.5 & 98.00\% & 100.00\% & ~\textbf{99.00\%} & ~\textbf{99.00\%} \\
      & ChatGPT & 97.00\% & 100.00\% & 98.50\% & 98.50\% \\
      \hline
      Qwen-turbo & Qwen-turbo & 98.00\% & 98.40\% & ~\textbf{98.20\%} & ~\textbf{98.20\%} \\
      & ChatGPT & 99.70\% & 94.40\% & 97.05\% & 97.13\% \\
      \hline
      Llama-3-70B & Llama-3-70B & 96.60\% & 100.00\% & 98.30\% & 98.30\% \\
      & ChatGPT & 99.70\% & 99.40\% & ~\textbf{99.55\%} & ~\textbf{99.55\%} \\
      \hline
    \end{tabular}
  }
  \caption{IPAD with ~\textit{Regeneration Comparator} performance on different LLMs}
  \label{tab:performance_metrics_setting2}
\end{table}


\subsubsection{Comparison of IPAD with other detectors in and out of distribution}
Table \ref{tab:performance_metrics_detection} compares the performance of two versions of IPAD with other detection methods in the OUTFOX dataset with and without attacks~\cite{r3}. The results show that both versions of IPAD generally outperform other detectors, while that IPAD with ~\textit{Prompt-Text Consistency Verifier} for detecting ChatGPT with DIPPER attack performs worse. These results imply that IPAD with ~\textit{Regeneration Comparator} demonstrates superior robustness compared to alternative detection methods in the OUTFOX dataset with and without attacks.

\begin{table}[ht!]
  \centering
  \resizebox{0.5\textwidth}{!}{
    \begin{tabular}{cccccc}
      \hline
      \multirow{2}{*}{\textbf{Original Generator}} & \multirow{2}{*}{\textbf{Detection Methods}} & \multicolumn{4}{c}{\textbf{Metrics (\%)}} \\
      \cline{3-6}
      & & \textbf{HumanRec} & \textbf{MachineRec} & \textbf{AvgRec} & \textbf{F1} \\
      \hline
      ChatGPT & RoBERTa-base & 93.80\% & 92.20\% & 93.00\% & 92.90\% \\
      & RoBERTa-large & 91.60\% & 90.00\% & 90.80\% & 90.70\% \\
      & HC3 detector & 79.20\% & 70.60\% & 74.90\% & 73.80\% \\
      & OUTFOX & 97.80\% & 92.40\% & 95.10\% & 95.00\% \\
      & IPAD version1 & 98.00\% & 99.80\% & 98.90\% & 98.89\%\\
      & IPAD version2& 99.70\% & 100.00\% & \textbf{99.85\%} & \textbf{99.85\%} \\
      \hline
      GPT-3.5 & RoBERTa-base & 93.80\% & 92.00\% & 92.90\% & 92.80\% \\
      & RoBERTa-large & 92.60\% & 92.00\% & 92.30\% & 92.30\% \\
      & HC3 detector & 79.20\% & 85.00\% & 82.10\% & 82.60\% \\
      & OUTFOX & 97.60\% & 96.20\% & 96.90\% & 96.90\% \\
      & IPAD version1 & 97.20\% & 99.90\% & \textbf{98.55\%} & \textbf{98.53\%}\\
      & IPAD version2& 97.00\% & 100.00\% & 98.50\% & 98.50\% \\
      \hline
      ChatGPT with DIPPER Attack & RoBERTa-base & 93.80\% & 89.20\% & 91.50\% & 91.30\% \\
      & RoBERTa-large & 91.60\% & 97.00\% & 94.30\% & 94.40\% \\
      & HC3 detector & 79.20\% & 3.40\% & 41.30\% & 5.50\% \\
      & OUTFOX & 98.60\% & 66.20\% & 82.40\% & 79.00\% \\
      & IPAD version1 & 98.00\% & 75.10\% & 86.55\% & 87.93\%\\
      & IPAD version2& 99.70\% & 95.40\% & \textbf{97.55\%} & \textbf{97.60\%} \\
      \hline
      ChatGPT with OUTFOX Attack & RoBERTa-base & 93.80\% & 69.20\% & 81.50\% & 78.90\% \\
      & RoBERTa-large & 91.60\% & 56.20\% & 73.90\% & 68.30\% \\
      & HC3 detector & 79.20\% & 0.40\% & 39.80\% & 0.70\% \\
      & OUTFOX & 98.80\% & 24.80\% & 61.80\% & 39.40\% \\
      & IPAD version1 & 98.00\% & 95.40\% & 96.70\% & 96.74\%\\
      & IPAD version2& 99.70\% & 98.00\% & \textbf{98.85\%} & \textbf{98.86\%} \\
      \hline
    \end{tabular}
    }
  \caption{Comparison of IPAD with other detectors on in-distribution data, where ~\textbf{IPAD version1} stands for ~\textbf{IPAD with ~\textit{Prompt-Text Consistency Verifier}} and ~\textbf{IPAD version2} stands for ~\textbf{IPAD with ~\textit{Regeneration Comparator}}}
  \label{tab:performance_metrics_detection}
\end{table}

Table \ref{tab:OOD_performance} presents the performance of various detection methods on OOD datasets to assess their generalizability, where the baseline data refer to DetectRL ~\cite{r58}.  The results demonstrate that IPAD with ~\textit{Regeneration Comparator} consistently outperforms all other baselines in all OOD datasets with and without attacks. In contrast, IPAD with ~\textit{Prompt-Text Consistency Verifier} exhibits strong performance on OOD datasets without attacks but shows a noticeable drop in effectiveness when subjected to attacks. For instance, while it achieves competitive results on datasets like XSum (99.90\%) and Writing (99.20\%), its performance against attacks, such as Prompt Attack (86.90\%) and Paraphrase Attack (82.72\%), is significantly lower than IPAD with ~\textit{Regeneration Comparator}. This suggests that \textbf{IPAD with ~\textit{Regeneration Comparator} demonstrates better generalizability and robustness.}

\begin{table}[h]
    \centering
    \resizebox{0.5\textwidth}{!}{ 
    \begin{tabular}{cccccccc}
        \toprule
        \hline
        \multirow{2}{*}{\textbf{OOD Datasets or attack type}} & \multicolumn{5}{c}{\textbf{Detection Methods}} \\
        \cmidrule(lr){2-6}
         & \textbf{LRR} & \textbf{Fast-DetectGPT} & \textbf{Rob-Base} & IPAD with version1 & IPAD version2 \\
        \midrule
        \hline
        Arxiv & 48.17\% & 42.00\% & 81.06\% & 84.47\% & \textbf{98.60\%} \\
        XSum & 48.41\% & 45.72\% & 76.81\% & \textbf{99.90\%} & 98.90\% \\
        Writing & 58.70\% & 51.13\% & 86.29\%& \textbf{99.20\%} & 95.80\% \\
        Review & 58.21\% & 54.55\% & 87.84\% &98.50\% & \textbf{89.30\%} \\
        \hline
        Avg. for non-attacked datasets & 53.37\% & 48.35\% & 83.00\% &95.52\% & \textbf{95.65\%} \\
        \hline
        Prompt Attack & 54.97\% & 43.89\% & 92.81\%& 86.90\% & \textbf{93.05\%}\\
        Paraphrase Attack & 49.23\% & 41.15\% & 90.02\%&82.72\% & \textbf{95.89\%}\\
        Perturbation Attack & 53.62\% & 44.38\% & 92.12\% & 94.96\% & \textbf{95.32\%} \\
        \hline
        Avg. for attacked datasets & 52.61\% & 43.14\% & 91.65\% & 88.26\% & \textbf{94.75\%}\\
        \hline
        Avg. & 53.04\% & 46.12\% & 86.70\%&92.41\%&\textbf{95.26\%}\\
        \hline
        \bottomrule
    \end{tabular}
    }
    \caption{The performance of IPAD in generalization assessment (AUROC). The selected detectors are evaluated on OOD data, all sourced from and processed using the DetectRL baseline, where ~\textbf{IPAD version1} stands for ~\textbf{IPAD with ~\textit{Prompt-Text Consistency Verifier}} and ~\textbf{IPAD version2} stands for ~\textbf{IPAD with ~\textit{Regeneration Comparator}.}}
    \label{tab:OOD_performance}
\end{table}

\vspace{-0.3cm}

\subsubsection{Robustness conclusion}

Our experimental results demonstrate that both IPAD versions exhibit strong performance across different LLMs, outperforming existing detection methods and maintaining robustness on OOD datasets. The IPAD with ~\textit{Regeneration Comparator} outperforming baselines by 9.73\% (F1-score) on in-distribution data and 12.65\% (AUROC) OOD data. Notably, IPAD with ~\textit{Regeneration Comparator} achieves significantly better performance than IPAD with ~\textit{Prompt-Text Consistency Verifier} in attack scenarios of 3.78\% (F1-score). While IPAD with ~\textit{Prompt-Text Consistency Verifier} performs robustly in standard settings, its performance declines when facing attacks. The calculation of these statistics are shown in Appendix ~\ref{Calculation}.
%

\vspace{-0.3cm}
\subsection{Necessity and Effectiveness of \textbf{Prompt Inverter} and \textbf{Distinguishers}}

\subsubsection{Necissity of the \textbf{Prompt Inverter} and \textbf{Distinguishers}}
To prove that it is necessary to fine-tune on IPAD with IPAD with ~\textit{Prompt-Text Consistency Verifier} and ~\textit{Regeneration Comparator}, we conducted ablation study to use the same finetune method on only ~\textit{input texts} and only ~\textit{predicted prompts}. The instructions are ~\textit{"Is this text generated by LLM?"}, and ~\textit{"Prompt Inverter predicts prompt that could have generated the input texts. Is this prompt predicted by an input texts written by LLM?"}, respectively.

The results shown in Figure ~\ref{fig:ablation} from the ablation study show that fine-tuning on either only the ~\textit{input text} or only the ~\textit{predicted prompt} leads to poor performance. This underscores the importance of fine-tuning on a combination of both the input text and predicted prompt, as explored in the ~\textit{Prompt-Text Consistency Verifier}, or on the input text and regenerated text, as examined in the ~\textit{Regeneration Comparator}, for more effective detection.
\begin{figure}[t]
  \centering
  \includegraphics[width=0.5\textwidth]{ablation.png}
  \caption{Ablation Study Results. The ~\textbf{IPAD version1} stands for ~\textbf{IPAD with ~\textit{Prompt-Text Consistency Verifier}} and ~\textbf{IPAD version2} stands for ~\textbf{IPAD with ~\textit{Regeneration Comparator}.}}
  \label{fig:ablation}
\end{figure}
\vspace{-0.3cm}
% \begin{table}[ht!]
%   \centering
%   \resizebox{0.5\textwidth}{!}{
%     \begin{tabular}{cccccc}
%       \hline
%       \multirow{}{}{\textbf{Original Generator}} & \multirow{}{}{\textbf{Detection Methods}} & \multicolumn{4}{c}{\textbf{Metrics (\%)}} \\
%       \cline{3-6}
%       & & \textbf{HumanRec} & \textbf{MachineRec} & \textbf{AvgRec} & \textbf{F1} \\
%       \hline
%       ChatGPT & Finetune with only Input & 10.80\% & 12.20\% & 11.00\% & 92.90\% \\
%       & Finetune with only Prompt & 91.60\% & 90.00\% & 90.80\% & 90.70\% \\
%       & IPAD Setting1 & 98.00\% & 99.80\% & 98.90\% & 98.89\%\\
%       & IPAD Setting2& 99.70\% & 100.00\% & \textbf{99.85\%} & \textbf{99.85\%} \\
%       \hline
%       GPT-3.5 & Finetune with only Input & 00\% & 00\% & 00\% & 00\% \\
%       & Finetune with only Prompt & 91.60\% & 90.00\% & 90.80\% & 90.70\% \\
%       & IPAD Setting1 & 98.00\% & 99.80\% & 98.90\% & 98.89\%\\
%       & IPAD Setting2& 99.70\% & 100.00\% & \textbf{99.85\%} & \textbf{99.85\%} \\
%       \hline
%       Qwen & 93.80\% & 92.20\% & 93.00\% & 92.90\% \\
%       & Finetune with only Prompt & 91.60\% & 90.00\% & 90.80\% & 90.70\% \\
%       & IPAD Setting1 & 98.00\% & 99.80\% & 98.90\% & 98.89\%\\
%       & IPAD Setting2& 99.70\% & 100.00\% & \textbf{99.85\%} & \textbf{99.85\%} \\
%       \hline
%       LLAMA & Finetune with only Input & 93.80\% & 92.20\% & 93.00\% & 92.90\% \\
%       & Finetune with only Prompt & 91.60\% & 90.00\% & 90.80\% & 90.70\% \\
%       & IPAD Setting1 & 98.00\% & 99.80\% & 98.90\% & 98.89\%\\
%       & IPAD Setting2& 99.70\% & 100.00\% & \textbf{99.85\%} & \textbf{99.85\%} \\
%       \hline
%     \end{tabular}
%     }
%   \caption{Ablation study (DATA NOT COMPLETED)}
%   \label{tab:performance_metrics_detection}
% \end{table}

\subsubsection{The effectivenss of the IPAD \textbf{Prompt Inverter}}

We use DPIC~\cite{r62} and PE~\cite{r65} as baseline methods for prompt extraction. DPIC employs a zero-shot approach using the prompt states in Appendix ~\ref{sec:DPIC prompt}, while PE uses adversarial attacks to recover system prompts.

In our evaluation, we tested 1000 LGT and 1000 HWT samples. We use only in-distribution data for testing since only these datasets include original prompts. The metrics are all tested on comparing the similarity of the original prompts and the predicted prompts. The results shown in Table ~\ref{tab:model_comparison} illustrate that IPAD consistently outperforms both DPIC and PE across all four metrics (BartScore~\cite{r64}, Sentence-Bert Cosine Similarity~\cite{r63}, BLEU~\cite{r66}, and ROUGE-1~\cite{r67}), which highlight the effectiveness of the IPAD ~\textbf{Prompt Inverter}.

\begin{table}[htbp]
\centering
\resizebox{0.5\textwidth}{!}{
\begin{tabular}{l|c|c|c|c}
\hline
\textbf{Evaluation} & \textbf{Bart-large-cnn} & \textbf{Sentence-Bert} & \textbf{BLEU} & \textbf{ROUGE-1} \\
\hline
\multicolumn{5}{c}{\textbf{LGT}} \\
\hline
DPIC & -2.12 & 0.46 & 5.61E-05 & 0.04 \\
PE & -2.23 & 0.58 & 3.21E-04 & 0.25 \\
IPAD & ~\textbf{-1.84} & ~\textbf{0.69} & ~\textbf{0.24} & ~\textbf{0.51} \\
\hline
\multicolumn{5}{c}{\textbf{HWT}} \\
\hline
DPIC & -2.47 & 0.42 & 8.75E-06 & 0.06 \\
PE & -2.39 & 0.53 & 2.56E-08 & 0.13 \\
IPAD & ~\textbf{-2.22} & ~\textbf{0.57} & ~\textbf{1.30E-01} & ~\textbf{0.39} \\
\hline
\end{tabular}
}
\caption{Comparison of the IPAD \textbf{Prompt Inverter} with other prompt extractors}
\label{tab:model_comparison}
\end{table}
\vspace{-0.3cm}


\subsubsection{The Effectiveness of the IPAD Distinguishers}

To examine the effectiveness of the IPAD ~\textbf{Distinguishers}, we conducted a comparison study using the same dataset but different distinguishing methods. The first and second methods employed Sentence-Bert ~\cite{r63} and Bart-large-cnn ~\cite{r64} to compute the similarity score between the input texts and the regenerated texts. We selected thresholds that maximized AvgRec, which were 0.67 for Sentence-Bert and -2.52 for Bart-large-cnn. The classification rule is that the texts with scores greater than the threshold will be classified as LGT, while the texts with scores less than or equal to the threshold will be classified as HWT.

The third and fourth methods involved directly prompting ChatGPT as follows: 

\textbf{Instruction:} ~\textit{"Text 1 is generated by an LLM. Determine whether Text 2 is also generated by an LLM with a similar prompt. Answer with only YES or NO."}  ~\textbf{Input: }~\textit{"Text 1: \{Regenerated Text\}; Text 2: \{LGT\} or \{HWT\}"}. 

and ~\textbf{Instruction:} ~\textit{"Can LLM generate text2 through the prompt text1? Answer with only YES or NO."} with ~\textbf{Input:} ~\textit{"Text 1: \{Predicted Prompt\}; Text 2: \{Input text\}"}.

The final results demonstrated that the other distinguishing methods performed worse than the two IPAD ~\textbf{Distinguishers}, highlighting the superior effectiveness of the IPAD ~\textbf{Distinguishers}.


\begin{table}[ht]
\centering
\resizebox{0.5\textwidth}{!}{
\begin{tabular}{l|cccc}
\hline
\textbf{Distinguish Method} & \textbf{HumanRec} & \textbf{MachineRec} & \textbf{AvgRec} & \textbf{F1} \\
\hline
Sentence-Bert (Threshold 0.67) & 61.20\% & 95.20\% & 78.20\% & 63.51\% \\
Bart-large-cnn (Threshold -2.52) & 42.60\% & 97.20\% & 69.90\% & 43.96\% \\
Prompt to ChatGPT version 1 & 33.20\% & 64.50\% & 48.85\% & 44.77\% \\
Prompt to ChatGPT version 2 & 12.50\% & 100\% & 56.25\% & 12.50\% \\
IPAD version 1 & 98.00\% & 99.80\% & 98.90\% & 98.10\% \\
IPAD version 2 & ~\textbf{99.70\%} & ~\textbf{100\%} & ~\textbf{99.85\%} & ~\textbf{99.70\%} \\
\hline
\end{tabular}
}
\caption{Comparison of Different Distinguishers, where ~\textbf{IPAD version1} stands for ~\textbf{IPAD with ~\textit{Prompt-Text Consistency Verifier}} and ~\textbf{IPAD version2} stands for ~\textbf{IPAD with ~\textit{Regeneration Comparator}.}}
\label{tab:distinguishers_comparison}
\end{table}
\vspace{-0.3cm}





\subsection{Explanability Assessment of IPAD}
\subsubsection{Different Linguistic Features of HWT prompts and LGT prompts}

This subsection of the evaluation aims to explore the linguistic features of prompts generated by HWT and LGT through the \textbf{Prompt Inverter}. We analyzed 1000 samples generated by HWT and 1000 samples generated by LGT, which are randomy selected from both in-distribution data and OOD.

The analysis is first conducted using the Linguistic Feature Toolkik (lftk)\footnote{https://lftk.readthedocs.io/en/latest/}, a commonly used general-purpose tool for linguistic features extraction, which provides a total of 220 features for text analysis. Upon applying this toolkit, we identified 20 features with significant differences in average values between the two groups, out of which 3 features showed statistically significant differences with p-values less than 0.05. These 3 differences can be summarized as one main aspects: ~\textbf{syntactic complexity}. Beyond these, we referred to the LIWC framework \footnote{https://www.liwc.app/}, which defines 7 function words variables and 4 summary variables. By comparing the difference, two of these 11 features is significantly distinguishable: ~\textbf{the pronoun usage} and ~\textbf{the level of analytical thinking}.

% One of the primary distinctions between the HWT prompts and the LGT prompts is the \textbf{conceptual scope}. The analysis reveals that LGT prompts tend to generate more generalized concepts, while HWT prompts tend to provide more specific and detailed descriptions. Linguistic features show that HWT prompts have significantly more \textbf{geographical entities} (mean value of 0.127 and 0.113), \textbf{organizational entities} (mean value of 0.143 and 0.154), and \textbf{cardinal entitites} (mean value of 0.03 and 0.105) per sentence. These entities, however, are often indirectly related to the core meaning of the prompt, which serve more as supplements rather than integral components of the main topic. For example, as shown in Figure~\ref{fig:concept scope}, HWT prompts would include specific geographical names as examples when describing \textit{car-free zone issue}, detailed organizational sources when stating \textit{"Face on Mars" problem}, and specific reasons when talking about \textit{student sport activities}.

One of the primary distinctions between the HWT prompts and the LGT prompts is \textbf{sentence complexity}. LGT prompts are typically more complex, characterized by \textbf{longer sentence lengths} (mean value of 1.514 and 1.794), \textbf{higher syllable counts} (mean values of total syllabus three are 1.572 and 3.042), and \textbf{more stop-words} (mean values of 9.88 and 10.045). HWT prompts, on the other hand, are characterized by shorter, less complex sentences that are easier to process and understand, as examples shown in Appendix~\ref{sec:Linguistic Difference Examples} Figure~\ref{fig:sentence complexity}.


Beyond the differences in \textbf{syntactic complexity}, we also explored variables in LIWC. We did the difference comparison by using HWT and LGT prompts as inputs for ChatGPT, for example, instructing with the prompts \textit{'determine the pronoun usage of this sentence, answer first person, second person, or third person'} and \textit{'determine the level of analytical thinking of these sentences, answer a number from 1 to 5'}. The results show that there are distinguish difference in pronoun usage and analytical thinking level. The HWT prompts frequently use \textbf{second-person pronouns} (e.g., 'you') - 75 occurrences per 1,000 prompts - due to the subjective tone often employed in HWT. In contrast, LGT prompts primarily feature first- and third-person pronouns, with second-person pronouns appearing only 2 per 1,000 prompts. LGT prompts typically present instructions and questions in a more objective manner. As shown in Appendix ~\ref{sec:Linguistic Difference Examples} Figure ~\ref{fig:comparison}, LGT prompts show higher \textbf{analytical thinking levels} than HWT prompts. With level 1 as the lowest and level 5 as the highest, LGT has 68.9\% of level 4 and 24.3\% of level 5, but HWT has only 48.0\% of level 4, and 0.8\% of level 5. It suggests that LGT prompts encourage more analytical thinking, while HWT prompts tend to focus more on concrete examples, with less emphasis on critical analysis, as examples shown in Appendix~\ref{sec:Linguistic Difference Examples} Figure ~\ref{fig:person}.


\subsection{User Study}
To assess the explainability improvement of IPAD, we designed an IRB-approved user study with ten participants evaluating one HWT and one LGT article. We used IPAD version 2 due to its superior OOD performance and attack resistance. Participants compared three online detection platforms with screenshots shown in Appendix~\ref{User study}\footnote{https://www.scribbr.com/ai-detector/}\footnote{https://quillbot.com/ai-content-detector}\footnote{https://app.gptzero.me/} with IPAD's process (which displayed input texts, predicted prompts, regenerated texts, and final judgments). After evaluation, users rated IPAD on four key explainability dimensions. Transparency received strong ratings (40\%:5, 60\%:4), with users appreciating the visibility of intermediate processes. Trust scores were more varied (10\%:3, 70\%:4, 20\%:5), but IPAD was generally considered more convincing than single-score detectors. Satisfaction was mixed (30\%:3, 30\%:4, 40\%:5), with users acknowledging better detection but raising concerns about energy efficiency since IPAD runs three LLMs. Debugging received unanimous 5s, as users could easily analyze the predicted prompt and regenerated text to verify the decision-making process. If needed, users could refine the generated content by adjusting instructions, such as specifying a word count, making IPAD a more effective and user-friendly tool compared to black-box detectors.


%
\section{Related Work}\label{sec:Related Work}
\section{Rethinking Sparse Attention Methods}
\label{sec:critique}

Modern sparse attention methods have made significant strides in reducing the theoretical computational complexity of transformer models. However, most approaches predominantly apply sparsity during inference while retaining a pretrained Full Attention backbone, potentially introducing architectural bias that limits their ability to fully exploit sparse attention's advantages. Before introducing our native sparse architecture, we systematically analyze these limitations through two critical lenses.


\begin{figure*}[t] 
\centering 
\includegraphics[width=1\textwidth]{figures/fig2.pdf} 
\caption{Overview of \method{}'s architecture. Left: The framework processes input sequences through three parallel attention branches: For a given query, preceding keys and values are processed into compressed attention for coarse-grained patterns, selected attention for important token blocks, and sliding attention for local context. Right: Visualization of different attention patterns produced by each branch. Green areas indicate regions where attention scores need to be computed, while white areas represent regions that can be skipped.}
\label{fig:framework}
\end{figure*}


\subsection{The Illusion of Efficient Inference}

Despite achieving sparsity in attention computation, many methods fail to achieve corresponding reductions in inference latency, primarily due to two challenges:

\textbf{Phase-Restricted Sparsity.}
Methods such as H2O \citep{h2o} apply sparsity during autoregressive decoding while requiring computationally intensive pre-processing (e.g. attention map calculation, index building) during prefilling. In contrast, approaches like MInference \citep{minference} focus solely on prefilling sparsity. 
These methods fail to achieve acceleration across all inference stages, as at least one phase remains computational costs comparable to Full Attention.
The phase specialization reduces the speedup ability of these methods in prefilling-dominated workloads like book summarization and code completion, or decoding-dominated workloads like long chain-of-thought~\citep{cot} reasoning.

\textbf{Incompatibility with Advanced Attention Architecture.}
Some sparse attention methods fail to adapt to modern decoding efficient architectures like Mulitiple-Query Attention~(MQA) \citep{mqa} and Grouped-Query Attention~(GQA) \citep{gqa}, which significantly reduced the memory access bottleneck during decoding by sharing KV across multiple query heads. For instance, in approaches like Quest \citep{quest}, each attention head independently selects its KV-cache subset. Although it demonstrates consistent computation sparsity and memory access sparsity in Multi-Head Attention (MHA) models, it presents a different scenario in models based on architectures like GQA, where the memory access volume of KV-cache corresponds to the union of selections from all query heads within the same GQA group. This architectural characteristic means that while these methods can reduce computation operations, the required KV-cache memory access remains relatively high.
This limitation forces a critical choice: while some sparse attention methods reduce computation, their scattered memory access pattern conflicts with efficient memory access design from advanced architectures.

These limitations arise because many existing sparse attention methods focus on KV-cache reduction or theoretical computation reduction, but struggle to achieve significant latency reduction in advanced frameworks or backends.
This motivates us to develop algorithms that combine both advanced architectural and hardware-efficient implementation to fully leverage sparsity for improving model efficiency.


\subsection{The Myth of Trainable Sparsity}
Our pursuit of native trainable sparse attention is motivated by two key insights from analyzing inference-only approaches:
(1) \textbf{\textit{Performance Degradation}}: Applying sparsity post-hoc forces models to deviate from their pretrained optimization trajectory. As demonstrated by \citet{magicpig}, top 20\% attention can only cover 70\% of the total attention scores, rendering structures like retrieval heads in pretrained models vulnerable to pruning during inference.
(2)~\textbf{\textit{Training Efficiency Demands}}: 
Efficient handling of  long-sequence training is crucial for modern LLM development. This includes both pretraining on longer documents to enhance model capacity, and subsequent adaptation phases such as long-context fine-tuning and reinforcement learning. However, existing sparse attention methods primarily target inference, leaving the computational challenges in training largely unaddressed. This limitation hinders the development of more capable long-context models through efficient training. Additionally, efforts to adapt existing sparse attention for training also expose challenges:



\textbf{Non-Trainable Components.} Discrete operations in methods like ClusterKV~\citep{clusterkv} 
(includes k-means clustering) and MagicPIG~\citep{magicpig} (includes SimHash-based selecting) create discontinuities in the computational graph. These non-trainable components prevent gradient flow through the token selection process, limiting the model's ability to learn optimal sparse patterns. 

\textbf{Inefficient Back-propagation.} Some theoretically trainable sparse attention methods suffer from practical training inefficiencies. Token-granular selection strategy used in approaches like HashAttention~\citep{desai2024hashattention} leads to the need to load a large number of individual tokens from the KV cache during attention computation. 
This non-contiguous memory access prevents efficient adaptation of fast attention techniques like FlashAttention, which rely on contiguous memory access and blockwise computation to achieve high throughput.
As a result, implementations are forced to fall back to low hardware utilization, significantly degrading training efficiency.



\subsection{Native Sparsity as an Imperative}

These limitations in inference efficiency and training viability motivate our fundamental redesign of sparse attention mechanisms.
We propose \method{}, a natively sparse attention framework that addresses both computational efficiency and training requirements.
In the following sections, we detail the algorithmic design and operator implementation of \method{}.

\section{Conclusion}\label{sec:Conclusion}
This paper introduces ~\textbf{IPAD (Inverse Prompt for AI Detection)}, a framework consisting of a ~\textbf{Prompt Inverter} that identifies predicted prompts that could have generated the input text, and a ~\textbf{Distinguisher} that examines how well the input texts align with the predicted prompts. This design enables explainable evidence chains tracing unavailable in existing black-box detectors. Empirical results show that IPAD surpasses the baselines on all in-distribution, OOD, and attacked data. Furthermore, the ~\textbf{Distinguisher} (version2) - ~\textit{Regeneration Comparator} outperforms the ~\textbf{Distinguisher} (version1) - ~\textit{Prompt-Text Consistency Verifier}, especially on OOD and attacked data. While the local alignment in veresion1 approach provides explicit interpretability, it is more sensitive to adversarial attacks. In contrast, the global distribution in veresion2 matching approach implicitly learns generative LLM's distributional properties, which offers more robustness while maintaining explainability. This insight suggests that combining self-consistency checks of generative models with multi-step reasoning for evidential explainability holds promise for future AI detection systems in real-world scenarios. A user study reveals that IPAD enhances trust and transparency by allowing users to examine decision-making evidence. Overall, IPAD establishes a new paradigm for more robust, reliable, and interpretable AI detection systems to combat the misuse of LLMs.
\section{Limitations}\label{sec:Limitation}
While IPAD demonstrates SOTA performance, two limitations warrant discussion:
%
(1) The \textbf{Prompt Inverter} may not fully reconstruct prompts containing explicit in-context learning examples (e.g., formatted demonstrations), as it prioritizes semantic alignment over precise syntactic replication.
%
(2) Since IPAD achieves satisfactory OOD performance (12.65\% improvement over baselines) by only adopting essay writing datasets for the fine-tuning of \textbf{Distinguishers}, we strategically deferred the exploration of more datasets. 
%
We will incorporate a wider and more diverse range of data in future works to explore if it can enhance robustness even further, including: creative/news domains, and triplet data formats (i.e., ~\textit{"Can this \{predicted prompt\} generate the \{Input text\} using an LLM? One example generated by the predicted prompt is: \{regenerated text\}"})

% The \textbf{Prompt Inverter} may not always extract the exact prompt, especially when the prompt includes examples for in-context learning, as it might fail to capture these elements precisely. In the fine-tuning of the \textbf{Distinguishers}, we did not incorporate a wider variety of datasets, such as creative prompt datasets or news datasets, nor did we experiment with the triplet data format (~\textit{”Can this \{predicted prompt\} generate the \{Input text\} using an LLM? One example generated by the predicted prompt is: \{regenerated text\}}). This decision was based on the satisfactory performance of IPAD on out-of-distribution (OOD) data. However, we acknowledge that utilizing a broader and more diverse range of data could potentially improve the performance of IPAD.



\section*{Acknowledgments}

% Bibliography entries for the entire Anthology, followed by custom entries
%\bibliography{anthology,custom}
% Custom bibliography entries only
\bibliography{custom}

\appendix
\section{AUROC formula}
\label{sec:AUROC formula}
Since our model predicts binary labels, we follow the ~\textit{Wilcoxon-Mann-Whitney} statistic~\cite{r56} to calculate the Area Under Receiver Operating Characteristtic curve (AUROC):

\[
\text{AUC}(f) = \frac{\sum_{t_0 \in \mathcal{D}^0} \sum_{t_1 \in \mathcal{D}^1} \mathbf{1}[f(t_0) < f(t_1)]}{|\mathcal{D}^0| \cdot |\mathcal{D}^1|}
\]

where \( \mathbf{1}[f(t_0) < f(t_1)] \) denotes an indicator function which returns 1 if \( f(t_0) < f(t_1) \) and 0 otherwise. \( \mathcal{D}^0 \) is the set of negative examples, and \( \mathcal{D}^1 \) is the set of positive examples.

\section{Calculation of Summary Statistics}
\label{Calculation}
\begin{itemize}

\item[$\bullet$] IPAD with ~\textit{Regeneration Comparator} outperforms the baselines by 9.73\% on in-distribution data. As shown in Table~\ref{tab:performance_metrics_detection}, RoBERTa-base has the best average F1 score of (92.9\% + 92.8\% + 91.3\% + 78.9\%) / 4. In comparison, the average F1 score for IPAD version 2 is (99.85\% + 98.5\% + 97.6\% + 98.86\%) / 4, showing an improvement of 9.73\%.

\item[$\bullet$] IPAD with ~\textit{Regeneration Comparator} outperforms the baselines by 12.65\% on in-distribution data. As shown in Table~\ref{tab:OOD_performance}, RoBERTa-base achieves the highest average AUROC score, but since the F1-score is not available for the baseline, we use the AUROC difference to calculate the improvement, which is (95.65\% - 83\%) = 12.65\%.

\item[$\bullet$] IPAD with ~\textit{Regeneration Comparator} outperforms IPAD with ~\textit{Prompt-Text Consistency Verifier} by 0.13\% on out-of-distribution (OOD) data. As shown in Table~\ref{tab:OOD_performance}, IPAD version 2 has the highest AUROC of 95.65\%, while IPAD version 1 has an AUROC of 95.52\%, resulting in a 0.13\% difference.

\item[$\bullet$] IPAD with ~\textit{Regeneration Comparator} outperforms IPAD with ~\textit{Prompt-Text Consistency Verifier} by 3.78\% on attacked data. As shown in Table~\ref{tab:performance_metrics_detection} (rows 3-4) and Table~\ref{tab:OOD_performance} (rows 6-8), IPAD version 2 achieves the best F1 score and AUROC scores. To calculate the overall attacked dataset score, we calculate the F1 scores for Table~\ref{tab:OOD_performance}: 94.82\%, 95.35\%, 95.31\% for IPAD version 2, and 83.58\%, 88.34\%, and 94.70\% for IPAD version 1. The average F1 score difference is thus (94.82\% + 95.35\% + 95.31\% - 83.58\% - 88.34\% - 94.70\% + 97.60\% + 98.86\% - 97.55\% - 98.85\%) / 5 = 3.78\%.

\end{itemize}


\section{DPIC (decouple prompt and intrinsic characteristics) Prompt Extraction Zero-shot Prompts}
\label{sec:DPIC prompt}
~\textit{"I want you to play the role of the questioner. I will type an answer in English, and you will ask me a question based on the answer in the same language. Don’t write any explanations or other text, just give me the question. <TEXT>."}. 


\section{Linguistic Difference Examples}
\label{sec:Linguistic Difference Examples}
Figure ~\ref{fig:sentence complexity} shows examples where HWT and LGT prompts with different sentence complexity. Figure ~\ref{fig:comparison} shows the results of analytical thinking level statistics. Figure ~\ref{fig:person} shows examples of using different personas and different analytical thinking levels.

% \begin{figure}[t]
%   \centering
%   \includegraphics[width=0.5\textwidth]{concept scope.png}
%   \caption{Concept Scope Examples}
%   \label{fig:concept scope}
% \end{figure}


\begin{figure}[t]
  \centering
  \includegraphics[width=0.5\textwidth]{sentence_complexity.png}
  \caption{Sentence Complexity Examples, where ~\textbf{HWT Prompt} stands for prompt generated by the Prompt Inverter from HWT, and ~\textbf{LGT Prompt} stands for prompt generated by the Prompt Inverter from LGT. The HWT Prompts have longer sentence lengths, more words with more than three syllabus (as shown in bold), and more stop-words (as shown with underline).}
  \label{fig:sentence complexity}
\end{figure}


\begin{figure}[t]
  \centering
  \includegraphics[width=0.5\textwidth]{comparison.png}
  \caption{Comparison of different analytical thinking levels, with LGT has higher percentage of level 4 and level 5.}
  \label{fig:comparison}
\end{figure}

\begin{figure}[t]
  \centering
  \includegraphics[width=0.5\textwidth]{person.png}
  \caption{Examples that use different persona usage (above), and different analytical thinking levels (below left has level 2, and below right has level 5, they are prompts generated by the same problem statements).}
  \label{fig:person}
\end{figure}

\section{User Study}
Figure ~\ref{fig:gptzero}~\ref{fig:quillbot} and ~\ref{fig:scribbr} shows the screenshots of online AI detectors. Figure ~\ref{fig:questionnaire} shows the questionnaire questions. Figure ~\ref{fig:guide} shows the user guide.
\label{User study}
\subsection{Online AI Detectors Screenshots}


\begin{figure}[t]
  \centering
  \includegraphics[width=0.5\textwidth]{gptzero.png}
  \caption{GPTZero Online Detector Screenshot}
  \label{fig:gptzero}
\end{figure}

\begin{figure}[t]
  \centering
  \includegraphics[width=0.5\textwidth]{quillbot.png}
  \caption{Quillbot Online Detector Screenshot}
  \label{fig:quillbot}
\end{figure}

\begin{figure}[t]
  \centering
  \includegraphics[width=0.5\textwidth]{scribbr.png}
  \caption{Scribbr Online Detector Screenshot}
  \label{fig:scribbr}
\end{figure}

\subsection{Questionnaire questions}
\begin{figure}[t]
  \centering
  \includegraphics[width=0.5\textwidth]{questionnaire.png}
  \caption{Questionnaire questions}
  \label{fig:questionnaire}
\end{figure}


\begin{figure}[t]
  \centering
  \includegraphics[width=0.5\textwidth]{IRB.png}
  \caption{User Study User guide}
  \label{fig:guide}
\end{figure}
\end{document}
