% This must be in the first 5 lines to tell arXiv to use pdfLaTeX, which is strongly recommended.
\pdfoutput=1
% In particular, the hyperref package requires pdfLaTeX in order to break URLs across lines.

\documentclass[11pt]{article}

% Change "review" to "final" to generate the final (sometimes called camera-ready) version.
% Change to "preprint" to generate a non-anonymous version with page numbers.
\usepackage[preprint]{acl}

% Standard package includes
\usepackage{times}
\usepackage{latexsym}

% For proper rendering and hyphenation of words containing Latin characters (including in bib files)
\usepackage[T1]{fontenc}
% For Vietnamese characters
% \usepackage[T5]{fontenc}
% See https://www.latex-project.org/help/documentation/encguide.pdf for other character sets

% This assumes your files are encoded as UTF8
\usepackage[utf8]{inputenc}

% This is not strictly necessary, and may be commented out,
% but it will improve the layout of the manuscript,
% and will typically save some space.
\usepackage{microtype}

% This is also not strictly necessary, and may be commented out.
% However, it will improve the aesthetics of text in
% the typewriter font.
\usepackage{inconsolata}

%Including images in your LaTeX document requires adding
%additional package(s)
\usepackage{graphicx}
\usepackage{amsmath}
\usepackage{enumitem}
\usepackage{xcolor}
\usepackage{todonotes}

\usepackage{multirow} 
\usepackage{hyperref}
\usepackage{graphicx}
\usepackage{array}
\usepackage{booktabs}

% If the title and author information does not fit in the area allocated, uncomment the following
%
%\setlength\titlebox{<dim>}
%
% and set <dim> to something 5cm or larger.

\newcommand{\fys}[1]{{\color{orange}{[fys: #1]}}}

\title{IPAD: Inverse Prompt for AI Detection - A Robust and Explainable LLM-Generated Text Detector}

% Author information can be set in various styles:
% For several authors from the same institution:
% \author{Author 1 \and ... \and Author n \\
%         Address line \\ ... \\ Address line}
% if the names do not fit well on one line use
%         Author 1 \\ {\bf Author 2} \\ ... \\ {\bf Author n} \\
% For authors from different institutions:
% \author{Author 1 \\ Address line \\  ... \\ Address line
%         \And  ... \And
%         Author n \\ Address line \\ ... \\ Address line}
% To start a separate ``row'' of authors use \AND, as in
% \author{Author 1 \\ Address line \\  ... \\ Address line
%         \AND
%         Author 2 \\ Address line \\ ... \\ Address line \And
%         Author 3 \\ Address line \\ ... \\ Address line}


\author{%
  \textbf{Zheng Chen\textsuperscript{1}\thanks{Authors contributed equally to this work.}},%
  \textbf{Yushi Feng\textsuperscript{2}\footnotemark[1]},%
  \textbf{Changyang He\textsuperscript{3}},%
  \textbf{Yue Deng\textsuperscript{1}},%
  \textbf{Hongxi Pu\textsuperscript{4}},%
  \textbf{Bo Li\textsuperscript{1}}\\[1ex]
  \textsuperscript{1}Hong Kong University of Science and Technology, Hong Kong\\
  \textsuperscript{2}The University of Hong Kong, Hong Kong\\
  \textsuperscript{3}Max Planck Institute, Germany\\
  \textsuperscript{4}University of Michigan, United States\\[1ex]
  \small{\textbf{Correspondence:} \href{mailto:zchenin@connect.ust.hk}{zchenin@connect.ust.hk}}
}


%\author{
%  \textbf{First Author\textsuperscript{1}},
%  \textbf{Second Author\textsuperscript{1,2}},
%  \textbf{Third T. Author\textsuperscript{1}},
%  \textbf{Fourth Author\textsuperscript{1}},
%\\
%  \textbf{Fifth Author\textsuperscript{1,2}},
%  \textbf{Sixth Author\textsuperscript{1}},
%  \textbf{Seventh Author\textsuperscript{1}},
%  \textbf{Eighth Author \textsuperscript{1,2,3,4}},
%\\
%  \textbf{Ninth Author\textsuperscript{1}},
%  \textbf{Tenth Author\textsuperscript{1}},
%  \textbf{Eleventh E. Author\textsuperscript{1,2,3,4,5}},
%  \textbf{Twelfth Author\textsuperscript{1}},
%\\
%  \textbf{Thirteenth Author\textsuperscript{3}},
%  \textbf{Fourteenth F. Author\textsuperscript{2,4}},
%  \textbf{Fifteenth Author\textsuperscript{1}},
%  \textbf{Sixteenth Author\textsuperscript{1}},
%\\
%  \textbf{Seventeenth S. Author\textsuperscript{4,5}},
%  \textbf{Eighteenth Author\textsuperscript{3,4}},
%  \textbf{Nineteenth N. Author\textsuperscript{2,5}},
%  \textbf{Twentieth Author\textsuperscript{1}}
%\\
%\\
%  \textsuperscript{1}Affiliation 1,
%  \textsuperscript{2}Affiliation 2,
%  \textsuperscript{3}Affiliation 3,
%  \textsuperscript{4}Affiliation 4,
%  \textsuperscript{5}Affiliation 5
%\\
%  \small{
%    \textbf{Correspondence:} \href{mailto:email@domain}{email@domain}
%  }
%}

\begin{document}
\maketitle
\begin{abstract}
Large Language Models (LLMs) have attained human-level fluency in text generation, which complicates the distinguishing between human-written and LLM-generated texts. This increases the risk of misuse and highlights the need for reliable detectors. Yet, existing detectors exhibit poor robustness on out-of-distribution (OOD) data and attacked data, which is critical for real-world scenarios. Also, they struggle to provide explainable evidence to support their decisions, thus undermining the reliability. In light of these challenges, we propose ~\textbf{IPAD (Inverse Prompt for AI Detection)}, a novel framework consisting of a ~\textbf{Prompt Inverter} that identifies predicted prompts that could have generated the input text, and a ~\textbf{Distinguisher} that examines how well the input texts align with the predicted prompts. We develop and examine two versions of ~\textbf{Distinguishers}. Empirical evaluations demonstrate that both ~\textbf{Distinguishers} perform significantly better than the baseline methods, with version2 outperforming baselines by 9.73\% on in-distribution data (F1-score) and 12.65\% on OOD data (AUROC). Furthermore, a user study is conducted to illustrate that IPAD enhances the AI detection trustworthiness by allowing users to directly examine the decision-making evidence, which provides interpretable support for its state-of-the-art detection results.


%: the ~\textit{Prompt-Text Consistency Verifier} that verifies whether the predicted prompt could plausibly generate the input text, and the ~\textit{Regeneration Comparator} that evaluates the similarity between the regenerated texts and the input texts. 

% a novel framework for detecting AI-generated text by identifying the most likely prompts that could have produced it. IPAD experiments in two settings: \textbf{Setting1} verifies whether the predicted prompt could plausibly generate the input text, and \textbf{Setting2} further assesses the similarity between the re-generated texts and the input texts. Empirical evaluations demonstrate that both settings significantly surpass baseline detectors, with Setting2 outperforming baselines by 11.3\% on in-distribution data and 12.65\% on out-of-distribution (ood) data. Additionally, Setting2 exhibits superior robustness on ood data and attacked data. These results highlight the enhanced reliability, robustness, and explainability of IPAD. Furthermore, a user study indicates that IPAD enhances the AI detection experience by providing more explainable and reliable results.
%也许太长不放在abstract里. If the result is affirmative, it suggests that the possible prompt is likely to have generated the user input, implying that the content is probably AI-generated. Conversely, if the result is negative, it indicates that even the most probable prompt cannot generate the user input, suggesting that it is most likely human-written.
%说AI DETECTOR还是LLM DETECTOR也许还要斟酌一下。

\end{abstract}

\section{Introduction}\label{sec:Introduction}
%!TEX root=main.tex

\section{Introduction}
% Decision-makers, analysts, data scientists, and policymakers frequently rely on data to draw conclusions and extract insights. This data-driven approach helps them identify actionable recommendations aimed at influencing an outcome of interest, such as increasing product satisfaction or income levels or decreasing the likelihood of experiencing serious health conditions \cite{galhotra2022hyper,lakkaraju2016interpretable,agrawal1994fast}. 
\revc{Prescriptions, or actionable recommendations, are commonly generated across various fields to influence key outcomes such as improving product satisfaction, enhancing economic policies, or increasing business efficiency. 
%Decision- or policy-makers, analysts, data scientists, and 
Policymakers in government, decision-makers in businesses, and data scientists in various fields, often rely on data-driven approaches to identify 
%actionable recommendations 
potential actions to influence an outcome of interest, such as increasing income levels or loan approval rates}.
% , or decreasing the likelihood of experiencing serious health conditions. 
%
While association or prediction-based methods are extensively used in practice to draw useful insights from data, they typically identify correlations among variables and may fail to reveal the underlying causal factors, i.e., which actions may result in an improved outcome, needed for informed decision-making. 
%For recommendations to be truly impactful, there must be a clear  explanation that justifies why a particular decision is appropriate for a specific subpopulation~\cite{sun2021treatment,plecko2022causal}. 

\emph{Causal analysis} or {\em causal inference}, therefore, is considered one of the most important requirements to generate prescriptions that are {\em actionable} and aligned with human reasoning~\cite{imbens2024causal}. Causal inference, and in particular {\em observational studies} for causal inference on collected data (when controlled trials are impossible due to cost or ethical reasons), have been extensively studied in the statistics and artificial intelligence (AI) literature for several decades \cite{rubin2005causal, pearl2009causal}. Motivated by this foundational work on causal inference, the notion of causality has also influenced the field of database research. The causal models from AI have been extended to relational databases \cite{salimi2020causal},  and causality has been incorporated into various data management tasks such as finding responsibilities of inputs toward query answers ~\cite{meliou2010causality, meliou2009so, meliou2014causality}, explanations for query answers \cite{roy2014formal, DBLP:journals/pacmmod/YoungmannCGR24}, data discovery~\cite{galhotra2023metam,youngmann2023causal}, data cleaning~\cite{pirhadi2024otclean,salimi2019interventional}, hypothetical reasoning \cite{galhotra2022causal}, and large system diagnostics~\cite{markakis2024sawmill,causalsim,sage, gudmundsdottir2017demonstration}. 


\revc{If-then rules are generally considered interpretable by humans~\cite{lakkaraju2016interpretable,guidotti2018local,van2021evaluating,pradhan2022interpretable,chen2018optimization}.
We give a concrete example of the difference between association and causation in generating prescriptions or recommended actions in the form of if-then rules below}:
\begin{example}	%
\label{example:ex1} {\bf Importance of causal prescriptions:}
Consider the Stack Overflow (SO) annual developer survey
\cite{stackoverflowreport}, where respondents from around the world answer
questions about their jobs and demographics. A sample of the dataset \reva{with a subset of the
attributes (there are 20 attributes)} is presented in \cref{tab:data}.
%
Alice, a researcher in the United Nations (UN) finance department, is interested in discovering ways to increase the salaries of high-tech employees worldwide. She is looking for a set of actionable recommendations 
%(that we call a prescription rules) 
to raise the overall average salary.
%
Using association-based approaches~\cite{chen2018optimization,lakkaraju2016interpretable}, she may discover that individuals residing in the US who identify as straight or heterosexual tend to earn higher salaries (see \cref{exp:quality} for full details). However, this observation merely indicates a correlation: people living in the US, for example, generally earn more than those outside the country. Their comparatively higher salaries are primarily attributable to the country's economy and are unrelated to their sexual orientation. Thus, this observation cannot be used as a prescription rule to increase salary. 
Our causal analysis, on the other hand, reveals that individuals aged 25-34 with dependents would benefit from working as front-end developers.
This results in a \$44,009 annual salary increase on average. \reva{Another observation is that students should pursue an
undergraduate major in CS. %Computer Science (CS). 
This can boost their salary by \$22,174 per year} (see details in \cref{sec:casestudy}).
\end{example}

%It has been incorporated into various tasks including . 
%Causal interventions are often more relatable and easier to understand, as they offer insight into the underlying reasons behind the recommendations and allow unraveling complex cause-effect relationships that govern our world~\cite{pearl2009causality}. Furthermore, causal interventions often have long-lasting effects~\cite{imbens2024causal}.

%, making it essential that the prescribed actions are not only actionable but also 

%causally consistent. 

%Decision makings, in particular, high-stak

\cut{
In this work, {we study the problem of generating causal insights (referred to as \emph{prescription rules}), which serve as actionable recommendations} to improve an outcome of interest.
Recent works have introduced causality to the field of database research~\cite{meliou2010causality,  meliou2014causality,salimi2020causal,10.14778/3554821.3554902}. It has been incorporated into various tasks including data discovery~\cite{galhotra2023metam,youngmann2023causal}, data cleaning~\cite{pirhadi2024otclean,salimi2019interventional}, and large system diagnostics~\cite{markakis2024sawmill,causalsim,sage, gudmundsdottir2017demonstration}. 
We propose using causal inference to generate prescription rules that are both actionable and justifiable.
}

While generating prescriptions based on causal inference may help in robust decision-making, causal prescriptions that solely consider the betterment of an outcome (like salary) are not enough in practice. 
It is well-known that decision-making in many high-stake applications (like hiring policy, or policy for approving loans by banks) may lead to disparate societal or economic impact on different sub-populations. 
As a shocking example from a recent work called 
%For example, 
CauSumX~\cite{DBLP:journals/pacmmod/YoungmannCGR24} that generates a set of causal explanations for an aggregated view, the explanations generated %by CauSumX %recommendations which 
suggest that male individuals do a Bachelor's degree to increase their salary while %suggesting that 
being an unmarried woman 
%the recommendation for women includes getting married 
has the most adverse effect on salary
(borrowed directly 
from Fig.~19 in~\cite{youngmann2024summarizedcausalexplanationsaggregate}). 
%We demonstrate the advantage of using causal reasoning to generate actionable recommendations and the limitations of not considering fairness requirements in the following example. 
We explored this further in the context of generating prescriptions and observed that prescriptions that are not fairness-aware can generate unfair outcomes to some subpopulations which we refer to as the {\em protected group}. Examples include women, Black, Latino, or Native Americans, individuals with a disability, countries with a weaker economy, or other protected groups specific to an application. %Here is a concrete example:


% Understanding the causal factors behind these recommendations is crucial to ensuring that decisions lead to fair and equitable outcomes, particularly in sensitive applications where biased decisions can perpetuate or even exacerbate societal inequalities.
% While prior work has extensively explored techniques for association rule mining~\cite{kumbhare2014overview}, recent efforts have focused on deriving causal explanations for individual data points or entire datasets~\cite{salimi2018bias,youngmann2022explaining,ma2023xinsight}. Although some of these methods produce causally consistent insights, the absence of fairness considerations in the process can lead to unfair outcomes, further reinforcing existing biases. For example, CauSumX~\cite{DBLP:journals/pacmmod/YoungmannCGR24} generates causal recommendation which suggest male individuals to do a Bachelor's degree to increase salary while the recommendation for women include getting married (borrowed directly from Figure~19 in the paper~\cite{youngmann2024summarizedcausalexplanationsaggregate}). 





%\emph{Causal inference} has been thoroughly studied in AI and Statistics~\cite{pearl2009causal,rubin2005causal}. Causal analysis is a vital tool in determining the effect of a \emph{treatment} on an \emph{outcome}, and has been used in decision-making in medicine \cite{robins2000marginal}, economics \cite{banerjee2011poor}, biology \cite{shipley2016cause}, and in high-stakes areas such as identifying the root causes of failures in critical infrastructure systems to prevent catastrophic outcomes. Recent works have introduced causality to the field of database research~\cite{meliou2010causality,  meliou2014causality,salimi2020causal,10.14778/3554821.3554902}. It has been incorporated into various tasks including data discovery~\cite{galhotra2023metam,youngmann2023causal}, query result explanation~\cite{salimi2018bias,youngmann2022explaining,DBLP:journals/pacmmod/YoungmannCGR24}, and large system diagnostics~\cite{markakis2024sawmill,causalsim,sage, gudmundsdottir2017demonstration}. We propose leveraging causal inference to generate interpretable and justifiable insights (referred to as \emph{prescription rules}), which serve as actionable recommendations to improve an outcome of interest. Causal reasoning is considered one of the most important requirements,  to generate insights that are actionable and aligned with human reasoning.




\begin{table*}[]
\footnotesize
    \centering
    	\caption{\textnormal{A subset of the Stack Overflow dataset.}}
         \label{tab:data}
    	% \vspace{-4mm}
  			\begin{tabular}[b]{|l|l|l|c|l|l|c|l|c|}
  			
				%\multicolumn{9}{c}{\textbf{Users}}\\ 
				\hline

				\textbf{ID}
    
    % \textbf{Country}& \textbf{Continent} 
    
    &\textbf{Gender} &\textbf{Ethnicity}&
				\textbf{Age} &\textbf{Role} &
				\textbf{Education} &\textbf{Country}&\textbf{Undergrad Major}&\textbf{Salary}
				\\ \hline

				1 &Male&White&26&Data Scientist & PhD& US&Computer Science&180k\\
    		2 &Non-binary&White&32&QA developer & Bachelor's degree& US&Mechanical Eng.&83k\\

 3 &Male&South Asian&29&C-suite executive  & Bachelor's degree & India&Computer Science&24k\\

  % 4 &Female&South Asian&25&Back-end developer  & Master's degree & India&Mathematics&7.5k\\

  4 &Female&East Asian&21&Back-end developer & Bachelor's degree & China&Computer Science&19k\\
  

        % $\ldots$ &  $\ldots$&  $\ldots$&  $\ldots$&  $\ldots$&  $\ldots$&  $\ldots$&  $\ldots$&  $\ldots$&  $\ldots$&  $\ldots$\\
    \hline
			\end{tabular}
            \vspace{-5mm}
\end{table*}




\begin{example}	%
\label{example:ex2}
{\bf Importance of fair prescriptions:}
Continuing Example~\ref{example:ex1}, while those causal prescription rules are highly beneficial for the overall population, they are considerably less effective for individuals residing in countries with a low GDP (indicating a weaker economy). For this group, the average expected increase in salary is only approximately \$13,000 per year (in contrast to \$44,009 for the entire group). % \sr{add which rule 44k or 25k} 
Consequently, implementing these rules would exacerbate the disparity between those living in countries with strong economies and those in countries with weaker economies.
\end{example}




% Our objective is to generate a small set of prescription rules aimed at increasing (or decreasing) an outcome of interest. This is framed as an optimization problem where the goal is to select the fewest prescription rules that maximize utility (i.e., the expected increase or decrease in the outcome). However, 

The example above shows that focusing solely on maximizing utility (\revc{i.e., increasing income}) can result in a scenario where only some of the population receive significant improvement, while others experience no benefit (\revc{only a small benefit for individuals from countries with weaker economies in our example}). Additionally, even if a large portion of the population receives recommendations, a protected subpopulation might not share the benefits and, worse, their situation could deteriorate, exacerbating inequalities.

Examples~\ref{example:ex1} and \ref{example:ex2} show that it is crucial to provide recommendations that are (1) {\em causal} for the outcome (beyond associations),  and (2) also {\em fair or equitable} in terms of the outcome for both the protected and non-protected groups. While recent work in database research
has focused on deriving {\em causal explanations} for individual data points, aggregated view, or entire datasets~\cite{salimi2018bias,youngmann2022explaining,ma2023xinsight, DBLP:journals/pacmmod/YoungmannCGR24}, and in particular \cite{DBLP:journals/pacmmod/YoungmannCGR24} has considered generating a set of causal explanations for an aggregated view that resemble a ruleset, 
%Although some of these methods produce causally consistent insights, 
the absence of fairness considerations in generating these causal explanations can lead to unfair outcomes for the protected group.
%further reinforcing existing biases.


%\red{We, therefore, enable users to incorporate various \emph{coverage and fairness constraints} along with the overall objective of improving an outcome of interest. }

\medskip
\noindent
\textbf{Our contributions.~} 
Motivated by the dual goals of generating causal and fair prescriptions for the betterment of an outcome, we introduce a {\em fairness-aware framework leveraging causal reasoning for generating a set of actionable prescription rules (ruleset)} called \sysName\ (\underline{Fair} \underline{CA}usal \underline{P}rescription).
%
Following research on fairness in data management~\cite{stoyanovich2020responsible,galhotra2022causal}, we assume the existence of a \emph{protected subpopulation}, defined by an attribute such as gender or race for people, or GDP of a country. Motivated by the causal explanation rules for an aggregated view \cite{DBLP:journals/pacmmod/YoungmannCGR24}, each prescription rule in our ruleset applies to a sub-population defined by a {\em grouping attribute}, and prescribes a {\em treatment or intervention} to improve the {\em outcome} for this sub-population. Fairness constraints ensure that the expected utility of the protected population is {\em comparable} to the utility of the unprotected individuals. We borrow the notions of \emph{group and individual fairness} from the fairness literature but tailor them for prescription rules. In addition to the fairness constraints, our coverage constraints ensure that a substantial fraction of the population and protected subpopulation receives at least one recommendation. 
%We demonstrate how such constraints ensure that the generated rules apply to a large portion of the population and ensure fairness through the following example.

\begin{example}
\label{ex:intro_example_3}
Continuing Examples~\ref{example:ex1} and \ref{example:ex2}, Alice uses our proposed system, called \sysName, to impose fairness and coverage constraints to discover useful and equitable recommendations for increasing salaries worldwide. In particular,
Alice chooses to implement a coverage constraint to ensure that the selected rules apply to a significant portion of people worldwide, including a sufficiently large number of individuals from countries with low GDP (the protected group). She also imposes a fairness constraint to ensure that the expected gains for both protected and non-protected groups are comparable.
\reva{She discovers, for example, that for individuals with 6-8 years of coding experience (a subpopulation comprising 21\% of the entire dataset and 25\% of the protected group), pursuing a bachelor’s degree in computer science will increase the expected salary by $\$14.9k$ for protected and by $\$17.8k$ for non-protected}. (See \cref{sec:casestudy} for more details.) This prescription rule applies to a large portion of the population and ensures fairness by providing a similar expected gain for both protected and non-protected groups, and the allowed difference of outcomes between these two populations may be adjusted by choosing appropriate thresholds in the fairness definitions. 
\end{example}


\noindent
Our main contributions are as follows. \\
%\begin{itemize}[leftmargin=*,topsep=0pt]
{\bf (1)} We {\bf develop a framework that generates a set of prescription rules to enhance an outcome of interest (Section~\ref{sec:problem})}. A prescription rule consists of a \emph{grouping pattern} and an \emph{intervention pattern}, representing the target subpopulation and the actionable recommendation for that group, respectively. The strength of the {\em conditional causal effect} (Section~\ref{sec:background-causal}) of this intervention on the subgroup is used to measure the expected utility of a rule. Our objective is to identify the smallest set of rules that maximizes overall expected utility. We refer to this problem as the {\em \probName} problem.
We adopt several notions of fairness (individual vs. group, statistical parity vs. bounded group loss) from the literature to define the {\bf fairness constraints} for our problem. In addition, {\bf coverage constraints} (for individual rules or for a group) ensure that the solution for the \probName\ problem is applied to a sufficient number of individuals and to minimize inequalities. We show NP-hardness for different variants of the problems and properties (matroid) useful in our algorithms. 
%We establish several definitions for group and individual fairness constraints tailored for prescription rules.
\smallskip
    \par
    \noindent
{\bf (2)} We {\bf develop a general three-step algorithm named \sysName to solve the optimization problem of selecting a fair prescription ruleset (Section~\ref{sec:algo})}. The first step involves mining frequent grouping patterns using the Apriori algorithm~\cite{agrawal1994fast}. In the second step, we employ a lattice-based algorithm to find high utility and fair intervention patterns for grouping patterns identified in the previous step. Finally, the third step applies a greedy approach to determine a solution. \sysName\ can be easily adapted to accommodate all variants of the \probName\ problem.

\smallskip
\par
\noindent
{\bf (3) We provide a detailed  case study  (Section~\ref{sec:casestudy}) and experimental analysis (Section~\ref{sec:experiments}) to evaluate our framework and algorithms.}
The case study shows the qualitative difference of different variants of our problem for different choices of the fairness and coverage constraints. The experiments include two datasets, three baselines, and 18 variations of our problem with different constraints. Our evaluations suggest that fairness may come at the cost of expected
utility for everyone. However, without fairness constraints, we often observe a significant disparity between the protected and non-protected. We also observe that
achieving individual fairness is harder than group fairness,
as most high-utility or high-coverage rules are unfair. Lastly, we show that \sysName\ can generate  prescription rules over large datasets in a reasonable time. 

%\end{itemize}


%\paragraph*{Paper outline} 
We discuss related work in \cref{sec:related}, review background on causal inference in \Cref{sec:background-causal}, %and our problem formulation can be found in \cref{sec:problem}. Our algorithmic framework is presented in \cref{sec:algo}. A case study demonstrating the impact of different constraint configurations on the solution is given in \cref{exp:problem_variants}, and our experimental evaluation is detailed in \cref{sec:experiments}. Finally, we 
and discuss the limitations of our framework and future work in \cref{sec:conc}.

% \noindent
% \boxed{\parbox{\columnwidth}{$\bullet$ 
% For people with a professional degree, move to the United Kingdom
%  (coverage = 435 (20), coverage-protected = 20 (13), utility = 186855, utility-protected = 0.)\\
% $\bullet$ For graphic developers, move to the	United States
%  (coverage = 116 (29), coverage-protected = 8 (2), utility = 169431, utility-protected = 0).\\
% $\bullet$ For people who have no formal education, move to the United States
%  (coverage = 123 (34), coverage-protected = 7 (2), utility = 206742, utility-protected = 0).\\
% % \textcolor{red}{size = 38, length = 76, overlap = 64029181, utility = 1659307}\\
% \textcolor{blue}{overall coverage =674, expected utility = 187485
% coverage-protected = 35, expected utility-protected = 0}
% \sr{should mention protected group, and possibly not mention coverage in the intro or just intuitively like high coverage}
% }}


% Alice notes that although these rules result in a \$187,485 increase in the overall salary for those to whom they apply, they only affect a small fraction of the population, specifically 674 individuals. Additionally, although the expected salary increase is substantial, there is no expected increase in salary for non-males, a subpopulation of particular interest to Alice. In other words, applying these rules would result in no gain for non-males.
% \end{example}

% \begin{example}[Episode 2 - coverage and fairness constraints]
% Alice introduces coverage and fairness constraints to ensure that enough people will benefit from the rules and that they will be \emph{fair} with respect to non-males. Specifically, she demands that the benefit for a randomly chosen individual to whom one of the rules applies is nearly the same as the benefit for a randomly chosen individual who identifies as non-male and to whom one of the rules applies.

% After adding these constraints, \sysName\ recommends the following set of prescription rules:



% \noindent
% \boxed{\parbox{\columnwidth}{$\bullet$ 
% For people who have no formal education, move to the United States
%  (coverage = 123 (34), coverage-protected = 7 (2), utility = 206742, utility-protected = 0)\\
% $\bullet$ 
% For females, change role to	DevOps specialist (coverage = 2256 (47), coverage-protected = 2256 (47), utility = 90023, utility-protected = 90023).\\
% $\bullet$ For people with a Master's degree, move to the	United States
%  (coverage = 9097 (2222), coverage-protected = 642 (236), utility = 85390, utility-protected = 84201).\\
% % \textcolor{red}{size = 38, length = 76, overlap = 64029181, utility = 1659307}\\
% \textcolor{blue}{overall coverage =11476	
% , expected utility = 87601,
% coverage-protected = 2905, expected utility-protected = 88519}
% }} 







% \begin{figure}[t]
%         \centering
%         \begin{minipage}[b]{1.0\linewidth}
%             \small
%             \begin{tcolorbox}[colback=white]
%             \vspace{-2mm}
% $\bullet$ For backend developers, the treatment with the highest effect on salary is “Country = US” effect size = 78646
% \begin{itemize}
%     \item For non-male the effect is only: 59429
%     \item For male the effect is 80454
% \end{itemize}

% $\bullet$ For frontend developers, the treatment with the highest effect is :Formal Education = Bachelor's degree” effect size: 17340
% \begin{itemize}
%     \item For white the effect is 33464
%     \item For non-white the effect is 15320
% \end{itemize}


% $\bullet$ For people in Europe, the treatment with the highest effect on salary is “DevType = C-suite executive” effect size = 53254
% \begin{itemize}
%     \item For white the effect is 55112
%     \item For non-white 35249
% \end{itemize}



%             \vspace{-2mm}
%             \end{tcolorbox}
%         \end{minipage}%%
%          % \vspace{-4mm}
%         \caption{Set of prescription rules.}
%         \label{fig:so-explanation}
%     \end{figure}

\section{Methodology}\label{sec:Method}
\section{Hierarchical Scene Synthesis with LLM}

Given a text description $t_r$ and the scene size $s_r$ as conditions, our approach is composed of three stages, as illustrated in Figure~\ref{pipeline}. First, we prompt the pre-trained LLM to generate the hierarchical structure with text descriptions. Second, we train a hierarchy-aware graph neural network to infer the relative placement coordinates between objects.
% the fine-grained relative placements between objects corresponding to the LLM-generated textual spatial relations. 
Third, we design a divide-and-conquer optimization which optimizes the sub-layout for each functional area and then arranges their placements to form the entire scene.  

% \vspace{-5pt}
\subsection{Hierarchical Structure Generation with LLM} 
Given the user requirement, the pre-trained LLM takes the constructed prompt as input and outputs structured text to describe the hierarchical scene representation, including the node attributes. The key challenge is to generate reasonable and informative spatial relations to specify the scene layout.

Although existing works define dense object relations to describe layouts, the more detailed the descriptions are, the more incorrect or self-contradictory results they make, due to the lack of spatial reasoning ability of the LLM. Therefore, in our approach, we require the LLM to generate a hierarchical structure to ground the objects and only the spatial relations between objects belonging to the same area, only to roughly specify their arrangements.

We construct the input prompt with three components: 1) a description of the LLM's role and task, including a brief definition of the hierarchical structure with the meaning of the nodes and connections; 2) a description of the preferred data format and pre-defined constraints, including the types of functional areas, possible anchor objects, and spatial relations; and 3) an example of a simple scene in the preferred format and the specific user requirements. We don't require the example to be selected corresponding to the user requirement, but only to demonstrate the output format. In this stage, the LLM generates textual descriptions and size attributes of the functional areas and objects, as well as the textual descriptions of spatial relations.

% \subsection{Data-Driven Inference of Relative Placements}
% \subsection{Hierarchy-Aware Network Inference}
\subsection{Fine-Grained Relative Placement Inference}
We propose a hierarchy-aware graph neural network to infer the fine-grained relative placements between correlated objects. The relative placements within each functional area exhibit a more compact and generalizable prior, allowing us to train a network to infer the placements for various scenes.

% As illustrated in Figure~\ref{pipeline}, given the LLM-generated hierarchy, we construct the input graph $G=(V,E)$ with the nodes as objects and edges connecting all the objects belonging to the same functional area. Although the input includes all the objects in the scene, the functional areas are isolated from each other. We use Linear embeddings for the object sizes $s_o$ and the relative placement coordinates $[p_e, \theta_e, d_e]$, where $d_e$ is a binary indicator of the alignment between two objects, and the pre-trained CLIP text encoder~\cite{radford2021learningtransferablevisualmodels} for descriptions of objects and spatial relations,
As illustrated in Figure~\ref{pipeline}, given the LLM-generated hierarchy, we construct the input graph $G=(O,E)$ with the nodes as objects and edges connecting all objects belonging to the same functional area. Although the input includes all objects in the scene, the functional areas are isolated from each other. We use Linear embeddings for the object sizes $s_o$ and the ground truth relative placement coordinates $[p_e, \theta_e, d_e]$, where $d_e$ is a binary indicator of the alignment between two objects, and the pre-trained CLIP text encoder~\cite{radford2021learningtransferablevisualmodels} for descriptions of objects $t_o$ and spatial relations $t_e$. They are organized as node features $h_o$ and edge features $h_e$, i.e.
%Note that the $[p_e, \theta_e, d_e]$ is only used during the training process. We incorporate this information into each node and edge of the $G$ to conceptualize it as the contextual graph $G_c$ with the node embedding $h_o$ and edge embedding $h_e$, 

\begin{equation}
\begin{aligned}
& h_o = [\mathrm{CLIP_t}(t_o), \mathrm{LINEAR(s_o)}], \\
& h_e = [\mathrm{CLIP_t}(t_e), \mathrm{LINEAR}(p_e, \theta_e, d_e)],
\end{aligned}
\end{equation}
which forms the contextual graph for the following network processing. Note that since we only have textual spatial relations between the anchor object and the others, we use all-zero vectors as the text embeddings for the edges without corresponding textual spatial relations (dotted arrows). 

We adopt the variational graph neural network~\cite{zhai2024commonscenes} for the contextual graph with the $h_o$ and $h_e$. Both the encoder and decoder are composed of several MLPs for 5 rounds of message passing, including $g_e^{(k)}$ for updating the edge features with connected node features in the $k$th round and $g_o^{(k)}$ for updating the node features with the 1-ring neighbor nodes, i.e. 
\begin{equation}
\begin{aligned}
h_{e_{i\xrightarrow{} j}}^{(k+1)}&=g_e^{(k)}(h_{o_i}^{(k)}, h_{e_{i\xrightarrow{} j}}^{(k)}, h_{o_j}^{(k)}) \\
h_{o_i}^{(k+1)} &=h_{o_i}^{(k)} + g_o^{(k)}(\mathrm{AVG}(h_{o_j}^{(k)}|o_j \in N_ \mathcal{G}(o_i))),
\end{aligned}
\end{equation}
where $e_{i\xrightarrow{} j}$ represents an edge connecting two objects $o_i$ and $o_j$, $N_\mathcal{G}(o_i)$ represents the set of neighbor nodes connected with object $o_i$. The encoder takes the contextual graph as input and outputs the graph with updated features, where the edge features (specifically the relative placement components of edge features, as shown in Figure~\ref{pipeline}) are parameterized as a Gaussian distribution. The decoder takes the updated graph as input and randomly samples from the Gaussian distribution. Finally, we use separate MLPs to decode the relative placement $[\hat{p}_e, \hat{\theta}_e, \hat{d}_e]$.

During training, we freeze the CLIP text encoder and update all other network layers. The loss function is 
\begin{equation}
L=L_{KL} + L_{ep} + L_{e\theta} + L_{ed},
\end{equation}
where $L_{KL}$ is the Kullback-Liebler divergence between the Gaussian distribution and
posterior distribution of the edge feature components. $L_{ep}$ is L1 loss on the relative positions $p_e$. $L_{e\theta}$ and $L_{ed}$ are cross-entropy loss on the discretized relative orientation angles and the binary alignment indicator.

\subsection{Divide-and-Conquer Layout Optimization} 

Given the hierarchical scene with the relative placements between correlated objects, we develop a divide-and-conquer optimization to solve for the final layout. Our solution includes a local optimization for each functional area and then a global optimization to organize the areas into scenes. This optimization produces reasonable and physically feasible layouts more effectively than a simple global optimization or iteratively optimizing each object's placements.

\begin{figure*}
\centering
\includegraphics[width=0.95\linewidth]{CameraReady/Figures/comparison.pdf}
\caption{The scenes generated from different approaches. The end-to-end data-driven approaches often produce infeasible object placements including overlap and out-of-boundary cases (red boxes). On the other hand, HOLODECK sometimes produces unreasonable results such as two nightstands on the same side of the bed (top row) or a tv stand on the left side of the sofa (second row). By contrast, our approach is able to more effectively produce reasonable and physically feasible scene layouts.}
\label{fig:comparison_topview}
\end{figure*}

\noindent \textbf{Local optimization.} For each functional area, we use local optimization to solve the object placements w.r.t. the bounding box of the functional area. The local optimization is formulated to minimize the objects' relative placements and those inferred by the network, with constraints to avoid object overlap and out-of-boundary, i.e.
\begin{equation}
\begin{aligned}
\min_{o'_i\in O} &  \sum_{o_i\in N_\mathcal{G}(o_a)} |\mathrm{REL}(o'_i, o'_a)-[p_{e_{i\xrightarrow{} a}},\theta_{e_{i\xrightarrow{} a}}]|, \quad \\
s.t. \quad & C_{overlap}(o'_i, o'_j), \quad \forall o_i,o_j \in A \\
& C_{OOB}(o'_i, s_a), \quad \forall o_i \in A 
\end{aligned}
\end{equation}
where $o'_i$ and $o'_a$ refers to the placements (center positions and orientations) w.r.t. the functional area of an object $o_i$ and the anchor object $o_a$, respectively. $\mathrm{REL}$ computes the relative placements between two objects and $[p_{e_{i\xrightarrow{} a}},\theta_{e_{i\xrightarrow{} a}}]$ is the relative positions between object $o_i$ and $o_a$ predicted by the network. $A$ represents the set of objects within the area. $C_{overlap}$ constrains the overlap between oriented bounding boxes of any two objects as small as possible, and $C_{OOB}$ aims to avoid the object boxes lying out of the area boundary, whose size $s_a$ is generated from pre-trained LLM in stage 1.


\noindent \textbf{Global optimization.} We then organize the areas to form scenes with global optimization. Each functional area takes the orientation of its anchor object as its own orientation. Based on observations in our daily life, the optimization is formulated to place the functional areas against the walls and far from each other with orientations pointing inside the scene, while avoiding object overlap and out-of-boundary:
\begin{equation}
\begin{aligned}
\min_{a_i} & \sum_{a_i} |\mathrm{D_w}(a_i, s_r)| - \sum_{a_i, a_j} |\mathrm{D_a}(a_i, a_j)|, \\
s.t. \quad & C_{overlap}(a_i, a_j), \quad \forall a_i,a_j \in S \\
& C_{OOB}(a_i, s_r), \quad \forall a_i \in S  
\end{aligned}
\end{equation}
where $a_i$ denotes area placement (center position and orientation). $D_w$ is the distance between back side of the area and the boundary of the scene. $D_a$ is the distance between two areas' bounding boxes. $S$ is the set of areas within the scene. $C_{overlap}$ and $C_{OOB}$ are same as in local optimization.

After the optimizations, we transform the coordinate systems to obtain object positions and orientations in the frame of scenes. Finally, we retrieve 3D object models from Objaverse~\cite{objaverseXL} and 3D-Front datasets~\cite{fu20213d} based on the CLIP scores, i.e. cosine similarity between object images and text embeddings. The object models are then scaled and placed according to the scene layouts.



\section{Experiments}\label{sec:Evaluation}
We investigate the following questions through our experiments:  

\begin{itemize}[itemsep=1pt, topsep=1pt]

\item[$\bullet$]Assess the robustness of IPAD (using various LLMs as generators, comparing with other detectors, and evaluating on out-of-distribution (OOD) datasets).  
\item[$\bullet$]Independently analyze the necessity and effectiveness of the \textbf{Prompt Inverter} and the \textbf{Distinguishers}.  
\item[$\bullet$]Explore the explainability of IPAD (through a user study and analysis of linguistic differences between prompts generated by HWT and LGT).
\end{itemize}

\subsection{Robustness of IPAD}
\subsubsection{Evaluation Baselines and Metrics}
The in-distribution experiments refer to the testing results presented in ~\cite{r3}, where the data aligns with the training data used for the IPAD ~\textbf{Distinguishers}, thereby serving as our baseline. The OOD experiments refer to the DetectRL baseline ~\cite{r58}, which is a comprehensive benchmark consisting of academic abstracts from the arXiv Archive (covering the years 2002 to 2017)\footnote{http://kaggle.com/datasets/spsayakpaul/arxiv-paper-abstracts/data}, news articles from the XSum dataset ~\cite{r59}, creative stories from Writing Prompts ~\cite{r60}, and social reviews from Yelp Reviews~\cite{r61}. It also employs three attack methods to simulate complex real-world detection scenarios, which includes the prompt attacks, paraphrase attacks, and perturbation attacks~\cite{r58}. All the testing sets have 1,000 samples in our experiments.

The ~\textbf{Area Under Receiver Operating Characteristic curve (AUROC)} is widely used for assessing detection method ~\cite{r55} because it considers the True Positive Rate (TPR) and False
Positive Rate (FPR) across different classification thresholds. Since our models predicts binary labels, we follow the ~\textit{Wilcoxon-Mann-Whitney} statistic~\cite{r56}, and the formula is shown in appendix ~\ref{sec:AUROC formula}. 
The ~\textbf{AvgRec} is the average of ~\textbf{HumanRec} and ~\textbf{MachineRec}. In our evaluation, ~\textbf{HumanRec} is the recall for detecting Human-written texts, and ~\textbf{MachineRec} is the recall for detecting LLM-generated texts~\cite{r57}.  The ~\textbf{F1 Score} provides a comprehensive evaluation of detector capabilities by balancing the model’s Precision and Recall. We use ~\textbf{AvgRec} and ~\textbf{F1} on in-distribution data, and we use ~\textbf{AUROC} for OOD data to align the test benchmarks for the same dataset.


\subsubsection{Robustness across different LLMs}
The results of IPAD for detecting the dataset OUTFOX~\cite{r3} across LLMs are presented in Table \ref{tab:performance_metrics_setting1} and Table \ref{tab:performance_metrics_setting2}, respectively. They show that both versions are highly robust across various LLMs, while ~\textit{Regeneration Comparator} is a bit more efficient. 

As for ~\textit{Regeneration Comparator}, when the original generator and re-generator are the same model, the performance is optimal. However, even when the re-generator is different from the original generator, the results remain impressive with ChatGPT used as the re-generator. These results imply that, in practical applications, it is possible to use a common set of LLMs as re-generators. If one or more correponding ~\textbf{Distinguishers} from different LLMs classify the results as 'yes', it can be inferred that the text is likely to be LGT, whereas if all ~\textbf{Distinguishers} classify the results as 'no', the text is more likely to be HWT. Furthermore, for applications aiming to save computational resources and improve efficiency, using ChatGPT as the sole re-generator still yields robust performance across all tested models.

\begin{table}[ht!]
  \centering
  \resizebox{0.5\textwidth}{!}{%
    \begin{tabular}{ccccc}
      \hline
      \multirow{2}{*}{\textbf{Original Generator}} &  \multicolumn{4}{c}{\textbf{Metrics (\%)}} \\
      \cline{2-5}
      & \textbf{HumanRec} & \textbf{MachineRec} & \textbf{AvgRec} & \textbf{F1} \\
      \hline
      ChatGPT     & 98.00\% & 99.80\%  & 98.90\% & 98.89\% \\
      \hline
      GPT-3.5     & 97.20\% & 99.90\%  & 98.55\% & 98.53\% \\
      \hline
      Qwen-turbo  & 98.00\% & 98.10\%  & 98.05\% & 98.05\% \\
      \hline
      Llama-3-70B & 98.00\% & 100.00\% & 99.00\% & 98.99\% \\
      \hline
    \end{tabular}%
  }
  \caption{IPAD with ~\textit{Prompt-Text Consistency Verifier} performance on different LLMs}
  \label{tab:performance_metrics_setting1}
\end{table}

\begin{table}[ht!]
  \centering
  \resizebox{0.5\textwidth}{!}{
    \begin{tabular}{cccccc}
      \hline
      \multirow{2}{*}{\textbf{Original Generator}} & \multirow{2}{*}{\textbf{Re-Generator}} & \multicolumn{4}{c}{\textbf{Metrics (\%)}} \\
      \cline{3-6}
      & & \textbf{HumanRec} & \textbf{MachineRec} & \textbf{AvgRec} & \textbf{F1} \\
      \hline
      ChatGPT & ChatGPT & 99.70\% & 100.00\% & \textbf{99.85\%} & \textbf{99.85\%} \\
      \hline
      GPT-3.5 & GPT-3.5 & 98.00\% & 100.00\% & ~\textbf{99.00\%} & ~\textbf{99.00\%} \\
      & ChatGPT & 97.00\% & 100.00\% & 98.50\% & 98.50\% \\
      \hline
      Qwen-turbo & Qwen-turbo & 98.00\% & 98.40\% & ~\textbf{98.20\%} & ~\textbf{98.20\%} \\
      & ChatGPT & 99.70\% & 94.40\% & 97.05\% & 97.13\% \\
      \hline
      Llama-3-70B & Llama-3-70B & 96.60\% & 100.00\% & 98.30\% & 98.30\% \\
      & ChatGPT & 99.70\% & 99.40\% & ~\textbf{99.55\%} & ~\textbf{99.55\%} \\
      \hline
    \end{tabular}
  }
  \caption{IPAD with ~\textit{Regeneration Comparator} performance on different LLMs}
  \label{tab:performance_metrics_setting2}
\end{table}


\subsubsection{Comparison of IPAD with other detectors in and out of distribution}
Table \ref{tab:performance_metrics_detection} compares the performance of two versions of IPAD with other detection methods in the OUTFOX dataset with and without attacks~\cite{r3}. The results show that both versions of IPAD generally outperform other detectors, while that IPAD with ~\textit{Prompt-Text Consistency Verifier} for detecting ChatGPT with DIPPER attack performs worse. These results imply that IPAD with ~\textit{Regeneration Comparator} demonstrates superior robustness compared to alternative detection methods in the OUTFOX dataset with and without attacks.

\begin{table}[ht!]
  \centering
  \resizebox{0.5\textwidth}{!}{
    \begin{tabular}{cccccc}
      \hline
      \multirow{2}{*}{\textbf{Original Generator}} & \multirow{2}{*}{\textbf{Detection Methods}} & \multicolumn{4}{c}{\textbf{Metrics (\%)}} \\
      \cline{3-6}
      & & \textbf{HumanRec} & \textbf{MachineRec} & \textbf{AvgRec} & \textbf{F1} \\
      \hline
      ChatGPT & RoBERTa-base & 93.80\% & 92.20\% & 93.00\% & 92.90\% \\
      & RoBERTa-large & 91.60\% & 90.00\% & 90.80\% & 90.70\% \\
      & HC3 detector & 79.20\% & 70.60\% & 74.90\% & 73.80\% \\
      & OUTFOX & 97.80\% & 92.40\% & 95.10\% & 95.00\% \\
      & IPAD version1 & 98.00\% & 99.80\% & 98.90\% & 98.89\%\\
      & IPAD version2& 99.70\% & 100.00\% & \textbf{99.85\%} & \textbf{99.85\%} \\
      \hline
      GPT-3.5 & RoBERTa-base & 93.80\% & 92.00\% & 92.90\% & 92.80\% \\
      & RoBERTa-large & 92.60\% & 92.00\% & 92.30\% & 92.30\% \\
      & HC3 detector & 79.20\% & 85.00\% & 82.10\% & 82.60\% \\
      & OUTFOX & 97.60\% & 96.20\% & 96.90\% & 96.90\% \\
      & IPAD version1 & 97.20\% & 99.90\% & \textbf{98.55\%} & \textbf{98.53\%}\\
      & IPAD version2& 97.00\% & 100.00\% & 98.50\% & 98.50\% \\
      \hline
      ChatGPT with DIPPER Attack & RoBERTa-base & 93.80\% & 89.20\% & 91.50\% & 91.30\% \\
      & RoBERTa-large & 91.60\% & 97.00\% & 94.30\% & 94.40\% \\
      & HC3 detector & 79.20\% & 3.40\% & 41.30\% & 5.50\% \\
      & OUTFOX & 98.60\% & 66.20\% & 82.40\% & 79.00\% \\
      & IPAD version1 & 98.00\% & 75.10\% & 86.55\% & 87.93\%\\
      & IPAD version2& 99.70\% & 95.40\% & \textbf{97.55\%} & \textbf{97.60\%} \\
      \hline
      ChatGPT with OUTFOX Attack & RoBERTa-base & 93.80\% & 69.20\% & 81.50\% & 78.90\% \\
      & RoBERTa-large & 91.60\% & 56.20\% & 73.90\% & 68.30\% \\
      & HC3 detector & 79.20\% & 0.40\% & 39.80\% & 0.70\% \\
      & OUTFOX & 98.80\% & 24.80\% & 61.80\% & 39.40\% \\
      & IPAD version1 & 98.00\% & 95.40\% & 96.70\% & 96.74\%\\
      & IPAD version2& 99.70\% & 98.00\% & \textbf{98.85\%} & \textbf{98.86\%} \\
      \hline
    \end{tabular}
    }
  \caption{Comparison of IPAD with other detectors on in-distribution data, where ~\textbf{IPAD version1} stands for ~\textbf{IPAD with ~\textit{Prompt-Text Consistency Verifier}} and ~\textbf{IPAD version2} stands for ~\textbf{IPAD with ~\textit{Regeneration Comparator}}}
  \label{tab:performance_metrics_detection}
\end{table}

Table \ref{tab:OOD_performance} presents the performance of various detection methods on OOD datasets to assess their generalizability, where the baseline data refer to DetectRL ~\cite{r58}.  The results demonstrate that IPAD with ~\textit{Regeneration Comparator} consistently outperforms all other baselines in all OOD datasets with and without attacks. In contrast, IPAD with ~\textit{Prompt-Text Consistency Verifier} exhibits strong performance on OOD datasets without attacks but shows a noticeable drop in effectiveness when subjected to attacks. For instance, while it achieves competitive results on datasets like XSum (99.90\%) and Writing (99.20\%), its performance against attacks, such as Prompt Attack (86.90\%) and Paraphrase Attack (82.72\%), is significantly lower than IPAD with ~\textit{Regeneration Comparator}. This suggests that \textbf{IPAD with ~\textit{Regeneration Comparator} demonstrates better generalizability and robustness.}

\begin{table}[h]
    \centering
    \resizebox{0.5\textwidth}{!}{ 
    \begin{tabular}{cccccccc}
        \toprule
        \hline
        \multirow{2}{*}{\textbf{OOD Datasets or attack type}} & \multicolumn{5}{c}{\textbf{Detection Methods}} \\
        \cmidrule(lr){2-6}
         & \textbf{LRR} & \textbf{Fast-DetectGPT} & \textbf{Rob-Base} & IPAD with version1 & IPAD version2 \\
        \midrule
        \hline
        Arxiv & 48.17\% & 42.00\% & 81.06\% & 84.47\% & \textbf{98.60\%} \\
        XSum & 48.41\% & 45.72\% & 76.81\% & \textbf{99.90\%} & 98.90\% \\
        Writing & 58.70\% & 51.13\% & 86.29\%& \textbf{99.20\%} & 95.80\% \\
        Review & 58.21\% & 54.55\% & 87.84\% &98.50\% & \textbf{89.30\%} \\
        \hline
        Avg. for non-attacked datasets & 53.37\% & 48.35\% & 83.00\% &95.52\% & \textbf{95.65\%} \\
        \hline
        Prompt Attack & 54.97\% & 43.89\% & 92.81\%& 86.90\% & \textbf{93.05\%}\\
        Paraphrase Attack & 49.23\% & 41.15\% & 90.02\%&82.72\% & \textbf{95.89\%}\\
        Perturbation Attack & 53.62\% & 44.38\% & 92.12\% & 94.96\% & \textbf{95.32\%} \\
        \hline
        Avg. for attacked datasets & 52.61\% & 43.14\% & 91.65\% & 88.26\% & \textbf{94.75\%}\\
        \hline
        Avg. & 53.04\% & 46.12\% & 86.70\%&92.41\%&\textbf{95.26\%}\\
        \hline
        \bottomrule
    \end{tabular}
    }
    \caption{The performance of IPAD in generalization assessment (AUROC). The selected detectors are evaluated on OOD data, all sourced from and processed using the DetectRL baseline, where ~\textbf{IPAD version1} stands for ~\textbf{IPAD with ~\textit{Prompt-Text Consistency Verifier}} and ~\textbf{IPAD version2} stands for ~\textbf{IPAD with ~\textit{Regeneration Comparator}.}}
    \label{tab:OOD_performance}
\end{table}

\vspace{-0.3cm}

\subsubsection{Robustness conclusion}

Our experimental results demonstrate that both IPAD versions exhibit strong performance across different LLMs, outperforming existing detection methods and maintaining robustness on OOD datasets. The IPAD with ~\textit{Regeneration Comparator} outperforming baselines by 9.73\% (F1-score) on in-distribution data and 12.65\% (AUROC) OOD data. Notably, IPAD with ~\textit{Regeneration Comparator} achieves significantly better performance than IPAD with ~\textit{Prompt-Text Consistency Verifier} in attack scenarios of 3.78\% (F1-score). While IPAD with ~\textit{Prompt-Text Consistency Verifier} performs robustly in standard settings, its performance declines when facing attacks. The calculation of these statistics are shown in Appendix ~\ref{Calculation}.
%

\vspace{-0.3cm}
\subsection{Necessity and Effectiveness of \textbf{Prompt Inverter} and \textbf{Distinguishers}}

\subsubsection{Necissity of the \textbf{Prompt Inverter} and \textbf{Distinguishers}}
To prove that it is necessary to fine-tune on IPAD with IPAD with ~\textit{Prompt-Text Consistency Verifier} and ~\textit{Regeneration Comparator}, we conducted ablation study to use the same finetune method on only ~\textit{input texts} and only ~\textit{predicted prompts}. The instructions are ~\textit{"Is this text generated by LLM?"}, and ~\textit{"Prompt Inverter predicts prompt that could have generated the input texts. Is this prompt predicted by an input texts written by LLM?"}, respectively.

The results shown in Figure ~\ref{fig:ablation} from the ablation study show that fine-tuning on either only the ~\textit{input text} or only the ~\textit{predicted prompt} leads to poor performance. This underscores the importance of fine-tuning on a combination of both the input text and predicted prompt, as explored in the ~\textit{Prompt-Text Consistency Verifier}, or on the input text and regenerated text, as examined in the ~\textit{Regeneration Comparator}, for more effective detection.
\begin{figure}[t]
  \centering
  \includegraphics[width=0.5\textwidth]{ablation.png}
  \caption{Ablation Study Results. The ~\textbf{IPAD version1} stands for ~\textbf{IPAD with ~\textit{Prompt-Text Consistency Verifier}} and ~\textbf{IPAD version2} stands for ~\textbf{IPAD with ~\textit{Regeneration Comparator}.}}
  \label{fig:ablation}
\end{figure}
\vspace{-0.3cm}
% \begin{table}[ht!]
%   \centering
%   \resizebox{0.5\textwidth}{!}{
%     \begin{tabular}{cccccc}
%       \hline
%       \multirow{}{}{\textbf{Original Generator}} & \multirow{}{}{\textbf{Detection Methods}} & \multicolumn{4}{c}{\textbf{Metrics (\%)}} \\
%       \cline{3-6}
%       & & \textbf{HumanRec} & \textbf{MachineRec} & \textbf{AvgRec} & \textbf{F1} \\
%       \hline
%       ChatGPT & Finetune with only Input & 10.80\% & 12.20\% & 11.00\% & 92.90\% \\
%       & Finetune with only Prompt & 91.60\% & 90.00\% & 90.80\% & 90.70\% \\
%       & IPAD Setting1 & 98.00\% & 99.80\% & 98.90\% & 98.89\%\\
%       & IPAD Setting2& 99.70\% & 100.00\% & \textbf{99.85\%} & \textbf{99.85\%} \\
%       \hline
%       GPT-3.5 & Finetune with only Input & 00\% & 00\% & 00\% & 00\% \\
%       & Finetune with only Prompt & 91.60\% & 90.00\% & 90.80\% & 90.70\% \\
%       & IPAD Setting1 & 98.00\% & 99.80\% & 98.90\% & 98.89\%\\
%       & IPAD Setting2& 99.70\% & 100.00\% & \textbf{99.85\%} & \textbf{99.85\%} \\
%       \hline
%       Qwen & 93.80\% & 92.20\% & 93.00\% & 92.90\% \\
%       & Finetune with only Prompt & 91.60\% & 90.00\% & 90.80\% & 90.70\% \\
%       & IPAD Setting1 & 98.00\% & 99.80\% & 98.90\% & 98.89\%\\
%       & IPAD Setting2& 99.70\% & 100.00\% & \textbf{99.85\%} & \textbf{99.85\%} \\
%       \hline
%       LLAMA & Finetune with only Input & 93.80\% & 92.20\% & 93.00\% & 92.90\% \\
%       & Finetune with only Prompt & 91.60\% & 90.00\% & 90.80\% & 90.70\% \\
%       & IPAD Setting1 & 98.00\% & 99.80\% & 98.90\% & 98.89\%\\
%       & IPAD Setting2& 99.70\% & 100.00\% & \textbf{99.85\%} & \textbf{99.85\%} \\
%       \hline
%     \end{tabular}
%     }
%   \caption{Ablation study (DATA NOT COMPLETED)}
%   \label{tab:performance_metrics_detection}
% \end{table}

\subsubsection{The effectivenss of the IPAD \textbf{Prompt Inverter}}

We use DPIC~\cite{r62} and PE~\cite{r65} as baseline methods for prompt extraction. DPIC employs a zero-shot approach using the prompt states in Appendix ~\ref{sec:DPIC prompt}, while PE uses adversarial attacks to recover system prompts.

In our evaluation, we tested 1000 LGT and 1000 HWT samples. We use only in-distribution data for testing since only these datasets include original prompts. The metrics are all tested on comparing the similarity of the original prompts and the predicted prompts. The results shown in Table ~\ref{tab:model_comparison} illustrate that IPAD consistently outperforms both DPIC and PE across all four metrics (BartScore~\cite{r64}, Sentence-Bert Cosine Similarity~\cite{r63}, BLEU~\cite{r66}, and ROUGE-1~\cite{r67}), which highlight the effectiveness of the IPAD ~\textbf{Prompt Inverter}.

\begin{table}[htbp]
\centering
\resizebox{0.5\textwidth}{!}{
\begin{tabular}{l|c|c|c|c}
\hline
\textbf{Evaluation} & \textbf{Bart-large-cnn} & \textbf{Sentence-Bert} & \textbf{BLEU} & \textbf{ROUGE-1} \\
\hline
\multicolumn{5}{c}{\textbf{LGT}} \\
\hline
DPIC & -2.12 & 0.46 & 5.61E-05 & 0.04 \\
PE & -2.23 & 0.58 & 3.21E-04 & 0.25 \\
IPAD & ~\textbf{-1.84} & ~\textbf{0.69} & ~\textbf{0.24} & ~\textbf{0.51} \\
\hline
\multicolumn{5}{c}{\textbf{HWT}} \\
\hline
DPIC & -2.47 & 0.42 & 8.75E-06 & 0.06 \\
PE & -2.39 & 0.53 & 2.56E-08 & 0.13 \\
IPAD & ~\textbf{-2.22} & ~\textbf{0.57} & ~\textbf{1.30E-01} & ~\textbf{0.39} \\
\hline
\end{tabular}
}
\caption{Comparison of the IPAD \textbf{Prompt Inverter} with other prompt extractors}
\label{tab:model_comparison}
\end{table}
\vspace{-0.3cm}


\subsubsection{The Effectiveness of the IPAD Distinguishers}

To examine the effectiveness of the IPAD ~\textbf{Distinguishers}, we conducted a comparison study using the same dataset but different distinguishing methods. The first and second methods employed Sentence-Bert ~\cite{r63} and Bart-large-cnn ~\cite{r64} to compute the similarity score between the input texts and the regenerated texts. We selected thresholds that maximized AvgRec, which were 0.67 for Sentence-Bert and -2.52 for Bart-large-cnn. The classification rule is that the texts with scores greater than the threshold will be classified as LGT, while the texts with scores less than or equal to the threshold will be classified as HWT.

The third and fourth methods involved directly prompting ChatGPT as follows: 

\textbf{Instruction:} ~\textit{"Text 1 is generated by an LLM. Determine whether Text 2 is also generated by an LLM with a similar prompt. Answer with only YES or NO."}  ~\textbf{Input: }~\textit{"Text 1: \{Regenerated Text\}; Text 2: \{LGT\} or \{HWT\}"}. 

and ~\textbf{Instruction:} ~\textit{"Can LLM generate text2 through the prompt text1? Answer with only YES or NO."} with ~\textbf{Input:} ~\textit{"Text 1: \{Predicted Prompt\}; Text 2: \{Input text\}"}.

The final results demonstrated that the other distinguishing methods performed worse than the two IPAD ~\textbf{Distinguishers}, highlighting the superior effectiveness of the IPAD ~\textbf{Distinguishers}.


\begin{table}[ht]
\centering
\resizebox{0.5\textwidth}{!}{
\begin{tabular}{l|cccc}
\hline
\textbf{Distinguish Method} & \textbf{HumanRec} & \textbf{MachineRec} & \textbf{AvgRec} & \textbf{F1} \\
\hline
Sentence-Bert (Threshold 0.67) & 61.20\% & 95.20\% & 78.20\% & 63.51\% \\
Bart-large-cnn (Threshold -2.52) & 42.60\% & 97.20\% & 69.90\% & 43.96\% \\
Prompt to ChatGPT version 1 & 33.20\% & 64.50\% & 48.85\% & 44.77\% \\
Prompt to ChatGPT version 2 & 12.50\% & 100\% & 56.25\% & 12.50\% \\
IPAD version 1 & 98.00\% & 99.80\% & 98.90\% & 98.10\% \\
IPAD version 2 & ~\textbf{99.70\%} & ~\textbf{100\%} & ~\textbf{99.85\%} & ~\textbf{99.70\%} \\
\hline
\end{tabular}
}
\caption{Comparison of Different Distinguishers, where ~\textbf{IPAD version1} stands for ~\textbf{IPAD with ~\textit{Prompt-Text Consistency Verifier}} and ~\textbf{IPAD version2} stands for ~\textbf{IPAD with ~\textit{Regeneration Comparator}.}}
\label{tab:distinguishers_comparison}
\end{table}
\vspace{-0.3cm}





\subsection{Explanability Assessment of IPAD}
\subsubsection{Different Linguistic Features of HWT prompts and LGT prompts}

This subsection of the evaluation aims to explore the linguistic features of prompts generated by HWT and LGT through the \textbf{Prompt Inverter}. We analyzed 1000 samples generated by HWT and 1000 samples generated by LGT, which are randomy selected from both in-distribution data and OOD.

The analysis is first conducted using the Linguistic Feature Toolkik (lftk)\footnote{https://lftk.readthedocs.io/en/latest/}, a commonly used general-purpose tool for linguistic features extraction, which provides a total of 220 features for text analysis. Upon applying this toolkit, we identified 20 features with significant differences in average values between the two groups, out of which 3 features showed statistically significant differences with p-values less than 0.05. These 3 differences can be summarized as one main aspects: ~\textbf{syntactic complexity}. Beyond these, we referred to the LIWC framework \footnote{https://www.liwc.app/}, which defines 7 function words variables and 4 summary variables. By comparing the difference, two of these 11 features is significantly distinguishable: ~\textbf{the pronoun usage} and ~\textbf{the level of analytical thinking}.

% One of the primary distinctions between the HWT prompts and the LGT prompts is the \textbf{conceptual scope}. The analysis reveals that LGT prompts tend to generate more generalized concepts, while HWT prompts tend to provide more specific and detailed descriptions. Linguistic features show that HWT prompts have significantly more \textbf{geographical entities} (mean value of 0.127 and 0.113), \textbf{organizational entities} (mean value of 0.143 and 0.154), and \textbf{cardinal entitites} (mean value of 0.03 and 0.105) per sentence. These entities, however, are often indirectly related to the core meaning of the prompt, which serve more as supplements rather than integral components of the main topic. For example, as shown in Figure~\ref{fig:concept scope}, HWT prompts would include specific geographical names as examples when describing \textit{car-free zone issue}, detailed organizational sources when stating \textit{"Face on Mars" problem}, and specific reasons when talking about \textit{student sport activities}.

One of the primary distinctions between the HWT prompts and the LGT prompts is \textbf{sentence complexity}. LGT prompts are typically more complex, characterized by \textbf{longer sentence lengths} (mean value of 1.514 and 1.794), \textbf{higher syllable counts} (mean values of total syllabus three are 1.572 and 3.042), and \textbf{more stop-words} (mean values of 9.88 and 10.045). HWT prompts, on the other hand, are characterized by shorter, less complex sentences that are easier to process and understand, as examples shown in Appendix~\ref{sec:Linguistic Difference Examples} Figure~\ref{fig:sentence complexity}.


Beyond the differences in \textbf{syntactic complexity}, we also explored variables in LIWC. We did the difference comparison by using HWT and LGT prompts as inputs for ChatGPT, for example, instructing with the prompts \textit{'determine the pronoun usage of this sentence, answer first person, second person, or third person'} and \textit{'determine the level of analytical thinking of these sentences, answer a number from 1 to 5'}. The results show that there are distinguish difference in pronoun usage and analytical thinking level. The HWT prompts frequently use \textbf{second-person pronouns} (e.g., 'you') - 75 occurrences per 1,000 prompts - due to the subjective tone often employed in HWT. In contrast, LGT prompts primarily feature first- and third-person pronouns, with second-person pronouns appearing only 2 per 1,000 prompts. LGT prompts typically present instructions and questions in a more objective manner. As shown in Appendix ~\ref{sec:Linguistic Difference Examples} Figure ~\ref{fig:comparison}, LGT prompts show higher \textbf{analytical thinking levels} than HWT prompts. With level 1 as the lowest and level 5 as the highest, LGT has 68.9\% of level 4 and 24.3\% of level 5, but HWT has only 48.0\% of level 4, and 0.8\% of level 5. It suggests that LGT prompts encourage more analytical thinking, while HWT prompts tend to focus more on concrete examples, with less emphasis on critical analysis, as examples shown in Appendix~\ref{sec:Linguistic Difference Examples} Figure ~\ref{fig:person}.


\subsection{User Study}
To assess the explainability improvement of IPAD, we designed an IRB-approved user study with ten participants evaluating one HWT and one LGT article. We used IPAD version 2 due to its superior OOD performance and attack resistance. Participants compared three online detection platforms with screenshots shown in Appendix~\ref{User study}\footnote{https://www.scribbr.com/ai-detector/}\footnote{https://quillbot.com/ai-content-detector}\footnote{https://app.gptzero.me/} with IPAD's process (which displayed input texts, predicted prompts, regenerated texts, and final judgments). After evaluation, users rated IPAD on four key explainability dimensions. Transparency received strong ratings (40\%:5, 60\%:4), with users appreciating the visibility of intermediate processes. Trust scores were more varied (10\%:3, 70\%:4, 20\%:5), but IPAD was generally considered more convincing than single-score detectors. Satisfaction was mixed (30\%:3, 30\%:4, 40\%:5), with users acknowledging better detection but raising concerns about energy efficiency since IPAD runs three LLMs. Debugging received unanimous 5s, as users could easily analyze the predicted prompt and regenerated text to verify the decision-making process. If needed, users could refine the generated content by adjusting instructions, such as specifying a word count, making IPAD a more effective and user-friendly tool compared to black-box detectors.


%
\section{Related Work}\label{sec:Related Work}
\section{Rethinking Sparse Attention Methods}
\label{sec:critique}

Modern sparse attention methods have made significant strides in reducing the theoretical computational complexity of transformer models. However, most approaches predominantly apply sparsity during inference while retaining a pretrained Full Attention backbone, potentially introducing architectural bias that limits their ability to fully exploit sparse attention's advantages. Before introducing our native sparse architecture, we systematically analyze these limitations through two critical lenses.


\begin{figure*}[t] 
\centering 
\includegraphics[width=1\textwidth]{figures/fig2.pdf} 
\caption{Overview of \method{}'s architecture. Left: The framework processes input sequences through three parallel attention branches: For a given query, preceding keys and values are processed into compressed attention for coarse-grained patterns, selected attention for important token blocks, and sliding attention for local context. Right: Visualization of different attention patterns produced by each branch. Green areas indicate regions where attention scores need to be computed, while white areas represent regions that can be skipped.}
\label{fig:framework}
\end{figure*}


\subsection{The Illusion of Efficient Inference}

Despite achieving sparsity in attention computation, many methods fail to achieve corresponding reductions in inference latency, primarily due to two challenges:

\textbf{Phase-Restricted Sparsity.}
Methods such as H2O \citep{h2o} apply sparsity during autoregressive decoding while requiring computationally intensive pre-processing (e.g. attention map calculation, index building) during prefilling. In contrast, approaches like MInference \citep{minference} focus solely on prefilling sparsity. 
These methods fail to achieve acceleration across all inference stages, as at least one phase remains computational costs comparable to Full Attention.
The phase specialization reduces the speedup ability of these methods in prefilling-dominated workloads like book summarization and code completion, or decoding-dominated workloads like long chain-of-thought~\citep{cot} reasoning.

\textbf{Incompatibility with Advanced Attention Architecture.}
Some sparse attention methods fail to adapt to modern decoding efficient architectures like Mulitiple-Query Attention~(MQA) \citep{mqa} and Grouped-Query Attention~(GQA) \citep{gqa}, which significantly reduced the memory access bottleneck during decoding by sharing KV across multiple query heads. For instance, in approaches like Quest \citep{quest}, each attention head independently selects its KV-cache subset. Although it demonstrates consistent computation sparsity and memory access sparsity in Multi-Head Attention (MHA) models, it presents a different scenario in models based on architectures like GQA, where the memory access volume of KV-cache corresponds to the union of selections from all query heads within the same GQA group. This architectural characteristic means that while these methods can reduce computation operations, the required KV-cache memory access remains relatively high.
This limitation forces a critical choice: while some sparse attention methods reduce computation, their scattered memory access pattern conflicts with efficient memory access design from advanced architectures.

These limitations arise because many existing sparse attention methods focus on KV-cache reduction or theoretical computation reduction, but struggle to achieve significant latency reduction in advanced frameworks or backends.
This motivates us to develop algorithms that combine both advanced architectural and hardware-efficient implementation to fully leverage sparsity for improving model efficiency.


\subsection{The Myth of Trainable Sparsity}
Our pursuit of native trainable sparse attention is motivated by two key insights from analyzing inference-only approaches:
(1) \textbf{\textit{Performance Degradation}}: Applying sparsity post-hoc forces models to deviate from their pretrained optimization trajectory. As demonstrated by \citet{magicpig}, top 20\% attention can only cover 70\% of the total attention scores, rendering structures like retrieval heads in pretrained models vulnerable to pruning during inference.
(2)~\textbf{\textit{Training Efficiency Demands}}: 
Efficient handling of  long-sequence training is crucial for modern LLM development. This includes both pretraining on longer documents to enhance model capacity, and subsequent adaptation phases such as long-context fine-tuning and reinforcement learning. However, existing sparse attention methods primarily target inference, leaving the computational challenges in training largely unaddressed. This limitation hinders the development of more capable long-context models through efficient training. Additionally, efforts to adapt existing sparse attention for training also expose challenges:



\textbf{Non-Trainable Components.} Discrete operations in methods like ClusterKV~\citep{clusterkv} 
(includes k-means clustering) and MagicPIG~\citep{magicpig} (includes SimHash-based selecting) create discontinuities in the computational graph. These non-trainable components prevent gradient flow through the token selection process, limiting the model's ability to learn optimal sparse patterns. 

\textbf{Inefficient Back-propagation.} Some theoretically trainable sparse attention methods suffer from practical training inefficiencies. Token-granular selection strategy used in approaches like HashAttention~\citep{desai2024hashattention} leads to the need to load a large number of individual tokens from the KV cache during attention computation. 
This non-contiguous memory access prevents efficient adaptation of fast attention techniques like FlashAttention, which rely on contiguous memory access and blockwise computation to achieve high throughput.
As a result, implementations are forced to fall back to low hardware utilization, significantly degrading training efficiency.



\subsection{Native Sparsity as an Imperative}

These limitations in inference efficiency and training viability motivate our fundamental redesign of sparse attention mechanisms.
We propose \method{}, a natively sparse attention framework that addresses both computational efficiency and training requirements.
In the following sections, we detail the algorithmic design and operator implementation of \method{}.

\section{Conclusion}\label{sec:Conclusion}
This paper introduces ~\textbf{IPAD (Inverse Prompt for AI Detection)}, a framework consisting of a ~\textbf{Prompt Inverter} that identifies predicted prompts that could have generated the input text, and a ~\textbf{Distinguisher} that examines how well the input texts align with the predicted prompts. This design enables explainable evidence chains tracing unavailable in existing black-box detectors. Empirical results show that IPAD surpasses the baselines on all in-distribution, OOD, and attacked data. Furthermore, the ~\textbf{Distinguisher} (version2) - ~\textit{Regeneration Comparator} outperforms the ~\textbf{Distinguisher} (version1) - ~\textit{Prompt-Text Consistency Verifier}, especially on OOD and attacked data. While the local alignment in veresion1 approach provides explicit interpretability, it is more sensitive to adversarial attacks. In contrast, the global distribution in veresion2 matching approach implicitly learns generative LLM's distributional properties, which offers more robustness while maintaining explainability. This insight suggests that combining self-consistency checks of generative models with multi-step reasoning for evidential explainability holds promise for future AI detection systems in real-world scenarios. A user study reveals that IPAD enhances trust and transparency by allowing users to examine decision-making evidence. Overall, IPAD establishes a new paradigm for more robust, reliable, and interpretable AI detection systems to combat the misuse of LLMs.
\section{Limitations}\label{sec:Limitation}
While IPAD demonstrates SOTA performance, two limitations warrant discussion:
%
(1) The \textbf{Prompt Inverter} may not fully reconstruct prompts containing explicit in-context learning examples (e.g., formatted demonstrations), as it prioritizes semantic alignment over precise syntactic replication.
%
(2) Since IPAD achieves satisfactory OOD performance (12.65\% improvement over baselines) by only adopting essay writing datasets for the fine-tuning of \textbf{Distinguishers}, we strategically deferred the exploration of more datasets. 
%
We will incorporate a wider and more diverse range of data in future works to explore if it can enhance robustness even further, including: creative/news domains, and triplet data formats (i.e., ~\textit{"Can this \{predicted prompt\} generate the \{Input text\} using an LLM? One example generated by the predicted prompt is: \{regenerated text\}"})

% The \textbf{Prompt Inverter} may not always extract the exact prompt, especially when the prompt includes examples for in-context learning, as it might fail to capture these elements precisely. In the fine-tuning of the \textbf{Distinguishers}, we did not incorporate a wider variety of datasets, such as creative prompt datasets or news datasets, nor did we experiment with the triplet data format (~\textit{”Can this \{predicted prompt\} generate the \{Input text\} using an LLM? One example generated by the predicted prompt is: \{regenerated text\}}). This decision was based on the satisfactory performance of IPAD on out-of-distribution (OOD) data. However, we acknowledge that utilizing a broader and more diverse range of data could potentially improve the performance of IPAD.



\section*{Acknowledgments}

% Bibliography entries for the entire Anthology, followed by custom entries
%\bibliography{anthology,custom}
% Custom bibliography entries only
\bibliography{custom}

\appendix
\section{AUROC formula}
\label{sec:AUROC formula}
Since our model predicts binary labels, we follow the ~\textit{Wilcoxon-Mann-Whitney} statistic~\cite{r56} to calculate the Area Under Receiver Operating Characteristtic curve (AUROC):

\[
\text{AUC}(f) = \frac{\sum_{t_0 \in \mathcal{D}^0} \sum_{t_1 \in \mathcal{D}^1} \mathbf{1}[f(t_0) < f(t_1)]}{|\mathcal{D}^0| \cdot |\mathcal{D}^1|}
\]

where \( \mathbf{1}[f(t_0) < f(t_1)] \) denotes an indicator function which returns 1 if \( f(t_0) < f(t_1) \) and 0 otherwise. \( \mathcal{D}^0 \) is the set of negative examples, and \( \mathcal{D}^1 \) is the set of positive examples.

\section{Calculation of Summary Statistics}
\label{Calculation}
\begin{itemize}

\item[$\bullet$] IPAD with ~\textit{Regeneration Comparator} outperforms the baselines by 9.73\% on in-distribution data. As shown in Table~\ref{tab:performance_metrics_detection}, RoBERTa-base has the best average F1 score of (92.9\% + 92.8\% + 91.3\% + 78.9\%) / 4. In comparison, the average F1 score for IPAD version 2 is (99.85\% + 98.5\% + 97.6\% + 98.86\%) / 4, showing an improvement of 9.73\%.

\item[$\bullet$] IPAD with ~\textit{Regeneration Comparator} outperforms the baselines by 12.65\% on in-distribution data. As shown in Table~\ref{tab:OOD_performance}, RoBERTa-base achieves the highest average AUROC score, but since the F1-score is not available for the baseline, we use the AUROC difference to calculate the improvement, which is (95.65\% - 83\%) = 12.65\%.

\item[$\bullet$] IPAD with ~\textit{Regeneration Comparator} outperforms IPAD with ~\textit{Prompt-Text Consistency Verifier} by 0.13\% on out-of-distribution (OOD) data. As shown in Table~\ref{tab:OOD_performance}, IPAD version 2 has the highest AUROC of 95.65\%, while IPAD version 1 has an AUROC of 95.52\%, resulting in a 0.13\% difference.

\item[$\bullet$] IPAD with ~\textit{Regeneration Comparator} outperforms IPAD with ~\textit{Prompt-Text Consistency Verifier} by 3.78\% on attacked data. As shown in Table~\ref{tab:performance_metrics_detection} (rows 3-4) and Table~\ref{tab:OOD_performance} (rows 6-8), IPAD version 2 achieves the best F1 score and AUROC scores. To calculate the overall attacked dataset score, we calculate the F1 scores for Table~\ref{tab:OOD_performance}: 94.82\%, 95.35\%, 95.31\% for IPAD version 2, and 83.58\%, 88.34\%, and 94.70\% for IPAD version 1. The average F1 score difference is thus (94.82\% + 95.35\% + 95.31\% - 83.58\% - 88.34\% - 94.70\% + 97.60\% + 98.86\% - 97.55\% - 98.85\%) / 5 = 3.78\%.

\end{itemize}


\section{DPIC (decouple prompt and intrinsic characteristics) Prompt Extraction Zero-shot Prompts}
\label{sec:DPIC prompt}
~\textit{"I want you to play the role of the questioner. I will type an answer in English, and you will ask me a question based on the answer in the same language. Don’t write any explanations or other text, just give me the question. <TEXT>."}. 


\section{Linguistic Difference Examples}
\label{sec:Linguistic Difference Examples}
Figure ~\ref{fig:sentence complexity} shows examples where HWT and LGT prompts with different sentence complexity. Figure ~\ref{fig:comparison} shows the results of analytical thinking level statistics. Figure ~\ref{fig:person} shows examples of using different personas and different analytical thinking levels.

% \begin{figure}[t]
%   \centering
%   \includegraphics[width=0.5\textwidth]{concept scope.png}
%   \caption{Concept Scope Examples}
%   \label{fig:concept scope}
% \end{figure}


\begin{figure}[t]
  \centering
  \includegraphics[width=0.5\textwidth]{sentence_complexity.png}
  \caption{Sentence Complexity Examples, where ~\textbf{HWT Prompt} stands for prompt generated by the Prompt Inverter from HWT, and ~\textbf{LGT Prompt} stands for prompt generated by the Prompt Inverter from LGT. The HWT Prompts have longer sentence lengths, more words with more than three syllabus (as shown in bold), and more stop-words (as shown with underline).}
  \label{fig:sentence complexity}
\end{figure}


\begin{figure}[t]
  \centering
  \includegraphics[width=0.5\textwidth]{comparison.png}
  \caption{Comparison of different analytical thinking levels, with LGT has higher percentage of level 4 and level 5.}
  \label{fig:comparison}
\end{figure}

\begin{figure}[t]
  \centering
  \includegraphics[width=0.5\textwidth]{person.png}
  \caption{Examples that use different persona usage (above), and different analytical thinking levels (below left has level 2, and below right has level 5, they are prompts generated by the same problem statements).}
  \label{fig:person}
\end{figure}

\section{User Study}
Figure ~\ref{fig:gptzero}~\ref{fig:quillbot} and ~\ref{fig:scribbr} shows the screenshots of online AI detectors. Figure ~\ref{fig:questionnaire} shows the questionnaire questions. Figure ~\ref{fig:guide} shows the user guide.
\label{User study}
\subsection{Online AI Detectors Screenshots}


\begin{figure}[t]
  \centering
  \includegraphics[width=0.5\textwidth]{gptzero.png}
  \caption{GPTZero Online Detector Screenshot}
  \label{fig:gptzero}
\end{figure}

\begin{figure}[t]
  \centering
  \includegraphics[width=0.5\textwidth]{quillbot.png}
  \caption{Quillbot Online Detector Screenshot}
  \label{fig:quillbot}
\end{figure}

\begin{figure}[t]
  \centering
  \includegraphics[width=0.5\textwidth]{scribbr.png}
  \caption{Scribbr Online Detector Screenshot}
  \label{fig:scribbr}
\end{figure}

\subsection{Questionnaire questions}
\begin{figure}[t]
  \centering
  \includegraphics[width=0.5\textwidth]{questionnaire.png}
  \caption{Questionnaire questions}
  \label{fig:questionnaire}
\end{figure}


\begin{figure}[t]
  \centering
  \includegraphics[width=0.5\textwidth]{IRB.png}
  \caption{User Study User guide}
  \label{fig:guide}
\end{figure}
\end{document}
