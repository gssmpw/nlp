\section{Literature Review}
\begin{figure}
\begin{center}
\includegraphics[height=9cm]{figures/fig1.png}
\end{center}
\caption 
{ \label{fig1}
Unique Fire Locations.} 
\end{figure} 

The foundation of fire prediction modeling was built on statistical approaches, primarily using logistic regression and historical fire occurrence data [4,5,6]. These methods, while fundamental, often struggled to capture the complex, non-linear relationships inherent in fire behavior. Early works by Preisler [7,8] established baseline methodologies using generalized linear models, which, though limited, provided the scaffolding for more sophisticated approaches. The initial statistical frameworks employed analysis for binary prediction outcomes, Poisson regression for count-based fire occurrence modeling, maximum likelihood estimation techniques, and Bayesian inference methods for parameter estimation. These early methods were notably constrained by their limited ability to capture non-linear relationships, difficulty in handling multiple interactive variables, reduced effectiveness with sparse or imbalanced datasets, and the inherent assumption of independence between observations [9,10].

\begin{table}[ht!]
\centering
\caption{Unique Values per Column}
\small
\begin{tabular}{l r}
\toprule
Column Name & Number of Unique Values \\
\midrule
type name              & 5 \\
type id                & 5 \\
lon                    & 25708 \\
lat                    & 23547 \\
temperature (c)         & 2581 \\
precipitation (mm)      & 571 \\
relative humidity     & 4028 \\
wind speed (ms)         & 532 \\
solar radiation       & 731 \\
\bottomrule
\end{tabular}
\label{tab:unique_values}
\end{table}

The transition to machine learning marked a significant advancement in prediction capabilities. Ensemble methods emerged as powerful tools, with Random Forests leading the transformation [11]. Studies [12,13] demonstrated how Random Forests could effectively process high-dimensional feature spaces while maintaining interpretability, a crucial factor for practical implementation. The later works showcased superior performance in feature importance ranking, handling missing data, managing high-dimensional feature spaces, and maintaining strong performance across diverse geographical regions. The development of gradient boosting methods, particularly XGBoost and LightGBM implementations, further enhanced the field by providing improved computational efficiency on structured weather data, better handling of imbalanced datasets, enhanced feature selection capabilities, and superior cross-validation performance [14,15,16].

The paradigm shift toward deep learning has revolutionized fire prediction capabilities. Convolutional Neural Networks (CNNs) have proven particularly effective in processing satellite imagery and spatial data. Notable work by Zhang et al. [17] demonstrated how multi-layer CNNs can integrate multiple data streams, from meteorological conditions to vegetation indices, achieving prediction accuracies exceeding 85 percent. Their implementation demonstrated high-level performance across diverse landscapes, effective feature extraction from complex spatial data, and improved generalization across different geographical regions. Long Short-Term Memory (LSTM) networks have excelled in capturing temporal dependencies in fire progression, with recent implementations showing particular promise in predicting fire spread patterns over extended periods [18,19]. Such networks have advanced capabilities in sequential pattern recognition, long-term dependency modeling, dynamic time series analysis, and adaptive learning rates for varying time scales.

Contemporary research has focused on incorporating both spatial and temporal dimensions simultaneously through hybrid approaches. Graph Neural Networks (GNNs) have emerged as powerful tools for modeling spatial relationships between different regions, while attention mechanisms help focus on critical temporal patterns [20,21]. GNNs have demonstrated exceptional capabilities in spatial relationship modeling, network-based fire spread prediction, regional interdependence analysis, and topological feature extraction. The attention mechanisms have further enhanced predictions through dynamic feature weighting, temporal pattern recognition, multi-scale temporal analysis, and context-aware prediction frameworks.

The operational requirements of fire prediction have driven research into efficient computing approaches. Edge computing solutions have been developed to process satellite data and sensor inputs in real-time, enabling rapid response capabilities. These developments include real-time data processing capabilities, distributed computing architectures, low-latency prediction systems, and resource-optimized deployment strategies [22]. System integration advances have focused on sensor network integration, satellite data processing pipelines, multi-source data fusion frameworks, and scalable computing infrastructure [23].

Emerging areas of investigation in fire prediction modeling include explainable AI frameworks, transfer learning applications, multi-modal integration, and computational efficiency improvements [24]. Research in explainable AI focuses on model interpretation methods, uncertainty quantification, confidence scoring systems, and decision support integration. Transfer learning applications explore cross-region model adaptation, domain-specific fine-tuning, feature transfer optimization, and model portability enhancement. Multi-modal integration advances investigate sensor fusion techniques, data synchronization methods, cross-platform compatibility, and real-time data integration. Computational efficiency research examines model compression techniques, inference optimization, resource allocation strategies, and distributed computing frameworks [25,26].