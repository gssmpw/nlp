\documentclass[conference]{IEEEtran}
\IEEEoverridecommandlockouts
% The preceding line is only needed to identify funding in the first footnote. If that is unneeded, please comment it out.
\usepackage{cite}

\usepackage{amsmath,amssymb,amsfonts}

\usepackage{algorithmicx}

\usepackage{algpseudocode}
\usepackage{tikz}
\usetikzlibrary{fit}
\usetikzlibrary{tikzmark}
\usepackage{graphicx}
\usepackage{wrapfig}
\usepackage{textcomp}
\usepackage{pifont}
\usepackage{xcolor}
\usepackage{caption}
\usepackage{subcaption}
\captionsetup{compatibility=false}
\usepackage{balance} 
\usepackage{xspace}
\usepackage[T1]{fontenc}
\usepackage[scaled=0.81]{beramono}
\usepackage{balance}
\let\endproof\relax
\let\proof\relax


\let\labelitemi\labelitemii
\usepackage{enumitem}  
\usepackage{listings}
\usepackage{url}
\usepackage[ruled,lined,linesnumbered,vlined]{algorithm2e}
\usepackage{multirow}
\usepackage{lscape}
\usepackage{longtable}
\usepackage{array} 
\usepackage{array}
\usepackage{tcolorbox}
\usepackage{color}
%\usepackage[utf8]{inputenc}
\usepackage[T1]{fontenc}
\usepackage[scaled=0.81]{beramono}
%\usepackage{zlmtt}
\usepackage{fancyvrb}
\usepackage{listings}
\usepackage{multicol}
\usepackage{lipsum}
\usepackage{booktabs}
\usepackage{listings}
\usepackage{color}
\usepackage{graphicx}
\usepackage{calc}
\usepackage{amsmath}
\usepackage{upgreek}
\usepackage{float}
\usepackage{mathrsfs}
\usepackage{amssymb}
\usepackage{threeparttable}
\usepackage[framemethod=tikz]{mdframed}
\usepackage{lipsum}
%\usepackage{lstlinebgrd}
\usepackage{float}
\usepackage{amsthm}
\usepackage{pifont}

\usepackage{color}
\usepackage{wasysym}
\usepackage{colortbl}

\usepackage{perpage} %the perpage package
\MakePerPage{footnote} %the perpage package command

\input{solidity-highlighting.tex}
\definecolor{darkgreen}{rgb}{0.0, 0.5, 0.13}
\definecolor{C-gray}{gray}{0.85}
\definecolor{B-gray}{gray}{0.65}
\definecolor{A-gray}{gray}{0.95}

\lstset{
	basicstyle=\ttfamily\small,
	%	numbers=left,
	xleftmargin=2em,
	%	numbersep=0.1pt,
	keywordstyle=\color{blue},
	commentstyle=\color{darkgreen},
}


%\usepackage[
%bookmarksopen,
%bookmarksdepth=2,
%breaklinks=true
%] {hyperref}

%~~~~~~~~~~~~~~~~~~~~~~~~~~~~~~~~~~~~~~~~~~~~~~~~~~~~~~~~~~~~~~~~~~~~~~~~~~~~~~
% Macros																{{{1
% Terminology
\newcommand{\tool}{\hbox{\textsc{UAT20}}\xspace}
\newcommand{\bitcoin}{\hbox{Bitcoin}\xspace}
\newcommand{\btc}{\hbox{BTC}\xspace}
\newcommand{\layertwo}{\hbox{Layer2}\xspace}
\newcommand{\defi}{\hbox{DeFi}\xspace}

\newcommand{\ethereum}{\hbox{Ethereum}\xspace}
\newcommand{\imc}{\hbox{UP}\xspace}
\newcommand{\fve}{\hbox{FVE}\xspace}
\newcommand{\cfg}{\hbox{CFG}\xspace}
\newcommand{\solidity}{\hbox{Solidity}\xspace}
\newcommand{\evm}{\hbox{EVM}\xspace}
\newcommand{\argmax}{\operatornamewithlimits{argmax}}
\newcommand{\srt}{\hbox{TR}\xspace}
\newcommand{\crt}{\hbox{DR}\xspace}
\newcommand{\crtm}{\hbox{IM-D}\xspace}
\newcommand{\srtm}{\hbox{IM-T}\xspace}
\newcommand{\srnt}{\hbox{SRNT}\xspace}
\newcommand{\crnt}{\hbox{CRNT}\xspace}
\newcommand{\crtnm}{\hbox{CRTNM}\xspace}
\newcommand{\srtnm}{\hbox{SRTNM}\xspace}
\newcommand{\rts}{\hbox{RTS}\xspace}
\newcommand{\manipulated}{\hbox{MC}\xspace}
\newcommand{\broken}{\hbox{BC}\xspace}
\newcommand{\stealing}{\hbox{SC}\xspace}
\newcommand{\gas}{\hbox{GC}\xspace}
\newcommand{\oyente}{\hbox{\textsf{Oyente}}\xspace}
\newcommand{\securify}{\hbox{\textsf{Securify}}\xspace}
\newcommand{\mythril}{\hbox{\textsf{Mythril}}\xspace}
\newcommand{\zthree}{\hbox{\textsf{Z3}}\xspace}
% Command
\newcommand{\myparagraph}[1]{\vspace*{0.14cm}\noindent\textbf{\emph{#1.}}\quad}
\theoremstyle{definition}
\newtheorem{thm}{Theorem}
\newtheorem{defn}[thm]{Definition}
\newcommand{\specialcell}[2][c]{\begin{tabular}[#1]{@{}c@{}}#2\end{tabular}}
% English
\newcommand{\cf}{\hbox{\emph{cf.}}\xspace}
\newcommand{\deletia}{\ldots [deletia] \ldots}
\newcommand{\etal}{\hbox{\emph{et al.}}\xspace}
\newcommand{\eg}{\hbox{\emph{e.g.}}\xspace}
\newcommand{\ie}{\hbox{\emph{i.e.}}\xspace}
\newcommand{\scil}{\hbox{\emph{sc.}}\xspace} %scilicet: it is 
%permitted to know
%\newcommand{\st}{\hbox{\emph{s.t.}}\xspace}
\newcommand{\wrt}{\hbox{\emph{w.r.t.}}\xspace}
\newcommand{\etc}{\hbox{\emph{etc.}}\xspace}
\newcommand{\viz}{\hbox{\emph{viz.}}\xspace} %videlicet: it is 
%permitted to see
\newcommand{\toolUrl}{\texttt{\url{https://github.com/njaliu/ConFu-release}}} 
\newcommand{\videoUrl}{\texttt{\url{https://youtu.be/AqCSxdXdQds}}}


\begin{document}
\lstset{%
	escapeinside={(*@}{@*)}
}

\title{\textsc{UAT20}: Unifying Liquidity Across Rollups}

%\title{Paper Title*\\
%{\footnotesize \textsuperscript{*}Note: Sub-titles are not captured in Xplore and
%should not be used}
%\thanks{Identify applicable funding agency here. If none, delete this.}
%}

%\author{\IEEEauthorblockN{1\textsuperscript{st} Given Name Surname}
%\IEEEauthorblockA{\textit{dept. name of organization (of Aff.)}\\
%City, Country \\
%email address}
%\and
%\IEEEauthorblockN{2\textsuperscript{nd} Given Name Surname}
%\IEEEauthorblockA{\textit{dept. name of organization (of Aff.)} \\
%City, Country \\
%email address}
%\and
%\IEEEauthorblockN{3\textsuperscript{rd} Given Name Surname}
%\IEEEauthorblockA{\textit{dept. name of organization (of Aff.)} \\
%City, Country \\
%email address}
%\and
%\IEEEauthorblockN{4\textsuperscript{th} Given Name Surname}
%\IEEEauthorblockA{\textit{dept. name of organization (of Aff.)} \\
%City, Country \\
%email address}
%\and
%\IEEEauthorblockN{5\textsuperscript{th} Given Name Surname}
%\IEEEauthorblockA{\textit{dept. name of organization (of Aff.)} \\
%City, Country \\
%email address}
%\and
%\IEEEauthorblockN{6\textsuperscript{th} Given Name Surname}
%\IEEEauthorblockA{\textit{dept. name of organization (of Aff.)} \\
%City, Country \\
%email address}
%}

\author{
	\IEEEauthorblockN{Yue Li \quad Han Liu}
	\IEEEauthorblockA{\textit{UAT Team} \\
	} 
}

\maketitle

\begin{abstract}
Retrieval-Augmented Generation (RAG) is often used with Large Language Models (LLMs) to infuse domain knowledge or user-specific information. In RAG, given a user query, a retriever extracts chunks of relevant text from a knowledge base. These chunks are sent to an LLM as part of the input prompt. Typically, any given chunk is repeatedly retrieved across user questions. However, currently, for every question, attention-layers in LLMs fully compute the key values (KVs) repeatedly for the input chunks, as state-of-the-art methods cannot reuse KV-caches when chunks appear at arbitrary locations with arbitrary contexts. Naive reuse leads to output quality degradation.  This leads to potentially redundant computations on expensive GPUs and increases latency. In this work, we propose \sys, a system for managing and reusing precomputed KVs corresponding to the text chunks (we call \textit{chunk-caches}) in RAG-based systems. We present how to identify \hl{\textit{chunk-caches} that are reusable}, how to efficiently perform a small fraction of recomputation to \textit{fix} the cache to maintain output quality, and how to efficiently store and evict \textit{chunk-caches} in the hardware for maximizing reuse while masking any overheads. With real production workloads as well as synthetic datasets, we show that \sys reduces redundant computation by \textbf{51\%} over SOTA prefix-caching and \textbf{75\%} over full recomputation.
\hl{Additionally, with continuous batching on a real production workload, we get a \textbf{1.6$\times$} speedup in throughput and a \textbf{2$\times$} reduction in end-to-end response latency over prefix-caching while maintaining quality, for both the \llama-3-8B and \llama-3-70B models. 
}
\end{abstract}








\section{Introduction}
\label{sec:intro}
\section{Introduction}
\label{sec:intro}

\begin{figure*}[tb]
    \centering
    \includegraphics[width=0.848\linewidth]{figs/circuitnn.pdf} 
    \caption{Illustration of differentiable CircuitNN. CircuitNN is designed based on differentiable NAND gates. After DAS is guided by PI and PO pairs of the truth table, CircuitNN can get the precise circuit architecture logic equivalent to the truth table.}
    \label{fig:circuitnn}
\end{figure*}

% 1. Describe the importance of logic synthesis
% 2. Existing Problems
% (a) Neural Architecture Search: Unstable, Predefined Setting, etc.
% (b) Circuit Generation: Probabilistic Model, Logic Equivalence

With the rapid advancement of technology, the scale of integrated circuits (ICs) has expanded exponentially. 
This expansion has introduced significant challenges in chip manufacturing, particularly concerning power and area metrics.
A primary objective in IC design is achieving the same circuit function with fewer transistors, thereby reducing power usage and area occupancy.

Logic synthesis~\cite{hachtel2005logicsynth}, a critical step in electronic design automation (EDA), transforms behavioral-level circuit designs into optimized gate-level circuits, ultimately yielding the final IC layout. 
The primary goal of logic synthesis is to identify the physical implementation with the fewest gates for a given circuit function. 
This task constitutes a challenging NP-hard combinatorial optimization problem. 
Current logic synthesis tools~\cite{brayton2010abc, wolf2013yosys} rely on human-designed heuristics, often leading to sub-optimal outcomes.

Differentiable architecture search (DAS) techniques~\cite{liu2018darts, chu2020darts} offer novel perspectives on addressing challenges in this problem.
Circuit functions can be represented through truth tables, which map binary inputs to their corresponding outputs. 
Truth tables provide a precise representation of input-output relationships, ensuring the design of functionally equivalent circuits.
Inspired by this, researchers~\cite{deepmind2024ai4sys, wang2024tnet} have begun exploring the application of DAS to synthesize circuits directly from truth tables.
Specifically, \citet{deepmind2024ai4sys} proposed CircuitNN, a framework that learns differentiable connection structures with logic gates, enabling the automatic generation of logic circuits from truth tables.
This approach significantly reduces the complexity of traditional circuit generation. 
Building on this, \citet{wang2024tnet} introduced T-Net, a triangle-shaped variant of CircuitNN, incorporating regularization techniques to enhance the efficiency of DAS.

Despite these advancements, several challenges remain. 
The computational complexity of DAS grows quadratically with the number of gates, posing scalability issues.
Although triangle-shaped architecture~\cite{wang2024tnet} partially mitigates this problem, redundancy persists. 
%Additionally, DAS is susceptible to converging to local optima, limiting the ability to search architectures that satisfy the given truth tables~\cite{liu2018darts}. 
%Furthermore, hyperparameters (network depth and layer width) require extensive searches, introducing complexity and prolonging the synthesis process. 
Additionally, DAS is susceptible to converging to local optima~\cite{liu2018darts} and hyperparameters (network depth and layer width) require extensive searches. 
The challenges arise from the vast search space in DAS. 
% Even with predefined settings for CircuitNN, finding a configuration that meets the truth table requires extensive trial and error during the DAS process. 
Intuitively, limiting the search space through predefined parameters (network depth, gates per layer, and connection probabilities) can significantly reduce the complexity.

Recent advances~\cite{openai2023gpt4, abramson2024alphafold3, esser2024sd3, li2024mar} in conditional generative models have demonstrated remarkable performance across language, vision, and graph generation tasks. 
Motivated by these developments, we propose a novel approach to circuit generation that generates preliminary circuit structures to guide DAS in generating refined circuits matching specified truth tables. 
Firstly, we introduce CircuitVQ, a tokenizer with a discrete codebook for circuit tokenization. 
Built upon our Circuit AutoEncoder framework~\cite{hou2022graphmae,li2023maskgae,wu2025mgvga}, CircuitVQ is trained through a circuit reconstruction task. 
Specifically, the CircuitVQ encoder encodes input circuits into discrete tokens using a learnable codebook, while the decoder reconstructs the circuit adjacency matrix based on these tokens.
Subsequently, the CircuitVQ encoder serves as a circuit tokenizer for CircuitAR pretraining, which employs a masked autoregressive modeling paradigm~\cite{chang2022maskgit, li2023mage}. 
In this process, the discrete codes function as supervision signals. 
After training, CircuitAR can generate discrete tokens progressively, which can be decoded into initial circuit structures by the decoder of the CircuitVQ. 
These prior insights can guide DAS in producing refined circuits that match the target truth tables precisely.

Our key contributions can be summarized as follows:
\begin{itemize}
\item We introduce CircuitVQ, a circuit tokenizer that facilitates graph autoregressive modeling for circuit generation, based on our Circuit AutoEncoder framework;
\item Develop CircuitAR, a model trained using masked autoregressive modeling, which generates initial circuit structures conditioned on given truth tables;
\item Propose a refinement framework that integrates differentiable architecture search to produce functionally equivalent circuits guided by target truth tables;
\item Comprehensive experiments demonstrating the scalability and capability emergence of our CircuitAR and the superior performance of the proposed circuit generation approach.
\end{itemize}

% Motivation
% (a) Diffusion (Vision, Graph), Autoregressive (Language, Vision)
% (b) Circuit Generation for Predefined Setting
% (c) Neural Architecture Search for Strict Logic Equivalence

% Contribution
% (a) Circuit Tokenizer (new transformer arch, training strategy)
% (b) CircuitAR (train and gen strategies, post-ar strategy)
% (c) Extensive Evaluation including BitD (Bit Distance) for Scalability


% \section{Background}
% \label{sec:bg}
% \section{Background and Motivation} \label{s:bg}
\subsection{TEE Data Interaction} \label{s:params}
As shown in Fig.~\ref{fig:datacom}, the normal world and TEE are two independent environments, separated to ensure the security of sensitive functions and data. In this architecture, the communication between TEE and the normal world involves three types of parameters: input, output, and shared memory~\cite{s20041090}.

\begin{figure}[t]
    \centering
    \includegraphics[width=0.5\linewidth]{figures/Fig_2.drawio.pdf}
    \caption{Data communication between TEE and the normal world.}
    \label{fig:datacom}
\end{figure}

Input parameters are used to transfer data from the normal side to TEE, while outputs handle the results or send processed data back. Inputs and outputs are simple mechanisms that allow users to temporarily transfer small amounts of data between the normal world and TEE. However, temporary inputs/outputs realize data transfer through memory copying, slightly lowering the performance of TEE applications due to additional memory copy. They are mainly used to transmit lightweight data, such as user commands and data for cryptographic operations.

Shared memory provides a zero-copy memory block to exchange larger data sets (\eg, multimedia files or bulk data) between two sides~\cite{optee}. It allows both sides to access the same memory space efficiently, which avoids frequent memory copying. Shared memory can also remain valid across different TEE invocation sessions, making it suitable for scenarios that require data to be reused multiple times.

Moreover, an SDK is responsible for managing these parameters and communication in the normal side. For example, TrustZone-based OP-TEE uses \texttt{TEEC\_InvokeCommand()} function to execute the in-TEE code, enabling the shared memory or temporary buffers to transfer data between the two sides~\cite{8684292}. 
Then, OP-TEE can handle these interactions through its internal APIs, which process incoming requests, perform secure computations, and return responses to the normal world.
Similarly, Intel SGX utilizes the ECALL and OCALL interfaces to achieve these functionalities~\cite{10632129}. ECALLs allow the normal world to securely invoke functions within the SGX enclave, passing data into the trusted environment for processing, while OCALLs enable the enclave to request services or share results with the untrusted normal world.

Table~\ref{tbl:comp_params} illustrates some differences between input/output parameters and shared memory. It is important to note that, while the normal side cannot directly access the memory copies of input and output parameters in TEE, an attacker with the permissions of the normal world can still tamper with the inputs before they are transmitted to TEE or intercept and read the outputs after they are returned from TEE~\cite{9925569, 10477533}.
Additionally, since shared memory relies on address-based data transfer between the two sides, any modifications made to the data on one side will be instantly mirrored on the other opposing side.
Therefore, data interactions controlled by the normal world code are the root cause of the bad partitioning issues in TEE applications.

\begin{table}[t]
    \caption{Comparison of input/output parameters and shared memory.}
    \label{tbl:comp_params}
    % \renewcommand{\arraystretch}{1.3}
    % \footnotesize
    \setlength{\tabcolsep}{3mm}
    \centering
	\begin{tabular}{lp{4.5cm}p{4.5cm}}
		\toprule
		\textbf{Feature} & \textbf{Input/Output} & \textbf{Shared Memory} \\
		\midrule
            Data Size & Small & Big  \\
            Efficiency & Low, memory copying required & High, no need of additional memory copy\\
            Life-time & Temporary, valid for a single TEE invocation & Valid for a long time and shared among several TEE invocations \\
            Privilege & The normal world cannot reach memory copy in TEE & Both the normal world and TEE can access at the same time \\
		\bottomrule
	\end{tabular}
\end{table}


\section{The \tool Protocol}
\label{sec:protocol}
\begin{figure}[ht]
    \centering
    % \includegraphics[width=0.45\textwidth]{image/protocol.png}
    \includegraphics[width=0.45\textwidth]{image/protocol_new.png}
    \caption{The master node broadcasts UDP messages containing its details to the broadcast address (e.g., \textit{192.168.0.255}) at port 7720. Each IP address in the diagram (e.g., \textit{192.168.0.100/0}) represents a device's unique address on the network, where '0' indicates a dynamically assigned port provided by the operating system. XR nodes, upon startup, listen on the broadcast port (7720) to receive these messages, extract the master node's IP and ZMQ socket address, and build a stable connection. This architecture supports both request-response and publish-subscribe communication patterns, ensuring robust, multi-device connectivity with automatic reconnection capabilities.}
    \label{fig:protocol}
\end{figure}

% \section{Security Analysis}
% \begin{figure*}[!htb]
\begin{tikzpicture}
\node[draw,thick,inner sep=10pt] (box) {
  \begin{minipage}{0.96\textwidth} % Adjust width to fit your content    
        \vspace{-5pt} 
        \begin{multicols}{2}
  	% \begin{multicols}{1}  % Begin the two-column environment
  		% Now, format the text as it appears in the 'picture'
        % \overview{Parties: Users {$U_1,...,U_m$}, Certificate Authority $A$, Service Providers {$P_1,...,P_l$}, and Access Controllers {$C_1,...,C_n$}.}
        % \overview{Parameters: Access policy \(\phi\), issuance criteria \(\ell\), service to access data, computation and authorization $S_{data}, S_{comp}, S_{auth}$, and secret $s$.}

        \vspace{5pt} 
        
        \func{Setup}{$1^\lambda$}
        \step{1. Let $\mathcal{R}$ be a circuit with public input $\textsf{cred}$ and $\textsf{EI}$ asserting: \\ $\textsf{EI} = \textsf{Ec}(e, \textsf{attrs}) \land \textsf{cred}$ opens to $(\textsf{nk,attrs})$}
        \step{2. Compute $\textsf{crs}_\mathcal{R} := \textsf{G16.Setup}(1^\lambda, \mathcal{R})$}
        \step{3. Let $\textsf{pp} := (1^\lambda, \textsf{crs}_\mathcal{R})$ and return $\textsf{pp}$}

        \vspace{5pt} 

        \func{{IssueReq_\textit{U}}}{\textsf{pp}, \textsf{attrs}}
        \step{1. Sample $r_\textsf{cred}, \textsf{nk}$ // commitment nonce, pseudonym key}
        \step{2. Commit $\textsf{cred} := \textsf{Com}(\textsf{nk}, \textsf{attrs}; r_\textsf{cred})$}
        \step{3. Let $\omega := (\textsf{nk}, \textsf{attrs}, r_\textsf{cred})$}
        \step{4. Send ($\omega$, $\textsf{cred}$) to $A$ and receive a proof $\theta$ attesting to $\textsf{cred} \in \textsf{MT}'$}

        \vspace{5pt} 

        \func{{IssueGrant_\textit{A}}}{$\textsf{pp}, \textsf{MT}, \textsf{cred}, \omega$}
        \step{1. Check $\textsf{cred} = \textsf{Com}(\textsf{nk,attrs}; r_\textsf{cred})$}
        \step{2. Let $\textsf{MT}' := \textsf{MT}.\textsf{Add}(\textsf{pk}^{U}, \textsf{cred})$ and $\gamma := \textsf{MT}'.\textsf{Root}$}
        \step{3. Let $\theta := \textsf{MT}'.\textsf{Prove}(\textsf{pk}^{U})$ // $\theta$ attests to $\textsf{cred} \in \textsf{MT}'$}
        \step{4. Send \(\theta\) to \(U\) and publish $\gamma$}

        \vspace{5pt} 

        % Service provider setup for (service, policy, secret, share, time-bound) core part!!!
        \func{{SetPolicy_\textit{P}}}{$\textsf{pp}, \textsf{S}, (f_\phi, param_\phi), T_1, T_2$}
        \step{1. Sample $s$ // secret}
        \step{2. Let $(\textsf{F}_\phi, (e^U, e^P), d) := \textsf{Gb}(1^\lambda, f_\phi)$}
        \step{3. Let $\textsf{EP} := \textsf{Ec}(e^P, param_\phi)$}
        % \step{3. Commit $\textsf{pol} := \textsf{Com}(\textsf{pk}^P, \phi; r_\textsf{pol})$}
        % \step{4. Let $\omega_P := (\phi, e, r_\textsf{pol})$}
        \step{4. Let $\textsf{pp}_\textsf{PVTSS}$ := $\textsf{PVTSS.Setup}(1^\lambda, T_1, T_2)$}
        \step{5. Let $(\{c_i\}_{i\in[n]}, \pi_D) := \textsf{PVTSS.Sharing}(\textsf{pp}_\textsf{PVTSS}, s, \{\textsf{pk}^{C_i}\}_{i\in[n]})$ // locked encrypted shares $\{c_i\}_{i\in[n]}$ with time parameter $T_1$}
        % \step{7. Publish ($\textsf{pol}, \textsf{S}$)}
        \step{6. Send ($\textsf{S}, \textsf{F}_\phi, \textsf{EP}, d, \{c_i\}_{i\in[n]}, \pi_D$) to each party $C_i \in C$}

        \columnbreak
        
        % Access Controllers receive shares and verify
        \func{{VerifyShare_\textit{C}}}{$\textsf{pp}, \textsf{pp}_\textsf{PVTSS}, \{c_i\}_{i\in[n]}, \pi_D, \textsf{F}_\phi, d$}
        \step{1. Check $\textsf{PVTSS.Verify1}(\textsf{pp}_\textsf{PVTSS}, \{\textsf{pk}^{C_i}, c_i\}_{i\in[n]}, \pi_D)$}
        \step{2. Let $\{\hat{s}_i, \pi_i\} := \textsf{PVTSS.Recover}(\textsf{pp}_\textsf{PVTSS}, c_i, \textsf{pk}^{C_i}, \textsf{sk}^{C_i})$ 
        // recover the decrypted share $\hat{s}_i$ together with proof $\pi_i$ of valid decryption}
        \step{3. Store ($\textsf{F}_\phi, d, \hat{s}_i, \pi_i$)}

        \vspace{5pt} 

        \func{{Encode_\textit{U}}}{$\textsf{pp}, \textsf{cred}, \textsf{attrs}, \textsf{S}, \omega$}
        % \step{2. Let ($r, \textsf{attrs}$) := ${\textsf{UserCreds}_\textit{U}}[\textsf{cred}]$}
        % \step{3. If \({\phi}_{vsb}\) is $\textsf{public}$: let ${\textsf{ShowProofs}_\textit{U}}[\textsf{sn}]$ := ($\phi, \textsf{cred}, \theta, r, \textsf{aux}$)}
        % \step{1. \(U\), \(P\) and \(C\) negotiate the secure MPC function $\textsf{Ec}(X, Y)$}
        % \step{2. Public input $(\textsf{cred,pol}, \gamma)$}
        % \step{3. Private input $\omega_P, \omega_U$ by $P$ and $U$}
        % \step{4. Only $C$ receive the initial input $\textsf{EI} := \textsf{Ec}(e, \textsf{attrs})$ if $\textsf{cred}$ opens to $(\textsf{nk,attrs}) \land \textsf{cred} \in \textsf{MT}$}
        \step{1. Request encoding information $e^U$ form $P$ and let $\textsf{EI} := \textsf{Ec}(e^U, \textsf{attrs})$}
        \step{2. Let $\pi_\mathcal{R} := \textsf{G16.Prove}(\textsf{crs}_\mathcal{R}, (\textsf{cred}, \textsf{EI}), (\omega, e^U))$}
        \step{3. Send ($\textsf{EI}, \textsf{cred}, \pi_\mathcal{R}, \textsf{S}$) to each party $C_i \in C$}

        \vspace{5pt} 
        
        \func{{Authenticate_\textit{C}}}{$\textsf{pp}, \textsf{F}_\phi, \textsf{cred}, \textsf{EI}, \textsf{EP}, \textsf{S}, d, \textsf{MT}, \gamma, \theta, \hat{s}_i, \pi_i$}
        \step{1. Check $\textsf{G16.Verify}(\textsf{crs}_\mathcal{R}, \pi_\mathcal{R}, (\textsf{cred}, \textsf{EI})) \land \textsf{MT}.\textsf{Verify}(\gamma, \textsf{pk}^{U}, \textsf{cred}, \theta)$}
        \step{2. Let encoded output $\textsf{EO} := \textsf{Ev}(\textsf{F}_\phi, (\textsf{EI}, \textsf{EP}))$}
        \step{3. Check the final output $\textsf{FO} := \textsf{De}(\textsf{EO}, d)$}
        \step{4. If check pass and over $T_1$, send ($\textsf{S}, c_i, \hat{s}_i, \pi_i$) to $U$}

        \vspace{5pt}

        \func{{AccessService_\textit{U}}}{$\textsf{pp}, \textsf{pp}_\textsf{PVTSS}, \textsf{S}, \{c_i\}, \{\hat{s}_i\}, \{\pi_i\}$}
        \step{1. Check $\textsf{PVTSS.Verify2}(\textsf{pp}_\textsf{PVTSS}, c_i, \hat{s}_i, \pi_i)$}
        \step{2. Add the valid $\hat{s}_i$ to a set $\mathcal{S}$}
        \step{3. If $|\mathcal{S}| > t$ and $T_2$ has not elapsed:\\ let $s := \textsf{PVTSS.Pool}(\textsf{pp}_\textsf{PVTSS}, \mathcal{S}, T_2)$ and $\textsf{Result} := \textsf{S}(s)$}

        \vspace{5pt} 

        \func{Revoke_\textit{A}}{$\textsf{pp}, \textsf{MT}, \textsf{pk}^{U}$}
        \step{1. Let $\textsf{MT}' := \textsf{MT}.\textsf{Remove}(\textsf{pk}^{U})$}
        \step{2. Return $\textsf{MT}'$}
    \end{multicols}
    \end{minipage}
};


\end{tikzpicture}
\caption{\tool construction.} %NB: Each participant maintains a list of public keys.
\label{fig:construction}
\end{figure*}

\section{Empirical Study}
\label{sec:discussion}
This work identifies signal collapse as a critical bottleneck in one-shot neural network pruning. Performance loss in pruned networks is due to \textbf{signal collapse} in addition to the removal of critical parameters. We propose \textbf{REFLOW} (\textbf{Re}storing \textbf{F}low of \textbf{Low}-variance signals), a simple yet effective method that mitigates signal collapse without computationally expensive weight updates. By focusing on signal preservation, REFLOW highlights the importance of mitigating signal collapse in sparse networks and enables magnitude pruning to match or surpass state-of-the-art one-shot pruning methods such as CHITA, CBS, and WF.

REFLOW consistently achieves state-of-the-art accuracy across diverse architectures, restoring ResNeXt-101 from under 4.1\% to 78.9\% top-1 accuracy at 80\% sparsity on ImageNet. Its lightweight design makes it a practical solution for both research and deployment, delivering high-quality sparse models without the overhead of traditional approaches. These findings challenge the traditional emphasis on weight selection strategies and underscore the critical role of signal propagation for achieving high-quality sparse networks in the context of one-shot pruning.




% \section{Related Work}
% \label{sec:rw}
% \section{Related Work}\label{sec:relatedwork}

Internet of Things (IoT) has seen rapid advancements in recent years, becoming an integral part of various domains, such as smart industries and homes, and serving as a key enabler in modern society.
However, despite its growth, IoT continues to face numerous security challenges, prompting significant research efforts aimed at improving IoT security.
With the rise of artificial intelligence (AI), machine learning (ML) and deep learning (DL)-based approaches have become increasingly popular in designing defense mechanisms for IoT devices, including malicious traffic classification~\cite{luo2022transformer,shafiq2020corrauc}, malware detection~\cite{vasan2020mthael,chaganti2022deep,aung2022atlas}, vulnerability discovery~\cite{neshenko2019demystifying}, and others~\cite{al2020survey,otoum2022dl,tambe2019detection}.

More recently, inspired by the success of large language models (LLMs), researchers have begun exploring the potential of LLMs to enhance IoT-related security tasks.
For instance, LLMs have been applied to existing IoT security challenges such as threat detection and fuzzing. Ferrag \etal~\cite{sokiotllm} introduced a BERT-based model, SecurityBERT, to achieve better cyber threat detection accuracy over traditional ML and DL-based methods. 
Similarly, Ma \etal~\cite{ma} and Wang \etal~\cite{llmiotfuz} proposed LLM-assisted fuzzing methods to uncover hidden bugs in IoT devices, enabling the detection of complex vulnerabilities that traditional techniques might miss.
Additionally, Yang \etal~\cite{yang2023iot} combined LLMs with static code analysis using prompt engineering to create a cost-effective solution for IoT vulnerability detection.
\cite{ji2024sevenllm} collected cybersecurity raw texts to train cybersecurity LLM to augment the analysis of cybersecurity events, and \cite{llmtikg} made use of a larger LLM to build knowledge graphs from public threat intelligence and use GPT to create datasets to fine-tune a smaller LLM to extract entities and TTPs from attack description.
Ferraris \etal~\cite{ferraris2024ici} proposed utilizing ChatGPT to enhance IoT trust semantics, aligning with W3C Web of Things (WoT) recommendations\footnote{\scriptsize \url{https://www.w3.org/WoT/}}.
This work extends the TrUStAPIS framework~\cite{ferraris2020trustapis}.

Beyond the above tasks, LLMs have been employed in other IoT challenges.
Meyuhas \etal~\cite{meyuhas2024iotlabel} used LLMs to address the problem of labeling previously unseen IoT devices.
\cite{llmiotcontrol,cui2024llmind} explored leveraging LLMs to control IoT devices and facilitate effective collaboration among them.
Mo \etal~\cite{mo2024iot} collected IoT sensor-natural language paired data and trained IoT-LM to interpret and interact with physical IoT sensors.
Xu \etal~\cite{xu2024penetrative} employed ChatGPT to interpret IoT sensor data and reason over tasks in the physical realm, introducing novel ways of integrating human knowledge into cyber-physical systems. 

Recently, Deldari \etal~\cite{deldari2024auditnet} proposed AuditNet, a conversational AI-based security assistant, which is most similar to \chatiot\ and also augmented by external knowledge.
However, AuditNet focused on standards, policies, and regulations of portable document format (PDF), and aimed to reduce the manual effort of security experts involved in compliance checks of IoT. 
On the other hand, we integrate IoT threat intelligence of various sources into \chatiot\ and can assist multiple kinds of users. Besides, we provide an end-to-end toolkit to process data in various formats, not limited to PDF. 

Together, these studies indicate that LLMs have great potential to improve the security of IoT systems in various domains, from vulnerability discovery to trustworthiness management. 
By integrating LLMs with IoT-specific threat intelligence, these models can be guided to meet the unique challenges posed by the IoT ecosystem.
Moreover, the continuous advancements in the LLM community, combined with increasingly accessible IoT datasets, are likely to further drive the adoption of LLMs in IoT-related research and practical applications.


\section{Conclusion}
\label{sec:conclusion}
\section*{Conclusion}
This paper aims to enhance our understanding of the computational complexity of computing various Shapley value variants. We found that for various ML models --- including decision trees, regression tree ensembles, weighted automata, and linear regression --- both local and global interventional and baseline SHAP can be computed in polynomial time under HMM modeled distributions. This extends popular algorithms, such as TreeSHAP, beyond their empirical distributional scope. We also establish strict complexity gaps between the various SHAP variants (baseline, interventional, and conditional) and prove the intractability of computing SHAP for tree ensembles and neural networks in simplified scenarios. Overall, we present SHAP as a versatile framework whose complexity depends on four key factors: \begin{inparaenum}[(i)] \item model type, \item SHAP variant, \item distribution modeling approach, \item and local vs. global explanations\end{inparaenum}. We believe this perspective provides deeper insight into the computational complexity of SHAP, paving the way for future work.




%We believe that our framework provides a more intricate understanding of SHAP computation complexity across different models, distributions, and variants, paving the way for further research.

Our work opens promising directions for future research. First, expanding our computational analysis to other SHAP-related metrics, such as asymmetric SHAP~\citep{frye20} and SAGE~\citep{covert2020understanding}, would be valuable. Additionally, we aim to explore more expressive distribution classes and relaxed assumptions beyond those in Section \ref{sec:tractable} while maintaining tractable SHAP computation. Finally, when exact computation is intractable (Section \ref{sec:intractable}), investigating the approximability of SHAP metrics through approximation and parameterized complexity theory~\citep{downey2012parameterized} is an important direction.

%Our work opens several promising avenues for future research on the computational properties of explainable AI methods, with a particular focus on SHAP. First, it would be interesting to broaden the computational analysis conducted in this work to include other popular SHAP-related metrics in the literature, such as asymmetric SHAP \cite{frye20} and SAGE \cite{covert2020understanding}. Also, in the future, we aim to explore more expressive distribution classes and relaxed distributional assumptions—extending beyond those examined in Section \ref{sec:tractable} —that still yield tractable SHAP computation. Finally, when exact computation proves intractable (Section \ref{sec:intractable}), it is worthwhile to theoretically investigate the question of the approximability of computing the SHAP metrics across various configurations, through the lens of approximation and parametrized complexity theory \cite{arora2009computational}.

%This paper aims to deepen our understanding of the computational complexity involved in obtaining different Shapley value variants. We found that for a variety of ML models, including decision trees, tree ensembles for regression, weighted automata, and linear regression models — computing both local and global interventional and baseline SHAP can be done in polynomial time when distributions are modeled by HMMs. This extends the distributional scope of popular algorithms like TreeSHAP, which is limited to empirical distributions. Additionally, we demonstrate a strict complexity gap between SHAP variants, showing that interventional and baseline SHAP can be strictly easier to compute than conditional SHAP. Despite these positive results, we uncovered intractability for various SHAP variants in neural networks and tree ensembles. Finally, we provided generalized complexity relations across SHAP variants. We believe that our framework offers a deeper understanding of the complexity involved in computing SHAP across various variants, models, distributions, as well as in both local and global computations, laying the groundwork for future research.







%\section*{Acknowledgment}
%
%The preferred spelling of the word ``acknowledgment'' in America is without 
%an ``e'' after the ``g''. Avoid the stilted expression ``one of us (R. B. 
%G.) thanks $\ldots$''. Instead, try ``R. B. G. thanks$\ldots$''. Put sponsor 
%acknowledgments in the unnumbered footnote on the first page.

%\section*{References}
%
%Please number citations consecutively within brackets \cite{b1}. The 
%sentence punctuation follows the bracket \cite{b2}. Refer simply to the reference 
%number, as in \cite{b3}---do not use ``Ref. \cite{b3}'' or ``reference \cite{b3}'' except at 
%the beginning of a sentence: ``Reference \cite{b3} was the first $\ldots$''
%
%Number footnotes separately in superscripts. Place the actual footnote at 
%the bottom of the column in which it was cited. Do not put footnotes in the 
%abstract or reference list. Use letters for table footnotes.
%
%Unless there are six authors or more give all authors' names; do not use 
%``et al.''. Papers that have not been published, even if they have been 
%submitted for publication, should be cited as ``unpublished'' \cite{b4}. Papers 
%that have been accepted for publication should be cited as ``in press'' \cite{b5}. 
%Capitalize only the first word in a paper title, except for proper nouns and 
%element symbols.
%
%For papers published in translation journals, please give the English 
%citation first, followed by the original foreign-language citation \cite{b6}.

\bibliographystyle{IEEEtran}
\bibliography{ethereum,pl} 

%\begin{thebibliography}{00}
%\bibitem{b1} G. Eason, B. Noble, and I. N. Sneddon, ``On certain integrals of Lipschitz-Hankel type involving products of Bessel functions,'' Phil. Trans. Roy. Soc. London, vol. A247, pp. 529--551, April 1955.
%\bibitem{b2} J. Clerk Maxwell, A Treatise on Electricity and Magnetism, 3rd ed., vol. 2. Oxford: Clarendon, 1892, pp.68--73.
%\bibitem{b3} I. S. Jacobs and C. P. Bean, ``Fine particles, thin films and exchange anisotropy,'' in Magnetism, vol. III, G. T. Rado and H. Suhl, Eds. New York: Academic, 1963, pp. 271--350.
%\bibitem{b4} K. Elissa, ``Title of paper if known,'' unpublished.
%\bibitem{b5} R. Nicole, ``Title of paper with only first word capitalized,'' J. Name Stand. Abbrev., in press.
%\bibitem{b6} Y. Yorozu, M. Hirano, K. Oka, and Y. Tagawa, ``Electron spectroscopy studies on magneto-optical media and plastic substrate interface,'' IEEE Transl. J. Magn. Japan, vol. 2, pp. 740--741, August 1987 [Digests 9th Annual Conf. Magnetics Japan, p. 301, 1982].
%\bibitem{b7} M. Young, The Technical Writer's Handbook. Mill Valley, CA: University Science, 1989.
%\end{thebibliography}
%\vspace{12pt}
%\color{red}
%IEEE conference templates contain guidance text for composing and formatting conference papers. Please ensure that all template text is removed from your conference paper prior to submission to the conference. Failure to remove the template text from your paper may result in your paper not being published.

\end{document}
