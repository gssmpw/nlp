\documentclass[conference]{IEEEtran}
\IEEEoverridecommandlockouts
% The preceding line is only needed to identify funding in the first footnote. If that is unneeded, please comment it out.
\usepackage{cite}

\usepackage{amsmath,amssymb,amsfonts}

\usepackage{algorithmicx}

\usepackage{algpseudocode}
\usepackage{tikz}
\usetikzlibrary{fit}
\usetikzlibrary{tikzmark}
\usepackage{graphicx}
\usepackage{wrapfig}
\usepackage{textcomp}
\usepackage{pifont}
\usepackage{xcolor}
\usepackage{caption}
\usepackage{subcaption}
\captionsetup{compatibility=false}
\usepackage{balance} 
\usepackage{xspace}
\usepackage[T1]{fontenc}
\usepackage[scaled=0.81]{beramono}
\usepackage{balance}
\let\endproof\relax
\let\proof\relax


\let\labelitemi\labelitemii
\usepackage{enumitem}  
\usepackage{listings}
\usepackage{url}
\usepackage[ruled,lined,linesnumbered,vlined]{algorithm2e}
\usepackage{multirow}
\usepackage{lscape}
\usepackage{longtable}
\usepackage{array} 
\usepackage{array}
\usepackage{tcolorbox}
\usepackage{color}
%\usepackage[utf8]{inputenc}
\usepackage[T1]{fontenc}
\usepackage[scaled=0.81]{beramono}
%\usepackage{zlmtt}
\usepackage{fancyvrb}
\usepackage{listings}
\usepackage{multicol}
\usepackage{lipsum}
\usepackage{booktabs}
\usepackage{listings}
\usepackage{color}
\usepackage{graphicx}
\usepackage{calc}
\usepackage{amsmath}
\usepackage{upgreek}
\usepackage{float}
\usepackage{mathrsfs}
\usepackage{amssymb}
\usepackage{threeparttable}
\usepackage[framemethod=tikz]{mdframed}
\usepackage{lipsum}
%\usepackage{lstlinebgrd}
\usepackage{float}
\usepackage{amsthm}
\usepackage{pifont}

\usepackage{color}
\usepackage{wasysym}
\usepackage{colortbl}

\usepackage{perpage} %the perpage package
\MakePerPage{footnote} %the perpage package command



\definecolor{verylightgray}{rgb}{.97,.97,.97}

\lstdefinelanguage{Solidity}{
	keywords=[1]{anonymous, assembly, assert, balance, break, call, callcode, case, catch, class, constant, continue, constructor, contract, debugger, default, delegatecall, delete, do, else, emit, event, experimental, export, external, false, finally, for, function, gas, if, implements, import, in, indexed, instanceof, interface, internal, is, length, library, log0, log1, log2, log3, log4, memory, modifier, new, payable, pragma, private, protected, public, pure, push, require, return, returns, revert, selfdestruct, send, solidity, storage, struct, suicide, super, switch, then, this, throw, transfer, true, try, typeof, using, value, view, while, with, addmod, ecrecover, keccak256, mulmod, ripemd160, sha256, sha3}, %
	keywordstyle=[1]\color{blue}\bfseries,
	keywords=[2]{address, bool, byte, bytes, bytes1, bytes2, bytes3, bytes4, bytes5, bytes6, bytes7, bytes8, bytes9, bytes10, bytes11, bytes12, bytes13, bytes14, bytes15, bytes16, bytes17, bytes18, bytes19, bytes20, bytes21, bytes22, bytes23, bytes24, bytes25, bytes26, bytes27, bytes28, bytes29, bytes30, bytes31, bytes32, enum, int, int8, int16, int24, int32, int40, int48, int56, int64, int72, int80, int88, int96, int104, int112, int120, int128, int136, int144, int152, int160, int168, int176, int184, int192, int200, int208, int216, int224, int232, int240, int248, int256, mapping, string, uint, uint8, uint16, uint24, uint32, uint40, uint48, uint56, uint64, uint72, uint80, uint88, uint96, uint104, uint112, uint120, uint128, uint136, uint144, uint152, uint160, uint168, uint176, uint184, uint192, uint200, uint208, uint216, uint224, uint232, uint240, uint248, uint256, var, void, ether, finney, szabo, wei, days, hours, minutes, seconds, weeks, years},	%
	keywordstyle=[2]\color{teal}\bfseries,
	keywords=[3]{block, blockhash, coinbase, difficulty, gaslimit, number, timestamp, msg, data, gas, sender, sig, value, now, tx, gasprice, origin},	%
	keywordstyle=[3]\color{violet}\bfseries,
	identifierstyle=\color{black},
	sensitive=true,
	comment=[l]{//},
	morecomment=[s]{/*}{*/},
	commentstyle=\color{gray}\ttfamily,
	stringstyle=\color{red}\ttfamily,
	morestring=[b]',
	morestring=[b]"
}

\lstset{
	language=Solidity,
	backgroundcolor=\color{verylightgray},
	extendedchars=true,
	basicstyle=\footnotesize\ttfamily,
	showstringspaces=false,
	showspaces=false,
	numbers=none,
	numberstyle=\footnotesize,
	numbersep=9pt,
	tabsize=2,
	breaklines=true,
	showtabs=false,
	captionpos=b
}

\definecolor{darkgreen}{rgb}{0.0, 0.5, 0.13}
\definecolor{C-gray}{gray}{0.85}
\definecolor{B-gray}{gray}{0.65}
\definecolor{A-gray}{gray}{0.95}

\lstset{
	basicstyle=\ttfamily\small,
	%	numbers=left,
	xleftmargin=2em,
	%	numbersep=0.1pt,
	keywordstyle=\color{blue},
	commentstyle=\color{darkgreen},
}


%\usepackage[
%bookmarksopen,
%bookmarksdepth=2,
%breaklinks=true
%] {hyperref}

%~~~~~~~~~~~~~~~~~~~~~~~~~~~~~~~~~~~~~~~~~~~~~~~~~~~~~~~~~~~~~~~~~~~~~~~~~~~~~~
% Macros																{{{1
% Terminology
\newcommand{\tool}{\hbox{\textsc{UAT20}}\xspace}
\newcommand{\bitcoin}{\hbox{Bitcoin}\xspace}
\newcommand{\btc}{\hbox{BTC}\xspace}
\newcommand{\layertwo}{\hbox{Layer2}\xspace}
\newcommand{\defi}{\hbox{DeFi}\xspace}

\newcommand{\ethereum}{\hbox{Ethereum}\xspace}
\newcommand{\imc}{\hbox{UP}\xspace}
\newcommand{\fve}{\hbox{FVE}\xspace}
\newcommand{\cfg}{\hbox{CFG}\xspace}
\newcommand{\solidity}{\hbox{Solidity}\xspace}
\newcommand{\evm}{\hbox{EVM}\xspace}
\newcommand{\argmax}{\operatornamewithlimits{argmax}}
\newcommand{\srt}{\hbox{TR}\xspace}
\newcommand{\crt}{\hbox{DR}\xspace}
\newcommand{\crtm}{\hbox{IM-D}\xspace}
\newcommand{\srtm}{\hbox{IM-T}\xspace}
\newcommand{\srnt}{\hbox{SRNT}\xspace}
\newcommand{\crnt}{\hbox{CRNT}\xspace}
\newcommand{\crtnm}{\hbox{CRTNM}\xspace}
\newcommand{\srtnm}{\hbox{SRTNM}\xspace}
\newcommand{\rts}{\hbox{RTS}\xspace}
\newcommand{\manipulated}{\hbox{MC}\xspace}
\newcommand{\broken}{\hbox{BC}\xspace}
\newcommand{\stealing}{\hbox{SC}\xspace}
\newcommand{\gas}{\hbox{GC}\xspace}
\newcommand{\oyente}{\hbox{\textsf{Oyente}}\xspace}
\newcommand{\securify}{\hbox{\textsf{Securify}}\xspace}
\newcommand{\mythril}{\hbox{\textsf{Mythril}}\xspace}
\newcommand{\zthree}{\hbox{\textsf{Z3}}\xspace}
% Command
\newcommand{\myparagraph}[1]{\vspace*{0.14cm}\noindent\textbf{\emph{#1.}}\quad}
\theoremstyle{definition}
\newtheorem{thm}{Theorem}
\newtheorem{defn}[thm]{Definition}
\newcommand{\specialcell}[2][c]{\begin{tabular}[#1]{@{}c@{}}#2\end{tabular}}
% English
\newcommand{\cf}{\hbox{\emph{cf.}}\xspace}
\newcommand{\deletia}{\ldots [deletia] \ldots}
\newcommand{\etal}{\hbox{\emph{et al.}}\xspace}
\newcommand{\eg}{\hbox{\emph{e.g.}}\xspace}
\newcommand{\ie}{\hbox{\emph{i.e.}}\xspace}
\newcommand{\scil}{\hbox{\emph{sc.}}\xspace} %scilicet: it is 
%permitted to know
%\newcommand{\st}{\hbox{\emph{s.t.}}\xspace}
\newcommand{\wrt}{\hbox{\emph{w.r.t.}}\xspace}
\newcommand{\etc}{\hbox{\emph{etc.}}\xspace}
\newcommand{\viz}{\hbox{\emph{viz.}}\xspace} %videlicet: it is 
%permitted to see
\newcommand{\toolUrl}{\texttt{\url{https://github.com/njaliu/ConFu-release}}} 
\newcommand{\videoUrl}{\texttt{\url{https://youtu.be/AqCSxdXdQds}}}


\begin{document}
\lstset{%
	escapeinside={(*@}{@*)}
}

\title{\textsc{UAT20}: Unifying Liquidity Across Rollups}

%\title{Paper Title*\\
%{\footnotesize \textsuperscript{*}Note: Sub-titles are not captured in Xplore and
%should not be used}
%\thanks{Identify applicable funding agency here. If none, delete this.}
%}

%\author{\IEEEauthorblockN{1\textsuperscript{st} Given Name Surname}
%\IEEEauthorblockA{\textit{dept. name of organization (of Aff.)}\\
%City, Country \\
%email address}
%\and
%\IEEEauthorblockN{2\textsuperscript{nd} Given Name Surname}
%\IEEEauthorblockA{\textit{dept. name of organization (of Aff.)} \\
%City, Country \\
%email address}
%\and
%\IEEEauthorblockN{3\textsuperscript{rd} Given Name Surname}
%\IEEEauthorblockA{\textit{dept. name of organization (of Aff.)} \\
%City, Country \\
%email address}
%\and
%\IEEEauthorblockN{4\textsuperscript{th} Given Name Surname}
%\IEEEauthorblockA{\textit{dept. name of organization (of Aff.)} \\
%City, Country \\
%email address}
%\and
%\IEEEauthorblockN{5\textsuperscript{th} Given Name Surname}
%\IEEEauthorblockA{\textit{dept. name of organization (of Aff.)} \\
%City, Country \\
%email address}
%\and
%\IEEEauthorblockN{6\textsuperscript{th} Given Name Surname}
%\IEEEauthorblockA{\textit{dept. name of organization (of Aff.)} \\
%City, Country \\
%email address}
%}

\author{
	\IEEEauthorblockN{Yue Li \quad Han Liu}
	\IEEEauthorblockA{\textit{UAT Team} \\
	} 
}

\maketitle

\begin{abstract}  
Test time scaling is currently one of the most active research areas that shows promise after training time scaling has reached its limits.
Deep-thinking (DT) models are a class of recurrent models that can perform easy-to-hard generalization by assigning more compute to harder test samples.
However, due to their inability to determine the complexity of a test sample, DT models have to use a large amount of computation for both easy and hard test samples.
Excessive test time computation is wasteful and can cause the ``overthinking'' problem where more test time computation leads to worse results.
In this paper, we introduce a test time training method for determining the optimal amount of computation needed for each sample during test time.
We also propose Conv-LiGRU, a novel recurrent architecture for efficient and robust visual reasoning. 
Extensive experiments demonstrate that Conv-LiGRU is more stable than DT, effectively mitigates the ``overthinking'' phenomenon, and achieves superior accuracy.
\end{abstract}  



\section{Introduction}
\label{sec:intro}
\section{Introduction}


\begin{figure}[t]
\centering
\includegraphics[width=0.6\columnwidth]{figures/evaluation_desiderata_V5.pdf}
\vspace{-0.5cm}
\caption{\systemName is a platform for conducting realistic evaluations of code LLMs, collecting human preferences of coding models with real users, real tasks, and in realistic environments, aimed at addressing the limitations of existing evaluations.
}
\label{fig:motivation}
\end{figure}

\begin{figure*}[t]
\centering
\includegraphics[width=\textwidth]{figures/system_design_v2.png}
\caption{We introduce \systemName, a VSCode extension to collect human preferences of code directly in a developer's IDE. \systemName enables developers to use code completions from various models. The system comprises a) the interface in the user's IDE which presents paired completions to users (left), b) a sampling strategy that picks model pairs to reduce latency (right, top), and c) a prompting scheme that allows diverse LLMs to perform code completions with high fidelity.
Users can select between the top completion (green box) using \texttt{tab} or the bottom completion (blue box) using \texttt{shift+tab}.}
\label{fig:overview}
\end{figure*}

As model capabilities improve, large language models (LLMs) are increasingly integrated into user environments and workflows.
For example, software developers code with AI in integrated developer environments (IDEs)~\citep{peng2023impact}, doctors rely on notes generated through ambient listening~\citep{oberst2024science}, and lawyers consider case evidence identified by electronic discovery systems~\citep{yang2024beyond}.
Increasing deployment of models in productivity tools demands evaluation that more closely reflects real-world circumstances~\citep{hutchinson2022evaluation, saxon2024benchmarks, kapoor2024ai}.
While newer benchmarks and live platforms incorporate human feedback to capture real-world usage, they almost exclusively focus on evaluating LLMs in chat conversations~\citep{zheng2023judging,dubois2023alpacafarm,chiang2024chatbot, kirk2024the}.
Model evaluation must move beyond chat-based interactions and into specialized user environments.



 

In this work, we focus on evaluating LLM-based coding assistants. 
Despite the popularity of these tools---millions of developers use Github Copilot~\citep{Copilot}---existing
evaluations of the coding capabilities of new models exhibit multiple limitations (Figure~\ref{fig:motivation}, bottom).
Traditional ML benchmarks evaluate LLM capabilities by measuring how well a model can complete static, interview-style coding tasks~\citep{chen2021evaluating,austin2021program,jain2024livecodebench, white2024livebench} and lack \emph{real users}. 
User studies recruit real users to evaluate the effectiveness of LLMs as coding assistants, but are often limited to simple programming tasks as opposed to \emph{real tasks}~\citep{vaithilingam2022expectation,ross2023programmer, mozannar2024realhumaneval}.
Recent efforts to collect human feedback such as Chatbot Arena~\citep{chiang2024chatbot} are still removed from a \emph{realistic environment}, resulting in users and data that deviate from typical software development processes.
We introduce \systemName to address these limitations (Figure~\ref{fig:motivation}, top), and we describe our three main contributions below.


\textbf{We deploy \systemName in-the-wild to collect human preferences on code.} 
\systemName is a Visual Studio Code extension, collecting preferences directly in a developer's IDE within their actual workflow (Figure~\ref{fig:overview}).
\systemName provides developers with code completions, akin to the type of support provided by Github Copilot~\citep{Copilot}. 
Over the past 3 months, \systemName has served over~\completions suggestions from 10 state-of-the-art LLMs, 
gathering \sampleCount~votes from \userCount~users.
To collect user preferences,
\systemName presents a novel interface that shows users paired code completions from two different LLMs, which are determined based on a sampling strategy that aims to 
mitigate latency while preserving coverage across model comparisons.
Additionally, we devise a prompting scheme that allows a diverse set of models to perform code completions with high fidelity.
See Section~\ref{sec:system} and Section~\ref{sec:deployment} for details about system design and deployment respectively.



\textbf{We construct a leaderboard of user preferences and find notable differences from existing static benchmarks and human preference leaderboards.}
In general, we observe that smaller models seem to overperform in static benchmarks compared to our leaderboard, while performance among larger models is mixed (Section~\ref{sec:leaderboard_calculation}).
We attribute these differences to the fact that \systemName is exposed to users and tasks that differ drastically from code evaluations in the past. 
Our data spans 103 programming languages and 24 natural languages as well as a variety of real-world applications and code structures, while static benchmarks tend to focus on a specific programming and natural language and task (e.g. coding competition problems).
Additionally, while all of \systemName interactions contain code contexts and the majority involve infilling tasks, a much smaller fraction of Chatbot Arena's coding tasks contain code context, with infilling tasks appearing even more rarely. 
We analyze our data in depth in Section~\ref{subsec:comparison}.



\textbf{We derive new insights into user preferences of code by analyzing \systemName's diverse and distinct data distribution.}
We compare user preferences across different stratifications of input data (e.g., common versus rare languages) and observe which affect observed preferences most (Section~\ref{sec:analysis}).
For example, while user preferences stay relatively consistent across various programming languages, they differ drastically between different task categories (e.g. frontend/backend versus algorithm design).
We also observe variations in user preference due to different features related to code structure 
(e.g., context length and completion patterns).
We open-source \systemName and release a curated subset of code contexts.
Altogether, our results highlight the necessity of model evaluation in realistic and domain-specific settings.






% \section{Background}
% \label{sec:bg}
% \section{Background and Motivation} \label{s:bg}
\subsection{TEE Data Interaction} \label{s:params}
As shown in Fig.~\ref{fig:datacom}, the normal world and TEE are two independent environments, separated to ensure the security of sensitive functions and data. In this architecture, the communication between TEE and the normal world involves three types of parameters: input, output, and shared memory~\cite{s20041090}.

\begin{figure}[t]
    \centering
    \includegraphics[width=0.5\linewidth]{figures/Fig_2.drawio.pdf}
    \caption{Data communication between TEE and the normal world.}
    \label{fig:datacom}
\end{figure}

Input parameters are used to transfer data from the normal side to TEE, while outputs handle the results or send processed data back. Inputs and outputs are simple mechanisms that allow users to temporarily transfer small amounts of data between the normal world and TEE. However, temporary inputs/outputs realize data transfer through memory copying, slightly lowering the performance of TEE applications due to additional memory copy. They are mainly used to transmit lightweight data, such as user commands and data for cryptographic operations.

Shared memory provides a zero-copy memory block to exchange larger data sets (\eg, multimedia files or bulk data) between two sides~\cite{optee}. It allows both sides to access the same memory space efficiently, which avoids frequent memory copying. Shared memory can also remain valid across different TEE invocation sessions, making it suitable for scenarios that require data to be reused multiple times.

Moreover, an SDK is responsible for managing these parameters and communication in the normal side. For example, TrustZone-based OP-TEE uses \texttt{TEEC\_InvokeCommand()} function to execute the in-TEE code, enabling the shared memory or temporary buffers to transfer data between the two sides~\cite{8684292}. 
Then, OP-TEE can handle these interactions through its internal APIs, which process incoming requests, perform secure computations, and return responses to the normal world.
Similarly, Intel SGX utilizes the ECALL and OCALL interfaces to achieve these functionalities~\cite{10632129}. ECALLs allow the normal world to securely invoke functions within the SGX enclave, passing data into the trusted environment for processing, while OCALLs enable the enclave to request services or share results with the untrusted normal world.

Table~\ref{tbl:comp_params} illustrates some differences between input/output parameters and shared memory. It is important to note that, while the normal side cannot directly access the memory copies of input and output parameters in TEE, an attacker with the permissions of the normal world can still tamper with the inputs before they are transmitted to TEE or intercept and read the outputs after they are returned from TEE~\cite{9925569, 10477533}.
Additionally, since shared memory relies on address-based data transfer between the two sides, any modifications made to the data on one side will be instantly mirrored on the other opposing side.
Therefore, data interactions controlled by the normal world code are the root cause of the bad partitioning issues in TEE applications.

\begin{table}[t]
    \caption{Comparison of input/output parameters and shared memory.}
    \label{tbl:comp_params}
    % \renewcommand{\arraystretch}{1.3}
    % \footnotesize
    \setlength{\tabcolsep}{3mm}
    \centering
	\begin{tabular}{lp{4.5cm}p{4.5cm}}
		\toprule
		\textbf{Feature} & \textbf{Input/Output} & \textbf{Shared Memory} \\
		\midrule
            Data Size & Small & Big  \\
            Efficiency & Low, memory copying required & High, no need of additional memory copy\\
            Life-time & Temporary, valid for a single TEE invocation & Valid for a long time and shared among several TEE invocations \\
            Privilege & The normal world cannot reach memory copy in TEE & Both the normal world and TEE can access at the same time \\
		\bottomrule
	\end{tabular}
\end{table}


\section{The \tool Protocol}
\label{sec:protocol}
\subsection*{Introduction}
Welcome, and thank you for participating in our design workshop. Before we begin, I want to ensure that you've read the consent form I sent earlier. Have you had a chance to review it?

\textbf{[Response]}

\noindent Great! Just to reiterate, this session will be recorded for research purposes. Do I have your consent to proceed with the recording?

\textbf{[Response]}

\noindent Thank you. If at any point you feel uncomfortable or wish to stop the interview, please let me know.

\subsection*{General}
Let's start with some general questions about your experiences with 3D gaming and Harry Potter.

\begin{itemize}
    \item Which 3D games do you play and how long have you been playing them?
    \item (Which is your favorite?) Can you describe the 3D elements or the spatial elements in that game a little bit, please?
    \item Have you ever played in VR mode if that is available?
    \item Any AR games you have played? Like Pokémon Go?
    \item What Harry Potter books or movies have you watched?
    \item Which is your favorite?
    \item Which Hogwarts house would you put yourself in?
\end{itemize}

\subsection*{Narrative/Plot}
As mentioned in the consent form, our goal today is to explore innovative ways to improve social media. We particularly invited 3D game players with a love for Harry Potter because we believe this will help us think creatively and ``outside the box'' about what an ideal, magical social media could look like, beyond the confines of a small 2D smartphone screen.

Don't worry about being too creative---we're here to guide you through the design process, not to evaluate you. We want to explore different design directions with your help, and we'll be actively involved while also giving you the space to share your ideas.

To give our brainstorming some structure, we'll start with a specific scenario to solve during this session. I'm going to ask you some guiding questions to help you address the scenario:

\begin{quote}
\textit{Imagine you are a student at Hogwarts. You have friends who are Muggle-born, friends from magical families, and even friends at other wizarding schools around the world. You're also getting to know me, as I'm a new student you've just met at Hogwarts. Keeping in touch with everyone is tough because traditional Muggle social media like Instagram doesn't really portray your wizard self very well. You don't have a smartphone in the first place because Hogwarts people don't need electronic devices, so you can't really charge your smartphone.}\\
\textit{One day, a brilliant Muggle-born student had an idea. They decided to use magic to create a magical social media where everyone---Muggles, Hogwarts students, professors, and friends from other magical schools---could all communicate and share their lives in the most ideal way. Imagine you're using this new magical platform---or maybe it's not even a platform. Whatever it is, what do you think this would look like?}
\end{quote}

\subsection*{Character Map}
Before we dive into the designing part, let's first map out your relationships in your new life as a Hogwarts student. Use the Miro board link (Figure \ref{fig:miro}) to write down the names or nicknames of people that you would like to put in each section. For example, you might want to put ``younger sister'' or ``Jane'' under Family, some of your coworkers under Quidditch Team members, yourself under the house that you think you belong to, your older middle school friends as Muggles, etc.

Please note the two boundaries in the four rectangles for each house. You can place your closer friends in the inner box and not-so-close ones in the outer box. And you can put the same name in multiple boxes (or parallelograms) as you wish.

\begin{itemize}
    \item How do you currently communicate with or stay connected with each of these people via Muggle social media?
    \item What are your biggest pet peeves/issues with Muggle social media that you'd like to solve with magical powers?
    \item What moments do you feel bad/guilty about using social media, if at all?
    \item When does time spent on social media feel meaningless/meaningful/fulfilling, if at all?
    \item Pet peeves around privacy settings?
    \item When do you feel most connected to your friends when interacting with them on social media?
    \item What would a better social media be if you were to use adjectives to describe it? What does “better” mean to you?
\end{itemize}

\subsection*{Design}
Now, let's think about what this magical world social media might look like. Remember, there are no right or wrong answers here. We're looking for your imagination and creativity, so feel free to think outside the box and have fun with it!

\begin{itemize}
    \item What key capabilities would this thing have for communicating with [GROUP]*?
    \item If you could design this thing with any magical powers, what form would it take? (Is it a magical mushroom? A secret room? A creature?)
    \item What would you want to share about yourself on this ideal magical ``thing'' with [GROUP]?
    \item If you had [GROUP'] join this ``thing,'' how might its form adapt to include this new group?
\end{itemize}

*: Iterate for 1) Romantic Relationships and Closest Friends, 2) Closer Hogwarts Friends, 3) Muggle Friends, 4) Friends at Other Magical Schools, 5) Quidditch Team Members, 6) House Elf


\subsection*{Connecting Back}
\begin{itemize}
    \item What do you like about this magical “thing” compared to Muggle social media?
    \item What do you dislike about it?
    \item Imagine using the magical “thing” we discussed earlier. How do you think it could resolve some of the issues people might face with Muggle social media?
    \item How might this new magical “thing” better support your needs for self-presentation and identity management compared to traditional social media?
    \item What challenges might arise in using such a “thing” at Hogwarts?
\end{itemize}

Thank you so much for your time and insights. Your contributions are incredibly valuable to our research. Do you have any final thoughts or questions before we conclude?


% \section{Security Analysis}
% \begin{figure*}[!htb]
\begin{tikzpicture}
\node[draw,thick,inner sep=10pt] (box) {
  \begin{minipage}{0.96\textwidth} % Adjust width to fit your content    
        \vspace{-5pt} 
        \begin{multicols}{2}
  	% \begin{multicols}{1}  % Begin the two-column environment
  		% Now, format the text as it appears in the 'picture'
        % \overview{Parties: Users {$U_1,...,U_m$}, Certificate Authority $A$, Service Providers {$P_1,...,P_l$}, and Access Controllers {$C_1,...,C_n$}.}
        % \overview{Parameters: Access policy \(\phi\), issuance criteria \(\ell\), service to access data, computation and authorization $S_{data}, S_{comp}, S_{auth}$, and secret $s$.}

        \vspace{5pt} 
        
        \func{Setup}{$1^\lambda$}
        \step{1. Let $\mathcal{R}$ be a circuit with public input $\textsf{cred}$ and $\textsf{EI}$ asserting: \\ $\textsf{EI} = \textsf{Ec}(e, \textsf{attrs}) \land \textsf{cred}$ opens to $(\textsf{nk,attrs})$}
        \step{2. Compute $\textsf{crs}_\mathcal{R} := \textsf{G16.Setup}(1^\lambda, \mathcal{R})$}
        \step{3. Let $\textsf{pp} := (1^\lambda, \textsf{crs}_\mathcal{R})$ and return $\textsf{pp}$}

        \vspace{5pt} 

        \func{{IssueReq_\textit{U}}}{\textsf{pp}, \textsf{attrs}}
        \step{1. Sample $r_\textsf{cred}, \textsf{nk}$ // commitment nonce, pseudonym key}
        \step{2. Commit $\textsf{cred} := \textsf{Com}(\textsf{nk}, \textsf{attrs}; r_\textsf{cred})$}
        \step{3. Let $\omega := (\textsf{nk}, \textsf{attrs}, r_\textsf{cred})$}
        \step{4. Send ($\omega$, $\textsf{cred}$) to $A$ and receive a proof $\theta$ attesting to $\textsf{cred} \in \textsf{MT}'$}

        \vspace{5pt} 

        \func{{IssueGrant_\textit{A}}}{$\textsf{pp}, \textsf{MT}, \textsf{cred}, \omega$}
        \step{1. Check $\textsf{cred} = \textsf{Com}(\textsf{nk,attrs}; r_\textsf{cred})$}
        \step{2. Let $\textsf{MT}' := \textsf{MT}.\textsf{Add}(\textsf{pk}^{U}, \textsf{cred})$ and $\gamma := \textsf{MT}'.\textsf{Root}$}
        \step{3. Let $\theta := \textsf{MT}'.\textsf{Prove}(\textsf{pk}^{U})$ // $\theta$ attests to $\textsf{cred} \in \textsf{MT}'$}
        \step{4. Send \(\theta\) to \(U\) and publish $\gamma$}

        \vspace{5pt} 

        % Service provider setup for (service, policy, secret, share, time-bound) core part!!!
        \func{{SetPolicy_\textit{P}}}{$\textsf{pp}, \textsf{S}, (f_\phi, param_\phi), T_1, T_2$}
        \step{1. Sample $s$ // secret}
        \step{2. Let $(\textsf{F}_\phi, (e^U, e^P), d) := \textsf{Gb}(1^\lambda, f_\phi)$}
        \step{3. Let $\textsf{EP} := \textsf{Ec}(e^P, param_\phi)$}
        % \step{3. Commit $\textsf{pol} := \textsf{Com}(\textsf{pk}^P, \phi; r_\textsf{pol})$}
        % \step{4. Let $\omega_P := (\phi, e, r_\textsf{pol})$}
        \step{4. Let $\textsf{pp}_\textsf{PVTSS}$ := $\textsf{PVTSS.Setup}(1^\lambda, T_1, T_2)$}
        \step{5. Let $(\{c_i\}_{i\in[n]}, \pi_D) := \textsf{PVTSS.Sharing}(\textsf{pp}_\textsf{PVTSS}, s, \{\textsf{pk}^{C_i}\}_{i\in[n]})$ // locked encrypted shares $\{c_i\}_{i\in[n]}$ with time parameter $T_1$}
        % \step{7. Publish ($\textsf{pol}, \textsf{S}$)}
        \step{6. Send ($\textsf{S}, \textsf{F}_\phi, \textsf{EP}, d, \{c_i\}_{i\in[n]}, \pi_D$) to each party $C_i \in C$}

        \columnbreak
        
        % Access Controllers receive shares and verify
        \func{{VerifyShare_\textit{C}}}{$\textsf{pp}, \textsf{pp}_\textsf{PVTSS}, \{c_i\}_{i\in[n]}, \pi_D, \textsf{F}_\phi, d$}
        \step{1. Check $\textsf{PVTSS.Verify1}(\textsf{pp}_\textsf{PVTSS}, \{\textsf{pk}^{C_i}, c_i\}_{i\in[n]}, \pi_D)$}
        \step{2. Let $\{\hat{s}_i, \pi_i\} := \textsf{PVTSS.Recover}(\textsf{pp}_\textsf{PVTSS}, c_i, \textsf{pk}^{C_i}, \textsf{sk}^{C_i})$ 
        // recover the decrypted share $\hat{s}_i$ together with proof $\pi_i$ of valid decryption}
        \step{3. Store ($\textsf{F}_\phi, d, \hat{s}_i, \pi_i$)}

        \vspace{5pt} 

        \func{{Encode_\textit{U}}}{$\textsf{pp}, \textsf{cred}, \textsf{attrs}, \textsf{S}, \omega$}
        % \step{2. Let ($r, \textsf{attrs}$) := ${\textsf{UserCreds}_\textit{U}}[\textsf{cred}]$}
        % \step{3. If \({\phi}_{vsb}\) is $\textsf{public}$: let ${\textsf{ShowProofs}_\textit{U}}[\textsf{sn}]$ := ($\phi, \textsf{cred}, \theta, r, \textsf{aux}$)}
        % \step{1. \(U\), \(P\) and \(C\) negotiate the secure MPC function $\textsf{Ec}(X, Y)$}
        % \step{2. Public input $(\textsf{cred,pol}, \gamma)$}
        % \step{3. Private input $\omega_P, \omega_U$ by $P$ and $U$}
        % \step{4. Only $C$ receive the initial input $\textsf{EI} := \textsf{Ec}(e, \textsf{attrs})$ if $\textsf{cred}$ opens to $(\textsf{nk,attrs}) \land \textsf{cred} \in \textsf{MT}$}
        \step{1. Request encoding information $e^U$ form $P$ and let $\textsf{EI} := \textsf{Ec}(e^U, \textsf{attrs})$}
        \step{2. Let $\pi_\mathcal{R} := \textsf{G16.Prove}(\textsf{crs}_\mathcal{R}, (\textsf{cred}, \textsf{EI}), (\omega, e^U))$}
        \step{3. Send ($\textsf{EI}, \textsf{cred}, \pi_\mathcal{R}, \textsf{S}$) to each party $C_i \in C$}

        \vspace{5pt} 
        
        \func{{Authenticate_\textit{C}}}{$\textsf{pp}, \textsf{F}_\phi, \textsf{cred}, \textsf{EI}, \textsf{EP}, \textsf{S}, d, \textsf{MT}, \gamma, \theta, \hat{s}_i, \pi_i$}
        \step{1. Check $\textsf{G16.Verify}(\textsf{crs}_\mathcal{R}, \pi_\mathcal{R}, (\textsf{cred}, \textsf{EI})) \land \textsf{MT}.\textsf{Verify}(\gamma, \textsf{pk}^{U}, \textsf{cred}, \theta)$}
        \step{2. Let encoded output $\textsf{EO} := \textsf{Ev}(\textsf{F}_\phi, (\textsf{EI}, \textsf{EP}))$}
        \step{3. Check the final output $\textsf{FO} := \textsf{De}(\textsf{EO}, d)$}
        \step{4. If check pass and over $T_1$, send ($\textsf{S}, c_i, \hat{s}_i, \pi_i$) to $U$}

        \vspace{5pt}

        \func{{AccessService_\textit{U}}}{$\textsf{pp}, \textsf{pp}_\textsf{PVTSS}, \textsf{S}, \{c_i\}, \{\hat{s}_i\}, \{\pi_i\}$}
        \step{1. Check $\textsf{PVTSS.Verify2}(\textsf{pp}_\textsf{PVTSS}, c_i, \hat{s}_i, \pi_i)$}
        \step{2. Add the valid $\hat{s}_i$ to a set $\mathcal{S}$}
        \step{3. If $|\mathcal{S}| > t$ and $T_2$ has not elapsed:\\ let $s := \textsf{PVTSS.Pool}(\textsf{pp}_\textsf{PVTSS}, \mathcal{S}, T_2)$ and $\textsf{Result} := \textsf{S}(s)$}

        \vspace{5pt} 

        \func{Revoke_\textit{A}}{$\textsf{pp}, \textsf{MT}, \textsf{pk}^{U}$}
        \step{1. Let $\textsf{MT}' := \textsf{MT}.\textsf{Remove}(\textsf{pk}^{U})$}
        \step{2. Return $\textsf{MT}'$}
    \end{multicols}
    \end{minipage}
};


\end{tikzpicture}
\caption{\tool construction.} %NB: Each participant maintains a list of public keys.
\label{fig:construction}
\end{figure*}

\section{Empirical Study}
\label{sec:discussion}
\section{Discussion of Assumptions}\label{sec:discussion}
In this paper, we have made several assumptions for the sake of clarity and simplicity. In this section, we discuss the rationale behind these assumptions, the extent to which these assumptions hold in practice, and the consequences for our protocol when these assumptions hold.

\subsection{Assumptions on the Demand}

There are two simplifying assumptions we make about the demand. First, we assume the demand at any time is relatively small compared to the channel capacities. Second, we take the demand to be constant over time. We elaborate upon both these points below.

\paragraph{Small demands} The assumption that demands are small relative to channel capacities is made precise in \eqref{eq:large_capacity_assumption}. This assumption simplifies two major aspects of our protocol. First, it largely removes congestion from consideration. In \eqref{eq:primal_problem}, there is no constraint ensuring that total flow in both directions stays below capacity--this is always met. Consequently, there is no Lagrange multiplier for congestion and no congestion pricing; only imbalance penalties apply. In contrast, protocols in \cite{sivaraman2020high, varma2021throughput, wang2024fence} include congestion fees due to explicit congestion constraints. Second, the bound \eqref{eq:large_capacity_assumption} ensures that as long as channels remain balanced, the network can always meet demand, no matter how the demand is routed. Since channels can rebalance when necessary, they never drop transactions. This allows prices and flows to adjust as per the equations in \eqref{eq:algorithm}, which makes it easier to prove the protocol's convergence guarantees. This also preserves the key property that a channel's price remains proportional to net money flow through it.

In practice, payment channel networks are used most often for micro-payments, for which on-chain transactions are prohibitively expensive; large transactions typically take place directly on the blockchain. For example, according to \cite{river2023lightning}, the average channel capacity is roughly $0.1$ BTC ($5,000$ BTC distributed over $50,000$ channels), while the average transaction amount is less than $0.0004$ BTC ($44.7k$ satoshis). Thus, the small demand assumption is not too unrealistic. Additionally, the occasional large transaction can be treated as a sequence of smaller transactions by breaking it into packets and executing each packet serially (as done by \cite{sivaraman2020high}).
Lastly, a good path discovery process that favors large capacity channels over small capacity ones can help ensure that the bound in \eqref{eq:large_capacity_assumption} holds.

\paragraph{Constant demands} 
In this work, we assume that any transacting pair of nodes have a steady transaction demand between them (see Section \ref{sec:transaction_requests}). Making this assumption is necessary to obtain the kind of guarantees that we have presented in this paper. Unless the demand is steady, it is unreasonable to expect that the flows converge to a steady value. Weaker assumptions on the demand lead to weaker guarantees. For example, with the more general setting of stochastic, but i.i.d. demand between any two nodes, \cite{varma2021throughput} shows that the channel queue lengths are bounded in expectation. If the demand can be arbitrary, then it is very hard to get any meaningful performance guarantees; \cite{wang2024fence} shows that even for a single bidirectional channel, the competitive ratio is infinite. Indeed, because a PCN is a decentralized system and decisions must be made based on local information alone, it is difficult for the network to find the optimal detailed balance flow at every time step with a time-varying demand.  With a steady demand, the network can discover the optimal flows in a reasonably short time, as our work shows.

We view the constant demand assumption as an approximation for a more general demand process that could be piece-wise constant, stochastic, or both (see simulations in Figure \ref{fig:five_nodes_variable_demand}).
We believe it should be possible to merge ideas from our work and \cite{varma2021throughput} to provide guarantees in a setting with random demands with arbitrary means. We leave this for future work. In addition, our work suggests that a reasonable method of handling stochastic demands is to queue the transaction requests \textit{at the source node} itself. This queuing action should be viewed in conjunction with flow-control. Indeed, a temporarily high unidirectional demand would raise prices for the sender, incentivizing the sender to stop sending the transactions. If the sender queues the transactions, they can send them later when prices drop. This form of queuing does not require any overhaul of the basic PCN infrastructure and is therefore simpler to implement than per-channel queues as suggested by \cite{sivaraman2020high} and \cite{varma2021throughput}.

\subsection{The Incentive of Channels}
The actions of the channels as prescribed by the DEBT control protocol can be summarized as follows. Channels adjust their prices in proportion to the net flow through them. They rebalance themselves whenever necessary and execute any transaction request that has been made of them. We discuss both these aspects below.

\paragraph{On Prices}
In this work, the exclusive role of channel prices is to ensure that the flows through each channel remains balanced. In practice, it would be important to include other components in a channel's price/fee as well: a congestion price  and an incentive price. The congestion price, as suggested by \cite{varma2021throughput}, would depend on the total flow of transactions through the channel, and would incentivize nodes to balance the load over different paths. The incentive price, which is commonly used in practice \cite{river2023lightning}, is necessary to provide channels with an incentive to serve as an intermediary for different channels. In practice, we expect both these components to be smaller than the imbalance price. Consequently, we expect the behavior of our protocol to be similar to our theoretical results even with these additional prices.

A key aspect of our protocol is that channel fees are allowed to be negative. Although the original Lightning network whitepaper \cite{poon2016bitcoin} suggests that negative channel prices may be a good solution to promote rebalancing, the idea of negative prices in not very popular in the literature. To our knowledge, the only prior work with this feature is \cite{varma2021throughput}. Indeed, in papers such as \cite{van2021merchant} and \cite{wang2024fence}, the price function is explicitly modified such that the channel price is never negative. The results of our paper show the benefits of negative prices. For one, in steady state, equal flows in both directions ensure that a channel doesn't loose any money (the other price components mentioned above ensure that the channel will only gain money). More importantly, negative prices are important to ensure that the protocol selectively stifles acyclic flows while allowing circulations to flow. Indeed, in the example of Section \ref{sec:flow_control_example}, the flows between nodes $A$ and $C$ are left on only because the large positive price over one channel is canceled by the corresponding negative price over the other channel, leading to a net zero price.

Lastly, observe that in the DEBT control protocol, the price charged by a channel does not depend on its capacity. This is a natural consequence of the price being the Lagrange multiplier for the net-zero flow constraint, which also does not depend on the channel capacity. In contrast, in many other works, the imbalance price is normalized by the channel capacity \cite{ren2018optimal, lin2020funds, wang2024fence}; this is shown to work well in practice. The rationale for such a price structure is explained well in \cite{wang2024fence}, where this fee is derived with the aim of always maintaining some balance (liquidity) at each end of every channel. This is a reasonable aim if a channel is to never rebalance itself; the experiments of the aforementioned papers are conducted in such a regime. In this work, however, we allow the channels to rebalance themselves a few times in order to settle on a detailed balance flow. This is because our focus is on the long-term steady state performance of the protocol. This difference in perspective also shows up in how the price depends on the channel imbalance. \cite{lin2020funds} and \cite{wang2024fence} advocate for strictly convex prices whereas this work and \cite{varma2021throughput} propose linear prices.

\paragraph{On Rebalancing} 
Recall that the DEBT control protocol ensures that the flows in the network converge to a detailed balance flow, which can be sustained perpetually without any rebalancing. However, during the transient phase (before convergence), channels may have to perform on-chain rebalancing a few times. Since rebalancing is an expensive operation, it is worthwhile discussing methods by which channels can reduce the extent of rebalancing. One option for the channels to reduce the extent of rebalancing is to increase their capacity; however, this comes at the cost of locking in more capital. Each channel can decide for itself the optimum amount of capital to lock in. Another option, which we discuss in Section \ref{sec:five_node}, is for channels to increase the rate $\gamma$ at which they adjust prices. 

Ultimately, whether or not it is beneficial for a channel to rebalance depends on the time-horizon under consideration. Our protocol is based on the assumption that the demand remains steady for a long period of time. If this is indeed the case, it would be worthwhile for a channel to rebalance itself as it can make up this cost through the incentive fees gained from the flow of transactions through it in steady state. If a channel chooses not to rebalance itself, however, there is a risk of being trapped in a deadlock, which is suboptimal for not only the nodes but also the channel.

\section{Conclusion}
This work presents DEBT control: a protocol for payment channel networks that uses source routing and flow control based on channel prices. The protocol is derived by posing a network utility maximization problem and analyzing its dual minimization. It is shown that under steady demands, the protocol guides the network to an optimal, sustainable point. Simulations show its robustness to demand variations. The work demonstrates that simple protocols with strong theoretical guarantees are possible for PCNs and we hope it inspires further theoretical research in this direction.

% \section{Related Work}
% \label{sec:rw}
% Our work draws heavily from the literature on semiparametric inference and double machine learning~\citep{robins1994estimation,robins1995semiparametric,tsiatis2006semiparametric,chernozhukov2018double}. In particular, our estimator is an optimal combination of several Augmented Inverse Probability Weighting~(\aipw) estimators, whose outcome regressions are replaced with foundation models. Importantly, the standard $\aipw$ estimator, which relies on an outcome regression estimated using experimental data alone, is also included in the combination. This approach allows \ours~to significantly reduce finite sample (and potentially asymptotic) variance while attaining the semiparametric \emph{efficiency bound}---the smallest asymptotic variance among all consistent and asymptotically normal estimators of the average treatment effect---even when the foundation models are arbitrarily biased.


\paragraph{Integrating foundation models}
Prediction-powered inference~(\ppi)~\citep{angelopoulos2023prediction} is a statistical framework that constructs valid confidence intervals using a small labeled dataset and a large unlabeled dataset imputed by a foundation model. $\ppi$ has been applied in various domains, including generalization of causal inferences~\citep{demirel24prediction}, large language model evaluation~\citep{fisch2024stratified,dorner2024limitsscalableevaluationfrontier}, and improving the efficiency of social science experiments~\citep{broskamixed,egami2024using}. However, unlike our approach, $\ppi$ requires access to an additional unlabeled dataset from the same distribution as the experimental sample, which may be as costly as labeled data. Recent work by \citet{poulet2025prediction} introduces 
Prediction-powered inference for clinical trials ($\ppct$), an adaptation of $\ppi$ to estimate  average treatment effects in randomized experiments without any additional  external data. $\ppct$ combines the difference in means estimator with an 
$\aipw$ estimator that integrates the same foundation model as the outcome regression for both treatment and control groups. However, our work differs in two key aspects:
(i) $\ppct$ integrates a single foundation model, and (ii) $\ppct$ does not include the standard $\aipw$ estimator with the outcome regression estimated from experimental data. As a result, $\ppct$ cannot achieve the efficiency bound unless the foundation model is almost surely equal to the underlying outcome regression. 


 



\paragraph{Integrating observational data} There is growing interest in augmenting randomized experiments with data from observational studies to improve statistical precision. One approach involves first testing whether the observational data is compatible with the experimental data~\citep{dahabreh2024using}---for instance, using a statistical test to assess if the mean of the outcome conditional on the covariates is invariant across studies \cite{luedtke2019omnibus,hussain2023falsification,de2024detecting}—and then combining the datasets to improve precision, if the test does not reject. These tests, however, have low statistical power, especially when the experimental sample size is small, which is precisely when leveraging observational data would be most beneficial. To overcome this, a recent line of work integrates a prognostic score estimated from observational data as a covariate when estimating the outcome regression~\citep{schuler2022increasing,liao2023prognostic}. However, increasing the dimensionality of the problem---by adding an additional covariate---can increase estimation error and inflate the finite sample variance. Finally, the work most closely related to ours is \citet{karlsson2024robust}, that integrates an outcome regression estimated from observational data into the \aipw~estimator. In contrast, our approach is not constrained by the availability of well-structured observational data, since it leverages black-box foundation models trained on external data sources.

\section{Conclusion}
\label{sec:conclusion}
\section{Conclusion}
In this work, we propose a simple yet effective approach, called SMILE, for graph few-shot learning with fewer tasks. Specifically, we introduce a novel dual-level mixup strategy, including within-task and across-task mixup, for enriching the diversity of nodes within each task and the diversity of tasks. Also, we incorporate the degree-based prior information to learn expressive node embeddings. Theoretically, we prove that SMILE effectively enhances the model's generalization performance. Empirically, we conduct extensive experiments on multiple benchmarks and the results suggest that SMILE significantly outperforms other baselines, including both in-domain and cross-domain few-shot settings.







%\section*{Acknowledgment}
%
%The preferred spelling of the word ``acknowledgment'' in America is without 
%an ``e'' after the ``g''. Avoid the stilted expression ``one of us (R. B. 
%G.) thanks $\ldots$''. Instead, try ``R. B. G. thanks$\ldots$''. Put sponsor 
%acknowledgments in the unnumbered footnote on the first page.

%\section*{References}
%
%Please number citations consecutively within brackets \cite{b1}. The 
%sentence punctuation follows the bracket \cite{b2}. Refer simply to the reference 
%number, as in \cite{b3}---do not use ``Ref. \cite{b3}'' or ``reference \cite{b3}'' except at 
%the beginning of a sentence: ``Reference \cite{b3} was the first $\ldots$''
%
%Number footnotes separately in superscripts. Place the actual footnote at 
%the bottom of the column in which it was cited. Do not put footnotes in the 
%abstract or reference list. Use letters for table footnotes.
%
%Unless there are six authors or more give all authors' names; do not use 
%``et al.''. Papers that have not been published, even if they have been 
%submitted for publication, should be cited as ``unpublished'' \cite{b4}. Papers 
%that have been accepted for publication should be cited as ``in press'' \cite{b5}. 
%Capitalize only the first word in a paper title, except for proper nouns and 
%element symbols.
%
%For papers published in translation journals, please give the English 
%citation first, followed by the original foreign-language citation \cite{b6}.

\bibliographystyle{IEEEtran}
\bibliography{ethereum,pl} 

%\begin{thebibliography}{00}
%\bibitem{b1} G. Eason, B. Noble, and I. N. Sneddon, ``On certain integrals of Lipschitz-Hankel type involving products of Bessel functions,'' Phil. Trans. Roy. Soc. London, vol. A247, pp. 529--551, April 1955.
%\bibitem{b2} J. Clerk Maxwell, A Treatise on Electricity and Magnetism, 3rd ed., vol. 2. Oxford: Clarendon, 1892, pp.68--73.
%\bibitem{b3} I. S. Jacobs and C. P. Bean, ``Fine particles, thin films and exchange anisotropy,'' in Magnetism, vol. III, G. T. Rado and H. Suhl, Eds. New York: Academic, 1963, pp. 271--350.
%\bibitem{b4} K. Elissa, ``Title of paper if known,'' unpublished.
%\bibitem{b5} R. Nicole, ``Title of paper with only first word capitalized,'' J. Name Stand. Abbrev., in press.
%\bibitem{b6} Y. Yorozu, M. Hirano, K. Oka, and Y. Tagawa, ``Electron spectroscopy studies on magneto-optical media and plastic substrate interface,'' IEEE Transl. J. Magn. Japan, vol. 2, pp. 740--741, August 1987 [Digests 9th Annual Conf. Magnetics Japan, p. 301, 1982].
%\bibitem{b7} M. Young, The Technical Writer's Handbook. Mill Valley, CA: University Science, 1989.
%\end{thebibliography}
%\vspace{12pt}
%\color{red}
%IEEE conference templates contain guidance text for composing and formatting conference papers. Please ensure that all template text is removed from your conference paper prior to submission to the conference. Failure to remove the template text from your paper may result in your paper not being published.

\end{document}
