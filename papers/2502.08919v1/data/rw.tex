% Recent research has explored different ways to improve the scalability of Bitcoin. Payment channels are the first proposed solution to enable fast Bitcoin exchange between parties via off-chain transactions. Instead of conducting each transaction on-chain, they only settle the final payment balances with Bitcoin\cite{malavolta2017concurrency,dziembowski2018general,dziembowski2019perun}. Sidechain is another popular scaling solution for Bitcoin, which involves creating a separate public blockchain and linking it to Bitcoin through specific cross-chain bridges\cite{gavzi2019proof}.
% Liquid network is a consortium chain with multi-parties that allows users to lock BTC on the mainnet and then mint an equivalent amount of wrapped BTC in the sidechain (\ie, peg-in)\cite{nick2020liquid}. 
% Meanwhile, the user can redeem bitcoin on the main chain if holding the multi-signature agreement from these parties (\ie, peg-out).
% Rootstock\cite{lerner2022rsk} and Drivechain\cite{drivechain} are similar to Liquid network, but they allow merged mining which enables Bitcoin’s existing miners to effectively secure other blockchains.
% Babylon\cite{tas2023bitcoin} is another project that will enable BTC as a “staking” assets to secure other blockchains.

% To add contract execution capabilities to Bitcoin without undermining Bitcoin’s security, a set of layer 2 protocols built on the top of Bitcoin is proposed.
% Stacks\cite{ali2020stacks} is a layer 2 protocol that proposes an innovative Proof of Transfer (PoX) consensus protocol to support smart contracts that directly interact with Bitcoin state and transactions.
% The $B^2$ Network\cite{b2network} is a zkRollup Layer2 with Bitcoin as the settlement layer and DA layer. Users' transactions are first submitted and processed in the Rollup Layer, which uses a zkEVM scheme to execute user transactions and output the associated proofs. The batch transactions and generated zero-knowledge proofs are forwarded to the bitcoin for validation.
% The CKB is another Bitcoin Layer2 project that aims to leverage the security and decentralization benefits of Bitcoin's PoW consensus mechanism while introducing a more scalable and flexible UTXO model for handling transactions\cite{ckb}. BitVM\cite{bitvm} introduces a protocol that makes any computation verifiable on Bitcoin and leverages Bitcoin scripts and Taproot to implement optimistic Rollups.
% Based on the computing paradigm of BitVM, Citrea\cite{citrea}, ZKbytes\cite{ZKByte}, and Bitlayer\cite{bitlayer} are proposed to process a large number of complex transaction batches off-chain and submit a succinct zero-knowledge proof on-chain that allows easy verification of the batch validity in Bitcoin. 

