Ethereum, as one of the earliest smart contract platforms, continues to play a pivotal role in the evolving multi-chain ecosystem\cite{Wood2014Ethereum}. The introduction of rollup-based scaling solutions, such as Arbitrum and Optimism, paved the way for designing Ethereum Layer\cite{kalodner2018arbitrum, donno2022optimistic, scroll,zksync}. Furthermore, the modular and replicable nature of underlying technology has made it possible to create new rollups quickly.
For example, over 50 chains have been launched based on the OP Stack's Superchain\cite{superchain}. These rollups enhance Ethereum's scalability and expand the development landscape, further enriching Ethereum's ecosystem. However, this rapid proliferation of rollups has also introduced issues, with \emph{liquidity fragmentation} emerging as a significant concern. Specifically, 
%However, this rapid growth has also brought about the issue of liquidity fragmentation.

\myparagraph{Liquidity Fragmentation} refers to the fragmentation of available liquidity for a particular asset across multiple chains. This fragmentation can lead to i) inefficiencies in trading and lending activities in a specific chain; ii) asset price inefficiencies and increased volatility, even small trades can significantly impact asset prices\cite{lehar2023liquidity}.
In \ethereum's rollup ecosystem, asset issuers tend to deploy their token on multiple rollups to maximize economic opportunities and attract a broader user base. Moreover, cross-chain bridges also enable users to transfer their assets among rollups\cite{whinfrey2021hop, xie2022zkbridge, lan2021horizon,zetachain,chainflip,Interlay}.
Thus, the liquidity fragmentation has emerged as a significant challenge within Ethereum's rollup ecosystem.

Currently, users often rely on cross-chain bridges to manually transfer their liquidity, which not only increases the complexity and cost of participating in DeFi but also introduces additional security risks\cite{lehar2023liquidity, augusto2024sok}.
Moreover, a range of DEX aggregators\cite{1inch,swoopexchange,matcha} has been proposed to provide a seamless trading experience. The underlying technologies enabling these aggregators typically involve cross-chain bridges and token pre-authorization mechanisms. However, these works have the following limitations: (i) only limited transaction scenarios are supported, such as cross-chain token swaps; (ii) a third-party authorization is required, which also introduces additional security risks.


% An increasing number of token issuers are deploying their tokens on multiple blockchains to maximize economic opportunities and attract a broader user base. 

%   Consequently, Ethereum's liquidity is dispersed, such fragmentation not only reduces the efficient utilization of liquidity (\eg, participating in trading or lending with a minimum token requirement in a rollup) but also increases the complexity for users when trading across different platforms, as pointed out by Vitalik Buterin.
% More specifically, the fragmentation of user assets constrains trading and lending activities on individual chains, thereby limiting access to larger markets.


%Decentralized exchange (DEX) aggregators mitigate this problem by facilitating multi-chain interactions through cross-chain bridges, enabling users to seamlessly transfer assets across multiple chains (\eg, multiple rollups) via pre-authorized operations.


% To tackle liquidity fragmentation between rollups, a set of DEX aggregators that provide a seamless trading experience and enhance the liquidity options available to users are proposed.
% The underlying technologies of DEX aggregators are commonly cross-chain bridges and asset pre-authorization. Hence, these works have the following limitations: (i) only limited transaction scenarios are supported, such as cross-chain asset swaps; (ii) a third-party authorization is required, which introduces additional security risks.


% We argue that Conflict-free Replicated Data Types (CRDT) are a promising approach 



% As the blockchain ecosystem expands, liquidity becomes increasingly fragmented across multiple chains due to a lack of interoperability which hinders efficient trading and asset management. Liquidity fragmentation poses a significant challenge in the industry by impeding the optimal utilization of liquidity, resulting in volatility and inefficient market-making. 

% While the \bitcoin blockchain is robust against security attacks, the scalability of the 
% network (\ie, throughput, latency) is largely limited by the way consensus is formed 
% through a large collection of peer-to-peer nodes. With the current block size and 
% creation frequency, the average number of transactions \bitcoin can handle 
% is roughly between 3.3 and 7, making the cryptocurrency infeasible in many throughput-critical 
% use cases. Under such circumstance, scaling solutions for \bitcoin have long been a fundamental 
% desire for the ecosystem, \ie, improve the capacity of the network. Representative proposals include 
% increasing the block size, payment channels, sidechain \etc. In recent years, the idea of 
% building \layertwo on \bitcoin to scale the network becomes increasingly popular and accepted. 
% As the main scaling direction for the \ethereum ecosystem\cite{Wood2014Ethereum}, \layertwo refers to off-chain networks 
% whose transactions are verified and confirmed by the underlying host-blockchain to enable a much 
% greater level of network capabilities\cite{}, \eg, The \layertwo Arbitrum\cite{kalodner2018arbitrum} claims to handle 
% around 40,000 transactions per second for \ethereum (currently around 15 transactions per second). 
% However, in the case of \bitcoin which also suffers from limited scalability, the problem of scaling is 
% much more challenging. The main reason for the difference comes from the 
% fact that \bitcoin does not provide turing-complete programmability as \ethereum therefore can 
% hardly natively verify the transactions submitted from a \layertwo. We are aware of the recent 
% idea of BitVM\cite{bitvm} to introduce full programmability on \bitcoin and choose not to consider it as a 
% mature solution in this paper since it is still at a very early stage. In the majority of existing 
% \bitcoin \layertwo projects, they propose to verify transactions and commit new states through 
% an independent set of nodes rather than \bitcoin miners hence do not deliver the same security 
% guarantee as \bitcoin. On the other hand, a naive inheritance of \bitcoin security would expectedly 
% degrade the scalability of a \layertwo to a large extent and accordingly the necessity of it as well.

\myparagraph{The Insight of \tool} 
% Our goal in this paper is to mitigate liquidity fragmentation across Rollups without introducing the above limitations. Specifically, we proposed a universal abstract token standard called UAT20, which always brings together the user's ERC20 token from multiple rollups.
% We utilize Conflict-free Replicated Data Types (CRDTs), \ie, abstract data types that converge to
%  the same state in a distributed environment, to create the data structure in UAT20.
% Specifically, the UAT20 token serves as a smart contract on each rollup, allowing users on any rollup to access the total amount of ERC 20 tokens in all Rollups.
% To enable users to operate their UAT20 on any rollup, we design a coordinator on layer 1 that infers the commutative operations from these rollups and updates the state on each rollup. 
The goal of this paper is to mitigate liquidity fragmentation across rollups without introducing the aforementioned limitations. We propose leveraging Conflict-free Replicated Data Types (CRDTs)—abstract data types that converge to the same state in distributed environments\cite{preguicca2018conflict,enes2019efficient}—as a promising solution. Specifically, we introduce a universal abstract token standard, UAT20, which unifies a user's ERC20 tokens across multiple rollups. The CRDT forms the foundation of the UAT20 data structure, ensuring consistency across multiple rollups. To enable seamless operation of UAT20 tokens on any rollup, we design a two-phase commit protocol to resolve conflicts arising from token transactions across rollups. This protocol ensures that 
unified liquidity is updated consistently across all rollups, preserving convergence and reliability within Rollups.



% To mitigate the aforementioned problem, \ie, balance the tradeoff between security and scalability 
% for \bitcoin \layertwo, we introduce the idea of \emph{Trust-Minimized} \bitcoin \layertwo through 
% the \tool protocol. The key insight is a modular security pegging mechanism that allows a \layertwo 
% to inherit full security guarantee from \bitcoin for its fundamental assets but still achieve 
% high scalability for the rest of it. More specifically, this is realized via extending \bitcoin 
% (without any hardfork) with \ethereum alike programmability to manage core assets (\eg, popular 
% stablecoins, gas tokens \etc) and submit all the \layertwo transactions to \bitcoin. 
% Key capabilities of the extension are to offer the exactly same safety and liveness 
% as the underlying \bitcoin. The scalability of \layertwo is little compromised to enable 
% large-scale applications, \eg, \defi, gaming \etc. Any potential security attack on one or 
% multiple \layertwo would not cause the loss of fundamental assets since they are secured 
% in the extension by \bitcoin miners.

\myparagraph{Main Contributions} 
We summarize the main contributions of this paper as below.

\begin{itemize}[leftmargin=*]


% and examine the possible effect of UAT20 in the Ethereum ecosystem.

\item We proposed a novel token standard called universal abstract token (UAT20) to solve liquidity fragmentation in Rollups without introducing security risks.


\item We present a two-phase commit protocol for UAT20, which combines token operations on different Rollups while handling conflicts between them in a uniform way.

\item We empirically studied historical transactions from three rollups to quantify the generality of liquidity fragmentation in Ethereum's rollup ecosystem, reflecting the necessity and effectiveness of UAT20.


% demonstrate a novel approach for creating UAT20 based on CRDTs and design a coordinator on layer 1 to enable the update of UAT 20.


% assess the potential impact of UAT20 within the Ethereum ecosystem
% quantify the 
% makes it possible to incentivize the widely use of UAT20.
    

\end{itemize}
