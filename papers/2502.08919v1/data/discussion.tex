This study seeks to empirically investigate the extent of liquidity fragmentation by analyzing token distribution and user behavior in Ethereum rollups so as to gain insight into the severity and impact of fragmentation. 
% We measure liquidity fragmentation across two Ethereum rollups Arbitrum and Optimism, focusing on the top 20 tokens ranked by circulating market capitalization. We evaluate the distribution of these tokens based on user holdings and identify cases where users hold the same token across both rollups. Additionally, the study assesses the frequency and volume of liquidity aggregation transactions and thus gains insight into the severity and impact of fragmentation.
% % By analyzing historical transactions, we aim to quantify fragmentation and measure the implications for the broader decentralized finance (DeFi) ecosystem.


\myparagraph{Data Collection and Methodology}  We leverage the Google BigQuery platform \cite{google_bigquery} to access Arbitrum dataset and Optimism dataset. ZKsync is not included in Google BigQuery public datasets, thus we  downloaded the data from ZKSync via its Block Explorer API\cite{zksyncapi}. Overall, We gather state and historical transactions for 11 popular ERC20 tokens deployed on these three rollups: \textbf{USDT}, \textbf{USDC}, \textbf{wBTC}, \textbf{wETH}, \textbf{DAI}, \textbf{LINK}, \textbf{ZRO}, \textbf{HOP}, \textbf{AAVE}, \textbf{SYN} and \textbf{UNI}. Using SQL queries and Python scripts, we analyze token distribution, focusing on cases where users hold the same token across both rollups. Historical transaction data related to token-bridging is also examined to measure liquidity aggregation activity from January 1, 2024 to December 20, 2024.
A liquidity unification behavior is identified as follows:
a user executes a token transfer-out transaction on one rollup, following one or more token transfer-in transactions. Additionally, the user has carried out one or more token transfer-out transactions on other rollups in the two hours before.


\myparagraph{The result of liquidity fragmentation} The results indicate that 4,142,268 accounts hold the same token across multiple rollups. Among these, 1,096,997 accounts hold USDT on two rollups, while 145,193 accounts hold wETH on two rollups. Additionally, a total of 1,952,503 addresses hold the same token across three rollups, further highlighting the significant fragmentation of liquidity within the ecosystem.

This fragmentation reflects the diverse engagement of users with multiple rollups, which introduces significant complexity in liquidity management. These insights highlight the growing need for UAT20 to streamline user experience and unify liquidity.

% approximately 5.17\% of users hold a given token on more than two rollups, indicating a high degree of liquidity dispersion across multiple rollups. Furthermore, over 0.52\% of users hold the same token on three rollups. This fragmentation highlights the extent to which users engage with multiple rollups, potentially increasing complexity in liquidity management and emphasizing the importance of effective aggregation mechanisms.

% \begin{figure}[h!]
% \centering
% \includegraphics[width=.8\linewidth]{ieee-4p/data/fig01.pdf}
% \caption{\label{fig:r1} The result of token fragmentation}
% \end{figure}



% This study aims to m

\myparagraph{The result of liquidity unification}
As shown in Figure \ref{fig:r2}, we analyzed transactions of 11 ERC20 tokens across three rollups to identify those related to liquidity unification. Among these, approximately 12.37\% of the transactions involved liquidity aggregation, with a daily peak reaching 22.21\%. 

This finding highlights that liquidity aggregation transactions constitute a significant portion of the overall token activity, resulting in notable resource consumption (\eg, gas consumption). Adopting the UAT20 offers a solution by unifying liquidity, thereby reducing the reliance on such transactions and enhancing overall efficiency.

\begin{figure}[h!]
\centering
\includegraphics[width=.8\linewidth]{./data/fig2.pdf}
\caption{\label{fig:r2} The result of liquidity unification}
\end{figure}
% Leveraging Google BigQuery, we access and process blockchain datasets from two rollups called , focusing on the top 20 tokens ranked by circulating market capitalization.  Our analysis evaluates token fragmentation based on user holdings, identifying cases where users hold the same token across both rollups. Additionally, we measured the frequency of liquidity aggregation transactions by analyzing historical transactions to understand the demand for liquidity aggregation and thus gain insight into the severity and impact of fragmentation.





% The findings of this study directly support the usability and importance of a Unified Abstracted Token (UAT) in addressing liquidity fragmentation across Ethereum Rollups.