\section{Related Work}
\label{sec:related}

Traditionally, TAMP has been addressed \textit{without considering uncertainty, with decoupled task and motion planning problems}, \emph{i.e.}, solved as distinct sub-problems to combine, usually tasks first then motions.
Examples of this approach are found in the robotic navigation and object manipulation with single and multiple robots ____, and in robotic exploration ____. 
Existing work has also explored interleaving or simultaneously addressing task and motion planning without uncertainty ____. 
For instance, ____ combines heuristic-based task planner with geometric reasoners for manipulation planning in table-top object manipulation problems for bi-manual robots. 
In this line of work, motion planning serves as a central part of the action selection process in task planning, \emph{e.g.}, for computing task feasibility at a particular state. 

One of the central themes in our work is addressing \textit{uncertainty} in TAMP. 
Here, an agent plans using incomplete information or probabilistic models of the environment. 
Most existing work on this front interleaves task and motion planning, where the uncertainty is incorporated in task planning with discrete actions and states. 
However, even with discrete spaces, the problem quickly becomes intractable as the search tree branches in both actions and observations. 
In ____, conditional or contingent task planning is the core process with geometric evaluation used as low-level feasibility checks for task selection. 
____ uses the same principle but employs a probabilistic task planning approach.

We address \textit{TAMP under uncertainty} in the form of chance-constrained TAMP (CC-TAMP), where the system considers constraints on the probability of failure to execute a control plan.
Directly related work includes ____, which solves a CC-HPOMDP formulation for autonomous vehicles and Mars rovers in a hierarchical fashion.   
Also, ____ study adaptive sampling in robotic ocean exploration to track a science phenomenon modeled as a Gaussian process. 
We address the CC-TAMP under uncertainty by simultaneously addressing task and motion planning, rather than decoupling them.
Our formulations consider continuous state spaces and both discrete and continuous action spaces that significantly increase the search space and computational complexity.

\begin{table}[t]
    \centering
    \caption{Representative related work landscape}
    \begin{tabular}{l|cccc}
    \hline
    \hline
    \multirow{2}{*}{Work} & Integrated & Under & Chance & Compute \\
    & TAMP & Uncertainty & Constraints & Efficiency \\
    \hline
    ____
            &              &               &    &      \\
    \hline
    ____
            & \checkmark    &               &      &    \\
    \hline
    ____
            &  \checkmark   &  \checkmark   &       &   \\
    \hline
    ____
            &               &  \checkmark   & \checkmark  &\\
    \hline
    This work
            & \checkmark    &  \checkmark   & \checkmark  & \checkmark\\
    \hline
    \hline
    \end{tabular}    
    \label{tab:relatedwork}
    \vspace{-4mm}
\end{table}

Given our focus on risk-awareness, we address the computationally challenging \emph{risk-aware motion planning problem} in uncertain environments.
Sampling-based methods, such as those by ____, estimate safety constraint satisfaction probabilities by sampling uncertainties. 
____ reformulates the problem as a MILP for polygon obstacles and linear Gaussian systems, while ____ proposes a linear program assuming convex safe regions.
By adapting traditional algorithms, ____ develops a chance-constrained RRT for motion planning under uncertainty. 
For nonlinear constraints, ____ use linearized models and Gaussian approximations, whereas ____ propose nonlinear algorithms for non-Gaussian uncertainties, considering higher-order moments.
Our proposed method builds upon existing work in risk-aware motion planning, while providing direct uncertainty incorporation rather than uncertainty sampling.


Our contributions are summarized as follows:\\
i) We introduce BLISS for systems with limited simultaneous motion and localization to achieve blind mobility and schedule intermittent scans for localization and obstacle avoidance.\\
ii) We formulate BLISS as a CC-HPOMDP to integrate task and motion planning under uncertainties, which has not been achieved by existing methods (Table\ref{tab:relatedwork}).\\
iii) We provide an efficient convex deterministic optimization that ensures safety guarantees by satisfying probabilistic constraints without relying on uncertainty samples.