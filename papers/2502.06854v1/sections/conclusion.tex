\section{Conclusion and Future Work}
\label{sec:conclusion}

%This study evaluates the ability of LLMs to comprehend various aspects of IRs, focusing on structural analysis, syntax, semantics, and execution reasoning. Our results indicate that while LLMs demonstrate competence in handling static IR features and basic control flow analysis, they struggle with more complex constructs such as loop recognition and precise execution reasoning. In the future work, improving control flow awareness, refining multi-level semantic analysis, and strengthening loop handling mechanisms could significantly enhance LLMs’ effectiveness in IR-related tasks.
This study evaluates LLMs’ ability to comprehend various aspects of IRs, focusing on structural analysis, syntax, semantics, and execution reasoning. Our findings reveal that while LLMs effectively recognize static IR features and basic control flow structures, they struggle with more complex constructs such as loop reasoning and execution simulation. Specifically, LLMs often fail to reconstruct precise control dependencies, omit key instructions during decompilation, and approximate execution in broad semantic steps rather than following instruction-level behavior.

Despite these limitations, LLMs exhibit strong performance in source-code-level tasks, highlighting the effectiveness of code-specific pretraining. This suggests that enhancing LLMs with IR-specific training, incorporating control-flow-sensitive modeling, and refining token-efficient IR representations could significantly improve their accuracy in handling IR-based tasks.

Future work should explore techniques for better control flow awareness, multi-granularity semantic understanding, and robust loop handling mechanisms. Additionally, fine-tuning LLMs on IR-annotated datasets or integrating graph-based neural representations may further enhance their capacity for structural and execution reasoning, bridging the gap between source code comprehension and IR analysis.


% \hy{the impact statement is not in 8 page main text}\\
% \hailong{Then what we do? Can we put the limitations back to main text as a Section Discussion?}
% \hy{see if you have more important contents to make good use of the space}

%\hy{maybe this section could be Conculsion only, but move the Limitations to  Impact Statement}




% \noindentparagraph{\textup{\textbf{Data Availability.}}}
% \section{Data Availability.}
% All the experimental data and code used in this paper are available at: 
% \texttt{\url{https://anonymous.4open.science/r/LLMs-in-IR-558F/
% }}.
% All study methods, results, and case studies,
% including additional details, are available on our website~\cite{This_work}.