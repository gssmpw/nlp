\section{Preliminaries}
\paragraph{LLVM and IRs}
LLVM is a widely adopted compiler framework that optimizes source code through Intermediate Representations (LLVM IRs)~\cite{llvm}. LLVM IRs are \emph{low-level, typed,} and \emph{structured}, offering a universal representation compatible with numerous high-level languages. LLVM enables control flow graph (CFG) analysis and supports the integration of custom compiler passes for additional transformations. The key components of LLVM IRs include:
\begin{itemize}[nolistsep,leftmargin=*]
    \item \emph{Modules}: Groups of functions representing a complete IR-based program unit.
    \item \emph{Functions}: Collections of basic blocks, each representing an independent unit for analysis or optimization.
    \item \emph{Basic Blocks}: Sequences of instructions without internal branches, forming the building blocks of control flow.
    \item \emph{Instructions}: Defined by an opcode, type, and operands governed by a static type system.   
\end{itemize}

\mypara{Control Flow Graphs (CFGs)}
A CFG models all possible execution paths in a program. Nodes represent basic blocks, while directed edges capture potential control flow transitions. CFGs underpin many software engineering tasks, including vulnerability detection~\cite{zhou2019method,anju2010malware}, code optimization~\cite{mcconnell1993tree}, and program analysis~\cite{fechete2008framework}. 


