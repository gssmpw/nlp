\lstset{
  basicstyle=\ttfamily\footnotesize,
  breaklines=true,
  numbers=left,
  numberstyle=\tiny,
  keywordstyle=\color{blue},
  commentstyle=\color{green!50!black},
  stringstyle=\color{orange},
}

\begin{figure}[h]
    \centering
    \begin{minipage}{0.98\textwidth}
      \begin{tcolorbox}[title= \footnotesize LLVM IR Source Code]
        \begin{lstlisting}[language=C]
    ; ModuleID = '../../Human_xBenchmarks_164/CPP_2/CPP_2.cpp'
    source_filename = "../../Human_xBenchmarks_164/CPP_2/CPP_2.cpp"
    target datalayout = "e-m:e-p270:32:32-p271:32:32-p272:64:64-i64:64-f80:128-n8:16:32:64-S128"
    target triple = "x86_64-unknown-linux-gnu"
    ; Function Attrs: mustprogress nofree norecurse nosync nounwind readnone uwtable willreturn
    define dso_local float @_Z15truncate_numberf(float %0) local_unnamed_addr #0 {
      %2 = fptosi float %0 to i32
      %3 = sitofp i32 %2 to float
      %4 = fsub float %0, %3
      ret float %4
    }
    attributes #0 = { mustprogress nofree norecurse nosync nounwind readnone 
                      uwtable willreturn "frame-pointer"="none" 
                      "min-legal-vector-width"="0" "no-trapping-math"="true" 
                      "stack-protector-buffer-size"="8" "target-cpu"="x86-64" 
                      "target-features"="+cx8,+fxsr,+mmx,+sse,+sse2,+x87" 
                      "tune-cpu"="generic" }

    !llvm.module.flags = !{!0, !1}
    !llvm.ident = !{!2}

    !0 = !{i32 1, !"wchar_size", i32 4}
    !1 = !{i32 7, !"uwtable", i32 1}
    !2 = !{!"clang version 13.0.0 (https://github.com/llvm/llvm-project.git d7b669b3a30345cfcdb2fde2af6f48aa4b94845d)"}
     \end{lstlisting}
      \end{tcolorbox}
    \end{minipage}
    \caption{An example of IR}
    \label{list:summarization_result}
\end{figure}

\begin{table}[h]
    \centering
    \renewcommand{\arraystretch}{1.3} % Adjust row spacing
    \begin{tabular}{ll}
        \toprule
        \textbf{LLMs} & \textbf{Summarizing of Example IR} \\
        \midrule
        \textbf{GPT-4} & There is 1 function in the code, function[1] takes 1 input: float number. \\
                      & function[1] is truncating the float number to its integer part \\
                      & and outputs the fractional part as a float. \\
        \midrule
        \textbf{GPT-3} & There is 1 function in the code, function[1] takes 1 input: float number. \\
                      & Function[1] is truncating the decimal part of the input number \\
                      & and outputs the truncated value. \\
        \midrule
        \textbf{Gemma 2} & There is 1 function in the code: function[1] takes 1 input: float number. \\
                        & function[1] is truncating the fractional part of the input number \\
                        & on number, and outputs float. \\
        \midrule
        \textbf{LLaMA 3.1} & There is 1 function in the code, function[1] takes 1 input: float number. \\
                          & function[1] is truncating the decimal part of the input number \\
                          & on number, and outputs the decimal part of the input number. \\
        \midrule
        \textbf{Code Llama} & There is 1 function in the code, function[1] takes 2 inputs: vector<float> numbers, \\
                            & float threshold. function[1] is checking if any two elements in the vector \\
                            & are closer to each other than the given threshold on numbers, \\
                            & and outputs a boolean value indicating if such elements exist. \\
        \bottomrule
    \end{tabular}
    \caption{Summarization of Example IR by Different LLMs}
    \label{tab:summarization_result}
\end{table}