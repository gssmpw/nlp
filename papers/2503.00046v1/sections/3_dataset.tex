\section{Dataset}

\begin{table}[htbp]
\caption{Basic Shapes 1 Questions}
\label{table:questions-basic_shapes_1}
\begin{tabularx}{\textwidth}{@{}lX@{}}
\hline
\textbf{Shape} & \textbf{Question} \\ \hline
Circle & Write a Python code to generate GDSII for a circle on layer 0, radius = 10 mm, center at 0,0. \\ \hline
Donut & Generate a donut shape with 10 mm outer radius and 5 mm inner radius. Make the circle smoother by setting max distance between point 0.01mm. \\ \hline
Oval & Generate an oval with major axis of 20 mm, minor axis of 13 mm, on layer 0, center at 0,0. \\ \hline
Square & Generate a square with width 10 mm, put lower right corner of the square at 0,0. \\ \hline
Triangle & Generate a triangle with each edge 10 mm, center at 0,0. \\ \hline
Grid & Draw the GDSII for a grid: Grid on Layer 1, DATATYPE 4, 5 µm grid, and total width is 200 µm and height is 400 µm, placed at coordinates (100,800) nanometers. \\ \hline
\end{tabularx}
\end{table}

\begin{table}[htbp]
\caption{Basic Shapes 2 Questions}
\label{table:questions-basic_shapes_2}
\begin{tabularx}{\textwidth}{@{}lX@{}}
\hline
\textbf{Shape} & \textbf{Question} \\ \hline
Heptagon & Generate a Heptagon with each edge 10 mm, center at 0,0. \\ \hline
Octagon & Generate an Octagon with each edge 10 mm, center at 0,0. \\ \hline
Trapezoid & Generate a Trapezoid with upper edge 10 mm, lower edge 20 mm, height 8 mm, center at 0,0. \\ \hline
Hexagon & Generate a regular hexagon with each edge 10 mm, center at 0,0. \\ \hline
Pentagon & Generate a regular pentagon with each edge 10 mm, center at 0,0. \\ \hline
Text & Generate a GDS file with the text "Hello, GDS!" centered at (0,0), with a height of 5 mm, on layer 1. \\ \hline
\end{tabularx}
\end{table}

\begin{table}[htbp]
\caption{Advanced Shapes Questions}
\label{table:questions-advanced_shapes}
\begin{tabularx}{\textwidth}{@{}lX@{}}
\hline
\textbf{Shape} & \textbf{Question} \\ \hline
Arrow & Generate an Arrow pointing to the right with length 10 mm, make the body 1/3 width of the head, start at 0,0. \\ \hline
SquareArray & Generate a square array with 5*5 mm square, for 10 columns and 10 rows, each 20 mm apart, the lower left corner of the upper right square is at 0,0. \\ \hline
Serpentine & Generate a serpentine pattern with a path width of 1 µm, 15 turns, each segment being 50 µm long and tall, starting at (0,0), on layer 2, datatype 6. \\ \hline
RoundedSquare & Draw a 10*10 mm square, and do corner rounding for each corner with r=1 mm. \\ \hline
Spiral & Generate a Parametric spiral with r(t) = e\^{}(-0.1t), for 0 <= t <= 6pi, line width 1. \\ \hline
BasicLayout & 1. Draw a rectangular active region with dimensions 10 µm x 5 µm.
2. Place a polysilicon gate that crosses the active region vertically at its center, with a width of 1 µm.
3. Add two square contact holes, each 1 µm x 1 µm, positioned 1 µm away from the gate on either side along the active region. \\ \hline
\end{tabularx}
\end{table}

\begin{table}[htbp]
\caption{Complex Structures Questions}
\label{table:questions-complex_structures}
\begin{tabularx}{\textwidth}{@{}lX@{}}
\hline
\textbf{Shape} & \textbf{Question} \\ \hline
RectangleWithText & Generate a GDS with a 30*10 mm rectangle on layer 0 with a text "IBM Research" at the center of the rectangle. Put the text on layer 1. \\ \hline
MicrofluidicChip & Draw a design of a microfluidic chip. On layer 0, it is the bulk of the chip. It is a 30 * 20 mm rectangle. On layer 2 (via level), draw two circular vias, with 2 mm radius, and 20 mm apart horizontally. On layer 3 (channel level), draw a rectangular shaped channel (width = 1 mm) that connects the two vias at their center. \\ \hline
ViaConnection & Create a design with three layers: via layer (yellow), metal layer (blue), and pad layer (red). The via radius is 10 units, pad radius is 30 units, and metal connection width is 40 units with a total length of 600 units. Position the first via at (50, 150) and the second via at (550, 150). Ensure the metal connection fully covers the vias and leaves a margin of 10 units between the edge of the metal and the pads. Leave a space of 50 units between the vias and the edges of the metal connection. \\ \hline
FiducialCircle & Draw a 3.2 mm circle, with fiducial marks inside. The fiducial marks should be a "+" sign, with equal length and width. Each marker should be 200 um apart. There will be annotations next to each marker. Row: A -> Z, column: start from 1. \\ \hline
ComplexLayout & 1. Draw three rectangular active regions with dimensions 20 µm x 5 µm, positioned horizontally with 5 µm spacing between them.
2. Create a complex polysilicon gate pattern consisting of multiple vertical and horizontal lines, with widths of 0.5 µm, forming a grid-like structure.
3. Add several contact holes (each 1 µm x 1 µm) positioned at the intersections of the polysilicon gate pattern and the active regions. \\ \hline
DLDChip & Draw a deterministic lateral displacement chip - include channel that can hold the array has gap size = 225 nm, circular pillar size = 400 nm, width = 30 pillars, row shift fraction = 0.1, add an inlet and outlet 40 µm diameter before and after the channel, use a 20*50 µm bus to connect the inlet and outlet to the channel. \\ \hline
FinFET & Draw a FinFET with the following specifications:
- Fin width: 0.1 µm
- Fin height: 0.2 µm
- Fin length: 1.0 µm
- Gate length: 0.1 µm
- Source/drain length: 0.4 µm
- Source/drain extension beyond the fin: 0.2 µm
Use separate layers for the fin, gate, and source/drain regions. \\ \hline
\end{tabularx}
\end{table}



We use the Pitt Ads Dataset (referred to as \texttt{Pitt-Ads}) as our basis, where each ad image is annotated with its topic, expected actions from viewers after seeing the ad, binary labels of atypical objects in it (when applicable), and the topic of the ad (10 topic groups in total) \cite{Hussain2017AutomaticUO,ye2019interpreting}. We sample 100 ads and collect fine-grained human creativity annotations (\texttt{Creative-100} in \ref{sec:fine_grained_creativity_data}); we also sample an additional 300 ads from the remaining data points for atypicality prediction(\texttt{Atypical-300} in \ref{sec:atypicality_data}).


\subsection{\texttt{Creative-100}}
\label{sec:fine_grained_creativity_data}
\texttt{Creative-100} consists of 100 ads, with 10 from each topic group: food, pet, drinks, automobile, electronics, service, education, beauty, healthcare, clothing, home, leisure, shopping, and non-commercial. To do quality creativity evaluation, we break down creativity into two dimensions: originality and atypicality, the two most influential dimensions for ads creativity according to \cite{modeling_determinants}. Human annotations are then collected via Amazon Mechanical Turk (Mturk) to represent fine-grained ratings in all three dimensions: \textbf{originality}, \textbf{atypicality}, and \textbf{creativity} (see Figure \ref{fig:mturk} for Mturk annotation interface). 

Due to the inherent subjectivity of the creativity judgment, we formulate the measurement of creativity as several multiple-choice questions with possible answers as a categorical distribution of those choices. In other words, the predictive target is not a single label (e.g., ``creative'') but a distribution of human ratings. This motivates us to collect 25 annotations per ad image to approximate the true rating distribution within certain error rate ~\cite{mchugh2012interrater}. 
Refer to Appendix \ref{sec:appendix_num_samples} for more details.


% \paragraph{Human Annotation}
For atypicality and originality, we follow \citet{modeling_determinants} and record responses about various statements (Table \ref{table:mturk_questions}). For creativity, we record an overall score from 1 to 5 and convert it to a 3-scale, aligning with other dimensions. We also include a quality check question by asking annotators to choose the action after seeing a given ad (e.g., ``I should go to Chick-fil-A'' for Ad A in Figure \ref{fig:intro}). Five actions are given, with one correct action and four randomly sampled from \texttt{Pitt-Ads}. Annotators get 96.88\% accuracy in this question, highlighting their accurate understanding of visual advertisements. More dataset construction details are in Appendix \ref{sec:data_collection}.





\subsection{\texttt{Atypical-300}}
\label{sec:atypicality_data}
We also randomly sampled 300 ads (\texttt{Atypical-300}) from \texttt{Pitt-Ads}, where 185(62\%) include atypical object(s). Different from \texttt{Creative-100}, each ad in this set only has three binary annotations on atypicality. Both ~\citet{modeling_determinants} and \texttt{Creative-100} (see Appendix \ref{sec:appendix_connection}) show that atypicality has a positive correlation with creativity  Thus, we include this dataset to gain further insight into VLM's ability to evaluate ad creativity.

% % Please add the following required packages to your document preamble:
% % \usepackage{graphicx}
% \begin{table*}[!ht]
% \centering
% \caption{}
% \label{tab:my-table}
% \resizebox{\textwidth}{!}{%
% \begin{tabular}{lcccccc}
% \toprule
%  & \multicolumn{2}{c}{ours} & \multicolumn{2}{c}{Gurobi} & \multicolumn{2}{c}{MOSEK} \\
%  & \multicolumn{1}{l}{time (s)} & \multicolumn{1}{l}{optimality gap (\%)} & \multicolumn{1}{l}{time (s)} & \multicolumn{1}{l}{optimality gap (\%)} & \multicolumn{1}{l}{time (s)} & \multicolumn{1}{l}{optimality gap (\%)} \\ \hline
% \begin{tabular}[c]{@{}l@{}}Linear Regression\\ Synthetic \\ (n=16000, p=16000)\end{tabular} & 57 & 0.0 & 3351 & - & 2148 & - \\ \hline
% \begin{tabular}[c]{@{}l@{}}Linear Regression\\ Cancer Drug Response\\ (n=822, p=2300)\end{tabular} & 47 & 0.0 & 1800 & 0.31 & 212 & 0.0 \\ \hline
% \begin{tabular}[c]{@{}l@{}}Logistic Regression\\ Synthetic\\ (n=16000, p=16000)\end{tabular} & 271 & 0.0 & N/A & N/A & 1800 & - \\ \hline
% \begin{tabular}[c]{@{}l@{}}Logistic Regression\\ Dorothea\\ (n=1150, p=91598)\end{tabular} & 62 & 0.0 & N/A & N/A & 600 & 0.0 \\
% \bottomrule
% \end{tabular}%
% }
% \end{table*}

% Please add the following required packages to your document preamble:
% \usepackage{multirow}
% \usepackage{graphicx}
\begin{table*}[]
\centering
\caption{Certifying optimality on large-scale and real-world datasets.}
\vspace{2mm}
\label{tab:my-table}
\resizebox{\textwidth}{!}{%
\begin{tabular}{llcccccc}
\toprule
 &  & \multicolumn{2}{c}{ours} & \multicolumn{2}{c}{Gurobi} & \multicolumn{2}{c}{MOSEK} \\
 &  & time (s) & opt. gap (\%) & time (s) & opt. gap (\%) & time (s) & opt. gap (\%) \\ \hline
\multirow{2}{*}{Linear Regression} & \begin{tabular}[c]{@{}l@{}}synthetic ($k=10, M=2$)\\ (n=16k, p=16k, seed=0)\end{tabular} & 79 & 0.0 & 1800 & - & 1915 & - \\ \cline{2-8}
 & \begin{tabular}[c]{@{}l@{}}Cancer Drug Response ($k=5, M=5$)\\ (n=822, p=2300)\end{tabular} & 41 & 0.0 & 1800 & 0.89 & 188 & 0.0 \\ \hline
\multirow{2}{*}{Logistic Regression} & \begin{tabular}[c]{@{}l@{}}Synthetic ($k=10, M=2$)\\ (n=16k, p=16k, seed=0)\end{tabular} & 626 & 0.0 & N/A & N/A & 2446 & - \\ \cline{2-8}
 & \begin{tabular}[c]{@{}l@{}}DOROTHEA ($k=15, M=2$)\\ (n=1150, p=91598)\end{tabular} & 91 & 0.0 & N/A & N/A & 634 & 0.0 \\
 \bottomrule
\end{tabular}%
}
% \vspace{-3mm}
\end{table*}