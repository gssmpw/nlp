\appendix

\section{Ads Dataset}
\label{sec:appendix_dataset}

\subsection{Terminology}
\label{sec:appendix_terminologies}
In the original study \cite{modeling_determinants} that proposed the breakdown of creativity, they use the term ``synthesis'' instead of atypicality. However, they defined ``synthesis'' as ``\textit{...combine, connect, or blend normally unrelated objects or ideas}'' which is very similiar to ``atypicality''. We use the term ``atypicality'' because that is the term used in the \texttt{Pitt-Ads} Dataset and want to keep the terminology aligned. %From now on, we refer to both of them as ``atypicality''.

\subsection{Atypicality and Originality as Selected Dimensions of Creativity}
The original marketing research \cite{modeling_determinants} conducted an exploratory factor analysis (EFA) on five dimensions of divergence in ads creativity (numbers in parenthesis are loading factors of EFA, higher means more correlated with creativity): originality (0.23), synthesis (0.23), artistic value (0.20), flexibility (0.18), and elaboration (0.15). As mentioned previously, their definition of synthesis is very similar to ``atypicality'' as defined in \texttt{Pitt-Ads} Dataset. Therefore, we keep the highest-scoring two aspects (originality and synthesis/atypicality) in our study. 

\subsection{Sampling Process}
% \label{sec:appendix_dataset}
The original \texttt{Pitt Ads} dataset consists of 38 topics in total \cite{Hussain2017AutomaticUO}. We group those topics into 10 ``topic groups'': clothing, food, automobile, beauty, leisure, electronics, drinks, service, non-commercial, and healthcare. From each topic group, we sampled 10 ads, which makes up the 100 images in \texttt{Creativity-100}.

\paragraph{Details about ``topic groups''}
\begin{itemize}
    \item food: ``Restaurants, cafe, fast food'', ``Chocolate, cookies, candy, ice cream'', ``Chips, snacks, nuts, fruit, gum, cereal, yogurt, soups'', ``Seasoning, condiments, ketchup''
    \item pet: ``Pet food''
    \item drinks: ``Alcohol"'', ``Coffee, tea"'', ``Soda, juice, milk, energy drinks, water"''
    \item automobile: ``Cars, automobiles'' 
    \item electronics: ``Electronics'' 
    \item service: ``Phone, TV and internet service providers'', ``Financial services , ``Security and safety services'' , ``Software'' , ``Other services'' 
    \item education: ``Education''
    \item beauty: ``Beauty products and cosmetics''
    \item healthcare: ``Healthcare and medications''
    \item clothing: ``Clothing and accessories''
    \item home: ``Baby products'' , ``Cleaning products'' , ``Home improvements and repairs'' , ``Home appliances'' ]
    \item leisure: ``Games and toys'' , ``Vacation and travel'' , ``Media and arts'' , ``Sports equipment and activities'', ``Gambling'' 
    \item shopping: ``Shopping''
    \item non-commercial: ``Environment, nature, pollution, wildlife'', ``Animal rights, animal abuse'', ``Human rights'', ``Safety, safe driving, fire safety'', ``Smoking, alcohol abuse'', ``Domestic violence'', ``Self esteem, bullying, cyber bullying'', ``Political candidates'', ``Charities''
\end{itemize}

\begin{figure*}
    \centering
        \includegraphics[width=\textwidth]{image/mturk.png}
    % \caption{Task formulation}
    \caption{Amazon Mechanical Turk interface.}
    \label{fig:mturk}
    % \vspace{-5mm}
\end{figure*}


\section{Data Collection and Post Processing}
\label{sec:data_collection}
\subsection{Amazon Mechanical Turk Details}
% \label{sec:}
\label{sec:appendix_mturk}
\paragraph{Payment for worker}
Each HIT receives \$0.5 compensation (estimated \$15/hour).

\paragraph{Annotation interface}
See Figure \ref{fig:mturk} for the annotation interface. Note that there is a section ``artistic values''. We dropped that section in the later parts of the experiment because 1) it is very subjective and could be further broken down into more fine-grained subcategories, and 2) to keep our focus on atypicality and originality.

In total, 43 workers contributed to our task and finished 2500 HITs. Their background can be found in Figure \ref{fig:mturk_countries}, \ref{fig:mturk_age}, 
\ref{fig:mturk_race}, \ref{fig:mturk_ethnicity}, and \ref{fig:mturk_gender}. As we can see, the annotators are strongly skewed towards the US-based, white, female, middle age group, which should be kept in mind when applying our methodology when it comes to people from another background. 

\begin{figure}
    \centering
        \includegraphics[width=0.4\textwidth]{image/mturk_countries.png}
    \caption{Distribution of workers' response to ``In which country did you live the longest time so far?''}
    \label{fig:mturk_countries}
    % \vspace{-8mm}
\end{figure}

\begin{figure}
    \centering
        \includegraphics[width=0.4\textwidth]{image/mturk_age.png}
    \caption{Distribution of workers' response to ``What is your age?''}
    \label{fig:mturk_age}
    % \vspace{-8mm}
\end{figure}

\begin{figure}
    \centering
        \includegraphics[width=0.4\textwidth]{image/mturk_race.png}
    \caption{Distribution of workers' response to ``Race Identification (select one or more)''}
    \label{fig:mturk_race}
    % \vspace{-8mm}
\end{figure}

\begin{figure}
    \centering
        \includegraphics[width=0.4\textwidth]{image/mturk_ethnicity.png}
    \caption{Distribution of workers' response to ``Are you of Hispanic or Latino/a origin?''}
    \label{fig:mturk_ethnicity}
    % \vspace{-8mm}
\end{figure}

\begin{figure}
    \centering
        \includegraphics[width=0.4\textwidth]{image/mturk_gender.png}
    \caption{Distribution of workers' response to ``What is your gender?''}
    \label{fig:mturk_gender}
    % \vspace{-8mm}
\end{figure}

\subsection{Number of Samples for Distribution Task}
\label{sec:appendix_num_samples}
Following previous works \cite{mchugh2012interrater, cheng2024every}, the number of samples required to approximate the real distribution can be calculated as follows: 
\[ P(D_{KL}(g_{n, k}||f) > \epsilon) \leq e^{-n\epsilon} \left[  \frac{3c_1}{c_2}\sum_{i=0}^{k-2} {k_{i-1}} (\frac{e\sqrt{n}}{2\pi})^i \right] \]
$c_1$ and $c_2$ are constant values (based on \cite{mchugh2012interrater} $c_1 = 2, c_2 = \frac{\pi}{2}$), k is the number of categories in the categorical distribution (in our case, $k = 3$), and n is the number of samples. If we fix the left-hand side to be less than 0.1, we would get $n$ has to be 25 (see Figure \ref{fig:sample_size_error}).
\begin{figure}
    \centering
        \includegraphics[width=0.4\textwidth]{image/sample_size_error.png}
    \caption{Upper-bound of the error based on calculation.}
    \label{fig:sample_size_error}
    % \vspace{-8mm}
\end{figure}


\subsection{Label Processing}
\label{sec:appendix_label_proc}

We process the annotation by first converting the categorical data to numerical values. For atypicality and originality, we code agree, neutral, and disagreement choices as 1, 0, and -1. As there are three subquestions for both atypicality and originality, we simply add up the three scores from each dimension and get one accumulated score for each. For overall creativity, we keep the raw score (an integer number between 1 and 5). Thus each annotation data point consists of three integer scores, corresponding to atypicality, originality, and overall creativity.

We then normalize the score by individual annotators to mitigate the differences in people's rating preferences. In particular, for each score dimension, we group the scores provided by each annotator and standardize them (subtract mean and divide by standard deviation). We then map the standardized score to an integer (1, 2, or 3) by dividing the standardized score interval into three bins.

\subsection{Connection between atypicality and creativity}
\label{sec:appendix_connection}
%As mentioned above, all of three score categories on their own are very subjective. However, if we consider the connection among three categories, 
After analyzing the fine-grained creativity data we collected (Sec.~\ref{sec:fine_grained_creativity_data}), we find out that the Pearson R correlation between the normalized atypicality and overall creativity score is 0.4017 ($p<0.01$), a positive correlation (the sample size is 2500: 100 ads with 25 annotations each). Therefore, it makes sense to evaluate the same methodology on data with only atypicality annotation to prove its effectiveness at a larger scale. 

\section{Experiment Details}
\label{sec:appendix_experiment_details}
\subsection{Configurations}
\begin{itemize}
    \item Temperature: 0.75 (for distribution prediction) and 0.01 (for disagreement prediction)
    \item Max New Token: 256
    \item Model Checkpoint
    \begin{itemize}
        \item GPT-4: \texttt{gpt-4-vision-preview}
        \item LLaVa-7B: \texttt{llava-v1.6-mistral-7b-hf}
        \item LLaVa-13B: \texttt{llava-v1.6-vicuna-13b-hf}
        \item InternVL2-8B: \texttt{InternVL2-8B-MPO}
    \end{itemize}
    \item Number of pairwise samples (\% of label ``1'')
    \begin{itemize}
        \item creativity: 938 (46\%)
        \item atypicality: 2631 (51\%)
        \item originality: 2708 (50\%)
    \end{itemize}
    \item Device: single card NVIDIA L40S GPU
\end{itemize}

\subsection{Running Time}(Roughly, all tasks combined) 
\begin{itemize}
    \item \texttt{Creative-100}
    \begin{itemize}
        \item GPT4-v: 1 hr
        \item LLaVA-7B: 4 hr
        \item LLaVA-13B: 5 hr
        \item InternVL: 4.5 hr
    \end{itemize}
    \item \texttt{Atypical-300} (atypical data only)
    \begin{itemize}
        \item GPT4-v: 0.2 hr
        \item LLaVA7B: 0.5 hr
        \item LLaVA13B: 0.5 hr
        \item InternVL: 0.5 hr
    \end{itemize}
\end{itemize}

\subsection{Multiple Run Results}
For the distribution modeling task, since there is inherent randomness in the modeling process (we set the temperature for LLM generation to 0.75), we run the same setting for 3 times to see if the randomness will have a huge impact. As shown in Table \ref{table:multi-run results}, for the same task, same model, the performance is still within reasonable margin. We believe it is safe to believe that the results is not affected by randomness in LLM.

\begin{table}[t]
\small
\setlength{\tabcolsep}{3pt}
\centering
  \begin{tabular}{cc|cc}
    \toprule
    % \hline
    \multirow{2}{*}{\textbf{Dimension}} &
    \multirow{2}{*}{\textbf{Model}} &
      \multicolumn{2}{c}{\textbf{Distribution Modeling}}  \\
        &&  R ({\textit{p-value}}) & {KL-Divergence}   \\
        
    \midrule     
        \multirow{11}{*}{\shortstack{\textbf{Creativity}}}
            & LLaVA-7B & 0.6363 (<.01) & 1.0721  \\
            & LLaVA-7B & 0.6548 (<.01) & 1.0127  \\
            & LLaVA-7B & 0.6392 (<.01) & 1.0464  \\
            &&&\\
            & LLaVA-13B   & 0.6548 (<.01) & 0.3734 \\
            & LLaVA-13B   & 0.6289 (<.01) & 0.3933 \\
            & LLaVA-13B   & 0.6257 (<.01) & 0.3706 \\
            &&&\\
            & InternVL2-8B   & 0.6363 (<.01) & 1.4903 \\
            & InternVL2-8B   & 0.6548 (<.01) & 1.4542 \\
            & InternVL2-8B   & 0.7153 (<.01) & 1.4658 \\
            
    \midrule    
        \multirow{11}{*}{\shortstack{\textbf{Originality}}}
            & LLaVA-7B &  0.7573 (<.01) & 0.5649 \\
            & LLaVA-7B &  0.7623 (<.01) & 0.5946 \\
            & LLaVA-7B &  0.7781 (<.01) & 0.5942 \\
            &&&\\
            & LLaVA-13B   & 0.7108 (<.01) & 0.2983 \\
            & LLaVA-13B   & 0.6757 (<.01) & 0.3297 \\
            & LLaVA-13B   & 0.7262 (<.01) & 0.3448 \\
            &&&\\
            & InternVL2-8B   & 0.8169 (<.01) & 0.5707 \\
            & InternVL2-8B   & 0.8044 (<.01) & 0.6203 \\
            & InternVL2-8B   & 0.8124 (<.01) & 0.5935 \\
            
    \midrule    
        \multirow{11}{*}{\shortstack{\textbf{Atypicality}}}
            & LLaVA-7B & 0.7282 (<.01) & 0.4226 \\
            & LLaVA-7B & 0.7470 (<.01) & 0.4043 \\
            & LLaVA-7B & 0.7061 (<.01) & 0.4301 \\ 
            &&&\\
            & LLaVA-13B   & 0.6332 (<.01) & 0.2617 \\
            & LLaVA-13B   & 0.5732 (<.01) & 0.2692 \\
            & LLaVA-13B   & 0.7126 (<.01) & 0.2367 \\
            &&&\\
            & InternVL2-8B   & 0.7838 (<.01) & 0.4041 \\
            & InternVL2-8B   & 0.7821 (<.01) & 0.4083 \\
            & InternVL2-8B   & 0.7838 (<.01) & 0.4041 \\
    \bottomrule
  \end{tabular}
  
  \caption{Results for multiple runs}
  \label{table:multi-run results}
\end{table}



% \subsection{VLM Prompts}
% \label{sec:appendix_prompt}
\subsection{VLM Prompts - Creativity}
\paragraph{Distribution Modeling} 

\textit{How creative is this advertisement? Give your answer that follows these guidelines:\\
- Choose your answer from [not creative, neutral, very creative], answer 1 for not creative, 2 for neutral, and 3 for very creative \\
- Show your reasons and give a final answer (in a single number), in the following format: ``reasoning:\{reasoning\}, answer: \{score\}''}


\paragraph{Disagreement}
\textit{You are a marketing analyst tasked with predicting the audience's perception of an advertisement's creativity. In next week, 25 individuals will rate this ad's creativity on a scale of 1 to 3. Your goal is to predict the level of disagreement among these ratings. As a reference, ads that are more challenging to understand generally result in higher disagreement, while simpler ads lead to greater consensus.} \\
\textit{Provide your assessment using a single score:\\
1: Low disagreement (high agreement),\\
2: Neutral disagreement,\\
3: High disagreement (low agreement).\\
Your response should follow this format:\\
``answer: \{score\}; explanation: \{reasoning\}''}


\paragraph{Pairwise}
\textit{Here are two images of advertisement. Which one is more creative? Answer 1 for the one on the left and 2 for the one on the right. Give your answer in the following format: ``explanation: \{reasoning\}; answer: \{choice\}''}

\subsection{VLM Prompts - Atypicality}
\paragraph{Distribution Modeling}
\textit{How atypical is this advertisement? Give your answer that follows these guidelines:\\
- Atypical advertisement either connected objects that are usually unrelated, contained unusual connections, or brought unusual items together;\\
- Choose your answer from [not atypical, neutral, very atypical], answer 1 for not atypical, 2 for neutral, and 3 for very atypical\\
- Show your reasons and give a final answer (in a single number), in the following format: ``reasoning: \{reasoning\}, answer: \{score\}; ''}

\paragraph{Disagreement} 
\textit{You are a marketing expert task with predicting how audiences perceive the atypicality of an advertisement. Atypicality refers to ads that connect objects typically unrelated, feature unusual connections, or bring unexpected items together. In next week, 25 individuals will rate this ad's atypicality on a scale of 1 to 3. Your task is to predict the level of disagreement among their ratings. As a reference, Ads that are more difficult to interpret tend to generate higher disagreement, whereas more straightforward ads lead to greater agreement.\\
Provide your answer using a single score:\\
1: Low disagreement (high agreement),\\
2: Neutral disagreement,\\
3: High disagreement (low agreement).\\
Format your response as follows:\\
``answer: \{score\}; explanation: \{reasoning\}''}

\paragraph{Pairwise}
\textit{Here are two images of advertisement. Which one is more abnormal and unusual? Answer 1 for the one on the left and 2 for the one on the right. Give your answer in the following format: ``explanation: \{reasoning\}; answer: \{choice\}''}






\subsection{VLM Prompts - Originality}
\paragraph{Distribution Modeling}
\textit{How novel is this advertisement? Give your answer that follows these guidelines:\\
- Novel advertisement either is out of the ordinary, break away from habit-bound and stereotypical thinking, or is unique;\\
- Choose your answer from [not novel, neutral, very novel], answer 1 for not novel, 2 for neutral, and 3 for very novel\\
- Show your reasons and give a final answer (in a single number), in the following format: ``reasoning: \{reasoning\}, answer: \{score\};''}

\paragraph{Disagreement} 
\textit{You are a marketing analyst tasked with predicting how audiences perceive the novelty of an advertisement. Novelty refers to ads that are out of the ordinary, break free from stereotypical thinking, or exhibit uniqueness. In next week, 25 individuals will rate this ad's novelty on a scale of 1 to 3. Your goal is to predict the level of disagreement among their ratings. Ads that are harder to interpret typically result in higher disagreement, while clearer ads lead to more agreement.\\
Provide your assessment using a single score:\\
1: Low disagreement (high agreement),\\
2: Neutral disagreement,\\
3: High disagreement (low agreement).\\
Format your response as follows:\\
``answer: \{score\}; explanation: \{reasoning\}''}

\paragraph{Pairwise}
\textit{Here are two images of advertisement. Which one is more unique compared with other ads in the same product category? Answer 1 for the one on the left and 2 for the one on the right. Give your answer in the following format: : ``explanation: \{reasoning\}; answer: \{choice\}''}


\subsection{\texttt{Atypical-300} Prompts (atypicality only)}
\paragraph{Distribution Modeling}
\textit{How atypical is this advertisement? Give your answer that follows these guidelines:\\
- Atypical advertisement either connected objects that are usually unrelated, contained unusual connections, or brought unusual items together;\\
- Choose your answer from [not atypical, neutral, very atypical], answer 0 for not atypical and 1 for very atypical\\
- Show your reasons and give a final answer (in a single number), in the following format: ``reasoning: \{reasoning\}, answer: \{score\};''}

\paragraph{Disagreement} 
\textit{You are a marketing expert analyzing how audiences perceive the atypicality of an advertisement. Atypicality refers to ads that connect objects typically unrelated, feature unusual connections, or bring unexpected items together. Imagine 25 individuals have rated the ad's atypicality either 0 or 1. Your task is to determine the level of disagreement among their ratings. Ads that are more difficult to interpret tend to generate higher disagreement, whereas more straightforward ads lead to greater agreement.\\
Provide your answer using a single score:\\
0: Low disagreement (high agreement),\\
1: High disagreement (low agreement).\\
Format your response as follows:\\
``answer: \{score\}; explanation: \{reasoning\}''}
\paragraph{Pairwise} 
\textit{Here are two images of advertisement. Which one is more abnormal and unusual? 1 for the left image and 2 for the right image. Give your answer in the following format: ``answer: \{number\}; explanation: \{reasoning\}''}


\section{Output Analysis}
\label{sec:appendix_output_analysis}
\subsection{Error Analysis on Pairwise Outputs}
\label{sec:appendix_error_analysis}
Here, we present two error analysis examples on the pairwise task. Detailed analysis are in the caption of each image pair (Figure \ref{fig:error_analysis_1_img}, \ref{fig:error_analysis_2_img}).

%%%%%%%%%%%%%%%%%%%%%% Error Analysis 1 %%%%%%%%%%%%%%%%%%%%%%%%%%%%%%%%%%%%%%%%%%%%%%%
\begin{figure*}
    \centering
    \begin{minipage}[b]{0.45\textwidth}
        \includegraphics[width=\textwidth]{image/171216.png}
        % \caption{Flower one.}
    \end{minipage}
    \hfill
    \begin{minipage}[b]{0.45\textwidth}
        \includegraphics[width=\textwidth]{image/172123.png}
    \end{minipage}
    \caption{Task: \textbf{atypicality}; model outputs are in Table \ref{table:error_analysis_1}}
    \label{fig:error_analysis_1_img}
\end{figure*}

\begin{table*}
\small
\centering
    \begin{tabular}{p{0.12\textwidth} p{0.71\textwidth} p{0.09\textwidth}}
        \toprule
        \textbf{Model} & \textbf{Model Output}  & \textbf{Prediction} \\
        \midrule
        LLaVA-7B & explanation: The image on the left features a man sitting on a chair with a cup of soda in front of him. The text on the ad is about the risks of consuming sugary drinks and the importance of cutting one's portion sizes. This is a typical advertisement that uses a real-life scenario to convey a message. On the other hand, the image on the right is more unusual and abnormal. It features a large, brown, cylindrical structure that resembles a bottle or a vase. The text on this ad is about the risks of consuming sugary drinks and the importance of cutting one's portion sizes. The presence of the large structure in the ad is unexpected and out of place, making it more unusual and abnormal compared to the first image. answer: 2 & right (2) \\
        \midrule
        LLaVA-13B & The image on the left is more abnormal and unusual because it features a person with a significantly larger waistline than the rest of their body, which is not a typical human proportion. \textbf{The image on the right shows a large, ancient-looking structure, which is not an unusual sight in itself, but it is presented in a way that suggests it is a part of the advertisement, which is not a common practice for advertising.} & left (1)\\
        \midrule
        InternVL2-8B & explanation: The left advertisement features a person with a large belly, which is a common depiction in media to represent obesity. The use of a person in the image is typical in health campaigns to visually convey the message about portion sizes and health risks. The right advertisement, on the other hand, uses the Parthenon in Athens, an ancient monument, to convey the message about portion sizes. Using a historical or cultural landmark in an advertisement is less common and more unusual compared to using a person. Therefore, the right advertisement is more abnormal and unusual. answer: 2 \textbf{\textit{\{repeating answer 4 times\}}} & right (2) \\
        \midrule
        GPT-4v & explanation: The advertisement on the right is more abnormal and unusual because it portrays a well-known architectural column, which is typically straight, as a bottle of Absolut Vodka. This is a surreal and artistic reinterpretation of a cultural landmark, intended to resemble the shape of the iconic Absolut bottle, which is not a normal sight in real-world settings. On the other hand, the advertisement on the left, while addressing a serious health issue regarding portion sizes and diabetes, doesn't alter common objects in an unexpected way but uses real-life elements (a person and soft drink cups) to convey its message. Therefore, the right image is more surreal and unconventional in its approach. answer: 2 & right (2)\\
        \bottomrule
    \end{tabular}
    \caption{Model output for ads pair in Figure \ref{fig:error_analysis_1_img}. The correct answer is right (2). We can see LLaVA-13B is the only model that gets the answer incorrectly and its explanation is also self-contradicting (the bold part); for InternVL2, the problem is that it always repeat the answer for multiple times, even when it answers correctly.}
    \label{table:error_analysis_1}
\end{table*}

%%%%%%%%%%%%%%%%%%%%%% Error Analysis 2 %%%%%%%%%%%%%%%%%%%%%%%%%%%%%%%%%%%%%%%%%%%%%%%
\begin{figure*}
    \centering
    \begin{minipage}[b]{0.45\textwidth}
        \includegraphics[width=\textwidth]{image/172667.png}
        % \caption{Flower one.}
    \end{minipage}
    \hfill
    \begin{minipage}[b]{0.45\textwidth}
        \includegraphics[width=\textwidth]{image/108011.jpg}
    \end{minipage}
    \caption{Task: \textbf{originality}; model outputs are in Table \ref{table:error_analysis_2}}
    \label{fig:error_analysis_2_img}
\end{figure*}

\begin{table*}
\small
\centering
    \begin{tabular}{p{0.12\textwidth} p{0.71\textwidth} p{0.09\textwidth}}
        \toprule
        \textbf{Model} & \textbf{Model Output}  & \textbf{Prediction} \\
        \midrule
        LLaVA-7B & explanation: The left image features a woman with a caption that reads ``When I sneeze, I wet my pants. Normal?'' This ad is unique because it addresses a common but often embarrassing issue, \textbf{using humor to connect with the audience}. The right image, on the other hand, shows a group of women in a bathing suit, with a caption that reads: \textbf{Seh die Schönheit! Seh die Schönheit! Seh die Schönheit! Seh die Schönheit! Seh die Schönheit! Seh die Schönheit! Seh die Schönheit!} \textbf{\textit{\{repeat lot of times\}}} & left (1) \\
        \midrule
        LLaVA-13B & The left image is more unique compared to the right one. The reason is that the left image features a modern woman with a contemporary hairstyle and clothing, while the right image is an old-fashioned advertisement with a vintage design and illustration style. \textbf{The juxtaposition of the modern woman with the vintage advertisement creates a contrast that makes the left image more unique.} & left (1) \\
        \midrule
        InternVL2-8B & explanation: The left advertisement is more unique compared to the right one because it uses a modern approach with a relatable and humorous question about a common pregnancy symptom, which can engage and resonate with potential customers. The use of a smiling woman and a direct question makes it more approachable and relevant to current audiences. In contrast, the right advertisement uses a vintage style and focuses on a specific product, which might not immediately connect with modern viewers as well. \textbf{The humor and relatability in the left ad make it stand out in the same product category}. Answer: 1 \textbf{\textit{\{repeating answer 4 times\}}} & left (1) \\
        \midrule
        GPT-4v & explanation: The advertisement on the left provides a candid, contemporary approach to discussing health issues, specifically addressing pregnancy-related concerns in a way that openly discusses a commonly experienced, yet often unspoken issue. This upfront and direct style can be considered unique in comparison to typical pregnancy services ads which may traditionally use softer language and imagery. On the right, the \textbf{Nivea advertisement} takes a more classic approach, using vintage-style artwork to represent the product in a historical context, \textbf{which is creative but not necessarily unique as it aligns with a familiar tradition of period-specific advertising.} This style, though distinctive, may not stand out as much in the skincare product category where vintage and nostalgic themes are frequently employed. answer: 1 &  left (1) \\
        \bottomrule
    \end{tabular}
    \caption{Model output for ads pair in Figure \ref{fig:error_analysis_2_img}. The correct answer is right (2). ALLaVA-13B is the only one that incorrectly combined the visual element in the left image (``modern woman'') and the right image (``vintage advertisement'') together as if they are in the same ad (``the juxtaposition of...''); LLaVA-7B and InternVL2-8B are able to capture the humor in the left ad but still make the mistake. In particular, LLaVA-7B is stuck in consistently generating ``Seh die Schönheit!'', which is a description of the left image, without even get a chance to get to the second image. This might be the reason for its failure. Note that GPT-4v is the only model that recognizes the brand name (``Nivea''), highlighting its strong capability in recognizing details.}
    \label{table:error_analysis_2}
\end{table*}


%%%%%%%%%%%%%%%%%%%%%%%%%%%%%%%%%%%%%%%%%%%%%%%%%%%%%%%%%%%%%%%%%%%%%%%%%%%%%%%%%%%%%%%%%%%%%%%
\subsection{Distribution Modeling Examples}
\label{sec:appendix_distribution_output}
We have three examples with all the scoring metrics; see Figure \ref{fig:output_example_1}, \ref{fig:output_example_2}, \ref{fig:output_example_3}. We have also plotted WordClouds for the \textit{reasoning} part of the output (Figure \ref{fig:wordcloud_1}, \ref{fig:wordcloud_2}, \ref{fig:wordcloud_3}). As shown in those word clouds, the commonly used phrases generally closely correspond to the task definition:  `creative', `message', `imagery', `effective' for creativity task, `unique', `novel', `habit' for originality task, and `atypical', `connection', `
unusual' for atypicality task).

%%%%%%%%%%%%%%%%%%%%%%%%%%%%%%%%%%%%%%%%%%%%%%%%%%%%%%%%%%%%%%%%%%%%%%%%%%%%%%%%%%%%%%%%%%%%%%%
\begin{figure*}
    \centering
    \begin{minipage}[b]{0.45\textwidth}
        \includegraphics[width=\textwidth]{image/kl_low_example.jpg}
        % \caption{Flower one.}
    \end{minipage}
    \hfill
    \begin{minipage}[b]{0.45\textwidth}
        \includegraphics[width=\textwidth]{image/kl_low_v2.png}
    \end{minipage}
    \caption{Example (A) and \texttt{creativity} predictions by models; complete output in Table \ref{table:output_example_1}}
    \label{fig:output_example_1}
\end{figure*}

\begin{table*}
\small
\centering
    \begin{tabular}{ccccc}
        \toprule
        \textbf{Aspect} & \textbf{Human} & \textbf{LLaVA-7B \textit{(KL)}} & \textbf{LLaVA-13B \textit{(KL)}}& \textbf{InternVL \textit{(KL)}}  \\
        \midrule
        Creativity & 2.60 & 3.00 (1.0054) & 2.76 (0.3986) & 3.00 (1.0054) \\
        \midrule
        Originality & 2.92 & 2.92 (0.9643) & 2.88 (0.3144) & 3.00 (1.0054)  \\
         \midrule
        Atypicality & 2.92 & 2.88 (0.6030) & 2.64 (0.1191) & 2.84 (0.2223) \\
        \bottomrule
    \end{tabular}
    \caption{Model output (average across 25 runs) and human ratings for Example (A), see ad image and distribution modeling result in Figure \ref{fig:output_example_1}; \textit{KL} refers to $KL(Human||Model)$}
    \label{table:output_example_1}
\end{table*}

%%%%%%%%%%%%%%%%%%%%%%%%%%%%%%%%%%%%%%%%%%%%%%%%%%%%%%%%%%%%%%%%%%%%%%%%%%%%%%%%%%%%%%%%%%%%%%%
\begin{figure*}
    \centering
    \begin{minipage}[b]{0.45\textwidth}
        \includegraphics[width=\textwidth]{image/kl_mid_example.jpg}
        % \caption{Flower one.}
    \end{minipage}
    \hfill
    \begin{minipage}[b]{0.45\textwidth}
        \includegraphics[width=\textwidth]{image/kl_mid_v2.png}
    \end{minipage}
    \caption{Example (B) and \texttt{creativity} predictions by models; complete output in Table \ref{table:output_example_2}}
    \label{fig:output_example_2}
\end{figure*}

\begin{table*}
\small
\centering
    \begin{tabular}{ccccc}
        \toprule
        \textbf{Aspect} & \textbf{Human} & \textbf{LLaVA-7B \textit{(KL)}} & \textbf{LLaVA-13B \textit{(KL)}}& \textbf{InternVL \textit{(KL)}}  \\
        \midrule
        Creativity & 2.60 & 2.52 (0.7701) & 2.60 (0.8803) & 2.88 (1.6395) \\
        \midrule
        Originality & 2.92 & 2.28 (0.2762) & 1.60 (0.0963) & 2.12 (0.1791) \\
         \midrule
        Atypicality & 2.92 & 1.76 (0.2560) & 1.68 (0.070) & 1.36 (0.5113) \\
        \bottomrule
    \end{tabular}
    \caption{Model output and human ratings for Example (B), see ad image and distribution modeling result in Figure \ref{fig:output_example_2}; \textit{KL} refers to $KL(Human||Model)$}
    % \vspace{-6mm}
    \label{table:output_example_2}
\end{table*}


%%%%%%%%%%%%%%%%%%%%%%%%%%%%%%%%%%%%%%%%%%%%%%%%%%%%%%%%%%%%%%%%%%%%%%%%%%%%%%%%%%%%%%%%%%%%%%%
% \newpage
\begin{figure*}
    \centering
    \begin{minipage}[b]{0.45\textwidth}
        \includegraphics[width=\textwidth]{image/kl_high_example.jpg}
        % \caption{Flower one.}
    \end{minipage}
    \hfill
    \begin{minipage}[b]{0.45\textwidth}
        \includegraphics[width=\textwidth]{image/kl_high_v2.png}
    \end{minipage}
    \caption{Example (C) and \texttt{creativity} predictions by models; complete output in Table \ref{table:output_example_3}}
    \label{fig:output_example_3}
\end{figure*}

\begin{table*}
\small
\centering
    \begin{tabular}{ccccc}
        \toprule
        \textbf{Aspect} & \textbf{Human} & \textbf{LLaVA-7B \textit{(KL)}} & \textbf{LLaVA-13B \textit{(KL)}}& \textbf{InternVL \textit{(KL)}}  \\
        \midrule
        Creativity & 2.60 & 2.64 (1.3395) & 2.20 (0.4060) & 2.00 (0.3070) \\
        \midrule
        Originality & 2.92 & 2.36 (0.7893) & 1.56 (0.0643) & 1.60 (0.3263) \\
         \midrule
        Atypicality & 2.92 & 1.80 (0.6703) & 1.60 (0.5048) & 1.16 (0.8899) \\
        \bottomrule
    \end{tabular}
    \caption{Model output and human ratings for Example (C), see ad image and distribution modeling result in Figure \ref{fig:output_example_3}; \textit{KL} refers to $KL(Human||Model)$}
    % \vspace{-6mm}
    \label{table:output_example_3}
\end{table*}


%%%%%%%%%%%%%%%%%%%%%%%%%%%%%%%%%%%%%%%%%%%%%%%%%%%%%%%%%%%%%%%%%%%%%%%%%%%%%%%%%%%%%%%%%%%%%%%
\clearpage
\begin{figure*}
    \centering
    
    \begin{minipage}[b]{0.25\textwidth}
        \includegraphics[width=\textwidth]{image/creativity_llava7b.png}
        \caption{Task: Creativity, Model: LLaVA-7B}
    \end{minipage}
    \hfill
    \begin{minipage}[b]{0.25\textwidth}
        \includegraphics[width=\textwidth]{image/creativity_llava13b.png}
        \caption{Task: Creativity, Model: LLaVA-13B}
    \end{minipage}
    \hfill
    \begin{minipage}[b]{0.25\textwidth}
        \includegraphics[width=\textwidth]{image/creativity_internvl.png}
        \caption{Task: Creativity, Model: InternVL2-8B}
        \label{fig:wordcloud_1}
    \end{minipage}
    
\end{figure*}

\begin{figure*}
    \centering
    \begin{minipage}[b]{0.25\textwidth}
        \includegraphics[width=\textwidth]{image/originality_llava7b.png}
        \caption{Task: Originality, Model: LLaVA-7B}
    \end{minipage}
    \hfill
    \begin{minipage}[b]{0.25\textwidth}
        \includegraphics[width=\textwidth]{image/originality_llava13b.png}
        \caption{Task: Originality, Model: LLaVA-13B}
    \end{minipage}
    \hfill
    \begin{minipage}[b]{0.25\textwidth}
        \includegraphics[width=\textwidth]{image/originality_internvl.png}
        \caption{Task: Originality, Model: InternVL2-8B}
        \label{fig:wordcloud_2}
    \end{minipage}
    
\end{figure*}

\begin{figure*}
    \centering
    \begin{minipage}[b]{0.25\textwidth}
        \includegraphics[width=\textwidth]{image/atypicality_llava7b.png}
        \caption{Task: Atypicality, Model: LLaVA-7B}
    \end{minipage}
    \hfill
    \begin{minipage}[b]{0.25\textwidth}
        \includegraphics[width=\textwidth]{image/atypicality_llava13b.png}
        \caption{Task: Atypicality, Model: LLaVA-13B}
    \end{minipage}
    \hfill
    \begin{minipage}[b]{0.25\textwidth}
        \includegraphics[width=\textwidth]{image/atypicality_internvl.png}
        \caption{Task: Atypicality, Model: InternVL2-8B}
        \label{fig:wordcloud_3}
    \end{minipage}
    
\end{figure*}

%%%%%%%%%%%%%%%%%%%%%%%%%%%%%%%%%%%%%%%%%%%%%%%%%%%%%%%%%%%%%%%%%%%%%%%%%%%%%%%%%%%%%%%%%%%%%%%
