\section{Literature Review}
\label{sec:lit_review}

\subsection{Theoretical Foundations of Rent-Seeking Behavior}

Rent-seeking, a concept first introduced by \citet{tullock2008welfare}, refers to efforts by individuals or enterprises to secure economic gains through manipulation of the institutional or informational environment without contributing new value. Early works framed rent-seeking as a source of inefficiency that diverts resources from productive uses, resulting in welfare losses. For example, \citet{acemoglu2015democracy} argue that rent-seeking reinforces class inequality by consolidating wealth among elites, thereby reducing social mobility. Similarly, studies by \citet{rodriguez2004inequality} and \citet{stiglitz2012price} document how rent-seeking exacerbates income inequality through distorted resource allocation.

Despite their contributions, these traditional models exhibit notable limitations. Many assume homogenous agents and static environments, failing to capture the dynamic and heterogeneous nature of modern economies. While \citet{krueger2008political} and \citet{zingales2017towards} emphasize the role of government and institutional structures in shaping rent-seeking, their models often overlook how technological changes can alter rent-seeking incentives. More recent studies, such as those by \citet{catalini2020some} on blockchain and \citet{prat2022attention} on digital markets, have extended the discussion by demonstrating that technological innovations can both reduce traditional information asymmetries and create novel rent-seeking opportunities. However, these works tend to analyze either the efficiency gains or the new distortions in isolation, leaving a gap in our comprehensive understanding of the dynamic interplay between technology and rent-seeking behavior.

\subsection{Development of Generative AI and Its Impact on the Information Environment}

The emergence of generative AI—exemplified by generative adversarial networks \citep{goodfellow2014generative} and large-scale language models \citep{brown2020language}—marks a fundamental shift from static prediction to active information synthesis. This technology not only enhances content creation but also transforms the underlying information environment by increasing data accessibility and transparency. Research in this area has shown promising applications in labor markets \citep{zarifhonarvar2024economics, hui2024short}, market simulation \citep{takahashi2019modeling, xu2019modeling}, and even mechanism design \citep{calvano2020artificial}.

Nevertheless, the literature on generative AI also reveals significant controversies and limitations. While many studies praise generative models for reducing reliance on manual labeling and democratizing content production, concerns remain about ethical risks such as bias, discrimination, and the potential for misuse \citep{weidinger2022taxonomy, vinuesa2020role}. Critically, the capability of generative AI to produce synthetic content creates new avenues for rent-seeking—enabling agents to manipulate information ecosystems through content manipulation and algorithmic interference. Although some researchers acknowledge these risks, a comprehensive framework that captures both the transparency-enhancing benefits and the novel rent-seeking strategies enabled by generative AI is still lacking.

\subsection{Research Gaps}

In summary, current research lacks a unified, dynamic framework that captures the interplay between traditional and AI-enabled rent-seeking. Our study addresses this gap by proposing a dynamic economic model that accounts for the evolving information environment and the strategic adaptations of agents. This approach not only reveals the limitations of existing theories but also offers valuable insights for policymakers balancing innovation with social welfare.