\section{Introduction}

%%% parental involvement significance and challenges
Parental involvement is essential to children's early education \cite{ma2016meta, harris2008parents}. However, despite parents' desire to participate in their children's learning, many of them face challenges, including limited capability, availability, and motivation \cite{green2007parents, ho2024s, hara1998parent}, which hinder effective participation. For example, one parent may struggle to come up with developmentally appropriate questions to ask their child, while another may know what to discuss but feel unable to dedicate sufficient time due to demanding work-home schedules. 

%%% AI-enabled robots as potential solution 
AI-assisted robots hold significant potential to transform both the scope and methods of parental involvement in children's early education. Prior work has highlighted that AI-powered systems can personalize educational experiences to address individual learning needs \cite{zhang2024mathemyths, xu2022elinor, xu2024artificial} and that social robots can facilitate rich interactions that support children's cognitive and emotional development in the physical world \cite{ho2023designing, leyzberg2012physical, belpaeme2018social, kim2024understanding, lee2022unboxing}. Yet, existing studies have primarily focused on addressing child-centric needs, overlooking parental roles. Additionally, parents often raise concerns about privacy, misinformation, and content appropriateness \cite{oswald2020psychological, david2007electronic, howard2021digital}, emphasizing the need for parental supervision to ensure alignment of AI and robots with parents' educational goals \cite{ho2024s, han2024teachers}.

%%% Research gap for developing such AI-enabled robots
In light of these gaps, it is essential to understand the contexts in which parents support their children's learning, investigate how AI-assisted robots can be designed to help parents overcome barriers in these contexts, and address concerns regarding children's interactions with these technologies. We propose that providing parents with \textit{flexible control} over AI-generated content and \textit{collaborative involvement} with educational robots can effectively support parents in responding to dynamic contexts and barriers, balancing the benefits and risks of AI-assisted robots. More specifically, this paper explores the following research questions:
\begin{itemize}
    \itemsep0.4em
    \item \textbf{RQ1:} What contexts do parents encounter when involving in young children's learning activities?
    \item \textbf{RQ2:} How do parents perceive AI-generated learning content for young children?
    \item \textbf{RQ3:} How would parents prefer to collaborate with LLM on supervising content creation under different contexts?
    % \item \textbf{RQ4:} How would parents prefer to collaborate with an AI-assisted robot to engage in learning activities with their children under different contexts?
\end{itemize}

\begin{itemize}
    \item \textbf{RQ4:} How would parents prefer to collaborate with an AI-assisted robot to engage in learning activities with their children under different contexts?
\end{itemize}

%%% what we have done to address the questions
%% design
To address these questions, we developed two tools: (1) a card-based activity kit named \texttt{SET} for systematically understanding parental involvement contexts, and (2) a prototype AI-assisted robot named \texttt{PAiREd} to explore flexible parent-AI and parent-robot collaboration mechanisms. The \texttt{SET} kit, with its name representing the key factors of \textbf{S}kill, \textbf{E}nergy, and \textbf{T}ime \cite{ho2024s, green2007parents}, helps parents reflect on their experiences in supporting their children's learning. Parents use the kit to create scenario cards that capture relatable contexts framed around these factors, serving as both study findings and research tools for contextualizing parent's interaction with the AI-assisted robot. In addition, the \texttt{PAiREd} prototype (\textbf{P}arenting with \textbf{Ai}-assisted \textbf{R}obot for \textbf{Ed}ucation) enables parents to generate, edit, and revise storybook content in collaboration with a large language model (LLM). During parent-child reading sessions, parents can adjust task delegation between themselves and the robot via a flexible mechanism, allowing control over the autonomy of the AI-assisted robot. This approach allows parents to balance their involvement based on the specific needs of different scenarios.

We evaluated the \texttt{SET-PAiREd} system through a four-stage study. First (\textbf{RQ1}), parents used \texttt{SET} to document real-life involvement scenarios, revealing contextual needs and scenarios. Second (\textbf{RQ2}), interviews about AI-generated content showed that parents value adaptability, efficiency, and affordability, but worry about age appropriateness, inaccuracies, and over-reliance. Third (\textbf{RQ3}), parents co-created learning content for two books using \texttt{PAiREd}, indicating that difficulty, variety, and quality drive their evaluation of LLM-generated content. Under the \texttt{SET} scenarios, their collaboration with the LLM varied depending on their time, energy, trust in the LLM, and interest in editing content. Finally (\textbf{RQ4}), parents read the books with their children, adjusted role delegation between themselves and the robot according to the scenarios, and provided feedback. We identified several collaboration patterns, including using \textit{parent takeover} mode with LLM-generated content, \textit{collaborative} modes guided by parent skill and motivation, and \textit{robot takeover} mode under parental supervision or with reviewed LLM content. Our work makes the following contributions:
\begin{enumerate} 
    \item \textit{Prototype Artifact}: We developed the \texttt{SET} activity kit to map parents' real-life contexts around their skill, energy, and time. In addition, we developed and tested \texttt{PAiREd}, a system supporting responsible supervision of LLM-generated content and enabling flexible parent-robot collaboration.
    \item \textit{Novel Insights}: Field testing provided insights into systematic understanding of parental involvement scenarios through \texttt{SET} (RQ1), parents' perceptions of AI-generated content (RQ2), parents' preferences for \texttt{PAiREd} during parent-AI collaboration in content creation (RQ3), and parent-robot collaboration in facilitating learning activities (RQ4). The insights reveal novel interaction paradigms for human-robot collaboration in the context of parenting and education.
    \item \textit{Design Recommendations}: We synthesized our findings into actionable recommendations to enhance parental involvement through AI-assisted learning companion robots.
\end{enumerate}