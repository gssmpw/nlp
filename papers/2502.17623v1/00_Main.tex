
\documentclass[sigconf]{acmart}


%%% Revision packages
%\usepackage{color}
%\usepackage{soul}

\usepackage{acmart-taps}
\usepackage{stfloats}

%	\newif\ifCOMMENTS
%	\COMMENTStrue
	
%	\ifCOMMENTS
%	\newcommand{\revise}[2]{\textcolor{red}{\sout{#1}}\hl{#2}}
%	\else
%	\newcommand{\revise}[2]{#2}
%	\fi

%%% TODO packages
% \newcommand{\todo}[1]{{\color{orange}\textbf{To do:} #1}}


\usepackage{microtype}
\usepackage{balance}
\usepackage{booktabs}
\usepackage{graphicx}
\usepackage{multirow}
% \usepackage[figurename=Figure]{caption}
% \renewcommand{\figurename}{Figure}
% \usepackage{caption}
% \captionsetup[figure]{name={Figure},}


%% Fonts used in the template cannot be substituted; margin 
%% adjustments are not allowed.
%%
%% \BibTeX command to typeset BibTeX logo in the docs
\AtBeginDocument{%
  \providecommand\BibTeX{{%
    \normalfont B\kern-0.5em{\scshape i\kern-0.25em b}\kern-0.8em\TeX}}}

\makeatletter
\def\@ACM@copyright@check@cc{}
\makeatother

\copyrightyear{2025}
\acmYear{2025}
\setcopyright{cc}
\setcctype{by}
\acmConference[CHI '25]{CHI Conference on Human Factors in Computing Systems}{April 26-May 1, 2025}{Yokohama, Japan}
\acmBooktitle{CHI Conference on Human Factors in Computing Systems (CHI '25), April 26-May 1, 2025, Yokohama, Japan}\acmDOI{10.1145/3706598.3713330}
\acmISBN{979-8-4007-1394-1/2025/04}


\acmSubmissionID{7119}

\begin{document}


\title[\texttt{SET-PAiREd}: Designing for Parental Involvement with an AI-Assisted Educational Robot]{\texttt{SET-PAiREd}: Designing for Parental Involvement in Learning with an AI-Assisted Educational Robot}


%% authors
\author{Hui-Ru Ho}
\orcid{0009-0000-3701-2521}
\affiliation{%
  \institution{Department of Computer Sciences\\
  University of Wisconsin--Madison}
  %\streetaddress{}
  \city{Madison}
  \state{Wisconsin}
  \country{USA}
}
\email{hho24@cs.wisc.edu}

\author{Nitigya Kargeti}
\orcid{0000-0001-5970-7332}
\affiliation{%
  \institution{Department of Computer Sciences\\
  University of Wisconsin--Madison}
  %\streetaddress{}
  \city{Madison}
  \state{Wisconsin}
  \country{USA}
}
\email{kargeti@wisc.edu}

\author{Ziqi Liu}
\orcid{0009-0007-8755-5744}
\affiliation{%
  \institution{Department of Computer Sciences\\
  University of Wisconsin--Madison}
  %\streetaddress{}
  \city{Madison}
  \state{Wisconsin}
  \country{USA}
}
\email{ziqil@cs.wisc.edu}

\author{Bilge Mutlu}
\orcid{0000-0002-9456-1495}
\affiliation{%
  \institution{Department of Computer Sciences\\
  University of Wisconsin--Madison}
  %\streetaddress{}
  \city{Madison}
  \state{Wisconsin}
  \country{USA}
}
\email{bilge@cs.wisc.edu}


%%
%% The "author" command and its associated commands are used to define
%% the authors and their affiliations.
%% Of note is the shared affiliation of the first two authors, and the
%% "authornote" and "authornotemark" commands
%% used to denote shared contribution to the research.

\renewcommand{\shortauthors}{Ho et al.}

\begin{abstract} %<150words
AI-assisted learning companion robots are increasingly used in early education. Many parents express concerns about content appropriateness, while they also value how AI and robots could supplement their limited skill, time, and energy to support their children's learning. We designed a card-based kit, \texttt{SET}, to systematically capture scenarios that have different extents of parental involvement. We developed a prototype interface, \texttt{PAiREd}, with a learning companion robot to deliver LLM-generated educational content that can be reviewed and revised by parents. Parents can flexibly adjust their involvement in the activity by determining what they want the robot to help with. We conducted an in-home field study involving 20 families with children aged 3--5. Our work contributes to an empirical understanding of the level of support parents with different expectations may need from AI and robots and a prototype that demonstrates an innovative interaction paradigm for flexibly including parents in supporting their children.
\end{abstract}

\begin{CCSXML}
<ccs2012>
   <concept>
       <concept_id>10003120.10003121.10003124.10011751</concept_id>
       <concept_desc>Human-centered computing~Collaborative interaction</concept_desc>
       <concept_significance>500</concept_significance>
       </concept>
   <concept>
       <concept_id>10003120.10003121.10011748</concept_id>
       <concept_desc>Human-centered computing~Empirical studies in HCI</concept_desc>
       <concept_significance>500</concept_significance>
       </concept>
   <concept>
       <concept_id>10010405.10010489.10010491</concept_id>
       <concept_desc>Applied computing~Interactive learning environments</concept_desc>
       <concept_significance>500</concept_significance>
       </concept>
 </ccs2012>
\end{CCSXML}

\ccsdesc[500]{Human-centered computing~Collaborative interaction}
\ccsdesc[500]{Human-centered computing~Empirical studies in HCI}
\ccsdesc[500]{Applied computing~Interactive learning environments}

\keywords{Human-robot interaction, human-AI interaction, large language model (LLM), flexible parental involvment, parent-child dyads, informal learning, young children, home, field study}

%%%%% teaser %%%%%
\begin{teaserfigure}
  \includegraphics[width=\textwidth]{figures/figure-teasor-hho-v2.pdf}
  %\vspace{-12pt}
  \caption{We explored parental involvement scenarios using the \texttt{SET} activity kit, examined parents' perceptions of AI-generated content, and analyzed their use of \texttt{PAiREd} for collaborating with LLM and robots to support their children's learning.}
  \label{fig:individual}
   %\vspace{-6pt}
\end{teaserfigure}
%%%%%%%%%%%%%%%%%%
\sloppy
\maketitle





\section{Introduction}

\begin{figure*}
    \centering
    \includegraphics[width=\textwidth]{figures/Introduction.pdf}
    \caption{Showing the novel problem statement applied to traffic prediction use case. Multiple unstructured observations from the past are used to reconstruct a hidden traffic state from which a full traffic state is forecast with a set of query locations. }
    \label{fig:intro}
\end{figure*}

% Was sagen denn die anderen warum Traffic Prediction gut ist? 
Forecasting the traffic in the near future is an important task for city management.
Data from the near past is used to predict future traffic states with spatio-temporal Graph Neural Networks \cite{bui22}.
Accurate prediction provides the opportunity to optimize traffic flow, reduce traffic jams and increase air quality \cite{Po19}.

% Wieso ist Sparsity in allen Dimensionen wichtig.
While traffic prediction relies on the availability of data from traffic sensors, there exists a plethora of reasons why sensors may stop working temporarily, such as simple errors, energy saving, or overloaded communication systems.
Considering small- or medium-sized cities, the coverage of sensors may be low because the sensors are too expensive or not available.
Also, the sensors are typically static and do not adapt to changes in the traffic flow (e.g. caused by a construction site), which motivates moving sensors that for example could be mounted on cars. 
However, both missing and moving sensors introduce sparsity, since measurements may not be available for all locations at all times.
This sparsity must be explicitly addressed in traffic prediction for a realistic application scenario, which is illustrated in figure \ref{fig:intro}.
From one hour of data on Sunday morning, only few observations of the traffic state are available at each timestep.
The number of observations may differ throughout the observed time and the observation itself can be distributed arbitrarily in the city. 
We assume a relatively low number of sensors to account for resource saving and sensor failure in our proposed framework SUSTeR.
The task is to predict the dense traffic state one timestep after the observations at all possible sensor locations.
We study this problem on the traffic dataset Metr-LA and PEMS-BAY to test our assumption that only a fraction of the sensor values would be enough for good predictions.
By modifying an existing traffic dataset, we are able to compare our results from very sparse observations to the bottom line with all information available.
A successful study will provide insights in how sensors in new cities can be reduced before installing them and further mobile sensors would save more resources and are able to adapt to new traffic situations.
We argue that in order to be adaptable to other cities and changes in traffic flows, prior information like the road network should be neglected and just the sparse observations considered.
This comes with the added benefit of making our solution applicable in regions where no openly available road network is maintained or pathways change frequently (e.g. flood areas, animal observations). 


The aforementioned problem is novel and more challenging than the commonly considered traffic prediction problem, since there exist very few observations in each input sample.
Current works for the traffic prediction problem do not consider any missing values. \cite{Li2021, Shao22}
A common method among state of the art approaches is the usage of Graph Neural Networks on graphs that model the sensor network \cite{bui22}.
The values of a sensor are applied to the same graph node for each timestep which prohibits any non-stationary sensors . 
With fixed sensor locations, the resulting sensor network is highly correlated with the road network.
Streets connecting two intersections with sensors should be also an interesting point for correlations in the sensor network.
However, variable observations and high temporal sparsity rates can not be modeled adequately in a static network.
We show in our experiments that the road network has only a small influence on the traffic predictions.

Besides the traffic prediction for future timesteps, some works explore the field of traffic speed imputation \cite{Cini22, Cuza22} where missing sensor values are predicted.
But the amount of missing values is assumed to be at most 80\%, which on average are still over 40 given sensors in each timestep in the Metr-LA dataset with a total of 207 sensors.
We consider up to 99.9\% missing values which are on average 2.4 observations in each timestep that are used as input.
Such high sparsity rates drastically decrease the chance that multiple values are present in one input sample from the same sensor location, which makes it challenging to recognize and learn temporal correlations for each location on its own.

High sparsity rates (>95\%) result in few sensor values, but if a reconstruction of the traffic state would be possible, we question if spatio-temporal graphs require nodes for each sensor.
In SUSTeR we utilize only a small amount of graph nodes for the encoding of information and do not relate such nodes to the sensor network.
We call this the hidden graph (see figure \ref{fig:intro}), which is still able to reconstruct the complete traffic state.
Due to the reduced number of nodes SUSTeR achieves faster runtimes, as shown in the experiments.
This hidden graph is not embedded directly in the spatial domain, which is why the assignment of observations, as well as the querying of the future traffic, is done with an encoder and a decoder, implemented as neural networks.
The decoding from the hidden graph to future values depends on a set of query locations.
Figure \ref{fig:intro} shows the query locations as given from outside and in combination with the reconstructed traffic state the future values are predicted.

To construct the hidden graph we encode observations from each timestep into from multiple graphs, one for each timestep. 
The graphs are created in a residual style and information is added to the node embeddings from the previous timesteps.
We choose this method to incorporate all timesteps equally into the hidden state because the redundant information along the past is non-existing for high sparsity rates.
From the sequence of graphs where our framework inserted the observations step by step we apply STGCN \cite{Yu18}, an algorithm for traffic prediction to find and learn the spatio-temporal correlations on our small number of graph nodes.
The first future timestep of the STGCN is our hidden graph in which the traffic state is reconstructed. 

% Recent work has an implicit embedding of the graph nodes into the spatial domain as the assignment from the sensor to graph node is fixed one by one.
% Because the graph has the same structure as the road network spatio-temporal correlations can be learned between those sensors.
% We reduce the number of nodes and use a non-linear assignment learned data-driven from the observations.

We find in the experiments that SUSTeR outperforms the plain STGCN and modern traffic prediction frameworks like D2STGNN for high sparsity rates $(\geq 99\%)$.
This is equivalent to only $0.2$ to $2.4$ observation for each timestep on average.
SUSTeR uses fewer parameters than the baselines and can train faster and with less training data.
Our main contributions can be summarized as follows:
\begin{itemize}
    \item We introduce a sparse and unstructured variant of the traffic prediction problem with sparsity in all dimensions. The sensors report only a fraction of their values and are arbitrarily distributed in the spatial domain.
    \item We propose SUSTeR, a framework around the STGCN architecture, which maps sparse observations onto a dense hidden graph to reconstruct the complete traffic state.
    Our code is available at github.\footnote{https://github.com/ywoelker/SUSTeR}
    \item We conducts experiments that show that SUSTeR outperforms the baselines in very sparse situations ($\geq 95\%$) and has a competitive performance in low sparsity rates.
    % \item SUSTeR trains a third faster than the next competitor.
\end{itemize}

\section{Related Work}
\label{sec:related_work}

\subsection{Robustness of Audio-Visual Speech Recognition} 

The robustness of AVSR systems has significantly advanced by integrating auditory and visual cues to improve speech recognition, especially in noisy environments. Conventional ASR methods have evolved from relying solely on audio signals \cite{schneider2019wav2vec, gulati2020conformer, baevski2020wav2vec, hsu2021hubert, chen2022wavlm, chiu2022self, radford2023robust} to incorporating visual data from speech videos \citep{makino2019recurrent}.
The multimodal AVSR methods \citep{pan2022leveraging, shi2022learning, seo2023avformer, ma2023auto} have enhanced robustness under audio-corrupted conditions, leveraging visual details like speaker's face or lip movements as well as acoustic features of speech. These advancements have been driven by various approaches, including end-to-end learning frameworks \citep{dupont2000audio, ma2021end, hong2022visual, burchi2023audio} and self-supervised pretraining \citep{ma2021lira, qu2022lipsound2, seo2023avformer, zhu2023vatlm, kim2025multitask}, which focus on audio-visual alignment and the joint training of modalities~\citep{zhang2023self, lian2023av, haliassos2022jointly, haliassos2024braven}.


Furthermore, recent advancements in AVSR highlight the importance of visual understanding alongside audio \citep{dai2024study, kim2024learning}. While initial research primarily targeted audio disturbances \citep{shi2022robust, hu2023hearing, hu2023cross, chen2023leveraging}, latest studies increasingly focus on the visual robustness to address challenges such as real-world audio-visual corruptions~\citep{hong2023watch, wang2024restoring, kim2025multitask} or modality asynchrony~\citep{zhang2024visual, fu2024boosting, li2024unified}. These efforts remark a shift towards a more balanced use of audio and visual modalities. Yet, there has been limited exploration in scaling model capacity or introducing innovative architectural designs, leaving room for further developments in AVSR system that can meticulously balance audio and visual modalities.



\subsection{MoE for Language, Vision, and Speech Models}

Mixture-of-Experts (MoE), first introduced by \citet{jacobs1991adaptive}, is a hybrid structure incorporating multiple sub-models, \ie experts, within a unified framework. The essence of sparsely-gated MoE \cite{shazeer2017outrageously, lepikhin2021gshard, dai2022stablemoe} lies in its routing mechanism where a learned router activates only a subset of experts for processing each token, significantly enhancing computational efficiency. Initially applied within LLMs using Transformer blocks, this structure has enabled unprecedented scalability \cite{fedus2022switch, zoph2022st, jiang2024mixtral, guo2025deepseek} and has been progressively adopted in multimodal models, especially in large vision-language models (LVLMs) \cite{mustafa2022multimodal, lin2024moellava, mckinzie2025mm1}.
Among these multimodal MoEs, \citet{zhu2022uni, shen2023scaling, li2023pace, li2024uni} and \citet{lee2025moai} share the similar philosophy to ours, assigning specific roles to each expert and decoupling them based on distinct modalities or tasks. These models design an expert to focus on specialized segments of input and enhance the targeted processing.

Beyond its applications in LLMs and LVLMs, the MoE framework has also been applied for speech processing \cite{you2021speechmoe, you2022speechmoe2, hu2023mixture, wang2023language}, where it has shown remarkable effectiveness in multilingual and code-switching ASR tasks. In addition, MoE has been employed in audio-visual models \cite{cheng2024mixtures, wu2024robust}, although they primarily focus on general video processing and not specifically on human speech videos. These approaches leverage MoE to model interactions between audio and visual tokens without directly processing multimodal tokens.
Our research advances the application of the MoE framework to AVSR by designing a modality-aware hierarchical gating mechanism, which categorizes experts into audio and visual groups and effectively dispatches multimodal tokens to each expert group. 
This tailored design enhances the adaptability in managing audio-visual speech inputs, which often vary in complexity due to diverse noise conditions.

\section{Card-Based Activity Kit Design} \label{sec-card}

\begin{table*}[!t]
    \caption{Factors and dimensions used in the Card-based Activity Kit}
    \label{tab:kit-example}
    \centering
    \small
    \begin{tabular}{p{0.12\linewidth}p{0.4\linewidth}p{0.4\linewidth}}%
        \toprule
         \textbf{Factor-level} & \textbf{Description} & \textbf{Example} \\
        \midrule
         Skill-high & Parents feels \textit{very} confident in a specific task or topic. & ``\textit{I am good at math-related activities.}'' \\
         \midrule
         Skill-low & Parents feels \textit{not} confident in a specific task or topic. & ``\textit{I am not good at coming up with good questions to ask.}'' \\
         \midrule
         Energy-high & Parents feels \textit{highly} motivated. & ``\textit{I am highly motivated when my child invites me.}'' \\
         \midrule
         Energy-low & Parents feels \textit{not} motivated. & ``\textit{I am not motivated when I am tired.}'' \\
         \midrule
         Time-high & Parents are \textit{available and present}. & ``\textit{I am available and present when my work is done.}'' \\
         \midrule
         Time-low & Parents \textit{need to be absent}. & ``\textit{I need to be absent when my younger child is crying.}'' \\     
         \bottomrule
    \end{tabular}
\end{table*}

To systematically explore parents' contextual, real-life scenarios where varying levels of parental involvement in children's learning may occur, we designed a card-based activity kit named \textbf{\texttt{SET}}, representing \textbf{\texttt{S}}kill, \textbf{\texttt{E}}nergy, and \textbf{\texttt{T}}ime. The design of the kit is grounded in frameworks from education \cite{green2007parents} and HCI \cite{ho2024s}. Specifically, we leveraged the ``\textit{Parent Perceived Life Context}'' construct from the Hoover-Dempsey and Sandler (HDS) framework for parental involvement \cite{green2007parents} and the key factors influencing parent-robot collaboration defined by \citet{ho2024s}. We detailed the theoretical foundation in Section \ref{sec-rw-2.1}.

The kit consists of three primary materials: \texttt{SET-banners}, \texttt{SET-} \texttt{experience banks}, and \texttt{SET-scenario cards}\footnote{\url{https://osf.io/zfksg/?view_only=b59bd41287f543ce82ab85950aaf004f}}. Each material is characterized by the same two-dimensional factors (\textit{i.e.,} skill (high/low), energy (high/low), and time (high/low)) but is designed in different formats to serve distinct purposes during the card activity (see Table \ref{tab:kit-example} for details about each factor and its dimensions). Although the factors theoretically exist on a continuous spectrum, we divided them into binary dimensions to ensure the scale of the activity is manageable for participants. Below, we describe the purpose of each material. For more details on how the kit was used, refer to Section \ref{sec-procedure}.

\begin{enumerate}
    \item The \texttt{SET-banners} present a hierarchical diagram that depicts how the three two-dimensional factors combine to create \textit{eight} distinct scenarios (Figure \ref{fig:card-kit}), forming the \textit{eight} \texttt{SET-scenario cards}.
    \item The \texttt{SET-experience banks} enable participants to document real-life examples for each dimension of every factor using post-it notes. This activity encourages focused reflection on individual factors and dimensions, serving as a stepping stone for constructing scenarios that integrate all three factors during the \texttt{SET-scenario cards} activity (Figure \ref{fig:card-kit}).
    \item The \texttt{SET-scenario cards} guide participants in creating \textit{eight} real-life scenarios, each defined by unique combinations of the three two-dimensional factors. The front of each card displays the dimensional status of the factors, while the back includes detailed descriptions of each factor's status and a blank space for participants to document a scenario that aligns with the corresponding dimensions (Figure \ref{fig:card-kit}).
\end{enumerate}

\begin{figure*}[b!]
  \includegraphics[width=\textwidth]{figures/figure-card-hho-2.pdf}
   \vspace{-6pt}
  \caption{The \texttt{SET} Card-Based Activity Kit. (1) Banners are used to visualize a hierarchical diagram. (2) Experience banks allow participants to generate examples for each factor dimension. (3) Scenario cards guide participants to create real-life scenarios.}
  \label{fig:card-kit}
  \vspace{-3pt}
\end{figure*}




\section{System Design} \label{sec-system}

We designed a prototype system, \texttt{PAiREd}, an LLM-driven interface integrated with an educational robot that uses LLMs to generate learning content based on established educational frameworks for reading activities. The system enables parents to review and revise LLM-generated learning content \textit{before} the activity and adjust their involvement by modifying the role delegation between the robot and themselves \textit{during} the activity. The system consists of three core design components: (1) \textit{LLM-driven content generation}, (2) \textit{Interface design}, and (3) \textit{Robot interaction design} (Figure \ref{fig:syste-diagram}).

\begin{figure*}[t!]
  \includegraphics[width=\textwidth]{figures/figure-system-hho.pdf}
   \vspace{-6pt}
  \caption{\texttt{PAiREd} System Architecture Overview. The system consists of three key components: prompt instructions and data, the web application (which includes both the editor and activity interfaces), and the robot interaction module.}
  \label{fig:syste-diagram}
   \vspace{-10pt}
\end{figure*}

\subsection{LLM-driven Content Generation}
% limitation of gpt-4o
Content generation is powered by a state-of-the-art LLM, ChatGPT-4o \cite{achiam2023gpt} developed by OpenAI\footnote{OpenAI ChatGPT-4o: \url{https://openai.com/index/hello-gpt-4o/}} through prompt engineering \cite{sahoo2024systematic}. To generate the learning content for each book page, the input we provided for GPT-4o includes: (1) \textit{prompts}\footnote{Prompts: \url{https://osf.io/zfksg/?view_only=b59bd41287f543ce82ab85950aaf004f}}: instructions to generate learning content, (2) \textit{book image}: an image of the chosen book page, (3) \textit{visual context}\footnote{Visual context: \url{https://osf.io/zfksg/?view_only=b59bd41287f543ce82ab85950aaf004f}}: a manually-created JSON-based dataset containing structured visual information of the book page, and (4) \textit{educational frameworks} \footnote{Frameworks: \url{https://osf.io/zfksg/?view_only=b59bd41287f543ce82ab85950aaf004f}}: theoretical-driven frameworks defining critical pre-school concepts for math and literacy. The books, visual contexts, and educational frameworks are placed in our database and retrieved using prompt instructions. We explain \textit{visual context} and \textit{educational frameworks} below.

\subsubsection{Visual Context} 
The ChatGPT-4o model faces limitations in quantitative and spatial reasoning on images and recognition of animated or anthropomorphic drawings (\textit{e.g.,} mistaking clothed animals for humans), limiting its ability to accurately auto-generate question-answer pairs for literacy and math based on storybook images. To enhance content generation accuracy, we created a JSON-based dataset to capture the \textit{visual context} of each storybook page. Each page is represented as a unique entity with detailed descriptions of objects (\textit{e.g.,} characters, animals, and environmental elements), properties, as well as their spatial relationships. This context was embedded in the prompt design, instructing GPT-4o to retrieve precise object properties such as color, count, and location. Integrating this structured approach significantly improved the accuracy of ChatGPT-4o to create question-answer pairs based on storybooks, reducing object-identification errors and enhancing consistency across pages. Systematic evaluations by the first and third authors confirmed these improvements.

% \begin{figure}[b!]
%   \includegraphics[width=\columnwidth]{figures/figure-editor-amy.pdf}
%    \vspace{-6pt}
%   \caption{Top: Editor Interface of the PAiREd system. This interface allows users to navigate book content, modify LLM-generated learning content through regeneration, or manually edit it. Bottom: Activity Interface of the PAiREd system. This interface enables parents to flexibly adjust their involvement and seamlessly share responsibilities with the robot through the mode-switching and role-delegation mechanisms. }
%   \label{fig:editor}
%    \vspace{-6pt}
% \end{figure}

\begin{figure*}[b!]
  \includegraphics[width=\textwidth]{figures/figure-editor-amy.pdf}
     \caption{Editor Interface of the PAiREd system. This interface allows users to navigate book content, modify LLM-generated learning content through regeneration, or manually edit it.}
  \label{fig:editor}
\end{figure*}

\begin{figure*}[b!]
\centering
  \includegraphics[width=\textwidth]{figures/figure-activity-amy.pdf}
  \caption{Activity Interface of the PAiREd system. This interface enables parents to flexibly adjust their involvement and seamlessly share responsibilities with the robot through the mode-switching and role-delegation mechanisms.}
  \label{fig:activity}
   \vspace{-10pt}
\end{figure*}
 
\subsubsection{Educational Frameworks for Math and Literacy} 
We developed frameworks outlining varying proficiency levels in math \cite{purpura2013informal, engel2013teaching} and literacy \cite{kaminski2014preschool} concepts for preschoolers, drawing on educational psychology research. Math and literacy were selected as they are foundational skills for preschoolers \cite{weiland2013impacts}. The math framework incorporated informal numeracy skills identified by \citet{purpura2013informal} and aligned them with the four proficiency levels proposed by \citet{engel2013teaching}. The literacy framework was based on the PELI assessment benchmarks \cite{kaminski2014preschool}.

\subsection{Interface Design}
We developed a web-based interface using a React front-end\footnote{\url{https://reactnative.dev}} and a FAST API back-end,\footnote{\url{https://fastapi.tiangolo.com}} supported by a MongoDB database.\footnote{\url{https://www.mongodb.com}} The interface includes three key components: (1) \textit{LLM content generation}: parents can request LLM to generate a new set of learning content (\textit{e.g.,} question-answer pairs); (2) \textit{Editor interface}: parents can review and revise LLM-generated content to varying extents; and (3) \textit{Activity interface}: parents can flexibly adjust their involvement and role delegation between themselves and the robot. 

\subsubsection{LLM Content Generation}
Parents can choose a book from the digital library in the system and generate a new set of learning content using the LLM embedded within the platform. Before generating content, parents are prompted to select \textit{grade level} (\textit{e.g.,} preschool) and \textit{subject} (\textit{e.g.,} math or literacy). Once the content is generated, parents can either \textit{edit and review} the material or \textit{launch} it directly. For each page of the book, the LLM creates content following a structured approach, including three main components: (1) \textit{question and multiple choices}: a question related to a selected concept from the educational framework, with four answer options, one being correct; (2) \textit{explanation}: a description of how the correct answer is derived; (3) \textit{motivation}: encouraging words or a fun fact related to the content of the book page.


\subsubsection{Editor Interface}
After the LLM generated the content, if the user choose to \textit{edit and review}, the editor interface will display the content page by page. The main \textit{book content} is shown in the left panel, while the \textit{LLM-generated content} appears on the right. The \textit{concept}, initially selected by the LLM from the chosen educational framework, is displayed at the top of the book content. For the \textit{book content}, parents could navigate through the book and review it page by page. The \textit{LLM-generated content} was organized into its main components: \textit{questions and choices}, \textit{explanations}, and \textit{motivational content}. Parents had the option to either regenerate individual components or all components at once using the LLM, or manually edit the content. Additionally, the \textit{concept} was presented as a clickable drop-down list of all available concepts from the chosen framework. Hovering over the information icon provided details about each concept. If parents wanted to change the concept, they could select a new one from the list, triggering the LLM to automatically regenerate content based on the new concept (Figure \ref{fig:editor}).


\subsubsection{Activity Interface}
When an activity was launched, regardless of whether the parent had edited it, the \textit{book content} was presented on the right panel, with the \textit{LLM-generated content} and the \textit{modes-switching and role-delegation mechanisms} shown on the left. For each page, the components of the \textit{LLM-generated content} were organized into colored blocks, showing the \textit{book text}, \textit{questions and choices}, \textit{explanation}, and \textit{motivational feedback} in sequence. The \textit{mode-switching} mechanism enabled parents to control their involvement on a macro level by adjusting the overall task distribution for all components based on preset configurations. This mechanism used a driving metaphor, with positions for a driver, co-pilot, and exit. The parent was represented by a blue icon, and the robot by a green icon. By dragging the icons to the appropriate positions, parents could select their preferred mode for each task. The \textit{role-delegation} mechanism provided more detailed control, allowing parents to assign individual components of the activity to either themselves or the robot (Figure \ref{fig:activity}).

The system defined four modes: (1) \textit{parent-takeover} mode: the parent facilitates all content components; (2) \textit{parent-led} mode: the parent leads the activity, with the robot helping with specific tasks; (3) \textit{robot-led} mode: the robot leads, with the parent helping as needed; (4) \textit{robot-takeover} mode: the robot facilitates all components. The \textit{mode-switching} mechanism automatically adjusted the task roles based on the selected mode, assigning all roles to the parent in \textit{parent-takeover} and \textit{parent-led} modes, and all to the robot in \textit{robot-takeover} and \textit{robot-led} modes. In \textit{parent-led} and \textit{robot-led} modes, parents could further adjust the role distribution by delegating individual components using the \textit{role delegation} mechanism. Figure~\ref{fig:mode} details how icon positions correspond to specific modes and provides examples of a role delegation pattern for each mode.

\begin{figure*}[t!]
  \includegraphics[width=\textwidth]{figures/figure-mode-hho.pdf}
   \vspace{-6pt}
  \caption{Mode-Switching and Role Delegation Mechanisms of the PAiREd system. The system offers four modes: robot takeover, robot-led, parent-led, and parent takeover. Parents can drag their icon to the ``driver,'' ``co-driver,'' or ``exit'' position to select the desired mode. Additionally, they can fine-tune the role delegation by assigning specific tasks to either themselves or the robot.}
  \label{fig:mode}
   \vspace{-6pt}
\end{figure*}


   \vspace*{-3pt}
\subsection{Robot Interaction Design}
We used a Misty II social robot,\footnote{Misty Robot: \url{https://www.mistyrobotics.com/products/misty-ii/}} which is a semi-humanoid robot with a four-inch LCD display for its face where customizable facial expressions can be displayed. Misty robot was connected to our interface system through the Internet and the RestAPI.\footnote{\url{https://github.com/MistyCommunity/REST-API}} During the activities, the robot autonomously facilitated a component according to the role delegation, remaining silent when the parent was responsible for a component. The interface also displayed the robot's status and reminded parents when it was their turn to engage. For each component assigned to the robot, it verbalized the content using audio generated by Google's Cloud Text-to-Speech engine.\footnote{Google Cloud Text-to-Speech: \url{https://cloud.google.com/text-to-speech/}} Additionally, the system leveraged LLM to select an appropriate robot expressions per content component from a predefined database to enhance interaction through expressive gestures, which had been used in previous studies with children \cite{white2021designing}. The robot's front bumpers provided interactive controls, allowing young children to navigate the activity more easily. Pressing the left-front bumper \textit{repeated} the ongoing component, while pressing the right-front bumper \textit{proceeded} to the next component. When progressing to the next component, the interface updated accordingly by closing the current component and expanding the next one (Figure \ref{fig:robot}).






\begin{table}[t!]
    \caption{Participant Demographics}
    \label{tab:demographics}
    \centering
\fontsize{6.3}{8.3}\selectfont
    \begin{tabular}{p{0.02\linewidth}p{0.06\linewidth}p{0.09\linewidth}p{0.09\linewidth}p{0.09\linewidth}p{0.13\linewidth}p{0.23\linewidth}}%
        \toprule
         \textbf{ID} & \textbf{Child Age} & \textbf{Child Gender} & \textbf{Parent Gender} & \textbf{Ethnicity} & \textbf{Maternal Education} & \textbf{Household Income} \\
        \midrule
         1 & 3Y8M & Male & Female & Biracial & Master's & \$200,000 and more \\
         \midrule
         2 & 3Y4M & Female & Male & Biracial & Master's & \$150,000-\$199,999\\
         \midrule
         3 & 3Y2M & Female & Female & White & Master's & \$100,000-\$149,999\\
         \midrule
         4 & 4Y6M & Male & Female & White & Ph.D. & \$100,000-\$149,999\\
         \midrule
         5 & 4Y1M & Female & Female & White & Master's & \$100,000-\$149,999\\
         \midrule
         6 & 3Y3M & Male & Female & Black & Master's & \$35,000 -\$49,999\\
         \midrule
         7 & 3Y5M & Female & Female & White & Ph.D. & \$100,000-\$149,999\\
         \midrule
         8 & 4Y2M & Male & Female & White & Bachelor's & \$100,000-\$149,999\\
         \midrule
         9 & 3Y2M & Female & Female & Biracial & Ph.D. & \$50,000-\$74,999\\
         \midrule
         10 & 5Y0M & Male & Male & White & Bachelor's & \$100,000-\$149,999\\
         \midrule
         11 & 5Y1M & Male & Female & White & Ph.D. & \$200,000 and more\\
         \midrule
         12 & 4Y10M & Female & Female & White & Master's & \$100,000-\$149,999\\
         \midrule
         13 & 3Y9M & Male & Male & White & Master's & \$100,000-\$149,999\\
         \midrule
         14 & 4Y0M & Female & Female & Asian & Master's & \$50,000-\$74,999\\
         \midrule
         15 & 4Y6M & Female & Female & Biracial & Ph.D. & \$50,000-\$74,999\\
         \midrule
         16 & 4Y10M & Female & Female & White & Master's & \$150,000-\$199,999\\
         \midrule
         17 & 4Y7M & Female & Female & White & Master's & \$100,000-\$149,999\\
         \midrule
         18 & 4Y2M & Female & Female & White & Ph.D. & \$200,000 and more\\
         \midrule
         19 & 4Y9M & Female & Female & Hispanic & Bachelor's & \$200,000 and more\\
         \midrule
         20 & 3Y2M & Female & Female & White & Ph.D. & \$200,000 and more\\
         
         \bottomrule
    \end{tabular}
\end{table}

\begin{figure*}[b!]
  \includegraphics[width=\textwidth]{figures/figure-robot-hho.pdf}
   \vspace{-6pt}
  \caption{Robot Interaction Module of the PAiREd System and Study Setup. The robot facilitates the activity based on its assigned role, providing behavioral expressions to engage the child. The child can interact with the robot by pressing its bumper to navigate through the activity.}
  \label{fig:robot}
   \vspace{-6pt}
\end{figure*}

\section{User Study} \label{sec-study}

\subsection{Participants} 
Following protocols fully approved by the responsible Institutional Review Board (IRB), we recruited families from U.S. Midwest cities via email distributed to university employee mailing lists. We selected participants based on the following criteria: (1) one parent and one child participated together; (2) the child was aged 3--5 years; and (3) both parent and child could communicate in English. This age range was chosen as it represents a critical stage for parental involvement in early education \cite{purpura2013informal}, prior to formal schooling. Families received \$50 USD upon completing the study.

Our analysis includes data from 20 parent-child dyads with children aged 3–5 years (13 female, 7 male; $M = 4.1$, $SD = 0.67$ years). Each session involved one parent and one child, although other family members were sometimes present but did not participate. Participant demographics are summarized in Table \ref{tab:demographics}.
Most participating parents were mothers (17 mothers, 3 fathers), reflecting the greater likelihood of mothers serving as primary caregivers, consistent with previous research \aptLtoX[graphic=no,type=html]{\cite[{e.g.,}][]{schoppe2013comparisons}}{\cite[\textit{e.g.,}][]{schoppe2013comparisons}}. Despite efforts to recruit a diverse sample, the sociodemographic representation is limited: 85\% of mothers held at least a Master's degree, and 80\% of families reported an annual income of \$100,000 or more. This limits the generalizability of our findings and may omit insights into underrepresented groups, a limitation further discussed in Section~\ref{sec-7.4}.



\begin{figure*}[b!]
  \includegraphics[width=\textwidth]{figures/figure-procedure-hho.pdf}
   \vspace{-6pt}
  \caption{Three-phase study procedure. Phase 1 (40 min) and Phase 2 (40 min) only involved the parent, and Phase 3 (60 min) involved both the parent and the child. }
  \label{fig:procedure}
   \vspace{-6pt}
\end{figure*}

\subsection{Study Design} 
We conducted in-home visits (2.5 hours per visit) with families. We used \texttt{SET} (Section~\ref{sec-card} and Figure~\ref{fig:card-kit}) to foster discussion with parents about their real-life scenarios in which they are able or unable to facilitate learning activities for their child. In addition, we used \texttt{PAiREd}, an AI-assisted robot as a \textit{technology probe} \cite{hutchinson2003technology} to facilitate discussions on how parents may, under different \texttt{SET} scenarios, prefer to supervise the AI-generated content for their child and adjust their participation in learning activities.

\subsubsection{Study Materials and Setup}
The study materials included (1) a Misty II social robot; (2) a Microsoft Surface laptop to present user interface; (3) the \texttt{SET} card-based activity kit; (4) recording devices (\textit{i.e.,} a video camera, a webcam, and an audio recorder) positioned to capture participants' behavior and conversations (Figure~\ref{fig:robot}). 

\subsubsection{Study Procedure} \label{sec-procedure} 
Before the study, the experimenter provided families with an overview of the procedure and obtained informed consent: the parent signed a consent form, and the child gave verbal assent. The study was conducted in three phases: Phase 1 and Phase 2 (40 minutes each) involved \textit{only the parent} and Phase 3 (60 minutes) included \textit{both the parent and the child}. To minimize interruptions during the first two phases, the research team offered childcare assistance upon request (Figure \ref{fig:procedure}).

% phase 1
In Phase 1, the experimenter began by asking the parent to describe their involvement in their child's learning activities at home (\textit{e.g.,} ``\textit{What learning activities does your child typically do at home? Which ones involve you?}''). Next, the experimenter introduced the \texttt{SET} activity kit (Figure~\ref{fig:card-kit}, Section~\ref{sec-card}) and explained the factors and their dimensions with concrete examples (Table~\ref{tab:kit-example}), arranging the \texttt{SET-banners} into a hierarchical diagram for visualization (Figure~\ref{fig:card-kit}). The parent was then asked to write 2–3 real-life examples for each dimension of every factor on sticky notes and place them on the \texttt{SET-experience banks} (Figure~\ref{fig:card-kit}). Lastly, the experimenter introduced the \texttt{SET scenario cards}, explaining how the hierarchical diagram leads to eight scenarios by combining three two-level factors (Figure~\ref{fig:card-kit}). The parent was asked to write one real-life example for each scenario card for later use in Phase 3.

In Phase 2, the parent was first asked about their thoughts on AI-generated content (\textit{e.g.,} ``What are your thoughts on AI-generated content?''). The experimenter then introduced the \textit{editor interface} (Figure~\ref{fig:editor}), explaining its features for content modification. The parent reviewed and edited LLM-generated content for two books, one on math and one on literacy, while using the \textit{think-aloud} method to vocalize their thought process. Afterward, the experimenter discussed the parent's experience and used the \texttt{SET-scenario cards} to explore how they might use the interface in different scenarios.

In Phase 3, the parent first familiarized themselves with the robot interaction, learning to use the \textit{mode-switching} and \textit{role-delegation} mechanisms. The experimenter interviewed them about their perceptions of these mechanisms and their application in situations specified in \texttt{SET-scenario cards}. Next, the parent and child read two books together, incorporating activities created by the parent in Phase 2. At the start of each book, parents were asked to situate themselves in one of the \texttt{SET-scenario cards}, adjusting the mode and delegating jobs according to what they think they might do in that scenario. The experimenter gave the parent a different scenario card in the middle of the reading, and the parent responded to the situation, \textit{i.e.,} change to a different mode, accordingly. In addition, they also completed surveys on their perceptions of their child's math and literacy proficiency before and after the reading. Finally, they discussed their experiences collaborating with the robot and provided feedback on the designed mechanisms.

\begin{table*}
    \caption{Example of scenarios created using the {\texttt{SET-scenario cards}} by P2. P2 selected {\textit{``Teach child to build block towers''}} as the high-skill activity and {\textit{``Process feelings''}} as the low-skill activity for all examples. Each row represents one scenario card.}
\label{tab:scneario-example}
        \centering
        \renewcommand{\arraystretch}{1.2}
    \small
    \begin{tabular}{p{0.12\linewidth}p{0.12\linewidth}p{0.12\linewidth}p{0.12\linewidth}p{0.4\linewidth}}
        \toprule
        \hline
        \textbf{Scenario Card} & \textbf{Skill} & \textbf{Energy} & \textbf{Time} & \textbf{Example}\\
        \midrule
         1 & High & High & High & \textit{I am well rested and my partner is taking care of dinner.} \\
\hline
         2 & High & High & Low & \textit{Kids are being well-behaved but I have to mow the lawn.} \\
\hline
         3 & High & Low & High & \textit{I need to rest while my partner is making dinner.} \\
\hline
         4 & High & Low & Low & \textit{Kids are whining and I have to be at work.} \\
\hline
         5 & Low & High & High & \textit{Free from work and we don't have any plans.} \\
\hline
         6 & Low & High & Low & \textit{Well rested but have to do chores.} \\
\hline
         7 & Low & Low & High & \textit{Not feeling well but we have free time.} \\
\hline
         8 & Low & Low & Low & \textit{Not feeling well and have chores to do.} \\
\hline
         \bottomrule
    \end{tabular}
\end{table*}

\subsection{Data Analysis}
To anonymize participants, we assigned IDs based on the order of study completion (\textit{e.g.,} P1 refers to the first participant). For qualitative data, we conducted a reflexive \textit{Thematic Analysis (TA)} following \citet{clarke2014thematic} and \citet{mcdonald2019reliability} to identify patterns and understand parental perspectives. The first three authors, experienced in qualitative coding, transcribed and familiarized themselves with the data from audio and video recordings. Initial semantic codes were generated by the first author and organized into categories as the initial codebook. The team collaboratively coded data from P1 using the initial codebook, discussed interpretations, and created a consensus-based codebook. Using the codebook, the remaining data were independently coded by the three authors, with peer reviews and iterative discussions to finalize codes. We constructed final themes from recurring, meaningful patterns in the data. For the quantitative survey data (\textit{i.e.,} parent perception on child's literacy and math abilities), we first conducted a One-Way ANOVA to evaluate overall significance across difficulty levels for each subject (\textit{i.e.,} math and literacy), revealing significant differences between levels. We then performed Repeated Measures ANOVAs for each level, followed by post-hoc pairwise comparisons using non-parametric (Wilcoxon Signed-Rank) tests.
\section{Results}
We identified key contexts of parental involvement, perceptions of AI-generated content, preferences for AI-assisted content creation, and collaborative patterns in shared interactions with a robot. We present the findings based on our research questions as follows.

%We identified four key areas that inform the design of an AI-assisted educational robot to support parental involvement in young children's learning activities. We present our results based on the study phases as follows: (1) \textit{Phase 1: Parent Contextual Needs and Scenarios}: understanding the real-life contexts and challenges parents face; (2) \textit{Phase 2.1: Parent Perspectives on AI-Generated Learning Content}: capturing parents' attitudes and concerns regarding the use of AI in generating educational content; (3) \textit{Phase 2.2: Parent Use of the LLM-Assisted Content Supervision Mechanism}: examining how parents review, edit, and supervise AI-generated content; and (4) \textit{Parent Use of the Robot Involvement Adjustment Mechanism}: exploring how parents delegate roles between themselves and the robot during learning activities.

\subsection{(RQ1) What contexts do parents encounter when involving in young children's learning activities?}

Each participant provided examples for the eight scenarios representing unique combinations of the three two-dimensional factors using the \texttt{SET-scenario cards}. We present P2's scenarios as an example in Table~\ref{tab:scneario-example}. The full set of scenario examples from all participants is documented as supplementary materials.\footnote{SET Scenarios: \url{https://osf.io/zfksg/?view_only=b59bd41287f543ce82ab85950aaf004f}} Beyond capturing the contexts shared by each parent, we analyzed these examples to identify and summarize key contextual patterns for the three factors of parental involvement: \textit{skills}, \textit{energy}, and \textit{time}.

\subsubsection{\textbf{Skill:} Parents face challenges in pedagogical skills, particularly with advanced or unfamiliar concepts.}
Parents mentioned several \textit{skills} in supporting their child's (1) intellectual, (2) pedagogical, and (3) social-emotional development, highlighting key challenges across these areas. While some parents (7/20) reported low confidence in \textit{intellectual} activities, especially advanced STEM topics (P1, P7, P12, P14, P17–19), many (16/20) felt confident in literacy (\textit{e.g.,} reading, spelling; P1–6, P8, P9, P11, P13, P14, P19, P20) and basic STEM (P6, P8–10, P12–15). Confidence often stemmed from personal expertise or interests, consistent with the Hoover-Dempsey and Sandler (HDS) framework \cite{green2007parents}. For example, P15, a physicist, felt confident teaching physics-related activities. The majority of parents (14/20) struggled with \textit{pedagogical} skills, such as explaining concepts (P7, P8, P13, P17, P18), answering or formulating questions (P3, P4, P6, P7, P9, P13), identifying developmental benchmarks (P4, P6, P10, P11), and allowing their child to learn from mistakes (P2, P12, P19). A smaller group of parents (7/20) expressed confidence in these areas, particularly explaining concepts (P14, P18, P19) and answering questions (P4, P5, P10, P14). \textit{Social-emotional} skills presented additional challenges. Some parents (6/20) struggled with teaching emotion regulation (P2, P17), behavioral management (P5, P15, P20), and interpersonal conflict resolution (P3, P15). Others (5/20) lacked confidence in encouraging participation in learning activities (P5, P11) or maintaining patience during learning support (P4, P10, P18, P19). Conversely, several parents (10/20) felt confident teaching emotion regulation (P1, P3, P10–13, P16, P18–20) and norms of polite communication (P7, P12, P16, P20).

\subsubsection{\textbf{Energy:} Parents' motivation depends on their physical and emotional status as well as the child's willingness to learn.}
%\paragraph{Parents' motivation depends on their physical and emotional status, time, and the child's willingness to learn.}
Parents suggested that their motivation to facilitate learning activities was affected by (1) physical status, (2) emotional status, and (3) time. Commenting on their \textit{physical status}, most parents (16/20) indicated low motivation when they need rest due to feeling ``\textit{hungry},'' ``\textit{sick},'' or ``\textit{tired}'' (P1--11, P14--17, P19), and many parents (9/20) reported being highly motivated when they are ``\textit{well rested}'' or after having ``\textit{a really good meal}'' (P1--4, P6, P7, P9, P11, P16). Regarding \textit{emotional state}, many parents (9/20) lacked motivation when they needed a mental break or ``\textit{me time}'' if they felt emotionally exhausted (P3, P7, P11--14, P17--19) or after spending time with their child (P5, P8, P20). In addition, some parents (7/20) lost motivation if their child appeared to be disinterested (P11, P12, P14) or poorly behaved (P4, P15, P17, P20). In contrast, many parents (16/20) were motivated when their child needed support (P12, P15), expressed interest and invited the parent to participate (P1, P4, P7, P8, P10, P12–14, P16–19), or is well behaved and ready to learn (P2, P5, P13, P18, P19, P20). Some parents (6/20) were highly motivated when they wanted to connect with their child (P3, P14, P15) or when they were personally interested in the activity (P5, P10, P12, P14).


\subsubsection{\textbf{Time:} Parents' availability depended on work, chores, other family members.}
Parents discussed (1) work and commitment, (2) household chores, and (3) family needs as factors that determined whether they had time, \textit{i.e.,} availability and presence, to facilitate learning activities. Most parents (19/20) were not available when they needed to be at \textit{work} (P1--4, P6, P7, P10--14, P16--19) and had other personal or professional engagements (P4, P5, P9, P11, P15, p20). Many parents (17/20) stated that \textit{household chores}, such as laundry, meal preparation, and cleaning, also determined their availability to be with their child (P1--6, P9--17, P20). Although some parents involved their child in chores (P1, P2, P7, P10, P15, P17), not all chores were seen as being appropriate or safe for children. Parents' availability also depended on the ability of other family members to provide support (P2--4, P6--8, P16--20), \textit{e.g.,} when a spouse helped with chores or an older child watches a younger sibling. Parents had less time if other family members needed them (P3, P7, P8, P12, P13, P16--18, P20), \textit{e.g.,} when a younger child is crying or a family member is sick. Finally, parents described their availability using specific time frames, \textit{e.g.,} ``\textit{weekday mornings} (P5, P8, P19),'' ``\textit{weekdays after dinner and before bedtime} (P5, P10, P18),'' ``\textit{anytime on weekends} (P1, P3, P4, P6, P8, P10, P16),'' or ``\textit{unstructured time} (P2, P11, P14, P16, P17, P20).'' They often structured their time and consider themselves available when they are physically present with their child (P1, P3--7, P14--16), such as during grocery shopping, car rides, or trips to the park together.





\subsection{(RQ2) How do parents perceive AI-generated content for young children?}\label{sec-result-2}

Parents showed mixed attitudes toward AI-generated learning content for young children. They discussed their perceived benefits and risks and envisioned ways to mitigate their concerns.

\subsubsection{Mixed Attitude towards AI-generated content}
Parents expressed a range of attitudes towards allowing AI to generate content for young children, ranging from skepticism and concern (P2--6, P10, P13--15) to open-minded caution (P9, P8, P16, P19, P20), acceptance (P6, P7, P11, P17, P18) and, in some cases, neutral (P1, P12). Parents who were \textit{\textbf{skeptical and concerned}} questioned whether AI-generated content met quality and safety standards, \textit{e.g.,} P3 questioned, ``\textit{Who's generating the content? Where is it getting the content from? Is it good? Is it safe?}'' On the other hand, parents who hold an attitude of \textit{\textbf{open-minded caution}} recognize the risks of using AI-generated content but feel open to use it under specific conditions. P16 highlighted model training, stating, ``\textit{I wouldn't be against it if the people training it were proficient in what the AI is teaching.}'' Similarly, P20 emphasized personal oversight, explaining, ``\textit{I can do my own evaluation to determine whether or not I think the content is good regardless of who it came from.}'' Furthermore, parents who have an attitude of \textit{\textbf{acceptance}} assume people who created the system have already ensure the appropriateness for children, \textit{e.g.,} P7 stated, ``\textit{I'm assuming because it's AI, there would be more research behind it.  So I would be okay with it.}'' Finally, parents who hold a \textit{\textbf{neutral}} attitude typically don't have much experience with AI and therefore feel unsure about their attitude for AI-generated content, \textit{e.g.,} P1 had ``\textit{not even thought about it until before this study.}''

\subsubsection{Perceived Benefits and Risks}
Parents identified several benefits of AI-generated content for young children. Some parents (P2, P4, P8, P10, P11, P16) highlighted AI's potential in \textit{\textbf{adaptability}} to adjust learning content to their child's evolving developmental needs, \textit{e.g.,} P11 expected AI to help ``\textit{adjust content as the child grows.}'' In addition, parents (P2, P3, P4, P16, P19) discussed \textit{\textbf{customization}}, illustrating that ``\textit{one of the big benefits would be to create material that are related to his[child's] interests and things that would be motivating to him[child]} (P4).'' Parents (P6, P7, P8, P10, P12, P17) also emphasized \textit{\textbf{efficiency}} of AI, explaining ``\textit{because it[AI] can access a huge amount of information very fast} (P12),'' enabling a ``\textit{quicker way to learn or to see something} (P7).'' Moreoever, a few parents (P1, P11, P18, P20) noted AI's potential to foster \textit{\textbf{affordability}}, suggesting that AI-generated content could enhance the scalability and accessibility of learning resources, making ``\textit{more learning materials available, more variety available} (P1),'' and making things ``\textit{cheaper and more accessible for people} (P20).'' Finally, a few parents (P14, P15) expected easier \textit{\textbf{pedagogical integration}} with AI, enabling parents to ``\textit{teach children things that sometimes parents don't know because not all parents know everything} (P14).''

Meanwhile, parents described their perceived risks of AI-generated content for children. Most parents (P1--3, P5, P11--15, P17, P19) were concerned about \textit{\textbf{age-inappropriateness}} of the content, which could be ``\textit{violent and don’t match family values} (P1),'' ``\textit{physically harmful and sexually inappropriate} (P3),'' and ``\textit{stuff about body image and certain people being better than other people} (P5).'' In addition many parents (P2, P3, P9, P14, P16, P17, P18) expressed concerns about the \textit{\textbf{inaccuracy}} of the information presented through AI-generated content, worrying that AI could provide ``\textit{factually inaccurate}'' learning materials or content that might imply theories that are ``\textit{misframed or misconstructed} (P2).'' Moreover, parents (P2--4, P14, P15) raised concerns about the \textit{\textbf{training data quality}} for AI models. P2 emphasized transparency stating, ``\textit{I'd want to know a lot more about where that training data came from or who supervised that learning process}.'' A few parents (P6, P7) expressed concerns about children's \textit{\textbf{over dependence}} on AI instead of developing their own cognitive abilities. P6 worried that constant use of AI could discourage critical thinking, stating, ``\textit{if they have a question, instead of thinking through the question, they just ask AI, not using their own brain}.'' Finally, two parents shared concerns over \textit{\textbf{message dilution}}, where AI oversimplifies complex ideas and diminishes their original intent. P15 worried that AI might dilute sociopolitical issues, such as racial diversity and gender identity. Similarly, P20 emphasized concern about whether the core message being conveyed to the child aligns with parental values, stating ``\textit{I'm more concerned about the message the book is trying to impart on the child}.

\subsubsection{Envisioned Risk Mitigation Methods}
Parents described what methods they envisioned to address their concerns. First, some parents (P3, P5, P11, P12, P14, P16) stressed the need to enable \textit{\textbf{parental review and verification}}. For example, P5 stated ``\textit{I would read it to make sure that it was actually something I wanted to read with her}.'' In addition, a few parents (P2, P17, P19) expressed that \textit{\textbf{social and public validation}} could also enhance their trust in AI-generated content, \textit{e.g.,} P2 described that ``\textit{if a thousand people used it...and endorsed this model, that would give me more confidence in it}.'' Moreover, some parents (P2, P9, P15) discussed \textit{\textbf{model and data transparency}}, emphasizing the need to understand how AI models are trained. As explained by P9, ``\textit{being able to know exactly what's going on or how it works...would make me feel more secure about what my child is learning}.'' Lastly, a few parents (P1, P15, P18) highlighted the importance of \textbf{expert involvement} in creating AI-generated content. For instance, P1 emphasized the need for oversight by ``\textit{people with a background in human development.}''

\subsection{(RQ3) How would parents prefer to collaborate with LLM on supervising content creation under different contexts?}

\begin{figure*}[b]
\includegraphics[width=\textwidth]{figures/figure-result-03-hho.pdf}
   \vspace{-6pt}
  \caption{Summary of parent's use of LLM-assisted content supervision mechanism: (1) content evaluation criteria, (2) use pattern, (3) perceived value.}
  \label{fig:result-03}
   \vspace{-6pt}
\end{figure*}

We found three main themes for parent-AI collaboration on content creation using the \textit{editor interface}: (1) \textit{Content evaluation and criteria}, referring to what parents pay attention to when reviewing and revising LLM-generated content. (2) \textit{Contextual usage patterns}, describing how parents envision using the LLM-powered interface in various contexts. (3) \textit{Perceived value and benefits}, covering what values parents believe LLM brings.

\subsubsection{Theme 1: Content evaluation and criteria}
Parents focus on balancing \textit{difficulty} and \textit{variety} of concepts as well as ensuring the \textit{quality} of questions when reviewing, regenerating, and revising LLM-generated content for young children.

\textit{\textbf{Parents aim to give their children the right level of challenge while reinforcing skills they can confidently accomplish.}} Many (8/20) avoided overly easy questions to prevent boredom but strategically included them at the start or after difficult questions to build confidence. As P2 explained, ``\textit{I want her to get the answers and then have it get increasingly difficult as she goes so she doesn't get discouraged at the beginning}.'' Meanwhile, most parents (11/20) valued challenges that stretch their child's abilities without overwhelming them. P12 concerned that ``\textit{underestimating her would be damaging for her},'' while P20 expressed interest in seeing how his child would handle harder concepts, saying, ``\textit{I'm actually really interested to see if she can answer.}'' Finally, parents (7/20) were also cautious of content that might be too advanced, \textit{e.g.,}``\textit{she[child] doesn't know uppercase or lowercase yet, so that doesn't mean anything to her} (P20).'' Additionally, \textit{\textbf{parents aim to maintain engagement by introducing diverse concepts and question types throughout the activity.}} Many parents (11/20) expressed concerns over repetitive content and preferred diverse topics to challenge their child differently. For instance, P20 changed the concept of a question to ``addition'', explaining, ``\textit{I just made the last question a `how many,' so this one I want a different concept}.'' Finally, \textit{\textbf{parents evaluate the quality of LLM-generated learning content based on standards} such as question clarity and coherence (9/20), wording precision (6/20), visual clarity (5/20), and cognitive load (P12, P20).} P12 raised issues with wording, stating, ``\textit{I don't think she's going to fully understand front legs versus back legs when it's a front view},'' while P6 expressed concerns about visual clarity: ``\textit{from the pictures, you can't really tell how many bugs with black bodies are flying in the air}.'' P20 also reflected on cognitive load, saying, ``\textit{I think it's just too long, too much information for her to process}.'' 

\subsubsection{Theme 2: Contextual usage patterns}
We discussed parents' preferences and behaviors when collaborating with LLM under two main contexts: (1) when parents have limited time or energy and (2) when they have sufficient time and energy.

\textit{When parents have limited time or energy}, most were still \textit{willing} to invest minimal effort (P4, P6--9, P11, P12, P15, P18), often opting to \textit{\textbf{skim through the LLM-generated content with minor self-editing}}. For example, P9 shared, ``\textit{I might skip quickly, skim through it, make sure there isn't anything that I feel is not appropriate}.'' This approach allows involvement with minimal time commitment. However, some parents (P4, P11, P15) emphasized that the \textit{\textbf{LLM output must be high-quality enough to require minimal editing}}, otherwise they may not use it at all. P11 explained, ``\textit{The more that stuff can be in really good shape before it gets to parents, the more we can minimize how much work we have to do ahead of time}.''

Some parents were \textit{unwilling} to invest effort when time or energy was limited. They preferred to either \textit{\textbf{reuse previously reviewed activities}} (P8, P11, P12) or directly \textit{\textbf{use LLM-generated content without review}} (P5, P6, P9, P10, P18, P20). As P8 explained, ``\textit{if I don't have time, I would have to be using something he's already done before, so I don't have to supervise it},'' while P10 noted, ``\textit{if the AI-generated questions were enough to keep him engaged, then it would be worth it}.'' A few parents (P1, P7) preferred to \textit{\textbf{avoid using the system entirely}}, as they feel uncomfortable leaving their child engaged with the content without supervision. As P1 explained, ``\textit{if I'm either physically or mentally not present. It's just not happening}.''

\textit{When parents have sufficient time and energy}, most of them (P6–P10, P12, P18, P20) choose to \textit{\textbf{review and edit the content in detail, even customizing questions}} to better supervise and personalize learning for their child. P9 shared, ``\textit{If I had more time and motivation, I would take the time to do it myself. I enjoy writing, so I'd probably spend time customizing the content}.'' Similarly, P12 noted, ``\textit{If I had all the time, I would go through and be picky with the wording and content of the questions}.''

In contrast, some parents (P1, P2, P7, P11, P12, P18) still prefer to \textit{\textbf{skim through the content with minor editing}}, as they found the detailed process too effortful even when time allowed, but they cannot fully trust LLM or themselves to come up with good questions. For example, P11 shared, ``\textit{I would probably scroll through and try to do as little editing as possible},'' while P7 expressed doubt, stating, ``\textit{I don't know that I would come up with better questions than this one from AI}.'' A few parents opted to \textit{\textbf{avoid using the system entirely}}, preferring to spend their time on other activities (P15) or relying on their ability to engage their child without the system (P4, P5). For instance, P15 shared, ``\textit{I would rather spend that time playing an imaginative game with her than spending time designing this,}'' P5 similarly expressed confidence, saying, ``\textit{I think I can and do ask her questions about stuff we read}.''

\subsubsection{Theme 3: Perceived value and benefits}
We found that parents perceive the value of the system to include not only \textit{content supervision}, but also \textit{content co-creation with LLM} and \textit{parent empowerment through pedagogical insights}.

First, and unsurprisingly, most parents suggested that the system allows them to \textit{\textbf{supervise the learning content generated by LLM}}. For example, P2, while feeling skeptical about trusting AI, noted, ``\textit{I don't know what AI model was used, still, I can confirm everything myself},'' reflecting the value parents place on maintaining oversight of the content presented to their children. Second, some parents (P6–10, P12, P18, P20) appreciated that the system allows them to \textit{\textbf{co-create personalized learning content with the LLM}} for their child without having to start from scratch. For example, P18 appreciated the ability to adapt the content to their child's needs, saying, ``\textit{tailoring it to her difficulty levels and seeing the ability to modify the content alleviates some concerns}.'' P10 highlighted how the LLM creates a draft to work from, stating, ``\textit{I do appreciate the concepts and the kinds of questions that it [LLM] provides, and how it has that template there}.'' This flexibility allowed parents to easily modify content while leveraging the assistance from LLM. Third, some parents found that the system \textit{\textbf{empowered parents with pedagogical insights}}. As many parents do not possess formal pedagogical knowledge--such as understanding how to effectively teach their child--they often struggle with determining what questions to ask or which concepts are age-appropriate. Since parents brought up the same value after interacting with the robot as well, we discuss this value more in-depth in Section \ref{sec-6.4.2}.

\subsection{(RQ4) How would parents prefer to collaborate with an AI-assisted robot to engage in learning activities with their children under different contexts?}

\begin{figure*}[b!]
\includegraphics[width=\textwidth]{figures/figure-result-04-hho.pdf}
   \vspace{-6pt}
  \caption{Summary of parent's use of robot involvement adjustment mechanisms: (1) usage pattern, (2) parenting education.}
  \label{fig:result-03}
   \vspace{-6pt}
\end{figure*}

We identified two major themes in the use of parent-robot collaboration mechanisms (\textit{i.e.,} \textit{mode-switching} and \textit{role-delegation}) within the \textit{activity interface}: (1) \textit{Contextual mode utilization}, referring to how parents adjust their involvement based on varying time and energy levels, and (2) \textit{Perceived educational impact on parenting}, highlighting how parents value the process for enhancing their skills and knowledge in parenting.

\subsubsection{Theme 1: Contextual mode utilization}

We discussed parents' preferences when collaborating with the AI-assisted robot across four contexts: (1) sufficient energy and time, (2) sufficient time but low energy, (3) sufficient energy but limited time, and (4) low energy and time. The impact of parental skill is discussed in specific cases.

(1) \textit{Parents have sufficient energy and time}: many parents (P2, P5–7, P9, P15, P18) preferred the \textit{\textbf{parent takeover mode}}, where they facilitate activities themselves while using LLM-generated content as a resource. For example, P18 shared, ``\textit{if I'm feeling motivated, I'd probably take over, but still look at some AI-generated questions to prompt me or remind me of things to ask or do with her}.'' Similarly, P15 noted, ``\textit{with full energy and time, I would use the parent-only mode because I want to interact with her and give her all my attention}.'' Parents valued the ability to take full control while using LLM-generated content for supplemental support when they have sufficient energy and time.

In addition, some parents (P2, P8–12, P15, P20) envisioned using \textit{\textbf{collaboration modes}}--where both the parent and the robot share responsibilities (\textit{i.e.,} parent-led or robot-led mode)--with the parents' \textit{skills} in specific areas relative to the robot playing a critical role in determining the pattern of role delegation. Parents often chose to involve the robot when they felt it could enhance their child's engagement especially in high-stakes tasks like quizzes. For example, P15 noted, ``\textit{I would use the robot for quizzes as a playful element to keep her engaged}.'' Similarly, P2 highlighted the objectivity of the robot in quizzing: ``\textit{I like the idea of reading her the book and then a neutral third party gets to test her on it}.'' On the other hand, parents took on specific roles when they believed their involvement would benefit their child more. P11 shared, ``\textit{I would let the robot read and ask questions but step in if he wasn't understanding or needed guidance},'' while P12 emphasized the emotional aspect of teaching: ``\textit{I can explain in a way that she understands, whereas the robot might come across as too harsh}.''

(2) \textit{Parents have sufficient time but lack energy}: some parents (P3, P7, P10--12, P15, P18, P20) opted for \textit{\textbf{collaboration modes}}, with their involvement influenced by their motivation levels and partially by their \textit{skill} relative to the robot. For example, P10 noted, ``\textit{when I'm not motivated, having the robot do the quiz takes some heat off me}.'' Similarly, P9 mentioned, ``\textit{I'd probably read the book, but have the robot do everything else}.'' Additionally, some parents (P1, P2, P4, P6–8, P15) chose to use \textit{\textbf{robot takeover mode}}--where the robot facilitates everything--while they remained nearby to supervise. For instance, P15 noted, ``\textit{I'd be around, but I wouldn't physically do much because I'm not feeling well}.'' Similarly, P2 noted, ``\textit{If I'm not motivated, I could see myself handing it all over to the robot}.''

(3) \textit{Parents have sufficient energy but lack time}: many parents (P2, P3, P5, P7--11, P15) opted for the \textit{\textbf{robot takeover mode}}--where the robot facilitates everything--while adjusting their usage based on \textit{how much they trust LLM}. Parents with higher trust allowed their child to use the content directly without review (P3, P5, P7, P9, P10). For example, P7 mentioned, ``\textit{If I'm trying to take a walk, I might do the robot takeover, then I can physically be gone}.'' In contrast, parents with less trust preferred to supervise while multitasking (P2, P7, P8), review content beforehand using the editor (P2, P8), or use the system only if the LLM model met high-quality standards (P11, P15). For example, P2 shared, ``\textit{If I'm not there, I wouldn't want them to do it, unless I had used the editor to review},'' while P7 described a multitasking scenario: ``\textit{I could be working from home while the robot takes over, and I'm nearby to supervise}.'' Moreover, a few parents (P2, P12) chose to \textit{\textbf{avoid using the system entirely}} due to their lack of trust in using LLM-generated content directly and insufficient time to review it, or because the system design did not support independent use for young children (P4, P12, P20). For instance, P2 noted, ``\textit{If I'm absent, I don't know if I'd want them to do any of this},'' while P12 stated, ``\textit{I know my daughter is sensitive, and if [the questions are too hard and] the robot keeps telling her she's wrong, she might take it personally and give up}.''

(4) \textit{Parents lack both energy and time}: Some parents chose \textit{\textbf{robot takeover mode}}, adjusting their usage based on their trust in LLM-generated content. Others \textit{\textbf{avoided using the system entirely}} due to low trust and insufficient time to review (P2, P12), or because the system design did not support independent use by young children (P4, P12, P20). Refer to the previous case—parents with sufficient energy but lacking time—as the usage patterns and contextual reasons are very similar.

\begin{figure*}[!t]
  \includegraphics[width=\textwidth]{figures/figure-quant-hho.pdf}
  \caption{Parent perception on child's math and literacy ability before and after the reading session. The result suggested that parents adjusted their perception after observing their child doing the activity and they tend to underestimate them, especially for advanced math concepts and phonological awareness concept for literacy. The horizontal lines represents significance from the Wilcoxon Signed-Ranked Test: $p < .01^{**}$, $p < .05^{*}$.}
  \label{fig:quant-result}
   \vspace*{-10pt}
\end{figure*}

\subsubsection{Theme 2: Perceived Educational Impact on Parenting} \label{sec-6.4.2}

Supported by mixed-method data, many parents thought \texttt{PAiREd} has value in parenting education, providing them with pedagogical strategies and giving them opportunities to observe their child's proficiency level systematically through observation. If the system provides a comprehensive framework and ample ideas, parents may not decrease their involvement in an activity just because they don't have the pedagogical skill; in fact, they may even increase their involvement. In addition, parents will be able to observe and adjust their understanding about what their child can do or cannot do, instead of under- or over- estimate their child's ability.

Several parents (P1, P4, P6, P7, P10--12, P18) appreciated that the system \textit{\textbf{offered ideas they might not have considered on their own}}, providing new topics to explore with their child. For instance, P1 emphasized, ``\textit{I hadn't even thought of all the different types of concepts},'' and P10 highlighted that the system ``\textit{gives more of a structure…even the drop down list of concepts is insightful, offering lenses I wouldn't normally consider when reading}.'' Others (P2, P6--9, P11, P12) valued that the LLM \textit{\textbf{generated example questions for each concept}}, allowing them to start with ready-made content without worrying whether their own questions reflected the intended learning goals. For example, P8 mentioned, ``\textit{What's nice about the AI-generated ones is that you can specifically choose a variety of concepts, whereas creating them on your own, you don't always know what the concepts are}.'' Additionally, some parents noted that the system allowed them to systematically select questions they were unsure their child could answer, which \textit{\textbf{provided a structured way to observe and assess their child's proficiency level}}. For example, P7 remarked, ``\textit{it gives her the opportunity to show me things she knows that I otherwise wouldn't have asked},'' while P8 shared, ``\textit{I was curious to see how he does if he doesn't know how to answer this, rather than just setting him up to succeed}.''

Before and after parent-child pairs engage in the activity, we asked parents to rate their perception on their child's math and literacy abilities. Our quantitative results suggest that parents adjusted their understanding of their child's proficiency in certain concepts after reading together. Specifically, \textit{\textbf{parents tended to underestimate what their child can or cannot do, especially with more advanced math concepts}} (Math-L3: $p < .01^{**}$, Math-L4: $p < .01^{**}$) and the phonological awareness concept in literacy ($p < .05^{*}$). Figure \ref{fig:quant-result} summarizes the significance results from the Wilcoxon Signed-Rank Test.

%A One-Way ANOVA revealed significant differences between levels (L1-L4), necessitating separate analyses. Subsequent Repeated Measures ANOVAs for each level showed significant improvement in Math L4 post-intervention. Post-hoc pairwise tests, using both parametric (Paired T-Tests) and non-parametric (Wilcoxon Signed-Rank) methods to ensure robustness, confirmed significance in Literacy L2, Math L3, and Math L4. This multi-layered approach—combining One-Way ANOVA, Repeated Measures ANOVA, and diverse post-hoc tests—ensured level independence, accounted for within-subject variability, and provided a comprehensive, robust understanding of level-specific improvements while minimising misinterpretation risks.






\section{Conclusion}

This paper introduces a variant of the multivariate time series traffic prediction problem with a focus on highly sparse and unstructured observations.
To address this problem we propose SUSTeR, a framework which handles sparse unstructured observations by creating hidden graphs in a residual fashion, which are then used with a conventional spatio-temporal GNN.
SUSTeR achieves better predictions for high sparsity (80\% - 99.9\% missing data) than existing baselines and remains competitive in denser settings or even when using only half the amount of the training data.
In addition, its training is considerably faster than the next-best competitor due to a smaller model size.

% We conduct experiments on a unstructured and sparse version of the traffic dataset Metr-LA and compare the performance of SUSTeR with traffic prediction baselines.
% The consideration of the sparsity within SUSTeR outperforms other approaches at sparsity rates $\geq$99\%.
% Experiments were performed up to a sparsity with only 2.4 observations within a sample where without missing data such a sample contains 12$\times$207 values.
% Further, the ablation studies explore the influence of our design choices and show the robustness of our framework.


\section{Future Work}

We plan to explore the interpretability within SUSTeR to obtain an intuitive understanding of the graph nodes within the hidden graph.
Small design choices are made within SUSTeR to make this possible, from observations that are not relying on each other in the same timestep, variable amounts of observations, a learnable assignment function from the observation to the hidden node, and an explicit learned laplacian matrix. 
The problem of sparse unstructured observations, which should be reconstructed into a hidden state, is present in many other domains.
In particular ocean data is a very promising application field for SUSTeR where sparse ARGO\footnote{https://argo.ucsd.edu} observations would perfectly match the problem definition to predict ocean states. 
There, observations are typically spatially and temporally sparse - comparable to the highest dropout rate in this paper - and observations are non-stationary and change their position freely.
We see SUSTeR as a bridge of the well-studied spatio-temporal mining methods into a new area of domains, in which such methods previously were not applicable.

\subsection{Design Implications} %\label{sec-5.2}
We proposed five design implications based on the discussion of the findings to inform future design for AI-assisted robot.

\subsubsection{Design Implication \#1: } \textit{Tailoring the Quality of LLM Models with Minimal Parental Feedback}.  
When parents have limited time or energy, they often provide minimal content review and may avoid using the system if they lack trust in the model or cannot supervise. This suggests that LLM-generated content must be trustworthy enough for parents to use with little oversight. Beyond employing advanced LLMs, designers should incorporate features that encourage parents to supervise the content considering their availability, \textit{e.g.,} summarization versus detailed view of all content. Additionally, integrating evaluations for factors like age appropriateness can increase parents' trust and reduce the likelihood of disuse.

\subsubsection{Design Implication \#2:} \textit{Explicit Design for Varying Levels of Control over LLM-Generated Content}.  
Parents engage with LLM-generated content differently depending on their circumstances—ranging from using it as is, reusing previously reviewed content, quickly skimming, or extensively customizing. This variation calls for a more explicit design of control options within the system. Instead of presenting all features (\textit{e.g.,} regenerate, edit) simultaneously, the interface should adapt to users' needs. Those skimming for a quick review could access a concise summary, while parents interested in customization might see detailed content and easy-to-use editing and regeneration tools.

\subsubsection{Design Implication \#3: } \textit{Mode-Switching as a Shift in User Center, Not Just Task Delegation}.
Currently, the mode-switching feature focuses on dividing tasks between the parent and the robot, but it should also reflect changing interaction dynamics between parent and child. For example, when switching from parent-led to parent-takeover mode, a parent may want to access LLM-generated content without the robot's presence. Yet, the current design presents content uniformly across modes. Similarly, shifting from robot-led to robot-takeover mode implies independent child use, but the system currently remains parent-centered. To address this, mode-switching should not only redistribute tasks but also adjust content presentation and interaction methods. In parent-takeover mode, content could be simplified for easier navigation, and in robot-takeover mode, interaction could shift from touch-based to voice-based, fostering a child-friendly environment without parental intervention.

\subsubsection{Design Implication \#4: } \textit{Maximizing Parent-Child Interaction Value through Appropriate Difficulty Levels}.
Mode-switching should also consider optimizing educational impact for different interaction dynamics by adjusting content difficulty. Parents generally excel at reading, explaining, and encouraging their children, while the robot's role is to supplement limited time and energy and offer alternative ways to engage children. When parents are involved, slightly more challenging content can increase the educational value of their engagement. Conversely, in child-robot interactions, the system should present content that children can handle independently, avoiding frustration or loss of confidence.

\subsubsection{Design Implication \#5: } \textit{Integrating Parenting Education into the System}.
Parents view the system not only as a tool for engaging their children, but also for improving their own parenting skills, \textit{e.g.,} effective questioning technique and pedagogical strategies. While familiar with preschool-level concepts, they may struggle to generate questions spontaneously or accurately gauge their child's abilities. To address this, the system should incorporate parenting education features. It could, for example, guide parents in developing more intuitive questioning techniques and provide tools to track and benchmark their child's progress. By doing so, the system empowers parents to focus their efforts and better support their children's learning.

\section{Limitations \& Future Work}\label{sec-7.4}

A notable limitation of our study is the lack of sociodemographic diversity in our participant sample. Specifically, (1) \textit{location:} all participants were based in the U.S. Midwest; (2) \textit{gender:} the 20 parents who participated in the study included only three fathers; (3) \textit{income:} the majority of families reported middle to high household incomes; and (4) \textit{education:} most parents had a Master's degree or higher. These factors may limit the broader applicability of our findings. Consistent with previous education studies, a predominance of mothers were included due to self-selection bias \cite{schoppe2013comparisons, mcbride1993comparison}, and we faced difficulties recruiting lower-income families \cite{nicholson2011recruitment}. Children from higher-income and more highly educated families tend to perform better academically \cite{sirin2005socioeconomic}. These may have influenced the types of support families wanted from the robot, which need to be addressed in future research. we aim to collaborate with local community centers, libraries and schools to reach a more diverse population in our future work.

In addition, while cost remains a common accessibility limitation, some educational robots, such as Miko,\footnote{Miko Robot: \url{https://shorturl.at/XSyM5}} are now priced similarly to smartphones, improving feasibility for families. Future work should design novel interaction paradigms for AI-assisted educational robots in public spaces (\textit{e.g.,} libraries, schools, museums) to broaden accessibility. Moreover, our brief home visits captured immediate reactions rather than long-term changes in parent-child interactions, trust in AI/robots, or learning outcomes. Extended research are needed to observe how these factors evolve as AI-assisted robots integrate into daily life.

Additionally, we acknowledge the limitation of simplifying parental involvement factors into binary levels, resulting in eight scenarios. This decision was practical, as more granular factors (\textit{e.g.,} three levels each; 27 scenarios) would yield in a large number of scenarios not manageable for human participants. Future work should explore methods for representing a continuous spectrum of factors to capture more accurate scenarios. Furthermore, our system currently targets literacy and math. Beyond these academic subjects, other domains like social-emotional skills and creativity warrant exploration. Investigating parent-AI-robot collaboration in these areas could yield broader insights. Finally, ethical concerns motivated our work (Section~\ref{sec-rw-2.2}), but were not our main focus. We presented some ethical issues in Section~\ref{sec-result-2} and discussed in Section~\ref{sec-dis-2}, but future research should delve more deeply into these topics. A comprehensive understanding of ethical considerations, from data sourcing to model training, will help ensure that AI-generated content meets appropriate standards for children.
\section{Conclusion}
Towards realizing a scalable map-less guide system that assists blind people in exploring, we developed WanderGuide, a robotic guide  system designed to provide real-time descriptions of surroundings and to offer conversation functionalities that allow users to specify their destinations or ask questions.
The formative study with ten blind participants revealed that there are three types of preferences over the levels of details of the descriptions generated by the system.
In a subsequent main study with five blind participants, all of them expressed appreciation for the experience of wandering freely without a fixed destination, as well as a desire to use the system for exploring both familiar and unfamiliar areas. 
The study further revealed that including audio recognition would be the immediate next step for developing our system. 
It also revealed that customizing to diverse user preferences is important and that MLLM is the key bottleneck of the technology development of our system.
We hope this research contributes to the potential deployment of robotic guide systems in general use cases, enabling blind users to explore independently.

%%
%% The acknowledgments section is defined using the "acks" environment
%% (and NOT an unnumbered section). This ensures the proper
%% identification of the section in the article metadata, and the
%% consistent spelling of the heading.
\begin{acks}
This work was supported by the National Science Foundation awards 2202803. We thank Nathan White for his guidance with the technical development of our robot system. We also thank Amy Koike for her help in creating Figures 4 and 5.
\end{acks}

%%
%% The next two lines define the bibliography style to be used, and
%% the bibliography file.
\balance
\bibliographystyle{ACM-Reference-Format}
\bibliography{Bibliography}

%%
%% If your work has an appendix, this is the place to put it.
%\appendix
%\section{Research Methods}
%\subsection{Part One}



\end{document}
\endinput
%%
%% End of file `sample-authordraft.tex'.
