
\documentclass[sigconf]{acmart}


%%% Revision packages
%\usepackage{color}
%\usepackage{soul}

\usepackage{acmart-taps}
\usepackage{stfloats}

%	\newif\ifCOMMENTS
%	\COMMENTStrue
	
%	\ifCOMMENTS
%	\newcommand{\revise}[2]{\textcolor{red}{\sout{#1}}\hl{#2}}
%	\else
%	\newcommand{\revise}[2]{#2}
%	\fi

%%% TODO packages
% \newcommand{\todo}[1]{{\color{orange}\textbf{To do:} #1}}


\usepackage{microtype}
\usepackage{balance}
\usepackage{booktabs}
\usepackage{graphicx}
\usepackage{multirow}
% \usepackage[figurename=Figure]{caption}
% \renewcommand{\figurename}{Figure}
% \usepackage{caption}
% \captionsetup[figure]{name={Figure},}


%% Fonts used in the template cannot be substituted; margin 
%% adjustments are not allowed.
%%
%% \BibTeX command to typeset BibTeX logo in the docs
\AtBeginDocument{%
  \providecommand\BibTeX{{%
    \normalfont B\kern-0.5em{\scshape i\kern-0.25em b}\kern-0.8em\TeX}}}

\makeatletter
\def\@ACM@copyright@check@cc{}
\makeatother

\copyrightyear{2025}
\acmYear{2025}
\setcopyright{cc}
\setcctype{by}
\acmConference[CHI '25]{CHI Conference on Human Factors in Computing Systems}{April 26-May 1, 2025}{Yokohama, Japan}
\acmBooktitle{CHI Conference on Human Factors in Computing Systems (CHI '25), April 26-May 1, 2025, Yokohama, Japan}\acmDOI{10.1145/3706598.3713330}
\acmISBN{979-8-4007-1394-1/2025/04}


\acmSubmissionID{7119}

\begin{document}


\title[\texttt{SET-PAiREd}: Designing for Parental Involvement with an AI-Assisted Educational Robot]{\texttt{SET-PAiREd}: Designing for Parental Involvement in Learning with an AI-Assisted Educational Robot}


%% authors
\author{Hui-Ru Ho}
\orcid{0009-0000-3701-2521}
\affiliation{%
  \institution{Department of Computer Sciences\\
  University of Wisconsin--Madison}
  %\streetaddress{}
  \city{Madison}
  \state{Wisconsin}
  \country{USA}
}
\email{hho24@cs.wisc.edu}

\author{Nitigya Kargeti}
\orcid{0000-0001-5970-7332}
\affiliation{%
  \institution{Department of Computer Sciences\\
  University of Wisconsin--Madison}
  %\streetaddress{}
  \city{Madison}
  \state{Wisconsin}
  \country{USA}
}
\email{kargeti@wisc.edu}

\author{Ziqi Liu}
\orcid{0009-0007-8755-5744}
\affiliation{%
  \institution{Department of Computer Sciences\\
  University of Wisconsin--Madison}
  %\streetaddress{}
  \city{Madison}
  \state{Wisconsin}
  \country{USA}
}
\email{ziqil@cs.wisc.edu}

\author{Bilge Mutlu}
\orcid{0000-0002-9456-1495}
\affiliation{%
  \institution{Department of Computer Sciences\\
  University of Wisconsin--Madison}
  %\streetaddress{}
  \city{Madison}
  \state{Wisconsin}
  \country{USA}
}
\email{bilge@cs.wisc.edu}


%%
%% The "author" command and its associated commands are used to define
%% the authors and their affiliations.
%% Of note is the shared affiliation of the first two authors, and the
%% "authornote" and "authornotemark" commands
%% used to denote shared contribution to the research.

\renewcommand{\shortauthors}{Ho et al.}

\begin{abstract} %<150words
AI-assisted learning companion robots are increasingly used in early education. Many parents express concerns about content appropriateness, while they also value how AI and robots could supplement their limited skill, time, and energy to support their children's learning. We designed a card-based kit, \texttt{SET}, to systematically capture scenarios that have different extents of parental involvement. We developed a prototype interface, \texttt{PAiREd}, with a learning companion robot to deliver LLM-generated educational content that can be reviewed and revised by parents. Parents can flexibly adjust their involvement in the activity by determining what they want the robot to help with. We conducted an in-home field study involving 20 families with children aged 3--5. Our work contributes to an empirical understanding of the level of support parents with different expectations may need from AI and robots and a prototype that demonstrates an innovative interaction paradigm for flexibly including parents in supporting their children.
\end{abstract}

\begin{CCSXML}
<ccs2012>
   <concept>
       <concept_id>10003120.10003121.10003124.10011751</concept_id>
       <concept_desc>Human-centered computing~Collaborative interaction</concept_desc>
       <concept_significance>500</concept_significance>
       </concept>
   <concept>
       <concept_id>10003120.10003121.10011748</concept_id>
       <concept_desc>Human-centered computing~Empirical studies in HCI</concept_desc>
       <concept_significance>500</concept_significance>
       </concept>
   <concept>
       <concept_id>10010405.10010489.10010491</concept_id>
       <concept_desc>Applied computing~Interactive learning environments</concept_desc>
       <concept_significance>500</concept_significance>
       </concept>
 </ccs2012>
\end{CCSXML}

\ccsdesc[500]{Human-centered computing~Collaborative interaction}
\ccsdesc[500]{Human-centered computing~Empirical studies in HCI}
\ccsdesc[500]{Applied computing~Interactive learning environments}

\keywords{Human-robot interaction, human-AI interaction, large language model (LLM), flexible parental involvment, parent-child dyads, informal learning, young children, home, field study}

%%%%% teaser %%%%%
\begin{teaserfigure}
  \includegraphics[width=\textwidth]{figures/figure-teasor-hho-v2.pdf}
  %\vspace{-12pt}
  \caption{We explored parental involvement scenarios using the \texttt{SET} activity kit, examined parents' perceptions of AI-generated content, and analyzed their use of \texttt{PAiREd} for collaborating with LLM and robots to support their children's learning.}
  \label{fig:individual}
   %\vspace{-6pt}
\end{teaserfigure}
%%%%%%%%%%%%%%%%%%
\sloppy
\maketitle




\section{Introduction}

% \textcolor{red}{Still on working}

% \textcolor{red}{add label for each section}


Robot learning relies on diverse and high-quality data to learn complex behaviors \cite{aldaco2024aloha, wang2024dexcap}.
Recent studies highlight that models trained on datasets with greater complexity and variation in the domain tend to generalize more effectively across broader scenarios \cite{mann2020language, radford2021learning, gao2024efficient}.
% However, creating such diverse datasets in the real world presents significant challenges.
% Modifying physical environments and adjusting robot hardware settings require considerable time, effort, and financial resources.
% In contrast, simulation environments offer a flexible and efficient alternative.
% Simulations allow for the creation and modification of digital environments with a wide range of object shapes, weights, materials, lighting, textures, friction coefficients, and so on to incorporate domain randomization,
% which helps improve the robustness of models when deployed in real-world conditions.
% These environments can be easily adjusted and reset, enabling faster iterations and data collection.
% Additionally, simulations provide the ability to consistently reproduce scenarios, which is essential for benchmarking and model evaluation.
% Another advantage of simulations is their flexibility in sensor integration. Sensors such as cameras, LiDARs, and tactile sensors can be added or repositioned without the physical limitations present in real-world setups. Simulations also eliminate the risk of damaging expensive hardware during edge-case experiments, making them an ideal platform for testing rare or dangerous scenarios that are impractical to explore in real life.
By leveraging immersive perspectives and interactions, Extended Reality\footnote{Extended Reality is an umbrella term to refer to Augmented Reality, Mixed Reality, and Virtual Reality \cite{wikipediaExtendedReality}}
(XR)
is a promising candidate for efficient and intuitive large scale data collection \cite{jiang2024comprehensive, arcade}
% With the demand for collecting data, XR provides a promising approach for humans to teach robots by offering users an immersive experience.
in simulation \cite{jiang2024comprehensive, arcade, dexhub-park} and real-world scenarios \cite{openteach, opentelevision}.
However, reusing and reproducing current XR approaches for robot data collection for new settings and scenarios is complicated and requires significant effort.
% are difficult to reuse and reproduce system makes it hard to reuse and reproduce in another data collection pipeline.
This bottleneck arises from three main limitations of current XR data collection and interaction frameworks: \textit{asset limitation}, \textit{simulator limitation}, and \textit{device limitation}.
% \textcolor{red}{ASSIGN THESE CITATION PROPERLY:}
% \textcolor{red}{list them by time order???}
% of collecting data by using XR have three main limitations.
Current approaches suffering from \textit{asset limitation} \cite{arclfd, jiang2024comprehensive, arcade, george2025openvr, vicarios}
% Firstly, recent works \cite{jiang2024comprehensive, arcade, dexhub-park}
can only use predefined robot models and task scenes. Configuring new tasks requires significant effort, since each new object or model must be specifically integrated into the XR application.
% and it takes too much effort to configure new tasks in their systems since they cannot spawn arbitrary models in the XR application.
The vast majority of application are developed for specific simulators or real-world scenarios. This \textit{simulator limitation} \cite{mosbach2022accelerating, lipton2017baxter, dexhub-park, arcade}
% Secondly, existing systems are limited to a single simulation platform or real-world scenarios.
significantly reduces reusability and makes adaptation to new simulation platforms challenging.
Additionally, most current XR frameworks are designed for a specific version of a single XR headset, leading to a \textit{device limitation} 
\cite{lipton2017baxter, armada, openteach, meng2023virtual}.
% and there is no work working on the extendability of transferring to a new headsets as far as we know.
To the best of our knowledge, no existing work has explored the extensibility or transferability of their framework to different headsets.
These limitations hamper reproducibility and broader contributions of XR based data collection and interaction to the research community.
% as each research group typically has its own data collection pipeline.
% In addition to these main limitations, existing XR systems are not well suited for managing multiple robot systems,
% as they are often designed for single-operator use.

In addition to these main limitations, existing XR systems are often designed for single-operator use, prohibiting collaborative data collection.
At the same time, controlling multiple robots at once can be very difficult for a single operator,
making data collection in multi-robot scenarios particularly challenging \cite{orun2019effect}.
Although there are some works using collaborative data collection in the context of tele-operation \cite{tung2021learning, Qin2023AnyTeleopAG},
there is no XR-based data collection system supporting collaborative data collection.
This limitation highlights the need for more advanced XR solutions that can better support multi-robot and multi-user scenarios.
% \textcolor{red}{more papers about collaborative data collection}

To address all of these issues, we propose \textbf{IRIS},
an \textbf{I}mmersive \textbf{R}obot \textbf{I}nteraction \textbf{S}ystem.
This general system supports various simulators, benchmarks and real-world scenarios.
It is easily extensible to new simulators and XR headsets.
IRIS achieves generalization across six dimensions:
% \begin{itemize}
%     \item \textit{Cross-scene} : diverse object models;
%     \item \textit{Cross-embodiment}: diverse robot models;
%     \item \textit{Cross-simulator}: 
%     \item \textit{Cross-reality}: fd
%     \item \textit{Cross-platform}: fd
%     \item \textit{Cross-users}: fd
% \end{itemize}
\textbf{Cross-Scene}, \textbf{Cross-Embodiment}, \textbf{Cross-Simulator}, \textbf{Cross-Reality}, \textbf{Cross-Platform}, and \textbf{Cross-User}.

\textbf{Cross-Scene} and \textbf{Cross-Embodiment} allow the system to handle arbitrary objects and robots in the simulation,
eliminating restrictions about predefined models in XR applications.
IRIS achieves these generalizations by introducing a unified scene specification, representing all objects,
including robots, as data structures with meshes, materials, and textures.
The unified scene specification is transmitted to the XR application to create and visualize an identical scene.
By treating robots as standard objects, the system simplifies XR integration,
allowing researchers to work with various robots without special robot-specific configurations.
\textbf{Cross-Simulator} ensures compatibility with various simulation engines.
IRIS simplifies adaptation by parsing simulated scenes into the unified scene specification, eliminating the need for XR application modifications when switching simulators.
New simulators can be integrated by creating a parser to convert their scenes into the unified format.
This flexibility is demonstrated by IRIS’ support for Mujoco \cite{todorov2012mujoco}, IsaacSim \cite{mittal2023orbit}, CoppeliaSim \cite{coppeliaSim}, and even the recent Genesis \cite{Genesis} simulator.
\textbf{Cross-Reality} enables the system to function seamlessly in both virtual simulations and real-world applications.
IRIS enables real-world data collection through camera-based point cloud visualization.
\textbf{Cross-Platform} allows for compatibility across various XR devices.
Since XR device APIs differ significantly, making a single codebase impractical, IRIS XR application decouples its modules to maximize code reuse.
This application, developed by Unity \cite{unity3dUnityManual}, separates scene visualization and interaction, allowing developers to integrate new headsets by reusing the visualization code and only implementing input handling for hand, head, and motion controller tracking.
IRIS provides an implementation of the XR application in the Unity framework, allowing for a straightforward deployment to any device that supports Unity. 
So far, IRIS was successfully deployed to the Meta Quest 3 and HoloLens 2.
Finally, the \textbf{Cross-User} ability allows multiple users to interact within a shared scene.
IRIS achieves this ability by introducing a protocol to establish the communication between multiple XR headsets and the simulation or real-world scenarios.
Additionally, IRIS leverages spatial anchors to support the alignment of virtual scenes from all deployed XR headsets.
% To make an seamless user experience for robot learning data collection,
% IRIS also tested in three different robot control interface
% Furthermore, to demonstrate the extensibility of our approach, we have implemented a robot-world pipeline for real robot data collection, ensuring that the system can be used in both simulated and real-world environments.
The Immersive Robot Interaction System makes the following contributions\\
\textbf{(1) A unified scene specification} that is compatible with multiple robot simulators. It enables various XR headsets to visualize and interact with simulated objects and robots, providing an immersive experience while ensuring straightforward reusability and reproducibility.\\
\textbf{(2) A collaborative data collection framework} designed for XR environments. The framework facilitates enhanced robot data acquisition.\\
\textbf{(3) A user study} demonstrating that IRIS significantly improves data collection efficiency and intuitiveness compared to the LIBERO baseline.

% \begin{table*}[t]
%     \centering
%     \begin{tabular}{lccccccc}
%         \toprule
%         & \makecell{Physical\\Interaction}
%         & \makecell{XR\\Enabled}
%         & \makecell{Free\\View}
%         & \makecell{Multiple\\Robots}
%         & \makecell{Robot\\Control}
%         % Force Feedback???
%         & \makecell{Soft Object\\Supported}
%         & \makecell{Collaborative\\Data} \\
%         \midrule
%         ARC-LfD \cite{arclfd}                              & Real        & \cmark & \xmark & \xmark & Joint              & \xmark & \xmark \\
%         DART \cite{dexhub-park}                            & Sim         & \cmark & \cmark & \cmark & Cartesian          & \xmark & \xmark \\
%         \citet{jiang2024comprehensive}                     & Sim         & \cmark & \xmark & \xmark & Joint \& Cartesian & \xmark & \xmark \\
%         \citet{mosbach2022accelerating}                    & Sim         & \cmark & \cmark & \xmark & Cartesian          & \xmark & \xmark \\
%         ARCADE \cite{arcade}                               & Real        & \cmark & \cmark & \xmark & Cartesian          & \xmark & \xmark \\
%         Holo-Dex \cite{holodex}                            & Real        & \cmark & \xmark & \cmark & Cartesian          & \cmark & \xmark \\
%         ARMADA \cite{armada}                               & Real        & \cmark & \xmark & \cmark & Cartesian          & \cmark & \xmark \\
%         Open-TeleVision \cite{opentelevision}              & Real        & \cmark & \cmark & \cmark & Cartesian          & \cmark & \xmark \\
%         OPEN TEACH \cite{openteach}                        & Real        & \cmark & \xmark & \cmark & Cartesian          & \cmark & \cmark \\
%         GELLO \cite{wu2023gello}                           & Real        & \xmark & \cmark & \cmark & Joint              & \cmark & \xmark \\
%         DexCap \cite{wang2024dexcap}                       & Real        & \xmark & \cmark & \xmark & Cartesian          & \cmark & \xmark \\
%         AnyTeleop \cite{Qin2023AnyTeleopAG}                & Real        & \xmark & \xmark & \cmark & Cartesian          & \cmark & \cmark \\
%         Vicarios \cite{vicarios}                           & Real        & \cmark & \xmark & \xmark & Cartesian          & \cmark & \xmark \\     
%         Augmented Visual Cues \cite{augmentedvisualcues}   & Real        & \cmark & \cmark & \xmark & Cartesian          & \xmark & \xmark \\ 
%         \citet{wang2024robotic}                            & Real        & \cmark & \cmark & \xmark & Cartesian          & \cmark & \xmark \\
%         Bunny-VisionPro \cite{bunnyvisionpro}              & Real        & \cmark & \cmark & \cmark & Cartesian          & \cmark & \xmark \\
%         IMMERTWIN \cite{immertwin}                         & Real        & \cmark & \cmark & \cmark & Cartesian          & \xmark & \xmark \\
%         \citet{meng2023virtual}                            & Sim \& Real & \cmark & \cmark & \xmark & Cartesian          & \xmark & \xmark \\
%         Shared Control Framework \cite{sharedctlframework} & Real        & \cmark & \cmark & \cmark & Cartesian          & \xmark & \xmark \\
%         OpenVR \cite{openvr}                               & Real        & \cmark & \cmark & \xmark & Cartesian          & \xmark & \xmark \\
%         \citet{digitaltwinmr}                              & Real        & \cmark & \cmark & \xmark & Cartesian          & \cmark & \xmark \\
        
%         \midrule
%         \textbf{Ours} & Sim \& Real & \cmark & \cmark & \cmark & Joint \& Cartesian  & \cmark & \cmark \\
%         \bottomrule
%     \end{tabular}
%     \caption{This is a cross-column table with automatic line breaking.}
%     \label{tab:cross-column}
% \end{table*}

% \begin{table*}[t]
%     \centering
%     \begin{tabular}{lccccccc}
%         \toprule
%         & \makecell{Cross-Embodiment}
%         & \makecell{Cross-Scene}
%         & \makecell{Cross-Simulator}
%         & \makecell{Cross-Reality}
%         & \makecell{Cross-Platform}
%         & \makecell{Cross-User} \\
%         \midrule
%         ARC-LfD \cite{arclfd}                              & \xmark & \xmark & \xmark & \xmark & \xmark & \xmark \\
%         DART \cite{dexhub-park}                            & \cmark & \cmark & \xmark & \xmark & \xmark & \xmark \\
%         \citet{jiang2024comprehensive}                     & \xmark & \cmark & \xmark & \xmark & \xmark & \xmark \\
%         \citet{mosbach2022accelerating}                    & \xmark & \cmark & \xmark & \xmark & \xmark & \xmark \\
%         ARCADE \cite{arcade}                               & \xmark & \xmark & \xmark & \xmark & \xmark & \xmark \\
%         Holo-Dex \cite{holodex}                            & \cmark & \xmark & \xmark & \xmark & \xmark & \xmark \\
%         ARMADA \cite{armada}                               & \cmark & \xmark & \xmark & \xmark & \xmark & \xmark \\
%         Open-TeleVision \cite{opentelevision}              & \cmark & \xmark & \xmark & \xmark & \cmark & \xmark \\
%         OPEN TEACH \cite{openteach}                        & \cmark & \xmark & \xmark & \xmark & \xmark & \cmark \\
%         GELLO \cite{wu2023gello}                           & \cmark & \xmark & \xmark & \xmark & \xmark & \xmark \\
%         DexCap \cite{wang2024dexcap}                       & \xmark & \xmark & \xmark & \xmark & \xmark & \xmark \\
%         AnyTeleop \cite{Qin2023AnyTeleopAG}                & \cmark & \cmark & \cmark & \cmark & \xmark & \cmark \\
%         Vicarios \cite{vicarios}                           & \xmark & \xmark & \xmark & \xmark & \xmark & \xmark \\     
%         Augmented Visual Cues \cite{augmentedvisualcues}   & \xmark & \xmark & \xmark & \xmark & \xmark & \xmark \\ 
%         \citet{wang2024robotic}                            & \xmark & \xmark & \xmark & \xmark & \xmark & \xmark \\
%         Bunny-VisionPro \cite{bunnyvisionpro}              & \cmark & \xmark & \xmark & \xmark & \xmark & \xmark \\
%         IMMERTWIN \cite{immertwin}                         & \cmark & \xmark & \xmark & \xmark & \xmark & \xmark \\
%         \citet{meng2023virtual}                            & \xmark & \cmark & \xmark & \cmark & \xmark & \xmark \\
%         \citet{sharedctlframework}                         & \cmark & \xmark & \xmark & \xmark & \xmark & \xmark \\
%         OpenVR \cite{george2025openvr}                               & \xmark & \xmark & \xmark & \xmark & \xmark & \xmark \\
%         \citet{digitaltwinmr}                              & \xmark & \xmark & \xmark & \xmark & \xmark & \xmark \\
        
%         \midrule
%         \textbf{Ours} & \cmark & \cmark & \cmark & \cmark & \cmark & \cmark \\
%         \bottomrule
%     \end{tabular}
%     \caption{This is a cross-column table with automatic line breaking.}
% \end{table*}

% \begin{table*}[t]
%     \centering
%     \begin{tabular}{lccccccc}
%         \toprule
%         & \makecell{Cross-Scene}
%         & \makecell{Cross-Embodiment}
%         & \makecell{Cross-Simulator}
%         & \makecell{Cross-Reality}
%         & \makecell{Cross-Platform}
%         & \makecell{Cross-User}
%         & \makecell{Control Space} \\
%         \midrule
%         % Vicarios \cite{vicarios}                           & \xmark & \xmark & \xmark & \xmark & \xmark & \xmark \\     
%         % Augmented Visual Cues \cite{augmentedvisualcues}   & \xmark & \xmark & \xmark & \xmark & \xmark & \xmark \\ 
%         % OpenVR \cite{george2025openvr}                     & \xmark & \xmark & \xmark & \xmark & \xmark & \xmark \\
%         \citet{digitaltwinmr}                              & \xmark & \xmark & \xmark & \xmark & \xmark & \xmark &  \\
%         ARC-LfD \cite{arclfd}                              & \xmark & \xmark & \xmark & \xmark & \xmark & \xmark &  \\
%         \citet{sharedctlframework}                         & \cmark & \xmark & \xmark & \xmark & \xmark & \xmark &  \\
%         \citet{jiang2024comprehensive}                     & \cmark & \xmark & \xmark & \xmark & \xmark & \xmark &  \\
%         \citet{mosbach2022accelerating}                    & \cmark & \xmark & \xmark & \xmark & \xmark & \xmark & \\
%         Holo-Dex \cite{holodex}                            & \cmark & \xmark & \xmark & \xmark & \xmark & \xmark & \\
%         ARCADE \cite{arcade}                               & \cmark & \cmark & \xmark & \xmark & \xmark & \xmark & \\
%         DART \cite{dexhub-park}                            & Limited & Limited & Mujoco & Sim & Vision Pro & \xmark &  Cartesian\\
%         ARMADA \cite{armada}                               & \cmark & \cmark & \xmark & \xmark & \xmark & \xmark & \\
%         \citet{meng2023virtual}                            & \cmark & \cmark & \xmark & \cmark & \xmark & \xmark & \\
%         % GELLO \cite{wu2023gello}                           & \cmark & \xmark & \xmark & \xmark & \xmark & \xmark \\
%         % DexCap \cite{wang2024dexcap}                       & \xmark & \xmark & \xmark & \xmark & \xmark & \xmark \\
%         % AnyTeleop \cite{Qin2023AnyTeleopAG}                & \cmark & \cmark & \cmark & \cmark & \xmark & \cmark \\
%         % \citet{wang2024robotic}                            & \xmark & \xmark & \xmark & \xmark & \xmark & \xmark \\
%         Bunny-VisionPro \cite{bunnyvisionpro}              & \cmark & \cmark & \xmark & \xmark & \xmark & \xmark & \\
%         IMMERTWIN \cite{immertwin}                         & \cmark & \cmark & \xmark & \xmark & \xmark & \xmark & \\
%         Open-TeleVision \cite{opentelevision}              & \cmark & \cmark & \xmark & \xmark & \cmark & \xmark & \\
%         \citet{szczurek2023multimodal}                     & \xmark & \xmark & \xmark & Real & \xmark & \cmark & \\
%         OPEN TEACH \cite{openteach}                        & \cmark & \cmark & \xmark & \xmark & \xmark & \cmark & \\
%         \midrule
%         \textbf{Ours} & \cmark & \cmark & \cmark & \cmark & \cmark & \cmark \\
%         \bottomrule
%     \end{tabular}
%     \caption{TODO, Bruce: this table can be further optimized.}
% \end{table*}

\definecolor{goodgreen}{HTML}{228833}
\definecolor{goodred}{HTML}{EE6677}
\definecolor{goodgray}{HTML}{BBBBBB}

\begin{table*}[t]
    \centering
    \begin{adjustbox}{max width=\textwidth}
    \renewcommand{\arraystretch}{1.2}    
    \begin{tabular}{lccccccc}
        \toprule
        & \makecell{Cross-Scene}
        & \makecell{Cross-Embodiment}
        & \makecell{Cross-Simulator}
        & \makecell{Cross-Reality}
        & \makecell{Cross-Platform}
        & \makecell{Cross-User}
        & \makecell{Control Space} \\
        \midrule
        % Vicarios \cite{vicarios}                           & \xmark & \xmark & \xmark & \xmark & \xmark & \xmark \\     
        % Augmented Visual Cues \cite{augmentedvisualcues}   & \xmark & \xmark & \xmark & \xmark & \xmark & \xmark \\ 
        % OpenVR \cite{george2025openvr}                     & \xmark & \xmark & \xmark & \xmark & \xmark & \xmark \\
        \citet{digitaltwinmr}                              & \textcolor{goodred}{Limited}     & \textcolor{goodred}{Single Robot} & \textcolor{goodred}{Unity}    & \textcolor{goodred}{Real}          & \textcolor{goodred}{Meta Quest 2} & \textcolor{goodgray}{N/A} & \textcolor{goodred}{Cartesian} \\
        ARC-LfD \cite{arclfd}                              & \textcolor{goodgray}{N/A}        & \textcolor{goodred}{Single Robot} & \textcolor{goodgray}{N/A}     & \textcolor{goodred}{Real}          & \textcolor{goodred}{HoloLens}     & \textcolor{goodgray}{N/A} & \textcolor{goodred}{Cartesian} \\
        \citet{sharedctlframework}                         & \textcolor{goodred}{Limited}     & \textcolor{goodred}{Single Robot} & \textcolor{goodgray}{N/A}     & \textcolor{goodred}{Real}          & \textcolor{goodred}{HTC Vive Pro} & \textcolor{goodgray}{N/A} & \textcolor{goodred}{Cartesian} \\
        \citet{jiang2024comprehensive}                     & \textcolor{goodred}{Limited}     & \textcolor{goodred}{Single Robot} & \textcolor{goodgray}{N/A}     & \textcolor{goodred}{Real}          & \textcolor{goodred}{HoloLens 2}   & \textcolor{goodgray}{N/A} & \textcolor{goodgreen}{Joint \& Cartesian} \\
        \citet{mosbach2022accelerating}                    & \textcolor{goodgreen}{Available} & \textcolor{goodred}{Single Robot} & \textcolor{goodred}{IsaacGym} & \textcolor{goodred}{Sim}           & \textcolor{goodred}{Vive}         & \textcolor{goodgray}{N/A} & \textcolor{goodgreen}{Joint \& Cartesian} \\
        Holo-Dex \cite{holodex}                            & \textcolor{goodgray}{N/A}        & \textcolor{goodred}{Single Robot} & \textcolor{goodgray}{N/A}     & \textcolor{goodred}{Real}          & \textcolor{goodred}{Meta Quest 2} & \textcolor{goodgray}{N/A} & \textcolor{goodred}{Joint} \\
        ARCADE \cite{arcade}                               & \textcolor{goodgray}{N/A}        & \textcolor{goodred}{Single Robot} & \textcolor{goodgray}{N/A}     & \textcolor{goodred}{Real}          & \textcolor{goodred}{HoloLens 2}   & \textcolor{goodgray}{N/A} & \textcolor{goodred}{Cartesian} \\
        DART \cite{dexhub-park}                            & \textcolor{goodred}{Limited}     & \textcolor{goodred}{Limited}      & \textcolor{goodred}{Mujoco}   & \textcolor{goodred}{Sim}           & \textcolor{goodred}{Vision Pro}   & \textcolor{goodgray}{N/A} & \textcolor{goodred}{Cartesian} \\
        ARMADA \cite{armada}                               & \textcolor{goodgray}{N/A}        & \textcolor{goodred}{Limited}      & \textcolor{goodgray}{N/A}     & \textcolor{goodred}{Real}          & \textcolor{goodred}{Vision Pro}   & \textcolor{goodgray}{N/A} & \textcolor{goodred}{Cartesian} \\
        \citet{meng2023virtual}                            & \textcolor{goodred}{Limited}     & \textcolor{goodred}{Single Robot} & \textcolor{goodred}{PhysX}   & \textcolor{goodgreen}{Sim \& Real} & \textcolor{goodred}{HoloLens 2}   & \textcolor{goodgray}{N/A} & \textcolor{goodred}{Cartesian} \\
        % GELLO \cite{wu2023gello}                           & \cmark & \xmark & \xmark & \xmark & \xmark & \xmark \\
        % DexCap \cite{wang2024dexcap}                       & \xmark & \xmark & \xmark & \xmark & \xmark & \xmark \\
        % AnyTeleop \cite{Qin2023AnyTeleopAG}                & \cmark & \cmark & \cmark & \cmark & \xmark & \cmark \\
        % \citet{wang2024robotic}                            & \xmark & \xmark & \xmark & \xmark & \xmark & \xmark \\
        Bunny-VisionPro \cite{bunnyvisionpro}              & \textcolor{goodgray}{N/A}        & \textcolor{goodred}{Single Robot} & \textcolor{goodgray}{N/A}     & \textcolor{goodred}{Real}          & \textcolor{goodred}{Vision Pro}   & \textcolor{goodgray}{N/A} & \textcolor{goodred}{Cartesian} \\
        IMMERTWIN \cite{immertwin}                         & \textcolor{goodgray}{N/A}        & \textcolor{goodred}{Limited}      & \textcolor{goodgray}{N/A}     & \textcolor{goodred}{Real}          & \textcolor{goodred}{HTC Vive}     & \textcolor{goodgray}{N/A} & \textcolor{goodred}{Cartesian} \\
        Open-TeleVision \cite{opentelevision}              & \textcolor{goodgray}{N/A}        & \textcolor{goodred}{Limited}      & \textcolor{goodgray}{N/A}     & \textcolor{goodred}{Real}          & \textcolor{goodgreen}{Meta Quest, Vision Pro} & \textcolor{goodgray}{N/A} & \textcolor{goodred}{Cartesian} \\
        \citet{szczurek2023multimodal}                     & \textcolor{goodgray}{N/A}        & \textcolor{goodred}{Limited}      & \textcolor{goodgray}{N/A}     & \textcolor{goodred}{Real}          & \textcolor{goodred}{HoloLens 2}   & \textcolor{goodgreen}{Available} & \textcolor{goodred}{Joint \& Cartesian} \\
        OPEN TEACH \cite{openteach}                        & \textcolor{goodgray}{N/A}        & \textcolor{goodgreen}{Available}  & \textcolor{goodgray}{N/A}     & \textcolor{goodred}{Real}          & \textcolor{goodred}{Meta Quest 3} & \textcolor{goodred}{N/A} & \textcolor{goodgreen}{Joint \& Cartesian} \\
        \midrule
        \textbf{Ours}                                      & \textcolor{goodgreen}{Available} & \textcolor{goodgreen}{Available}  & \textcolor{goodgreen}{Mujoco, CoppeliaSim, IsaacSim} & \textcolor{goodgreen}{Sim \& Real} & \textcolor{goodgreen}{Meta Quest 3, HoloLens 2} & \textcolor{goodgreen}{Available} & \textcolor{goodgreen}{Joint \& Cartesian} \\
        \bottomrule
        \end{tabular}
    \end{adjustbox}
    \caption{Comparison of XR-based system for robots. IRIS is compared with related works in different dimensions.}
\end{table*}


\section{Related Work}

In this section, we review research related to the importance and barriers to parental involvement; parental use of learning technologies; and the use of generative AI and robot in educational and parenting scenarios.

\subsection{Importance and Barriers to Parental Involvement}\label{sec-rw-2.1}

% 79 words
Early childhood is a critical period for predicting future success and well-being, with early education investments resulting in higher returns than later interventions \cite{duncan2007school, doyle2009investing}. Effective parental involvement fosters cognitive and social skills, especially in younger children \cite{blevins2016early, peck1992parent}. Parents are encouraged to prioritize home-based involvement to maximize their influence \cite{ma2016meta}, as their involvement has a greater impact on children's learning outcomes \cite{hoffner2002parents, fehrmann1987home, hill2004parent} within the family setting than partnerships with schools or communities \cite{ma2016meta, harris2008parents, fantuzzo2004multiple, sui1996effects}.

However, parents' involvement in their children's education is often constrained by practical challenges related to parents' \textit{skills}, \textit{time}, and \textit{energy}. The Hoover-Dempsey and Sandler (HDS) framework \cite{green2007parents} and the CAM framework \cite{ho2024s} both highlight these factors-- parents' perceived \textit{skills and knowledge} (capability), \textit{time} (availability), and \textit{motivation} (energy)--influence the extent of their engagement. For instance, a parent confident in math may choose to engage more in math-related tasks, while those facing inflexible schedules may participate less \cite{green2007parents}. Unlike teachers, parents often lack formal pedagogical training and may underestimate their role in supporting children's learning, particularly as young children struggle to articulate their needs \cite{hara1998parent}. The CAM framework similarly suggests parents delegate tasks to a robot when they feel less capable, have limited time, or are unmotivated. These factors reflect parents' life contexts, shaped by demographic backgrounds, occupations, and parenting responsibilities \cite{grolnick1997predictors}, highlighting the need to help parents overcome barriers to effective involvement in early education within their life contexts.

\subsection{Parental Use of Learning Technologies}\label{sec-rw-2.2}
% 207 words
Technology encourages parental involvement by facilitating parent-child engagement in learning activities while introducing risks that require active parental mediation \cite{gonzalez2022parental}. On the positive side, technology offers novel opportunities for parental engagement and enhances children's learning outcomes. For example, e-books promote interactive behaviors between parents and children better than print books \cite{korat2010new}. In addition, having access to computers at home significantly boost academic achievement of young children when parents actively mediate their use \cite{hofferth2010home, espinosa2006technology}. However, the effectiveness of these tools often depends on parents' familiarity with and attitude toward technology. Mobile applications, for instance, can improve learning outcomes but require parents to possess sufficient technology efficacy to guide their use \cite{papadakis2019parental}.

On the negative side, technology introduces risks such as excessive screen time, exposure to inappropriate content, and misinformation, which necessitate parental intervention \cite{oswald2020psychological, howard2021digital}. According to parental mediation theory, parents mitigate these risks through restrictive mediation (e.g., setting limits), active mediation (e.g., discussing content), and co-use (e.g., shared use of technology) \cite{valkenburg1999developing}. Modern technologies like video games, location-based games (\textit{e.g.,} Pokemon Go), and conversational agents (\textit{e.g.,} Alexa) also require parents to adapt their mediation strategies to ensure responsible use \cite{valkenburg1999developing, nikken2006parental, sobel2017wasn, beneteau2020parenting, yu2024parent}. Overall, parents seek to leverage technology to support their children's learning due to ite effectivenss but are also mindful of its risks. Their involvement is therefore driven by both opportunities and concerns, highlighting the need to design tools that effectively involve parents to balance benefits and risks.

\subsection{Generative AI and Companion Robots for Parenting and Education}
Generative AI and companion robots offer human-like affordances, with AI simulating human intelligence and robots providing physical human-like features. Compared to conventional models (\textit{e.g.,} machine learning) and devices (\textit{e.g.,} laptops), these emerging technologies enable natural and social interactions, creating opportunities for novel paradigms to enhance parental involvement and children's learning while introducing their unique challenges.

\subsubsection{Generative AI}
GAI offers promising support for parents by enhancing their ability to educate and engage with their children. Prior work suggested that AI-driven systems can support parenting education \cite{petsolari2024socio} and provide evidence-based advice through applications and chatbots, delivering micro-interventions such as teaching parents how to praise their children effectively \cite{davis2017parent, entenberg2023user} or offering strategies to teach complex concepts \cite{mogavi2024chatgpt, su2023unlocking}. Many parents also prefer using GAI to create educational materials tailored to their children's needs, rather than granting children direct access to these tools \cite{han2023design}. Beyond educational support, AI-based storytelling tools address practical challenges (\textit{e.g.,} time constraints) by alleviating physical labor while fostering parent-child interactions \cite{sun2024exploring}. Furthermore, GAI offers advantages to children's learning directly. It can help create personalized learning experiences by providing timely feedback and tailoring content \cite{su2023unlocking, mogavi2024chatgpt, han2024teachers}, enhancing positive learning experiences \cite{jauhiainen2023generative}. For example, a LLM-driven conversational system can teach children mathematical concepts through co-creative storytelling, achieving learning outcomes similar to human-led instructions \cite{zhang2024mathemyths}.

Despite these benefits, several concerns persist regarding the use of GAI in education. Prior work highlighted the limitations of GAI, such as its limited effectiveness in more complex learning tasks,the limited quality of the training data, and its inability to offer comprehensive educational support \cite{su2023unlocking}. There is also a significant risk of GAI producing inaccurate or biased information, discouraging independent thought among children, and threatening user privacy \cite{su2023unlocking, han2023design, han2024teachers}. Many parents are skeptical about the role of AI in their children's academic processes, concerned about the accuracy of AI-generated content, and worry that over-reliance on AI could stifle independent thinking \cite{han2023design}.

%\todo{might need to add some structural transition here}
\subsubsection{Social companion robots}
Social companion robots have proven potential to assist parents in home education settings through studies in \textit{parent-child-robot} interactions. \citet{gvirsman2020patricc} showed that the robotic system, \texttt{Patricc}, fostered more triadic interaction between parents and toddlers than a tablet, and \citet{gvirsman2024effect} found that, in a parent-toddler-robot interaction, parents tend to decrease their scaffolding affectively when the robot increases its scaffolding behavior. Similarly, \citet{chen2022designing} found that social robots enhanced parent-child co-reading activities, while \citet{chan2017wakey} demonstrated that the WAKEY robot improved morning routines and reduced parental frustration. Beyond educational support, \citet{ho2024s} uncovered that parents envisioned robots as their \textit{collaborators} to support their children's learning at home and that their collaboration patterns can be determined by the parents' capability, availability, and motivation. Although parents generally have positive attitudes toward incorporating robots into their children's learning, they remain concerned about the risk of disrupting school-based learning and potential teacher replacement \cite{tolksdorf2020parents, lee2008elementary, louie2021desire}.

In addition to parental support, social companion robots also support children in education directly through \textit{child-robot interactions}. Physically embodied robots provide adaptive assistance and verbal interaction similar to virtual or conversational agents \cite{ramachandran2019personalized, leyzberg2014personalizing, schodde2017adaptive, brown2014positive}, yet they foster greater engagement with the physical environment and encourage more advanced social behaviors during learning \cite{belpaeme2018social}, leading to improved learning outcomes \cite{leyzberg2012physical}. Prior work demonstrated that companion robots can effectively support both school-based learning (\textit{e.g.,} math \cite{lopez2018robotic}, literacy \cite{kennedy2016social, gordon2016affective}, and science \cite{davison2020working}) and home-based learning activities (\textit{e.g.,} reading \cite{michaelis2018reading, michaelis2019supporting}, number board games \cite{ho2021robomath}, and math-oriented conversations with parents \cite{ho2023designing}). For example, \citet{kennedy2016social} suggested that children can learn elements of a second language from a robot in short-term interactions, and \citet{tanaka2009use} found that children who took on the role of teaching the robot gained confidence and improved learning outcomes.

%\todo{may need to make this a separate section and explain why we propose AI-assisted robots}

% \subsubsection{Research Gap}
Parental involvement in early education is crucial and AI-assisted robots can offer promising support by helping parents overcome practical barriers (\textit{i.e.,} time, energy, and skills) and addressing concerns about technological risks. Yet, limited research has examined how technology design can simultaneously alleviate these barriers and concerns. Though \citet{zhang2022storybuddy} emphasized the importance of flexible parental involvement during reading through a system called \textit{Storybuddy}, yet they focused on a virtual chatbot rather than a physical robot, and how the flexible modes may be used in different scenarios remain unknown. Similarly, \textit{ContextQ} \cite{dietz2024contextq} presented auto-generated dialogic questions to caregivers for dialogic reading, but primarily considered situations where parents are actively involved, not scenarios where parents cannot participate fully.

In this work, we address these gaps by exploring parental involvement contexts, understanding parents' perceptions of AI-generated content, and examining how parents collaborate with AI and robots under different scenarios. In the following sections, we describe our development of  \texttt{SET}, a card-based activity, to understand parental involvement contexts (Section~\ref{sec-card}), the design of the \texttt{PAiREd} system to enable parents to co-create learning activities with an LLM (Section~\ref{sec-system}), and user study aimed to discover use patterns and understand user perceptions of the system (Section~\ref{sec-study}).
\section{Card-Based Activity Kit Design} \label{sec-card}

\begin{table*}[!t]
    \caption{Factors and dimensions used in the Card-based Activity Kit}
    \label{tab:kit-example}
    \centering
    \small
    \begin{tabular}{p{0.12\linewidth}p{0.4\linewidth}p{0.4\linewidth}}%
        \toprule
         \textbf{Factor-level} & \textbf{Description} & \textbf{Example} \\
        \midrule
         Skill-high & Parents feels \textit{very} confident in a specific task or topic. & ``\textit{I am good at math-related activities.}'' \\
         \midrule
         Skill-low & Parents feels \textit{not} confident in a specific task or topic. & ``\textit{I am not good at coming up with good questions to ask.}'' \\
         \midrule
         Energy-high & Parents feels \textit{highly} motivated. & ``\textit{I am highly motivated when my child invites me.}'' \\
         \midrule
         Energy-low & Parents feels \textit{not} motivated. & ``\textit{I am not motivated when I am tired.}'' \\
         \midrule
         Time-high & Parents are \textit{available and present}. & ``\textit{I am available and present when my work is done.}'' \\
         \midrule
         Time-low & Parents \textit{need to be absent}. & ``\textit{I need to be absent when my younger child is crying.}'' \\     
         \bottomrule
    \end{tabular}
\end{table*}

To systematically explore parents' contextual, real-life scenarios where varying levels of parental involvement in children's learning may occur, we designed a card-based activity kit named \textbf{\texttt{SET}}, representing \textbf{\texttt{S}}kill, \textbf{\texttt{E}}nergy, and \textbf{\texttt{T}}ime. The design of the kit is grounded in frameworks from education \cite{green2007parents} and HCI \cite{ho2024s}. Specifically, we leveraged the ``\textit{Parent Perceived Life Context}'' construct from the Hoover-Dempsey and Sandler (HDS) framework for parental involvement \cite{green2007parents} and the key factors influencing parent-robot collaboration defined by \citet{ho2024s}. We detailed the theoretical foundation in Section \ref{sec-rw-2.1}.

The kit consists of three primary materials: \texttt{SET-banners}, \texttt{SET-} \texttt{experience banks}, and \texttt{SET-scenario cards}\footnote{\url{https://osf.io/zfksg/?view_only=b59bd41287f543ce82ab85950aaf004f}}. Each material is characterized by the same two-dimensional factors (\textit{i.e.,} skill (high/low), energy (high/low), and time (high/low)) but is designed in different formats to serve distinct purposes during the card activity (see Table \ref{tab:kit-example} for details about each factor and its dimensions). Although the factors theoretically exist on a continuous spectrum, we divided them into binary dimensions to ensure the scale of the activity is manageable for participants. Below, we describe the purpose of each material. For more details on how the kit was used, refer to Section \ref{sec-procedure}.

\begin{enumerate}
    \item The \texttt{SET-banners} present a hierarchical diagram that depicts how the three two-dimensional factors combine to create \textit{eight} distinct scenarios (Figure \ref{fig:card-kit}), forming the \textit{eight} \texttt{SET-scenario cards}.
    \item The \texttt{SET-experience banks} enable participants to document real-life examples for each dimension of every factor using post-it notes. This activity encourages focused reflection on individual factors and dimensions, serving as a stepping stone for constructing scenarios that integrate all three factors during the \texttt{SET-scenario cards} activity (Figure \ref{fig:card-kit}).
    \item The \texttt{SET-scenario cards} guide participants in creating \textit{eight} real-life scenarios, each defined by unique combinations of the three two-dimensional factors. The front of each card displays the dimensional status of the factors, while the back includes detailed descriptions of each factor's status and a blank space for participants to document a scenario that aligns with the corresponding dimensions (Figure \ref{fig:card-kit}).
\end{enumerate}

\begin{figure*}[b!]
  \includegraphics[width=\textwidth]{figures/figure-card-hho-2.pdf}
   \vspace{-6pt}
  \caption{The \texttt{SET} Card-Based Activity Kit. (1) Banners are used to visualize a hierarchical diagram. (2) Experience banks allow participants to generate examples for each factor dimension. (3) Scenario cards guide participants to create real-life scenarios.}
  \label{fig:card-kit}
  \vspace{-3pt}
\end{figure*}




\section{System Design} \label{sec-system}

We designed a prototype system, \texttt{PAiREd}, an LLM-driven interface integrated with an educational robot that uses LLMs to generate learning content based on established educational frameworks for reading activities. The system enables parents to review and revise LLM-generated learning content \textit{before} the activity and adjust their involvement by modifying the role delegation between the robot and themselves \textit{during} the activity. The system consists of three core design components: (1) \textit{LLM-driven content generation}, (2) \textit{Interface design}, and (3) \textit{Robot interaction design} (Figure \ref{fig:syste-diagram}).

\begin{figure*}[t!]
  \includegraphics[width=\textwidth]{figures/figure-system-hho.pdf}
   \vspace{-6pt}
  \caption{\texttt{PAiREd} System Architecture Overview. The system consists of three key components: prompt instructions and data, the web application (which includes both the editor and activity interfaces), and the robot interaction module.}
  \label{fig:syste-diagram}
   \vspace{-10pt}
\end{figure*}

\subsection{LLM-driven Content Generation}
% limitation of gpt-4o
Content generation is powered by a state-of-the-art LLM, ChatGPT-4o \cite{achiam2023gpt} developed by OpenAI\footnote{OpenAI ChatGPT-4o: \url{https://openai.com/index/hello-gpt-4o/}} through prompt engineering \cite{sahoo2024systematic}. To generate the learning content for each book page, the input we provided for GPT-4o includes: (1) \textit{prompts}\footnote{Prompts: \url{https://osf.io/zfksg/?view_only=b59bd41287f543ce82ab85950aaf004f}}: instructions to generate learning content, (2) \textit{book image}: an image of the chosen book page, (3) \textit{visual context}\footnote{Visual context: \url{https://osf.io/zfksg/?view_only=b59bd41287f543ce82ab85950aaf004f}}: a manually-created JSON-based dataset containing structured visual information of the book page, and (4) \textit{educational frameworks} \footnote{Frameworks: \url{https://osf.io/zfksg/?view_only=b59bd41287f543ce82ab85950aaf004f}}: theoretical-driven frameworks defining critical pre-school concepts for math and literacy. The books, visual contexts, and educational frameworks are placed in our database and retrieved using prompt instructions. We explain \textit{visual context} and \textit{educational frameworks} below.

\subsubsection{Visual Context} 
The ChatGPT-4o model faces limitations in quantitative and spatial reasoning on images and recognition of animated or anthropomorphic drawings (\textit{e.g.,} mistaking clothed animals for humans), limiting its ability to accurately auto-generate question-answer pairs for literacy and math based on storybook images. To enhance content generation accuracy, we created a JSON-based dataset to capture the \textit{visual context} of each storybook page. Each page is represented as a unique entity with detailed descriptions of objects (\textit{e.g.,} characters, animals, and environmental elements), properties, as well as their spatial relationships. This context was embedded in the prompt design, instructing GPT-4o to retrieve precise object properties such as color, count, and location. Integrating this structured approach significantly improved the accuracy of ChatGPT-4o to create question-answer pairs based on storybooks, reducing object-identification errors and enhancing consistency across pages. Systematic evaluations by the first and third authors confirmed these improvements.

% \begin{figure}[b!]
%   \includegraphics[width=\columnwidth]{figures/figure-editor-amy.pdf}
%    \vspace{-6pt}
%   \caption{Top: Editor Interface of the PAiREd system. This interface allows users to navigate book content, modify LLM-generated learning content through regeneration, or manually edit it. Bottom: Activity Interface of the PAiREd system. This interface enables parents to flexibly adjust their involvement and seamlessly share responsibilities with the robot through the mode-switching and role-delegation mechanisms. }
%   \label{fig:editor}
%    \vspace{-6pt}
% \end{figure}

\begin{figure*}[b!]
  \includegraphics[width=\textwidth]{figures/figure-editor-amy.pdf}
     \caption{Editor Interface of the PAiREd system. This interface allows users to navigate book content, modify LLM-generated learning content through regeneration, or manually edit it.}
  \label{fig:editor}
\end{figure*}

\begin{figure*}[b!]
\centering
  \includegraphics[width=\textwidth]{figures/figure-activity-amy.pdf}
  \caption{Activity Interface of the PAiREd system. This interface enables parents to flexibly adjust their involvement and seamlessly share responsibilities with the robot through the mode-switching and role-delegation mechanisms.}
  \label{fig:activity}
   \vspace{-10pt}
\end{figure*}
 
\subsubsection{Educational Frameworks for Math and Literacy} 
We developed frameworks outlining varying proficiency levels in math \cite{purpura2013informal, engel2013teaching} and literacy \cite{kaminski2014preschool} concepts for preschoolers, drawing on educational psychology research. Math and literacy were selected as they are foundational skills for preschoolers \cite{weiland2013impacts}. The math framework incorporated informal numeracy skills identified by \citet{purpura2013informal} and aligned them with the four proficiency levels proposed by \citet{engel2013teaching}. The literacy framework was based on the PELI assessment benchmarks \cite{kaminski2014preschool}.

\subsection{Interface Design}
We developed a web-based interface using a React front-end\footnote{\url{https://reactnative.dev}} and a FAST API back-end,\footnote{\url{https://fastapi.tiangolo.com}} supported by a MongoDB database.\footnote{\url{https://www.mongodb.com}} The interface includes three key components: (1) \textit{LLM content generation}: parents can request LLM to generate a new set of learning content (\textit{e.g.,} question-answer pairs); (2) \textit{Editor interface}: parents can review and revise LLM-generated content to varying extents; and (3) \textit{Activity interface}: parents can flexibly adjust their involvement and role delegation between themselves and the robot. 

\subsubsection{LLM Content Generation}
Parents can choose a book from the digital library in the system and generate a new set of learning content using the LLM embedded within the platform. Before generating content, parents are prompted to select \textit{grade level} (\textit{e.g.,} preschool) and \textit{subject} (\textit{e.g.,} math or literacy). Once the content is generated, parents can either \textit{edit and review} the material or \textit{launch} it directly. For each page of the book, the LLM creates content following a structured approach, including three main components: (1) \textit{question and multiple choices}: a question related to a selected concept from the educational framework, with four answer options, one being correct; (2) \textit{explanation}: a description of how the correct answer is derived; (3) \textit{motivation}: encouraging words or a fun fact related to the content of the book page.


\subsubsection{Editor Interface}
After the LLM generated the content, if the user choose to \textit{edit and review}, the editor interface will display the content page by page. The main \textit{book content} is shown in the left panel, while the \textit{LLM-generated content} appears on the right. The \textit{concept}, initially selected by the LLM from the chosen educational framework, is displayed at the top of the book content. For the \textit{book content}, parents could navigate through the book and review it page by page. The \textit{LLM-generated content} was organized into its main components: \textit{questions and choices}, \textit{explanations}, and \textit{motivational content}. Parents had the option to either regenerate individual components or all components at once using the LLM, or manually edit the content. Additionally, the \textit{concept} was presented as a clickable drop-down list of all available concepts from the chosen framework. Hovering over the information icon provided details about each concept. If parents wanted to change the concept, they could select a new one from the list, triggering the LLM to automatically regenerate content based on the new concept (Figure \ref{fig:editor}).


\subsubsection{Activity Interface}
When an activity was launched, regardless of whether the parent had edited it, the \textit{book content} was presented on the right panel, with the \textit{LLM-generated content} and the \textit{modes-switching and role-delegation mechanisms} shown on the left. For each page, the components of the \textit{LLM-generated content} were organized into colored blocks, showing the \textit{book text}, \textit{questions and choices}, \textit{explanation}, and \textit{motivational feedback} in sequence. The \textit{mode-switching} mechanism enabled parents to control their involvement on a macro level by adjusting the overall task distribution for all components based on preset configurations. This mechanism used a driving metaphor, with positions for a driver, co-pilot, and exit. The parent was represented by a blue icon, and the robot by a green icon. By dragging the icons to the appropriate positions, parents could select their preferred mode for each task. The \textit{role-delegation} mechanism provided more detailed control, allowing parents to assign individual components of the activity to either themselves or the robot (Figure \ref{fig:activity}).

The system defined four modes: (1) \textit{parent-takeover} mode: the parent facilitates all content components; (2) \textit{parent-led} mode: the parent leads the activity, with the robot helping with specific tasks; (3) \textit{robot-led} mode: the robot leads, with the parent helping as needed; (4) \textit{robot-takeover} mode: the robot facilitates all components. The \textit{mode-switching} mechanism automatically adjusted the task roles based on the selected mode, assigning all roles to the parent in \textit{parent-takeover} and \textit{parent-led} modes, and all to the robot in \textit{robot-takeover} and \textit{robot-led} modes. In \textit{parent-led} and \textit{robot-led} modes, parents could further adjust the role distribution by delegating individual components using the \textit{role delegation} mechanism. Figure~\ref{fig:mode} details how icon positions correspond to specific modes and provides examples of a role delegation pattern for each mode.

\begin{figure*}[t!]
  \includegraphics[width=\textwidth]{figures/figure-mode-hho.pdf}
   \vspace{-6pt}
  \caption{Mode-Switching and Role Delegation Mechanisms of the PAiREd system. The system offers four modes: robot takeover, robot-led, parent-led, and parent takeover. Parents can drag their icon to the ``driver,'' ``co-driver,'' or ``exit'' position to select the desired mode. Additionally, they can fine-tune the role delegation by assigning specific tasks to either themselves or the robot.}
  \label{fig:mode}
   \vspace{-6pt}
\end{figure*}


   \vspace*{-3pt}
\subsection{Robot Interaction Design}
We used a Misty II social robot,\footnote{Misty Robot: \url{https://www.mistyrobotics.com/products/misty-ii/}} which is a semi-humanoid robot with a four-inch LCD display for its face where customizable facial expressions can be displayed. Misty robot was connected to our interface system through the Internet and the RestAPI.\footnote{\url{https://github.com/MistyCommunity/REST-API}} During the activities, the robot autonomously facilitated a component according to the role delegation, remaining silent when the parent was responsible for a component. The interface also displayed the robot's status and reminded parents when it was their turn to engage. For each component assigned to the robot, it verbalized the content using audio generated by Google's Cloud Text-to-Speech engine.\footnote{Google Cloud Text-to-Speech: \url{https://cloud.google.com/text-to-speech/}} Additionally, the system leveraged LLM to select an appropriate robot expressions per content component from a predefined database to enhance interaction through expressive gestures, which had been used in previous studies with children \cite{white2021designing}. The robot's front bumpers provided interactive controls, allowing young children to navigate the activity more easily. Pressing the left-front bumper \textit{repeated} the ongoing component, while pressing the right-front bumper \textit{proceeded} to the next component. When progressing to the next component, the interface updated accordingly by closing the current component and expanding the next one (Figure \ref{fig:robot}).






\begin{table}[t!]
    \caption{Participant Demographics}
    \label{tab:demographics}
    \centering
\fontsize{6.3}{8.3}\selectfont
    \begin{tabular}{p{0.02\linewidth}p{0.06\linewidth}p{0.09\linewidth}p{0.09\linewidth}p{0.09\linewidth}p{0.13\linewidth}p{0.23\linewidth}}%
        \toprule
         \textbf{ID} & \textbf{Child Age} & \textbf{Child Gender} & \textbf{Parent Gender} & \textbf{Ethnicity} & \textbf{Maternal Education} & \textbf{Household Income} \\
        \midrule
         1 & 3Y8M & Male & Female & Biracial & Master's & \$200,000 and more \\
         \midrule
         2 & 3Y4M & Female & Male & Biracial & Master's & \$150,000-\$199,999\\
         \midrule
         3 & 3Y2M & Female & Female & White & Master's & \$100,000-\$149,999\\
         \midrule
         4 & 4Y6M & Male & Female & White & Ph.D. & \$100,000-\$149,999\\
         \midrule
         5 & 4Y1M & Female & Female & White & Master's & \$100,000-\$149,999\\
         \midrule
         6 & 3Y3M & Male & Female & Black & Master's & \$35,000 -\$49,999\\
         \midrule
         7 & 3Y5M & Female & Female & White & Ph.D. & \$100,000-\$149,999\\
         \midrule
         8 & 4Y2M & Male & Female & White & Bachelor's & \$100,000-\$149,999\\
         \midrule
         9 & 3Y2M & Female & Female & Biracial & Ph.D. & \$50,000-\$74,999\\
         \midrule
         10 & 5Y0M & Male & Male & White & Bachelor's & \$100,000-\$149,999\\
         \midrule
         11 & 5Y1M & Male & Female & White & Ph.D. & \$200,000 and more\\
         \midrule
         12 & 4Y10M & Female & Female & White & Master's & \$100,000-\$149,999\\
         \midrule
         13 & 3Y9M & Male & Male & White & Master's & \$100,000-\$149,999\\
         \midrule
         14 & 4Y0M & Female & Female & Asian & Master's & \$50,000-\$74,999\\
         \midrule
         15 & 4Y6M & Female & Female & Biracial & Ph.D. & \$50,000-\$74,999\\
         \midrule
         16 & 4Y10M & Female & Female & White & Master's & \$150,000-\$199,999\\
         \midrule
         17 & 4Y7M & Female & Female & White & Master's & \$100,000-\$149,999\\
         \midrule
         18 & 4Y2M & Female & Female & White & Ph.D. & \$200,000 and more\\
         \midrule
         19 & 4Y9M & Female & Female & Hispanic & Bachelor's & \$200,000 and more\\
         \midrule
         20 & 3Y2M & Female & Female & White & Ph.D. & \$200,000 and more\\
         
         \bottomrule
    \end{tabular}
\end{table}

\begin{figure*}[b!]
  \includegraphics[width=\textwidth]{figures/figure-robot-hho.pdf}
   \vspace{-6pt}
  \caption{Robot Interaction Module of the PAiREd System and Study Setup. The robot facilitates the activity based on its assigned role, providing behavioral expressions to engage the child. The child can interact with the robot by pressing its bumper to navigate through the activity.}
  \label{fig:robot}
   \vspace{-6pt}
\end{figure*}

\section{User Study} \label{sec-study}

\subsection{Participants} 
Following protocols fully approved by the responsible Institutional Review Board (IRB), we recruited families from U.S. Midwest cities via email distributed to university employee mailing lists. We selected participants based on the following criteria: (1) one parent and one child participated together; (2) the child was aged 3--5 years; and (3) both parent and child could communicate in English. This age range was chosen as it represents a critical stage for parental involvement in early education \cite{purpura2013informal}, prior to formal schooling. Families received \$50 USD upon completing the study.

Our analysis includes data from 20 parent-child dyads with children aged 3–5 years (13 female, 7 male; $M = 4.1$, $SD = 0.67$ years). Each session involved one parent and one child, although other family members were sometimes present but did not participate. Participant demographics are summarized in Table \ref{tab:demographics}.
Most participating parents were mothers (17 mothers, 3 fathers), reflecting the greater likelihood of mothers serving as primary caregivers, consistent with previous research \aptLtoX[graphic=no,type=html]{\cite[{e.g.,}][]{schoppe2013comparisons}}{\cite[\textit{e.g.,}][]{schoppe2013comparisons}}. Despite efforts to recruit a diverse sample, the sociodemographic representation is limited: 85\% of mothers held at least a Master's degree, and 80\% of families reported an annual income of \$100,000 or more. This limits the generalizability of our findings and may omit insights into underrepresented groups, a limitation further discussed in Section~\ref{sec-7.4}.



\begin{figure*}[b!]
  \includegraphics[width=\textwidth]{figures/figure-procedure-hho.pdf}
   \vspace{-6pt}
  \caption{Three-phase study procedure. Phase 1 (40 min) and Phase 2 (40 min) only involved the parent, and Phase 3 (60 min) involved both the parent and the child. }
  \label{fig:procedure}
   \vspace{-6pt}
\end{figure*}

\subsection{Study Design} 
We conducted in-home visits (2.5 hours per visit) with families. We used \texttt{SET} (Section~\ref{sec-card} and Figure~\ref{fig:card-kit}) to foster discussion with parents about their real-life scenarios in which they are able or unable to facilitate learning activities for their child. In addition, we used \texttt{PAiREd}, an AI-assisted robot as a \textit{technology probe} \cite{hutchinson2003technology} to facilitate discussions on how parents may, under different \texttt{SET} scenarios, prefer to supervise the AI-generated content for their child and adjust their participation in learning activities.

\subsubsection{Study Materials and Setup}
The study materials included (1) a Misty II social robot; (2) a Microsoft Surface laptop to present user interface; (3) the \texttt{SET} card-based activity kit; (4) recording devices (\textit{i.e.,} a video camera, a webcam, and an audio recorder) positioned to capture participants' behavior and conversations (Figure~\ref{fig:robot}). 

\subsubsection{Study Procedure} \label{sec-procedure} 
Before the study, the experimenter provided families with an overview of the procedure and obtained informed consent: the parent signed a consent form, and the child gave verbal assent. The study was conducted in three phases: Phase 1 and Phase 2 (40 minutes each) involved \textit{only the parent} and Phase 3 (60 minutes) included \textit{both the parent and the child}. To minimize interruptions during the first two phases, the research team offered childcare assistance upon request (Figure \ref{fig:procedure}).

% phase 1
In Phase 1, the experimenter began by asking the parent to describe their involvement in their child's learning activities at home (\textit{e.g.,} ``\textit{What learning activities does your child typically do at home? Which ones involve you?}''). Next, the experimenter introduced the \texttt{SET} activity kit (Figure~\ref{fig:card-kit}, Section~\ref{sec-card}) and explained the factors and their dimensions with concrete examples (Table~\ref{tab:kit-example}), arranging the \texttt{SET-banners} into a hierarchical diagram for visualization (Figure~\ref{fig:card-kit}). The parent was then asked to write 2–3 real-life examples for each dimension of every factor on sticky notes and place them on the \texttt{SET-experience banks} (Figure~\ref{fig:card-kit}). Lastly, the experimenter introduced the \texttt{SET scenario cards}, explaining how the hierarchical diagram leads to eight scenarios by combining three two-level factors (Figure~\ref{fig:card-kit}). The parent was asked to write one real-life example for each scenario card for later use in Phase 3.

In Phase 2, the parent was first asked about their thoughts on AI-generated content (\textit{e.g.,} ``What are your thoughts on AI-generated content?''). The experimenter then introduced the \textit{editor interface} (Figure~\ref{fig:editor}), explaining its features for content modification. The parent reviewed and edited LLM-generated content for two books, one on math and one on literacy, while using the \textit{think-aloud} method to vocalize their thought process. Afterward, the experimenter discussed the parent's experience and used the \texttt{SET-scenario cards} to explore how they might use the interface in different scenarios.

In Phase 3, the parent first familiarized themselves with the robot interaction, learning to use the \textit{mode-switching} and \textit{role-delegation} mechanisms. The experimenter interviewed them about their perceptions of these mechanisms and their application in situations specified in \texttt{SET-scenario cards}. Next, the parent and child read two books together, incorporating activities created by the parent in Phase 2. At the start of each book, parents were asked to situate themselves in one of the \texttt{SET-scenario cards}, adjusting the mode and delegating jobs according to what they think they might do in that scenario. The experimenter gave the parent a different scenario card in the middle of the reading, and the parent responded to the situation, \textit{i.e.,} change to a different mode, accordingly. In addition, they also completed surveys on their perceptions of their child's math and literacy proficiency before and after the reading. Finally, they discussed their experiences collaborating with the robot and provided feedback on the designed mechanisms.

\begin{table*}
    \caption{Example of scenarios created using the {\texttt{SET-scenario cards}} by P2. P2 selected {\textit{``Teach child to build block towers''}} as the high-skill activity and {\textit{``Process feelings''}} as the low-skill activity for all examples. Each row represents one scenario card.}
\label{tab:scneario-example}
        \centering
        \renewcommand{\arraystretch}{1.2}
    \small
    \begin{tabular}{p{0.12\linewidth}p{0.12\linewidth}p{0.12\linewidth}p{0.12\linewidth}p{0.4\linewidth}}
        \toprule
        \hline
        \textbf{Scenario Card} & \textbf{Skill} & \textbf{Energy} & \textbf{Time} & \textbf{Example}\\
        \midrule
         1 & High & High & High & \textit{I am well rested and my partner is taking care of dinner.} \\
\hline
         2 & High & High & Low & \textit{Kids are being well-behaved but I have to mow the lawn.} \\
\hline
         3 & High & Low & High & \textit{I need to rest while my partner is making dinner.} \\
\hline
         4 & High & Low & Low & \textit{Kids are whining and I have to be at work.} \\
\hline
         5 & Low & High & High & \textit{Free from work and we don't have any plans.} \\
\hline
         6 & Low & High & Low & \textit{Well rested but have to do chores.} \\
\hline
         7 & Low & Low & High & \textit{Not feeling well but we have free time.} \\
\hline
         8 & Low & Low & Low & \textit{Not feeling well and have chores to do.} \\
\hline
         \bottomrule
    \end{tabular}
\end{table*}

\subsection{Data Analysis}
To anonymize participants, we assigned IDs based on the order of study completion (\textit{e.g.,} P1 refers to the first participant). For qualitative data, we conducted a reflexive \textit{Thematic Analysis (TA)} following \citet{clarke2014thematic} and \citet{mcdonald2019reliability} to identify patterns and understand parental perspectives. The first three authors, experienced in qualitative coding, transcribed and familiarized themselves with the data from audio and video recordings. Initial semantic codes were generated by the first author and organized into categories as the initial codebook. The team collaboratively coded data from P1 using the initial codebook, discussed interpretations, and created a consensus-based codebook. Using the codebook, the remaining data were independently coded by the three authors, with peer reviews and iterative discussions to finalize codes. We constructed final themes from recurring, meaningful patterns in the data. For the quantitative survey data (\textit{i.e.,} parent perception on child's literacy and math abilities), we first conducted a One-Way ANOVA to evaluate overall significance across difficulty levels for each subject (\textit{i.e.,} math and literacy), revealing significant differences between levels. We then performed Repeated Measures ANOVAs for each level, followed by post-hoc pairwise comparisons using non-parametric (Wilcoxon Signed-Rank) tests.
\section{Results}
\label{sec:results}
% 
Figures~\ref{fig:LLM}-\ref{fig:DC} present results from applying counterfactual cross-validation (Algorithm~\ref{alg:C-CV}) across six benchmark scenarios detailed in \S\ref{sec:Benchmark_Toolbox}. Below, we outline our implementation approach and key findings.

For the LLM-based social network model, we conducted 10 distinct runs, constrained by OpenAI API limitations; the current simulation, with $N=1000$ and $T=30$, required approximately 100,000 GPT-3.5 API calls to generate experimental and ground truth results. Each run employs a unique treatment allocation following a staggered rollout design across three stages with $\Vec{\expr} = (0.2, 0.5, 0.8)$, each spanning 10 periods. This design implies that on average 20\% of units received the intervention in the initial 10 periods, followed by an additional 30\% in the subsequent 10 periods, and so forth.

For the remaining five experiments, we conducted 100 independent runs for each setting, utilizing a fresh treatment allocation for each run through a staggered rollout design. The design comprises four equal-length stages with treatment probabilities $\Vec{\expr} = (0, 0.2, 0.4, 0.6)$. In each figure's leftmost panel, we display the temporal evolution of outcomes through their mean and standard deviation, along with the 95th percentile across runs.

\begin{figure}
    \centering
    \includegraphics[width=1\linewidth]{plots/LLM.pdf}
    \caption{LLM-based social network with $N=1,000$ agents.}
    \label{fig:LLM}
\end{figure}

\begin{figure}
    \centering
    \includegraphics[width=1\linewidth]{plots/BAM1.pdf}
    \caption{Belief adoption model with Krupina network with $N=3,366$ users.}
    \label{fig:BAM1}
\end{figure}

\begin{figure}
    \centering
    \includegraphics[width=1\linewidth]{plots/BAM2.pdf}
    \caption{Belief adoption model with Topolcany network with $N=18,246$ users.}
    \label{fig:BAM2}
\end{figure}

\begin{figure}
    \centering
    \includegraphics[width=1\linewidth]{plots/BAM3.pdf}
    \caption{Belief adoption model with Zilina network with $N=42,971$ users.}
    \label{fig:BAM3}
\end{figure}

\begin{figure}
    \centering
    \includegraphics[width=1\linewidth]{plots/Auction.pdf}
    \caption{Ascending auction model with $N=500$ objects.}
    \label{fig:auction}
\end{figure}

The second panels of Figures~\ref{fig:LLM}-\ref{fig:DC} display the box plot of the average total treatment effect (TTE) across multiple time periods. The TTE contrasts the counterfactual of all units under treatment against all units under control:
% 
\begin{align}
    \label{eq:TTE}
    \text{TTE} :=
    \frac{1}{LN} \sum_{t=T-L+1}^T \sum_{i=1}^\UN 
    \left[
    \outcomeDW{\mathbf{1}}{i}{t}
    -
    \outcomeDW{\mathbf{0}}{i}{t}
    \right].
\end{align}
% 
In each setting, we carefully select $L$ so that the TTE in \eqref{eq:TTE} covers all periods with nonzero treatment probability, ensuring our benchmark estimators remain meaningful. The results compare ground truth (GT) values against estimates obtained from three methods: our proposed causal message-passing approach (CMP), the difference-in-means estimator (DM), and the Horvitz-Thompson estimator (HT)%
% 
\footnote{Difference-in-means (DM) and Horvitz-Thompson (HT) are expressed as:
\begin{align*}
%
\DIME :=  \frac{1}{L} \sum_{t=T-L+1}^T 
\Big(
\frac{\sum_{i=1}^N\outcomeD{}{i}{t}\treatment{i}{t}}{\sum_{i=1}^N\treatment{i}{t}} - \frac{\sum_{i=1}^N\outcomeD{}{i}{t}(1-\treatment{i}{t})}{\sum_{i=1}^N(1-\treatment{i}{t})} \Big),
%
\quad\quad
% 
\HTE :=  \frac{1}{LN} \sum_{t=T-L+1}^T \sum_{i=1}^N \left( \frac{\outcomeD{}{i}{t} \treatment{i}{t}}{\E[\treatment{i}{t}]} - \frac{\outcomeD{}{i}{t} (1 - \treatment{i}{t})}{\E[1 - \treatment{i}{t}]} \right).
%
\end{align*}}
%
\citep{savje2021average}. Finally, the rightmost two panels in each figure display the CFE under the ground truth and CMP estimates for all-control and all-treatment conditions, along with their respective 95th percentiles.

In implementing Algorithm~\ref{alg:C-CV}, we employ five validation batches ($b_v=5$). 
To select candidate estimators, we begin with a base model where each outcome is expressed as a linear function of two components: the sample mean of outcomes from the previous round and the current treatment allocation means. We then systematically modify this model by incorporating additional first-order and higher-order terms. The configurations also span batch counts from 200 to 2000 and batch sizes ranging from 0.1 to 20 percent of the population size. To estimate parameters, we employ Ridge regression with penalty parameters logarithmically spaced from $10^{-4}$ to $10^{4}$. These parameters are comprehensively combined to generate a diverse set of potential estimators, with time blocks aligned to experimental stages. For example, when $T=40$ and the design $\Vec{\expr} = (0, 0.2, 0.4, 0.6)$ with equal length blocks is used, 
$\tblockList$ has four elements, one corresponding to each block with a fixed treatment probability.
Then, the selection process incorporates both domain knowledge and observed data characteristics. For instance, the pronounced temporal patterns evident in the left panels of Figures~\ref{fig:NYC_taxi}-\ref{fig:DC}, observed in the New York City taxi model, exercise encouragement program, and data center model, necessitate estimators with detrending steps (see Remark~\ref{rem:proprocessing}). Computational efficiency is maintained by constraining the estimator search space based on the experimental context.

Overall, our framework demonstrates robust performance across all six scenarios, successfully estimating counterfactual evolutions despite strong seasonality patterns and without requiring information about the underlying interference network. The proposed method achieves significantly better performance than both DM and HT estimators, even in settings with subtle treatment effects. As illustrated in Figures~\ref{fig:LLM}-\ref{fig:NYC_taxi}, CMP yields estimates with both smaller bias and variance in different scenarios. The effectiveness of our method is particularly evident in the challenging scenarios presented in Figures~\ref{fig:BAM1}-\ref{fig:auction} and \ref{fig:DC}, where conventional estimators struggle to reliably determine even the direction of treatment effects. These comprehensive results establish our framework's capability to deliver precise estimates of counterfactual evolutions and treatment effects across diverse experimental settings.

\begin{figure}
    \centering
    \includegraphics[width=1\linewidth]{plots/NYC_taxi.pdf}
    \caption{New York City Taxi model with $N=18,768$ Routes.}
    \label{fig:NYC_taxi}
\end{figure}

\begin{figure}
    \centering
    \includegraphics[width=1\linewidth]{plots/EEP.pdf}
    \caption{Exercise encouragement program with $N=30,162$ users.}
    \label{fig:EP}
\end{figure}

\begin{figure}
    \centering
    \includegraphics[width=1\linewidth]{plots/DC_N1k.pdf}
    \caption{Data Center model with $N=1,000$ servers.}
    \label{fig:DC}
\end{figure}

\begin{remark}
    \label{rem:batch_generation}
    % 
    Selecting a predetermined number of batches for a given batch size $n^\batch$ presents a significant computational challenge, particularly in large-scale problems with time-varying treatment allocations across units. For staggered rollout designs, we implement a heuristic approach while deferring comprehensive analysis to future research. Our heuristic consists of three steps. First, we order units by their treatment duration, defined as the number of time periods under treatment. Second, we select two blocks of size $n^\batch$—one that slides through the ordered list to cover all treatment durations, and another chosen randomly to ensure sufficient between-batch variation. Third, we select individual units from these merged blocks with equal probability to generate batches with average size $n^\batch$. This procedure maintains computational efficiency while ensuring batches with diverse treatment allocations with high probability.
\end{remark}
\section{DISCUSSION}  

% In this work, we compared behavioural responses to robot failures, examining how individuals reacted to these failures and their subsequent perceptions of the robot. The failures in our experiment varied in type, timing, and the robot's acknowledgement of the failure. Our findings indicate differences in both user gaze and perception of the robot, specifically in feelings of anxiety, and perceptions of the robot being skilled and sensible. Based on these results, we will discuss the impact of two failure types, two timings, and two levels of the robot's acknowledgement of the failure on user gaze and perception.

This study compared behavioural responses to robot failures, focusing on how individuals reacted and perceived the robot. Failures varied by type, timing, and acknowledgement. The findings revealed that robot failures affect user gaze and perceptions. These findings are discussed further in the following section.

\subsection{Behavioural Response}

To address the first research question, we analysed user gaze behaviour in multiple ways: the number of gaze shifts, gaze distribution during puzzle-solving, and gaze entropy based on transition matrices. These measures allowed us to examine how the type and timing of failures, as well as whether the robot acknowledged its failure, influenced user gaze patterns and whether gaze behaviour varied across different failure scenarios. Our results showed that user gaze is a reliable indicator of robot failures. When the robot made a failure, participants exhibited more frequent gaze shifts between different AoIs, likely due to confusion and an attempt to understand what was happening. This finding is similar to the results of Kontogiorgos et al. \cite{kontogiorgos_embodiment_2020}, who found that people tend to gaze more at the robot when it makes a mistake. The literature suggests that different types of failures influence user perceptions of the robot \cite{morales_interaction_2019}, and our findings support this by showing that users exhibit distinct gaze behaviours in response to various failure types. For example, when the failure was executional, the number of gaze shifts towards the robot was significantly higher compared to when the failure was decisional. Moreover, during executional failures, the proportion of time spent looking at the robot was much higher compared to decisional failures. It is crucial for the robot to recognize the type of failure it has made so that it can determine the appropriate strategy for recovery and regain the user's trust.


The timing of the failure is also crucial for the robot, as it requires different approaches for recovery and repair. In our research, while the timing of the failure—whether at the start or end of the interaction—did not significantly affect gaze shifts, it did influence gaze transition matrices, and gaze distribution across AoIs. Failures at the beginning of the interaction led to higher median gaze transition values, indicating more randomness early on. Additionally, participants' focus on the Tangram figure was more when the failure occurred at the beginning of the interaction compared to later ones, while their focus on the robot's body or end effector was more during late failures than early ones. 


% \subsubsection{Failure Acknowledgement}

In our research, after committing a failure, the robot could either acknowledge the failure and then continue its action, or proceed without acknowledgement. We could not find significant differences in users' gaze behaviour when the robot acknowledged its failure and when it did not. As the literature suggests \cite{esterwood_you_2021, karli_what_2023, wachowiak_when_2024}, there are other verbal approaches to failure recovery, such as promises and technical explanations, which might influence users' gaze differently. Verbal failure recovery is important for robots, as it demonstrates an awareness of mistakes. This, in turn, can make the robot appear more intelligent and encourage users to engage with it more.


% \subsubsection{Anticipatory Gaze Behavior}
Our study also explored changes in users' anticipatory gaze behaviour during the task and its potential role in assisting the robot to recover from failures. Participants frequently anticipated the placement of the object before the robot executed the action, even when the robot made an error. This anticipatory gaze behaviour could serve as a valuable cue for the robot to detect its failures and initiate appropriate recovery strategies. However, we observed a decrease in participants' anticipatory gaze behaviour as the number of tasks increased. This decline may indicate reduced engagement over time, with participants being more actively collaborative at the beginning of the interaction. It also suggests that users' gaze behaviour might change throughout the interaction. These findings highlight the dynamic nature of gaze behaviour throughout the interaction.


\subsection{Subjective Measures}

To address the second research question, we examined user perceptions of the robot in three areas: perceived intelligence, sense of safety, and trust during failures. The analysis revealed how these measures varied with the type and timing of failure and whether the robot acknowledged its mistake.

The results of the subjective evaluation revealed that users' perceptions of the robot's intelligence and safety were not significantly influenced by the type of failure. However, users exhibited higher levels of trust in the robot during executional failures compared to decisional failures, suggesting that placing an object in an incorrect location reduces trust more than making an incorrect decision. Additionally, we observed interesting findings regarding the timing of the robot's failures. When failures occurred early in the interaction, users rated the robot as more intelligent and trustworthy compared to failures that occurred later. For the measure of "Sensible," this difference was statistically significant. These findings are consistent with previous research by Morales et al. \cite{morales_interaction_2019} and Lucas et al. \cite{lucas_getting_2018}. Interestingly, users reported feeling more relaxed when failures occurred later in the interaction, aligning with results from Desai et al. \cite{desai_impact_2013} and Rossi et al. \cite{rossi_how_2017}.

When the robot acknowledged its failures, users perceived it as slightly more intelligent and trustworthy but also experienced increased anxiety. This finding may be explained by the robot’s consistent physical repair actions a few seconds after each failure. When the robot did not explicitly acknowledge its failures, users might not have interpreted these actions as errors, reducing their perception of failure events.

\subsection{Limitations and Future Work}
 There were instances where participants were preoccupied with determining the placement of their next piece, which occasionally led them to overlook the robot's movements. However, these occurrences were minimal. Another limitation is the restriction to only two types of failure and whether the robot acknowledges its failure or not. The effect size in our study was medium; however, to obtain more robust results, a larger sample size would be beneficial.
 %Additionally, focusing solely on participants' gaze behaviour may not provide a comprehensive measure of failure detection. Incorporating other non-verbal cues, such as gestures or facial expressions, alongside gaze, could improve the accuracy of failure detection. 
 Furthermore, for safety reasons, the robot's arm movement was slowed and the experimenter was in the room, which may have influenced participants' perceptions. 
 Future research could address these limitations by exploring a broader range of failure types and incorporating explanatory feedback from the robot.




\subsection{Design Implications} %\label{sec-5.2}
We proposed five design implications based on the discussion of the findings to inform future design for AI-assisted robot.

\subsubsection{Design Implication \#1: } \textit{Tailoring the Quality of LLM Models with Minimal Parental Feedback}.  
When parents have limited time or energy, they often provide minimal content review and may avoid using the system if they lack trust in the model or cannot supervise. This suggests that LLM-generated content must be trustworthy enough for parents to use with little oversight. Beyond employing advanced LLMs, designers should incorporate features that encourage parents to supervise the content considering their availability, \textit{e.g.,} summarization versus detailed view of all content. Additionally, integrating evaluations for factors like age appropriateness can increase parents' trust and reduce the likelihood of disuse.

\subsubsection{Design Implication \#2:} \textit{Explicit Design for Varying Levels of Control over LLM-Generated Content}.  
Parents engage with LLM-generated content differently depending on their circumstances—ranging from using it as is, reusing previously reviewed content, quickly skimming, or extensively customizing. This variation calls for a more explicit design of control options within the system. Instead of presenting all features (\textit{e.g.,} regenerate, edit) simultaneously, the interface should adapt to users' needs. Those skimming for a quick review could access a concise summary, while parents interested in customization might see detailed content and easy-to-use editing and regeneration tools.

\subsubsection{Design Implication \#3: } \textit{Mode-Switching as a Shift in User Center, Not Just Task Delegation}.
Currently, the mode-switching feature focuses on dividing tasks between the parent and the robot, but it should also reflect changing interaction dynamics between parent and child. For example, when switching from parent-led to parent-takeover mode, a parent may want to access LLM-generated content without the robot's presence. Yet, the current design presents content uniformly across modes. Similarly, shifting from robot-led to robot-takeover mode implies independent child use, but the system currently remains parent-centered. To address this, mode-switching should not only redistribute tasks but also adjust content presentation and interaction methods. In parent-takeover mode, content could be simplified for easier navigation, and in robot-takeover mode, interaction could shift from touch-based to voice-based, fostering a child-friendly environment without parental intervention.

\subsubsection{Design Implication \#4: } \textit{Maximizing Parent-Child Interaction Value through Appropriate Difficulty Levels}.
Mode-switching should also consider optimizing educational impact for different interaction dynamics by adjusting content difficulty. Parents generally excel at reading, explaining, and encouraging their children, while the robot's role is to supplement limited time and energy and offer alternative ways to engage children. When parents are involved, slightly more challenging content can increase the educational value of their engagement. Conversely, in child-robot interactions, the system should present content that children can handle independently, avoiding frustration or loss of confidence.

\subsubsection{Design Implication \#5: } \textit{Integrating Parenting Education into the System}.
Parents view the system not only as a tool for engaging their children, but also for improving their own parenting skills, \textit{e.g.,} effective questioning technique and pedagogical strategies. While familiar with preschool-level concepts, they may struggle to generate questions spontaneously or accurately gauge their child's abilities. To address this, the system should incorporate parenting education features. It could, for example, guide parents in developing more intuitive questioning techniques and provide tools to track and benchmark their child's progress. By doing so, the system empowers parents to focus their efforts and better support their children's learning.

\section{Limitations \& Future Work}\label{sec-7.4}

A notable limitation of our study is the lack of sociodemographic diversity in our participant sample. Specifically, (1) \textit{location:} all participants were based in the U.S. Midwest; (2) \textit{gender:} the 20 parents who participated in the study included only three fathers; (3) \textit{income:} the majority of families reported middle to high household incomes; and (4) \textit{education:} most parents had a Master's degree or higher. These factors may limit the broader applicability of our findings. Consistent with previous education studies, a predominance of mothers were included due to self-selection bias \cite{schoppe2013comparisons, mcbride1993comparison}, and we faced difficulties recruiting lower-income families \cite{nicholson2011recruitment}. Children from higher-income and more highly educated families tend to perform better academically \cite{sirin2005socioeconomic}. These may have influenced the types of support families wanted from the robot, which need to be addressed in future research. we aim to collaborate with local community centers, libraries and schools to reach a more diverse population in our future work.

In addition, while cost remains a common accessibility limitation, some educational robots, such as Miko,\footnote{Miko Robot: \url{https://shorturl.at/XSyM5}} are now priced similarly to smartphones, improving feasibility for families. Future work should design novel interaction paradigms for AI-assisted educational robots in public spaces (\textit{e.g.,} libraries, schools, museums) to broaden accessibility. Moreover, our brief home visits captured immediate reactions rather than long-term changes in parent-child interactions, trust in AI/robots, or learning outcomes. Extended research are needed to observe how these factors evolve as AI-assisted robots integrate into daily life.

Additionally, we acknowledge the limitation of simplifying parental involvement factors into binary levels, resulting in eight scenarios. This decision was practical, as more granular factors (\textit{e.g.,} three levels each; 27 scenarios) would yield in a large number of scenarios not manageable for human participants. Future work should explore methods for representing a continuous spectrum of factors to capture more accurate scenarios. Furthermore, our system currently targets literacy and math. Beyond these academic subjects, other domains like social-emotional skills and creativity warrant exploration. Investigating parent-AI-robot collaboration in these areas could yield broader insights. Finally, ethical concerns motivated our work (Section~\ref{sec-rw-2.2}), but were not our main focus. We presented some ethical issues in Section~\ref{sec-result-2} and discussed in Section~\ref{sec-dis-2}, but future research should delve more deeply into these topics. A comprehensive understanding of ethical considerations, from data sourcing to model training, will help ensure that AI-generated content meets appropriate standards for children.


\vspace{-0.7em}

\section{Conclusions}
In this paper, we propose a new distribution-aware divergence-based metric, DistFaiR, for amortized fairness measurement. We identify metrics under DistFaiR with the useful property that group unfairness is upper bounded by individual unfairness. We show that we can reduce individual and group unfairness under DistFaiR for different choices of divergence measures. We emphasize query polarity as a crucial yet overlooked aspect in fair-ranking literature, noting that neglecting polarity can result in fairwashing. We also empirically demonstrate fairwashing effects due to a lack of query polarity consideration and propose/evaluate a method to mitigate this effect. 

Our work has some limitations. For example, we assume a position bias model of attention. However, we note that this assumption can be relaxed to consider more complex user attention patterns under our framework, with some modifications made to the cumulative attention formulation. We also make normative assumptions that the distribution of attention should be close to that of relevance. However, a different link function may be more appropriate~\cite{saito2022fair}. Additionally, scores allotted to minority groups may be under-estimates of their true value~\cite{pierson2021algorithmic,krieg2022perceived} and may need to be pre-processed ~\cite{liao2023social}.  Importantly, there may not be purely technical fixes for operationalizing real-world fair ranking~\cite{gichoya2021equity}. Our approach, we believe, is a step towards reducing the scale of such issues.








%%
%% The acknowledgments section is defined using the "acks" environment
%% (and NOT an unnumbered section). This ensures the proper
%% identification of the section in the article metadata, and the
%% consistent spelling of the heading.
\begin{acks}
This work was supported by the National Science Foundation awards 2202803. We thank Nathan White for his guidance with the technical development of our robot system. We also thank Amy Koike for her help in creating Figures 4 and 5.
\end{acks}

%%
%% The next two lines define the bibliography style to be used, and
%% the bibliography file.
\balance
\bibliographystyle{ACM-Reference-Format}
\bibliography{Bibliography}

%%
%% If your work has an appendix, this is the place to put it.
%\appendix
%\section{Research Methods}
%\subsection{Part One}



\end{document}
\endinput
%%
%% End of file `sample-authordraft.tex'.
