\subsection{Design Implications} %\label{sec-5.2}
We proposed five design implications based on the discussion of the findings to inform future design for AI-assisted robot.

\subsubsection{Design Implication \#1: } \textit{Tailoring the Quality of LLM Models with Minimal Parental Feedback}.  
When parents have limited time or energy, they often provide minimal content review and may avoid using the system if they lack trust in the model or cannot supervise. This suggests that LLM-generated content must be trustworthy enough for parents to use with little oversight. Beyond employing advanced LLMs, designers should incorporate features that encourage parents to supervise the content considering their availability, \textit{e.g.,} summarization versus detailed view of all content. Additionally, integrating evaluations for factors like age appropriateness can increase parents' trust and reduce the likelihood of disuse.

\subsubsection{Design Implication \#2:} \textit{Explicit Design for Varying Levels of Control over LLM-Generated Content}.  
Parents engage with LLM-generated content differently depending on their circumstances—ranging from using it as is, reusing previously reviewed content, quickly skimming, or extensively customizing. This variation calls for a more explicit design of control options within the system. Instead of presenting all features (\textit{e.g.,} regenerate, edit) simultaneously, the interface should adapt to users' needs. Those skimming for a quick review could access a concise summary, while parents interested in customization might see detailed content and easy-to-use editing and regeneration tools.

\subsubsection{Design Implication \#3: } \textit{Mode-Switching as a Shift in User Center, Not Just Task Delegation}.
Currently, the mode-switching feature focuses on dividing tasks between the parent and the robot, but it should also reflect changing interaction dynamics between parent and child. For example, when switching from parent-led to parent-takeover mode, a parent may want to access LLM-generated content without the robot's presence. Yet, the current design presents content uniformly across modes. Similarly, shifting from robot-led to robot-takeover mode implies independent child use, but the system currently remains parent-centered. To address this, mode-switching should not only redistribute tasks but also adjust content presentation and interaction methods. In parent-takeover mode, content could be simplified for easier navigation, and in robot-takeover mode, interaction could shift from touch-based to voice-based, fostering a child-friendly environment without parental intervention.

\subsubsection{Design Implication \#4: } \textit{Maximizing Parent-Child Interaction Value through Appropriate Difficulty Levels}.
Mode-switching should also consider optimizing educational impact for different interaction dynamics by adjusting content difficulty. Parents generally excel at reading, explaining, and encouraging their children, while the robot's role is to supplement limited time and energy and offer alternative ways to engage children. When parents are involved, slightly more challenging content can increase the educational value of their engagement. Conversely, in child-robot interactions, the system should present content that children can handle independently, avoiding frustration or loss of confidence.

\subsubsection{Design Implication \#5: } \textit{Integrating Parenting Education into the System}.
Parents view the system not only as a tool for engaging their children, but also for improving their own parenting skills, \textit{e.g.,} effective questioning technique and pedagogical strategies. While familiar with preschool-level concepts, they may struggle to generate questions spontaneously or accurately gauge their child's abilities. To address this, the system should incorporate parenting education features. It could, for example, guide parents in developing more intuitive questioning techniques and provide tools to track and benchmark their child's progress. By doing so, the system empowers parents to focus their efforts and better support their children's learning.
