
\subsection{(RQ4) How would parents prefer to collaborate with an AI-assisted robot to engage in learning activities with their children under different contexts?}

\begin{figure*}[b!]
\includegraphics[width=\textwidth]{figures/figure-result-04-hho.pdf}
   \vspace{-6pt}
  \caption{Summary of parent's use of robot involvement adjustment mechanisms: (1) usage pattern, (2) parenting education.}
  \label{fig:result-03}
   \vspace{-6pt}
\end{figure*}

We identified two major themes in the use of parent-robot collaboration mechanisms (\textit{i.e.,} \textit{mode-switching} and \textit{role-delegation}) within the \textit{activity interface}: (1) \textit{Contextual mode utilization}, referring to how parents adjust their involvement based on varying time and energy levels, and (2) \textit{Perceived educational impact on parenting}, highlighting how parents value the process for enhancing their skills and knowledge in parenting.

\subsubsection{Theme 1: Contextual mode utilization}

We discussed parents' preferences when collaborating with the AI-assisted robot across four contexts: (1) sufficient energy and time, (2) sufficient time but low energy, (3) sufficient energy but limited time, and (4) low energy and time. The impact of parental skill is discussed in specific cases.

(1) \textit{Parents have sufficient energy and time}: many parents (P2, P5–7, P9, P15, P18) preferred the \textit{\textbf{parent takeover mode}}, where they facilitate activities themselves while using LLM-generated content as a resource. For example, P18 shared, ``\textit{if I'm feeling motivated, I'd probably take over, but still look at some AI-generated questions to prompt me or remind me of things to ask or do with her}.'' Similarly, P15 noted, ``\textit{with full energy and time, I would use the parent-only mode because I want to interact with her and give her all my attention}.'' Parents valued the ability to take full control while using LLM-generated content for supplemental support when they have sufficient energy and time.

In addition, some parents (P2, P8–12, P15, P20) envisioned using \textit{\textbf{collaboration modes}}--where both the parent and the robot share responsibilities (\textit{i.e.,} parent-led or robot-led mode)--with the parents' \textit{skills} in specific areas relative to the robot playing a critical role in determining the pattern of role delegation. Parents often chose to involve the robot when they felt it could enhance their child's engagement especially in high-stakes tasks like quizzes. For example, P15 noted, ``\textit{I would use the robot for quizzes as a playful element to keep her engaged}.'' Similarly, P2 highlighted the objectivity of the robot in quizzing: ``\textit{I like the idea of reading her the book and then a neutral third party gets to test her on it}.'' On the other hand, parents took on specific roles when they believed their involvement would benefit their child more. P11 shared, ``\textit{I would let the robot read and ask questions but step in if he wasn't understanding or needed guidance},'' while P12 emphasized the emotional aspect of teaching: ``\textit{I can explain in a way that she understands, whereas the robot might come across as too harsh}.''

(2) \textit{Parents have sufficient time but lack energy}: some parents (P3, P7, P10--12, P15, P18, P20) opted for \textit{\textbf{collaboration modes}}, with their involvement influenced by their motivation levels and partially by their \textit{skill} relative to the robot. For example, P10 noted, ``\textit{when I'm not motivated, having the robot do the quiz takes some heat off me}.'' Similarly, P9 mentioned, ``\textit{I'd probably read the book, but have the robot do everything else}.'' Additionally, some parents (P1, P2, P4, P6–8, P15) chose to use \textit{\textbf{robot takeover mode}}--where the robot facilitates everything--while they remained nearby to supervise. For instance, P15 noted, ``\textit{I'd be around, but I wouldn't physically do much because I'm not feeling well}.'' Similarly, P2 noted, ``\textit{If I'm not motivated, I could see myself handing it all over to the robot}.''

(3) \textit{Parents have sufficient energy but lack time}: many parents (P2, P3, P5, P7--11, P15) opted for the \textit{\textbf{robot takeover mode}}--where the robot facilitates everything--while adjusting their usage based on \textit{how much they trust LLM}. Parents with higher trust allowed their child to use the content directly without review (P3, P5, P7, P9, P10). For example, P7 mentioned, ``\textit{If I'm trying to take a walk, I might do the robot takeover, then I can physically be gone}.'' In contrast, parents with less trust preferred to supervise while multitasking (P2, P7, P8), review content beforehand using the editor (P2, P8), or use the system only if the LLM model met high-quality standards (P11, P15). For example, P2 shared, ``\textit{If I'm not there, I wouldn't want them to do it, unless I had used the editor to review},'' while P7 described a multitasking scenario: ``\textit{I could be working from home while the robot takes over, and I'm nearby to supervise}.'' Moreover, a few parents (P2, P12) chose to \textit{\textbf{avoid using the system entirely}} due to their lack of trust in using LLM-generated content directly and insufficient time to review it, or because the system design did not support independent use for young children (P4, P12, P20). For instance, P2 noted, ``\textit{If I'm absent, I don't know if I'd want them to do any of this},'' while P12 stated, ``\textit{I know my daughter is sensitive, and if [the questions are too hard and] the robot keeps telling her she's wrong, she might take it personally and give up}.''

(4) \textit{Parents lack both energy and time}: Some parents chose \textit{\textbf{robot takeover mode}}, adjusting their usage based on their trust in LLM-generated content. Others \textit{\textbf{avoided using the system entirely}} due to low trust and insufficient time to review (P2, P12), or because the system design did not support independent use by young children (P4, P12, P20). Refer to the previous case—parents with sufficient energy but lacking time—as the usage patterns and contextual reasons are very similar.

\begin{figure*}[!t]
  \includegraphics[width=\textwidth]{figures/figure-quant-hho.pdf}
  \caption{Parent perception on child's math and literacy ability before and after the reading session. The result suggested that parents adjusted their perception after observing their child doing the activity and they tend to underestimate them, especially for advanced math concepts and phonological awareness concept for literacy. The horizontal lines represents significance from the Wilcoxon Signed-Ranked Test: $p < .01^{**}$, $p < .05^{*}$.}
  \label{fig:quant-result}
   \vspace*{-10pt}
\end{figure*}

\subsubsection{Theme 2: Perceived Educational Impact on Parenting} \label{sec-6.4.2}

Supported by mixed-method data, many parents thought \texttt{PAiREd} has value in parenting education, providing them with pedagogical strategies and giving them opportunities to observe their child's proficiency level systematically through observation. If the system provides a comprehensive framework and ample ideas, parents may not decrease their involvement in an activity just because they don't have the pedagogical skill; in fact, they may even increase their involvement. In addition, parents will be able to observe and adjust their understanding about what their child can do or cannot do, instead of under- or over- estimate their child's ability.

Several parents (P1, P4, P6, P7, P10--12, P18) appreciated that the system \textit{\textbf{offered ideas they might not have considered on their own}}, providing new topics to explore with their child. For instance, P1 emphasized, ``\textit{I hadn't even thought of all the different types of concepts},'' and P10 highlighted that the system ``\textit{gives more of a structure…even the drop down list of concepts is insightful, offering lenses I wouldn't normally consider when reading}.'' Others (P2, P6--9, P11, P12) valued that the LLM \textit{\textbf{generated example questions for each concept}}, allowing them to start with ready-made content without worrying whether their own questions reflected the intended learning goals. For example, P8 mentioned, ``\textit{What's nice about the AI-generated ones is that you can specifically choose a variety of concepts, whereas creating them on your own, you don't always know what the concepts are}.'' Additionally, some parents noted that the system allowed them to systematically select questions they were unsure their child could answer, which \textit{\textbf{provided a structured way to observe and assess their child's proficiency level}}. For example, P7 remarked, ``\textit{it gives her the opportunity to show me things she knows that I otherwise wouldn't have asked},'' while P8 shared, ``\textit{I was curious to see how he does if he doesn't know how to answer this, rather than just setting him up to succeed}.''

Before and after parent-child pairs engage in the activity, we asked parents to rate their perception on their child's math and literacy abilities. Our quantitative results suggest that parents adjusted their understanding of their child's proficiency in certain concepts after reading together. Specifically, \textit{\textbf{parents tended to underestimate what their child can or cannot do, especially with more advanced math concepts}} (Math-L3: $p < .01^{**}$, Math-L4: $p < .01^{**}$) and the phonological awareness concept in literacy ($p < .05^{*}$). Figure \ref{fig:quant-result} summarizes the significance results from the Wilcoxon Signed-Rank Test.

%A One-Way ANOVA revealed significant differences between levels (L1-L4), necessitating separate analyses. Subsequent Repeated Measures ANOVAs for each level showed significant improvement in Math L4 post-intervention. Post-hoc pairwise tests, using both parametric (Paired T-Tests) and non-parametric (Wilcoxon Signed-Rank) methods to ensure robustness, confirmed significance in Literacy L2, Math L3, and Math L4. This multi-layered approach—combining One-Way ANOVA, Repeated Measures ANOVA, and diverse post-hoc tests—ensured level independence, accounted for within-subject variability, and provided a comprehensive, robust understanding of level-specific improvements while minimising misinterpretation risks.




