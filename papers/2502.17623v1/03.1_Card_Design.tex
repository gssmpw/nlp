\section{Card-Based Activity Kit Design} \label{sec-card}

\begin{table*}[!t]
    \caption{Factors and dimensions used in the Card-based Activity Kit}
    \label{tab:kit-example}
    \centering
    \small
    \begin{tabular}{p{0.12\linewidth}p{0.4\linewidth}p{0.4\linewidth}}%
        \toprule
         \textbf{Factor-level} & \textbf{Description} & \textbf{Example} \\
        \midrule
         Skill-high & Parents feels \textit{very} confident in a specific task or topic. & ``\textit{I am good at math-related activities.}'' \\
         \midrule
         Skill-low & Parents feels \textit{not} confident in a specific task or topic. & ``\textit{I am not good at coming up with good questions to ask.}'' \\
         \midrule
         Energy-high & Parents feels \textit{highly} motivated. & ``\textit{I am highly motivated when my child invites me.}'' \\
         \midrule
         Energy-low & Parents feels \textit{not} motivated. & ``\textit{I am not motivated when I am tired.}'' \\
         \midrule
         Time-high & Parents are \textit{available and present}. & ``\textit{I am available and present when my work is done.}'' \\
         \midrule
         Time-low & Parents \textit{need to be absent}. & ``\textit{I need to be absent when my younger child is crying.}'' \\     
         \bottomrule
    \end{tabular}
\end{table*}

To systematically explore parents' contextual, real-life scenarios where varying levels of parental involvement in children's learning may occur, we designed a card-based activity kit named \textbf{\texttt{SET}}, representing \textbf{\texttt{S}}kill, \textbf{\texttt{E}}nergy, and \textbf{\texttt{T}}ime. The design of the kit is grounded in frameworks from education \cite{green2007parents} and HCI \cite{ho2024s}. Specifically, we leveraged the ``\textit{Parent Perceived Life Context}'' construct from the Hoover-Dempsey and Sandler (HDS) framework for parental involvement \cite{green2007parents} and the key factors influencing parent-robot collaboration defined by \citet{ho2024s}. We detailed the theoretical foundation in Section \ref{sec-rw-2.1}.

The kit consists of three primary materials: \texttt{SET-banners}, \texttt{SET-} \texttt{experience banks}, and \texttt{SET-scenario cards}\footnote{\url{https://osf.io/zfksg/?view_only=b59bd41287f543ce82ab85950aaf004f}}. Each material is characterized by the same two-dimensional factors (\textit{i.e.,} skill (high/low), energy (high/low), and time (high/low)) but is designed in different formats to serve distinct purposes during the card activity (see Table \ref{tab:kit-example} for details about each factor and its dimensions). Although the factors theoretically exist on a continuous spectrum, we divided them into binary dimensions to ensure the scale of the activity is manageable for participants. Below, we describe the purpose of each material. For more details on how the kit was used, refer to Section \ref{sec-procedure}.

\begin{enumerate}
    \item The \texttt{SET-banners} present a hierarchical diagram that depicts how the three two-dimensional factors combine to create \textit{eight} distinct scenarios (Figure \ref{fig:card-kit}), forming the \textit{eight} \texttt{SET-scenario cards}.
    \item The \texttt{SET-experience banks} enable participants to document real-life examples for each dimension of every factor using post-it notes. This activity encourages focused reflection on individual factors and dimensions, serving as a stepping stone for constructing scenarios that integrate all three factors during the \texttt{SET-scenario cards} activity (Figure \ref{fig:card-kit}).
    \item The \texttt{SET-scenario cards} guide participants in creating \textit{eight} real-life scenarios, each defined by unique combinations of the three two-dimensional factors. The front of each card displays the dimensional status of the factors, while the back includes detailed descriptions of each factor's status and a blank space for participants to document a scenario that aligns with the corresponding dimensions (Figure \ref{fig:card-kit}).
\end{enumerate}

\begin{figure*}[b!]
  \includegraphics[width=\textwidth]{figures/figure-card-hho-2.pdf}
   \vspace{-6pt}
  \caption{The \texttt{SET} Card-Based Activity Kit. (1) Banners are used to visualize a hierarchical diagram. (2) Experience banks allow participants to generate examples for each factor dimension. (3) Scenario cards guide participants to create real-life scenarios.}
  \label{fig:card-kit}
  \vspace{-3pt}
\end{figure*}



