\section{Limitations \& Future Work}\label{sec-7.4}

A notable limitation of our study is the lack of sociodemographic diversity in our participant sample. Specifically, (1) \textit{location:} all participants were based in the U.S. Midwest; (2) \textit{gender:} the 20 parents who participated in the study included only three fathers; (3) \textit{income:} the majority of families reported middle to high household incomes; and (4) \textit{education:} most parents had a Master's degree or higher. These factors may limit the broader applicability of our findings. Consistent with previous education studies, a predominance of mothers were included due to self-selection bias \cite{schoppe2013comparisons, mcbride1993comparison}, and we faced difficulties recruiting lower-income families \cite{nicholson2011recruitment}. Children from higher-income and more highly educated families tend to perform better academically \cite{sirin2005socioeconomic}. These may have influenced the types of support families wanted from the robot, which need to be addressed in future research. we aim to collaborate with local community centers, libraries and schools to reach a more diverse population in our future work.

In addition, while cost remains a common accessibility limitation, some educational robots, such as Miko,\footnote{Miko Robot: \url{https://shorturl.at/XSyM5}} are now priced similarly to smartphones, improving feasibility for families. Future work should design novel interaction paradigms for AI-assisted educational robots in public spaces (\textit{e.g.,} libraries, schools, museums) to broaden accessibility. Moreover, our brief home visits captured immediate reactions rather than long-term changes in parent-child interactions, trust in AI/robots, or learning outcomes. Extended research are needed to observe how these factors evolve as AI-assisted robots integrate into daily life.

Additionally, we acknowledge the limitation of simplifying parental involvement factors into binary levels, resulting in eight scenarios. This decision was practical, as more granular factors (\textit{e.g.,} three levels each; 27 scenarios) would yield in a large number of scenarios not manageable for human participants. Future work should explore methods for representing a continuous spectrum of factors to capture more accurate scenarios. Furthermore, our system currently targets literacy and math. Beyond these academic subjects, other domains like social-emotional skills and creativity warrant exploration. Investigating parent-AI-robot collaboration in these areas could yield broader insights. Finally, ethical concerns motivated our work (Section~\ref{sec-rw-2.2}), but were not our main focus. We presented some ethical issues in Section~\ref{sec-result-2} and discussed in Section~\ref{sec-dis-2}, but future research should delve more deeply into these topics. A comprehensive understanding of ethical considerations, from data sourcing to model training, will help ensure that AI-generated content meets appropriate standards for children.