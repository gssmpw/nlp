\section{Related Work}
In this section, we review research related to the importance and barriers to parental involvement; parental use of learning technologies; and the use of generative AI and robot in educational and parenting scenarios.

\subsection{Importance and Barriers to Parental Involvement}\label{sec-rw-2.1}

% 79 words
Early childhood is a critical period for predicting future success and well-being, with early education investments resulting in higher returns than later interventions \cite{duncan2007school, doyle2009investing}. Effective parental involvement fosters cognitive and social skills, especially in younger children \cite{blevins2016early, peck1992parent}. Parents are encouraged to prioritize home-based involvement to maximize their influence \cite{ma2016meta}, as their involvement has a greater impact on children's learning outcomes \cite{hoffner2002parents, fehrmann1987home, hill2004parent} within the family setting than partnerships with schools or communities \cite{ma2016meta, harris2008parents, fantuzzo2004multiple, sui1996effects}.

However, parents' involvement in their children's education is often constrained by practical challenges related to parents' \textit{skills}, \textit{time}, and \textit{energy}. The Hoover-Dempsey and Sandler (HDS) framework \cite{green2007parents} and the CAM framework \cite{ho2024s} both highlight these factors-- parents' perceived \textit{skills and knowledge} (capability), \textit{time} (availability), and \textit{motivation} (energy)--influence the extent of their engagement. For instance, a parent confident in math may choose to engage more in math-related tasks, while those facing inflexible schedules may participate less \cite{green2007parents}. Unlike teachers, parents often lack formal pedagogical training and may underestimate their role in supporting children's learning, particularly as young children struggle to articulate their needs \cite{hara1998parent}. The CAM framework similarly suggests parents delegate tasks to a robot when they feel less capable, have limited time, or are unmotivated. These factors reflect parents' life contexts, shaped by demographic backgrounds, occupations, and parenting responsibilities \cite{grolnick1997predictors}, highlighting the need to help parents overcome barriers to effective involvement in early education within their life contexts.

\subsection{Parental Use of Learning Technologies}\label{sec-rw-2.2}
% 207 words
Technology encourages parental involvement by facilitating parent-child engagement in learning activities while introducing risks that require active parental mediation \cite{gonzalez2022parental}. On the positive side, technology offers novel opportunities for parental engagement and enhances children's learning outcomes. For example, e-books promote interactive behaviors between parents and children better than print books \cite{korat2010new}. In addition, having access to computers at home significantly boost academic achievement of young children when parents actively mediate their use \cite{hofferth2010home, espinosa2006technology}. However, the effectiveness of these tools often depends on parents' familiarity with and attitude toward technology. Mobile applications, for instance, can improve learning outcomes but require parents to possess sufficient technology efficacy to guide their use \cite{papadakis2019parental}.

On the negative side, technology introduces risks such as excessive screen time, exposure to inappropriate content, and misinformation, which necessitate parental intervention \cite{oswald2020psychological, howard2021digital}. According to parental mediation theory, parents mitigate these risks through restrictive mediation (e.g., setting limits), active mediation (e.g., discussing content), and co-use (e.g., shared use of technology) \cite{valkenburg1999developing}. Modern technologies like video games, location-based games (\textit{e.g.,} Pokemon Go), and conversational agents (\textit{e.g.,} Alexa) also require parents to adapt their mediation strategies to ensure responsible use \cite{valkenburg1999developing, nikken2006parental, sobel2017wasn, beneteau2020parenting, yu2024parent}. Overall, parents seek to leverage technology to support their children's learning due to ite effectivenss but are also mindful of its risks. Their involvement is therefore driven by both opportunities and concerns, highlighting the need to design tools that effectively involve parents to balance benefits and risks.

\subsection{Generative AI and Companion Robots for Parenting and Education}
Generative AI and companion robots offer human-like affordances, with AI simulating human intelligence and robots providing physical human-like features. Compared to conventional models (\textit{e.g.,} machine learning) and devices (\textit{e.g.,} laptops), these emerging technologies enable natural and social interactions, creating opportunities for novel paradigms to enhance parental involvement and children's learning while introducing their unique challenges.

\subsubsection{Generative AI}
GAI offers promising support for parents by enhancing their ability to educate and engage with their children. Prior work suggested that AI-driven systems can support parenting education \cite{petsolari2024socio} and provide evidence-based advice through applications and chatbots, delivering micro-interventions such as teaching parents how to praise their children effectively \cite{davis2017parent, entenberg2023user} or offering strategies to teach complex concepts \cite{mogavi2024chatgpt, su2023unlocking}. Many parents also prefer using GAI to create educational materials tailored to their children's needs, rather than granting children direct access to these tools \cite{han2023design}. Beyond educational support, AI-based storytelling tools address practical challenges (\textit{e.g.,} time constraints) by alleviating physical labor while fostering parent-child interactions \cite{sun2024exploring}. Furthermore, GAI offers advantages to children's learning directly. It can help create personalized learning experiences by providing timely feedback and tailoring content \cite{su2023unlocking, mogavi2024chatgpt, han2024teachers}, enhancing positive learning experiences \cite{jauhiainen2023generative}. For example, a LLM-driven conversational system can teach children mathematical concepts through co-creative storytelling, achieving learning outcomes similar to human-led instructions \cite{zhang2024mathemyths}.

Despite these benefits, several concerns persist regarding the use of GAI in education. Prior work highlighted the limitations of GAI, such as its limited effectiveness in more complex learning tasks,the limited quality of the training data, and its inability to offer comprehensive educational support \cite{su2023unlocking}. There is also a significant risk of GAI producing inaccurate or biased information, discouraging independent thought among children, and threatening user privacy \cite{su2023unlocking, han2023design, han2024teachers}. Many parents are skeptical about the role of AI in their children's academic processes, concerned about the accuracy of AI-generated content, and worry that over-reliance on AI could stifle independent thinking \cite{han2023design}.

%\todo{might need to add some structural transition here}
\subsubsection{Social companion robots}
Social companion robots have proven potential to assist parents in home education settings through studies in \textit{parent-child-robot} interactions. \citet{gvirsman2020patricc} showed that the robotic system, \texttt{Patricc}, fostered more triadic interaction between parents and toddlers than a tablet, and \citet{gvirsman2024effect} found that, in a parent-toddler-robot interaction, parents tend to decrease their scaffolding affectively when the robot increases its scaffolding behavior. Similarly, \citet{chen2022designing} found that social robots enhanced parent-child co-reading activities, while \citet{chan2017wakey} demonstrated that the WAKEY robot improved morning routines and reduced parental frustration. Beyond educational support, \citet{ho2024s} uncovered that parents envisioned robots as their \textit{collaborators} to support their children's learning at home and that their collaboration patterns can be determined by the parents' capability, availability, and motivation. Although parents generally have positive attitudes toward incorporating robots into their children's learning, they remain concerned about the risk of disrupting school-based learning and potential teacher replacement \cite{tolksdorf2020parents, lee2008elementary, louie2021desire}.

In addition to parental support, social companion robots also support children in education directly through \textit{child-robot interactions}. Physically embodied robots provide adaptive assistance and verbal interaction similar to virtual or conversational agents \cite{ramachandran2019personalized, leyzberg2014personalizing, schodde2017adaptive, brown2014positive}, yet they foster greater engagement with the physical environment and encourage more advanced social behaviors during learning \cite{belpaeme2018social}, leading to improved learning outcomes \cite{leyzberg2012physical}. Prior work demonstrated that companion robots can effectively support both school-based learning (\textit{e.g.,} math \cite{lopez2018robotic}, literacy \cite{kennedy2016social, gordon2016affective}, and science \cite{davison2020working}) and home-based learning activities (\textit{e.g.,} reading \cite{michaelis2018reading, michaelis2019supporting}, number board games \cite{ho2021robomath}, and math-oriented conversations with parents \cite{ho2023designing}). For example, \citet{kennedy2016social} suggested that children can learn elements of a second language from a robot in short-term interactions, and \citet{tanaka2009use} found that children who took on the role of teaching the robot gained confidence and improved learning outcomes.

%\todo{may need to make this a separate section and explain why we propose AI-assisted robots}

% \subsubsection{Research Gap}
Parental involvement in early education is crucial and AI-assisted robots can offer promising support by helping parents overcome practical barriers (\textit{i.e.,} time, energy, and skills) and addressing concerns about technological risks. Yet, limited research has examined how technology design can simultaneously alleviate these barriers and concerns. Though \citet{zhang2022storybuddy} emphasized the importance of flexible parental involvement during reading through a system called \textit{Storybuddy}, yet they focused on a virtual chatbot rather than a physical robot, and how the flexible modes may be used in different scenarios remain unknown. Similarly, \textit{ContextQ} \cite{dietz2024contextq} presented auto-generated dialogic questions to caregivers for dialogic reading, but primarily considered situations where parents are actively involved, not scenarios where parents cannot participate fully.

In this work, we address these gaps by exploring parental involvement contexts, understanding parents' perceptions of AI-generated content, and examining how parents collaborate with AI and robots under different scenarios. In the following sections, we describe our development of  \texttt{SET}, a card-based activity, to understand parental involvement contexts (Section~\ref{sec-card}), the design of the \texttt{PAiREd} system to enable parents to co-create learning activities with an LLM (Section~\ref{sec-system}), and user study aimed to discover use patterns and understand user perceptions of the system (Section~\ref{sec-study}).