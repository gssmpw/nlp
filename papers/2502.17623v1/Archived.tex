


Related work

--paernt involvement
%Parents support and facilitate their children’s education at home through several means: engaging them in learning-stimulating activities, discussing school and family issues, and conveying educational expectations \cite{mcwayne2004multivariate}. 
%parents can support their child's mathematical learning at home via engaging in different kinds of activities (e.g., number board games) and incorporating math concepts in their conversations with their children. This incorporation of math concepts in at-home activities is known as \textit{math talk}.
%In terms of developmentally appropriate practices, knowledgeable adults play important roles in scaffolding young children’s learning within the zone of proximal development \cite{wertsch1984zone}.

% in-home technology for children's learning

%Results showed that both interventions increased the number of divergent-thinking questions and the fluency of question asking, while they did not significantly alter children’s perception of curiosity despite their high intrinsic motivation scores. In addition, children’s curiosity trait has a mediating effect on question asking under the agent that promoted divergent-thinking, suggesting that question-asking interventions must be personalized to each student based on their tendency to be curious. \cite{alaimi2020pedagogical}

%First, a variety of technologies enhanced children’s collaboration and interaction with peers \cite{infante2010co}. Second, the technologies used at home facilitated adult-child interaction and maintained family relationships \cite{jessel2011different}. Researchers described how young children worked with adults (e.g., parents, grandparents, relatives) to achieve the shared goal of the technology-related activity and reinforce their ties with family members. 

--parent mediation
%This study contextual factors that elicit digital media practices in families with young children. They found that parents acknowledge the potential risks of media use and thereby restrict and monitor children's media use in various aspects. However, parents also recognized the potential benefits, especially ones that contributes to educational purposes. These factors reveal the paradoxical nature of parents' mediation practices and explain possible dynamics in media use. \cite{zaman2016qualitative}. 
%This study denotes that most parents of young children play the role of ‘gatekeepers’ when it comes to facilitating and constraining access and use of digital technologies.
% parent desired technology design
%parents desire devices such as conversational agents (CA) to include them in their children's learning activities, facilitate social engagement, and allow them to monitor their use of the technology \cite{garg2020conversational}.
%In a later paper, \citet{lovato2019hey} based on their finding that children used these devices to ask questions out of curiosity, suggested that CAs could be used for children?s self-directed learning. However, the authors suggested that that current smart speakers could better support question-asking behavior by tailoring their responses to specific users (e.g., children versus adults.)
%\citet{storer2020all} through their work with mixed-visual-ability families, found that smart speakers prove to be appropriate tools for supporting more accessible parent/child interactions. Examples include parents and childrens bonding while using smart speakers, including playing games together, sending each other jokes, and learning to use the devices. However, it also created new anxieties about children possibly gaining access to adult content
%\citet{park2020investigating}, after conducting a participatory user study, suggested that devices should be able to develop an understanding of family dynamics and rituals and adapt their features to social versus private situations (e.g., avoid revealing personal and private information in shared spaces).
%\citet{chen2019understanding} used mixed-method interviews and surveys with the parents of 2- to 6-year-olds in eight families that used storyboards as prompts, and examined barriers to the adoption of technology during mealtimes. They found that the parents preferred screen-based technology over voice-based devices and smart objects. In the case of voice-based devices, the personification of CAs led them to believe these devices could intrude upon their relationship with their children (e.g., by disrupting interpersonal interactions and displacing parenting relationships).
%We argue that children’s learning with technology is conditioned by several factors categorized into children, adults, and technology aspects. This typology shows the reciprocal interplay between children and technology, and adults play as mediators between children and technology. It should be noted that the effects of these factors are not conclusive \cite{hsin2014influence}.

%%% factors
%Parent’s perceptions are key to parental mediation style and therefore to children’s provision and protection in respect of digital technologies. Our analysis linking parents’ perceptions and mediation strategies shows that the more parents are confident with digital technologies themselves, the more they perceive them as positive and the more confident and active they are in their mediation. On the contrary, parents who lack digital skills and knowledge are less confident, find greater difficulties in managing their children’s access and protection, and are more fearful of losing control. These parents often chose more restrictive and controlling mediation strategies \cite{dias2016role}.
%This study illustrated that, within frame of this study, parental mediation of their young children’s *internet use levels were found to differentiate based on levels of age, education and internet usage experience through the scanning done by utilizing the scale created as a result of the scale adaptation. Additionally, a positive and meaningful relation between parental mediation scores and digital data security was found in the study \cite{durak2020parental}.
%\citet{sciuto2018hey} and \citet{garg2020he} found that, initially, younger children face more difficulties than older children in interacting with Conversational Agents because of their verbal intonations and cadences.

-- social robot
%Compared with virtual agents, physically embodied robots offer three advantages: (i) they can be used for curricula or populations that require engagement with the physical world, (ii) users show more social behaviors that are beneficial for learning when engaging with a physically embodied system, and (iii) users show increased learning gains when interacting with physically embodied systems over virtual agents \cite{belpaeme2018social}. Robots can be more engaging and enjoyable than a virtual agent in cooperative tasks \cite{kidd2004effect, wainer2007embodiment, kose2015effect} and are often perceived more positively \cite{wainer2007embodiment, powers2007comparing, li2015benefit}. physical robots have enhanced learning and affected later behavioral choice more substantially than virtual agents. Compared with instructions from virtual characters, videos of robots, or audio-only lessons, robots have produced more rapid learning in cognitive puzzles \cite{leyzberg2012physical}.
%robots that personalize what content to provide based on user performance during an interaction can increase cognitive learning gains \cite{leyzberg2014personalizing, schodde2017adaptive}.  
%Personalized social support, such as using a child’s name or referring to previous interactions \cite{janssen2011motivating, henkemans2013using}, is the low-hanging fruit of social interaction. 
%More complex prosocial behavior, such as attention-guiding \cite{saerbeck2010expressive}, displaying congruent gaze behavior \cite{huang2013repertoire}, nonverbal immediacy (3), or showing empathy with the learner \cite{leite2014empathic}, not only has a positive impact on affective outcomes but also results in increased learning.

%-Considerable educational benefits can also be obtained from a robot that takes the role of a novice, allowing the student to take on the role of an instructor that typically improves confidence while, at the same time, establishing learning outcomes. The care-receiving robot (CRR) was the first robot designed with the concept of a teachable robot for education \cite{tanaka2009use}
%As expected, the children did learn elements of a second language from the robot. This was measured immediately after the interaction and also some days later. These findings suggest that in this short-term dyadic interaction context, additional effort in developing social aspects of a robot’s verbal behaviour may not return the desired positive impact on learning gains \cite{kennedy2016social}.
%We have studied the effects of an autonomous social robot’s verbal and nonverbal behavior on children’s creativity as measured through the Droodle Creativity Game. We verified our hypotheses that children interacting with the highly creative robot generated more ideas, explored more themes of ideas, and generated more creative ideas as compared to children interacting with the non-creative robot \cite{ali2019can}.
%\citet{chen2020teaching} - Three child-agent interaction paradigm and adaptive role-switching model, impact of different roles on children's learning, bidirectional child-agent peer-learning paradigm --> this work developed reinforcement learning model that supports young children's one-on-one interaction with robot. This synergistic effect between tutor and tutee roles suggests why our adaptive peer-like agent brought the most benefit to children’s vocabulary learning and affective engagement, as compared to an agent that interacts only as a tutor or tutee for the child.
%-A long-term primary school study showed that a peer-like humanoid robot able to personalize the interaction could increase child learning of novel subjects \cite{baxter2017robot}

%Social robot as a teaching assistant in classroom for children's collaborative learning. The result shows that the social behavior encouraged children to work more in the first two lessons, but did not affect them in later lessons. On the other hand, social behavior contributed to building relationships and attaining better social acceptance \cite{kanda2012children}.

%We introduce a humanoid robot into a English class to assist the teacher in teaching and motivate students to involve in learning activities. Most students in the three classes had a positive attitude to this robot, and had great interest in the robot's performance. As a tutor or teacher, robots provide direct curriculum support through hints, tutorials, and supervision. These types of educational robots, including teaching assistant robots \cite{you2006robot}.

%However, the role of those robots sometimes becomes ambiguous (tutor versus peer), and it is difficult to place one above the other in general.

%Educational robots can interact with children in various ways, from one individual to a group, involving different numbers and types of people. 
%-A good deal of studies on educational robots focus on \textit{single child-robot} interactions in the home learning environment. 
%For example, \citet{michaelis2018reading} suggested that a social robot could maintain regular reading activity of a child (aged 10-12) at home while transforming the experience into a social one, and \citet{brown2013engaging} demonstrated that social interaction with robots help promote a child's tablet-based algebra test performance. These studies point to the potential support that social robots could offer for children in at-home education surrounding dyadic child-robot interaction design. 

%-Some other studies investigated how \textit{multiple children interact with a single robot}. \citet{strohkorb2015classification} examined social dominance in a group of children involving a social robot and identified a novel model for classifying social dominance level using non-behavioral features. 

%-On the other hand, \citet{leite2015emotional} studied how \textit{multiple robots interact with a single child} within a storytelling scenario and discovered that involving multiple robots could benefit children's social development. Other studies, yet relatively rare, involve \textit{parent-child dyads} in the interaction with a robot, leading to a triadic social dynamic. \citet{ho2021robomath} explored the design of a robot in supporting young children's math learning at home through a board game, with the presence and support of a parent. This study mainly analyzed the pros and cons of different cues from the robot and did not focus on the design for parent-child-robot interaction. Another study specifically explored how parents can support young children's learning experience with a robot in the field of programming education \cite{relkin2020parents}, but focused on highlighting parents' strategies instead of uncovering the design space for a robot in triadic interaction. 

%Few prior studies have explored the multi-person child-robot interaction paradigm \cite{gvirsman2020patricc, strohkorb2015classification}. Among those, robot interaction with a child group has been more widely examined, such as a robot mediating children’s conflict resolution \cite{shen2018stop} and recognizing social dominance among children [15]. Only a few MHRI studies involved both adults and children (e.g., \cite{gvirsman2020patricc}, \cite{rudovic2017measuring]}). In a crosscultural study \cite{rudovic2017measuring]}, children with autism engaged in robot-assisted therapy with an adult, but the researchers focused exclusively on the child-robot interaction. In another study \cite{gvirsman2020patricc}, Patricc, a robotic platform designed for toddler-parent-robot triadic interaction, was found to promote more triadic interactions than a tablet.

%- (parent observe, not actively involved) This study suggested that parents recognize the robots’ potential for language learning within a playful interaction with their children. However, as parents reported, the technical challenges for an adaptive and smooth interaction might impede children’s learning gains in the long-term\cite{tolksdorf2020parents}

%- (survey-only study) Results showed that the parents generally accepted a robot as a teaching and learning tool in educational scenarios but were cautious about using the robot as a teacher. Instead, they preferred to assign it a supportive role \cite{lee2008elementary}.

%- (survey only on parents, classroom, no tech) Oros et al. found that in general parents have positive attitudes towards robots being a part of children’s education [25]. Moreover, the authors showed that such a positive attitude towards robots was even greater in people with a higher educational level in comparison to people with a lower educational level \cite{oros2014children}.

%- (group interview, classroom, no tech) This study suggested parents' positive attitude towards social robots for children's learning. Parents expressed a strong desire for educational robots to support their children at school. While children embraced the idea of a robot at school, some expressed concerns about the potential for robots to be disruptive \cite{louie2021desire}.

%In a crosscultural study \cite{rudovic2017measuring}, children with autism engaged in robot-assisted therapy with an adult, but the researchers focused exclusively on the child-robot interaction. 

%- (live interaction, parent involved) In another study \cite{gvirsman2020patricc}, Patricc, a robotic platform designed for toddler-parent-robot triadic interaction, was found to promote more triadic interactions than a tablet.

%- Another work designed WAKEY, a technology-based intervention aimed at helping parents teach preschool children to carry out their morning routines via more effective communication tactics. In their intervention, parents reported much less frustration during morning routines and greater independent behavior by children, as well as improvements in their parenting attitudes and new insights into communication \cite{chan2017wakey}.

%- A participatory design study examined how families view in-home social robots partaking in shared activities and found common themes around robot roles at home, different preferred robot interaction styles, and privacy and ethics concerns. However, this study did not directly illuminate the design space for parent-child-robot triadic interaction, nor provide families with a lived technology experience to develop experiential knowledge \cite{cagiltay2020investigating}. 

%-Parents in general showed positive attitude towards a social robot in educational setting. In summary, all of our participants saw the potential benefits of educational robots to provide individual engagement and instruction for ELLs at school. Teachers and parents both recognized that an educational robot cannot replace the teacher and positive robot attributes varied across all three samples. \cite{alam2022social}.