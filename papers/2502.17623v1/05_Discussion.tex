\section{Discussion}
Below, we draw upon our findings to explore the idea of enabling parents to collaborate with an AI-assisted robot to supervise generative learning materials and flexibly involve in their child's learning.

\subsection{Parental Involvement Support Needs}
We found that while parents generally have the \textit{skill} to support their children's development, they seek more guidance on pedagogical strategies, especially for advanced concepts. In addition, our quantitative results further indicate that parents tend to underestimate their children's ability in more advanced concepts. In fact, \citet{claessens2014academic} and \citet{engel2013teaching} suggested that preschoolers benefit from exposure to advanced content in literacy and math, compared to basic content. Parents' lack of ability or confidence to guide children in advanced content hinders more effective support for children. Therefore, supporting parents with pedagogical strategies in advanced content and providing them with opportunities to observe their children's developmental progress can further optimize the value of parental involvement. In addition, we discovered that parents' \textit{motivation} to engage in learning activities is influenced by their physical and emotional state and their children's willingness to learn, while parental \textit{availability} is shaped by factors such as work, household responsibilities, and family needs. These results exemplify prior parental involvement frameworks \cite{green2007parents, ho2024s}, and future studies should consider how to detect these moments reliably to automate intelligent decisions to support parents.

\subsection{Ethical Consideration in the Use of AI-Generated Learning Content}\label{sec-dis-2}
We found that parents have mixed attitudes toward AI-generated learning content, expressing concerns about age appropriateness, inaccuracies, data quality, over-reliance, and message dilution. Similar concerns were raised by \citet{han2024teachers}, who noted parents' challenges in distinguishing elementary-school students' authorship and controlling misinformation. \citet{yu2024exploring} and \citet{ho2024s} also highlighted worries about misinformation, appropriateness, and privacy for teenagers. To mitigate harm from AI for children, prior research has outlined five key principles for age-appropriate AI: fairness and equality, transparency and accountability, privacy and exploitation, safety and safeguarding, and sustainability and age-appropriateness \cite{wang2022informing}. These insights not only underscore the principles AI-generated content should adhere to but also emphasize parents' concerns about the use of such content for children. Researchers should continue developing principles to guide generative AI for children while also proposing effective interaction paradigms that facilitate parental supervision. Furthermore, enhancing the explainability and transparency of how learning materials are generated by AI could help alleviate parental concerns and encourage more effective use. While creating AI models that produce safe and accurate outputs is crucial, designing mechanisms that build parents' trust and understanding of the content's source could improve the adoption of AI-generated materials.

\subsection{Optimizing Parental Control and Supervision for LLM-Generated Content}
We identified three key factors--difficulty, variety, and quality--that parents considered when reviewing LLM-generated learning content, providing essential criteria for developing specialized AI models for early education. In addition, many parents expressed an interest in leveraging content generated by AI but also voiced concerns, consistent with prior studies on parental control over media content \cite{yu2024parent, ho2024s} and preferences for creating their own learning materials using AI \cite{han2024teachers}. We found that parents' willingness to control and supervise the content varied by their \textit{level of trust in LLMs} and \textit{personal preferences}. When time or energy was limited, many parents opted for minimal supervision, skimming or directly relying on LLM-generated content. Those with lower trust tended to rely on previously reviewed materials or avoid the system entirely. Even when parents had more time and energy, their approaches diverged--some enthusiastically customized the AI-generated content, while others viewed customization as a burden and minimized or avoided using the system.

The range of use cases indicate that, first, the questions auto-generated by LLMs need to meet parents' expectations for quality and safety. This would allow parents to trust and make minimal edits, thereby improving their confidence in LLMs. Second, to further reduce parental effort, especially when time and energy are limited, LLM-based content generation should adapt by learning from parents' and children's preferences over time, minimizing repetitive errors. Third, the design for varying levels of control over LLM autonomy should be refined to suit different use cases. For example, a flexible design could allow parents to toggle between reviewing only critical information and in-depth editing. Future research and design should prioritize improving the quality of LLM-generated content, develop mechanisms for LLMs to incorporate parental feedback over time, and design user-friendly interfaces that provide flexible control options, reducing cognitive load across different use scenarios.

\subsection{Design for Complex and Contextualized Interaction Dynamics with AI-based Robot} %- complex  %- blockers
We examined parents' contextual usage patterns of the editor interface, revealing a spectrum of approaches: editing, skimming, reusing previously reviewed content, using LLM-generated content directly, or avoiding the system altogether. Similarly, in the activity interface, we observed various usage patterns: \textit{parent takeover} mode leveraging LLM content as support; \textit{collaborative} modes determined by parent skill and motivation; \textit{robot takeover} mode under supervision, with previewed content, or with original LLM content; and opting to avoid the system entirely. \citet{zhang2022storybuddy} similarly emphasized the importance of flexible parental involvement during reading through a system called \textit{Storybuddy} using a virtual chatbot, while \citet{dietz2024contextq} presented auto-generated dialogic questions to caregivers for dialogic reading that is similar to our \textit{parent-takeover} mode. 
%These work highlight parents' desire for flexible control to supervise, co-create, and personalize content, empowering them to adapt LLM usage to their dynamic contexts.

The parent-child-robot interaction must adapt to changing scenarios. For example, when a parent becomes busy or loses motivation, the dynamic may shift from parent-child-robot interaction to child-robot interaction. To adapt to this shift, designing mechanisms such as mode-switching and flexible role delegation is essential. This approach allows parents and children to utilize the robot across a broader range of contexts, rather than being restricted to specific scenarios, thereby catering to individual needs and preferences. Although we made an initial attempt to design a mechanism that enables mode-switching and flexible role delegation between the parent and the robot, our findings revealed areas for improvement. We also identified factors that can hinder parents and children in utilizing this design to its full potential. 

First, when parents choose the \textit{parent takeover mode}, they often do not require the robot at all. Instead, they rely on LLM-generated content as a resource to guide their conversations, pacing the interaction according to their own and their child's needs. The current presentation of information may not optimally support parents in leading these activities.  Thus, the \textit{parent takeover mode} should focus on delivering suggested interaction content more efficiently, enabling parents to access and integrate the material seamlessly. P4, for example, found this mode challenging because her child was too captivated and distracted by the robot, reducing her ability to engage fully. This observation suggests that the design should allow parents to easily separate the robot from the content delivery, providing them with full control. From an educational standpoint, since parents are deeply involved in this mode, it presents opportunities to introduce more advanced content. Parents can explain complex concepts in a manner tailored to their child’s understanding compared to the robot. Therefore, integrating or suggesting more advanced content in this mode could enhance parents' engagement and maximize the educational value of their involvement.

Second, when parents select \textit{collaboration modes} (either parent-led or robot-led), they often relied on the robot to either engage their child or reduce their own involvement when feeling tired. 
These different needs--enhancing engagement versus offloading activity leadership--require tailored design solutions. For instance, the robot could not only facilitate learning but also offer entertaining and interactive responses. Parents could activate these features as needed, drawing attention to the content or creating excitement.
% These two situations indicate different needs: one require the robot to be fun and engaging and the other require the robot to lead the learning activity. This suggests that the design of the collaboration modes should take these varying parental intentions into account and offer appropriate options. 

Third, when parents selected the \textit{robot takeover mode}, they wanted to either reduce their involvement while maintaining supervision or be completely absent. In the first case, they preferred the robot to lead the activity while they remained nearby to assist if needed. In the second, they wanted the robot to manage the entire activity independently for the child. However, our current design of the \textit{robot takeover mode} remains too ``parent-centered,'' hindering young children's independent use due to their limited reading skill, touchscreen proficiency, and attention spans. Many parents noted that older children might handle the system independently, suggesting that the \textit{robot takeover mode} should incorporate more interaction modalities, adjustable information density, and varying activity lengths to accommodate development needs of children as they mature. From an educational standpoint, younger children benefit from tasks that are neither too challenging nor too easy, as overly difficult tasks can lead to frustration and loss of confidence. Thus, when the \textit{robot takeover mode} is active, ensuring that the learning content presents an appropriate level of challenge to maintain engagement and support children's learning is essential.

%\todo{Add a paragraph related to ethical considerations in the use of AI-assisted robots with children}



