\begin{table}[t!]
    \caption{Participant Demographics}
    \label{tab:demographics}
    \centering
\fontsize{6.3}{8.3}\selectfont
    \begin{tabular}{p{0.02\linewidth}p{0.06\linewidth}p{0.09\linewidth}p{0.09\linewidth}p{0.09\linewidth}p{0.13\linewidth}p{0.23\linewidth}}%
        \toprule
         \textbf{ID} & \textbf{Child Age} & \textbf{Child Gender} & \textbf{Parent Gender} & \textbf{Ethnicity} & \textbf{Maternal Education} & \textbf{Household Income} \\
        \midrule
         1 & 3Y8M & Male & Female & Biracial & Master's & \$200,000 and more \\
         \midrule
         2 & 3Y4M & Female & Male & Biracial & Master's & \$150,000-\$199,999\\
         \midrule
         3 & 3Y2M & Female & Female & White & Master's & \$100,000-\$149,999\\
         \midrule
         4 & 4Y6M & Male & Female & White & Ph.D. & \$100,000-\$149,999\\
         \midrule
         5 & 4Y1M & Female & Female & White & Master's & \$100,000-\$149,999\\
         \midrule
         6 & 3Y3M & Male & Female & Black & Master's & \$35,000 -\$49,999\\
         \midrule
         7 & 3Y5M & Female & Female & White & Ph.D. & \$100,000-\$149,999\\
         \midrule
         8 & 4Y2M & Male & Female & White & Bachelor's & \$100,000-\$149,999\\
         \midrule
         9 & 3Y2M & Female & Female & Biracial & Ph.D. & \$50,000-\$74,999\\
         \midrule
         10 & 5Y0M & Male & Male & White & Bachelor's & \$100,000-\$149,999\\
         \midrule
         11 & 5Y1M & Male & Female & White & Ph.D. & \$200,000 and more\\
         \midrule
         12 & 4Y10M & Female & Female & White & Master's & \$100,000-\$149,999\\
         \midrule
         13 & 3Y9M & Male & Male & White & Master's & \$100,000-\$149,999\\
         \midrule
         14 & 4Y0M & Female & Female & Asian & Master's & \$50,000-\$74,999\\
         \midrule
         15 & 4Y6M & Female & Female & Biracial & Ph.D. & \$50,000-\$74,999\\
         \midrule
         16 & 4Y10M & Female & Female & White & Master's & \$150,000-\$199,999\\
         \midrule
         17 & 4Y7M & Female & Female & White & Master's & \$100,000-\$149,999\\
         \midrule
         18 & 4Y2M & Female & Female & White & Ph.D. & \$200,000 and more\\
         \midrule
         19 & 4Y9M & Female & Female & Hispanic & Bachelor's & \$200,000 and more\\
         \midrule
         20 & 3Y2M & Female & Female & White & Ph.D. & \$200,000 and more\\
         
         \bottomrule
    \end{tabular}
\end{table}

\begin{figure*}[b!]
  \includegraphics[width=\textwidth]{figures/figure-robot-hho.pdf}
   \vspace{-6pt}
  \caption{Robot Interaction Module of the PAiREd System and Study Setup. The robot facilitates the activity based on its assigned role, providing behavioral expressions to engage the child. The child can interact with the robot by pressing its bumper to navigate through the activity.}
  \label{fig:robot}
   \vspace{-6pt}
\end{figure*}

\section{User Study} \label{sec-study}

\subsection{Participants} 
Following protocols fully approved by the responsible Institutional Review Board (IRB), we recruited families from U.S. Midwest cities via email distributed to university employee mailing lists. We selected participants based on the following criteria: (1) one parent and one child participated together; (2) the child was aged 3--5 years; and (3) both parent and child could communicate in English. This age range was chosen as it represents a critical stage for parental involvement in early education \cite{purpura2013informal}, prior to formal schooling. Families received \$50 USD upon completing the study.

Our analysis includes data from 20 parent-child dyads with children aged 3–5 years (13 female, 7 male; $M = 4.1$, $SD = 0.67$ years). Each session involved one parent and one child, although other family members were sometimes present but did not participate. Participant demographics are summarized in Table \ref{tab:demographics}.
Most participating parents were mothers (17 mothers, 3 fathers), reflecting the greater likelihood of mothers serving as primary caregivers, consistent with previous research \aptLtoX[graphic=no,type=html]{\cite[{e.g.,}][]{schoppe2013comparisons}}{\cite[\textit{e.g.,}][]{schoppe2013comparisons}}. Despite efforts to recruit a diverse sample, the sociodemographic representation is limited: 85\% of mothers held at least a Master's degree, and 80\% of families reported an annual income of \$100,000 or more. This limits the generalizability of our findings and may omit insights into underrepresented groups, a limitation further discussed in Section~\ref{sec-7.4}.



\begin{figure*}[b!]
  \includegraphics[width=\textwidth]{figures/figure-procedure-hho.pdf}
   \vspace{-6pt}
  \caption{Three-phase study procedure. Phase 1 (40 min) and Phase 2 (40 min) only involved the parent, and Phase 3 (60 min) involved both the parent and the child. }
  \label{fig:procedure}
   \vspace{-6pt}
\end{figure*}

\subsection{Study Design} 
We conducted in-home visits (2.5 hours per visit) with families. We used \texttt{SET} (Section~\ref{sec-card} and Figure~\ref{fig:card-kit}) to foster discussion with parents about their real-life scenarios in which they are able or unable to facilitate learning activities for their child. In addition, we used \texttt{PAiREd}, an AI-assisted robot as a \textit{technology probe} \cite{hutchinson2003technology} to facilitate discussions on how parents may, under different \texttt{SET} scenarios, prefer to supervise the AI-generated content for their child and adjust their participation in learning activities.

\subsubsection{Study Materials and Setup}
The study materials included (1) a Misty II social robot; (2) a Microsoft Surface laptop to present user interface; (3) the \texttt{SET} card-based activity kit; (4) recording devices (\textit{i.e.,} a video camera, a webcam, and an audio recorder) positioned to capture participants' behavior and conversations (Figure~\ref{fig:robot}). 

\subsubsection{Study Procedure} \label{sec-procedure} 
Before the study, the experimenter provided families with an overview of the procedure and obtained informed consent: the parent signed a consent form, and the child gave verbal assent. The study was conducted in three phases: Phase 1 and Phase 2 (40 minutes each) involved \textit{only the parent} and Phase 3 (60 minutes) included \textit{both the parent and the child}. To minimize interruptions during the first two phases, the research team offered childcare assistance upon request (Figure \ref{fig:procedure}).

% phase 1
In Phase 1, the experimenter began by asking the parent to describe their involvement in their child's learning activities at home (\textit{e.g.,} ``\textit{What learning activities does your child typically do at home? Which ones involve you?}''). Next, the experimenter introduced the \texttt{SET} activity kit (Figure~\ref{fig:card-kit}, Section~\ref{sec-card}) and explained the factors and their dimensions with concrete examples (Table~\ref{tab:kit-example}), arranging the \texttt{SET-banners} into a hierarchical diagram for visualization (Figure~\ref{fig:card-kit}). The parent was then asked to write 2–3 real-life examples for each dimension of every factor on sticky notes and place them on the \texttt{SET-experience banks} (Figure~\ref{fig:card-kit}). Lastly, the experimenter introduced the \texttt{SET scenario cards}, explaining how the hierarchical diagram leads to eight scenarios by combining three two-level factors (Figure~\ref{fig:card-kit}). The parent was asked to write one real-life example for each scenario card for later use in Phase 3.

In Phase 2, the parent was first asked about their thoughts on AI-generated content (\textit{e.g.,} ``What are your thoughts on AI-generated content?''). The experimenter then introduced the \textit{editor interface} (Figure~\ref{fig:editor}), explaining its features for content modification. The parent reviewed and edited LLM-generated content for two books, one on math and one on literacy, while using the \textit{think-aloud} method to vocalize their thought process. Afterward, the experimenter discussed the parent's experience and used the \texttt{SET-scenario cards} to explore how they might use the interface in different scenarios.

In Phase 3, the parent first familiarized themselves with the robot interaction, learning to use the \textit{mode-switching} and \textit{role-delegation} mechanisms. The experimenter interviewed them about their perceptions of these mechanisms and their application in situations specified in \texttt{SET-scenario cards}. Next, the parent and child read two books together, incorporating activities created by the parent in Phase 2. At the start of each book, parents were asked to situate themselves in one of the \texttt{SET-scenario cards}, adjusting the mode and delegating jobs according to what they think they might do in that scenario. The experimenter gave the parent a different scenario card in the middle of the reading, and the parent responded to the situation, \textit{i.e.,} change to a different mode, accordingly. In addition, they also completed surveys on their perceptions of their child's math and literacy proficiency before and after the reading. Finally, they discussed their experiences collaborating with the robot and provided feedback on the designed mechanisms.

\begin{table*}
    \caption{Example of scenarios created using the {\texttt{SET-scenario cards}} by P2. P2 selected {\textit{``Teach child to build block towers''}} as the high-skill activity and {\textit{``Process feelings''}} as the low-skill activity for all examples. Each row represents one scenario card.}
\label{tab:scneario-example}
        \centering
        \renewcommand{\arraystretch}{1.2}
    \small
    \begin{tabular}{p{0.12\linewidth}p{0.12\linewidth}p{0.12\linewidth}p{0.12\linewidth}p{0.4\linewidth}}
        \toprule
        \hline
        \textbf{Scenario Card} & \textbf{Skill} & \textbf{Energy} & \textbf{Time} & \textbf{Example}\\
        \midrule
         1 & High & High & High & \textit{I am well rested and my partner is taking care of dinner.} \\
\hline
         2 & High & High & Low & \textit{Kids are being well-behaved but I have to mow the lawn.} \\
\hline
         3 & High & Low & High & \textit{I need to rest while my partner is making dinner.} \\
\hline
         4 & High & Low & Low & \textit{Kids are whining and I have to be at work.} \\
\hline
         5 & Low & High & High & \textit{Free from work and we don't have any plans.} \\
\hline
         6 & Low & High & Low & \textit{Well rested but have to do chores.} \\
\hline
         7 & Low & Low & High & \textit{Not feeling well but we have free time.} \\
\hline
         8 & Low & Low & Low & \textit{Not feeling well and have chores to do.} \\
\hline
         \bottomrule
    \end{tabular}
\end{table*}

\subsection{Data Analysis}
To anonymize participants, we assigned IDs based on the order of study completion (\textit{e.g.,} P1 refers to the first participant). For qualitative data, we conducted a reflexive \textit{Thematic Analysis (TA)} following \citet{clarke2014thematic} and \citet{mcdonald2019reliability} to identify patterns and understand parental perspectives. The first three authors, experienced in qualitative coding, transcribed and familiarized themselves with the data from audio and video recordings. Initial semantic codes were generated by the first author and organized into categories as the initial codebook. The team collaboratively coded data from P1 using the initial codebook, discussed interpretations, and created a consensus-based codebook. Using the codebook, the remaining data were independently coded by the three authors, with peer reviews and iterative discussions to finalize codes. We constructed final themes from recurring, meaningful patterns in the data. For the quantitative survey data (\textit{i.e.,} parent perception on child's literacy and math abilities), we first conducted a One-Way ANOVA to evaluate overall significance across difficulty levels for each subject (\textit{i.e.,} math and literacy), revealing significant differences between levels. We then performed Repeated Measures ANOVAs for each level, followed by post-hoc pairwise comparisons using non-parametric (Wilcoxon Signed-Rank) tests.