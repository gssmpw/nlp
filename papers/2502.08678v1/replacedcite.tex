\section{Related Work}
\vspace{-1mm}
\label{sec:Related work}
\textcolor{black}
In the domain of precision agriculture, several studies have employed deep learning models for weed and crop classification using multispectral data. This section reviews relevant literature that informs our approach, focusing on how previous researchers have tackled similar problems.

\textcolor{black}Bah et al.  ____ used ResNet for weed detection in polyhouse-grown bell peppers. They demonstrated that ResNet's deep architecture and residual connections can effectively handle the complex variability in agricultural imagery, resulting in high classification accuracy. This work supports our choice of ResNet for its robustness in handling complex and high-dimensional multispectral data. \textcolor{black}Ronneberger et al. ____ introduced U-Net for image segmentation, which has been successfully adapted for agricultural applications____. U-Net's encoder-decoder structure and its capability to perform precise pixel-level segmentation make it an ideal choice for distinguishing between crops and weeds in high-resolution multispectral images.

\textcolor{black}Chen et al.  ____ applied the DeepLab model for semantic segmentation in agricultural fields. DeepLab's use of atrous convolution and Conditional Random Fields (CRFs) enhances boundary delineation and maintains high-resolution features____, which is critical for accurately segmenting the intricate patterns of weeds and crops. \textcolor{black}Badrinarayanan et al. ____ developed SegNet for efficient semantic segmentation, demonstrating its effectiveness in processing high-resolution remote sensing data. SegNet’s encoder-decoder framework is particularly suited for applications requiring detailed segmentation of agricultural fields, where computational efficiency is also a priority .

\textcolor{black}The work by Szegedy et al. ____ on InceptionV3 showed how factorized convolutions and diverse filter sizes could handle large and varied datasets. This makes InceptionV3 a strong candidate for our study, given its ability to manage the diverse spectral bands in multispectral agricultural data and provide accurate classification results.

\textcolor{black}Recent advancements have focused on unsupervised domain adaptation (UDA) techniques to improve model performance across different fields. Huang et al. ____ proposed an unsupervised domain adaptation framework using greedy pseudo-labeling, which optimizes pseudo-label selection to enhance weed segmentation under varied conditions . This approach mitigates overfitting by monitoring covariance during co-training, ensuring robust model adaptation across different agricultural contexts.
\textcolor{black}In another study, an unsupervised classification algorithm was developed for early weed detection in row-crops by combining spatial and spectral information. This method leveraged Fourier Transform for row orientation detection and NDVI(Normalized Difference Vegetation Index, an index that quantifies vegetation by measuring the difference between near-infrared and red light reflectance.) for vegetation discrimination, improving classification results by integrating spatial and spectral data ____.

\textcolor{black}These studies collectively highlight the effectiveness of advanced deep learning models in the context of weed and crop classification using multispectral datasets. They provide a solid foundation for our approach, validating our model choices and guiding our methodological framework.

%-----------------------------------------------------------------------------%
\vspace{-1mm}