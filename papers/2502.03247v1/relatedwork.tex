\section{Related work}
\label{sec:relwork}
In this section, we mention some directly related recent work on practical integration of threshold cryptography, excluding theoretical contributions as the literature on threshold cryptography is vast.
Two orthogonal lines of research are relevant: integration of threshold cryptography into distributed systems to enhance security and relax system-wide trust assumptions, and easier implementation of more complex multi-round protocols, relying on the coordination of replicated systems. 

As for the first, 
F3B~\cite{DBLP:conf/aft/ZhangMQBEF23} is an extension to a blockchain that protects against front-running attacks through threshold encryption.  It employs two committees: one for consensus and one for decryption, and the two do not necessarily overlap.  Since the solution is tailored to Ethereum, it is specific to its consensus model, exploiting long finality times. 
Shutter Network~\cite{shutter} provides an identity-based threshold decryption service for Ethereum smart contracts. Leveraging an ad-hoc committee, the network allows in every epoch decryption of encrypted transactions on the blockchain in the previous epoch, resulting in a delay of one epoch. Shutter relies on a Tendermint-based atomic broadcast channel for communication. This ensures fault-tolerance when the threshold parameter is chosen accordingly, sacrificing latency due to ordering.
The Internet Computer~\cite{DBLP:conf/podc/CamenischDHPS022} demonstrates the successful deployment of threshold protocols in practice, using secure randomness for leader selection and quorum certificates for consensus. However, the cryptographic methods are integrated closely with the whole Internet Computer platform.
Also Aptos recently released a design for blockchain-native threshold cryptography~\cite{DBLP:journals/corr/abs-2407-12172}.  It provides cryptographic randomness for the platform, but it is also intertwined with the consensus mechanism to optimize performance.

A separate line of work has proposed to use blockchains as a coordination layer for implementing complex, multi-round protocols. For instance, Ferveo~\cite{DBLP:journals/iacr/BebelO22} introduces a DKG protocol that relies on BFT-based blockchain synchronization, enabling threshold encryption for mempool privacy. Similarly, CHURP~\cite{DBLP:conf/ccs/MaramZWLZJS19} focuses on secure reconfiguration and resharing strategies, while Scrape~\cite{DBLP:conf/acns/CascudoD17} proposes randomness beacon generation using blockchain as a communication medium.

Of a different interest is the performance of threshold schemes in practical systems, with measurements that go beyond microbenchmarks. A recent benchmarking efforts compares multiple libraries with respect to their effectiveness for BFT replication protocols~\cite{vonseck2024thresh}. 
Similar to our evaluation, this study demonstrates the crucial importance of the environment and network deployment on the observed performance of threshold cryptography.