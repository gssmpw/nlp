%% 
%% Copyright 2007-2024 Elsevier Ltd
%% 
%% This file is part of the 'Elsarticle Bundle'.
%% ---------------------------------------------
%% 
%% It may be distributed under the conditions of the LaTeX Project Public
%% License, either version 1.3 of this license or (at your option) any
%% later version.  The latest version of this license is in
%%    http://www.latex-project.org/lppl.txt
%% and version 1.3 or later is part of all distributions of LaTeX
%% version 1999/12/01 or later.
%% 
%% The list of all files belonging to the 'Elsarticle Bundle' is
%% given in the file `manifest.txt'.
%% 
%% Template article for Elsevier's document class `elsarticle'
%% with numbered style bibliographic references
%% SP 2008/03/01
%% $Id: elsarticle-template-num.tex 249 2024-04-06 10:51:24Z rishi $
%%
\documentclass[preprint,12pt]{elsarticle}
\usepackage[a4paper, margin=1.5in]{geometry}
%% Use the option review to obtain double line spacing
%% \documentclass[authoryear,preprint,review,12pt]{elsarticle}

%% Use the options 1p,twocolumn; 3p; 3p,twocolumn; 5p; or 5p,twocolumn
%% for a journal layout:
%% \documentclass[final,1p,times]{elsarticle}
%% \documentclass[final,1p,times,twocolumn]{elsarticle}
%% \documentclass[final,3p,times]{elsarticle}
%% \documentclass[final,3p,times,twocolumn]{elsarticle}
%% \documentclass[final,5p,times]{elsarticle}
%% \documentclass[final,5p,times,twocolumn]{elsarticle}

%% For including figures, graphicx.sty has been loaded in
%% elsarticle.cls. If you prefer to use the old commands
%% please give \usepackage{epsfig}

%% The amssymb package provides various useful mathematical symbols
\usepackage{amssymb}
%% The amsmath package provides various useful equation environments.
\usepackage{amsmath}
%% The amsthm package provides extended theorem environments
%% \usepackage{amsthm}

% Preamble - load necessary packages
\newcommand{\CG}{\mathcal{G}\xspace}
\newcommand{\CV}{\mathcal{V}\xspace}
\newcommand{\CE}{\mathcal{E}\xspace}
\newcommand{\CA}{\mathcal{A}\xspace}
\newcommand{\CF}{\mathcal{F}\xspace}
\newcommand{\CR}{\mathcal{R}\xspace}
\newcommand{\CB}{\mathcal{B}\xspace}
\newcommand{\CX}{\mathcal{X}\xspace}
\newcommand{\CK}{\mathcal{K}\xspace}
\newcommand{\CM}{\mathcal{M}\xspace}
\newcommand{\CC}{\mathcal{C}\xspace}
\newcommand{\CL}{\mathcal{L}\xspace}
\newcommand{\CI}{\mathcal{I}\xspace}
\newcommand{\CQ}{\mathcal{Q}\xspace}
\newcommand{\CO}{\mathcal{O}\xspace}
\newcommand{\CP}{\mathcal{P}\xspace}
\newcommand{\CS}{\mathcal{S}\xspace}
\newcommand{\CT}{\mathcal{T}\xspace}
\newcommand{\CJ}{\mathcal{J}\xspace}
\usepackage[para]{footmisc}
\usepackage{subfig}
% \usepackage{subcaption}
% \usepackage{array}
% \usepackage{colortbl}


\usepackage{graphicx} % For including graphics if needed
\usepackage{adjustbox}

% Define the "definition" environment in the preamble
% \newtheorem{definition}{Definition}

% \usepackage{subcaption}

%% The lineno packages adds line numbers. Start line numbering with
%% \begin{linenumbers}, end it with \end{linenumbers}. Or switch it on
%% for the whole article with \linenumbers.
%% \usepackage{lineno}

% \journal{International Journal of Forecasting}
\makeatletter
\def\ps@pprintTitle{%
 \let\@oddhead\@empty
 \let\@evenhead\@empty
 \let\@oddfoot\@empty
 \let\@evenfoot\@empty
}
\begin{document}

\begin{frontmatter}

%% Title, authors and addresses

%% use the tnoteref command within \title for footnotes;
%% use the tnotetext command for theassociated footnote;
%% use the fnref command within \author or \affiliation for footnotes;
%% use the fntext command for theassociated footnote;
%% use the corref command within \author for corresponding author footnotes;
%% use the cortext command for theassociated footnote;
%% use the ead command for the email address,
%% and the form \ead[url] for the home page:
%% \title{Title\tnoteref{label1}}
%% \tnotetext[label1]{}
%% \author{Name\corref{cor1}\fnref{label2}}
%% \ead{email address}
%% \ead[url]{home page}
%% \fntext[label2]{}
%% \cortext[cor1]{}
%% \affiliation{organization={},
%%             addressline={},
%%             city={},
%%             postcode={},
%%             state={},
%%             country={}}
%% \fntext[label3]{}

\title{Generalized Factor Neural Network Model for High-dimensional Regression}

%% use optional labels to link authors explicitly to addresses:
%% \author[label1,label2]{}
%% \affiliation[label1]{organization={},
%%             addressline={},
%%             city={},
%%             postcode={},
%%             state={},
%%             country={}}
%%
%% \affiliation[label2]{organization={},
%%             addressline={},
%%             city={},
%%             postcode={},
%%             state={},
%%             country={}}

\author[oxstat,omi]{Zichuan Guo} %% Author name
\author[oxstat,omi,ucla]{Mihai Cucuringu} %% Author name
\author[qmmath,nlmathstat]{Alexander Y. Shestopaloff} %% Author name

%% Author affiliation
\affiliation[oxstat]{organization={Department of Statistics, University of Oxford},%Department and Organization
            addressline={24-29 St Giles'}, 
            city={Oxford},
            postcode={OX1 3LB}, 
            country={U.K.}}

%% Author affiliation
\affiliation[omi]{organization={Oxford-Man Institute, University of Oxford},%Department and Organization
            addressline={Eagle House, Walton Well Road}, 
            city={Oxford},
            postcode={OX2 6ED}, 
            country={U.K.}}

\affiliation[ucla]{organization={Department of Mathematics, University of California Los Angeles},%Department and Organization
            addressline={UCLA Mathematical Sciences Building, 520 Portola Plaza}, 
            city={Los Angeles},
            postcode={90095}, 
            country={US}}
            
%% Author affiliation
\affiliation[qmmath]{organization={School of Mathematical Sciences, Queen Mary University of London},%Department and Organization
            addressline={Mile End Road}, 
            city={London},
            postcode={E1 4NS},
            country={U.K.}}

%% Author affiliation
\affiliation[nlmathstat]{organization={Department of Mathematics and Statistics, Memorial University of Newfoundland},%Department and Organization
            addressline={230 Elizabeth Avenue}, 
            city={St. John's},
            postcode={A1C 5S7},
            state={NL},
            country={Canada}}


%% Abstract
\begin{abstract}
%% Text of abstract
We tackle the challenges of modeling high-dimensional data sets, particularly those with latent low-dimensional structures hidden within complex, non-linear, and noisy relationships. Our approach enables a seamless integration of concepts from non-parametric regression, factor models, and neural networks for high-dimensional regression. Our approach introduces PCA and Soft PCA layers, which can be embedded at any stage of a neural network architecture, allowing the model to alternate between factor modeling and non-linear transformations. This flexibility makes our method especially effective for processing hierarchical compositional data. We explore ours and other techniques for imposing low-rank structures on neural networks and examine how architectural design impacts model performance. The effectiveness of our method is demonstrated through simulation studies, as well as applications to forecasting future price movements of equity ETF indices and nowcasting with macroeconomic data.  
\end{abstract}

% %%Graphical abstract
% \begin{graphicalabstract}
% %\includegraphics{grabs}
% \end{graphicalabstract}


%% Keywords
\begin{keyword}
%% keywords here, in the form: keyword \sep keyword

%% PACS codes here, in the form: \PACS code \sep code

%% MSC codes here, in the form: \MSC code \sep code
%% or \MSC[2008] code \sep code (2000 is the default)
High-dimensional Regression\sep Generalized Additive Model\sep Factor Model\sep Neural Network\sep Hierarchical Composition Model\sep Generalized PCA
\end{keyword}

\end{frontmatter}

%% Add \usepackage{lineno} before \begin{document} and uncomment 
%% following line to enable line numbers
%% \linenumbers

\section{Introduction}
\label{sec:section1}

In the age of big data, an increasing amount of high-dimensional data is becoming available. Additionally, there is a growing need to make predictions based on a large number of variables \citep{Fan_2014, wainwright2019high}. As an illustration, consider the ImageNet data set \citep{deng2009large}, which consists of approximately $1.5\times10^7$ labeled images with high resolution and a size of around $2 \times 10^5$. Associated with high-dimensional features is the dependence among variables, and many laws of nature and human societies admit certain sparse compositional structures \citep{dahmen2022compositional}, for example, natural language \citep{farkas1986varieties} and gene sequencing data \citep{quinn2018understanding}. In the financial domain, researchers frequently face the challenge of building predictive models from limited data or generating a multitude of linear and non-linear features from low-dimensional data for a regression task \citep{https://doi.org/10.1111/jofi.13298}.

These observations underscore the necessity of tailoring statistical and machine learning models to capture low-dimensional structures within a high-dimensional regime, specifically in the regime $p \gg \log n$. Classical statistical tools, such as additive models \citep{friedman1981projection} or linear factor models used in factor analysis \citep{fruchter1954introduction}, explicitly delineate the structure of the data-generating process. However, relationships between factors representing the sparse structure and the dependent and independent variables, often extend beyond simple linearity. 

Considering the task of forecasting the returns of the S\&P500 index based on prices of its stock constituents, one may expect there exist factors that non-linearly affect the observed variables. \Citet{MCMILLAN2001353} posits a non-linear relationship between interest rates and stock returns. \Citet{reddy2019impact} demonstrate the asymmetric market response to credit rating changes, with more significant reactions to downgrades than upgrades. Similarly, \Citet{bernard1989post} reports a stronger market reaction to negative earnings surprises than positive ones. Additionally, we can also expect factors that have diminishing marginal impact --- fundamental factors like capital expenditures (CapEx) exhibit non-linear relationships with stock returns; initial CapEx investments can significantly enhance performance and competitive advantage, but the benefits diminish with subsequent investments due to saturation effects or strategic misalignments \citep{michelon2020capital}.

These phenomena suggest that although the data is governed by a few latent variables, the association between observed variables and latent factors can be non-linear. The versatility and superior performance of neural networks in handling non-linear patterns and complexities are well-documented \Citep{4341155, doi:10.1080/00031305.1996.10473554, somers2009using}. However, neural networks are data-intensive and may perform unstably in high-dimensional settings \Citep{saarinen1993ill}. This makes it imperative to seamlessly incorporate simple structures into neural network models, enhancing their efficiency while retaining their adaptability to non-linear dynamics. This can be achieved either by imposing low-dimensional structures on the data or on the regression function \citep{lu2023co, zhang2024graph}. In light of this, we will first provide an overview of the key concepts and methods that will be used to better address the high-dimensional regression problem.  

\subsection{Nonparametric Regression}
\label{subsec:nonpara}
Considering a covariate vector $x \in \mathbb{R}^p$ and a response variable $y$, originating from an unspecified distribution $\mu$, our objective revolves around the estimation of the regression function $m^*(\boldsymbol{x}) = \mathbb{E}[y \mid \boldsymbol{x}]$. This function is aimed at minimizing the population $L_2$ risk, defined as 

\begin{equation}
\mathrm{R}(m)=\int(y-m(x))^2 \mu(d x, d y)
\end{equation}

To address the challenges introduced by the \textit{curse of dimensionality}, adopting certain low-dimensional structures for the regression function $m^*(x)$ is a typical regularization strategy. Extensive research has been dedicated to characterizing inherent low-dimensional structures and devising efficient statistical approaches, including interaction models \citep{stone1994use}, and single-index models \citep{hardle1989investigating}. Implementing these low-dimensional frameworks proves beneficial in improving the rate of convergence. For instance, \citet{stone1985additive} applies an additive structure $m^*(x) = \sum_{j=1}^{p} m^*_j(x_j)$, demonstrating that a superior convergence rate of $n^{-2\beta/(2\beta+1)}$ is achievable with univariate functions that are $(\beta;C)$-smooth. In the nonlinear high-dimensional setting, \citet{scheidegger2023spectral} assumed an additive and sparse structure to develop a simple yet practical model.

\subsection{Factor Models}
Factor models are widely used in the analysis of high-dimensional data across various fields such as finance, economics, genomics, and social sciences. Factor analysis was developed by the British psychologist Charles Spearman in the early 20th century to analyze intelligence structures. Spearman's introduction of the general intelligence factor ("g") marked the beginning of factor model applications, showcasing their potential to reveal latent variables influencing observed outcomes \citep{YANAI2006257}. The statistical foundations of factor models were further solidified through efforts to mathematically construct factors for analysis, as seen in the work of \citet{krijnen2002construction}.

The primary motivation for using factor models is rooted in the idea that the observed variables in a data set may share common sources of variation that are not directly measured --- latent factors. Instead of treating each variable independently, factor models aim to capture the joint variability among variables by positing the existence of these latent factors. A linear factor model with covariate \( x \) admits
\begin{equation}
    x = Bz + u, 
    \label{eq:factor}
\end{equation}
where the latent factor \( z \in \mathbb{R}^r \) and the idiosyncratic component \( u \in \mathbb{R}^p \) is unobserved, the factor loading matrix \( B \in \mathbb{R}^{p \times r} \) is fixed but unknown. 

% Some factors influencing asset prices are not directly observable in the price data, but some can be effectively represented by proxies, including GDP growth rates and interest rates that influence the cost of capital, among other macroeconomic variables that reflect the broader state of the financial ecosystem. Models that incorporate these proxies play a crucial role in distilling the essence of market dynamics. They are extensively utilized in the development of risk assessment strategies and in optimizing portfolios, as evidenced by the seminal works of \citep{fama1993common} and \citep{carhart1997persistence}.


The literature on factor models is extensive, with numerous applications across finance. For example, the Fama-French Three-Factor Model extends the Capital Asset Pricing Model (CAPM) by incorporating size and value factors, providing a more comprehensive explanation of stock returns \citep{fama1992cross}. Similarly, the Carhart Four-Factor Model, which adds momentum as an additional factor, is widely used to explain mutual fund performance \citep{carhart1997persistence}. The Arbitrage Pricing Theory (APT), introduced by \citet{chen1986economic}, captures the influence of multiple macroeconomic factors on asset prices.

Recent developments include the asset pricing model proposed by \citet{kelly2019characteristics}, which assumes individual returns follow a 
$K-factor$ structure. Additionally, modern factor models play a critical role in estimating the covariance matrix of asset returns, a key component in portfolio construction and risk management \citep{zhang2023dynamic}. By reducing the dimensionality of the covariance matrix, factor models explain the co-movement of asset returns through a smaller set of common factors, improving estimation efficiency \citep{ledoit2003improved}.

\subsection{Neural Networks}

A neural network is a composition of multiple simple mathematical functions that implement more complex functions. \citet{cybenko1989approximation, hornik1991approximation,  pinkus1999approximation} showed that multi-layer perceptrons (i.e. feed-forward neural networks with at least one hidden layer) can approximate any continuous function, which is referred to as the universal approximation theorem. 

Recent progress in deep learning has led to significant breakthroughs in handling complex challenges, such as object detection from images and speech recognition \citep{lecun2015deep}. These tasks often involve analyzing data with a very high number of features relative to the number of samples available. This scenario underscores the capability of neural networks to manage high-dimensional data effectively. Notably, research by \citet{kohler2021rate} and \citet{schmidt2020nonparametric} has demonstrated that a deep ReLU neural network can adapt to the inherently low-dimensional structure of regression functions. This adaptability enables them to overcome the curse of dimensionality.



\subsection{The Problem Under Study}
Our research draws inspiration from the work of \citet{fan2023factor}, wherein they innovatively integrate factor models with neural network models through the use of diversified projections for estimating latent factor spaces. They leveraged deep ReLU networks for nonparametric factor regression, formulating a Factor Augmented Regression Using Neural Network (FAR-NN) estimator and a Factor Augmented Sparse Throughput Neural Network (FAST-NN) estimator. Their findings demonstrate that FAR-NN and FAST-NN estimators adeptly adjust to the underlying low-dimensional structures by employing hierarchical composition models, achieving nonasymptotic minimax rates.

In this methodology, the factor structure, commonly identified through methods such as PCA, is established in the initial step and subsequently integrated into the neural networks. This one-time factor identification process is particularly suited for observations with a low-dimensional data structure that can be modeled by a linear factor model, as shown in equation \eqref{eq:factor}.

Building on their contributions, our study seeks to broaden the applicability of this framework to high-dimensional datasets that exhibit more complex structures. Through this extension, we aim to uncover new insights and enhance the modeling precision for datasets characterized by intricate relationships and patterns. We posit that within these complex datasets, latent low-dimensional structures exist; however, it is conceivable that these low-dimensional structures might only manifest themselves following some non-linear transformations. We thus specify the observation follows a non-linear factor model with covariate \( x \) admits
\begin{equation}
    x = Bg(z) + u, 
    \label{eq:non-linear_factor}
\end{equation}
where \( g \) is a smooth, approximately monotonic, and non-degenerate non-linear function applied elementwise to \( \mathbf{z} \).
 With such observations, we aim to estimate the regression function
\[
\mathbb{E}[y | z] = m^*(z)
\]
using i.i.d. observations $(x_1, y_1), \ldots, (x_n, y_n)$, and
\begin{equation}
y_i = m^*(z_i) + \varepsilon_i, \quad \mathbb{E}[\varepsilon_i | z_i] = 0 
\label{eq:FAST}
\end{equation}
with i.i.d. noises $\varepsilon_1, \ldots, \varepsilon_n$. We aim to minimize the population $L_2$ error
$
\int ( \hat{m}(z) - m^*(z) )^2 \, \mu(dz)
$ with a focus on the setting that (1) $p$ is relatively high dimensional, and (2) the latent factor dimension $k$ is small.

In addressing this challenge, our approach involves introducing factor and additive structures to neural networks. This innovative methodology is designed to unveil the underlying non-linearity inherent in the data, providing a more comprehensive and interpretable representation of the complex relationships within the high-dimensional space.

\subsection{Our Contributions}
This paper aims to integrate factor model and additive model structures into neural networks for nonparametric regression modeling, with a focus on learning sparse \textbf{compositional structures} in a relatively high-dimensional regime. Our contribution can be summarized as follows:
\smallskip 
\begin{itemize}
    \item We explore Principal Component Analysis (PCA) as a method for factor estimation. In particular, we introduce a \textbf{differentiable PCA layer} that can be seamlessly integrated at any stage of a neural network model. To address the challenge of unstable factor axes, we have developed a training strategy that activates the PCA operation based on a predetermined operating schedule. This approach significantly improves the stability, performance, and efficiency of the PCA layers.

    \item We also introduce a \textbf{Soft PCA} layer designed to retain the variance of the output matrix while preserving a degree of orthogonality. This method integrates smoothly with the training process and does not require a predefined operating schedule.
    
    \item Leveraging the newly introduced factor layers, we propose the Generalized Factor Neural Network Model. Specifically, we present the Generalized Factor Augmented Neural Network (GFANN) and the Generalized Factor Additive Neural Network (GFADNN) architectures, which incorporate both factor layers and additive layers. This design enables neural networks to more effectively process data characterized by hierarchical composition models, as described in \citet{schmidt2020nonparametric}, particularly in high-dimensional settings.

    \item Our proposed Generalized Factor Neural Network Models demonstrate improved performance compared to the FAR-NN and FAST-NN across both simulated and real-world datasets. This advantage is particularly notable in scenarios involving non-linear relationships between latent factors and observed data. Our model's superior capability in capturing and modeling these complexities underscores its potential for broad application and further innovation in neural network design. 
    
\end{itemize} 



\subsection{Notation}
We use bold lower case letter $\boldsymbol{x}=\left(x_1, \ldots, x_d\right)^{\top}$ to represent a $d$-dimension vector, let $\|\boldsymbol{x}\|_q=$ $\left(\sum_{i=1}^d\left|x_i\right|^q\right)^{1 / q}$ be its $\ell_q$ norm, and let $\|x\|_{\infty}=\max _{1 \leq i \leq d}\left|x_i\right|$ be its $\ell_{\infty}$ norm. We use bold upper case $\boldsymbol{A}=\left[A_{i, j}\right]_{i \in[n], j \in[m]}$ to denote a matrix. We define $\|\boldsymbol{A}\|=\sup _{\boldsymbol{x} \in \mathbb{R}^m,\|\boldsymbol{x}\|_2=1}\|\boldsymbol{A} \boldsymbol{x}\|_2$, let $\|\boldsymbol{A}\|_F=$ $\sqrt{\sum_{i, j} A_{i, j}^2}$, and let $\|\boldsymbol{A}\|_{\max }=\max _{i \in[n], j \in[m]}\left|A_{i, j}\right|$. Moreover, we use $\lambda_{\min }(\boldsymbol{A})$ and $\lambda_{\max }(\boldsymbol{A})$ to denote its minimum and maximum eigenvalue respectively. Please refer to table \ref{table:notation_acronyms} in Section \ref{sec:table_notation} for the table of notations and acronyms.




\subsection{Paper structure}
In Section~\ref{sec:section2}, we introduce the foundational models considered in this study. Section~\ref{sec:section3} details the integration of structures derived from factor and additive models into neural networks, leading to the development of PCA, Soft PCA, and Additive Layers as fundamental components. This section also provides the implementation algorithms. Section~\ref{subsec:Model_Architecture} and Section~\ref{subsec:GFADNN} elucidate the rationale behind the construction of our Generalized Factor Neural Network Model. Section~\ref{sec:section4} explores simulation studies and discusses methodologies for conducting fair model comparisons within the inherently stochastic nature of neural network outcomes, even with the same data input. Finally, Section~\ref{sec:section5} presents an empirical study focused on forecasting S\&P 500 ETFs and macroeconomic data sets, demonstrating the practical application of our proposed model.

\section{Model}
\label{sec:section2}
\subsection{Additive Models and Factor Models}

In financial studies, the returns of asset \( X_{j, i} \), for a given asset \( j \), over period \( i \), are influenced by shared factors. \citet{fan2021robust} employ the following model  
\begin{equation}
X_{j, i} = \sum_{m=1}^{d} b_{j, m} Z_{m, i} + u_{j, i},
\end{equation}
where \( j \) ranging from 1 to \( J \) stands for the total number of assets in the portfolio—and \( i \) stands for the portfolio's time horizon from 1 to \( n \). Here, \( d \) is the count of factors that are measured, with \( Z_{m, i} \) signifying the \( m \)-th factor at time \( i \), and \( b_{j, m} \) representing the factor loading for asset \( j \) and factor \( m \). The model specification imposes that the relation between the common factors and returns are linear. By incorporating the additive model mentioned in Section~\ref{subsec:nonpara}, we can allow a more flexible relationship between the asset returns and factors, we assume that the returns evolve through the following model
\begin{equation}
X_{j, i} = f_{j}(Z_i) + u_{j, i},
\end{equation}
where
\begin{equation}
f_{j}(Z_i) := \sum_{m=1}^{d} f_{j, m}(Z_{m, i}),
\end{equation}
and \( f_{j} \) denotes the true but unobservable function that links the factor matrix \( Z_i = (Z_{1, i}, \ldots, Z_{m, i}, \ldots, Z_{d, i})^\prime \), a \( d \times 1 \) vector, with each asset's returns. The function \( f_{j, m} \) is the unknown function for the \( m \)-th factor during period \( i \). Different assets can link to common factors through different functional forms. Compared to the strict conventional linear factor models, this model specification increases the flexibility in the relationship between the factors and the assets.

% \subsection{High-dimensional augmented sparse nonparametric regression}

\subsection{ReLU Neural Networks}

We construct our model using a fully-connected deep neural network with ReLU activation $\sigma(\cdot) = \max\{\cdot, 0\}$, chosen for its empirical success. We refer to this as a deep ReLU network for brevity. Let $L \in \mathbb{N}$ be the depth of the network and $\boldsymbol{d} = (d_1, \ldots, d_{L+1}) \in \mathbb{N}^{L+1}$ define the dimensions of each layer. Following \citet{fan2023factor}'s definition, a deep ReLU network is a function mapping $\mathbb{R}^{d_0} \to \mathbb{R}^{d_{L+1}}$, expressed as
\begin{equation}
g(x) = \mathcal{L}_{L+1} \circ \bar{\sigma} \circ \mathcal{L}_L \circ \bar{\sigma} \circ \cdots \circ \mathcal{L}_2 \circ \bar{\sigma} \circ \mathcal{L}_1(x),
\label{eq:reluNN}
\end{equation}
where $\mathcal{L}_{\ell}(\boldsymbol{z}) = \boldsymbol{W}_{\ell} \boldsymbol{z} + \boldsymbol{b}_{\ell}$ represents an affine transformation with weight matrix $\boldsymbol{W}_{\ell} \in \mathbb{R}^{d_{\ell} \times d_{\ell-1}}$ and bias vector $\boldsymbol{b}_{\ell} \in \mathbb{R}^{d_{\ell}}$. The operator $\bar{\sigma}: \mathbb{R}^{d_{\ell}} \to \mathbb{R}^{d_{\ell}}$ applies the ReLU activation function elementwise to a $d_{\ell}$-dimensional vector. For simplicity, we refer to both $\boldsymbol{W}_{\ell}$ and $\boldsymbol{b}_{\ell}$ as the network's weights.

The family of deep ReLU networks, truncated at level $M$ with depth $L$, width parameter $\boldsymbol{d}$, and weights bounded by $B$, is defined as
\begin{equation}
\mathcal{G}(L, \boldsymbol{d}, M, B) = \left\{\widetilde{g}(\boldsymbol{x}) = \bar{T}_M(g(\boldsymbol{x})) : g \text{ of form \eqref{eq:reluNN} with } 
\|\boldsymbol{W}_{\ell}\|_{\max} \leq B, \|\boldsymbol{b}_{\ell}\|_{\max} \leq B\right\},
\end{equation}
where $\bar{T}_M(\cdot)$ is the truncation operator at level $M$ applied element-wise to a $d_{L+1}$-dimensional vector, more specifically defined as 
\[
\left[\bar{T}_M(z)\right]_i = \operatorname{sgn}(z_i)\min\{|z_i|, M\}.
\]
If the width parameter is given as $\boldsymbol{d} = (d_{\text{in}}, N, N, \ldots, N, d_{\text{out}})$, we denote it as $\mathcal{G}(L, d_{\text{in}}, d_{\text{out}}, N, M, B)$, referring to a deep ReLU network with depth $L$ and width $N$ for brevity.

For practical implementation, LeakyReLU is selected as the activation function due to its ability to mitigate the 'dying ReLU' issue \citet{maas2013rectifier}. By permitting a minor gradient even when the neuron is inactive, LeakyReLU ensures the continuity of gradient flow during the entire training cycle. This characteristic not only potentially accelerates the convergence rate, but also fosters more uniform learning across deep neural networks.



\section{Methodology}
\label{sec:section3}
In this section, we delineate the methodology underpinning our model for high-dimensional regression, focusing on two complementary approaches. First, we incorporate factor structure into the neural network using the standard PCA Layer and the novel Soft PCA Layer. Second, we introduce an additive structure through the Additive Layer to impose a low-dimensional structure on the regression function.

\subsection{Differentiable PCA Layer}
The work by \Citet{fan2023factor} introduced a model incorporating a diversified projection matrix prior to the neural network layer, estimated through Principal Component Analysis (PCA). However, their approach restricts PCA to the initial step of the model. When the data does not follow a linear factor model or exhibits a hierarchical factor structure, alternating between factor modeling and non-linear transformations becomes essential \citep{bengio2013representation}. Even when the data follows a linear factor structure, performing PCA multiple times at intermediate layers, as illustrated in Figure \ref{fig:pca_pca_nn}, can help refine feature representations, reduce redundancy, and mitigate noise accumulation, thereby improving training stability and representational efficiency \citep{goodfellow2016deep}. This necessitates the ability to insert the PCA operation at any stage of a neural network architecture, enhancing its adaptability and effectiveness in high-dimensional settings \citep{fan2014challenges}.

A significant challenge of this approach lies in ensuring the differentiability of the PCA operation while maintaining end-to-end training stability. This can be difficult to achieve if the PCA operation causes drastic changes to weight parameters after processing each batch of data. To address this challenge, we introduce methods to stabilize and seamlessly integrate PCA within the training process. Specifically, we implement PCA operations using differentiable eigenvector computations. The accompanying pseudo-code is provided in Algorithm 1, serving as a procedural guide. For more intricate technicalities, we refer to \textit{On differentiating eigenvalues and eigenvectors} by \citet{magnus1985differentiating}. It is imperative to note that the gradient can become numerically unstable when the disparity between any two eigenvalues approaches nullity, as it is contingent upon the eigenvalues \( \lambda \) through the computation of \( \frac{1}{\min_{i \neq j} (\lambda_i - \lambda_j)} \). To circumvent this, our algorithm instates a stability criterion that ascertains whether the minimal difference among the eigenvalues surpasses a predefined threshold. Should this not be the case, we introduce a perturbation by adding a scalar multiple of random values, scaled according to the magnitude of diagonal values, to the covariance matrix to enhance stability.

\begin{figure}[H]
\centering
    \includegraphics[width=\columnwidth]{images/Factor_Augmented_Neural_Networks_Composite_Model.png}
    \caption{A visual illustration of integrating multiple PCA layers into neural networks. The layers in red denote the PCA layers. }  
\label{fig:pca_pca_nn}
\end{figure}

\begin{algorithm}[H]
   \caption{PCA Layer}
   \label{alg:Factor Layer}
\begin{algorithmic}[1]  % The [1] ensures that lines are numbered
   \State {\bfseries Input:} data $A$, shape $n \times p$
   \State {\bfseries Output:} data $B$, shape $n \times k$
   \State Initialize placeholder $C$, shape $p \times k$
   \If{$\textit{initializing}$}
      \State $covMatrix \gets \text{cov}(A^T)$
      \State $eigenvalues, eigenvectors \gets \text{eigen}(covMatrix)$
      \Comment{Stability check}
      \While{$\min(eigenvalues.\text{diff}()) \leq 10^{-10}$}
         \State \# add noise to the cov matrix
         \State $covMatrix \mathrel{{+}{=}} 10^{-4} \times \text{mean}(\text{diag}(covMatrix)) \times \text{randLike}(covMatrix)$
         \State $eigenvalues, eigenvectors \gets \text{eigen}(covMatrix)$
      \EndWhile
      \State $C \gets eigenvectors[:,:k] / sqrt(p)$
      % \State $projection \gets eigenvectors$
      % \State $(factor, projection) \gets \text{pca}(A, C)$
      \State $factor \gets A \cdot C$
      
      \State \textbf{return} $factor$
   \Else
      \State $factor \gets A \cdot C$
      \State \textbf{return} $factor$
   \EndIf
\end{algorithmic}
\end{algorithm}


\subsubsection{Unstable Factor Axes}
However, when training a neural network model with multiple PCA layers as shown in Figure 1, we observed a markedly higher test error relative to vanilla neural networks. This phenomenon appears rooted in the instability of factor axes. Conceptually, PCA is akin to fitting a p-dimensional ellipsoid around the data, with each principal component acting as an ellipsoid axis. These axes, set sequentially based on the direction of maximum variance in projections, are susceptible to alteration with varying data batches and evolving input during training. Figure 3 illustrates the training progression within one of our simulated data experiments. It reveals notable instability in both training and validation error, with the test error also exhibiting inferior performance compared to that of a standard neural network.

\begin{figure}[H]
\centering
    \includegraphics[width=10cm ]{images/train_valid_loss_no_schedule.PNG}
    \caption{Training and validation error when training a Neural Network model with a PCA layer in the middle.}
\end{figure}

To address this instability, we propose an operating schedule for the model. This approach enables PCA decomposition exclusively during initialization (i.e., when $initializing==True$). This allows us to reduce the number of PCA operations and hence reduce noise introduced by the unstable factor axes during training. Intuitively, with each PCA update, the model requires a certain number of epochs to align its weights. As the model attains stability, the need for frequent PCA recalibrations diminishes. Consequently, a tailored operating schedule that incrementally decreases the PCA operations ensures stable factor axes while accommodating model evolution. Figure \ref{fig:train_valid_loss_schedule} presents the model's training process with an operating schedule, applying PCA operations at the initial batch of each specified epoch—namely epochs [1, 3, 5, 7, 9, 11, 13, 15, 20, 30, 40]. A rapid stabilization of both training and validation errors can be observed. The refinement not only enhances test results but also markedly decreases computational complexity.

\begin{figure}[H]
\centering
    \includegraphics[width=10cm ]{images/train_valid_loss_schedule.PNG}
    \caption{Training and validation error when training a Neural Network model with a PCA layer in the middle, with an operating schedule.}
    \label{fig:train_valid_loss_schedule}
\end{figure}

\subsubsection{Monitoring Explained Variance}
In typical applications, a PCA layer serves to reduce the dimensionality of the input data matrix from $p$ dimensions to $k$. It also enables us to monitor the variance explained throughout the training process. Figure~\ref{fig:monitorvariance} depicts the training process of the neural network model, with an operating schedule spanning epochs 1 through 40. The green line illustrates the percentage of variance explained by the first $k$ principal components. A discernible increasing trend in the percentage of variance explained is observed, and then it stabilizes, coinciding with the stabilization of the validation error. This trend not only sheds light on the sufficiency of the chosen width $k$ in capturing the majority of the variance, but also offers insights into the linear interpretability of the input matrix.

\begin{figure}[H]
\centering
    \includegraphics[width=10cm ]{images/pct_variance_explained_with_train_valid.PNG}
    \caption{Training and Validation error with an operating schedule of [1,2,...40] and the corresponding percentage variance explained by the first $k$ components}
    \label{fig:monitorvariance}
\end{figure}

\subsubsection{Limitations and Challenges}
Non-adaptive PCA layers are inherently rigid, requiring precomputed transformations or periodic updates based on fixed schedules. This rigidity becomes particularly limiting in dynamic or nonstationary data environments, where the statistical properties of the data evolve over time. Moreover, PCA-based factor estimation is highly sensitive to eigenvalue distributions; when eigenvalues are close in magnitude, small perturbations in the data can lead to instability. While perturbation methods can mitigate this issue, they introduce additional computational complexity. Additionally, freezing factor weights after initial training preserves exogeneity but limits adaptability when the data distribution shifts

\subsection{Soft PCA Layer}
In this subsection, we introduce the \textbf{Soft PCA Layer}, a novel neural network layer designed to approximate the effects of PCA without relying on a predefined operating schedule. Unlike traditional PCA, which relies on deterministic eigendecomposition, the Soft PCA Layer integrates PCA-like behavior into the learning process, enabling it to adapt dynamically during training. The Soft PCA Layer addresses above-mentioned challenges by embedding PCA-inspired objectives into a fully trainable framework, allowing it to:
\begin{itemize}
    \item Adapt dynamically to evolving data representations across network layers.
    \item Eliminate the need for predefined update schedules, reducing manual intervention.
    \item Achieve PCA-like variance maximization and orthogonality without explicit eigendecomposition.
\end{itemize}

\subsubsection{Design and Objectives}
The Soft PCA Layer achieves its objectives through two key components:

\textit{1. Variance Maximization Objective}: Traditional PCA maximizes variance along principal components. To approximate this behavior, we minimize the loss function
\[
\mathcal{L}_{\text{variance}} = \| \text{Var}(\mathbf{X}_{\text{in}}) - \text{Var}(\mathbf{X}_{\text{out}}) \|_F^2,
\]
where \( \mathbf{X}_{\text{in}} \) and \( \mathbf{X}_{\text{out}} \) represent the input and output of the layer, respectively, and \( \| \cdot \|_F \) is the Frobenius norm. This ensures that the transformation captures the most significant variations in the input.

\textit{2. Orthogonality Constraint}: To encourage outputs resembling the orthogonal projections of PCA, we impose a loss term on the covariance matrix \( \Sigma_{\text{out}} \) of the layer outputs
\[
\mathcal{L}_{\text{orthogonality}} = \| \Sigma_{\text{out}} - \text{diag}(\Sigma_{\text{out}}) \|_F^2,
\]
where \( \text{diag}(\Sigma_{\text{out}}) \) is the diagonal of the covariance matrix. By minimizing off-diagonal elements, the layer approximates the orthogonality of PCA's principal components.

\subsubsection{Integration into Neural Networks}
The Soft PCA Layer is implemented as a fully differentiable layer, making it compatible with standard backpropagation. Unlike the traditional PCA preprocessing step, it operates as a dynamic and trainable transformation embedded within the neural network. This flexibility enables its integration at any stage of the model, much like the standard PCA Layer described earlier.

\subsubsection{Comparison with Standard PCA Layer and Autoencoders}
While the standard PCA Layer provides deterministic dimensionality reduction and variance decomposition, it requires periodic updates or fixed preprocessing. In contrast, the Soft PCA Layer offers
\begin{itemize}
    \item \textbf{Adaptability}: It adapts to evolving feature representations during training.
    \item \textbf{End-to-End Learnability}: Integrated into the model, it eliminates the need for separate preprocessing.
    \item \textbf{Computational Efficiency}: Avoids the computational cost of explicit eigendecomposition during training.
\end{itemize}

In comparison to autoencoders, the Soft PCA Layer excels in scenarios involving high-dimensional data with low-rank structure due to its built-in orthogonality constraints and variance maximization objective. These properties make the Soft PCA Layer particularly suitable for extracting uncorrelated features and preserving the most significant variations in the data, aligning closely with the goals of PCA. While autoencoders are powerful for nonlinear feature learning, they introduce additional parameters and lack explicit decorrelation objectives, making them less ideal for problems where interpretability and efficient handling of low-rank structures are critical.

The flexibility and adaptability of the Soft PCA Layer make it particularly suitable for scenarios where data characteristics or feature representations evolve, such as in high-dimensional regression tasks or dynamic systems modeling.

% However, the following limitations still remain:
% \paragraph{Dependence on the Number of Factors (\(\bar{k}\))}
% The theory suggests that choosing \(\bar{k} \geq k\) ensures a good approximation of the factor space. However, determining the \textbf{optimal number of factors} remains challenging. Selecting an overly large \(\bar{k}\) may introduce unnecessary noise, while selecting a too-small \(\bar{k}\) may result in missing important factor structures.

% \paragraph{Lack of Interaction Modeling Among Latent Variables}
% Both the PCA and Soft PCA layers primarily perform \textbf{linear factor extraction} and do not inherently capture interaction effects among latent variables. In particular, while Soft PCA enforces an orthogonality constraint on the factor representations, it does not model direct dependencies or hierarchical interactions between factors. In contrast, methods such as deep autoencoders or tensor decomposition techniques can explicitly capture complex interactions within the latent space.



\subsection{Factor Estimation via Diversified Projection Matrix}
We introduce the idea of a \textit{diversified projection matrix} proposed by \citet{fan2022learning}, in order to provide more intuition for the frozen factor layer. 
\begin{definition}[Diversified projection matrix]
Let $\bar{k} \ge k$, and $c_1$ be a universal positive constant. A $p \times \bar{k}$ matrix $\mathbf{W}$ is said to be a diversified projection matrix if it satisfies
\begin{enumerate}
    \item \textbf{(Boundedness)} $\|\mathbf{W}\|_{\max} \le c_1$.
    \item \textbf{(Exogeneity)} $\mathbf{W}$ is independent of $x_1, \dots, x_n$ in \eqref{eq:non-linear_factor}.
    \item \textbf{(Significance)} The matrix $\mathbf{H} = p^{-1} \mathbf{W}^\top \mathbf{B} \in \mathbb{R}^{\bar{k} \times k}$ satisfies $\nu_{\min}(\mathbf{H}) \gg p^{-1/2}$.
\end{enumerate}
Each column of \( \mathbf{W} \) is referred to as a diversified weight, with \( \bar{k} \) representing the total number of diversified weights.
\end{definition}

The core concept behind the \textit{diversified projection matrix} is that we can construe
\[
\tilde{\mathbf{z}} = p^{-1} \mathbf{W}^\top \mathbf{x}
\]
as a proxy for the factor \( \mathbf{z} \) in downstream predictions, even when \( \bar{k} > k \), effectively overestimating the number of factors. To see why this works, we can substitute equation \eqref{eq:non-linear_factor} into the expression above, yielding
\begin{equation}
\tilde{\mathbf{z}} = \mathbf{H} g(\mathbf{z}) + \bm{\xi}, \quad \text{where } \bm{\xi} = p^{-1} \mathbf{W}^\top \mathbf{u}.
\label{eq:decompose_proj}
\end{equation}

This decomposition shows that, under mild conditions, \( \tilde{\mathbf{z}} \) provides a good estimate of an affine transformation of \( g(\mathbf{z}) \). The intuition here is that if the idiosyncratic component \( \mathbf{u} \) has weak dependence and uniformly bounded second moments, then \( \|\bm{\xi}\| = O_P(p^{-1/2}) \) due to the bounded variance. Meanwhile, due to the \textit{significance} condition and the presence of non-degenerate factors, the signal term satisfies
\[
\| \mathbf{H} g(\mathbf{z}) \|_2 \ge \nu_{\min}(\mathbf{H}) \| \mathbf{z} \|_2 \gg \| \bm{\xi} \|_2,
\]
implying that \( \mathbf{H} g(\mathbf{z}) \) is the dominant term in the decomposition \eqref{eq:decompose_proj}.

We can intuitively interpret the diversified projection matrix \( \mathbf{W} \) as an ``overestimate" of the factor loading matrix \( \mathbf{B} \). The term ``over-" indicates that it is not necessary to precisely determine the number of factors \( k \) in our framework; instead, we can simply choose a sufficiently large \( \bar{k} \). 

Moreover, the \textit{significance}  condition requires that the choice of \( \mathbf{W} \) be more informed than a random guess. For instance, for an \( n \times n \) random matrix \( \mathbf{X} \) with i.i.d. Gaussian entries, the smallest singular value \( \sigma_{\min}(\mathbf{X}) \) scales as \( O(n^{-1/2}) \). This does not satisfy the \textit{significance} condition for a diversified projection matrix. 

In the case of our (Soft) PCA layer, one may clip the weights to ensure we satisy the \textit{boundedness}  condition, although, in practice, the initial weight values also generally meet this requirement. Regarding the \textit{exogeneity} condition, \citet{fan2023factor} documented that when the weights are trained jointly with other layers (and thus become dependent on \( x_1, \dots, x_n \)), performance deteriorates significantly. A potential approach to maintain exogeneity is to estimate \( \mathbf{W} \) using an independent set of observations, separate from \( x_1, \dots, x_n \). However, in practice, we opt to freeze the weights after a few epochs, which prevents them from further adapting to \( x_1, \dots, x_n \).

Lastly, the (Soft) PCA layer satisfies the \textit{significance} condition, as it is either orthogonal due to the PCA operation or approximately orthogonal in the Soft PCA case. When a matrix has approximately orthogonal columns, its singular values are generally well-distributed and bounded away from zero. Additionally, because the layer approximates the projection from \( \mathbf{x} \) to \( \mathbf{g(z)} \), it aligns with \( \mathbf{B} \) to a certain extent after the initial training stage. This alignment prevents any particular direction from collapsing when multiplied by \( \mathbf{B} \), thereby preserving the rank and significance of \( \mathbf{H} \).


\subsection{Additive Layer}
In the next phase of our pipeline, we focus on integrating an additive structure into the neural network. This approach involves isolating specific sections of the previous layer to form sub-networks and combining them solely through addition. Consider a scenario where we have identified latent factors $X$ with dimensions $n \times p$ following a PCA layer. 

Rather than forwarding this output to subsequent fully connected layers or the final output layer, we apply the additive structure as shown in Algorithm \ref{alg: Additive_Layer}: we divide $X$ column-wise into segments $[X_{p_1}, X_{p_2}, ..., X_{p_j}]$. Each segment $g(X)_{p_i}$ is characterized by the dimensions $n \times p_i$, ensuring that the sum of all $p_i$ equals $p$ (i.e., $\sum^j_i p_i = p$). Each sub-matrix $X_{p_i}$ is then fed into a linear layer of size $p_i \times q_i$. Consequently, the output from each linear layer retains the shape $n \times q_i$. Then we horizontally concatenate the output matrices, resulting in a matrix $[X_{q_1}, X_{q_2}, ..., X_{q_j}]$ of shape $n \times q$, where $\sum^j_i q_i = q$. When constructing the layer,we need to specify the $input\_dimension\_list\ [p_1, p_2..p_i]$ and  $output\_dimension\_list\ [q_1, q_2..q_j]$ By aligning the $input\_dimension\_list$ of a subsequent Additive Layer with the $output\_dimension\_list$ of the preceding one, (i.e. the $input\_dimension\_list$ of the subsequent Additive Layer is $[q_1, q_2..q_j]$) we ensure that there is no interaction among the initial column segments $[X_{p_1}, X_{p_2}, ..., X_{p_j}$. Through one or more such Additive Layers, our neural network can be effectively segmented into multiple sub-networks. In the final step, we aggregate these sub-networks, forming the additive structure. Figure \ref{fig:additive_layer} in Section \ref{sec:additive_layer_graph} provides an illustration of the additive layers.

% \begin{figure}[H]
%     \centering
%     \includegraphics[width=\columnwidth]{images/additive_layer.PNG}
%     \caption{Additive networks, without interactions among neurons across different sub-networks.}
% \label{fig:additive_layer}
% \end{figure}

\begin{algorithm}[H]
   \caption{Additive Layer}
   \label{alg: Additive_Layer}
\begin{algorithmic}
   \State {\bfseries Input:} data $A$, shape $n \times p$; $input\_dim\_list = [p_1, p_2, \ldots]$ where $\sum^j_{i=1} p_i = p$
   \State {\bfseries Output:} data $B$, shape $n \times q$; $output\_dim\_list = [q_1, q_2, \ldots]$ where $\sum^j_{i=1} q_i = q$
   \State $n\_subNN \gets \text{len}(input\_dim\_list)$
   \State $subNN\_idx \gets [0] \cup \text{cumsum}(input\_dim\_list)$ \Comment{Indices to partition $A$}
   % \State $layer\_list \gets [\text{linear\_layer}(input\_dim\_list[i], output\_dim\_list[i]) \ \text{for} \ i \ \text{in} \ \text{range}(n\_subNN)]$
   % \State $output\_list \gets [\ layer\_list[i](A[:, subNN\_idx[i]:subNN\_idx[i+1]]) \ \text{for} \ i \ \text{in} \ \text{range}(n\_subNN)]$
    \State $layer\_list \gets [$
    $\text{linear\_layer}(input\_dim\_list[I], output\_dim\_list[i])$
    \State \hspace{2.5cm} $\text{for} \ i \ \text{in} \ \text{range}(n\_subNN)]$
    
    \State $output\_list \gets [$
    $layer\_list[i](A[:,$
    $subNN\_idx[i]:subNN\_idx[i+1]])$
    \State \hspace{2.5cm} $\text{for} \ i \ \text{in} \ \text{range}(n\_subNN)]$
   \State $B \gets \text{concat}(output\_list)$
   \State \textbf{return} $B$
\end{algorithmic}
\end{algorithm}



\subsection{Model Architecture}
\label{subsec:Model_Architecture}

We can now present some intuition on how to design the model architectures based on the data-generating process with an underlying low-dimensional structure. Assuming that we have observable $n\times 1$ target $Y$, and high-dimensional $n\times p$ features $X$ and there exist $n\times k$ latent factors $Z$ where $k << p$. In a linear setting, the observations have a factor structure in a way that $X=BZ+\epsilon$, where $B$ is the loading matrix and $\epsilon$ is the noise term. We specify the target $y=f(Z)+\mu$ to follow an additive structure: $y_i=\sum_{j \in \mathcal{J}} f_j\left(z_{i, j}\right)+\varepsilon_i \quad$ for $\quad i=1, \ldots, n$. In this data-generating process, the latent factors $Z$ bridges observations $X$ and the target $Y$ with $Z-X$ structure and $Z-Y$ structure. We can further add some non-linearity to the generating process by passing $Z$ through some simple non-linear functions $g$ like $B \times g_1\left(\left[z_1, z_2, z_3\right]\right)$. 
Under this more general setting, our model needs to pass the input through some layers to approximate $g$ and then pass to a Factor Layer to perform factor decomposition. Next, since we assume $Z-Y$ structure to be additive, we need to pass the data to multiple Additive Layers so that each subnetwork formed can approximate the non-linear $f_i$ function. Following this intuition, we can further model higher-order/hierarchical additive factor models.\\\\


\subsubsection{Generalized Factor Additive Neural Network Model}  
\label{subsec:GFADNN}

\begin{figure}[H]
\centering
\includegraphics[width=\columnwidth]{images/PCA_NN_PCA_ADD.png}
\vspace{1cm} 

\caption{An illustration of the General Factor Additive Neural Network Model. The red and blue layers represent generalized factor layers, while the final layers are additive layers. The input to the additive layers is already (partially) decorrelated before being passed through them.}
\label{fig:FADNN}
\end{figure}



The rationale for integrating fully-connected layers between the PCA layer and the Additive layers stems from the need to address potential non-linearity between the latent factors and the observations. However, considering that a primary motivation for employing Additive layers is to segregate information flows, the introduction of fully-connected layers might counteract this by amalgamating the orthogonal outputs derived from the PCA layers. Therefore, to preserve the discrete transfer of information to the Additive layers, it is preferable to limit neuron interactions, which can be achieved by directly connecting the Additive layers to the PCA layer.

% Incorporating Independent Component Analysis (ICA) could also be considered for enhancing this process.

\subsubsection{Generalized Factor Augmented Neural Network Model}  
\label{subsec:GFANN}
Following FAST-NN estimator, we propose the Generalized  Factor Augmented Neural Network (GFANN) Model. Recall the non-linear factor model specification in equation \eqref{eq:non-linear_factor}. With this latent factor structure, it is natural to consider a factor-augmented regression model where $z$ and $u$ serve as regressors. This approach is equivalent to using both $x$ and latent factors $z$ as regressors, but the former formulation results in weaker dependence among variables. We can then formulate the regression as such:
\[
\mathbb{E}[y | z, \mathbf{u}] = m^*(z, u_{\mathcal{J}}),
\]
where $\mathcal{J} \subset \{1, 2, \ldots, p\}$ is an unknown subset of indexes.In the GFANN model, the input matrix \( \mathbf{X} \in \mathbb{R}^{n \times p} \) is first passed through a layer that acts as a projection matrix \( \mathbf{P} \in \mathbb{R}^{p \times k} \) to produce the output \( o_1 \). Next, the residual matrix \( \mathbf{u} \in \mathbb{R}^{n \times p} \) is computed as \( \mathbf{u} = \mathbf{X} - \mathbf{P} \mathbf{P}^T \mathbf{X} \). This residual is then passed through a variable selection matrix \( \mathbf{Q} \in \mathbb{R}^{p \times m} \) to obtain the output \( o_2 \). Finally, \( o_1 \) and \( o_2 \) are concatenated and passed to the subsequent network layers. The Figure \ref{fig:GFANN} below illustrates the network architecture.

\begin{figure}[H]
\centering
\includegraphics[width=1.2\columnwidth]{images/GFANN_compact.png}
\caption{Visualization of the Generalized Factor Augmented Neural Network Model. Layers in red and blue are factor layers}
\label{fig:GFANN}
\end{figure}


\newpage
\section{Simulation Studies}
\label{sec:section4}
\subsection{Data Generating Process}

\begin{table}[H]
\centering
\begin{tabular}{|l|l|l|} 
\hline
Observations (Z-X structure) & Target (Z-Y structure)\\
\hline
{1. $B \times \left[z_1, z_2, z_3\right]$} & 1. $ f_1\left(z_1\right)+f_2\left(z_2\right)+f_3\left(z_3\right)$\\
\hline
{2. $B \times g_1\left(\left[z_1, z_2, z_3\right]\right)$ }  &  1. $ f_1\left(z_1\right)+f_2\left(z_2\right)+f_3\left(z_3\right)$ \\
\hline
\end{tabular}
\label{tab:Variations of data structuress} % For referencing the table in text
\vspace{5pt}
\caption{Variations of data structures of observations and the target}
\end{table}

For the data-generating process, we follow the basic setup in \citet{fan2023factor}. For the $Z-X$ structure, we first generate $k=5$ independent latent factors, which follow Uniform$[-1,1]$, so the shape of the latent factor matrix $Z$ is $n$ by 5. The factor loading matrix $B$ has i.i.d. Unif $[-\sqrt{3}, \sqrt{3}]$ entries, and the shape is 5 by $p$. To generate observations, for case 1 (obs-id 1), we multiply $Z$ and $B$, then add idiosyncratic components $\boldsymbol{u}$, which are independent and have i.i.d. Unif $[-1,1]$ entries; for case 2 (obs-id 2), we pass $Z$ to a non-linear function $g$ first, where $g$ assumed to be some intuitive non-linear functions, in this case $exp$, since it is common and have economic intuition. To get the observation matrix $X$, we multiply $g(Z)$ by $B$ then add idiosyncratic components $\boldsymbol{u}$. To generate the target $Y$, in case 1 (target-id 1), we pass each latent factor $z_i$ to a non-linear function $f_i$, where $f_i$ are selected randomly from the candidate function set $\left\{\cos (\pi x), \sin (x),(1-|x|)^2, 1 /\left(1+e^{-x}\right), 2 \sqrt{|x|}-1, x^2\right\},$ in each trial. Finally, we add an independent noise term $\varepsilon$ which follows a zero-mean Gaussian distribution with a variance of $\sigma_\varepsilon=0.3$. For a high-dimensional setup, we will use $n_{\text {train }}=500$ i.i.d. samples from the above data-generating process to train our neural network. We also use other $n_{\text {valid }}=150$ i.i.d. observations as a validation data set for model selection and $n_{\text {valid }}=10000$ i.i.d. observations as a test data set.
\subsection{Model Comparison}
There is a substantial body of literature that compares neural network models using identical hyperparameter settings. However, hyperparameter settings are highly specific to each model, and therefore, for a fair comparison, we aim to select the best hyperparameter setting for each individual model under multiple random seeds. For training details, please refer to Section \ref{sec:model_comparison}. At this stage, the following models are selected for comparison: 
\begin{itemize}
    \item A vanilla Relu-Neural network model with \textbf{latent factors} as input (oracleNN).
    \smallskip 
    \item A vanilla Relu-Neural network model (vanillaNN).
    \smallskip 
    \item FAR-NN estimator \citep{fan2023factor} that motivated this research ($ FAR-NN$).
    \smallskip 
    \item A Generalized Factor Neural Network with Soft PCA as the first layer ($SPCA\_NN$) for the linear case and the second layer  ($NN\_SPCA\_NN$) for the non-linear case.
    \smallskip 
    \item A GFADNN model with PCA layers ($PCA\_NN\_PCA\_ADD$).
    \smallskip 
    \item A GFADNN model with soft PCA layers ($SPCA\_NN\_SPCA\_ADD$).
    \smallskip 
\end{itemize}
\subsection{Results}


\begin{table}[h!]
\centering
\begin{tabular}{|c|c|l|c|}
\hline
\textbf{obs-id} & \textbf{target-id} & \textbf{model} & \textbf{test mse} \\ \hline
1 & 1 & SPCA\_NN\_SPCA\_ADD & 0.038 \\ \hline
1 & 1 & PCA\_NN\_PCA\_ADD & 0.052 \\ \hline
1 & 1 & oracleNN & 0.080 \\ \hline
1 & 1 & FAR-NN & 0.089 \\ \hline
1 & 1 & SPCA\_NN & 0.093 \\ \hline
1 & 1 & autoencoder & 0.230 \\ \hline
1 & 1 & vanillaNN & 0.539 \\ \hline
1 & 1 & lasso & 1.507 \\ \hline
1 & 1 & pcr & 2.138 \\ \hline
2 & 1 & PCA\_NN\_PCA\_ADD & 0.065 \\ \hline
2 & 1 & SPCA\_NN\_SPCA\_ADD & 0.078 \\ \hline
2 & 1 & oracleNN & 0.080 \\ \hline
2 & 1 & FAR-NN & 0.103 \\ \hline
2 & 1 & NN\_SPCA\_NN & 0.151 \\ \hline
2 & 1 & autoencoder & 0.455 \\ \hline
2 & 1 & vanillaNN & 0.722 \\ \hline
2 & 1 & lasso & 1.444 \\ \hline
2 & 1 & pcr & 1.511 \\ \hline
\end{tabular}
\vspace{5pt}
\caption{Performance table with input dimension $p=500$, for two observation scenarios (obs-id 1 and 2).}
\label{table:test_mse_combined}
\end{table}


When the observations take the form \( B \times \left[z_1, z_2, z_3\right] \), $obs-id 1$, the relationship between the observations and latent variables is linear. Model performance is evaluated based on test mean squared error (MSE), with no noise term included in the test data set. Starting from an input dimension of \( p = 500 \), our GFADNN models consistently outperform the other models, with the \( SPCA\_NN\_SPCA\_ADD \) model improving the test MSE of \( FAR\_NN \) by 57.3\%.

When the observations take the form \( B \times \exp\left(\left[z_1, z_2, z_3\right]\right) \), $obs-id 2$, introducing non-linearity into the relationship between observations and latent variables, our GFADNN models again consistently outperform all others. In this case, the \( PCA\_NN\_PCA\_ADD \) model improves the test MSE of \( FAR-NN \) by 36.9\%.
\newpage


\subsection{Performance against input dimension}



\begin{figure}[H]
% \captionsetup{labelformat=empty} 
\centering

% The first plot
\begin{subfigure} % Specify width as a fraction of text width
    \centering
    % \captionsetup{labelformat=empty}
    \includegraphics[width=0.95\textwidth]{images/ds02_table1.png}
    \caption*{(a)}
    \label{fig:ds02_table1}
\end{subfigure}
\hfill % Adds space between subfigures

% The second plot
\begin{subfigure}% Specify width as a fraction of text width
    \centering
    % \captionsetup{labelformat=empty}
    \includegraphics[width=0.95\textwidth]{images/ds02_table2.png}
    \caption*{(b)}
    \label{fig:ds02_table2}
\end{subfigure}
\caption{Dataset 1-1: test MSE against dimension. Figure (b) provides a zoomed-in view of the top-performing models.}
\label{fig:ds02_table}
\end{figure}

% \begin{figure}[H]
%     \centering
%     \includegraphics[width=\columnwidth]{images/ds02_table1.png}
%     \caption{Dataset 1-1: test MSE against dimension}
%     \label{fig:ds02_table1}
% \end{figure}

% \begin{figure}[H]
%     \centering
%     \includegraphics[width=\columnwidth]{images/ds02_table2.png}
%     \caption{Dataset 1-1: test MSE against dimension, a zoomed-in view of the top-performing models.}
%     \label{fig:ds02_table2}
% \end{figure}

\begin{figure}[H]
% \captionsetup{labelformat=empty} 
\centering

% The first plot
\begin{subfigure} % Specify width as a fraction of text width
    \centering
    % \captionsetup{labelformat=empty}
    \includegraphics[width=0.95\textwidth]{images/ds12_table1.png}
    \caption*{(a)}
    \label{fig:ds12_table1}
\end{subfigure}
\hfill % Adds space between subfigures

% The second plot
\begin{subfigure}% Specify width as a fraction of text width
    \centering
    % \captionsetup{labelformat=empty}
    \includegraphics[width=0.95\textwidth]{images/ds12_table2.png}
    \caption*{(b)}
    \label{fig:ds12_table2}
\end{subfigure}
\caption{Dataset 2-1: test MSE against dimension. Figure (b) provides a zoomed-in view of the top-performing models.}
\label{fig:ds12_table}
\end{figure}

% \begin{figure}[H]
%     \centering
%     \includegraphics[width=\columnwidth]{images/ds12_table1.png}
%     \caption{Dataset 2-1: test MSE against dimension}
%     \label{fig:ds12_table1}
% \end{figure}

% \begin{figure}[H]
%     \centering
%     \includegraphics[width=\columnwidth]{images/ds12_table2.png}
%     \caption{Dataset 2-1: test MSE against dimension, a zoomed-in view of the top-performing models.}
%     \label{fig:ds12_table2}
% \end{figure}

We vary the input dimension from \(p=500\) to \(p=3000\) and plot the test MSE against the dimension for all models. Figures \ref{fig:ds02_table} (a) and \ref{fig:ds12_table} (a) show that all neural network models outperform the classical PCR and Lasso models, while neural network models with a low-rank structure consistently outperform vanilla neural networks.

For Dataset 1-1, where the factor-observation relationship is linear, Figure \ref{fig:ds02_table} (b) demonstrates that our \(SPCA\_NN\) model performs similarly to or better than the FAR-NN and OracleNN models. In the case of Dataset 2-1, which exhibits a non-linear factor-observation relationship, we incorporate a soft PCA layer within the network architecture. As shown in Figure \ref{fig:ds12_table} (b), this approach achieves results comparable to FAR-NN and OracleNN when the number of dimensions \(p\) exceeds 1500.


Furthermore, for both data sets, the GFADNN model, structured as \( PCA\allowbreak\_NN\allowbreak\_PCA\allowbreak\_ADD \) and \( SPCA\allowbreak\_NN\allowbreak\_SPCA\allowbreak\_ADD \), where PCA and soft PCA operations are applied iteratively and fed into additive subnetworks, significantly reduces the test MSE across all dimensions. We also observe that models with soft PCA layer(s) exhibit a slight downward trend in test MSE as the dimensionality increases, while this trend is less clear for the vanillaNN and autoencoder models. This behavior is also documented in \citet{fan2023factor}, indicating that imposing a low-rank structure in neural network models, with the number of latent factors fixed, can even benefit from a reasonable and finite increase in dimension.




% \subsection{Tracking Explained Variance}
\section{Empirical Studies}
\label{sec:section5}
In this section, we evaluate our model by comparing it against benchmark neural network models and classical statistical models using real-world data sets. For each dataset, we construct a high-dimensional setting to perform regression tasks, with the primary focus on model performance comparison.
% \subsection{Sector ETFs}
% An Exchange-Traded Fund (ETF) targeting specific sectors offers a focused investment approach, concentrating on companies within a particular industrial sector. Our research aims to predict the next-day returns of such an ETF to develop an informed trading strategy. For instance, we focus on the XLI Industrials ETF from the Sector SPDR ETFs, which segments the $S\&P$ 500 into eleven distinct sector index funds. The analysis utilizes constituents from the XLI universe as of August 2023. After applying liquidity and data quality filters, we shortlist 60 stocks and compute their daily returns from January 1, 2000, to December 31, 2018. To mitigate external influences, we adjust these returns by subtracting the SPY return, thus isolating the idiosyncratic returns of each stock. We then calculate the 1-20 day moving average returns for each stock. For forecasting the subsequent day's ETF return, we employ a feature set comprising 20 moving average returns from each stock, resulting in a feature dimensionality of 20*60 = 1200. In this study, we employ a rolling training approach to train the models. Specifically, we allocate 252 trading days—reflecting the average number of trading days in a year—as our training window. For validation, we set aside 40 trading days, approximately the number of trading days spanning two months, followed by another 252 trading days dedicated to the testing window.

% The process begins with hyperparameter tuning, during which we identify and select the optimal model configuration based on the lowest mean squared error (MSE) observed in the validation phase. Subsequently, this chosen model is trained on the training dataset, with the validation MSE serving as a criterion for early stopping to prevent overfitting. Finally, we record the MSE obtained during the testing phase to gauge the model's predictive accuracy. 
% To further assess our model's performance, we devise a straightforward trading strategy based on the model's forecasts. This involves winsorizing the model's forecasts based on the historical range of forecasts, thereby utilizing the winsorized forecasts to determine the positions for the subsequent day. Our evaluation focuses on metrics that are independent of leverage, so there is no need to control leverage in this context. The profitability curve of this trading strategy is derived by multiplying the position vector with the subsequent day's ETF idiosyncratic returns.

% For a comprehensive performance evaluation, we resort to trading-related metrics, including directional accuracy, Information Coefficient (IC), Sharpe ratio, turnover, and percentage maximum drawdown. The outcomes are summarized in the table provided below.
% \\
% \begin{table}[H]
% \centering
% \begin{tabular}{|r|r|r|r|r|}
% \hline
% $model\_name$ & $vanillaNN$ & $FAR-NN$ & $FactorAdditiveNN$ \\
% \hline
% $valid\_mse$ & 7.677 & 5.294 & 5.251 \\
% \hline
% % $valid\_score$ & -17.589195 & -40.290296 & -41.402638 & -48.437935 \\
% % \hline
% $test\_mse$ & 30.227 & 4.851 & 4.676  \\
% \hline
% % $test\_score$ & 0.104977 & -2.684102 & -9.208489 & -8.821784 \\
% % \hline
% $dir\_accuracy$ & 0.498 & 0.499 & 0.507  \\
% \hline
% $IC$ & -0.018 & 0.007 & 0.024 \\
% \hline
% $turnover$ & 0.247 & 0.142 & 0.155 \\
% \hline
% $sharpe\_ratio$ & -0.166 & 0.118 & 0.433\\
% \hline
% \end{tabular}
% \caption{Sector ETF: Portfolio performance table}
% \end{table}



% We can observe that vanillaNN has the highest validation mean squared error (mse), test mse whereas our FANN achieves the lowest scores in both categories. Regarding portfolio-related metrics, the vanilla NN's performance is suboptimal, with a Sharpe ratio of -0.166, indicating it does not surpass random guessing. In contrast, the Factor Augmented Neural Network (FAR-NN) shows improvement with a Sharpe ratio of 0.118075, and our FANN model further enhances this figure to 0.506713, demonstrating significant advancement.

% Additionally, our model notably outperforms others in terms of the Information Coefficient (IC), indicating a more predictive power. Furthermore, portfolios constructed using both FAR-NN and FANN exhibit lower turnover rates and effectively reduce the maximum drawdown from $-28\%$ to $-20\%$ and $-17\%$, respectively. This indicates not only enhanced return predictability but also improved risk management.


\subsection{Forecasting SPY ETF Price Returns}
An Exchange-Traded Fund (ETF) targeting specific sectors offers a focused investment approach, concentrating on companies within a particular industrial sector. In this experiment, we explore Sector ETFs from S\&P Global, specifically their S\&P 500 Sector ETFs. These ETFs break down the S\&P 500 into eleven distinct sector index funds. Our experiment aims to predict the next-day returns of S\&P 500 ETFs using data from all eleven Sector ETFs. The data, spanning from August 1, 2014, to August 1, 2024, was sourced from S\&P Global.

To forecast the next day's S\&P 500 return, we utilize a feature set comprising moving average returns over multiple time frames: 1, 5, 10, 20, and 126 days for each ETF. This approach results in a total feature dimensionality of 55. Our methodology employs a rolling training model, where we allocate 504 trading days—corresponding to the average number of trading days in two years—as our training window. For validation, we set aside 60 trading days, approximately equivalent to three months, followed by an additional 252 trading days designated for the testing phase.

The process begins with hyperparameter tuning to identify and select the optimal model configuration, which is based on achieving the lowest mean squared error (MSE) during the validation phase. The hyperparameter space is consistent with that used in the simulation experiment, except for the number of latent factors, which is allowed to vary between 11 and 22. Once selected, this model is trained on the training dataset, using the validation MSE as a criterion for early stopping to mitigate overfitting. The MSE obtained during the testing phase is then used to gauge the model's predictive accuracy.

To further evaluate our model's effectiveness, we implement a simple trading strategy based on the model’s forecasts. The forecasted signal is scaled by dividing it by 1.2 times the maximum value observed during the training period, then capped within the range [-1, 1]. This scaled signal is then used as the target percentage position. The profitability of this strategy is assessed by multiplying the position vector with the subsequent day’s S\&P 500 returns. 

\subsubsection{Portfolio Evaluation Metrics}
To evaluate our trading strategy, we use several performance metrics, including Directional Accuracy, Information Coefficient (IC), Sharpe Ratio, Turnover,  Maximum Drawdown and Average Percentage Position. We define each metric as follows:



1. \underline{Daily Returns}: The daily return \( r_t \) of the portfolio on day \( t \) is calculated as
   \[
   r_t = \frac{P_t - P_{t-1}}{P_{t-1}},
   \]
   where \( P_t \) is the portfolio value at the end of day \( t \).

2. \underline{Directional Accuracy} Directional Accuracy measures the percentage of days when the sign of the portfolio return aligns with the actual market direction. It is defined as
   \[
   \text{Directional Accuracy (Dir)} = \frac{\sum_{t=1}^{T} \mathbb{I} \left( \operatorname{sign}(r_t) = \operatorname{sign}(m_t) \right)}{T} \times 100\%,
   \]
   where \( \mathbb{I}(\cdot) \) is the indicator function, \( r_t \) is the portfolio return on day \( t \), \( m_t \) is the market return on day \( t \), and \( T \) is the total number of trading days.

3. \underline{Information Coefficient (IC)}: The Information Coefficient measures the correlation between the predicted returns and the actual returns, indicating the accuracy of the strategy's predictions. It is calculated as
   \[
   \text{IC} = \frac{\sum_{t=1}^{T} (f_t - \bar{f})(r_t - \bar{r})}{\sqrt{\sum_{t=1}^{T} (f_t - \bar{f})^2 \sum_{t=1}^{T} (r_t - \bar{r})^2}},
   \]
   where \( f_t \) is the predicted return, \( r_t \) is the actual return, \( \bar{f} \) and \( \bar{r} \) are the mean predicted and actual returns over the period.

4. \underline{Sharpe Ratio}: The Sharpe Ratio is used to evaluate the risk-adjusted return of the portfolio. It is defined as
   \[
   \text{Sharpe Ratio} = \frac{\mathbb{E}[r_t - r_f]}{\sigma_r},
   \]
   where \( r_t \) is the daily return, \( r_f \) is the risk-free rate, and \( \sigma_r \) is the standard deviation of daily returns.

5. \underline{Turnover}: Turnover measures the trading activity within the portfolio. It is calculated as
   \[
   \text{Turnover} = \frac{1}{T} \sum_{t=1}^{T} \sum_{i=1}^{N} \left| w_{i, t} - w_{i, t-1} \right|,
   \]
   where \( w_{i, t} \) is the portfolio weight of asset \( i \) at time \( t \), and \( N \) is the number of assets in the portfolio, and $N=1$ for this experiment.

6. \underline{Maximum Drawdown}: Maximum Drawdown represents the maximum observed loss from a peak to a trough in the portfolio's value, expressed as a percentage. It is defined as
   \[
   \text{Maximum Drawdown} = \max_{t \in [0, T]} \left( \frac{\max_{0 \le s \le t} P_s - P_t}{\max_{0 \le s \le t} P_s} \right) \times 100\%,
   \]
   where \( P_t \) is the portfolio value at time \( t \), and \( T \) is the end of the evaluation period.

7. \underline{Average Percentage Position}: Average Percentage Position represents the mean absolute position within the portfolio, calculated as the average of the absolute portfolio weights over the evaluation period. It is defined as

\[
\text{Average Percentage Position} = \frac{1}{T} \sum_{t=1}^{T} \frac{1}{N} \sum_{i=1}^{N} |w_{i,t}| \times 100\%,
\]
\noindent where \( w_{i,t} \) is the portfolio weight of asset \( i \) at time \( t \), \( N \) is the number of assets in the portfolio, and \( T \) is the end of the evaluation period.


Each of these metrics provides a different perspective on the portfolio's performance, helping us to assess returns, risk, accuracy of predictions, and trading costs comprehensively.

\subsubsection{Portfolio Performance Summary}
The outcomes are summarized in the table provided below.

\begin{table}[h!]
\centering
\resizebox{\textwidth}{!}{
\begin{tabular}{|l|r|r|r|r|r|r|r|}
\hline
\textbf{Model} & \textbf{Ret} & \textbf{Sharpe} & \textbf{MaxDD} & \textbf{Turnover} & \textbf{Dir} & \textbf{IC} & \textbf{AvgPos} \\ \hline
pcr & 0.061 & 0.528 & -0.150 & 0.232 & 0.509 & 0.045 & 0.327 \\
\hline
lasso & 0.082 & 0.436 & -0.234 & 0.057 & 0.534 & 0.017 & 0.919 \\
\hline
vanillaNN & 0.041 & 0.284 & -0.289 & 0.226 & 0.534 & 0.014 & 0.348 \\
\hline
autoencoder & 0.034 & 0.263 & -0.204 & 0.312 & 0.514 & 0.023 & 0.374 \\
\hline
FAR-NN & 0.073 & 0.523 & -0.185 & 0.320 & 0.521 & \textbf{0.059} & 0.445 \\
\hline
\textbf{PCA\_NN\_PCA\_ADD} & 0.080 & \textbf{0.767} & -0.159 & 0.150 & \textbf{0.539} & 0.057 & 0.286 \\
\hline
\textbf{SPCA\_NN\_SPCA\_ADD} & 0.139 & \textbf{1.244} & -0.099 & 0.133 & \textbf{0.543} & \textbf{0.071} & 0.495 \\
\hline
\end{tabular}
}
\caption{SP500 ETF Trading Strategies Performance Table, where Return and Sharpe ratio are annualized. }
\label{table:performance_metrics1}
\end{table}


The tables above show that our models achieve superior Sharpe ratios of 0.767 and 1.244, lower maximum drawdowns of -15.9\% and -9.9\%, and relatively lower turnover.

It is worth noting that the average percentage position of these portfolios is only around 30\%. To enhance the strategy, we decided to increase the position size and reduce turnover. First, we applied a 20-day rolling mean to smooth the position, followed by volatility targeting to achieve an annualized volatility of 20\%, similar to that of the S\&P 500 index. Finally, we capped the position within the range [-2, 2] to manage leverage. The resulting performance is presented in Table \ref{table:performance_metrics2}.

\begin{table}[h!]
\centering
\resizebox{\textwidth}{!}{
\begin{tabular}{|l|r|r|r|r|r|r|r|}
\hline
\textbf{Model} & \textbf{Ret} & \textbf{Sharpe} & \textbf{MaxDD} & \textbf{Turnover} & \textbf{Dir} & \textbf{IC} & \textbf{AvgPos} \\ \hline
pcr & 0.097 & 0.505 & -0.311 & 0.150 & 0.507 & 0.024 & 1.248 \\
\hline
lasso & 0.155 & 0.805 & -0.168 & 0.052 & 0.525 & 0.027 & 1.407 \\
\hline
vanillaNN & 0.124 & 0.652 & -0.330 & 0.116 & 0.531 & \textbf{0.041} & 1.238 \\
\hline
autoencoder & 0.062 & 0.318 & -0.414 & 0.151 & 0.510 & -0.012 & 1.266 \\
\hline
FAR-NN & 0.152 & 0.779 & -0.229 & 0.114 & 0.527 & 0.034 & 1.276 \\
\hline
\textbf{PCA\_NN\_PCA\_ADD} & 0.247 & \textbf{1.276} & -0.170 & 0.102 & 0.545 & \textbf{0.067} & 1.285 \\
\hline
\textbf{SPCA\_NN\_SPCA\_ADD} & 0.196 & \textbf{1.005} & -0.176 & 0.050 & \textbf{0.546} & 0.031 & 1.401 \\
\hline
\end{tabular}
}
\caption{Performance metrics for various models}
\label{table:performance_metrics2}
\end{table}



The tables above demonstrate that our model maintains a Sharpe ratio of 1.276 and 1.005 while enhancing the annualized return and significantly reducing turnover.


\subsection{FRED-MD Dataset}
In this section, we evaluate the performance of our FANN estimator against other high-dimensional linear estimators using the macroeconomics dataset FRED-MD \citep{mccracken2016fred}. FRED-MD has been used as one of the benchmarks for applying big data techniques, such as random subspace methods \citep{boot2019forecasting}, sufficient dimension reduction \citep{barbarino2017unified}, and various lasso-type regressions \citep{smeekes2018macroeconomic}. Smeekes and Wijler demonstrated that lasso-type estimators perform well with this dataset and are more robust to model misspecification than factor models, even when the underlying data-generating process has a factor structure. Additionally, research by \citet{medeiros2021forecasting} noted that the DeepNN model generally exhibits the poorest performance with this dataset. Consequently, this dataset poses significant challenges for our neural network model. 

The FRED-MD dataset comprises $p = 134$ monthly U.S. macroeconomic variables from January 1959, including metrics such as the unemployment rate and Treasuray bill rate. \citet{mccracken2016fred} demonstrate that these variables are largely describable through several latent factors. For each target response variable $y$, we utilize the vector of all other variables $\mathbf{x}_t$ to predict $y_t$ at each time index $t \in [T]$. This analysis demonstrates the ability of our estimator to capture relationships in high-dimensional datasets and generalize well in different scenarios.

The data spans from January 1980 to July 2022, processed according to \citet{mccracken2016fred}, resulting in a sample size of $n = 330$. Specifically, data from January 1980 to September 2009 (200 months, 60\% of the data) are used for training and validation, while data from October 2009 to July 2022 (130 months, denoted as $D_{\text{test}}$, 40\% of the data) are utilized for testing. Within the former period, a random sample of 70\% ($D_{\text{train}}$) is used for model training, and the remaining 30\% ($D_{\text{valid}}$) for validation and model selection. The performance of the estimator $\hat{m}_b$ is assessed using the out-of-sample $R^2$, defined as
\[
R^2_{\text{OOS}} = 1 - \frac{\sum_{(x,y) \in D_{\text{test}}} (\hat{m}(x) - y)^2}{\sum_{(x,y) \in D_{\text{test}}} (\bar{y}_{\text{train}} - y)^2} \quad \text{with} \quad \bar{y}_{\text{train}} = \sum_{(x,y) \in D_{\text{train}}} y.
\]


We begin with the model specification from \citet{fan2023factor}, setting the hyperparameter \( r \in \{4, 5\} \), depth \( L \in \{2, 3\} \), and width \( N = 32 \). This configuration is chosen to balance the moderate sample size \( n \) and ambient dimension \( p \). All other regularization hyperparameters ranges are kept consistent with those used in the simulation experiments.

Each variable is treated as a target variable in the regression, resulting in a total of 127 regression tasks. Performance was evaluated using the negative R-squared metric, with results reported for some selected variables for demonstration purposes, as shown in Table~\ref{table:FRED_model_performance}. Our analysis reveals that the GFANN model outperforms the Lasso model in 77 out of 127 cases, demonstrating a significant advantage. In comparison, FAST-NN surpasses the Lasso model in only 30 cases. These findings highlight the robustness and efficiency of the GFANN model in handling complex high-dimensional regression tasks relative to traditional methods.

To further compare the relative performance across models, we also record the relative rank of the test scores for each regression task, assigning a rank of 1 to the lowest (best) and 5 to the highest score. The average rank across all 127 regression tasks is summarized in Table~\ref{table:model_metrics}, showing that our GFANN model achieves the lowest average rank, even on a dataset that poses substantial challenges for neural network models.


\begin{table}[h!]
\centering
\begin{tabular}{|l|r|r|r|r|r|}
\hline
\textbf{target} & \textbf{lasso} & \textbf{pcr} & \textbf{vanillaNN} & \textbf{FAST-NN} & \textbf{GFANN} \\ \hline
CP3Mx         & -0.581 & 0.623 & 0.194 & 1.170 & -0.629 \\ \hline
USWTRADE      & -0.026 & -0.028 & 0.189 & 0.690 & -0.136 \\ \hline
FEDFUNDS      & -0.377 & 0.396 & 1.847 & 0.250 & -0.471 \\ \hline
CES2000000008 & -0.782 & 0.030 & -0.006 & -0.162 & -0.776 \\ \hline
CUSR0000SAS   & -0.812 & 0.265 & 0.475 & -0.273 & -0.875 \\ \hline
EXUSUKx       & -0.474 & 0.053 & -0.030 & 0.149 & -0.431 \\ \hline
UEMP5TO14     & -0.428 & 0.075 & 0.045 & 0.078 & -0.415 \\ \hline
TOTRESNS      & -0.767 & 0.013 & -0.164 & -0.300 & -0.773 \\ \hline
ACOGNO        & -0.418 & -0.241 & -0.227 & 0.043 & -0.426 \\ \hline
EXJPUSx       & -0.321 & 0.190 & 0.073 & 0.277 & -0.156 \\ \hline
\end{tabular}
\caption{Comparison of model performance for selected targets; the metric employed here is the negative $R^2_{\text{OOS}}$.}
\label{table:FRED_model_performance}
\end{table}

\begin{table}[h!]
\centering
\begin{tabular}{|l|r|r|r|r|r|}
\hline
\textbf{lasso} & \textbf{pcr} & \textbf{vanillaNN} & \textbf{FAST-NN} & \textbf{GFANN} \\ \hline
1.961 & 4.378 & 4.0 & 2.89 & 1.772 \\ \hline
\end{tabular}
\caption{Average rank across all 127 regression tasks, with a rank of 1 assigned to the lowest (best) test score and 5 to the highest. Lower ranks indicate better performance.}
\label{table:model_metrics}
\end{table}

\section{Conclusion}
\label{sec:section6}
This study has addressed the challenges involved in analyzing high-dimensional data sets, particularly those embodying latent low-dimensional structures concealed by non-linear, and noise-laden relationships. Our novel % pioneering 
strategy enhances neural networks through the incorporation of additive and factor layers, which can be effortlessly embedded at any step  within the neural network architecture. This flexibility proves especially effective in managing hierarchical and compositional data. A notable finding is that the sequential arrangement of additive layers following factor layers constitutes a highly efficient architecture for factor additive data-generating processes, showcasing superior performance with minimal parameter requirements and robustness against variations in architecture-related hyperparameters. Beyond integrating additive and factor models within neural networks seamlessly, our research validates the effectiveness of these innovations through comprehensive simulation studies and practical applications to the prediction of SPY ETF price returns and FRED-MD macroeconomic indicators.

%% For citations use: 
%%       \cite{<label>} ==> [1]
%%
%% Example citation, See \cite{lamport94}.



%% If you have bib database file and want bibtex to generate the
%% bibitems, please use
%%
% \bibliographystyle{elsarticle-num} 
\newpage
During the preparation of this work, the author(s) used Chatgpt in order to polish the writing. After using this tool/service, the author(s) reviewed and edited the content as needed and take(s) full responsibility for the content of the publication.

\bibliographystyle{elsarticle-num-names} 
% \bibliography{reference}
\documentclass{MITstyle}

%\usepackage[table]{xcolor}
\usepackage{chngcntr}
\usepackage{hyperref}
\usepackage{microtype}

\title{A Lightweight and Extensible Cell Segmentation and Classification Model for Whole Slide Images}

\author{Nikita Shvetsov~$^{1, }$\footnote{Correspondence e-mail: nikita.shvetsov@uit.no}, Thomas K. Kilvaer~$^{2, 3}$, Masoud Tafavvoghi~$^{4}$, Anders Sildnes~$^{1}$, \\ Kajsa Møllersen~$^{4}$, Lill-Tove Rasmussen Busund~$^{5, 6}$, Lars Ailo Bongo~$^{1}$ \\
%
\vspace{1em} % Space between authors and afilliations
%
\normalfont{\small $^{1}$Department of Computer Science, UiT The Arctic University of Norway}\\
\normalfont{\small $^{2}$Department of Oncology, University Hospital of North Norway}\\
\normalfont{\small $^{3}$Department of Clinical Medicine, UiT The Arctic University of Norway}\\
\normalfont{\small $^{4}$Department of Community Medicine, UiT The Arctic University of Norway}\\
\normalfont{\small $^{5}$Department of Medical Biology, UiT The Arctic University of Norway} \\
\normalfont{\small $^{6}$Department of Clinical Pathology, University Hospital of North Norway} %\vspace{2em}
}

\begin{document}
\maketitle

\section*{Abstract}

% \begin{abstract}
% Developing clinically useful cell-level analysis tools in digital pathology remains challenging due to limitations in dataset granularity, inconsistent annotations, computational demands of advanced models, and difficulties in integrating new technologies into clinical workflows. To address these challenges, we propose a multi-faceted solution that enhances data quality, model performance, and usability to create a lightweight and extensible cell segmentation and classification model.

% First, we update data labels by employing a cross-relabeling process that refines the labels of two existing datasets, PanNuke and MoNuSAC, to create a new unified dataset with enhanced granularity, encompassing seven distinct cell types. Second, we leverage the H-Optimus foundation model as a fixed encoder to improve feature representation for simultaneous cell segmentation and classification tasks. Third, to address the computational demands of foundation models, we employ knowledge distillation to reduce model size and complexity while maintaining comparable performance. Finally, to facilitate integration into clinical workflows, we integrate the distilled model into the QuPath software, a widely used open-source platform in digital pathology.

% Our results demonstrate improvements in cell segmentation and classification performance using the H‑Optimus-based model compared to a CNN-based model. Specifically, the average $R^2$ improved from 0.575 to 0.871, and the average $PQ$ score improved from 0.450 to 0.492, indicating better alignment with actual cell counts and enhanced segmentation and classification quality. Furthermore, the distilled student model maintains performance comparable to the larger foundation model while reducing the parameter count by a factor of 48.
% Overall, by reducing computational complexity and integrating it into existing workflows, the proposed approach may significantly impact diagnostic processes, reduce the workload of pathologists, and contribute to improved patient outcomes. Though our approach shows potential enhancements in efficiency and usability of cell segmentation and classification models in digital pathology, extensive validation is needed to deploy these models in clinical practice.
% \end{abstract}

%%% shortened abstract
\begin{abstract}
Developing clinically useful cell-level analysis tools in digital pathology remains challenging due to limitations in dataset granularity, inconsistent annotations, high computational demands, and difficulties integrating new technologies into workflows. To address these issues, we propose a solution that enhances data quality, model performance, and usability by creating a lightweight, extensible cell segmentation and classification model. 

First, we update data labels through cross-relabeling to refine annotations of PanNuke and MoNuSAC, producing a unified dataset with seven distinct cell types. Second, we leverage the H-Optimus foundation model as a fixed encoder to improve feature representation for simultaneous segmentation and classification tasks. Third, to address foundation models' computational demands, we distill knowledge to reduce model size and complexity while maintaining comparable performance. Finally, we integrate the distilled model into QuPath, a widely used open-source digital pathology platform. 

Results demonstrate improved segmentation and classification performance using the H-Optimus-based model compared to a CNN-based model. Specifically, average $R^2$ improved from 0.575 to 0.871, and average $PQ$ score improved from 0.450 to 0.492, indicating better alignment with actual cell counts and enhanced segmentation quality. The distilled model maintains comparable performance while reducing parameter count by a factor of 48. By reducing computational complexity and integrating into workflows, this approach may significantly impact diagnostics, reduce pathologist workload, and improve outcomes. Although the method shows promise, extensive validation is necessary prior to clinical deployment.
\end{abstract}
\clearpage

\section{Introduction}
In digital pathology, accurate segmentation and classification of cells are crucial for many diagnostic, prognostic, and predictive analyses \cite{Jaber_Beziaeva_etal._2019,Lin_Pan_etal._2022,Park_Ock_etal._2022,Shen_Choi_etal._2024}. Nowadays, developments in computational pathology offer multiple solutions \cite{H._Qu_P._Wu_etal._2020,Javed_Mahmood_etal._2020} to utilize cell-level datasets to train machine learning models that solve these problems. The quality and specificity of training datasets are critical for robust and accurate models. Adhering to the principle of "garbage in, garbage out", it is essential to ensure that these datasets are extensively and accurately labeled with distinct classes that reflect the diverse biological characteristics of different cell types. Unfortunately, the number of open-source datasets comprising such high-quality annotations is limited. Existing cell segmentation datasets \cite{Gamper_Koohbanani_etal._2019,Graham_Vu_etal._2019,Verma_Kumar_etal._2021} may offer extensive annotations for certain cell types while providing more general labels for others. For example, in PanNuke, which is one of the largest open-source datasets comprising labeled cells, various types of morphologically and functionally different inflammatory cells like macrophages and lymphocytes are clustered in a broad "inflammatory" class. Consequently, these classes are frequently omitted from analyses or aggregated into broader meta-classes \cite{Gamper_Koohbanani_etal._2020} and likely interfere with other cell classes included in the dataset. This and similar inconsistencies in annotation granularity limit the ability of machine learning models to learn the comprehensive and nuanced features necessary for accurate cell segmentation and classification. To address these challenges, methods for refining and standardizing dataset annotations are essential to enhance the quality of training data.

A complementary approach to mitigate the absence of high-quality training data is the use of foundation models. Foundation models as encoders are defined as large-scale, versatile networks pre-trained on vast, diverse datasets using self-supervised learning, contrasting with convolutional neural network (CNN) pre-trained encoders that rely on supervised learning with labeled data. In practice, foundation models leverage enormous amounts of weakly or unlabeled data from millions of whole slide images (WSIs) and employ self-attention mechanisms to capture long-range dependencies and global context \cite{Chen_Ding_etal._2024,Saillard_Jenatton_etal._2024,Vorontsov_Bozkurt_etal._2024,Xu_Usuyama_etal._2024}. As a consequence, foundation models are able to produce transferable feature representations across different cell types and tissue environments. The feature representations can be leveraged by decoder networks to produce segmentation masks and pixel-level classifications. Because foundation models have comprehensive feature representations, they can be effectively fine-tuned using much smaller amounts of cell-level data compared to the large datasets needed to train models from scratch. Furthermore, foundation models incorporate adversarial training elements or contrastive learning \cite{Chen_Ding_etal._2024,Xu_Usuyama_etal._2024}, enhancing their resilience and adaptability by exposing them to challenging and varied scenarios during training. This may result in more generalizable models, often making them well-suited for diverse and complex tasks in digital pathology.

Despite the inherent advantages of foundation models, their deployment for practical use faces its own obstacles. In particular, they require substantial computational power, financial investments and rigorous testing to ensure reliability and efficacy for a given task \cite{Akkus_Dangott_etal._2022,Dragomir_Cocuz_etal._2022,Go_2022,Jafri_Farooqui_etal._2024}. Moreover, while foundation models enhance feature representation and performance, they depend on the quality of available annotations for decoder fine-tuning and, like any other model, cannot resolve existing inconsistencies or ambiguities in data labels. Therefore, there remains a critical need for solutions that address both data quality and practical deployment considerations.
Further, integrating new technologies into existing clinical workflows often encounters resistance, as it necessitates adjustments to established diagnostic processes. So, there is a need to develop solutions that could be integrated into current practices, minimizing the burden on medical professionals to adopt new tools \cite{King_Williams_etal._2023}.

Existing solutions \cite{Goldsborough_Philps_etal._2024,Hörst_Rempe_etal._2024}, while addressing some aspects of these challenges, fall short in providing a comprehensive approach. To address the data quality and clinical deployment issues, we propose a multi-faceted solution that encompasses data refinement, model optimization, and integration with existing pathology tools (\hyperref[fig:fig1]{Figure 1}). The outcome is a lightweight cell segmentation and classification model that can be integrated into digital pathology workflows for practical clinical use.

\begin{figure}[h!]
    \centering
    \includegraphics[width=\textwidth, height=0.82\textheight, keepaspectratio]{images/Figure_1.pdf}
    \caption{Overview of the proposed solution, including 1) Data refinement using cross-relabeling, 2) Teacher model development and fine tuning, 3) Student model optimization with knowledge distillation and 4) Student model and QuPath integration}
    \label{fig:fig1}
\end{figure}
\clearpage

Our approach begins with preparing the data for the fine-tuning and training of the machine learning models. We create a refined dataset, acquired via cross-relabeling two cell-level datasets, enhancing annotation specificity and consistency of the labeled data. Subsequently, we create a cell segmentation and classification model based on the foundation model. We leverage the foundation model as a fixed encoder and fine-tune a decoder using the refined dataset to improve generalization across diverse tissue- and cell types.
To ensure that the model remains lightweight and deployable in a possibly resource-constrained environment, we employ knowledge distillation to approximate the functionality of the foundation model. Finally, to facilitate the practical application of our model in digital pathology workflows, we integrate it with the QuPath \cite{Bankhead_Loughrey_etal._2017} application. Each methodological component contributes to the overarching goal of enhancing model performance, generalizability, and usability in clinical settings.

The primary contributions of this paper are:
\begin{enumerate}
    \item \textit{Data labels refinement through cross-relabeling:}
    
    We propose a new method for refining labels of cell-level datasets through cross-relabeling. This method employs classification models to re-label broad and ambiguous instances, resulting in a more diverse dataset. Our evaluation demonstrates that these classification models achieve high accuracy on test subsets, indicating the reliability of the method for label refinement.

    \item \textit{Enhanced model performance via foundation models:}
    
    We employ a foundation model as a feature extractor for the cell segmentation and classification task. In comparison with training a CNN model from scratch, the foundation model backbone only needs fine-tuning, which significantly reduces training time, computational resources and data requirements. We show that using a foundation model encoder leads to better performance in cell segmentation and classification networks than using a CNN-based encoder. This improvement may enable the model to generalize more effectively across various tissue types and imaging methods.
    
    \item \textit{Model optimization through knowledge distillation:}
    
    We show that a smaller student model trained using knowledge distillation on the refined dataset obtained via our cross-relabeling approach from a foundation model achieves comparable performance in cell segmentation and quantification tasks. As a result, this model is more suitable for deployment in environments without high-performance computing resources.
    
    \item \textit{Integration with QuPath:}
    
    We integrate the distilled cell segmentation and classification model into QuPath, a widely used open-source digital pathology platform, to accelerate clinical adaptation by enabling pathologists to more easily incorporate advanced computational tools into their existing workflows.
\end{enumerate}

Through these methodological steps, we aim to bridge the gap between advanced machine learning techniques and practical clinical applications, making accurate and efficient digital pathology accessible in a broader range of healthcare settings.

\section{Refining Existing Datasets Using Cross-Relabeling}
To address the limitations of sparse and ambiguous labeling of cell-level datasets, we propose a generalizable cross-relabeling strategy that can be applied to any dataset containing broadly categorized or imprecisely labeled cell types. This approach involves training and subsequently leveraging classification models to refine broad categories into more specific or biologically relevant classes.
When applied to cell-level data, the methodology includes extracting individual cell images from the dataset patches, preprocessing these images to standardize the size and accommodate partial cells, and then training deep learning classifiers capable of distinguishing between the finer cell subtypes within the coarser categories. 
To illustrate our approach, we focus on the PanNuke \cite{Gamper_Koohbanani_etal._2020, Gamper_Koohbanani_etal._2019} and MoNuSAC \cite{Verma_Kumar_etal._2021} datasets that we have used to train models for cell quantification in our previous works \cite{Shvetsov_Grønnesby_etal._2022,Shvetsov_Sildnes_etal._2024}. We find that for better cell differentiation we have to introduce more granular labels. PanNuke includes a broad classification of "inflammatory" cells, encompassing lymphocytes, macrophages, and neutrophils. Each cell type differs significantly in structure, function, and clinical relevance. Conversely, MoNuSAC uses the label "epithelial" for a class that comprises both benign epithelial cells and malignant neoplastic cells. This practice makes it challenging to differentiate between benign and malignant epithelial cells in the dataset, which is a critical distinction when identifying tumor areas within tissue samples. To address these issues, we implement a cross-relabeling strategy as shown in \hyperref[fig:fig2]{Figure 2}. The key components are two classification models: one is trained on singular cell images from PanNuke data to classify the epithelial meta-class into epithelial and neoplastic classes. The other is trained on MoNuSAC to refine the inflammatory class into lymphocytes, neutrophils, and macrophages.

\begin{figure}[h!]
    \centering
    \includegraphics[width=\textwidth]{images/Figure_2.pdf}
    \caption{Refined dataset generation via cross relabeling}
    \label{fig:fig2}
\end{figure}

The refining approach consists of three consecutive steps. The first is the preprocessing step, in which we extract individual cells from both datasets (\hyperref[fig:fig3]{Figure 3}). The specifics of PanNuke and MoNuSAC patch preparation before cell preprocessing are provided in \hyperref[chap:S1]{Appendix S1}.

\begin{figure}[h!]
    \centering
    \includegraphics[width=\textwidth]{images/Figure_3.pdf}
    \caption{Cell instances preprocessing including (1) cell map extraction, (2) bounding box delineation, (3) adjusting cell boxes and (4) cropping and resizing of cell images}
    \label{fig:fig3}
\end{figure}

During preprocessing, we extract cell type maps from the ground truth label mask and calculate bounding boxes around each cell instance. To accommodate partial cells at patch borders, a common issue in cropped patch images, we employ mirror padding and extend the field of view of the cell label by 15 pixels to capture adjacent cells. We then crop and resize the identified regions to $64 \times 64$ pixels using bicubic interpolation.

The preprocessed PanNuke dataset comprises 68,031 neoplastic and 23,207 epithelial cell images, while MoNuSAC comprises  33,104 lymphocytes, 1,252 neutrophils, and 1,695 macrophages, which we subsequently use in training cell classification models and classifying the cell image data \hyperref[fig:S2]{Appendix Figure S2 (1)}. 

The next step is to train two distinct ResNet50-based classifiers tailored to address the specific labeling challenges inherent in each dataset. We use ResNet50 for classification models due to its proven effectiveness for image classification tasks in histopathology \cite{pan2022reviewmachinelearningapproaches}, and its compatibility with small images. For the PanNuke dataset, we design the classifier, trained on MoNuSAC data, to disaggregate the heterogeneous "inflammatory" cell category into distinct subtypes: lymphocytes, macrophages, and neutrophils. Similarly, for the MoNuSAC dataset, the classifier is trained on PanNuke data and distinguishes between benign and malignant epithelial cells within the overarching "epithelial" label. By applying these targeted classifiers to their respective datasets, we assign more specific labels to individual cell instances, thus enabling us to create a unified dataset.
To ensure a balanced representation of classes, we train both models on datasets that had been equalized to match the size of the least represented class. Thus, we obtain datasets comprising 23,207 samples per class for PanNuke and 1,252 samples per class for MoNuSAC data. Next, we partition both of them into training (70\%), validation (20\%), and testing (10\%) subsets. To mitigate the risk of overfitting, we use a single dropout layer with a rate of p=0.5 in both models and data augmentation using randomized color perturbations, rotation, and horizontal and vertical flipping. We employ AdamW optimizer and the cross-entropy loss function for the training criterion.

To evaluate the two trained models, we measure the classification accuracy on the respective test subsets. The accuracies on the test subset for both classifiers are presented in \hyperref[tab:1]{Table 1}. The PanNuke model achieves an average accuracy of 93.57\%, with higher accuracy for neoplastic cells (96.06\%) compared to epithelial cells (86.26\%). The confusion matrix in Figure A3.1 shows that the model predominantly distinguishes accurately between epithelial and neoplastic tissues, with a substantial number of correct classifications and relatively few misclassifications. The MoNuSAC model demonstrates an average accuracy of 98.92\%, excelling in classifying lymphocytes (99.67\%) and macrophages (94.12\%), with lower performance for neutrophils (85.71\%). The confusion matrix in Figure A3.2 shows that the model identifies lymphocytes and performs reasonably well with macrophages and neutrophils.

\begin{table}[h!]
\renewcommand{\arraystretch}{1.5}
  \centering
  \caption{Cell classification results for PanNuke and MoNuSAC trained models (CI 95\%).}
  \label{tab:1}
  \begin{tabular}{|l|c|c|}
   \hline
   %\rowcolor{gray!30}
    Accuracy               & PanNuke model              & MoNuSAC model              \\
    \hline
    Average      & 0.936 (0.931--0.941)         & 0.989 (0.986--0.993)        \\
    \hline
    Neoplastic   & 0.961 (0.956--0.965)         & -                          \\
    \hline
    Epithelial   & 0.863 (0.849--0.877)         & -                          \\
    \hline
    Lymphocytes  & -                          & 0.997 (0.995--0.999)        \\
    \hline
    Neutrophils  & -                          & 0.857 (0.796--0.918)        \\
    \hline
    Macrophages  & -                          & 0.941 (0.906--0.976)        \\
    \hline
  \end{tabular}
\end{table}

Finally, during the last step, we use the model trained on PanNuke data for epithelial cells in MoNuSAC and the model trained on MoNuSAC for the inflammatory cells class in PanNuke. Specifically, we use classifier models to relabel epithelial cells in MoNuSAC and inflammatory cells in PanNuke data. Then we combine cells with refined labels and the rest of the cells in both datasets to create a refined dataset (\hyperref[fig:S2]{Appendix Figure S2 (2)}). The process of relabeling cells and visualizing them on a patch is shown in \hyperref[fig:fig4]{Figure 4}. The cell counts in the refined dataset are provided in \hyperref[tab:S4]{Appendix Table S4}.

\begin{figure}[h!]
    \centering
    \includegraphics[width=\textwidth, height=0.42\textheight, keepaspectratio]{images/Figure_4.pdf}
    \caption{Cell relabeling procedure for epithelial and inflammatory cell classes}
    \label{fig:fig4}
\end{figure}

%\hfill

Relabeling and combining datasets have been explored in a prior study \cite{Parulekar_Kanwat_etal._2023}, where consecutive fine-tuning on multiple datasets was employed to account for hierarchical class label structures. While the method presented in \cite{Parulekar_Kanwat_etal._2023} is intuitive, it often lacks consistency and requires multiple fine-tuning runs, which can be cumbersome and time-consuming. 
In contrast, cross-relabeling simplifies this process by using specialized classification models tailored to each dataset's specific labeling challenges. This approach provides better transparency and produces a unified dataset encompassing seven distinct cell types across multiple tissue samples, enhancing data diversity for further model training or fine-tuning.

Despite these improvements, cross-relabeling does not entirely resolve issues related to poor labeling quality or the amount of labeled data. Specifically, our results show lower accuracies persist for underrepresented classes, such as macrophages, which may stem from a limited sample availability and intrinsic challenges in distinguishing these cells based solely on H\&E staining. Furthermore, while our method enhances label specificity, it relies on the initial quality of the broad labels; thus, any fundamental inaccuracies in the original annotations can propagate through the relabeling process. Addressing the overall problem of limited data labels may require integrating additional data sources or utilizing complementary immunohistochemical staining methods.
Although the reported performance metrics are obtained from evaluations on the native test sets of each dataset, it is important to note that the primary application of these classifiers is to perform cross-relabeling, where a model trained on one dataset (e.g., PanNuke) is applied to another (e.g., MoNuSAC) and vice versa. We acknowledge that a more systematic evaluation of cross-dataset generalization is needed and could be performed in future work.

Overall, the refined dataset produced by our approach can enhance the supervised training or fine-tuning of cell segmentation and classification models, especially those that utilize pre-trained foundation models to improve feature extraction robustness. In addition, these models can detect nuanced classes that enable researchers to conduct more detailed analyses of biological processes in computational pathology.

\section{Foundation models for robust cell segmentation and classification}

Accurate cell segmentation and classification in digital pathology are hindered by limited labeled data and the fact that conventional CNNs are unable to capture global contextual information due to their local receptive field constraints \cite{Gheflati_Rivaz_2022,Yang_Marcus_etal.}. Traditional approaches in cell quantification have predominantly relied on CNN encoders, such as ResNet50, given their proven effectiveness in semantic segmentation tasks \cite{Deshmane_2023,Graham_Vu_etal._2019,Mukasheva_Koishiyeva_etal._2024,Stringer_Wang_etal._2021}. However, approaches that include fine-tuning of pretrained CNNs, data augmentation, and stain normalization to partially increase data variability and address staining differences often fail to achieve the necessary generalization and robustness across diverse tissue types and staining conditions \cite{G._Wang_W._Li_etal._2018,Gao_Bagci_etal._2018,Karim_El_Khoury_Martin_Fockedey_etal._2021}.

To overcome these challenges, we leverage an encoder-decoder network that uses a foundation model as the encoder and a CNN upsampling decoder (\hyperref[fig:fig5]{Figure 5}) for simultaneous cell segmentation and classification in 2D patches extracted from WSIs. Foundation models with transformer-based architectures are viable alternatives to CNN-based encoders \cite{Shamshad_Khan_etal._2023,Sourget_2023}. They enable the creation of more advanced architectures that can decode or transform learned features more effectively \cite{Chen_Duan_etal._2023,Cheng_Misra_etal._2022,Xie_Wang_etal._2021}.

\begin{figure}[h!]
    \centering
    \includegraphics[width=\textwidth]{images/Figure_5.pdf}
    \caption{UNETR-like model with foundational model as backbone}
    \label{fig:fig5}
\end{figure}

By utilizing a transformer-based encoder, we incorporate global contextual information into the feature extraction process, which is a key advantage of such architectures \cite{Chen_Lu_etal._2021}. This foundation model integration facilitates accurate pixel-wise segmentation and classification without the need for extensive encoder training, thereby potentially improving generalization across varied cellular structures and tissue types.
In our implementation, we employ a modified UNETR \cite{Hatamizadeh_Tang_etal._2021} architecture that combines a vision transformer (ViT) \cite{Dosovitskiy_Beyer_etal._2021} encoder with a CNN-based decoder. The encoder utilizes the pretrained H-Optimus foundation model, which contains 1.1 billion parameters and is trained on over 500,000 H\&E stained WSIs \cite{Saillard_Jenatton_etal._2024}. We extract outputs from four evenly spaced transformer blocks $Z_i$, where $i \in [1, 14, 26, 38]$, to serve as residual connections for the CNN decoder. We select these blocks based on our observation that features from non-adjacent levels of the encoder lead to better overall performance on the test subset.

The CNN decoder upsamples the feature representations, acquired from the transformer blocks, to generate an intermediate vector that is handled by two task-specific layers that generate cell segmentation and classification masks. The first task-specific layer is the ‘Cellpose head’,  which is used to delineate cell instances. The layer generates horizontal and vertical gradient maps to form vector fields that are refined through gradient tracking in a post-processing step using the Cellpose algorithm \cite{Stringer_Wang_etal._2021}, known for its efficacy in cell segmentation tasks and generalizability across multiple domains \cite{Pachitariu_Stringer_2022,Stringer_Pachitariu_2024}. The second task-specific layer is the "Cell type head", which assigns labels to individual pixels. In the post-processing step, we determine the output classification label of each segmented cell instance by majority voting over the labeled pixels that comprise the cell in the segmentation map.

To evaluate model performance and measure the impact of adding a foundation model as backbone, we compare it to a ResNet50-based model. ResNet50 is a widely used solution for encoders in segmentation architectures in the medical domain \cite{Deshmane_2023,Graham_Vu_etal._2019,Mukasheva_Koishiyeva_etal._2024,Stringer_Wang_etal._2021}. For the H-Optimus-based model, we utilize frozen weights for the encoder and only fine-tune the decoder to take advantage of the extensive pre-training of the foundation model. For the ResNet50-based model we start with ImageNet \cite{Deng_Dong_etal.} weights and train both encoder and decoder parts. Hyperparameters for the training step are set to be identical, where possible, for comparable evaluation. 
For this evaluation, we deliberately use the PanNuke dataset to provide a standardized and controlled comparison between the H‑Optimus and ResNet50-based models (\hyperref[fig:S2]{Appendix Figure S2 (3)}). Specifically, we use two of the default PanNuke dataset splits (66\%) for training and validation, and reserve the third split (33\%) for testing.

To address the challenge of cell class imbalance in the PanNuke dataset, which is a common characteristic in most cell-level H\&E patch datasets, both models’ training processes employ a weighted loss function comprising cross-entropy and focal loss \cite{Lin_Goyal_etal._2018}. The focal loss component is adjusted with coefficients derived from each cell class' instance frequency, emphasizing learning from underrepresented classes and enhancing the model's sensitivity to rare but significant cellular patterns. The cross-entropy loss is augmented with spectral decoupling regularization \cite{Pezeshki_Kaba_etal._2021,Pohjonen_Stürenberg_etal._2022} and spatially varying label smoothing \cite{Islam_Glocker_2021}, which potentially stabilizes training and improves generalization in case of complex tissue morphologies. For optimization, we employ the \textit{AdamW} \cite{Loshchilov_Hutter_2019} to counter unbalanced class scenarios, with cosine annealing learning rate scheduler.

We utilize the scikit-learn library \cite{Van_der_Walt_Schönberger_etal._2014} and HoVer-Net \cite{Graham_Vu_etal._2019} implementations of $R^2$ (the coefficient of determination) and $PQ$ (panoptic quality) to evaluate our experiments. Complete mathematical formulations and detailed explanations of these metrics are provided in \hyperref[chap:S5]{Appendix S5}. To compute confidence intervals, we use nonparametric bootstrapping, where after calculating the metric on the full sample, we generated 1000 bootstrap replicates by resampling with replacement and then determined the 95\% confidence intervals as the 2.5th and 97.5th percentiles of the resulting empirical distribution.

%\hfill

The model comparisons are summarized in \hyperref[tab:2]{Table 2}. The H‑Optimus-based model achieves higher $R^2$ across all cell classes compared to the ResNet50-based model, which means that its predictions are more closely aligned with the PanNuke cell counts, indicating a stronger correlation with the observed data. Notably, the improvement of $R^2_{dead}$ may be an indicator of better global contextual representations provided by the foundation model backbone. In terms of segmentation and classification quality combined, measured by the PQ score, the H‑Optimus-based model demonstrates notable improvements across most cell classes. Overall, the average $R^2$ improved from 0.575 to 0.871, while the average $PQ$ score improved from 0.450 to 0.492, demonstrating better performance of the H-Optimus-based model.

\begin{table}[h!]
\renewcommand{\arraystretch}{1.5}
  \centering
  \caption{Cell quantification metrics for baseline and proposed models (CI 95\%).}
  \label{tab:2}
  \begin{tabular}{|l|c|c|}
    \hline
    %\rowcolor{gray!30}
    Metric             & Resnet50-based            & H-optimus-based              \\
    \hline
    $R^2_{neoplastic}$    & 0.681 (0.576--0.769)       & \textbf{0.941 (0.917--0.960)} \\
    \hline
    $R^2_{inflammatory}$  & 0.863 (0.778--0.903)       & \textbf{0.949 (0.918--0.966)} \\
    \hline
    $R^2_{connective}$    & 0.600 (0.488--0.698)       & 0.609 (0.436--0.772)          \\
    \hline
    $R^2_{dead}$          & 0.097 (-11.389--0.669)     & 0.925 (0.404--0.982)          \\
    \hline
    $R^2_{epithelial}$    & 0.635 (0.490--0.747)       & \textbf{0.930 (0.886--0.964)} \\
    \hline
    $PQ_{neoplastic}$       & 0.517 (0.499--0.535)       & \textbf{0.589 (0.575--0.604)} \\
    \hline
    $PQ_{inflammatory}$     & 0.455 (0.429--0.482)       & \textbf{0.528 (0.507--0.549)} \\
    \hline
    $PQ_{connective}$       & 0.416 (0.400--0.431)       & \textbf{0.451 (0.436--0.465)} \\
    \hline
    $PQ_{dead}$             & 0.374 (0.342--0.408)       & 0.292 (0.209--0.365)          \\
    \hline
    $PQ_{epithelial}$       & 0.488 (0.460--0.519)       & \textbf{0.599 (0.579--0.618)} \\
    \hline
  \end{tabular}
\end{table}

Our results  show that integrating the H‑Optimus foundation model within the UNETR architecture enhances the model's ability to segment and classify cells across diverse tissues from PanNuke data. The pretrained transformer encoder provides robust feature representations, resulting in higher average $R^2$ and $PQ$ scores compared to the CNN-based model. This leads to more reliable cell quantification and more accurate downstream analysis. Additionally, the streamlined fine-tuning process reduces computational overhead and training time, making the model more adaptable for new data.

Despite these advancements, the foundation model-based approach does not fully resolve all challenges related to cell segmentation and classification. We observe lower metric scores for underrepresented classes in the training data. Furthermore, foundation models typically encompass billions of parameters, resulting in substantial computational and memory requirements. It therefore poses challenges for deployment in resource-constrained environments, limiting their practical applicability in certain clinical settings.

\section{Model optimization via Knowledge Distillation}

To address the limitations posed by the extensive size of foundation models, we implement knowledge distillation — a model compression technique that leverages the teacher-student paradigm \cite{Hinton_Vinyals_etal._2015}. By training a smaller, more efficient student model to replicate the output of a larger, pre-trained teacher model, we retain performance while significantly reducing the model's complexity and resource requirements (\hyperref[fig:fig6]{Figure 6}).

\begin{figure}[h!]
    \centering
    \includegraphics[width=\textwidth, height=0.45\textheight, keepaspectratio]{images/Figure_6.pdf}
    \caption{Knowledge distillation framework for training a student model using a pre-trained teacher}
    \label{fig:fig6}
\end{figure}

We employ knowledge distillation to compress the H‑Optimus-based teacher model into a more efficient student model. The teacher model is the modified UNETR architecture with the H‑Optimus foundation model described in the previous chapter. The student model is based on a UNet architecture augmented with residual connections and incorporates a smaller ViT encoder with 9 million parameters \cite{Steiner_Kolesnikov_etal._2022,Wightman_2019}. 

First, we fine-tune the teacher model using the refined dataset from the cross-relabeling procedure (Section 2). Initially we train the decoder of the teacher model while keeping the encoder weights frozen. We split the refined dataset into train (70\%), validation (20\%) and test (10\%) subsets (\hyperref[fig:S2]{Appendix Figure S2 (4)}). During fine-tuning, we use the train and validation subsets, while leaving the test subset for model evaluation. We set the training procedure and model hyperparameters to be identical to those that were used to demonstrate the utility of foundation models for the simultaneous cell segmentation and classification task.

Next, we perform knowledge distillation from teacher to student using the refined dataset used to fine-tune the teacher model. The student model is trained to replicate the teacher model's outputs. We utilize a specialized loss function that aligns the student's predicted probability distribution with the teacher's, incorporating the teacher's class probability distribution derived from the output. Following the methodology of Hinton et al. \cite{Hinton_Vinyals_etal._2015}, we experiment with various hyperparameter settings for the temperature ($T$) and the balancing coefficients ($\alpha$ and $\beta$) in the loss function. We vary $T$ from 1 to 20 and adjust $\alpha$ and $\beta$ to balance the distillation and student losses. Through iterative tuning and evaluation, we identify that setting $T=14$, $\alpha=0.3$, and $\beta=0.7$ yields a configuration that converges and closely approximates the teacher model's performance during training.

Finally, we assess the performance of both models using the $R^2$ and $PQ$ (defined in \hyperref[chap:S5]{Appendix S5}) on the test set of the refined dataset (\hyperref[tab:3]{Table 3}). We observe that the 95\% confidence intervals overlap for most cell types, so we cannot claim statistically significant performance differences between the teacher and student models. One exception appears in the neoplastic class. The teacher model produces an $R^2$ of 0.919, while the student model shows an $R^2$ of 0.852. In addition, the student model achieves higher $PQ$ values for the neoplastic and connective classes, though the confidence intervals show overlap.

\begin{table}[h!]
\renewcommand{\arraystretch}{1.5}
  \centering
  \caption{Cell quantification metrics for teacher and distilled student models (CI 95\%).}
  \label{tab:3}
  \begin{tabular}{|l|c|c|}
    \hline
    %\rowcolor{gray!30}
    Metric & Teacher & Student \\
    \hline
    $R^2_{neoplastic}$    & \textbf{0.919} (0.898--0.939) & 0.852 (0.800--0.891) \\
    \hline
    $R^2_{lymphocyte}$    & 0.969 (0.956--0.977)         & 0.969 (0.956--0.978) \\
    \hline
    $R^2_{connective}$    & 0.694 (0.548--0.809)         & 0.618 (0.469--0.741) \\
    \hline
    $R^2_{dead}$          & 0.755 (0.400--0.908)         & 0.424 (0.100--0.731) \\
    \hline
    $R^2_{epithelial}$    & 0.922 (0.870--0.958)         & 0.843 (0.738--0.917) \\
    \hline
    $R^2_{macrophage}$    & 0.384 (-0.369--0.724)        & 0.704 (0.352--0.859) \\
    \hline
    $R^2_{neutrofil}$     & 0.854 (0.578--0.929)         & 0.833 (0.502--0.925) \\
    \hline
    $PQ_{neoplastic}$       & 0.581 (0.569--0.593)         & 0.601 (0.588--0.613) \\
    \hline
    $PQ_{lymphocyte}$       & 0.536 (0.520--0.553)         & 0.563 (0.544--0.579) \\
    \hline
    $PQ_{connective}$       & 0.436 (0.421--0.451)         & 0.457 (0.441--0.474) \\
    \hline
    $PQ_{dead}$             & 0.272 (0.235--0.315)         & 0.279 (0.201--0.369) \\
    \hline
    $PQ_{epithelial}$       & 0.522 (0.500--0.545)         & 0.530 (0.506--0.555) \\
    \hline
    $PQ_{macrophage}$       & 0.524 (0.459--0.588)         & 0.474 (0.405--0.543) \\
    \hline
    $PQ_{neutrofil}$        & 0.541 (0.490--0.592)         & 0.565 (0.522--0.607) \\
    \hline
  \end{tabular}
\end{table}


We further decompose the $PQ$ metric into its $SQ$ and $DQ$ components (\hyperref[tab:S6]{Appendix Table S6}). Both models produce nearly identical $SQ$ values, which indicates that they predict instance boundaries with similar precision. Although the student model shows some improvement in $DQ$ scores for certain classes, the confidence intervals overlap and do not confirm a statistically significant difference.

We observe that the student and teacher models yield comparable detection performance despite the student model using a much smaller and simpler architecture. A model with fewer parameters reduces the risk of overfitting when training data are scarce relative to the model’s complexity \cite{Farias_Ludermir_etal._2022}. The knowledge distillation process also encourages the student model to focus on the most generalizable detection features learned from the teacher. These factors enable the student model to achieve similar detection performance across different cell types.

Additionally, considering the model sizes reported in \hyperref[tab:4]{Table 4}, the distilled model achieves a significant reduction compared to the teacher model, with a 48-fold decrease in parameter count and a 5.5-fold reduction in on-disk size. In inference mode, the teacher model requires 16 GB of VRAM for a batch size of 32, while the distilled model only needs 3 GB of VRAM for the same batch size. These reductions make the distilled model significantly more practical for fine-tuning and deployment in resource-constrained environments.

\begin{table}[h!]
\renewcommand{\arraystretch}{1.5}
  \centering
  \caption{Parameter counts and size of teacher and distilled model}
  \label{tab:4}
  \adjustbox{max width=\textwidth}{%
  \begin{tabular}{|l|c|c|c|}
    \hline
    %\rowcolor{gray!30}
    Metric & H-optimus-based (Teacher) & mobileViT-based (Student) & Magnitude of difference \\
    \hline
    Parameters count       & 1,158,917,906   & \textbf{24,093,393}   & \textbf{48x}  \\
    \hline
    Estimated Total Size (MB) & 87,912       & \textbf{15,935}    & \textbf{5.5x} \\
    \hline
  \end{tabular}%
}
\end{table}

%\hfill

With recent advancements in complex network architectures and the use of pretrained encoders to achieve state-of-the-art performance \cite{Baumann_Dislich_etal._2024,Hörst_Rempe_etal._2024} in cell segmentation and classification tasks, model size, computational complexity, and processing times have increased. This limits the scalability and accessibility of these models. As we demonstrate, this may be mitigated using knowledge distillation. Studies in the field of natural language processing have demonstrated the efficacy of knowledge distillation in retaining the capabilities of the teacher model while achieving significant reductions in size and complexity \cite{Huangpu_Gao_2024,Sun_Yu_etal.}. 

We demonstrate the feasibility of knowledge distillation in digital pathology, specifically for cell segmentation and classification tasks. Moreover, we achieve this performance while also significantly reducing the parameter count. In addressing the challenge of knowledge transfer, we found that distillation from a transformer-based model to a smaller transformer is more straightforward than attempting to map transformer features to CNN blocks. In our experiments, using a CNN-based network as a student results in worse cell quantification performance due to the structural constraints of CNN feature space dimensions. 

Although our primary approach relies on a transformer-based student model that performs well, it can be further optimized to incorporate advantages from CNN architectures. For example, employing alternative techniques such as using ViT adapters \cite{Chen_Duan_etal._2023} or $1 \times 1$ convolutions to adjust feature map sizes may be beneficial for harnessing CNN advantages like enhanced local feature extraction. Moreover, if additional performance improvements are desired, the process can be further enhanced by applying supplementary knowledge distillation techniques, such as self-distillation \cite{Zhang_Song_etal._2019} or online distillation \cite{Houyon_Cioppa_etal._2023}.

Despite these promising results, further validation on independent datasets is necessary to fully understand the model's limitations. Underrepresented classes may pose challenges when addressing complex cases. Pathologists need to validate these models to adopt them in clinical settings. While the distilled models are smaller and more deployable, a technological gap persists because pathologists traditionally rely on established methods for inspecting WSIs and diagnosing diseases. Addressing the complexities involved in deploying models for inference and supporting pathologists in adopting new tools is essential for integrating these models into clinical workflows.

\section{Model integration with QuPath}
Digital pathology tools with graphical user interfaces are essential for visualizing and analyzing WSIs. To make our student model useful in clinical pathology workflows, it needs to be integrated into a tool that enables inspecting regions, creating annotations, and providing quantitative analyses of biomarkers. Therefore, we integrate the trained student model from the previous chapter into the QuPath open‑source platform \cite{Bankhead_Loughrey_etal._2017}. QuPath provides the required annotation, visualization, and analysis tools to interpret complex histological data, including workflows for cell segmentation, classification, and quantification (\hyperref[fig:fig7]{Figure 7}). 

\begin{figure}[h!]
    \centering
    \includegraphics[width=\textwidth]{images/Figure_7.pdf}
    \caption{Visualization of model-generated cell quantification annotations (left) and the corresponding unannotated slide (right) in QuPath}
    \label{fig:fig7}
\end{figure}

To identify the regions in a WSI critical for prognosticating tumor development, such as specific tumor areas or border regions without overlapping healthy tissue, the pathologist uses QuPath to outline these regions. Then, the pathologist initiates a cell segmentation and classification script through the QuPath interface for the selected regions. The resulting annotations and quantified cell information are then directly overlaid onto the WSI in the QuPath interface. Additional design and implementation details are in \hyperref[chap:S7]{Appendix S7}. 

Two common approaches for integrating deep learning models into QuPath are Java‑based native QuPath extensions \cite{Goldsborough_Philps_etal._2024} and the execution of RESTful API requests to a model server coupled with handling the response via an extension, as demonstrated in the application of cell segmentation models applied to immunofluorescence images \cite{Sugawara_2023}. While the community is actively working on these integration strategies, there is currently no universal solution that fully addresses all integration and performance requirements.

Extensions may offer better integration with QuPath, allowing slightly improved performance and more widespread usage of the built-in QuPath models, but they lack the flexibility to customize models and modify their behavior. For example, the newest version of QuPath includes models such as StarDist \cite{Weigert_Schmidt} and InstanSeg \cite{Goldsborough_Philps_etal._2024} that can perform cell segmentation. Both models pose limitations when applied to simultaneous cell segmentation and classification. StarDist performs well only on convex, round shapes by design, whereas some neoplastic, inflammatory, and connective cells exhibit complex and non-convex shapes. InstanSeg provides only semantic segmentation without assigning classes to the segmented cells.

%\hfill

In contrast, our approach offers an alternative integration strategy. It utilizes the paquo library to directly interact with QuPath’s internal application programming interface from within Python. This enables data exchange and processing without the need for intermediate conversion steps and provides greater control over model customization, retraining, and the incorporation of custom processing steps.

The integration of our custom model with QuPath underscores its potential to significantly enhance the diagnostic process by reducing the time burden on pathologists and enabling them to focus on more complex interpretative tasks using familiar software. Leveraging a tool that is already well-established among pathologists increases the likelihood of its adoption into daily clinical workflows. The quantitative data generated through the automated workflow is critical for both clinical decision-making and research, facilitating more accurate biomarker analysis, enabling robust statistical evaluations, and supporting hypothesis generation and testing. Additionally, by streamlining cell segmentation and classification, the tool enhances the scalability and reproducibility of pathological assessments, ultimately contributing to improved diagnostic accuracy and patient outcomes.

\section{Conclusion and future work}

In this study, we address critical challenges in digital pathology and tackle the usability and deployment issues of the developed models in standard computing environments without the need for high-performance computing systems. Our multi-faceted approach encompasses data refinement through cross-relabeling, leveraging foundation models for robust cell segmentation and classification, optimizing model performance via knowledge distillation, and integrating the optimized model into the QuPath software for practical application. This approach is used to construct a capable, versatile, and adjustable model for cell segmentation and classification, with enhanced performance and usability.

\begin{sloppypar}
While our approach shows potential in the field of computational pathology, certain limitations persist. 
For example, our implementation currently exhibits lower performance in detecting macrophages. 
This serves as an instance of the broader challenge of accurately identifying complex cell types. In order to address this issue, extending our approach to incorporate additional data sources, exploring alternative modeling approaches, and integrating other imaging modalities such as immunohistochemical staining may help improve detection accuracy. Moreover, although the distilled model reduces computational demands, integrating advanced deep learning models into clinical practice requires addressing technological gaps and potential resistance to adopting new tools within established diagnostic processes.
\end{sloppypar}

Future work could focus on several key areas to refine the proposed approach and facilitate its adoption in clinical environments. Enhancing the cell-relabeling process with additional datasets \cite{Graham_Jahanifar_etal._2021} could improve the representation of underrepresented cell types and enhance overall model performance. Also, incorporating additional data sources, such as multi-modal imaging or complementary staining methods, may address limitations related to cell type differentiation and class imbalance. Exploring other foundation models \cite{Vorontsov_Bozkurt_etal._2024,Zimmermann_Vorontsov_etal._2024} or introducing additional modalities \cite{Ding_Wagner_etal._2024,Vaidya_Zhang_etal._2025} may provide alternative architectures better suited to specific tasks or offer improved efficiency. Implementing more complex knowledge distillation techniques \cite{Houyon_Cioppa_etal._2023,Zhang_Song_etal._2019} could further optimize the model's performance and adaptability. Additionally, deeper integration with QuPath or other digital pathology software could provide pathologists more control over cell quantification analysis directly within the QuPath interface, thereby increasing accessibility and usability. Such enhancements would not only refine model performance but also ensure greater adaptability and scalability within various clinical environments. Finally, extensive validation of the model by pathologists and benchmarking against independent datasets are essential steps toward establishing the model's reliability and fostering confidence in its clinical utility.

\section*{Acknowledgments} 
This work was funded in part by the Research Council of Norway grant no. 309439 SFI Visual Intelligence, and the North Norwegian Health Authority grant no. HNF1521-20.

\bibliographystyle{IEEEtran}
\begin{sloppypar}
\begin{thebibliography}{99}

\bibitem{chaplot2020neural} Chaplot, Devendra Singh, et al. "Neural topological slam for visual navigation." Proceedings of the IEEE/CVF conference on computer vision and pattern recognition. 2020.

\bibitem{maksymets2021thda} Maksymets, Oleksandr, et al. "Thda: Treasure hunt data augmentation for semantic navigation." Proceedings of the IEEE/CVF International Conference on Computer Vision. 2021.

\bibitem{mezghan2022memory} Mezghan, Lina, et al. "Memory-augmented reinforcement learning for image-goal navigation." 2022 IEEE/RSJ International Conference on Intelligent Robots and Systems (IROS). IEEE, 2022.

\bibitem{al2022zero} Al-Halah, Ziad, Santhosh Kumar Ramakrishnan, and Kristen Grauman. "Zero experience required: Plug \& play modular transfer learning for semantic visual navigation." Proceedings of the IEEE/CVF Conference on Computer Vision and Pattern Recognition. 2022.

\bibitem{ye2021auxiliary} Ye, Joel, et al. "Auxiliary tasks and exploration enable objectgoal navigation." Proceedings of the IEEE/CVF international conference on computer vision. 2021.

\bibitem{chaplot2020object} Chaplot, Devendra Singh, et al. "Object goal navigation using goal-oriented semantic exploration." Advances in Neural Information Processing Systems 33 (2020)

\bibitem{ramakrishnan2022poni} Ramakrishnan, Santhosh Kumar, et al. "Poni: Potential functions for objectgoal navigation with interaction-free learning." Proceedings of the IEEE/CVF Conference on Computer Vision and Pattern Recognition. 2022.

\bibitem{ramrakhya2022habitat} Ramrakhya, Ram, et al. "Habitat-web: Learning embodied object-search strategies from human demonstrations at scale." Proceedings of the IEEE/CVF Conference on Computer Vision and Pattern Recognition. 2022.

\bibitem{mousavian2019visual} Mousavian, Arsalan, et al. "Visual representations for semantic target driven navigation." 2019 International Conference on Robotics and Automation (ICRA). IEEE, 2019.

\bibitem{dhariwal2021diffusion} Dhariwal, Prafulla, and Alexander Nichol. "Diffusion models beat gans on image synthesis." Advances in neural information processing systems 34 (2021)

\bibitem{ho2022classifier} Ho, Jonathan, and Tim Salimans. "Classifier-free diffusion guidance." arXiv preprint arXiv:2207.12598 (2022).

\bibitem{nichol2021glide} Nichol, Alex, et al. "Glide: Towards photorealistic image generation and editing with text-guided diffusion models." arXiv preprint arXiv:2112.10741 (2021)

\bibitem{brooks2023instructpix2pix} Brooks, Tim, Aleksander Holynski, and Alexei A. Efros. "Instructpix2pix: Learning to follow image editing instructions." Proceedings of the IEEE/CVF Conference on Computer Vision and Pattern Recognition. 2023.

\bibitem{fu2023guiding} Fu, Tsu-Jui, et al. "Guiding instruction-based image editing via multimodal large language models." arXiv preprint arXiv:2309.17102 (2023).

\bibitem{geng2024instructdiffusion} Geng, Zigang, et al. "Instructdiffusion: A generalist modeling interface for vision tasks." Proceedings of the IEEE/CVF Conference on Computer Vision and Pattern Recognition. 2024.

\bibitem{zhou2024minedreamer} Zhou, Enshen, et al. "Minedreamer: Learning to follow instructions via chain-of-imagination for simulated-world control." arXiv preprint arXiv:2403.12037 (2024).

\bibitem{zhou2023esc} Zhou, Kaiwen, et al. "Esc: Exploration with soft commonsense constraints for zero-shot object navigation." International Conference on Machine Learning. PMLR, 2023.

\bibitem{yu2023l3mvn} Yu, Bangguo, Hamidreza Kasaei, and Ming Cao. "L3mvn: Leveraging large language models for visual target navigation." 2023 IEEE/RSJ International Conference on Intelligent Robots and Systems (IROS). IEEE, 2023.

\bibitem{gadre2023cows} Gadre, Samir Yitzhak, et al. "Cows on pasture: Baselines and benchmarks for language-driven zero-shot object navigation." Proceedings of the IEEE/CVF Conference on Computer Vision and Pattern Recognition. 2023.

\bibitem{shah2023navigation} Shah, Dhruv, et al. "Navigation with large language models: Semantic guesswork as a heuristic for planning." Conference on Robot Learning. PMLR, 2023.

\bibitem{cai2024bridging} Cai, Wenzhe, et al. "Bridging zero-shot object navigation and foundation models through pixel-guided navigation skill." 2024 IEEE International Conference on Robotics and Automation (ICRA). IEEE, 2024.

\bibitem{yu2023co} Yu, Bangguo, Hamidreza Kasaei, and Ming Cao. "Co-NavGPT: Multi-robot cooperative visual semantic navigation using large language models." arXiv preprint arXiv:2310.07937 (2023).

\bibitem{wu2024voronav} Wu, Pengying, et al. "Voronav: Voronoi-based zero-shot object navigation with large language model." arXiv preprint arXiv:2401.02695 (2024).

\bibitem{qin2023mp5} Qin, Yiran, et al. "Mp5: A multi-modal open-ended embodied system in minecraft via active perception." arXiv preprint arXiv:2312.07472 (2023).

\bibitem{du2024learning} Du, Yilun, et al. "Learning universal policies via text-guided video generation." Advances in Neural Information Processing Systems 36 (2024).

\bibitem{ajay2024compositional} Ajay, Anurag, et al. "Compositional foundation models for hierarchical planning." Advances in Neural Information Processing Systems 36 (2024).

\bibitem{liang2024skilldiffuser} Liang, Zhixuan, et al. "Skilldiffuser: Interpretable hierarchical planning via skill abstractions in diffusion-based task execution." Proceedings of the IEEE/CVF Conference on Computer Vision and Pattern Recognition. 2024.

\bibitem{heusel2017gans} Heusel, Martin, et al. "Gans trained by a two time-scale update rule converge to a local nash equilibrium." Advances in neural information processing systems 30 (2017).

\bibitem{zhang2018unreasonable} Zhang, Richard, et al. "The unreasonable effectiveness of deep features as a perceptual metric." Proceedings of the IEEE conference on computer vision and pattern recognition. 2018.

\bibitem{brown2020language} Brown, Tom B. "Language models are few-shot learners." arXiv preprint arXiv:2005.14165 (2020).

\bibitem{podell2023sdxl} Podell, Dustin, et al. "Sdxl: Improving latent diffusion models for high-resolution image synthesis." arXiv preprint arXiv:2307.01952 (2023).

\bibitem{brohan2022rt} Brohan, Anthony, et al. "Rt-1: Robotics transformer for real-world control at scale." arXiv preprint arXiv:2212.06817 (2022).

\bibitem{brohan2023rt} Brohan, Anthony, et al. "Rt-2: Vision-language-action models transfer web knowledge to robotic control." arXiv preprint arXiv:2307.15818 (2023).

\bibitem{li2024manipllm} Li, Xiaoqi, et al. "Manipllm: Embodied multimodal large language model for object-centric robotic manipulation." Proceedings of the IEEE/CVF Conference on Computer Vision and Pattern Recognition. 2024.

\bibitem{shah2023vint} Shah, Dhruv, et al. "ViNT: A foundation model for visual navigation." arXiv preprint arXiv:2306.14846 (2023).

\bibitem{liu2024visual} Liu, Haotian, et al. "Visual instruction tuning." Advances in neural information processing systems 36 (2024).

\bibitem{hu2021lora} Hu, Edward J., et al. "Lora: Low-rank adaptation of large language models." arXiv preprint arXiv:2106.09685 (2021).

\bibitem{qin2023supfusion} Qin, Yiran, et al. "SupFusion: Supervised LiDAR-camera fusion for 3D object detection." Proceedings of the IEEE/CVF International Conference on Computer Vision. 2023.

\bibitem{qin2024worldsimbench} Qin, Yiran, et al. "Worldsimbench: Towards video generation models as world simulators." arXiv preprint arXiv:2410.18072 (2024).

\bibitem{yu2025gamefactory} Yu, Jiwen, et al. "GameFactory: Creating New Games with Generative Interactive Videos." arXiv preprint arXiv:2501.08325 (2025).

\bibitem{zhou2024code} Zhou, Enshen, et al. "Code-as-Monitor: Constraint-aware Visual Programming for Reactive and Proactive Robotic Failure Detection." arXiv preprint arXiv:2412.04455 (2024).

\bibitem{zhang2024ad} Zhang, Zaibin, et al. "AD-H: Autonomous Driving with Hierarchical Agents." arXiv preprint arXiv:2406.03474 (2024).

\bibitem{wang2024toward} Wang, Chaoqun, et al. "Toward Accurate Camera-based 3D Object Detection via Cascade Depth Estimation and Calibration." arXiv preprint arXiv:2402.04883 (2024).

\bibitem{huang2024story3d} Huang, Yuzhou, et al. "Story3d-agent: Exploring 3d storytelling visualization with large language models." arXiv preprint arXiv:2408.11801 (2024).

\bibitem{savinov2018semi} Savinov, Nikolay, Alexey Dosovitskiy, and Vladlen Koltun. "Semi-parametric topological memory for navigation." arXiv preprint arXiv:1803.00653 (2018).

\bibitem{majumdar2022zson} Majumdar, Arjun, et al. "Zson: Zero-shot object-goal navigation using multimodal goal embeddings." Advances in Neural Information Processing Systems 35 (2022): 32340-32352.

\bibitem{yadav2023offline} Yadav, Karmesh, et al. "Offline visual representation learning for embodied navigation." Workshop on Reincarnating Reinforcement Learning at ICLR 2023. 2023.

\bibitem{yadav2023ovrl} Yadav, Karmesh, et al. "Ovrl-v2: A simple state-of-art baseline for imagenav and objectnav." arXiv preprint arXiv:2303.07798 (2023).

\bibitem{sun2024fgprompt} Sun, Xinyu, et al. "FGPrompt: fine-grained goal prompting for image-goal navigation." Advances in Neural Information Processing Systems 36 (2024).

\bibitem{zhu2017target} Zhu, Yuke, et al. "Target-driven visual navigation in indoor scenes using deep reinforcement learning." 2017 IEEE international conference on robotics and automation (ICRA). IEEE, 2017.

\bibitem{koh2024generating} Koh, Jing Yu, Daniel Fried, and Russ R. Salakhutdinov. "Generating images with multimodal language models." Advances in Neural Information Processing Systems 36 (2024).

\bibitem{krantz2022instance} Krantz, Jacob, et al. "Instance-specific image goal navigation: Training embodied agents to find object instances." arXiv preprint arXiv:2211.15876 (2022).

\bibitem{schulman2017proximal} Schulman, John, et al. "Proximal policy optimization algorithms." arXiv preprint arXiv:1707.06347 (2017).

\bibitem{anderson2018evaluation} Anderson, Peter, et al. "On evaluation of embodied navigation agents." arXiv preprint arXiv:1807.06757 (2018).

\bibitem{lin2024navcot} Lin, Bingqian, et al. "NavCoT: Boosting LLM-Based Vision-and-Language Navigation via Learning Disentangled Reasoning." arXiv preprint arXiv:2403.07376 (2024).

\bibitem{NavGPT} Zhou, Gengze, Yicong Hong, and Qi Wu. "Navgpt: Explicit reasoning in vision-and-language navigation with large language models." Proceedings of the AAAI Conference on Artificial Intelligence.

\bibitem{hahn2021no} Hahn, Meera, et al. "No rl, no simulation: Learning to navigate without navigating." Advances in Neural Information Processing Systems 34 (2021): 26661-26673.

\bibitem{li2025t2isafety} Li, Lijun, et al. "T2ISafety: Benchmark for Assessing Fairness, Toxicity, and Privacy in Image Generation." arXiv preprint arXiv:2501.12612 (2025).

\bibitem{an2024agfsync} An, Jingkun, et al. "AGFSync: Leveraging AI-Generated Feedback for Preference Optimization in Text-to-Image Generation." arXiv preprint arXiv:2403.13352 (2024).


\end{thebibliography}
\end{sloppypar}

\clearpage
\beginsupplement
\section*{Appendix}
\renewcommand{\thesubsection}{S\arabic{subsection}}

\subsection{\label{chap:S1}PanNuke and MoNuSAC preprocessing}
The PanNuke dataset comprises a set of 7,901 RGB patches, each with dimensions of $256 \times 256$ pixels, which we set as the standard patch size for our analysis. In contrast, the MoNuSAC dataset encompasses 294 images of heterogeneous dimensions. To standardize the MoNuSAC images with our experiments, we implement a standardization protocol. Specifically, for images exceeding the dimensions of $256 \times 256$ pixels, we segment them into equal-sized patches and apply mirror padding to the remaining portions to avoid information loss at the peripherals. Patches with dimensions less than $128 \times 128$ pixels are excluded from the dataset due to the insufficient resolution to capture relevant cellular details. For patches where either dimension falls between 128 and 256 pixels, we employ upsampling to achieve the standard patch size. As a result, we obtain a total of 2,823 RGB patches derived from the MoNuSAC dataset for subsequent analysis. For additional details on the MoNuSAC data preparation process, refer to the source code \cite{Shvetsov_2025a}.
\clearpage

\subsection{\label{chap:S2}Data usage for the methodology}

\counterwithin{figure}{subsection}
\renewcommand{\thefigure}{S\arabic{subsection}}

\begin{figure}[h!]
    \centering
    \includegraphics[width=\textwidth, height=0.85\textheight, keepaspectratio]{images/A2.pdf}
    \caption{Overview of the methodology for cross-labeling, dataset refinement, and model comparison. (1) Cross-relabeling - training and testing cell classification models, (2) Cross-relabeling - using cell classification models to create refined dataset, (3) Fine-tuning and training models for comparison, (4) Student knowledge distillation with refined dataset}
    \label{fig:S2}
\end{figure}
\clearpage

\subsection{\label{chap:S3}Confusion matrices for classification models}
\counterwithin{figure}{subsection}
\renewcommand{\thefigure}{S\arabic{subsection}.\arabic{figure}}

\begin{figure}[h!]
    \centering
    \includegraphics[width=\textwidth, height=0.4\textheight, keepaspectratio]{images/A3_1.pdf}
    \caption{Confusion matrix for PanNuke trained model}
    \label{fig:S3.1}
\end{figure}

\begin{figure}[h!]
    \centering
    \includegraphics[width=\textwidth, height=0.4\textheight, keepaspectratio]{images/A3_2.pdf}
    \caption{Confusion matrix for MoNuSAC trained model}
    \label{fig:S3.2}
\end{figure}

\clearpage

\subsection{\label{chap:S4}Datasets cell counts}

\counterwithin{table}{subsection}
\renewcommand{\thetable}{S\arabic{subsection}}

\begin{table}[h!]
\renewcommand{\arraystretch}{2.0}
\centering
\caption{\label{tab:S4}Cell counts for PanNuke, MoNuSAC and refined datasets. Numbers in parentheses indicate preprocessed cell counts for cell classifier models training and testing.}
%\adjustbox{max width=\textwidth}{%
\begin{tabular}{|l|c|c|c|}
\hline
%\rowcolor{gray!30}
Cell type & PanNuke & MoNuSAC & Refined \\
\hline
Neoplastic & 77,403 (68,031) & - & 105,451 \\
\hline
Epithelial & 26,572 (23,207) & - & 29,926 \\
\hline
Epithelial (benign and malignant) & - & 31,402 & - \\
\hline
Inflammatory & 32,276 & - & - \\
\hline
Lymphocytes & - & 37,045 (33,104) & 65,275 \\
\hline
Neutrophils & - & 1,355 (1,252) & 3,833 \\
\hline
Macrophage & - & 1,842 (1,695) & 3,410 \\
\hline
Dead & 2,908 & - & 2,908 \\
\hline
Connective & 50,585 & - & 50,585 \\
\hline
\end{tabular}
%
%}
\end{table}



\clearpage

\subsection{\label{chap:S5}Definition of validation metrics}
\counterwithin{equation}{subsection}
\renewcommand{\theequation}{\arabic{equation}}

\subsubsection{\label{chap:S5.1}R\textsuperscript{2}}
The coefficient of determination, denoted as $R^2$, is a statistical measure that represents the proportion of variance in the dependent variable that is predictable from the independent variables. In the context of cell quantification in pathology, $R^2$ is used to assess how well the predicted quantities of different cell types in a patch align with the actual quantities observed in the ground truth data, with higher values representing more accurate quantification. $R^2$ is defined as
\begin{equation*}
R^2 = 1 - \frac{\sum_{i=1}^n (y_i - \hat{y}_i)^2}{\sum_{i=1}^n (y_i - \bar{y})^2},
\end{equation*}
where $y_i$ represents the actual number of cells of a specific type in the $i$-th image, $\hat{y}_i$ represents the predicted number of cells of that type in the $i$-th image, $\bar{y}$ is the mean of the actual numbers across all images, and $n$ is the total number of images in the dataset.

The $R^2$ metric has a range of $(-\infty, 1]$. An $R^2$ of 1 indicates perfect prediction, where all predicted values exactly match the actual values. An $R^2$ of 0 suggests that the model explains none of the variability of the response data around its mean. If $R^2$ is negative, it indicates that the model performs worse than a model that simply predicts the mean of the actual values for all observations.

\subsubsection{\label{chap:S5.2}PQ}
Panoptic Quality ($PQ$) is a comprehensive metric used to evaluate the performance of segmentation models in tasks that require both instance segmentation and classification. $PQ$ provides a single score that encapsulates both the detection accuracy (i.e., how many objects were correctly identified) and the segmentation quality (i.e., how accurately the objects' boundaries were delineated). This metric is particularly useful in multiclass scenarios where each pixel is classified into distinct categories, such as different cell types in pathology images.

$PQ$ is calculated as the product of two terms: Detection Quality ($DQ$) and Segmentation Quality ($SQ$). It can be expressed as
\begin{equation*}
PQ = DQ \cdot SQ,
\end{equation*}
where
\begin{equation*}
DQ = \frac{TP}{TP + 0.5\, FP + 0.5\, FN},
\end{equation*}
\begin{equation*}
SQ = \frac{\sum_{(p, g) \in \mathcal{M}} IoU(p, g)}{TP}.
\end{equation*}
In these formulas, $TP$ denotes the number of correctly matched instances between ground truth and prediction, $FP$ denotes the predicted instances that have no corresponding ground truth, $FN$ denotes the ground truth instances that were not detected, $IoU(p, g)$ is the Intersection over Union for a pair of matched instances $p$ (prediction) and $g$ (ground truth), and $\mathcal{M}$ is the set of matched pairs.

The $PQ$ metric is calculated for each class and is averaged across classes to provide a global performance measure.

The $PQ$ score has a range of $[0, 1.0]$, where a higher score indicates better performance in both detecting and segmenting the instances correctly. A $PQ$ of 1 signifies perfect identification and segmentation of all instances, whereas a $PQ$ of 0 indicates that no instances were correctly identified and segmented.

\clearpage

\subsection{\label{chap:S6}Segmentation and Detection quality metrics for teacher and student models}

\begin{table}[h!]
\renewcommand{\arraystretch}{2.0}
\centering
\caption{Segmentation and detection quality for student and teacher models (CI 95\%)}
\label{tab:S6}
%\adjustbox{max width=\textwidth}{%
\begin{tabular}{|l|c|c|}
\hline
%\rowcolor{gray!30}
Metric & Teacher & Student \\
\hline
$SQ_{neoplastic}$ & 0.819 (0.815--0.823) & 0.824 (0.819--0.828) \\
\hline
$SQ_{lymphocyte}$ & 0.795 (0.788--0.802) & 0.790 (0.783--0.796) \\
\hline
$SQ_{connective}$ & 0.770 (0.762--0.776) & 0.780 (0.772--0.786) \\
\hline
$SQ_{dead}$ & 0.659 (0.623--0.688) & 0.657 (0.624--0.695) \\
\hline
$SQ_{epithelial}$ & 0.780 (0.770--0.790) & 0.788 (0.779--0.797) \\
\hline
$SQ_{macrophage}$ & 0.788 (0.760--0.810) & 0.757 (0.730--0.783) \\
\hline
$SQ_{neutrofil}$ & 0.782 (0.761--0.801) & 0.775 (0.759--0.792) \\
\hline
$DQ_{neoplastic}$ & 0.706 (0.692--0.719) & 0.727 (0.712--0.741) \\
\hline
$DQ_{lymphocyte}$ & 0.675 (0.656--0.698) & 0.713 (0.691--0.734) \\
\hline
$DQ_{connective}$ & 0.566 (0.546--0.584) & 0.583 (0.565--0.602) \\
\hline
$DQ_{dead}$ & 0.410 (0.361--0.465) & 0.435 (0.306--0.561) \\
\hline
$DQ_{epithelial}$ & 0.668 (0.639--0.694) & 0.673 (0.644--0.702) \\
\hline
$DQ_{macrophage}$ & 0.657 (0.583--0.727) & 0.615 (0.531--0.703) \\
\hline
$DQ_{neutrofil}$ & 0.691 (0.625--0.753) & 0.729 (0.679--0.778) \\
\hline
\end{tabular}
%
%}
\end{table}

\clearpage

\subsection{\label{chap:S7}QuPath integration method}
We adopt an integration strategy leveraging the paquo \cite{Bayer_AG} library, a Python package that enables direct interaction with QuPath’s internal API, thereby facilitating seamless data exchange without intermediate conversion steps. The data processing pipeline (\hyperref[fig:S7]{Appendix Figure S7}) begins with the acquisition of WSIs and their associated annotations from QuPath, which are represented as Shapely \cite{Gillies_Wel_etal._2024} polygons. Utilizing paquo, we directly read, create, and modify these annotations and detections within a QuPath project in the Python environment. Images are then cropped using these polygons and processed by cell segmentation and classification models employing standard vision processing toolkits such as OpenCV, pyvips, and PyTorch. Additionally, QuPath employs Groovy scripts to initiate a Python process that starts the entire pipeline from QuPath graphical interface: fetching polygons, extracting images from them, and running deep learning model inference on the cropped images. 
The results are returned to QuPath, leveraging paquo's Python bindings to manipulate QuPath data while minimizing the computational overhead typically associated with cross-environment communication.

\counterwithin{figure}{subsection}
\renewcommand{\thefigure}{S\arabic{subsection}}

\begin{figure}[h!]
    \centering
    \includegraphics[width=\textwidth]{images/A7.pdf}
    \caption{QuPath integration workflow using Python environment}
    \label{fig:S7}
\end{figure}

Compared to traditional workflows that involve exporting annotations as GeoJSON, classifying them in Python, and reimporting them into QuPath, our approach offers several advantages. We eliminate the need to switch between programming languages, providing a cohesive and streamlined development process entirely within QuPath software and removing the necessity to use other tools. Meanwhile, we avoid storing annotations as intermediate JSON files unless required for external use or archiving. By conducting the entire inference and post-processing workflow within the Python environment, we leverage the power and flexibility of Python libraries for image processing and machine learning. This approach also enables adjustments to any set of labels and models, thereby improving its applicability.

%\hfill

The distilled model and QuPath integration code are packaged into a Docker container, enabling streamlined execution with the Docker engine. Detailed integration code and deployment instructions can be found in the GitHub repository \cite{Shvetsov_2025b}.

Despite these benefits, we acknowledge that the paquo library is a proof‑of‑concept project in its early development stage and has not been tested across all versions of QuPath.

\clearpage

\subsection{\label{chap:S8}Data and code availability statement}
All datasets, models, and code used in this study are publicly available and can be obtained from the repositories listed below. 
The PanNuke \cite{Gamper_Koohbanani_etal._2019} and MoNuSAC \cite{Verma_Kumar_etal._2021} datasets are publicly accessible, and download information along with detailed descriptions can be found in their respective articles. Preprocessing scripts for PanNuke and MoNuSAC data, as well as individual cell extraction scripts, are available on GitHub \cite{Shvetsov_2025a}. The H-Optimus foundation model used in our experiments can be downloaded from the HuggingFace repository \cite{hoptimus2024}, and model information is available on GitHub \cite{Saillard_Jenatton_etal._2024}. In addition, the integration code for QuPath and the distilled model packaged in a Docker container are provided in the repository \cite{Shvetsov_2025b}, and paquo Python library is available from the authors GitHub repository \cite{Bayer_AG}.
\clearpage

\end{document}

\newpage
\section{Appendices}

\subsection{Table of Notations and Acronyms} \label{sec:table_notation}
\begin{table}[h!]
\centering
\begin{tabular}{|l|p{0.65\textwidth}|}
\hline
\textbf{Notation/Acronym} & \textbf{Description} \\ \hline
$k$ & number of latent factors \\ \hline
$p$ & input dimension \\ \hline
PCA & Principal Component Analysis \\ \hline
PCR & Principal Component Regression \\ \hline
vanillaNN & A vanilla ReLU-Neural Network model \\ \hline
oracleNN & A vanilla ReLU-Neural Network model fed with \textbf{latent factors} \\ \hline
FAR-NN & Factor Augmented Regression Using Neural Network \\ \hline
FAST-NN & Factor Augmented Sparse Throughput Neural Network \\ \hline
GFNN & Generalized Factor Neural Network Model \\ \hline
GFADNN & Generalized Factor Additive Neural Network Model, GFNN with additive layers. \\ \hline
GFANN & Generalized Factor Augmented Neural Network Model, GFNN with augmented residual components. \\ \hline
$PCA\_NN$ & A GFNN model with the PCA layer as the first layer, followed by standard layers. \\ \hline
$SPCA\_NN$ & A GFNN model with the Soft PCA layer as the first layer, followed by standard layers. \\ \hline
$PCA\_NN\_PCA\_ADD$ & A GFADNN model with a PCA layer, standard layer(s), another PCA layer, then followed by additive layer(s). \\ \hline
$PCA\_NN\_ADD\_PCA$ & A GFADNN model with a PCA layer, standard layer(s), additive layer(s), then followed by another PCA layer. \\ \hline
\end{tabular}
\caption{Table of Notations and Acronyms}
\label{table:notation_acronyms}
\end{table}

\newpage

\subsection{Rationale for Model Comparison in Simulation Studies} \label{sec:model_comparison}
There is a substantial body of literature that compares neural network models using identical hyperparameter settings. However, hyperparameter settings are highly specific to each model, and therefore, for a fair comparison, we aim to select the best hyperparameter setting for each individual model. To achieve this, we systematically vary crucial hyperparameters, such as learning rates, network width, network depth, and batch size. Upon establishing the search space, hyperparameter optimization is conducted using Optuna, with the adoption of the Tree-structured Parzen Estimator (TPE) algorithm—a Bayesian optimization technique—for the tuning process. The configuration that results in the minimal validation error is selected for each model. Subsequently, we vary the random seeds to generate over 20 random samples. For each sample, models are trained, the validation data is used for early stopping, and test errors are recorded. Ultimately, model performance is compared based on their average test error. In particular, the performance of the estimator $\hat{m}(x)$ is evaluated via the empirical mean squared error.

\subsection{Hyperparameter Sensitivity}

Through the hyperparameter tuning process, we gain insights into the significance of various hyperparameters for each model type, specifically focusing on \(FAR\text{-}NN\), \(PCA\_NN\_PCA\_ADD\), and \(SPCA\_NN\_SPCA\_ADD\). By evaluating the relative importance of these hyperparameters, with the learning rate as a reference point, we observe that the learning rate has larger relative importance in Figure~\ref{fig:hyper_importance}(b) \(PCA\_NN\_PCA\_ADD\) and Figure~\ref{fig:hyper_importance}(c) \(SPCA\_NN\_SPCA\_ADD\). In contrast, in Figure~\ref{fig:hyper_importance}(a) \(FAR\text{-}NN\), hyperparameters such as batch size and architecture-related features, including width and depth, exhibit comparable influence to the learning rate. This suggests a heightened sensitivity of the model to architectural adjustments, relative to learning rate changes.

Conversely, for \(PCA\_NN\_PCA\_ADD\) and \(SPCA\_NN\_SPCA\_ADD\), architecture-related hyperparameters demonstrate less significance relative to the learning rate, providing insights into the robustness of this model's architecture.


\begin{figure}[H]
% \captionsetup{labelformat=empty} 
\centering

% The first plot
\begin{subfigure} % Specify width as a fraction of text width
    \centering
    % \captionsetup{labelformat=empty}
    \includegraphics[width=0.95\textwidth]{images/hyperpara_farnn.png}
    \caption*{(a) FAR-NN}
    \label{subfig:hyper_importance_FAR-NN}
\end{subfigure}
\hfill % Adds space between subfigures

% The second plot
\begin{subfigure}% Specify width as a fraction of text width
    \centering
    % \captionsetup{labelformat=empty}
    \includegraphics[width=0.95\textwidth]{images/hyperpara_ori_pca_pca_add.png}
    \caption*{(b) PCA\_NN\_PCA\_ADD}
    \label{subfig:hyper_importance_PCA_NN_PCA_ADD}
\end{subfigure}

\vspace{0.5cm} % Adds vertical space between rows of subfigures

% The third plot
\begin{subfigure} % Full width for this subfigure
    \centering
    \includegraphics[width=0.95\textwidth]{images/hyperpara_pca_pca_add.png}
    \caption*{(c) SPCA\_NN\_SPCA\_ADD}
    \label{subfig:hyper_importance_SPCA_NN_SPCA_ADD}
\end{subfigure}

\caption{Hyperparameter Importances diagram of (a) FAR-NN, (b) PCA\_NN\_PCA\_ADD, and (c) SPCA\_NN\_SPCA\_ADD. This presents the relative importance of the hyperparameters tuned in the training process}
\label{fig:hyper_importance}
\end{figure}




\subsection{Ablation study}


\begin{figure}[H]
    \centering
    \includegraphics[width=\columnwidth]{images/ablation1.png}
    \caption{test mse against dimension when observations follow a linear factor model. The benchmark models are PCA\_NN\_PCA\_ADD in blue and SPCA\_NN\_SPCA\_ADD in red.}
    \label{fig:ablation1}
\end{figure}

% \begin{tabular}{|l|l|} 
% \hline
% Observations & {$B \times g_1\left(\left[z_1, z_2, z_3\right]\right)$} \\
% \hline
% \end{tabular}\\

\begin{figure}[H]
    \centering
    \includegraphics[width=\columnwidth]{images/ablation2.png}
    \caption{test mse against dimension when observations follow a non-linear factor model. The benchmark models are PCA\_NN\_PCA\_ADD in blue and SPCA\_NN\_SPCA\_ADD in red.}
    \label{fig:ablation2}
\end{figure}

In this section, we evaluate the performance implications of modifying key components within our GFADNN model. Specifically, we examine the impact of replacing the Additive layer with a simple linear layer, resulting in the models \( PCA\_NN\_PCA \) and \( SPCA\_NN\_SPCA \). Additionally, we investigate the effect of swapping the second (Soft) PCA layer with the Additive layers, producing the models \( PCA\_NN\_ADD\_PCA \) and \( SPCA\_NN\_ADD\_SPCA \). Figures \ref{fig:ablation1} and \ref{fig:ablation2} display the average test MSE of these models across 20 random seeds for different dimensions. The results indicate that either removing the Additive layers or changing their order negatively impacts performance, highlighting their significance in the model's architecture.




% \subsection{An illustration of multiple PCA layers}
% \begin{figure}[H]
% \centering
%     \includegraphics[width=\columnwidth]{images/Factor_Augmented_Neural_Networks_Composite_Model.png}
%     \caption{A visual illustration of integrating multiple PCA layers into neural networks. The layers in red denote the PCA layers. }  
% \label{fig:pca_pca_nn}
% \end{figure}


\subsection{An illustration of additive layers} \label{sec:additive_layer_graph}
\begin{figure}[H]
    \centering
    \includegraphics[width=\columnwidth]{images/additive_layer.PNG}
    \caption{Additive networks, without interactions among neurons across different sub-networks.}
\label{fig:additive_layer}
\end{figure}
\subsection{Computational graph involving the PCA layer}
\label{sec:computation_graph}
\begin{figure}[!h]
\centering
    \includegraphics[height = 21cm, clip ]{images/loss_simplify_torch.png}
    \caption{An illustration of the computational graph, highlighting the tracking of gradients during a PCA operation.}
\label{fig:loss_nontorch}
\end{figure}

\end{document}

\endinput
%%
%% End of file `elsarticle-template-num.tex'.
