\vspace{-2mm}
\section{Method}
\label{Method}
In this section, we introduce ParallelComp, our proposed approach for achieving efficient long-context inference. Then we introduce its unique bias phenomenon. Figure~\ref{fig:overview} offers a high-level overview of  ParallelComp. 
% \vspace{-1mm}
\subsection{ParallelComp}
\label{ParallelComp}
% Next, we describe ParallelComp. We will first introduce some of our key observations, followed by the details of the ParallelComp algorithm inspired by our insights. Figure~\ref{fig:overview} provides a high-level overview of inference using ParallelComp, which includes the compression and ranking components.



% \vspace{-2mm}
\paragraph{Parallel Attention.}
Inspired by previous studies~\citep{chen2023longlora,an2024training}, we split the text into chunks according to the model's maximum context size and concatenate them with the input query for parallel encoding. This step is typically performed using local attention. For a given input sequence \( X \in \mathbb{R}^{N \times d} \), the sequence is divided into \( C = \lceil N / w \rceil\) chunks, each containing at most \( w \) tokens (the maximum context length of each chunk). Let \( f_Q(\cdot) \), \( f_K(\cdot) \), and \( f_V(\cdot) \) represent the linear transformation functions for query, key, and value projections, respectively. Then, the attention computation is performed parallelly within each chunk\footnote{We use multi-head attention in implementation but omit the head information for simplicity in our description.}:
\begin{equation}
A^{c}_\mathfrak{l} = \text{Softmax}\left(\frac{f_Q({X^{c}}) \cdot f_K(X^{c})^T}{\sqrt{d}}\right),
\end{equation}
where \( X^{c} \in \mathbb{R}^{w \times d} \) represents the \( c \)-th chunk of the input sequence and \( A^{c}_\mathfrak{l} \in \mathbb{R}^{w \times w} \) is the corresponding attention matrix. The feature update is performed for each chunk:
\begin{equation}
F^{c} = A^{c}_\mathfrak{l} \cdot f_V(X^{c}),
\end{equation}
where \( F^{c} \in \mathbb{R}^{w \times d} \) is the updated feature for the \( c \)-th chunk. 



Below, we discuss how to design chunk eviction strategies and parallel KV cache eviction strategies to maintain high throughput while minimizing redundant computations.

\vspace{-3mm}
\paragraph{Chunk Eviction.}

To ensure that the computation of parallel attention can be performed on a single 80GB A100 GPU, we design a chunk eviction strategy to control memory overhead as shown in Figure~\ref{fig:overview} step 3. Inspired by \citet{ye2023compositional}, we introduce a chunk eviction mechanism that leverages the self-information of the query tokens \( X^q \) to further enhance parallel processing efficiency. This mechanism utilizes an online priority queue to manage memory, retaining only the most relevant chunks, thereby enhancing language modeling. For a given chunk \( c \), the self-information score for the query tokens \( X^q \) is calculated as follows: 
\begin{equation}
I_c = -\log P({X^q} \mid X^c),
\end{equation}
where \( X^c \) represents the context of chunk \( c \) and \( X^q \) corresponds to the chunk of the query. Chunks with lower self-information scores are considered more relevant and are retained. The set of indices \(c\) corresponding to the selected chunks is denoted by: 
\begin{equation}
S = \{ c \mid I_c \leq \tau \},  
\end{equation}  
where \( \tau \) is a threshold that determines whether a chunk will be selected or not. The selected set of chunks is stored in a fixed-size priority queue to ensure that the prefilling stage remains within the memory limit.


\paragraph{Parallel KV Cache Eviction.}


% where \( S_c \in \mathbb{R}^{w}\) represents the cumulative attention score for each chunk token, \( A_l \) is the local attention score between the \( i \)-th token of the query \( X^q \) and the \( j \)-th token of the chunk \( X^c \), and \( w_q \) denotes the length of the query sequence, and \( w \) is the length of the chunk sequence. The cumulative attention score sums the attention distributions of each token in the query to each token in the chunk. Tokens with low cumulative attention scores within the chunk are evicted, and the retained tokens are used to form the compressed KV cache.


To further increase the throughput of chunks, we design a KV cache eviction strategy as shown in Figure~\ref{fig:overview} step 2. We utilize Flash Attention to perform fast attention computation. Since it is hard to obtain the complete attention distribution from Flash Attention~\citep{dao2023flashattention}, we use the \textbf{local attention} of \( X^q \) to quickly estimate tokens with relatively low attention scores and evict them in advance before sending them into Flash Attention:
\begin{equation}
\label{cum}
S_{c,j} = \sum_{i=1}^{w_q} A^c_\mathfrak{l}(X^q_i, X^c_j), \quad j = 1, 2,...,w,
\end{equation}
where \( S_c \in \mathbb{R}^{w} \) represents the cumulative attention score for each token in the chunk and \( A^c_\mathfrak{l} \) is the local attention score between the \( i \)-th token of the query \( X^q \) and the \( j \)-th token of the chunk \( X^c \), \( w_q \) denotes the length of the query sequence. The cumulative attention score aggregates the attention distributions from each token in the query to each token in the chunk, thereby measuring the relevance of each token in the chunk to the query $X^q$. Tokens with low cumulative attention scores within the chunk are evicted, and the retained tokens are used to form the compressed KV cache.
\begin{equation}
K_r = K_x[R_l], \quad V_r = V_x[R_l],
\end{equation}
where \( K_x \) and \( V_x \) represent the KV cache of the input chunk, and \( R_l \) denotes the set of indices corresponding to the evicted tokens with low attention scores. The notation \( [\cdot] \) indicates indexing into \( K_x \) and \( V_x \) to evict only the tokens corresponding to the indices in \( R_l \). $K_r$ and $V_r$ denote the retained KV cache. The compression strategy of the KV cache typically helps reduce memory overhead while increasing chunk throughput, but it often exacerbates attention bias. Next, we introduce a simple and effective strategy to calibrate attention distribution.

\paragraph{Attention Calibration.} To alleviate the attention bias exacerbated by Parallel KV Cache Eviction, we design another token eviction strategy using Eq.~\ref{cum}. Specifically, we evict tokens with excessively high attention scores. Let \( R_h \) correspond to those with extremely high attention scores exceeding \( \lambda \) (a manually-set threshold), then we have:
\begin{equation}  
K'_r = K_r[R_h], \quad V'_r = V_r[R_h].  
\end{equation}  
 Evicting tokens with exceptionally high scores guarantees that the segmented computation of softmax in Flash Attention can produce calibrated attention distributions. We will thoroughly investigate the impact of this calibration method on the attention distribution in Section~\ref{Empirical Study of Parallel Attention Bias}.




\paragraph{Global Attention.}  

After obtaining the attention outputs for each chunk, we concatenate the key-value (KV) caches from all chunks into a unified representation. Specifically, the concatenated KV cache is given by: 
\begin{equation}
\small
\begin{aligned}
K =& \big[K^{X^1}, K^{X^2}, \dots, K^{X^C}, K^{X^q}\big], \\
V =& \big[V^{X^1}, V^{X^2}, \dots, V^{X^C}, V^{X^q}\big],
\end{aligned}
\end{equation}
where \( K^{X^c} \) and \( V^{X^c} \) are the key and value projections of the \( c \)-th chunk.


Next, we perform a global attention operation. This global attention enables the model to aggregate information across all chunks, ensuring that global dependencies are captured. The global attention computation for \( X^q \) is given by:
\begin{equation}
A_{\mathfrak{g}} = \text{Softmax}\left(\frac{f_Q(X^q) \cdot K^T}{\sqrt{d}}\right),
\end{equation}
where \( A_{\mathfrak{g}} \in \mathbb{R}^{w_q \times (C \cdot w + w_q)} \) is the global attention score matrix for the query chunk. The corresponding output of the global attention is computed as:
\begin{equation}
F_{\mathfrak{g}} = A_{\mathfrak{g}} \cdot V,
\end{equation}
where \( F_{\mathfrak{g}} \in \mathbb{R}^{w_q \times d} \) represents the globally updated features for the query chunk. Finally, the updated global representation is passed through the feedforward and autoregressive decoding stages, enabling the model to generate outputs while leveraging information from all chunks efficiently.














% 
\section{\thename}
\subsection{End-to-End Driving Policy}
The overall framework of \thename{} is depicted in Fig.~\ref{fig:framework}. 
\thename{} takes multi-view image sequences as input, transforms the sensor data into scene token embeddings, outputs the probabilistic distribution of actions, and samples an action to control the vehicle. 

\boldparagraph{BEV Encoder.} 
We first employ a BEV encoder~\cite{li2022bevformer} to transform multi-view image features from the perspective view to the Bird's Eye View (BEV), obtaining a feature map in the BEV space. This feature map is then used to learn instance-level map features and agent features.

\boldparagraph{Map Head.} 
Then we utilize a group of map tokens~\cite{maptrv2, liao2022maptr, lanegap} to learn the vectorized map elements of the driving scene from the BEV feature map, including lane centerlines, lane dividers, road boundaries, arrows, traffic signals, \etc.

\boldparagraph{Agent Head.} 
Besides, a group of agent tokens~\cite{jiang2022pip} is adopted to predict the motion information of other traffic participants, including location, orientation, size, speed, and multi-mode future trajectories.

\boldparagraph{Image Encoder.} 
Apart from the above instance-level map and agent tokens, we also use an individual image encoder~\cite{vit,he2016resnet} to transform the original images into image tokens. These image tokens provide dense and rich scene information for planning, complementary to the instance-level tokens.

\begin{figure}[t]
\centering
\includegraphics[width=0.98\linewidth]{fig/post-training-2.pdf} 
\caption{\textbf{Post-training.}  $N$  workers parallelly run. The generated rollout data $(s_t,a_t, r_{t+1},s_{t+1},...)$ are recorded in a rollout buffer. Rollout data and human driving demonstrations are used in RL- and IL-training steps to fine-tune the AD policy synergistically.
}
\label{fig:post-training}
\end{figure}

\boldparagraph{Action Space.} 
To accelerate the convergence of RL training, we design a decoupled discrete action representation. 
We divide the action into two independent components: lateral action and longitudinal action. 
The action space is constructed over a short $0.5$-second time horizon, during which the vehicle's motion is approximated by assuming constant linear and angular velocities. 
Under this assumption, the lateral action $a^x$ and longitudinal action $a^y$ can be directly computed based on the current linear and angular velocities.
By combining decoupling with a limited temporal scope and simplified motion model, our approach effectively reduces the dimensionality of the action space, accelerating training convergence.


\boldparagraph{Planning Head.} 
We use $E_\text{scene}$ to denote the scene representation, which consists of map tokens, agent tokens, and image tokens. We initialize a planning embedding denoted as $E_\text{plan}$. A cascaded Transformer decoder $\phi$ takes the planning embedding $E_\text{plan}$ as the query and the scene representation $E_\text{scene}$ as both key and value.

The output of the decoder $\phi$ is then combined with navigation information $E_\text{navi}$ and ego state $E_\text{state}$ to output the probabilistic distributions of the lateral action $a^x$ and the longitudinal action $a^y$:
\begin{equation}
\begin{aligned}
     \pi(a^x\mid s) = & \text{softmax}(\text{MLP}(\phi(E_\text{plan}, E_\text{scene}) \\
    & + E_\text{navi} + E_\text{state})), \\
     \pi(a^y\mid s) = & \text{softmax}(\text{MLP}(\phi(E_\text{plan}, E_\text{scene}) \\
     & + E_\text{navi} + E_\text{state})),
\label{eq:action distribution}
\end{aligned}
\end{equation}
where $E_\text{plan}$, $E_\text{navi}$, $E_\text{state}$, and the output of $\text{MLP}$ are all of the same dimension ($1 \times D$).

The planning head also outputs the value functions $V_x(s)$ and $V_y(s)$, which estimate the expected cumulative rewards for the lateral and longitudinal actions, respectively: 
\begin{equation}
\begin{aligned}
    & V_x(s) = \text{MLP}(\phi(E_\text{plan}, E_\text{scene}) + E_\text{navi} + E_\text{state}), \\
    & V_y(s) = \text{MLP}(\phi(E_\text{plan}, E_\text{scene}) + E_\text{navi} + E_\text{state}).
\end{aligned}
\end{equation}
The value functions are used in RL training (Sec.~\ref{sec:optimization}).

\subsection{Training Paradigm}
We adopt a three-stage training paradigm: perception pre-training, planning pre-training, and reinforced post-training, as shown in Fig.~\ref{fig:framework}.

\boldparagraph{Perception Pre-Training.} 
Information in the image is sparse and low-level. In the first stage,  
the map head and the agent head explicitly output map elements and agent motion information, which are supervised with ground-truth labels. Consequently,  
map tokens and agent tokens implicitly encode the corresponding high-level information.  
In this stage, we only update the parameters of the BEV encoder, the map head, and the agent head.



\boldparagraph{Planning Pre-Training.} 
In the second stage, to prevent the unstable cold start of RL training, IL is first performed to initialize the probabilistic distribution of actions based on large-scale real-world driving demonstrations from expert drivers. In this stage, we only update the parameters of the image encoder and the planning head, while the parameters of the BEV encoder, map head, and agent head are frozen. The optimization objectives of perception tasks and planning tasks may conflict with each other. However, with the training stage and parameters decoupled, such conflicts are mostly avoided.

\boldparagraph{Reinforced Post-Training.} 
In the reinforced post-training, RL and IL synergistically fine-tune the distribution. RL aims to guide the policy to be sensitive to critical risky events and adaptive to out-of-distribution situations. IL serves as the regularization term to keep the policy's behavior similar to that of humans.

We select a large amount of risky dense-traffic clips from collected driving demonstrations. For each clip, we train an independent 3DGS model that reconstructs the clip and serves as a digital driving environment.  
As shown in Fig.~\ref{fig:post-training}, we set $N$ parallel workers.  
Each worker randomly samples a 3DGS environment and begins rollout, i.e., the AD policy controls the ego vehicle to move and iteratively interacts with the 3DGS environment. After the rollout process of this 3DGS environment ends, the generated rollout data $(s_t,a_t, r_{t+1},s_{t+1},...)$ are recorded in a rollout buffer, and the worker will sample a new 3DGS environment for another round of rollout.

As for policy optimization, we iteratively perform RL-training steps and IL-training steps. For RL-training steps, we sample data from the rollout buffer and follow the Proximal Policy Optimization (PPO) framework~\cite{PPO} to update the AD policy. For IL-training steps, we use real-world driving demonstrations to update the policy. After a fixed number of training steps, the updated AD policy is sent to every worker to replace the old one, to avoid a distribution shift between data collection and optimization.
We only update the parameters of the image encoder and the planning head. The parameters of the BEV encoder, the map head, and the agent head are frozen.  
The detailed RL design is presented below.

\subsection{Interaction Mechanism between AD Policy and 3DGS Environment}
In the 3DGS environment, the ego vehicle acts according to the AD policy. Other traffic participants act according to real-world data in a log-replay manner.  
A simplified kinematic bicycle model is employed to iteratively update the ego vehicle's pose at every $\Delta t$ seconds as follows:  
\begin{equation}
\begin{aligned}
x_{t+1}^{w} & = x_{t}^w + v_t \cos \left(\psi_{t}^w\right) \Delta t, \\
y_{t+1}^{w} & = y_{t}^w + v_t \sin \left(\psi_{t}^w\right) \Delta t, \\
\psi_{t+1}^{w} & = \psi_{t}^w + \frac{v_t}{L} \tan \left(\delta_t\right) \Delta t,
\label{equation:kinematic_model}
\end{aligned}
\end{equation}  
where $x_t^{w}$ and $y_t^{w}$ denote the position of the ego vehicle relative to the world coordinate; $\psi_t^w$ is the heading angle that defines the vehicle's orientation with respect to the world $x$-coordinate; $v_t$ is the linear velocity of the ego vehicle; $\delta_t$ is the steering angle of the front wheels; and $L$ is the wheelbase, i.e., the distance between the front and rear axles.

During the rollout process, the AD policy outputs actions $(a_t^x, a_t^y)$ for a $0.5$-second time horizon at time step $t$. We derive the linear velocity $v_t$ and steering angle $\delta_t$ based on $(a_t^x, a_t^y)$.  
Based on the kinematic model in Eq.~\ref{equation:kinematic_model},  
the pose of the ego vehicle in the world coordinate system is updated from ${p}_t = (x_{t}^w, y_{t}^w, \psi_{t}^w)$ to ${p}_{t+1} = (x_{t+1}^{w}, y_{t+1}^{w}, \psi_{t+1}^{w})$.  

Based on the updated ${p}_{t+1}$, the 3DGS environment computes the new ego vehicle's state $s_{t+1}$. The updated pose ${p}_{t+1}$ and state $s_{t+1}$ serve as the input for the next iteration of the inference process.

The 3DGS environment also generates rewards $\mathcal{R}$ (Sec.~\ref{sec:reward}) according to multi-source information (including trajectories of other agents, map information, the expert trajectory of the ego vehicle, and the parameters of Gaussians), which are used to optimize the AD policy (Sec.~\ref{sec:optimization}).

\begin{figure}[t]
\centering
\includegraphics[width=1.0\linewidth]{fig/reward.pdf} 
\caption{\textbf{Example diagram of four types of reward sources.}  (1): Collision with a dynamic obstacle ahead triggers a reward $r_{\text{dc}}$. (2): Hitting a static roadside obstacle incurs a reward $r_{\text{sc}}$. (3): Moving onto the curb exceeds the positional deviation threshold $d_{\text{max}}$, triggering a reward $r_{\text{pd}}$. (4): Drifting toward the adjacent lane exceeds the heading deviation threshold $\psi_{\text{max}}$, triggering a reward $r_{\text{hd}}$.
}
\label{fig: reward source}
\end{figure}
\subsection{Reward Modeling}
\label{sec:reward}
The reward is the source of the training signal, which determines the optimization direction of RL. The reward function is designed to guide the ego vehicle's behavior by penalizing unsafe actions and encouraging alignment with the expert trajectory. It is composed of four reward components: (1) collision with dynamic obstacles, (2) collision with static obstacles, (3) positional deviation from the expert trajectory, and (4) heading deviation from the expert trajectory:
\begin{equation}
\begin{aligned}
\mathcal{R} = \{r_{\text{dc}}, r_{\text{sc}}, r_{\text{pd}}, r_{\text{hd}}  \}. 
\end{aligned}
\end{equation}

As illustrated in Fig.~\ref{fig: reward source}, these reward components are triggered under specific conditions.  
In the 3DGS environment, dynamic collision is detected if the ego vehicle's bounding box overlaps with the annotated bounding boxes of dynamic obstacles, triggering a negative reward $r_{\text{dc}}$. Similarly, static collision is identified when the ego vehicle's bounding box overlaps with the Gaussians of static obstacles, resulting in a negative reward $r_{\text{sc}}$.  
Positional deviation is measured as the Euclidean distance between the ego vehicle's current position and the closest point on the expert trajectory. A deviation beyond a predefined threshold $d_{\text{max}}$ incurs a negative reward $r_{\text{pd}}$.  
Heading deviation is calculated as the angular difference between the ego vehicle's current heading angle $ \psi_t $ and the expert trajectory's matched heading angle $\psi_{\text{expert}}$. A deviation beyond a threshold $ \psi_{\text{max}}$ results in a negative reward $r_{\text{hd}}$.

Any of these events, including dynamic collision, static collision, excessive positional deviation, or excessive heading deviation, triggers immediate episode termination. Because after such events occur, the 3DGS environment typically generates noisy sensor data, which is detrimental to RL training.

\subsection{Policy Optimization}
\label{sec:optimization}
In the closed-loop environment, the error in each single step accumulates over time. The aforementioned rewards are not only caused by the current action but also by the actions of the preceding steps.  
The rewards are propagated forward with Generalized Advantage Estimation (GAE)~\cite{gae} to optimize the action distribution of the preceding steps.

Specifically, for each time step $t$, we store the current state $s_t$, action $a_t$, reward $r_t$, and the estimate of the value $V(s_t)$.  
Based on the decoupled action space, and considering that different rewards have different correlations to lateral and longitudinal actions, the reward $r_t$ is divided into lateral reward $r_t^x$ and longitudinal reward $r_t^y$:
\begin{equation}
\begin{aligned}
r_t^x &= r_t^{\text{sc}} + r_t^{\text{pd}} + r_t^{\text{hd}}, \\
r_t^y &= r_t^{\text{dc}}.
\label{eq:reward-decouple}
\end{aligned}
\end{equation}
Similarly, the value function $V(s_t)$ is decoupled into two components: $V_x(s_t)$ for the lateral dimension and $V_y(s_t)$ for the longitudinal dimension. These value functions estimate the expected cumulative rewards for the lateral and longitudinal actions, respectively. The advantage estimates $\hat{A}_t^x$ and $\hat{A}_t^y$ are then computed as follows:
\begin{equation}
\begin{aligned}
\delta_t^x &= r_t^x + \gamma V_x(s_{t+1}) - V_x(s_t), \\
\delta_t^y &= r_t^y + \gamma V_y(s_{t+1}) - V_y(s_t), \\
\hat{A}_t^x &= \sum_{l=0}^{\infty}(\gamma \lambda)^l \delta_{t+l}^x, \\
\hat{A}_t^y &= \sum_{l=0}^{\infty}(\gamma \lambda)^l \delta_{t+l}^y,
\label{eq:advantage}
\end{aligned}
\end{equation}
where $\delta_t^x$ and $\delta_t^y$ are the temporal difference errors for the lateral and longitudinal dimensions, $\gamma$ is the discount factor, and $\lambda$ is the GAE parameter that controls the trade-off between bias and variance.

To further clarify the relationship between the advantage estimates and the reward components, we decompose $\hat{A}_t^x$ and $\hat{A}_t^y$ based on the reward decomposition in Eq.~\ref{eq:reward-decouple} and the advantage estimation in Eq.~\ref{eq:advantage}. Specifically, we derive the following decomposition:
\begin{equation}
\begin{aligned}
\hat{A}_t^x &= \hat{A}_t^{\text{sc}} + \hat{A}_t^{\text{pd}} + \hat{A}_t^{\text{hd}}, \\
\hat{A}_t^y &= \hat{A}_t^{\text{dc}},
\end{aligned}
\end{equation}
where $\hat{A}_t^{\text{sc}}$ is the advantage estimate for avoiding static collisions, $\hat{A}_t^{\text{pd}}$ is the advantage estimate for minimizing positional deviations, $\hat{A}_t^{\text{hd}}$ is the advantage estimate for minimizing heading deviations, and $\hat{A}_t^{\text{dc}}$ is the advantage estimate for avoiding dynamic collisions.

These advantage estimates are used to guide the update of the AD policy $\pi_{\theta}$, following the PPO framework~\cite{PPO}. By leveraging the decomposed advantage estimates $\hat{A}_t^x$ and $\hat{A}_t^y$, we can independently optimize the lateral and longitudinal dimensions of the policy. This is achieved by defining separate objective functions $\mathcal{L}_x^{\text{CLIP}}(\theta)$ and $\mathcal{L}_y^{\text{CLIP}}(\theta)$ for each dimension,  as follows:
\begin{equation}
\begin{aligned}
\mathcal{L}_x^{\text{PPO}}(\theta) &= \mathbb{E}_t \left[ \min \left( \rho_t^x \hat{A}_t^x, \ \text{clip}(\rho_t^x, 1-\epsilon_x, 1+\epsilon_x) \hat{A}_t^x \right) \right], \\
\mathcal{L}_y^{\text{PPO}}(\theta) &= \mathbb{E}_t \left[ \min \left( \rho_t^y \hat{A}_t^y, \ \text{clip}(\rho_t^y, 1-\epsilon_y, 1+\epsilon_y) \hat{A}_t^y \right) \right], \\
\mathcal{L}^{\text{PPO}}(\theta) &= \mathcal{L}_x^{\text{PPO}}(\theta) + \mathcal{L}_y^{\text{PPO}}(\theta),
\end{aligned}
\end{equation}
where $\rho_t^x = \frac{\pi_{\theta}(a_t^x \mid s_t)}{\pi_{\theta_{\text{old}}}(a_t^x \mid s_t)}$ is the importance sampling ratio for the lateral dimension, $\rho_t^y = \frac{\pi_{\theta}(a_t^y \mid s_t)}{\pi_{\theta_{\text{old}}}(a_t^y \mid s_t)}$ is the importance sampling ratio for the longitudinal dimension, $\epsilon_x$ and $\epsilon_y$ are small constants that control the clipping range for the lateral and longitudinal dimensions, ensuring stable policy updates.

The clipped objective function $\mathcal{L}^{\text{PPO}}(\theta)$ prevents excessively large updates to the policy parameters $\theta$, thereby maintaining training stability.

\begin{table*}[ht]
    \centering
{
\begin{tabular}{lccccccccc}
    \toprule
    RL:IL & CR$\downarrow$ & DCR$\downarrow$ & SCR$\downarrow$ & DR$\downarrow$ & PDR$\downarrow$ & HDR$\downarrow$ &ADD$\downarrow$ & Long. Jerk$\downarrow$ & Lat. Jerk$\downarrow$ \\
    \midrule
     0:1  & 0.229 & 0.211 & 0.018 & 0.066 & 0.039 & 0.027  & 0.238 & 3.928 & 0.103\\
     1:0  & 0.143 & 0.128 & 0.015 &0.080 &0.065 &0.015 &0.345 &4.204 &0.085\\
     2:1 & 0.137 & 0.125 & 0.012 & 0.059 & 0.050 & 0.009  & 0.274 & 4.538 & 0.092\\
     4:1 & 0.089 & 0.080 & 0.009 & 0.063 & 0.042 & 0.021  & 0.257 & 4.495 & 0.082 \\
     8:1 & 0.125 & 0.116 & 0.009 & 0.084 & 0.045 & 0.039  & 0.323 & 5.285 & 0.115\\
    \bottomrule
\end{tabular}
}
    \caption{\textbf{Ablation on RL-to-IL step mixing ratios in the reinforced post-training stage.}}
    \label{tab:ratio}
\end{table*}

\subsection{Auxiliary Objective}
RL usually faces the challenge of sparse rewards, which makes the convergence process unstable and slow. To speed up convergence, we introduce auxiliary objectives that provide dense guidance to the entire action distribution.

The auxiliary objectives are designed to penalize undesirable behaviors by incorporating specific reward sources, including dynamic collisions, static collisions, positional deviations, and heading deviations. These objectives are computed based on the actions \( a_t^{x, \text{old}} \) and \( a_t^{y, \text{old}} \) selected by the old AD policy \( \pi_{\theta_{\text{old}}} \) at time step \( t \). To facilitate the evaluation of these actions, we separate the probability distribution of the action into four parts:
\begin{equation}
\begin{aligned}
\Delta \pi_y^{\text{dec}} &= \sum_{a_t^y < a_t^{y, \text{old}}} \pi_\theta(a_t^y \mid s_t), \\
\Delta \pi_y^{\text{acc}} &= \sum_{a_t^y > a_t^{y, \text{old}}} \pi_\theta(a_t^y \mid s_t), \\
\Delta \pi_x^{\text{left}} &= \sum_{a_t^x < a_t^{x, \text{old}}} \pi_\theta(a_t^x \mid s_t), \\
\Delta \pi_x^{\text{right}} &= \sum_{a_t^x > a_t^{x, \text{old}}} \pi_\theta(a_t^x \mid s_t).
\end{aligned}
\end{equation}
Here, \( \Delta \pi_y^{\text{dec}} \) represents the total probability of deceleration actions, \( \Delta \pi_y^{\text{acc}} \) represents the total probability of acceleration actions, \( \Delta \pi_x^{\text{left}} \) represents the total probability of leftward steering actions, and \( \Delta \pi_x^{\text{right}} \) represents the total probability of rightward steering actions.

\boldparagraph{Dynamic Collision Auxiliary Objective.}  
The dynamic collision auxiliary objective adjusts the longitudinal control action \(a_t^y\) based on the location of potential collisions relative to the ego vehicle. If a collision is detected ahead, the policy prioritizes deceleration actions (\(a_t^y < a_t^{y, \text{old}}\)); if a collision is detected behind, it encourages acceleration actions (\(a_t^y > a_t^{y, \text{old}}\)). To formalize this behavior, we define a directional factor \(f_\text{dc}\):
\begin{equation}
\begin{aligned}
f_\text{dc} = \begin{cases} 
1 & \text{if the collision is ahead}, \\
-1 & \text{if the collision is behind}.
\end{cases} 
\end{aligned}
\end{equation}

The auxiliary objective for dynamic collision avoidance is defined as:
\begin{equation}
\begin{aligned}
\mathcal{L}_\text{dc}(\theta_y) = \mathbb{E}_t \left[ 
    \hat{A}_t^\text{dc} \cdot f_\text{dc} \cdot (\Delta \pi_y^{\text{dec}} - \Delta \pi_y^{\text{acc}})
\right],
\end{aligned}
\end{equation}
where \(\hat{A}_t^\text{dc}\) is the advantage estimate for dynamic collision avoidance.

\boldparagraph{Static Collision Auxiliary Objective.}  
The static collision auxiliary objective adjusts the steering control action $a_t^x$ based on the proximity to static obstacles. If the static obstacle is detected on the left side, the policy promotes rightward steering actions ($a_t^x > a_t^{x,\text{old}}$); if the static obstacle is detected on the right side, it promotes leftward steering actions ($a_t^x < a_t^{x,\text{old}}$). To formalize this behavior, we define a directional factor $f_\text{sc}$:  
\begin{equation}
\begin{aligned}
f_\text{sc} = \begin{cases} 
1 & \text{if static obstacle is on the left}, \\
-1 & \text{if static obstacle is on the right}.
\end{cases} 
\end{aligned}
\end{equation}

The auxiliary objective for static collision avoidance is defined as:  
\begin{equation}
\begin{aligned}
\mathcal{L}_\text{sc}(\theta_x) = \mathbb{E}_t \left[ 
    \hat{A}_t^\text{sc} \cdot f_\text{sc} \cdot (\Delta \pi_x^{\text{right}} - \Delta \pi_x^{\text{left}})
\right],
\end{aligned}
\end{equation}  
where $\hat{A}_t^\text{sc}$ is the advantage estimate for static collision avoidance.  

\boldparagraph{Positional Deviation Auxiliary Objective.}  
The positional deviation auxiliary objective adjusts the steering control action $a_t^x$ based on the ego vehicle's lateral deviation from the expert trajectory. If the ego vehicle deviates leftward, the policy promotes rightward corrections ($a_t^x > a_t^{x,\text{old}}$); if it deviates rightward, it promotes leftward corrections ($a_t^x < a_t^{x,\text{old}}$). We formalize this with a directional factor $f_\text{pd}$:  
\begin{equation}
\begin{aligned}
f_\text{pd} = \begin{cases} 
1 & \text{if ego vehicle deviates leftward}, \\
-1 & \text{if ego vehicle deviates rightward}.
\end{cases} 
\end{aligned}
\end{equation}

The auxiliary objective for positional deviation correction is:
\begin{equation}
\begin{aligned}
\mathcal{L}_\text{pd}(\theta_x) = \mathbb{E}_t \left[ 
    \hat{A}_t^\text{pd} \cdot f_\text{pd} \cdot (\Delta \pi_x^{\text{right}} - \Delta \pi_x^{\text{left}})
\right],
\end{aligned}
\end{equation}  
where $\hat{A}_t^\text{pd}$ estimates the advantage of trajectory alignment.

\boldparagraph{Heading Deviation Auxiliary Objective.}  
The heading deviation auxiliary objective adjusts the steering control action $a_t^x$ based on the angular difference between the ego vehicle’s current heading and the expert’s reference heading. If the ego vehicle deviates counterclockwise, the policy promotes clockwise corrections ($a_t^x > a_t^{x,\text{old}}$); if it deviates clockwise, it promotes counterclockwise corrections ($a_t^x < a_t^{x,\text{old}}$). To formalize this behavior, we define a directional factor $f_\text{hd}$:  
\begin{equation}
\begin{aligned}
f_\text{hd} = \begin{cases} 
1 & \text{if ego vehicle deviates clockwise}, \\
-1 & \text{if ego vehicle deviates counterclockwise}.
\end{cases} 
\end{aligned}
\end{equation}

The auxiliary objective for heading deviation correction is then defined as:  
\begin{equation}
\begin{aligned}
\mathcal{L}_\text{hd}(\theta_x) = \mathbb{E}_t \left[ 
    \hat{A}_t^\text{hd} \cdot f_\text{hd} \cdot (\Delta \pi_x^{\text{right}} - \Delta \pi_x^{\text{left}})
\right],
\end{aligned}
\end{equation}  
where $\hat{A}_t^\text{hd}$ is the advantage estimate for heading alignment.  

\begin{table*}[ht]
\begin{center}
\centering
\resizebox{0.98\textwidth}{!}{
\begin{tabular}{cccccccccccccc}
\toprule
\multirow{2}{*}{ID} & Dynamic & Static & Position & Heading & \multirow{2}{*}{CR$\downarrow$} &\multirow{2}{*}{DCR$\downarrow$} &\multirow{2}{*}{SCR$\downarrow$} &\multirow{2}{*}{DR$\downarrow$} &\multirow{2}{*}{PDR$\downarrow$} &\multirow{2}{*}{HDR$\downarrow$} &\multirow{2}{*}{ADD$\downarrow$} &\multirow{2}{*}{Long. Jerk$\downarrow$} &\multirow{2}{*}{Lat. Jerk$\downarrow$}\\
& Collision & Collision & Deviation & Deviation & & & & & & & & & \\
\midrule
1 & \cmark  &  &  &  & 0.172 & 0.154 & 0.018 & 0.092 & 0.033 & 0.059  & 0.259 & 4.211 & 0.095 \\
2 &  & \cmark & \cmark & \cmark & 0.238 & 0.217 & 0.021 & 0.090 & 0.045 & 0.045  & 0.241 & 3.937 & 0.098 \\
3 & \cmark &  & \cmark & \cmark & 0.146 & 0.128 & 0.018 & 0.060 & 0.030 & 0.030  & 0.263 & 3.729 & 0.083\\
4 & \cmark & \cmark &  & \cmark & 0.151 & 0.142 & 0.009 & 0.069 & 0.042 & 0.027 & 0.303 & 3.938 & 0.079\\
5 & \cmark & \cmark & \cmark &  & 0.166 & 0.157 & 0.009 & 0.048 & 0.036 & 0.012 & 0.243 & 3.334 & 0.067\\
6 & \cmark & \cmark & \cmark & \cmark & 0.089 & 0.080 & 0.009 & 0.063 & 0.042 & 0.021 & 0.257 & 4.495 & 0.082 \\
\bottomrule
\end{tabular}
}
\end{center}
\vspace{-2mm}
\caption{\textbf{Ablation on reward sources.} The table shows the impact of different reward components on performance.}
\label{tab:reward_ablation}
\end{table*}

\begin{table*}[ht]
\begin{center}
\centering
\resizebox{0.98\textwidth}{!}{
\begin{tabular}{ccccccccccccccc}
\toprule
\multirow{2}{*}{ID} & \multirow{2}{*}{PPO Obj.}  & Dynamic Col. & Static Col. & Position Dev. & Heading Dev. & \multirow{2}{*}{CR$\downarrow$} & \multirow{2}{*}{DCR$\downarrow$}  & \multirow{2}{*}{SCR$\downarrow$} & \multirow{2}{*}{DR$\downarrow$} & \multirow{2}{*}{PDR$\downarrow$} & \multirow{2}{*}{HDR$\downarrow$} & \multirow{2}{*}{ADD$\downarrow$} & \multirow{2}{*}{Long. Jerk$\downarrow$} & \multirow{2}{*}{Lat. Jerk$\downarrow$} \\
& & Auxiliary Obj. & Auxiliary Obj. & Auxiliary Obj. & Auxiliary Obj. & & & & & & & & & \\
\midrule
1 &\cmark&  &  &  &  & 0.249 & 0.223 & 0.026 & 0.077 & 0.047 & 0.030  & 0.266 & 4.209 & 0.104 \\
2 &\cmark& \cmark &  &  &  & 0.178 & 0.163 & 0.015 & 0.151 & 0.101 & 0.050 & 0.301 & 3.906 & 0.085 \\
3 &\cmark&  & \cmark & \cmark & \cmark & 0.137 & 0.125 & 0.012 & 0.157 & 0.145 & 0.012 & 0.296 & 3.419 & 0.071 \\
4 &\cmark& \cmark &  & \cmark & \cmark & 0.169 & 0.151 & 0.018 & 0.075 & 0.042 & 0.033 & 0.254 & 4.450 & 0.098 \\
5 &\cmark& \cmark & \cmark &  & \cmark & 0.149 & 0.134 & 0.015 & 0.063 & 0.057 & 0.006 & 0.324 & 3.980 & 0.086 \\
6 &\cmark& \cmark & \cmark & \cmark & & 0.128 & 0.119  & 0.009 & 0.066 & 0.030 & 0.036  & 0.254 & 4.102 & 0.092 \\
7 &&\cmark  &\cmark  &\cmark  &\cmark  & 0.187 &0.175  &0.012 &0.077 &0.056  &0.021  &0.309  &5.014  &0.112  \\
8 &\cmark& \cmark & \cmark & \cmark & \cmark & 0.089 & 0.080 & 0.009 & 0.063 & 0.042 & 0.021  & 0.257 & 4.495 & 0.082 \\
\bottomrule
\end{tabular}
}
\end{center}
\vspace{-2mm}
\caption{\textbf{Ablation on auxiliary objectives.} The table shows the impact of different auxiliary objectives on performance.}
\label{tab:auxiliary_ablation}
\end{table*}

\boldparagraph{Overall Auxiliary Objectives.}  
The overall auxiliary objectives are a weighted sum of the individual objectives:
\begin{equation}
\begin{aligned}
\mathcal{L}_\text{aux}(\theta) = &\lambda_1 \mathcal{L}_\text{dc}(\theta_y) + \lambda_2 \mathcal{L}_\text{sc}(\theta_x)  + \\ 
&\lambda_3 \mathcal{L}_\text{pd}(\theta_x) +\lambda_4 \mathcal{L}_\text{hd}(\theta_x),
\end{aligned}
\end{equation}
where $\lambda_1$, $\lambda_2$, $\lambda_3$, and $\lambda_4$ are weighting coefficients that balance the contributions of each auxiliary objective.

\boldparagraph{Optimization Objective.}  
The final optimization objective combines the clipped PPO objective with the auxiliary objective:
\begin{equation}
\mathcal{L}(\theta) = \mathcal{L}^{\text{PPO}}(\theta) + \mathcal{L}_\text{aux}(\theta).
\end{equation}

\vspace{-1mm}
\subsection{Parallel Attention Bias}
\label{Parallel Attention Bias}



\paragraph{Theoretical Insights into Parallel Attention Bias.}\label{sec:parallel_attn_collapse}

In this section, we provide a theoretical framework for understanding \textit{Parallel Attention Bias}, extending the concept of attention collapse~\citep{dong2021attention} to parallel attention mechanisms described in Section~\ref{ParallelComp}. Specifically, we analyze the sparsity behavior of the local attention matrices computed over parallel chunks and its implications for efficiency and accuracy.

\begin{theorem}\label{thm:parallel_attn_collapse_main}
    Consider the following setup:
      \vspace{-2mm}
    \begin{itemize}
        \item {\bf Part 1:} For any \(\epsilon > 0\), the sparsity threshold of effective entries in \(A^{c}_\mathfrak{l}\) increases with \(w\). \( \epsilon \) is a user-defined threshold that controls sparsity in the attention matrix. With more chunks (\( C \)), \( \epsilon \) affects the balance between retaining information within each chunk and computational efficiency.
        \item {\bf Part 2:} The number of effective entries \(k\) in each row of \(A^{c}_\mathfrak{l}\) is upper-bounded as: 
        \[
        k \leq w - \exp\left(O\left(\frac{\log^2(\epsilon \cdot w)}{R^2}\right)\right) \cdot \frac{\delta}{wd},
        \] where \(R\) represents the rank of the sparse attention matrix, which determines the effective dimensionality of retained attention entries, and \(\delta\) is a probability bound controlling the confidence level of the sparsity constraint. 
        
        % A larger \(R\) implies more retained attention entries, whereas a higher \(\delta\) indicates a looser constraint on sparsity.
        \item {\bf Part 3:} With high probability (\(1 - \delta\)), the number of ineffective entries in each row satisfies:
        \[
        \lim_{w \to \infty} | \mathcal{S}_\epsilon^{(c)}(A^{c}_\mathfrak{l}[i,:]) | = w - k.
        \]
    \end{itemize}
\end{theorem}

\begin{proof}[Proof Sketch of Theorem~\ref{thm:parallel_attn_collapse_main}]
{\bf Proof sketch of Part 1:} From the softmax scaling property, the sparsity threshold for effective entries in \( A^{c}_\mathfrak{l} \) can be bounded as:
\[
\epsilon \geq \exp\left(O(R) \cdot \sqrt{\log(w \cdot (w-k)/\delta)}\right).
\]
This inequality shows that as \( w \) increases, the threshold for retaining effective entries becomes stricter, limiting the number of such entries. {\bf Proof sketch of Part 2:} Rearranging the above inequality, we derive an upper bound on \( k \), the number of effective entries:
\[
k \leq w - \exp\left(O\left(\frac{\log^2(\epsilon \cdot w)}{R^2}\right)\right) \cdot \frac{\delta}{wd}.
\]
This implies that the number of effective entries in each row of the attention matrix is given by \( w - k \). {\bf Proof sketch of Part 3:} Substituting the bound of \( k \) into the definition of \( |\mathcal{S}_\epsilon^{(c)}| \), the number of ineffective entries, we find:
\[
\lim_{w \to \infty} | \mathcal{S}_\epsilon^{c}(A_\mathfrak{l}^{c}[i,:]) | \geq w - k.
\]
Finally, observing that \( R = O(\sqrt{\log(w)}) \) ensures that the sparsity growth is bounded as \( w \to \infty \). Appendix~\ref{sec:parallel_attn_collapse_aa} provides a more detailed proof.
\end{proof} 


\vspace{-6mm}
\paragraph{Discussion.}

Theorem~\ref{thm:parallel_attn_collapse_main} highlights the inevitability of attention collapse in parallel attention mechanisms. Despite dividing the input sequence into \( C \) smaller chunks, the effective number of attention entries within each chunk diminishes as the chunk size \( w \) increases. Key observations include: \textit{i)} Each local attention matrix \( A^{c}_\mathfrak{l} \) exhibits sparsity behavior similar to the global attention matrix, with most entries becoming negligible for large \( w \). \textit{ii)} When a super-long matrix is input into attention in parallel, {\it attention bias} inevitably occurs. The attention mechanism repeatedly focuses on a small number of tokens due to its inherent limitations, even with more information provided. Selecting an appropriate sparsity parameter \( \epsilon \) can mitigate this issue. \textit{iii)} Dividing the input into chunks reduces computational overhead while maintaining sparsity within each chunk.


\section{Empirical Study of Parallel Attention Bias}
\label{Empirical Study of Parallel Attention Bias}
In this section, we investigate the attention sink phenomenon in parallel attention mechanisms and compare its similarities and differences with the regular attention sink phenomenon. Specifically, we explore the following question:  

\vspace{-4mm}

\paragraph{Q1: What types of attention patterns can be summarized?}
 
In summary, three main types of attention patterns emerge, as illustrated in Figure~\ref{fig:attention_bias_pattern}: U-shape, Mountain-shape, and Uniform-shape. 
% \vspace{-10mm}
\begin{figure}[H]
% \vskip 0.2in
% \vspace{-3mm}
\begin{center}
\centerline{\includegraphics[width=0.5\textwidth]{figure/Attention_pattern.pdf}}
\caption{Several types of attention distribution. The Token ID represents the token position in the input text.}
\label{fig:attention_bias_pattern}
\vspace{-8mm}
\end{center}
% \vskip -0.2in
\end{figure}

\paragraph{Observations.} These attention distributions give rise to three corresponding biases: \textit{i)} Attention sink, where focus is concentrated on the initial few tokens. \textit{ii)} Recency bias, where attention is more strongly concentrated at the tail. \textit{iii)} Middle bias, where attention is disproportionately focused on a few tokens in the middle of a sequence. \textit{iv)} These biases manifest in a wavelike pattern, with \( R_h \) containing three token types (\( R_s, R_m, R_r \)) corresponding to these biases.

% In summary, there are three main types of attention patterns, as illustrated in Figure~\ref{fig:attention_bias_pattern}, namely U-shape, Mountain-shape, and Uniform-shape. The shapes of the first two attention distributions lead to three types of bias, which can be briefly described as follows: \textit{i)} Attention sink where the focus is concentrated on the initial few tokens. \textit{ii)} The model's attention sink phenomenon is more severe at the tail, indicating a stronger recency bias. \textit{iii)} The model exhibits severe attention concentration in the middle part, where attention focuses on only a few tokens among thousands. This bias is referred to as an middle bias.



% \paragraph{Answer.}
% Three types of attention bias—sink, recency, and middle—emerge in a wavelike pattern, with \( R_h \) containing three token types (\( R_s, R_m, R_r \)) that correspond to these biases, where \( R_s \) is the anomalous token.

\paragraph{Q2: Is there any difference between the attention bias in parallel attention and the attention bias in classical attention?}
In this part, we provide a detailed analysis of some bias phenomena in parallel attention mechanisms. We observe in Figure~\ref{fig:parrallel_attention} that there are relatively more peaks within the contexts. 

\begin{figure}[H]
% \vskip -0.2in
\begin{center}
\centerline{\includegraphics[width=0.5\textwidth]{figure/layer_1_head_1_horizontal.pdf}}
\caption{Comparison of local and parallel attention patterns. The blue lines show the local attention distribution within a chunk, while the yellow lines represent the parallel attention patterns in global attention.}
\label{fig:parrallel_attention}
\end{center}
% \vskip -0.2in
\vspace{-8mm}
\end{figure}

\vspace{-2mm}
\begin{figure*}[ht]
\begin{center}
% \centering
\includegraphics[width=0.99\textwidth]{figure/eviction.pdf}
\caption{Several types of attention bias and patterns. In the figure, \textbf{Parallel KV Cache Eviction} performs independent KV cache eviction within each chunk, while \textbf{KV Cache Eviction} unifies this process during global attention. \textbf{Parallel KV Cache Eviction} significantly reduces the computational load of global attention.}
\label{fig:attention_pattern_comparison}
\end{center}
% \vskip -0.2in
\end{figure*}
\FloatBarrier


% \vspace{-3mm}
\begin{figure}[H]
% \vskip -0.2in
\begin{center}
\centerline{\includegraphics[width=0.5\textwidth]{figure/Comparison_of_bias_before_and_after.pdf}}
\caption{The distribution of tokens with abnormally high attention scores. Blue represents outliers.}
\label{fig:parrallel_attention_count}
\end{center}
\vspace{-2mm}
\end{figure}


\paragraph{Observations.} \textit{i)} Similar to the blue local attention, the yellow curve shows the U-shaped attention sink phenomenon repeatedly appearing. \textit{ii)} Parallel attention and local attention both exhibit severe recency bias, but the effect is significantly mitigated in parallel attention compared to local attention. \textit{iii)} When computing global attention \( A{\mathfrak{g}} \), the model suffers from the most severe recency bias, though it is still less pronounced than within \( A^c_{\mathfrak{l}} \) (blue line). \textit{iv)} Compared to the classical attention distribution, i.e., the local attention, the peaks of \( A{\mathfrak{g}} \) within the chunk are significantly weakened, indicating that global attention can significantly mitigate recency bias. \textbf{In other words, the parallel attention mechanism itself can mitigate attention bias.}

\paragraph{Q3: Can the calibration strategy alleviate attention bias?}
\label{Q3}

By evicting different types of \(R_h\) at different layers, we have the following observations:
\vspace{-2mm}
\paragraph{Observations.} \textit{i):} From Figure~\ref{fig:attention_pattern_comparison}, we can see that KV cache eviction exacerbates the bias phenomenon. However, parallel KV cache eviction can achieve a more stable distribution. \textit{ii):} Evicting sink bias tokens in the early layers may exacerbate attention bias, but evicting them in the deeper layers can mitigate this attention bias. \textit{iii):} Evicting recency bias tokens in the intermediate layers can mitigate attention bias, while evicting recency bias tokens in the deeper layers redistributes the attention scores obtained by the recency bias tokens to the intermediate tokens. \textit{iv):}  Simultaneously evicting sink bias and recency bias tokens can alleviate attention bias in the intermediate layers (Layer 16). \textit{v):} From Figure~\ref{fig:parrallel_attention_count}, by evicting tokens with abnormally high attention scores, the model can mitigate attention bias, which means it can also alleviate the performance loss. We will further validate this in our experiments.





