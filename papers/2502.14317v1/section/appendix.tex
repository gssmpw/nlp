\appendix
\onecolumn

\section{Proof for Parallel Attention Collapse}
\label{sec:parallel_attn_collapse_aa}
\begin{theorem}\label{thm:parallel_attn_collapse}
    Consider a parallel attention mechanism where:
    \begin{itemize}
        \item The input sequence \( X \in \mathbb{R}^{N \times d} \) is divided into \( C \) chunks, where each chunk \( X^{c} \in \mathbb{R}^{w \times d} \) contains at most \( w \) tokens, for \( c \in [C] \).
        \item Define the query, key, and value matrices for each chunk as:
        \[
        Q^{c} = f_Q(X^{c}), \quad K^{c} = f_K(X^{c}), \quad V^{c} = f_V(X^{c}),
        \]
        where \( Q^{c}, K^{c}, V^{c} \in \mathbb{R}^{w \times d} \).
        \item The local attention matrix for chunk \( c \) is:
        \[
        A^{c}_\mathfrak{l} = \operatorname{Softmax}\left(\frac{Q^{c} {K^{c}}^\top}{\sqrt{d}}\right) \in \mathbb{R}^{w \times w}.
        \]
        \item Let \( \mathcal{S}_\epsilon^{(c)} \) denote the set of effective entries in the normalized attention matrix row for chunk \( c \), where:
        \[
        \mathcal{S}_\epsilon^{(c)}(A^{c}_\mathfrak{l}[i,:]) = \{ j \mid A^{c}_\mathfrak{l}[i,j] > \epsilon \}, \quad i \in [w].
        \]
        \item Assume the sparsity parameter \( R = O(\sqrt{\log(w)}) \).
    \end{itemize}
    Then, for any \(\epsilon > 0\), with high probability (\(1 - \delta\)), the number of ineffective entries in each row satisfies:
    \[
    \lim_{w \to \infty} | \mathcal{S}_\epsilon^{(c)}(A^{c}_\mathfrak{l}[i,:]) | = w - k.
    \]
\end{theorem}

\begin{proof}
    In the parallel attention setting, the sparsity of the local attention mechanism is analyzed within each chunk. For a specific chunk \( c \), the softmax scaling property governs the behavior of the attention matrix \( A^{c}_\mathfrak{l} \). The sparsity threshold \( \epsilon \) for effective entries in \( A^{c}_\mathfrak{l} \) can be expressed as:
    \[
    \epsilon \geq \exp\left(O(R) \cdot \sqrt{\log(w \cdot (w - k) / \delta)}\right),
    \]
    where \( k \) represents the number of ineffective entries in a row of the attention matrix. This inequality indicates that as the chunk size \( w \) increases, the threshold for retaining effective entries becomes stricter, reducing the number of effective entries.

    Rearranging the inequality, we derive an upper bound on \( k \), the number of effective entries:
    \[
    k \leq w - \exp\left(O\left(\frac{\log^2(\epsilon \cdot w)}{R^2}\right)\right) \cdot \frac{\delta}{wd}.
    \]
    Substituting this bound into the definition of \( |\mathcal{S}_\epsilon^{(c)}| \), we find that the number of ineffective entries is:
    \[
    | \mathcal{S}_\epsilon^{(c)}(A^{c}_\mathfrak{l}[i,:]) | \geq w - k.
    \]

    As \( w \to \infty \), the remaining effective entries approach \( w - 1 \), as \( R = O(\sqrt{\log(w)}) \) ensures that the sparsity growth remains bounded. Thus, the number of ineffective entries in each row satisfies:
    \[
    \lim_{w \to \infty} | \mathcal{S}_\epsilon^{(c)}(A^{c}_\mathfrak{l}[i,:]) | = w - k,
    \]
    which completes the proof.
\end{proof}

\FloatBarrier
\begin{table*}[!t]
\vspace{-1mm}
\centering
\adjustbox{max width=\textwidth}{%
\scriptsize
\begin{tabular}{c@{}c@{}c@{}c@{} c@{}c@{}c@{} c@{}c@{}c@{} c@{}c@{}c@{} c@{}c@{} c@{}c@{} c}
\toprule
\multirow{2}{*}{\raisebox{-4ex}{\textbf{Methods}}}  % Moves the text down
& \multicolumn{3}{c}{\textbf{Single-Document QA}} 
& \multicolumn{3}{c}{\textbf{Multi-Document QA}} 
& \multicolumn{3}{c}{\textbf{Summarization}} 
& \multicolumn{3}{c}{\textbf{Few-shot Learning}} 
& \multicolumn{2}{c}{\textbf{Synthetic}} 
& \multicolumn{2}{c}{\textbf{Code}} 
& \multirow{2}{*}{\raisebox{-4ex}{\textbf{Avg.}}} 
% & \multirow{2}{*}{\raisebox{-4ex}{\textbf{ Time}}} 
% & \multirow{2}{*}{\raisebox{-6ex}{\textbf{\shortstack{Time \\ (s / sample)}}}}
\\  % Moves the text down

\cmidrule(lr){2-4} \cmidrule(lr){5-7} \cmidrule(lr){8-10} \cmidrule(lr){11-13} \cmidrule(lr){14-15} \cmidrule(lr){16-17}
\setlength{\tabcolsep}{1pt} % 默认值是6pt,减小这个值来减少列间距
& \makebox[1cm]{\raisebox{0.5ex}{\rotatebox{30}{\textbf{NtrvQA}}}} 
& \makebox[1cm]{\raisebox{0.7ex}{\rotatebox{30}{\textbf{Qasper}}}} 
& \makebox[1cm]{\raisebox{0.8ex}{\rotatebox{30}{\textbf{MF-en}}}} 
& \makebox[1cm]{\raisebox{0.4ex}{\rotatebox{30}{\textbf{HotpotQA}}}} 
& \makebox[1cm]{\raisebox{0.3ex}{\rotatebox{30}{\textbf{2WikiMQA}}}} 
& \makebox[1cm]{\raisebox{0.7ex}{\rotatebox{30}{\textbf{Musique}}}} 
& \makebox[1cm]{\raisebox{0.5ex}{\rotatebox{30}{\textbf{GovReport}}}} 
& \makebox[1cm]{\raisebox{0.8ex}{\rotatebox{30}{\textbf{QMSum}}}} 
& \makebox[1cm]{\raisebox{0.6ex}{\rotatebox{30}{\textbf{MultiNews}}}} 
& \makebox[1cm]{\raisebox{0.8ex}{\rotatebox{30}{\textbf{TREC}}}} 
& \makebox[1cm]{\raisebox{0.6ex}{\rotatebox{30}{\textbf{TriviaQA}}}} 
& \makebox[1cm]{\raisebox{0.6ex}{\rotatebox{30}{\textbf{SAMSum}}}} 
& \makebox[1cm]{\raisebox{0.6ex}{\rotatebox{30}{\textbf{PCount}}}}  
& \makebox[1cm]{\raisebox{1.4ex}{\rotatebox{30}{\textbf{PRe}}}}    
& \makebox[1cm]{\raisebox{1.6ex}{\rotatebox{30}{\textbf{Lcc}}}} 
& \makebox[1cm]{\raisebox{1.4ex}{\rotatebox{30}{\textbf{RB-P}}}} \\

% Llama-2-7B-chat-hf
\midrule
Max Length & 84123 & 24204 & 17727 & 20325 & 19001 & 20520  & 60515 & 34477 & 16271 & 13049 & 26756 & 21884 & 32699 & 17158  & 37628 & 58822 & 30657 \\
\midrule
\multicolumn{18}{c}{\textbf{Llama2-7B-chat-hf(4k)}} \\
\arrayrulecolor[gray]{0.8}
\midrule
\arrayrulecolor{black}
No-eviction & 23.20 & 17.50 & 37.07 & 38.67 & 32.68 & 20.22 & 25.00 & 22.79 & 25.84 & \textbf{64.00} & 84.63 & 40.67 & 4.00 & 31.50 & 59.37 & 58.53 & 36.60 \\
Sink-eviction-layer-1-8 & 23.75 & 18.69 & \textbf{38.41} & 39.86 & 32.91 & 20.75 & 24.86 & 22.10 & 25.56 & 63.00 & 84.42 & 40.78 & 4.50 & 30.00 & 54.67 & \textbf{59.30} & 36.47 \\
Sink-eviction-layer-9-16 & 23.34 & \textbf{19.10} & 38.21 & 38.73 & 30.42 & 21.04 & 25.31 & 21.86 & 25.16 & 62.00 & 85.53 & \textbf{41.26} & 3.00 & 29.50 & 56.75 & 58.31 & 36.22 \\
Sink-eviction-layer-17-24 & 24.46 & 17.75 & 36.84 & 38.79 & 30.59 & 19.67 & 25.42 & 22.20 & 25.58 & 62.00 & 85.35 & 40.24 & 4.00 & 28.50 & 58.41 & 58.17 & 36.12 \\
Sink-eviction-layer-25-32 & 23.87 & 18.40 & 35.91 & 38.96 & 31.02 & 20.21 & 25.32 & 22.00 & 25.81 & 64.00 & 84.19 & 39.77 & 3.50 & 30.00 & 58.58 & 58.21 & 36.23 \\

Recency-eviction-layer-1-8 & 22.71 & 16.95 & 35.24 & 36.14 & 30.60 & 17.19 & 25.21 & 22.11 & \textbf{26.22} & 59.00 & 68.00 & 40.03 & 2.50 & 31.00 & 58.07 & 51.47 & 33.90 \\
Recency-eviction-layer-9-16 & 24.95 & 13.54 & 35.67 & 34.13 & 30.69 & 17.77 & 25.14 & 22.85 & 25.43 & 54.50 & 79.23 & 39.16 & 5.00 & 27.50 & 57.73 & 57.57 & 34.43 \\
Recency-eviction-layer-17-24 & 21.68 & 15.17 & 34.97 & 32.79 & 26.93 & 13.95 & 25.29 & 22.10 & 25.42 & 62.50 & 80.47 & 38.60 & 6.00 & 34.00 & 58.31 & 57.76 & 34.75 \\
Recency-eviction-layer-25-32 & 24.15 & 17.32 & 37.82 & 36.76 & 29.86 & 18.48 & 25.21 & 22.06 & 25.67 & 64.00 & 83.20 & 36.67 & 5.00 & 30.50 & 56.13 & 56.52 & 35.58 \\

Middle-eviction-layer-1-8 & 22.41 & 16.84 & 37.94 & 39.99 & \textbf{33.24} & 19.62 & 24.74 & 22.02 & 25.80 & 63.50 & 83.67 & 40.00 & 5.00 & 32.50 & 59.22 & 56.71 & 36.45 \\
Middle-eviction-layer-9-16 & 22.96 & 17.63 & 37.39 & \textbf{40.51} & 30.68 & 21.09 & 25.14 & 21.94 & 25.66 & 62.00 & 85.02 & 40.57 & 4.00 & 35.50 & 59.28 & 57.88 & 36.70 \\
Middle-eviction-layer-17-24 & 21.72 & 17.06 & 35.88 & 39.99 & 31.76 & 19.06 & 25.00 & 22.20 & 25.77 & 58.00 & 83.62 & 40.02 & 4.50 & 38.00 & \textbf{59.38} & 57.98 & 36.25 \\
Middle-eviction-layer-25-32 & 21.64 & 17.04 & 36.32 & \textbf{40.51} & 32.80 & 19.14 & 25.07 & 22.14 & 25.86 & 63.50 & 84.39 & 40.50 & 3.00 & 30.00 & 59.15 & 57.86 & 36.18 \\
All-eviction-layer-1-8 & 0.33 & 0.05 & 1.24 & 0.32 & 0.68 & 0.38 & 1.76 & 3.25 & 1.59 & 1.50 & 5.40 & 1.75 & 0.41 & 0.50 & 23.61 & 12.17 & 3.43 \\
All-eviction-layer-9-16 & 1.60 & 1.97 & 5.06 & 0.65 & 1.44 & 0.81 & 21.75 & 36.22 & 1.81 & 35.00 & 30.77 & 10.54 & 3.00 & 0.50 & 33.25 & 22.89 & 12.95 \\
All-eviction-layer-17-24 & 12.20 & 10.26 & 16.30 & 20.30 & 18.10 & 8.54 & 13.63 & 20.43 & 17.61 & 49.00 & 11.93 & 29.74 & 5.50 & 26.00 & 41.92 & 23.67 & 20.32 \\
All-eviction-layer-25-32 & 11.19 & 7.62 & 11.11 & 12.49 & 8.83 & 1.79 & 11.45 & 16.21 & 12.87 & 43.00 & 30.77 & 9.94 & 3.50 & 22.00 & 24.42 & 22.19 & 15.59 \\


Ours-calibration & \textbf{24.95} & 19.07 & 38.16 & 39.53 & 32.62 & \textbf{22.64} & \textbf{25.42} & \textbf{22.82} & 26.01 & 63.00 & \textbf{85.41} & 40.36 & \textbf{5.00} & \textbf{32.50} & 59.04 & 58.84 & \textbf{37.21} \\

\bottomrule
\end{tabular}
}
\caption{Bias Token Eviction Ablation. Sink-eviction-layer-1-8 typically means evicting sink bias tokens in layers 1 to 8, and other naming conventions follow the same pattern. Ours-calibration refers to the approach where layers 9-16 adopt the recency bias token eviction method, while layers 1-8 evict middle bias tokens, and layers 25-32 evict sink bias tokens.} 
\vspace{-4mm}
\label{longbench-ablation}
\end{table*}



\FloatBarrier


\section{Ablation Study}
\label{Ablation Study}

In this section, we analyze the impact of different attention biases on the LongBench dataset. As shown in Table~\ref{longbench-ablation}, the exceptionally low performance of Recency-eviction-layer-1-8 on both in-context learning tasks, TREC and TriviaQA, as well as SAMSum, indicates that the recency bias tokens in the model's early layers are crucial for developing in-context learning abilities. 
\section{Hyperparameter}
\label{Hyperparameter}
The Dynamic-PI method interpolates dynamically according to the length of the input token. NTK-Aware refer to~\citep{fixedNTK} and the maximum length is set to 280k. ChunkLlama, InfLLM and AttenCalibration-NTK use hyperparameters from open source repositories. About our method, when performing parallel KV Cache compression, we use last 8 token's cumulative attention scores to compress the KV cache size within each chunk to 2000. On Longbench, we retain 3 chunks from the priority queue, while on InfiniteBench, we retain 1 chunk for retrieval tasks and 3 chunks for other tasks from the priority queue. \textbf{In our setup, the total number of chunks processed in parallel and the chunks in the priority queue is 23.} In all datasets, the context length of each chunk, including the query, is the maximum pre-training length of the model. \( \tau \)  is set to 0.1 for llama2 and 0.3 for llama3. \( R_s \) is obtained from the first 200 tokens of the chunk, \( R_r \) is obtained from the last 200 tokens of the chunk, and the remaining part of the chunk obtains \( R_m \).  All experiments are performed on 80G A100.