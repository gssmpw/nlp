
\section{Experiments}\label{sec::exp}
In this section, we conduct experiments to study the ability of LLM agents, aiming to answer the following research questions (RQ).
\begin{itemize}[leftmargin=*]
    \item \textbf{RQ1}: How do the LLM-empowered agents behave in the simulation environment, compared with the traditional agents?
    \item \textbf{RQ2}: How about LLM agent's strategy? Could it provide reasonable explanations for the actions?
    % \item \textbf{RQ3}: What about the simulation results if there is internal interventions on the environment?
\end{itemize}

% We present a series of experiments designed to evaluate the performance and decision-making capabilities of the LLM-empowered agent within our macroeconomic simulation.

\subsection{Experimental Setup}
% Key metrics were established to gauge both macroeconomic indicators and the decision-making accuracy of our LLM agent.
We commence by defining the experimental landscape. We investigate some phenomena of paramount interest in macroeconomics, including several macroeconomics indicators and two macroeconomics regularities. For comparative analysis, we select two representative macroeconomic simulations and adopt their carefully designed rules of working and consumption as baselines~\cite{lengnick2013agent,gatti2011macroeconomics}.

\subsubsection{Definition of macroeconomics indicators} Annual nominal GDP is defined as the sum of $S\times P$ over one year. As for real GDP, we set the first year in the simulation as the reference year and replace $P$ with $P_0$, where $P_0$ is the goods price in the reference year. The definition of the annual (price) inflation rate and the unemployment rate is shown in Eq. \ref{def-indicator}. For wage inflation, the definition is similar to that of price inflation, where the average price is replaced with the average wage across all the agents.
% As for quarter rates, the calculation is similar, taking the average over one quarter.

% Metrics include inflation rate, GDP, unemployment rate, and others pivotal for macroeconomic analysis.

\subsubsection{Baselines} We select \textbf{LEN}~\cite{lengnick2013agent} and \textbf{CATS}~\cite{gatti2011macroeconomics} as the baselines because 1) they partially reproduce the aforementioned macroeconomics phenomena within their own (more complex) simulation frameworks, and 2) carefully designed decision rules of working and consumption are representative, reflecting typical decision-making observed in real-life scenarios.
% Baselines were chosen considering their relevance and performance in prior works [citation].

\textbf{Consumption}. In LEN, the consumption decision is \textit{memory-based}, which means that consumption is influenced not only by current income but also by past accumulated savings. Besides, the goods price is another important factor. 

Conversely, in CATS, it is \textit{non-memory-based} consumption decisions suggesting that consumption is solely related to the current income. The agent aims to keep a desired ratio between savings and income, and consumption is only a proportion of the income. For more human-like decisions, we also introduce the influence of interest rate in the decision rule.

\textbf{Work}. The rule of working in LEN and CATS can not be directly used in our simulation framework because we don't simulate firms. Therefore, we follow the intuitions of their designs and define a formula implying that a higher expected income, lower savings, or a lower interest rate lead to a greater propensity to work.

Considering the importance of agents' heterogeneity in macroeconomic simulation, we also combine these two baselines as an additional baseline \textbf{Composite}, where each agent randomly adopts one rule of them.

Besides, we also select an AI method, AI-Economist~\cite{zheng2022ai} as a baseline \textbf{AI-Eco}, which builds on the assumption of rational decision-making and employs reinforcement learning~(RL)~\cite{arulkumaran2017deep} to maximize the agent's utility. The utility is a satisfaction function positively correlated with savings and consumption but negatively correlated with labor. Maximizing utility implies that the agent always desires more savings and consumption but prefers less labor. The policy network for the agent's work and consumption decisions is a multi-layer perceptron~(MLP), where the input includes various environment information, such as monthly wage, interest rate, goods price, tax rates, \textit{etc}.

The details of decision rules in LEN and CATS and the training process of AI-Economist are provided in the supplementary materials.

\subsubsection{Simulation parameters} 
We simulate $N=100$ agents. The productivity is set as $A = 1$ for simplicity. The initial goods price is the average hourly wage across all the agents. For the labor and consumption dynamics, $\alpha_w = 0.05$ and $\alpha_P = 0.10$. For the financial market, $r^n = 0.01$, $\pi^t = 0.02$, $u^n = 0.04$, and $\alpha^{\pi} = \alpha^u = 0.5$. Note that our results and conclusions are not sensitive to these parameters.
% Various parameters, such as the number of agents, economic policies in place, and initial economic conditions, were set [unknown].

Our simulation is implemented with Python. We use GPT-3.5-turbo provided by OpenAI API as the LLM\footnote{https://platform.openai.com/}. Other detailed simulation parameters, crucial for replicability and deeper understanding, are provided in the supplementary materials.

\subsection{Comparison with Baselines}
The LLM agent's performance was compared with that of representative rule-based baselines, as detailed in two referenced works and their combination~\cite{lengnick2013agent,gatti2011macroeconomics}. Our evaluation encapsulates a broad spectrum of macroeconomic indicators and regularities~\cite{lengnick2013agent,gatti2011macroeconomics,dawid2018agent,axtell2022agent}.

\textbf{Macroeconomic indicators}. In Figure~\ref{fig::annual-indicators}, we depict the fluctuations of the annual inflation rate, unemployment rate, nominal GDP, and growth rate of nominal GDP. Note that the unreasonable unemployment rate~(around 46\%) and nominal GDP for AI-Eco are not reported. Both rule-based and RL-driven baselines produce anomalous indicators and large fluctuations. In contrast, agent decision-making based on the LLM has manifested more stable and numerically plausible macroeconomic phenomena across multiple facets, even without fine-grained calibration. Specifically, the inflation rate consistently fluctuated within a -5\% to 5\% range after the 3-th year, whereas the baselines exhibited significantly larger oscillations, at times even surpassing 20\%. This indicates that LLM-based decision-making is coherent and more closely emulates real-world human choices, leading to easier attainment of equilibrium between supply and demand in the consumption market. The unemployment rate generally varied between 2\% and 12\%\footnote{The increase in the unemployment rate after year 15 is actually a normal fluctuation. See Figure \ref{fig::unemployment-c}.}, which aligns well with real-world economic activities~\cite{gatti2011macroeconomics}. Both the nominal GDP and its growth rate also fluctuated within more reasonable numerical bounds like the inflation rate does. We also provide quarterly indicators in the supplementary materials.

% As can be observed, these indicators exhibit reasonable variations, reflecting real-world dynamics.

\begin{figure}[h!]
\centering
\includegraphics[width=0.86\linewidth]{figs/legend-line.pdf} \
\subfloat[Inflation Rate]{\includegraphics[width=0.7\linewidth]{figs/inflation-year.pdf}} \
\subfloat[Unemployment Rate]{\includegraphics[width=0.7\linewidth]{figs/unemployment-year.pdf}} \
\subfloat[Nominal GDP]{\includegraphics[width=0.7\linewidth]{figs/nominal_gdp-year.pdf}} \
\subfloat[Nominal GDP Growth Rate]{\includegraphics[width=0.7\linewidth]{figs/nominal_gdp_growth-year.pdf}} \
\caption{Annual variations of macroeconomic indicators.}\label{fig::annual-indicators}
\end{figure}

% \begin{figure}[t]
% \centering
% \begin{subfigure}{0.99\columnwidth}
%     \includegraphics[width=\linewidth]{figs/legend-line.pdf}
%     % \caption{Inflation Rate}
% \end{subfigure}

% \begin{subfigure}{0.7\columnwidth}
%     \includegraphics[width=\linewidth]{figs/inflation-year.pdf}
%     % \caption{Inflation Rate}
% \end{subfigure}
% % \hspace{1cm}
% \begin{subfigure}{0.7\columnwidth}
%     \includegraphics[width=\linewidth]{figs/unemployment-year.pdf}
%     % \caption{Unemployment Rate}
% \end{subfigure}

% \begin{subfigure}{0.7\columnwidth}
%     \includegraphics[width=\linewidth]{figs/nominal_gdp-year.pdf}
%     % \caption{Nominal GDP}
% \end{subfigure}
% \begin{subfigure}{0.7\columnwidth}
%     \includegraphics[width=\linewidth]{figs/nominal_gdp_growth-year.pdf}
%     % \caption{Nominal GDP Growth Rate}
% \end{subfigure}
% \caption{Annual variations of macroeconomic indicators.} \label{fig::annual-indicators}
% \end{figure}
% \begin{figure}[t]
% \centering
% \begin{subfigure}{0.92\columnwidth}
%     \includegraphics[width=\linewidth]{figs/legend-line.pdf}
%     % \caption{Inflation Rate}
% \end{subfigure}

% \begin{subfigure}{0.47\columnwidth}
%     \includegraphics[width=\linewidth]{figs/inflation.pdf}
%     \caption{Inflation Rate}
% \end{subfigure}\hfill
% \begin{subfigure}{0.47\columnwidth}
%     \includegraphics[width=\linewidth]{figs/unemployment.pdf}
%     \caption{Unemployment Rate}
% \end{subfigure}

% \begin{subfigure}{0.47\columnwidth}
%     \includegraphics[width=\linewidth]{figs/nominal_gdp.pdf}
%     \caption{Nominal GDP}
% \end{subfigure}\hfill
% \begin{subfigure}{0.47\columnwidth}
%     \includegraphics[width=\linewidth]{figs/nominal_gdp_growth.pdf}
%     \caption{Nominal GDP Growth}
% \end{subfigure}
% \caption{Annual variations of macroeconomic indicators.} \label{fig::annual-indicators}
% \end{figure}

\textbf{Macroeconomic regularities}. As the two most commonly used regularities in macroeconomic simulations for validating the plausibility of simulation results, the Phillips Curve~\cite{phelps1967phillips} and Okun's Law~\cite{okun1963potential} respectively describe the negative correlations between the annual unemployment rate and wage inflation, and the annual growth rate of unemployment and real GDP growth. As shown in Figure~\ref{fig::correlation-curves}, the decision-making of agents based on the LLM has correctly manifested phenomena in accordance with these two regularities~(Pearson correlation coefficient is -0.619, $p<0.01$ and -0.918, $p<0.001$). Notably, the rule-based baseline method displayed an incorrect positive relationship on the Phillips Curve. We attribute this advantage to the LLM's accurate perception that consumption should be reduced when unemployed, a point which will be elaborated upon in the subsequent section. Note that the Phillips Curve for AI-Eco is not shown due to the very large unemployment rate.
% The Phillips curve and the Beveridge curve, displayed in Figure~\ref{fig::correlation-curves}, demonstrate the negative correlation between the variables, a hallmark of real-world macroeconomic systems.
\begin{figure}[t]
\centering
\begin{subfigure}{0.8\columnwidth}
    \includegraphics[width=\linewidth]{figs/legend-scatter.pdf}
    % \caption{Inflation Rate}
\end{subfigure}

\includegraphics[width=0.44\columnwidth]{figs/Phillips_Curve.pdf}
\includegraphics[width=0.44\columnwidth]{figs/Okun_Law.pdf}
\caption{Phillips Curve and Okun's Law.} \label{fig::correlation-curves}
\end{figure}

\subsection{Decision-Making Abilities}
Our experiments further zoomed into the LLM agent's decision-making abilities, spanning various economic actions.
\subsubsection{Decision Rationality}
Given that an agent's decisions, including their propensities of working and consumption, might be influenced by multiple economic variables such as income and savings, we employ regression analysis to delve into the factors affecting these decisions. Specifically, the regression equation is as follows:
\begin{equation}
    p^w_i, p^c_i \sim v_i + \hat{c}_i + T(z_i) + z^r + P + s_i + r,
\end{equation}
where $\hat{c}_i, T(z_i), z^r$ denotes real consumption, the tax paid, and redistribution in the previous month. These independent variables are embedded in the LLM prompts to influence the agent's decisions. We have conducted individual regression analyses on all $N=100$ agents' 240 decisions (spanning 20 years) after applying z-score normalization on all the variables and tabulated the significance of the impact of each economic variable on their decisions. Table \ref{table:num_agents} presents the number of agents for whom the effects of the variables are statistically significant, \textit{i.e.}, $p \leq 0.05$. We observe that 1) the effects of taxation, redistribution, and expected monthly income on the work propensity of the majority of agents were significant, 2) in comparison to work propensity, the previous month's consumption, current savings, and bank interest rates significantly influence the consumption propensity of a greater number of agents, and 3) goods price has a significant impact on both work and consumption propensity for most agents. These phenomena are consistent with economic common sense about how people make decisions in daily life. Consequently, we delve further into analyzing how these economic variables impact the propensity of working and consumption.

\begin{table}[t]
\caption{The number of agents for whom the effects of the variables on decisions are statistically significant.}
    \centering
    \begin{tabular}{c|c|c|c|c|c|c|c}
    \toprule
     & $v_i$ & $\hat{c}_i$ & $T(z_i)$ & $z^r$ & $P$ & $s_i$ & $r$\\
    \hline
    $p^w_i$ & 60 & 37 & 60	& 65 & 58& 56& 31 \\
    $p^c_i$ & 65 & 73 & 51 & 52 & 62 & 100 & 49 \\
    \bottomrule
    \end{tabular}
    \label{table:num_agents}
\end{table}

% Four key decision-making actions (two for employment decisions and two for consumption decisions) were evaluated. The relationship between economic variables and these decisions, visualized in Figure~\ref{fig::decision-rationality}, validates the LLM's predictive capabilities.
\textbf{Work propensity}. Figure \ref{fig::work-decision-rationality} presents the regression coefficients with respect to the tax paid, redistribution~(left), and expected monthly income~(right, empirical cumulative distribution function, (eCDF)), where $p\leq 0.05$. Clearly, when agents paid less tax in the previous month or received greater redistribution, their propensity to work increases. The overall negative correlation between these two coefficients also suggests that agents sensitive to taxation are equally responsive to redistribution when considering their propensity to work. Besides, the coefficients of income are greater than zero for more than 60\% of agents, indicating that when agents anticipate a higher income for the current month, their propensity to work also rises.
\begin{figure}[t]
\centering
\includegraphics[width=0.44\columnwidth]{figs/tax_paid-redistribution-work.pdf}
\includegraphics[width=0.42\columnwidth]{figs/income-work.pdf}
\caption{Rationality of the LLM agent's work decisions in relation to the tax paid, redistribution, and expected monthly income.} \label{fig::work-decision-rationality}
\end{figure}

\textbf{Consumption propensity}. Figure \ref{fig::consumption-decision-rationality} shows the coefficient eCDF of savings, consumption in the previous month, and interest rate, as well as agents' real consumption $\hat{c}_j$ along with the simulation. We observe that with greater savings, an agent's consumption propensity decreases, represented as the ratio of consumption to savings and income. Further inspection of average consumption trends throughout the simulation reveals that agents tend to stabilize their consumption within a consistent range. Moreover, although the impact of consumption in the previous month is significant for more than 60\% of agents, the absolute value of the coefficient does not exceed 0.05. This confirms that agents tend to not change consumption propensity when other economic factors like savings remain stable, reflecting certain status quo bias~\cite{kahneman1991endowment}. Fiscal policies also influence agents' consumption decisions slightly. When interest rates increase, savings yield greater returns, making agents more willing to spend. 
% This will be further investigated in the next section.

% Moreover, as the prices of essential goods rise, the consumption propensity diminishes, indicating that agents prioritize meeting basic survival needs while minimizing consumption where possible.

\begin{figure}[t]
\centering
\includegraphics[width=0.44\columnwidth]{figs/savings-consumption.pdf}
\hspace{0.5cm}
\includegraphics[width=0.44\columnwidth]{figs/consumption_Coin.pdf}

\includegraphics[width=0.44\columnwidth]{figs/consumption-consumption.pdf}
\hspace{0.5cm}
\includegraphics[width=0.44\columnwidth]{figs/interest_rate-consumption.pdf}
\caption{Rationality of the LLM agent's consumption decisions in relation to current savings, consumption in the previous month, and interest rate.} 
\vspace{-0.2cm}
\label{fig::consumption-decision-rationality}
\end{figure}

\textbf{The impact of goods price}. As a key indicator reflecting the dynamics of the consumption market, the impact of goods prices on agent decisions is a crucial aspect of measuring decision-making rationality. Figure \ref{fig::goods_price} presents the coefficient eCDF. Compared to the coefficients associated with the aforementioned factors, the coefficient of goods prices is noticeably larger, indicating that agents pay much more attention to inflation or deflation in the consumption market when making decisions. Moreover, during inflation, around 60\% agents tend to reduce consumption and work propensity, representing a pessimistic view of the consumption market.


\begin{figure}[t]
\centering
\includegraphics[width=0.44\columnwidth]{figs/goods_price-work.pdf}
\includegraphics[width=0.44\columnwidth]{figs/goods_price-consumption.pdf}
\caption{The impact of goods price on the LLM agent's work and consumption decisions.} \label{fig::goods_price}
\end{figure}

\subsubsection{Interactive Analysis}
Through interactions with the model, we decipher the underlying reasons for the emergence of negative correlations in the Phillips curve. We first calculate the average consumption propensity across all the agents of two years with the highest and lowest unemployment rates. Figure \ref{fig::explain-Phillips} shows the comparison results, where *** denotes a significant disparity with $p<0.001$. Obviously, high unemployment leads to low consumption propensity significantly. To delve deeper into the reasons why agents opt to reduce consumption in the labor market of high unemployment rates, we randomly select an agent and input its conversations with LLM from the year with the highest unemployment rate back into the LLM. We then prompt the LLM to summarize the economic dynamics for each quarter and provide rationales for the consumption decisions made. The following responses demonstrate that agents will be cautious about their spending when facing deflation in the labor market under a high unemployment rate.

% \vspace{0.6em}
\begin{center}
\begin{minipage}{0.92\linewidth}
\begin{shaded}
In the last quarter, I have adjusted my \textbf{willingness to work} and my \textbf{planned expenditures} on essential goods slightly \textbf{downwards}. This decision is primarily influenced by the \textbf{continued deflation in the labor market}, resulting in a decrease in my expected income. With a lower income, I need to be \textbf{cautious about my spending} and ensure that I have enough savings for \textbf{future expenses and unforeseen circumstances}\ldots
% Additionally, the decrease in the average price of essential goods due to deflation has allowed me to allocate a slightly lower proportion of my savings and income towards consumption. 
% This helps me maintain a more conservative approach to my finances and prioritize saving for the future. 
% Overall, these adjustments reflect my desire to be financially responsible and adapt to the current economic trends, ensuring that I can meet my living costs and maintain a stable financial position.
\end{shaded}
\end{minipage}
\end{center}
% \vspace{0.6em}
% Further, a deep dive into rule-based decision-making revealed why such decisions failed to manifest the said correlations [unknown].

\begin{figure}[t]
\centering
\includegraphics[width=0.44\columnwidth]{figs/unemployment-consumption.pdf}
\caption{High unemployment leads to low consumption propensity.} \label{fig::explain-Phillips}
\end{figure}


% \subsection{Policy Analysis}
% We subjected our simulation to external interventions for policy analysis, including fiscal and monetary policy.

% \textbf{Fiscal policy}. The aftermath of heightened taxes on high-income groups is presented in Figure~\ref{fig::study-of-intervention-tax}. We observed shifts in employment willingness, unemployment rate, and inflation rate.
% \begin{figure}[t]
% \centering
% \includegraphics[width=0.47\columnwidth]{figs/study-of-intervention-tax.pdf}
% \caption{Behavioral shifts post-tax interventions.} \label{fig::study-of-intervention-tax}
% \end{figure}

% \textbf{Monetary policy}. Post elevating interest rates, we examined changes in employment willingness, consumption, unemployment rate, and inflation rate, as portrayed in Figure~\ref{fig::study-of-intervention-interest}.
% \begin{figure}[t]
% \centering
% \includegraphics[width=0.47\columnwidth]{figs/study-of-intervention-interest.pdf}
% \caption{Macroeconomic changes following interest rate adjustments.} \label{fig::study-of-intervention-interest}
% \end{figure}

% We anticipate these experiments, analyses, and findings to bolster the application and understanding of LLM-empowered agents in macroeconomic simulations. Further studies can delve deeper into unexplored territories, ensuring more robust, accurate, and insightful outcomes.



% \subsection{Experimental Setup}


% \subsection{Basic Macroeconomic Observations}


% \subsection{Comparison on the utility between LLM agent and baseline}


% \subsection{Comparison on the robustness between LLM agent and baseline}


% \subsection{Study on intervention strategy}



% \subsection{Case study on LLM-agent's decision-making ability}