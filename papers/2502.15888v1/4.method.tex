\section{Data Bias Intensifies 3D Large Model Hallucinations}\label{sec::system}
% 第四部分分析数据集要使用的图
\begin{figure*}[htb] 
\centering
\includegraphics[width=\textwidth]{figs_evaluation/multi_object_relations_plot_horizontal_adjusted.png} % 使用 \textwidth 适应双栏宽度
     \caption{(1): The relationship between object hallucination rates in 3DLLM and LL3DA and object occurrence frequencies in the dataset is shown in figures a and b.(2): The relationship between strong object correlations and object hallucination rates are shown in figure c.}
     \label{fig:influenceOfDataset}
\end{figure*}
In the previous section, we briefly examined the significant hallucinations present in existing 3D large models and provided an analysis and definition of hallucinations in 3D scenes. In this section, we will delve into the underlying causes of this phenomenon.
In Section 3 of our study, we evaluated the occurrence of object hallucination in  large 3D point cloud models. We found that  model often describes objects that do not exist in the actual scene.We hypothesize that imbalanced object frequencies and object corelation in the dataset contribute heavily to hallucination. 
\subsection{Imbalanced Frequency Distribution of Objects}
We performed statistical analysis on the hallucination rate and occurrence frequency of objects. The hallucination rate$(HR)$ of an object is defined as the ratio of scenes in which the object is incorrectly identified as present, even though it does not actually exist, to the total number of scenes where the object is absent in the test set. The occurrence frequency of an object is defined as the ratio of scenes where the object is present to the total number of scenes. As shown in Figure \ref{fig:influenceOfDataset}, $a$ represents the object hallucination rate results for 3DLLM, and $b$ represents the object hallucination rate results for LL3DA. From the figures, it can be observed that the curve representing the hallucination rate closely follows the curve representing the occurrence frequency. This suggests that objects with a high frequency of occurrence are more likely to be accurately described by the model, as it tends to repeat the most common elements. In other words, \textbf{objects with higher occurrence frequencies are more prone to hallucination}, being more likely to be incorrectly identified as present when they are actually absent.

However,in the Scannet dataset, certain objects such as the floor, wall, and door appear very frequently across many scenes.\textbf{Floor} appeared in \textbf{1506 out of 1513} scenes.\textbf{Wall} appeared in \textbf{1473} scenes.\textbf{Door} appeared in \textbf{1015} scenes.These data demonstrate that scene similarity in ScanNet is high, with the same object appearing repeatedly across multiple scenes.Based on the conclusion that excessively high occurrence frequencies can exacerbate hallucinations, we can infer that \textbf{the high overlap of objects across different scenes in the dataset} is one of the key factors \textbf{contributing to the strong hallucinations} observed in 3D large language models.
\subsection{Potential Influence of Object Correlation}
 In Figure \ref{fig:influenceOfDataset}, the y-axis represents the conditional probability \( P(AB|A) \), where A denotes the presence of object a in the scene and B denotes the presence of object b. A higher value of \( P(AB|A) \) indicates a higher likelihood that if object a is present, object b is also likely to be present. The objects b labeled on the x-axis, such as floor, wall, and door, are arranged in descending order of their hallucination rates, and the conditional probabilities also exhibit a downward trend. This suggests that \textbf{objects frequently co-occurring with others are more likely to be incorrectly identified as present}, thereby inducing hallucinations. For example, if \textbf{chairs} and \textbf{tables} often appear together in the same scene, the model might learn an implicit dependency between them. When the chair is present, the model may "hallucinate" the table, even if it isn't present in the actual scene.\\
 ScanNet is an indoor scene dataset containing environments such as bedrooms, bathrooms, and offices. Due to the specific nature of these scenes, they consistently include certain objects—such as toilets, sinks, and toilet paper—always appearing together in bathrooms. This strong correlation between objects in the dataset means that during training, the model may receive rewards for providing answers based on these associations rather than point clouds. As a result, the model may incorrectly associate these objects with one another, leading to hallucinations when detecting one object.

\begin{figure*}[t] % [t] 表示图片放置在页面顶部,跨双栏
\centering
\includegraphics[width=\textwidth]{figs_evaluation/evaluationProcess.png} % 使用 \textwidth 适应双栏宽度
     \caption{In the evaluation process, we generate new QA pairs by changing the scene while keeping the questions fixed: different scenes are randomly selected to form new QA pairs. Additionally, we modify the questions while keeping the scene fixed: spatial relationship-related questions are selected, and all QA pairs are transformed such that the object A is the focus. Then, the spatial relationship in the questions is inverted, generating new QA pairs.}
     \label{fig:evaluationProcess}
\end{figure*}



