% This must be in the first 5 lines to tell arXiv to use pdfLaTeX, which is strongly recommended.
\pdfoutput=1
% In particular, the hyperref package requires pdfLaTeX in order to break URLs across lines.

\documentclass[11pt]{article}

% Change "review" to "final" to generate the final (sometimes called camera-ready) version.
% Change to "preprint" to generate a non-anonymous version with page numbers.
\usepackage[preprint]{acl}

% Standard package includes
\usepackage{times}
\usepackage{latexsym}

% For proper rendering and hyphenation of words containing Latin characters (including in bib files)
\usepackage[T1]{fontenc}
% For Vietnamese characters
% \usepackage[T5]{fontenc}
% See https://www.latex-project.org/help/documentation/encguide.pdf for other character sets

% This assumes your files are encoded as UTF8
\usepackage[utf8]{inputenc}

% This is not strictly necessary, and may be commented out,
% but it will improve the layout of the manuscript,
% and will typically save some space.
\usepackage{microtype}

% This is also not strictly necessary, and may be commented out.
% However, it will improve the aesthetics of text in
% the typewriter font.
\usepackage{inconsolata}

% If the title and author information does not fit in the area allocated, uncomment the following
%
%\setlength\titlebox{<dim>}
%
% and set <dim> to something 5cm or larger.

\usepackage{graphicx} %
% \urlstyle{rm} %
% \def\UrlFont{\rm}  %
\usepackage{natbib}  %
\usepackage{caption} %
\usepackage{algorithm}
\usepackage{listings}
\usepackage{algorithmic}
\usepackage{amsmath}
\usepackage{booktabs}
\usepackage{multirow}
\usepackage[outdir=./]{epstopdf}
\usepackage{enumitem}
\usepackage{caption}
\usepackage{subcaption}
\usepackage{subfloat}
\usepackage{newfloat}
\usepackage{graphicx}
\usepackage{svg} % svg
\usepackage[normalem]{ulem}
\usepackage{framed}
\usepackage{mdframed}
\usepackage{xcolor}
\usepackage{lipsum}
\usepackage{float}
\usepackage{hyperref}
\usepackage{amssymb}  % For checkmarks and crosses
\usepackage{geometry} % To adjust page margins
\definecolor{shadecolor}{gray}{0.9}

\DeclareMathOperator*{\argmax}{arg\,max}
\newlist{todolist}{itemize}{2}
\setlist[todolist]{label=$\square$}
\usepackage{pifont}
\newcommand{\cmark}{\ding{51}}%
\newcommand{\xmark}{\ding{55}}%
\newcommand{\done}{\rlap{$\square$}{\raisebox{2pt}{\large\hspace{1pt}\cmark}}\hspace{-2.5pt}}
\newcommand{\wontfix}{\rlap{$\square$}{\large\hspace{1pt}\xmark}}

\newcommand{\quanming}[1]{{\color{blue} \textbf{QM}. #1}}
\newcommand{\yong}[1]{{\color{red} \textbf{Yong}: #1}}
\newcommand{\gao}[1]{{\color{pink} \textbf{Gao}: #1}}
\newcommand{\chen}[1]{{\color{red} \textbf{Chen}: #1}}
\newcommand{\chenf}[1]{\footnote{+chen+:#1}}
\newcommand{\yongrevised}[1]{{\color{pink} \textbf{Yong}: #1}}
\newcommand{\req}[1]{\textcolor{black}{#1}}
\newcommand{\rev}[1]{#1}
\newcommand{\yantex}[1]{{\color{black} \textbf{Yan}: #1}}
\newcommand{\yanf}[1]{\footnote{+yanf+:#1}}
\newcommand{\reviewf}[1]{\footnote{\color{blue}+rev+:#1}}


\newtheorem{mdefinition}{Definition}

\newtheorem{problem}{Problem}
\newcommand{\para}[1]{{\vspace{4pt} \bf \noindent #1 \hspace{0pt}}}

\setlength{\belowcaptionskip}{-0.1cm} 

\usepackage{array}
\newcolumntype{L}[1]{>{\raggedright\let\newline\\\arraybackslash\hspace{0pt}}m{#1}}
\newcolumntype{C}[1]{>{\centering\let\newline  \\\arraybackslash\hspace{0pt}}m{#1}}
\newcolumntype{R}[1]{>{\raggedleft\let\newline \\\arraybackslash\hspace{0pt}}m{#1}}


\AtBeginDocument{%
  \providecommand\BibTeX{{%
    \normalfont B\kern-0.5em{\scshape i\kern-0.25em b}\kern-0.8em\TeX}}
}


% \copyrightyear{2023}
% \acmYear{2023}
% \setcopyright{acmlicensed}
% \acmConference[WWW '24] {The Web Conference}{May 13--17, 2024}{Singapore}
% \acmBooktitle{Proceedings of The Web Conference 2024 (WWW '24), May 13--17, 2024, Singapore}
% \acmPrice{15.00}
% \acmISBN{979-8-4007-0103-0/23/08}
% \acmDOI{10.1145/XXXXXX.XXXXXX}
% \acmDOI{10.1145/3580305.3599322}

% \settopmatter{printacmref=true}

% \linespread{0.97}


% \copyrightyear{2024}
% \acmYear{2024}
% \setcopyright{acmcopyright}\acmConference[KDD'24]{Proceedings of the 30th ACM SIGKDD Conference on Knowledge Discovery and Data Mining}{August 25--29, 2024}{Barcelona, Spain}
% \acmBooktitle{Proceedings of the 30th ACM SIGKDD Conference on Knowledge Discovery and Data Mining (KDD'24), Barcelona, Spain}

% \author{Nian Li, Chen Gao, Yong Li, Qingmin Liao\\
% Tsinghua University\\
% \texttt{linian21@mails.tsinghua.edu.cn, \{chgao96, liyong07, liaoqm\}@tsinghua.edu.cn}}
% \author{Nian Li, Chen Gao, Yong Li, Qingmin Liao}
% \affiliation{Tsinghua University}
% \email{linian21@mails.tsinghua.edu.cn, {chgao96, liyong07, liaoqm}@tsinghua.edu.cn}

% \orcid{0000-0003-4689-2289}
% \affiliation{
%   \institution{Shenzhen International Graduate School, Tsinghua University}
%   \city{Shenzhen}
%   \country{China}
% }

% \author{Chen Gao}
% \authornote{Corresponding author (chgao96@gmail.com).}
% \author{Yong Li}
% \affiliation{%
%   \institution{Department of Electronic Engineering, Tsinghua University}
%   \city{Beijing}
%   \country{China}
% }
% \author{Qingmin Liao}
% \affiliation{
%   \institution{Shenzhen International Graduate School, Tsinghua University}
%   \city{Shenzhen}
%   \country{China}
% }
% \renewcommand{\shortauthors}{Wen, et. al.}
%\renewcommand{\shortauthors}{Efficient and Joint Hyperparameter and Architecture Search for Collaborative Filtering}
% \makeatletter
% \def\@copyrightspace{\relax}
% \makeatother
% \settopmatter{printacmref=false}
% \setlength{\abovedisplayskip}{5pt}    % 调整所有公式上方的间距
% \setlength{\belowdisplayskip}{5pt}

\title{Understanding and Evaluating Hallucinations in 3D Visual Language Models}
\author{
    Ruiying Peng\thanks{Co-first author.}, 
    Kaiyuan Li\footnotemark[1],  % 共享相同的标注
    Weichen Zhang, 
    Chen Gao, 
    Xinlei Chen, 
    Yong Li \\ 
    Tsinghua University
}
\begin{document}
\maketitle
\begin{abstract}
% With the rapid development of large language models, they have demonstrated impressive performance in fields such as reasoning and question answering. 
Recently, 3D-LLMs, which combine point-cloud encoders with large models, have been proposed to tackle complex tasks in embodied intelligence and scene understanding. In addition to showing promising results on 3D tasks, we found that they are significantly affected by hallucinations. For instance, they may generate objects that do not exist in the scene or produce incorrect relationships between objects. To investigate this issue, this work presents the first systematic study of hallucinations in 3D-LLMs. We begin by quickly evaluating hallucinations in several representative 3D-LLMs and reveal that they are all significantly affected by hallucinations. We then define hallucinations in 3D scenes and, through a detailed analysis of datasets, uncover the underlying causes of these hallucinations. We find three main causes: (1) Uneven frequency distribution of objects in the dataset. (2) Strong correlations between objects. (3) Limited diversity in object attributes. Additionally, we propose new evaluation metrics for hallucinations, including Random Point Cloud Pair and Opposite Question Evaluations, to assess whether the model generates responses based on visual information and aligns it with the text's meaning.









 
\end{abstract}

% \keywords{Macroeconomic Simulation; Large Language Models; Agent-based Modeling}

% \maketitle

% This work pioneers the integration of LLMs into macroeconomic simulations, bridging fields, and enhancing realism, promising to reshape our understanding of global economies.



% \begin{CCSXML}
% <ccs2012>
%    <concept>
%        <concept_id>10010405.10010455.10010460</concept_id>
%        <concept_desc>Applied computing~Economics</concept_desc>
%        <concept_significance>300</concept_significance>
%        </concept>
%    <concept>
%        <concept_id>10010147.10010341</concept_id>
%        <concept_desc>Computing methodologies~Modeling and simulation</concept_desc>
%        <concept_significance>300</concept_significance>
%        </concept>
%  </ccs2012>
% \end{CCSXML}

% \ccsdesc[300]{Applied computing~Economics}
% \ccsdesc[300]{Computing methodologies~Modeling and simulation}

% \begin{CCSXML}
% <ccs2012>
% <concept>
% <concept_id>10002951.10003317.10003331.10003271</concept_id>
% <concept_desc>Information systems~Recommender Systems</concept_desc>
% <concept_significance>500</concept_significance>
% </concept>
% </ccs2012>
% \end{CCSXML}

% \ccsdesc[500]{Data Science~Recommender Systems}
% \ccsdesc[500]{Information systems~Recommender systems}
% \keywords{Economic Simulation; Large Language Models; Web and Economics}
% \keywords{Recommendation System; Collaborative Filtering; Automated Machine Learning}
%\settopmatter{printfolios=true} % add pages
% \textcolor{blue}{
% TO DO
% \begin{itemize}[leftmargin=*]
%     \item Intervention Experiments
%     \item Adjust all the figures for aesthetics
%     \item Reorganize System, LLM agents, and Experiments to ensure clear logic and add details to supplementary materials
%     \item Add AI-Economist~(MLP) as additional baseline~(Codes for training and simulation have been prepared)
% \end{itemize}
% }

% \textcolor{blue}{Framework:
% \begin{itemize}[leftmargin=*]
%     \item Intro
%     \item Related Work
%         \begin{enumerate}
%             \item Economic simulation
%             \item LLM agents
%         \end{enumerate}
%     \item System Overview
%     \item Methodology
%         \begin{enumerate}
%             \item Environment (with basic verification on the environment)
%             \item Agent Design
%         \end{enumerate}
%     \item Results
%         \begin{enumerate}
%             \item Basic Macroeconomic Observations
%             \item Comparison on the utility between LLM agent and baseline
%             \item Comparison on the robustness between LLM agent and baseline
%             \item Study on intervention strategy
%             \item Case study on LLM-agent's decision-making ability
%         \end{enumerate}
%     \item Conclusion and Future works
% \end{itemize}
% }


\section{Introduction}
\label{sec::intro}

Embodied Question Answering (EQA) \cite{das2018embodied} represents a challenging task at the intersection of natural language processing, computer vision, and robotics, where an embodied agent (e.g., a UAV) must actively explore its environment to answer questions posed in natural language. While most existing research has concentrated on indoor EQA tasks \cite{gao2023room, pena2023visual}, such as exploring and answering questions within confined spaces like homes or offices \cite{liu2024aligning}, relatively little attention has been dedicated to EQA tasks in  open-ended city space. Nevertheless, extending EQA to city space is crucial for numerous real-world applications, including autonomous systems \cite{kalinowska2023embodied}, urban region profiling \cite{yan2024urbanclip}, and city planning \cite{gao2024embodiedcity}. 
% 1. 环境复杂性   
%    - 地标重复性问题(如区分相似建筑)  
%    - 动态干扰因素(交通流、行人)  
% 2. 行动复杂性  
%    - 长程导航路径规划  
%    - 移动控制、角度等  
% 3. 感知复杂性  
%    - 复合空间关系推理("A楼东侧商铺西边的车辆")  
%    - 时序依赖的观察结果整合

EQA tasks in city space (referred to as CityEQA) introduce a unique set of challenges that fundamentally differ from those encountered in indoor environments. Compared to indoor EQA, CityEQA faces three main challenges: 

1) \textbf{Environmental complexity with ambiguous objects}: 
Urban environments are inherently more complex,  featuring a diverse range of objects and structures, many of which are visually similar and difficult to distinguish without detailed semantic information (e.g., buildings, roads, and vehicles). This complexity makes it challenging to construct task instructions and specify the desired information accurately, as shown in Figure \ref{fig:example}. 

2) \textbf{Action complexity in cross-scale space}: 
The vast geographical scale of city space compels agents to adopt larger movement amplitudes to enhance exploration efficiency. However, it might risk overlooking detailed information within the scene. Therefore, agents require cross-scale action adjustment capabilities to effectively balance long-distance path planning with fine-grained movement and angular control.

3) \textbf{Perception complexity with observation dynamics}: 
% Rich semantic information in urban settings leads to varying observations depending on distance and orientation, which can impact the accuracy of answer generation. 
Observations can vary greatly depending on distance, orientation, and perspective. For example, an object may look completely different up close than it does from afar or from different angles. These differences pose challenges for consistency and can affect the accuracy of answer generation, as embodied agents must adapt to the dynamic and complex nature of urban environments.


\begin{table}
\centering
\caption{CityEQA-EC vs existing benchmarks.}
\label{table:dataset}
\renewcommand\arraystretch{1.2}
\resizebox{\linewidth}{!}{
\begin{tabular}{cccccc}
             & Place  & Open Vocab & Active & Platform  & Reference \\ \hline
EQA-v1      & Indoor & \textcolor{red}{\ding{55}}          & \textcolor{green}{\ding{51}}      & House3D      & \cite{das2018embodied}  \\
IQUAD        & Indoor & \textcolor{red}{\ding{55}}          & \textcolor{green}{\ding{51}}      & AI2-THOR     & \cite{gordon2018iqa} \\
MP3D-EQA     & Indoor & \textcolor{red}{\ding{55}}          & \textcolor{green}{\ding{51}}      & Matterport3D & \cite{wijmans2019embodied} \\
MT-EQA       & Indoor & \textcolor{red}{\ding{55}}          & \textcolor{green}{\ding{51}}      & House3D      & \cite{yu2019multi} \\
ScanQA       & Indoor & \textcolor{red}{\ding{55}}          & \textcolor{red}{\ding{55}}      & -            & \cite{azuma2022scanqa} \\
SQA3D        & Indoor & \textcolor{red}{\ding{55}}          & \textcolor{red}{\ding{55}}      & -            & \cite{masqa3d} \\
K-EQA        & Indoor & \textcolor{green}{\ding{51}}          & \textcolor{green}{\ding{51}}      & AI2-THOR     & \cite{tan2023knowledge} \\
OpenEQA      & Indoor & \textcolor{green}{\ding{51}}          & \textcolor{green}{\ding{51}}      & ScanNet/HM3D & \cite{majumdar2024openeqa} \\
 \hline
CityEQA-EC   & City (Outdoor)  & \textcolor{green}{\ding{51}}          & \textcolor{green}{\ding{51}}      & EmbodiedCity & - \\ \hline
\end{tabular}}
\end{table}

\begin{figure*}[!htb]
\centering
    \includegraphics[width=0.78\linewidth]{figures/example.pdf}
% \vspace{-0.2cm}
\caption{The typical workflow of the PMA to address City EQA tasks. There are two cars in this area, thus a valid question must contain landmarks and spatial relationships to specify a car. Given the task, PMA will sequentially complete multiple sub-tasks to find the answer.}
% \vspace{-0.2cm}
\label{fig:example}
\end{figure*}

As an initial step toward CityEQA, we developed \textbf{CityEQA-EC}, a benchmark dataset to evaluate embodied agents' performance on CityEQA tasks. The distinctions between this dataset and other EQA benchmarks are summarized in Table \ref{table:dataset}. CityEQA-EC comprises six task types characterized by open-vocabulary questions. These tasks utilize urban landmarks and spatial relationships to delineate the expected answer, adhering to human conventions while addressing object ambiguity. This design introduces significant complexity, turning CityEQA into long-horizon tasks that require embodied agents to identify and use landmarks, explore urban environments effectively, and refine observation to generate high-quality answers.

To address CityEQA tasks, we introduce the \textbf{Planner-Manager-Actor (PMA)}, a novel baseline agent powered by large models, designed to emulate human-like rationale for solving long-horizon tasks in urban environments, as illustrated in Figure \ref{fig:example}. PMA employs a hierarchical framework to generate actions and derive answers. The Planner module parses tasks and creates plans consisting of three sub-task types: navigation, exploration, and collection. The Manager oversees the execution of these plans while maintaining a global object-centric cognitive map \cite{deng2024opengraph}. This 2D grid-based representation enables precise object identification (retrieval) and efficient management of long-term landmark information. The Actor generates specific actions based on the Manager's instructions through its components: Navigator, Explorer, and Collector. Notably, the Collector integrates a Multi-Modal Large Language Model (MM-LLM) as its Vision-Language-Action (VLA) module to refine observations and generate high-quality answers.
PMA's performance is assessed against four baselines, including humans. 
Results show that humans perform best in CityEQA, while PMA achieves 60.73\% of human accuracy in answering questions, highlighting both the challenge and validity of the proposed benchmarks. 

% The Frontier-Based Exploration (FBE) Agent, widely used in indoor EQA tasks, performs worse than even a blind LLM. This underscores the importance of PMA's hierarchical framework and its use of landmarks and spatial relationships for tackling CityEQA tasks.

In summary, this paper makes the following significant contributions:
\vspace{-8pt}
\begin{itemize}[leftmargin=*]
    \item To the best of our knowledge, we present the first open-ended embodied question answering benchmark for city space, namely CityEQA-EC.
    \vspace{-7pt}
    \item We propose a novel baseline model, PMA, which is capable of solving long-horizon tasks for CityEQA tasks with a human-like rationale.
     \vspace{-7pt}
    \item Experimental results demonstrate that our approach outperforms existing baselines in tackling the CityEQA task. However, the gap with human performance highlights opportunities for future research to improve visual thinking and reasoning in embodied intelligence for city spaces.
\end{itemize}




\section{Related Works}


\noindent\textbf{3D Point Cloud Domain Adaptation and Generalization.}
Early endeavors within 3D domain adaptation (3DDA) focused on extending 2D adversarial methodologies~\cite{qin2019pointdan} to align point cloud features. Alternative methods have delved into geometry-aware self-supervised pre-tasks. Achituve \etal~\cite{achituve2021self} introduced DefRec, a technique employing self-complement tasks by reconstructing point clouds from a non-rigid distorted version, while Zou \etal~\cite{zou2021geometry} incorporating norm curves prediction as an auxiliary task. Liang \etal~\cite{liang2022point} put forth MLSP, focusing on point estimation tasks like cardinality, position, and normal. SDDA~\cite{cardace2023self} employs self-distillation to learn the point-based features. Additionally, post-hoc self-paced training~\cite{zou2021geometry,fan2022self,park2023pcadapter} has been embraced to refine adaptation to target distributions by accessing target data and conducting further finetuning based on prior knowledge from the source domain.
In contrast, the landscape of 3D domain generalization (3DDG) research remains nascent. Metasets~\cite{huang2021metasets} leverage meta-learning to address geometric variations, while PDG~\cite{wei2022learning} decomposes 3D shapes into part-based features to enhance generalization capabilities.
Despite the remarkable progress, existing studies assume that objects in both the source and target domains share the same orientation, limiting their practical application. This limitation propels our exploration into orientation-aware 3D domain generalization through intricate orientation learning.


\noindent\textbf{Rotation-generalizable Point Cloud Analysis.}
Previous works in point cloud analysis~\cite{qi2017pointnet, wang2019dynamic} enhance rotation robustness by introducing random rotations to augment point clouds. {However, generating a comprehensive set of rotated data is impractical, resulting in variable model performance across different scenes. To robustify the networks \wrt randomly rotated point clouds,} rotation-equivariance methods explore equivalent model architectures by incorporating equivalent operations~\cite{su2022svnet, Deng_2021_ICCV, luo2022equivariant} or steerable convolutions~\cite{chen2021equivariant, poulenard2021functional}.
Alternatively, rotation-invariance approaches aim to identify geometric descriptors invariant to rotations, such as distances and angles between local points~\cite{chen2019clusternet, zhang2020global} or point norms~\cite{zhao2019rotation, li2021rotation}. Besides, {Li \etal~\cite{li2021closer} have explored disambiguating the number of PCA-based canonical poses, while Kim \etal~\cite{kim2020rotation} and Chen \etal~\cite{chen2022devil} have transformed local point coordinates according to local reference frames to maintain rotation invariance. However, these methods focus on improving in-domain rotation robustness, neglecting domain shift and consequently exhibiting limited performance when applied to diverse domains. This study addresses the challenge of cross-domain generalizability together with rotation robustness and proposes novel solutions.} 

\noindent\textbf{Intricate Sample Mining}, aimed at identifying or synthesizing challenging samples that are difficult to classify correctly, seeks to rectify the imbalance between positive and negative samples for enhancing a model's discriminability. While traditional works have explored this concept in SVM optimization~\cite{felzenszwalb2009object}, shallow neural networks~\cite{dollar2009integral}, and boosted decision trees~\cite{yu2019unsupervised}, recent advances in deep learning have catalyzed a proliferation of researches in this area across various computer vision tasks. For instance, 
Lin \etal~\cite{lin2017focal} proposed a focal loss to concentrate training efforts on a selected group of hard examples in object detection, while Yu \etal~\cite{yu2019unsupervised} devised a soft multilabel-guided hard negative mining method to learn discriminative embeddings for person Re-ID. Schroff \etal~\cite{schroff2015facenet} introduced an online negative exemplar mining process to encourage spherical clusters in face embeddings for individual recognition, and Wang \etal~\cite{wang2021instance} designed an adversarially trained negative generator to yield instance-wise negative samples, bolstering the learning of unpaired image-to-image translation. In contrast to existing studies, our work presents the first attempt to mitigate the orientational shift in 3D point cloud domain generalization, by developing an effective intricate orientation mining strategy to achieve orientation-aware learning.


\section{3D Hallucination}
\label{sec::system}
In this section, we first validate the existence of significant hallucination issues in the current popular 3D-LLMs on the 3D captioning task using traditional object-centric method which is used in image hallucination evaluate. We then define 3D hallucinations and compare them with the multimodal hallucinations defined in previous works.

\subsection{Simple Evaluation Based on Traditional Detection Methods}
% \begin{table}[H]
%     \centering
%    \resizebox{1.0\linewidth}{!}{ \begin{tabular}{c|c|c|c|c|c}
%     \toprule
%      & Precision & Recall & F1Score & Rouge & Meteor \\
%     \hline
%     ll3da & 36.36 & 16.67 & 22.86	& 25.87 & 14.98 \\
%     3dllm & 22.97 & 8.20 & 10.92 & 9.94 & 4.37  \\
%     \bottomrule
%     \end{tabular}}
%     \label{table:num_agents}
%     \caption{Evaluate Result of Sota 3D-LLM}
% \end{table}
\begin{table}[H]
    \centering
   \resizebox{1.0\linewidth}{!}{ \begin{tabular}{cccccc}
    \toprule
     & Precision & Recall & F1Score & Rouge & Meteor \\
    \midrule
    ll3da & 36.36 & 16.67 & 22.86 & 25.87 & 14.98 \\
    3D-LLM & 22.97 & 8.20  & 10.92 & 9.94  & 4.37  \\
    \bottomrule
    \end{tabular}}
    \caption{Evaluate Result of Sota 3D-LLM}
    \label{table:hallucinatiovalid}
\end{table}
First, we evaluate whether existing 3D-LLMs are affected by object hallucinations in tasks where it is relatively easier compare with relation hallucinations. Here, we use traditional object hallucination definitaion in image-text area which defines object hallucinations as situations where the items described in the model's output do not exist in the real scene. If the object described by 3D-LLM mismatch with the ground truth, we consider that a hallucination has occurred. Formally, we define \( A \) as the set of items output by the model, and \( B \) as the set of items present in the real scene. The evaluation metric can be defined as:
% \setlength{\abovedisplayskip}{8pt} 
% \setlength{\belowdisplayskip}{8pt}
 \begin{align}
      Precision= \frac{|A \cap B|}{|A|} 
\end{align}
 % \vspace{-5pt}
 \begin{equation}
      Recall= \frac{|A \cap B|}{|B|} 
 \end{equation}
 
To validate that existing 3D models suffer from significant object hallucinations, we select two representative 3D models : ll3da and 3D-LLM for evaluation. We use the metric defined above. The results are presented in Table \ref{table:hallucinatiovalid}. As we can see, both models perform badly and exhibit significant hallucination issues in the object description task. To better illustrate the evaluation of hallucinations, we present our evaluation of LL3DA on the description task as a Recall-Precision plot, as shown in Fig. \ref{fig:hallucination_eva}. The image is divided into the bottom-left corner and the top-right corner. The bottom-left corner indicates that the model struggles with hallucinations in the object description task, while the top-right corner demonstrates that the model performs well.  It can be observed that most of the samples are concentrated in the lower-left corner of the plot, which reflects the presence of severe hallucinations in the majority of examples produced by the current state-of-the-art models.

\begin{figure}[H]
\begin{center}
\begin{minipage}{0.9\linewidth}
% \centering
    \includegraphics[width=1.0\linewidth]{figs_evaluation/PRF_scatter_2d_ll3da.jpg}
     \caption{Object hallucination evaluation for 3D LLMs. Precision measures the proportion of described objects that exist in the scene, while recall represents the proportion of scene objects that are described.}
     \label{fig:hallucination_eva}
\end{minipage}
\end{center}
\end{figure}
\subsection{3D Hallucination Definition}

\subsubsection{Modality Difference}
Previous work on hallucinations has primarily focused on text and image modalities, as well as their interactions. Since the main difference between 3D-LLMs and earlier vision-based large models lies in the input modalities, we approach the analysis and comparison of the three types of hallucinations from the perspective of input modalities.
As shown in Table \ref{table:Different_modality}, unlike text-based LLMs and text-image-based LVLMs, 3D-LLMs primarily use text and point cloud modalities as inputs, which brings extra depth information.

\begin{table}[h]
\centering
\Large
\resizebox{1.0\linewidth}{!}{
\begin{tabular}{ccccccc}
\toprule
\multirow{3}{*}{Model Type} & 
\multicolumn{3}{c}{Input Modality} &
\multicolumn{3}{c}{Modality Conflict} \\ \cmidrule(lr){2-7}
& Text    & Vision   & Depth   & Knowledge Conflict & Text-Image Conflict & Scene Conflict \\ \midrule
\multirow{1}{*}{LLM} & \checkmark & \xmark  & \xmark  & \checkmark & \xmark & \xmark \\
\multirow{1}{*}{LVLM} & \checkmark & \checkmark & \xmark & \checkmark & \checkmark & \xmark  \\
\multirow{1}{*}{3D-LLM} & \checkmark & \checkmark & \checkmark & \checkmark & \checkmark & \checkmark  \\
\bottomrule
\end{tabular}}
\caption{Modality Difference}
\label{table:Different_modality}
\end{table}

\begin{table}[h]
\centering
\Large
\resizebox{1.0\linewidth}{!}{
\begin{tabular}{ccccccc}
\toprule
\multirow{3}{*}{Model Type} & 
\multicolumn{3}{c}{Object Hallucination} &
\multicolumn{3}{c}{Relation Hallucination} \\ \cmidrule(lr){2-7}
&Color & Shape & Size & Abstract& Relative & Accurate \\ \midrule
\multirow{1}{*}{Text Hallucination} & \checkmark & \xmark  & \xmark  & \checkmark & \xmark & \xmark \\
\multirow{1}{*}{Image Hallucination} & \checkmark & \checkmark & \xmark & \checkmark & \checkmark & \xmark  \\
\multirow{1}{*}{3D Hallucination} & \checkmark & \checkmark & \checkmark & \checkmark & \checkmark & \checkmark  \\
\bottomrule
\end{tabular}}
\caption{Classification of Hallucinations}
\label{table:classification}
\end{table}

The uniqueness of the input modalities leads to differences in the interactions between modalities. In text hallucinations, conflicts only arise between different textual knowledge, i.e., knowledge conflicts, which are also presented in LVLMs and 3D-LLMs, as both are built on LLMs. In image hallucinations, conflicts occur between textual and visual information. However, in 3D hallucinations, the depth information leads to conflicts where 3D-LLMs generates fictitious spatial relationships within the scene. We refer to this phenomenon as \textbf{scene conflict}.

\subsubsection{Hallucination Definition}

To define hallucination types appeared in scene conflict more concretely and accurately, we abstract the 3D scene into objects and relationships, thus defining two types of hallucinations: \textbf{Object hallucinations} and \textbf{Relation hallucinations}. We present the classification in Table \ref{table:classification}.

Object hallucinations are primarily related to the attributes of objects, such as color, shape, and size. Among these attributes, \textbf{size attribute} requires accurate depth information for proper evaluation, making this a hallucination type unique to 3D scenes. Formally, we use \(H_{obj}\) to represent object hallucination, \(S\) to represent the attributes set. \(Attr^{i}_{true} \in S\) represents the real object's attribute. \(Attr^{i}_{pred}\) represents the attributes in the prediction of 3D-LLM. 

\begin{equation}
    H_{obj} = S[Attr^{i}_{true} \neq Attr^{i}_{pred}]
\end{equation}


Relation hallucinations, on the other hand, are primarily concerned with the relationships between objects.
Among these relations, \textbf{Abstract relationship hallucinations} refer to the functional relationships between objects. \textbf{Relative positional relationships} refer to broader postion relationships, such as left-right orientation, which can usually be inferred from a given view. However, because a single view lacks depth information, precise positional relationships, such as "hanging" or "standing on," cannot be determined. In 3D scene, we can deduce \textbf{accurate spatial relations} among objects. Formally, we use \(O_i\) and \(O_j\) to represent two objects, use \(\stackrel{rel}{\longrightarrow}\) to represent relationship between two objects, use \(\stackrel{pred}{\longrightarrow}\) to represent predicted relationship. The we can define relation hallucination as:

\begin{equation}
    O_i \stackrel{rel}{\longrightarrow} O_j \neq O_i \stackrel{pred}{\longrightarrow} O_j
\end{equation}

\section{SAPS: Semantic Alignment for Policy Stitching}\label{sec:method-alignment}
Relative representations \citep{Moschella2022-yf}, used as a base for zero-shot stitching in R3L, involve computing a distance function between a set of samples, called \say{anchors}, to project the output of each encoder to a shared latent space, enabling the subsequent training of a universal policy. Semantic alignment, instead, estimates a direct mapping between latent spaces.

Consider the environment $\mathcal{M}_u^j$ for which no dedicated policy exists. However, we do have an encoder $\phi_u^i$ and a controller $\psi_v^j$, extracted from policies $\pi_u^i$ and $\pi_v^j$, respectively. 
We estimate an affine transformation $\tau_u^v$: $\mathcal{X}_{u}^{i} \mapsto \mathcal{X}_{v}^{j}$, mapping embeddings produced by $\phi_u^i$ into the space of $\pi_v^j$. This yields a new latent space:

\begin{align}
    & \tau_u^v(\enc_u^i(\mathbf{o}_u)) \approx \enc_v^j(\mathbf{o}_v)\\
    & \tau_u^v(\mathbf{x}_{u}^i) \approx \mathbf{x}_{v}^j
\end{align}

that is compatible with the existing $\psi_v^j$.
This enables the stitching of encoders and controllers from $\pi_u^i$ and $\pi_v^j$, respectively, to obtain a new policy $\tilde{\pi}_u^j$ that can act in $\mathcal{M}_u^j$, without additional training:
\begin{equation}\label{eq:2}
    \tilde{\pi}_u^j(o_u) = \con_v^j[\tau_u^v(\enc_u^i(\mathbf{o}_u))]
\end{equation}

\paragraph{Estimating $\tau$}
As in \cite{maiorca2023latent}, assume to be given latent spaces $\mathbf{X}_u$ and $\mathbf{X}_v$ which here correspond to the embedding of two visual variations in the space of observations.
We use SVD to obtain an affine transformation $\tau_u^v(\mathbf{x}_u) = \mathbf{R} \mathbf{X}_u + \mathbf{b}$.


\paragraph{Collecting the Dataset.}
The anchor embeddings $\mathbf{X}_u$ and $\mathbf{X}_v$ derive from sets of anchor points $\mathbf{A}_u$ and $\mathbf{A}_v$. Following previous works \citep{maiorca2023latent, Moschella2022-yf, ricciardi2025r3lrelativerepresentationsreinforcement} anchor pair ($\mathbf{a}_u$, $\mathbf{a}_v$) must share a semantic correspondence, meaning both samples represent the same underlying concept (e.g., the same spatial position in a racing track, viewed under two different visual styles).
In supervised learning contexts, anchor pairs can come from paired datasets (e.g., bilingual corpora). In the context of online RL, however, such datasets do not naturally exist. Hence, we collect datasets sharing a correspondence.
This correspondence can be obtained by either rolling out a policy and replaying the same set of actions with different visual variations, as already done in \cite{jian2023policy, ricciardi2025r3lrelativerepresentationsreinforcement}, or by simply applying visual transformations to the image in pixel space. This yields corresponding observation sets $\mathbf{A}_u$ and $\mathbf{A}_v$ that can be embedded by each domain’s encoder to create $\mathbf{X}_u$ and $\mathbf{X}_v$. Finally, we solve for $\tau_u^v$ using the SVD-based procedure above.


%\AR{da inserire forse: Specifically, we estimate $\tau_u^v$, following the technique used in \cite{maiorca2023latent}. which suggests that, given two latent spaces $\mathbf{X} \in \mathbb{R}^{n \times d1}$ and $\mathbf{Y} \in \mathbb{R}^{m \times d2}$ from independently trained deep neural networks, the transformation $\tau$ that directly maps $\mathbf{X}$ to $\mathbf{Y}$: (i) is mostly orthogonal and (ii) can be estimated from a few corresponding elements between the two spaces. In our work, $\mathbf{X}$ and $\mathbf{Y}$ are produced by $\enc_u$ and $\enc_v$, respectively. As in \cite{maiorca2023latent}, we use \textit{Singular Value Decomposition} (SVD) to estimate the optimal orthogonal transformation.}

In our context, we assume that an agent trained end-to-end to solve a specific task in a specific environment will generate a comprehensive set of observations, providing a reasonable approximation of the entire latent space. Nevertheless, forcing the agent to explore more could be beneficial in this context.
In our experiments, we gather parallel samples either by directly translating the observation in pixel space, when there is a well-defined known visual variation between the environments, or by replaying the same sequence of actions in both environments, that in this case must be deterministic and initialized with the same random seed. We leave to future research other possible approximation techniques for translating observations between different environments.
\makeatletter
\renewcommand{\subsubsection}[1]{%
  \par\noindent\textbf{#1}\par % 不换行且加粗
}
\makeatother

\section{Evaluation and Detection}\label{sec::exp}

% \begin{figure*}[t] % [t] 表示图片放置在页面顶部,跨双栏
% \centering
% \includegraphics[width=\textwidth]{figs_evaluation/evaluationProcess.png} % 使用 \textwidth 适应双栏宽度
%      \caption{In the evaluation process, we generate new QA pairs by changing the scene while keeping the questions fixed: different scenes are randomly selected to form new QA pairs. Additionally, we modify the questions while keeping the scene fixed: spatial relationship-related questions are selected, and all QA pairs are transformed such that the object A is the focus. Then, the spatial relationship in the questions is inverted, generating new QA pairs.}
%      \label{fig:evaluationProcess}
% \end{figure*}



\subsection{ Inadequacy of Existing Evaluation Frameworks}
Existing evaluation frameworks for 2D multimodal models, such as POPE~\cite{li2023evaluating}, are insufficient for addressing the challenges in 3D point cloud large language models (LLMs). Since the POPE view uses yes/no questions to evaluate model object hallucinations, which cannot accurately assess the model's understanding of spatial relationships or visual details such as attributes.In Section 3, we assess hallucinations in 3D point cloud models by evaluating object hallucination in description tasks. However, this method has two main limitations: 1) It only detects hallucinations in description tasks, as not all responses involve objects. 2) It doesn't analyze other types of hallucinations, such as attribute or relational errors.

 Therefore, we aim to propose a more stable, fair, and flexible evaluation framework for evaluating hallucinations in 3D point clouds.
% 随机场景实验的结果
\begin{table*}[h]
\centering
\small
\begin{tabular}{llccc|ccc}
\hline
\multicolumn{2}{c} {\multirow{3}{*}{Type}} & \multicolumn{3}{c|}{ll3da} & \multicolumn{3}{c}{3dllm} \\ \cline{3-8}
 
     &   & \multicolumn{2}{c}{Accuracy} & \multirow{2}{*}{\textbf{$HR_{random}$\%}}  & \multicolumn{2}{c}{Accuracy} & \multirow{2}{*}{\textbf{$HR_{random}$\%}}  \\
     
     &   & Rouge-L & Meteor&  & Rouge-L & Meteor&  \\ \hline


\multirow{4}{*}{Relation} & Direction  & 30.62  & 19.53 &\textbf{33.21} & 30.32  & 19.77 &\textbf{30.43}\\
                              & Containment  & 43.28  & 35.27&\textbf{36.89}  & 42.51  & 31.98&\textbf{43.69} \\ 
                              & Contact & 35.08  & 23.55 &\textbf{34.72}  & 35.58  & 24.2 &\textbf{36.79}\\ 
                              & Distance    & 32.02  & 22.71 &\textbf{31.49} & 32.5  & 21.36&\textbf{28.94} \\ \hline
                             
\multirow{3}{*}{Property}  & Color                    & 47.38  & 41.9 &\textbf{62.69} & 51.72   & 47.38&\textbf{61.77} \\ 
                                      & Shape                    & 42.74  & 31.9  &\textbf{49.48}& 44.56  & 32.94 &\textbf{46.39}\\ 
                                      & Size                     & 43.74  & 39.01 &\textbf{74.29} & 47.48  & 37.57&\textbf{51.43} \\ \hline
                                     
\multicolumn{2}{c}{Comparison}                 & 24.75  & 17.65 &\textbf{63.16} & 29.43  & 21.82&\textbf{42.11} \\ 
\multicolumn{2}{c}{Quantity} & 50.18  & 41.84&\textbf{63.93}  & 49.85  & 42.68 &\textbf{53.88} \\ 
\multicolumn{2}{c}{Usage} & 32.22  & 21.62 &\textbf{34.78} & 30.88  & 22.03&\textbf{26.09} \\ 
\multicolumn{2}{c}{Other}  & 37.22  & 31.72&\textbf{38.89}  & 39.8  & 32.82&\textbf{25.93} \\ 
\hline
\end{tabular}
\caption{Model Performance and Hallucination Rate in Random Scenarios.Accuracy refers to the evaluation result between the model's response and the ground truth. $HR_{random}$ is the hallucination rate calculated based on random scenes as defined in Section 5.}
\label{table:randomSceneResult}
\end{table*}
% 随机场景实验的分析图
\begin{figure*}[h]
\begin{center}
\centering
    \includegraphics[width=0.8\linewidth]{figs_evaluation/propertydistribution.png}
     \caption{Impact of Attribute Simplicity on Accuracy.ROUGE represents the average quality of question-answer pairs for a specific item, while the Top 3 Ratio is the proportion of the three most common attributes of the item.}
     \label{fig:topkRatio}
\end{center}
\end{figure*}
\subsection{Proposed Evaluation Framework}
We propose two strategies for detecting hallucinations in 3D point cloud models.\\
\textbf{\emph{Random Point Cloud Pair Evaluation}} ~We select a random point cloud and ask the model the same question on both the original and new point clouds. If the answers are identical, it's considered a hallucination, indicating the model doesn't integrate visual context and just maps the question to a fixed answer.\\
\textbf{\emph{Opposite Question Evaluation}} ~For a fixed point cloud, we ask two Opposite questions (e.g., "What is on the right of the table?" and "What is on the left?"). If the model gives the same answer, it's a hallucination, suggesting the model isn't using the spatial information from the point cloud.

 By employing these two strategies, we aim to identify cases where the model fails to distinguish between spatially different scenarios or produces inconsistent responses to questions.
\subsection{ Inadequacy of Existing Evaluation Frameworks}
The entire pipeline is illustrated in Figure \ref{fig:evaluationProcess}. \\
\textbf{Data Generation}: In the \textbf{change scene} experiment, for each $(Q_i,A_i,S_i)$ pair, we randomly select a different $S_j$ from the scene set to create a new $(Q_i,A_i,\{S_i, S_j\})$ pair dataset.In the \textbf{change question} experiment, we first select questions related to spatial relationships and use GPT-4 to convert each QA pair into a dataset where the answer is an object, resulting in the \textit{scanqa-SR} dataset. For each spatial relationship question $Q_i$ in \textit{scanqa-SR}, we generate its opposite $Q_j$ to form $(\{Q_i, Q_j\},A_i,S_i)$ pairs, creating the \textit{scanqa-SR-Opposite} dataset.\\
\textbf{Experiment}: We then conduct tests using the aforementioned data on different models. 
In Experiment 1, for a given question \( q_i \), we generate two answers, \( a_{ij} \) and \( a_{ik} \), corresponding to two different scenes, \( s_{j} \) and \( s_{k} \), respectively.
We use BLEU-4~\cite{papineni2002bleu}, ROUGE~\cite{lin-2004-rouge}, and METEOR~\cite{banerjee2005meteor} metrics to measure the similarity between two answers. The hallucination rate($HR_{random}$) is calculated as follows:
\begin{equation}
HR_{\text{random}} = \frac{1}{N} \sum_{i=1}^{N} \mathbf{1}(\text{ROUGE}(a_{ij}, a_{ik}) > 0.66)
\end{equation}
In Experiment 2, for a fixed scene \( s_i \), we generate two answers, \( a_{ji} \) and \( a_{jk} \), for two semantically opposite questions, \( q_j \) and \( q_k \).The hallucination rate($HR_{opposite}$) is calculated as follows:
\begin{equation}
HR_{\text{opposite}} = \frac{1}{N} \sum_{i=1}^{N} \mathbf{1}(\text{ROUGE}(a_{ji}, a_{jk}) > 0.66)
\end{equation}
% \textbf{Hallucination Rate}: We use BLEU-4\cite{papineni2002bleu}, ROUGE\cite{lin-2004-rouge}, and METEOR\cite{banerjee2005meteor} metrics to measure the similarity between two answers. We define hallucination as occurring when the ROUGE score between two answers is greater than 0.66, indicating that the answers are semantically identical. The hallucination rate(\(HR\)) is computed as the ratio of hallucinations to the total number of experiments (\(N\)).
% \begin{equation}
% HR_{\text{random}} = \frac{1}{N} \sum_{i=1}^{N} \mathbf{1}(\text{ROUGE}(a_{ij}, a_{ik}) > 0.66)
% \end{equation}
% \begin{equation}
% HR_{\text{opposite}} = \frac{1}{N} \sum_{i=1}^{N} \mathbf{1}(\text{ROUGE}(a_{ji}, a_{jk}) > 0.66)
% \end{equation}

\begin{table*}[h]
\centering
\small
\begin{tabular}{lllcccc}
\hline
Dataset & Task &Model & Bleu-4  & Rouge-L & Meteor   &\textbf{$HR_{opposite}$\%}  \\ \hline
\multirow{6}{*}{scannet}&\multirow{2}{*}{scanqa}&ll3da&7.64&36.56&26.95&/\\ 
& &3dllm&0.80&37.46&28.18&/ \\  \cline{2-7}
&\multirow{2}{*}{scanqa-SR}&ll3da&0.02&13.34&9.68&/ \\ 
&&3dllm&0.0&15.55&10.28&/ \\   \cline{2-7}
&\multirow{2}{*}{scanqa-SR-Opposite}&ll3da&/&/&/&\textbf{56.27} \\ 
&&3dllm&/&/&/&\textbf{52.25} \\ \hline
\end{tabular}
\caption{Model Performance and Hallucination Rate on Semantically Opposite Questions.BLEU-4, ROUGE, and METEOR are evaluation metrics for model response quality based on ground truth, while $HR_{opposite}$ represents the hallucination rate in the opposite-question experiment.}
\label{table:oppositeQuestionResult}
\end{table*}
\section{Evaluation on 3dllm and ll3da}
\subsection{Hallucinations in Random Scene Queries}
We evaluate two models using the approach above. Table \ref{table:randomSceneResult} presents the results for random scenes. ROUGE and METEOR measure performance on ScanQA, while $HR_{random}$ is defined in Section 5.2. The table shows a positive correlation between accuracy and hallucination rate. LL3DA and 3DLLM both exhibit low accuracy and hallucination rates for spatial questions but higher rates for object attributes.\\
This suggests that the model exhibits significant hallucination issues, where it answers questions without considering the visual context, yet its responses appear 'better' or closer to the ground truth. Upon examining the training set, we find that object attributes often align with typical characteristics—for example, tables are usually black, white, or brown, and televisions are typically rectangular. This indicates that the model learns attribute associations due to the homogeneous nature of indoor scenes and the limited diversity of attributes.
% 模型准确率与物品之间的关系
\subsection{Relationship Between Attribute Uniformity and Answer Accuracy}
We plotted Figure \ref{fig:topkRatio} to illustrate the relationship between the uniformity of an object's properties and the accuracy of the answers. For instance, chair color is queried 346($N$) times, with black ($T_1$ times), brown ($T_2$ times), and gray ($T_3$ times) as the most frequent colors. To quantify attribute uniformity, we introduce the "Top-K Ratio," where the Top-3 Ratio for the chair can be calculated as: 
\begin{equation}
\text{Top-3 Ratio} = \frac{T_1 + T_2 + T_3}{N}.
\end{equation} The x-axis of the figure represents the average ROUGE score for questions related to a specific object, with higher ROUGE scores indicating that the questions regarding the object's properties are more easily answered correctly. The three plots from left to right show the relationship between the accuracy of answers and the uniformity of the object's properties, specifically color, shape, and size.
In the plots for color and shape, the distribution of points is approximately linear, confirming a strong positive correlation between the accuracy of the answers and the uniformity of the object's properties. Additionally, we observed that many points clustered near a Top-3 Ratio of 1, suggesting that the dataset contains objects with highly uniform attributes. Such objects tend to exhibit a strong correlation between the object and a specific attribute, which makes it easier for the model to hallucinate the correct attribute. 
\subsection{Hallucinations in Opposite-Question Queries}The results for testing with 
opposite questions within the same scene are presented in Table \ref{table:oppositeQuestionResult}.The ScanQA dataset includes a wide range of QA pairs involving various attributes, spatial relationships, and other data types. In contrast, ScanQA-SR focuses solely on spatial relationships and transforms all QA pairs into those where the answer is the object itself.\\
By comparing the results from these two datasets, we observe that the ROUGE scores for ScanQA-SR are significantly lower than those for ScanQA. This indicates that the model is more prone to errors when dealing with spatial relationship tasks. To investigate whether the model truly understands the meaning of spatial relationships, we created a dataset of opposite questions specifically for spatial relationships. The goal was to assess the model’s ability to handle questions about opposing spatial positions.\\
However, we found that the hallucination rate for both models exceeded 50\%. This suggests that when posed with opposite questions about the same scene, the model has a 50\% chance of giving the same answer. This further demonstrates that the model is prone to errors and hallucinations when handling spatial relationship queries. The results imply that the model may lack a proper visual-semantic understanding of spatial relationships, leading it to answer incorrectly without considering point cloud data.

% \begin{table*}[t]
% \centering
% \begin{tabular}{lllccccc}
% \hline
% Dataset & Task &Model & Bleu-4  & Rouge-L & Meteor & Acc & Hallucination Rate  \\ \hline
% \multirow{6}{*}{scannet}&\multirow{2}{*}{scanqa}&ll3da&7.64&36.56&26.95&26.46&/\\ 
% & &3dllm&0.80&37.46&28.18&26.57&/ \\  \cline{2-8}
% &\multirow{2}{*}{scanqa-object-spatial}&ll3da&0.02&13.34&9.68&7.85&/ \\ 
% &&3dllm&0.0&15.55&10.28&8.34&/ \\   \cline{2-8}
% &\multirow{2}{*}{scanqa-spatial-hallucination}&ll3da&1.32&57.02&56.21&/&56.27 \\ 
% &&3dllm&1.08&55.65&52.60&/&52.25 \\ \hline
% \end{tabular}
% \caption{detection}
% \label{table:performance}
% \end{table*}




% We have merged several spatial relationships into broader categories to simplify classification. The relationships are grouped as follows:

% \begin{itemize}
%     \item \textbf{Directional Relationships}: These include terms such as \textit{opposite}, \textit{under}, \textit{above}, \textit{outside}, \textit{right}, \textit{left}, \textit{front}, \textit{behind}, \textit{in front}, \textit{diagonal}, and \textit{facing}.
%     \item \textbf{Contact/Support Relationships}: This category includes relationships like \textit{resting on}, \textit{lying on}, \textit{standing on}, \textit{on}, \textit{on top of}, \textit{hanging on}, \textit{attached to}, and \textit{against}.
%     \item \textbf{Containment/Surrounding Relationships}: Terms like \textit{inside}, \textit{outside}, \textit{surrounding}, and \textit{enclosed by} fall under this group.
%     \item \textbf{Distance Relationships}: This group includes \textit{far away}, \textit{away}, \textit{near}, \textit{close to}, and \textit{detached from}.
%     \item \textbf{Adjacency/Relative Location Relationships}: This category involves relationships such as \textit{next to}, \textit{adjacent to}, \textit{on each side of the couch}, \textit{between}, and \textit{between two walls}.
%     \item \textbf{Other Descriptive Relationships}: This group includes \textit{centered on}, \textit{off the wall}, and \textit{described as}.
% \end{itemize}
















% \begin{figure*}[ht]
% \begin{center}
% \centering
%     \includegraphics[width=0.93\linewidth]{figs_evaluation/question_reverse_line.jpg}
%      \caption{contract question}
% \end{center}
% \end{figure*}


% \section{Related Works}


\noindent\textbf{3D Point Cloud Domain Adaptation and Generalization.}
Early endeavors within 3D domain adaptation (3DDA) focused on extending 2D adversarial methodologies~\cite{qin2019pointdan} to align point cloud features. Alternative methods have delved into geometry-aware self-supervised pre-tasks. Achituve \etal~\cite{achituve2021self} introduced DefRec, a technique employing self-complement tasks by reconstructing point clouds from a non-rigid distorted version, while Zou \etal~\cite{zou2021geometry} incorporating norm curves prediction as an auxiliary task. Liang \etal~\cite{liang2022point} put forth MLSP, focusing on point estimation tasks like cardinality, position, and normal. SDDA~\cite{cardace2023self} employs self-distillation to learn the point-based features. Additionally, post-hoc self-paced training~\cite{zou2021geometry,fan2022self,park2023pcadapter} has been embraced to refine adaptation to target distributions by accessing target data and conducting further finetuning based on prior knowledge from the source domain.
In contrast, the landscape of 3D domain generalization (3DDG) research remains nascent. Metasets~\cite{huang2021metasets} leverage meta-learning to address geometric variations, while PDG~\cite{wei2022learning} decomposes 3D shapes into part-based features to enhance generalization capabilities.
Despite the remarkable progress, existing studies assume that objects in both the source and target domains share the same orientation, limiting their practical application. This limitation propels our exploration into orientation-aware 3D domain generalization through intricate orientation learning.


\noindent\textbf{Rotation-generalizable Point Cloud Analysis.}
Previous works in point cloud analysis~\cite{qi2017pointnet, wang2019dynamic} enhance rotation robustness by introducing random rotations to augment point clouds. {However, generating a comprehensive set of rotated data is impractical, resulting in variable model performance across different scenes. To robustify the networks \wrt randomly rotated point clouds,} rotation-equivariance methods explore equivalent model architectures by incorporating equivalent operations~\cite{su2022svnet, Deng_2021_ICCV, luo2022equivariant} or steerable convolutions~\cite{chen2021equivariant, poulenard2021functional}.
Alternatively, rotation-invariance approaches aim to identify geometric descriptors invariant to rotations, such as distances and angles between local points~\cite{chen2019clusternet, zhang2020global} or point norms~\cite{zhao2019rotation, li2021rotation}. Besides, {Li \etal~\cite{li2021closer} have explored disambiguating the number of PCA-based canonical poses, while Kim \etal~\cite{kim2020rotation} and Chen \etal~\cite{chen2022devil} have transformed local point coordinates according to local reference frames to maintain rotation invariance. However, these methods focus on improving in-domain rotation robustness, neglecting domain shift and consequently exhibiting limited performance when applied to diverse domains. This study addresses the challenge of cross-domain generalizability together with rotation robustness and proposes novel solutions.} 

\noindent\textbf{Intricate Sample Mining}, aimed at identifying or synthesizing challenging samples that are difficult to classify correctly, seeks to rectify the imbalance between positive and negative samples for enhancing a model's discriminability. While traditional works have explored this concept in SVM optimization~\cite{felzenszwalb2009object}, shallow neural networks~\cite{dollar2009integral}, and boosted decision trees~\cite{yu2019unsupervised}, recent advances in deep learning have catalyzed a proliferation of researches in this area across various computer vision tasks. For instance, 
Lin \etal~\cite{lin2017focal} proposed a focal loss to concentrate training efforts on a selected group of hard examples in object detection, while Yu \etal~\cite{yu2019unsupervised} devised a soft multilabel-guided hard negative mining method to learn discriminative embeddings for person Re-ID. Schroff \etal~\cite{schroff2015facenet} introduced an online negative exemplar mining process to encourage spherical clusters in face embeddings for individual recognition, and Wang \etal~\cite{wang2021instance} designed an adversarially trained negative generator to yield instance-wise negative samples, bolstering the learning of unpaired image-to-image translation. In contrast to existing studies, our work presents the first attempt to mitigate the orientational shift in 3D point cloud domain generalization, by developing an effective intricate orientation mining strategy to achieve orientation-aware learning.




\section{Conclusion}\label{sec::conclusion}
This study classifies 3D hallucinations and evaluates the severity of hallucinations in the large-scale point cloud models 3DLLM and LL3DA through description and QA tasks. By analyzing hallucination rates across datasets, we identify that high object frequency, strong correlations, and attribute singularity contribute to hallucinations. We explore whether models rely on visual information, but current tasks and metrics only measure text similarity to ground truth. To address this, we design two experiments and define hallucinations based on the results. Our findings show that models struggle to answer contextually accurate questions and align spatial relationships with visual concepts.

% In this study, we classify 3D hallucinations and conduct evaluation experiments on the large-scale point cloud models 3DLLM and LL3DA. These experiments assess the severity of hallucinations in both description and QA tasks. By analyzing the hallucination rates in relation to dataset distributions, we identify that high object frequency, strong object correlations, and the singularity of object attributes contribute to the occurrence of hallucinations. We aim to explore whether the models respond based on visual information. However, current tasks and evaluation metrics only measure the similarity between the generated text responses and the ground truth. To address this, we design two experiments and propose a definition of hallucinations based on the experimental findings. The results reveal that the models exhibit significant issues, such as failing to answer questions in accordance with the scene context and struggling to align textual expressions of spatial relationships with visual concepts.



% \clearpage

\section{Limitations}

In this study, we provide a detailed classification of hallucination types specifically for the QA task. Each QA pair is classified to detect corresponding hallucinations. However, for the description task and other long-text tasks, no specific approach is proposed to detect the types of hallucinations present in the generated answers. This limitation means that our evaluation only demonstrates the significant hallucination issues within 3D point cloud models, and uses different types of short QA pairs to explore the following questions: 1) Which types of questions are more likely to induce hallucinations in the model? 2) How does the dataset distribution impact the occurrence of hallucinations in the model?

Furthermore, we identify that models are particularly prone to attribute hallucinations and investigate the relationship between dataset distribution and hallucination rates. Regarding spatial relationship hallucinations, our experiments only reveal that the models lack understanding of spatial relationships, but do not explain why the models perform worse on spatial relationship-related questions compared to other question types.

Third, in our experiments designed to explore whether the models answer based on visual information or rely on textual inputs alone, the results indicate that the current dataset is overly simple and highly regular, which allows the models to disregard visual information in favor of answering based on text alone. However, we do not provide insights into why the models do not incorporate point cloud information in their responses from an architectural perspective.

Finally, we utilize GPT-4 to generate a new annotated dataset, which, compared to manual annotation, may contain some minor errors. Although we have discussed hallucination issues in 3D large language models and highlighted the problem of models not responding based on point cloud data, this should not be interpreted as a pessimistic view of the development of 3D language models. On the contrary, we aim to identify the reasons behind their suboptimal performance, such as the dataset distribution issues discussed in this paper. We hope that our work can provide new insights and ideas for further improving the performance of 3D large language models.




% \bibliographystyle{ACM-Reference-Format}
% % % \balance
% % % \bibliographystyle{plain}
\bibliography{bibliography}
% \nobalance

% \nobalance 

% \appendix
% \cleardoublepage
\section{Appendix}

\subsection{Details of Evaluation Settings}\label{asec:eval-setting}

\subsubsection{Benchmarks}\label{asec:eval-setting-bench}

Since PLPHP maintains the computational integrity of the LVLMs' Prefilling Stage, its efficiency advantage is primarily reflected in the low decoding latency during the subsequent Decoding Stage. Therefore, we mainly choose benchmarks composed of open-ended VQA and image captioning tasks. The benchmarks we select encompasses both multi-image task benchmarks and single-image task benchmarks.

$\bullet$ \textbf{Multi-Image benchmarks}: The LLaVA-Interleave Bench is a comprehensive benchmark dataset designed to evaluate the performance of LVLMs in multi-image scenarios. It consists of 13 challenging tasks with a total of 17,000 instances. We curated four subsets consisting of open-ended VQA tasks from LLaVA-NeXT-Interleave-Bench: Spot-the-Diff, Image-Edit, Visual-Story-Telling, and Multi-View.

$\bullet$ \textbf{Single-Image benchmarks}: The Flickr30k dataset is a widely used benchmark in the field of image captioning and visual understanding. It consists of 31,783 images collected from the Flickr platform, each paired with five human-annotated captions. The COCO2017 Caption subset contains more than 45,000 images, each annotated with five captions written by human annotators, describing the visual content of the images in detail, including objects, their attributes, and the relationships between them. DetailCaps4870 provides more fine-grained and specific image content descriptions than standard captioning datasets, which is more useful for efficiency analysis. 

\subsubsection{Baselines}\label{asec:eval-setting-baseline}

We select FastV and VTW as our baselines in our experiments. Notably, FastV offers two versions of implementation: one that supports KV cache and one that does not. Since the non-KV-cache implementation introduces substantial computational overhead, we use the version that supports KV cache to ensure a fair comparison. For both of the baselines, we refer to the official open source code \footnote{\url{https://github.com/pkunlp-icler/FastV}} \footnote{\url{https://github.com/lzhxmu/VTW}} and implement them on the models we evaluate.

\subsubsection{Models}\label{asec:eval-setting-impl}

For Qwen2-VL, we set \texttt{max\_pixels} to $1280 \times 28 \times 28$ and \texttt{min\_pixels} to $256 \times 28 \times 28$ according to the official recommendation. The Mantis model that we choose is Mantis-8B-SigLIP-LLaMA3. For LLaVA-OneVision and Mantis, we use the official original versions \footnote{\url{https://huggingface.co/lmms-lab/llava-onevision-qwen2-7b-ov}} \footnote{\url{https://huggingface.co/TIGER-Lab/Mantis-8B-siglip-llama3}}, while using the versions provided by the transformers library \cite{wolf-etal-2020-transformers} for all other models.

\subsection{Case Study}

To showcase the effectiveness of our proposed method, we present a series of case studies in the form of multimodal chatbots, as shown in Figure \ref{fig:case-studies}.

\begin{figure*}[ht]
	\centering
	\subfloat[]{
		\includegraphics[width=0.48\textwidth]{figs/appendix-case3.png}}
        \subfloat[]{
		\includegraphics[width=0.48\textwidth]{figs/appendix-case4.png}}
        \\ \quad \\ \quad \\
	\subfloat[]{
		\includegraphics[width=0.48\textwidth]{figs/appendix-case1.png}}
        \subfloat[]{
		\includegraphics[width=0.48\textwidth]{figs/appendix-case2.png}}
 %        \\
	% \subfloat[]{
	% 	\includegraphics[width=0.9\textwidth]{figs/appendix-case5.png}}
	\caption{\textbf{Multimodal Chatbots with different pruning methods.}}
		\label{fig:case-studies}
\end{figure*}


% \nobalance 

\end{document}
