\subsection{Proposed taxonomy for Neuromorphic Architectures}
\label{sec:taxonomy}
Schuman et al.\cite{schuman_opportunities_2022} suggest that for a system to be called \textit{neuromorphic}, it must fulfill the following requirements: \textbf{(i)} it must compute using neurons and synapses (as opposed to instructions of a classical computer), \textbf{(ii)} it must communicate with spikes (as opposed to multi-bit values in a classical system), \textbf{(iii)}
it is massively parallel (that is, it can compute all neurons/synapses fully in parallel), \textbf{(iv)} it collocates computation and memory (unlike von Neumann-based classical system), \textbf{(v)} it operates in an event-driven manner (the network state can be updated asynchronously, i.e., the neurons are not updated in sequence). Table~\ref{tab:schuman} contrasts these properties with that of a classical (von Neurmann) computer system.
\begin{wrapfigure}[12]{h}{0.5\textwidth}
  \centering
    \vspace{-10pt}
    \includegraphics[width=\linewidth]{Figs/taxonomy_venn.png}
    \caption{Proposed taxonomy - classes.}
    \label{fig:taxonomy_venn}
\end{wrapfigure}
We propose a taxonomy for hardware neuromorphic architectures (NMAs) that builds upon that of Schuman et al., but relaxes the constraints. We argue that while a neuromorphic system must be programmed using neurons and synapses and communicate with spikes \textbf{(i--ii)}, the remaining requirements \textbf{(iii-v)} are, in fact, architectural design decisions that define traits and properties of the implementation of a hardware neuromorphic architecture (NMA).
To summarize, we propose that a NMA \textbf{must} perform computations using neurons and synapses and communicate with spikes, but it \textbf{may} also be massively parallel, be asynchronous, and/or co-locate computations and memory. Ultimately, this results in a taxonomy consisting of eight classes, shown in Figure~\ref{fig:taxonomy_venn}. The latter three optional characteristics will be referred to as \textbf{Traits} for the remainder of this paper.
It is important to mention that the Classes are not synonymous to grades, i.e., there are no \textit{better} or \textit{worse} Classes. Instead, the classification is intended to compare NMAs and implementations similar in terms of hardware and relate them to each other, like Flynn's taxonomy~\cite{skillicorn1988taxonomy}. Every Class comes with its own advantages and disadvantages.

\begin{table}
  \caption{Comparison between von Neumann architectures and NMAs according to Schuman et al. Adapted from the Figure in \cite{schuman_opportunities_2022}.}
  \label{tab:schuman}
  \begin{tabular}{c|cc}
    \toprule
    Property&von Neumann&Neuromorphic\\
    \midrule
    Operation & Sequential processing & Massively parallel processing \\
    Organization & Separated computation and memory & Collocated computation and memory \\
    Programming & Code as binary instructions & Spiking Neural Network \\
    Communication & Binary data & Spikes \\
    Network update & Synchronous (clock-driven) & Asynchronous (event-driven)  \\
  \bottomrule
\end{tabular}
\end{table}