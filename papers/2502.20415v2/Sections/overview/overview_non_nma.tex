\subsection{Non-NMA structures related to SNN research}
\label{sec:overview_non_nma}

In this section, we present systems that could not be classified as NMAs, presented interesting optimizations of NMAs sub-units or were single neuron implementations. Interested readers are referred to the referenced works.

There is a number of structures that we were not able to fairly compare with the implementations listed in Sections from \ref{sec:overview_class0} to \ref{sec:overview_class7}, due to the lack of distinctive features of NMAs, as we defined them in Section \ref{sec:taxonomy}. There are many such implementations reported in the literature, and we are listing two of them as examples. \parhl{Palumbo et al. (2017) \cite{palumbo_feasibility_2017}} presented an architecture for simulating Swarm Intelligence (SI) using SNNs,it employed a soft-core processor and a designated coprocessor to provide a systolic array for realizing partial calculations for neuronal dynamics. \parhl{Shahsavari et al. (2021)} presented \textbf{Partially Ordered Event-Triggered System (POETS)} architecture - a general-purpose parallel architecture of a number of RISC-V cores arranged in a NoC on multiple FPGAs.

In many papers, the authors were focusing on optimizing parts of NMAs. As an interested reader may find information on optimizing specific their NMA designs extremely useful, we list such papers, grouped by the sub-units being optimized: spike encoding - \cite{gerlinghoff_resource-efficient_2022}, exponential function unit - \cite{kim_hardware-efficient_2021}, STDP-based learning - \cite{gomar_digital_2018, nouri_digital_2018, pedroni_forward_2016, belhadj_fpga-based_2008}, ion channel dynamics unit - \cite{jokar_digital_2017, mak_field_2005}, soma unit - \cite{pourhaj_fpga_2010}, alternative to SNN-based processing (P systems) - \cite{pena_efficient_2019}.

While surveying the neuromorphic architectures, we realized that many researchers were focusing on implementing single neurons on FPGAs with hardware optimizations, accuracy of simulation and biological plausibility of the results in mind. Examples of such works, grouped by the neuron model the authors aimed at implementing, are as follows: HH - \cite{levi_digital_2018, bonabi_fpga_2014}, LIF - \cite{grassia_digital_2016}, AdEx - \cite{heidarpour_cordic_2016, gomar_digital_2014}, HR - \cite{heidarpur_digital_2017, hayati_digital_2016, kazemi_digital_2014}, ML - \cite{hayati_digital_2015}, IZH - \cite{pu_low-cost_2021, haghiri_multiplierless_2018}, WIL - \cite{imani_digital_2018}, Pinsky-Rinzel (PR) - \cite{rahimian_digital_2018},  multiple - \cite{soleimani_efficient_2018}, other - \cite{zhang_neural_2022}.