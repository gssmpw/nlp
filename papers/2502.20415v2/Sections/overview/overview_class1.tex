\subsection{Class 1 (Traits: fully parallel)}
\label{sec:overview_class1}
\begin{wrapfigure}[13]{h}{0.5\textwidth} % Class 1 is too short to hold this reliably
  \centering
    \vspace{-15pt}
    \includegraphics[width=0.9\linewidth]{Figs/class1.png}
    \caption{Simplified diagram of a Class 1 NMA.}
    \label{fig:class1}
\end{wrapfigure}
Class 1 architectures support only fully parallel operation Trait. Those systems include number of PEs equal to the number of simulated neurons, allowing for higher speed of operation than Class 0. However, the number of neurons is bound by the amount of available logic resources on an FPGA.

\mypar{Early implementations}
\parhl{Maya et al. (2000) \cite{maya_compact_2000}} presented an architecture that comprised multiple PEs that computed a pulsed neuron - three in total - to solve the XOR problem. The weights were obtained through BP in MATLAB, and the structure was clocked at 90 MHz on Virtex VX50-6. 
\parhl{Xicotencatl et al. (2003) \cite{xicotencatl_fpga_2003}} presented a network of 1120 \textit{pulsed neurons} with 1000 synapses. The architecture was implemented on Virtex-V2000 and was clocked at 35.6 MHz. The topology of the implemented network was a 2D array of neurons.

\mypar{Early modern implementations}
\parhl{Moctezuma et al. (2013) \cite{moctezuma_numerically_2013}} presented an architecture for simulating ion-channel dynamics and membrane voltage changes of neurons defined by HH and Traub neuron models. The models utilized floating-point arithmetic, and exponential functions were implemented as LUTs. The system was using a MicroBlaze processor for configuration and supervision, as well as communication between subunits. The entire system operated at 100 MHz and allowed the simulation of six different configurations of soma, dendrite and synapse types from the aforementioned neuron models.