\section{Conclusion}
The FPGA-based neuromorphic systems can be implemented in various ways, with different advantages and disadvantages, which is reflected by the Classes of the Taxonomy we provided. Representatives of those Classes fit diverse sets of requirements and allow for achieving diverse goals - from classification tasks to neurocientific simulations. We expect that the popularity of the digital NMAs will continue to grow and that even more brain-like systems will be presented, as new discoveries related to human brain are made.  Moreover, it is apparent, due to the stark differences in design choices between the Classes, that creating a unified Design Space Exploration framework that would allow for convenient and automatic search of optimal architectural Class for given requirements (e.g., number of neurons, synapses, neuron models and available FPGA platform) is troublesome, but we believe it is necessary in order to simplify the NMA design process.