 % This must be in the first 5 lines to tell arXiv to use pdfLaTeX, which is strongly recommended.
\pdfoutput=1
% In particular, the hyperref package requires pdfLaTeX in order to break URLs across lines.

\documentclass[11pt]{article}
\usepackage{natbib}
% Remove the "review" option to generate the final version.
\usepackage[dvipsnames]{xcolor}
\usepackage{tcolorbox}
\usepackage{authblk}
\usepackage{ACL2025}
\usepackage{hyperref}
% Standard package includes
\usepackage{times}
\usepackage{latexsym}
\usepackage{booktabs}
% For proper rendering and hyphenation of words containing Latin characters (including in bib files)
\usepackage[T1]{fontenc}
% For Vietnamese characters
% \usepackage[T5]{fontenc}
% See https://www.latex-project.org/help/documentation/encguide.pdf for other character sets

% This assumes your files are encoded as UTF8
\usepackage[utf8]{inputenc}

% This is not strictly necessary and may be commented out.
% However, it will improve the layout of the manuscript,
% and will typically save some space.
\usepackage{microtype}

% This is also not strictly necessary and may be commented out.
% However, it will improve the aesthetics of text in
% the typewriter font.
\usepackage{inconsolata}
\usepackage{array} % required for text wrapping in tables


% If the title and author information does not fit in the area allocated, uncomment the following
%
%\setlength\titlebox{4.2cm}
%
% and set <dim> to something 5cm or larger.

\usepackage{graphicx}
\usepackage{todonotes}
\usepackage{amsmath}
\usepackage{multirow}
\usepackage{multicol}
\usepackage{caption}
\usepackage{subcaption}
\usepackage{float}

\usepackage[labelfont=bf]{caption}
\captionsetup{labelfont=bf}

\usepackage{adjustbox}


\title{\textit{ReVision:} A Dataset and Baseline VLM for Privacy-Preserving Task-Oriented Visual Instruction Rewriting}  
% \author{Authors - TBD}
% \author {
%     % Authors
%    \textbf{ Abhijit Mishra}, \textbf{Richard  Noh},\textbf{ Hsiang Fu},, \textbf{Mingda Li}$\dagger$, \textbf{Minji Kim }
% }
% \affiliations {
%     % Affiliations
%     School of Information, University of Texas at Austin \\
%     \{abhijitmishra, rjnoh, seanfu, minji.kim\}@utexas.edu \\
%     \textsuperscript Department of Statistics and Data Science, Yale University \\
%     mingdali@utexas.edu
% }

% Author information can be set in various styles:
% For several authors from the same institution:
% \author{Author 1 \and ... \and Author n \\
%         Address line \\ ... \\ Address line}
% if the names do not fit well on one line use
%         Author 1 \\ {\bf Author 2} \\ ... \\ {\bf Author n} \\
% For authors from different institutions:
% \author{Author 1 \\ Address line \\  ... \\ Address line
%         \And  ... \And
%         Author n \\ Address line \\ ... \\ Address line}
% To start a separate ``row'' of authors use \AND, as in
% \author{Author 1 \\ Address line \\  ... \\ Address line
%         \AND
%         Author 2 \\ Address line \\ ... \\ Address line \And
%         Author 3 \\ Address line \\ ... \\ Address line}

% For several authors from the same institution:
% if the names do not fit well on one line use
%         Author 1 \\ {\bf Author 2} \\ ... \\ {\bf Author n} \\


% \author{First Author \\
%   Affiliation / Address line 1 \\
%   Affiliation / Address line 2 \\
%   Affiliation / Address line 3 \\
%   \texttt{email@domain} \\\And
%   Second Author \\
%   Affiliation / Address line 1 \\
%   Affiliation / Address line 2 \\
%   Affiliation / Address line 3 \\
%   \texttt{email@domain} \\}
  
\author {
    % Authors
    \textbf{ Abhijit Mishra$^*$}, \textbf{Richard  Noh$^*$},\textbf{ Hsiang Fu$^*$}, \textbf{Mingda Li}$\dagger$, \textbf{Minji Kim }\\
    School of Information, University of Texas at Austin \\
    \{abhijitmishra, rjnoh, seanfu, minji.kim\}@utexas.edu \\
    \textsuperscript{$\dagger$}Department of Statistics and Data Science, Yale University \\
     mingda.li@yale.edu
}

\begin{document}
\maketitle

\def\thefootnote{*}\footnotetext{Equal Contribution}\def\thefootnote{\arabic{footnote}}

\begin{abstract}
Efficient and privacy-preserving multimodal interaction is essential as AR, VR, and modern smartphones with powerful cameras become primary interfaces for human-computer communication. Existing powerful large vision-language models (VLMs) enabling multimodal interaction often rely on cloud-based processing, raising significant concerns about (1) visual privacy by transmitting sensitive vision data to servers, and (2) their limited real-time, on-device usability. This paper explores \textit{Visual Instruction Rewriting}, a novel approach that transforms multimodal instructions into text-only commands, allowing seamless integration of lightweight on-device instruction rewriter VLMs (250M parameters) with existing conversational AI systems, enhancing vision data privacy. To achieve this, we present a dataset of over 39,000 examples across 14 domains and develop a compact VLM, pretrained on image captioning datasets and fine-tuned for instruction rewriting. Experimental results, evaluated through NLG metrics such as BLEU, METEOR, and ROUGE, along with semantic parsing analysis, demonstrate that even a quantized version of the model (<500MB storage footprint) can achieve effective instruction rewriting, thus enabling privacy-focused, multimodal AI applications.
\end{abstract}

\section{Introduction}\label{sec:intro}

In computational finance, Monte Carlo simulations are used extensively to estimate the expected value of financial payoffs based on the solution of stochastic differential equations (SDEs) which model the evolution of stock prices, interest rates, exchange rates and other quantities \cite{glasserman04}.  Monte Carlo methods are very general and flexible, but for high accuracy it requires generating a large number of costly SDE path approximations, which has motivated research into a number of variance reduction or, equivalently, cost reduction techniques. One such method is
Multilevel Monte Carlo (MLMC), which was proposed in \cite{GILES2008} and was adapted for various applications that are summarised in \cite{Giles_overview17} and successfully combined with other methods such as quasi-Monte Carlo methods. The main idea of MLMC is to approximate the payoff using different time stepping resolutions when numerically solving the underlying SDE and to generate an optimal number of samples on each level, such that the overall computational cost is minimised subject to the desired bound on the variance. %, such that the total computational cost is minimised. 
The computational savings come from the fact that most samples are computed on the coarser levels and hence are less expensive while only a few samples from the finest levels are required \cite{GILES2008}.


Among the directions in which the computational cost 
of MLMC methods could further be reduced, an important avenue is the use of lower precision calculations, especially for the first Monte Carlo levels where the targeted accuracy is relatively low. 
 An overview of the research on mixed precision for the standard Monte Carlo (MC) framework is provided in \cite{ChowMixedPrecisionStandardMC} but only a few references study the potential of low precision computation in the MLMC framework \cite{Rounding_error_oliver}. To the best of our knowledge, the only MLMC framework with customised precision in the literature is \cite{brugger2014mixed}, but they use a uniform precision for all operations on each Monte Carlo level instead of optimising 
 the precision of each intermediary variable to reduce as much as possible the cost of path generation.
 
An important motivation for an MLMC framework with variable precision would be performing the low precision computations on reconfigurable hardware devices such as Field Programmable Gate Arrays (FPGAs). FPGAs contain customizable logic blocks and connectors that make it easy to adapt the digital circuit architecture for a specific application, leading to a highly parallel and optimised implementation. Therefore they are successfully exploited in applications that require high speed and have high computational workload, such as signal processing \cite{woods2008fpga}, and real time applications like high frequency trading \cite{HFT1,HFT2}. That is why a number of previous works in hardware architecture design implemented the MLMC algorithm to price financial options using FPGAs as accelerators, which resulted in improved speed and power efficiency compared to full CPU architectures \cite{Schryver2013AMM}. The paper \cite{lindsey2016domain} also proposed 
a Domain Specific Language to automate the configuration of FPGAs for this specific application. However, only \cite{brugger2014mixed} proposed a heuristic to reduce the precision in calculations.

In addition, all aforementioned works considered that the random number generation (RNG) is performed in single or double precision. Yet in most cases an important portion of the workload in the overall MLMC simulation comes from the RNG and in \cite{brugger2014mixed} this limited the total computational savings.
To reduce the cost of MLMC simulations in particular those based on the Geometric Brownian Motion (GBM), \cite{approximateICDF_Oliver, NestedOliver} have proposed to use approximate random numbers that are generated by applying an approximation of the inverse CDF to uniform random numbers. In \cite{NestedOliver}, the authors proposed a way to integrate these lower precision random variables into a \textit{nested} MLMC framework and completed a numerical analysis to bound the resulting error at each MC level by a product of the time step and the error in the random number approximation. The same authors show in \cite{approximateICDF_Oliver} that using approximate random variables reduces the cost of path generation by a factor 7.


In this paper we propose a nested MLMC framework that combines the use of approximate random normal variables and lower precision calculations to reduce the computational cost of MLMC even further than \cite{brugger2014mixed,NestedOliver}. We illustrate the efficiency of our framework in Matlab, after making several assumptions on the cost of operations and size of the errors that we carefully justify. We focus on the case of GBM and use the approximate RNG methods presented in \cite{approximateICDF_Oliver} as well as a new slightly modified method that combines CDF inversion and the central limit theorem. To choose the precision of the variables in the low precision path generation, we introduce a novel method to optimise the bit-widths. This optimisation is performed before the main path generation loop is executed and is based on a linear model of the payoff error  
due to rounding when computing in low precision. The error model relies on algorithmic differentiation in a similar manner to \cite{unifying-bwoptim,bitwidth-AD,ADAPT}. The bit-width optimisation procedure can be performed off-line, so this stage can be excluded from the on-line time complexity of our framework. The user specified desired accuracy is then enforced by calculating on-line the number of samples that need to be generated.

In terms of hardware design, we suggest implementing the low precision path generation on FPGAs and the full-precision ones on a CPU or GPU. 
The FPGA offers enough flexibility to define a separate bit-width for every variable in the low precision path generation, and can be reconfigured periodically to update the bit-widths when the market parameters have changed considerably. 


The paper is organized as follows : \Cref{sec:MLMC} introduces MLMC and nested MLMC to make clear the estimator that is implemented in our framework. Then in \Cref{sec:RNG} we detail the methods that could be used to obtain approximate random normally distributed numbers very cheaply for the low precision path generation. In \Cref{sec:error_model} and \Cref{sec:costModel} we propose an error model and a cost model (resp.) that we then use to formulate the optimisation problem that is solved to obtain the optimal bit-widths of fixed point variables in \Cref{sec:optimisation}. Finally we summarise our results and future directions in \Cref{sec:conclusion}.



\section{Related Work}
\label{sec:related work}
% In this section, we review the existing literature on point cloud denoising and unsupervised image denoising.
%-------------------------------------------------------------------------
\subsection{Point cloud denoising}

    \subsubsection{Traditional methods}
Traditional approaches to point cloud denoising include statistical methods \cite{computingpointset2003,definingpointset2004,wlop2009HH}, filtering techniques\cite{pointsetsurfaces2001,Robustmoving2005, zaman2017density}, and optimization-based methods \cite{l1sparse2010,clop2014PR,digne2017bilateral,multi-projection2018duan,hu2020featuregraph} . These techniques often rely on handcrafted features and heuristics to distinguish signal from noise. For example, statistical methods may use distribution models to identify and remove outliers. Filtering methods, such as mean or median filters, operate under the assumption that noise is statistically different from the signal. Optimization-based methods formulate denoising as an energy minimization problem, where regularization terms constrain the solution to ensure certain smoothness cirterion or adherence to prior knowledge.

%-------------------------------------------------------------------------
    \subsubsection{Supervised learning based methods}
In recent years, several deep learning-based methods \cite{rakotosaona2020PCN,hermosilla2019TotalDenoising,luo2020DMR,luo_score-based_2021} have been proposed for point cloud denoising. NPD \cite{NPD2019} is the first learning-based point cloud denoising method that directly processes noisy data without requiring noise characteristics or neighboring point definitions. This approach combines local and global information by projecting noisy points onto estimated reference planes, effectively removing noise and enhancing robustness against variations in noise intensity and curvature. PointCleanNet\cite{rakotosaona2020PCN} first removes outlier points and then combines them with residual connectivity to predict the inverse displacement \cite{Guerrero2017PCPNetLL}, and iteratively shifts noisy points to remove noise. Pistilli \etal proposed GPDNet \cite{gpdnet2020}, which is a graph convolutional network to improve denoising robustness at high noise levels. Luo \etal also proposed  DMRDenoise \cite{luo2020DMR}, which filter
points by first downsampling the noisy inputs and reconstructing the local subsurface to perform point upsampling. However, this resampling method is difficult to maintain a good local shape. ScoreDenoise \cite{luo_score-based_2021} is proposed to tackle the aforementioned issues by iteratively updating the point position in implicit gradient fields learned by neural networks. For inference, they follows an iterative procedure with a decaying step size, which stabilizes point movement and prevents over-correction, allowing points to converge gradually toward the underlying geometry. The authors of \cite{de_Silva_Edirimuni_2023_CVPR} proposed IterativePFN - an iterative method that use a novel loss function that utilizes an adaptive ground truth target at each iteration to capture the relationship between intermediate filtering results during training. Zheng \etal proposed a end-to-end network for joint normal filtering-based point cloud denoising \cite{10173632}. They introduce an auxiliary normal filtering task to enhance the network's ability to remove noise while preserving geometric features more accurately.

Supervised methods can achieve impressive results, but are limited by the availability and quality of the training data, as they typically require paired noisy and clean point clouds to train the neural network. The absence of clean data in real-world scenario pose a significant challenge on applicability of these algorithms.

%-------------------------------------------------------------------------
    \subsubsection{Unsupervised learning methods}
Unsupervised learning-based methods for point cloud denoising do not require ground-truth clean data. Instead, these methods leverage the inherent structure or distribution of the point cloud to guide the denoising process. Unsupervised methods show promise in scenarios where clean data is absent or hard to obtain.

Hermosilla \etal first introduced Total Denoising (TotalDn) \cite{hermosilla2019TotalDenoising} as an unsupervised learning approach for point cloud denoising, relying solely on noisy data without requiring clean ground truth. TotalDn approximates the underlying surfaces by regressing points from the distribution of unstructured total noise, utilizing a spatial prior term to refine the estimation of geometry. 

In DMRDenoise \cite{luo2020DMR}, an unsupervised version is proposed which utilizes a loss function that identify local neighborhoods using a probabilistic Gaussian mask on the k-nearest neighbors, which selectively retains points likely to represent the underlying surface. By leveraging an Earth Mover's Distance (EMD) assignment, it achieves a one-to-one correspondence between input and predicted points, aligning them to reduce noise within local neighborhoods.
This approach enhances robustness to noise and adapts well to varied surface geometries. However, the probabilistic masking and EMD calculation add computational complexity, which can slow down inference in dense or noisy point clouds. 

ScoreDenoise \cite{luo_score-based_2021} also introduced an unsupervised version that employs ensemble score function and an adaptive neighborhood-covering loss for model training.  
Score-u is probably the most relevant work to our method. However, the defined score in \cite{luo_score-based_2021} is only an displacement-alike version of the score function and there is no explicit formula relating the estimated score to the final denoising result. Also, the iterative process is computationally expensive, and can suffer from fluctuation, leading to perturbed and unstable solution.

Most recently, Noise4Denoise \cite{noise4Wang2024} method is proposed that use an additional doubly-noisy point cloud to learn noise displacement by comparing the two noise levels. This approach effectively leverages synthetic noise for training, allowing the model to estimate residuals without relying on clean data.
However, in practical applications, noise parameters are often unknown and variable, making it challenging to replicate the exact conditions assumed during training. This reliance on predefined noise characteristics can limit the model's applicability to real-world scenarios where noise distributions may differ significantly from synthetic settings. 
%-------------------------------------------------------------------------
\subsection{Unsupervised image denoising}
Recently unsupervised image denoising has made significant progress. Non-Bayesian methods include PURE \cite{luisier2010image}, SURE \cite{SURE2018} \textit{etc.}, which are based on various unbiased risk estimator under certain noise distribution. Other methods explore self-similarity in natural images \cite{xu2015patch, doi:10.1137/23M1614456} or exploits the statistical properties of noise to achieve denoising effect \cite{gravel2004method}.  

Noise2Noise \cite{2018Noise2NoiseLI} is a pioneering method that does not require clean data, but it requires multiple noisy versions of the same image for training. To address this limitation, methods such as Noise2Void \cite{2018Noise2VoidL}, Noise2Self \cite{2019Noise2SelfBD}, \textit{etc.}, have been developed, which use only a single noisy image. These methods are particularly important for practical applications where obtaining clean images or multiple noisy realizations of the same image is difficult or impossible. Neighbor2Neighbor \cite{huang2021neighbor2neighbor} proposed a two-step method with a a random neighbor sub-sampler that generates training  pairs and a denosing network. Kim \etal proposed Noise2Score\cite{kim_noise2score_2021}, a novel Bayesian framework for self-supervised image denoising without clean data. The core of Noise2Score is the usage of Tweedie's formula, which provides an explicit representation of the denoised image through a score function. Combined with score function estimation, Noise2Score can be applied to image denoising with any exponential family noise. Kim \etal also proposed the Noise Distribution Adaptive Self-Supervised Image Denoising method \cite{kim_noise_2022}, which further extends the application of Noise2Score by combining the Tweedie distribution with score matching. This enables adaptive handling of various noise distributions and dynamically adjusts the denoising process by estimating noise parameters. On the other hand, Xie \etal \cite{scoreXie2024} broadened the denoising scope of Noise2Score by allowing it to handle complex noise models, such as multiplicative and structurally correlated noise, demonstrating broad applicability to diverse noise models.

These development of unsupervised image denoising method motivate us to explore similar ideas in 3D point cloud denoising.




\begin{figure*}[t]
  \centering
  \includegraphics[width=\textwidth, height=10cm]{images/mindmap2.pdf} 
  \caption{Mindmap showing Data Collection and Rewrite Desiderata}
  \label{fig:mindmap}
\end{figure*}
% \begin{figure*}[t]
%   \centering
%   \includegraphics[width=\textwidth]{images/process.pdf} 
%   \caption{Dataset Creation Pipeline}
%   \label{fig:process}
% \end{figure*}
\section{Constructing a Dataset for Visual Instruction Rewriting}
\label{sec:datasets}

Task-oriented conversational AI systems rely on a semantic parser to interpret user intent and extract structured arguments \cite{louvan2020recent,aghajanyan2020conversational}. For example, when a user says,\textit{ "Add the team meeting to my calendar for Friday at 3 PM"}, the system must parse the intent (\textit{CreateCalendarEvent}) and extract arguments such as the \textit{EventTitle} (``team meeting''), \textit{EventDate} (``Friday''), and \textit{EventTime} (``3 PM'') to schedule the event correctly. Unlike purely text-based interactions, multimodal instructions, particularly those directed at conversational AI assistants on AR/VR devices (\textit{e.g.,} Apple's Siri for Apple Vision Pro), introduce additional challenges such as ellipsis and coreference resolution. For instance, a user may look at a book cover and ask, \textit{“Who wrote this?”} or point at a product in an AR interface and say, \textit{“How much does this cost?”} Traditional text-based semantic parsers struggle with such instructions since critical visual context is missing. Thus, to bridge the gap between multimodal input and existing conversational AI stacks, we introduce a dataset specifically designed for \textit{rewriting multimodal instructions} into structured text that can be processed by standard text-based semantic parsers. Figure \ref{fig:mindmap} illustrates a representation of the dataset collection requirement, highlighting the transformation of multimodal inputs into text-based rewrites.

To construct our dataset, we first define an ontology of intents and arguments, as existing ontologies in conversational AI and semantic parsing are often proprietary and unavailable for research use. We take inspiration from \newcite{goel2023presto} for ontology and extend it to accommodate multimodal task-oriented interactions. Figure \ref{fig:intent_argument_box} (ref. Appendix) presents an overview of the intents and arguments in our ontology. Next, we curate a diverse set of images covering various real-world multimodal interaction scenarios, including book covers, product packaging, paintings, mobile screenshots, flyers, signboards, and landmarks. These images are sourced from publicly available academic datasets, such as OCR-VQA\footnote{\url{https://ocr-vqa.github.io/}}, CD and book cover datasets, Stanford mobile image datasets\footnote{\url{http://web.cs.wpi.edu/~claypool/mmsys-dataset/2011/stanford/}}, flyer OCR datasets\footnote{\url{https://github.com/Skeletonboi/ocr-nlp-flyer.git}}, signboard classification datasets\footnote{\url{https://github.com/madrugado/signboard-classification-dataset}}, Google Landmarks\footnote{\url{https://github.com/cvdfoundation/google-landmark}}, and Products-10K\footnote{\url{https://products-10k.github.io/}}.

\begin{table}[t]
    \centering
    \scriptsize
    \label{tab:dataset_statistics}
    \begin{tabular}{llccc}
        \toprule
        \textbf{Category} & \textbf{Total} & \textbf{Train} & \textbf{Test} \\
        \midrule
        Book              & 485 / 500                               & 386 / 399                               & 101 / 101                               \\
        Business Card     & 26 / 960                                & 26 / 772                                & 26 / 188                                \\
        CD               & 27 / 1,020                              & 27 / 835                                & 27 / 185                                \\
        Flyer & 159 / 5,940                             & 159 / 4,742                             & 159 / 1,198                             \\
        Landmark         & 511 / 19,274                            & 511 / 15,420                            & 511 / 3,854                             \\
        Painting & 27 / 980                                & 27 / 774                                & 27 / 206                                \\
        Product          & 499 / 10,349                            & 499 / 8,276                             & 492 / 2,073                             \\
        \midrule
        \textbf{Total}   & \textbf{1,734 / 39,023}                 & \textbf{1,635 / 31,218}                 & \textbf{1,343 / 7,805}                  \\
        \bottomrule
    \end{tabular}
    \caption{Number of Images/Instructions per Category}
    \label{tab:sources}
\end{table}
\begin{table}[t]
    \centering
    \footnotesize
    \begin{tabular}{l  c}
        \toprule
         \textbf{Annotator}& \textbf{Percentage of Correct Captions}\\ 
         \midrule
         Annotator 1	& 90.62\%\\ 
         Annotator 2	& 87.23\%\\
         Annotator 3	& 86.35\%\\
         \midrule
         \textbf{At least two }& \textbf{92.18}\%\\
         \midrule
         \textit{All three }& \textit{74.63}\% \\
         \bottomrule
    \end{tabular}
    \caption{GPT-4 Instruction Rewriting Validation Results from Amazon Mechanical Turk }
    \label{tab:annotator_data}
\end{table}
\begin{figure}[t]
\includegraphics[width=\columnwidth]{images/intent.png}
  \caption{Dataset Distributions By Intent}
  \label{fig:intent}
\end{figure}
Upon identifying and verifying the images, we employ the GPT-4 model from OpenAI \cite{achiam2023gpt} to systematically generate and refine multimodal instructions into rewritten text-based instructions. The process begins with a bootstrap phase, where GPT-4 is prompted to generate 20 direct questions per image by explicitly referencing visible objects or textual elements while adhering to the intent list defined in Figure \ref{fig:intent_argument_box}. A second prompting phase then validates the generated questions against the corresponding image, filtering out ambiguous or irrelevant instructions to ensure alignment with the visual context. 

In the rewriting phase, GPT-4 is tasked with paraphrasing the validated instructions, ensuring that the transformed questions are fully self-contained and interpretable without requiring the image. This transformation is crucial for enabling multimodal conversational AI systems to process instructions using purely text-based stacks. Finally, a verification phase prompts the model to assess the rewritten questions in relation to both the original instruction and the image, ensuring semantic fidelity and eliminating inconsistencies. This multi-stage prompting strategy resulted in a dataset of 39,023 original-rewritten instruction pairs, derived from 1,734 images, with an 80\%-20\% train-test split. Table \ref{tab:sources} provides a breakdown of image sources.

While automated validation ensures consistency across different stages, human evaluation remains critical for verifying the dataset’s reliability. To this end, we conducted an annotation task via Amazon Mechanical Turk (AMT) to validate rewritten instructions within the test set for indirect image-based instructions. Each annotation task followed a structured validation guideline, where annotators reviewed an image, its original multimodal instruction, and the rewritten text-only instruction, determining whether the reformulation preserved the intent and meaning of the original instruction. Annotators were instructed to select "Accept" if the rewritten instruction was correct or "Reject" if it failed to capture the original meaning. Annotators are incentivized appropriately for this binary grading task. Agreement analysis, as shown in Table \ref{tab:annotator_data}, indicates that in 92.2\% of cases, at least two annotators agreed on "Accept," while 74.6\% of instructions achieved full consensus across all three annotators. Despite a Fleiss' Kappa score of 0.278—suggesting fair inter-annotator agreement—the high rate of majority consensus supports the dataset’s reliability for real-world use. Given these results, we publicly release the full dataset along with raw AMT responses, enabling further analysis, filtering, and refinements by the research community.

Figure \ref{fig:intent} presents the distribution of intents in our dataset, categorized into training and test splits. The distribution reflects practical usage patterns in real-world multimodal conversational AI systems, with a higher occurrence of general QA and web search, alongside diverse task-oriented intents such as reminders, messaging, and navigation, ensuring coverage of frequent user interactions.



% In this study, we utilize a comprehensive multimodal dataset curated from various sources to facilitate research in multimodal instruction rewriting using compact models. Table~\ref{tab:dataset_statistics} provides an overview of the dataset's composition, detailing the number of images and corresponding instructions sourced from different domains. This diverse dataset is designed to challenge models in interpreting and rewriting instructions based on both visual and textual information embedded within images.

% The dataset is organized into a single TSV file, \texttt{all\_data.tsv}, which consolidates all the data for streamlined processing and analysis.

% The dataset is publicly accessible and can be downloaded from our Hugging Face repository:
% \url{https://huggingface.co/datasets/utischoolnlp/multimodal_instruction_rewrites}.

% \begin{table}[h]
%     \centering
%     \caption{Dataset Statistics}
%     \label{tab:dataset_statistics}
%     \resizebox{0.5\textwidth}{!}{%
%         \begin{tabular}{|l|l|c|c|}
%             \hline
%             \textbf{Data Source} & \textbf{Type} & \textbf{Number of Images} & \textbf{Number of instructions} \\ \hline
%             \href{https://github.com/gulvarol/grocerydataset}{Grocery Store Dataset} & Grocery Dataset & 287 & 5,945 \\ \hline
%             \href{https://amazon-berkeley-objects.s3.amazonaws.com/index.html}{Amazon Berkeley Objects} & Amazon Dataset & 187 & 3,890 \\ \hline
%             \href{https://products-10k.github.io/}{Products-10K} & E-commerce Dataset & 23 & 472 \\ \hline
%             \href{https://www.kaggle.com/datasets/vikashrajluhaniwal/fashion-images}{Fashion Images} & Fashion Clothing Dataset & 2 & 42 \\ \hline
%             \textbf{Total} & & \textbf{499} & \textbf{10,349} \\ \hline
%         \end{tabular}
%     }
% \end{table}


% \subsection*{Additional Dataset Statistics}

% To provide a deeper understanding of the dataset's characteristics, we present the following statistics derived from \texttt{all\_data.tsv}:

% \begin{itemize}
%     \item \textbf{Prompt Length}:
%     \begin{itemize}
%         \item \textbf{Average Prompt Length}: 80.99 tokens
%         \item \textbf{Maximum Prompt Length}: 160 tokens
%         \item \textbf{Minimum Prompt Length}: 28 tokens
%     \end{itemize}
    
%     \item \textbf{Rewritten Question Length}:
%     \begin{itemize}
%         \item \textbf{Average Rewritten Question Length}: 56.94 tokens
%         \item \textbf{Maximum Rewritten Question Length}: 160 tokens
%         \item \textbf{Minimum Rewritten Question Length}: 28 tokens
%     \end{itemize}
% \end{itemize}

% These statistics highlight the complexity and variability of the prompts and their corresponding rewritten questions, providing a robust foundation for training and evaluating multimodal instruction rewriting models.

% \subsection*{Dataset Composition}

% The dataset is consolidated into a single TSV file, \texttt{all\_data.tsv}, which includes all image-instruction pairs. This unified format simplifies data handling and ensures consistency across training and evaluation phases. The structure of \texttt{all\_data.tsv} is as follows:


% \begin{itemize}
%     \item \textbf{Columns}:
%     \begin{itemize}
%         \item \texttt{Image\_ID}: Unique identifier for each image.
%         \item \texttt{Image\_URL}: Direct link to the image file.
%         \item \texttt{Prompt}: Original instruction associated with the image.
%         \item \texttt{Rewritten\_Question}: Reformulated version of the original instruction.
%     \end{itemize}
% \end{itemize}

% \subsection*{Dataset Accessibility}

% Researchers and practitioners can access the dataset and its associated resources through our Hugging Face repository:
% \url{https://huggingface.co/datasets/utischoolnlp/multimodal_instruction_rewrites}.

% The dataset is organized in a structured format, including:
% \begin{itemize}
%     \item \texttt{all\_data.tsv}: Consolidated dataset containing all image-instruction pairs.
%     \item \texttt{images.zip}: Compressed archive of all dataset images.
%     \item \texttt{README.md}: Detailed instructions and metadata descriptions for dataset usage.
% \end{itemize}

% \subsection*{Discussion}

% The diversity of data sources, ranging from grocery items to fashion clothing, ensures that the dataset covers a wide array of visual and textual contexts. This variety is crucial for training models that are robust and generalizable across different domains. The substantial number of instructions relative to images indicates that each image is associated with multiple instructions, providing ample data for effective model training and evaluation.

% By consolidating all data into a single TSV file, we streamline the data processing pipeline, facilitating easier integration with various modeling frameworks and tools. The comprehensive statistics on prompt and rewritten question lengths further underscore the dataset's complexity, challenging models to handle a wide range of instruction formulations.

% \section*{Conclusion}

% Our multimodal instruction rewriting dataset offers a comprehensive resource for researchers aiming to develop and evaluate models in this domain. By providing a diverse and sizeable dataset, we aim to facilitate advancements in multimodal understanding and contribute to the broader field of artificial intelligence.

% \section*{References}

% \begin{itemize}
%     \item \href{https://github.com/gulvarol/grocerydataset}{Grocery Store Dataset}
%     \item \href{https://amazon-berkeley-objects.s3.amazonaws.com/index.html}{Amazon Berkeley Objects}
%     \item \href{https://products-10k.github.io/}{Products-10K}
%     \item \href{https://www.kaggle.com/datasets/vikashrajluhaniwal/fashion-images}{Fashion Images Dataset}
% \end{itemize}

% \label{sec:dataset}
\section{Verification via Confined Boxes}
\label{sec:method}

Towards formally verifying recourse over an entire region, we formulate a \emph{mixed-integer quadratically constrained program} (MIQCP) to solve the RVP. 

\paragraph{Characterizing Regions with Boxes}
We focus on a special case of the RVP that finds the largest confined \textit{box}. A box is a set defined by simple upper and lower bound constraints on each dimension. Let $U_j = \max_{x \in {\cal R}}x_j$, $L_j = \min_{x \in {\cal R}}x_j$ be the upper and lower bound for each feature $j$ in the region. Given an upper bound, $\mathbf{u} \in \mathbb{R}^d: \mathbf{u} \leq \mathbf{U}$, and lower bound, $\mathbf{l} \in \mathbb{R}^d: \mathbf{l} \geq \mathbf{L}$, a box $B_{\cal R}(\mathbf{u},\mathbf{l})$ is defined as
$
B_{\cal R}(\mathbf{u},\mathbf{l}) = \{\mathbf{x} \in {\cal R}: \mathbf{l} \leq x \leq \mathbf{u}\}
$.
We focus on boxes due their interpretability, which can help model developers understand the source of fixed predictions. Boxes can be viewed as a type of \emph{decision rule}, which have been widely studied for their interpretability within the broader ML community (e.g., \cite{lawless2023interpretable, lawless2022interpretable, lawless2023cluster}). For ease of notation we drop the explicit dependence on ${\cal R}$ and refer to boxes as $B(\mathbf{u}, \mathbf{l})$. We define the size of a box $B(\mathbf{u},\mathbf{l})$ as the sum of the normalized ranges of each feature:%
\vspace{-0.5em}
\begin{equation} \label{def:boxsize}
\text{Size}(B(\mathbf{u}, \mathbf{l})) = \sum_{j=1}^d \frac{u_j - l_j}{U_j - L_j}
\end{equation}

\paragraph{Generating Confined Boxes}
We start by formulating the related problem of auditing whether a given box $B(\mathbf{u}, \mathbf{l})$ in region ${\cal R}$ contains any data points with recourse, which we denote the \emph{Region Recourse Existence Problem (REP)}. Let $\mathbf{x} \in \mathbb{R}^{d-q} \times \mathbb{Z}^q$ be a decision variable representing an individual, and $\mathbf{a} \in \mathbb{R}^{d-q} \times \mathbb{Z}^q$ represent an action. We assume that the region ${\cal R}$, feature space ${\cal X}$, and action set ${\cal A}$ can be represented by a set of constraints over a mixed-integer set (see \cref{fig:summary} for an example). This general assumption encompasses a variety of potential regions and feature sets. We model the REP as a mixed-integer linear program (MILP) over $\mathbf{x}$ and $\mathbf{a}$ (see Appendix \ref{app:rep_form} for details). 

Recall that the RVP can be cast as an optimization problem to find the largest confined region within ${\cal R}$. By definition the REP is infeasible for \emph{every confined box}. To certify that the REP is infeasible for a given box, and by extension certify that the box is confined, we leverage a classical result from linear optimization called Farkas' lemma: 

\begin{theorem}[\citet{farkas}]\normalfont
Let $A \in \mathbb{R}^{m \times n}$ and $b \in \mathbb{R}^m$. Then exactly one of the following two assertions is true:
\begin{enumerate}[label={\Roman*.},leftmargin=*,itemsep=0.1em]
    \item There exists $x \in \mathbb{R}^n$ such that $Ax \leq b$
    \item There exists $y \geq 0$ such that $A^T y = 0$ and $b^\top y = -1$
\end{enumerate}
\end{theorem}

Farkas' lemma states that we can certify that a system of inequalities over continuous variables $Ax \leq b$ is infeasible by finding a \emph{Farkas certificate} $y \geq 0$ such that $A^\top y = 0$ and $b^\top y = -1$. In our context, we can thus view the problem of finding a confined box as a joint problem of selecting a box and finding an associated Farkas certificate for the REP. However, Farkas' lemma only applies to \emph{continuous} variables, and the REP can include discrete variables.

We extend Farkas' certificates to the discrete setting using a simple strategy that simultaneously generates certificates for all possible continuous restrictions of the REP. A \emph{continuous restriction} of a MILP is a restricted version of the optimization problem where all discrete variables are fixed to specific values. Note that a box is confined if and only if every continuous restriction of the REP is infeasible.

Let  ${\cal C}$ be the set of continuous restrictions, where each restriction $c \in {\cal C}$ corresponds to a specific set of fixed values for the discrete variables (e.g., $x_1 = 1, x_2 = 2$ for a problem with two discrete variables $x_1, x_2 \in \mathbb{Z}^2$). Note that the set ${\cal C}$ is finite, from the assumption ${\cal R}$ is bounded and only discrete variables are fixed, but grows exponentially with respect to the number of discrete variables. If there are no discrete variables in the REP, there is a single continuous restriction representing the full problem with no fixed values. In settings where there are a large number of discrete variables, enumerating all possible continuous restrictions may become computationally intractable. However, we prove in \cref{sec:scaling} that under very general constraints and minimal assumptions we can relax many if not all of the discrete variables in the REP. Under these new theoretical results, the set of restrictions that the algorithm must consider is often incredibly small (e.g., $|{\cal C}| \leq 4$ for all the datasets and actionability constraints considered in \citet{kothari2023prediction}). 

We formulate a continuous restriction $c \in {\cal C}$ of the REP as a linear program (LP) (see Appendix \ref{app:rep_form}), which we represent in the following standard form:
\begin{align*}
C_c\mathbf{x} + D_c\mathbf{a} \leq b_c(\mathbf{u}, \mathbf{l})
\end{align*}
where where $C_c$ and $D_c$ are $m \times d$ matrices and $b_c(u,l)$ is a $m$-dimensional vector that is a linear function of the box upper and lower bounds $\mathbf{u}, \mathbf{l}$. Here $m$ represents the number of constraints in the continuous restriction of the REP.

\paragraph{MIQCP Formulation} We can now formulate the RVP as MIQCP that finds the largest box with Farkas certificates of infeasibility for every continuous restriction. Let $\mathbf{y}_c \in \mathbb{R}^{m}$ be decision variables representing the Farkas certificate for a continuous restriction $c \in {\cal C}$, and $\mathbf{u}, \mathbf{l} \in \mathbb{Z}^d$ represent the upper and lower bounds of a box. Note that there is one variable in $\mathbf{y}$ for every constraint in the continuous restriction. We can now find the largest confined box $B(\mathbf{u}, \mathbf{l})$ with associated certificates of infeasibility $y_c$ for $c \in {\cal C}$ using the \emph{Farkas Certificate Problem (FCP)}:
\begin{subequations}
\allowdisplaybreaks
\begin{align}
	\maximize_{\mathbf{y}_c, \mathbf{u}, \mathbf{l}}\quad&& \sum_d \frac{u_d - l_d}{U_d - L_d} \label{obj:f_size}\\[.1cm]
	\st\quad&& b_c(\mathbf{u}, \mathbf{l})^\top \mathbf{y}_c &= -1 ~~&& \forall c \in {\cal C} \label{const:f_neg_ray}\\
	&& C_c^\top \mathbf{y}_c &= 0 && \forall c \in {\cal C} \label{const:f_dual_feas_a} \\
	&& D_c^\top \mathbf{y}_c &= 0 && \forall c \in {\cal C}\label{const:f_dual_feas_b} \\
	&& \mathbf{y}_c &\geq 0 && \forall c \in {\cal C}\label{const:f_non_neg_y} \\
	&& \mathbf{L} \leq \mathbf{l} \leq \mathbf{u} &\leq \mathbf{U} && \label{const:f_box_bounds} \\
	&& \mathbf{u}, \mathbf{l} &\in \mathbb{Z}^d \label{const:f_ul_int}
\end{align}
\end{subequations}
The objective of the problem is to maximize the size of the box, as defined in Equation \eqref{def:boxsize}. Constraints \eqref{const:f_neg_ray}-\eqref{const:f_non_neg_y} follow from Farkas' lemma and ensure that $y_c$ is a valid certificate of infeasibility for the continuous restriction $c$ of the REP. Constraint \eqref{const:f_box_bounds} ensures the FCP generates a valid box within the region ${\cal R}$. We restrict $\mathbf{u}, \mathbf{l}$ to be integer variables to prevent numerical precision issues when solving this MIQCP in practice. This is not an onerous assumption as any continuous variable $x_j$ with a desired precision $10^{-p}$ can be re-scaled and rounded to an integer variable $10^p x_j$. The problem is quadratically constrained due to the inner product of $b_c(\mathbf{u},\mathbf{l})$ and $\mathbf{y}_c$ in constraint \eqref{const:f_neg_ray}. While MIQCPs are often more computationally demanding than MILPs, the FCP can be solved in seconds on real-world datasets using commercial solvers~\citep[e.g.,][]{achterberg2019gurobi}, as the problem scales with the number of features and actionability constraints (which are typically small) rather than the number of data points in the data set. 

When verifying recourse over a \emph{fixed} box $B(\mathbf{u}, \mathbf{l})$ the FCP can be decomposed into $|{\cal C}|$ problems (solved independently for each continuous restriction). If the FCP is infeasible for any continuous restriction $c$, then the RVP is infeasible for the box. If the FCP is feasible for all continuous restrictions $c \in {\cal C}$, then the box is responsive. Alas, when optimizing over potential boxes, the FCP cannot be decomposed as the variables $\mathbf{u}, \mathbf{l}$ link all the continuous restrictions. 

\paragraph{Generating Multiple Boxes} Solving an instance of the FCP generates a single confined box or certifies that the region is responsive. However, in practice, a given region may contain multiple confined regions. To provide model developers and stakeholders with a comprehensive view of individuals with fixed predictions, the FCP can be run sequentially to enumerate multiple (or all) confined boxes with the region. It does so by iteratively adding \emph{no-good cuts} to exclude previously discovered confined regions from ${\cal R}$ (see \cref{app:multi_boxes} for details).

\subsection{Handling Discrete Variables} \label{sec:scaling}
\begin{table*}[t]
    \centering
    \resizebox{\linewidth}{!}{
    \begin{tabular}{l@{\hspace*{4mm}}R{0.4\linewidth}lR{0.6\linewidth}}
         \textbf{Class} &
         \textbf{Description} &
         \textbf{Formulation} &
         \textbf{Discussion} 
         \\
    \cmidrule(lr){1-4} %\cmidrule(lr){2-4} \cmidrule(lr){5-7}

    $K$-Hot Constraint &
    Preserves that the unweighted sum of a set of variables $\{v_j\}_{j \in J}$ is at most $K \in \mathbb{Z}$. &
    $
    \sum_{j \in J}  \pm~v_j \leq K.
    $ &
    Generalizes the popular \emph{one-hot encoding} for categorical variables. \\
    \cmidrule(lr){1-4} %\cmidrule(lr){2-4} \cmidrule(lr){5-7}

    \makecell{Directional Linkage Constraints}&
    Ensures that one feature, $v_{j}$ is greater than or equal to another feature $v_{k}$ &
    $v_{j} \leq v_{k}.$ &
    Ensures a broad class of non-separable constraints (i.e., constraints that act on multiple features) including thermometer encodings, and deterministic causal constraints (e.g., increasing years of account history implies a commensurate increase in Age). \\
    \cmidrule(lr){1-4} %\cmidrule(lr){2-4} \cmidrule(lr){5-7}
    
    Integer Bound Constraints&
    Places an integer upper or lower bound on a variable &
    $
L_j \leq  v_j \leq U_j.$
&
    Encompasses a wide range of separable constraints including monotonicity, actionability, and bounds on the action step size \cite{kothari2023prediction} \\
    \cmidrule(lr){1-4} %\cmidrule(lr){2-4} \cmidrule(lr){5-7}


    \end{tabular}
    

    }
    \caption{Linear Recourse Constraints Classes. Variables $v_j$ used in the constraints may represent $x$ variables (i.e., constrain the region), $a$ variables (i.e., constrain the actions), or $x + a$ (i.e., constrain the resulting feature vector). This restricted set of constraints encompasses a broad set of existing actionability constraints considered in previous literature.} \label{tab:linear_recourse_const}
\end{table*}

In the preceding section, the RVP was solved by enumerating and finding Farkas' certificates for all continuous restrictions of the REP. However, this approach scales exponentially with respect to the number of discrete variables in the REP. In this section, we show that under a very broad set of actionability constraints and general assumptions we can relax all the discrete variables in the REP and still verify recourse over the entire region.

\paragraph{Linear Recourse Constraints} 
We consider a restricted set of constraints, which we call \emph{linear recourse constraints} (detailed in \cref{tab:linear_recourse_const}). These constraints include a broad class of actionability constraints such as monotonicity, categorical encodings, and immutability. They can be used to define the feature space ${\cal X}$, the region ${\cal R}$, or the action set $A$. Linear recourse constraints encompass many actionability constraints considered in previous literature including all the constraints in \citep{ustun2019actionable, russell2019efficient,kothari2023prediction}. We denote an action set comprised only of these constraints as \textit{linear recourse constraints}. %These constraints can act on either $x$ variables (i.e., constrain the region), $a$ variables (i.e., constrain the actions), or $x + a$ (i.e., constrain the resulting feature vector). Let $v_j$ represent a set of variables corresponding to feature $j$ (i.e., $x_j, a_j$, or $x_j+a_j$). 

%and all but 2 of the 100+ constraints used in the experiments of \citet{}. 
%
% \begin{assumption}[A1, Informal\label{a1:onehot}] 
% All variables participate in at most one $K$-hot constraint.
% \end{assumption}

% \begin{assumption}[A2, Informal\label{a2:directional_linkage}] 
% The set of directional linkage constraints do not imply any relationships between variables participating in $K$-hot constraints.
% \end{assumption}
%
\paragraph{Key Result} 

\cref{thm:tum} shows that we can recover the solution to the REP by solving a \emph{linear relaxation} if:
%
\begin{enumerate}[label={A.\arabic*}, itemsep=0pt]
\item 
% All variables participate in at most one $K$-hot constraint.\label{a1:onehot} 
No variable appears in more than one $K$-hot constraint.\label{a1:onehot} 
\item The directional linkage constraints do not enforce relationships between variables appearing in $K$-hot constraints.\label{a2:directional_linkage}
\item The directional linkage constraints do not imply any circular relationships between variables. \label{a3:cycles}
\end{enumerate}
%
%of the REP (i.e., the problem with only \emph{continuous variables}) is equivalent to solving the original REP. 
Practically, \cref{thm:tum} shows we can solve the FCP with a single continuous restriction (i.e., $|{\cal C}| = 1$),
relaxing all discrete variables in the problem.
%
\begin{theorem}\label{thm:tum}
Under Assumptions \ref{a1:onehot}- \ref{a3:cycles}, the linear relaxation of the REP is feasible iff the REP is feasible for any problem with linear recourse constraints.
\end{theorem}
%
For a full proof and formal definitions of the assumptions, see \cref{app:tum_pf}. The assumptions for \cref{thm:tum} are general and hold in many realistic settings. For instance, $K$-hot constraints are often used to encode categorical features (e.g., via a one-hot encoding). Assumption \ref{a1:onehot} holds in this setting as each associated variable only corresponds to one encoding (i.e., one $K$-hot constraint). Similarly, Assumption \ref{a2:directional_linkage} holds as long as there are no logical implications between the categorical features. Finally, Assumption \ref{a3:cycles} holds as long as there are no circular implications between variables. Circular implications between variables represent flaws in constructing the action set and should be caught prior to solving the RVP.

%Assumption \ref{a1:onehot} holds if $K$-hot constraints are used to encode categorical features, as each feature is only represented in one encoding (i.e., one $K$-hot constraint). Assumption \ref{a2:directional_linkage} holds as long as there are not logical implications between categorical variables encoded using $K$-hot constraints. 
%\textit{Proof Sketch.} We prove this result by showing that the polyhedron defining feasible $x$ and actions $a$ under linear recourse constraints is \emph{totally unimodular}, which means that all extreme points of the polyhedron are integral. 
%Consequently, the linear relaxation of the REP is feasible 
%if and only the discrete REP is feasible. For a full proof and formal definitions of the assumptions, see \cref{app:tum_pf}.

\cref{thm:tum} holds under linear recourse constraints
% which encompass a broad class of potential actionability constraints, 
but not under more general constraints. 
In \cref{app:relax_disc} we discuss how to extend our approach to general constraints, and provide practical guidelines on how to select continuous restrictions to include in the FCP.

% \section{Simulation Evaluation \& Results}\label{sec:results}

\subsection{Baseline Planners}

To evaluate the performance of \PlannerName, we compare it against several baseline methods. In the following section, we describe these baselines, their implementation details, and their respective advantages and limitations, particularly in the context of information gathering in large, high-dimensional search spaces. The simulation framework and vehicle parameters remain consistent across all planners, and each method is allowed to replan during testing.

\subsubsection{Monte-Carlo Tree Search}

Monte Carlo Tree Search (MCTS) can be a powerful technique for finding feasible and optimal paths in complex environments. It is a heuristic search algorithm that builds a search tree incrementally through repeated simulations. At each iteration, it selects a node to explore based on a selection policy (often the Upper Confidence Bound or UCB1 algorithm), expands the tree by adding possible actions from that node, runs a simulation from the newly added node, and updates the statistics of nodes along the path traversed during the simulation. 

The UCB1 (Upper Confidence Bound) algorithm is a technique commonly used in the context of multi-armed bandit problems and Monte Carlo Tree Search (MCTS) for balancing exploration and exploitation. It helps in selecting actions or nodes that are likely to yield high rewards while also exploring less-frequented options to gather more information about their potential rewards. 

We formulate our UCB score in the following manner, \\
\begin{equation*}
    UCB_\text{node} = \frac{I(X_{\text{node}})}{\alpha} + C \times \sqrt{\frac{\ln(N_\text{tree})}{N_\text{node}}}
\end{equation*}
%  $
% UCB_\text{node} = \frac{\overline{X_\text{node}}}{\alpha} + C \times \sqrt{\frac{\ln(N_\text{tree})}{N_\text{node}}}
% $ \\
Here $I(X_{\text{node}})$ denotes the estimated information gain from the node, $\alpha$ denotes the normalization factor which is given by $\frac{B}{v_\text{desired}}$, $B$ being the maximum planning budget and $v_\text{desired}$ being the desired speed of our UAV. $C$ denotes the exploration weight, and $N_\text{tree}$ denotes the number of visits to the tree root node while $N_\text{node}$ denotes the number of times the present node has been visited.

After selecting a candidate node, if it has been visited before, it is expanded by applying motion primitives to generate child nodes, growing the tree. Unvisited nodes skip this step. Following expansion, either the unvisited candidate node or one of its children is selected for the simulation phase, where the future values of nodes along the path are estimated to update the total potential information gain. This informs the selection policy in subsequent iterations. Once planning time is exhausted, the path with the highest information gain is returned.

% with authors goes here
\begin{figure}[t]
\centering
\includegraphics[trim={.7cm 0cm .5cm 1.4cm},clip,width=\columnwidth]{figs/5_/Results1v3.pdf}
\caption{The Monte Carlo simulation results for the planners. The plots show the average percent reduction in entropy over the course of the simulations, and the shading shows the 95\% confidence intervals. IA-TIGRIS outperforms all of the baselines.}
\label{fig:mc_results}
\end{figure}

While MCTS is probabilistically guaranteed to converge to the optimal path \cite{mcts_ref_1}, it is constrained to actions within a predefined set of motion primitives. Its reliance on random sampling to estimate the future value of nodes can result in poor approximations, particularly in environments with sparse, localized pockets of high information gain. This limitation is especially pronounced in large search areas or scenarios with large budgets constraints, where estimating future node values becomes increasingly expensive. As a result, in such scenarios, MCTS is often implemented with a finite planning horizon, which can restrict its ability to account for long-term consequences or dependencies in the environment.

% This property of MCTS, which causes unguided exploration of the environment, leads to increased convergence times on the optimal path, as a result of a lot of budget being spent in exploring information sparse areas of the map. 
% Also, the computation time of MCTS increases exponentially with the depth of the search tree. The time complexity of MCTS is given by $\mathcal{O}(\frac{T}{t_\text{iter}} \cdot |A|^d)$. Here, $T$ is the total planning time and $t_\text{iter}$ is the time taken per iteration of the planning loop. $|A|$ is the number of actions and $d$ represents the average depth of the search tree. 

% The above limitations are not inconsequential in the context of performing informative path planning in large high-dimensional search spaces. We compare MCTS with \PlannerName, in \ref{}, and empirically demonstrate its drawbacks and how \PlannerName, is able to outperform MCTS in the context of the mission parameters we examine in this work.  

\subsubsection{Greedy}

For the greedy planner, we iterated through each cell within the search bounds and calculated the reward for a given cell $i$ as $g_i = R(X_i) / d_i$ where $R(X_i)$ is given through \eqref{equ:reward} and $d_i$ represents the Euclidean distance between the current position the robot at the current time $t$ and the closest viewpoint to the cell. To compute this viewpoint, the yaw between the current pose of the robot and the intersected cell is first calculated. Using the robot's sensor configuration and this yaw, $x$ and $y$ coordinates are calculated that view the cell at the desired flight altitude. With this formulation, the planner prioritizes regions with a high ratio of entropy to distance. This can lead to locally optimal choices that contradict with paths that lead to higher information gain over the entire trajectory. 

% without authors goes here
% \begin{figure}[t]
% \centering
% \includegraphics[trim={.7cm 0cm .5cm 1.4cm},clip,width=\columnwidth]{figs/5_/Results1v3.pdf}
% \caption{The Monte Carlo simulation results for the planners. The plots show the average percent reduction in entropy over the course of the simulations, and the shading shows the 95\% confidence intervals. IA-TIGRIS outperforms all of the baselines.}
% \label{fig:mc_results}
% \end{figure}


\begin{figure*}[t]
    \centering
    \begin{subfigure}[b]{0.99\textwidth}
        \centering
        \includegraphics[trim={0cm 0.3cm 0cm 0cm},clip,width=\textwidth]{figs/5_/Fig2v1_target.png}
        % \caption{Slice by targets}
        % \vspace{.1cm}
    \end{subfigure}
    
    \begin{subfigure}[b]{0.99\textwidth}
        \centering
        \includegraphics[trim={0cm 0cm 0cm 0cm},clip,width=\textwidth]{figs/5_/Fig2v1_sigma.png}
        % \caption{Slice by sigma }
    \end{subfigure}
    \caption{A comparison of the methods based on the number of sampled prior clusters and the standard deviation of sampled prior clusters. IA-TIGRIS is most effective compared to the baselines when there is high variation in the search space. As the search space prior information becomes more evenly spread out, the performance gap between the methods tends to decrease.}
    \label{fig:targets_sigmas}
\end{figure*}

\subsubsection{Random}

The random planner operates by iteratively sampling points within the defined search bounds and calculating the minimum-cost path to observe each sampled point. This process is repeated until the available budget is fully expended. The random planner does not utilize any prior information about the environment or target distribution. Additionally, it does not optimize the sequence of actions, instead treating each sampled point independently without considering the global structure of the search problem. This simplicity allows the random planner to highlight the performance benefits of more sophisticated methods by providing a lower-bound comparison for evaluation.

\subsubsection{Coverage}

The coverage planner generates a plan that systematically covers the entire search space using a straightforward lawn-mower pattern. The spacing between each pass is set to match the width of the projected observation footprint at 20\% from the bottom, ensuring that no grid cells are missed. This spacing also maintains a distance that enables high-quality sensor measurements. However, due to the size of the search spaces considered, the coverage planner spends significant time surveying empty regions. This approach results in inefficient use of the budget, as it prioritizes full coverage with safe sensor overlap, even in areas with little or no valuable information. While simple and robust, this method highlights the tradeoff between exhaustive coverage and efficient, targeted exploration.

% \subsubsection{Branch and Bound}
% The branch and bound baseline is based on motion primitive planning. In each future step the drone has a set of motion primitives with future states and each of these future states also has a set of motion primitives. In this way, a tree can be built with multiple path candidates. The path candidate with the highest information gain will be selected and form the output. 

% By adding branch and bound, there will be an estimation of a node's upper bound information reward, using the node's current information reward, updated information map and the remaining budget. If this upper bound is already lower than the information reward of any other node in the tree, the corresponding node will be closed and not expanded in the future to accelerate the expansion of the tree. 



\subsection{Tests and Analysis}
% To evaluate the efficacy of IA-TIGRIS compared to the baseline methods, we conduct Monte Carlo testing as well as analyze how the prior and budget affect the performance of each method. In all of these test cases, there are no time-based or priority rewards and have horizon lengths set to the full budget. All tests were performed using an Intel Xeon CPU E5-2620 v4 @ 2.10GHz.
To evaluate the efficacy of IA-TIGRIS against baseline methods, we perform Monte Carlo testing and analyze the impact of the prior and budget on the performance of each method. In all test cases, rewards are calculated using \eqref{equ:reward}, and horizon lengths are set to match the full budget. The tests are conducted on an Intel Xeon CPU E5-2620 v4 @ 2.10GHz, ensuring consistent computational conditions across all evaluations.

% Random sample across which parameters.

% Quantitative ideas. Look into number and std of prior (metric for this? std of grid cell values, mediuan, mean,). 
% Uniform prior? 
% Split distinct regions, not smooth. 
% Compare to coverage and amount of time to reach specific amount. 
% Compare with different budgets. 
% Repeatability test. 
% Graph size vs time. 
% Look at coverage with different altitudes or widths. Something that shows long horizon vs not nature of things?
% Shape of search space?
% Time/budget to get x\% of all info gain. Have to do moving horizon. 
% Targets detected? 

% Key thought for results where I show time, our optimization does not optimize for time, only final value. Key thing to show across the different budgets. 

% \BM{Qualitative. Nayana idea of plot with example sampled case. Should add one here.} 



\subsubsection{Monte Carlo Testing}
Our simulated testing environment is a $5000\times5000$ m square with Gaussian-distributed prior information randomly placed throughout the search space. The number of prior clusters was sampled uniformly between $[4,20]$, with standard deviations between $[60,450]$, and maximum value between $[0.05,0.5]$. 

The results of $100$ Monte Carlo tests are shown in Fig.~\ref{fig:mc_results}. IA-TIGRIS clearly outperforms the other methods, achieving nearly a $40\%$ greater reduction in entropy than the next best method. Early in the simulation, the greedy method initially gains information more quickly, as expected, but this does not translate to better long-term performance. Since our method optimizes for total information gain, it generates paths that maximize information collection over the entire budget. MCTS performed slightly worse than the greedy approach.

The random paths slightly outperformed the coverage paths. This is likely because the lawnmower strategy requires sufficient overlap between passes to avoid missing areas, and its long straight paths often lead to redundant observations due to the UAV’s forward-facing camera. Changing the heading of the UAV is beneficial to viewing more of the search space, which may explain why random paths performed better.

We also conducted Monte Carlo tests where either the number of prior clusters or their standard deviation was held constant to analyze how variations in the information map affect planner performance. The results, shown in Fig.~\ref{fig:targets_sigmas}, include two cases: the upper figure fixes the number of priors, while the lower figure fixes their standard deviation. All other agent and simulation parameters remained unchanged.


% The first thing to note from these results is that for all tests the proportional performance gap between IA-TIGRIS and the baselines increases as the number and standard deviation of the Gaussian priors decreases. As the search space becomes more uniformly filled with entropy in the information map, the need for longer-horizon planning decreases and other simple or random approaches can perform satisfactorily given the testing budget. As the information becomes more sparsely distribution in the space, such as when the information is contained in separated pockets of areas, there is a greater need to plan longer-horizon paths that reason about the given budget.
% \BM{Could have figures here or refer to others}

Across these tests, the performance gap between IA-TIGRIS and the baselines widens as the number and standard deviation of the Gaussian priors decrease. When entropy is more uniformly distributed across the search space, simpler methods perform reasonably well within the given budget. However, when information is concentrated in sparse, distinct regions, longer-horizon planning becomes essential. In such cases, IA-TIGRIS demonstrates a significant advantage by effectively reasoning about the budget and prioritizing high-value regions.

% Show plot of first plans expected info gain versus planning time. (plans not executed)


\subsubsection{Budget Analysis}
To evaluate the impact of budget constraints on performance, we conducted additional tests beyond our initial Monte Carlo experiments, evaluating budgets of $5000$ m, $10000$ m, $30000$ m, and $60000$ m. Table~\ref{tab:budgets} summarizes the average entropy reduction across these budgets.

\definecolor{tabfirst}{rgb}{1, 0.7, 0.7} % red
\definecolor{tabsecond}{rgb}{1, 0.85, 0.7} % orange
\definecolor{tabthird}{rgb}{1, 1, 0.7} % yellow
\begin{table}[t]
    \centering
    \resizebox{\linewidth}{!}{
    \begin{tabular}{l|ccccc}
    & $5000$ m & 10000 m  & 15000 m& 30000 m& 60000 m\\ \hline

    % \hline
    IA-TIGRIS  &  \cellcolor{tabfirst}$9.41\pm1.0$ &  \cellcolor{tabfirst}$18.28\pm1.8$ & \cellcolor{tabfirst}$25.36\pm2.3$ & \cellcolor{tabfirst}$41.08\pm2.9$ & \cellcolor{tabfirst}$58.85\pm2.9$ \\
    Greedy  &  \cellcolor{tabsecond}$6.99\pm0.8$ &  \cellcolor{tabsecond}$13.10\pm1.5$ & \cellcolor{tabsecond}$17.97\pm2.0$ & \cellcolor{tabthird}$30.00\pm2.3$ & \cellcolor{tabsecond}$49.38\pm3.5$ \\
    MCTS  &  \cellcolor{tabthird}$6.06\pm0.7$ &  \cellcolor{tabthird}$11.80\pm1.1$ & \cellcolor{tabthird}$17.11\pm1.4$ & \cellcolor{tabsecond}$30.21\pm2.2$ & \cellcolor{tabthird}$48.68\pm2.7$ \\
    Random  &  $2.19\pm0.3$ & $4.29\pm0.7$ & $6.61\pm0.6$ & $17.50\pm1.2$ & $22.47\pm1.4$ \\
    Coverage  &  $1.58\pm0.3$ &  $2.82\pm0.4$ & $4.09\pm0.7$ & $12.04\pm1.9$ & $16.77\pm2.4$ \\

    \end{tabular}
    }
    \caption{Monte Carlo testing results given different budgets. The values are the average percent reduction in entropy and the 95\% confidence bounds. \mbox{IA-TIGRIS} had the best performance for all budgets.}
    \label{tab:budgets}
\end{table}
%$\uparrow$ 

IA-TIGRIS consistently achieved the highest entropy reduction across all budget constraints, with a statistically significant margin over alternative methods. Greedy generally ranked second but was slightly outperformed by MCTS at the $30000$ m budget level. Greedy and MCTS exhibited comparable performance throughout the tests, with their results closely tracking each other. Consistent with our previous findings, Random and Coverage methods yielded the lowest results.


Among the tested methods, only IA-TIGRIS and MCTS explicitly incorporate budget constraints into their planning algorithms. Notably, at lower budgets ($5000$ m and $10000$ m), these methods achieved higher entropy reduction compared to the equivalent time steps ($200$ s and $400$ s) in the $15000$ m budget scenario shown in Fig.~\ref{fig:mc_results}. This improved performance stems from IA-TIGRIS's optimization of total path reward under budget constraints, contrasting with the myopic next-best-action approach of the greedy method. The remaining methods---Greedy, Random, and Coverage---maintain consistent behavior regardless of budget constraints, as their planning strategies do not account for resource limitations.


The performance gap between IA-TIGRIS and the next-best method varied with budget size, showing margins of $34.6\%$, $39.5\%$, $41.1\%$, $36.0\%$, and $19.2\%$ in ascending budget order. This gap widened through the first three budget levels as problem complexity increased, before declining significantly at higher budgets. This performance pattern suggests that implementing a planning horizon could enhance efficiency by limiting tree search depth, enabling the planner to prioritize path quality optimization over exhaustive space exploration.


% percent improved from next best
% 34.6, 39.5, 41.1, 36.0, 19.2
% reasons, too long horizon is a larger search space, so less quality paths closer. Or larger horizon, more packing in


% with authors goes here
\begin{figure}[t] 
    \centering
    \renewcommand\arraystretch{0} % Adjust the height between rows here
    \setlength{\tabcolsep}{1pt} % Adjust the column separation here
    \begin{tabular}{c}
        \begin{tikzpicture}
            \node[anchor=south west, inner sep=0] (image) at (0,0) {
                \includegraphics[width=0.9\linewidth]{figs/5_/google_earth_prior.png}
            };
            \begin{scope}[x={(image.south east)},y={(image.north west)}]
                % \fill[OrangeRed] (0.02, 0.03) circle (2pt); 
                % \fill[OrangeRed] (0.51, 0.04) circle (2pt); 
                % \fill[OrangeRed] (0.61, 0.04) arc (0:90:2pt); 
                \fill[Orange, opacity=0.8] (0.74, 0.45) circle (3pt); % Adjust 
                \fill[Orange, opacity=0.8] (0.27, 0.42) circle (3pt); % Adjust 
                \fill[Orange, opacity=0.8] (0.39, 0.63) circle (3pt); % Adjust 
            \end{scope}
        \end{tikzpicture} \\
        % \includegraphics[width=0.9\linewidth]{figs/5_/google_earth_prior.png} \\
        \\
        \includegraphics[width=0.9\linewidth]{figs/5_/google_earth_path.png} 
    \end{tabular}
    \caption{Google Earth screenshots illustrating the mission planning process and execution. Top: Areas of high entropy targeted for search are highlighted in red, representing regions with a binary occupied/unoccupied probability of 0.2. Three points of particular interest, each assigned a 0.5 probability, are marked in orange. Bottom: The executed drone flight path (yellow) shows the optimized path for maximum information gain across the search space.} 
    \label{fig:google_earth}
\end{figure}
\begin{figure}[t]
\centering
% https://docs.google.com/presentation/d/1RjI-QqHpBRLHN60UAxzmQYs4EaWaVCOoSBkEkA39kk0/edit?usp=sharing
\includegraphics[width=\columnwidth]{figs/5_/m600_labeled.jpg}
\caption{Hexarotor system (DJI M600 Pro) with onboard compute and camera. Left image shows drone on the ground, right image shows drone in flight.}
\label{fig:m600}
\end{figure}


\section{Field Deployments}\label{sec:field}


\subsection{Hexarotor Deployment}
The first field experiment that we present uses a hexarotor drone to cover an urban area shown in Fig.~\ref{fig:fig1}.
We designed this field experiment to simulate classifying where cars are within a search area.  
Hence, we set the plan request to focus on parking lots at the field test site (Fig.~\ref{fig:google_earth}, top), with the addition of three chosen grid cells within the parking lots being marked as having a higher uncertainty. The plan request boundaries and priors were created with GPS coordinates in Google Earth, exported as kml files, and then converted into our plan request message format. 

The following sections details the hardware, autonomy, and experimental results for our hexarotor deployments.

% without the authors goes here
% \begin{figure}[t] 
%     \centering
%     \renewcommand\arraystretch{0} % Adjust the height between rows here
%     \setlength{\tabcolsep}{1pt} % Adjust the column separation here
%     \begin{tabular}{c}
%         \begin{tikzpicture}
%             \node[anchor=south west, inner sep=0] (image) at (0,0) {
%                 \includegraphics[width=0.9\linewidth]{figs/5_/google_earth_prior.png}
%             };
%             \begin{scope}[x={(image.south east)},y={(image.north west)}]
%                 % \fill[OrangeRed] (0.02, 0.03) circle (2pt); 
%                 % \fill[OrangeRed] (0.51, 0.04) circle (2pt); 
%                 % \fill[OrangeRed] (0.61, 0.04) arc (0:90:2pt); 
%                 \fill[Orange, opacity=0.8] (0.74, 0.45) circle (3pt); % Adjust 
%                 \fill[Orange, opacity=0.8] (0.27, 0.42) circle (3pt); % Adjust 
%                 \fill[Orange, opacity=0.8] (0.39, 0.63) circle (3pt); % Adjust 
%             \end{scope}
%         \end{tikzpicture} \\
%         % \includegraphics[width=0.9\linewidth]{figs/5_/google_earth_prior.png} \\
%         \\
%         \includegraphics[width=0.9\linewidth]{figs/5_/google_earth_path.png} 
%     \end{tabular}
%     \caption{Google Earth screenshots illustrating the mission planning process and execution. Top: Areas of high entropy targeted for search are highlighted in red, representing regions with a binary occupied/unoccupied probability of 0.2. Three points of particular interest, each assigned a 0.5 probability, are marked in orange. Bottom: The executed drone flight path (yellow) shows the optimized path for maximum information gain across the search space.} 
%     \label{fig:google_earth}
% \end{figure}
% \begin{figure}[t]
% \centering
% % https://docs.google.com/presentation/d/1RjI-QqHpBRLHN60UAxzmQYs4EaWaVCOoSBkEkA39kk0/edit?usp=sharing
% \includegraphics[width=\columnwidth]{figs/5_/m600_labeled.jpg}
% \caption{Hexarotor system (DJI M600 Pro) with onboard compute and camera. Left image shows drone on the ground, right image shows drone in flight.}
% \label{fig:m600}
% \end{figure}

\subsubsection{Hardware System}
The hardware consists of the DJI M600 Pro, shown in Fig.~\ref{fig:m600}, along with the physical sensing and onboard computer payload. The DJI M600 Pro contains a flight controller that handles pose estimation and position-based control. The DJI M600 Pro’s flight controller also handles teleloperation if human intervention is necessary. Beneath the drone's base, we mount a custom hardware payload.
That payload consists of an onboard computer, a Jetson Xavier, to run the autonomy software shown in Fig.~\ref{fig:functional_diagram}.
The payload also contains a downward-facing a camera for sensing the environment. The camera is a Seek S304SP thermal camera.
The camera intrinsics are used to calculate the frustum's intersection with the search map's cells in IA-TIGRIS.

% without authors goes here
\begin{figure}[t]
\centering
% https://lucid.app/lucidchart/f750ddb4-2809-4773-8361-d5fbb1ba49eb/edit?viewport_loc=-257%2C-116%2C2219%2C1140%2C0_0&invitationId=inv_56e8a3a9-e8cf-4cad-a280-48bd967ff651
\includegraphics[trim={0cm 0cm 0cm 0cm},clip,width=\columnwidth]{figs/5_/functional_diagram.jpeg}
\caption{Functional diagram of the DJI M600 Pro autonomy software.}
\label{fig:functional_diagram}
\end{figure}
\begin{figure}[b]
    \centering
    \begin{subfigure}[b]{0.48\columnwidth}
        \centering
        \includegraphics[width=1.0\linewidth]{figs/5_/field_test_altitude_over_time.png}
        \caption{}
        \label{fig:m600_altitude_over_time}
    \end{subfigure}
    \begin{subfigure}[b]{0.48\columnwidth}
        \centering
        \includegraphics[width=1.0\linewidth]{figs/5_/field_test_entropy_over_time.png}
        \caption{}
        \label{fig:m600_entropy_over_time}
    \end{subfigure}
    \caption{The results for our hexarotor field deployment. (a) Plot of flown altitude over time, showing large variation throughout the experiment. (b) Reduction in entropy percentage over time of field experiment.}
\end{figure}

\subsubsection{Autonomy System}
Fig.~\ref{fig:functional_diagram} illustrates the functional system diagram for the real world field test on the DJI M600. The user specifies the initial plan request prior to takeoff. The TIGRIS planner makes an initial plan on that plan request and sends a global path to the waypoint manager. The waypoint manager tracks the current waypoint within the plan and sends the next waypoint to the DJI software development kit, which then sends actuation commands to the motors. The position of the drone is used to calculate the distance from the drone to the ground and sends that distance parameter to the sensor model. The sensor model's true positive and false positive rate is used to calculate the per-cell entropy updates in the search map manager. The search map manager publishes the current information map, and the replanning node sends an updated plan request to the IA-TIGRIS planner every ten seconds.

The drone started at an altitude of $50$ m above the origin of the reference frame. The informed sampler in IA-TIGRIS was set to add states at altitudes of either $30$ m or $60$ m, creating a trade-off between observation area and detector accuracy. The budget was $2000$ m, the planning horizon was $600$ m, and the planning time was $10$ seconds. 

% % without authors goes here
% \begin{figure}[t]
% \centering
% % https://lucid.app/lucidchart/f750ddb4-2809-4773-8361-d5fbb1ba49eb/edit?viewport_loc=-257%2C-116%2C2219%2C1140%2C0_0&invitationId=inv_56e8a3a9-e8cf-4cad-a280-48bd967ff651
% \includegraphics[trim={0cm 0cm 0cm 0cm},clip,width=\columnwidth]{figs/5_/functional_diagram.jpeg}
% \caption{Functional diagram of the DJI M600 Pro autonomy software.}
% \label{fig:functional_diagram}
% \end{figure}
% \begin{figure}[b]
%     \centering
%     \begin{subfigure}[b]{0.48\columnwidth}
%         \centering
%         \includegraphics[width=1.0\linewidth]{figs/5_/field_test_altitude_over_time.png}
%         \caption{}
%         \label{fig:m600_altitude_over_time}
%     \end{subfigure}
%     \begin{subfigure}[b]{0.48\columnwidth}
%         \centering
%         \includegraphics[width=1.0\linewidth]{figs/5_/field_test_entropy_over_time.png}
%         \caption{}
%         \label{fig:m600_entropy_over_time}
%     \end{subfigure}
%     \caption{The results for our hexarotor field deployment. (a) Plot of flown altitude over time, showing large variation throughout the experiment. (b) Reduction in entropy percentage over time of field experiment.}
% \end{figure}

\subsubsection{Experimental Results}


The bottom image of Fig.~\ref{fig:google_earth} shows the path selected by IA-TIGRIS in the search area. The figure highlights how the planner dynamically adjusts altitudes over time to balance coverage and sensing resolution, maximizing information gain. Higher altitudes allow for broader area coverage, while lower altitudes provide more detailed observations where needed. Additionally, the planner prioritizes revisiting the three regions of higher uncertainty, recognizing the need for repeated observations reduce entropy. This adaptive strategy ensures that uncertain areas receive sufficient attention to improve the belief map. As a result, the entropy of the information map decreases to near zero by the end of the mission, as shown in Fig.~\ref{fig:m600_entropy_over_time}, indicating that the planner has effectively gathered the necessary information. This behavior demonstrates the planner’s ability to optimize sensing actions, balancing altitude selection, revisit frequency, and exploration to maximize mission success.

\begin{figure}[t]
\centering
% \includegraphics[width=2.5in]{fig1}
\includegraphics[trim={4cm 4cm 0cm 4cm},clip,width=\columnwidth]{figs/5_/TL1.jpg}
\caption{Fixed-wing platform used for autonomous flights with an onboard camera pitched at 10 degrees\cite{alarewebsite}}
\label{fig:tl1}
\end{figure}






\subsection{Fixed-wing Deployments}

Our proposed approach was extensively tested on the fixed-wing AlareTech TL-1 UAV, shown in Fig.~\ref{fig:tl1}. The UAV is equipped with an onboard camera pitched at 10 degrees, which introduces a more challenging planning problem due to the non-holonomic motion model and the camera's field of view. Over more than 20 flight hours and 100 flights running IA-TIGRIS, we validated our approach with the objective to search for objects of interest in a large search space across a variety of test scenarios, including different terrain types, varying environmental conditions, and diverse target distributions. An example mission from these tests is shown in Fig.~\ref{fig:fwd}. In this scenario, the planner was given the search bounds and a designated high-priority region. The resulting flight path prioritized revisiting the high-priority area twice, optimizing sensor use and ensuring maximum information gain. This strategy led to the successful detection of the object of interest, with its estimated position marked by the red dot in the figure. 

The map on the upper right in Fig.~\ref{fig:fwd} shows the information map after plan execution was complete. Due to the UAV's limited budget, the upper right and lower left corners of the map are not searched by the agent. The budget is instead utilized to search over the area of higher priority two times. Compared to the paths in Fig.~\ref{fig:google_earth}, we observe that the paths for the fixed wing are smoother and have a larger turning radius, demonstrating how IA-TIGRIS respects the motion constraints of the vehicle. We can also see the effect of wind on the path execution, where the flown path shown in green deviates from the planned path shown in yellow. This illustrates the importance of online planning in the cases where this deviation is large or would accumulate over the course of a longer mission and cause the expected observed area to be much different than actual observed area. 

\begin{figure}[t]
\centering
% \includegraphics[width=2.5in]{fig1}
% [trim={left bottom right top},clip]
\includegraphics[trim={3.0cm, 1.0cm, 3.0cm, 1.0cm},clip,width=\columnwidth]{figs/5_/ONRFig_v3.pdf}
\caption{An example path generated for the fixed-wing platform conducting a large-area search for an object of interest. The larger black rectangle denotes the search bounds, while the smaller black rectangle highlights a region of higher uncertainty. The red dot marks the estimated position of the detected object based on image detections. The upper-right map displays the information state after planning is complete, while the middle plot shows the percent change in entropy over mission time. The flown path illustrates a balance between allocating resources to the high-priority region and exploring other areas within the search space.}
\label{fig:fwd}
\end{figure}

% Also tested extensively on the AlareTech TL-1 (citation?) tube launched UAV seen in Fig.~\ref{fig:tl1}.

% Talk about amount of flights, hours. Platform. Compute. Show visualization fo example flight. Talk about objects of interest in a broad sense (no mention of water/ocean/land for targets). Follow similar figure format as previous section. Main thing we want to highlight is the differences introduced in plans by having a fixed-wing platform compared to a drone. Include image of Alare TL-1 somewhere.

% One big figure showing all the info we want to convey. 

% \BM{Pitch 10 degrees, onboard computer type, etc}


% \subsection{VTOL?}
% what would it bring?


\section{Concluding Remarks}
In this paper, we proposed a novel approach utilizing multimodal LLMs to generate gesture-aware speech recognition transcripts for patients with language disorders. Our framework integrates verbal speech and iconic gestures, enabling the generation of enriched transcripts that capture the latent meaning conveyed through both modalities. Through extensive experimentation, we demonstrated that the proposed method effectively contextualizes incomplete or disfluent speech by incorporating gesture information, leading to more accurate and meaningful representations of the speaker's intent. These findings highlight the potential of our approach to significantly contribute to the field of speech and language therapy, offering innovative tools that can enhance the quality of life for individuals with language disorders by facilitating better communication and assessment methods.

\subsection{Ethical Statement} 
Our dataset was obtained from AphasiaBank with the approval of the Institutional Review Board (IRB) and adheres to the data sharing guidelines set by TalkBank\footnote{https://talkbank.org/share/ethics.html}. This includes complying with the Ground Rules for all TalkBank databases, which are based on the American Psychological Association Code of Ethics~\cite{american2002ethical}.

\subsection{Limitation \& Future Work} 
%This study represents a preliminary investigation into using multimodal LLMs to generate gesture-aware speech recognition transcripts. 
While the results are promising, we recognize several limitations and outline our plans to extend this work further.

One primary limitation is the absence of a definitive ground truth for quantitative evaluation. Since our model generates transcripts by synthesizing speech and gesture data from scratch, traditional benchmarks, such as comparisons with standard speech recognition outputs, are insufficient. Moreover, existing original transcripts lack gesture annotations, making direct comparisons challenging. In future work, we aim to address this gap by collaborating with certified pathologists to conduct qualitative assessments, such as A-B preference tests, to evaluate the effectiveness of gesture-enriched transcripts in accurately conveying the speaker's intentions.

To support quantitative evaluations, we plan to develop novel metrics that assess transcript quality, including grammar accuracy, semantic consistency, and the integration of multimodal information. Such metrics will provide a more objective basis for assessing our model's performance and facilitate comparisons with other multimodal and unimodal approaches.

Another limitation of this study is its focus on structured gestures from a specific task, the Peanut Butter Sandwich Task. While this task offers a controlled context for testing our approach, it does not encompass the diversity of gestures and communication patterns seen in everyday scenarios. As part of our future work, we plan to expand the scope of our model to include tasks such as the Cinderella Story Recall Task~\cite{bird1996cinderella}, which involves unstructured and complex narrative gestures. This expansion will allow us to evaluate the adaptability and robustness of our model in handling varied linguistic and gestural contexts.

In summary, while this study establishes a strong foundation for gesture-aware speech recognition, we aim to refine and extend our methods through collaborative qualitative evaluations, the development of robust quantitative metrics, and broader task applications. These efforts will ensure that our approach continues to evolve, ultimately contributing to more effective communication tools and interventions for individuals with language disorders.





\section*{Limitations}
While our approach demonstrates strong performance in Visual Instruction Rewriting, several limitations remain. First, image downsampling to $256 \times 256$ resolution can lead to the loss of fine-grained text details, affecting instructions that rely on small-font information, such as nutritional labels or product specifications. Second, deictic reference resolution remains challenging, especially in images with multiple similar objects where the model lacks explicit localization cues. The absence of bounding box annotations in our dataset limits the model’s ability to disambiguate references, leading to errors in object-grounded instructions. Additionally, while our model is lightweight and optimized for on-device execution, it still lags behind larger VLMs in handling complex multimodal instructions requiring deep reasoning and external world knowledge. Lastly, our dataset, while diverse across 14 domains, is monolingual, limiting applicability to multilingual and culturally varied settings. Future work can address these challenges by increasing dataset coverage, incorporating localized image region processing, and adding bounding box annotations to improve reference resolution and multimodal grounding.

\section*{Ethics Statement}
This work prioritizes privacy and ethical considerations by designing a lightweight, on-device Visual Instruction Rewriting system that eliminates the need to transmit personal vision-related data to external servers. By converting multimodal instructions into text-only commands, our approach reduces data exposure risks and ensures secure, user-controlled inference. Our dataset is sourced from publicly available and academic-use image collections, ensuring compliance with fair use and licensing policies. However, we acknowledge potential biases in data distribution and the need for greater multilingual and cultural inclusivity. Future efforts will focus on expanding dataset diversity, improving fairness in multimodal understanding, and ensuring responsible AI deployment in real-world applications.

Additionally, we acknowledge the use of OpenAI’s ChatGPT-4 system solely for enhancing writing efficiency, generating LaTeX code, and aiding in error debugging. No content related to the survey's research findings, citations, or factual discussions was autogenerated or retrieved using Generative AI-based search mechanisms. Our work remains grounded in peer-reviewed literature and ethical academic standards.
% \section*{Acknowledgements}

% Entries for the entire Anthology, followed by custom entries
\bibliography{acl2025}
\bibliographystyle{acl_natbib}


\appendix

\section{Appendix: Prompt}
\label{sec:appendix}
``Here is a sketch of an image. 
$\{input\_color\_mask\}$, while the rest of the white space is the background. 
I need you to infer details of the image based on the given sketch.
The details should include the possible background likely to be present with the $\{input\_color\_mask\}$, the attribute of each object (like wearing, texture, color etc.), the state (including action, posture, etc.) of each object, the direction of each object and the relationships between objects.

You should first analyze the mask carefully, considering the size, location, and relative position of each object mask. Ensure that specific actions are analyzed based on the mask, and infer each aspect with a reasoning process before providing the final output.
The final output format should be: $\{format\_example\}$, and you should refer to the example: $\{few\_shot\}$. You are going to complete the "" in each item, you need to complete them in multiple short phrases based on your above reasoning.

The state and relationship should be as detailed as possible while ensuring they align with the mask, formatted as: objectA action/spatial relation objectB, with both objectA and objectB included.
You should properly refer to some examples of attributes of object $\{attributes\}$ and relationships $\{relationships\}$.
Do not include words like `or', `possibly' in your final output, there should no ambiguity in your output.
Make sure all aspects of given mask is filled.''

% \newpage
% \pagebreak







\end{document}
