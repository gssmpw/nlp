Trusted Execution Environments (TEEs) isolate a special space within a device's memory that is not accessible to the normal world (also known as Untrusted Environment), even when the device is compromised. 
Thus, developers can utilize TEEs to provide strong security guarantees for their programs, making sensitive operations like encrypted data storage, fingerprint verification, and remote attestation protected from malicious attacks. 
Despite the strong protections offered by TEEs, adapting existing programs to leverage such security guarantees is non-trivial, often requiring extensive domain knowledge and manual intervention, which makes TEEs less accessible to developers. 
This motivates us to design \system, the first Large Language Model (LLM)-enabled approach that can automatically identify, partition, transform, and port sensitive functions into TEEs with minimal developer intervention. 
By manually reviewing 68 repositories, we constructed a benchmark dataset consisting of 385 sensitive functions eligible for transformation, on which \system achieves a high F1 score of 0.91. 
\system effectively transforms these sensitive functions into their TEE-compatible counterparts, achieving success rates of 90\% and 83\% for Java and Python, respectively. 
We further provide a mechanism to automatically port the transformed code to different TEE platforms, including Intel SGX and AMD SEV, demonstrating that the transformed programs run successfully and correctly on these platforms.