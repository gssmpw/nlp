We evaluate the performance of \system by answering the following research questions:

\begin{itemize}[leftmargin=*]
\item \textbf{RQ 1: Identification Accuracy.} Can \system accurately identify sensitive functions within a program?

\item \textbf{RQ 2: Transformation Consistency.} Can \system ensure that sensitive functions are correctly transformed, achieving the same functionality as the original code?

\item \textbf{RQ 3: Resource Consumption.} What is the resource consumption in code transformations?

%\item \textbf{RQ 4: Application.} How does \system perform in TEEs?

\end{itemize}
\subsection{Experiment Preparation}

\begin{description}[leftmargin = 0pt]
\item [Experiment Environment.]
We ran \system on a server equipped with an AMD EPYC 7763 processor and an NVIDIA RTX 6000 Ada graphics card.
We chose four LLMs: GPT-4o~\cite{openai2024gpt4o}, Qwen2.5-coder:32b~\cite{abs240912186}, Deepseek-v3~\cite{bi2024deepseek}, and \tool{LLama3.1:8b}~\cite{abs230709288} in our experiments, subsequently referred to as GPT-4o, Qwen2.5, Deepseek, and LLama3.1.
Qwen2.5 and LLama3.1 are deployed locally on our server.
For TEE platforms, we selected Intel SGX (Intel i7-9700 processor) for processor-based TEE experiments and AMD SEV (AMD EPYC 7763 processor) for VM-based TEE experiments.

\item [Dataset Construction.]
To the best of our knowledge, no existing dataset of functions to be protected under TEEs is available.
Thus, we manually constructed a benchmark for the task addressed in our paper.
To obtain repositories that potentially include sensitive operations, we conducted keyword searches using \emph{password}, \emph{credential}, \emph{seal}, \emph{serialization}, and \emph{cryptography} on \tool{GitHub}.
To ensure the quality of collected projects, we only kept projects with more than 50 stars, a commonly used threshold to exclude toy projects~\cite{han2023credential, feng2022automated}.
Finally, we obtained 38 Java repositories and 30 Python repositories.
By parsing these projects, we obtained 7,214 and 3,770 Java and Python functions, respectively.
We retained the leaf functions and conducted a manual review with three students who possess expertise in cryptography or software engineering.
Only those functions that were consistently identified as sensitive by all three reviewers were retained.
In total, we marked 232 sensitive functions for Java and 153 sensitive functions for Python.
Additionally, to construct our dataset, we randomly added 241 Java and 166 Python normal functions from the remaining code (i.e., normal functions).
\system utilizes LLMss to generate test inputs for sensitive functions.
We leverage \tool{JaCoCo} for Java and \tool{Pytest-cov} for Python to measure line coverage of these test inputs.
The Java test inputs achieve 89.3\% line coverage, while the Python test inputs achieve 94.6\%.
These coverage rates indicate that the test inputs generated by \system cover the vast majority of the code.

\end{description}

\subsection{RQ1: Identification Accuracy}
In this experiment, we applied \system to the dataset to evaluate its effectiveness in sensitive function identification.
As presented in Table~\ref{tab:eva:detection}, in general, LLMs exhibit high accuracy in identification, with an overall F1-score exceeding 86\%.
GPT-4o demonstrates the best performance in both Java and Python, achieving the highest precision and F1 scores.
Deepseek follows closely behind; Qwen2.5 and LLama3.1 show lower performance in comparison.

For these failure samples, we conducted a manual examination to analyze their characteristics.
Functions that contain multiple shift operations or array manipulations may be incorrectly identified as sensitive operations because shifting is a commonly used operation in cryptography for security handling.
Additionally, if a function involves operations related to a dictionary with key-value pairs, the term \dquote{key} may be improperly identified as referring to cryptographic actions.

\section{Detection results}
We show that combining (1) aggregation of layer-wise predictors and (2) RFM probes at each layer and (3) using RFM as an aggregation method gives state-of-the-art results for detecting concepts from LLM activations.  We outline results below and note that additional experimental details are provided in  Appendix~\ref{app: experimental details}.



\paragraph{Benchmark datasets.} To evaluate concept predictors, we consider seven commonly-used benchmarks in which the goal is to evaluate LLM prompts and responses for hallucinations, toxicity, harmful content, and truthfulness.  To detect hallucinations, we use HaluEval (HE) \citep{halueval}, FAVA \citep{fava}, and HaluEval-Wild (HE-Wild) \citep{haluevalwild}. To detect harmful content and prompts, we use AgentHarm \citep{agentharm} and ToxicChat \citep{toxicchat}. For truthfulness, we use the TruthGen benchmark \citep{truthgen}.  Labels for concepts in each benchmark were generated according to the following procedures.  The TruthGen, ToxicChat, HaluEval, and AgentHarm datasets contain labels for the prompts that we use directly. HE-Wild contains text queries a user might make to an LLM that are likely to induce hallucinated replies. For this benchmark, we perform multiclass classification to detect which of six categories each query belongs to. LAT \citep{representation_engineering} does not apply to multiclass classification and LLM-Check \citep{LLMcheck} detects hallucinations directly, not queries that may induce hallucinations. Hence, we mark these methods with `NA' for this task. For FAVA, we follow the binarization used by \citet{LLMcheck} on this dataset, detecting simply the presence of at least one hallucination in the text. For HE, we consider an additional detection task in which we transfer the directions learned for the QA task to the general hallucination detection (denoted HE(Gen.) in Figure~\ref{fig: f1+accuracies, halluc detection, llama}). 



\paragraph{Evaluation metrics.} On these benchmarks, we compare various predictors based on their ability to detect whether or not a given prompt contains the concept of interest.  We evaluate the performance of detectors based on their test accuracy as well as their test F1 score, a measure that is used in cases where the data is imbalanced across classes (see Appendix~\ref{app: metrics}).  For benchmarks without a specific train/validation/testing split of the data, we randomly sampled multiple splits and reported the average F1 scores and (standard) accuracies of the detector fit to the given train/validation set on the test set on instruction-tuned Llama-3.1-8B, Llama-3.3-70B, and Gemma-2-9B. 



% ADD THIS HEADER TO ALL NEW CHAPTER FILES FOR SUBFILES SUPPORT

% Allow independent compilation of this section for efficiency
\documentclass[../CLthesis.tex]{subfiles}

% Add the graphics path for subfiles support
\graphicspath{{\subfix{../images/}}}

% END OF SUBFILES HEADER

%%%%%%%%%%%%%%%%%%%%%%%%%%%%%%%%%%%%%%%%%%%%%%%%%%%%%%%%%%%%%%%%
% START OF DOCUMENT: Every chapter can be compiled separately
%%%%%%%%%%%%%%%%%%%%%%%%%%%%%%%%%%%%%%%%%%%%%%%%%%%%%%%%%%%%%%%%
\begin{document}
\chapter{Appendix}%

\label{appendix:Appendix}
\section{Neuron Depths}
\begin{table}[htbp]
\centering
\begin{tabular}{ll||ll||ll||ll}
\toprule
Neuron & Depth & Neuron & Depth & Neuron & Depth & Neuron & Depth \\
\midrule
N1  & 3380 & N20 & 2640 & N39 & 2460 & N58 & 2140 \\
N2  & 3220 & N21 & 2600 & N40 & 2440 & N59 & 2100 \\
N3  & 3200 & N22 & 2600 & N41 & 2440 & N60 & 1880 \\
N4  & 3180 & N23 & 2600 & N42 & 2420 & N61 & 1820 \\
N5  & 2980 & N24 & 2600 & N43 & 2400 & N62 & 1680 \\
N6  & 2960 & N25 & 2580 & N44 & 2380 & N63 & 1680 \\
N7  & 2880 & N26 & 2580 & N45 & 2380 & N64 & 1340 \\
N8  & 2860 & N27 & 2580 & N46 & 2360 & N65 & 1320 \\
N9  & 2820 & N28 & 2580 & N47 & 2360 & N66 & 1320 \\
N10 & 2740 & N29 & 2580 & N48 & 2340 & N67 & 1120 \\
N11 & 2720 & N30 & 2560 & N49 & 2320 & N68 & 1080 \\
N12 & 2720 & N31 & 2540 & N50 & 2300 & N69 & 1060 \\
N13 & 2700 & N32 & 2540 & N51 & 2280 & N70 & 1060 \\
N14 & 2680 & N33 & 2520 & N52 & 2280 & N71 & 840  \\
N15 & 2680 & N34 & 2520 & N53 & 2260 & N72 & 660  \\
N16 & 2660 & N35 & 2500 & N54 & 2240 & N73 & 480  \\
N17 & 2660 & N36 & 2480 & N55 & 2220 & N74 & 480  \\
N18 & 2640 & N37 & 2480 & N56 & 2180 & N75 & 200  \\
N19 & 2640 & N38 & 2460 & N57 & 2160 &     &      \\
\bottomrule
\end{tabular}
\caption{Depth ($\mu$m) to probe tip for all neurons used in experiment~\ref{exp:1}}
\label{tab:neuron_depths}
\end{table}
% \begin{table}[htbp]
%     \centering
%     {\footnotesize
%     \begin{tabular}{lcllcl}
%         \hline
%         Neuron & Depth & & Neuron & Depth & \\
%         \hline
%         N1 & 3380$\,\mu$m & & N39 & 2460$\,\mu$m & \\
%         N2 & 3220$\,\mu$m & & N40 & 2440$\,\mu$m & \\
%         N3 & 3200$\,\mu$m & & N41 & 2440$\,\mu$m & \\
%         N4 & 3180$\,\mu$m & & N42 & 2420$\,\mu$m & \\
%         N5 & 2980$\,\mu$m & & N43 & 2400$\,\mu$m & \\
%         N6 & 2960$\,\mu$m & & N44 & 2380$\,\mu$m & \\
%         N7 & 2880$\,\mu$m & & N45 & 2380$\,\mu$m & \\
%         N8 & 2860$\,\mu$m & & N46 & 2360$\,\mu$m & \\
%         N9 & 2820$\,\mu$m & & N47 & 2360$\,\mu$m & \\
%         N10 & 2740$\,\mu$m & & N48 & 2340$\,\mu$m & \\
%         N11 & 2720$\,\mu$m & & N49 & 2320$\,\mu$m & \\
%         N12 & 2720$\,\mu$m & & N50 & 2300$\,\mu$m & \\
%         N13 & 2700$\,\mu$m & & N51 & 2280$\,\mu$m & \\
%         N14 & 2680$\,\mu$m & & N52 & 2280$\,\mu$m & \\
%         N15 & 2680$\,\mu$m & & N53 & 2260$\,\mu$m & \\
%         N16 & 2660$\,\mu$m & & N54 & 2240$\,\mu$m & \\
%         N17 & 2660$\,\mu$m & & N55 & 2220$\,\mu$m & \\
%         N18 & 2640$\,\mu$m & & N56 & 2180$\,\mu$m & \\
%         N19 & 2640$\,\mu$m & & N57 & 2160$\,\mu$m & \\
%         N20 & 2640$\,\mu$m & & N58 & 2140$\,\mu$m & \\
%         N21 & 2600$\,\mu$m & & N59 & 2100$\,\mu$m & \\
%         N22 & 2600$\,\mu$m & & N60 & 1880$\,\mu$m & \\
%         N23 & 2600$\,\mu$m & & N61 & 1820$\,\mu$m & \\
%         N24 & 2600$\,\mu$m & & N62 & 1680$\,\mu$m & \\
%         N25 & 2580$\,\mu$m & & N63 & 1680$\,\mu$m & \\
%         N26 & 2580$\,\mu$m & & N64 & 1340$\,\mu$m & \\
%         N27 & 2580$\,\mu$m & & N65 & 1320$\,\mu$m & \\
%         N28 & 2580$\,\mu$m & & N66 & 1320$\,\mu$m & \\
%         N29 & 2580$\,\mu$m & & N67 & 1120$\,\mu$m & \\
%         N30 & 2560$\,\mu$m & & N68 & 1080$\,\mu$m & \\
%         N31 & 2540$\,\mu$m & & N69 & 1060$\,\mu$m & \\
%         N32 & 2540$\,\mu$m & & N70 & 1060$\,\mu$m & \\
%         N33 & 2520$\,\mu$m & & N71 & 840$\,\mu$m & \\
%         N34 & 2520$\,\mu$m & & N72 & 660$\,\mu$m & \\
%         N35 & 2500$\,\mu$m & & N73 & 480$\,\mu$m & \\
%         N36 & 2480$\,\mu$m & & N74 & 480$\,\mu$m & \\
%         N37 & 2480$\,\mu$m & & N75 & 200$\,\mu$m & \\
%         N38 & 2460$\,\mu$m & & & & \\
%         \hline
%     \end{tabular}
%     }
%     \caption{Depth to Probe Tip for All Neurons Used in Experiment 1}
%     \label{tab:neuron_depths}
% \end{table}

\section{Neural Information Integration}
\label{appendix:integration}
\begin{figure}[H]
    \centering
    \includegraphics[height=0.9\textheight]{images/accuracy_all_onsets.pdf}
    \caption{Classification accuracy at different onsets}
    \label{fig:neural_integration}
\end{figure}

\section{CEBRA Results}
\label{appendix:CEBRA}
\begin{figure}[htbp]
    \centering
    \includegraphics[width=0.9\textwidth]{images/embeddings_plot.png}
    \caption{Extra CEBRA embedding visualization from different parameters}
    \label{fig:all_cebra}
\end{figure}

\begin{figure}[H]
    \centering
    \includegraphics[width=0.9\textwidth]{images/cebra_loss.pdf}
    \caption{CEBRA training loss}
    \label{fig:cebra_loss}
\end{figure}

\begin{figure}[H]
    \centering
    \includegraphics[width=0.45\textwidth]{images/cebra_labels.pdf}
    \caption{Data distribution in CEBRA}
    \label{fig:cebra_labels}
\end{figure}

\section{VAE Results}
\label{appendix:VAE}
\begin{figure}[H]
   \begin{subfigure}[b]{0.32\textwidth}
       \centering
       \includegraphics[width=\textwidth]{images/average_syllable_2.pdf}
       \caption{Syllable 2}
       \label{fig:syllable_2}
   \end{subfigure}
   \hfill
   \begin{subfigure}[b]{0.32\textwidth}
       \centering
       \includegraphics[width=\textwidth]{images/average_syllable_3.pdf}
       \caption{Syllable 3}
       \label{fig:syllable_3}
   \end{subfigure}
   \hfill
   \begin{subfigure}[b]{0.32\textwidth}
       \centering
       \includegraphics[width=\textwidth]{images/average_syllable_4.pdf}
       \caption{Syllable 4}
       \label{fig:syllable_4}
   \end{subfigure}
\end{figure}
\begin{figure}[H]
   \begin{subfigure}[b]{0.32\textwidth}
       \centering
       \includegraphics[width=\textwidth]{images/average_syllable_5.pdf}
       \caption{Syllable 5}
       \label{fig:syllable_5}
   \end{subfigure}
   \hfill
   \begin{subfigure}[b]{0.32\textwidth}
       \centering
       \includegraphics[width=\textwidth]{images/average_syllable_6.pdf}
       \caption{Syllable 6}
       \label{fig:syllable_6}
   \end{subfigure}
   \hfill
   \begin{subfigure}[b]{0.32\textwidth}
       \centering
       \includegraphics[width=\textwidth]{images/average_syllable_7.pdf}
       \caption{Syllable 7}
       \label{fig:syllable_7}
   \end{subfigure}

   \begin{subfigure}[b]{0.32\textwidth}
       \centering
       \includegraphics[width=\textwidth]{images/average_syllable_8.pdf}
       \caption{Syllable 8}
       \label{fig:syllable_8}
   \end{subfigure}
   
   \caption{Original and reconstruction syllables of a motif}
   \label{fig:all_syllables}
\end{figure}

\begin{figure}[H]
    \centering
    \includegraphics[width=\linewidth]{images/2d_vae_vocal_warped.pdf}
    \caption{Reconstruction of warped vocal data}
    \label{fig:whole_motif}
\end{figure}

\begin{figure}[H]
    \centering
    \includegraphics[width=\linewidth]{images/neural2vocal_80ms.pdf}
    \caption{Generate 80\,ms vocalization from 80\,ms neural data}
    \label{fig:neuro2voc_80ms}
\end{figure}

% \begin{figure}
%     \centering
%     \includegraphics[width=\linewidth]{images/2d_vae_vocal_trimmed.pdf}
%     \caption{Trimmed Vocal Data to 80ms}
%     \label{fig:vocal_trimmed}
% \end{figure}

% \begin{figure}
%     \centering
%     \includegraphics[width=\linewidth]{images/2d_vae_vocal_padded.pdf}
%     \caption{Padded Vocal Data to 224ms}
%     \label{fig:vocal_padded}
% \end{figure}

% \begin{figure}
%     \centering
%     \includegraphics[width=\linewidth]{images/2d_vae_vocal_warped.pdf}
%     \caption{Warped Vocal Data to 224ms}
%     \label{fig:vocal_warped}
% \end{figure}


\end{document}




\paragraph{Results.} Among all detection methods based on activations for hallucination detection, RFM was a component of the winning model across all datasets. These include (1) RFM from the single best layer, or (2) aggregating layers with RFM as the layer-wise predictor and either linear regression or RFM as the aggregation model (Table~\ref{fig: f1+accuracies, halluc detection, llama}). For Gemma-2-9B, the best performing method, among those that learn from activations, used aggregation or RFM on the single best layer on three of the four datasets (Table~\ref{fig: f1+accuracies, halluc detection, gemma}). The detection performance for the best performing detector on Llama-3.1-8B was better than that of Gemma-2-9B across all hallucination datasets, hence the best performing detectors taken across both models utilized both aggregation and RFM. The results are similar for detecting harmful, toxic, and dishonest content (Table~\ref{fig: f1 scores, combined non-halluc detection}). Among all models tested, the best performing model utilized aggregation and/or RFM as one of its components. Further, among the smaller LLMs (Llama-3.1-8B and Gemma-2-9B), the overall best performing model for each dataset utilized RFM. 

Moreover, aggregation with RFM as a component (either as the layer-wise predictor or aggregation method) performed better than the state-of-the-art judge model, GPT-4o \citep{gpt4o}. For Truthgen, we found that our method out-performed GPT-4o on Llama-3.3-70B but not with Llama-3.1-8B. We note that GPT-4o may have an advantage on this task as the truthful and dishonest responses were generated from earlier version of this model (GPT-3.5 and 4).  

Our aggregation method also performs favorably compared to methods designed for specific tasks. For example, our aggregation method outperformed a recent hallucination detection method based on the self-consistency of the model (LLM-Check \citep{LLMcheck}), which was argued to be the prior state-of-the-art in their work on the FAVA dataset. Further, despite the generality of our method, we perform only slightly worse than the fine-tuned model (ToxicChat-T5-Large) for toxicity detection on the ToxicChat dataset in F1 score and accuracy (Tables~\ref{fig: f1 scores, combined non-halluc detection} and \ref{fig: accuracies, non-halluc detection combined}). 

\subsection{RQ2: Transformation Consistency}
This experiment employs 232 sensitive functions written in Java and 153 in Python.
For these samples, \system transforms them into Rust code.
We validated the consistency after transformation, analyzed the reasons for failures, and compared \system with other prompting methods.

\subsubsection{Metric}
Before launching the experiment, we established several metric measures.

\begin{itemize}[leftmargin=*]
    \item \textbf{\#Original:} Refers to the number of samples marked as sensitive, specifically those that require transformation.

    \item \textbf{\#Direct:} Refers to the number of samples that directly achieve executability and pass the consistency validation after an \emph{initial transformation}, with no need for activating \emph{iterative refinement}.

    \item \textbf{\#Succeed:} Refers to the number of samples that become executable and pass consistency validation following \emph{iterative refinement}.
    Note that {\#Succeed:} includes {\#Direct:}.

    \item \textbf{Avg. Iter.:} Refers to the average number of iterations the code successfully completes the transformation while achieving both TEE-executable and consistent functionality with the original code. 
    The {\#Direct:} samples do not included in the statistics.

\end{itemize}

\subsubsection{Initial Transformation}
Initially, referring to the first step in Section~\ref{subsub:initial_transformation}, we applied several prompts and examples to transform the code into Rust.
The \textit{\#Direct} row in Table~\ref{tab:eva:consistent} shows the result.
The table shows that, solely with the provided prompts and examples, the LLM agent is unable to successfully transform the majority of the samples, achieving a success rate of less than 21\%.
%This indicates that the remaining functions require further iterative refinement.
We manually analyzed these \emph{Direct} samples and summarized their characteristics.
Typically, they contain only a few simple statements, including direct processing of strings (e.g., hashing) and returning the result without any branching.
However, a high failure rate indicates that only providing prompts and examples cannot meet our transformation requirements, as other samples may contain complex operations, such as standard library imports, initialization of cryptographic algorithms, and pertinent error handling.

% Please add the following required packages to your document preamble:
% \usepackage{graphicx}
\begin{table}[h]
\centering
\caption{Consistency score (the probability that a synthetic example is closer to the training set rather than the holdout set). A score closer to 50\% is better.}
\label{Privacy}
\resizebox{\textwidth}{!}{%
\begin{tabular}{cccccccc}
\toprule[1pt]
 &
  \multicolumn{1}{c}{\textbf{Real Dataset}} &
  \multicolumn{2}{c}{\textbf{Traditional Methods}} &
  \multicolumn{3}{c}{\textbf{SOTA Methods}} &
  \begin{tabular}[c]{@{}c@{}}Our Workflow\\ +Tabsyn\end{tabular} \\ \midrule
\textbf{Methods} &
  \textbf{Real} &
  \textbf{SMOTE} &
  \textbf{CTGAN} &
  \textbf{Tabddpm} &
  \textbf{Tabsyn} &
  \textbf{Great}  \\ \midrule
\textbf{Consistency score} &
  \textbf{\large 100.00±0.00} &
  \textbf{\large 92.16±0.01} &
  \textbf{\large 36.71±0.01} &
  \textbf{\large 21.75±0.01} &
  \textbf{\large 70.61±0.01} &
  \textbf{\large 97.29±0.01}  \\ \bottomrule[1pt]
\end{tabular}%
}
\end{table}


\subsubsection{Iterative Refinement}
For remaining samples that did not achieve successful transformation, \system carried out \emph{iterative refinement} with the \tool{ReAct} strategy to modify the code.
The \textit{Succeed} row in Table~\ref{tab:eva:consistent} illustrates the results that achieve successful transformation.
Meanwhile, the \textit{Ave. Iter.} row represents the average number of iterations required to attain success.
Figure~\ref{code:eva:key_case} shows an example of successful transformation, which presents code for private and public key generation.
The code defines a method that generates an RSA key pair with a specified number of bits.
It use a random number as the seed to create the key pair and return.
The transformed code used Rust's library implement the same function, and it also incorporated additional exception handling.

\begin{figure}[htb]
    \centering
    \begin{minipage}{.48\textwidth}
\begin{tcolorbox}[colback=gray!5!white, colframe=cyan!40!black, boxsep=2pt, top=1mm, bottom=1mm, left=0pt, right=0pt, title=\centering{Original code}]
\begin{minted}[fontsize=\scriptsize, breaklines, autogobble]{python}
def genKeys(bits):
    random_generator = Random.new().read
    key = RSA.generate(int(bits), random_generator)
    privatekey = key.exportKey("PEM")
    publickey = key.publickey().exportKey("PEM")
    keys = [privatekey, publickey]
    return keys
\end{minted}
\end{tcolorbox}
\end{minipage}%
\vspace{1em} % Add some space between the two code blocks
\begin{minipage}{.48\textwidth}
\centering
\begin{tcolorbox}[colback=gray!5!white, colframe=cyan!40!black, boxsep=2pt, top=1mm, bottom=1mm, left=0pt, right=0pt, title=\centering{Transformed code}]
\begin{minted}[fontsize=\scriptsize, breaklines, autogobble]{rust}
fn gen_keys(bits: usize) -> (String, String) {
    let private_key = RsaPrivateKey::new(&mut OsRng, bits).expect("Failed");
    let public_key = RsaPublicKey::from(&private_key);
    let private_pem = private_key.to_pkcs8_pem(LineEnding::LF) .expect("Failed");
    let public_pem = public_key.to_public_key_pem(LineEnding::LF) .expect("Failed");
    (private_pem.to_string(), public_pem.to_string())
}
\end{minted}
\end{tcolorbox}
\end{minipage}
\caption{Example of key generation between original and transformed code.}
\label{code:eva:key_case}
\end{figure}



Overall, the success rate of GPT-4o is the highest, regardless of the programming language employed.
When processing Java transformations, GPT-4o achieved a 92.2\% success rate across 214 samples.
Qwen2.5 achieved an 82.3\% success rate on 191 samples, while Deepseek reached 87.9\% success on 204 samples.
LLama3.1 lagged behind, transforming only 4 samples with a 1.7\% success rate.
Python transformations require more iterations and yield lower overall success rates.
GPT-4o transformed 127 samples with an 83.6\% success rate.
Qwen2.5 managed a 66.1\% success rate on 101 samples, while Deepseek achieved 76.5\% success on 117 samples.
LLama3.1 again lagged significantly, transforming only 4 samples with a 2.6\% success rate.
The lowest success rate of LLama3.1 can be attributed to its smaller model size (8 billion parameters) in our evaluation, which has limited reasoning capacity to accurately leverage actions during the iteration process.
From this perspective, our approach requires the model to have strong reasoning capabilities.
Since LLama3.1 no longer has sufficient capability, subsequent experiments will exclude this model.




Part of Python's lower success rate compared to Java can be attributed to its dynamic typing characteristic.
As an interpreted language, Python's absence of explicit type declarations limits the information available to the LLM.
Conversely, the target language, Rust, being statically typed, requires more type information for transformation.
For verification, we added type annotations for the arguments and return values of failed samples to transform again.
Table~\ref{tab:eva:add_type} illustrates their results.
Among these samples, GPT-4o successfully transformed an additional 4, Qwen2.5 transformed 12, and Deepseek transformed 10.
This resulted in success rates of 85.6\%, 73.9\%, and 83.1\%, respectively.
Given that GPT-4o already achieved the highest success rate, its improvement is not substantial.
The success rate of Qwen2.5 has improved most significantly.
Deepseek had nearly matched the performance of GPT-4o.
It is evident that for interpreted languages, the success rate can be enhanced through the manual addition of type annotations.

\begin{table}[htb]
\centering
\caption{Python transformation results with additional type annotations.}
\label{tab:eva:add_type}
\begin{threeparttable}
\begin{tabular}{lccc}
\toprule[1.5pt]
\textbf{Consistent} & \textbf{GPT-4o} & \textbf{Qwen2.5} & \textbf{Deepseek} \\
\midrule[0.8pt]
W/O Ann. & 127(83.6\%) & 101(66.1\%) & 117(76.5\%)\\
W Ann. & 131(+4, 85.6\%) & 113(+12, 73.9\%) & 127(+10, 83.1\%) \\
\bottomrule[1.5pt]
\end{tabular}
\begin{tablenotes}
    \small
    \item[*] \textit{W/O Ann.} refers to \textit{without type annotation};
    \textit{W Ann.} refers to \textit{with type annotation};
\end{tablenotes}
\end{threeparttable}
\end{table}


\subsubsection{Failure Analysis}
For the code that failed to transform, we performed a manual examination of its implementation details and code structure, identifying the following issues:
\begin{description}[leftmargin = 0pt]
    \item [Missing flag variable.] There is a global static flag whose specific value the agent cannot determine, resulting in continuous substitution with incorrect variables.

    \item [Shift operations.]
    Code containing numerous shift operations, even if successfully transformed for execution, may terminate due to illegal shift operation errors.

    \item [Sophisticated cryptography.]
    If a code employs sophisticated and multiple algorithms, the agent may fail to accurately transform the code. 
    For instance, a Java code simultaneously utilizes \tool{Curve25519} public keys and \tool{XSalsa20} as a stream cipher for encryption, ultimately employing the \tool{Poly1305} algorithm for authentication.
    \system inaccurately transforms the code due to the complexity of the cryptographic operations, preventing successful consistency validation before the iteration threshold was reached.
    
    \item [Functionality change.]
    Among the samples that did not pass consistency validation, there are instances of \dquote{security upgrade} modifications.
    For example, the original code employed \emph{SHA-1} for hash computation, while the transformed code replaced it with \emph{SHA-256}, thereby enhancing security.
    Although this results in inconsistencies, from a security standpoint, \emph{SHA-256} offers greater reliability than \emph{SHA-1}, 
        given that \emph{SHA-1} has been demonstrated to be more vulnerable to collision attacks~\cite{Merrill17limits, wang2005finding}.
    Subjectively, we consider this to be a beneficial modification as it offers a more secure implementation of functionality.
    Unfortunately, the agents did not always exhibit such reliability.
    In the absence of sufficient semantic information, 
        particularly with Python code, 
        the agent may compromise the security of the transformed code while also producing inconsistent outputs.
    For example, Figure~\ref{code:eva:hashing} is the original and transformed Python code, which calculates the hash value using the hashing and salting method (recursive hash).
    Although the agent substituted \emph{SHA-1} with \emph{SHA-256}, the omission of the salt model still led to a functional inconsistency.
\end{description}

We also calculated the relationship between the success rate and the code size, as shown in Figure~\ref{fig:eva:success}.
The results indicate that the code size of the function has little impact on the transformation success rate. 
Among our experiments, the samples with a code size between 120-160 Lines are fewer, hence they show a relatively higher success rate.

\begin{figure}[htb]
    \centering
    \includegraphics[width=0.98\linewidth]{fig/eva/success_rate_by_line.pdf}
    \caption{The success rate under different function sizes.}
    \label{fig:eva:success}
\end{figure}

\begin{figure*}[htb]
\centering
\begin{minipage}{.49\textwidth}
\begin{tcolorbox}[colback=gray!5!white, colframe=cyan!40!black, boxsep=2pt, top=1mm, bottom=1mm, left=0pt, right=0pt, title=\centering{Original code}]
\begin{minted}[fontsize=\scriptsize, breaklines, autogobble]{python}
from passlib.hash import ldap_salted_sha1
def hash(a):
    # hash salt
    h = ldap_salted_sha1.hash(a)
    
    return h
\end{minted}
\end{tcolorbox}
\end{minipage}
\hfill
\begin{minipage}{.49\textwidth}
\centering
\begin{tcolorbox}[colback=gray!5!white, colframe=cyan!40!black, boxsep=2pt, top=1mm, bottom=1mm, left=0pt, right=0pt, title=\centering{Transformed code}]
\begin{minted}[fontsize=\scriptsize, breaklines, autogobble]{rust}
use sha2::{Sha256, Digest};use hex;
fn hash(a: &str) -> String {
    let mut hasher = Sha256::new();
    hasher.update(a);
    hex::encode(hasher.finalize())
}
\end{minted}
\end{tcolorbox}
\end{minipage}
\caption{Inconsistent functionality caused by security method change.}
\label{code:eva:hashing}
\end{figure*}



\subsubsection{Comparison and Analysis}
\system employs multiple methods to assist the agent in accomplishing the transformation tasks.
This sub-experiment still employs LLMs for transformation but uses different methods for comparison:
\begin{itemize}[leftmargin = *]
    \item \textbf{Zero-Shot:} This method directly prompted the LLM to transform sensitive functions into Rust code without providing any additional information.
    
    \item \textbf{One-Shot:} This method prompted the LLM to transform the code using three transformed examples. 
    However, unlike the \emph{initial transformation}, it did not require the LLM to analyze the code first.

    \item \textbf{Compiler Check:} Other processes are consistent with \system, except that there is no consistency validation during the iterative refinement.
    It only utilized compiler checks as feedback for the LLM agent.

\end{itemize}

\begin{table*}[htb]
    \centering
    \caption{Comparison of code transformations across different methods.}
    \label{tab:eva:cmp}
    \begin{tabular}{lcccccccc}
    \toprule[1.5pt]
    \multirow{2}{*}{\textbf{Method}} & \multicolumn{4}{c}{\textbf{Java (232 \#Original Samples)}} & \multicolumn{4}{c}{\textbf{Python (153 \#Original Samples)}} \\
    \cmidrule(lr){2-5} \cmidrule(lr){6-9}
     & \textbf{GPT-4o} & \textbf{Qwen2.5} & \textbf{Deepseek} & \textbf{LLama3.1} & \textbf{GPT-4o} & \textbf{Qwen2.5} & \textbf{Deepseek} & \textbf{LLama3.1}\\
    \midrule[0.8pt]
    
    Zero-Shot & 22 (9.4\%) & 18 (7.8\%) & 20 (8.6\%) & 0 (-) & 13 (8.5\%) & 11 (7.2\%) & 13 (8.5\%) & 0 (-) \\
    One-Shot & 44 (18.9\%)  & 38 (16.4\%) & 42 (18.1\%) & 0 (-) & 26 (16.9\%) & 18 (11.8\%)& 28 (18.3\%) & 0 (-)  \\
    Compiler Check & 158 (68.1\%) & 141 (60.8\%)& 149 (64.2\%)& 2 (0.8\%) & 61 (39.8\%) & 49 (32.1\%) & 56 (24.1\%) & 1 (0.6\%) \\
    \system & 214 (92.2\%) & 191 (82.3\%) & 204 (87.9\%) & 4 (1.7\%) & 127 (83.6\%) & 101 (66.1\%) & 117 (76.5\%) & 4 (2.6\%)\\

    \bottomrule[1.5pt]
    
    \end{tabular}
    \end{table*}

Table~\ref{tab:eva:cmp} presents the comparison results.
According to the table, the success rate for the \textit{Zero-Shot} is the lowest because of the lowest information provided.
When additional transformation examples were introduced (\textit{One-Shot}), the success rate improved.
Despite this enhancement, a significant number of samples (over 80\%) still remained unsuccessful.
Following the introduction of the \textit{Compiler Check}, the success rate exhibited a substantial increase.
\textit{Compiler Check} resulted in the majority of the transformed code being executable; however, it did not ensure their consistency with the original code.
In comparison, since consistency validation was introduced, \system achieved the highest success rate, ensuring both executable and consistent functionality of transformed code.

\subsubsection{Stability}
Utilizing LLMs as agents introduces an element of randomness, which can lead to different transformation outcomes for the same code.
To assess the stability of \system, we conducted multiple rounds of repeated experiments.
We conducted four more rounds of repeated experiments, and the results are presented in Table~\ref{tab:eva:rep}.
For these 37 samples, GPT-4o, Qwen2.5, and Deepseek successfully transformed 35, 30, and 33 samples, respectively, in the original experiment.
The table indicates that, in most cases, the success rate remained consistent.
However, the results also exhibited minor fluctuations.
In \textit{Exp1}, GPT-4o successfully transformed one additional sample, whereas Qwen2.5 in \textit{Exp3} failed to transform one previously successful sample.
The one more successful transformation is a data encryption process.
The failure instance arose from a functionality change, where the source code used \emph{SHA1}, while the transformed result applied \emph{SHA256}.
In general, despite the presence of some fluctuations, \system demonstrates stability in its transformation outcomes.

\begin{table}[ht]
    \centering
    \caption{Success rates of transformations in repeated experiments.}
    \label{tab:eva:rep}
    \begin{threeparttable}
    \begin{tabular}{l|c|cccc}
    \toprule[1.5pt]
    & \makecell[c]{\textbf{Original Exp} \\ \textbf{(37 Samples)}} & \textbf{Exp1} & \textbf{Exp2} & \textbf{Exp3} & \textbf{Exp4} \\
    \midrule[0.8pt]
    GPT-4o & 35 Consistent & 36 (+1) & 35 & 35 & 35 \\
    Qwen2.5 & 30 Consistent & 30 & 30 & 29 (-1) & 30 \\
    Deepseek & 33 Consistent & 33 & 33 & 33 & 33 \\
    \bottomrule[1.5pt]
    \end{tabular}
    \begin{tablenotes}
    \small
    \item[*] \textit{Exp} refers to \textit{Experiment}.
    \end{tablenotes}
    \end{threeparttable}
\end{table}


\subsection{RQ3: Resource Consumption}
The bottleneck of our \system is the response consumption of LLMs.
We recorded their average response time during our experiments, with the results presented in Table~\ref{tab:time_consumption}.
Here, \textit{Identify Sensitive Function} records the average response time taken to identify sensitive operations in code.
\textit{Thought \& Action} records the average response time taken to create a response in \emph{iterative refinement} with the \tool{ReAct} strategy.
Since each \emph{iterative refinement} carries out multiple thoughts and actions, this process is the most time-consuming in \system.
Figure~\ref{fig:eva:dis} illustrates the distribution of the number of iterations within this process.
From the figure, the majority of successful transformations require more than three iterations.
% which means it requires at least three responses, i.e., 24.9s, 54.6s, and 21.9s on average.

% \usepackage{booktabs}


\begin{table}
	\centering
	\caption{Analysis of models' inference speed.}
	\begin{tabular}{c|ccc} 
		\toprule
		& Parms (MB) & Speed & Moderate  \\ 
		\hline
		IA-SSD     & 2.7        & 84    & 79.12     \\
		PDM-SSD(A) & 3.3        & 84    & 79.37     \\
		PDM-SSD(J) & 3.3        & 68    & 79.75     \\
		\bottomrule
	\end{tabular}
\label{tabel8}
\end{table}

\begin{figure}[htb]
    \centering
    \includegraphics[width=0.98\linewidth]{fig/eva/distribution.pdf}
    \caption{Iteration count distribution for iterative refinement across models and languages.}
    \label{fig:eva:dis}
\end{figure}

% \subsection{Case Study}
% In this section, we utilize two cases in the GPT-4o experiment to illustrate how \system protects a program using TEEs.
% \system identified their sensitive operations and transformed them into Rust code passing compiler checks and consistency validation.
% Besides, we also developed them into both SGX and SEV TEE and utilized test inputs to evaluate the changes in operational overhead.
% For SGX, we utilized the \tool{Rust-SGX}~\cite{rustsgx} tool to load our transformed code into the TEE.
% For AMD SEV, we launched an Ubuntu 20.04 virtual machine through SEV to execute transformed code.
% Table~\ref{tab:time_run} presents the recorded execution times for both SGX and SEV, along with the changes compared to the original code.

% \begin{table}[htb]
\centering
\caption{Performance overhead during runtime.}
\label{tab:time_run}
\begin{threeparttable}
\begin{tabular}{lccccc}
\toprule[1.5pt]
\multirow{2}{*}{\textbf{Case}} & \multirow{2}{*}{\textbf{Orig.}} & \multicolumn{2}{c}{\textbf{SGX}} & \multicolumn{2}{c}{\textbf{SEV}} \\ 
\cmidrule(lr){3-4} \cmidrule(lr){5-6}
&  & \textbf{TEE Ver.} & \textbf{$\Delta$}  & \textbf{TEE Ver.} & \textbf{$\Delta$} \\
\midrule[0.8pt]
Encode Data & 0.03s  & 0.14s & 4.67 $\times$  & 0.07s & 2.33 $\times$ \\ 
Create Key (Lib) & 0.22s  & 18.31s & 82.2 $\times$  & 0.26s & 1.18 $\times$ \\ 
\bottomrule[1.5pt]
\end{tabular}
\begin{tablenotes}
    \small
    \item \textit{Lib} refers to \textit{Library};
    \textit{Orig.} refers to \textit{Original};
    \textit{TEE Ver.} refers to \textit{TEE Version};
    \textit{$\Delta$} refers to the factor of increase.
\end{tablenotes}
\end{threeparttable}
\end{table}

% \subsubsection{Encoding Data}
% Figure~\ref{code:eva:encode_case} illustrates the original and transformed code of a function implementing data encoding (serialization).
% This function does not involve any third-party libraries and encodes the input data according to its rules.
% For operational overhead, due to the simplicity of the functionality, the original code requires only 0.03s to complete its execution.
% SGX and SEV require 0.14s and 0.07s, respectively.
% SGX experiences higher time consumption due to environment switches (TEE and normal world) compared to VM-based SEV, which has latency similar to standard network communications.
% However, initiating SEV demands significant resources as it involves launching a complete operating system in device memory.
% Furthermore, it does not actively release these resources after the operation has concluded.

% \begin{figure}[htb]
\centering
\begin{minipage}{.95\linewidth}
\begin{tcolorbox}[colback=gray!5!white, colframe=cyan!40!black, boxsep=2pt, top=1mm, bottom=1mm, left=0pt, right=0pt, title=\centering{Original code}]
\begin{minted}[fontsize=\scriptsize, breaklines, autogobble]{java}
static char[] encode(byte[] in, int iOff, int iLen) {
    int oDataLen = (iLen * 4 + 2) / 3;       
    // ...existing code...
    while (ip < iEnd) {
        int i0 = in[ip++] & 0xff;
        // ...existing code...
        out[op] = op < oDataLen ? map1[o3] : '='; op++;
    }
    return out;
}
\end{minted}
\end{tcolorbox}
\end{minipage}

\vspace{1em} % Add some space between the two code blocks

\begin{minipage}{.95\linewidth}
\centering
\begin{tcolorbox}[colback=gray!5!white, colframe=cyan!40!black, boxsep=2pt, top=1mm, bottom=1mm, left=0pt, right=0pt, title=\centering{Transformed code}]
\begin{minted}[fontsize=\scriptsize, breaklines, autogobble]{rust}
fn encode(input: &[u8], i_off: usize, i_len: usize) -> Vec<char> {
    let o_data_len = (i_len * 4 + 2) / 3;
    // ...existing code...
    while ip < i_end {
        let i0 = input[ip] & 0xff;
        // ...existing code...
        out[op] = if op < o_data_len { MAP1[o3] } else { '=' };op += 1;
    }
    out
}
\end{minted}
\end{tcolorbox}
\end{minipage}
\caption{Example of encode function between original and transformed code.}
    \label{code:eva:encode_case}
\end{figure}


% \subsubsection{Create Key}
% Figure~\ref{code:eva:key_case} presents code for private and public key generation.
% The code defines a method that generates an RSA key pair with a specified number of bits.
% For operational overhead, the original code requires 0.22s to create keys.
% SGX requires 18.31s to complete its execution, representing a significant increase in time.
% This is due to the Rust code's reliance on the standard library, which frequently switches back to the normal world for library calls and subsequently returns to the TEE.
% This results in frequent context switching.
% A more effective solution to eliminate this problem is to implement the same library within the TEE; however, this topic is outside the scope of this paper.
% \begin{figure}[htb]
    \centering
    \begin{minipage}{.48\textwidth}
\begin{tcolorbox}[colback=gray!5!white, colframe=cyan!40!black, boxsep=2pt, top=1mm, bottom=1mm, left=0pt, right=0pt, title=\centering{Original code}]
\begin{minted}[fontsize=\scriptsize, breaklines, autogobble]{python}
def genKeys(bits):
    random_generator = Random.new().read
    key = RSA.generate(int(bits), random_generator)
    privatekey = key.exportKey("PEM")
    publickey = key.publickey().exportKey("PEM")
    keys = [privatekey, publickey]
    return keys
\end{minted}
\end{tcolorbox}
\end{minipage}%
\vspace{1em} % Add some space between the two code blocks
\begin{minipage}{.48\textwidth}
\centering
\begin{tcolorbox}[colback=gray!5!white, colframe=cyan!40!black, boxsep=2pt, top=1mm, bottom=1mm, left=0pt, right=0pt, title=\centering{Transformed code}]
\begin{minted}[fontsize=\scriptsize, breaklines, autogobble]{rust}
fn gen_keys(bits: usize) -> (String, String) {
    let private_key = RsaPrivateKey::new(&mut OsRng, bits).expect("Failed");
    let public_key = RsaPublicKey::from(&private_key);
    let private_pem = private_key.to_pkcs8_pem(LineEnding::LF) .expect("Failed");
    let public_pem = public_key.to_public_key_pem(LineEnding::LF) .expect("Failed");
    (private_pem.to_string(), public_pem.to_string())
}
\end{minted}
\end{tcolorbox}
\end{minipage}
\caption{Example of key generation between original and transformed code.}
\label{code:eva:key_case}
\end{figure}



\subsection{Threats to Validity}
One potential threat to validity arises from the fact that the input tests utilized in consistency verification are inadequate to comprehensively cover the original function. 
In order to address this concern and to enhance the accuracy of consistency verification, we employed line coverage tools (i.e., JaCoCo and Pytest-cov) to assess the coverage achieved by the input tests generated by the LLM. 
The coverage information motivates the LLM to optimize coverage of input tests to the greatest extent possible.
Another potential threat to validity pertains to the labeling process. Specifically, during the identification of functions that contain sensitive operations, there exists a risk of human error in judgment.
We engaged three authors with expertise in software engineering and security to collaboratively assess the labeling, thereby enhancing the accuracy of our dataset.
Another threat to validity is the reasoning capabilities of LLMs. 
Our experiments have demonstrated that a smaller parameter model, like Llama 3.1:8B, is unable to successfully perform the transformation tasks that we require.
Therefore, we employ models with higher reasoning capabilities, such as DeepSeek and GPT-4o.