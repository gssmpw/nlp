\begin{figure}[htb]
    \centering
    \begin{minipage}{.48\textwidth}
\begin{tcolorbox}[colback=gray!5!white, colframe=cyan!40!black, boxsep=2pt, top=1mm, bottom=1mm, left=0pt, right=0pt, title=\centering{Original code}]
\begin{minted}[fontsize=\scriptsize, breaklines, autogobble]{python}
def genKeys(bits):
    random_generator = Random.new().read
    key = RSA.generate(int(bits), random_generator)
    privatekey = key.exportKey("PEM")
    publickey = key.publickey().exportKey("PEM")
    keys = [privatekey, publickey]
    return keys
\end{minted}
\end{tcolorbox}
\end{minipage}%
\vspace{1em} % Add some space between the two code blocks
\begin{minipage}{.48\textwidth}
\centering
\begin{tcolorbox}[colback=gray!5!white, colframe=cyan!40!black, boxsep=2pt, top=1mm, bottom=1mm, left=0pt, right=0pt, title=\centering{Transformed code}]
\begin{minted}[fontsize=\scriptsize, breaklines, autogobble]{rust}
fn gen_keys(bits: usize) -> (String, String) {
    let private_key = RsaPrivateKey::new(&mut OsRng, bits).expect("Failed");
    let public_key = RsaPublicKey::from(&private_key);
    let private_pem = private_key.to_pkcs8_pem(LineEnding::LF) .expect("Failed");
    let public_pem = public_key.to_public_key_pem(LineEnding::LF) .expect("Failed");
    (private_pem.to_string(), public_pem.to_string())
}
\end{minted}
\end{tcolorbox}
\end{minipage}
\caption{Example of key generation between original and transformed code.}
\label{code:eva:key_case}
\end{figure}

