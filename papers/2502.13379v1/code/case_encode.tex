\begin{figure}[htb]
\centering
\begin{minipage}{.95\linewidth}
\begin{tcolorbox}[colback=gray!5!white, colframe=cyan!40!black, boxsep=2pt, top=1mm, bottom=1mm, left=0pt, right=0pt, title=\centering{Original code}]
\begin{minted}[fontsize=\scriptsize, breaklines, autogobble]{java}
static char[] encode(byte[] in, int iOff, int iLen) {
    int oDataLen = (iLen * 4 + 2) / 3;       
    // ...existing code...
    while (ip < iEnd) {
        int i0 = in[ip++] & 0xff;
        // ...existing code...
        out[op] = op < oDataLen ? map1[o3] : '='; op++;
    }
    return out;
}
\end{minted}
\end{tcolorbox}
\end{minipage}

\vspace{1em} % Add some space between the two code blocks

\begin{minipage}{.95\linewidth}
\centering
\begin{tcolorbox}[colback=gray!5!white, colframe=cyan!40!black, boxsep=2pt, top=1mm, bottom=1mm, left=0pt, right=0pt, title=\centering{Transformed code}]
\begin{minted}[fontsize=\scriptsize, breaklines, autogobble]{rust}
fn encode(input: &[u8], i_off: usize, i_len: usize) -> Vec<char> {
    let o_data_len = (i_len * 4 + 2) / 3;
    // ...existing code...
    while ip < i_end {
        let i0 = input[ip] & 0xff;
        // ...existing code...
        out[op] = if op < o_data_len { MAP1[o3] } else { '=' };op += 1;
    }
    out
}
\end{minted}
\end{tcolorbox}
\end{minipage}
\caption{Example of encode function between original and transformed code.}
    \label{code:eva:encode_case}
\end{figure}
