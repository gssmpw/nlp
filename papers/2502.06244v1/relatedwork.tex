\section{Related Work}
\label{sec: related}

\textbf{\small Data Curation and Selection.} The effectiveness of language models heavily depends on the quality of the pre-training corpus. Consequently, significant efforts have been made to enhance pre-training data. These efforts include heuristic-based filtering~\citep{raffel2020exploring, rae2021scaling, laurenccon2022bigscience, penedo2023refinedweb, soldaini2024dolma} and deduplication~\citep{abbas2023semdedup, lee2021deduplicating, chowdhery2022palm, dubey2024llama}. Recently, \cite{vo2024automatic} proposed an automated method for constructing large, diverse, and balanced datasets for self-supervised learning by applying hierarchical k-means clustering. \cite{sachdeva2024train} introduced techniques that leverage instruction-tuned models to assess and select high-quality training examples, along with density sampling to ensure diverse data coverage by modeling the data distribution. Additionally, \cite{guu2023simfluence} simulated training runs to model the non-additive effects of individual training examples, enabling the analysis of their influence on a model's predictions.

\textbf{\small Multitask Learning Optimization}  
The approach most closely related to our method is multitask learning (MTL) optimization, which modifies gradient updates to mitigate gradient conflicts—situations where task gradients point in opposing directions, slowing down optimization~\citep{vandenhende2021multi, yu2020gradient}. The Multiple Gradient Descent Algorithm (MGDA)~\citep{desideri2012multiple, sener2018multi} updates the model by optimizing the worst improvement across all tasks, aiming for equal descent in task losses. Projected Gradient Descent (PCGrad)~\citep{yu2020gradient} modifies task gradients by iteratively removing conflicting components in a randomized order, ensuring that updates do not interfere destructively across tasks. Conflict-Averse Gradient Descent (CAGRAD)~\citep{liu2021conflict} optimizes for the worst task improvement while ensuring a decrease in the average loss. NASHMTL~\citep{navon2022multi} determines gradient directions by solving a bargaining game that maximizes the sum of log utility functions. While these methods improve performance, they introduce significant computational and memory overhead, making them impractical for large-scale models with numerous tasks~\citep{xin2022current}. Similar challenges exist in AdaTask~\citep{yang2023adatask}, which improves multitask learning by balancing parameter updates using task-wise adaptive learning rates, mitigating task dominance, and enhancing overall performance. Unlike previous approches that requires  requiring \(O(K)\) storage for task gradients (e.g. PCGrad) or optimizer states (e.g. AdaTask), FAMO~\citep{liu2024famo} balances task loss reductions efficiently using \(O(1)\) space and time. However, these methods fail to exploit the~\textit{non-conflicting} interactions among tasks, focusing instead on resolving conflicts that seldom arise. This highlights the need for a new approach that actively leverages lack of gradient conflicts to enhance training efficiency. 

Another line of work focuses on adjusting the domain mixture to improve data efficiency during training~\citep{xie2024doremi, xia2023sheared, jiang2024adaptive}. However, these methods require a target loss for optimization, which has been shown to not always correlate with downstream performance~\citep{tay2021scale, liu2023same, wettig2024qurating}. In contrast, our method leverages the absence of gradient conflict and the presence of positive gradient interactions between tasks or domains. This approach provides a more reliable and effective way to enhance the final model's performance.