\section{Related Works}
In this section, we explore recent research trends in spiking architectures, multi-modal classification techniques, and feature discrimination approaches.
Building on the aforementioned research fields, the design of our multi-modal spiking architecture has been shaped, combining the strengths of feature discrimination techniques with the efficiency of multi-modal integration.

\subsection{Spiking Neural Network}

SNNs have been extensively studied thanks to their high biological plausibility and the remarkably low computational complexities achieved with neuromorphic processors~\cite{davies2021advancing,massa2020efficient}. 
They have been designed with bio-plausible learning rules~\cite{iakymchuk2015simplified,hao2020biologically,kheradpisheh2018stdp}, by adapting prior knowledge of convolutional neural networks (CNNs) or through several ANN2SNN conversion approaches~\cite{bu2023optimal,wang2024universal,wang2023new}.
SNNs have also shown remarkable performance with backpropagation approaches~\cite{deng2022temporal,guo2023membrane, lee2020enabling} as a result of extensive research on efficient learning approaches~\cite{meng2023towards, wei2023temporal}.
Spiking networks such as spiking VGG~\cite{lee2020enabling,sengupta2019going} and spiking ResNet~\cite{shi2024spikingresformer,zheng2021going,fang2021deep,zhou2023spikingformer,hu2021spiking}, achieve high accuracy by taking advantage of the useful characteristics of the corresponding conventional DNN architectures.
In~\cite{hu2021spiking}, an ANN2SNN conversion is performed using a residual model to scale the activations along with a compensation mechanism to minimize the discretization errors.
In~\cite{sengupta2019going}, the authors suggested an ANN2SNN conversion technique to balance the SNN's threshold for the VGG and ResNet spiking approaches. 
Other approaches propose novel backpropagation algorithms to achieve state-of-the-art results with residual networks~\cite{lee2020enabling}.
Zheng et al.~\cite{zheng2021going} replaced the batch normalization layer (BN) with a custom threshold dependent, introducing a more bio-plausing normalization approach in spiking ResNet.
Fang et al.~\cite{fang2021deep}, introduced a spike-element-wise approach in the common ResNet architecture for residual learning, ensuring identity mapping and overcoming the vanish-exploding gradients problem.



\subsection{Multi-modal classification}

Multi-modal classification, particularly in the audio-visual domain, has gained significant attention for its ability to enhance model performance by leveraging complementary cross-modal features.
In early approaches, multi-modal recognition has been realized through feature level~\cite{natarajan2012multimodal} or decision level fusion~\cite{ebrahimi2015recurrent} based on the stage where the combination of the different modalities is performed. 
Gradually, given their efficiency in classification approaches, ANNs became the most widespread selection for audio-visual classification adopting a feature-level fusion methodology~\cite{metallinou2012context,eyben2011audiovisual,8937495}.
In addition, recurrent neural networks (RNNs) were frequently used in corresponding techniques to effectively capture the temporal properties of a given task~\cite{feng2017audio,makino2019recurrent}.
Yet, despite the wide adoption of several DNN architectures in multi-modal tasks and the promising results of SNNs, the utilization of the latter in multi-modal classification tasks is still in its infancy.
To that end, an unsupervised method using multi-modal SNNs that exploits cross-modal connections and spike-timing-dependent plasticity to interpret visual and auditory data has been presented~\cite{rathi2018stdp}.
Similarly, by dynamically modifying the weights allocated to each modality, the authors in~\cite{liu2022event} introduced the first supervised multi-modal SNN intended for audio-visual classification, also employing an attention mechanism.
Meanwhile, a spiking cross-modal attention mechanism for audio-visual classification with SNN has been proposed, displaying enhanced performance~\cite{10293172}

\subsection{Feature discrimination}
Feature discrimination has gained great attention over the last few years.
Maintaining relatively high intra-class compactness compared to inter-class discrepancy has been a key focus in many learning methodologies in the field~\cite{sun2014deep, wen2016discriminative, adeli2018semi}.
Contemporary methods involve adding angular margins between classes to reinforce the loss function of DNNs to increase the separability between feature vectors. 
To achieve that, the Large-Margin Softmax Loss (L-Softmax)~\cite{liu2016large} combines a softmax function and a normalization scheme applied to the feature vectors of the final fully connected layer to produce tighter angular bounds.
Further empirical and theoretical studies demonstrated that adding hyperspherical or decoupled convolution operations to CNNs can enhance performance~\cite{liu2017deep}. 
For instance, L2-normalization combined with angular margin-based loss functions~\cite{deng2019arcface,liu2017sphereface} has significantly enhanced performance by enforcing intra-class compactness and inter-class separability on the studied hypersphere.
The abovementioned methods produce robust embeddings, while also presenting highly discriminative features.