% CVPR 2025 Paper Template; see https://github.com/cvpr-org/author-kit

\documentclass[10pt,twocolumn,letterpaper]{article}

%%%%%%%%% PAPER TYPE  - PLEASE UPDATE FOR FINAL VERSION
\usepackage{cvpr}              % To produce the CAMERA-READY version
% \usepackage[review]{cvpr}      % To produce the REVIEW version
% \usepackage[pagenumbers]{cvpr} % To force page numbers, e.g. for an arXiv version

% Import additional packages in the preamble file, before hyperref
%
% --- inline annotations
%
\newcommand{\red}[1]{{\color{red}#1}}
\newcommand{\todo}[1]{{\color{red}#1}}
\newcommand{\TODO}[1]{\textbf{\color{red}[TODO: #1]}}
% --- disable by uncommenting  
% \renewcommand{\TODO}[1]{}
% \renewcommand{\todo}[1]{#1}



\newcommand{\VLM}{LVLM\xspace} 
\newcommand{\ours}{PeKit\xspace}
\newcommand{\yollava}{Yo’LLaVA\xspace}

\newcommand{\thisismy}{This-Is-My-Img\xspace}
\newcommand{\myparagraph}[1]{\noindent\textbf{#1}}
\newcommand{\vdoro}[1]{{\color[rgb]{0.4, 0.18, 0.78} {[V] #1}}}
% --- disable by uncommenting  
% \renewcommand{\TODO}[1]{}
% \renewcommand{\todo}[1]{#1}
\usepackage{slashbox}
% Vectors
\newcommand{\bB}{\mathcal{B}}
\newcommand{\bw}{\mathbf{w}}
\newcommand{\bs}{\mathbf{s}}
\newcommand{\bo}{\mathbf{o}}
\newcommand{\bn}{\mathbf{n}}
\newcommand{\bc}{\mathbf{c}}
\newcommand{\bp}{\mathbf{p}}
\newcommand{\bS}{\mathbf{S}}
\newcommand{\bk}{\mathbf{k}}
\newcommand{\bmu}{\boldsymbol{\mu}}
\newcommand{\bx}{\mathbf{x}}
\newcommand{\bg}{\mathbf{g}}
\newcommand{\be}{\mathbf{e}}
\newcommand{\bX}{\mathbf{X}}
\newcommand{\by}{\mathbf{y}}
\newcommand{\bv}{\mathbf{v}}
\newcommand{\bz}{\mathbf{z}}
\newcommand{\bq}{\mathbf{q}}
\newcommand{\bff}{\mathbf{f}}
\newcommand{\bu}{\mathbf{u}}
\newcommand{\bh}{\mathbf{h}}
\newcommand{\bb}{\mathbf{b}}

\newcommand{\rone}{\textcolor{green}{R1}}
\newcommand{\rtwo}{\textcolor{orange}{R2}}
\newcommand{\rthree}{\textcolor{red}{R3}}
\usepackage{amsmath}
%\usepackage{arydshln}
\DeclareMathOperator{\similarity}{sim}
\DeclareMathOperator{\AvgPool}{AvgPool}

\newcommand{\argmax}{\mathop{\mathrm{argmax}}}     



% It is strongly recommended to use hyperref, especially for the review version.
% hyperref with option pagebackref eases the reviewers' job.
% Please disable hyperref *only* if you encounter grave issues, 
% e.g. with the file validation for the camera-ready version.
%
% If you comment hyperref and then uncomment it, you should delete *.aux before re-running LaTeX.
% (Or just hit 'q' on the first LaTeX run, let it finish, and you should be clear).
\definecolor{cvprblue}{rgb}{0.21,0.49,0.74}
\usepackage[pagebackref,breaklinks,colorlinks,allcolors=cvprblue]{hyperref}
\usepackage{makecell}
\usepackage{multirow}

\usepackage[hypcap=false]{caption}

%%%%%%%%% PAPER ID  - PLEASE UPDATE
\def\paperID{15096} % *** Enter the Paper ID here
\def\confName{CVPR}
\def\confYear{2025}

%%%%%%%%% TITLE - PLEASE UPDATE
\title{MSVCOD:A Large-Scale Multi-Scene Dataset for Video \\ Camouflage Object Detection}

%%%%%%%%% AUTHORS - PLEASE UPDATE
% \author{First Author\\
% Institution1\\
% Institution1 address\\
% {\tt\small firstauthor@i1.org}
% % For a paper whose authors are all at the same institution,
% % omit the following lines up until the closing ``}''.
% % Additional authors and addresses can be added with ``\and'',
% % just like the second author.
% % To save space, use either the email address or home page, not both
% \and
% % Second Author\\
% % Institution2\\
% % First line of institution2 address\\
% % {\tt\small secondauthor@i2.org}
% % }

\author{
	Shuyong Gao\textsuperscript{\rm 1,2}, Yu'ang Feng\textsuperscript{\rm 1}, Qishan Wang\textsuperscript{\rm 1},\\
	Lingyi Hong\textsuperscript{\rm 1}, Xinyu Zhou\textsuperscript{\rm 1}, Liu Fei\textsuperscript{\rm 2}, Yan Wang\textsuperscript{\rm 1}, Wenqiang Zhang\textsuperscript{\rm 1} \\ 
    \textsuperscript{\rm 1} Fudan University, Shanghai, China \\
    \textsuperscript{\rm 2} Keenon Robotics Co. Ltd, Shanghai, China
    }


\begin{document}



% \maketitle
% \begin{abstract}


The choice of representation for geographic location significantly impacts the accuracy of models for a broad range of geospatial tasks, including fine-grained species classification, population density estimation, and biome classification. Recent works like SatCLIP and GeoCLIP learn such representations by contrastively aligning geolocation with co-located images. While these methods work exceptionally well, in this paper, we posit that the current training strategies fail to fully capture the important visual features. We provide an information theoretic perspective on why the resulting embeddings from these methods discard crucial visual information that is important for many downstream tasks. To solve this problem, we propose a novel retrieval-augmented strategy called RANGE. We build our method on the intuition that the visual features of a location can be estimated by combining the visual features from multiple similar-looking locations. We evaluate our method across a wide variety of tasks. Our results show that RANGE outperforms the existing state-of-the-art models with significant margins in most tasks. We show gains of up to 13.1\% on classification tasks and 0.145 $R^2$ on regression tasks. All our code and models will be made available at: \href{https://github.com/mvrl/RANGE}{https://github.com/mvrl/RANGE}.

\end{abstract}

    
% \section{Introduction}
Backdoor attacks pose a concealed yet profound security risk to machine learning (ML) models, for which the adversaries can inject a stealth backdoor into the model during training, enabling them to illicitly control the model's output upon encountering predefined inputs. These attacks can even occur without the knowledge of developers or end-users, thereby undermining the trust in ML systems. As ML becomes more deeply embedded in critical sectors like finance, healthcare, and autonomous driving \citep{he2016deep, liu2020computing, tournier2019mrtrix3, adjabi2020past}, the potential damage from backdoor attacks grows, underscoring the emergency for developing robust defense mechanisms against backdoor attacks.

To address the threat of backdoor attacks, researchers have developed a variety of strategies \cite{liu2018fine,wu2021adversarial,wang2019neural,zeng2022adversarial,zhu2023neural,Zhu_2023_ICCV, wei2024shared,wei2024d3}, aimed at purifying backdoors within victim models. These methods are designed to integrate with current deployment workflows seamlessly and have demonstrated significant success in mitigating the effects of backdoor triggers \cite{wubackdoorbench, wu2023defenses, wu2024backdoorbench,dunnett2024countering}.  However, most state-of-the-art (SOTA) backdoor purification methods operate under the assumption that a small clean dataset, often referred to as \textbf{auxiliary dataset}, is available for purification. Such an assumption poses practical challenges, especially in scenarios where data is scarce. To tackle this challenge, efforts have been made to reduce the size of the required auxiliary dataset~\cite{chai2022oneshot,li2023reconstructive, Zhu_2023_ICCV} and even explore dataset-free purification techniques~\cite{zheng2022data,hong2023revisiting,lin2024fusing}. Although these approaches offer some improvements, recent evaluations \cite{dunnett2024countering, wu2024backdoorbench} continue to highlight the importance of sufficient auxiliary data for achieving robust defenses against backdoor attacks.

While significant progress has been made in reducing the size of auxiliary datasets, an equally critical yet underexplored question remains: \emph{how does the nature of the auxiliary dataset affect purification effectiveness?} In  real-world  applications, auxiliary datasets can vary widely, encompassing in-distribution data, synthetic data, or external data from different sources. Understanding how each type of auxiliary dataset influences the purification effectiveness is vital for selecting or constructing the most suitable auxiliary dataset and the corresponding technique. For instance, when multiple datasets are available, understanding how different datasets contribute to purification can guide defenders in selecting or crafting the most appropriate dataset. Conversely, when only limited auxiliary data is accessible, knowing which purification technique works best under those constraints is critical. Therefore, there is an urgent need for a thorough investigation into the impact of auxiliary datasets on purification effectiveness to guide defenders in  enhancing the security of ML systems. 

In this paper, we systematically investigate the critical role of auxiliary datasets in backdoor purification, aiming to bridge the gap between idealized and practical purification scenarios.  Specifically, we first construct a diverse set of auxiliary datasets to emulate real-world conditions, as summarized in Table~\ref{overall}. These datasets include in-distribution data, synthetic data, and external data from other sources. Through an evaluation of SOTA backdoor purification methods across these datasets, we uncover several critical insights: \textbf{1)} In-distribution datasets, particularly those carefully filtered from the original training data of the victim model, effectively preserve the model’s utility for its intended tasks but may fall short in eliminating backdoors. \textbf{2)} Incorporating OOD datasets can help the model forget backdoors but also bring the risk of forgetting critical learned knowledge, significantly degrading its overall performance. Building on these findings, we propose Guided Input Calibration (GIC), a novel technique that enhances backdoor purification by adaptively transforming auxiliary data to better align with the victim model’s learned representations. By leveraging the victim model itself to guide this transformation, GIC optimizes the purification process, striking a balance between preserving model utility and mitigating backdoor threats. Extensive experiments demonstrate that GIC significantly improves the effectiveness of backdoor purification across diverse auxiliary datasets, providing a practical and robust defense solution.

Our main contributions are threefold:
\textbf{1) Impact analysis of auxiliary datasets:} We take the \textbf{first step}  in systematically investigating how different types of auxiliary datasets influence backdoor purification effectiveness. Our findings provide novel insights and serve as a foundation for future research on optimizing dataset selection and construction for enhanced backdoor defense.
%
\textbf{2) Compilation and evaluation of diverse auxiliary datasets:}  We have compiled and rigorously evaluated a diverse set of auxiliary datasets using SOTA purification methods, making our datasets and code publicly available to facilitate and support future research on practical backdoor defense strategies.
%
\textbf{3) Introduction of GIC:} We introduce GIC, the \textbf{first} dedicated solution designed to align auxiliary datasets with the model’s learned representations, significantly enhancing backdoor mitigation across various dataset types. Our approach sets a new benchmark for practical and effective backdoor defense.



% \section{Related work}
\label{sec:formatting}

\subsection{Text-to-Video Generation}

T2V generation has made notable progress, evolving from early GAN-based models \cite{saito2017temporal,tulyakov2018mocogan,fu2023tell,li2018video,wu2022nuwa,yu2022generating} to newer transformer \cite{yan2021videogpt,arnab2021vivit,esser2021taming,ramesh2021zero,yu2022scaling} and diffusion models \cite{kirkpatrick2017overcoming,sohl2015deep,song2020denoising,zhang2022gddim}. Early efforts like MoCoGAN~\cite{tulyakov2018mocogan} focused on short video clips but faced issues with stability and coherence. The introduction of transformers improved sequential data handling, enhancing video generation, while diffusion models further improved video quality by progressively denoising the input. 
Despite these advances, T2V models still struggle to reflect human preferences, with the generated videos generally lacking aesthetic quality. Additionally, the scarcity of paired video preference data hinders effective model training and may lead to insufficient flexibility and poor quality in the generated videos.


\subsection{RLHF}

\iffalse
Aligning LLMs \cite{dai1901transformer,radford2019language,zhang2023opt} typically involves two steps: supervised fine-tuning followed by Reinforcement Learning with Human Feedback (RLHF) \cite{gao2023scaling,stiennon2020learning,rafailov2024direct}. Although effective, RLHF is computationally expensive and can lead to issues like reward hacking. Methods like DPO have streamlined alignment by leveraging feedback data directly, improving efficiency.

In contrast, diffusion model alignment is still evolving, focusing mainly on enhancing visual quality through curated datasets. Techniques like DOODL \cite{wallace2023end} and AlignProp \cite{prabhudesai2023aligning} target aesthetic improvements but face challenges with complex tasks such as text-image alignment. Reinforcement learning methods like DPOK \cite{fan2024reinforcement} and DDPO \cite{black2023training} improve reward optimization but struggle with scalability. DPO-SDXL integrates DPO into T2I generation, boosting both alignment and aesthetics.

However, aligning video generation remains a largely unaddressed challenge, especially when dealing with motion consistency and semantic coherence across frames.
\fi

RLHF \cite{gao2023scaling,stiennon2020learning,rafailov2024direct} is a method that utilizes human feedback to guide machine learning models. Early RLHF algorithms, such as DDPG~\cite{lillicrap2015continuous} and PPO~\cite{schulman2017proximal}, typically relied on complex reward models to quantify human feedback. These reward models require a large amount of annotated data and face challenges during tuning. As research has progressed, more efficient preference learning methods have emerged, among which DPO has become a new framework. DPO does not depend on a separate reward model; instead, it obtains human preferences through pairwise comparisons and directly optimizes these preferences. This shift not only simplifies the application of RLHF but also enhances the alignment of models with human values. Furthermore, DPO has been successfully introduced into T2I tasks~\cite{wallace2024diffusion,yang2024using}, providing new insights for generative models in addressing the alignment of human preferences and showcasing DPO's potential in the field of AIGC~\cite{shi2024instantbooth,
qing2024hierarchical,menapace2024snap,koley2024s}. However, there remains a gap in current research regarding the application of DPO strategies to T2V tasks. Effectively integrating DPO into T2V tasks presents a challenging endeavor.


% \section{Preliminary}
\label{sec:preliminary}
In this section, we first introduce the mathematical formulation of flow-based text-to-image generative models~\cite{Xingchao_2022,Lipman_2022}, which forms the foundation of modern T2I systems~\cite{sd3,sdxl,imagen3,imagen}. We then describe classifier-free guidance~\cite{ho2022classifier}, a key technique to control the generation process through text conditioning.

\subsection{Flow-based text-to-image generative models}
In state-of-the-art T2I models~\cite{sd3}, the image generation process is modeled by learning, through a neural network, a flow $\psi$ that generates a probability path $(p_t)_{0\le t\le 1}$ bridging the source distribution $p_0$ and the target distribution $p_1$ ~\cite{Xingchao_2022,Lipman_2022}. This framework encompasses diffusion models~\cite{sohl2015deep,ddpm} as a special case. In particular, a commonly used formulation sets a Gaussian distribution as the source: $p_0 = \mathcal{N}(\mathbf{0}, \mathbf{I})$ and a delta distribution centered on a sample $\mathbf{x}_1$ from the data distribution $q$ as the target: $p_1 = \delta_{\mathbf{x}_1}$.
Then, it incorporates an affine conditional flow $\psi_t(\mathbf{x} | \mathbf{x}_1) = a_t \mathbf{x}_1 + b_t \mathbf{x}$ with the boundary condition $(a_0, b_0) = (0, 1),\ (a_1, b_1) = (1, 0)$ to bridge them. The neural network typically approximates quantities such as velocity fields, $x_0$ prediction or $x_1$ prediction. In this modeling, these quantities can be viewed as affine transformations of the marginal probability path score $\nabla_{\mathbf{x}} \log p_t(\mathbf{x})$.

\subsection{Classifier-free guidance in flow-based models}
Classifier-free guidance~\cite{ho2022classifier} is a method for sampling from a model conditioned by a text input $\mathbf{y}$ by guiding an unconditional image generation model modeled using the score $\nabla_{\mathbf{x}} \log p_t(\mathbf{x})$. This enables the sampling from
\[
q_w(\mathbf{x}, \mathbf{y}) \propto q(\mathbf{x})q(\mathbf{y}|\mathbf{x})^w \propto q(\mathbf{x})^{1-w}q(\mathbf{x}|\mathbf{y})^w
\]
where $w \in \mathbb{R}$ is the guidance scale typically used with $w > 1$. The score satisfies
\[
\nabla_{\mathbf{x}} \log q_w(\mathbf{x}, \mathbf{y}) = (1-w)\nabla_{\mathbf{x}} \log q(\mathbf{x}) + w\nabla_{\mathbf{x}} \log q(\mathbf{x}|\mathbf{y})
\]
so by training the network to learn both the unconditional score $\nabla_{\mathbf{x}} \log q(\mathbf{x})$ and conditional score $\nabla_{\mathbf{x}} \log q(\mathbf{x}|\mathbf{y})$, flexible sampling from the conditional distribution can be achieved through a weighted sum of the network outputs.


\twocolumn[{
\vspace{-0.3cm}
\maketitle 
\vspace{-0.6cm}
\renewcommand\twocolumn[1][]{#1}%
\vspace{-0.4cm}
\begin{center}
\centering
\includegraphics[width=\textwidth]{./images/overview.png}
\vspace{-0.7cm}
\captionof{figure}[An overview of MSVCOD]{An overview of MSVCOD composed of video frames from seven scenarios and four types of object.}\label{fig:fig1}
\vspace{-0.1cm}
\end{center}}]


\begin{abstract}
  Video Camouflaged Object Detection (VCOD) is a challenging task which aims to identify objects that seamlessly concealed within the background in videos. The dynamic properties of video enable detection of camouflaged objects through motion cues or varied perspectives. Previous VCOD datasets primarily contain animal objects, limiting the scope of research to wildlife scenarios. However, the applications of VCOD extend beyond wildlife and have significant implications in security, art, and medical fields. Addressing this problem, we construct a new large-scale multi-domain VCOD dataset MSVCOD. To achieve high-quality annotations, we design a semi-automatic iterative annotation pipeline that reduces costs while maintaining annotation accuracy. Our MSVCOD is the largest VCOD dataset to date, introducing multiple object categories including human, animal, medical, and vehicle objects for the first time, while also expanding background diversity across various environments. This expanded scope increases the practical applicability of the VCOD task in camouflaged object detection. Alongside this dataset, we introduce a one-steam video camouflage object detection model that performs both feature extraction and information fusion without additional motion feature fusion modules. Our framework achieves state-of-the-art results on the existing VCOD animal dataset and the proposed MSVCOD. The dataset and code will be made publicly available.
\end{abstract}


\section{Introduction}

\begin{figure*}
    \centering
    \includegraphics[width=1\linewidth]{./images/big_small_ins.png}
    \caption{Detail illustration of the data collection and annotation pipline.}
    \label{fig:big_small}
    \vspace{-0.4cm}
\end{figure*}


Camouflaged Object Detection (COD) \cite{fan2020camouflaged} focuses on identifying and segmenting the hidden objects that closely resemble their backgrounds. Those camouflaged objects exhibit intricate visual patterns, edge breaking, texture similarities, and color matching, making them blend seamlessly into their surroundings and challenging to detect compared to traditional object detection methods \cite{Skelhorn2016-CODcognition, Zhao2019-TraditionalOD}. COD has applications in diverse fields such as medical image segmentation \cite{fan2020inf,FanDP2020-Pranet}, enemy detection in the battlefield \cite{LinCJ2019-Metaheuristic}, art \cite{Ge2018-Art,chu2010camouflage}, industry \cite{Zeng2022-DefectDetection}, scientific research \cite{Perez2012-SpeciesDiscovery}, and agriculture \cite{Rustia2020-Agriculture}. Given that motion and multi-angle visual information can help to detect camouflaged objects, Video Camouflage Object Detection (VCOD) was developed. VCOD, a sub-domain of COD, focuses on segmenting out camouflage targets in the videos by utilizing motion and multi-angle visual cues.


Although video can effectively reveal camouflage, VCOD datasets are relatively scarce due to the limited amount of video camouflage data and the time-consuming, labor-intensive nature of manual labeling. In 2016, Pia Bideau and Erik Learned-Miller introduced the first VCOD dataset, the Camouflaged Animal Dataset (CAD) \cite{bideau2016s}, which includes 9 short clips of camouflaged animals from YouTube videos and provides 191 frames with manual ground truth mask annotations. In 2020, Hala Lamdouar \emph {et al.} proposed the first large-scale VCOD dataset, Moving Camouflaged Animals (MoCA) \cite{lamdouar2020betrayed}, which consists of 141 video clips featuring 67 animals and provides bounding box annotations. On this basis, Cheng \emph {et al.} reorganized and annotated MoCA, resulting in 87 video sequences with ground truth mask annotations every 5 frames, totaling 5,750 annotations. This reorganized dataset, called MoCA-Mask \cite{cheng2022implicit}, is currently the only large-scale dataset in the field.Based on this, Cheng \emph {et al.} reorganized and annotated MoCA, creating MoCA-Mask \cite{cheng2022implicit}, a large-scale dataset with 87 video sequences and 5,750 ground truth mask annotations, which is currently the large-scale dataset in the field.


\begin{table}[ht]
  \centering
  % \fontsize{12}{15}\selectfont  % 设置表格字体为12pt,行距为15pt
  \Large
    \caption{Comparison of MVCOD with other video camouflage object detection dataset benchmark.(Img. = Number of frames in the dataset; Obj. = Types of object; BBox. = Bounding box level; Pix. = Pixel level; Ins. = Instance level; Cate. = Category; Spi. means explicitly splitting the Training and Testing Set)}
  \label{tab_comparison}
  \resizebox{0.47\textwidth}{!}{
  \begin{tabular}{lccc ccccc}
    \toprule
    % \Xhline{4\arrayrulewidth}
    %\multicolumn{2}{c}{Part}                   \\
    %\cmidrule(r){2-4}
    Dataset      & Clips &Img. &Obj. &BBox. &Pix. &Ins. &Cate. &Spi\\
    \midrule
    % CAMO &2019 & \textbackslash  &1250 & \textbackslash  \\
    % CHAMELEON \\
    % COD10K \\
    % NC4K \\
    CAD$_{2016}$  & 9 &191  &1 & &\checkmark & & &\\
    MoCA$_{2020}$  & 141 &7617  &1 &\checkmark & & & &\\
    MoCA-Mask$_{2022}$  & 87 &5750 &1 &\checkmark &\checkmark & &\checkmark & \checkmark \\
    \midrule
    Ours  &162 &9486 &4 &\checkmark &\checkmark &\checkmark &\checkmark &\checkmark  \\
    \bottomrule
    % \Xhline{4\arrayrulewidth}

  \end{tabular}
  }
\vspace{-0.3cm}
\end{table}

Current video camouflage datasets primarily focus on animal scenes, overlooking the diverse camouflage modalities human society. This limitation restricts the broader applicability of VCOD, particularly in fields such as medicine, security, search-and-rescue, and art. The lack of diverse datasets results in researchers not having enough data to train and test models on benchmark datasets. To enhance model generalization across scenes and objects, existing VCOD models \cite{lamdouar2020betrayed, zhang2024explicit} are often pretrained on static image datasets \cite{fan2020camouflaged}. To address this gap, we built a novel, large-scale multi-scene VCOD benchmark dataset. It includes 162 video clips across four object categories (human, animal, medical, and vehicle) and seven scenarios (aquatic, field, medical, art, jungle, desert, snowfield). Our dataset provides 6 frames of ground truth mask annotations per second, totaling 9,486 frame annotations, making it the largest VCOD dataset to date. Table \ref{tab_comparison} and Figure \ref{fig:statistic graphs} show the characteristics of our dataset.


% Some still-image-based methods detect camouflaged targets from still images by localizing them first and then refining them, or by combining multitasking. By combining simple images or feature amplification, some models have achieved large performance gains. However, this line of models focuses only on processing still images and cannot utilize the motion information of the video. 
% In order to utilize the motion characteristics, the. 
% \cite{bideau2016s} develop a method using a variety of motion models derived from dense optical flow. Also using optical flow information, \cite{lamdouar2020betrayed} propose a network for video registration and segmentation that utilizes optical flow and difference images to detect camouflaged objects. However, limited by the dataset, \cite{lamdouar2020betrayed} can only output detection results from the box level. Optical flow does provide motion information, but it can be disruptive to the model when the object is not moving, or when only the camera is moving, and the computational expense of optical flow is large. \cite{cheng2022implicit} propose a two-stage model that implicitly incorporates motion information in terms of long and short time, and predict pixel level masks. Further, explicitly handling motion cues, \cite{zhang2024explicit} propose a two stream model to estimate optical flow and segment camouflage objects. Different from the above methods, accompanied by the MVCOD dataset, we propose a single-stream camouflage object detection model without explicitly computing the optical flow information.

Some static image-based methods detect camouflaged objects from still images by first localizing and then refining \cite{FanDP2020-Pranet, fan2020camouflaged, FanDP2021-ConcealedOD}, sometimes in combination with multitasking \cite{zhai2021MGL, Lv2021-RankNet, he2023camouflaged}. Other models \cite{pang2022zoom, jia2022-CODSegMar, Xing2023-SARNet} achieve significant performance gains by incorporating simple images or feature amplification. However, these models focus solely on still images and cannot take advantage of motion information in videos. To address this, some models \cite{lamdouar2020betrayed, zhang2024explicit} explicitly integrate optical flow information for camouflaged object detection, drawing inspiration from the neighboring field of video salient object detection \cite{lan2022siamese, gao2022weakly, ji2021full}. SLT-Net \cite{cheng2022implicit} extracts image features from consecutive frames separately and then fuses short-term and long-term features to detect camouflaged objects.

The VCOD models discussed above all follow a two-stream architecture, where feature extraction and information fusion are performed separately. This approach is computationally expensive and often leads to poor performance due to the difficulty of adaptively extracting exploitable features. In contrast, inspired by video object tracking and video object segmentation \cite{cui2022mixformer, hong2023simulflow}, and in conjunction with MSVCOD, we propose a one-stream VCOD model. This model simultaneously extracts image features and motion information, eliminating the need for optical flow as input. At the decoding layer, we design a simple, fully-connected, UNet-like decoder that relies on linear adapter layer, without any unnecessary complexity, achieving state-of-the-art performance. Our main contributions are as follows:



\begin{itemize}
% \item We construct a novel multi-scene large-scale video camouflage object detection dataset MSVCOD. The dataset contains 216 clips, 11679 frames with 7 scenes and 4 major categories,introducing a large number of non-wildlife objects for the first time.and for the first time included objects other than animals in the dataset, especially a large number of humans. This helps to apply and research video camouflaged object detection in more scenarios.

\item  We design a semi-automatic iterative annotation pipeline and construct a novel, large-scale multi-scene video camouflage object detection dataset, MSVCOD. The dataset consists of 162 clips and 9,486 frames, covering 7 scenes across 4 major categories, and introduces a wide range of non-wildlife targets for the first time. It provides annotations at the box, mask, instance, and category levels.

\item We develop a simple, one-stream camouflage object detection model equipped with fully-connected UNet-like decoder, enabling simultaneous extraction of image features and fusion of motion features

% \item We have proposed a single-stream framework for implicitly processing motion information in VCOD. This framework, while cost-effective, achieves state-of-the-art performance on previous animal VCOD benchmarks. Combined with our dataset, it is significantly better than existing methods in comprehensive object tests.

\item Extensive experiments demonstrate that our proposed dataset enhances model performance and improves generalization across multiple scenarios. Additionally, numerous experiments show that our model significantly outperforms previous VCOD models.


\end{itemize}


% \begin{table}[ht]
%   \centering
%     \caption{Comparison of MVCOD with other video camouflage object detection dataset benchmark.(Obj. = types of object. Img. = Image. Cls. = Class. BBox. = Bounding box; Ins. = Instance; Cate. = Category. Spi. means explicitly splitting the Training and Testing Set)}
%   \label{tab_comparison}
%   \resizebox{0.5\textwidth}{!}{
%   \begin{tabular}{l|l|l|l|l|l |lllll}
%     \toprule
%     %\multicolumn{2}{c}{Part}                   \\
%     %\cmidrule(r){2-4}
%     Dataset     & year & clips &images &resolution &Obj. &BBox. &Pix. &Ins. &Cate. &Spi\\
%     \midrule
%     % CAMO &2019 & \textbackslash  &1250 & \textbackslash  \\
%     % CHAMELEON \\
%     % COD10K \\
%     % NC4K \\
%     CAD &2016 & 9 &191 &360P &1 & &\checkmark & & &\\
%     MoCA &2020 & 141 &7617 &720P &1 &\checkmark & & & &\\
%     MoCA-Mask &2022 & 87 &5750 &720P &1 &\checkmark &\checkmark & &\checkmark & \checkmark \\
%     \midrule
%     Ours &2024 &162 &9486 &1080P &4 &\checkmark &\checkmark &\checkmark &\checkmark &\checkmark  \\
%     \bottomrule
%   \end{tabular}
%   }

% \end{table}

% \renewcommand{\arraystretch}{1.5}  % 1.5 倍行高(默认是1)










\section{Related Work}

\subsection{Dataset for Camouflage Object Detection}

% The rapid development of the field of camouflage target detection in recent years is partly due to the enhancement of deep learning algorithms, and partly attributed to the emergence of large-scale camouflage object detection datasets, which are the basis for the training and benchmarking the COD models. Data sets for static camouflage target detection include CAMO \cite{camo}, CHAMELEON \cite{chameleon}, COD10K \cite{fan2020camouflaged}, and NC4K \cite{Lv2021-RankNet}. The CAMO \cite{camo} contain 1,000 traing images and 250 test images, and covers a variety of challenging scenarios, including camouflaging animals and artificial camouflage. CHAMELEON \cite{chameleon} contains 76 images, focusing on camouflage animals. COD10K \cite{fan2020camouflaged} is the largest camouflage target dataset, and contains around 7000 camouflage images, and is notably often used in pre-training of VCOD models. On the VCOD dataset, the CAD dataset \cite{bideau2016s} is a small VCOD dataset, comprising 9 short sequences from YouTube with hand-labeled ground truth masks every 5 frames. The original Moving Camouflaged Animals (MoCA) dataset \cite{lamdouar2020betrayed} includes 37K frames from 141 YouTube videos at 720 × 1280 resolution and 24 fps, featuring 67 animal species in natural scenes, though not all are camouflaged. The ground truth in MoCA is provided as bounding boxes, which makes it difficult to evaluate the segmentation performance of the VCOD. \cite{cheng2022implicit} reorganize this dataset into MoCA-Mask, establishing a benchmark with more comprehensive evaluation criteria. Nevertheless, all of the VCOD datasets are animal-specific, different from these datasets, we propose MSVCOD dataset for multiple types of objects and scenarios.


The rapid development of camouflage object detection in recent years is partly due to advancements in deep learning algorithms and the emergence of large-scale camouflage object detection (COD) datasets, which form the foundation for training and benchmarking COD models. Datasets for static camouflage object detection include CAMO \cite{camo}, CHAMELEON \cite{chameleon}, COD10K \cite{fan2020camouflaged}, and NC4K \cite{Lv2021-RankNet}. The CAMO dataset \cite{camo} contains 1,000 training images and 250 test images, covering various challenging scenarios, including camouflaged animals and artificial camouflage. CHAMELEON \cite{chameleon} consists of 76 images, focusing on camouflaged animals. COD10K \cite{fan2020camouflaged}, the largest camouflage object dataset, includes around 10000 images and is frequently used for pre-training VCOD models.

For VCOD, the CAD dataset \cite{bideau2016s} is a small dataset, containing 9 short sequences from YouTube with hand-labeled ground truth masks every 5 frames. The original Moving Camouflaged Animals (MoCA) dataset \cite{lamdouar2020betrayed} includes 37K frames from 141 YouTube videos at 720 × 1280 resolution and 24 fps, featuring 67 animal species in natural scenes (though not all are camouflaged). MoCA provides ground truth as bounding boxes, making segmentation evaluation difficult. Cheng \emph {et al.} \cite{cheng2022implicit} reorganized MoCA into MoCA-Mask, establishing a benchmark with more comprehensive evaluation criteria. However, all existing VCOD datasets are animal-specific. In contrast, we propose the MSVCOD dataset, which covers multiple object types and scenarios.



\subsection{Image-based Camouflage Object Detection}

% Image-based COD aims to identify camouflage objects from still images. Those early methods \cite{pan2011study, liu2012foreground} leverage hand-designed features to identify camouflage objects from backgrounds environment. Then with the development of deep learning and the proposal of large-scale COD datasets \cite{camo,chameleon,fan2020camouflaged}, COD has been rapidly developed. Some methods \cite{FanDP2020-Pranet,fan2020camouflaged,FanDP2021-ConcealedOD} develop a coarse-to-fine approach to recognize the camouflages objects progressively. To further enhance performance, some studies \cite{zhai2021MGL,Lv2021-RankNet,he2023camouflaged} incorporate auxiliary tasks into a joint learning framework. MGL \cite{zhai2021MGL} combine classification or boundary detection tasks with COD, and LSR \cite{Lv2021-RankNet} propose a multi-task framework for COD to simultaneously localize, segment,and rank camouflaged objects. Combined with edge extraction, FEDER \cite{he2023camouflaged} propose a network to decompose features into
% different frequency bands with learnable wavelets. Some works \cite{pang2022zoom,jia2022-CODSegMar,Xing2023-SARNet} find that image or feature amplification can be useful to recognize camouflaged objects. ZoomNet \cite{pang2022zoom} employs a zoom-in-and-out technique to handle appearance features across three different scales. SegMaR \cite{jia2022-CODSegMar} utilizes a segment, magnify, and reiterate method in a multistage, coarse-to-fine process, effectively replicating human behavior in analyzing complex situations. SARNet \cite{Xing2023-SARNet} introduces a search, amplify, and recognize framework by enhancing the resolution of the target region to detect camouflaged objects. Since the above model is only focused on still images, 
% Since the above model focuses on still images, it cannot utilize motion information and perform poorly on video camouflage object detection tasks. 



Image-based COD aims to identify camouflaged objects in still images. Early methods \cite{pan2011study, liu2012foreground} relied on hand-designed features to distinguish camouflaged objects from their background. With the development of deep learning and large-scale COD datasets \cite{camo, chameleon, fan2020camouflaged}, the field has advanced rapidly. Some methods \cite{FanDP2020-Pranet, fan2020camouflaged, FanDP2021-ConcealedOD} use a coarse-to-fine approach to progressively identify camouflaged objects. To further enhance performance, some studies \cite{zhai2021MGL, Lv2021-RankNet, he2023camouflaged} integrate auxiliary tasks within a joint learning framework. Additionally, some works \cite{pang2022zoom, jia2022-CODSegMar, Xing2023-SARNet} explore image or feature amplification to improve camouflage recognition. For example, ZoomNet \cite{pang2022zoom} employs a zoom-in-and-out technique to process appearance features across three different scales. Other methods \cite{lin2023frequency, sun2025frequency, zhang2024frequency} attempt to segment camouflaged objects through frequency analysis. However, since these models are designed for still images, they cannot utilize motion information, which limits their performance in video camouflage object detection tasks.





\subsection{Video Camouflage Object Detection}
% Video camouflage object detection is a recently developed research field. Just like other video segment tasks (e.g., VSOD \cite{lan2022siamese,gao2022weakly,ji2021full} and VOS \cite{hong2023simulflow, miao2024region,seong2022video,yang2022decoupling}), motion cues are considered to be an effective means to break the camouflage of objects. Bideau et al.\cite{bideau2016s} use a variety of motion models derived from dense optical flow to detect camouflage object. Also using optical flow information, Lamdouar et al. \cite{lamdouar2020betrayed} propose a network for video registration and segmentation that utilizes optical flow and difference images to detect camouflaged objects. However, Lamdouar et al. \cite{lamdouar2020betrayed} can only output detection results at the bounding box level, due to dataset limitations. Optical flow \cite{deng2023explicit, teed2020raft, dosovitskiy2015flownet,sun2018pwc} provide motion information, but it can be disruptive to the model when the object is not moving, or when only the camera is moving, and the computational expense of optical flow is large. Cheng et al. \cite{cheng2022implicit} propose a two-stage model that implicitly incorporates motion information in terms of long and short time, and predict pixel level masks. Further, explicitly handling motion cues, Zhang et al\cite{zhang2024explicit} propose a two stream model to estimate optical flow and segment camouflage objects. Different from the above methods, accompanied by the MSVCOD dataset, we propose a single-stream camouflage object detection model without explicitly computing the optical flow information.


Video camouflage object detection is a recently developed research field. Similar to other video segmentation tasks (\emph {e.g.}, VSOD \cite{lan2022siamese, gao2022weakly, ji2021full} and VOS \cite{hong2023simulflow, miao2024region, seong2022video, yang2022decoupling}), motion cues are considered an effective means to break the camouflage of objects. Bideau \emph {et al.} \cite{bideau2016s} use various motion models derived from dense optical flow to detect camouflaged objects. Lamdouar \emph {et al.} \cite{lamdouar2020betrayed}, also using optical flow, propose a network for video registration and segmentation that utilizes optical flow and difference images to detect camouflaged objects. However, their model can only output detection results at the bounding box level due to dataset limitations.

While optical flow \cite{deng2023explicit, teed2020raft, dosovitskiy2015flownet, sun2018pwc} provides motion information, it can be problematic when the object is stationary or only the camera is moving. Additionally, the computational cost of optical flow is high. Cheng \emph {et al.} \cite{cheng2022implicit} propose a two-stage model that implicitly incorporates motion information over long and short time intervals and predicts pixel-level masks. Further, explicitly handling motion cues, Zhang \emph {et al.} \cite{zhang2024explicit} introduce a two-stream model to estimate optical flow and segment camouflaged objects. In contrast to these approaches, and with the support of the MSVCOD dataset, We propose a one-stream camouflage object detection model that does not require explicitly computing optical flow as input.



\section{MSVCOD Dataset}



\begin{figure*}
    \centering
    \includegraphics[width=0.8\linewidth]{./images/collect_pip.jpg}
    \caption{Detail illustration of the data collection and annotation pipline.}
    \label{fig:annotation pipline}

\vspace{-0.3cm}
\end{figure*}


\subsection{Dataset Construction}           
% \textbf{Dataset Design and Data Collection.} To address the problem that the current VCOD dataset is only for animal camouflage, with fewer scenes and less data, MSVCOD aims to establish a novel VCOD dataset for training and evaluating robust VOS models. Our primary focus is on camouflaged object segmentation in challenging and complex scenes that include various camouflage objects. Therefore, we set several rules during the data collection and construction process to ensure the creation of a high-quality VCOD dataset. The requirements are summarized as follows:

\textbf{Dataset Design and Data Collection.} To address the limitation that current VCOD datasets focus primarily on animal camouflage, with fewer scenes and less data, MSVCOD aims to establish a novel VCOD dataset for training and evaluating robust VCOD models. Our primary focus is on camouflaged object detection in challenging and complex scenes that feature a variety of camouflage objects. To ensure the creation of a high-quality VCOD dataset, we set several guidelines during the data collection and construction process. These requirements are summarized as follows:


\textbf{R1:} Camouflage objects in the videos should be well-camouflaged. Specifically, the objects should be similar to the background and not be easily recognizable by annotators at first glance. The objects can be one or multiple objects.

\textbf{R2:} The collected videos should cover a variety of scenes, including underwater, land, desert, jungle, and other diverse environments.

\textbf{R3:} The dataset should contain various types of targets, including animals, humans, and vehicles.

\textbf{R4:} As shown in Figure \ref{fig:big_small}, targets in the dataset should vary in size and shape, including large, small, elongated targets, or those with gradually changing scales.

\textbf{R5:} The dataset should include various motion patterns, such as camera motion, object motion, or simultaneous motion of both the object and the camera.

\textbf{R6:} The dataset should be large-scale and densely annotated with high-quality data. The scale and quality of the dataset will ensure its long-term utility. Therefore, MSVCOD includes a substantial number of video clips. Given the time-consuming nature of densely annotating high-quality camouflage images, we have developed a semi-automatic iterative annotation method. The data collection process is illustrated in Figure \ref{fig:annotation pipline} (a).

% To collect the data meeting the above requirements, we searched for 227 videos containing camouflaged objects. Based on the dataset design rules, we carefully select seven scenarios that include four types of targets: animal, human, medical target, and vehicle. For the collected long-duration videos, we divided them into multiple video clips, similar to previous work\cite{wang2022ferv39k,ding2023mose}. The same object may appear in multiple video clips.

% \textcolor{red}{Seven annotators participated in the annotation task, but only three annotators (researchers working on still-image camouflage object detection or video camouflage object detection) selected the available camouflage video clips. The selection process was also relatively simple, based on their existing experience with camouflage object detection, and for each video clip, clips that two or more people voted that they needed to be eliminated were removed. In fact, there is no standardized criterion for the determination of artifacts, nor was there a clear criterion during the previous construction of the COD dataset or VCOD dataset.}

To collect the data meeting the above requirements, we searched for 227 videos containing camouflaged objects and selected seven scenarios with four types of targets: animal, human, medical, and vehicle. The long-duration videos were divided into multiple clips, as in previous work \cite{wang2022ferv39k,ding2023mose}, and the same object potentially appeared in multiple clips. Seven annotators participated in the task, with three experts in camouflage object detection selecting the relevant video clips based on their experience.

\begin{table}[h!]
\centering
\caption{Statistical table of three typical movement patterns.}
\begin{tabular}{lccc}
\toprule
\textbf{Name} & \textbf{Train} & \textbf{Test} & \textbf{Total} \\
\midrule
Object Motion         & 53 & 20 & 73 \\
Camera Motion         & 22 & 7  & 29 \\
Simultaneous Motion   & 46 & 14 & 60 \\
\bottomrule
\end{tabular}
\label{tab:motion patterns}
\end{table}

In the videos, if camouflaged objects become visible due to changes in the background or the objects themselves, we segment and exclude the corresponding clips, as referencing visible parts would reduce the level of camouflage. Finally, considering both video quality and segmentation difficulty, we selected 162 video clips to form MSVCOD.

To comply with \textbf{R4}, We manually selected video clips that contain targets with scale variations and size transformations. For example, objects occupying a larger portion of the frame include artist and octopus; smaller objects include humans in outdoor environments and various insects; and elongated objects include snake and eel. The scatterplot in Figure \ref{fig:msvcod_vs_moca} visually demonstrates the large scale distribution of our dataset. \textbf{R5} requires a variety of motion patterns. Here are some examples. \textbf{Object motion}: In artificial environments, the camera remains stationary while the object moves. \textbf{Camera motion}: In underwater environments, it is sometimes challenging to stabilize the filming equipment. \textbf{Simultaneous motion}: When filming wildlife in the field, the camera tracks the animal’s movement while it moves.
More visual examples are provided in the supplementary material. Additionally, we performed statistics on the video clips across different motion models, as shown in table \ref{tab:motion patterns}.






\textbf{Annotation Method.} To obtain accurate pixel-level annotations for video clips, a purely manual annotation method is feasible when the dataset is small or the objects are easy to annotate. However, with 9.5K frames in this dataset, manually annotating such a large number of camouflage images frame by frame is costly and time-consuming. To address this, we designed a semi-automatic iterative annotation pipeline to progressively improve annotation quality, as shown in Figure \ref{fig:annotation pipline} (b).


\textbf{Step 1: Selection and Segmentation of Representative Frames.} In the automatic annotation process, we utilize semi-supervised video object segmentation algorithms (SwinB-DeAOT-L \cite{yang2022decoupling}) to generate pseudo-labels. The semi-VOS model uses the first frame and its corresponding mask as a reference to identify objects in subsequent frames. Unlike VCOD, where camouflaged objects may not appear in the first frame, the semi-VOS model assumes that objects are already segmented in the first frame. To adapt VOS models for VCOD, we manually select representative frames from each video clip to serve as the reference frame. If the camouflaged object appears in the first frame, we use it as the reference. If the camouflaged object does not appear in the first frame, we choose the frame closest to the first frame that contains the camouflage object as the reference.

% --------------SwinB-DeAOT-L 输入512
\textbf{Step 2: Mask Forward Propagation and 1 FPS Manual Correction.} To automatically generate pseudo-labels based on the manual annotations, we use the VOS model (SwinB-DeAOT-L \cite{yang2022decoupling}) for pseudo-label generation. All input frames are resized to 512 pixels. The VOS model then outputs pseudo-labels for the unannotated frames in each video clip. Following this, we manually correct one frame every six frames (corresponding to our sampling rate) by reviewing and refining the generated pseudo-labels.

\textbf{Step 3: Mask Bidirectional Propagation.} Using the manually corrected mask as the reference frame, we apply the DeATO model \cite{yang2022decoupling} in both the forward and backward directions. As a result, each intermediate frame receives two pseudo-labels. These two pseudo-labels are then combined using the AND and OR operations, generating a total of four pseudo-labels, as shown in Figure \ref{fig:annotation pipline} (b) Step 3. Finally, the pseudo-labels are converted into polygon masks. Instead of selecting the consistency region as the pseudo-label, we allow the annotator to choose the most appropriate mask for annotation from the four polygon masks generated by the pseudo-labels.

\textbf{Step 4: Manual Correction.} Since the pseudo-labels generated in Step 3 may still contain errors, annotators iteratively refine these labels until they meet our quality standards. A total of seven people were involved in labeling the dataset: three were researchers specializing in camouflage object detection (including both video and still image camouflage detection), while the remaining four were not. In this step, annotators perform two main tasks: 1) Compare multiple frames before and after the current frame to identify inaccuracies or errors in labeling, such as missing objects, incorrect object boundaries, background incorrectly labeled as the object, or newly appearing object that have not been labeled in cases with multiple camouflage objects. 2) Manually correct these errors.


\subsection{Dataset Statistics and Characteristics}



\begin{figure}
    \centering
    \includegraphics[width=1\linewidth]{./images/statistic_graph.jpg}
    \caption{Statistical charts of the MSVCOD. Figure (a) represents the percentage of occurrence of the 4 types of targets in each scenario.The inner circle of (b) indicates the number of clips for the seven scenes, and the outer circle indicates the number of frames in each scene. (c) shows the word cloud of MSVCOD.}
    \label{fig:statistic graphs}
\vspace{-0.4cm}
\end{figure}


%表1给出了MVCOD的视频级信息。MVCOD包含720个视频,每个视频的平均持续时间为1.14分钟,相当于大约412帧,帧率为5 FPS(短期数据集为3-10秒)。总共有296,401帧和407,945个注释,提供的帧数至少是其他数据集的两倍[8],[9],[20],[21],[35]。视频分为5个父类和44个子类。

% 为了进一步分析,图x显示了每个场景中

%表1给出了MVCOD的视频级信息。MSVCOD包含153个视频,每个视频的平均持续时间是xxx秒,帧率5FPS,平均每个视频包含xxx帧。视频的分辨率全部为高分辨率视频,1080*1920, 视频分为7个场景:水中、陆地、沙漠、丛林、雪地、医学、艺术,涉及的伪装目标包括动物、人类、医学目标、工具四类目标。数据集给出了包围框、mask级别、实例级别、和种类信息,并且分训练集和测试集。

%很明显,与之前的数据集相比,我们的数据集包含的目标种类更加丰富,有更加丰富的注释,并且有更高质量的分辨率。在数据规模方面,虽然MoCA与我们数据集规模接近(clips:141 vs 153, frame:7614 vs 8985),但是MoCA只提供了box级别的注释,无法进行像素级别分割任务。在MoCA基础上的改进MoCA-mask虽然提供了部分像素级别的手工标注,但是片段数目和图像数据较少,我们数据集的规模几乎是它的两倍。

As shown in Figure \ref{fig:fig1}, the video clips cover seven distinct scenes, each representing different environmental conditions: Aquatic, depicting underwater environments with various aquatic organisms; Field, showcasing terrestrial landscapes such as rocky terrains or grasslands with sparse vegetation; Desert, featuring arid, sandy environments with minimal plant life; Jungle, characterized by dense, tropical forests with abundant foliage; Snowfield, highlighting icy, snow-covered terrains; Medical, focusing on clinical settings; and Artificial, which includes various man-made environments, such as urban areas and indoor spaces, where individuals attempt to camouflage within diverse artificial surroundings.

As shown in the bottom part of Figure \ref{fig:fig1},  the camouflage objects are categorized into four types, representing different subjects: animal, encompassing both wild and domesticated species that blend into their environments; human, referring to individuals who employ various techniques to remain concealed in different settings; medical target, such as objects requiring segmentation in clinical videos; and vehicle, involving transportation means like cars or tanks designed to evade detection in their respective environments.

%%%%%%%%%%%%%%%%

% Figure \ref{fig:statistic graphs} presents video-level information for MSVCOD, which contains 162 video clips, each with an average duration of 3$\sim$40 seconds and a frame rate of 6 FPS, with an average of 59 frames per video. The dataset provides bounding box level, mask level, instance level, and category annotation, and is divided into training set (121 clips, 75\%) and test set (41 clips, 25\%). Figure \ref{fig:statistic graphs} shows the attribute distribution of MSVCOD.

Figure \ref{fig:statistic graphs} presents the video-level statistics for MSVCOD, which consists of 162 video clips. Each clip has an average duration ranging from 3 to 40 seconds and a frame rate of 6 FPS, with an average of 59 frames per video. The dataset includes bounding box, mask, instance, and category annotations, and is split into a training set (121 clips, 75\%) and a test set (41 clips, 25\%). The attribute distribution of MSVCOD is shown in Figure \ref{fig:statistic graphs}.

As demonstrated in Table \ref{tab_comparison}, compared to previous datasets, our dataset offers a richer variety of target categories, more detailed annotations. In terms of data size, although MoCA is similar in size to our dataset (141 clips vs. 162, 7617 frames vs. 9486), it only provides bounding box annotations and cannot be used for pixel-level segmentation tasks. MoCA-Mask, an improved version of MoCA, includes some pixel-level manual annotations; however, it has fewer clips and frames, with our dataset nearly doubling its size (162 clips vs. 87, 9486 frames vs. 5750).

\section{Benchmark Performance}


In this section, we present a one-stream VCOD framework that overcomes the multi-stage complexity of previous models and outperforms the current state-of-the-art video camouflage object detection models by utilizing only two frames of short-term motion features. Additionally, we provide a comprehensive evaluation of existing VCOD models.

\subsection{One-stream VCOD}

\begin{figure}
    \centering
    \includegraphics[width=1\linewidth]{./images/method_pipline.png}
    \caption{ Illustration of the our one-stream model pipline.}
    \label{fig:method_pipline}
\vspace{-0.3cm}
\end{figure}




\textbf{Pipline overview.} 
% We propose a new VCOD framework that inputs the current and previous frames and handles motion implicitly. This one-stream architecture processes two frames simultaneously after encoding, performing feature extraction and object recognition concurrently, eliminating the need for additional fusion modules. 
% #######################################################################################################################################################################
Inspired by the one-stream architecture of tracking \cite{cui2022mixformer} and video object segmentation \cite{hong2023simulflow}, we provide a novel One-Stream video Camouflage object detection network, called OSCNet, which processes the current and previous frames simultaneously, handling motion implicitly within a one-stream architecture. The framework comprises an encoder for concurrent feature extraction and object recognition, and a fully-connected UNet-like decoder, using linear adapter layer, as illustrated in Figure \ref{fig:method_pipline}. Initially, we utilize a four-stage asymmetrical transformer as the feature extraction backbone, each stage incorporating a patch embedding layer and $N_{i}$ asymmetrical attention layers. Then the tokens at four level are fed correspondingly into the Unet-like decoder to predict the camouflage object in current frame. Specifically, given the current frame image $I_{t}$ and the previous frame image $I_{t-1}$ with a resolution of $H\times  W\times 3$, we first divide them into patches of size $4\times 4$ and use a patch embedding layer to project these patches into the feature space. These patches are then input into an asymmetrical transformer with a hierarchical design to obtain multi-level feature representations of appearance. 


The appearance feature tokens of the $i$ stage $\left ( i\in \left \{ 1, 2, 3, 4 \right \}  \right )$ are denoted as $F_{i}$. The features $F_{1}, F_{2}, F_{3}$ and $F_{4}$ are sequentially input into the MLP decoder, and through upsampling and multiple MLP layers. Finally, the decoder head output the prediction $P_0$ of $I_{t}$. 

\textbf{Learning Strategy.} During training, we use the weighted Binary Cross Entropy (BCE) Loss and the weighted Intersection over union (IoU) loss to optimize the model: $\mathcal{L}_{all}= \sum_{j=0}^{4}(\mathcal{L}_{bce}^{w}(P_{j},G)+\mathcal{L}_{iou}^{w}(P_{j},G))$,
% \begin{equation}
% \begin{aligned}
% \mathcal{L}_{all}= \sum_{j=0}^{4}(\mathcal{L}_{bce}^{w}(P_{j},G)+\mathcal{L}_{iou}^{w}(P_{j},G)),
% \end{aligned}
% \end{equation}\\
where $G$ is the ground truth label, and $P_j$ are the predicted maps. When $j=1,2,3,4$, $P_j$ denotes the auxiliary loss for the four stages in Figure \ref{fig:method_pipline}. $P_0$ denotes the final prediction.


% #######################################################################################################################################################################

% \textbf{Design of decoder}

% % #######################################################################################################################################################################
% For the task of multi-scene video camouflage object detection, we build upon previous work in Video Object Segmentation \cite{hong2023simulflow} and Video Object tracking \cite{cui2022mixformer} to obtain final predictions. However, considering the diversity and complexity of scenes, as well as the challenges posed by camouflage objects, it is difficult for a previous model to accurately extract the camouflage targets. Therefore, we propose a layer-wise query decoder (LQD) to enhance the use of query embedding and assist in modeling the complex training data distribution.

% #######################################################################################################################################################################

\subsection{Experiments}

         % \multirow{2}{*}{\textbf{Model}} & \multirow{2}{c}{\textbf{Input}} & \multicolumn{5}{c}{\textbf{MSVCOD}} \\



% \begin{table}[!ht]
%     \centering
%     \caption{Quantitative comparisons with state-of-the-art methods on MSVCOD datasets.  "$\downarrow$"/"$\uparrow$"indicates that smaller/larger is better. 
%     % Top three results are highlighted in \textcolor{red}{red}, \textcolor{blue}{blue} and \textcolor{green}{green}.}
%     \label{tab:comparison}

%     \resizebox{0.5\textwidth}{!}{
%     \begin{tabular}{c|c|ccccc}
%         \toprule
%          \multirow{2}{*}{\textbf{Model}} & \multirow{2}{*}{\textbf{Input}} & \multicolumn{5}{c}{\textbf{MSVCOD}} \\
%         \cline{3-7}
%            & &$\mathbf{S_\alpha} \uparrow$ & $\mathbf{F^{w}_{\beta}} \uparrow$ & $\mathbf{M} \downarrow$ & \textbf{mDice} $\uparrow$ & \textbf{mIoU} $\uparrow$ \\
%         \hline
%          SINet_{20} &Image& 0.750 & 0.541 & 0.045 & 0.587 & 0.477 \\
%          SINet-V2_{22} &Image& 0.804 & 0.635 & 0.036 & \textcolor{green}{0.688} & 0.584 \\
%          ZoomNet_{22} &Image& \textcolor{green}{0.810} & \textcolor{green}{0.656} & \textcolor{blue}{0.028} & 0.684 & \textcolor{green}{0.604} \\
%         \hline

%         PNS-Net_{21}	&Video &0.540	&0.147	&0.117	&0.181	&0.118 \\

%         MG_{21}	&Video &0.431	&0.219	&0.439	&0.350	&0.246  \\
        
%          SLT-Net_{22} &Video & \textcolor{blue}{0.841} & \textcolor{blue}{0.716} & \textcolor{green}{0.029} & \textcolor{blue}{0.766} & \textcolor{blue}{0.673} \\
%         \hline
%          \textbf{Ours} &Video & \textcolor{red}{0.847} & \textcolor{red}{0.758} & \textcolor{red}{0.021} & \textcolor{red}{0.773} & \textcolor{red}{0.697} \\
%         \bottomrule
%     \end{tabular}
%     }
% \end{table}

% \begin{table*}[!ht]
%     \centering
%     \caption{Quantitative comparisons with state-of-the-art methods on MoCA-Mask and CAD , "$\downarrow$"/"$\uparrow$"indicates that smaller/larger is better. }
%     \label{tab:comparison moca-mask}

%     \resizebox{\textwidth}{!}{
%     \begin{tabular}{c c ccccc ccccc}
%         \toprule
%          \multirow{2}{*}{\textbf{Model}}& \multirow{2}{*}{\textbf{Input}} & \multicolumn{5}{c}{\textbf{MoCA-Mask}} & \multicolumn{5}{c}{\textbf{CAD}} \\
%         \cline{3-7} \cline{8-12}
%            & &$\mathbf{S_\alpha} \uparrow$ & $\mathbf{F^{w}_{\beta}} \uparrow$ & $\mathbf{M} \downarrow$ & \textbf{mDice} $\uparrow$ & \textbf{mIoU} $\uparrow$ & $\mathbf{S_\alpha} \uparrow$ & $\mathbf{F^{w}_{\beta}} \uparrow$ & $\mathbf{M} \downarrow$ & \textbf{mDice} $\uparrow$ & \textbf{mIoU} $\uparrow$\\
%         \hline
%          SINet(CVPR'20)\cite{fan2020camouflaged} &Image & 0.598 & 0.231 & 0.028 & 0.276 & 0.202 & 0.636 & 0.346 & 0.041 & 0.381 & 0.283 \\
%          SINet-V2(TPAMI'21)\cite{FanDP2021-ConcealedOD} &Image & 0.588 & 0.204 & 0.031 & 0.245 & 0.180 & 0.653 & 0.382 & 0.039 & 0.413 & 0.318 \\
%          ZoomNet(CVPR'22)\cite{pang2022zoom} &Image & 0.582 & 0.211 & 0.033 & 0.224 & 0.167 & 0.633 & 0.349 & 0.033 & 0.349 & 0.273 \\
%          DGNet(MIR'23)\cite{ji2023deep} &Image & 0.581 & 0.184 & 0.024 & 0.222 & 0.156 & 0.686 & 0.416 & 0.037 & 0.456 & 0.340 \\
%          FSPNet(CVPR'23)\cite{huang2023feature} &Image & 0.594 & 0.182 & 0.044 & 0.238 & 0.167 & 0.681 & 0.401 & {0.044} & 0.238 & 0.167 \\ 
%          FEDER(CVPR'23)\cite{He2023-FEDER} &Image & 0.560 & 0.165 & 0.033 & 0.194 & 0.137 & 0.691 & 0.444 & {0.029} & 0.474 & 0.375 \\
%          HitNet (AAAI'23)\cite{hu2023high} &Image & 0.623 & 0.299 & 0.019 & 0.318 & 0.254 & 0.685 & 0.463 & 0.031 & 0.389 & 0.375 \\
%         \hline
%         RCRNet(ICCV'19)\cite{yan2019semi} &Video & 0.555 & 0.138 & 0.033 & 0.171 & 0.116 & 0.627 & 0.287 & 0.048 & 0.380 & 0.229 \\
%          PNS-Net (MICCAI'21)\cite{ji2021progressively} &Video & 0.544 & 0.093 & 0.036 & 0.195 & 0.101 & 0.655 & 0.334 & 0.032 & 0.390 & 0.290 \\
%          MG (ICCV'21)\cite{yang2021self} &Video & 0.530 & 0.168 & 0.067 & 0.181 & 0.127 & 0.606 & 0.203 & 0.059 & 0.310 & 0.176 \\
%          SLT-Net (CVPR'22)\cite{cheng2022implicit} &Video  & 0.634 & 0.317 & 0.027 & 0.356 & 0.271 & \textbf{0.696} & {0.471} & 0.031 & 0.480 & {0.392} \\
%          SLT-Net-Long (CVPR'22)\cite{cheng2022implicit} &Video  & {0.631} & {0.311} & {0.026} & {0.367} & {0.279} & {0.691} & {0.481} & \textbf{0.030} & {0.493} &{0.401} \\
%          IMEX(TMM'24)\cite{hui2024implicit}  &Video & {0.661}  & {0.371} & {0.020} &{0.409} & {0.319} &{0.695} & \textbf{0.490} & \textbf{0.030} & \textbf{0.501} & \textbf{0.412} \\
%         % \hline
%          \textbf{Ours}  &Video  & \textbf{0.672} &\textbf{0.384} & \textbf{0.013} &\textbf{0.426} & \textbf{0.345} & {0.689} & {0.453} & \textbf{0.030} & {0.481} & {0.394} \\


%         \bottomrule
%     \end{tabular}
%     }
% \vspace{-0.2cm}
% \end{table*}


% \begin{table*}[!ht]
%     \centering
%     \caption{Quantitative comparisons with state-of-the-art methods on MoCA-Mask and CAD, "$\downarrow$"/"$\uparrow$" indicates that smaller/larger is better. }
%     \label{tab:comparison moca-mask}

%     \resizebox{\textwidth}{!}{
%     \begin{tabular}{c c ccccc c ccccc}
%         \toprule
%          \multirow{2}{*}{\textbf{Model}} & \multirow{2}{*}{\textbf{Input}} & \multicolumn{5}{c}{\textbf{MoCA-Mask}} & \multirow{2}{*}{} & \multicolumn{5}{c}{\textbf{CAD}} \\
%         \cline{3-7} \cline{9-13}
%            & & $\mathbf{S_\alpha} \uparrow$ & $\mathbf{F^{w}_{\beta}} \uparrow$ & $\mathbf{M} \downarrow$ & \textbf{mDice} $\uparrow$ & \textbf{mIoU} $\uparrow$ & & $\mathbf{S_\alpha} \uparrow$ & $\mathbf{F^{w}_{\beta}} \uparrow$ & $\mathbf{M} \downarrow$ & \textbf{mDice} $\uparrow$ & \textbf{mIoU} $\uparrow$ \\
%         \cline{1-7} \cline{9-13}
%          SINet(CVPR'20)\cite{fan2020camouflaged} &Image & 0.598 & 0.231 & 0.028 & 0.276 & 0.202 & & 0.636 & 0.346 & 0.041 & 0.381 & 0.283 \\
%          SINet-V2(TPAMI'21)\cite{FanDP2021-ConcealedOD} &Image & 0.588 & 0.204 & 0.031 & 0.245 & 0.180 & & 0.653 & 0.382 & 0.039 & 0.413 & 0.318 \\
%          ZoomNet(CVPR'22)\cite{pang2022zoom} &Image & 0.582 & 0.211 & 0.033 & 0.224 & 0.167 & & 0.633 & 0.349 & 0.033 & 0.349 & 0.273 \\
%          DGNet(MIR'23)\cite{ji2023deep} &Image & 0.581 & 0.184 & 0.024 & 0.222 & 0.156 & & 0.686 & 0.416 & 0.037 & 0.456 & 0.340 \\
%          FSPNet(CVPR'23)\cite{huang2023feature} &Image & 0.594 & 0.182 & 0.044 & 0.238 & 0.167 & & 0.681 & 0.401 & {0.044} & 0.238 & 0.167 \\ 
%          FEDER(CVPR'23)\cite{He2023-FEDER} &Image & 0.560 & 0.165 & 0.033 & 0.194 & 0.137 & & 0.691 & 0.444 & {0.029} & 0.474 & 0.375 \\
%          HitNet (AAAI'23)\cite{hu2023high} &Image & 0.623 & 0.299 & 0.019 & 0.318 & 0.254 & & 0.685 & 0.463 & 0.031 & 0.389 & 0.375 \\
%         \cline{1-7} \cline{9-13}
%         RCRNet(ICCV'19)\cite{yan2019semi} &Video & 0.555 & 0.138 & 0.033 & 0.171 & 0.116 & & 0.627 & 0.287 & 0.048 & 0.380 & 0.229 \\
%          PNS-Net (MICCAI'21)\cite{ji2021progressively} &Video & 0.544 & 0.093 & 0.036 & 0.195 & 0.101 & & 0.655 & 0.334 & 0.032 & 0.390 & 0.290 \\
%          MG (ICCV'21)\cite{yang2021self} &Video & 0.530 & 0.168 & 0.067 & 0.181 & 0.127 & & 0.606 & 0.203 & 0.059 & 0.310 & 0.176 \\
%          SLT-Net (CVPR'22)\cite{cheng2022implicit} &Video  & 0.634 & 0.317 & 0.027 & 0.356 & 0.271 & & \textbf{0.696} & {0.471} & 0.031 & 0.480 & {0.392} \\
%          SLT-Net-Long (CVPR'22)\cite{cheng2022implicit} &Video  & {0.631} & {0.311} & {0.026} & {0.367} & {0.279} & & {0.691} & {0.481} & \textbf{0.030} & {0.493} &{0.401} \\
%          IMEX(TMM'24)\cite{hui2024implicit}  &Video & {0.661}  & {0.371} & {0.020} &{0.409} & {0.319} & &{0.695} & \textbf{0.490} & \textbf{0.030} & \textbf{0.501} & \textbf{0.412} \\
%         % \hline
%          \textbf{Ours(OSCNet)}  &Video  & \textbf{0.672} &\textbf{0.384} & \textbf{0.013} &\textbf{0.426} & \textbf{0.345} & & {0.689} & {0.453} & \textbf{0.030} & {0.481} & {0.394} \\
%         \bottomrule
%     \end{tabular}
%     }
% \end{table*}

\begin{table*}[!ht]
    \centering
    \caption{Quantitative comparisons with state-of-the-art methods on MoCA-Mask and CAD, "$\downarrow$"/"$\uparrow$" indicates that smaller/larger is better. }
    \label{tab:comparison moca-mask}

    \resizebox{0.95\textwidth}{!}{
    \begin{tabular}{c c ccccc c ccccc}
        \toprule
         \multirow{2}{*}{\textbf{Model}} & \multirow{2}{*}{\textbf{Input}} & \multicolumn{5}{c}{\textbf{MoCA-Mask}} & \multirow{2}{*}{} & \multicolumn{5}{c}{\textbf{CAD}} \\
        \cline{3-7} \cline{9-13}
           & & $\mathbf{S_\alpha} \uparrow$ & $\mathbf{F^{w}_{\beta}} \uparrow$ & $\mathbf{M} \downarrow$ & \textbf{mDice} $\uparrow$ & \textbf{mIoU} $\uparrow$ & & $\mathbf{S_\alpha} \uparrow$ & $\mathbf{F^{w}_{\beta}} \uparrow$ & $\mathbf{M} \downarrow$ & \textbf{mDice} $\uparrow$ & \textbf{mIoU} $\uparrow$ \\
        \cline{1-7} \cline{9-13}
         SINet(CVPR'20)\cite{fan2020camouflaged} &Image & 0.598 & 0.231 & 0.028 & 0.276 & 0.202 & & 0.636 & 0.346 & 0.041 & 0.381 & 0.283 \\
         SINet-V2(TPAMI'21)\cite{FanDP2021-ConcealedOD} &Image & 0.588 & 0.204 & 0.031 & 0.245 & 0.180 & & 0.653 & 0.382 & 0.039 & 0.413 & 0.318 \\
         ZoomNet(CVPR'22)\cite{pang2022zoom} &Image & 0.582 & 0.211 & 0.033 & 0.224 & 0.167 & & 0.633 & 0.349 & 0.033 & 0.349 & 0.273 \\
         DGNet(MIR'23)\cite{ji2023deep} &Image & 0.581 & 0.184 & 0.024 & 0.222 & 0.156 & & 0.686 & 0.416 & 0.037 & 0.456 & 0.340 \\
         FSPNet(CVPR'23)\cite{huang2023feature} &Image & 0.594 & 0.182 & 0.044 & 0.238 & 0.167 & & 0.681 & 0.401 & {0.044} & 0.238 & 0.167 \\ 
         FEDER(CVPR'23)\cite{He2023-FEDER} &Image & 0.560 & 0.165 & 0.033 & 0.194 & 0.137 & & 0.691 & 0.444 & {0.029} & 0.474 & 0.375 \\
         HitNet (AAAI'23)\cite{hu2023high} &Image & 0.623 & 0.299 & 0.019 & 0.318 & 0.254 & & 0.685 & 0.463 & 0.031 & 0.389 & 0.375 \\
        \cline{1-7} \cline{9-13}
        RCRNet(ICCV'19)\cite{yan2019semi} &Video & 0.555 & 0.138 & 0.033 & 0.171 & 0.116 & & 0.627 & 0.287 & 0.048 & 0.380 & 0.229 \\
         PNS-Net (MICCAI'21)\cite{ji2021progressively} &Video & 0.544 & 0.093 & 0.036 & 0.195 & 0.101 & & 0.655 & 0.334 & 0.032 & 0.390 & 0.290 \\
         MG (ICCV'21)\cite{yang2021self} &Video & 0.530 & 0.168 & 0.067 & 0.181 & 0.127 & & 0.606 & 0.203 & 0.059 & 0.310 & 0.176 \\
         SLT-Net (CVPR'22)\cite{cheng2022implicit} &Video  & 0.634 & 0.317 & 0.027 & 0.356 & 0.271 & & \textbf{0.696} & {0.471} & 0.031 & 0.480 & {0.392} \\
         SLT-Net-Long (CVPR'22)\cite{cheng2022implicit} &Video  & {0.631} & {0.311} & {0.026} & {0.367} & {0.279} & & {0.691} & {0.481} & \textbf{0.030} & {0.493} &{0.401} \\
         IMEX(TMM'24)\cite{hui2024implicit}  &Video & {0.661}  & {0.371} & {0.020} &{0.409} & {0.319} & &{0.695} & \textbf{0.490} & \textbf{0.030} & \textbf{0.501} & \textbf{0.412} \\
         TSP-SAM-Point(CVPR'24)  &Video & {0.673}  & {0.400} & {0.012} &{0.421} & {0.345} & &{0.695} & \textbf{0.490} & \textbf{0.030} & \textbf{0.501} & \textbf{0.412} \\
         TSP-SAM-Bbox(CVPR'24)  &Video & {0.689}  & {0.444} & \textbf{0.008} &{0.458} & {0.388} & &{0.695} & \textbf{0.490} & \textbf{0.030} & \textbf{0.501} & \textbf{0.412} \\
        % \hline
         \textbf{Ours}  &Video  & \textbf{0.709} &\textbf{0.451} & \textbf{0.008} &\textbf{0.473} & \textbf{0.392} & & {0.689} & {0.453} & \textbf{0.030} & {0.481} & {0.394} \\
        \bottomrule
    \end{tabular}
    }
\end{table*}
% \begin{table*}[!ht]
%     \centering
%     \caption{Quantitative comparisons with state-of-the-art methods on MoCA-Mask and CAD, "$\downarrow$"/"$\uparrow$" indicates that smaller/larger is better. }
%     \label{tab:comparison moca-mask}

%     \resizebox{\textwidth}{!}{
%     \begin{tabular}{c c ccccc c ccccc}
%         \toprule
%          \multirow{2}{*}{\textbf{Model}} & \multirow{2}{*}{\textbf{Input}} & \multicolumn{5}{c}{\textbf{MoCA-Mask}} & \multirow{2}{*}{} & \multicolumn{5}{c}{\textbf{CAD}} \\
%         \cline{3-7} \cline{9-13}
%            & & $\mathbf{S_\alpha} \uparrow$ & $\mathbf{F^{w}_{\beta}} \uparrow$ & $\mathbf{M} \downarrow$ & \textbf{mDice} $\uparrow$ & \textbf{mIoU} $\uparrow$ & & $\mathbf{S_\alpha} \uparrow$ & $\mathbf{F^{w}_{\beta}} \uparrow$ & $\mathbf{M} \downarrow$ & \textbf{mDice} $\uparrow$ & \textbf{mIoU} $\uparrow$ \\
%         \cline{1-7} \cline{9-13}
%          SINet(CVPR'20)\cite{fan2020camouflaged} &Image & 0.598 & 0.231 & 0.028 & 0.276 & 0.202 & & 0.636 & 0.346 & 0.041 & 0.381 & 0.283 \\
%          SINet-V2(TPAMI'21)\cite{FanDP2021-ConcealedOD} &Image & 0.588 & 0.204 & 0.031 & 0.245 & 0.180 & & 0.653 & 0.382 & 0.039 & 0.413 & 0.318 \\
%          ZoomNet(CVPR'22)\cite{pang2022zoom} &Image & 0.582 & 0.211 & 0.033 & 0.224 & 0.167 & & 0.633 & 0.349 & 0.033 & 0.349 & 0.273 \\
%          DGNet(MIR'23)\cite{ji2023deep} &Image & 0.581 & 0.184 & 0.024 & 0.222 & 0.156 & & 0.686 & 0.416 & 0.037 & 0.456 & 0.340 \\
%          FSPNet(CVPR'23)\cite{huang2023feature} &Image & 0.594 & 0.182 & 0.044 & 0.238 & 0.167 & & 0.681 & 0.401 & {0.044} & 0.238 & 0.167 \\ 
%          FEDER(CVPR'23)\cite{He2023-FEDER} &Image & 0.560 & 0.165 & 0.033 & 0.194 & 0.137 & & 0.691 & 0.444 & {0.029} & 0.474 & 0.375 \\
%          HitNet (AAAI'23)\cite{hu2023high} &Image & 0.623 & 0.299 & 0.019 & 0.318 & 0.254 & & 0.685 & 0.463 & 0.031 & 0.389 & 0.375 \\
%         \cline{1-7} \cline{9-13}
%         RCRNet(ICCV'19)\cite{yan2019semi} &Video & 0.555 & 0.138 & 0.033 & 0.171 & 0.116 & & 0.627 & 0.287 & 0.048 & 0.380 & 0.229 \\
%          PNS-Net (MICCAI'21)\cite{ji2021progressively} &Video & 0.544 & 0.093 & 0.036 & 0.195 & 0.101 & & 0.655 & 0.334 & 0.032 & 0.390 & 0.290 \\
%          MG (ICCV'21)\cite{yang2021self} &Video & 0.530 & 0.168 & 0.067 & 0.181 & 0.127 & & 0.606 & 0.203 & 0.059 & 0.310 & 0.176 \\
%          SLT-Net (CVPR'22)\cite{cheng2022implicit} &Video  & 0.634 & 0.317 & 0.027 & 0.356 & 0.271 & & \textbf{0.696} & {0.471} & 0.031 & 0.480 & {0.392} \\
%          SLT-Net-Long (CVPR'22)\cite{cheng2022implicit} &Video  & {0.631} & {0.311} & {0.026} & {0.367} & {0.279} & & {0.691} & {0.481} & \textbf{0.030} & {0.493} &{0.401} \\
%          IMEX(TMM'24)\cite{hui2024implicit}  &Video & {0.661}  & {0.371} & {0.020} &{0.409} & {0.319} & &{0.695} & \textbf{0.490} & \textbf{0.030} & \textbf{0.501} & \textbf{0.412} \\
%         % \hline
%          \textbf{Ours(OSCNet)}  &Video  & \textbf{0.672} &\textbf{0.384} & \textbf{0.013} &\textbf{0.426} & \textbf{0.345} & & {0.689} & {0.453} & \textbf{0.030} & {0.481} & {0.394} \\
%         \bottomrule
%     \end{tabular}
%     }
% \end{table*}



\begin{table}[!ht]
    \centering
    \caption{Quantitative comparisons with state-of-the-art methods on MSVCOD datasets. 
    % "$\downarrow$"/"$\uparrow$"indicates that smaller/larger is better. 
    % Top three results are highlighted in \textcolor{red}{red}, \textcolor{blue}{blue} and \textcolor{green}{green}.
    }
    \label{tab:comparison msvcod}

    \resizebox{0.47\textwidth}{!}{
    \begin{tabular}{c c ccccc}
        \toprule
         \multirow{2}{*}{\textbf{Model}} & \multirow{2}{*}{\textbf{Input}} & \multicolumn{5}{c}{\textbf{MSVCOD}} \\
        \cline{3-7}
           & &$\mathbf{S_\alpha}$ $\uparrow$ & $\mathbf{F^{w}_{\beta}}$ $\uparrow$ & $\mathbf{M}$ $\downarrow$ & \textbf{mDice} $\uparrow$ & \textbf{mIoU} $\uparrow$ \\
        \hline
         SINet\cite{fan2020camouflaged} &Image& 0.750 & 0.541 & 0.045 & 0.587 & 0.477 \\
         SINet-V2\cite{FanDP2021-ConcealedOD} &Image& 0.804 & 0.635 & 0.036 & {0.688} & 0.584 \\
         ZoomNet\cite{pang2022zoom} &Image& {0.810} & {0.656} & {0.028} & 0.684 & {0.604} \\
        \hline

        PNS-Net\cite{ji2021progressively}	&Video &0.540	&0.147	&0.117	&0.181	&0.118 \\

        MG\cite{yang2021self}	&Video &0.431	&0.219	&0.439	&0.350	&0.246  \\
        
         SLT-Net\cite{cheng2022implicit}  &Video & {0.841} & {0.716} & {0.029} & {0.766} & {0.673} \\
        \hline
         \textbf{Ours(OSCNet)} &Video & \textbf{0.845} & \textbf{0.744} & \textbf{0.026} & \textbf{0.771} & \textbf{0.695} \\
        \bottomrule
    \end{tabular}
    }
\vspace{-0.3cm}
\end{table}




\subsubsection{Implementation Details}

All input images were first randomly resized to a scale ranging from 0.5 to 2 times their original size, followed by random cropping to produce 512x512 patches. The patches were then augmented using random flips and photometric distortion. The batch size was set to 12. The entire model was optimized using the AdamW \cite{Loshchilov2017DecoupledWD} optimizer. In terms of the training strategy, and to ensure fairness, we followed a two-stage model training approach, similar to Cheng \emph {et al.} \cite{cheng2022implicit}. In the first stage, the backbone was trained on the COD10k dataset, followed by fine-tuning on either the MoCA-mask or MSVCOD training set. Evaluation was then conducted on three video camouflage object detection datasets: CAD, MoCA-mask, and MSVCOD. During the pre-training phase, the learning rate was set to 1e-4 and decayed to 0 after 80,000 iterations. For fine-tuning, the learning rate was set to 1e-5, with a maximum of 10,000 training iterations.


\textbf{Evaluation Metrics.} We use five metrics for evaluation: S-measure ($S_\alpha$) \cite{fan2017structure} for structural similarity, F-measure with weights ($F_w^{\beta}$) \cite{margolin2014evaluate} for precision and recall, Mean Absolute Error (MAE) for pixel-level differences, Average Dice (mDice) for data similarity, and Average IoU (mIoU) for mask overlap. 
% These metrics assess performance at the pixel, local, and overall levels.



\subsubsection{Baseline Evalution}

To assess our dataset and model, we compared our model with existing state-of-the-art methods. For fairness, the performance in Table \ref{tab:comparison moca-mask} was either evaluated using the authors' published prediction images or tested using the authors' open-source weights. For testing on MSVCOD, we use the authors' open source code for training, and testing. The Table \ref{tab:comparison moca-mask} uses the follow setting: methods trained on MoCA-Mask training dataset, comparison on our MoCA-Mask and CAD test dataset; Table \ref{tab:comparison msvcod} uses the follow setting: methods trained on our MSVCOD training dataset, comparison on our MSVCOD test dataset.

% It can be seen that with the assistance of the new framework and new dataset, our results are significantly leading on the MoCA-Mask dataset. Our proposed multi-level decoder and one-stream framework can perform well in the task of animal camouflage object detection. 

% In the experiment of the CAD dataset, our results also improved compared to the comparison method. 

% Since the CAD dataset is not included in MoCA, the performance improvement of the generalization experiment on CAD is mostly due to the expansion of new data.

From the results of Table \ref{tab:comparison moca-mask} and Table \ref{tab:comparison msvcod}, it is clear that our one-stream model performs well. Unlike previous methods that extract features from adjacent frames or image and optical flow separately and then fuse them, our model simultaneously extracts image features and incorporates motion information. Self-attention handles image feature extraction, while cross-attention uses motion features from adjacent frames. This alternating process of interactive extraction and fusion optimizes feature extraction for camouflaged objects, setting our approach apart from previous methods.

Table \ref{tab:comparison moca-mask} shows that our one-stream model consistently outperforms the previous SoTA model on both MoCA-Mask datasets. MoCA-Mask contains 87 video clips 5750 image frames (CAD only contains 9 clips, 191 frames), and our model exhibits consistent performance improvement in MoCA-Mask compared to IMEX:2.6\% improvement in mIoU, 1.7\% improvement in mDice, 1.1\% improvement in $S_\alpha$ 1.3\% improvement in $F^{w}_{\beta}$ , and 0.7\% reduction in MAE. 

Similar to SLT-Net \cite{cheng2022implicit} and IMEX \cite{hui2024implicit}, we also compared RCRNet \cite{yan2019semi}, PNS-Net \cite{ji2021progressively}, and MG \cite{yang2021self} in Table \ref{tab:comparison moca-mask}. PNS-Net, RCRNet, and MG are not methods specialized for video camouflage target detection models. PNS-Net is a model for medical polyp segmentation, while RCRNet is a semi-supervised video salient object segmentation method using only pseudo-labels, and MG is a self-supervised video object segmentation method using only optical flow. So their performance is lower than our model, SLT-Net and IMEX. To explore future possibilities on multiple tasks, we also trained and tested PNS-Net, the MG on our MSVCOD in Table \ref{tab:comparison msvcod}.


%%%%%%%%%%%%%%%%%%%%%%%%%%%

The results in Table \ref{tab:comparison msvcod} show that static image-based camouflage object detection methods (\emph {e.g.}, SINet, SINet-V2, and ZoomNet) on MSVCOD also perform well, but there is a significant gap compared to advanced VCOD methods. Since IMEX does not have available code, its performance was not evaluated. Our model outperforms SLTNET on all five metrics,
particularly with 3.9\% improvement in $F^{w}_{\beta}$, 3.2\% improvement in mIoU.





\subsubsection{Dataset Analysis.}

Comparing the test results in Exp 1 (Table \ref{tab:comparison moca-mask}) and Exp 2 (Table \ref{tab:comparison msvcod}), models generally perform worse on the MoCA-Mask test set. However, this does not mean MoCA-Mask is more complex than MSVCOD. As shown in Figure \ref{fig:msvcod_vs_moca} and \textbf{the Figure 1 and Figure 2 in the supplementary material}, the MoCA-Mask test set contains 16 clips, all with small objects (mostly between 0 and 0.04 in size). Whereas in the training set, the distribution of large and small targets is more uniform, which may be the reason why most models perform poorly on the test set of MoCA-Mask. Small objects, especially camouflaged ones, are inherently harder to detect. In contrast, the MSVCOD test set, with 41 clips, has a more diverse range of object sizes, as shown in Figures 2 and Figures 3. Thus, the MoCA-Mask test set's focus on small objects makes it more challenging, whereas MSVCOD provides a more balanced representation of object types. 


Figure \ref{fig:visual_comparison} provides a qualitative comparison of our proposed model with existing methods on representative objects (human, animal, vehicle) from MSVCOD. Our method outperforms others in both target integrity and local details (the horns of the chameleon in line 3), owing to our one-stream camouflage object detection framework.

To analyze the model performance across four object types and seven scenarios, we plotted the line chart in Figure \ref{fig:Line chart}. In Figure \ref{fig:Line chart} (a), the performance of different models (including SINetv2, ZoomNet, and SLTNet) varies across different camouflaged objects (\emph {e.g.}, animal, human, vehicle, and medical objects). All models perform well on animals, with the best performance of $\mathbf{F^{w}_{\beta}}$ reaches 0.825, but their performance is lower for other object categories (humans, vehicles, medical), particularly for medical objects, due to the relatively small number of medical objects in MSVCOD. Figure \ref{fig:Line chart} (b) compares the model's detection performance across different scenes (Aquatic, Artificial, Desert, Field, Jungle, Medical, Snowfield). The models show significant performance differences across these scenes, which suggests that the scenarios also make a large difference to the camouflage detection model.


%%%%%%%%%%%%%%%%%%%%%%%%

% \textcolor{red}{PNS-Net [2], RCRNet [3], and MG [4] are not methods specialized for video camouflage target detection models. PNS-Net [2] is a model for medical polyp segmentation, while RCRNet [3] is a semi-supervised video object segmentation method using only pseudo-labels, and MG [4] is a semi-supervised salient object detection method using only optical flow.
% SLTNet is the first video camouflage object detection model which is trained on COD10k and then fine-tuned on MoCA-Mask. It compares two two-stage training methods, PNS-Net (2021) and RCRNet (2019). Possibly to explore the potential of self-supervised approaches for video camouflage object detection tasks, the MG (2021) [4] method is tested in this paper, which was trained using optical flow supervision. Here we trained and tested two relatively new models (PNS-Net and MG) on MSVCOD, as shown in the table below.}



\begin{figure}
    \centering
    \includegraphics[width=1\linewidth]{./images/visual_comparison2.png}
    \caption{Visual comparisons on the MSVCOD benchmark demonstrate that our model predicts camouflaged objects more accurately across a range of challenging scenarios.}
    \label{fig:visual_comparison}
\vspace{-0.3cm}
\end{figure}


\begin{figure}
    \centering
    \includegraphics[width=1\linewidth]{./images/Line_chart.jpg}
    \caption{The results of group experiments on the MSVCOD dataset. Subgraph (a) compares the performance across major categories, and subgraph (b) contrasts the results across different scenarios. The weighted F-measure is evaluation metric.}
    \label{fig:Line chart}
\end{figure}


\begin{figure}
    \centering
    \includegraphics[width=1\linewidth]{./images/msvcod_vs_moca.jpg}
    \caption{The scale distribution of camouflaged objects in MSVCOD and MoCA-Mask. The horizontal coordinate represents the number of frames in the clips, and the vertical coordinate represents the ratio of the camouflage object to the image size. It can be seen that our MSVCOD dataset has a much larger range of camouflage object scales.}
    \label{fig:msvcod_vs_moca}
\vspace{-0.4cm}
\end{figure}

% In Figure \ref{fig:msvcod_vs_moca}, The horizontal coordinate represents the number of frames in the clips, and the vertical coordinate represents the ratio of the camouflage object to the image size. It can be seen that our MSVCOD dataset has a much larger range of camouflage object scales. And by visualizing the scale distributions of the data in the training and test sets separately \textbf{in the Supplementary Material}, it can be observed that most of the samples in the MoCA-Mask test set are small targets, whereas in the training set, the distribution of large and small targets is more uniform, which may be the reason why most models perform poorly on the test set of MoCA-Mask.



\subsection{Limitation Discussion}
In Figure. \ref{fig:statistic graphs}, it can be seen that the number of medical image videos is limited, and the unbalanced distribution of data is a limitation of our dataset. On the other hand, the length of the videos is not long enough (only 10-15 seconds on average) due to the limitation of available data. Although we had access to a small number of videos lasting longer than 1 minute, only a very small portion of them were available. We had to intercept some of them to keep close to other parts of the video clips. Limited by the videos we could collect, the number of camouflage instances in our videos is relatively small, with most video clips having only one camouflage object.

\section{Conclusion}

% In this paper, we introduce an MSVCOD dataset, a novel video camouflage target detection dataset containing multiple scenes and multiple targets. To construct the MSVCOD dataset, we design a semi-automatic iterative annotation pipline to annotate the large-scale VCOD dataset while saving the annotation cost, and this annotation method can provide a reference for similar tasks. 
% % In this dataset, all videos are in HD with a resolution of 1080 × 1920.
% The videos are categorized into seven scenarios: underwater, land, desert, jungle, snowfield, medical, and art. Camouflaged targets include four categories: animal, human, medical, and vehicle. The dataset provides boundaries, mask level, instance level, and category information, and is divided into a training set and a test set. The dataset includes 162 clips, which substantially exceeds the current VCOD dataset. Moreover, we propose a single-stage decoder for video camouflage target recognition as a baseline model, which achieves state-of-the-art performance on multiple datasets. In the experimental section, we conduct a large number of experiments, and the results show that the present dataset can improve the scene adaptation capability of the model. By proposing MSVCOD, we hope to facilitate that video camouflage target detection can be extended to more realistic scenarios.


This paper introduces the MSVCOD dataset, a novel video camouflage target detection dataset featuring multiple scenes and target categories. We design a semi-automatic iterative annotation pipeline to efficiently annotate this large-scale dataset, which can serve as a reference for similar tasks. The dataset includes 162 video clips across seven scenarios (underwater, land, desert, jungle, snowfield, medical, and art) and four object categories (animal, human, medical, and vehicle). It provides bounding box, mask, instance, and category annotations, divided into training and test sets. We also propose a one-stream model for video camouflage object detection, achieving state-of-the-art performance on multiple datasets. Experimental results show that MSVCOD facilitates the extension of video camouflage object detection to diverse scenes and object types.


{
    \small
    \bibliographystyle{ieeenat_fullname}
    \bibliography{main}
}

% WARNING: do not forget to delete the supplementary pages from your submission 
% \clearpage
\pagenumbering{gobble}
\maketitlesupplementary

\section{Additional Results on Embodied Tasks}

To evaluate the broader applicability of our EgoAgent's learned representation beyond video-conditioned 3D human motion prediction, we test its ability to improve visual policy learning for embodiments other than the human skeleton.
Following the methodology in~\cite{majumdar2023we}, we conduct experiments on the TriFinger benchmark~\cite{wuthrich2020trifinger}, which involves a three-finger robot performing two tasks: reach cube and move cube. 
We freeze the pretrained representations and use a 3-layer MLP as the policy network, training each task with 100 demonstrations.

\begin{table}[h]
\centering
\caption{Success rate (\%) on the TriFinger benchmark, where each model's pretrained representation is fixed, and additional linear layers are trained as the policy network.}
\label{tab:trifinger}
\resizebox{\linewidth}{!}{%
\begin{tabular}{llcc}
\toprule
Methods       & Training Dataset & Reach Cube & Move Cube \\
\midrule
DINO~\cite{caron2021emerging}         & WT Venice        & 78.03     & 47.42     \\
DoRA~\cite{venkataramanan2023imagenet}          & WT Venice        & 81.62     & 53.76     \\
DoRA~\cite{venkataramanan2023imagenet}          & WT All           & 82.40     & 48.13     \\
\midrule
EgoAgent-300M & WT+Ego-Exo4D      & 82.61    & 54.21      \\
EgoAgent-1B   & WT+Ego-Exo4D      & \textbf{85.72}      & \textbf{57.66}   \\
\bottomrule
\end{tabular}%
}
\end{table}

As shown in Table~\ref{tab:trifinger}, EgoAgent achieves the highest success rates on both tasks, outperforming the best models from DoRA~\cite{venkataramanan2023imagenet} with increases of +3.32\% and +3.9\% respectively.
This result shows that by incorporating human action prediction into the learning process, EgoAgent demonstrates the ability to learn more effective representations that benefit both image classification and embodied manipulation tasks.
This highlights the potential of leveraging human-centric motion data to bridge the gap between visual understanding and actionable policy learning.



\section{Additional Results on Egocentric Future State Prediction}

In this section, we provide additional qualitative results on the egocentric future state prediction task. Additionally, we describe our approach to finetune video diffusion model on the Ego-Exo4D dataset~\cite{grauman2024ego} and generate future video frames conditioned on initial frames as shown in Figure~\ref{fig:opensora_finetune}.

\begin{figure}[b]
    \centering
    \includegraphics[width=\linewidth]{figures/opensora_finetune.pdf}
    \caption{Comparison of OpenSora V1.1 first-frame-conditioned video generation results before and after finetuning on Ego-Exo4D. Fine-tuning enhances temporal consistency, but the predicted pixel-space future states still exhibit errors, such as inaccuracies in the basketball's trajectory.}
    \label{fig:opensora_finetune}
\end{figure}

\subsection{Visualizations and Comparisons}

More visualizations of our method, DoRA, and OpenSora in different scenes (as shown in Figure~\ref{fig:supp pred}). For OpenSora, when predicting the states of $t_k$, we use all the ground truth frames from $t_{0}$ to $t_{k-1}$ as conditions. As OpenSora takes only past observations as input and neglects human motion, it performs well only when the human has relatively small motions (see top cases in Figure~\ref{fig:supp pred}), but can not adjust to large movements of the human body or quick viewpoint changes (see bottom cases in Figure~\ref{fig:supp pred}).

\begin{figure*}
    \centering
    \includegraphics[width=\linewidth]{figures/supp_pred.pdf}
    \caption{Retrieval and generation results for egocentric future state prediction. Correct and wrong retrieval images are marked with green and red boundaries, respectively.}
    \label{fig:supp pred}
\end{figure*}

\begin{figure*}[t]
    \centering
    \includegraphics[width=0.9\linewidth]{figures/motion_prediction.pdf}
    \vspace{-0.5mm}
    \caption{Motion prediction results in scenes with minor changes in observation.}
    \vspace{-1.5mm}
    \label{fig:motion_prediction}
\end{figure*}

\subsection{Finetuning OpenSora on Ego-Exo4D}

OpenSora V1.1~\cite{opensora}, initially trained on internet videos and images, produces severely inconsistent results when directly applied to infer future videos on the Ego-Exo4D dataset, as illustrated in Figure~\ref{fig:opensora_finetune}.
To address the gap between general internet content and egocentric video data, we fine-tune the official checkpoint on the Ego-Exo4D training set for 50 epochs.
OpenSora V1.1 proposed a random mask strategy during training to enable video generation by image and video conditioning. We adopted the default masking rate, which applies: 75\% with no masking, 2.5\% with random masking of 1 frame to 1/4 of the total frames, 2.5\% with masking at either the beginning or the end for 1 frame to 1/4 of the total frames, and 5\% with random masking spanning 1 frame to 1/4 of the total frames at both the beginning and the end.

As shown in Fig.~\ref{fig:opensora_finetune}, despite being trained on a large dataset, OpenSora struggles to generalize to the Ego-Exo4D dataset, producing future video frames with minimal consistency relative to the conditioning frame. While fine-tuning improves temporal consistency, the moving trajectories of objects like the basketball and soccer ball still deviate from realistic physical laws. Compared with our feature space prediction results, this suggests that training world models in a reconstructive latent space is more challenging than training them in a feature space.


\section{Additional Results on 3D Human Motion Prediction}

We present additional qualitative results for the 3D human motion prediction task, highlighting a particularly challenging scenario where egocentric observations exhibit minimal variation. This scenario poses significant difficulties for video-conditioned motion prediction, as the model must effectively capture and interpret subtle changes. As demonstrated in Fig.~\ref{fig:motion_prediction}, EgoAgent successfully generates accurate predictions that closely align with the ground truth motion, showcasing its ability to handle fine-grained temporal dynamics and nuanced contextual cues.

\section{OpenSora for Image Classification}

In this section, we detail the process of extracting features from OpenSora V1.1~\cite{opensora} (without fine-tuning) for an image classification task. Following the approach of~\cite{xiang2023denoising}, we leverage the insight that diffusion models can be interpreted as multi-level denoising autoencoders. These models inherently learn linearly separable representations within their intermediate layers, without relying on auxiliary encoders. The quality of the extracted features depends on both the layer depth and the noise level applied during extraction.


\begin{table}[h]
\centering
\caption{$k$-NN evaluation results of OpenSora V1.1 features from different layer depths and noising scales on ImageNet-100. Top1 and Top5 accuracy (\%) are reported.}
\label{tab:opensora-knn}
\resizebox{0.95\linewidth}{!}{%
\begin{tabular}{lcccccc}
\toprule
\multirow{2}{*}{Timesteps} & \multicolumn{2}{c}{First Layer} & \multicolumn{2}{c}{Middle Layer} & \multicolumn{2}{c}{Last Layer} \\
\cmidrule(r){2-3}   \cmidrule(r){4-5}  \cmidrule(r){6-7}  & Top1           & Top5           & Top1            & Top5           & Top1           & Top5          \\
\midrule
32        &  6.10           & 18.20             & 34.04               & 59.50             & 30.40             & 55.74             \\
64        & 6.12              & 18.48              & 36.04               & 61.84              & 31.80         & 57.06         \\
128       & 5.84             & 18.14             & 38.08               & 64.16              & 33.44       & 58.42 \\
256       & 5.60             & 16.58              & 30.34               & 56.38              &28.14          & 52.32        \\
512       & 3.66              & 11.70            & 6.24              & 17.62              & 7.24              & 19.44  \\ 
\bottomrule
\end{tabular}%
}
\end{table}

As shown in Table~\ref{tab:opensora-knn}, we first evaluate $k$-NN classification performance on the ImageNet-100 dataset using three intermediate layers and five different noise scales. We find that a noise timestep of 128 yields the best results, with the middle and last layers performing significantly better than the first layer.
We then test this optimal configuration on ImageNet-1K and find that the last layer with 128 noising timesteps achieves the best classification accuracy.

\section{Data Preprocess}
For egocentric video sequences, we utilize videos from the Ego-Exo4D~\cite{grauman2024ego} and WT~\cite{venkataramanan2023imagenet} datasets.
The original resolution of Ego-Exo4D videos is 1408×1408, captured at 30 fps. We sample one frame every five frames and use the original resolution to crop local views (224×224) for computing the self-supervised representation loss. For computing the prediction and action loss, the videos are downsampled to 224×224 resolution.
WT primarily consists of 4K videos (3840×2160) recorded at 60 or 30 fps. Similar to Ego-Exo4D, we use the original resolution and downsample the frame rate to 6 fps for representation loss computation.
As Ego-Exo4D employs fisheye cameras, we undistort the images to a pinhole camera model using the official Project Aria Tools to align them with the WT videos.

For motion sequences, the Ego-Exo4D dataset provides synchronized 3D motion annotations and camera extrinsic parameters for various tasks and scenes. While some annotations are manually labeled, others are automatically generated using 3D motion estimation algorithms from multiple exocentric views. To maximize data utility and maintain high-quality annotations, manual labels are prioritized wherever available, and automated annotations are used only when manual labels are absent.
Each pose is converted into the egocentric camera's coordinate system using transformation matrices derived from the camera extrinsics. These matrices also enable the computation of trajectory vectors for each frame in a sequence. Beyond the x, y, z coordinates, a visibility dimension is appended to account for keypoints invisible to all exocentric views. Finally, a sliding window approach segments sequences into fixed-size windows to serve as input for the model. Note that we do not downsample the frame rate of 3D motions.

\section{Training Details}
\subsection{Architecture Configurations}
In Table~\ref{tab:arch}, we provide detailed architecture configurations for EgoAgent following the scaling-up strategy of InternLM~\cite{team2023internlm}. To ensure the generalization, we do not modify the internal modules in InternML, \emph{i.e.}, we adopt the RMSNorm and 1D RoPE. We show that, without specific modules designed for vision tasks, EgoAgent can perform well on vision and action tasks.

\begin{table}[ht]
  \centering
  \caption{Architecture configurations of EgoAgent.}
  \resizebox{0.8\linewidth}{!}{%
    \begin{tabular}{lcc}
    \toprule
          & EgoAgent-300M & EgoAgent-1B \\
          \midrule
    Depth & 22    & 22 \\
    Embedding dim & 1024  & 2048 \\
    Number of heads & 8     & 16 \\
    MLP ratio &    8/3   & 8/3 \\
    $\#$param.  & 284M & 1.13B \\
    \bottomrule
    \end{tabular}%
    }
  \label{tab:arch}%
\end{table}%

Table~\ref{tab:io_structure} presents the detailed configuration of the embedding and prediction modules in EgoAgent, including the image projector ($\text{Proj}_i$), representation head/state prediction head ($\text{MLP}_i$), action projector ($\text{Proj}_a$) and action prediction head ($\text{MLP}_a$).
Note that the representation head and the state prediction head share the same architecture but have distinct weights.

\begin{table}[t]
\centering
\caption{Architecture of the embedding ($\text{Proj}_i$, $\text{Proj}_a$) and prediction ($\text{MLP}_i$, $\text{MLP}_a$) modules in EgoAgent. For details on module connections and functions, please refer to Fig.~2 in the main paper.}
\label{tab:io_structure}
\resizebox{\linewidth}{!}{%
\begin{tabular}{lcl}
\toprule
       & \multicolumn{1}{c}{Norm \& Activation} & \multicolumn{1}{c}{Output Shape}  \\
\midrule
\multicolumn{3}{l}{$\text{Proj}_i$ (\textit{Image projector})} \\
\midrule
Input image  & -          & 3$\times$224$\times$224 \\
Conv 2D (16$\times$16) & -       & Embedding dim$\times$14$\times$14    \\
\midrule
\multicolumn{3}{l}{$\text{MLP}_i$ (\textit{State prediction head} \& \textit{Representation head)}} \\
\midrule
Input embedding  & -          & Embedding dim \\
Linear & GELU       & 2048          \\
Linear & GELU       & 2048          \\
Linear & -          & 256           \\
Linear & -          & 65536     \\
\midrule
\multicolumn{3}{l}{$\text{Proj}_a$ (\textit{Action projector})} \\
\midrule
Input pose sequence  & -          & 4$\times$5$\times$17 \\
Conv 2D (5$\times$17) & LN, GELU   & Embedding dim$\times$1$\times$1    \\
\midrule
\multicolumn{3}{l}{$\text{MLP}_a$ (\textit{Action prediction head})} \\
\midrule
Input embedding  & -          & Embedding dim$\times$1$\times$1 \\
Linear & -          & 4$\times$5$\times$17     \\
\bottomrule
\end{tabular}%
}
\end{table}


\subsection{Training Configurations}
In Table~\ref{tab:training hyper}, we provide the detailed training hyper-parameters for experiments in the main manuscripts.

\begin{table}[ht]
  \centering
  \caption{Hyper-parameters for training EgoAgent.}
  \resizebox{0.86\linewidth}{!}{%
    \begin{tabular}{lc}
    \toprule
    Training Configuration & EgoAgent-300M/1B \\
    \midrule
    Training recipe: &  \\
    optimizer & AdamW~\cite{loshchilov2017decoupled} \\
    optimizer momentum & $\beta_1=0.9, \beta_2=0.999$ \\
    \midrule
    Learning hyper-parameters: &  \\
    base learning rate & 6.0E-04 \\
    learning rate schedule & cosine \\
    base weight decay & 0.04 \\
    end weight decay & 0.4 \\
    batch size & 1920 \\
    training iters & 72,000 \\
    lr warmup iters & 1,800 \\
    warmup schedule & linear \\
    gradient clip & 1.0 \\
    data type & float16 \\
    norm epsilon & 1.0E-06 \\
    \midrule
    EMA hyper-parameters: &  \\
    momentum & 0.996 \\
    \bottomrule
    \end{tabular}%
    }
  \label{tab:training hyper}%
\end{table}%

\clearpage


\end{document}
