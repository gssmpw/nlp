%%%% ijcai25.tex

\typeout{IJCAI--25 Instructions for Authors}

% These are the instructions for authors for IJCAI-25.

\documentclass{article}
\pdfpagewidth=8.5in
\pdfpageheight=11in

% The file ijcai25.sty is a copy from ijcai22.sty
% The file ijcai22.sty is NOT the same as previous years'
\usepackage{ijcai25}

% Use the postscript times font!
\usepackage{times}
\usepackage{soul}
\usepackage{url}
\usepackage[hidelinks]{hyperref}
\usepackage[utf8]{inputenc}
\usepackage[small]{caption}
\usepackage{graphicx}
\usepackage{amsmath}
\usepackage{amsthm}
\usepackage{booktabs}
\usepackage{algorithm}
\usepackage{algorithmic}
\usepackage[switch]{lineno}

% Comment out this line in the camera-ready submission
\linenumbers

\urlstyle{same}

% add by user
\usepackage{multirow}
\usepackage{arydshln}
\usepackage{makecell}
\usepackage{gensymb}
\def\swtwo{0.48\linewidth}
\def\swone{0.98\linewidth}

% the following package is optional:
%\usepackage{latexsym}

% See https://www.overleaf.com/learn/latex/theorems_and_proofs
% for a nice explanation of how to define new theorems, but keep
% in mind that the amsthm package is already included in this
% template and that you must *not* alter the styling.
\newtheorem{example}{Example}
\newtheorem{theorem}{Theorem}

% Following comment is from ijcai97-submit.tex:
% The preparation of these files was supported by Schlumberger Palo Alto
% Research, AT\&T Bell Laboratories, and Morgan Kaufmann Publishers.
% Shirley Jowell, of Morgan Kaufmann Publishers, and Peter F.
% Patel-Schneider, of AT\&T Bell Laboratories collaborated on their
% preparation.

% These instructions can be modified and used in other conferences as long
% as credit to the authors and supporting agencies is retained, this notice
% is not changed, and further modification or reuse is not restricted.
% Neither Shirley Jowell nor Peter F. Patel-Schneider can be listed as
% contacts for providing assistance without their prior permission.

% To use for other conferences, change references to files and the
% conference appropriate and use other authors, contacts, publishers, and
% organizations.
% Also change the deadline and address for returning papers and the length and
% page charge instructions.
% Put where the files are available in the appropriate places.


% PDF Info Is REQUIRED.

% Please leave this \pdfinfo block untouched both for the submission and
% Camera Ready Copy. Do not include Title and Author information in the pdfinfo section
\pdfinfo{
/TemplateVersion (IJCAI.2025.0)
}

\title{PoI: Pixel of Interest for Novel View Synthesis Assisted Scene Coordinate Regression \\ – Supplementary Materials –}


% Single author syntax
\author{
    Author Name
    \affiliations
    Affiliation
    \emails
    email@example.com
}

% Multiple author syntax (remove the single-author syntax above and the \iffalse ... \fi here)
\iffalse
\author{
First Author$^1$
\and
Second Author$^2$\and
Third Author$^{2,3}$\And
Fourth Author$^4$\\
\affiliations
$^1$First Affiliation\\
$^2$Second Affiliation\\
$^3$Third Affiliation\\
$^4$Fourth Affiliation\\
\emails
\{first, second\}@example.com,
third@other.example.com,
fourth@example.com
}
\fi

\begin{document}

\maketitle

\section{Visulized Results of sparse input}

\begin{figure}[!hb]
	\begin{center}
		\begin{tabular}{cc}
            \includegraphics[width=\swtwo]{figures/c2f/coarse.png}  &
            \includegraphics[width=\swtwo]{figures/c2f/fine.png}  \\
            
            \small{(a) Coarse stage results. } &
            \small{(b) Fine stage results. } \\
            
        \end{tabular}
    \end{center}
    \setlength{\abovecaptionskip}{-4pt}%
    \caption{The localization trajectories of the coarse-to-fine method for sparse view circumstances.}
	\label{fig:c2f}
\end{figure}

We construct the mesh based on the estimated scene coordinates of the coarse stage and the fine stage and visualize the camera pose estimation results in Figure~\ref{fig:c2f}. We may find that in the coarse stage, the pose estimation error is relatively large, and the quality of the reconstructed details is low. In the refined stage, the performance is much better.



\section{Visulized Results of PoI}

The visualized results of the 7Scenes camera pose estimation are shown in Figure~\ref{fig:vis_loc}. The trajectories of the ground truth camera pose are drawn in white, while the color of the predicted trajectories is set according to the estimated translation error. As translation errors increase, the color tends to change from purple to red, following the color spectrum of the rainbow.
% the same color map is also used in the error histograms on the top.
To make the camera pose prediction results clearer, we also draw a mesh rendering view built from the estimated scene coordinates of the training data in the same frame for correspondence.
\begin{figure*}[ht]
\begin{center}
   \includegraphics[width=1.\linewidth]{figures/vis_supp.pdf}
\end{center}
   \caption{Visualized camera pose estimation results of 7scenes dataset.}
\label{fig:vis_loc}
\end{figure*}

\end{document}

