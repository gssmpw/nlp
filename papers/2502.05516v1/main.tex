% This is samplepaper.tex, a sample chapter demonstrating the
% LLNCS macro package for Springer Computer Science proceedings;
% Version 2.21 of 2022/01/12
%
\documentclass[runningheads]{llncs}
%
\usepackage[T1]{fontenc}
% T1 fonts will be used to generate the final print and online PDFs,
% so please use T1 fonts in your manuscript whenever possible.
% Other font encondings may result in incorrect characters.
%
\usepackage{graphicx}
% Used for displaying a sample figure. If possible, figure files should
% be included in EPS format.
%
% If you use the hyperref package, please uncomment the following two lines
% to display URLs in blue roman font according to Springer's eBook style:
%\usepackage{color}
%\renewcommand\UrlFont{\color{blue}\rmfamily}
%\urlstyle{rm}
%

% basic
%\usepackage{color,xcolor}
\usepackage{color}
\usepackage{epsfig}
\usepackage{graphicx}
\usepackage{algorithm,algorithmic}
% \usepackage{algpseudocode}
%\usepackage{ulem}

% figure and table
\usepackage{adjustbox}
\usepackage{array}
\usepackage{booktabs}
\usepackage{colortbl}
\usepackage{float,wrapfig}
\usepackage{framed}
\usepackage{hhline}
\usepackage{multirow}
% \usepackage{subcaption} % issues a warning with CVPR/ICCV format
% \usepackage[font=small]{caption}
\usepackage[percent]{overpic}
%\usepackage{tikz} % conflict with ECCV format

% font and character
\usepackage{amsmath,amsfonts,amssymb}
% \let\proof\relax      % for ECCV llncs class
% \let\endproof\relax   % for ECCV llncs class
\usepackage{amsthm} 
\usepackage{bm}
\usepackage{nicefrac}
\usepackage{microtype}
\usepackage{contour}
\usepackage{courier}
%\usepackage{palatino}
%\usepackage{times}

% layout
\usepackage{changepage}
\usepackage{extramarks}
\usepackage{fancyhdr}
\usepackage{lastpage}
\usepackage{setspace}
\usepackage{soul}
\usepackage{xspace}
\usepackage{cuted}
\usepackage{fancybox}
\usepackage{afterpage}
%\usepackage{enumitem} % conflict with IEEE format
%\usepackage{titlesec} % conflict with ECCV format

% ref
% commenting these two out for this submission so it looks the same as RSS example
% \usepackage[breaklinks=true,colorlinks,backref=True]{hyperref}
% \hypersetup{colorlinks,linkcolor={black},citecolor={MSBlue},urlcolor={magenta}}
\usepackage{url}
\usepackage{quoting}
\usepackage{epigraph}

% misc
\usepackage{enumerate}
\usepackage{paralist,tabularx}
\usepackage{comment}
\usepackage{pdfpages}
% \usepackage[draft]{todonotes} % conflict with CVPR/ICCV/ECCV format



% \usepackage{todonotes}
% \usepackage{caption}
% \usepackage{subcaption}

\usepackage{pifont}% http://ctan.org/pkg/pifont

% extra symbols
\usepackage{MnSymbol}



%
\setlength\unitlength{1mm}
\newcommand{\twodots}{\mathinner {\ldotp \ldotp}}
% bb font symbols
\newcommand{\Rho}{\mathrm{P}}
\newcommand{\Tau}{\mathrm{T}}

\newfont{\bbb}{msbm10 scaled 700}
\newcommand{\CCC}{\mbox{\bbb C}}

\newfont{\bb}{msbm10 scaled 1100}
\newcommand{\CC}{\mbox{\bb C}}
\newcommand{\PP}{\mbox{\bb P}}
\newcommand{\RR}{\mbox{\bb R}}
\newcommand{\QQ}{\mbox{\bb Q}}
\newcommand{\ZZ}{\mbox{\bb Z}}
\newcommand{\FF}{\mbox{\bb F}}
\newcommand{\GG}{\mbox{\bb G}}
\newcommand{\EE}{\mbox{\bb E}}
\newcommand{\NN}{\mbox{\bb N}}
\newcommand{\KK}{\mbox{\bb K}}
\newcommand{\HH}{\mbox{\bb H}}
\newcommand{\SSS}{\mbox{\bb S}}
\newcommand{\UU}{\mbox{\bb U}}
\newcommand{\VV}{\mbox{\bb V}}


\newcommand{\yy}{\mathbbm{y}}
\newcommand{\xx}{\mathbbm{x}}
\newcommand{\zz}{\mathbbm{z}}
\newcommand{\sss}{\mathbbm{s}}
\newcommand{\rr}{\mathbbm{r}}
\newcommand{\pp}{\mathbbm{p}}
\newcommand{\qq}{\mathbbm{q}}
\newcommand{\ww}{\mathbbm{w}}
\newcommand{\hh}{\mathbbm{h}}
\newcommand{\vvv}{\mathbbm{v}}

% Vectors

\newcommand{\av}{{\bf a}}
\newcommand{\bv}{{\bf b}}
\newcommand{\cv}{{\bf c}}
\newcommand{\dv}{{\bf d}}
\newcommand{\ev}{{\bf e}}
\newcommand{\fv}{{\bf f}}
\newcommand{\gv}{{\bf g}}
\newcommand{\hv}{{\bf h}}
\newcommand{\iv}{{\bf i}}
\newcommand{\jv}{{\bf j}}
\newcommand{\kv}{{\bf k}}
\newcommand{\lv}{{\bf l}}
\newcommand{\mv}{{\bf m}}
\newcommand{\nv}{{\bf n}}
\newcommand{\ov}{{\bf o}}
\newcommand{\pv}{{\bf p}}
\newcommand{\qv}{{\bf q}}
\newcommand{\rv}{{\bf r}}
\newcommand{\sv}{{\bf s}}
\newcommand{\tv}{{\bf t}}
\newcommand{\uv}{{\bf u}}
\newcommand{\wv}{{\bf w}}
\newcommand{\vv}{{\bf v}}
\newcommand{\xv}{{\bf x}}
\newcommand{\yv}{{\bf y}}
\newcommand{\zv}{{\bf z}}
\newcommand{\zerov}{{\bf 0}}
\newcommand{\onev}{{\bf 1}}

% Matrices

\newcommand{\Am}{{\bf A}}
\newcommand{\Bm}{{\bf B}}
\newcommand{\Cm}{{\bf C}}
\newcommand{\Dm}{{\bf D}}
\newcommand{\Em}{{\bf E}}
\newcommand{\Fm}{{\bf F}}
\newcommand{\Gm}{{\bf G}}
\newcommand{\Hm}{{\bf H}}
\newcommand{\Id}{{\bf I}}
\newcommand{\Jm}{{\bf J}}
\newcommand{\Km}{{\bf K}}
\newcommand{\Lm}{{\bf L}}
\newcommand{\Mm}{{\bf M}}
\newcommand{\Nm}{{\bf N}}
\newcommand{\Om}{{\bf O}}
\newcommand{\Pm}{{\bf P}}
\newcommand{\Qm}{{\bf Q}}
\newcommand{\Rm}{{\bf R}}
\newcommand{\Sm}{{\bf S}}
\newcommand{\Tm}{{\bf T}}
\newcommand{\Um}{{\bf U}}
\newcommand{\Wm}{{\bf W}}
\newcommand{\Vm}{{\bf V}}
\newcommand{\Xm}{{\bf X}}
\newcommand{\Ym}{{\bf Y}}
\newcommand{\Zm}{{\bf Z}}

% Calligraphic

\newcommand{\Ac}{{\cal A}}
\newcommand{\Bc}{{\cal B}}
\newcommand{\Cc}{{\cal C}}
\newcommand{\Dc}{{\cal D}}
\newcommand{\Ec}{{\cal E}}
\newcommand{\Fc}{{\cal F}}
\newcommand{\Gc}{{\cal G}}
\newcommand{\Hc}{{\cal H}}
\newcommand{\Ic}{{\cal I}}
\newcommand{\Jc}{{\cal J}}
\newcommand{\Kc}{{\cal K}}
\newcommand{\Lc}{{\cal L}}
\newcommand{\Mc}{{\cal M}}
\newcommand{\Nc}{{\cal N}}
\newcommand{\nc}{{\cal n}}
\newcommand{\Oc}{{\cal O}}
\newcommand{\Pc}{{\cal P}}
\newcommand{\Qc}{{\cal Q}}
\newcommand{\Rc}{{\cal R}}
\newcommand{\Sc}{{\cal S}}
\newcommand{\Tc}{{\cal T}}
\newcommand{\Uc}{{\cal U}}
\newcommand{\Wc}{{\cal W}}
\newcommand{\Vc}{{\cal V}}
\newcommand{\Xc}{{\cal X}}
\newcommand{\Yc}{{\cal Y}}
\newcommand{\Zc}{{\cal Z}}

% Bold greek letters

\newcommand{\alphav}{\hbox{\boldmath$\alpha$}}
\newcommand{\betav}{\hbox{\boldmath$\beta$}}
\newcommand{\gammav}{\hbox{\boldmath$\gamma$}}
\newcommand{\deltav}{\hbox{\boldmath$\delta$}}
\newcommand{\etav}{\hbox{\boldmath$\eta$}}
\newcommand{\lambdav}{\hbox{\boldmath$\lambda$}}
\newcommand{\epsilonv}{\hbox{\boldmath$\epsilon$}}
\newcommand{\nuv}{\hbox{\boldmath$\nu$}}
\newcommand{\muv}{\hbox{\boldmath$\mu$}}
\newcommand{\zetav}{\hbox{\boldmath$\zeta$}}
\newcommand{\phiv}{\hbox{\boldmath$\phi$}}
\newcommand{\psiv}{\hbox{\boldmath$\psi$}}
\newcommand{\thetav}{\hbox{\boldmath$\theta$}}
\newcommand{\tauv}{\hbox{\boldmath$\tau$}}
\newcommand{\omegav}{\hbox{\boldmath$\omega$}}
\newcommand{\xiv}{\hbox{\boldmath$\xi$}}
\newcommand{\sigmav}{\hbox{\boldmath$\sigma$}}
\newcommand{\piv}{\hbox{\boldmath$\pi$}}
\newcommand{\rhov}{\hbox{\boldmath$\rho$}}
\newcommand{\upsilonv}{\hbox{\boldmath$\upsilon$}}

\newcommand{\Gammam}{\hbox{\boldmath$\Gamma$}}
\newcommand{\Lambdam}{\hbox{\boldmath$\Lambda$}}
\newcommand{\Deltam}{\hbox{\boldmath$\Delta$}}
\newcommand{\Sigmam}{\hbox{\boldmath$\Sigma$}}
\newcommand{\Phim}{\hbox{\boldmath$\Phi$}}
\newcommand{\Pim}{\hbox{\boldmath$\Pi$}}
\newcommand{\Psim}{\hbox{\boldmath$\Psi$}}
\newcommand{\Thetam}{\hbox{\boldmath$\Theta$}}
\newcommand{\Omegam}{\hbox{\boldmath$\Omega$}}
\newcommand{\Xim}{\hbox{\boldmath$\Xi$}}


% Sans Serif small case

\newcommand{\Gsf}{{\sf G}}

\newcommand{\asf}{{\sf a}}
\newcommand{\bsf}{{\sf b}}
\newcommand{\csf}{{\sf c}}
\newcommand{\dsf}{{\sf d}}
\newcommand{\esf}{{\sf e}}
\newcommand{\fsf}{{\sf f}}
\newcommand{\gsf}{{\sf g}}
\newcommand{\hsf}{{\sf h}}
\newcommand{\isf}{{\sf i}}
\newcommand{\jsf}{{\sf j}}
\newcommand{\ksf}{{\sf k}}
\newcommand{\lsf}{{\sf l}}
\newcommand{\msf}{{\sf m}}
\newcommand{\nsf}{{\sf n}}
\newcommand{\osf}{{\sf o}}
\newcommand{\psf}{{\sf p}}
\newcommand{\qsf}{{\sf q}}
\newcommand{\rsf}{{\sf r}}
\newcommand{\ssf}{{\sf s}}
\newcommand{\tsf}{{\sf t}}
\newcommand{\usf}{{\sf u}}
\newcommand{\wsf}{{\sf w}}
\newcommand{\vsf}{{\sf v}}
\newcommand{\xsf}{{\sf x}}
\newcommand{\ysf}{{\sf y}}
\newcommand{\zsf}{{\sf z}}


% mixed symbols

\newcommand{\sinc}{{\hbox{sinc}}}
\newcommand{\diag}{{\hbox{diag}}}
\renewcommand{\det}{{\hbox{det}}}
\newcommand{\trace}{{\hbox{tr}}}
\newcommand{\sign}{{\hbox{sign}}}
\renewcommand{\arg}{{\hbox{arg}}}
\newcommand{\var}{{\hbox{var}}}
\newcommand{\cov}{{\hbox{cov}}}
\newcommand{\Ei}{{\rm E}_{\rm i}}
\renewcommand{\Re}{{\rm Re}}
\renewcommand{\Im}{{\rm Im}}
\newcommand{\eqdef}{\stackrel{\Delta}{=}}
\newcommand{\defines}{{\,\,\stackrel{\scriptscriptstyle \bigtriangleup}{=}\,\,}}
\newcommand{\<}{\left\langle}
\renewcommand{\>}{\right\rangle}
\newcommand{\herm}{{\sf H}}
\newcommand{\trasp}{{\sf T}}
\newcommand{\transp}{{\sf T}}
\renewcommand{\vec}{{\rm vec}}
\newcommand{\Psf}{{\sf P}}
\newcommand{\SINR}{{\sf SINR}}
\newcommand{\SNR}{{\sf SNR}}
\newcommand{\MMSE}{{\sf MMSE}}
\newcommand{\REF}{{\RED [REF]}}

% Markov chain
\usepackage{stmaryrd} % for \mkv 
\newcommand{\mkv}{-\!\!\!\!\minuso\!\!\!\!-}

% Colors

\newcommand{\RED}{\color[rgb]{1.00,0.10,0.10}}
\newcommand{\BLUE}{\color[rgb]{0,0,0.90}}
\newcommand{\GREEN}{\color[rgb]{0,0.80,0.20}}

%%%%%%%%%%%%%%%%%%%%%%%%%%%%%%%%%%%%%%%%%%
\usepackage{hyperref}
\hypersetup{
    bookmarks=true,         % show bookmarks bar?
    unicode=false,          % non-Latin characters in AcrobatÕs bookmarks
    pdftoolbar=true,        % show AcrobatÕs toolbar?
    pdfmenubar=true,        % show AcrobatÕs menu?
    pdffitwindow=false,     % window fit to page when opened
    pdfstartview={FitH},    % fits the width of the page to the window
%    pdftitle={My title},    % title
%    pdfauthor={Author},     % author
%    pdfsubject={Subject},   % subject of the document
%    pdfcreator={Creator},   % creator of the document
%    pdfproducer={Producer}, % producer of the document
%    pdfkeywords={keyword1} {key2} {key3}, % list of keywords
    pdfnewwindow=true,      % links in new window
    colorlinks=true,       % false: boxed links; true: colored links
    linkcolor=red,          % color of internal links (change box color with linkbordercolor)
    citecolor=green,        % color of links to bibliography
    filecolor=blue,      % color of file links
    urlcolor=blue           % color of external links
}
%%%%%%%%%%%%%%%%%%%%%%%%%%%%%%%%%%%%%%%%%%%



\begin{document}
%
\title{Evaluating Differential Privacy on Correlated Datasets Using Pointwise Maximal Leakage}
%
\titlerunning{Evaluating DP on Correlated Datasets Using PML}
% If the paper title is too long for the running head, you can set
% an abbreviated paper title here
%
\author{Sara~Saeidian \and Tobias~J.~Oechtering \and Mikael~Skoglund}
%
\authorrunning{S.~Saeidian et al.}
% First names are abbreviated in the running head.
% If there are more than two authors, 'et al.' is used.
%
\institute{KTH Royal Institute of Technology, 100 44 Stockholm, Sweden\\
\email{\{saeidian,oech,skoglund\}@kth.se}}
%
\maketitle              % typeset the header of the contribution
%
\begin{abstract}
Data-driven advancements significantly contribute to societal progress, yet they also pose substantial risks to privacy. In this landscape, \emph{differential privacy} (DP) has become a cornerstone in privacy preservation efforts. However, the adequacy of DP in scenarios involving correlated datasets has sometimes been questioned and multiple studies have hinted at potential vulnerabilities. In this work, we delve into the nuances of applying DP to correlated datasets by leveraging the concept of \emph{pointwise maximal leakage} (PML) for a quantitative assessment of information leakage. Our investigation reveals that DP's guarantees can be arbitrarily weak for correlated databases when assessed through the lens of PML. More precisely, we prove the existence of a pure DP mechanism with PML levels arbitrarily close to that of a mechanism which releases individual entries from a database without any perturbation. By shedding light on the limitations of DP on correlated datasets, our work aims to foster a deeper understanding of subtle privacy risks and highlight the need for the development of more effective privacy-preserving mechanisms tailored to diverse scenarios.


\keywords{Pointwise maximal leakage  \and Differential privacy \and Correlated data.}
\end{abstract}
%
%
%
\section{Introduction}
\label{sec:intro}
\section{Introduction}


\begin{figure}[t]
\centering
\includegraphics[width=0.6\columnwidth]{figures/evaluation_desiderata_V5.pdf}
\vspace{-0.5cm}
\caption{\systemName is a platform for conducting realistic evaluations of code LLMs, collecting human preferences of coding models with real users, real tasks, and in realistic environments, aimed at addressing the limitations of existing evaluations.
}
\label{fig:motivation}
\end{figure}

\begin{figure*}[t]
\centering
\includegraphics[width=\textwidth]{figures/system_design_v2.png}
\caption{We introduce \systemName, a VSCode extension to collect human preferences of code directly in a developer's IDE. \systemName enables developers to use code completions from various models. The system comprises a) the interface in the user's IDE which presents paired completions to users (left), b) a sampling strategy that picks model pairs to reduce latency (right, top), and c) a prompting scheme that allows diverse LLMs to perform code completions with high fidelity.
Users can select between the top completion (green box) using \texttt{tab} or the bottom completion (blue box) using \texttt{shift+tab}.}
\label{fig:overview}
\end{figure*}

As model capabilities improve, large language models (LLMs) are increasingly integrated into user environments and workflows.
For example, software developers code with AI in integrated developer environments (IDEs)~\citep{peng2023impact}, doctors rely on notes generated through ambient listening~\citep{oberst2024science}, and lawyers consider case evidence identified by electronic discovery systems~\citep{yang2024beyond}.
Increasing deployment of models in productivity tools demands evaluation that more closely reflects real-world circumstances~\citep{hutchinson2022evaluation, saxon2024benchmarks, kapoor2024ai}.
While newer benchmarks and live platforms incorporate human feedback to capture real-world usage, they almost exclusively focus on evaluating LLMs in chat conversations~\citep{zheng2023judging,dubois2023alpacafarm,chiang2024chatbot, kirk2024the}.
Model evaluation must move beyond chat-based interactions and into specialized user environments.



 

In this work, we focus on evaluating LLM-based coding assistants. 
Despite the popularity of these tools---millions of developers use Github Copilot~\citep{Copilot}---existing
evaluations of the coding capabilities of new models exhibit multiple limitations (Figure~\ref{fig:motivation}, bottom).
Traditional ML benchmarks evaluate LLM capabilities by measuring how well a model can complete static, interview-style coding tasks~\citep{chen2021evaluating,austin2021program,jain2024livecodebench, white2024livebench} and lack \emph{real users}. 
User studies recruit real users to evaluate the effectiveness of LLMs as coding assistants, but are often limited to simple programming tasks as opposed to \emph{real tasks}~\citep{vaithilingam2022expectation,ross2023programmer, mozannar2024realhumaneval}.
Recent efforts to collect human feedback such as Chatbot Arena~\citep{chiang2024chatbot} are still removed from a \emph{realistic environment}, resulting in users and data that deviate from typical software development processes.
We introduce \systemName to address these limitations (Figure~\ref{fig:motivation}, top), and we describe our three main contributions below.


\textbf{We deploy \systemName in-the-wild to collect human preferences on code.} 
\systemName is a Visual Studio Code extension, collecting preferences directly in a developer's IDE within their actual workflow (Figure~\ref{fig:overview}).
\systemName provides developers with code completions, akin to the type of support provided by Github Copilot~\citep{Copilot}. 
Over the past 3 months, \systemName has served over~\completions suggestions from 10 state-of-the-art LLMs, 
gathering \sampleCount~votes from \userCount~users.
To collect user preferences,
\systemName presents a novel interface that shows users paired code completions from two different LLMs, which are determined based on a sampling strategy that aims to 
mitigate latency while preserving coverage across model comparisons.
Additionally, we devise a prompting scheme that allows a diverse set of models to perform code completions with high fidelity.
See Section~\ref{sec:system} and Section~\ref{sec:deployment} for details about system design and deployment respectively.



\textbf{We construct a leaderboard of user preferences and find notable differences from existing static benchmarks and human preference leaderboards.}
In general, we observe that smaller models seem to overperform in static benchmarks compared to our leaderboard, while performance among larger models is mixed (Section~\ref{sec:leaderboard_calculation}).
We attribute these differences to the fact that \systemName is exposed to users and tasks that differ drastically from code evaluations in the past. 
Our data spans 103 programming languages and 24 natural languages as well as a variety of real-world applications and code structures, while static benchmarks tend to focus on a specific programming and natural language and task (e.g. coding competition problems).
Additionally, while all of \systemName interactions contain code contexts and the majority involve infilling tasks, a much smaller fraction of Chatbot Arena's coding tasks contain code context, with infilling tasks appearing even more rarely. 
We analyze our data in depth in Section~\ref{subsec:comparison}.



\textbf{We derive new insights into user preferences of code by analyzing \systemName's diverse and distinct data distribution.}
We compare user preferences across different stratifications of input data (e.g., common versus rare languages) and observe which affect observed preferences most (Section~\ref{sec:analysis}).
For example, while user preferences stay relatively consistent across various programming languages, they differ drastically between different task categories (e.g. frontend/backend versus algorithm design).
We also observe variations in user preference due to different features related to code structure 
(e.g., context length and completion patterns).
We open-source \systemName and release a curated subset of code contexts.
Altogether, our results highlight the necessity of model evaluation in realistic and domain-specific settings.






\section{Preliminaries}
\label{sec:background}
\section{Background}\label{sec:backgrnd}

\subsection{Cold Start Latency and Mitigation Techniques}

Traditional FaaS platforms mitigate cold starts through snapshotting, lightweight virtualization, and warm-state management. Snapshot-based methods like \textbf{REAP} and \textbf{Catalyzer} reduce initialization time by preloading or restoring container states but require significant memory and I/O resources, limiting scalability~\cite{dong_catalyzer_2020, ustiugov_benchmarking_2021}. Lightweight virtualization solutions, such as \textbf{Firecracker} microVMs, achieve fast startup times with strong isolation but depend on robust infrastructure, making them less adaptable to fluctuating workloads~\cite{agache_firecracker_2020}. Warm-state management techniques like \textbf{Faa\$T}~\cite{romero_faa_2021} and \textbf{Kraken}~\cite{vivek_kraken_2021} keep frequently invoked containers ready, balancing readiness and cost efficiency under predictable workloads but incurring overhead when demand is erratic~\cite{romero_faa_2021, vivek_kraken_2021}. While these methods perform well in resource-rich cloud environments, their resource intensity challenges applicability in edge settings.

\subsubsection{Edge FaaS Perspective}

In edge environments, cold start mitigation emphasizes lightweight designs, resource sharing, and hybrid task distribution. Lightweight execution environments like unikernels~\cite{edward_sock_2018} and \textbf{Firecracker}~\cite{agache_firecracker_2020}, as used by \textbf{TinyFaaS}~\cite{pfandzelter_tinyfaas_2020}, minimize resource usage and initialization delays but require careful orchestration to avoid resource contention. Function co-location, demonstrated by \textbf{Photons}~\cite{v_dukic_photons_2020}, reduces redundant initializations by sharing runtime resources among related functions, though this complicates isolation in multi-tenant setups~\cite{v_dukic_photons_2020}. Hybrid offloading frameworks like \textbf{GeoFaaS}~\cite{malekabbasi_geofaas_2024} balance edge-cloud workloads by offloading latency-tolerant tasks to the cloud and reserving edge resources for real-time operations, requiring reliable connectivity and efficient task management. These edge-specific strategies address cold starts effectively but introduce challenges in scalability and orchestration.

\subsection{Predictive Scaling and Caching Techniques}

Efficient resource allocation is vital for maintaining low latency and high availability in serverless platforms. Predictive scaling and caching techniques dynamically provision resources and reduce cold start latency by leveraging workload prediction and state retention.
Traditional FaaS platforms use predictive scaling and caching to optimize resources, employing techniques (OFC, FaasCache) to reduce cold starts. However, these methods rely on centralized orchestration and workload predictability, limiting their effectiveness in dynamic, resource-constrained edge environments.



\subsubsection{Edge FaaS Perspective}

Edge FaaS platforms adapt predictive scaling and caching techniques to constrain resources and heterogeneous environments. \textbf{EDGE-Cache}~\cite{kim_delay-aware_2022} uses traffic profiling to selectively retain high-priority functions, reducing memory overhead while maintaining readiness for frequent requests. Hybrid frameworks like \textbf{GeoFaaS}~\cite{malekabbasi_geofaas_2024} implement distributed caching to balance resources between edge and cloud nodes, enabling low-latency processing for critical tasks while offloading less critical workloads. Machine learning methods, such as clustering-based workload predictors~\cite{gao_machine_2020} and GRU-based models~\cite{guo_applying_2018}, enhance resource provisioning in edge systems by efficiently forecasting workload spikes. These innovations effectively address cold start challenges in edge environments, though their dependency on accurate predictions and robust orchestration poses scalability challenges.

\subsection{Decentralized Orchestration, Function Placement, and Scheduling}

Efficient orchestration in serverless platforms involves workload distribution, resource optimization, and performance assurance. While traditional FaaS platforms rely on centralized control, edge environments require decentralized and adaptive strategies to address unique challenges such as resource constraints and heterogeneous hardware.



\subsubsection{Edge FaaS Perspective}

Edge FaaS platforms adopt decentralized and adaptive orchestration frameworks to meet the demands of resource-constrained environments. Systems like \textbf{Wukong} distribute scheduling across edge nodes, enhancing data locality and scalability while reducing network latency. Lightweight frameworks such as \textbf{OpenWhisk Lite}~\cite{kravchenko_kpavelopenwhisk-light_2024} optimize resource allocation by decentralizing scheduling policies, minimizing cold starts and latency in edge setups~\cite{benjamin_wukong_2020}. Hybrid solutions like \textbf{OpenFaaS}~\cite{noauthor_openfaasfaas_2024} and \textbf{EdgeMatrix}~\cite{shen_edgematrix_2023} combine edge-cloud orchestration to balance resource utilization, retaining latency-sensitive functions at the edge while offloading non-critical workloads to the cloud. While these approaches improve flexibility, they face challenges in maintaining coordination and ensuring consistent performance across distributed nodes.



\section{Privacy for Correlated Databases: PML vs. DP}
\label{sec:main}
As discussed in Section~\ref{sec:intro}, the objective of our work is to understand the relation between PML and DP when $X$ is a database with correlated entries. It turns out that pure DP mechanisms can have poor privacy performance when assessed through the lens of PML. Before we formally state our result, we need to define the PML of a mechanism that releases an entry from the database without perturbation. 

Let $X = (D_1, \ldots, D_n)$ be a database. Given a distribution $P_X \in \cP_\cX$ over $X$ and $i \in [n]$, let $P_{D_i}$ denote the distribution of the $i$-th entry in a database, obtained by marginalizing $P_X$ over $P_{D_{-i}}$, described by~\eqref{eq:marginal}. We use
\begin{equation*}
    \varepsilon_\mathrm{max} (D_i) \coloneqq \log \frac{1}{\min\limits_{d \in \cD} \, P_{D_i}(d)},
\end{equation*}
 to denote the largest amount of information that can leak about $D_i$ through any mechanism. By \eqref{eq:pml_bounds}, $\varepsilon_\mathrm{max} (D_i)$ is equal to the PML of a mechanism that releases $D_i$ with no randomization.  

\begin{theorem}
\label{thm:main}
For each $\delta >0$ and $\varepsilon >0$ there exists a database $X = (D_1, \ldots, D_n)$, $i \in [n]$, a mechanism $P_{Y \mid X}$ satisfying $\varepsilon$-DP, and $y \in \bR$ such that
\begin{equation*}
    \ell(D_i \to y) > \varepsilon_\mathrm{max}(D_i) - \delta.
\end{equation*}
\end{theorem}

\begin{proof}
To prove the statement, we construct a binary database where the entries are strongly correlated: If one entry is zero (resp. one), then the other entries are also likely to be zero (resp. one). We then demonstrate that if we use the Laplace mechanism to answer the counting query on this database, then the resulting PML is significantly high.

Let $X = (D_1, \ldots, D_{n +1})$ be a database containing $n+1$ binary entries.\footnote{We consider a database of size $n+1$ instead of $n$ for notational convenience.} Suppose $P_{D_1}(0) = 1 - P_{D_1}(1) = \alpha$ with $0 < \alpha < 0.5$. Let $D_-=(D_2, \ldots, D_{n+1})$. Fix a constant $0 < \eta < 1$ and suppose the distribution of $D_-$ depends on $D_1$ as follows:  
\begin{gather*}
    P_{D_- \mid D_1=1}(d_-) = \begin{cases}
        \eta, &\quad \mathrm{if} \; d_- = 1^{n},\\
        \frac{1 - \eta}{2^{n} -1}, &\quad \mathrm{otherwise}.
    \end{cases}\\
    P_{D_- \mid D_1=0}(d_-) = \begin{cases}
        \eta, &\quad \mathrm{if} \; d_- = 0^{n},\\
        \frac{1 - \eta}{2^{n} -1}, &\quad \mathrm{otherwise}.
    \end{cases}
\end{gather*}
Suppose our goal is to release the empirical frequency of the ones in the database using the Laplace mechanism, i.e., $Y \mid X=x \sim \mathrm{Lap}(\frac{\norm{x}_1}{n+1}, b)$, where $\norm{x}_1$ denotes the $\ell_1$-norm of $x \in \{0,1\}^{n+1}$. Note that the empirical frequency has global sensitivity $\frac{1}{n+1}$, thus the Laplace mechanism with scale parameter $b = \frac{1}{\varepsilon (n+1)}$ satisfies $\varepsilon$-DP. However, here we show that the Laplace mechanism is insufficient for protecting $D_1$. To demonstrate this, we calculate the PML $\ell(D_1 \to y)$ with $y \leq 0$, which depends on the distributions $P_{Y \mid D_1=1}, P_{Y \mid D_1=0}$ and $P_{Y}$. 

First, we calculate $P_{Y \mid D_1=1}(y)$ assuming $y \leq 0$:
\begin{align*}
    &P_{Y \mid D_1=1}(y) = \sum_{d_- \in \{0,1\}^{n}} P_{Y \mid D_1=1, D_-=d_-}(y) \cdot P_{D_- \mid D_1=1}(d_-)\\
    &= \eta \; P_{Y \mid D_1=1, D_-= 1^n}(y) + \frac{1-\eta}{2^{n} -1} \sum_{d_- \in \{0,1\}^{n} \setminus 1^{n}} P_{Y \mid D_1=1, D_-=d_-}(y)\\
    &= \frac{\eta}{2b} \exp \left(- \frac{\abs{y - 1}}{b} \right) + \frac{1-\eta}{2b (2^{n} -1)} \sum_{d_- \in \{0,1\}^{n} \setminus 1^{n}} \exp \left(- \frac{\abs{y - \frac{1}{n+1} - \frac{\norm{d_-}_1}{n+1}}}{b} \right)\\
    &= \frac{\eta}{2b} \exp \left(- \frac{\abs{y - 1}}{b} \right) + \frac{1-\eta}{2b (2^{n} -1)} \cdot \sum_{i=0}^{n-1} \binom{n}{i} \exp \left(- \frac{\abs{y - \frac{1}{n+1} - \frac{i}{n+1}}}{b} \right)\\
    &= \frac{\eta}{2b} \exp \left(- \frac{\abs{y - 1}}{b} \right) + \\
    &\hspace{3em} \frac{1-\eta}{2b (2^{n} -1)} \bigg[ \sum_{i=0}^{n} \binom{n}{i} \exp \left(- \frac{\abs{y - \frac{1}{n+1} - \frac{i}{n+1}}}{b} \right) - \exp \left(- \frac{\abs{y -1}}{b} \right) \bigg]\\
    &= \frac{2^n \eta - 1}{2b (2^n - 1)} \exp \left(\frac{y - 1}{b} \right) + \\
    &\hspace{.3\textwidth} \frac{1-\eta}{2b (2^{n} -1)} \exp \left(\frac{y - \frac{1}{n+1}}{b}\right) \cdot \left(1 + \exp (- \frac{1}{b(n+1)}) \right)^n\\
    &= \frac{1}{2b (2^{n} -1)}\exp(\frac{y}{b}) \bigg[\Big(2^n \eta - 1\Big) \exp \left(-\frac{1}{b} \right) + \\
    &\hspace{.3\textwidth} (1-\eta) \, \exp \left(-\frac{1}{b(n+1)}\right) \left(1 + \exp (- \frac{1}{b(n+1)}) \right)^n \bigg].
\end{align*}

Similarly, we can calculate $P_{Y \mid D_1=0}(y)$ assuming $y \leq 0$:  
\begin{align*}
    &P_{Y \mid D_1=0}(y) = \sum_{d_- \in \{0,1\}^{n}} P_{Y \mid D_1=0, D_-=d_-}(y) \cdot P_{D_- \mid D_1=0}(d_-) \\
    &= \eta \; P_{Y \mid D_1=0, D_-= 0^n}(y) + \frac{1-\eta}{2^{n} -1} \; \sum_{d_- \in \{0,1\}^{n} \setminus 0^{n}} P_{Y \mid D_1=0, D_-=d_-}(y)\\
    &= \frac{\eta}{2b} \exp \left(\frac{y}{b} \right) + \frac{1-\eta}{2b (2^{n} -1)} \; \sum_{d_- \in \{0,1\}^{n} \setminus 0^{n}} \exp \left(\frac{y - \frac{\norm{d_-}_1}{n+1}}{b} \right)\\
    &= \frac{1}{2b} \exp \left(\frac{y}{b} \right) \left[\eta + \frac{1-\eta}{2^{n} -1}  \sum_{i=1}^{n} \binom{n}{i} \exp \left(- \frac{i}{b(n+1)} \right) \right] \\  
    &= \frac{1}{2b} \exp \left(\frac{y}{b} \right) \left[\eta - \frac{1-\eta}{2^{n} -1} + \frac{1-\eta}{2^{n} -1}  \sum_{i=0}^{n} \binom{n}{i} \exp \left(- \frac{i}{b(n+1)} \right) \right] \\  
    &= \frac{1}{2b (2^{n} -1)} \exp \left(\frac{y}{b} \right) \left[2^n  \eta - 1 + (1-\eta) \left(1 + \exp (- \frac{1}{b(n +1)}) \right)^n\right].  \\  
\end{align*}
Since $\exp(-x) \leq 1$ for $x \geq 0$, then $P_{Y \mid D_1=1}(y) \leq P_{Y \mid D_1=0}(y)$ when $y \leq 0$. Next, we calculate $P_Y(y)$ for $y \leq 0$: 
\begin{align*}
    &P_Y(y) = (1 - \alpha) P_{Y \mid D_1 = 1}(y) + \alpha P_{Y \mid D_1=0}(y) \\
    &= \frac{1}{2b (2^{n} -1)} \exp \left(\frac{y}{b} \right) \Bigg[\Big(2^n \eta - 1\Big) (1 - \alpha) \exp \left(-\frac{1}{b} \right) + \\
    &\hspace{7em} (1-\eta) (1 - \alpha)\, \exp \left(-\frac{1}{b(n+1)}\right) \left(1 + \exp (- \frac{1}{b(n+1)}) \right)^n + \\
    &\hspace{7em} \big(2^n \eta - 1\big) \alpha +  (1-\eta) \alpha \left(1 + \exp (- \frac{1}{b(n +1)}) \right)^n \Bigg]\\
    &\leq \frac{1}{2b (2^{n} -1)} \exp \left(\frac{y}{b} \right) \Bigg[\Big(2^n \eta - 1\Big) \Big( (1 - \alpha)\, \exp \left(-\frac{1}{b}\right) + \alpha \Big) + \\
    &\hspace{1em} (1-\eta) (1 - \alpha)\, \left(1 + \exp (- \frac{1}{b(n+1)}) \right)^n + (1-\eta) \alpha \left(1 + \exp (- \frac{1}{b(n +1)}) \right)^n \Bigg]\\
    &= \frac{1}{2b (2^{n} -1)} \exp \left(\frac{y}{b} \right) \Bigg[\Big(2^n \eta - 1\Big) \Big( (1 - \alpha)\, \exp \left(-\frac{1}{b}\right) + \alpha \Big) +\\
    &\hspace{18em}(1-\eta) \left(1 + \exp (- \frac{1}{b(n+1)}) \right)^n  \Bigg]\\
    &\leq \frac{1}{2b (2^{n} -1)} \exp \left(\frac{y}{b} \right) \Bigg[2^n  \eta \Big( (1 - \alpha)\, \exp \left(-\frac{1}{b}\right) + \alpha \Big) +\\
    &\hspace{18em} (1-\eta) \left(1 + \exp (- \frac{1}{b(n+1)}) \right)^n  \Bigg]. 
\end{align*}
Using $b(n+1) = \frac{1}{\varepsilon}$ to achieve $\varepsilon$-DP, we obtain the following lower bound on $\ell(D_1 \to y)$ with $y \leq 0$:
\begin{align*}
    \ell(D_1 \to y) &= \log \frac{\max\limits_{d_1 \in \{0,1\}} P_{Y \mid D_1 = d_1}(y)}{P_Y(y)} = \log \frac{P_{Y \mid D_1 = 0}(y)}{P_Y(y)}\\[0.5em]
    &\geq \log \frac{2^n \eta + \left(1 + e^{-\varepsilon} \right)^n (1-\eta) - 1}{2^n  \eta \Big( (1 - \alpha)\, \exp (-\varepsilon  n-\varepsilon) + \alpha \Big) + \left(1 + e^{-\varepsilon} \right)^n (1-\eta)}\\[0.5em]
    &= \log \frac{2^n  \eta +  \left(1 + e^{-\varepsilon} \right)^n \, (1-\eta) - 1}{2^n  \eta \, \alpha  + \left(\frac{2}{e^\varepsilon}\right)^n \eta (e^{-\varepsilon}) (1 - \alpha) +  \left(1 + e^{-\varepsilon} \right)^n \, (1-\eta) }.
 \end{align*}    

Note that $1 + e^{-\varepsilon} < 2$ and $\frac{2}{e^\varepsilon} < 2$ for all $\varepsilon > 0$. Therefore, when $n$ is large the dominating term in the numerator is $2^n \eta$ and the dominating term in the denominator is $2^n \eta \, \alpha$. Hence, as $n \to \infty$, the lower bound on $\ell(D_1 \to y)$ approaches $\varepsilon_\mathrm{max} (D_1) = \log \frac{1}{\alpha}$. This proves that for each $\delta >0$, there exists an integer $n$, a database $X$ of size $n$, and a mechanism satisfying $\varepsilon$-DP such that 
\begin{equation*}
    \ell(D_1 \to y) > \varepsilon_\mathrm{max}(D_1) - \delta.
\end{equation*}
\qed
\end{proof}



The proof of Theorem~\ref{thm:main} relies on a database exhibiting what may be considered as pathologically strong correlations: If the first entry is zero (resp. one) then all other entries are likely to be zero (resp. one) with a constant probability that does not diminish with growing database size $n$. However, it is important to note that the theorem holds true even in more realistic scenarios characterized by weaker correlations. Specifically, the asymptotic lower bound of $\varepsilon_\mathrm{max}(D_1)$ for PML remains applicable even if $\eta$ diminishes at a polynomial rate, i.e., if $\eta = \Theta(\frac{1}{n^r})$ for some constant $r \geq 1$.



\section{Conclusions}
\label{sec:discussion}
\section{Discussion of Assumptions}\label{sec:discussion}
In this paper, we have made several assumptions for the sake of clarity and simplicity. In this section, we discuss the rationale behind these assumptions, the extent to which these assumptions hold in practice, and the consequences for our protocol when these assumptions hold.

\subsection{Assumptions on the Demand}

There are two simplifying assumptions we make about the demand. First, we assume the demand at any time is relatively small compared to the channel capacities. Second, we take the demand to be constant over time. We elaborate upon both these points below.

\paragraph{Small demands} The assumption that demands are small relative to channel capacities is made precise in \eqref{eq:large_capacity_assumption}. This assumption simplifies two major aspects of our protocol. First, it largely removes congestion from consideration. In \eqref{eq:primal_problem}, there is no constraint ensuring that total flow in both directions stays below capacity--this is always met. Consequently, there is no Lagrange multiplier for congestion and no congestion pricing; only imbalance penalties apply. In contrast, protocols in \cite{sivaraman2020high, varma2021throughput, wang2024fence} include congestion fees due to explicit congestion constraints. Second, the bound \eqref{eq:large_capacity_assumption} ensures that as long as channels remain balanced, the network can always meet demand, no matter how the demand is routed. Since channels can rebalance when necessary, they never drop transactions. This allows prices and flows to adjust as per the equations in \eqref{eq:algorithm}, which makes it easier to prove the protocol's convergence guarantees. This also preserves the key property that a channel's price remains proportional to net money flow through it.

In practice, payment channel networks are used most often for micro-payments, for which on-chain transactions are prohibitively expensive; large transactions typically take place directly on the blockchain. For example, according to \cite{river2023lightning}, the average channel capacity is roughly $0.1$ BTC ($5,000$ BTC distributed over $50,000$ channels), while the average transaction amount is less than $0.0004$ BTC ($44.7k$ satoshis). Thus, the small demand assumption is not too unrealistic. Additionally, the occasional large transaction can be treated as a sequence of smaller transactions by breaking it into packets and executing each packet serially (as done by \cite{sivaraman2020high}).
Lastly, a good path discovery process that favors large capacity channels over small capacity ones can help ensure that the bound in \eqref{eq:large_capacity_assumption} holds.

\paragraph{Constant demands} 
In this work, we assume that any transacting pair of nodes have a steady transaction demand between them (see Section \ref{sec:transaction_requests}). Making this assumption is necessary to obtain the kind of guarantees that we have presented in this paper. Unless the demand is steady, it is unreasonable to expect that the flows converge to a steady value. Weaker assumptions on the demand lead to weaker guarantees. For example, with the more general setting of stochastic, but i.i.d. demand between any two nodes, \cite{varma2021throughput} shows that the channel queue lengths are bounded in expectation. If the demand can be arbitrary, then it is very hard to get any meaningful performance guarantees; \cite{wang2024fence} shows that even for a single bidirectional channel, the competitive ratio is infinite. Indeed, because a PCN is a decentralized system and decisions must be made based on local information alone, it is difficult for the network to find the optimal detailed balance flow at every time step with a time-varying demand.  With a steady demand, the network can discover the optimal flows in a reasonably short time, as our work shows.

We view the constant demand assumption as an approximation for a more general demand process that could be piece-wise constant, stochastic, or both (see simulations in Figure \ref{fig:five_nodes_variable_demand}).
We believe it should be possible to merge ideas from our work and \cite{varma2021throughput} to provide guarantees in a setting with random demands with arbitrary means. We leave this for future work. In addition, our work suggests that a reasonable method of handling stochastic demands is to queue the transaction requests \textit{at the source node} itself. This queuing action should be viewed in conjunction with flow-control. Indeed, a temporarily high unidirectional demand would raise prices for the sender, incentivizing the sender to stop sending the transactions. If the sender queues the transactions, they can send them later when prices drop. This form of queuing does not require any overhaul of the basic PCN infrastructure and is therefore simpler to implement than per-channel queues as suggested by \cite{sivaraman2020high} and \cite{varma2021throughput}.

\subsection{The Incentive of Channels}
The actions of the channels as prescribed by the DEBT control protocol can be summarized as follows. Channels adjust their prices in proportion to the net flow through them. They rebalance themselves whenever necessary and execute any transaction request that has been made of them. We discuss both these aspects below.

\paragraph{On Prices}
In this work, the exclusive role of channel prices is to ensure that the flows through each channel remains balanced. In practice, it would be important to include other components in a channel's price/fee as well: a congestion price  and an incentive price. The congestion price, as suggested by \cite{varma2021throughput}, would depend on the total flow of transactions through the channel, and would incentivize nodes to balance the load over different paths. The incentive price, which is commonly used in practice \cite{river2023lightning}, is necessary to provide channels with an incentive to serve as an intermediary for different channels. In practice, we expect both these components to be smaller than the imbalance price. Consequently, we expect the behavior of our protocol to be similar to our theoretical results even with these additional prices.

A key aspect of our protocol is that channel fees are allowed to be negative. Although the original Lightning network whitepaper \cite{poon2016bitcoin} suggests that negative channel prices may be a good solution to promote rebalancing, the idea of negative prices in not very popular in the literature. To our knowledge, the only prior work with this feature is \cite{varma2021throughput}. Indeed, in papers such as \cite{van2021merchant} and \cite{wang2024fence}, the price function is explicitly modified such that the channel price is never negative. The results of our paper show the benefits of negative prices. For one, in steady state, equal flows in both directions ensure that a channel doesn't loose any money (the other price components mentioned above ensure that the channel will only gain money). More importantly, negative prices are important to ensure that the protocol selectively stifles acyclic flows while allowing circulations to flow. Indeed, in the example of Section \ref{sec:flow_control_example}, the flows between nodes $A$ and $C$ are left on only because the large positive price over one channel is canceled by the corresponding negative price over the other channel, leading to a net zero price.

Lastly, observe that in the DEBT control protocol, the price charged by a channel does not depend on its capacity. This is a natural consequence of the price being the Lagrange multiplier for the net-zero flow constraint, which also does not depend on the channel capacity. In contrast, in many other works, the imbalance price is normalized by the channel capacity \cite{ren2018optimal, lin2020funds, wang2024fence}; this is shown to work well in practice. The rationale for such a price structure is explained well in \cite{wang2024fence}, where this fee is derived with the aim of always maintaining some balance (liquidity) at each end of every channel. This is a reasonable aim if a channel is to never rebalance itself; the experiments of the aforementioned papers are conducted in such a regime. In this work, however, we allow the channels to rebalance themselves a few times in order to settle on a detailed balance flow. This is because our focus is on the long-term steady state performance of the protocol. This difference in perspective also shows up in how the price depends on the channel imbalance. \cite{lin2020funds} and \cite{wang2024fence} advocate for strictly convex prices whereas this work and \cite{varma2021throughput} propose linear prices.

\paragraph{On Rebalancing} 
Recall that the DEBT control protocol ensures that the flows in the network converge to a detailed balance flow, which can be sustained perpetually without any rebalancing. However, during the transient phase (before convergence), channels may have to perform on-chain rebalancing a few times. Since rebalancing is an expensive operation, it is worthwhile discussing methods by which channels can reduce the extent of rebalancing. One option for the channels to reduce the extent of rebalancing is to increase their capacity; however, this comes at the cost of locking in more capital. Each channel can decide for itself the optimum amount of capital to lock in. Another option, which we discuss in Section \ref{sec:five_node}, is for channels to increase the rate $\gamma$ at which they adjust prices. 

Ultimately, whether or not it is beneficial for a channel to rebalance depends on the time-horizon under consideration. Our protocol is based on the assumption that the demand remains steady for a long period of time. If this is indeed the case, it would be worthwhile for a channel to rebalance itself as it can make up this cost through the incentive fees gained from the flow of transactions through it in steady state. If a channel chooses not to rebalance itself, however, there is a risk of being trapped in a deadlock, which is suboptimal for not only the nodes but also the channel.

\section{Conclusion}
This work presents DEBT control: a protocol for payment channel networks that uses source routing and flow control based on channel prices. The protocol is derived by posing a network utility maximization problem and analyzing its dual minimization. It is shown that under steady demands, the protocol guides the network to an optimal, sustainable point. Simulations show its robustness to demand variations. The work demonstrates that simple protocols with strong theoretical guarantees are possible for PCNs and we hope it inspires further theoretical research in this direction.


\begin{credits}
\subsubsection{\ackname} This work has been supported by the Swedish Research Council (VR) under the grant 2023-04787 and Digital Futures center within the collaborative project DataLEASH. 


\subsubsection{\discintname}
The authors have no competing interests to declare that are
relevant to the content of this article.
\end{credits}
%
% ---- Bibliography ----
%
% BibTeX users should specify bibliography style 'splncs04'.
% References will then be sorted and formatted in the correct style.
%
\bibliographystyle{splncs04}
\bibliography{main}
%

\end{document}
