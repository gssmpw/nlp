% This is samplepaper.tex, a sample chapter demonstrating the
% LLNCS macro package for Springer Computer Science proceedings;
% Version 2.21 of 2022/01/12
%
\documentclass[runningheads]{llncs}
%
\usepackage[T1]{fontenc}
% T1 fonts will be used to generate the final print and online PDFs,
% so please use T1 fonts in your manuscript whenever possible.
% Other font encondings may result in incorrect characters.
%
\usepackage{graphicx}
% Used for displaying a sample figure. If possible, figure files should
% be included in EPS format.
%
% If you use the hyperref package, please uncomment the following two lines
% to display URLs in blue roman font according to Springer's eBook style:
%\usepackage{color}
%\renewcommand\UrlFont{\color{blue}\rmfamily}
%\urlstyle{rm}
%

%%%%%%%%%%%%%%%%%%%%%%%%%%%%%%%%%%%%%%%%%%%%%%%%%%%%%%%%%%%%%%%%%%%%%%%%%%%%%%

%% Beautiful mathematics
\usepackage{amsmath, amssymb, amsfonts} 
\usepackage{nicefrac}
\usepackage{mathtools}
\usepackage{bm, bbm}
\usepackage[scr=boondoxo,scrscaled=1.05]{mathalfa}

%% References in the correct format 
%\usepackage[square,numbers]{natbib}
%\def\bibfont{\footnotesize} % fix to have the same font size as without natbib

\usepackage[sort, compress, space]{cite}            


%% Enumerate nicely 
\usepackage{enumitem}

%% Different color comments and commenting large parts of the text
\usepackage{xcolor}
\usepackage{comment}
\usepackage{soul}

%% Hyper references
\usepackage{hyperref}
\usepackage{cleveref}
%\usepackage[numbers]{natbib}

\usepackage{tikz}
%\usepackage{thm-restate}
%% Appendix package
%\usepackage{appendix}

%% Random text to test spacing 
\usepackage{blindtext}

\usepackage{afterpage}

\usepackage{algorithm, algorithmic}    



\usepackage{dsfont}

\usepackage{tikz}
\usepackage{graphicx}
\usepackage{tikzscale}
\usepackage{pgfplots}
\pgfplotsset{compat=newest}
\usepackage{xfrac}

\usepackage{thm-restate}

%\usepackage{subcaption}

\usepackage{balance}

\usepackage{cite}
\usepackage{amsmath,amssymb,amsfonts}
\usepackage{balance}
\usepackage{algorithmic}
\usepackage{graphicx}
\usepackage{textcomp}
\usepackage{xcolor}
\usepackage{amsmath}
\usepackage{amssymb}
\usepackage[mathscr]{euscript}
\usepackage{comment}
\usepackage{xcolor}
\usepackage{enumitem} 
\usepackage{amsthm}


\newcommand{\thought}[1]{{\color[rgb]{0.2,0.39,0.66}(#1)}}
\newcommand{\todo}[1]{{\color[rgb]{1.0,0.0,0.0}(#1)}}
\newcommand{\hsh}[1]{{\color{green!50!black} Henrik: #1}}
\newcommand{\st}[1]{{\color{red!50!black} Sebastian: #1}}

\newcommand{\ulm}[1]{_{\scaleto{\mathrm{#1}}{3pt}}}
\newcommand\at[2]{\left.#1\right|_{#2}}











\newtheorem{assumption}{Assumption}

\DeclareMathOperator*{\argmax}{arg\,max}
\DeclareMathOperator*{\argmin}{arg\,min}

\newcommand{\swname}[1]{\texttt{#1}}
\newcommand{\ie}{i\/.\/e\/.,\/~}
\newcommand{\eg}{e\/.\/g\/.,\/~}
\newcommand{\cf}{cf\/.\/~}

\newcommand{\fig}{Fig\/.\/~}
\newcommand{\defn}{Def\/.\/~}
\newcommand{\sect}{Sec\/.\/~}
\newcommand{\tabl}{Tab\/.\/~}
\newcommand{\algo}{Algorithm~}
\newcommand{\theo}{Theorem~}

\newcommand{\bnnl}{3 hidden layers}
\newcommand{\bnnn}{50 neurons}
\newcommand{\bnna}{tanh activations}

\newcommand{\capt}[1]{\mdseries{\emph{#1}}}

\newcommand{\videolink}{at \url{https://youtu.be/_d7AqTRjz6g}}
\newcommand{\codelink}{\url{https://github.com/wheelbot/mini-wheelbot}}

\newcommand{\fakepar}[1]{\vspace{0mm}\noindent\textbf{#1.}}

\newcommand{\needref}{\textcolor{red}{[REF]}}

\newcommand{\plotfontsize}{9pt}


\begin{document}
%
\title{Evaluating Differential Privacy on Correlated Datasets Using Pointwise Maximal Leakage}
%
\titlerunning{Evaluating DP on Correlated Datasets Using PML}
% If the paper title is too long for the running head, you can set
% an abbreviated paper title here
%
\author{Sara~Saeidian \and Tobias~J.~Oechtering \and Mikael~Skoglund}
%
\authorrunning{S.~Saeidian et al.}
% First names are abbreviated in the running head.
% If there are more than two authors, 'et al.' is used.
%
\institute{KTH Royal Institute of Technology, 100 44 Stockholm, Sweden\\
\email{\{saeidian,oech,skoglund\}@kth.se}}
%
\maketitle              % typeset the header of the contribution
%
\begin{abstract}
Data-driven advancements significantly contribute to societal progress, yet they also pose substantial risks to privacy. In this landscape, \emph{differential privacy} (DP) has become a cornerstone in privacy preservation efforts. However, the adequacy of DP in scenarios involving correlated datasets has sometimes been questioned and multiple studies have hinted at potential vulnerabilities. In this work, we delve into the nuances of applying DP to correlated datasets by leveraging the concept of \emph{pointwise maximal leakage} (PML) for a quantitative assessment of information leakage. Our investigation reveals that DP's guarantees can be arbitrarily weak for correlated databases when assessed through the lens of PML. More precisely, we prove the existence of a pure DP mechanism with PML levels arbitrarily close to that of a mechanism which releases individual entries from a database without any perturbation. By shedding light on the limitations of DP on correlated datasets, our work aims to foster a deeper understanding of subtle privacy risks and highlight the need for the development of more effective privacy-preserving mechanisms tailored to diverse scenarios.


\keywords{Pointwise maximal leakage  \and Differential privacy \and Correlated data.}
\end{abstract}
%
%
%
\section{Introduction}
\label{sec:intro}
\section{Introduction}
\label{sec:intro}

\begin{figure*}[tb]
    \centering
    \includegraphics[width=0.848\linewidth]{figs/circuitnn.pdf} 
    \caption{Illustration of differentiable CircuitNN. CircuitNN is designed based on differentiable NAND gates. After DAS is guided by PI and PO pairs of the truth table, CircuitNN can get the precise circuit architecture logic equivalent to the truth table.}
    \label{fig:circuitnn}
\end{figure*}

% 1. Describe the importance of logic synthesis
% 2. Existing Problems
% (a) Neural Architecture Search: Unstable, Predefined Setting, etc.
% (b) Circuit Generation: Probabilistic Model, Logic Equivalence

With the rapid advancement of technology, the scale of integrated circuits (ICs) has expanded exponentially. 
This expansion has introduced significant challenges in chip manufacturing, particularly concerning power and area metrics.
A primary objective in IC design is achieving the same circuit function with fewer transistors, thereby reducing power usage and area occupancy.

Logic synthesis~\cite{hachtel2005logicsynth}, a critical step in electronic design automation (EDA), transforms behavioral-level circuit designs into optimized gate-level circuits, ultimately yielding the final IC layout. 
The primary goal of logic synthesis is to identify the physical implementation with the fewest gates for a given circuit function. 
This task constitutes a challenging NP-hard combinatorial optimization problem. 
Current logic synthesis tools~\cite{brayton2010abc, wolf2013yosys} rely on human-designed heuristics, often leading to sub-optimal outcomes.

Differentiable architecture search (DAS) techniques~\cite{liu2018darts, chu2020darts} offer novel perspectives on addressing challenges in this problem.
Circuit functions can be represented through truth tables, which map binary inputs to their corresponding outputs. 
Truth tables provide a precise representation of input-output relationships, ensuring the design of functionally equivalent circuits.
Inspired by this, researchers~\cite{deepmind2024ai4sys, wang2024tnet} have begun exploring the application of DAS to synthesize circuits directly from truth tables.
Specifically, \citet{deepmind2024ai4sys} proposed CircuitNN, a framework that learns differentiable connection structures with logic gates, enabling the automatic generation of logic circuits from truth tables.
This approach significantly reduces the complexity of traditional circuit generation. 
Building on this, \citet{wang2024tnet} introduced T-Net, a triangle-shaped variant of CircuitNN, incorporating regularization techniques to enhance the efficiency of DAS.

Despite these advancements, several challenges remain. 
The computational complexity of DAS grows quadratically with the number of gates, posing scalability issues.
Although triangle-shaped architecture~\cite{wang2024tnet} partially mitigates this problem, redundancy persists. 
%Additionally, DAS is susceptible to converging to local optima, limiting the ability to search architectures that satisfy the given truth tables~\cite{liu2018darts}. 
%Furthermore, hyperparameters (network depth and layer width) require extensive searches, introducing complexity and prolonging the synthesis process. 
Additionally, DAS is susceptible to converging to local optima~\cite{liu2018darts} and hyperparameters (network depth and layer width) require extensive searches. 
The challenges arise from the vast search space in DAS. 
% Even with predefined settings for CircuitNN, finding a configuration that meets the truth table requires extensive trial and error during the DAS process. 
Intuitively, limiting the search space through predefined parameters (network depth, gates per layer, and connection probabilities) can significantly reduce the complexity.

Recent advances~\cite{openai2023gpt4, abramson2024alphafold3, esser2024sd3, li2024mar} in conditional generative models have demonstrated remarkable performance across language, vision, and graph generation tasks. 
Motivated by these developments, we propose a novel approach to circuit generation that generates preliminary circuit structures to guide DAS in generating refined circuits matching specified truth tables. 
Firstly, we introduce CircuitVQ, a tokenizer with a discrete codebook for circuit tokenization. 
Built upon our Circuit AutoEncoder framework~\cite{hou2022graphmae,li2023maskgae,wu2025mgvga}, CircuitVQ is trained through a circuit reconstruction task. 
Specifically, the CircuitVQ encoder encodes input circuits into discrete tokens using a learnable codebook, while the decoder reconstructs the circuit adjacency matrix based on these tokens.
Subsequently, the CircuitVQ encoder serves as a circuit tokenizer for CircuitAR pretraining, which employs a masked autoregressive modeling paradigm~\cite{chang2022maskgit, li2023mage}. 
In this process, the discrete codes function as supervision signals. 
After training, CircuitAR can generate discrete tokens progressively, which can be decoded into initial circuit structures by the decoder of the CircuitVQ. 
These prior insights can guide DAS in producing refined circuits that match the target truth tables precisely.

Our key contributions can be summarized as follows:
\begin{itemize}
\item We introduce CircuitVQ, a circuit tokenizer that facilitates graph autoregressive modeling for circuit generation, based on our Circuit AutoEncoder framework;
\item Develop CircuitAR, a model trained using masked autoregressive modeling, which generates initial circuit structures conditioned on given truth tables;
\item Propose a refinement framework that integrates differentiable architecture search to produce functionally equivalent circuits guided by target truth tables;
\item Comprehensive experiments demonstrating the scalability and capability emergence of our CircuitAR and the superior performance of the proposed circuit generation approach.
\end{itemize}

% Motivation
% (a) Diffusion (Vision, Graph), Autoregressive (Language, Vision)
% (b) Circuit Generation for Predefined Setting
% (c) Neural Architecture Search for Strict Logic Equivalence

% Contribution
% (a) Circuit Tokenizer (new transformer arch, training strategy)
% (b) CircuitAR (train and gen strategies, post-ar strategy)
% (c) Extensive Evaluation including BitD (Bit Distance) for Scalability


\section{Preliminaries}
\label{sec:background}
\section{Basic Background: Supervised Learning and the PAC Model}
\label{sec:background}

At this point almost everyone has heard of machine learning (ML). Anyone likely to stumble upon this article will have also heard of its most influential special case, supervised learning, and those theoretically inclined will also be familiar with the PAC model. Nonetheless, I will set the stage by  recapping the basics.

\subsection{Basics of Supervised Learning}%Let's set the stage in any case

\emph{Supervised Learning} is the task of ``coming up'' with a function $f: \X \to \Y$ to ``explain'' or ``fit'' a sequence of input/output examples   $(x_1,y_1), \ldots, (x_n,y_n)$, with $x_i \in \X$ and $y_i \in \Y$.  Here $\X$ is a \emph{data domain} consisting of \emph{datapoints} $x \in \X$, $\Y$ is a \emph{label set} consisting of \emph{labels} $y \in \Y$, and the sequence $(x_1,y_1),\ldots,(x_n,y_n)$ is the \emph{training data} consisting of \emph{labeled examples (a.k.a. samples)}~$(x_i,y_i)$.  I~will refer to the chosen function $f$ as a \emph{predictor}, and to $n$ as the \emph{sample size}. A \emph{learning algorithm} takes as input training data, and outputs (some representation of) a predictor $f \in \Y^\X$.\footnote{Note that this describes the usual \emph{batch}, a.k.a.~\emph{offline}, setting of supervised learning. I do not discuss other paradigms such as online or active learning in this article.} 



Success in supervised learning is defined as \emph{generalization} to  future examples: For a typical \emph{test example}  $(x_{\tst},y_{\tst})$, the predicted label $y'_{\tst}=f(x_{\tst})$ should ``equal'' $y_{\tst}$, perhaps approximately. We usually assume the test example is drawn from the same  ``source'' as the training data  --- commonly, i.i.d.~from the same distribution. The quality of the prediction is quantified by $\ell(y'_{\tst},y_{\tst})$, where $\ell:~\Y~\times~\Y \to \RR_{\geq 0}$ is a \emph{loss function} chosen as part of the problem definition. Common loss functions include the 0-1 loss $\ell_{0-1}(y',y) = [y' \neq y]$ for \emph{classification} problems,\footnote{The notation $[P]$ denotes $1$ when predicate $P$ is true, and denotes $0$ when $P$ is false.} as well as the absolute loss $|y'-y|$ or squared loss $(y'-y)^2$ for \emph{regression problems} featuring $\Y  \sse \RR$.

Nontrivial generalization properties are typically only possible if one assumes something about the data.\footnote{The need for such an assumption is formalized by the  \emph{no free lunch theorems} of supervised learning \cite{wolpert_connection_1992,wolpert_lack_1996,schaffer_conservation_1994}.} The Bayesian approach to  machine learning, common in many applications, assumes some parametric form for the distribution generating the data, and postulates a prior on the parameters. This is not the approach I will take in this article. Instead, I will focus on the frequentist --- and some would say ``worst-case'' or ``adversarial'' ---  approach that is common in the computational learning theory community, embodied by the PAC model. Here we assume that the (training and test) data can be explained, perhaps approximately, by a function in some ``simple enough to learn'' class of functions $\H \sse \Y^\X$, often called the \emph{hypotheses}. Equivalently, we  seek a predictor which explains the unseen data roughly  as well as the best hypothesis $h^* \in \H$, whether or not we assume that $h^*$ itself provides a perfect explanation.



 \paragraph{Common Algorithmic Templates.} Perhaps the best known general-purpose supervised learning algorithm is \emph{empirical risk minimization (ERM)}, which chooses as its predictor a hypothesis $f \in \H$ minimizing $\frac{1}{n} \sum_{i=1}^n \ell(f(x_i),y_i)$ --- a quantity called the \emph{training error}, \emph{empirical error}, or \emph{empirical risk} of $f$. %\footnote{When multiple hypotheses minimize the empirical risk, we assume ERM breaks ties arbitrarily.}
A common template for generalizing ERM involves adding a \emph{regularization term} $\psi(f)$ to the  objective function, typically chosen to measure some notion of ``hypothesis complexity.'' An algorithm instantiating this template is known as a \emph{structural risk minimizer (SRM)}, and chooses as its predictor the hypothesis $f \in \H$ minimizing the \emph{structural risk} $\frac{1}{n} \sum_{i=1}^n \ell(f(x_i),y_i) + \psi(f)$. Other well-known algorithms, such as gradient descent and its variations,  can frequently be interpreted as approximate implementations of ERM or SRM.


\paragraph{Proper vs Improper Learning.} A learning algorithm is said to be \emph{proper} if its predictor $f$ is always chosen from the hypothesis class, i.e., $f \in \H$, otherwise it is said to be \emph{improper}. ERM  is an example of a proper learning algorithm, as are SRM algorithms of the form described above.  In the \emph{proper regime} of learning, algorithms are required to be proper. This article will be concerned with the more flexible \emph{improper regime} (a.k.a \emph{representation-independent learning}), where no such constraint is placed on the learner. In other words, all we care about is predictive power at test time, rather than any insights derived from the functional form or representation of the predictor~itself.


\subsection{The PAC Model}
A standard mathematical setup for evaluation of supervised learning algorithms, at least in the theoretical computer science community, is Valiant's \emph{Probably Approximately Correct (PAC) model} of learning (see e.g.~\cite{kearns_introduction_1994,mohri_foundations_2018}). Here, we assume there is an unknown distribution $\D$ on $\X \times \Y$ from which training and test data are  drawn.  Specifically, the labeled datapoints of the training set  $(x_1,y_1), \ldots, (x_n,y_n)$, as well as the test data  $(x_\tst,y_\tst)$, are i.i.d.~from $\D$. Often it is assumed that $\D$ lies in some class of distributions of interest. The \emph{true expected loss}, or simply \emph{loss}, of a predictor $f: \X \to \Y$ is the expected loss it incurs on draws from $\D$, written $L_\D(f) = \Ex_{(x,y) \sim \D} \ell(f(x),y)$.


There are two main ``settings'' in PAC learning. The  \emph{realizable setting} only requires that the data be perfectly explained by some hypothesis in $\H$. More generally, the \emph{agnostic setting} makes no assumption relating the data to the hypotheses, but shifts the goalposts as necessary to allow nontrivial guarantees: the expected loss at test time is evaluated only ``relative'' to that of the best hypothesis $h^* \in \H$. There are other settings which make more nuanced assumptions, such as $\D$ being of a particular parametric form or its support living in some (unknown) lower-dimensional space, etc. I will mostly discuss the realizable and agnostic settings in this article, those being the simplest and most studied from a theoretical perspective. %TODO:We will briefly discuss other settings in Section ??

The PAC model demands high probability guarantees of learners, in the worst case over distributions of interest. Consider first the realizable setting, where $\D$ is such that $\min_{h \in \H} L_{\D}(h) = 0$. A PAC learner has \emph{error} $\epsilon=\epsilon(n)$ and \emph{confidence} $\delta=\delta(n)$ if, when training data consists of $n$ i.i.d~samples from a realizable distribution $\D$, it produces a predictor $f$  satisfying $L_\D(f) \leq \epsilon$ with probability at least $1-\delta$. In the agnostic setting, where $\D$ can be arbitrary, we require $L_\D(f) - \min_{h \in \H} L_\D(h) \leq \epsilon$ with probability $1-\delta$.

In both the realizable and agnostic settings, we look for PAC learners with small $\epsilon$ and $\delta$ as a function of the sample size $n$. An equivalent perspective looks at the sample complexity $m(\epsilon,\delta)$, which is the minimum sample size which guarantees error  at most $\epsilon$ with probability at least $1-\delta$. We say a problem is \emph{PAC learnable} if its PAC sample complexity is finite whenever $\epsilon,\delta > 0$.

For most PAC learning problems, learnability and sample complexity are characterized in terms of a  ``dimension'' of the hypothesis class. Most prominently this is the \emph{VC dimension} for binary classification, the \emph{fat shattering dimension} for agnostic regression, and the \emph{DS dimension} for multiclass classification (see \cite{anthony_neural_1999,daniely_optimal_2014,brukhim_characterization_2022}). Treatment of these is beyond the scope of this article. The unfamiliar reader need not worry, however,  as dimensions will feature only tangentially in our~discussion.




%\paragraph{Learning settings: Realizable, Agnostic, etc.} In learning theory, evaluating a supervised learning algorithm requires specifying a data model and an objective. We will leave the details of the data model flexible for now, to allow for both the PAC model and the adversarial transductive model. Nonetheless we will describe two variations, which we call ``settings'', which cut across different models. The  \emph{realizable setting}  requires only that the data be perfectly explained by some hypothesis $h \in \H$ --- i.e., there exists a hypothesis which is guaranteed to suffer a loss of $0$ on training and test data. The performance of the learning algorithm is its expected loss at test time for some ``worst case'' realizable instance. More generally, the \emph{agnostic setting} makes no assumption relating the data to the hypotheses, but shifts the goalposts as necessary to allow nontrivial guarantees: the expected loss at test time is evaluated only ``relative'' to that of the best hypothesis $h^* \in \H$, again for some ``worst case'' instance. There are other settings which make more nuanced assumptions about the data, such as it is drawn from a distribution of a particular parametric form, or that it lives in some (unknown) lower-dimensional space, etc. We will mostly discuss the realizable and agnostic settings, those being the simplest and most studied from a theoretical perspective.




%%% Local Variables:
%%% mode: latex
%%% TeX-master: "learning_matching"
%%% End:


\section{Privacy for Correlated Databases: PML vs. DP}
\label{sec:main}
As discussed in Section~\ref{sec:intro}, the objective of our work is to understand the relation between PML and DP when $X$ is a database with correlated entries. It turns out that pure DP mechanisms can have poor privacy performance when assessed through the lens of PML. Before we formally state our result, we need to define the PML of a mechanism that releases an entry from the database without perturbation. 

Let $X = (D_1, \ldots, D_n)$ be a database. Given a distribution $P_X \in \cP_\cX$ over $X$ and $i \in [n]$, let $P_{D_i}$ denote the distribution of the $i$-th entry in a database, obtained by marginalizing $P_X$ over $P_{D_{-i}}$, described by~\eqref{eq:marginal}. We use
\begin{equation*}
    \varepsilon_\mathrm{max} (D_i) \coloneqq \log \frac{1}{\min\limits_{d \in \cD} \, P_{D_i}(d)},
\end{equation*}
 to denote the largest amount of information that can leak about $D_i$ through any mechanism. By \eqref{eq:pml_bounds}, $\varepsilon_\mathrm{max} (D_i)$ is equal to the PML of a mechanism that releases $D_i$ with no randomization.  

\begin{theorem}
\label{thm:main}
For each $\delta >0$ and $\varepsilon >0$ there exists a database $X = (D_1, \ldots, D_n)$, $i \in [n]$, a mechanism $P_{Y \mid X}$ satisfying $\varepsilon$-DP, and $y \in \bR$ such that
\begin{equation*}
    \ell(D_i \to y) > \varepsilon_\mathrm{max}(D_i) - \delta.
\end{equation*}
\end{theorem}

\begin{proof}
To prove the statement, we construct a binary database where the entries are strongly correlated: If one entry is zero (resp. one), then the other entries are also likely to be zero (resp. one). We then demonstrate that if we use the Laplace mechanism to answer the counting query on this database, then the resulting PML is significantly high.

Let $X = (D_1, \ldots, D_{n +1})$ be a database containing $n+1$ binary entries.\footnote{We consider a database of size $n+1$ instead of $n$ for notational convenience.} Suppose $P_{D_1}(0) = 1 - P_{D_1}(1) = \alpha$ with $0 < \alpha < 0.5$. Let $D_-=(D_2, \ldots, D_{n+1})$. Fix a constant $0 < \eta < 1$ and suppose the distribution of $D_-$ depends on $D_1$ as follows:  
\begin{gather*}
    P_{D_- \mid D_1=1}(d_-) = \begin{cases}
        \eta, &\quad \mathrm{if} \; d_- = 1^{n},\\
        \frac{1 - \eta}{2^{n} -1}, &\quad \mathrm{otherwise}.
    \end{cases}\\
    P_{D_- \mid D_1=0}(d_-) = \begin{cases}
        \eta, &\quad \mathrm{if} \; d_- = 0^{n},\\
        \frac{1 - \eta}{2^{n} -1}, &\quad \mathrm{otherwise}.
    \end{cases}
\end{gather*}
Suppose our goal is to release the empirical frequency of the ones in the database using the Laplace mechanism, i.e., $Y \mid X=x \sim \mathrm{Lap}(\frac{\norm{x}_1}{n+1}, b)$, where $\norm{x}_1$ denotes the $\ell_1$-norm of $x \in \{0,1\}^{n+1}$. Note that the empirical frequency has global sensitivity $\frac{1}{n+1}$, thus the Laplace mechanism with scale parameter $b = \frac{1}{\varepsilon (n+1)}$ satisfies $\varepsilon$-DP. However, here we show that the Laplace mechanism is insufficient for protecting $D_1$. To demonstrate this, we calculate the PML $\ell(D_1 \to y)$ with $y \leq 0$, which depends on the distributions $P_{Y \mid D_1=1}, P_{Y \mid D_1=0}$ and $P_{Y}$. 

First, we calculate $P_{Y \mid D_1=1}(y)$ assuming $y \leq 0$:
\begin{align*}
    &P_{Y \mid D_1=1}(y) = \sum_{d_- \in \{0,1\}^{n}} P_{Y \mid D_1=1, D_-=d_-}(y) \cdot P_{D_- \mid D_1=1}(d_-)\\
    &= \eta \; P_{Y \mid D_1=1, D_-= 1^n}(y) + \frac{1-\eta}{2^{n} -1} \sum_{d_- \in \{0,1\}^{n} \setminus 1^{n}} P_{Y \mid D_1=1, D_-=d_-}(y)\\
    &= \frac{\eta}{2b} \exp \left(- \frac{\abs{y - 1}}{b} \right) + \frac{1-\eta}{2b (2^{n} -1)} \sum_{d_- \in \{0,1\}^{n} \setminus 1^{n}} \exp \left(- \frac{\abs{y - \frac{1}{n+1} - \frac{\norm{d_-}_1}{n+1}}}{b} \right)\\
    &= \frac{\eta}{2b} \exp \left(- \frac{\abs{y - 1}}{b} \right) + \frac{1-\eta}{2b (2^{n} -1)} \cdot \sum_{i=0}^{n-1} \binom{n}{i} \exp \left(- \frac{\abs{y - \frac{1}{n+1} - \frac{i}{n+1}}}{b} \right)\\
    &= \frac{\eta}{2b} \exp \left(- \frac{\abs{y - 1}}{b} \right) + \\
    &\hspace{3em} \frac{1-\eta}{2b (2^{n} -1)} \bigg[ \sum_{i=0}^{n} \binom{n}{i} \exp \left(- \frac{\abs{y - \frac{1}{n+1} - \frac{i}{n+1}}}{b} \right) - \exp \left(- \frac{\abs{y -1}}{b} \right) \bigg]\\
    &= \frac{2^n \eta - 1}{2b (2^n - 1)} \exp \left(\frac{y - 1}{b} \right) + \\
    &\hspace{.3\textwidth} \frac{1-\eta}{2b (2^{n} -1)} \exp \left(\frac{y - \frac{1}{n+1}}{b}\right) \cdot \left(1 + \exp (- \frac{1}{b(n+1)}) \right)^n\\
    &= \frac{1}{2b (2^{n} -1)}\exp(\frac{y}{b}) \bigg[\Big(2^n \eta - 1\Big) \exp \left(-\frac{1}{b} \right) + \\
    &\hspace{.3\textwidth} (1-\eta) \, \exp \left(-\frac{1}{b(n+1)}\right) \left(1 + \exp (- \frac{1}{b(n+1)}) \right)^n \bigg].
\end{align*}

Similarly, we can calculate $P_{Y \mid D_1=0}(y)$ assuming $y \leq 0$:  
\begin{align*}
    &P_{Y \mid D_1=0}(y) = \sum_{d_- \in \{0,1\}^{n}} P_{Y \mid D_1=0, D_-=d_-}(y) \cdot P_{D_- \mid D_1=0}(d_-) \\
    &= \eta \; P_{Y \mid D_1=0, D_-= 0^n}(y) + \frac{1-\eta}{2^{n} -1} \; \sum_{d_- \in \{0,1\}^{n} \setminus 0^{n}} P_{Y \mid D_1=0, D_-=d_-}(y)\\
    &= \frac{\eta}{2b} \exp \left(\frac{y}{b} \right) + \frac{1-\eta}{2b (2^{n} -1)} \; \sum_{d_- \in \{0,1\}^{n} \setminus 0^{n}} \exp \left(\frac{y - \frac{\norm{d_-}_1}{n+1}}{b} \right)\\
    &= \frac{1}{2b} \exp \left(\frac{y}{b} \right) \left[\eta + \frac{1-\eta}{2^{n} -1}  \sum_{i=1}^{n} \binom{n}{i} \exp \left(- \frac{i}{b(n+1)} \right) \right] \\  
    &= \frac{1}{2b} \exp \left(\frac{y}{b} \right) \left[\eta - \frac{1-\eta}{2^{n} -1} + \frac{1-\eta}{2^{n} -1}  \sum_{i=0}^{n} \binom{n}{i} \exp \left(- \frac{i}{b(n+1)} \right) \right] \\  
    &= \frac{1}{2b (2^{n} -1)} \exp \left(\frac{y}{b} \right) \left[2^n  \eta - 1 + (1-\eta) \left(1 + \exp (- \frac{1}{b(n +1)}) \right)^n\right].  \\  
\end{align*}
Since $\exp(-x) \leq 1$ for $x \geq 0$, then $P_{Y \mid D_1=1}(y) \leq P_{Y \mid D_1=0}(y)$ when $y \leq 0$. Next, we calculate $P_Y(y)$ for $y \leq 0$: 
\begin{align*}
    &P_Y(y) = (1 - \alpha) P_{Y \mid D_1 = 1}(y) + \alpha P_{Y \mid D_1=0}(y) \\
    &= \frac{1}{2b (2^{n} -1)} \exp \left(\frac{y}{b} \right) \Bigg[\Big(2^n \eta - 1\Big) (1 - \alpha) \exp \left(-\frac{1}{b} \right) + \\
    &\hspace{7em} (1-\eta) (1 - \alpha)\, \exp \left(-\frac{1}{b(n+1)}\right) \left(1 + \exp (- \frac{1}{b(n+1)}) \right)^n + \\
    &\hspace{7em} \big(2^n \eta - 1\big) \alpha +  (1-\eta) \alpha \left(1 + \exp (- \frac{1}{b(n +1)}) \right)^n \Bigg]\\
    &\leq \frac{1}{2b (2^{n} -1)} \exp \left(\frac{y}{b} \right) \Bigg[\Big(2^n \eta - 1\Big) \Big( (1 - \alpha)\, \exp \left(-\frac{1}{b}\right) + \alpha \Big) + \\
    &\hspace{1em} (1-\eta) (1 - \alpha)\, \left(1 + \exp (- \frac{1}{b(n+1)}) \right)^n + (1-\eta) \alpha \left(1 + \exp (- \frac{1}{b(n +1)}) \right)^n \Bigg]\\
    &= \frac{1}{2b (2^{n} -1)} \exp \left(\frac{y}{b} \right) \Bigg[\Big(2^n \eta - 1\Big) \Big( (1 - \alpha)\, \exp \left(-\frac{1}{b}\right) + \alpha \Big) +\\
    &\hspace{18em}(1-\eta) \left(1 + \exp (- \frac{1}{b(n+1)}) \right)^n  \Bigg]\\
    &\leq \frac{1}{2b (2^{n} -1)} \exp \left(\frac{y}{b} \right) \Bigg[2^n  \eta \Big( (1 - \alpha)\, \exp \left(-\frac{1}{b}\right) + \alpha \Big) +\\
    &\hspace{18em} (1-\eta) \left(1 + \exp (- \frac{1}{b(n+1)}) \right)^n  \Bigg]. 
\end{align*}
Using $b(n+1) = \frac{1}{\varepsilon}$ to achieve $\varepsilon$-DP, we obtain the following lower bound on $\ell(D_1 \to y)$ with $y \leq 0$:
\begin{align*}
    \ell(D_1 \to y) &= \log \frac{\max\limits_{d_1 \in \{0,1\}} P_{Y \mid D_1 = d_1}(y)}{P_Y(y)} = \log \frac{P_{Y \mid D_1 = 0}(y)}{P_Y(y)}\\[0.5em]
    &\geq \log \frac{2^n \eta + \left(1 + e^{-\varepsilon} \right)^n (1-\eta) - 1}{2^n  \eta \Big( (1 - \alpha)\, \exp (-\varepsilon  n-\varepsilon) + \alpha \Big) + \left(1 + e^{-\varepsilon} \right)^n (1-\eta)}\\[0.5em]
    &= \log \frac{2^n  \eta +  \left(1 + e^{-\varepsilon} \right)^n \, (1-\eta) - 1}{2^n  \eta \, \alpha  + \left(\frac{2}{e^\varepsilon}\right)^n \eta (e^{-\varepsilon}) (1 - \alpha) +  \left(1 + e^{-\varepsilon} \right)^n \, (1-\eta) }.
 \end{align*}    

Note that $1 + e^{-\varepsilon} < 2$ and $\frac{2}{e^\varepsilon} < 2$ for all $\varepsilon > 0$. Therefore, when $n$ is large the dominating term in the numerator is $2^n \eta$ and the dominating term in the denominator is $2^n \eta \, \alpha$. Hence, as $n \to \infty$, the lower bound on $\ell(D_1 \to y)$ approaches $\varepsilon_\mathrm{max} (D_1) = \log \frac{1}{\alpha}$. This proves that for each $\delta >0$, there exists an integer $n$, a database $X$ of size $n$, and a mechanism satisfying $\varepsilon$-DP such that 
\begin{equation*}
    \ell(D_1 \to y) > \varepsilon_\mathrm{max}(D_1) - \delta.
\end{equation*}
\qed
\end{proof}



The proof of Theorem~\ref{thm:main} relies on a database exhibiting what may be considered as pathologically strong correlations: If the first entry is zero (resp. one) then all other entries are likely to be zero (resp. one) with a constant probability that does not diminish with growing database size $n$. However, it is important to note that the theorem holds true even in more realistic scenarios characterized by weaker correlations. Specifically, the asymptotic lower bound of $\varepsilon_\mathrm{max}(D_1)$ for PML remains applicable even if $\eta$ diminishes at a polynomial rate, i.e., if $\eta = \Theta(\frac{1}{n^r})$ for some constant $r \geq 1$.



\section{Conclusions}
\label{sec:discussion}
This work identifies signal collapse as a critical bottleneck in one-shot neural network pruning. Performance loss in pruned networks is due to \textbf{signal collapse} in addition to the removal of critical parameters. We propose \textbf{REFLOW} (\textbf{Re}storing \textbf{F}low of \textbf{Low}-variance signals), a simple yet effective method that mitigates signal collapse without computationally expensive weight updates. By focusing on signal preservation, REFLOW highlights the importance of mitigating signal collapse in sparse networks and enables magnitude pruning to match or surpass state-of-the-art one-shot pruning methods such as CHITA, CBS, and WF.

REFLOW consistently achieves state-of-the-art accuracy across diverse architectures, restoring ResNeXt-101 from under 4.1\% to 78.9\% top-1 accuracy at 80\% sparsity on ImageNet. Its lightweight design makes it a practical solution for both research and deployment, delivering high-quality sparse models without the overhead of traditional approaches. These findings challenge the traditional emphasis on weight selection strategies and underscore the critical role of signal propagation for achieving high-quality sparse networks in the context of one-shot pruning.





\begin{credits}
\subsubsection{\ackname} This work has been supported by the Swedish Research Council (VR) under the grant 2023-04787 and Digital Futures center within the collaborative project DataLEASH. 


\subsubsection{\discintname}
The authors have no competing interests to declare that are
relevant to the content of this article.
\end{credits}
%
% ---- Bibliography ----
%
% BibTeX users should specify bibliography style 'splncs04'.
% References will then be sorted and formatted in the correct style.
%
\bibliographystyle{splncs04}
\bibliography{main}
%

\end{document}
