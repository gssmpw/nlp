As discussed in Section~\ref{sec:intro}, the objective of our work is to understand the relation between PML and DP when $X$ is a database with correlated entries. It turns out that pure DP mechanisms can have poor privacy performance when assessed through the lens of PML. Before we formally state our result, we need to define the PML of a mechanism that releases an entry from the database without perturbation. 

Let $X = (D_1, \ldots, D_n)$ be a database. Given a distribution $P_X \in \cP_\cX$ over $X$ and $i \in [n]$, let $P_{D_i}$ denote the distribution of the $i$-th entry in a database, obtained by marginalizing $P_X$ over $P_{D_{-i}}$, described by~\eqref{eq:marginal}. We use
\begin{equation*}
    \varepsilon_\mathrm{max} (D_i) \coloneqq \log \frac{1}{\min\limits_{d \in \cD} \, P_{D_i}(d)},
\end{equation*}
 to denote the largest amount of information that can leak about $D_i$ through any mechanism. By \eqref{eq:pml_bounds}, $\varepsilon_\mathrm{max} (D_i)$ is equal to the PML of a mechanism that releases $D_i$ with no randomization.  

\begin{theorem}
\label{thm:main}
For each $\delta >0$ and $\varepsilon >0$ there exists a database $X = (D_1, \ldots, D_n)$, $i \in [n]$, a mechanism $P_{Y \mid X}$ satisfying $\varepsilon$-DP, and $y \in \bR$ such that
\begin{equation*}
    \ell(D_i \to y) > \varepsilon_\mathrm{max}(D_i) - \delta.
\end{equation*}
\end{theorem}

\begin{proof}
To prove the statement, we construct a binary database where the entries are strongly correlated: If one entry is zero (resp. one), then the other entries are also likely to be zero (resp. one). We then demonstrate that if we use the Laplace mechanism to answer the counting query on this database, then the resulting PML is significantly high.

Let $X = (D_1, \ldots, D_{n +1})$ be a database containing $n+1$ binary entries.\footnote{We consider a database of size $n+1$ instead of $n$ for notational convenience.} Suppose $P_{D_1}(0) = 1 - P_{D_1}(1) = \alpha$ with $0 < \alpha < 0.5$. Let $D_-=(D_2, \ldots, D_{n+1})$. Fix a constant $0 < \eta < 1$ and suppose the distribution of $D_-$ depends on $D_1$ as follows:  
\begin{gather*}
    P_{D_- \mid D_1=1}(d_-) = \begin{cases}
        \eta, &\quad \mathrm{if} \; d_- = 1^{n},\\
        \frac{1 - \eta}{2^{n} -1}, &\quad \mathrm{otherwise}.
    \end{cases}\\
    P_{D_- \mid D_1=0}(d_-) = \begin{cases}
        \eta, &\quad \mathrm{if} \; d_- = 0^{n},\\
        \frac{1 - \eta}{2^{n} -1}, &\quad \mathrm{otherwise}.
    \end{cases}
\end{gather*}
Suppose our goal is to release the empirical frequency of the ones in the database using the Laplace mechanism, i.e., $Y \mid X=x \sim \mathrm{Lap}(\frac{\norm{x}_1}{n+1}, b)$, where $\norm{x}_1$ denotes the $\ell_1$-norm of $x \in \{0,1\}^{n+1}$. Note that the empirical frequency has global sensitivity $\frac{1}{n+1}$, thus the Laplace mechanism with scale parameter $b = \frac{1}{\varepsilon (n+1)}$ satisfies $\varepsilon$-DP. However, here we show that the Laplace mechanism is insufficient for protecting $D_1$. To demonstrate this, we calculate the PML $\ell(D_1 \to y)$ with $y \leq 0$, which depends on the distributions $P_{Y \mid D_1=1}, P_{Y \mid D_1=0}$ and $P_{Y}$. 

First, we calculate $P_{Y \mid D_1=1}(y)$ assuming $y \leq 0$:
\begin{align*}
    &P_{Y \mid D_1=1}(y) = \sum_{d_- \in \{0,1\}^{n}} P_{Y \mid D_1=1, D_-=d_-}(y) \cdot P_{D_- \mid D_1=1}(d_-)\\
    &= \eta \; P_{Y \mid D_1=1, D_-= 1^n}(y) + \frac{1-\eta}{2^{n} -1} \sum_{d_- \in \{0,1\}^{n} \setminus 1^{n}} P_{Y \mid D_1=1, D_-=d_-}(y)\\
    &= \frac{\eta}{2b} \exp \left(- \frac{\abs{y - 1}}{b} \right) + \frac{1-\eta}{2b (2^{n} -1)} \sum_{d_- \in \{0,1\}^{n} \setminus 1^{n}} \exp \left(- \frac{\abs{y - \frac{1}{n+1} - \frac{\norm{d_-}_1}{n+1}}}{b} \right)\\
    &= \frac{\eta}{2b} \exp \left(- \frac{\abs{y - 1}}{b} \right) + \frac{1-\eta}{2b (2^{n} -1)} \cdot \sum_{i=0}^{n-1} \binom{n}{i} \exp \left(- \frac{\abs{y - \frac{1}{n+1} - \frac{i}{n+1}}}{b} \right)\\
    &= \frac{\eta}{2b} \exp \left(- \frac{\abs{y - 1}}{b} \right) + \\
    &\hspace{3em} \frac{1-\eta}{2b (2^{n} -1)} \bigg[ \sum_{i=0}^{n} \binom{n}{i} \exp \left(- \frac{\abs{y - \frac{1}{n+1} - \frac{i}{n+1}}}{b} \right) - \exp \left(- \frac{\abs{y -1}}{b} \right) \bigg]\\
    &= \frac{2^n \eta - 1}{2b (2^n - 1)} \exp \left(\frac{y - 1}{b} \right) + \\
    &\hspace{.3\textwidth} \frac{1-\eta}{2b (2^{n} -1)} \exp \left(\frac{y - \frac{1}{n+1}}{b}\right) \cdot \left(1 + \exp (- \frac{1}{b(n+1)}) \right)^n\\
    &= \frac{1}{2b (2^{n} -1)}\exp(\frac{y}{b}) \bigg[\Big(2^n \eta - 1\Big) \exp \left(-\frac{1}{b} \right) + \\
    &\hspace{.3\textwidth} (1-\eta) \, \exp \left(-\frac{1}{b(n+1)}\right) \left(1 + \exp (- \frac{1}{b(n+1)}) \right)^n \bigg].
\end{align*}

Similarly, we can calculate $P_{Y \mid D_1=0}(y)$ assuming $y \leq 0$:  
\begin{align*}
    &P_{Y \mid D_1=0}(y) = \sum_{d_- \in \{0,1\}^{n}} P_{Y \mid D_1=0, D_-=d_-}(y) \cdot P_{D_- \mid D_1=0}(d_-) \\
    &= \eta \; P_{Y \mid D_1=0, D_-= 0^n}(y) + \frac{1-\eta}{2^{n} -1} \; \sum_{d_- \in \{0,1\}^{n} \setminus 0^{n}} P_{Y \mid D_1=0, D_-=d_-}(y)\\
    &= \frac{\eta}{2b} \exp \left(\frac{y}{b} \right) + \frac{1-\eta}{2b (2^{n} -1)} \; \sum_{d_- \in \{0,1\}^{n} \setminus 0^{n}} \exp \left(\frac{y - \frac{\norm{d_-}_1}{n+1}}{b} \right)\\
    &= \frac{1}{2b} \exp \left(\frac{y}{b} \right) \left[\eta + \frac{1-\eta}{2^{n} -1}  \sum_{i=1}^{n} \binom{n}{i} \exp \left(- \frac{i}{b(n+1)} \right) \right] \\  
    &= \frac{1}{2b} \exp \left(\frac{y}{b} \right) \left[\eta - \frac{1-\eta}{2^{n} -1} + \frac{1-\eta}{2^{n} -1}  \sum_{i=0}^{n} \binom{n}{i} \exp \left(- \frac{i}{b(n+1)} \right) \right] \\  
    &= \frac{1}{2b (2^{n} -1)} \exp \left(\frac{y}{b} \right) \left[2^n  \eta - 1 + (1-\eta) \left(1 + \exp (- \frac{1}{b(n +1)}) \right)^n\right].  \\  
\end{align*}
Since $\exp(-x) \leq 1$ for $x \geq 0$, then $P_{Y \mid D_1=1}(y) \leq P_{Y \mid D_1=0}(y)$ when $y \leq 0$. Next, we calculate $P_Y(y)$ for $y \leq 0$: 
\begin{align*}
    &P_Y(y) = (1 - \alpha) P_{Y \mid D_1 = 1}(y) + \alpha P_{Y \mid D_1=0}(y) \\
    &= \frac{1}{2b (2^{n} -1)} \exp \left(\frac{y}{b} \right) \Bigg[\Big(2^n \eta - 1\Big) (1 - \alpha) \exp \left(-\frac{1}{b} \right) + \\
    &\hspace{7em} (1-\eta) (1 - \alpha)\, \exp \left(-\frac{1}{b(n+1)}\right) \left(1 + \exp (- \frac{1}{b(n+1)}) \right)^n + \\
    &\hspace{7em} \big(2^n \eta - 1\big) \alpha +  (1-\eta) \alpha \left(1 + \exp (- \frac{1}{b(n +1)}) \right)^n \Bigg]\\
    &\leq \frac{1}{2b (2^{n} -1)} \exp \left(\frac{y}{b} \right) \Bigg[\Big(2^n \eta - 1\Big) \Big( (1 - \alpha)\, \exp \left(-\frac{1}{b}\right) + \alpha \Big) + \\
    &\hspace{1em} (1-\eta) (1 - \alpha)\, \left(1 + \exp (- \frac{1}{b(n+1)}) \right)^n + (1-\eta) \alpha \left(1 + \exp (- \frac{1}{b(n +1)}) \right)^n \Bigg]\\
    &= \frac{1}{2b (2^{n} -1)} \exp \left(\frac{y}{b} \right) \Bigg[\Big(2^n \eta - 1\Big) \Big( (1 - \alpha)\, \exp \left(-\frac{1}{b}\right) + \alpha \Big) +\\
    &\hspace{18em}(1-\eta) \left(1 + \exp (- \frac{1}{b(n+1)}) \right)^n  \Bigg]\\
    &\leq \frac{1}{2b (2^{n} -1)} \exp \left(\frac{y}{b} \right) \Bigg[2^n  \eta \Big( (1 - \alpha)\, \exp \left(-\frac{1}{b}\right) + \alpha \Big) +\\
    &\hspace{18em} (1-\eta) \left(1 + \exp (- \frac{1}{b(n+1)}) \right)^n  \Bigg]. 
\end{align*}
Using $b(n+1) = \frac{1}{\varepsilon}$ to achieve $\varepsilon$-DP, we obtain the following lower bound on $\ell(D_1 \to y)$ with $y \leq 0$:
\begin{align*}
    \ell(D_1 \to y) &= \log \frac{\max\limits_{d_1 \in \{0,1\}} P_{Y \mid D_1 = d_1}(y)}{P_Y(y)} = \log \frac{P_{Y \mid D_1 = 0}(y)}{P_Y(y)}\\[0.5em]
    &\geq \log \frac{2^n \eta + \left(1 + e^{-\varepsilon} \right)^n (1-\eta) - 1}{2^n  \eta \Big( (1 - \alpha)\, \exp (-\varepsilon  n-\varepsilon) + \alpha \Big) + \left(1 + e^{-\varepsilon} \right)^n (1-\eta)}\\[0.5em]
    &= \log \frac{2^n  \eta +  \left(1 + e^{-\varepsilon} \right)^n \, (1-\eta) - 1}{2^n  \eta \, \alpha  + \left(\frac{2}{e^\varepsilon}\right)^n \eta (e^{-\varepsilon}) (1 - \alpha) +  \left(1 + e^{-\varepsilon} \right)^n \, (1-\eta) }.
 \end{align*}    

Note that $1 + e^{-\varepsilon} < 2$ and $\frac{2}{e^\varepsilon} < 2$ for all $\varepsilon > 0$. Therefore, when $n$ is large the dominating term in the numerator is $2^n \eta$ and the dominating term in the denominator is $2^n \eta \, \alpha$. Hence, as $n \to \infty$, the lower bound on $\ell(D_1 \to y)$ approaches $\varepsilon_\mathrm{max} (D_1) = \log \frac{1}{\alpha}$. This proves that for each $\delta >0$, there exists an integer $n$, a database $X$ of size $n$, and a mechanism satisfying $\varepsilon$-DP such that 
\begin{equation*}
    \ell(D_1 \to y) > \varepsilon_\mathrm{max}(D_1) - \delta.
\end{equation*}
\qed
\end{proof}



The proof of Theorem~\ref{thm:main} relies on a database exhibiting what may be considered as pathologically strong correlations: If the first entry is zero (resp. one) then all other entries are likely to be zero (resp. one) with a constant probability that does not diminish with growing database size $n$. However, it is important to note that the theorem holds true even in more realistic scenarios characterized by weaker correlations. Specifically, the asymptotic lower bound of $\varepsilon_\mathrm{max}(D_1)$ for PML remains applicable even if $\eta$ diminishes at a polynomial rate, i.e., if $\eta = \Theta(\frac{1}{n^r})$ for some constant $r \geq 1$.

