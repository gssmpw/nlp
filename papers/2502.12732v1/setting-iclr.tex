% =============================================================
%         Setting for ICLR format
%
%   Author      : 
%   Last Update : 
% =============================================================


\usepackage{amsfonts}
\usepackage{amsmath}
\usepackage{amssymb}
% \usepackage{bbm} % forbidden by aaai
\usepackage{xcolor}
\usepackage{cleveref}
\usepackage[subrefformat=parens,labelformat=parens]{subfig}
\usepackage{threeparttable,booktabs}
\usepackage{makecell}
\usepackage{multirow}
\usepackage[]{algpseudocode}                               % algorithm package
%\usepackage[noend]{algpseudocode}
\algrenewcommand\textproc{\texttt}
\makeatletter\let\float@addtolists\relax\makeatother
\usepackage{algorithm}
\renewcommand{\algorithmicrequire}{\textbf{Input:} }       % Use Input in the format of Algorithm
\renewcommand{\algorithmicensure} {\textbf{Output:}}       % Use Output in the format of Algorithm
\usepackage{pgfplots}
\usepackage{pgfplotstable}
\pgfplotsset{compat=newest}
\usepackage{courier}
\usepackage[inline]{enumitem}
\usepackage{tabularx}
\usepackage{soul}
\usepackage{bm}
\usepackage{pifont}                                        % add circle
\usepackage{bbding}                                        % Checkmark
\usepackage[super]{nth}

\newcommand{\minisection}[1]{\noindent{\textbf{#1}}}
\newcommand{\minienumerate}[1]{\noindent{\textbf{#1}}}
\newcommand{\tabincell}[2]{
    \begin{tabular}{@{}#1@{}}#2\end{tabular}
}


% === Local new commands
% ==== Local new commands
\newcommand{\calH}{\mathcal{H}}
\newcommand{\calN}{\mathcal{N}}
\newcommand{\calO}{\mathcal{O}}
\newcommand{\calP}{\mathcal{P}}
\newcommand{\calV}{\mathcal{V}}
\newcommand{\etal}{\textit{et al.}}
\newcommand{\norm}[1]{\left\lVert#1\right\rVert}
\newcommand{\tool}[1]{$\mathsf{#1}$}
\newcommand{\m}[1]{\boldsymbol{#1}}
\newcommand{\abs}[1]{\ensuremath{\left\lvert #1\right\rvert}}
\newcommand{\tensor}[1]{\mathcal{#1}}   
% re-define vec command
\newcommand{\mathbbm}[1]{\text{\usefont{U}{bbm}{m}{n}#1}}
\renewcommand{\vec}[1]{\mathbf{#1}}

% === Theorem Definitions
\newtheorem{myproblem}{\textbf{Problem}}
\newtheorem{myassumption}{\textbf{Assumption}}
\newtheorem{mydefinition}{\textbf{Definition}}
\newtheorem{mytheorem}{\textbf{Theorem}}
\newtheorem{mycorollary}{\textbf{Corollary}}
\newtheorem{mylemma}{\textbf{Lemma}}
\newtheorem{myclaim}{\textbf{Claim}}
\newtheorem{myapplication}{\textbf{Application}}
\newtheorem{myexample}{\textbf{Example}}
\newtheorem{myconjecture}{Conjecture}
\newtheorem{myprop}{Property}
\newtheorem{remark}{Remark}

% === shrink page num
\setlength{\textfloatsep}{6pt plus 1pt minus 1pt}          % set space between float and text
% \setlength{\floatsep}{5pt plus 1pt minus 1pt}              % set space between two floats
% \setlength{\intextsep}{6pt plus 1pt minus 1pt}             % set space between text and float
\setlength{\belowdisplayskip}{2pt} \setlength{\belowdisplayshortskip}{2pt}
\setlength{\abovedisplayskip}{2pt} \setlength{\abovedisplayshortskip}{2pt}

\usepackage{graphicx}


