\begin{thebibliography}{00}

\bibitem{1} Bhide, Nirmal, and Christopher M. Bishop. "Pathophysiology of Non-Dopaminergic Monoamine Systems in Parkinson's Disease: Implications for Mood Dysfunction." InTech EBooks, 2011,  https://doi.org/10.5772/21140.


\bibitem{2} What Is Parkinson’s? — Parkinson’s Foundation. Parkinson’s Foundation. https://www.parkinson.org/understanding-parkinsons/what-is-parkinsons . Accessed 20 Aug 2022

\bibitem{3} Hamzehei, Sahand. "Gateways and Wearable Tools for Monitoring Patient Movements in a Hospital Environment-Webthesis." (2022).

\bibitem{4} Karami, Mostafa. "Machine Learning Algorithms for Radiogenomics: Application to Prediction of the MGMT promoter methylation status in mpMRI scans." PhD diss., Politecnico di Torino, 2022.

\bibitem{5} Andreas Maier, Christopher Syben, Tobias Lasser, Christian Riess, A gentle introduction to deep learning in medical image processing, Zeitschrift für Medizinische Physik, Volume 29, Issue 2,2019,Pages 86-101,ISSN 0939-3889

\bibitem{6} Tsanas A, Little MA, McSharry PE, Ramig LO (2010) Accurate telemonitoring of Parkinson’s disease progression by noninvasive speech tests. IEEE Trans Biomed Eng 57(4):884–93. https://doi.org/10.1109/TBME.2009.2036000. Epub 2009 Nov 20 PMID: 19932995

\bibitem{7} Tin Kam Ho, "Random decision forests," Proceedings of 3rd International Conference on Document Analysis and Recognition, Montreal, QC, Canada, 1995, pp. 278-282 vol.1, doi: 10.1109/ICDAR.1995.598994.

\bibitem{8} Guyon I.; Weston J.; Barnhill S.; Vapnik V. (2002). "Gene selection for cancer classification using support vector machines". Machine Learning. 46 (1–3): 389–422. doi:10.1023/A:1012487302797

\bibitem{9} Wolaver, Dan H. (1991). Phase-Locked Loop Circuit Design. Prentice Hall. p. 211. ISBN 978-0-13-662743-2.

\bibitem{10} Hamzehei, S., Akbarzadeh, O., Attar, H. et al. Predicting the total Unified Parkinson’s Disease Rating Scale (UPDRS) based on ML techniques and cloud-based update. J Cloud Comp 12, 12 (2023). https://doi.org/10.1186/s13677-022-00388-1

\bibitem{11} "Mean Squared Error (MSE)" www.probabilitycourse.com. Retrieved 2020-09-12.

\bibitem{12} Steel, R. G. D.; Torrie, J. H. (1960). Principles and Procedures of Statistics with Special Reference to the Biological Sciences. McGraw Hill.

%%%%%%%%%%% LR
\bibitem{13} A. U. Haq et al., "Feature Selection Based on L1-Norm Support Vector Machine and Effective Recognition System for Parkinson’s Disease Using Voice Recordings," in IEEE Access, vol. 7, pp. 37718-37734, 2019, doi: 10.1109/ACCESS.2019.2906350.

\bibitem{14} A. A. Spadoto, R. C. Guido, F. L. Carnevali, A. F. Pagnin, A. X. Falcão and J. P. Papa, "Improving Parkinson's disease identification through evolutionary-based feature selection," 2011 Annual International Conference of the IEEE Engineering in Medicine and Biology Society, Boston, MA, USA, 2011, pp. 7857-7860, doi: 10.1109/IEMBS.2011.6091936.

\bibitem{15} S. Aich, K. Younga, K. L. Hui, A. A. Al-Absi and M. Sain, "A nonlinear decision tree based classification approach to predict the Parkinson's disease using different feature sets of voice data," 2018 20th International Conference on Advanced Communication Technology (ICACT), Chuncheon, Korea (South), 2018, pp. 638-642, doi: 10.23919/ICACT.2018.8323864.

\bibitem{16} Amato F, Borzì L, Olmo G, Orozco-Arroyave JR. An algorithm for Parkinson's disease speech classification based on isolated words analysis. Health Inf Sci Syst. 2021 Jul 30;9(1):32. doi: 10.1007/s13755-021-00162-8. PMID: 34422258; PMCID: PMC8324609.





\end{thebibliography}

