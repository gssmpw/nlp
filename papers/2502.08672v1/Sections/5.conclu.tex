\section{Conclusion}

The findings of this research highlight the considerable promise of employing the LSTM model with Attention Mechanism in the field of medical diagnostics, particularly for the timely identification of Parkinson's Disease. The model has higher prediction accuracy than typical machine learning approaches, as seen by the lowered MSE values observed throughout the training, validation, and test stages. This progress is more than just a statistical achievement; it carries significant implications for the timely implementation of therapeutic measures and improved patient outcomes.

Incorporating attention processes into the LSTM framework facilitates the model's ability to discern and assign varying degrees of importance to different data pieces. Within the framework of Parkinson's Disease, this implies that the model exhibits proficiency in discerning particular patterns or irregularities that might potentially serve as indicators for the first stages of the disease. The achievement of this degree of detail and concentration may provide a challenge for conventional models, highlighting the effectiveness of attention processes in improving forecast precision.

Nevertheless, it is essential to approach these findings with a comprehensive understanding of the intricacies inherent in real-world clinical circumstances. While extensive, the dataset utilized in this research may not fully capture the extensive diversity observed within the worldwide patient cohort. Therefore, it is imperative to conduct additional evaluations of our model's generalizability by utilizing varied datasets that encompass a more comprehensive range of patient demographics and clinical manifestations.

In conclusion, our study provides a promising prospect for the timely and precise identification of Parkinson's Disease, utilizing deep learning and attention processes. In the future, adopting a comprehensive strategy that integrates technical improvements with clinical findings will be crucial. With more study, validation, and cooperation, we aspire that models like ours may be effectively incorporated into healthcare systems, serving as indispensable resources for physicians globally.

\section{acknowledgment}
Mostafa Karami is currently a graduate assistant at the University of Connecticut's Computer Science and Engineering Department.

% The dataset used in this study is available on GitHub, along with the source code:
% \url{https://github.com/Dechosenone/CNS_prediction.git}
