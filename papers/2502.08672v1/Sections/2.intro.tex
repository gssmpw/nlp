\section{Introduction}
Parkinson's disease (PD) is a neurodegenerative disorder affecting millions worldwide\cite{1}. As the disease advances, individuals with this condition encounter motor manifestations, including tremors, stiffness, and bradykinesia, alongside several non-motor symptoms. The analysis, comprehension, and anticipation of the manifestation of PD have significant importance due to the absence of a cure\cite{2} and limited treatment choices, which mostly consist of pharmaceutical interventions, lifestyle adjustments, and surgical procedures. The Unified Parkinson's Disease Rating Scale (UPDRS) is a widely recognized instrument used to assess the severity of symptoms associated with Parkinson's disease. The precise estimation of UPDRS scores provides a valuable understanding of the advancement of the disease. It also assists in customizing personalized therapy approaches for individuals with PD. 

In several areas of health care, such as smart hospital \cite{3}, cancer detection\cite{4}, and medical image processing\cite{5}, the importance of machine learning (ML) and deep learning (DL) is evident. ML and DL have demonstrated considerable promise in the realm of predicting, categorizing, and assessing diverse medical problems and disorders. Specifically, the utilization of these methodologies in the context of neurological illnesses such as PD has the potential to provide significant advancements in the areas of early detection, prognosis assessment, and tailored treatment strategies. The utilization of extensive data, the identification of complex patterns, and the development of predictive models have the potential to assist healthcare professionals in making better informed and evidence-based choices on patient care. For instance, the ML models can result in enhanced precision and prompt detection of alterations in UPDRS scores, enabling a more proactive approach to intervention.





%%%%%%%%%%%%% Subsection
\subsection{Research problem and objectives}
The accurate estimation of UPDRS scores presents a complex and diverse task. The intricate structure of the disease, the temporal course it exhibits, and the inherent heterogeneity in symptoms provide significant challenges for standard statistical methodologies. The existing methodologies, albeit somewhat successful, frequently fail to fully capture the many intricacies and interconnections inherent in the dataset. The limits of this approach become apparent when considering both the accuracy of the results and the ability to apply them to a wide range of datasets.

Long Short-Term Memory (LSTM) networks, which are a specific type of recurrent neural networks (RNN), have demonstrated potential in the modeling of time-series and sequential data. In the setting of Parkinson's disease, where temporal variations in symptoms play a crucial role as indications, LSTM models can be particularly relevant. Nevertheless, a rudimentary LSTM model may fail to comprehend the intricate connections within the dataset, hence requiring the incorporation of sophisticated components such as attention processes.

This study introduces a complete methodology for predicting Unified UPDRS scores. The proposed strategy leverages advanced LSTM models, incorporating attention processes, data augmentation, and robust feature selection strategies. Our proposed methodology demonstrates enhanced predictive accuracy and introduces a novel approach to the field of medical data analysis for PD.



%%%%%%%%%%%%% Subsection
\subsection{Related Works}
The prediction of UPDRS scores in Parkinson's Disease using learning methods has been an area of increasing interest over the last decade. Several methodologies have been proposed, demonstrating varying degrees of success. For instance, a prominent research\cite{13} investigation was conducted to examine the presence of speech articulation issues as an initial indication of Parkinson's Disease. The study utilized three data mining techniques to analyze voice measures obtained from a sample of 31 participants, 23 of whom were diagnosed with Parkinson's Disease. The Sequential Minimization Optimization (SVM) algorithm demonstrated notable accuracy (0.76) and a high level of sensitivity (0.97). In contrast, Logistic Regression exhibited a modest accuracy of 0.64, with a balanced sensitivity and specificity of 0.64 and 0.62, respectively. The equilibrium mentioned above implies a diminished necessity for additional diagnostic assessments compared to SVM's high sensitivity but low specificity (0.13). The findings underscore the need to employ a comprehensive diagnostic strategy to diagnose Parkinson's disease.

A further research\cite{14} examined several feature selection methodologies to enhance the accuracy of Parkinson's disease detection. The study compared Optimum-Path Forest (OPF) and three evolutionary-based approaches, namely PSO-OPF, HS-OPF, and GSA-OPF, specifically focusing on 22 voice-related parameters. Notably, all evolutionary strategies showed superior performance compared to the conventional OPF method. It is worth mentioning that HS-OPF demonstrated superior efficiency by picking 10 out of the total 22 characteristics and attaining an accuracy rate of around 92.78\%. However, it is important to note that this performance fell short of the 100\% standard established by a separate study.


Previous studies have separately examined gait and speech analysis. However, this study\cite{15} distinguishes itself by comparing performance metrics of voice data using different feature sets, with a specific focus on non-linear classification. By employing Principal Component Analysis (PCA) as a method for feature selection, a notable accuracy rate of 96.83\% was attained through the utilization of the random forest classifier. These insights play a crucial role for doctors as they guide in prioritizing certain symptoms during the first diagnosis of Parkinson's disease. In the future, researchers will investigate different feature reduction and classification techniques to enhance the accuracy and precision of performance measurements.

By employing a multi-level analysis of diverse speech segments, this study\cite{16} optimized a k-Nearest Neighbors model and attained a cross-validation accuracy of 94.3\% in tests. The results indicate the possibility of utilizing voice-recorded evaluations for Parkinson's disease, maybe through smartphone applications.