\section{Related Works}
\subsection{Generalizable GS}
3D Gaussian Splatting (3DGS) \cite{kerbl20233d} has emerged as a powerful method for reconstructing and representing 3D scenes using millions of 3D Gaussians. There has been a significant amount of work that has achieved good results in the reconstruction of small objects \cite{szymanowicz2024splatter} \cite{boss2024sf3d} and scenes \cite{chen2025mvsplat} \cite{charatan2024pixelsplat}. Additionally, some studies have utilized diffusion models to realize text-to-3D generation \cite{li2024instant3d}, which also achieves promising outcomes.

\subsection{3D Steganography}
Steganography has been evolving over the decades. Some research has made 3D steganography achieve good results in explicit geometry like meshes and point clouds \cite{ohbuchi2002frequency} \cite{zhu2024rethinking} \cite{ferreira2020robust} or implicit geometer like nerf \cite{li2023steganerf}\cite{luo2023copyrnerf}. 3DGS is a new technology to represent explicit geometry. However, few works have been researched in steganography for 3DGS. GS-Hider \cite{zhang2024gs}design a coupled secured feature attribute to replace the original 3DGS’s spherical harmonics coefficients and then use a scene decoder and a message decoder to disentangle the original RGB scene and the hidden message. GaussianStego\cite{li2024gaussianstego} use dino to add image watermarks and use U-Net to extract them via multi rendered views. But there is not research about watermarking 1 to 3D messages to single 3DGS scene simultaneous while keeping GS parameters or model structure unchanged.