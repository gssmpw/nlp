\section{Related Works}
\label{L_R}
Ref ____ proposed two unitary gates for reducing the dimension of the input gates and another two unitary gates to reconstruct an original number of gates. It addresses the question of the possibility of super-replication of the gates, which is considered totally from an image perspective. A cloud computing-based quantum autoencoder was proposed in ____ for reducing the number of qubit resources for communication. It addresses the problem of more quantum resources between client and server regarding communication purposes. However, it considers single qubits, whereas most generalized quantum circuits are multi-qubits.  

A higher fidelity-based QAE was proposed in ____ for enhancing data loader efficiency. It presents the enhanced fea-
ture quantum autoencoder (EF-QAE) and demonstrates its
performance through simulations. There is still a gap in how compression happens through latent space. In addition, the standard representation model used was not mentioned clearly. Moreover, it is limited to two qubits  latent space only, which is very tiny to represent a size image. Also, how only two qubits will represent the colour and its corresponding state is not appropriately investigated. Furthermore, another issue is which representation model has been used to incorporate qubit connections, such as connecting state qubits and colour qubits. Also, how compression happens in the ground states must be mentioned clearly. Besides, rate-distortion performance metrics such as PSNR versus the required number of gates per pixel were not measured based on the reconstruction image quality. 

A single photon-based quantum autoencoder has been proposed to compress quantum resources in ____. It uses a classical optimization approach to utilize quantum resources such as state. It can learn the fixed structure of the data. The reconstruction of the state, including color, is still a gap. Also, how it applies in the case of image compression is not mentioned clearly. In ____, a swap test  QAE approach has been proposed to identify fingerprints. It compares two states of the fingerprint to identify the original one. Ref____ proposed a quantum autoencoder circuit for quantum secret sharing (QSS). It addresses the gap in the speed required to run thousands of qubits. A high-rank mixed-state quantum autoencoder was proposed in ____. Another gap is how compression happens with image color, including state connection.

In____, a hybrid quantum autoencoder was proposed to compress image data into lower latent space. The cost function and gradient were measured to evaluate model efficiency. There is still a gap in how quantum information is mapped into latent space incorporating qubit connection. A machine learning quantum autoencoder approach was proposed in ____ that compresses two-qubit states into one qubit. It provides a lossless compression scheme. It is limited to a single latent space qubit. Ref____ proposed an unsupervised quantum autoencoder for denoising cluster state. In ____, an effective QAE  learning protocol was proposed to address the reduction technique of the high dimensional space into lower dimensional latent space. It can calculate corresponding eigenvalues with a lower rank of state property to measure error. 

Tom et al. (2020) reported a QAE circuit to correct Green-berger-Horne-Zeilinger (GHZ) states using depolarizing and bit flip channels ____. Gong et al. (2024) explored a quantum neural network using a quantum variational circuit angle and amplitude encoding. Due to angle and amplitude encoding, it generates probability outcomes. It is used for image classification purposes, not for compression. In 2022, Zhang et al. proposed a high-dimensional quantum autoencoder for teleportation of compressed quantum data ____. It uses integrated photonic in a compress-teleport-decompress manner. The key concept of this approach is to compress dimensional by throwing redundant information and recovering the original state using a teleportation mechanism. Park et al. (2023) described a variational one-class classifier to address the one-class classification issue using a semi-supervised fully-parameterized quantum autoencoder ____. It uses a probabilistic approach using a control rotational gate. 

To enhance predicted output accuracy toward mitigating error, a fixed-scale quantum circuit including one hot encoding approach was implemented in ____. Quantum circuit consists of a sequence of gates and the nature of timing. Anand et al.(2022) proposed classically immutable circuits for the compression of the quantum information ____. It demonstrated various quantum state compression using a low representation of quantum data. There is still a gap in how the image is encoded for both color and corresponding states. In quantum machine learning, a high dimensional function estimator is another approach to avoid over-fitting issues via parameterized quantum circuit____. It described error scalability using ansatz saturation and related between generation and expressibility using circuit depth.  

Tabi et al. (2022)  proposed a hybrid-classical quantum neural network to mitigate the noise of the quantum processor using the noisy channel. A factorized quantum measurement circuit has been presented to map the probability distribution of the state via quasi-stochastic and local unitary matrices____. A statical floating-point lossy quantum algorithm presented in ____ incorporating binary learning approach. Besides, the bias correlation approach is also applied to minimize error via inexact reconstructed data. A tree structure hybrid amplitude encoding approach has been proposed to classify image data complying with angle encoding____. It demonstrated the width and height adjustment of the image related to the quantum circuit resources. Liu et al.(2024) designed a quantum algorithm to compress classical information using hidden sub-group concept____. A flexible representation of the quantum image (FRQI) approach is generally used in the encoder and decoder section for quantum autoencoder architecture. It is proposed by Le, which encodes the pixel value using a control rotational gate ____. Figure ~\ref{FRQI_A} represents $2\times2$ FRQI image, including its states. 
\begin{figure}[htbp]
\centerline{\includegraphics[width=0.45\linewidth, height=0.35\linewidth]{result/FRQI_2_2_Gray.png}}
\caption{A $2\times2$ FRQI quantum image}
\label{FRQI_A}
\end{figure}
It can be expressed as, \\

\begin{math}
%\resizebox{1\hsize}{!}{$
|I_{FRQI}\rangle=\frac{1}{2}[\left(cos\theta_{0}|0\rangle +sin\theta_{0}|1\rangle \right) \otimes |00\rangle 
+\left(cos\theta_{1}|0\rangle+sin\theta_{1}|1\rangle \right)\otimes|01\rangle +  \left(cos\theta_{2}|0\rangle+sin\theta_{2}|1\rangle \right)\otimes|10\rangle+ \left(cos\theta_{3}|0\rangle+sin\theta_{3}|1\rangle \right)\otimes|11\rangle ]\nonumber
%$}
\end{math}

Figure ~\ref{FRQI_Circuit} depicts the circuit for representing   $|I_{FRQI}\rangle$ image____. 

\begin{figure}[t!]
\centerline{\includegraphics[width=0.8\linewidth,height=4cm]{result/FRQI_Circuit.png}}
\caption{An FRQI circuit for representing $|I_{FRQI}\rangle$ image}
\label{FRQI_Circuit}
\end{figure}

Where, $R_y(2\theta)$ indicates standard rotation metric and is given as,\\
$R_y(2\theta)$=
$\begin{pmatrix}
\centering
  cos\theta_i & -sin\theta_i \\ 
  sin\theta_i & cos\theta_i
\end{pmatrix}$ 
\\
Generally, it represents image pixel value using a control rotational gate. The control rotation gate can be implemented using standard rotation and c-not gate. It is unable to represent pixel-wise grey-scale complex operational value due to the use of a single qubit. It represents color information as an angle and position in a qubit sequence. Generally, it uses a control rotating gate and stores pixel value in the Bloch sphere. Due to the probabilistic measurement approach, it is very challenging to reconstruct the original image. Most of the autoencoder is based on FRQI architecture. 

After that, Zhang et al. (2013) proposed a novel enhanced quantum representation(NEQR) approach ____ that represents the grey-scale pixel value. It converts pixel values into a binary system, where only a frequent number of ones are considered when mapping the pixel value. In a quantum system, pixels and positions representing qubits are generally used to map quantum circuits. For example, an image containing pixel values are $0 (Y=0, X=0), 100 (Y=0, X=1), 200 (Y=1, X=0),$ and $255(Y=1, X=1)$. The pixel value's quantum representation is known as the NEQR approach and is expressed as $|I_{NEQR}\rangle$.    

\begin{flalign}
\resizebox{0.9\hsize}{!}{$
|I_{NEQR}\rangle=\frac{1}{2}[|0\rangle\otimes|00\rangle 
+|100\rangle\otimes|01\rangle+|200\rangle\otimes|10\rangle+|255\rangle \otimes|11\rangle]\nonumber
$}
\end{flalign}
Figure~\ref{NEQR_Circuit} shows the circuit diagram of NEQR approach for $|I_{NEQR}\rangle$ image.   

\begin{figure*}[htbp]
\centerline{\includegraphics[width=\linewidth,height=8cm]{result/NEQR_Approach.png}}
\caption{An NEQR circuit for pixel values representation}
\label{NEQR_Circuit}
\end{figure*}

It addresses FRQI issues and provides two qubits: one for color representation and another for the state of the grey-scale image's pixel value. An improved architecture of the NEQR circuit has been proposed in ____. It addresses the higher resource requirement of the quantum gates to connect the pixel value. It uses the transfer coefficient value and its corresponding position to implement the NZ-NEQR circuit, which requires several connections to complete the image connection. For example, Figure~\ref{fig_nzneqr_diagram}, shows the circuit diagram of the NZ-NEQR approach for pixel values of $5(X = 0, Y = 0), 248(X = 1, Y = 0), 1(X = 2, Y = 0), 8(X = 4, Y = 0)$ and $32(X=0, Y=1)$ respectively.

\begin{figure*}[htbp]
\centerline{\includegraphics[width=\linewidth,height=8cm]{result/Proposed_NEQR.png}}
\caption{A NZ-NEQR circuit for pixel values representation where an initial connection (marked as red) and zero-discarded zone (marked as orange). Zero is discarded because the identity gate has no control over the c-not gate.}
\label{fig_nzneqr_diagram}
\end{figure*}
In ____, a modified version of the SCMNEQR (state connection modification novel enhanced quantum representation) approach  ____ is known as ZSCNEQR (zero-discarded state connection novel enhance quantum representation) approach that minimizes the required number of gate connections. It is more efficient than the SCMNEQR approach because it provides better rate-distortion performance metrics via block-wise division of the transfer coefficient value. For example, Figure~\ref{fig_proposed_SCMNEQR_diagram} shows ZSCNEQR circuit diagram for quantized transfer coefficient,  $125(X=0, Y=0), 1(X=1, Y=0), 1(X=4, Y=0), 4(X=0, Y=1),$ and $16(Y=3, X=0)$ values.

\begin{figure*}[!t]
\centerline{\includegraphics[width=\linewidth,height=7cm]{result/Zero-discared_SCMNEQR.png}}
\caption{Quantized transform coefficient representation based on ZSCNEQR circuit. It includes an initial connection (marked in red) and a zero-discarded zone (marked in green)}
\label{fig_proposed_SCMNEQR_diagram}
\end{figure*}

Flip et al. 2021 demonstrated that NEQR quantum autoencoders perform better than the FRQI approach for $2\times2$ image metric____. Although it provides a better result than the FRQI approach, there are still many challenges with the NEQR-based quantum autoencoder, such as limited to $2\times2$ image. Moreover, the state label encoding system is not mentioned correctly. Every time, the state position relates to 1's connection with the transfer coefficient value, which is challenging. Performance is not assessed to measure its capacity for compression and reconstruction. Due to the probabilistic outcome, it does not mention how compression happens regarding quantum information. Besides, reconstruction quality measures were not investigated. Moreover, how it relates to the standard NEQR approach was not investigated. In addition, how medium and higher-resolution images are encoded using NEQR autoencoder is another gap. To address the above issues of the existing NEQR QAE, further study needs to improve the architecture.