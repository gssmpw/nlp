\chapter{Publications and Collaborations}

\section{Publications}
\begin{itemize}
    \item[] \href{https://www.xjenza.org/JOURNAL/OLD/7-1-2019/07.pdf}{Zerafa, C., Galea, P. \& Sebu, C. (2019). "Learning to Invert Pseudo-Spectral Data for Seismic Waveforms". \textit{Xjenza Online} \textbf{7}(1),3-17}.
    \item[] \href{https://www.earthdoc.org/content/papers/10.3997/2214-4609.201803015}{Zerafa, C. (2018) "DNN application of pseudo-spectral FWI". \textit{First EAGE/PESGB Workshop on Machine Learning: European Edition.}}
    \item[] \textbf{\textit{TBD}} \emph{Parts of this work are planned for publishing in \href{https://library.seg.org/journal/gpysa7}{SEG Geophysics} and \href{https://academic.oup.com/gji}{OUP Geophysical Journal International}}.
\end{itemize}

\section{Conferences \& Poster Presentations}
\begin{itemize}
    \item \textbf{Jul 2020} Zerafa, C. "Overview of Machine Learning Applications in Geophysics and Seismology". \textit{Department of Geosciences Summer Seminar, University of Malta}.
    \item \href{https://www.southampton.ac.uk/the-alan-turing-institute/news/events/2019/10/optimization-and-machine-learning.page}{\textbf{Nov 2019}. Zerafa, C., Galea, P. \& Sebu, C. "Learning to Invert Pseudo-Spectral Data for Seismic Waveforms". \textit{OptML: Optimization and Machine Learning, University of Southampton, UK}}.
    \item \textbf{Mar 2018}. Zerafa, C. "An Introduction to High Resolution Seismic Imaging". \textit{Scubed Annual Scientific Conference}.
\end{itemize}

\section{Appointments}
\begin{itemize}
    \item \textbf{Jan 2022} Post-Doc at Istituto Nazionale di Geofisica e Vulcanologia, Pisa.
    \newline Will be joining Istituto Nazionale di Geofisica e Vulcanologia for 2-year Post-Doc at Istituto Nazionale di Geofisica e Vulcanologia for project titled \emph{SOME Seismological Oriented Machine lEarning}. I will be involved in two main working groups:
    \begin{itemize}
        \item \emph{WP2}: Earthquake detection and characterization using large seismic datasets from tectonic/volcanic processes and hydrocarbon/geothermal exploitation. In this WP, we plan to extensively test published or newly developed supervised and unsupervised Machine Learning algorithms on the wealth of already available seismic datasets. We will use seismic datasets from regional (e.g., INSTANCE, AlpArray, Amatrice/Norcia) and local scale (Mugello, Irpinia fault system, Val d’Agri, Amiata, Larderello, Campi Flegrei, Etna).
        \item \emph{WP4}: Automatic extraction of phase and group velocity surface dispersion curves. We plan to build up on the recent work by \citet{Zhang2020}, applying the method to a various range of datasets already available at Istituto Nazionale di Geofisica e Vulcanologia \citep{Molinari2015,Molinari2020} or available in the near future as results of Department Projects at local and regional scales and in a wide frequency range.
    \end{itemize}
    \item \textbf{Apr 2020} - CA17137 WG2 - Co-Leader - \href{https://www.g2net.eu/wgs/wg2-machine-learning-for-low-frequency-seismic-measurement}{\url{www.g2net.eu/wgs/wg2}}. \newline I joined a leadership position within the working group together with Dr. Ilec. My involvement includes overseeing and organising the research initiatives within the group, introduced a common code repository to encourage collaboration and hold weekly meetings to steer the working group forward.
    \item \textbf{Sep 2018} - CA17137 University of Malta Representative - \url{https://www.g2net.eu/}. \newline Active member within CA17137 - g2net - A network for Gravitational Waves, Geophysics and Machine Learning, with particular focus to Working Group 2 - Machine Learning for low-frequency seismic measurement. Research deals with acquisition, processing and interpretation of seismic data, with the goal of combating the seismic influences at Gravitational Wave detector site, using multi-disciplinary research focussing on advanced techniques available from state of the art machine learning algorithms. 
\end{itemize}

\section{Collaborations}
\begin{itemize}
    \item \textbf{Apr 2020} Short Term Scientific Mission at Istituto Nazionale di Geofisica e Vulcanologia, Pisa. \newline In collaboration with Giunchi, C., De Matteo, G., Gaviano, S., developed two CNN approaches able to classify local earthquakes into 13 classes with different epicentre and magnitude characterizations.
\end{itemize}

\section{Organisation of Events and Guest Lecturing}
\begin{itemize}
    \item \textbf{Sep 2020} Kaggle Competition - \href{https://www.kaggle.com/c/g2net-gravitational-wave-detection}{Classification of Gravitational Wave Glitches}. Cuoco, E., Zerafa, C., Messenger, C., Williams, M. \newline Collaborating with Kaggle and other g2net members to host an international competition using gravitational wave data which could lead to novel was of automatic detection of gravitational wave detection. In turn, this could directly influence the next generation of gravitational wave research and better understanding of the known and unknown universe.
    
    \item \textbf{Mar 2020} CA17137 – g2net – 2nd Training School. \newline Hosted an international training school for CA17137 within the University of Malta. Was mainly responsible within the Scientific Organizing Committee, Local Organizing Committee and one of the lecturers. Classes though included a course in Machine Learning and hosted a hackathon for all participants. Testimonials, lecture notes, video recordings and code can  be found at \url{https://github.com/zerafachris/g2net_2nd_training_school_malta_mar_2020}.
    
    \item  \textbf{Sep 2019} PyMalta: First Steps towards Machine Learning \newline Hosted a two-part training introductory series on Machine Learning using Python. All code can be found in the repo \url{https://github.com/PyMalta/Introduction_to_ML_CZ}.
\end{itemize}

\section{Relevant Training \& Certification}
\begin{itemize}
    \item \textbf{Nov 2019} "Oberwolfach Graduate Seminar: Mathematics of Deep Learning", \textit{MFO}, \textit{Oberwalfach}, \textit{Germany}. Tutorial based course focussing on state-of-the-art mathematical analysis of deep learning algorithms. Training focussed on (i) approximation theory, (ii) expressivity, (iii) generalization, and (iv) interpretability. See~\href{https://www.mfo.de/occasion/1947a}{\url{MFO}}.
    
    \item \textbf{Mar 2019} "ML and Statistical Analysis", Distinction, \textit{The Data Incubator}. Tutorial based course on the use of ML on real world data set, with a heavy emphasis on creative use of different data science techniques to solve problems from multiple perspectives. Projects included NLP, clustering, Time series analysis and anomaly detection. See~\href{https://wqu.org/programs/data-science}{\url{DataIncubator}}.
    
    \item \textbf{Sep 2018} "Full Waveform Inversion: Maths and Geophysics", \textit{KIT, Karlsruhe, Germany}. Tutorials for high performance computing with specific focus on computational mathematics. See~\href{https://www.waves.kit.edu/summerschool2018.php}{\url{KIT2018}}.
    
    \item \textbf{Jul 2018} "International Summer School on Deep Learning", \textit{Gdansk University of Technology, Poland}. In-depth training and mini-projects on a range of learning methods, with particular focus on DNNs. See~\href{http://2018.dl-lab.eu/}{\url{ISSDL}}.
    
    \item \textbf{Jun 2018} "Scientific Computing and Python for Data Science", Distinction, \textit{The Data Incubator}. Tutorial based course covering linear regression and gradient descent techniques. Particular emphasis included cost function analysis, dimensionality reduction, regularization and feature engineering. See~\href{https://wqu.org/programs/data-science}{\url{DataIncubator}}.
    
    \item \textbf{Apr 2018} "Graphical Processing Unit-based analytics and data science, \textit{Vicomtech Research Center, Spain}. Intensive training on use of GPUs using PyCuda for big data analytics as applied to machine learning for image processing, segmentation, de-noising, filtering, interpolation and reconstruction. See~\href{https://bigskyearth.eu/apply-to-the-bigskyearth-training-school-2018-gpu-based-analytics-and-data-science/}{\url{BigSkyEarth}}.
    
\end{itemize}
