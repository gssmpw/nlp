%% For tips on how to write a great abstract, have a look at
%%	-	https://www.cdc.gov/stdconference/2018/How-to-Write-an-Abstract_v4.pdf (presentation, start here)
%%	-	https://users.ece.cmu.edu/~koopman/essays/abstract.html
%%	-	https://search.proquest.com/docview/1417403858
%%  - 	https://www.sciencedirect.com/science/article/pii/S037837821830402X

\begin{abstract}
    Full Waveform Inversion seeks to achieve a high-resolution model of the subsurface through the application of multi-variate optimization to the seismic inverse problem. Although now a mature technology, FWI has limitations related to the choice of the appropriate solver for the forward problem in challenging environments requiring complex assumptions, and very wide angle and multi-azimuth data necessary for full reconstruction are often not available.
    
    Deep Learning techniques have emerged as excellent optimization frameworks. These exist between data and theory-guided methods. Data-driven methods do not impose a wave propagation model and are not exposed to modelling errors. On the contrary, deterministic models are governed by the laws of physics. In between, there are theory-guided methods which have some fixed parameters able mimic physical processes. This enables more intelligibility as compared to purely data driven approach.
    
    Application of seismic FWI has recently started to be investigated within Deep Learning. This has focussed on the time-domain approach, while the pseudo-spectral domain has not been yet explored. However, classical FWI experienced major breakthroughs when pseudo-spectral approaches were employed. This thesis addresses the lacuna that exists in incorporating the pseudo-spectral approach within Deep Learning. This has been done by re-formulating the pseudo-spectral FWI problem as a Deep Learning algorithm for both a data-driven and a theory-guided pseudo-spectral approach. A deep neural network (DNN) and recurrent neural network (RNN) framework are derived. Either was formulated theoretically, qualitatively assessed on synthetic data, applied to a two-dimensional Marmousi dataset and evaluated against deterministic and time-based approaches.
    
    Inversion of data-driven pseudo-spectral DNN was found to outperform classical FWI for deeper and over-thrust areas. This is due to the global approximator nature of the technique and hence not bound by forward-modelling physical constraints from ray-tracing. Pseudo-spectral theory-guided FWI using RNN was shown to be more accurate with only 0.05 error tolerance and 1.45\% relative percentage error. Indeed, this provides more stable convergence, able to identify faults and has more low frequency content than classical FWI. From the comparative analysis of data-driven DNN and theory-guided RNN approaches, DNN was better performing, and recovered more of the velocity contrast, whilst RNN was better at edge definition. In general, RNN was more suited in shallow and deep sections due to cleaner receiver residuals. 
    
    Besides showing the improved performance of FWI formulated as a Deep Learning approach, this thesis highlighted the significant potential of such methods in other fields which have so far not been explored. New research avenues resulting from the shift in the inversion paradigm were identified and the next steps on how to continue developing these two frameworks presented.
 
\end{abstract}