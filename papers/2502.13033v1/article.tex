\pdfoutput=1
\documentclass[conference]{IEEEtran}
\usepackage{etoolbox}
\newbool{doubleblind}
\boolfalse{doubleblind}
\newbool{arxiv}
\booltrue{arxiv}
\def\aujourdhui{February 18th, 2025}
\newbool{natbib}
\booltrue{natbib}

\IEEEoverridecommandlockouts


\IEEEtriggercmd{\newpage}
\IEEEtriggeratref{57}


\usepackage{amsmath,amssymb,amsfonts}
\usepackage{mathtools}
\usepackage{algorithmic}
\usepackage{graphicx}
\usepackage{textcomp}
\usepackage{xcolor}
\usepackage{varwidth}
\usepackage{cprotect}
\usepackage{todonotes}\setlength{\marginparwidth}{1.7cm}
\usepackage{adjustbox}
\usepackage{mdframed}

\usepackage{array}


  \definecolor{dark-red}{rgb}{0.6,0,0}
  \definecolor{dark-blue}{rgb}{0,0.0,0.6}
  \definecolor{blue}{rgb}{0,0,0.8}
\def\pdfauthor{Éléonore Mangel, Paul-André Melliès and Guillaume Munch-Maccagnoni}
\usepackage[unicode=true,pdfusetitle,pdfauthor={\pdfauthor},
 bookmarks=true,bookmarksnumbered=false,bookmarksopen=false,
 breaklinks=true,pdfborder={0 0 0},pdfborderstyle={},backref=page,colorlinks=true]
{hyperref}
\hypersetup{
 ocgcolorlinks,linkcolor={dark-red},citecolor={dark-blue},urlcolor={blue}}


\definecolor{red}{rgb}{.8,0.2,0.2}
\definecolor{rouge}{rgb}{.8,0,0}
\definecolor{green}{rgb}{0,.5,0}
\definecolor{antigreen}{rgb}{.8,0,.8}
\definecolor{blue}{rgb}{0,0,1}
\definecolor{littleblue}{rgb}{.2,.2,.6}
\definecolor{graille}{rgb}{.01,.01,.00}
\definecolor{grispale}{rgb}{.6,.6,.6}
\definecolor{yellow}{rgb}{.5,.5,.2}
\definecolor{cyan}{rgb}{0.75,1,1}

\newcommand{\golden}[1]{\textcolor{yellow}{#1}}
\newcommand{\noire}[1]{\textcolor{black}{#1}}
\newcommand{\rouge}[1]{\textcolor{rouge}{#1}}
\newcommand{\blanche}[1]{\textcolor{white}{#1}}
\newcommand{\verte}[1]{\textcolor{green}{#1}}
\newcommand{\bleue}[1]{\textcolor{blue}{#1}}
\newcommand{\grease}[1]{\textcolor{graille}{#1}}
\newcommand{\grispale}[1]{\textcolor{grispale}{#1}}
\newcommand{\surligner}[1]{\colorbox{yellow}{#1}}
\newcommand{\blancheligner}[1]{\colorbox{white}{#1}}
\newcommand{\allumer}[1]{\colorbox{green}{#1}}
\newcommand{\remark}[1]{\textcolor{goldie}{#1}}

\usepackage[utf8]{inputenc}
\ifbool{arxiv}{\usepackage[a4paper,DIV=19]{typearea}}{}
\usepackage[T1]{fontenc}
\usepackage{amsmath}
\usepackage{amssymb}
\usepackage{amsthm}
\usepackage{amsfonts}
\usepackage{stmaryrd}
\usepackage{mathrsfs}
\usepackage{cmll}
\usepackage{perfectcut}
\usepackage{ebproof}
\usepackage{tikz-cd}
\usepackage{bbold}
\usepackage[
]{cleveref}
\crefname{theorem}{thm.}{thms.}\crefname{figure}{fig.}{figs.}\crefname{example}{ex.}{exs.}\crefname{equation}{eq.}{eqs.}\crefname{lemma}{lem.}{lems.}\crefname{definition}{def.}{defs.}\crefname{proposition}{prop.}{props.}\Crefname{theorem}{Theorem}{Theorems}\Crefname{figure}{Figure}{Figures}\Crefname{example}{Example}{Examples}\Crefname{equation}{Equation}{Equations}\Crefname{lemma}{Lemma}{Lemmas}\Crefname{definition}{Definition}{Definitions}\Crefname{proposition}{Proposition}{Propositions}

\ifbool{natbib}{
  \usepackage[numbers]{natbib}
  \usepackage{flushend}
}{}

\makeatletter
\@ifundefined{showcaptionsetup}{}{\PassOptionsToPackage{caption=false}{subfig}}
\usepackage{subfig}
\makeatother


\usepackage{microtype}\usepackage[font=small]{caption}


\makeatletter
\g@addto@macro\bfseries{\boldmath}
\makeatother

\makeatletter
\newcommand{\ostar}{\mathbin{\mathpalette\make@circled\star}}
\newcommand{\make@circled}[2]{\ooalign{$\m@th#1\smallbigcirc{#1}$\cr\hidewidth$\m@th#1#2$\hidewidth\cr}}
\newcommand{\smallbigcirc}[1]{\vcenter{\hbox{\scalebox{0.77778}{$\m@th#1\bigcirc$}}}}
\makeatother
\newtheorem{theorem}{Theorem}[section]
\newtheorem{lemma}[theorem]{Lemma}
\newtheorem{proposition}[theorem]{Proposition}
\newtheorem{definition}[theorem]{Definition}
\newtheorem{axiom}[theorem]{Axiom}
\newtheorem{corollary}[theorem]{Corollary}

\newcommand{\define}[1]{{\bf #1}}
\newcommand{\dom}{\mathsf{dom}\hspace{0.5mm}}
\newcommand{\fv}{\upharpoonright\mathbf{fv}\hspace{0.4mm}}
\newcommand{\fvwo}{\mathbf{fv}\hspace{0.4mm}}
\newcommand{\fcv}{\upharpoonright\mathbf{fcv}\hspace{0.4mm}}
\newcommand{\fcvwo}{\mathbf{fcv}\hspace{0.4mm}}
\newcommand{\tsigma}{\tilde\sigma}
\newcommand{\tmu}{\tilde\mu}
\newcommand{\negN}{-^*}
\newcommand{\unegN}{\underline\neg^-}
\newcommand{\negP}{+^*}
\newcommand{\unegP}{\underline\neg^+}
\newcommand{\nuparrow}{\mathord{\uparrow}}
\newcommand{\ndownarrow}{\mathord{\downarrow}}
\newcommand{\veps}{\varepsilon}
\newcommand{\1}{\mathsf{1}}
\newcommand{\nUparrow}{\mathord{\Uparrow}}
\newcommand{\nDownarrow}{\mathord{\Downarrow}}
\newcommand{\op}{\mathsf{op}}
\newcommand{\dupl}{\mathcal Dupl}
\newcommand{\Dduploid}{\mathcal D}
\newcommand{\Eduploid}{\mathcal E}
\newcommand{\inclusion}{\hookrightarrow}

\newcommand{\chirdual}[1]{#1^{*}}
\newcommand{\chirleftdual}[1]{^{*}#1}
\newcommand{\chirrightdual}[1]{#1^{*}}
\newcommand{\dupldual}[1]{#1^{*}}
\newcommand{\dupldoubledual}[1]{#1^{**}}

\newcommand{\command}[1]{\mathsf{#1}}

\newcommand{\pcomp}{\mathbin{\bullet}}
\newcommand{\ncomp}{\mathbin{\circ}}
\newcommand{\ccomp}{\mathbin{\mathrlap{\mathchoice{\mspace{2mu}}{\mspace{2mu}}{\mspace{1.85mu}}{\mspace{1.7mu}}\mathord{\cdot}}\mathord{\circ}}}

\newcommand{\keyword}[1]{\mathop{\mathbf{#1}}{}}

\def\letinspace{3mu}
\newcommand{\letin}[3]{\keyword{let}\mkern\letinspace#1\mkern\letinspace\mathbin{\boldsymbol{=}}\mkern\letinspace#2\mkern\letinspace\keyword{in}\mkern\letinspace#3}
\newcommand{\letinx}[4]{\keyword{let}\mkern\letinspace#1\mkern\letinspace\mathbin{\stackrel{#4}{\boldsymbol{=}}}\mkern\letinspace#2\mkern\letinspace\keyword{in}\mkern\letinspace#3}
\newcommand{\letinplus}[3]{\letinx{#1}{#2}{#3}{\oplus}}
\newcommand{\letinminus}[3]{\letinx{#1}{#2}{#3}{\ominus}}
\newcommand{\letinepsilon}[3]{\letinx{#1}{#2}{#3}{\varepsilon}}
\newcommand{\letinepsilonprime}[3]{\letinx{#1}{#2}{#3}{\varepsilon'}}
\newcommand{\true}{\textbf{true}}
\newcommand{\false}{\textbf{false}}


\newcommand{\Acategory}{\mathscr{A}}
\newcommand{\Bcategory}{\mathscr{B}}
\newcommand{\Ccategory}{\mathscr{C}}
\newcommand{\Dcategory}{\mathscr{D}}
\newcommand{\Ecategory}{\mathscr{E}}
\newcommand{\Mcategory}{\mathscr{M}}
\newcommand{\Ncategory}{\mathscr{N}}
\newcommand{\Pcategory}{\mathscr{P}}
\newcommand{\Ncategoryl}{\mathscr{N}_l}
\newcommand{\Pcategoryt}{\mathscr{P}_t}
\newcommand{\twocategory}{\mathbf{2}}
\newcommand{\kleisli}[2]{\mathbf{Kl}{[#1,#2]}}
\newcommand{\cokleisli}[2]{\mathbf{coKl}{[#1,#2]}}
\newcommand{\bikleisli}[3]{\mathbf{biKl}{[#1,#2,#3]}}
\newcommand{\collage}[2]{\mathbf{coll}_{#1,#2}}
\newcommand{\UpMonad}{\mathord{\uparrow}}
\newcommand{\DownComonad}{\mathord{\downarrow}}


\newcommand{\LC}{\textbf{LC}}
\newcommand{\LKT}{\textbf{LKT}}
\newcommand{\LKQ}{\textbf{LKQ}}
\newcommand{\polsystem}[1]{\textbf{#1}^\eta_p}

\newcommand{\tensorialand}{\varowedge}
\newcommand{\tensorialor}{\varovee}
\newcommand{\tensorialtrue}{\mathbf{\mathit{true}}}
\newcommand{\tensorialfalse}{\mathbf{\mathit{false}}}

\newcommand{\Lprime}{L_{\Ecategory}}
\newcommand{\Rprime}{R_{\Ecategory}}
\newcommand{\duploid}[2]{\mathbf{dupl}_{#1,#2}}

\newcommand{\median}{\mathbf{trans}}
\newcommand{\id}[1]{\mathrm{id}_{#1}}
\newcommand{\tensor}{\otimes}
\newcommand{\positivetensor}{\ostar}

\newcommand{\Set}{\mathit{Set}}

\newcommand{\bmid}{\mathrel{\boldsymbol{\nthleft{1}|}}}
\newcommand{\stretcharray}{\arraycolsep=0.5ex\def\arraystretch{1.3}}

\newcommand{\symbolscriptstyle}[1]{{\raisebox{0.07ex}{\hbox{\mbox{\ensuremath{{\scriptstyle #1}}}}}}}
\newcommand{\symbolscriptscriptstyle}[1]{{\raisebox{0.05ex}{\hbox{\mbox{\ensuremath{{\scriptscriptstyle #1}}}}}}}
\newcommand{\smallsymbol}[1]{\mkern1mu{\mathchoice{\symbolscriptstyle{#1}}{\symbolscriptstyle{#1}}{\symbolscriptscriptstyle{#1}}{\symbolscriptscriptstyle{#1}}}\mkern1mu}
\newcommand{\smallotimes}{\smallsymbol{\otimes}}
\newcommand{\smallparr}{\smallsymbol{\parr}}

\newcommand{\ltensortimes}{\ltensor}
\newcommand{\rtensortimes}{\rtensor}
\newcommand{\lparrtimes}{\ltimes}
\newcommand{\rparrtimes}{\rtimes}



\RequirePackage{graphicx}
\RequirePackage{calc}

\makeatletter
\newsavebox\gadmm@boxsrc
\newsavebox\gadmm@boxtrg
\newcommand{\gadmm@resize}[4]{\sbox{\gadmm@boxtrg}{$\m@th#4$}\raisebox{-#3}{\resizebox{#1}{#2+#3}{\raisebox{\dp\gadmm@boxtrg}{\usebox{\gadmm@boxtrg}}}}}
\newcommand{\gadmm@resizeto}[2]{\sbox{\gadmm@boxsrc}{$\m@th#1$}\gadmm@resize{\wd\gadmm@boxsrc}{\ht\gadmm@boxsrc}{\dp\gadmm@boxsrc}{#2}
}
\newcommand{\gadmm@resizeadjust}[3]{\sbox{\gadmm@boxsrc}{$\m@th#1$}\gadmm@resize{\wd\gadmm@boxsrc-#3-#3-#3}{\ht\gadmm@boxsrc-#3-#3-#3}{\dp\gadmm@boxsrc+#3}{#2}
}

\newcommand{\rtensornormal}{\mkern2mu\gadmm@resizeadjust{\rtimes}{\mathbin{>\mkern-4mu\blacktriangleleft}}{0.8pt}}
\newcommand{\rtensorscript}{\mkern2.5mu\gadmm@resizeadjust{\scriptstyle{\rtimes}}{\mathbin{\scriptstyle{>\mkern-4mu\blacktriangleleft}}}{0.8pt}}
\newcommand{\rtensorsscript}{\mkern3mu\gadmm@resizeadjust{\scriptscriptstyle{\rtimes}}{\mathbin{\scriptscriptstyle{>\mkern-4mu\blacktriangleleft}}}{0.8pt}}

\newcommand{\rtensor}{\mathbin{\mathrlap{\mathchoice{\rtensornormal}{\rtensornormal}{\rtensorscript}{\rtensorsscript}}\rtimes}}

\newcommand{\ltensornormal}{\mkern2.3mu\gadmm@resizeadjust{\ltimes}{\mathbin{\blacktriangleright\mkern-4mu<}}{0.8pt}}
\newcommand{\ltensorscript}{\mkern2.5mu\gadmm@resizeadjust{\scriptstyle{\ltimes}}{\mathbin{\scriptstyle{\blacktriangleright\mkern-4mu<}}}{0.8pt}}
\newcommand{\ltensorsscript}{\mkern2.8mu\gadmm@resizeadjust{\scriptscriptstyle{\ltimes}}{\mathbin{\scriptscriptstyle{\blacktriangleright\mkern-4mu<}}}{0.8pt}}

\newcommand{\ltensor}{\mathbin{\mathrlap{\mathchoice{\ltensornormal}{\ltensornormal}{\ltensorscript}{\ltensorsscript}}\ltimes}}

\makeatother



\newcommand{\joyallemma}{Joyal's obstruction theorem}

\usepackage{tikz}
\usepackage{tikz-cd}
\usetikzlibrary{matrix}
\usetikzlibrary{arrows}
\usetikzlibrary{arrows.meta}
\usetikzlibrary{decorations.pathmorphing,shapes}
\usetikzlibrary{decorations.markings}



\tikzset{spanmap/.style={
            decoration={markings,
            mark= at position 0.5 with {
                  \node[transform shape] (tempnode) {$|$};
}
              },
              postaction={decorate}
}
}

\begin{document}

\def\FH{Hasegawa-Thielecke}
\title{\texorpdfstring
{Classical notions of computation\\
and the \FH{} theorem\ifbool{arxiv}{\thanks{\aujourdhui}}{}}
{Classical notions of computation and the \FH{} theorem}
}

\ifbool{doubleblind}{

\author{\IEEEauthorblockN{Intentionally left blank for double-blind reviewing}
\IEEEauthorblockA{}}

}{

\author{\IEEEauthorblockN{\'El{\'e}onore Mangel}
\IEEEauthorblockA{\textit{Univ. Paris Cit{\'e}, CNRS, INRIA}\\
Paris, France}
\and
\IEEEauthorblockN{Paul-André Melli{\`e}s}
\IEEEauthorblockA{\textit{Univ. Paris Cit{\'e}, CNRS, INRIA}\\
Paris, France}
\and
\IEEEauthorblockN{Guillaume Munch-Maccagnoni}
\IEEEauthorblockA{\textit{INRIA, LS2N CNRS} \\
Nantes, France}
}

}

\maketitle
\ifbool{arxiv}{
  \thispagestyle{plain}
  \pagestyle{plain}
}{}

\begin{abstract}
In the spirit of the Curry-Howard correspondence between proofs and
programs, we define and study a syntax and semantics for classical
logic equipped with a computationally involutive negation, using a
polarised effect calculus. A main challenge in designing a denotational semantics
is to accommodate both call-by-value and call-by-name evaluation
strategies, which leads to a failure of associativity of composition.
Building on the work of \ifbool{doubleblind}{Munch-Maccagnoni}{the
  third author}, we devise the notion of \emph{dialogue duploid},
which provides a non-associative and effectful counterpart to the
notion of dialogue category introduced by
\ifbool{doubleblind}{Melli{\`e}s}{the second author} in his
2-categorical account, based on adjunctions, of logical polarities and continuations.
We show that the syntax of the polarised calculus can be interpreted in any
dialogue duploid, and that it defines in fact a syntactic dialogue duploid.
As an application, we establish, by semantic as well as syntactic
means, the \FH{} theorem, which states that the notions
of central map and of thunkable map coincide in any dialogue duploid (in
particular, for any double negation monad on a symmetric monoidal
category).
\end{abstract}

\section{Introduction}

\subsection{Adjunctions, duploids, and notions of computation}



In this paper, we combine methods coming from
proof theory and programming language semantics to investigate
the meaning of effectful expressions
for proofs or programs,
starting with those of the form
\begin{equation}\label{equation/letin}
\letin{a}{u}{t}
\end{equation}
where $t$ and $u$ are effectful
expressions,
and where~$t$ possibly contains free instances
of the variable~$a$.
One main difficulty we face is that there are 
two canonical ways of assigning meaning to the $\keyword{let}$ construct~\eqref{equation/letin}
depending on the evaluation paradigm at work:


\medbreak
\noindent
\emph{In the call-by-value (CBV) paradigm,} the expression~$u$ 
performs a number of actions and returns a value $v$~; the value~$v$ 
is then substituted for every free instance of the variable~$a$ in the expression~$t$~;
it is then the turn of the expression $t[a:=v]$ to perform its actions and to return a value.


\medbreak 

\noindent
\emph{In the call-by-name (CBN) paradigm,}
the expression~$t$ performs its actions and is evaluated while
the expression~$u$ is ``frozen'' and substituted for each free instance 
of the variable~$a$ in~$t$~;~a
new copy of the expression~$u$ performs its actions and is evaluated
each time a free instance of the variable~$a$
appears as head variable during the evaluation of the expression $t[a:=u]$.
\medbreak



\noindent
\textbf{Kleisli categories.}
The seminal work on computational effects by Moggi
\cite{Moggi89computationallambda-calculus, Moggi1991} initiated a
well-established tradition~\cite{Filinski94representingmonads,PowerRobinson97,fuhrmanndirectmodels,Levy99CBPV,Plotkin_2002,Power2002,ThieleckeFuhrmann04,Lindley_2005,Katsumata2005}
of interpreting CBV expressions of type $a:A\vdash t:B$ as maps
$t:A\to B$ in the Kleisli category~$\kleisli{\Ccategory}{T}$
associated to a monad~$(T,\mu,\eta)$ on a category~$\Ccategory$.
Recall that a map $f:A\to B$ in the Kleisli category is a map $f:A\to TB$ in the original category~$\Ccategory$
and that two maps~$f:A\to TB$ and~$g:B\to TC$ are composed 
using the multiplication~$\mu$ of the monad:
$$
\begin{tikzcd}[column sep=1.em]
g\pcomp f \quad = \quad A\arrow[rr,"f"] && TB \arrow[rr,"Tg"] && TTC \arrow[rr,"\mu_C"] && TC
\end{tikzcd}
$$
Symmetrically, there is a well-established tradition after Girard~\cite{Gir87}
of interpreting CBN expressions of type $a:A\vdash t:B$ as maps $t:A\to B$ 
in the co-Kleisli category~$\cokleisli{\Ccategory}{K}$ associated to a computational comonad~$(K,\delta,\varepsilon)$
on a given category~$\Ccategory$ of types and pure programs.
Recall that a map $f:A\to B$ in the co-Kleisli category is a map $f:KA\to B$ in the original category $\Ccategory$
and that two maps $f:A\to B$ and $g:B\to C$ are composed in the co-Kleisli category
using the comultiplication~$\delta$ of the comonad:
$$
\begin{tikzcd}[column sep=1.em]
g\ncomp f \quad = \quad KA\arrow[rr,"\delta_A"] && KKA \arrow[rr,"Kf"] && KB \arrow[rr,"g"] && C
\end{tikzcd}
$$
\noindent
The {mathematical property}
that composition is \emph{associative} in $\kleisli{\Ccategory}{T}$ and $\cokleisli{\Ccategory}{K}$,
in the sense that
\[
\begin{array}{ccc}
h\pcomp (g\pcomp f) \, = \, (h\pcomp g)\pcomp f
& & 
h\ncomp (g\ncomp f) \, = \, (h\ncomp g)\ncomp f
\end{array}
\]
reflects the {computational property} that for all effectful expressions
$\vdash f:A$, $a:A\vdash g:B$ and $b:B\vdash h:C,$
the two effectful expressions~$(i)$ and~$(ii)$ defined below
\begin{center}
\fbox{
\begin{tabular}{ccc}
\vspace{-1.2em}
\\
$(i)$ & $\letinepsilon{a}{f}{(\letinepsilonprime{b}{g}{h})}$ &
\\
\vspace{-.8em}
\\
$(ii)$ & $\letinepsilonprime{b}{(\letinepsilon{a}{f}{g})}{h}$ &
\\
\vspace{-1em}
\\
\end{tabular}
}
\end{center}
are equal whenever the \emph{polarities}
$\varepsilon,\varepsilon'\in\{\oplus,\ominus\}$
of the $\keyword{let}$ constructs are the same.
Here, we use the polarity $\varepsilon\in\{\oplus,\ominus\}$
to indicate in which style $\letinepsilon{a}{u}{t}$ should be evaluated:
CBV ($\varepsilon=\oplus$) or CBN ($\varepsilon=\oplus$).
The fact that the expressions~$(i)$ and~$(ii)$ behave in the same way
implies in particular that they evaluate $f$, $g$ and~$h$ in the same order
in CBV as well as in CBN, as shown below.
\[
\begin{array}{|c|c|}
\hline
\mbox{composition style}
&
\mbox{order of evaluation}
\\
\hline
(\varepsilon,\varepsilon') = (\oplus,\oplus) & 
\hspace{-2.9mm}
\begin{array}{cccc}
\vspace{-.8em}
\\
& (i) = (ii) &
\mbox{$f$ then $g$ then $h$} 
&
\\
\vspace{-1em}
\\
\end{array}
\\
\hline
(\varepsilon,\varepsilon') = (\ominus,\ominus) 
&
\hspace{-2.9mm}
\begin{array}{cccc}
\vspace{-.8em}
\\
& (i) = (ii) & 
\mbox{$h$ then $g$ then $f$}
&
\\
\vspace{-1em}
\\
\end{array}
\\
\hline
\end{array}
\]


\medskip

\noindent
\textbf{Mixing call by name and call by value.} In many concrete
situations, the programmer would like to control and reason about the
order of evaluation.
This can be modelled by letting
both styles of $\keyword{let}$ constructs appear inside expressions.
Inspecting the two effectful expressions~$(i)$ and~$(ii)$ again in that hybrid scenario,
we see that the two expressions~$(i)$ and~$(ii)$ behave in the same way when $(\varepsilon,\varepsilon') = (\ominus,\oplus)$
but behave differently when $(\varepsilon,\varepsilon') = (\oplus,\ominus)$.
In particular, in that latter case, the expression~$f$ is evaluated before $h$ and then $g$ in~$(i)$
whereas the expression~$h$ is evaluated before $f$ and then $g$ in~$(ii)$.
\[
\begin{array}{|c|c|}
\hline
\mbox{composition style}
&
\mbox{order of evaluation}
\\
\hline
(\varepsilon,\varepsilon') = (\ominus,\oplus) &
\hspace{-2.9mm}
\begin{array}{cccc}
\vspace{-.8em}
\\
& (i) = (ii) & 
\mbox{$g$ then $h$ then $f$}
\\
\vspace{-1em}
\\
\end{array}
\\
\hline
(\varepsilon,\varepsilon') = (\oplus,\ominus) & 
\begin{array}{cccc}
\vspace{-.8em}
\\
&
\hspace{.8em} (i) \hspace{.8em}
& 
\mbox{$f$ then $h$ then $g$}
& \\
& 
\hspace{.8em} (ii) \hspace{.8em}
&
\mbox{$h$ then $f$ then $g$}
&
\\
\vspace{-1em}
\\
\end{array}
\\
\hline
\end{array}
\]
A natural question is how we could develop a mathematical framework
that considers seriously the combination of evaluation paradigms,
without a priori biases towards monads nor comonads.
In order to reflect these equations,
such a framework needs to integrate both Kleisli and co-Kleisli categories,
where the former associativity equation holds
$$
(h\pcomp g)\ncomp f \hspace{.5em} = \hspace{.5em} h\pcomp (g\ncomp f)
$$
but where the latter associativity equation
$$
(h\ncomp g)\pcomp f \hspace{.5em} = \hspace{.5em} h\ncomp (g\pcomp f)
$$
does not necessarily hold in general.
There is no hope of defining categories and we thus need to move 
to ``non-associative'' forms of categories.
This is the direction taken by \ifbool{doubleblind}{Munch-Maccagnoni}{the
  third author}~\cite{munchduploids}
based on a non-associative and polarized notion of \ifbool{doubleblind}{so-called }{}\emph{duploid}.


The idea of non-associativity is far from new:
it appeared for the first time in Girard's ``constructive'' classical logic $\LC$, 
which introduced a formal distinction between ``positive'' and ``negative''
formulae~\cite{girardnew}.
The idea then resurfaced with the ``Blass problem'' in game semantics~\cite{Blass1992a,Abramsky2003}, 
whose origin was traced back to the existence of an adjunction between 
categories of ``positive'' and ``negative'' games~\cite{Mellies2005ag3}. 
However, non-associativity was mainly perceived as an anomaly
until the introduction of duploids~\cite{munchduploids}
and their \emph{computational account of adjunctions}
where it was shown that having ``three fourths'' of the associativity
equations captures directly effectful computation integrating both
monadic and comonadic effects.


\medbreak

\noindent
\textbf{Adjunctions.}
In order to intertwine the interpretations of CBV 
and CBN evaluation in a single mathematical structure including the Kleisli and co-Kleisli categories,
a good starting point is indeed to consider a pair of adjoint functors\begin{equation}\label{equation/adjunction-LR}
\begin{tikzcd}[column sep = 1em]
\Acategory \arrow[rr,"L", bend left] & \bot & \Bcategory\arrow[ll,"R",bend left]
\end{tikzcd}
\end{equation}
Incidentally, shifting attention from Moggi's monads to adjunctions is
now standard, notably after Levy's
Call-by-Push-Value~\cite{Levy99CBPV,Levy2004,Levy05adjunctionmodels}.

Recall that the left adjoint functor~$L$ and the right adjoint functor~$R$
are related by
a pair of natural transformations
\begin{center}
\begin{tikzcd}[column sep = 1em]
\eta : Id_{\Acategory} \arrow[Rightarrow,rr] && R\circ L
\end{tikzcd}
\quad\quad
\begin{tikzcd}[column sep = 1em]
\varepsilon : L\circ R\arrow[Rightarrow,rr] && Id_{\Bcategory} 
\end{tikzcd}
\end{center}
called \emph{unit} and \emph{counit} of the adjunction, satisfying the
triangular equations~\cite{Street_1972,PAM2009} depicted as zigzags in
the language of string diagrams:
\begin{center}
\begin{tabular}{cc}
\includegraphics[height=4.25em]{R-zig-zag-a.pdf}
\raisebox{1.8em}{$=$}
\includegraphics[height=4.25em]{R-zig-zag-b.pdf}
&\hspace{-.6em}
\includegraphics[height=4.25em]{L-zig-zag-a.pdf}
\raisebox{1.8em}{$=$}
\includegraphics[height=4.25em]{L-zig-zag-b.pdf}
\end{tabular}
\end{center}
The orientations of the strings~$L$ and~$R$ are imported 
from the functorial description of game semantics in string diagrams\ifbool{doubleblind}{ developed by Melli{\`e}s}{}~\cite{Mellies12}
where the functor~$R$ is understood as an input (or Opponent move)
and the functor~$L$ as an output (or Player move).
As we will see very soon, these orientations can be seen as describing
the flow of control in expressions, reframing and generalising an idea
by Jeffrey~\cite{Jeffrey1998}.

\medbreak

The adjunction induces a monad $T=R\circ L$ on the category~$\Acategory$ 
and a comonad~$K=L\circ R$ on the category~$\Bcategory$.
In order to mix the CBV style
and the CBN style
we need to combine the Kleisli category~$\kleisli{\Acategory}{T}$ and
the co-Kleisli category~$\cokleisli{\Bcategory}{K}$ in a single algebraic structure.
\medbreak

\noindent
\textbf{The collage category of an adjunction.}
It is well-known that an adjunction $L\dashv R$
can equivalently be seen as 
a bifibration 
$p:\Ecategory\to\twocategory$ over the order category $\twocategory={0\to 1}$
with two objects~$0$ and~$1$ and a unique map $\median:0\to 1$.
Here, the category $\Ecategory=\collage{L}{R}$
is defined as the \emph{collage} of the adjunction $L\dashv R$:
its objects are the pairs $(0,A)$ where $A$ is an object of $\Acategory$
and the pairs $(1,B)$ where $B$ is an object of $\Bcategory$, and
\begin{itemize}
\item its maps $(0,A)\to(0,A')$ are the maps $A\to A'$ in~$\Acategory$,
\item its maps $(1,B)\to(1,B')$ are the maps $B\to B'$ in~$\Bcategory$,
\item its maps $(0,A)\to(1,B)$ are the maps $A\to RB$ in $\Acategory$
or equivalently $LA\to B$ in $\Bcategory$,
\item there are no maps of the form $(1,B)\to(0,A)$.
\end{itemize}
The bifibration $p:\Ecategory\to\twocategory$ 
transports every object of the form~$(0,A)$ to $0$ 
and of the form $(1,B)$ to~$1$.
The category~$\Ecategory$ comes equipped with
two injective on objects and fully faithful functors
$
\begin{tikzcd}[column sep = 2em]
\Acategory \arrow[rr,"{inj_{\Acategory}}"] && \Ecategory 
&&
\Bcategory \arrow[ll,"{inj_{\Bcategory}}"{swap}]
\end{tikzcd}
$
identifying~$\Acategory$ and~$\Bcategory$ as the fibers over~$0$ and~$1$ respectively.
We find convenient to write $A$ for $(0,A)$ and $B$ for $(1,B)$ when there are no ambiguities.
We also call \emph{transverse} a map of the form $f:A\to B$ with image $p(f)=\median$.
A remarkable property of~$\Ecategory$ is that the adjunction $L\dashv R$ factors as a pair of adjunctions:
$$
\begin{tikzcd}[column sep = 1em]
\Acategory \arrow[rr,"{inj_{\Acategory}}", bend left] & \bot &
\Ecategory\arrow[ll,"{{\Rprime}}",bend left] \arrow[rr,"{{\Lprime}}", bend left] & \bot & \Bcategory\arrow[ll,"{inj_{\Bcategory}}",bend left]
\end{tikzcd}
$$
where the functors ${\Lprime}$ and ${\Rprime}$ are entirely determined 
by the factorisation property and the equations
$$
\begin{array}{ccc}
{\Rprime}\circ{inj_{\Bcategory}} = R
& \quad\quad &
{\Lprime}\circ{inj_{\Acategory}} = L
\\
{\Rprime}\circ{inj_{\Acategory}} = id_{\Acategory}
& \quad\quad &
{\Lprime}\circ{inj_{\Bcategory}} = id_{\Bcategory}
\end{array}
$$
From this pair of adjunctions, it follows that the category~$\Ecategory$ comes equipped 
with a comonad~$\DownComonad$ and a monad~$\UpMonad$ below
$$\DownComonad{} \, = \, inj_{\Acategory}\circ R
\quad\quad\quad
\UpMonad{} \, = \, inj_{\Bcategory}\circ L$$
defined as
$$
\DownComonad A = A \quad \DownComonad B = RB
\quad
\UpMonad A = LA
\quad
\UpMonad B = B
$$
An easy inspection shows that the monad~$\UpMonad$ and comonad~$\DownComonad$ are idempotent, 
in the strong sense that the multiplication $\mu_X:{\UpMonad\UpMonad X}\,\,{\to}\,\,{\UpMonad X}$ of the monad
and the comultiplication~$\delta_X:{\DownComonad X}\,\,{\to}\,\,{\DownComonad\DownComonad X}$ of the comonad are identities.


\medbreak

\noindent
\textbf{The duploid of an adjunction as a non-associative bi-Kleisli
  construction.} An enlightening way to understand the construction of
the duploid $\duploid{L}{R}$ associated to the adjunction~$L\dashv R$
in~\cite{munchduploids} is to see it as a non-associative variant (and
generalization) of the usual bi-Kleisli construction\footnote{We
discovered and developed this bi-Kleisli formulation of the duploid
construction before learning that this observation should probably be
attributed to T. Tsukada.}
\cite{Power2002,harmer-hyland-mellies:lics2007} on the
comonad~$\DownComonad$ and monad~$\UpMonad$ of the
collage~$\Ecategory=\collage{L}{R}$.
It appears indeed that there is a family of maps in the category~$\Ecategory$
\begin{equation}\label{equation/distributivity-law}
\begin{tikzcd}
\lambda_X \quad : \quad \DownComonad\, \UpMonad \, X \arrow[rr] && \UpMonad\, \DownComonad \, X 
\end{tikzcd}
\end{equation}
parametrized by the objects~$X$ of $\Ecategory$, which satisfies all
the equations of a distributivity law between a comonad and a monad
\emph{except for the naturality condition}, see
\S\ref{section/duploids-as-bikleisli} for details.
The duploid~$\duploid{L}{R}=\bikleisli{\Ecategory}{\UpMonad}{\DownComonad}$
can be then obtained as the non-associative category with bi-Kleisli maps $X\to Y$ 
defined as $\DownComonad X\to \UpMonad Y$ in $\Ecategory$.
The composite noted $g\ccomp f$ of two bi-Kleisli maps
$$
\begin{tikzcd}[column sep = 1em]
f : {\DownComonad X} \arrow[rr] && {\UpMonad Y}
\end{tikzcd}
\quad\quad
\begin{tikzcd}[column sep = 1em]
g : {\DownComonad Y} \arrow[rr] && {\UpMonad Z}
\end{tikzcd} 
$$
is defined in the same way as in usual (associative) bi-Kleisli categories,
using the distributivity law~$\lambda$:
$$
\begin{tikzcd}[column sep = .8em]
\DownComonad X \arrow[rr,"\delta_X"] && \DownComonad\DownComonad X \arrow[rr,"\DownComonad f"] 
&& \DownComonad\UpMonad Y \arrow[rr,"\lambda_Y"] && \UpMonad\DownComonad Y \arrow[rr,"\UpMonad g"] 
&& \UpMonad\UpMonad Z \arrow[rr,"\mu_Z"] && \UpMonad Z
\end{tikzcd}
$$
An easy computation shows that
\begin{itemize}
\item $\kleisli{\Acategory}{RL}$ coincides with the full subcategory
of positive objects (= objects of~$\Acategory$)
in $\duploid{L}{R}$,
\item $\cokleisli{\Bcategory}{LR}$ coincides with the full subcategory
of negative objects (= objects of~$\Bcategory$)
in $\duploid{L}{R}$.
\end{itemize}
For that reason, it makes sense to write the composite $g\ccomp f$ 
as $g\pcomp f$ when $Y$ is positive, and as $g\ncomp f$ when $Y$ is negative.

\medbreak 

\noindent
The bi-Kleisli construction
establishes the non-associative category~$\duploid{L}{R}$ 
as a simple and canonical way to integrate the Kleisli and co-Kleisli 
categories in a larger overarching mathematical structure.
The fact that bi-Kleisli composition
is not associative comes from the fact that three maps
\begin{center}
\begin{tikzcd}[column sep = 1em]
A'\arrow[rr,"f"] && A\arrow[rr,"g"] && B \arrow[rr,"h"] && B'
\end{tikzcd}
\end{center}
defining a path of length 3
in $\bikleisli{\Ecategory}{\UpMonad}{\DownComonad}$
may induce different composite maps 
$$(i):\;\;(h\ncomp g)\pcomp f \quad\quad\quad (ii):\;\;h\ncomp (g\pcomp f)$$
We will see in \S\ref{section/duploids-as-bikleisli} how to detect the difference
between the two maps $(i)$ and $(ii)$ by observing 
the flow of control determined by the trajectories
of the functors~$L$ and~$R$ 
as depicted in their string diagrams:\footnote{Note that each object $A$ of the category~$\Acategory$ is
seen here as a functor $A:\mathbb{1}\to\Acategory$ from the terminal
category~$\mathbb{1}$, and that each Kleisli map $f:A'\to RLA$ of the
monad $R\circ L$ is seen (up to bijection) as a natural transformation
$f$ from $L\circ A':\mathbb{1}\to\Bcategory$ to $L\circ
A:\mathbb{1}\to\Bcategory$ (and dually for objects~$B$ of $\Bcategory$
and co-Kleisli maps $h$ of the co-monad $L\circ R$).}
\begin{equation}\label{equation/non-associativity}
\hspace{.2em}
\raisebox{-6em}{\includegraphics[width=10.5em]{hg-f-indirect-style.pdf}}
\hspace{.8em}
\raisebox{-6em}{\includegraphics[width=10.5em]{h-gf-indirect-style.pdf}}
\end{equation}
\noindent This flow of control 
indicates that $(i)$ in the lefthand side diagram evaluates $f$ then $h$ then $g$,
while $(ii)$ in the righthand side diagram evaluates $h$ then $f$ then $g$.
These different evaluation orders reflect the behavior 
of the effectful expressions $(i)$ and $(ii)$ and the lack of associativity
described above for the polarities $(\varepsilon,\varepsilon') = (\oplus,\ominus)$.
We will introduce and justify the notation for the transverse maps of
the adjunction ($g$, above).

\subsection{Continuations, dialogue duploids, and classical notions of computation}
One fascinating aspect of duploids is that they exhibit
and preserve the perfect symmetry between the monadic and comonadic effects
of an adjunction, by treating on an equal footing the CBV and CBN evaluation policies.
Our main goal in the present paper is to explore how this symmetric account of effects
can benefit the long quest for a perfectly symmetric computational account of classical logic,
in the spirit and philosophy of the Curry-Howard correspondence.


\medbreak
\noindent
\textbf{Turning around Joyal obstruction theorem.}
The fact that duploids are non-associative categories is very meaningful from that point of view.
Indeed, Joyal made the well-known observation (recalled below, see \cref{thm/joyal})
that it is not possible to develop a proof-theoretic account of classical logic using
the language of usual (associative) cartesian categories.
\begin{definition}\label{definition/return-object}
An object $\bot$ is called a \emph{return object}
in a symmetric monoidal category~$(\Ccategory,\otimes,1)$
when it comes equipped with an object~$\bot^{A}$ and a family of bijections 
$$
\varphi_{A,B} \quad : \quad \Ccategory(A\tensor B,\bot) \quad \cong \quad \Ccategory(B,\bot^A)
$$
natural in~$B$, for every object~$A$ of the category~$\Ccategory$.
\end{definition}
\noindent
We may think of~$\bot^A$ as a negation of the object~$A$ and write it accordingly $\lnot A$.
A simple argument shows that every return object~$\bot$ induces 
a family of canonical maps
\begin{equation}\label{equation/A-implies-not-not-A}
\begin{tikzcd}[column sep = 1em]
{\eta_A} \quad : \quad A\arrow[rr] && \lnot\lnot A
\end{tikzcd}
\end{equation}
indexed by the objects~$A$ of the category~$\Ccategory$, 
which reflects the logical principle that every formula~$A$ implies its double negation~$\lnot\lnot A$.
A return object~$\bot$ is called \emph{dualizing} when the canonical map~\eqref{equation/A-implies-not-not-A}
is an isomorphism for every object~$A$.
A natural direction to resolve the quest for a proof-theoretic interpretation of classical logic
would be to look for a cartesian category~$(\Ccategory,\times,1)$ equipped with a dualizing object~$\bot$.
Unfortunately, Joyal observed that the search for such a simple solution cannot succeed:
\begin{theorem}[\joyallemma]\label{thm/joyal}
Any cartesian category $(\Ccategory,\times,1)$ with a dualizing object~$\bot$ is a preorder,
and thus defines a boolean algebra (up to equivalence).
\end{theorem}
\noindent
For a long time, this observation has been widely accepted 
as evidence that classical logic cannot be interpreted 
in a denotational and proof-relevant way.
The situation changed in the early 1990s
when Griffin~\cite{Griffin90aformulae-as-types} and
Murthy~\cite{Murthy91EvaluationSemantics} 
observed a fundamental and unexpected relationship
between proof systems for classical logic, 
and programs written with the control operator $\mathcal{C}$, 
a variant of Scheme's \emph{call-cc}.
Since then,
a large number of investigations have been made
to define a clean denotational and proof-theoretic
interpretation of classical logic.
Interestingly, each of the two main directions taken
can be seen as providing a specific way
to relax one of the hypothesis of Joyal's obstruction theorem:
\vspace{.3em}

\noindent
1)\, \emph{classical linear logic~\cite{Gir87}:}
the idea is to relax the cartesianity condition
and to work with $\ast$-autonomous categories,
defined as symmetric monoidal categories~$(\Ccategory,\tensor,1)$
equipped with a dualizing object~$\bot$,
possibly supplemented with an exponential modality~$A\mapsto{!A}$
to deal with non-linearity,


\vspace{.3em}

\noindent
2)\, \emph{continuation models:} the idea is to relax the dualizing condition,
and to work with cartesian categories~$(\Ccategory,\times,1)$ 
or symmetric monoidal categories~$(\Ccategory,\tensor,1)$
equipped with a return object~$\bot$ whose canonical maps~\eqref{equation/A-implies-not-not-A} are not necessarily required to be invertible.


\vspace{.3em}

\noindent
In these directions, influential and most notable works have been the
Lafont-Reus-Streicher translation~\cite{LRS93} as well as the later
works by Hofmann and Streicher~\cite{Hof02ComplLambdaMu} and by
Selinger~\cite{Sel01Control}.
Another important and early work has been the introduction of two dual
sequent calculi $\LKT$ and $\LKQ$ for classical logic, and their
translation in linear logic by Danos, Joinet and
Schellinx~\cite{danos93structure,DJS95LKQLKT,danos95new}, which turned
out to rephrase respectively the CBV and CBN continuation-passing
style (CPS) semantics~\cite{Ogata2000}. Interestingly, all these
models ``break the symmetry'' of classical logic by giving precedence
at some stage to the CBV or CBN side.
The symmetry between the two sides remains however, as a categorical
duality observed by Streicher and
Reus~\cite{StreicherReus98ContinuationAbstractMachines} and made
manifest by Selinger~\cite{Sel01Control} and Curien and
Herbelin~\cite{CH00Duality} (predated by, and in the spirit of,
Filinski's ``symmetric
$\lambda$-calculus''~\cite{filinski89declarative}). These works
explored in particular a syntactic symmetry between the CBN and CBV
calculi, which reflects the categorical duality.


\medbreak
\noindent
\textbf{A third direction: preserving the symmetries of classical logic at the expense of associativity.}
At about the same time, in the early 1990s,
an elegant and third direction was explored
by Girard with the classical logic $\LC$~\cite{girardnew}.
The goal was to preserve the symmetries of logic---in particular, an
involutive negation and various De Morgan identities present as type
isomorphisms---by giving a formal status to the notion of polarity of
a formula.
Girard's work on $\LC$ inspired many later
works~\cite{Murthy92LC,Quatrini96polarisationdes,Lau02PhD,Zeil2008Unity,Liang2009a,melliestabareau}
including in fact some of the works we already
mentioned~\cite{LRS93,DJS95LKQLKT,danos95new}.
However, the solution, which involves giving up the associativity of
composition precisely in the way which we have described, has not seen
much exploration from the angle of categorical proof theory. In fact,
the question of classical categorical proof theory is essentialy
mentioned as open in Hyland~\cite{Hyland2002}.
By continuing the duploid programme with classical logic in mind, we
aim to show that Girard's approach makes sense from both a syntactic
and a semantic point of view.


\medbreak
\noindent
\textbf{The self-adjunction of negation.}
At this stage, we find convenient and evocative to follow the terminology
used \ifbool{doubleblind}{by Melliès in his work}{in the second author's work}
on functorial game semantics~\cite{melliestabareau,Mellies12,melliesdialogue},
and to define a \emph{dialogue category} as a symmetric monoidal category~$(\Ccategory,\otimes,1)$
equipped with a return object~$\bot$ in the sense of \cref{definition/return-object}.
A well-known fact is that every dialogue category
comes equipped with a negation functor
$$
\begin{tikzcd}[column sep = 1em]
\lnot \quad : \quad \Ccategory\arrow[rr] && \Ccategory^{\op}
\end{tikzcd}
$$
defined as~$A\mapsto \lnot A := \bot^{A}$,
and that this negation functor defines an adjunction with itself:
\begin{equation}\label{equation/adjunction-neg-neg}
\begin{tikzcd}[column sep = 1em]
\Ccategory \arrow[rr,"{L\,=\,\neg}", bend left] & \bot & \Ccategory^{\mathrlap{\op}}\arrow[ll,"{R\,=\,\neg}",bend left]
\end{tikzcd}
\end{equation}
This observation, dating back to A.\@~Kock~\cite{Kock1970}, was given
emphasis in Thielecke's Ph.D. thesis on the structure of CPS
translations~\cite{Thielecke97Thesis}.


\medbreak
\noindent
We have seen that
the construction of the duploid~$\duploid{L}{R}$ associated to an adjunction~$L\dashv R$ 
amounts to building a direct computational interpretation combining 
and preserving the symmetries between the CBV and the CBN models.
Now, if we turn to the self-adjunction \eqref{equation/adjunction-neg-neg}
of the negation with itself in a dialogue category,
it appears that the duploid construction
coincides in fact with Girard's polarised translation for $\LC$
defined in~\cite{girardnew}\ifbool{doubleblind}{}{, which inspired duploids in the first place}.
In that sense, the duploid construction provides
in the case of dialogue categories a precise mathematical
and denotational counterpart to the multiplicative fragment
of the new form  of double-negation translation 
implemented by $\LC$
which contains the traditional CBV and the
CBN computational models as its \emph{positive} and \emph{negative subcategories}
respectively. 





\medbreak

\noindent
\textbf{Dialogue duploids.}
More generally, we believe that behind the superficial duality 
between CBV and CBN notions of control, 
there is more structure asking to be revealed on duploids 
associated to dialogue categories.
In order to uncover these structures, we start from the symmetric reformulation
(up to equivalence) of dialogue categories as \emph{dialogue chiralities} 
defined below:
\begin{definition}[\ifbool{doubleblind}{Melliès~}{}\cite{melliestabareau,Mellies12,melliesdialogue}]
A \emph{dialogue chirality} is a pair of symmetric monoidal categories
$(\Acategory,\tensorialand,\tensorialtrue)$ and $(\Bcategory,\tensorialor,\tensorialfalse)$
equipped with an adjunction~${L:\Acategory\rightleftarrows\Bcategory:R}$ as depicted in~\eqref{equation/adjunction-LR}
together with a symmetric monoidal equivalence:
\vspace{-1em}
\begin{equation}\label{equation/chiral-duality}
\begin{tikzcd}[column sep = 1em]
  (\Acategory,\tensorialand,\tensorialtrue) \arrow[rr,"\chirdual{(-)}", bend left] & \simeq &
  (\Bcategory,\tensorialor,\tensorialfalse)^{\mathrlap{\op}}\arrow[ll,"\chirdual{(-)}",bend left]
\end{tikzcd}
\end{equation}
and a family of bijections (called currifications)
\[
  \chi_{A_1,A_2,B} : \Acategory(A_1 \tensorialand A_2, RB)
  \longrightarrow \Acategory(A_1, R(\chirdual{A_2}\tensorialor B)) 
\]
natural in $A_1$, $A_2$ and $B$ and satisfying a coherence diagram.
\end{definition}
\noindent
In order to understand the specific nature of duploids associated
to dialogue chiralities, we will develop a general theory 
of duploids equipped with different forms of monoidal structures,
in link with classical logic and linear as well as non linear continuations.
In particular, we will define the notion of \emph{dialogue duploid}
which describes the structure of a duploid associated to a dialogue
chirality. In doing so, we will make explicit the structure of $\LC$'s
involutive negation with a connective materializing the
equivalence~\eqref{equation/chiral-duality}.
















\medbreak

\noindent
\textbf{The classical $L$-calculus.}
One final ingredient to our correspondence concerns
abstract-machine-like term calculi, or $L$-calculi. They are
$\lambda$-calculi (higher-order rewriting systems) which just so
happen to represent derivations of sequent calculus, and subsume the
rich relationship between CPS, abstract machines, proof search
(focusing), etc.
These calculi reflect categorical duality as a symmetry between player
and opponent, between expression and evaluation context. They were
discovered by Curien and Herbelin~\cite{CH00Duality,Herb05} through
the reunion of two research lines---the one we just mentioned after
Girard around the connection between constructive classical logic and
CPS~\cite{DJS95LKQLKT,danos95new,Ogata2000}, and one that investigated
well-behaved $\lambda$-calculi for classical logic
(Parigot~\cite{Parigot92}) and sequent calculus
(Herbelin~\cite{Herbelin1994}).





\medbreak
\noindent
\textbf{The \FH{} theorem} Replacing these logical considerations into
the context of computation, we are able to cast in a new light a
fundamental result about continuations.

It is natural to ask when an expression in an effectful language is
pure. One possible definition is that it can be substituted like a
value, a notion called \emph{algebraic value} or
\emph{thunkable}~\cite{Thielecke97Thesis} expression. In duploids,
thunkability for a map $f$ is characterised as associativity of
composition (quantifying over all
$g,h$)~\cite{munchduploids,munchlcalculi}:\[
\def\letinspace{1mu}
\letinplus{a}{f}{\!\letinminus{b}{g}{h}}\hspace{0.5em}=\hspace{0.5em}\letinminus{b}{(\letinplus{a}{f}{g})}{h}
\]
or in our sequent calculus:\vspace*{\medskipamount}

\newcommand{\stackhypo}[1]{\hypo{&#1}\infer[no rule]1}
\begin{adjustbox}{width=\columnwidth}
\begin{prooftree}[separation=1ex,rule margin=0.5ex]
\stackhypo{\scriptstyle{f}}{\scriptstyle{ \Gamma'\!' } & \scriptstyle{\vdash P, \Delta\!'\!'} }
\stackhypo{\scriptstyle{g}}{\scriptstyle{ \Gamma,P } & \scriptstyle{\vdash N, \Delta} }
\stackhypo{\scriptstyle{h}}{\scriptstyle{ \Gamma', N } & \scriptstyle{\vdash \Delta\!'} }
\infer2{\scriptstyle{ \Gamma,\Gamma'\!,P } & \scriptstyle{\vdash \Delta,\Delta\!'} }
\infer2[$\,\,\,\,\scriptstyle{=}$]{ \scriptstyle{\Gamma,\Gamma'\!,\Gamma'\!'} & \scriptstyle{\vdash \Delta,\Delta\!'\!,\Delta\!'\!'} }
\end{prooftree}\,\begin{prooftree}[separation=1ex,rule margin=0.5ex]
\stackhypo{\scriptstyle{f}}{\scriptstyle{ \Gamma'\!'} & \scriptstyle{\vdash P, \Delta\!'\!'}  }
\stackhypo{\scriptstyle{g}}{\scriptstyle{ \Gamma,P} & \scriptstyle{\vdash N, \Delta} }
\infer2{\scriptstyle{ \Gamma,\Gamma'\!'} & \scriptstyle{\vdash N, \Delta,\Delta\!'\!'} }
\stackhypo{\scriptstyle{h}}{\scriptstyle{ \Gamma'\!,N} & \scriptstyle{\vdash \Delta\!'} }
\infer2{\scriptstyle{ \Gamma,\Gamma'\!,\Gamma'\!'} & \scriptstyle{\vdash \Delta,\Delta\!'\!,\Delta\!'\!'} }
\end{prooftree}
\end{adjustbox}\vspace*{\bigskipamount}

\noindent A weaker concept of purity,
\emph{centrality}~\cite{PowerRobinson97}, captures the idea of
irrelevance of order of evaluation with a property of commutation
(again for all $g,h$):\[
\def\letinspace{2mu}
\letinplus{a}{f}{\!\letinplus{b}{g}{h}}\hspace{0.5em}=\hspace{0.5em}\letinplus{b}{g}{\!\letinplus{a}{f}{h}}
\]
or in our sequent calculus:\vspace*{\medskipamount}

\begin{adjustbox}{width=\columnwidth}
\begin{prooftree}[separation=1ex,rule margin=0.5ex]
\stackhypo{\scriptstyle{f}}{\scriptstyle{ \Gamma'\!' } & \scriptstyle{\vdash P, \Delta\!'\!'} }
\stackhypo{\scriptstyle{g}}{\scriptstyle{ \Gamma' } & \scriptstyle{\vdash Q, \Delta\!'} }
\stackhypo{\scriptstyle{h}}{\scriptstyle{ \Gamma,P,Q } & \scriptstyle{\vdash \Delta} }
\infer2{\scriptstyle{ \Gamma,\Gamma'\!,P } & \scriptstyle{\vdash \Delta,\Delta\!'} }
\infer2[$\,\,\,\,\scriptstyle{=}$]{ \scriptstyle{\Gamma,\Gamma'\!,\Gamma'\!'} & \scriptstyle{\vdash \Delta,\Delta\!'\!,\Delta\!'\!'} }
\end{prooftree}\,\begin{prooftree}[separation=1ex,rule margin=0.5ex]
\stackhypo{\scriptstyle{f}}{\scriptstyle{ \Gamma'\!'} & \scriptstyle{\vdash P, \Delta\!'\!'}  }
\stackhypo{\scriptstyle{g}}{\scriptstyle{ \Gamma,P} & \scriptstyle{\vdash N, \Delta} }
\infer2{\scriptstyle{ \Gamma,\Gamma'\!'} & \scriptstyle{\vdash N, \Delta,\Delta\!'\!'} }
\stackhypo{\scriptstyle{h}}{\scriptstyle{ \Gamma'\!,N} & \scriptstyle{\vdash \Delta\!'} }
\infer2{\scriptstyle{ \Gamma,\Gamma'\!,\Gamma'\!'} & \scriptstyle{\vdash \Delta,\Delta\!'\!,\Delta\!'\!'} }
\end{prooftree}
\end{adjustbox}\vspace*{\bigskipamount}

Strikingly, these two instances of commutations are the same up to
duality in the sequent calculus. Now, for the classical notions of
computation we are considering, another ingredient makes them actually
coincide: the presence of a negation connective which is involutive at
the level of proof denotation, whose rules in sequent calculus provide
a way to exchange between the left-hand and right-hand sides without
loss of information. We formalize this idea with our proof of
\Cref{theorem/fh-syntactic}.
It follows that in any symmetric monoidal category with a return object
(i.e. a dialogue category),
\emph{a map is thunkable if
and only if it is central}. In particular, \emph{the double-negation
monad is commutative if and only if it is idempotent}.


This property was noticed by Thielecke~\cite{Thielecke97Thesis} in the
context of categorical semantics for continuations, in which it plays
an important role~\cite{Thielecke97Thesis,Sel01Control,Hasegawa_2002}.
The essential status of thunkability as a concept distinct from
centrality became apparent in the works of Führmann on the direct
axiomatic theory of monadic effects~\cite{Fuhrmann2000PhD,fuhrmanndirectmodels}.
The refinement of this property
from the cartesian to the symmetric monoidal setting was suggested by
Hasegawa and played a key role in \ifbool{doubleblind}{Melliès's}{the
second author's} analysis of the Blass problem in game semantics as
a non-commutativity of the double-negation
monad~\cite{Mellies2005ag3}. We are not aware of a published proof of
this result in the symmetric monoidal case.



















\subsection{Summary and main contributions}
After this long but necessary introduction,
we provide in~\S\ref{section/duploids-as-bikleisli} more details
on the bi-Kleisli construction which turns every adjunction~$L\dashv R$
into a non-associative category~$\duploid{L}{R}$.
We then recall in~\S\ref{section/duploids} and~\S\ref{section/sm-freyd-categories}
the notions of duploid and of symmetric monoidal Freyd category.
We then start our journey towards the classical $L$-calculus 
by introducing in~\S\ref{section/sm-duploids} the notion of symmetric monoidal duploid,
followed in~\S\ref{section/dialogue-duploids} by the notion of dialogue duploid.
At this stage, we introduce in~\S\ref{section/classical-L-calculus}
the syntax of the classical $L$-calculus and establish a soundness theorem
of the interpretation of the $L$-calculus in any dialogue duploid.
Building on this result, we illustrate the relevance and robustness 
of our approach by defining the syntactic dialogue duploid in \S\ref{section/syntactic-duploid}
and by proving in \S\ref{section/fh-theorem} the \FH{} theorem 
using both semantic and syntactic methods.
We then conclude and give directions for future work in \S\ref{section/conclusion}.






\section{A non-associative bi-Kleisli construction}\label{section/duploids-as-bikleisli}
As discussed in the introduction, we want to see the construction
of the duploid associated to an adjunction $L\dashv R$ in~\cite{munchduploids}
as an instance of a bi-Kleisli construction
of a non-associative category~$\bikleisli{\Ecategory}{\UpMonad}{\DownComonad}$
on the collage category~$\Ecategory=\collage{L}{R}$ of the adjunction.
A preliminary observation in that direction is that a map~$f:A\to A'$ in the Kleisli category~$\kleisli{\Acategory}{T}$
for $T=RL$ can be equivalently seen as a map $LA\to LA'$ in the category $\Bcategory$,
along the back-and-forth translation:
$$
\begin{array}{lccc}
(a) & \begin{tikzcd}[column sep=.8em]
A\arrow[rr,"f"] && RLA'
\end{tikzcd}
& \hspace{-.5em} \mapsto \hspace{-.5em} &
\begin{tikzcd}[column sep=1em]
LA\arrow[rr,"Lf"] && LRLA' \arrow[rr,"\varepsilon_{LA'}"] && LA'
\end{tikzcd}
\\
(b) & \begin{tikzcd}[column sep=.8em]
LA\arrow[rr,"f"] && LA'
\end{tikzcd}
& \hspace{-.5em} \mapsto \hspace{-.5em} &
\begin{tikzcd}[column sep=1em]
A\arrow[rr,"\eta_A"] && RLA\arrow[rr,"Rf"] && RLA'
\end{tikzcd}
\end{array}
$$
The translation $(a)$ depicted in the language of string diagrams
amounts to ``bending'' the functor~$R$ 
into the functor~$L$ using the counit~$\varepsilon$ of the adjunction:
\begin{center}
\begin{tabular}{ccc}
\includegraphics[width=9em]{kleisli-before.pdf}
& 
\raisebox{4em}{$\mapsto$}
&
\includegraphics[width=9em]{kleisli-after.pdf}
\end{tabular}
\end{center}
One benefit of this alternative description of maps $f:A\to A'$ and $g:A'\to A''$ 
in the Kleisli category~$\kleisli{\Acategory}{T}$ as maps $f:LA\to LA'$ and $g:LA'\to LA''$
in the category~$\Bcategory$
is that the composite $g\pcomp f:A\to A''$ computed in the Kleisli category
happens to coincide with the composite $g\circ f:LA\to LA'\to LA''$ 
computed in the original category~$\Bcategory$, as depicted below:
\begin{center}
\begin{tabular}{ccc}
\includegraphics[width=9.5em]{kleisli-bullet-after.pdf}
& 
\raisebox{3em}{$=$}
&
\includegraphics[width=9.5em]{kleisli-g-circ-f-before.pdf}
\end{tabular}
\end{center} 


\medbreak
\noindent
Symmetrically, a map~$f:B\to B'$ in the co-Kleisli category~$\cokleisli{\Bcategory}{K}$
for $K=LR$ can be seen as a map $RB\to RB'$ in the category $\Acategory$.
Moreover, for all pairs of objects $A$ in~$\Acategory$ and $B$ in $\Bcategory$, we have the bijections
$$
\Acategory(A,RB) \quad \cong \quad \Ecategory(A,B)  \quad \cong \quad \Bcategory(LA,B)
$$
From this follows that every map~$g:A\to B$ in
$\Ecategory(A,B)$ above $\median:0\to 1$ can be equivalently 
seen as a map $A\to RB$ in~$\Acategory=\Ecategory_0$ and as a map $LA\to B$
in~$\Bcategory=\Ecategory_1$.
We find convenient to use the following notations
to depict these different ``incarnations'' of the transverse map $g:A\to B$
in the language of string diagrams:
\begin{center}
\begin{tabular}{c}
\includegraphics[width=6.5em]{transverse-map-R.pdf}
\,\raisebox{2em}{$\leftrightharpoons$}\,
\includegraphics[width=6.5em]{transverse-map-med.pdf}
\,\raisebox{2em}{$\leftrightharpoons$}\,
\includegraphics[width=6.5em]{transverse-map-L.pdf}
\end{tabular}
\end{center}
\noindent
The distributivity law~\eqref{equation/distributivity-law} 
for a positive object~$X=A$ and a negative object~$X=B$
is defined as the transverse map
$$
\begin{tikzcd}[column sep = 1em, row sep = 1.2em]
{\lambda_A} \,\, : \,\, RLA \arrow[rr] && LA
\end{tikzcd}
\hspace{1.5em}
\begin{tikzcd}[column sep = 1em, row sep = 1.2em]
{\lambda_B} \,\, : \,\, RB \arrow[rr] && LRB
\end{tikzcd}
$$
associated by the adjunction $L\dashv R$
to the identity maps $RLA\to RLA$ and $LRB\to LRB$, respectively.
The two maps $\lambda_A$ and $\lambda_B$
are depicted in string diagrams as follows:
\begin{center}
\begin{small}
\begin{tabular}{ccc}
\raisebox{2.2em}{$\lambda_A\,=\,$} 
\includegraphics[width=9em]{lambda-A.pdf}
&  \raisebox{2.2em}{$\lambda_B\,=\,$} 
\includegraphics[width=9em]{lambda-B.pdf}
\end{tabular}
\end{small}
\end{center}
Interestingly, the distributivity law satisfies all the properties of
a usual distributivity law between a monad and a comonad, as defined
in Power and Watanabe~\cite{Power2002}, except that it is not natural in general.
Indeed, given a map $f:A\to B$ from a positive object~$A$
to a negative object~$B$ in the collage category~$\Ecategory=\collage{L}{R}$,
an easy computation shows that the naturality diagram below does not commute in general:
$$
\begin{tikzcd}[column sep = 1.5em, row sep = 1em]
RLA\arrow[dd,"\DownComonad\UpMonad f"{swap}]\arrow[rr,"{\lambda_A}"]
&&
LA\arrow[dd,"\UpMonad\DownComonad f"]
\\
\\
RB\arrow[rr,"{\lambda_B}"] && LRB
\end{tikzcd}
$$
and that the two maps
$\UpMonad\DownComonad f \bullet \lambda_A$
and $\lambda_B\circ\DownComonad\UpMonad f$ 
have respective descriptions
in the language of string diagrams 
just introduced for the adjunction $L\dashv R$:
\medbreak
\begin{center}
\begin{tabular}{ccc}
\includegraphics[width=9.2em]{RLA-LRB-fR.pdf}
& \quad &
\includegraphics[width=9.2em]{RLA-LRB-fL.pdf}
\end{tabular}
\end{center}
This lack of naturality of the family of maps~$\lambda_{X}$ is the reason why
the bi-Kleisli construction defines a non-associative category in
general, as indicated in~\eqref{equation/non-associativity}.


In order to understand better why the two maps $(h\ncomp g)\pcomp f$ and
$h\ncomp (g\pcomp f)$ do not coincide in~\eqref{equation/non-associativity},
it is worth observing that given a transverse map $g:A\to B$ in
the bi-Kleisli construction,
the result~$g\pcomp f$ of compositing $g$ with a map $f:A'\to A$
of the Kleisli category~$\kleisli{\Acategory}{RL}$ and the result~$h\ncomp g$ 
of composing~$g$ with a map $h:B\to B'$ 
of the co-Kleisli category~$\cokleisli{\Bcategory}{LR}$ are depicted as follows:
\begin{center}
\begin{tabular}{c}
\includegraphics[width=8.5em]{bikleisli-g-circ-f.pdf}
\hspace{2em}
\includegraphics[width=8.5em]{bikleisli-h-bullet-g.pdf}
\end{tabular}
\end{center}
The flow of control described by the adjoint functors~$R$ and~$L$
clearly indicates that the Kleisli map~$f$ is executed before~$g$ in the composite~$g\bullet f$
and that, symmetrically, the co-Kleisli map~$h$ is executed before~$g$ in the composite~$h\circ g$.
The controntation of the call-by-value $g\bullet f$ and call-by-name $h\circ g$
policies of composition explains the non-associativity phenomenon observed in~\eqref{equation/non-associativity}.

\section{Duploids}\label{section/duploids}
We have observed in the introduction and in~\S\ref{section/duploids-as-bikleisli}
that composition of effectful programs $g,f\mapsto g\ccomp f$ is not associative in general
when one wants to make positive and negative types coexist in the same overarching mathematical structure.
That observation justifies to study and characterize the class 
of ``non-associative categories''  of the form~$\duploid{L}{R}$
associated to an adjunction~$L\dashv R$ in the way explained above.
This is precisely the purpose of the notion of \emph{duploid} \cite{munchduploids} 
which we find convenient to recall in this section.




\begin{definition}
  A \define{magmoid} $\mathcal M$ is defined as a graph with set of objects~$|\mathcal M|$
  equipped with a composition law
  $$
\begin{tikzcd}[row sep=-.3em,column sep=1em]
\ccomp_{A,B,C} \quad : \quad \mathcal M(B, C) \times  \mathcal M(A, B) \arrow[rr] && \mathcal M(A, C)
\end{tikzcd}
$$
which associates to every pair of maps $f:A\to B$ and $g:B\to C$ a composite map $g\ccomp f:A\to C$.
A \define{unital magmoid} or  \define{non-associative category} is a magmoid $\mathcal M$ equipped 
with a map $\mathsf{id}_A$ for all objects $A \in |\mathcal M|$ such that:
  \[
    f \ccomp \mathsf{id}_A = f = \mathsf{id}_A \ccomp f.
  \]
\end{definition}



\begin{definition}
  For all magmoids $\mathcal M$, we define $\mathcal M^{\mathsf{op}}$ to be the magmoid with the same objects as $\mathcal M$ but whose maps are reversed.
\end{definition}

\begin{definition}
  In a magmoid $\mathcal M$, a map $h$ is said to be \define{linear} if, for all maps $f$ and $g$, one has:
  \[
    (h \ccomp g) \ccomp f = h \ccomp (g \ccomp f)
  \]
  Symmetrically, a map $f$ of $\mathcal M$ is said to be \define{thunkable} if, for all maps $g$ and $h$, one has:
  \[
    (h \ccomp g) \ccomp f = h \ccomp (g \ccomp f)
  \]
\end{definition}

One useful observation is that it is possible to derive the polarity of an object~$A$ 
in a non-associative category~$\mathcal M$ just by observing the way maps associates.

\begin{definition}[Polarity]\label{definition/polarity}
An object~$A \in |\mathcal M|$ is called \define{positive} when, for all $B \in |\mathcal M|$, 
all maps of $\mathcal M(A,B)$ are linear.\\
Symmetrically, an object $B$ of $\mathcal M$ is \define{negative} when, 
for all $A \in |\mathcal M|$, all maps of $\mathcal M(A,B)$ are thunkable.
\end{definition}

\noindent
Note that an object~$A$ may be both positive and negative:
this is the case in particular for every object~$A$ of an usual (associative) category.
Note also that, if a map $f$ is linear in the magmoid $\mathcal M$, then $f$ is thunkable
in the opposite magmoid~$\mathcal M^\op$, and conversely. 
From this follows that $(-)^\op$ reverses the polarities.

Given $f \in \mathcal M(A, B)$ and $g \in \mathcal M(B,C)$, we find convenient to write~$g\ccomp f$
as $g \pcomp f$ when $B$ is positive and as $g \ncomp f$ when $B$ is negative.



\begin{definition}
  A \define{positive shift} on a unital magmoid $\mathcal M$ consists 
  of the data for every object~$A$ of a positive object~${\nDownarrow A}$ equipped 
  with a pair of thunkable maps
  $$
  \omega_A :A \to \nDownarrow A
  \quad\quad
  \overline{\omega}_A : \nDownarrow A \to A
  $$
such that $\overline{\omega}_A\ccomp \omega_A=\id{A}$ and $\omega_A\ccomp\overline{\omega}_A=\id{\nDownarrow A}$.
Dually, a \define{negative shift} $\nUparrow$ is a positive shift on $\mathcal M^{op}$.
\end{definition}

A nice and instructive exercise in non-associative categories
is to show that positive shifts are unique up to thunkable and linear isomorphisms, 
and similarly (by duality) for negative shifts.
We are ready now to give a slight variant of the original definition
of duploid formulated by \ifbool{doubleblind}{Munch-Maccagnoni}{the
  second author}~\cite{munchduploids} taken from more recent
\ifbool{doubleblind}{unpublished}{ongoing} work~\cite{newduploids}.
\begin{definition}
A \define{duploid} is a non-associative category equipped with a positive and a negative shift,
  and where every object is either positive or negative (or both).
\end{definition}

\begin{proposition}
  For $\Dduploid$ a duploid, $\Dduploid^\op$ is also a duploid.
\end{proposition}

\noindent
Given a duploid~$\Dduploid$, we find convenient to introduce below
notations for usual (associative) subcategories of $\Dduploid$:
\begin{itemize}
    \item $\Dduploid_l$ is the subcategory of linear maps,
    \item $\Dduploid_t$ is the subcategory of thunkable maps,
    \item $\Pcategory$ is the full subcategory of positive objects,
    \item $\Ncategory$ is the full subcategory of negative objects,
    \item $\Pcategoryt$ is the subcategory of thunkable maps of $\mathcal P$,
    \item $\Ncategoryl$ is the subcategory of linear maps of $\mathcal N$.
\end{itemize}
\medbreak
\noindent
The notion of duploid is justified in~\cite{munchduploids}
by the following characterization result:
\begin{theorem}[\ifbool{doubleblind}{Munch-Maccagnoni~}{}\cite{munchduploids,newduploids}]\label{theorem/adjunctions-duploids}
Every non-associative category~$\duploid{L}{R}$ associated to an
adjunction $L\dashv R$
comes equipped with a duploid structure, where $\Pcategory$ is
equivalent to the Kleisli category on the monad $T=R\circ L$, and
$\Ncategory$ is equivalent to the co-Kleisli category on the comonad
$K=L\circ R$. Moreover, $\duploid{L}{R}$ is associative if and only if
the monad, or equivalently the comonad, is idempotent.
Conversely, every duploid~$\Dduploid$ induces an
adjunction
\begin{equation}\label{equation/adjunction-PN}
\begin{tikzcd}[column sep = 1em]
{\Pcategoryt} \arrow[rr,, bend left] & \bot & {\Ncategoryl}\arrow[ll,,bend left]
\end{tikzcd}
\end{equation}
defined by restriction of the shifts, whose associated duploid is equivalent to $\Dduploid$.
\end{theorem}



\section{Symmetric monoidal Freyd categories}\label{section/sm-freyd-categories}
We have just seen (\cref{theorem/adjunctions-duploids}) how the notion 
of duploid introduced in~\cite{munchduploids} enables one to characterize 
the non-associative categories
associated to an adjunction $L\dashv R$.
Now, we want to describe in this section and in the next one~\S\ref{section/sm-duploids}
the structures inherited by a duploid~$\duploid{L}{R}$ 
associated to an adjunction~$L\dashv R$ of the form~\eqref{equation/adjunction-LR}
where the category~$\Acategory$ is equipped
with a symmetric monoidal structure $(\Acategory,\tensorialand,\tensorialtrue)$
and where the monad $T=R\circ L$ is equipped with a pair of 
left and right strengths related by symmetry:
$$
\begin{array}{ccc}
rstr_{A_1,A_2} & \quad : \quad & TA_1\,\tensorialand A_2 \longrightarrow T(A_1\tensorialand A_2)
\\
lstr_{A_1,A_2} & \quad : \quad & A_1\tensorialand \,TA_2 \longrightarrow T(A_1\tensorialand A_2)
\end{array}
$$
In that case, the Kleisli category~$\kleisli{\Acategory}{T}$ comes
equipped with a premonoidal structure compatible 
with the original tensor product.
The tensor product $f\ltimes A_2$ of a Kleisli map $f:A_1\to TA_1'$ and an object~$A_2$ is defined as
$$
\begin{tikzcd}[column sep = .85em]
\hspace{-.2em}f\ltimes A_2 \, : \, A_1\tensorialand A_2 \arrow[rrr,"f\tensorialand A_2"] &&& TA_1'\tensorialand A_2 \arrow[rrr,"rstr"] &&& T (A_1'\tensorialand A_2)
\end{tikzcd}
$$
and symmetrically, the tensor product of an object~$A_1$ and a Kleisli map $g:A_2\to TA_2'$ is defined as
$$
\begin{tikzcd}[column sep = .85em]
A_1\rtimes g \, : \, A_1\tensorialand A_2 \arrow[rrr,"A_1\tensorialand g"] &&& A\tensorialand TA_2' \arrow[rrr,"lstr"] &&& T (A_1\tensorialand A_2')
\end{tikzcd}
$$
The compatibility between the monoidal structure on~$\Acategory$
and the premonoidal structure on~$\kleisli{\Acategory}{T}$
is witnessed by the fact that the identity-on-object functor
$\iota : \Acategory\to \kleisli{\Acategory}{T}$
transports (strictly) the symmetric monoidal 
structure of~$\Acategory$ to the symmetric
premonoidal structure of~$\kleisli{\Acategory}{T}$.
Recall that given two maps $f:A_1\to A_1'$ and $g:A_2\to A_2'$
in a premonoidal category~$\Pcategory$,
the diagram below does not necessarily commute:
\begin{equation}\label{equation/premonoidal-square}
\begin{tikzcd}[column sep = 1.8em,row sep = .8em]
A_1\tensor A_2
\arrow[rr,"f\ltimes A_2"]\arrow[dd,"A_1\rtimes g"{swap}]
&&
A_1'\tensor A_2
\arrow[dd,"A_1'\rtimes g"]
\\
\\
A_1\tensor A_2'\arrow[rr,"f\ltimes A_2'"]
&&
A_1'\tensor A_2'
\end{tikzcd}
\end{equation}
where we use $f\ltimes A_2$ and $A_1\rtimes g$
as more explicit notations for $f\tensor A_2$ and $A_1\tensor g$, respectively.
We say that $f$ is orthogonal to $g$ when the
diagram~\eqref{equation/premonoidal-square} does commute.
A map~$f$ is called \emph{central} when it is orthogonal to all maps~$g$.
One shows that the functor $\iota$ transports every morphism in~$\Acategory$
into a central morphism in~$\kleisli{\Acategory}{T}$.
This structure has been recognised as important in the semantics on effects
and has been intensively studied under the name of symmetric monoidal Freyd category~\cite{PowerRobinson97,Staton14}.
\begin{definition}
A \define{symmetric monoidal Freyd category} 
is an identity-on-object functor
\begin{equation}\label{equation/Freyd-category-functor}
\begin{tikzcd}[column sep = 1.5em]
\iota \quad : \quad \Mcategory\arrow[rr] && \Pcategory
\end{tikzcd}
\end{equation}
between a symmetric monoidal category~$(\Mcategory,\tensor,1)$
and a symmetric premonoidal category~$\Pcategory$
which transports (strictly) the symmetric monoidal structure of~$\Mcategory$
to the symmetric premonoidal structure of~$\Pcategory$,
and such that every morphism~$\iota(f):A\to A'$ in $\Pcategory$
coming from a morphism~$f:A\to A'$ in $\Mcategory$
is central in $\Pcategory$.
\end{definition}

\section{Symmetric monoidal duploids}\label{section/sm-duploids}
We have seen at the end of~\S\ref{section/duploids} (\cref{equation/adjunction-PN})
that in the reconstruction of a given duploid~$\mathcal D$, the category~$\Pcategoryt$ of positive objects and thunkable morphisms
plays the role of the category~$\Acategory$, while the category of positive objects~$\Pcategory$
plays the role of the Kleisli category~$\kleisli{\Acategory}{T}$.
This leads us to the definition:
\begin{definition}\label{definition/sm-duploid}
A (positive) \define{symmetric monoidal duploid}~$(\Dduploid,\tensor,1)$
is a duploid whose inclusion functor
$\Pcategoryt\inclusion\Pcategory$ is equipped with the structure of
a symmetric monoidal Freyd category $(\Pcategoryt,\tensor,1)\to(\Pcategory,\tensor,1)$.
\end{definition}
The asynchronous product ${\mathscr G}\boxtimes {\mathscr H}$ of two reflexive graphs~${\mathscr G}$ and ${\mathscr H}$
is defined as the reflexive graph whose objects 
are pairs $(X,Y)$ of objects~$X$ of ${\mathscr G}$
and $Y$ of ${\mathscr H}$ and whose maps are of the form
$$
(f,Y) : (X,Y)\to (X',Y)
\quad 
(X,g) : (X,Y)\to (X,Y')
$$
with the maps $(\id{X},Y)$ and $(X,\id{Y})$ identified
and defining the identity map $\id{(X,Y)}$ of the object~$(X,Y)$.
A binoidal graph~${\mathscr G}$ is defined as a reflexive graph 
equipped with a reflexive graph homomorphism
\[
\begin{tikzcd}
\otimes \quad : \quad {\mathscr G}\boxtimes{\mathscr G} \arrow[rr] &&  {\mathscr G}
\end{tikzcd}
\]
We write $f\ltimes Y:X\tensor Y\to X'\tensor Y$ and
$X\rtimes g:X\tensor Y\to X\tensor Y'$ the image of $(f,Y)$ and
$(X,g)$, respectively.


One important observation is that every symmetric monoidal duploid
in the sense of \cref{definition/sm-duploid} comes equipped
with a binoidal structure on positive as well as negative objects.
In order to explain the construction, we find convenient to write 
$A_1\positivetensor A_2$ for the tensor product of two positive objects~$A_1$ and~$A_2$
of the symmetric monoidal category~$\Pcategoryt$ of positive objects 
and thunkable maps.
The tensor product is extended to every pair of objects $X$ and $Y$ as 
the tensor product of their positive shifts:
\begin{equation}\label{equation/definition-of-tensor}
X\tensor Y \quad := \quad \nDownarrow X \positivetensor \nDownarrow Y
\end{equation}
Accordingly, given a map $f:X\to X'$ and an object~$Y$, we define
$
f\ltensortimes Y \, = \, \nDownarrow f \ltensortimes \nDownarrow Y
$
and symmetrically, given an object $X$ and a map $g:Y\to Y'$,
we define
$
X\rtensortimes g \, = \, \nDownarrow X \rtensortimes \nDownarrow g
$
where we write $\rtensortimes$ and $\ltensortimes$ the premonoidal
structure between positive objects in~$\Pcategoryt$. 


One side consequence of the definition is that shifting positively
coincides in the monoidal duploid $(\mathcal D,\tensor,1)$ 
with the operation of tensoring with the unit~$1$,
up to a thunkable and linear isomorphism, what can be written:
$$
\nDownarrow X \quad \cong \quad X\tensor 1.
$$
An important point to stress is that although the tensor products
$X\tensor -$ and $-\tensor Y$ are functorial in the premonoidal 
category~$\Pcategory$ of positive objects,
it is not true in general when one considers the duploid~$\Dduploid$ itself,
in the sense that the functoriality diagram below does not commute in general:
\begin{equation}\label{equation/non-functoriality-of-tensor}
\begin{tikzcd}[column sep = 1.2em, row sep = 1.2em]
&&
X'\tensor Y
\arrow[rrdd,"f'\ltensortimes Y"]
\\
\\
X\tensor Y \arrow[rruu,"f\ltensortimes Y"] \arrow[rrrr,"(f'\ccomp f)\ltensortimes Y"] &&&& X''\tensor Y
\end{tikzcd}
\end{equation}
and similarly for $X\tensor -$.
This lack of functoriality of $X\tensor -$ and $-\tensor Y$
is the reason why we need to extend the notion 
of premonoidal category to the (more general)
notion of binoidal reflexive graph.
One nice consequence of the existence of positive shifts
in the definition of duploids is that one can easily characterize
when the functoriality diagram commutes:
\begin{proposition}
The diagram~\eqref{equation/non-functoriality-of-tensor}
commutes precisely when the triple below associates
$$
\begin{tikzcd}[column sep = 1.2em]
X\arrow[rr,"f"] && X'\arrow[rr,"f'"] && X''\arrow[rr,"\omega_{X''}"] && \nDownarrow X''
\end{tikzcd}
$$
in the sense that $\omega_{X''}\ccomp(f'\ccomp f)=(\omega_{X'}\ccomp f')\ccomp f$.
\end{proposition}
The notion of symmetric monoidal duploid is justified by the following theorem:
\begin{theorem}\label{theorem/adjunctions-monoidal-duploids}
Every non-associative category~$\duploid{L}{R}$ 
associated to an adjunction $L\dashv R$
where $\Acategory$ is symmetric monoidal
and the monad~$T=R\circ L$ has a left and right strength
comes equipped with a symmetric monoidal duploid structure.
Conversely, every symmetric monoidal duploid~$\Dduploid$ 
induces an adjunction~\eqref{equation/adjunction-PN}
where $\Pcategoryt$ is equipped with a symmetric monoidal structure
$(\Pcategoryt,\ostar,1)$ and the associated monad on $\Pcategoryt$ has
a left and right strength.
\end{theorem}
\noindent
In preparation for the \FH{} theorem 
in~\S\ref{section/fh-theorem},
we establish that every symmetric monoidal duploid~$(\Dduploid,\otimes,1)$
satisfies the cardinal property:
\begin{proposition}\label{lemma/thunkableimpliescentral}
Every thunkable map is central.
\end{proposition}
\noindent
The converse property is not true in general: 
consider for instance the symmetric monoidal duploid~$(\Dduploid,\otimes,1)$
associated to the finite probability monad~$T:\Set\to\Set$
which maps every set~$A$ to the set~$TA$ of its finite probability distributions.
The monad~$T$ is commutative, and every map in~$\Dduploid$ is thus central.
On the other hand, the two expressions $(i)$ and $(ii)$
discussed in the introduction are not necessarily equal in the case $(\varepsilon,\varepsilon')=(\oplus,\ominus)$
because of possible duplications of the variable~$b$ in the expression~$h$.
From this follows that every map is not thunkable in the duploid~$\Dduploid$.











\section{Dialogue duploids}\label{section/dialogue-duploids}
In this section, we will describe the structure inherited 
by a duploid associated to a dialogue chirality~\cite{melliesdialogue}
as we described it in the introduction.
Before introducing the notion of dialogue duploid,
we find convenient to define a notion of strong monoidal functor 
between symmetric monoidal duploids:
\begin{definition}
A strong monoidal functor 
\begin{center}
\begin{tikzcd}
F \quad : \quad (\Dduploid,\otimes,\1) \arrow[rr] &&  (\Eduploid,\otimes,\1)
\end{tikzcd}
\end{center}
between symmetric monoidal duploids consists of a function $F:{|\Dduploid|}\longrightarrow{|\Eduploid|}$
which preserves polarities of objects, together with a family of functions
\begin{center}
\begin{tikzcd}[column sep = 1.5em]
{F_{X,Y}} \quad : \quad {\Dduploid(X,Y)} \arrow[rr] && {\Eduploid(FX,FY)}
\end{tikzcd}
\end{center}
which preserves compositions and identities as well as linearity and thunkability.
One requires moreover that $F$ is equipped with a family of thunkable and linear isomorphisms
\begin{center}
\begin{tikzcd}[column sep = 1.5em, row sep=-.2em]
{m_{X,Y}} \quad : \quad {FX\otimes FY} \arrow[rr] && {F(X\otimes Y)}
\\
\quad {m_{1}} \quad \hspace{.15em}  :  \hspace{.85em} \quad \quad {1} \quad \quad\hspace{.1em} \arrow[rr] && \quad\hspace{.2em} {F(1)} \quad
\end{tikzcd}
\end{center}
natural in each component~$X$ and~$Y$ independently,
and making the same coherence diagrams commute
as in the usual case of a strong monoidal functor between symmetric monoidal categories.
\end{definition}





\begin{definition}
A pair of strong monoidal functors $F:\Dduploid\to\Eduploid$ and $G:\Eduploid\to\Dduploid$
between symmetric monoidal duploids $(\Dduploid,\otimes,\1)$ and $(\Eduploid,\otimes,\1)$
is called a \define{monoidal equivalence} when there exists two families of thunkable 
and linear isomorphisms $\nu_X : F(GX) \to X$ and $\nu'_X : G(FX) \to X$, both natural in $X$
and compatible with the tensor product.
\end{definition}

The De Morgan duality of classical logic implies to consider the
original definition~\ref{definition/sm-duploid} of (positive)
symmetric monoidal duploid together with its dual: a \define{negative
  symmetric monoidal structure}~$(\Dduploid,\parr,\bot)$ on a
duploid~$\Dduploid$ such that the inclusion functor
$\Ncategoryl\inclusion\Ncategory$ is equipped with of the structure of
a symmetric monoidal Freyd category
$(\Ncategoryl,\parr,\top)\to(\Ncategory,\parr,\top)$.
This
leads us to the following definition of a dialogue duploid.
\begin{definition}\label{dialogueduploidsym}
A \define{dialogue duploid} is a duploid $\mathcal D$
equipped with a positive and negative symmetric monoidal duploid 
structure $(\mathcal D, \otimes, \1)$ and $(\mathcal D, \parr, \bot)$
related by a strong monoidal equivalence
   \[
      \begin{tikzcd}
        (\mathcal D, \otimes, \1) \arrow[r, bend left, "{\dupldual{(-)}}"] & \arrow[l, bend left, "{\dupldual{(-)}}"] (\mathcal D, \parr, \bot)^{op}
      \end{tikzcd}
    \]
together with a family of bijections (called currifications)
$$\chi_{X, Y, Z} \quad : \quad \Dduploid (X \otimes Y, Z) \quad \simeq \quad \Dduploid (X, \dupldual{Y} \parr Z)$$
natural component-wise in~$X$, $Y$ and $Z$,\footnote{That is, $\chi$
is a natural transformation between graph homomorphisms
$\Dduploid^\op\boxtimes\Dduploid^\op\boxtimes\Dduploid\rightarrow\Set$.
This also amounts (modulo shifts) to a natural transformation between
functors of categories
$\Pcategory^\op\boxtimes\Pcategory^\op\boxtimes\Ncategory\rightarrow\Set$\ifbool{arxiv}{
 where $\boxtimes$ is extended into the ``funny'' tensor product of categories}{}.
}
and subject up to monoidality, symmetry and associativity to the
coherence condition \ifbool{arxiv}{between $\chi$ and monoidality}{}
$\chi_{A,B\otimes C,D}=\chi_{A,B,C^{*}\parr D}\circ\chi_{A\otimes B,C,D}$.
\end{definition}




\noindent Note that an associative dialogue duploid is the same thing
as a $*$-autonomous category.
The theorem below establishes in what sense the notion of dialogue duploid
can be seen as a direct and computational counterpart to dialogue chiralities,
which provides an overarching mathematical framework for reasoning in direct style
about (linear and non-linear continuations),
while preserving the perfect symmetry between CBV and CBN evaluation paradigms.
\begin{theorem}\label{theorem/adjunctions-monoidal-duploids}
Every duploid~$\duploid{L}{R}$ 
associated to a dialogue chirality $L\dashv R$
comes equipped with a dialogue duploid structure.
Conversely, every dialogue duploid~$(\Dduploid,\otimes,\parr)$ 
induces a dialogue chirality structure
on the adjunction~\eqref{equation/adjunction-PN},
whose associated dialogue duploid is equivalent to~$\Dduploid$
in the strong monoidal sense.
\end{theorem}

\section{The classical $L$-calculus}\label{section/classical-L-calculus}

\begin{figure*}[!t]
\vspace*{-4ex}
\begin{mdframed}[shadow=true,shadowsize=2pt,shadowcolor=black!8]
\begin{small}
\centering

\hspace*{\fill}\subfloat[Grammar]{
$
 \stretcharray
 \begin{array}{lrcccccccccccccccc}
   \mbox{Values:} & V, W & ::= & a &\bmid& \mu\alpha^-.\command c &\bmid& () &\bmid& V \otimes W &\bmid& \mu[].\command c &\bmid& \mu(\alpha \parr \beta).\command c&\bmid& [S] &\bmid& \mu[a].\command c\\
   \mbox{Expressions:} & t,u & ::= & V &\bmid& \mu\alpha^+.\command c\\
   \mbox{Stacks:} & S & ::= & \alpha &\bmid& \tmu a^+.\command c &\bmid& \tmu().\command c &\bmid& \tmu(a\otimes b).\command c &\bmid& [] &\bmid& S \parr S'&\bmid& \tmu[\alpha].\command c &\bmid& [V]\\
   \mbox{Contexts:} & e & ::= & S &\bmid& \tmu a^-.\command c\\
   \mbox{Commands:} & \command c & ::= & \perfectcut{V}{e}^- &\bmid& \perfectcut{t}{S}^+
 \end{array}
$
}\hspace*{\fill}

\hspace*{\fill}\subfloat[Conversions (Reduction and expansion rules)]{
 $
 \stretcharray
 \begin{array}{lccc}
   (R\tmu^\veps) & \perfectcut{V}{\tmu a^\veps.\command c}^\veps & \triangleright_R & c[V/a]\\
   (R\mu^\veps) & \perfectcut{\mu \alpha^\veps.\command c}{S}^\veps & \triangleright_R & c[S/\alpha]\\
   (R1) & \perfectcut{()}{\tmu().\command c}^+ & \triangleright_R & c\\
   (R\otimes) & \perfectcut{V\otimes W}{\tmu(a\otimes b).\command c}^+ & \triangleright_R & c[V/a,W/b]\\
   (R\bot) & \perfectcut{\mu[].\command c}{[]}^- & \triangleright_R & c\\
   (R\parr) & \perfectcut{\mu(\alpha\parr\beta).\command c}{S \parr S'}^- & \triangleright_R & c[S/\alpha,S'/\beta]\\
   (R\negN) & \perfectcut{[S]}{\tmu[\alpha].\command c}^+ & \triangleright_R & c[S/\alpha]\\
   (R\negP) & \perfectcut{\mu[a].\command c}{[V]}^- & \triangleright_R & c[V/a]
 \end{array}
 \quad\quad\quad\quad
 \begin{array}{lccc}
   (E\tmu^\veps) & e & \triangleright_E & \tmu a^\veps.\perfectcut{a}{e}^\veps\\
   (E\mu^\veps) & t & \triangleright_E & \mu\alpha^\veps.\perfectcut{t}{\alpha}^\veps \\
   (E1) & S & \triangleright_E & \tmu().\perfectcut{()}{S}^+ \\
   (E\otimes) & S & \triangleright_E & \tmu(a\otimes b).\perfectcut{a\otimes b}{S}^+ \\
   (E\bot) & V & \triangleright_E & \mu[].\perfectcut{V}{[]}^-\\
   (E\parr) & V & \triangleright_E & \mu(\alpha\parr\beta).\perfectcut{V}{\alpha\parr\beta}^-\\
   (E\negN) & S & \triangleright_E & \tmu[\alpha].\perfectcut{[\alpha]}{S}^+ \\
   (E\negP) & V & \triangleright_E & \mu[a].\perfectcut{V}{[a]}^- 
 \end{array}
 $
}\hspace*{\fill}

\hspace*{\fill}\subfloat[Judgements]{
$
 \command c : (\Gamma \vdash \Delta) \quad\quad\quad \Gamma \vdash V : B\ |\ \Delta\quad\quad\quad \Gamma \vdash t : B\ |\ \Delta\quad\quad\quad \Gamma\ |\ S : A \vdash \Delta\quad\quad\quad \Gamma\ |\ e : A \vdash \Delta
$
}\hspace*{\fill}

\hspace*{\fill}\subfloat[Typing rules (Identity and structural groups)]{
\begin{varwidth}{\linewidth}
\centering
 $
   \begin{prooftree}
     \infer0[$(\vdash\mathbf{ax})$]{a:A\vdash a : A\ |}
   \end{prooftree}
   \quad\quad
   \begin{prooftree}
     \infer0[$(\mathbf{ax}\vdash)$]{|\ \alpha:A\vdash \alpha : A}
   \end{prooftree}
   \quad
 $
\bigbreak

 $
   \begin{prooftree}
     \hypo{\command c:(\Gamma, a:A_\veps\vdash \Delta)}
     \infer1[$(\boldsymbol{\tmu}^\veps\vdash)$]{\Gamma\ |\ \tmu a^\veps.\command c:A_\veps\vdash\Delta}
   \end{prooftree}
   \quad\quad
   \begin{prooftree}
     \hypo{\command c:(\Gamma \vdash\alpha : A_\veps, \Delta)}
     \infer1[$(\vdash\boldsymbol{\mu}^\veps)$]{\Gamma\vdash \mu \alpha^\veps.\command c:A_\veps\ |\ \Delta}
   \end{prooftree}
   \quad\quad
   \begin{prooftree}
     \hypo{\Gamma\ |\ e : A_\veps\vdash \Delta}
     \hypo{\Gamma' \vdash t : A_\veps\ |\ \Delta'}
     \infer2[$(\mathbf{cut}^\veps)$]{\perfectcut{t}{e}^\veps : (\Gamma, \Gamma' \vdash \Delta, \Delta')}
   \end{prooftree}
 $
\bigbreak

$
 \forall\sigma\in\Sigma(\Gamma',\Gamma),\ \forall\tsigma\in\Sigma(\Delta,\Delta')\;: \quad\quad
 \begin{prooftree}
   \hypo{\Gamma \vdash t : A\ |\ \Delta}
   \infer1[$(\vdash\sigma, \tsigma)$]{\Gamma'\vdash t[\sigma,\tsigma]\ |\ \Delta'}
 \end{prooftree}
 \quad
 \begin{prooftree}
   \hypo{\Gamma\ |\ e : A\vdash\Delta}
   \infer1[$(\sigma, \tsigma\vdash)$]{\Gamma'\ |\ e[\sigma,\tsigma]\vdash\Delta'}
 \end{prooftree}
 \quad
 \begin{prooftree}
   \hypo{\command c:(\Gamma \vdash \Delta)}
   \infer1[$(\sigma, \tsigma)$]{c[\sigma,\tsigma]:(\Gamma'\vdash \Delta')}
 \end{prooftree}
$
\end{varwidth}
}\hspace*{\fill}

\hspace*{\fill}\subfloat[Typing rules (Logic group)]{
\begin{varwidth}{\linewidth}
\centering
$
 \begin{prooftree}
   \hypo{}
   \infer1[$(\vdash 1)$]{\vdash() : 1\ |}
 \end{prooftree}
 \quad
 \begin{prooftree}
   \hypo{\Gamma\vdash V : A\ |\ \Delta}
   \hypo{\Gamma'\vdash W : B\ |\ \Delta'}
   \infer2[$(\vdash\otimes)$]{\Gamma,\Gamma'\vdash V \otimes W:A\otimes B\ |\ \Delta,\Delta'}
 \end{prooftree}
 \quad
 \begin{prooftree}
   \hypo{\command c : (\Gamma\vdash\Delta)}
   \infer1[$(1 \vdash)$]{\Gamma\ |\ \tmu().\command c : 1 \vdash \Delta}
 \end{prooftree}
 \quad
 \begin{prooftree}
   \hypo{\command c : (\Gamma, a : A, b : B\vdash\Delta)}
   \infer1[$(\otimes\vdash)$]{\Gamma\ |\ \tmu(a \otimes b).\command c : A\otimes B\vdash\Delta}
 \end{prooftree}
$
\bigbreak

$
 \begin{prooftree}
   \hypo{}
   \infer1[$(\bot\vdash)$]{|\ []:\bot \vdash}
 \end{prooftree}
 \quad
 \begin{prooftree}
   \hypo{\command c:(\Gamma\vdash\Delta)}
   \infer1[$(\vdash\bot)$]{\Gamma \vdash \mu[].\command c : \bot \ |\ \Delta}
 \end{prooftree}
 \quad
 \begin{prooftree}
   \hypo{\Gamma\ |\ S : A\vdash \Delta}
   \hypo{\Gamma'\ |\ S' : B\vdash \Delta'}
   \infer2[$(\parr\vdash)$]{\Gamma,\Gamma'\ |\ S \parr S':A\parr B\vdash \Delta,\Delta'}
 \end{prooftree}
 \quad
 \begin{prooftree}
   \hypo{\command c:(\Gamma\vdash \alpha : A, \beta : B, \Delta)}
   \infer1[$(\vdash\parr)$]{\Gamma \vdash \mu(\alpha \parr \beta).\command c : A \parr B\ |\ \Delta}
 \end{prooftree}
$
\bigbreak

$
 \begin{prooftree}
   \hypo{\Gamma\ |\ S : N\vdash\Delta}
   \infer1[$(\vdash\negN)$]{\Gamma\vdash[S]:\dupldual{N}\ |\ \Delta}
 \end{prooftree}
 \quad
 \begin{prooftree}
   \hypo{\command c:(\Gamma\vdash\alpha:N, \Delta)}
   \infer1[$(\negN\vdash)$]{\Gamma\ |\ \tilde\mu[\alpha].\command c:\dupldual{N}\vdash\Delta}
 \end{prooftree}
 \quad
 \begin{prooftree}
   \hypo{\Gamma\vdash V : P\ |\ \Delta}
   \infer1[$(\negP\vdash)$]{\Gamma\ |\ [V]:\dupldual{P}\vdash\Delta}
 \end{prooftree}
 \quad
 \begin{prooftree}
   \hypo{\command c:(a:P,\Gamma\vdash\Delta)}
   \infer1[$(\vdash\negP)$]{\Gamma\vdash\mu[a].\command c:\dupldual{P}\ |\ \Delta}
 \end{prooftree}
$
\end{varwidth}
}\hspace*{\fill}
\end{small}
\end{mdframed}
\vspace*{-1ex}\caption{Syntax of the classical $L$-calculus}\vspace*{-2ex}
\label{figure/syntaxDialogue}
\end{figure*}

After Curien and Herbelin~\cite{CH00Duality}, $L$-calculi for sequent
calculus were extended to feature polarities, involutive negation, and
linearity~\cite{munchmonolateral,Munch14Involutive,munchlcalculi}.
Building upong these works, we introduce the classical $L$-calculus in
\cref{figure/syntaxDialogue}.
In the linear logic nomenclature, the underlying sequent calculus can
be called $\polsystem{MLL}$ (\emph{polarised} multiplicative linear
logic with \emph{$\eta$-restriction} in the terminology of Danos
\emph{et al.}~\cite{danos95new}).



The terms of the $L$-calculus come in five syntactic categories:
expressions, values, contexts, stacks and commands.
Values and stacks are particular expressions and contexts, respectively,
which can be understood as pure or effect-free.
In particular, variables (noted $a$, $b$, $c$, ...) are values, 
and dually, co-variables (noted $\alpha$, $\beta$, $\gamma$, ...) are stacks.


Each type comes with a polarity: 
\[
  \stretcharray
  \begin{array}{lccl}
\mbox{Negatives:} & N, M, A_- & ::= & X^+\bmid \bot\bmid A\parr B\bmid \dupldual{P}\\
    \mbox{Positives:} & P, Q, A_+ & ::= & X^-\bmid 1\bmid A\otimes B\bmid \dupldual{N}
  \end{array}
\]
We have the types $\1$ and $\bot$ corresponding to the unit of the conjunction and the disjunction respectively,
and for two types $A$ and $B$, we can construct the types $A \otimes B$ and $A \parr B$.
For the negation, we have two distinct connectives, one for each polarity as in~\cite{danos95new, Munch14Involutive},
and we note both of them $\dupldual{(-)}$ to simplify the notations.


Variables are bound by $\tmu$ to form a stack~$\tmu a^+.\command c$
when the variable~$a$ has a positive type, and to form 
a context~$\tmu a^-.\command c$ when the variable~$a$ has a negative type.
Dually, co-variables are bound by $\mu$ to form a value~$\mu \alpha^-.\command c$
when the co-variable~$\alpha$ has a negative type, or an expression~$\mu \alpha^+.\command c$
when the co-variable~$\alpha$ has a positive type.
The term $()$ and the nullary binder $\tmu().\command c$ are associated with the unit of the conjunction $\1$. We can construct conjunctive terms with either the binary binder $\tmu(\alpha \otimes \beta).\command c$ or the construction $V \otimes W$. Symmetrically, for the disjunction $\parr$, we have the nullary and binary binders $\mu[].\command c$ and $\mu(a\parr b).\command c$ and the constructions $[]$ and $S \parr S'$.
In order to model the rules of negation, we also have the unary
binders $\mu[a].\command c$ and $\tmu[\alpha].\command c$, as well as
the constructions $[V]$ and $[S]$ which turn terms into duals.
In this language, the $\keyword{let}$ construct is defined as:
\[
  \letinepsilon a t u \;\;:=\;\; \mu\alpha^{\veps'}.\perfectcut{t}{\tmu a^\veps.{\perfectcut u \alpha}^{\veps'}}^{\veps}\;.
\]

\noindent The figure defines a reduction relation $\triangleright_R$
($\beta$-like) and an expansion relation $\triangleright_E$
($\eta$-like) between terms.
We note $\to_{RE}$ the contextual closure of ($\beta\eta$)
\emph{reduction} $\triangleright_R \cup \triangleleft_E$, and
$\simeq_{RE}$ the symmetric, transitive and reflexive closure of
$\to_{RE}$.

Typing rules are used to define typing derivations and well-typed terms.
Each judgment has a context each side,
expressions and values have a distinguished type on the right, 
contexts and stacks a distinguished type on the left 
and commands don't have any.
A context on the left $\Gamma$ (resp. context on the right $\Delta$) is a map from an ordered finite set of variables (resp. co-variables) to types. 
The notations $\Gamma,\Gamma'$ and $\Delta,\Delta'$ imply that the contexts have disjoint domains.


Structural rules lets us \emph{rename} the (co-)variables of the contexts and
\emph{change their order}. To this effect, we define
\define{$\Sigma(\Gamma,\Gamma')$} the set of \emph{bijective} maps $\sigma :
\dom \Gamma \to \dom \Gamma'$ such that $\Gamma'(\sigma(a)) =
\Gamma(a)$ for all $a\in\dom \Gamma$.
Regarding the cartesian case, it is possible to obtain (non-linear)
classical logic---precisely the multiplicative fragment of Danos, Joinet and
Schellinx's $\polsystem{LK}$~\cite{danos95new}---by omitting the
bijection requirement, thus allowing \emph{weakening} and
\emph{contraction}. (This treatment of structural rules is reminiscent
of Atkey~\cite{Atkey2006}, Curien, Fiore and
Munch-Maccagnoni~\cite{munchlcalculi}.)

Unrestricted (non-\emph{focused}) rules for negation are derived from
their restrictions to values/stacks:
\begin{definition}
For $e$ a negative context, we define $[e] := \mu\alpha^+.\perfectcut{\mu\beta^-.\perfectcut{[\beta]}{\alpha}^+}{e}^-$. Symmetrically, for t a positive term, we define $[t] := \tilde\mu a^-.\perfectcut{t}{\tilde\mu b^+.\perfectcut{a}{[b]}^-}^+$. The following rules can be derived:
  \[
    \begin{prooftree}
      \hypo{\Gamma\ |\ e : N\vdash\Delta}
      \infer[double]1{\Gamma\vdash[e]:\dupldual N\ |\ \Delta}
    \end{prooftree}
    \quad
    \begin{prooftree}
      \hypo{\Gamma\vdash t : P\ |\ \Delta}
      \infer[double]1{\Gamma\ |\ [t]:\dupldual P\vdash\Delta}
    \end{prooftree}
  \]
\end{definition}

\noindent
In other words, computation of $[e]$ and $[t]$ in the general case
reduces inside the terms using (essentially) let-expansions:
\begin{align*}
\perfectcut{[e]}{S}^+ & \triangleright_R\perfectcut{\mu\beta^-.\perfectcut{[\beta]}{S}^+}{e}^- & \text{(\ensuremath{e} not a stack)}\\
\perfectcut{V}{[t]}^- & \triangleright_R\perfectcut{t}{\tilde\mu b^+.\perfectcut{V}{[b]}^-}^+ & \text{(\ensuremath{t} not a value)}
\end{align*}
Notice that these let-expansions crucially involve compositions of
both polarities on the right-hand side. This behaviour distinguishes
our interpretation of negation from calculi built around the idea of
(external) CBV/CBN duality~\cite{Wad03Dual,Laurent2008}. It
circumvents syntactic objections~\cite{Parigot00Negation} to an
involutive negation in a (non-linear) classical context. This explicit
treatment of the negation of $\LC$
follows~\cite{danos95new,munchmonolateral,Munch14Involutive}.

\begin{lemma}\label{lemma/negSwitchSide}
For $e$ a context and $\command c$ a command, one has:
\[
  \perfectcut{[e]}{\tmu[\alpha].\command c}^+\simeq_{RE}\perfectcut{\mu\alpha^-.\command c}{e}^-
\]
Likewise, for $t$ an expression and $\command c$ a command, one has:
\[
  \perfectcut{\mu [a].\command c}{[t]}^-\simeq_{RE}\perfectcut{t}{\tmu a^+.\command c}^+
\]
\end{lemma}


We have similar constructions and rules for $\otimes$ and $\parr$, with two versions for each one depending on which side of $\otimes$ or $\parr$ is evaluated first. It allows us to give a meaning to $\command c[t/a]$ and $\command c[e/\alpha]$ for any $t$ and $e$.

\begin{theorem}[Subject reduction]
  If $\command c \to_{RE} \command c'$ and $\command c : (\Gamma \vdash \Delta)$, then $\command c' : (\Gamma \vdash \Delta)$.
\end{theorem}

\begin{theorem}[Soundness of the classical $L$-calculus]\label{thm/soundness}
The interpretation of typed terms in any dialogue duploid is invariant modulo reductions and expansions.
\end{theorem}

\noindent A coherence result between dialogue categories and
chiralities~\cite{melliesdialogue} suggests, via
\cref{theorem/adjunctions-monoidal-duploids,thm/soundness}, that we
should see the simplification brought by a strictly-involutive
negation with all formulae on the right~\cite{Gir87,girardnew} as a
coherence property.

\section{The syntactic dialogue duploid}\label{section/syntactic-duploid}
We construct a dialogue duploid whose objects are the types of the classical $L$-calculus
and whose morphisms $\command c : A\to B$ between two types $A$ and $B$ are 
the commands $\command c : (a : A \vdash \beta : B)$ quotiented by the rewriting relation~$\simeq_{RE}$.
The composite of two maps 
$$\mathsf{c} : (a : A \vdash \beta : B)
\quad\quad\quad
\command c' : (b : B \vdash \gamma : C)$$
with respective typing derivations $\pi_1$ and $\pi_2$, 
is defined as the command of the $L$-calculus:
$$\perfectcut{\mu \beta^{\veps_B}.\command c}{\tmu b^{\veps_B}.\command c'}^{\veps_B}$$
with typing derivation:
  \[
    \begin{prooftree}
      \hypo{\pi_2}
      \infer1{\command c' : (b : B \vdash \gamma : C)}
      \infer1[$(\tmu\vdash)$]{|\ \tmu b^{\veps_B}.\command c' : B \vdash \gamma : C}
      \hypo{\pi_1}
      \infer1{\command c : (a : A \vdash \beta : B)}
      \infer1[$(\vdash\mu)$]{a : A \vdash \mu \beta^{\veps_B}.\command c : B\ |}
      \infer2[$(\mathsf{cut})$]{\perfectcut{\mu \beta^{\veps_B}.\command c}{\tmu b^{\veps_B}.\command c'}^{\veps_B} : (a : A \vdash \gamma : C)}
    \end{prooftree}
  \]

\begin{theorem}\label{thm/classical-syntactic-duploid}
  The construction just described
  defines a dialogue duploid called the syntactic dialogue duploid.
\end{theorem}


In order to establish the theorem, we give the following characterizations of thunkable maps
and of central maps in the non-associative category of commands.
Linear maps are characterized symmetrically.
\begin{lemma}\label{lemma/syntacticallythunkable}
  Let $t$ be an expression. The two following properties are equivalent :

\noindent
\emph{(1)} For all commands $c$, $\perfectcut{t}{\tmu a^\veps.c}^\veps \simeq_{RE} c[t/a]$;

\noindent
\emph{(2)} For all commands $c$ and contexts $e$,
\[
  \perfectcut{t}{\tmu a^{\veps_1}.\perfectcut{\mu \alpha^{\veps_2}.c}{e}^{\veps_2}}^{\veps_1} \simeq_{RE} \perfectcut{\mu \alpha^{\veps_2}.\perfectcut{t}{\tmu a^{\veps_1}.c}^{\veps_1}}{e}^{\veps_2}.
\]
\noindent
We say that an expression~$t$ is \define{syntactically thunkable}
  when it satisfies one of the above equivalent properties.
\end{lemma}
 
\begin{lemma}\label{lemma/thunkable}
A command $\command c : (a : A \vdash \beta : B)$ is thunkable
if and only if $\mu\beta^{\veps_B}.\command c$ is syntactically thunkable.
\end{lemma}

\noindent
This characterization based on the intuition that thunkable expression behave like values
plays a fundamental role in the proof that the syntactic polarity $\veps$ of a type $A_{\veps}$ in the $L$-calculus
coincides with its semantic polarity as an object of the non-assocative category,
as it is defined in \cref{definition/polarity}.


\begin{definition}\label{definition/syntacticallycentral}
An expression $t$ is \define{syntactically central} when the equality up to reduction and expansion
is satisfied
\[
\perfectcut{t}{\tmu q_1.\perfectcut{u}{\tmu q_2.c}^{\veps_2}}^{\veps_1}\simeq_{RE} \perfectcut{u}{\tmu q_2.\perfectcut{t}{\tmu q_1.c}^{\veps_1}}^{\veps_2}
\]
for all commands $\command c$, expressions $u$ and binders $q_1$ and $q_2$ (i.e.\ 
either $a$, $a \otimes b$, $()$ or $[\alpha]$) of polarity $\veps_1$
and $\veps_2$ respectively.
\end{definition}

\begin{lemma}\label{lemma/central}
A command $\command c : (a : A \vdash \beta : B)$ is central
if and only if the expression $\mu\beta^{\veps_B}.\command c$ is syntactically central.
\end{lemma}



\begin{proof}
The interested reader will find the proofs of the two lemmas in the Appendix.
\end{proof}

\section{The \FH{} theorem}\label{section/fh-theorem}

In this section, we formulate and establish the \FH{} theorem 
in the language of dialogue duploids.
We have seen in \cref{lemma/thunkableimpliescentral}
that every thunkable map is central in a symmetric monoidal duploid,
and that the converse property is not true in general.
We establish now that the two notions coincide in a dialogue duploid.
\begin{theorem}[\FH{} theorem]
  In a dialogue duploid, a morphism is central for $\otimes$ if and
  only if it is thunkable.
\end{theorem}
\begin{proof}
We want to prove that central morphisms are thunkable in any dialogue duploid.
This can be done by purely equational reasoning, 
using the observation that the composite $g \ccomp f$
can be expressed in every dialogue duploid as:
  \[
    g \ccomp f =  \chi_{A,\dupldual C,\bot}(\chi^{-1}_{A,\dupldual B,\bot}(f)
    \bullet (A \rtensortimes \dupldual g)) \hspace{.5em} : \hspace{.5em} A \to C
  \]
for every pair of maps $f:A\to B$ and $g:B\to C$, 
where we do not indicate for readability reasons
the units $A\tensor 1\to A$ for the tensor product 
and $\dupldoubledual{A}\to A$ for double negation.
The reader will find more details in the Appendix.
\end{proof}
One benefit of the classical $L$-calculus is that the same statement can be also established
by purely syntactic means, thanks to the equational theory of the classical $L$-calculus,
and the soundness theorem of its interpretation in dialogue duploids.
Seen from a purely syntactic point of view, the two notions of
centrality and thunkability can both be expressed as commutations in
the classical $L$-calculus, as explained in the introduction and shown
in \cref{lemma/syntacticallythunkable} and
\cref{definition/syntacticallycentral}.



We prove now that every syntactically central expression is syntactically thunkable in the classical $L$-calculus.
We will use the fact the two notions of commutation are related by \cref{lemma/negSwitchSide}
in the equational theory of the classical $L$-calculus.
Let $t$ be an expression, $\command c$ a commands and $e$ be a context. 
We assume that $t$ is syntactically central and we prove that $t$ is syntactically thunkable. 
The only difficult case is when $t$ is positive and $e$ is negative.
\begin{align*}
  & \perfectcut{t}{\tmu b^+.\perfectcut{\mu\gamma^-.\command c}{e}{}^-}{}^+ & &\\
  & \simeq_{RE} \perfectcut{t}{\tmu b^+.\perfectcut{[e]}{\tmu[\gamma].\command c}{}^+}{}^+ & & \mbox{By\ \cref{lemma/negSwitchSide}}\\
  & \simeq_{RE} \perfectcut{[e]}{\tmu[\gamma].\perfectcut{t}{\tmu b^+.\command c}{}^+}{}^+ & & \mbox{Centrality of } t\\
  & \simeq_{RE} \perfectcut{\mu \gamma^-.\perfectcut{t}{\tmu b^+.\command c}{}^+}{e}{}^- & & \mbox{By\ \cref{lemma/negSwitchSide}}
\end{align*}
This establishes that the expression $t$ is syntactically thunkable.
We conclude that:
\begin{theorem}[Syntactic \FH{} theorem]\label{theorem/fh-syntactic}
An expression of the classical $L$-calculus is syntactically central for $\otimes$ if and only if it is syntactically thunkable.
\end{theorem}
\noindent

Recall that the general situation of a duploid~$\Dduploid$ associated
to an adjunction~$L\dashv R$, one has that
\begin{center}
\fbox{
\begin{tabular}{c}
\emph{the monad $R\circ L$ is idempotent
if and only if}
\\
\emph{every morphism of the duploid~$\Dduploid$ is thunkable.}
\end{tabular}}
\end{center}
Also, it is not difficult to see that in the situation described in~\S\ref{section/sm-duploids}
of a symmetric monoidal duploid~$\Dduploid$ associated to 
an adjunction~$L\dashv R$ where~$\Acategory$ is symmetric monoidal
and where the monad~$T=R\circ L$ is strong, one has that
\begin{center}
\fbox{
\begin{tabular}{c}
\emph{the monad $T$ is commutative if and only if}
\\
\emph{every morphism of the duploid~$\Dduploid$ is central.}
\end{tabular}}
\end{center}
In the case of a dialogue duploid~$\Dduploid$ associated to a dialogue
category, this proves the following statement, attributed to Hasegawa
in \cite{melliestabareau}, as a corollary of
\cref{theorem/fh-syntactic}.
\begin{corollary}
The continuation monad of a dialogue category is commutative if and only if it is idempotent.
\end{corollary}













It is natural to wonder if we could not weaken the assump\-tions of
structure on duploids. Removing negation from
\cref{figure/syntaxDialogue} leads to consider a linearly distributive
structure on duploids:
\begin{definition}
  A \define{linearly distributive duploid} is a duploid equipped with
  a pair of positive and negative symmetric monoidal structures
  related by a family of mappings
$A \otimes (B \parr C) \to (A \otimes B) \parr C$ natural
  component-wise and that respects the usual coherence diagrams for a
  linearly distributive category~\cite{Cockett_1997,Mellies2017micrological}.
\end{definition}

\noindent Note in particular that a linearly distributive duploid that
is associative is the same thing as a linearly distributive category.
A variant of the syntactic argument given in\ifbool{doubleblind}{
  Munch-Maccagnoni}{}~\cite[p.262]{munchthese} then suggests the
following refinement of the \FH{} theorem (in the dual): in any
linearly distributive duploid which is closed
in the sense of a natural isomorphism $\Dduploid (X \otimes Y, Y'
\parr Z) \simeq \Dduploid (X, (Y\multimap Y') \parr Z)$, a morphism is
central for $\parr$ if and only if it is linear.

\section{Conclusion and future work}\label{section/conclusion}
We have introduced the syntax and semantics of classical $L$-calculus,
and developed a theory of dialogue duploids.
We see the framework 
as solid foundation for the study of non-associative and effectful logical systems 
and term calculi for classical logic, integrating the lessons of linear logic, continuation
models and functorial game semantics.


One interesting direction is in connection with programming language
semantics. For instance, Cong, Oswald, Essertel and
Rompf~\cite{Cong_2019} characterise a restriction to the usage of
continuations suitable for compilation, which crucially still permits
to copy and discard them. This is beyond the scope of dialogue
duploids and will probably involve the notion of linearly distributive
duploid just introduced.

$L$-calculi have also been given for other notions of effectful
computation. We believe that the notion of dialogue duploid can serve
as a blueprint for further connections, such as between models of
LCBPV~\cite{munchlcalculi} and ``symmetric monoidal closed'' duploids,
and likewise between CBPV~\cite{Levy2004} and ``(bi\nobreakdash-)cartesian
closed'' duploids. Such correspondances are for instance the right way
(in our opinion) to connect CBPV to focusing in proof theory.


\ifbool{arxiv}{}{\clearpage}

\ifbool{natbib}{
\begin{small}
  \bibliographystyle{IEEEtranN}
  \bibliography{biblio}
  \flushcolsend
  \end{small}
}{
\bibliographystyle{IEEEtran}
  \bibliography{biblio}
}

\newpage

\onecolumn
\appendix


\subsection{Proof of \joyallemma}

\global\long\def\pair#1#2{\langle#1,#2\rangle}
\begin{proof}[Proof of \joyallemma]
Observe that the object $\bot^{\bot}\simeq\bot^{\bot^{1}}\simeq1$
is the terminal object. Consequently, the set $\mathscr{C}(\bot\times\bot,\bot)$,
in bijection with $\mathscr{C}(\bot,\bot^{\bot})$, is a singleton;
in particular one has $\pi_{1}=\pi_{2}\in\mathscr{C}(\bot\times\bot,\bot)$.
Now consider the pairs $\pair fg\in\mathscr{C}(A,\bot\times\bot)$
for $f,g\in\mathscr{C}(A,\bot)$. By the identity of projections,
one has $f=g$ for any such pair of morphisms, in other words any
$\mathscr{C}(A,\bot)$ has at most one element. Thus, any hom-set
$\mathscr{C}(B,C)$ has at most one element as well, as witnessed
by the bijections: $\mathscr{C}(B,C)\simeq\mathscr{C}(B,\bot^{\bot^{C}})\simeq\mathscr{C}(B\times\bot^{C},\bot)$.
\end{proof}

\subsection{Discussions on duploids}


\begin{lemma}
  Let $(\nDownarrow, \omega)$ be a positive shift for $\mathcal M$. Then, for any object $A$, $\omega^{-1}_A$ is thunkable.
\end{lemma}
\begin{proof}
  Let $A$ be an object of $\mathcal M$. For all $f$ and $g$ two morphisms, one has :
    \begin{align*}
      & (f\ccomp g) \ccomp \omega^{-1}_A\\
      &= (f\ccomp (g \ccomp (\omega^{-1}_A \bullet \omega_A))) \ccomp \omega^{-1}_A\\ 
      &= (f\ccomp ((g \ccomp \omega^{-1}_A) \bullet \omega_A)) \ccomp \omega^{-1}_A && \mbox{by thunkability of }\omega_A\\ 
      &= (f\ccomp (g \ccomp \omega^{-1}_A)) \bullet \omega_A \ccomp \omega^{-1}_A && \mbox{by thunkability of }\omega_A\\ 
      &= f\ccomp (g \ccomp \omega^{-1}_A)
    \end{align*}
  So $\omega^{-1}_A$ is thunkable.
\end{proof}

\begin{proposition}\label{functorialshift}
  The mapping $\nDownarrow$ of a positive shift $(\nDownarrow, \omega)$ can be extended to morphisms such that identities and thunkability are conserved. Moreover, for $f$ and $g$,$\nDownarrow$ conserves their composition, i.e.\ that
  \[
    \nDownarrow(g \ccomp f) = \nDownarrow g \bullet \nDownarrow f.
  \]
 if and only if $\omega_C \ccomp g\ccomp f$ associates.
\end{proposition}

\begin{proof}
  \[
    \begin{array}{ccl}
      \forall f \in \mathcal M(A,B),& \nDownarrow f &:= (\omega_B \ccomp f) \ccomp \omega^{-1}_A
    \end{array}
  \]
  Identities are obviously conserved and, as $\omega$ and $\omega^{-1}$ are families of thunkable morphisms, if $f$ is thunkable, then $\nDownarrow f$ is thunkable too.

  Let $f \in \mathcal M(A,B)$ and $g \in \mathcal M(B,C)$.
  \begin{align*}
    & \nDownarrow g \bullet \nDownarrow f\\
    & = ((\omega_C \ccomp g) \ccomp \omega^{-1}_B) \bullet (\omega_B \ccomp f) \ccomp \omega^{-1}_A\\
    & = ((\omega_C \ccomp g) \ccomp \omega^{-1}_B) \bullet \omega_B \ccomp (f \ccomp \omega^{-1}_A) && \mbox{by thunkability of }\omega^{-1}\\
    & = ((\omega_C \ccomp g) \ccomp (\omega^{-1}_B \bullet \omega_B)) \ccomp (f \ccomp \omega^{-1}_A) && \mbox{by thunkability of }\omega\\
    & = ((\omega_C \ccomp g) \ccomp (f \ccomp \omega^{-1}_A)\\
    & = (((\omega_C \ccomp g) \ccomp f) \ccomp \omega^{-1}_A && \mbox{by thunkability of }\omega^{-1}
  \end{align*}
  Thus, if $\omega_C \ccomp g \ccomp f$ associates, then $\nDownarrow g \bullet \nDownarrow f = \nDownarrow (g \ccomp f)$. 

  Conversely, we now assume that $\nDownarrow$ conserves the composition of $f$ and $g$.
  \begin{align*}
    & (\omega_C \ccomp g) \ccomp f\\
    & = ((\omega_C \ccomp g) \ccomp f) \ccomp (\omega^{-1}_A \pcomp \omega_A)\\
    & = (((\omega_C \ccomp g) \ccomp f) \ccomp \omega^{-1}_A) \pcomp \omega_A &&\mbox{by thunk of }\omega\\
    & = ((\omega_C \ccomp (g \ccomp f)) \ccomp \omega^{-1}_A) \pcomp \omega_A &&\mbox{by hypothesis}\\
    & = (\omega_C \ccomp (g \ccomp f)) \ccomp (\omega^{-1}_A \pcomp \omega_A) &&\mbox{by thunk of }\omega\\
    & = (\omega_C \ccomp (g \ccomp f))
  \end{align*}
\end{proof}


\begin{lemma}\label{strongaltthunkable}
  Let $\mathcal M$ be a magmoid and $f$ a morphism of $\mathcal M$ from $A$ to $B$. If there exists $N$ and $P$ two objects of $\mathcal M$ respectively negative and positive and left-invertible morphisms $\delta \in \mathcal M(B,N)$ and $\omega \in \mathcal M(N,P)$ such that $\omega$ is thunkable, then $f$ is thunkable if and only if $\omega \circ \delta \ccomp f$ associates.
\end{lemma}
\begin{proof}
  We assume that $f$ verifies \cref{strongaltthunkable}. We will first show that, for any $h \in \mathcal M(N, C)$, $h \ccomp \delta \ccomp f$ associates.
    \begin{align*} 
      (h \circ \delta) \ccomp f &= ((h \circ (\omega^* \bullet \omega)) \circ \delta) \ccomp f\\ 
                               &= (h \circ \omega^*) \bullet (\omega \circ \delta) \ccomp f &&\mbox{by thunkability of }\omega\\ 
                               &= (h \circ \omega^*) \bullet \omega \circ (\delta \ccomp f) &&\mbox{by hypothesis}\\ 
                               &= (h \circ (\omega^* \bullet \omega)) \circ (\delta \ccomp f) &&\mbox{by thunkability of }\omega\\ 
                               &= h \circ (\delta \ccomp f)
    \end{align*}
  Now we prove that $f$ is thunkable. For any $g$ and $h$, one has
    \begin{align*}
      (h \ccomp g) \ccomp f &= (h \ccomp (g \ccomp \delta^* \circ \delta)) \ccomp f\\
                          &= (h \ccomp (g \ccomp \delta^*) \circ \delta) \ccomp f &&\mbox{by negativity of }N\\
                          &= h \ccomp (g \ccomp \delta^*) \circ (\delta \ccomp f) &&\mbox{as proved above}\\
                          &= h \ccomp ((g \ccomp \delta^* \circ \delta) \ccomp f) &&\mbox{as proved above}\\
                          &= h \ccomp (g \ccomp f)
    \end{align*}
  So $f$ is thunkable. The other direction is trivial.
\end{proof}
\begin{corollary}
  Let $\mathcal D$ be a duploid and $f$ be a morphism of $\mathcal D$ from $A$ to $B$. $f$ is thunkable if and only if:
  \begin{equation}\label{altthunkable}
    (\omega_{\nUparrow B} \circ \varphi^{-1}_B) \ccomp f = \omega_{\nUparrow B} \circ (\varphi^{-1}_B \ccomp f)
  \end{equation}
  Symmetrically, $f$ is linear if and only if:
  \[
    (f \ccomp \omega^{-1}_{A}) \bullet \varphi_{\nDownarrow A} = f \ccomp (\omega^{-1}_{A} \bullet \varphi_{\nDownarrow A})
  \]
\end{corollary}

\begin{definition}
  Let $\mathcal M$ and $\mathcal M'$ be two magmoids. A \define{functor of graphs} $F : \mathcal M \to \mathcal M'$ is given by:
  \begin{itemize}
    \item A mapping $F : |\mathcal M| \to |\mathcal M'|$,
    \item A family of mappings $F_{A,B} : \mathcal M(A,B) \to \mathcal M'(FA, FB)$.
  \end{itemize}
  Moreover, $F$ is a \define{functor of magmoids} if it preserves identities and composition. Finally, we say that $F$ is \define{polarised} if it preserves polarities, thunkability and linearity. A \define{duploid functor} is a polarised functor of magmoids between two duploids.
\end{definition}

\begin{proposition}
  Let $F : \mathcal D \to \mathcal D'$ be a functor of magmoids between two duploids. Then $F$ is a polarised functor if, for all $A$ in $\mathcal D$, $F(\nDownarrow A)$ is positive, $F(\nUparrow A)$ is negative, $F(\omega_A)$ is thunkable and $F(\varphi_A)$ is linear.
\end{proposition}

\begin{proof}
  We assume that these conditions holds for $F$.

  Let $f : A \to B$ be a thunkable morphism of $\mathcal D$. Then, $\omega_{\nUparrow B} \circ \varphi^{-1}_B \ccomp f$ associates. As $F$ preserves composition, $F(\omega_{\nUparrow B}) \circ F(\varphi^{-1}_B) \ccomp F(f)$ associates too. Thus, by \cref{strongaltthunkable}, $F(f)$ is thunkable. The proof that $F$ preserves linearity is symmetric.

  Let $P$ be a positive object of $\mathcal D$. For all morphisms $f$ from $F(P)$, we have that $f$ can be decomposed in $f \ccomp F(\omega^{-1}_P)$ and $F(\omega_P)$ as the latter is thunkable. $f \ccomp F(\omega^{-1}_P)$ is linear by positivity of $F(\nDownarrow P)$ and $F(\omega_P)$ is linear because $P$ is positive and $F$ preserves linearity. So, as the composition of two linear morphisms, $f$ is linear and $F(P)$ is positive. We can prove that $F$ preserves negativity dually.
\end{proof}



\subsubsection{Construction of a duploid from an adjunction}

Let $\mathcal A$ and $\mathcal B$ be two categories equipped with an adjunction:
\[
  \begin{tikzcd}
    \mathcal A \arrow[r, bend left, "L"] \arrow[r, phantom, "\bot"] & \arrow[l, bend left, "R"] \mathcal B
  \end{tikzcd}
\]

We will write $P$, $Q$, \dots\ the elements of $\mathcal A$ and $N$, $M$, \dots\ the elements of $\mathcal B$, as they will be respectively positive and negative objects in the constructed duploid.

We first define the notion of transverse morphisms from an object $P$ of $\mathcal A$ to an object $N$ of $\mathcal B$. We define $\mathcal O(P, N)$ to be $A(P, RN)$. This definition is biased towards $A$ but it is irrelevant as we have the following natural isomorphism:
\begin{equation} \label{adjunctioniso}
  \mathcal O(P,N) = \mathcal A(P, RN) \simeq \mathcal B(LP, N)
\end{equation}
We will note it $(-)^*$ and, by abuse of notation, we will also note its inverse $(-)^*$.

The magmoid $\mathcal D$ is defined the following way:
\begin{itemize}
  \item $|\mathcal D|$ is the disjoint union of $|\mathcal A|$ and $|\mathcal B|$,
  \item For any $A, B \in |\mathcal D|$, $\mathcal D(A, B) := \mathcal O(A^+,B^-)$ where:
    \[
      \begin{array}{cc}
        P^+ := P & P^- := LP\\
        N^+ := RN & N^- := N
      \end{array}
    \]
  \item Let $f \in \mathcal D(B,C)$ and $g \in \mathcal D(A,B)$.
    \begin{itemize}
      \item If $B \in |\mathcal A|$, then we use the composition of $\mathcal B$:
        \[
          \begin{array}{c}
            f^* \in \mathcal B(LB, C^-)\\
            g^* \in \mathcal B(LA^+, LB)\\
            f \bullet g := (f^* \circ^{\mathcal B} g^*)^* \in \mathcal O(A^+, C^-) = \mathcal D(A, C)
          \end{array}
        \]
      \item If $B \in |\mathcal B|$, then we use the composition of $\mathcal A$:
        \[
          \begin{array}{c}
            f \in \mathcal A(GB, GC^-)\\
            g \in \mathcal A(A^+, GB)\\
            f \circ g := f \circ^{\mathcal A} g \in \mathcal O(A^+, C^-) = \mathcal D(A,C)
          \end{array}
        \]
    \end{itemize}
  \item The identities are given by the identities of $\mathcal A$ and $\mathcal B$:
    \[
      \begin{array}{c}
        \mathsf{id}^{\mathcal D}_P := (\mathsf{id}^{\mathcal B}_{LP})^* \in \mathcal D(P, P)\\
        \mathsf{id}^{\mathcal D}_N := \mathsf{id}^{\mathcal B}_{RN} \in \mathcal D(N,N)
      \end{array}
    \]
  The identities are neutral by properties of the identities in $\mathcal A$ and $\mathcal B$.
\end{itemize}

\begin{lemma}
  Objects $P$ of $\mathcal A$ are positive in $\mathcal D$ and objects $N$ of $\mathcal B$ are negative in $\mathcal D$.
\end{lemma}
\begin{proof}
  Let $P$ be an object of $\mathcal A$. Let $f \in \mathcal D(P,C)$, $g\in \mathcal D(B,P)$ and $h\in \mathcal D(A,B)$.
  \begin{itemize}
    \item If $B$ is an object of $\mathcal A$, the two compositions are in $\mathcal B$ and by associativity in $\mathcal B$, $f \bullet g \bullet h$ associates.
    \item If $B$ is an object of $\mathcal B$, $f \bullet g \circ h$ associates by naturality of $(-)^*$.
  \end{itemize}
  So, $f$ is linear and, therefore, $P$ is positive. The proof that the objects of $B$ are negative is symmetric.
\end{proof}

\begin{corollary}
  $\mathcal D$ is a pre-duploid.
\end{corollary}

We define shifts on $\mathcal D$ the following way:

\[
  \begin{array}{c}
    \nUparrow A := A^- \quad \nDownarrow A := A^+\\
    \varphi_P := \mathsf{id}^{\mathcal A}_{RLP} \in \mathcal A(RLP,RLP) = \mathcal D(\nUparrow P, P) \\
    \varphi_P^{-1} := (\mathsf{id}^{\mathcal B}_{LP})^* \in \mathcal A(P,RLP) = \mathcal D(P,\nUparrow P) \\
    \omega_N := (\mathsf{id}^{\mathcal B}_{LRN})^* \in \mathcal A(RN, RLRN) = \mathcal D(N,\nDownarrow N)\\
    \omega_N^{-1} := \mathsf{id}^{\mathcal A}_{RN} \in \mathcal A(RN, RN) = \mathcal D(\nDownarrow N,N)
  \end{array}
\]
and $\varphi_N$ and $\omega_P$ are identities of $\mathcal D$. We can easily verify that $\varphi$ is a family of linear morphisms and $\omega$ is family of thunkable morphisms.

\begin{lemma}
  This construction defines a duploid for any adjunction.
\end{lemma}



\subsection{Linearly distributive duploids}\label{section/linearly-distributive-duploids}
We want to describe the structure inherited by a duploid associated to
an adjunction of the form~\eqref{equation/adjunction-LR} where both
categories $\Acategory$ and $\Bcategory$ come equipped with symmetric
monoidal structures noted $(\Acategory, \tensorialand,
\tensorialtrue)$ and $(\Bcategory, \tensorialor, \tensorialfalse)$,
generalising linearly-distributive categories~\cite{Cockett_1997}, in
the sense that there are four distributivity laws (or commutators)
\[
\begin{array}{ccc}
    ldistr^\tensorialand_{A_1,A_2,B} & \hspace{-.7em} :  \hspace{-.7em} & A_1 \tensorialand R(L(A_2) \tensorialor B) \to R(L(A_1 \tensorialand A_2) \tensorialor B)
        \vspace{.4em}
        \\
ldistr^\tensorialor_{A,B_1,B_2} &  \hspace{-.7em} :  \hspace{-.7em} & L(R(B_1 \tensorialor B_2) \tensorialand A) \to B_1 \tensorialor L(R(B_2) \tensorialand A)
        \vspace{.4em}
        \\
rdistr^\tensorialand_{A_1,A_2,B} &  \hspace{-.7em} :  \hspace{-.7em} & R(B \tensorialor L(A_1)) \tensorialand A_2 \to R(B \tensorialor L(A_1 \tensorialand A_2))
\vspace{.4em}
\\
rdistr^\tensorialor_{A,B_1,B_2} &  \hspace{-.7em} :  \hspace{-.7em} & L(A \tensorialand R(B_1 \tensorialor B_2)) \to L(A \tensorialand R(B_1)) \tensorialor B_2\\
  \end{array}
\]
introduced in~\cite{Mellies2017micrological} and assumed to make a number of coherence diagrams commute.
We note that the strengths for $\tensorialand$ and $\tensorialor$ can be deduced from the commutators.

When translating the commutators into the duploid framework, the four rules collapse into only two, as they were merely cases depending on the polarity of $A'$/$B$.
$ldistr^\tensorialand$ and $rdistr^\tensorialor$ become $\delta^l$ and $ldistr^\tensorialor$ and $rdistr^\tensorialand$ become $\delta^r$.

\begin{definition}
  A \define{linearly distributive duploid} $\mathcal D$ is a duploid equipped with a pair of 
  positive and negative symmetric monoidal structures related by two families of mappings:
  \[
    \begin{array}{ccc} 
      \delta^l_{A,B,C} & : & A \otimes (B \parr C) \to (A \otimes B) \parr C\\
      \delta^r_{A,B,C} & : & (A \parr B) \otimes C \to A \parr (B \otimes C)
    \end{array}
  \]
  natural for each component and that respects the usual coherence diagrams for a linearly distributive category.
\end{definition}

\begin{definition}\label{definition/thunkable-wrt-type}
  Let $\mathcal D$ be a linearly distributive duploid.
We say that a morphism $f \in \mathcal D(A \otimes B, C)$ is \define{linear wrt. $A$} (and we note it $f \in \mathcal D(\underline A \otimes B, C)$) when, for all $g \in \mathcal D(A',A)$ and $h \in \mathcal D(A'',A')$, we have
  \[
    f \pcomp ((g \ccomp h) \ltensortimes B) = f \pcomp (h \ltensortimes B) \pcomp (g \ltensortimes B).
  \]
  Dually, we say that a morphism $h$ from $A$ to $B \parr C$ is \define{thunkable wrt. $B$} (noted $h \in \mathcal D(A, \underline B \parr C)$) when, for all $g \in \mathcal D(B, B')$ and $f \in \mathcal D(B',B'')$, we have
  \[
    ((f \ccomp g) \lparrtimes C) \ncomp h = (f \lparrtimes C) \ncomp (g \lparrtimes C) \ncomp h.
  \]
\end{definition}

\begin{proposition}
  Let $f$ be a morphism from $A \otimes B$ to $C$. If $A$ is positive, then $f$ is linear wrt. $A$. Symmetrically, if $B$ is negative, then $g \in \Dduploid(A, B \parr C)$ is thunkable wrt. $B$.
\end{proposition}

\subsection{Notes about dialogue duploids}

\subsubsection{Coherence and naturality diagrams of dialogue duploids}

Here are the naturality and coherence conditions from the definition
of dialogue duploid spellt out in full.
\begin{equation}\label{chinaturality}
\begin{array}{ccc}
\begin{tikzcd}[column sep = .5em, row sep = 2em]
A'\arrow[rrrr, "{h_A}"] \arrow[rrdd, "\chi_{A',B,C}(f \pcomp (h_A \ltensortimes B))"'] 
&&&&
A \arrow[lldd, "\chi_{A,B,C}(f)"]
\\
\\
& & \dupldual{B} \parr C
\end{tikzcd}
&
\begin{tikzcd}[column sep = .5em, row sep = 1.2em]
&&
A 
\arrow[rrdd, "\chi_{A,B',C}(f \pcomp (A \rtensortimes h_B))"]
\arrow[lldd,"\chi_{A,B,C}(f)"{swap}]
&&
\\ 
\\
\dupldual{B} \parr C 
\arrow[rrrr,"\dupldual{h_B} \lparrtimes C"]
&&&&
\dupldual{B'} \parr C
\end{tikzcd}
&
\begin{tikzcd}[column sep = .5em, row sep = 1.2em]
 &&
A \arrow[lldd, "\chi_{A,B,C}(f)"{swap}]
 \arrow[rrdd, "\chi_{A,B,C'}(h_C \ccomp f)"] 
\\
\\
\dupldual{B} \parr C \arrow[rrrr, "\dupldual{B}\rparrtimes h_C"]
&&&&
\dupldual{B} \parr C'
\end{tikzcd}
\end{array}
\end{equation}
\begin{equation}\label{chicoherence}
    \begin{tikzcd}
      \mathcal D(A \otimes (B \otimes C), D) \arrow[rrrr, "\chi_{A,B \otimes C,D}"] \arrow[dd, "\text{associativity}"'] & & & & \mathcal D(A, \dupldual{(B \otimes C)} \parr D) \arrow[dd, "\begin{array}{c}\text{monoidality}\\\text{symmetry}\\\text{associativity}\end{array}"]\\ \\
      \mathcal D((A \otimes B)\otimes C, D) \arrow[rr, "\chi_{A\otimes B, C, D}"] & & \mathcal D(A \otimes B, \dupldual{C} \parr D) \arrow[rr, "\chi_{A,B,\dupldual{C} \parr D}"] & & \mathcal D(A, \dupldual{B} \parr (\dupldual{C} \parr D)) 
    \end{tikzcd}
  \end{equation}

\subsubsection{Dialogue duploid functors}

\begin{definition}
  A \define{dialogue duploid functor} $F : \mathcal D \to \mathcal D'$ is a duploid functor, lax monoidal for $\otimes$ and colax monoidal for $\parr$ (i.e.\ $F^\op$ is lax monoidal) and equipped with a family of natural central invertible morphisms $\widetilde F_A : F(\dupldual A) \simeq (FA)^\star$ such that the following coherence diagram commutes:
  \[
    \begin{tikzcd}
      \mathcal D(A \otimes B, C) \arrow[rr, "\chi_{A,B,C}"] \arrow[d, "F"'] & & \mathcal D(B, \dupldual A \parr C)\arrow[d, "F"]\\
      \mathcal D'(F(A \otimes B), FC) \arrow[dd, "\begin{array}{c}\text{monoidality}\\\text{($\tensor$) of }F\end{array}"'] & & \mathcal D'(FB, F(\dupldual A \parr C)) \arrow[d, "\begin{array}{c}\text{monoidality}\\\text{($\parr$) of }F\end{array}"]\\ 
                                                                            & & \mathcal D'(FB, F(\dupldual A) \parr' FC) \arrow[d, "\widetilde F_A"]\\\
      \mathcal D'(FA \otimes' FB, FC) \arrow[rr, "\chi'_{FA, FB, FC}"] & & \mathcal D'(FB, (FA)^\star \parr' FC)\\
    \end{tikzcd}
  \]
\end{definition}

\begin{definition}
  $Dia\dupl$ is the category whose objects are the dialogue duploids and whose morphisms are the dialogue duploid functors.
\end{definition}
\subsection{Detailed semantic proof of the \FH{} theorem}

\begin{lemma}\label{lemma/commutationSemantic}
  Let $f \in \mathcal D(B,C)$ and $g \in \mathcal D(A,B)$ be two morphisms of $\mathcal D$. We have:
  \[
    f \ccomp g = \nu_C \ccomp (\lambda'_{\dupldoubledual C} \circ \chi_{A,\dupldual C,\bot}(\chi^{-1}_{A,\dupldual B,\bot}((\nu^{-1}_B \lparrtimes \bot) \circ (\lambda'^{-1}_B \ccomp g)) \bullet (A \rtensortimes \dupldual f)))
  \]
\end{lemma}
\begin{proof}
  \begin{align*}
    f \ccomp g &= f \ccomp ((\lambda'_B \circ \lambda'^{-1}_B) \ccomp g)\\ 
    &= f \ccomp \lambda'_B \circ (\lambda'^{-1}_B \ccomp g) && \mbox{by linearity of }\lambda'\\ 
    &= (\nu_C \ccomp (\dupldoubledual f \ccomp \nu^{-1}_B)) \ccomp \lambda'_B \circ (\lambda'^{-1}_B \ccomp g) && \mbox{by naturality of }\nu\\ 
    &= \nu_C \ccomp ((\dupldoubledual f \ccomp \nu^{-1}_B) \ccomp \lambda'_B \circ (\lambda'^{-1}_B \ccomp g)) && \mbox{by linearity of }\nu\\ 
    &= \nu_C \ccomp (\lambda'_{\dupldoubledual C} \circ ((\dupldoubledual f \ccomp \nu^{-1}_B) \lparrtimes \bot) \circ (\lambda'^{-1}_B \ccomp g))\\ 
    &= \nu_C \ccomp (\lambda'_{\dupldoubledual C} \circ (\dupldoubledual f\lparrtimes \bot) \circ (\nu^{-1}_B\lparrtimes \bot) \circ (\lambda'^{-1}_B \ccomp g))\\ 
    &= \nu_C \ccomp (\lambda'_{\dupldoubledual C} \circ (\dupldoubledual f \lparrtimes \bot) \circ \chi_{A,\dupldual B,\bot}(\chi^{-1}_{A,\dupldual B,\bot}((\nu^{-1}_B \lparrtimes \bot) \circ (\lambda'^{-1}_B \ccomp g))))\\ 
    &= \nu_C \ccomp (\lambda'_{\dupldoubledual C} \circ \chi_{A,\dupldual C,\bot}(\chi^{-1}_{A,\dupldual B,\bot}((\nu^{-1}_B \lparrtimes \bot) \circ (\lambda'^{-1}_B \ccomp g)) \bullet (A \rtensortimes \dupldual f))) && \mbox{by \cref{chinaturality}}
  \end{align*}
\end{proof}

\begin{theorem}
  A morphism of $\mathcal D$ is thunkable if and only if it is central for $\otimes$.
\end{theorem}
\begin{proof}
  We know by definition that thunkable morphisms are central for $\otimes$, so we only have to prove that central morphisms are thunkable.

  Let $A,B,C,D \in|\mathcal D|$ and $f \in \mathcal D(C,D)$, $g \in \mathcal D(B,C)$ and $h \in \mathcal D(A,B)$ such that $h$ is central for $\otimes$.
  \begin{align*}
    & (f \ccomp g) \ccomp h \\
    &= (\nu_D \ccomp (\lambda'_{\dupldoubledual D} \circ \chi_{B,\dupldual D,\bot}(\chi^{-1}_{B,\dupldual C,\bot}((\nu^{-1}_C \lparrtimes \bot) \circ (\lambda'^{-1}_C \ccomp g)) \bullet (B \rtensortimes\dupldual f)))) \ccomp h && \mbox{by the previous lemma}\\ 
    &= \nu_D \ccomp ((\lambda'_{\dupldoubledual D} \circ \chi_{B,\dupldual D,\bot}(\chi^{-1}_{B,\dupldual C,\bot}((\nu^{-1}_C \lparrtimes \bot) \circ (\lambda'^{-1}_C \ccomp g)) \bullet (B \rtensortimes \dupldual f))) \ccomp h) && \mbox{by linearity of }\lambda'\\ 
    &= \nu_D \ccomp (\lambda'_{\dupldoubledual D} \circ (\chi_{B,\dupldual D,\bot}(\chi^{-1}_{B,\dupldual C,\bot}((\nu^{-1}_C \lparrtimes \bot) \circ (\lambda'^{-1}_C \ccomp g)) \bullet (B \rtensortimes\dupldual f)) \ccomp h))  && \mbox{by linearity of }\lambda'\\ 
    &= \nu_D \ccomp (\lambda'_{\dupldoubledual D} \circ (\chi_{B,\dupldual D,\bot}(\chi^{-1}_{B,\dupldual C,\bot}((\nu^{-1}_C \lparrtimes \bot) \circ (\lambda'^{-1}_C \ccomp g)) \bullet (B \rtensortimes \dupldual f) \bullet (h \ltensortimes \dupldual D))))  && \mbox{by \cref{chinaturality}}\\ 
    &= \nu_D \ccomp (\lambda'_{\dupldoubledual D} \circ (\chi_{B,\dupldual D,\bot}(\chi^{-1}_{B,\dupldual C,\bot}((\nu^{-1}_C \lparrtimes \bot) \circ (\lambda'^{-1}_C \ccomp g)) \bullet (h \ltensortimes \dupldual D) \bullet (B \rtensortimes \dupldual f)))) && \mbox{by centrality of }h\\ 
    &= \nu_D \ccomp (\lambda'_{\dupldoubledual D} \circ (\chi_{B,\dupldual D,\bot}(\chi^{-1}_{B,\dupldual C,\bot}(((\nu^{-1}_C \lparrtimes \bot) \circ (\lambda'^{-1}_C \ccomp g)) \ccomp h) \bullet (B \rtensortimes \dupldual f)))) && \mbox{by \cref{chinaturality}}\\ 
    &= \nu_D \ccomp (\lambda'_{\dupldoubledual D} \circ (\chi_{B,\dupldual D,\bot}(\chi^{-1}_{B,\dupldual C,\bot}((\nu^{-1}_C \lparrtimes \bot) \circ ((\lambda'^{-1}_C \ccomp g) \ccomp h)) \bullet (B \rtensortimes \dupldual f)))) && \mbox{by lin. of }\nu^{-1} \mbox{ preserved by } \lparrtimes\\ 
    &= \nu_D \ccomp (\lambda'_{\dupldoubledual D} \circ (\chi_{B,\dupldual D,\bot}(\chi^{-1}_{B,\dupldual C,\bot}((\nu^{-1}_C \lparrtimes \bot) \circ (\lambda'^{-1}_C \ccomp (g \ccomp h))) \bullet (B \rtensortimes \dupldual f)))) && \mbox{by linearity of }\lambda'^{-1}\\ 
    &= f \ccomp (g \ccomp h)  && \mbox{by the previous lemma}
  \end{align*}
  So $h$ is thunkable. This concludes the proof.
\end{proof}



\subsection{Interpretation of the syntax}

This section and the one that follows uses and adapts to the classical
case the technique used in \cite{munchlcalculi} which is detailed in
\cite{Munch-Maccagnoni2017curry}.

A context on the left $(a_1 : A_1, a_2 : A_2,\dots,a_n : A_n)$ is interpreted as $A_1 \otimes A_2 \otimes \dots \otimes A_n$ and a context on the right $(\beta_1 : B_1, \beta_2 : B_2, \dots, \beta_n : B_n)$ is interpreted as $B_1 \parr B_2 \parr \dots \parr B_n$.

Let $\Gamma$ and $\Gamma'$ be two contexts on the left and $\sigma$ an element of $\Sigma(\Gamma,\Gamma')$. We note $\llbracket \sigma \rrbracket$ the associated canonical isomorphism of $\Dduploid(\Gamma,\Gamma')$ constructed by composing symmmetries of $\otimes$. We stress on the fact that, as a composition of thunkable morphisms, $\llbracket \sigma \rrbracket$ is thunkable.
Dually, let $\Delta$ and $\Delta'$ be two contexts on the right and $\tsigma$ an element of $\Sigma(\Delta',\Delta)$. The associated canonical isomorphism of $\Dduploid(\Delta', \Delta)$ obtained by composing symmetries of $\parr$ is noted $\llbracket\tsigma \rrbracket$ and is linear. 

\subsubsection{Interpretation of judgments}

\begin{itemize}
  \setlength{\itemsep}{4pt}
  \item $\llbracket \Gamma \vdash t : A\ |\ \Delta\rrbracket \in \mathcal D(\Gamma, A \parr \Delta)$
  \item $\llbracket \Gamma \vdash V : A\ |\ \Delta\rrbracket \in \mathcal D(\Gamma, \underline A \parr \Delta)$
  \item $\llbracket \Gamma\ |\ e : A \vdash \Delta\rrbracket \in \mathcal D(\Gamma \otimes A, \Delta)$
  \item $\llbracket \Gamma\ |\ S : A \vdash \Delta\rrbracket \in \mathcal D(\Gamma \otimes \underline A, \Delta)$
  \item $\llbracket \command c : (\Gamma \vdash \Delta)\rrbracket \in \mathcal D(\Gamma, \Delta)$
\end{itemize}
See \cref{definition/thunkable-wrt-type} for the meaning of the notation $\underline A$.
\subsubsection{Interpretation of typing rules}

\paragraph{Identity rules}
\begin{itemize}
  \setlength{\itemsep}{4pt}
  \item $\llbracket a : A \vdash a : A\ |\rrbracket = \mathsf{id}_{A} \in \mathcal D_t(A,A)$
  \item $\llbracket |\ \alpha : A \vdash \alpha : A\rrbracket = \mathsf{id}_{A} \in \mathcal D_l(A,A)$
  \item $\llbracket \Gamma\ |\ \tmu a^\veps.\command c : A_\veps\vdash \Delta\rrbracket = \llbracket \command c : (\Gamma, x:A_\veps\vdash\Delta) \rrbracket \in \mathcal D(\Gamma \otimes A_\veps, \Delta)$
  \item $\llbracket \Gamma\vdash \mu \alpha^\veps.\command c : A_\veps\ |\ \Delta\rrbracket = \llbracket \command c : (\Gamma \vdash\alpha:A_\veps,\Delta) \rrbracket \in \mathcal D(\Gamma, A_\veps \parr \Delta)$
  \item $\begin{aligned}[t]
\llbracket \perfectcut{t}{e}^\veps : (\Gamma, \Gamma' \vdash \Delta, \Delta')\rrbracket &= (\llbracket \Gamma\ |\ e : A \vdash \Delta\rrbracket \lparrtimes \Delta') \ncomp (\delta^l_{\Gamma,A,\Delta'} \pcomp (\Gamma \rtensortimes \llbracket \Gamma' \vdash t : A\ |\ \Delta'\rrbracket)) \\
& \in \mathcal D(\Gamma \otimes \Gamma', \Delta \parr \Delta')
\end{aligned}$\\
where $\delta^l_{\Gamma,A,\Delta'} : \Gamma \otimes (A \parr \Delta') \to (\Gamma \otimes A) \parr \Delta'$ is the distributor.
\end{itemize}

\paragraph{Structural rules}

$\forall\sigma\in\Sigma(\Gamma',\Gamma),\ \forall\tsigma\in\Sigma(\Delta,\Delta')$
\begin{itemize}
  \setlength{\itemsep}{4pt}
  \item $\llbracket \Gamma'\vdash t[\sigma,\tsigma]: A\ |\ \Delta'\rrbracket = (A \rparrtimes \llbracket \tsigma \rrbracket) \ncomp (\llbracket \Gamma \vdash t : A\ |\ \Delta\rrbracket \ccomp \llbracket \sigma\rrbracket) \in \Dduploid(\Gamma', A \parr \Delta')$
  \item $\llbracket \Gamma'\ |\ e[\sigma,\tsigma] : A\vdash\Delta'\rrbracket = \llbracket \tsigma \rrbracket \ccomp (\llbracket \Gamma\ |\ e : A\vdash\Delta \rrbracket \pcomp (\llbracket \sigma\rrbracket \ltensortimes A)) \in \Dduploid(\Gamma' \tensor A, \Delta')$
  \item $\llbracket c[\sigma,\tsigma]:(\Gamma'\vdash \Delta')\rrbracket = \llbracket \tsigma \rrbracket \ccomp (\llbracket \command c:(\Gamma \vdash \Delta) \rrbracket \ccomp \llbracket \sigma\rrbracket) \in \Dduploid(\Gamma', \Delta')$
\end{itemize}

\paragraph{Conjunction rules}
\begin{itemize}
  \setlength{\itemsep}{4pt}
  \item $\llbracket \vdash () : \1\ |\rrbracket = \mathsf{id}_\1 \in \mathcal D_t(\1,\1)$
  \item $\llbracket \Gamma\ |\ \tmu ().\command c\vdash \Delta \rrbracket = \llbracket \command c : (\Gamma\vdash\Delta)\rrbracket \ccomp \rho_\Gamma \in \mathcal D(\Gamma \otimes \underline{\1},\Delta)$\\
    where $\rho_\Gamma : \Gamma \otimes 1 \to \Gamma$ is the right unitor of $\otimes$.
  \item $\begin{aligned}[t]
&\llbracket \Gamma,\Gamma'\vdash V \otimes W : A \otimes B\ |\ \Delta, \Delta'\rrbracket \\
&= ((((\sigma_{B,A} \lparrtimes \Delta) \ncomp \delta^l_{B,A,\Delta}) \pcomp \sigma_{A\parr \Delta, B})\lparrtimes \Delta')\ncomp\delta^l_{A\parr \Delta, B, \Delta'}) \pcomp (\llbracket \Gamma\vdash V : A\ |\ \Delta\rrbracket \otimes \llbracket \Gamma'\vdash W : B\ |\ \Delta'\rrbracket) \\
&\in \mathcal D(\Gamma \otimes \Gamma',\underline{(A \otimes B)} \parr \Delta \parr \Delta')
\end{aligned}$
  \item $\llbracket \Gamma\ |\ \tmu(a \otimes b).\command c : A\otimes B\vdash\Delta \rrbracket = \llbracket \Gamma, a : A, b : B\vdash\Delta\rrbracket \in \mathcal D(\Gamma \otimes \underline{A \otimes B},\Delta)$
\end{itemize}

\paragraph{Disjunction rules}
\begin{itemize}
  \setlength{\itemsep}{4pt}
  \item $\llbracket |\ [] : \bot \vdash \rrbracket = \mathsf{id}_\bot \in \mathcal D_l(\bot,\bot)$
  \item $\llbracket \Gamma \vdash \mu [].\command c\ |\ \Gamma \rrbracket = \lambda'_\Delta \ccomp \llbracket \command c : (\Gamma\vdash\Delta)\rrbracket \in \mathcal D(\Gamma,\underline\bot \parr \Delta)$\\
    where $\lambda'_\Delta : \Delta \to \bot\parr\Delta$ is the left unitor of $\parr$.
  \item $\begin{aligned}[t]
&\llbracket \Gamma,\Gamma' \ |\ S \parr S' : A \parr B \vdash\Delta, \Delta'\rrbracket\\
&= \llbracket \Gamma\ |\ S : A \vdash \Delta\rrbracket \parr_l\llbracket \Gamma'\ |\ S' : B\vdash \Delta'\rrbracket \circ (\delta^l_{\Gamma,A,(\Gamma'\otimes B)} \bullet (\Gamma \otimes (\sigma'_{(\Gamma'\otimes B), A} \circ \delta^l_{\Gamma',B,A})) \bullet (\Gamma \otimes \Gamma' \otimes \sigma'_{A,B}))\\
&\in \mathcal D(\Gamma \otimes \Gamma' \otimes \underline{(A \parr B)}, \Delta \parr \Delta')
\end{aligned}$
  \item $\llbracket \Gamma \vdash \mu(\alpha \parr \beta).\command c : A\parr B\ |\ \Delta \rrbracket = \llbracket \command c:(\Gamma\vdash \alpha : A, \beta : B, \Delta) \rrbracket \in \mathcal D(\Gamma, \underline{A \parr B} \parr \Delta)$
\end{itemize}

\paragraph{Negation rules}
\begin{itemize}
  \setlength{\itemsep}{4pt}
  \item $\llbracket \Gamma \vdash [S] : \dupldual N\ |\ \Delta\rrbracket = \chi_{\Gamma,N,\Delta}(\llbracket \Gamma\ |\ S : N \vdash \Delta \rrbracket) \in \mathcal D(\Gamma, \underline{\dupldual N} \parr \Delta)$
  \item $\llbracket \Gamma\ |\ [V] : \dupldual P \vdash \Delta\rrbracket = \chi^{-1}_{\Gamma,\dupldual P,\Delta}((\nu_P \lparrtimes \Delta) \ncomp \llbracket\Gamma \vdash V : P\ |\ \Delta\rrbracket) \in \mathcal D(\Gamma \otimes \underline{\dupldual P},\Delta)$
  \item $\llbracket \Gamma\ |\ \tmu[\alpha].\command c : \dupldual N \vdash \Delta\rrbracket = \chi^{-1}_{\Gamma,\dupldual N,\Delta}((\nu_N \lparrtimes \Delta) \ncomp \llbracket \command c : (\Gamma\vdash \alpha : N, \Delta)\rrbracket) \in \mathcal D(\Gamma \otimes \underline{\dupldual N}, \Delta) $
  \item $\llbracket \Gamma\vdash \mu[a].\command c : \dupldual P\ |\ \Delta\rrbracket = \chi_{\Gamma,P,\Delta}(\llbracket \command c : (\Gamma, a : P \vdash \Delta)\rrbracket) \in \mathcal D(\Gamma, \underline{\dupldual P} \parr \Delta)$
\end{itemize}
where $\nu_A : A \to \dupldoubledual A$.
\subsection{Soundness of the interpretation}\label{cohsoundcomp}

We follow again \cite{Munch-Maccagnoni2017curry} which we adapt to
classical logic with an involutive negation.
We start by proving coherence properties of the interpretation.
We say that two derivations are \define{equivalent} if their
interpretation are equal in all dialogue duploids.

\begin{lemma}\label{exaonestruct}
  For any typing derivations, there is an equivalent derivation starting by one structural rule.
\end{lemma}
\begin{proof}
  We treat the case of a typing derivation of $\Gamma \vdash t : A\ |\ \Delta$;
the other cases are similar. We look at the smallest equivalent
  typing derivation of $\Gamma \vdash t :A\ |\ \Delta$ in terms of
  number of rules used. If it starts by two structural rules
  $\tau,\tilde\tau$ and $\sigma, \tsigma$, then the derivation where
  the two first rules are replaced by the structural rule
  $\tau\circ\sigma, \tsigma\circ\tilde\tau$ is equivalent and uses
  strictly less rules, which is impossible by hypothesis.
So we have a derivation starting with at most one structural rule.
  If there is none, we can always add the structural rule of the
  identity, which is interpreted as the identity.
\end{proof}



For a term $\mathfrak g$, we will note $\fvwo \mathfrak g$ the set of free variables of $\mathfrak g$ and $\fcvwo \mathfrak g$ the set of free co-variables of $\mathfrak g$. For $\Gamma$ a context and $X$ a subset of the domain of $\Gamma$, we will note the restriction of $\Gamma$ to $X$ as $\Gamma_{\upharpoonright X}$.

\begin{lemma}
  For any derivation $\command c: (\Gamma\vdash\Delta)$, one has $\fvwo \command c = \dom \Gamma$ and $\fcvwo \command c = \dom \Delta$ and similarly for $t$ and $e$ replacing $\command c$.
\end{lemma}
\begin{proof}
  By induction on the derivation.
\end{proof}







We prove a \emph{coherent generation lemma} which says that, from the
form of the term, we can deduce the first rules of a derivation, or,
at least, find an equivalent derivation starting by those rules.

\begin{lemma}\label{cohgen}

$(\vdash \mathbf{ax})$ : Any derivation of $\Gamma\vdash x : A\ |\ \Delta$ satisfies $\Gamma = (x : A)$ and $\Delta = \emptyset$ and is equivalent to the derivation:
\[
  \begin{prooftree}
    \infer0[$(\vdash\mathbf{ax})$]{x:A \vdash x : A\ |\ }
  \end{prooftree}
\]

\medskip

$(\mathbf{cut}^\veps)$ : For any derivation of $\perfectcut{t}{e}^\veps : (\Gamma\vdash\Delta)$, there exists $A^\veps$ and an equivalent derivation ending with:
\[
  \begin{prooftree}
    \hypo{\Gamma_{\fv e}\ |\ e : A^\veps \vdash\Delta_{\fcv e}}
    \hypo{\Gamma_{\fv t}\vdash t : A^\veps \ |\ \Delta_{\fcv t}}
    \infer2[$(\mathbf{cut}^\veps)$]{\perfectcut{t}{e}^\veps : (\Gamma_{\fv e},\Gamma_{\fv t}\vdash\Delta_{\fcv e},\Delta_{\fcv t})}
    \infer1[$(\sigma,\tsigma)$]{\perfectcut{t}{e}^\veps : (\Gamma\vdash\Delta)}
  \end{prooftree}
\]
where $\sigma \in \Sigma(\Gamma,(\Gamma_{\fv t},\Gamma_{\fv e}))$ is the unique permutation without renaming from $\Gamma$ to $(\Gamma_{\fv t},\Gamma_{\fv e})$ and $\tsigma \in \Sigma((\Delta_{\fcv t},\Delta_{\fcv e}),\Delta)$ is the unique permutation without renaming from $(\Delta_{\fcv t},\Delta_{\fcv e})$ to $\Delta$.

\medskip

$(\vdash\negN)$ : For any derivation of $\Gamma\vdash [S] : A\ |\ \Delta$, one has $A$ of the form $\dupldual N$ and an equivalent derivation ending with:
\[
  \begin{prooftree}
    \hypo{\Gamma\ |\ S : N\vdash \Delta}
    \infer1[$(\vdash\negN)$]{\Gamma\vdash [S] : \dupldual N\ |\ \Delta}
  \end{prooftree}
\]

The other cases are similar.
\end{lemma}

\begin{proof}
  $(\vdash \mathsf{ax})$ By using the previous lemma, we have that $\dom\Gamma = \{A\}$ and $\Delta$ is empty. We know from \cref{exaonestruct} that we can assume that it starts with one structural rule but it's a renaming which is interpreted as the identity. Finally, the only non-structural rule that can be applied to $x : A\vdash x : A\ |$ is $(\vdash \mathsf{ax})$.

  $(\vdash \mathsf{cut}^\veps)$ From the previous lemma, we know that $\dom\Gamma = \fvwo \perfectcut{t}{e}^\veps = \fvwo t \uplus \fvwo e$, so $\sigma$ is well defined. We can say the same about $\Delta$ and $\tsigma$. By using \cref{exaonestruct} and the fact that there is only one non-structural rule that can be applied to $\perfectcut{t}{e}^\veps$, we have a type $A^\veps$ and a derivation of $\perfectcut{t}{e}^\veps : (\Gamma \vdash \Delta)$ of the form:
  \[
    \begin{prooftree}
      \hypo{\Gamma_1\ |\ e[\tau,\tilde\tau] : A^\veps\vdash \Delta_1}
      \hypo{\Gamma_2 \vdash t[\tau,\tilde\tau] : A^\veps\ |\ \Delta_2}
      \infer2[$(\mathsf{cut}^\veps)$]{\perfectcut{t[\tau,\tilde\tau]}{e[\tau,\tilde\tau]}^\veps : (\Gamma_1, \Gamma_2 \vdash \Delta_1, \Delta_2)}
      \infer1[$(\tau,\tilde\tau)$]{\perfectcut{t}{e}^\veps : (\Gamma \vdash \Delta)}
    \end{prooftree}
  \]
  We can add the structural rules $\sigma,\tsigma$ and $\sigma^{-1},\tsigma^{-1}$ and, by centrality of symmetries and by coherence between symmetries and distributors, we can commute the cut rule and the structural rules to obtain the following equivalent derivation:
  \[
    \begin{prooftree}
      \hypo{\Gamma_1\ |\ e[\tau,\tilde\tau] : A^\veps\vdash \Delta_1}
      \infer1[$(\sigma^{-1}\circ\tau,\tilde\tau\circ\sigma^{-1})$]{\Gamma_{\fv e}\ |\ e : A^\veps \vdash\Delta_{\fcv e}}
      \hypo{\Gamma_2 \vdash t[\tau,\tilde\tau] : A^\veps\ |\ \Delta_2}
      \infer1[$(\sigma^{-1}\circ\tau,\tilde\tau\circ\sigma^{-1})$]{\Gamma_{\fv t}\vdash t : A^\veps \ |\ \Delta_{\fcv t}}
      \infer2[$(\mathsf{cut}^\veps)$]{\perfectcut{t}{e}^\veps : (\Gamma_{\fv e},\Gamma_{\fv t}\vdash\Delta_{\fcv e},\Delta_{\fcv t})} 
      \infer1[$(\sigma,\tilde\sigma)$]{\perfectcut{t}{e}^\veps : (\Gamma \vdash \Delta)}
    \end{prooftree}
  \]

  $(\vdash\negN)$ : By using \cref{exaonestruct} and the fact that
  only the rule $(\vdash\negN)$ can be applied, we have a negative
  type $N$ and a derivation of the form:
  \[
    \begin{prooftree}
      \hypo{\Gamma'\ |\ S[\tau,\tilde\tau] : N\vdash \Delta'}
      \infer1[$(\vdash\negN)$]{\Gamma'\vdash [S[\tau,\tilde\tau]] : \dupldual N\ |\ \Delta'}
      \infer1[$(\tau,\tilde\tau)$]{\Gamma\vdash [S] : \dupldual N\ |\ \Delta}
    \end{prooftree}
  \]
  We can commute the negation and the structural rule by naturality
  component-wise of $\chi$ and we obtain the equivalent derivation we
  seek:
  \[
    \begin{prooftree}
      \hypo{\Gamma'\ |\ S[\tau,\tilde\tau] : N\vdash \Delta'}
      \infer1[$(\tau,\tilde\tau)$]{\Gamma\ |\ S : N\vdash \Delta}
      \infer1[$(\vdash\negN)$]{\Gamma\vdash [S] : \dupldual N\ |\ \Delta}
    \end{prooftree}
  \]
  The other cases are similar and rely on the two previous lemmas and
  the coherence between the operations we are using.
\end{proof}

\noindent Thanks to the previous lemma, we can now reason on
derivations up to equivalence by doing an induction on the structure
of the term.





\begin{lemma}
  We consider a derivation of $\Gamma \vdash V : A\ |\ \Delta$ and its interpretation $\llbracket V\rrbracket \in \Dduploid(\Gamma,\underline A \parr \Delta)$.
  \begin{itemize}
    \item For any derivation of $\command c : (\Gamma',a:A\vdash\Delta')$, there exists a derivation of $\command c[V/a] : (\Gamma',\Gamma\vdash\Delta',\Delta)$ such that:
      \[
        \llbracket c[V/a]\rrbracket = (\llbracket c\rrbracket \lparrtimes \Delta) \ncomp (\delta^l_{\Gamma',A,\Delta} \pcomp (\Gamma' \rtensortimes \llbracket V\rrbracket))
      \]
    \item For any derivation of $\Gamma', a:A\vdash t : B\ |\ \Delta'$, there exists a derivation of $\Gamma',\Gamma\vdash t[V/a] : B\ |\ \Delta',\Delta$ such that:
      \[
        \llbracket t[V/a]\rrbracket = (\llbracket t\rrbracket \lparrtimes \Delta) \ncomp (\delta^l_{\Gamma',A,\Delta} \pcomp (\Gamma' \rtensortimes \llbracket t\rrbracket))
      \]
    \item For any derivation of $\Gamma',a:A\ |\ e : B\vdash \Delta'$, there exists a derivation of $\Gamma',\Gamma\ |\ e[V/a] : B\vdash \Delta',\Delta$ such that:
      \[
        \llbracket e[V/x]\rrbracket = ((\llbracket e\rrbracket \pcomp (\Gamma' \rtensortimes \sigma^{-1}_{A,B})) \lparrtimes \Delta) \ncomp (\delta^l_{\Gamma'\otimes B,A,\Delta} \pcomp ((\Gamma' \otimes B) \rtensortimes \llbracket V\rrbracket) \pcomp (\Gamma' \rtensortimes \sigma_{\Gamma,B}))
      \]
  \end{itemize}
\end{lemma}
\begin{proof}
  We reason by induction on $\command c$, $t$, $e$ by using \cref{cohgen}.
In the case where the last rule used is $(\mathbf{cut}^\veps)$ and $\command c$ is of the form $\perfectcut{t}{e}^\veps$ with derivations $\Gamma'_{\fv t} \vdash t : B\ |\ \Delta'_{\fcv t}$ and $\Gamma'_{\fv e}\ |\ e : B \vdash \Delta'_{\fcv e}$:
If $a \in \fvwo t$, by induction, we know that we have a derivation of $\Gamma'_{\fv t}, \Gamma \vdash t[V/a] : B\ |\ \Delta'_{\fcv t}, \Delta$ and that :
  \[
    \llbracket t[V/a]\rrbracket = (\llbracket t\rrbracket \lparrtimes \Delta) \ncomp (\delta^l_{\Gamma'_{\fv t},A,\Delta} \pcomp (\Gamma'_{\fv t} \rtensortimes \llbracket V\rrbracket))
  \]
  So,
  \begin{align*}
    & \llbracket \perfectcut{t}{e}^\veps[V/a]\rrbracket \\
     &= \llbracket\tsigma\rrbracket \ncomp (((\llbracket e\rrbracket \lparrtimes (\Delta'_{\fcv t}\parr \Delta)) \ncomp (\delta^l_{\Gamma'_{\fv e},B,\Delta'_{\fcv t}\parr \Delta} \pcomp (\Gamma'_{\fv e} \rtensortimes \llbracket t[V/a]\rrbracket))) \pcomp \llbracket\sigma\rrbracket)\\ 
                                                      &= \llbracket\tsigma\rrbracket \ncomp (((\llbracket e\rrbracket \lparrtimes (\Delta'_{\fcv t}\parr \Delta)) \ncomp (\delta^l_{\Gamma'_{\fv e},B,\Delta'_{\fcv t}\parr \Delta} \pcomp (\Gamma'_{\fv e} \rtensortimes ((\llbracket t\rrbracket \lparrtimes \Delta) \ncomp (\delta^l_{\Gamma'_{\fv t},A,\Delta} \pcomp (\Gamma'_{\fv t} \rtensortimes \llbracket V\rrbracket)))))) \pcomp \llbracket\sigma\rrbracket)\\
                                                      &= \llbracket\tsigma\rrbracket \ncomp ((((\llbracket e\rrbracket \lparrtimes \Delta'_{\fcv t}) \ncomp (\delta^l_{\Gamma'_{\fv e},B,\Delta'_{\fcv t}} \pcomp (\Gamma'_{\fv e} \rtensortimes \llbracket t\rrbracket))) \lparrtimes \Delta) \ncomp (\delta^l_{(\Gamma'_{\fv e},\Gamma'_{\fv t}),A,\Delta} \pcomp ((\Gamma'_{\fv e},\Gamma'_{\fv t}) \rtensortimes \llbracket V\rrbracket) \pcomp \llbracket\sigma\rrbracket))\\
                                                      & \hspace{4em}\mbox{by thunkability of }V\mbox{ and }\delta^l\\
                                                      &= \llbracket\tsigma\rrbracket \ncomp ((((\llbracket e\rrbracket \lparrtimes \Delta'_{\fcv t}) \ncomp (\delta^l_{\Gamma'_{\fv e},B,\Delta'_{\fcv t}} \pcomp (\Gamma'_{\fv e} \rtensortimes \llbracket t\rrbracket))) \lparrtimes \Delta) \ncomp ((\llbracket\sigma'\rrbracket \lparrtimes \Delta) \ncomp (\delta^l_{\Gamma',A,\Delta} \pcomp (\Gamma' \rtensortimes \llbracket V\rrbracket))))\\
                                                      &\hspace{4em}\mbox{by centrality and compatibility with the distributor of symmmetries}\\
                                                      &= (\llbracket\tsigma'\rrbracket \ncomp ((\llbracket e\rrbracket \lparrtimes \Delta'_{\fcv t}) \ncomp (\delta^l_{\Gamma'_{\fv e},B,\Delta'_{\fcv t}} \pcomp (\Gamma'_{\fv e} \rtensortimes \llbracket t\rrbracket))) \lparrtimes \Delta) \ncomp ((\llbracket\sigma'\rrbracket \lparrtimes \Delta) \ncomp (\delta^l_{\Gamma',A,\Delta} \pcomp (\Gamma' \rtensortimes \llbracket V\rrbracket)))\\
                                                      & \hspace{4em}\mbox{by linearity of }\llbracket \tsigma\rrbracket \\
                                                      &= (\llbracket \perfectcut{t}{e}^\veps \rrbracket \lparrtimes \Delta) \ncomp (\delta^l_{\Gamma',A,\Delta} \pcomp (\Gamma' \rtensortimes \llbracket V\rrbracket))
  \end{align*}
  where:
\begin{align*}
&\sigma \in \Sigma((\Gamma',\Gamma), (\Gamma'_{\fv e},\Gamma'_{\fv t\setminus\{x\}},\Gamma))\\
&\sigma' \in \Sigma((\Gamma',x:A), (\Gamma'_{\fv e},\Gamma'_{\fv t\setminus\{x\}},x:A))\\
&\tsigma\in \Sigma((\Delta'_{\fcv e},\Delta'_{\fcv t},\Delta),(\Delta', \Delta))\\
&\tsigma'\in \Sigma((\Delta'_{\fcv e},\Delta'_{\fcv t}),\Delta')
\end{align*}
  The other cases are also straightforward, by using induction and the
  compatibility of the operations we are using.
\end{proof}



The following lemma is exactly the symmetric of the previous and is proved accordingly.
\begin{lemma}\label{soundstack}
  Let a derivation of $\Gamma\ |\ S : A\vdash \Delta$. We consider $\llbracket S\rrbracket \in \Dduploid(\Gamma \otimes A, \Delta)$ its interpretation.
  \begin{itemize}
    \item For any derivation of $\command c : (\Gamma'\vdash\alpha:A,\Delta')$, there exists a derivation of $\command c[S/\alpha] : (\Gamma,\Gamma'\vdash\Delta,\Delta')$ such that:
      \[
        \llbracket c[S/\alpha]\rrbracket = (\llbracket S\rrbracket \lparrtimes \Delta') \ncomp (\delta^l_{\Gamma,A,\Delta'} \pcomp (\Gamma \rtensortimes \llbracket c\rrbracket))
      \]
    \item For any derivation of $\Gamma'\vdash t : B\ |\ \alpha:A,\Delta'$, there exists a derivation of $\Gamma,\Gamma'\vdash t[S/\alpha] : B\ |\ \Delta,\Delta'$ such that:
      \[
        \llbracket t[S/\alpha]\rrbracket = ((\sigma'_{B, \Delta} \lparrtimes \Delta') \ncomp (\llbracket S\rrbracket \lparrtimes (B \parr \Delta')) \ncomp (\delta^l_{\Gamma,A,B \parr\Delta'} \pcomp (\Gamma \rtensortimes ((\sigma'_{A,B} \lparrtimes \Delta') \llbracket t\rrbracket))
      \]
    \item For any derivation of $\Gamma'\ |\ e : B\vdash \alpha:A,\Delta'$, there exists a derivation of $\Gamma,\Gamma'\ |\ e[S/\alpha] : B\vdash \Delta,\Delta'$ such that:
      \[
        \llbracket e[S/\alpha]\rrbracket = (\llbracket S\rrbracket \lparrtimes \Delta') \ncomp (\delta^l_{\Gamma,A,\Delta'} \pcomp (\Gamma \rtensortimes \llbracket e\rrbracket))
      \]
  \end{itemize}
\end{lemma}



\noindent We now prove the \emph{sound subject reduction} lemma.

\begin{lemma}
  $\triangleright_{RE}$ preserves typing, and, when restricted to typed terms, $\triangleright_{RE}$ preserves the interpretation.
\end{lemma}

\begin{proof}
  We reason by case analysis. We will treat in details the case of $(R\negN)$ and $(E\negN)$.

  $(R\negN)$ For any $\command c = \perfectcut{[S]}{\tmu[\alpha].\command c'}^+ \triangleright_R \command c'[S/\alpha]$ and derivation of $\command c :(\Gamma \vdash \Delta)$, by applying \cref{cohgen}, we have a negative type $N$ and an equivalent derivation of the form:
  \[
    \begin{prooftree}
      \hypo{\command c' : (\Gamma_{\fv \command c'}\vdash\alpha : N,\Delta_{\fcv \command c'\setminus\{\alpha\}})} 
      \infer1[$(\negN\vdash)$]{\Gamma_{\fv \command c'}\ |\ \tmu[\alpha].\command c' : \dupldual N \vdash \Delta_{\fcv \command c'\setminus\{\alpha\}}}
      \hypo{\Gamma_{\fv S}\ |\ S : N\vdash \Delta_{\fcv S}}
      \infer1[$(\vdash\negN)$]{\Gamma_{\fv S} \vdash [S] :\dupldual N\ |\ \Delta_{\fcv S}}
      \infer2[$(\mathsf{cut}^+)$]{\perfectcut{[S]}{\tmu[\alpha].\command c'}^+:(\Gamma_{\fv \command c'},\Gamma_{\fv S}\vdash\Delta_{\fcv \command c'\setminus\{\alpha\}},\Delta_{\fcv S})}
      \infer1[$(\sigma,\tsigma)$]{\perfectcut{[S]}{\tmu[\alpha].\command c'}^+:(\Gamma\vdash\Delta)}
    \end{prooftree}
  \] 
  where $\sigma \in \Sigma(\Gamma,(\Gamma_{\fv S},\Gamma_{\fv \command c}))$ is the unique permutation without renaming from $\Gamma$ to $(\Gamma_{\fv S},\Gamma_{\fv \command c})$ and $\tsigma \in \Sigma((\Delta_{\fcv S},\Delta_{\fcv \command c\setminus\{\alpha\}}),\Delta)$ is the unique permutation without renaming from $(\Delta_{\fcv S},\Delta_{\fcv \command c\setminus\{\alpha\}})$ to $\Delta$.
  
  So, by the previous lemma, we have a derivation of $\command c'[S/\alpha] : (\Gamma\vdash\Delta)$. Moreover, one has:
  \[
    \llbracket\perfectcut{[S]}{\tmu[\alpha].\command c'}^+:(\Gamma\vdash\Delta)\rrbracket = \llbracket \command c'|S/\alpha]:(\Gamma\vdash\Delta)\rrbracket
  \]
  The proof goes along the lines of the proof of \cref{lemma/commutationSemantic}.


  $(E\negN)$ For any $\tmu[\alpha].\perfectcut{[\alpha]}{S}^+ \triangleright_R S$ and derivation of $\Gamma\ |\ \tmu[\alpha].\perfectcut{[\alpha]}{S'}^+ : \dupldual N \vdash \Delta$, by applying \cref{cohgen}, we have an equivalent derivation of the form:
  \[
    \begin{prooftree}
      \hypo{\Gamma\ |\ S : \dupldual N\vdash \Delta}
      \infer0[$(\mathsf{ax})$]{|\ \alpha : N\vdash \alpha : N}
      \infer1[$(\vdash\negN)$]{\vdash [\alpha] : \dupldual N\ |\ \alpha : N}
      \infer2[$(\mathsf{cut}^+)$]{\perfectcut{[\alpha]}{S}^+ : (\Gamma\vdash\alpha : N,\Delta)}
      \infer1[$(\negN\vdash)$]{\Gamma\ |\ \tmu[\alpha].\perfectcut{[\alpha]}{S}^+ : \dupldual N \vdash \Delta}
    \end{prooftree}
  \]
  So, one has:
  \begin{align*}
    & \llbracket\Gamma\ |\ \tmu[\alpha].\perfectcut{[\alpha]}{S}^+ : \dupldual N\vdash \Delta\rrbracket \\
     &= \chi^{-1}_{\Gamma,\dupldual N,\Delta}((\nu_N \lparrtimes \Delta) \ncomp \llbracket \perfectcut{[\alpha]}{S}^+ \rrbracket)\\
                                                                                                      &= \chi^{-1}_{\Gamma,\dupldual N,\Delta}((\nu_N \lparrtimes \Delta) \ncomp \sigma'_{\Delta,N} \ncomp (\llbracket S\rrbracket \parr N) \ncomp (\delta^l_{\Gamma,\dupldual N,N} \pcomp (\Gamma \rtensortimes \llbracket [\alpha]\rrbracket)))\\
                                                                                                      &= \chi^{-1}_{\Gamma,\dupldual N,\Delta}((\dupldoubledual N \rparrtimes \llbracket S\rrbracket) \ncomp (\nu_N \lparrtimes (\Gamma \otimes \dupldual N)) \ncomp \sigma'_{\Gamma \otimes \dupldual N,N} \ncomp (\delta^l_{\Gamma,\dupldual N,N} \pcomp (\Gamma \rtensortimes \llbracket [\alpha]\rrbracket)))\\
                                                                                                      &= \llbracket S\rrbracket \pcomp \chi^{-1}_{\Gamma,\dupldual N,\Gamma \otimes \dupldual N}((\nu_N \lparrtimes (\Gamma \otimes \dupldual N)) \ncomp \sigma'_{\Gamma \otimes \dupldual N,N} \ncomp (\delta^l_{\Gamma,\dupldual N,N} \pcomp (\Gamma \rtensortimes \llbracket [\alpha]\rrbracket))) & \mbox{ by naturality of }\chi^{-1}\\
                                                                                                      &= \llbracket\Gamma\ |\ S : \dupldual N\vdash \Delta\rrbracket
  \end{align*}
  The other cases are treated similarly, by using the coherent generation lemma and the sound value/stack substitution.
\end{proof}

\begin{theorem}
  $\rightarrow_{RE}$ preserves typing.
\end{theorem}
\begin{proof}
  We reason by induction on $\rightarrow_{RE}$. On the base case, we use the previous lemma. On other cases, we use \cref{cohgen} and the induction hypothesis.
\end{proof}


\end{document}
