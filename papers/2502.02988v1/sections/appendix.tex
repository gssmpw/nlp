
\setcounter{section}{0}
\renewcommand{\thesection}{\Alph{section}}

\section{Detailed Scenarios} \label{app:scenario}

In this section, we detail the ten scenarios currently supported by \modelname as well as their descriptions. 
Judge criteria will be available in a future version due to the space constraint.

\stitle{Question-answer Scenarios}
\etitle{(1) Close QA} 
Solve a problem that may involve professional knowledge or real-world inquiries, such as historical facts or scientific laws, and the problem has a standard/reference answer.
\etitle{(2) Open QA} Open dialogue instructions, usually asking an open-field question, and responses are also open-ended, such as casual chats, advice consultations, recommendations, etc.
\etitle{(3) Math-related QA} Solve a problem involving mathematics, calculations, reasoning, etc., and the problem has a standard/reference answer.



\eat{
Judge criteria.   
1. \emph{Accuracy}: Answers must be accurate and factual, consistent with known scientific principles.
2. \emph{Relevance}: Answers should be direct and focused on the content of the question, avoiding unnecessary information and background.
3. \emph{Harmlessness}: Answers should avoid any potentially offensive content, ensuring appropriateness and cultural sensitivity, and adhere to ethical standards.
4. \emph{Completeness}: Answers should comprehensively cover all aspects of the question, with no key points omitted, while following user instructions.
5. \emph{Source credibility}: When providing factual information, authoritative, credible sources should be cited.
6. \emph{Clarity and structure}: Answers should be clearly structured and logical, making it easy for users to understand and follow the information.
7. \emph{Timeliness}: Information should be up-to-date, especially on questions in rapidly changing fields.
8. \emph{Adaptability to user level}: Answers should consider the user's knowledge level, ensuring the content is understandable to the user.
}



\eat{
Judge criteria.   
1. \emph{Accuracy}: Ensure the accuracy of the provided information, adhering to common sense and facts, avoiding misleading the user.
2. \emph{Relevance}: Answers must address the user's questions, avoiding irrelevant content, and ensuring the relevance of the information.
3. \emph{Cultural sensitivity}: Understand and respect the user's cultural background and differences, adhering to ethical standards, avoiding cultural bias and insensitive expressions, and avoiding potentially offensive content.
4. \emph{Information richness}: While ensuring accuracy, provide detailed information, especially background information that may not be explicitly requested by the user but is helpful in understanding the question.
5. \emph{Clarity}: Use clear and understandable language to answer questions, avoiding professional terms or complex constructions that could cause misunderstanding.
6. \emph{User engagement}: Encourage further interaction with the user, showing attention and thought to their questions, and promoting engagement through follow-up questions or feedback.
7. \emph{Empathy}: Consider the user's emotional state when responding, appropriately expressing empathy and understanding, especially when answering emotionally charged questions.
8. \emph{Constructive feedback}: Even when facing critical or negative questions, maintain a positive and constructive attitude, providing valuable responses and suggestions.
}



\eat{
Judge criteria.   
1. \emph{Accuracy}: Answers should be error-free, including the final result and every step of calculation and reasoning during the solution process.
2. \emph{Clarity}: The explanation of the solution process should be clear, easy to understand, unambiguous, and use correct mathematical terms and concepts.
3. \emph{Efficiency}: The solution should be direct and as concise as possible, avoiding unnecessary lengthy explanations while ensuring accuracy and completeness.
4. \emph{Instruction adherence}: Strictly follow the problem requirements and user instructions, including specific constraints and steps.
5. \emph{Formatting}: Mathematical symbols, formulas, and diagrams should comply with academic norms and maintain consistency and readability.
6. \emph{Methodological diversity}: Where possible, provide multiple solution methods and point out their respective advantages and disadvantages.
7. \emph{Answer structure}: First provide a clear answer, followed by steps and explanations, and finally summarize key points or common mistakes.
}

\stitle{Writing scenarios}
\etitle{(4) Creative writing} Writing that primarily expresses personalized imagination and emotions, focusing on literary quality and originality, such as creating essays, poems, lyrics, scripts, stories, speeches, social media posts, blogs, advertising materials, brainstorming, etc.
\etitle{(5) Informative and professional writing} Writing aimed at conveying key information and professional knowledge, focusing on accuracy, reliability, and authority, covering practical emails, job applications, product descriptions, user manuals, to in-depth academic papers, medical research, legal opinions, engineering design, industry analysis, economic forecasts, and other complex documents.
\etitle{(6) Rewriting}  Includes text simplification, language optimization, rewriting text according to instructions, text correction, text summarization and expansion, etc.



\eat{
Judge criteria.   
1. \emph{Originality}: The work should reflect the author's independent thinking, include original views and ideas, avoiding plagiarism.
2. \emph{Emotional expression}: The work should effectively convey the author's emotions, resonating with the reader.
3. \emph{Creativity}: The work should exhibit creativity, including unique thinking, language use, and plot construction.
4. \emph{Cultural sensitivity}: The work should respect and consider cultural diversity, avoiding cultural bias, and ensuring the content is free from offensive, inappropriate, or discriminatory material.
5. \emph{Text coherence}: The work should be fluid and logically coherent, with a clear storyline for narrative works.
6. \emph{Stylistic adaptability}: The work should match its style or genre with appropriate writing style and language use.
7. \emph{Imagery and language}: The work should engage readers with expressive language and powerful imagery, enhancing visual and sensory experiences.
8. \emph{User-friendly}: The work should consider the background of the target readers, ensuring content is accessible and readable.
}



\eat{
Judge criteria.   
1. \emph{Accuracy}: Content must be based on facts and reliable data, reflecting the latest research and real-world conditions to ensure accuracy.
2. \emph{Credibility}: Content should demonstrate authority in its scientific or professional field, supported by reliable sources to build credibility, and provide a detailed assessment framework to increase objectivity.
3. \emph{Harmlessness}: Writing content should meet ethical standards, avoid offensive, inappropriate, or discriminatory material, ensuring the text's harmlessness.
4. \emph{Clarity and coherence}: Information transmission should be clear, accurate, logically rigorous, and sound in arguments, ensuring all readers can easily and orderly understand the conveyed information.
5. \emph{Relevance}: Content should be directly relevant to the topic, focused on the target audience's needs and purposes, avoiding irrelevant information, and ensuring content is valuable to the audience.
6. \emph{Professionalism}: The text should correctly use professional terms and concepts, reflecting the author's deep knowledge and skills in the relevant field, and clearly articulated to suit readers of different levels.
7. \emph{Originality}: The text should reflect the author's original thinking, showcasing unique research, insights, or analysis, ensuring the content's uniqueness.
8. \emph{Formatting standards}: Documents should follow appropriate formatting and design standards, using appropriate professional terms, typesetting, and design elements to enhance content clarity and professional presentation.
9. \emph{Audience engagement}: The text should consider the audience's needs and interests, attract readers through content appeal, effective communication, and interactivity, expanding its impact and response among the audience.
}



\eat{
Judge criteria.   
1. \emph{Accuracy}: The rewritten text should faithfully convey the original information, avoiding misleading content.
2. \emph{Compliance}: The rewriting should strictly follow the key steps and specific constraints required by the instructions.
3. \emph{Harmlessness}: The text should avoid offensive, inappropriate or discriminatory content.
4. \emph{Text quality}: The text should be grammatically correct, free of spelling errors, and maintain consistency in style, tone, and information.
5. \emph{Relevance}: The rewritten text should be relevant to the target audience and context, ensuring adaptability and targeting.
6. \emph{Conciseness}: The text should be concise and clear, avoiding unnecessary redundancy to convey information clearly.
7. \emph{Originality}: The rewritten text should demonstrate originality, avoiding plagiarism, and providing unique insights or expressions.
8. \emph{Cultural sensitivity}: The text should consider diverse cultural backgrounds, respecting different cultural values and expression habits.
}

\stitle{Professional scenarios}
\etitle{(7) Translation} Translate the given text into another language without changing the original meaning.
\etitle{(8) Reading comprehension and extraction} Read materials and complete directive tasks based on the materials, such as Q\&A, summarization, keyword extraction, topic extraction, title generation, fact-checking, etc.
\etitle{(9) Role-playing} Pretend to be a particular person, character, profession, or identity, and complete the tasks in the instructions based on this role.
\etitle{(10) Programming-related}  Tasks related to computer code, including implementing code based on requirements, code modification and optimization, programming language conversion, analyzing code and responding to related questions, software development assistance, etc.
%  education, and learning,

\eat{
Judge criteria.   
1. \emph{Faithfulness}: The translation needs to maintain fidelity to the original content, ensuring accurate transmission of information, style, and cultural connotations, avoiding misunderstandings.
2. \emph{Fluency}: The translation should be natural and fluent, conforming to the linguistic habits of the target language and easy for readers to understand.
3. \emph{Accuracy}: Terminology and factual information in the translation should be accurate, especially professional terms and data.
4. \emph{Adaptability}: The translation should be adaptively adjusted according to different contexts and target audiences.
5. \emph{Coherence}: The translation should maintain internal logical consistency, ensuring the information's consistency throughout the text.
6. \emph{Cultural appropriateness}: The translation should respect and convey the original text's cultural characteristics while considering the target language's cultural acceptance.
7. \emph{Harmlessness}: The translation should avoid any potentially misleading or offensive content, ensuring cultural and contextual sensitivity.
8. \emph{Innovation}: Without violating the original meaning, the translation should demonstrate creativity, making the translation dynamic and closer to the target language's expression habits.
}



\eat{
Judge criteria.   
1. \emph{Accuracy}: Answers should strictly correspond to the given context information, correctly responding to the questions, even if the context information might be incorrect or outdated.
2. \emph{Relevance}: Answers should directly correspond to the text content or topic, avoiding irrelevant information, ensuring all provided information has a clear textual or thematic basis.
3. \emph{Instruction compliance}: The output should strictly follow the specific requirements of the instructions, including action steps and any constraints.
4. \emph{Text coherence}: Answers should have internal logical consistency and fluency, ensuring the consistency and coherence of the information.
5. \emph{User experience}: Answers should be presented in a user-friendly manner, easy to understand, and guide the user to obtain the needed information timely.
6. \emph{Contextual understanding}: The model should demonstrate the ability to understand complex contexts and implicit information.
7. \emph{Conciseness}: Information expression should be direct and concise, avoiding unnecessary elaboration or complexity.
8. \emph{Creativity}: In tasks requiring creative output (such as title or summary generation), answers should exhibit a certain degree of originality and appeal.
}



\eat{
Judge criteria.   
1. \emph{Role fidelity}: Responses should strictly adhere to the role setting, reflecting the role’s background, behavior patterns, and characteristics.
2. \emph{Instruction compliance}: Ensure responses follow the requirements in the user's instructions, including key steps and constraints, with no omissions.
3. \emph{Safety}: Responses should avoid including any harmful or offensive content, whether overt or covert, ensuring the interaction is safe and positive.
4. \emph{Creativity}: Encourage innovative and personalized responses, showcasing the unique traits of the role while complying with the role setting and user instructions.
5. \emph{Information quality}: Responses should provide high-quality information, both professional and relevant, consistent with the role’s knowledge background.
6. \emph{Engagement}: Responses should encourage user participation and interaction, enhancing the immersive experience and user engagement in role playing.
7. \emph{Text clarity}: Text should be clear and accurate, free of grammatical errors or typos, with straightforward expressions.
8. \emph{Response efficiency}: Responses should be timely and concise, avoiding unnecessary delays, and improving interaction smoothness.
}


\eat{
Judge criteria.   
1. \emph{Code correctness}: Code should strictly follow the requirements or specifications, free from syntax or logic errors, achieving the intended functionality.
2. \emph{Code maintainability}: Code should be easy to understand and modify, with good modularity, documentation comments, and adherence to coding standards.
3. \emph{Security and safety}: Code should not contain any potential security vulnerabilities or malicious behaviors, handle all types of input safely, and have a stable error-handling mechanism.
4. \emph{Performance efficiency}: Code should execute efficiently, with optimized resource utilization, considering the algorithm's complexity and data structure choice.
5. \emph{Analysis accuracy}: Code analysis should be thorough and accurate, correctly understanding the code logic, data flow, and structure.
6. \emph{Modularity}: The written code should be modular, clearly separating concerns. It should use appropriate functions, classes, and modules to facilitate reusability and maintainability.
7. \emph{Comprehensibility of explanations}: Explanations and analyses of the code should be clear, easy to understand by users, with appropriate terms and language style.
8. \emph{Problem-solving effectiveness}: In tasks involving code modification, optimization, and programming language conversion, the generated or analyzed code should effectively solve the specified problem.
9. \emph{Input/output adherence}: Code should handle input and output in accordance with user-specified requirements, including format, type, and size. In the absence of specific instructions, it should follow common standards.
}

\begin{table}[tbh!]
  \centering
  \caption{Scenario mapping between \modelname and Llama-3}
  \label{tab:scenarioComp}
  \begin{tabularx}{.47\textwidth}{X X}
    \toprule
    \textbf{\modelname scenarios} & \textbf{Llama-3 use cases} \\ \midrule
    (1*) Closed-QA & (1) Closed question answering \\ \midrule
    \multirow{3}{*}{(2*) Open-QA} & (2) Open question answering  \\
        & (3) Asking for advice  \\ 
        & (4) Brainstorming \\ \midrule
    (3) Math-related QA & (5) Reasoning  \\ \midrule 
    (4*) Creative writing & (6) Creative writing  \\ \midrule
    (5) Info. and prof. writing &  \texttt{NA} \\ \midrule
    (6*) Rewriting & (7) Rewriting  \\ \midrule
    (7) Translation & \texttt{NA} \\ \midrule
    \multirow{2}{*}{(8*) Reading compre. and extrac.} & (8) Extraction  \\
    & (9) Summarization  \\ \midrule
    (9*) Role-playing & (10) Inhabiting a character  \\ \midrule
    (10*) Programming-related & (11) Coding  \\ \midrule
    \texttt{NA} & (12) Classification  \\ 
    \bottomrule
  \end{tabularx}
\end{table}

\stitle{Justification for scenario design}
Table~\ref{tab:scenarioComp} illustrates the mapping between \modelname's scenarios and Llama-3's evaluation use cases~\cite{llama3tech}. 
From the table we can see that 7 out of \modelname's 10 scenarios, \ie those with *, have direct mapping to the ones of Llama-3. The remaining math-related QA, informative and professional writing, and translation are popular applications of LLMs and deserve context-specific evaluations.
On the other side, 11 out of Llama-3's 12 use cases, except for classification, are covered by \modelname.
As the scenario design of \modelname is independent with Llama-3, the above comparison could be regarded as an empirical justification for our scenario design through human-AI collaboration.


\section{Prompts}
\label{app:prompts}

In addition to Table~\ref{tab:prompt}, extra prompts related to \modelname training and inference include the ones for reference-guided grading, pairwise comparison, reference-based question synthesis, role-playing quizzing for three scenarios, and fine-tuning the scenario classification LLM. All these prompts will be presented in a future extended version for reproducibility.

%Note that \{\texttt{variable}\} represents a variable and should be filled in properly for all prompts. 

% We also illustrate a complete evaluation record for supervised fine-tuning, with the evaluation input in Table~\ref{tab: math example} and the evaluation result by GPT-4 in Table~\ref{tab: gpt-4's response}. Evaluations by Qwen-max, Qwen-14B, \modelname are also presented (Tables~\ref{tab: qwenmax's response}--\ref{tab: themis' response}). Among the three, we find only \modelname makes a reasonable evaluation. 