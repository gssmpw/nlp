%Version 3 October 2023
% See section 11 of the User Manual for version history
%
%%%%%%%%%%%%%%%%%%%%%%%%%%%%%%%%%%%%%%%%%%%%%%%%%%%%%%%%%%%%%%%%%%%%%%
%%                                                                 %%
%% Please do not use \input{...} to include other tex files.       %%
%% Submit your LaTeX manuscript as one .tex document.              %%
%%                                                                 %%
%% All additional figures and files should be attached             %%
%% separately and not embedded in the \TeX\ document itself.       %%
%%                                                                 %%
%%%%%%%%%%%%%%%%%%%%%%%%%%%%%%%%%%%%%%%%%%%%%%%%%%%%%%%%%%%%%%%%%%%%%

%%\documentclass[referee,sn-basic]{sn-jnl}% referee option is meant for double line spacing

%%=======================================================%%
%% to print line numbers in the margin use lineno option %%
%%=======================================================%%

%%\documentclass[lineno,sn-basic]{sn-jnl}% Basic Springer Nature Reference Style/Chemistry Reference Style

%%======================================================%%
%% to compile with pdflatex/xelatex use pdflatex option %%
%%======================================================%%

%%\documentclass[pdflatex,sn-basic]{sn-jnl}% Basic Springer Nature Reference Style/Chemistry Reference Style


%%Note: the following reference styles support Namedate and Numbered referencing. By default the style follows the most common style. To switch between the options you can add or remove “Numbered” in the optional parenthesis. 
%%The option is available for: sn-basic.bst, sn-vancouver.bst, sn-chicago.bst%  
 
%%\documentclass[sn-nature]{sn-jnl}% Style for submissions to Nature Portfolio journals
\documentclass[referee, sn-basic, Numbered]{sn-jnl}% Basic Springer Nature Reference Style/Chemistry Reference Style
% \documentclass[sn-mathphys-num]{sn-jnl}% Math and Physical Sciences Numbered Reference Style 
%%\documentclass[sn-mathphys-ay]{sn-jnl}% Math and Physical Sciences Author Year Reference Style
%%\documentclass[sn-aps]{sn-jnl}% American Physical Society (APS) Reference Style
%%\documentclass[sn-vancouver,Numbered]{sn-jnl}% Vancouver Reference Style
%%\documentclass[sn-apa]{sn-jnl}% APA Reference Style 
%%\documentclass[sn-chicago]{sn-jnl}% Chicago-based Humanities Reference Style

%%%% Standard Packages
%%<additional latex packages if required can be included here>

\usepackage{graphicx}%
\usepackage{multirow}%
\usepackage{amsmath,amssymb,amsfonts}%
\usepackage{amsthm}%
\usepackage{mathrsfs}%
\usepackage[title]{appendix}%
\usepackage{xcolor}%
\usepackage{textcomp}%
\usepackage{manyfoot}%
\usepackage{booktabs}%
\usepackage{algorithm}%
\usepackage{algorithmicx}%
\usepackage{algpseudocode}%
\usepackage{listings}%
\usepackage{amsmath}
% \usepackage{lineno}
\usepackage{diagbox}
\usepackage{hyperref}
\usepackage{xpatch}
\usepackage{bm}
\usepackage{braket}  % 狄拉克符号支持
\usepackage{geometry}%
\usepackage{ulem}%

\newcommand{\eref}{Eq. (\eqref)}
\newcommand{\erefs}{Eqs. (\eqref)}
\newcommand{\fref}{Fig.\ \ref}
\newcommand{\secref}{Section\ \ref}
\newcommand{\frefs}{\ref}
\newcommand{\tref}{Table\ \ref}
\newcommand{\rd}[1]{\textcolor{red}{#1}}
\newcommand{\bl}[1]{\textcolor{blue}{#1}}
\newcommand{\mgt}[1]{\textcolor{magenta}{#1}}
%\newcommand{\dgr}[1]{\textcolor[rgb]{0.1, 0.5, 0.1}{#1}}
\newcommand{\org}[1]{\textcolor{orange}{#1}}
\newcommand{\tea}[1]{\textcolor{teal}{#1}}
\newcommand{\pp}[1]{\textcolor{purple}{#1}}
\newcommand{\erase}[1]{\sout{\pp{#1}}}

%%%%

%%%%%=============================================================================%%%%
%%%%  Remarks: This template is provided to aid authors with the preparation
%%%%  of original research articles intended for submission to journals published 
%%%%  by Springer Nature. The guidance has been prepared in partnership with 
%%%%  production teams to conform to Springer Nature technical requirements. 
%%%%  Editorial and presentation requirements differ among journal portfolios and 
%%%%  research disciplines. You may find sections in this template are irrelevant 
%%%%  to your work and are empowered to omit any such section if allowed by the 
%%%%  journal you intend to submit to. The submission guidelines and policies 
%%%%  of the journal take precedence. A detailed User Manual is available in the 
%%%%  template package for technical guidance.
%%%%%=============================================================================%%%%
% paper size configurations:
\geometry{letterpaper, margin=1in}
%% as per the requirement new theorem styles can be included as shown below
\theoremstyle{thmstyleone}%
\newtheorem{theorem}{Theorem}%  meant for continuous numbers
%%\newtheorem{theorem}{Theorem}[section]% meant for sectionwise numbers
%% optional argument [theorem] produces theorem numbering sequence instead of independent numbers for Proposition
\newtheorem{proposition}[theorem]{Proposition}% 
%%\newtheorem{proposition}{Proposition}% to get separate numbers for theorem and proposition etc.

\theoremstyle{thmstyletwo}%
\newtheorem{example}{Example}%
\newtheorem{remark}{Remark}%

\theoremstyle{thmstylethree}%
\newtheorem{definition}{Definition}%

\raggedbottom
%%\unnumbered% uncomment this for unnumbered level heads

\begin{document}
\let\WriteBookmarks\relax
\def\floatpagepagefraction{1}
\def\textpagefraction{.001}
% \title[Article Title]{Application of Quantum Approximate Optimization Algorithm to Topology Optimization}
\title[Article Title]{An Efficient Quantum Approximate Optimization Algorithm with Fixed Linear Ramp Schedule for Truss Structure Optimization}
%%=============================================================%%
%% GivenName	-> \fnm{Joergen W.}
%% Particle	-> \spfx{van der} -> surname prefix
%% FamilyName	-> \sur{Ploeg}
%% Suffix	-> \sfx{IV}
%% \author*[1,2]{\fnm{Joergen W.} \spfx{van der} \sur{Ploeg} 
%%  \sfx{IV}}\email{iauthor@gmail.com}
%%=============================================================%%

\author[1]{\fnm{Junsen} \sur{Xiao}}\email{xiao.junsen.s2@dc.tohoku.ac.jp}

\author[2]{\fnm{Naruethep} \sur{Sukulthanasorn}}\email{sukulthanasorn.naruethep.c2@tohoku.ac.jp }

\author[2]{\fnm{Reika} \sur{Nomura}}\email{rnomura@tohoku.ac.jp}

\author[2]{\fnm{Shuji} \sur{Moriguchi}}\email{shuji.moriguchi.d6@tohoku.ac.jp}


\author*[1]{\fnm{Kenjiro} \sur{Terada}}\email{tei@tohoku.ac.jp}
\affil*[1]{\orgdiv{Department of Civil and Environmental Engineering}, \orgname{Tohoku University}, \orgaddress{\street{Aza-Aoba, 468-1, Aramaki, Aoba-ku}, \city{Sendai}, \postcode{980-8572}, \country{Japan}}}

\affil[2]{\orgdiv{International Research Institute of Disaster Science}, \orgname{Tohoku University}, \orgaddress{\street{Aza-Aoba, 468-1, Aramaki, Aoba-ku}, \city{Sendai}, \postcode{980-8572}, \country{Japan}}}

% \makeatletter
% \def\changeBibColor#1{%
%   \in@{#1}{
% }%  list of colored bib items
  
%   \ifin@\color{red}
%   \else\normalcolor\fi
% }
 
% \xpatchcmd\@bibitem
%   {\item}
%   {\changeBibColor{#1}\item}
%   {}{\fail}
 
% \xpatchcmd\@lbibitem
%   {\item}
%   {\changeBibColor{#2}\item}
%   {}{\fail}
% \makeatother
%%==================================%%
%% Sample for unstructured abstract %%
%%==================================%%

\abstract{
%To confront the increasingly complex and high-dimensional optimization challenges inherent in engineering design, quantum computing leveraging quantum superposition and entanglement is {changing the} paradigm {in recent years}. 
%{In recent years, quantum computing, which utilizes quantum superposition and entanglement, has begun to change the paradigm in order to tackle the complex and high-dimensional optimization problems inherent in engineering design. }
This study proposes a novel structural optimization framework based on quantum variational circuits, in which the multiplier acting on the cross-sectional area of each rod in a truss structure as an updater is used as a design variable.
Specifically, we employ a classical processor for structural analysis with the finite element method, and the Quantum Approximate Optimization Algorithm (QAOA) is subsequently performed to update the cross-sectional area so that the compliance is minimized. 
The advantages of this framework can be seen in three key aspects. First, by defining design variables as multipliers, rather than simply reducing the design variable to a binary candidate of inclusion or exclusion (corresponding to qubit states, ``0" and ``1"), it provides greater flexibility in adjusting the cross-sectional area of the rod at each iteration of the optimization process.
Second, the multipliers acting on rods are encoded with on-off encoding, eliminating additional constraints in the convergence judgement. As a result, the objective function is in a simple format, enabling efficient optimization using QAOA.
%the objective function is simple in the form and facilitates efficient solving by QAOA. 
Third, a fixed linear ramp schedule (FLRS) for variational parameter setting bypasses the classical optimization process, thereby improving the operational efficiency of the framework.
In the two structural cases investigated in this study, the proposed approach highlights the feasibility and applicability potential of quantum computing in advancing engineering design and optimization. 
%Numerical experiments substantiate the framework's effectiveness, providing a solid foundation for future research in quantum-assisted engineering optimization methodologies. 
Numerical experiments have demonstrated the effectiveness of this framework, providing a firm foundation for future research on quantum-assisted optimization methods in engineering fields.
}

%%================================%%
%% Sample for structured abstract %%
%%================================%%

% \abstract{\textbf{Purpose:} The abstract serves both as a general introduction to the topic and as a brief, non-technical summary of the main results and their implications. The abstract must not include subheadings (unless expressly permitted in the journal's Instructions to Authors), equations or citations. As a guide the abstract should not exceed 200 words. Most journals do not set a hard limit however authors are advised to check the author instructions for the journal they are submitting to.
% 
% \textbf{Methods:} The abstract serves both as a general introduction to the topic and as a brief, non-technical summary of the main results and their implications. The abstract must not include subheadings (unless expressly permitted in the journal's Instructions to Authors), equations or citations. As a guide the abstract should not exceed 200 words. Most journals do not set a hard limit however authors are advised to check the author instructions for the journal they are submitting to.
% 
% \textbf{Results:} The abstract serves both as a general introduction to the topic and as a brief, non-technical summary of the main results and their implications. The abstract must not include subheadings (unless expressly permitted in the journal's Instructions to Authors), equations or citations. As a guide the abstract should not exceed 200 words. Most journals do not set a hard limit however authors are advised to check the author instructions for the journal they are submitting to.
% 
% \textbf{Conclusion:} The abstract serves both as a general introduction to the topic and as a brief, non-technical summary of the main results and their implications. The abstract must not include subheadings (unless expressly permitted in the journal's Instructions to Authors), equations or citations. As a guide the abstract should not exceed 200 words. Most journals do not set a hard limit however authors are advised to check the author instructions for the journal they are submitting to.}

\keywords{Quantum approximate optimization algorithm, Topology optimization, On-off encoding, Fixed linear ramp schedule}

%%\pacs[JEL Classification]{D8, H51}

%%\pacs[MSC Classification]{35A01, 65L10, 65L12, 65L20, 65L70}

\maketitle
%%%%%%%%%%%%%%%%%%%%%%%%%%%%%%%%%%%%%%%%%%%%%
\section{Introduction}\label{sec1}
%%%%%%%%%%%%%%%%%%%%%%%%%%%%%%%%%%%%%%%%%%%%
%\linenumbers
%{In recent years, quantum computing, which utilizes quantum superposition and entanglement, has begun to change the paradigm in order to tackle the complex and high-dimensional optimization problems inherent in engineering design. }
Structural optimization is a computational design method aimed at determining the optimal size of structural members, geometrical shape, or distribution of materials to enhance performance while adhering to given constraints, such as material volume, stress distribution, and target strength. Since the pioneering works by Bendsøe and Kikuchi \cite{bendsoe2013topology, bendsoe1988generating}, topology optimization has evolved in particular and has come to be applied to various fields \cite{kiziltas2003topology, nomura2007structural, FUJII2020120082, IRADUKUNDA2020115723, CHEN2020112806, CHEN2021107054, yesilyurt2021efficient, mukherjee2021accelerating}. So far, numerous optimization algorithms have been developed to achieve efficient and effective structural designs. Prominent techniques include gradient-based methods \cite{Raphael2012elements, Andreassen2011matlab}, genetic algorithms \cite{DEB2001447, WANG20053749}, and evolutionary structural optimization \cite{hu2020fracture}, complemented by advanced computational strategies such as parallel computing \cite{MAKSUM2022100352}, dimensionality reduction strategies \cite{Gogu2015reduce}, and geometric primitives \cite{GUO2022103238}. While these approaches have historically demonstrated considerable success, recent years have witnessed challenges in achieving significant improvements in computational performance, largely due to the saturation of algorithmic advancements and the stagnation of progress in classical computational hardware \cite{moore1998cramming, Xu2024Gate}. To break through these bottlenecks, innovative optimization frameworks and technologies capable of surpassing current limitations and driving substantial advancements in computational efficiency are further required.

In recent years, quantum computing (QC) has gained tremendous attraction as an emerging and alternative computing paradigm that can theoretically be superior to classical computing for a variety of optimization problems\cite{Frank2019quantum}, including structural optimization. This superiority stems from the entanglement and superposition of quantum states. Generally, candidate solutions are encoded through a string of qubits in QC. The qubits can exist in 0 and 1 simultaneously, while the classical bit must be either 0 or 1. This superposition allows QC to search multiple solutions in parallel. With the theoretical foundation, QC has been applied to optimization problems in various disciplines in recent years. Brandhofe et al. \cite{brandhofer2022benchmarking} conducted a benchmarking test of portfolio optimization problems utilizing quantum optimization algorithms\cite{brandhofer2022benchmarking}. Xiao et al. \cite{xiao2024fmqa} examined the applicability of quantum computing for hyperparameter optimization and metamodel-based granular flow simulation optimization. Kurowski et al.\cite{KUROWSKI2023518} applied a quantum variational circuit to the job shop scheduling problem. 

To date, there are currently two main types of quantum hardware in the current stage as the noisy intermediate-scale quantum (NISQ) devices: gate-based universal quantum computers (GQC)\cite{Deutsch1985universal} and quantum annealers (QA) embrace the annealing techniques\cite{kadowaki1998annealing}. The former alters the amplitudes of the states in the superimposed Hamiltonian through a series of unitary operators (quantum gates), enabling the ground state to be observed with sufficiently high probability.
The latter obtains the ground state of the Hamiltonian based on the adiabatic evolution\cite{born1928beweis} of quantum fluctuations. Currently, the number of qubits that can be adopted in QA is roughly in the thousands, but only a few dozen can be reliably used in GQC. In addition, reliably executable GQC circuits are necessarily constrained to remain short in depth, because of the inherent noise in quantum gates and the substantial overhead associated with error mitigation. Given these circumstances, QA has been successfully applied to structural optimization problems with different proposed approaches \cite{honda2024development,ye2023quantum,sukulthanasorn2024quantum}. Nevertheless, since QA hardware is specifically designed for the application as optimization through the annealing process, its applicability across a broad range of applications may be limited and could potentially be replaced by the GQC in the long term. 

Based on existing literature, one of the most well-known algorithms tailored for solving optimization problems by GQC is Quantum Approximate Optimization Algorithm (QAOA)\cite{Edward2014qaoa}. It prepares the problem's Hamiltonian by a variational quantum circuit composed of a series of quantum operators, and the variational parameters in the operators are finely tuned by adiabatic evolution (consistent with QA) and a layering strategy. The circuit consists of phase-separating and mixer Hamiltonian operators, which are alternately applied, with each pair of operators corresponding to a layer. If the number of layers is denoted by $p$, then when $p \to \infty$, the circuit is theoretically closer to an ideal adiabatic evolution. 
The phase-separating operators symbolize the problem' Hamiltonian as a combination of multiple quantum gates, whereas the mixer operators alter the amplitudes of the states in Hamiltonian. The basis state can be observed with a sufficiently high probability once the variational parameters are optimized by a classical processor. In recent years, the application of QAOA has gradually expanded from graph theory problems such as Max-Cut\cite{Wang2018max}, multi-knapsack\cite{Awasthi2023}, and Max-Sat\cite{Yu2023sat}, to real-world optimization problems\cite{Pontus2020tail,Wang2012insp,Babbush2018depth}. Along with this trend, some researches have also proceeded to design the mixer operators to improve the approximation accuracy of QAOA\cite{marsh2019quantum,Wang2020xy,kim2023quantum}. So far, the application of QAOA to structural optimization and other computational mechanics problems remains limited, highlighting the need to explore a suitable integration framework between QAOA and structural optimization.

To the best of authors' knowledge, Kim and Wang\cite{Kim2023to} first applied QAOA to truss optimization problems. They directly defined the existing rod as binary values, representing the addition or removal of rods in a truss structure. All possible candidate rod members are incorporated into the objective function, formulating the problem as a well-known ground structure optimization problem. Although they successfully obtained a solution with acceptable accuracy for a small-scale truss structure with 6 rods, this approach faces challenges when extending to larger problem sizes due to the computational expense of encoding candidate solutions into the objective function. Consequently, the complexity of the objective function increases exponentially with the number of rods, resulting in significant computational time requirements. Additionally, Xu et al.\cite{Xu2024Gate} presented the alternative approach for representing the cross-sectional areas of rods as discrete variables using one-hot encoding, and defined an objective function that minimizes the volume of material usage to meet the load and deformation requirements. Their approach ensures that the objective function remains quadratic, but the cost of using one-hot encoding is a several-fold increase in qubits consumption. Besides, it should be noted here that the QAOA circuit parameters also require to be fine-tuned with iterative optimization. This makes structural optimization using QAOA expensive and remains a challenge in enhancing computational efficiency. 

In this study, taking the structural optimization problem of truss structures as an example, we propose an efficient iterative updating framework that effectively uses an updater that adjusts the cross-sectional area based on the optimization results obtained from QAOA. 
%In addition, this framework employs preselected parameters for the quantum variational circuit, eliminating the need for iterative fine-tuning and marking the first successful application of this approach to truss optimization problems. 
Furthermore, by using predetermined parameters in the quantum variational circuit, this framework eliminates the need for iterative fine-tuning by a classical processor, and we have successfully applied this approach to the truss optimization problem for the first time.
%In detail, the updaters adjust the cross-sectional area of each rod in each design iteration. 
To be more specific, in order to minimize compliance while meeting a volume constraint, the cross-sectional areas and volumes are adjusted in each design iteration by multiplying the cross-sectional area of each rod by the updater obtained from QAOA.
%The objective of each iterative update is to minimize compliance while satisfying a volume constraint. 
In the present study, 
%The objective function is a quadratic function for truss optimization problems, 
each updater is encoded by on-off encoding using two qubits, and the objective function is expressed as a quadratic function. 
%, {is also the target of encoding}. 
This setting allows the cross-sectional area of each rod to be dynamically and flexibly adjusted during the optimization process.
%The purpose of iterative updates is to minimize compliance while meeting volume constraints.
%In this framework, the form of the objective function is also quadratic. Simultaneously, each updater is encoded by on-off encoding with two qubits, which enable the cross-sectional areas of the rods to be dynamically and flexibly adjusted during optimization process.} 
%{ Rods experiencing high strain energy have their cross-sectional areas progressively increased, while those with lower strain energy are reduced, leading to a stable convergence of the material distribution toward an optimal configuration.} 
Specifically, the cross-sectional area of a rod that has experienced high strain energy gradually increases, while the cross-sectional area of a rod that has experienced low strain energy decreases. 
As a result, the arrangement of truss members (rods) gradually converges to an optimal configuration.
%the material distribution steadily converges towards an optimal configuration.
%To prevent hardware noise from the current quantum processor, the examples in this study are based on a quantum simulator rather than real quantum processor. The simulator is provided by Pennylane 0.39 (an open-source Python framework for quantum programming)\cite{bergholm2022penny}.
In order to prevent hardware noise, this research uses a quantum simulator rather than an actual quantum processor. 
This simulator is provided by \texttt{Pennylane 0.39}, an open-source Python library for quantum programming.

%%%%%%%%%%%%%%%%%%%%%%%%%%%%%%%%%%%%%%%%%%%%%%
\section{Quantum Approximate Optimization Algorithm}\label{sec2}
%%%%%%%%%%%%%%%%%%%%%%%%%%%%%%%%%%%%%%%%%%%%%%%
% Theoretically, it is challenging for classical computers to simulate the superposition of quantum computers. This difficulty arises because quantum computers operate in a $2^n$ dimensional Hilbert space\cite{Edward2014qaoa}, which makes the required computing resources are exponential\cite{Zhou2020limit}. Quantum optimization makes use of this to accelerate the search for the optimum. In an $n$ qubits quantum system, the optimization result exists in $2^n$ combinations, which can be written as 
% \begin{equation}
%     \ket{\phi_j} (j = 0, 1, 2,..., 2^n-1)\label{eq1_comb}{.}
% \end{equation}
% Such a quantum system itself can be expressed in the form of a tensor product $\ket{\psi}$ as
% \begin{equation}
%     \ket{\psi} = \sum_{j = 0}^{2^n-1}{\alpha_j\ket{\phi_j}} \quad \text{subject to  } \sum_{j=0}^{2^n-1}{\left|\alpha_j\right|^2} = 1\label{eq2_tensor}{.}
% \end{equation}
% Here, $\alpha_j$ is the complex probability amplitudes of each combination, which determines the probability of the quantum system collapsing into one of the combinations when measured. The goal of QAOA is precisely to prepare the quantum system $\ket{\psi}$ whose ground state $\ket{\phi}_{min}$ can be observed with a probability as high as feasible. 
%###########################################################
\begin{figure}[htbp]
    \centering
    \includegraphics[width=0.6\linewidth]{Figures/Fig_1_QAOA_circuit_c.eps}
    \caption{Architecture of QAOA circuit.
}
\label{fig:circuit}
\end{figure}
%###########################################################

In this section, we briefly summarize the architecture of the QAOA circuit and discuss fine-tuning approaches for QAOA circuit parameters, called variational parameters.
%++++++++++++++++++++++++++++++++++++++++++++++++++++++++
\subsection{Combinatorial optimization problem and QAOA principle}
%++++++++++++++++++++++++++++++++++++++++++++++++++++++++
Many interesting real-world problems can be framed as
combinatorial optimization problems, which are aimed at finding an optimal object from a finite set of objects. A combinatorial optimization problem can be phrased as a maximization of an objective function which is a sum of Boolean functions. Each Boolean function $C_d(\bm{z}) : \{-1, 1\}^{n} \rightarrow \{-1, 1\}$ takes the qubit string $z = z_1z_2\cdots z_n$ as input and a Boolean value (-1 or 1) as output. The objective function in Hamiltonian form can then be formulated as
%%%%%%%%
\begin{equation}
    C = \sum_{d=1}^{D}{w_dC_d(\bm{z})}\label{eq1_comb}{,}
\end{equation}
%%%%%%%%
where $D$ is the number of Boolean functions. In more general problems, $w_d$ are added as factors to introduce weights to different objects. QAOA was originally designed to leverage quantum acceleration for solving such combinatorial optimization problems by Farhi et al.\cite{Edward2014qaoa}. They took inspiration from the Trotterized version of the quantum adiabatic evolution theorem \cite{born1928beweis}. The evolution of a $n$-qubits system performed by QAOA quantum circuit is defined as
%%%%%%%%
\begin{equation}
    \ket{\psi} = \tilde{U}_{B}(B, \beta_p)\tilde{U}_{C}(C, \gamma_p) \cdots \tilde{U}_{B}(B, \beta_1)\tilde{U}_{C}(C, \gamma_1)\ket{+}^{\otimes n}\label{eq2_qaoa1}{.}
\end{equation}
%%%%%%%%
Here, $\ket{+}^{\otimes n}$ is uniform superposition initial state for the $n$-qubits system, which prepared by applying Hadamard gates to all qubits with the initial state at $\ket{0}$. Also, $\tilde{U}_{C}$ and $\tilde{U}_{B}$ represent the phase-separating operators and mixer operators, respectively. These two operators are alternately mounted on the circuit.

The phase-separating operator $\tilde{U}_{C}(C, \gamma)$ symbolizes the problem's Hamiltonian $C$ with multiple quantum gates. In particular, $C$ is obtained from the conventional objective function with Ising variables $z \in \{-1, +1\}$, which can be easily transferred from conventional binary variables $q \in \{0, 1\}$ as 
%with Ising variables $z \in \{-1, +1\}$ as}
%%%%%%%%
\begin{equation}
    q_j \rightarrow \frac{1}{2}(1-z_j) \quad (j = 1, \ldots ,n)\label{eq3_bintois}{.}
\end{equation}
%%%%%%%%
The Ising variables are encoded in $\tilde{U}_{C}$, which applies a Pauli $Z$ matrix on each qubit. Specifically, given the problem's Hamiltonian $C$ and variational parameter $\gamma$, $\tilde{U}_{C}$ is defined as 
%%%%%%%%
\begin{equation}
   \tilde{U}_{C}(C, \gamma) = e^{-i\gamma C} = \prod_{d=1}^{D}e^{-i\gamma w_dC_d(Z)}\label{eq4_uc}{.}
\end{equation}
%%%%%
On the other hand, the mixer operator $\tilde{U}_{B}$ alters the amplitudes of the states in Hamiltonian; i.e., a Pauli $X$ gate $\sigma^{x}$ is applied to each qubit in the circuit. The Pauli $X$ gate gives each qubit a counterclockwise rotation around the Bloch sphere along the $X$ axis, represented by the variational parameter $\beta$. To be specific, given the variational parameter $\beta$, $\tilde{U}_{B}$ is defined as 
%%%%%%%%
\begin{equation}
   \tilde{U}_{B}(B, \beta) = e^{-i\beta B} = \prod_{j=1}^{n}e^{-i\beta \sigma^{x}_{j}},\text{ where } B = \sum_{j=1}^{n}{\sigma^{x}_{j}}\label{eq5_uB}{.}
\end{equation}
%%%%%%%%
After repeating the sequence of $\tilde{U}_{C}$ and $\tilde{U}_{B}$ $p$ times, the measurement operators converges $\ket{\psi}$ to the single basis state $\ket{\psi_f}$, which is the optimal solution identified by the algorithm. 
%The objective {function} value corresponding to $\ket{\psi_f}$ is {approximately???} computed from the expectation of $C$\erase{, defined} as
The objective function value corresponding to the expectation of $C$ is approximately computed as  
%%%%%
\begin{equation}
   f(C, \ket{\psi_f}) \approx \bra{\psi_f} C \ket{\psi_f} \label{eq6_exp}{.}
\end{equation}
%%%%%

QAOA is a quantum-classical hybrid approach, with some of the classical processor acting as an optimizer for the variational parameters in the circuit as shown in \fref{fig:circuit}. The goal of the above-described process is to find the optimal set of variational parameters $(\bm{\gamma}^{*}, \bm{\beta}^{*})$ such that the expectation value $f(C, \ket{\psi_f})$ is maximized as
%%%%%

\begin{equation}
   (\bm{\gamma}^{*}, \bm{\beta}^{*}) = \underbrace{\mathrm{arg} \; \mathrm{max}}_{\forall \; \bm{\gamma}, \bm{\beta}}f(C, \ket{\psi_f}) \label{eq7_para}{.}
\end{equation}

%%%%%
Typically, the performance of the QAOA greatly depends on the classical optimizer adopted for determining the variational parameters. When the phase-separating and mixer Hamiltonian operators are alternately applied $p$ times, the circuit depth QAOA corresponds to $p$ layers, necessitating the optimization to determine $2p$ variational parameters. 
As mentioned in Introduction, as $p \rightarrow \infty$, the QAOA circuit approaches ideal adiabatic evolution and achieves optimal precision in theory\cite{Shor2014adia}.

%++++++++++++++++++++++++++++++++++++++++++++++++++++++++
\subsection{Fine-tuning of QAOA circuit parameters}
% with non-iterative scheme}}
%++++++++++++++++++++++++++++++++++++++++++++++++++++++++
%Currently, widely recognized classical optimization methods include 
The classical optimizers for fine-tuning QAOA circuit parameters that are widely recognized today include gradient-based Broyden-Fletcher-Goldfarb-Shanno (BFGS) \cite{Flecher1992} algorithms and linearly approximation approach COBYLA. 
Additionally, Kim and Wang \cite{kim2023quantum} proposed QABOA, which adopted Bayesian optimization as a classical optimizer. However, even with moderate $p$-values (e.g., 5–10), they face the barren plateau problem\cite{wang2021noise}, making it difficult to achieve global optimum. Furthermore, in the QAOA circuit, fine-tuning of its parameters is required before designing the target structure, and this can be another iterative process,  which is independent of iterative structural updating. 
This double iteration significantly reduces computational efficiency, making the QAOA approach less practical. Recent investigations have focused on this issue, and some studies have explored patterns of variational parameters to simplify the optimization process. 
For instance, Zhou et al.\cite{Zhou2020devices} and Montanez-Barrer and Michielsen\cite{Montanez2024protocol} identified general trends in optimal variational parameters in large-scale Max-cut problems. 
%The optimal parameters, $\gamma^{*}$ and $\beta^{*}$ increases steadily with $p$, while the optimal $\beta^{*}$ decreases steadily. 
The optimal parameters, $\gamma^{*}$ and $\beta^{*}$, increase and decreases steadily with $p$, respectively. 
The former study proposed a strategy that optimizes the initial variational parameters ($p = 1$) by BFGS, then iteratively sets the subsequent variational parameters ($p > 1$) using linear interpolation. 
The latter suggested using a fixed linear ramp schedule (FLRS) for variational parameter settings. Both approaches bypass most of the optimization process and achieve promising results.

In this study, the FLRS method is combined with the proposed update design scheme to significantly reduce
% to integrate with the proposed update design scheme, significantly reducing 
the computational cost associated with fine-tuning the QAOA circuit parameters. 
%to provide insights into comparing the performance of 
To investigate the applicability of QAOA to truss optimization problems, we conduct a detailed evaluation of the performance comparison of COBYLA, Bayesian optimization, and FLRS in terms of computational cost and accuracy for two optimization cases with $p$ set to 6 and 8. 

As a point to note, there are two unavoidable issues in the practical implementation of QAOA. 
%First, one cannot directly access the expectation of the observable through QAOA circuit, but can only estimate it by obtaining many measures of the problem variable and averaging them. 
First, it is not possible to directly access the observable expectation at each location in the QAOA circuit, and since measurements are limited to one per circuit execution, it is necessary to estimate the necessary information by executing the circuit many times to obtain many measurements of the problem variable and averaging them. 
Thus, the expectation can only be approximated by the average, and its uncertainty is inversely proportional to the square root of the number of measurements. In this study, the number of measurements is set to $10^{5}$ to mitigate the inherent uncertainty as much as possible. 
The second problem is that 
%Second, 
the result reported at the end of the QAOA experiment is the most frequently measured qubit string among all candidates, not the average value (expectation) of $C$ \cite{kim2023quantum,Larkin2022QAOA}. 

%%%%%%%%%%%%%%%%%%%%%%%%%%%%%%%%%%%%%%%%%%%
\section{QAOA-based framework for truss optimization}\label{sec3}
%%%%%%%%%%%%%%%%%%%%%%%%%%%%%%%%%%%%%%%%%%%%%%

%In the previous section, we introduced the architecture of the QAOA circuit. 
In this section, the details of the QAOA-based structural optimization framework is outlined 
%according to two cases of truss optimization instances. 
using two examples of truss optimization.
%{The framework consists of} the truss structure setup, the iterative updating scheme, the encoding operation on design variables, and the definition of the objective function.
After setting the static equilibrium problem of truss structures, we explain the iterative updating scheme, the encoding operation on design variables, and the definition of the objective function in turn.

%###########################################################
\begin{figure}[htbp]
    \centering
    \includegraphics[width=0.6\linewidth]{Figures/Fig2_truss_structure.eps}
    \caption{2D truss structures.
}
\label{fig:truss}
\end{figure}
%###########################################################

%%%%%%%%%%%%%%%%%%%%%%%%%%%%%%%%%%%%%%%%%%%%%%%%
%++++++++++++++++++++++++++++++++++++++++++++++++++++++++
\subsection{Truss structure setup}\label{subsec31}
%++++++++++++++++++++++++++++++++++++++++++++++++++++++++
This study configures two 2D truss structures consisting of 6 and 11 rods, as shown in \fref{fig:truss}, as optimization targets. Each rod has a length $L$ of either 1.0 m or $\sqrt{2}$ m, with a Young's modulus $E$ of $2\times10^{11}$ $\mathrm{N/m^2}$, and a initial cross-sectional area $A^{(0)}$ of 0.5 $\mathrm{m^2}$. Then, the initial volumes $V^{(0)}$ for Case 1 and Case 2 can be calculated as 3.414 $\mathrm{m^2}$ and 6.328 $\mathrm{m^2}$, respectively, which serve as volume constraints associated with the objective function for QAOA. A load of 100 kN is applied on the blue node, acting vertically downward. The vertical rods on the left in both of these cases do not transmit any load, but are still included in the optimization target for verification purposes. That is, with this inclusion, we can verify that QAOA is properly working by checking whether these rods are removed in the first few iterations. Note that the self-weight of the structure is not considered in this study.

%###########################################################
\begin{figure}[htbp]
    \centering
    \includegraphics[width=0.75\linewidth]{Figures/Fig3_iterative_update.eps}
    \caption{Iterative updating scheme.
}
\label{fig:update}
\end{figure}
%###########################################################

%%%%%%%%%%%%%%%%%%%%%%%%%%%%%%%%%%%%%%%%%%%%%%
%++++++++++++++++++++++++++++++++++++++++++++++++++++++++
\subsection{Iterative updating scheme and design variables}\label{subsec32}
%++++++++++++++++++++++++++++++++++++++++++++++++++++++++
As mentioned in Introduction, we design an optimization framework characterized by iterative updates, in which the multiplier $\alpha$ acting on cross-sectional areas $A$ is defined as a design variable.  
%This multiplier {works to adjust} the cross-sectional area of each rod {in each iteration}. 
The purpose of iterative updates is to minimize compliance while satisfying a volume constraint, and the multiplier works to adjust the cross-sectional area of each rod in each iteration.  
Taking Case 1 as an example, we can envision the optimization problem encountered by QAOA in each iteration with \fref{fig:update}.

According to the initial setup of the truss structure, before the iteration starts (Iteration 0 in \fref{fig:update}), we assume the same cross sectional area $A^{(0)}$ for all rods.
%the cross-sectional area of each rod is $A^{(0)}$. 
%{Subsequently}, a set of  $\alpha$ {is multiplied by the cross-sectional areas to update them, so is referred to as ``updater'' hereafter. 
Subsequently, the cross-sectional areas is updated by multiplying by a series of $\alpha$, and for this reason, this multiplier is referred to below as ``updater''.
In the figure, the superscript on $\alpha$ and $A$ represents the iteration index $i$, which can be assigned values of (0, 1, \ldots, $N-1$), and the subscript on $\alpha$ represents the rod number $e$, which takes values of ($e$ = 1, 2, 3, 4, 5, 6) in Case 1. 
Since the $\alpha$ is the design variable, QAOA is responsible for determining the value of the updater for each rod in each iteration based on the objective function. The range of updater is specified as 
%%%%%
\begin{equation}
   0 < \alpha \leq \theta \label{eq8_upda}{,}
\end{equation}
%%%%%
where the $\theta$ acts as an upper limit that the updaters can be set to within each iteration. In this framework, this value can be customized in a specific problem. It is obvious that when $\alpha > 1.0$, the cross-sectional area of the corresponding rod will be increased in this iterative process, and vice versa. For example, if $\alpha_1^{(0)}$ is optimized to 1.1, the area of rod 1 will be $1.1 \times A^{(0)}$; if $\alpha_2^{(0)}$ is optimized to 0.5, the area of rod 2 will be $0.5 \times A^{(0)}$. 
%This strategy encourages the truss structure to efficiently allocate material on each rod subject to the volume constraint. 
This operation efficiently allocates materials to each rod of the truss structure to meet volume constraints.
After $N$ iterations, the cross-sectional area of rod $e$ is updated such that
%%%%%
\begin{align}
   A_e^{N-1} &= \alpha_e^{(N-1)} \cdot\alpha_e^{(N-2)} \cdots \alpha_e^{(2)} \cdot\alpha_e^{(1)} \cdot\alpha_e^{(0)} \cdot A^{(0)}\label{eq9_updato}
   \\ 
   &= \prod_{i=0}^{N-1}{\alpha_e^{(i)} A^{(0)}} \label{eq10_updato}{.}
\end{align}
%%%%%
This update will ultimately be reflected in the stiffness matrix of this rod as
%%%%%
\begin{align} 
   \bm{K}_{e}^{(N-1)}= \prod_{i=0}^{N-1}{\alpha_e^{(i)} \bm{K}_{e}^{(0)}} =  \prod_{i=0}^{N-1} {\alpha_e^{(i)}} \frac{E{A}^{(0)}}{L_e} \bm{J}
%\quad \mbox{with} \quad 
%\bm{J}:= 
%\begin{bmatrix} 1 &-1 \\ -1& 1 \end{bmatrix} }
\label{eq11_updak}{,}
\end{align}
%%%%%
where $L_e$ is the length of rod $e$, and $\bm{J}$ is a square matrix of order 4 whose constant components are determined by the orientation of the rod axis. 

It should be noted that the conventional method for structural optimization such as the optimality criteria (OC) method\cite{Andreassen2011matlab} can easily define the updater $\alpha$ as a continuous variable. However, in the current situation, this cannot be easily implemented on a quantum processor for both QA and GQC. 
%Currently, the generally accepted method is to employ an encoding operation to represent variables with multiple binary variables (qubits) as discrete variables. 
In fact, the generally accepted method requires an encoding operation that uses multiple binary variables (qubits) as discrete variables.
Ideally, a discrete variable encoded by a sufficient number of qubits could be rich enough to an approximate continuous variable. 

%There have been several encoding operations 
So far, several encoding methods have been proposed, including one-hot encoding, binary encoding, on-off encoding\cite{endo2024encod}, etc. 
Xu et al.\cite{Xu2024Gate} proposed a non-iterative framework based on QAOA and implemented an algorithm for truss optimization with one-hot encoding. However, their method differs from ours in that they directly set the cross-sectional area of each rod as a design variable. 
The area of a rod $e$ is then encoded by multiplying a series of qubits using a pre-determined corresponding candidate vector $\bm{r}$
% determined in advance 
as follows: 
%%%%%
\begin{equation}
       A_{e} = \sum_{m=1}^{M}{r_{m}q_{e, m}} \text{ }(q \in \{0, 1\})\quad \textrm{s.t.} \quad \sum_{m=1}^{M}{q_{e, m}} =1
       \label{eq12_onehotencod}{,}
\end{equation}
%%%%%
where $M$ represents the number of pre-determined candidate values, $q$ refers to the qubits (binary variables) adopted for the rod $e$. Because one-hot encoding strictly enforces that for a given rod $e$, only one qubit in the associated set can be in state ``1". Therefore, it lacks a combinatorial effect, and $m$ candidate values should be expressed using $m$ qubits.
In their study, since five candidates were assigned to each rod for a simple truss structure consisting of three rods, fifteen qubits were required in total. 
%In addition, 
On the other hand, Sukulthanasorn et al.\cite{sukulthanasorn2024quantum} proposed a framework for truss and topology optimization using quantum annealing, in which only one qubit per rod or element is used to control the updater value. 
In the structural optimization of trusses using this method, the state ``1" represents an increase in the cross-sectional area of the rod, while the state ``0" signifies a decrease. This approach saves qubits, but when the truss volume approaches the upper limit of the volume constraint, there are not enough candidates to maintain a constant cross-sectional area. 
When the truss volume approaches the volume constraint, the design iteration is stopped manually by an appropriate criterion. 

In the framework proposed in this study, on-off encoding is adopted to encode the updater. 
In on-off encoding, the update data can be encoded as a linear sum of predefined digits, and the ``0" and ``1" states of each qubit can be controlled without constraint.
Specifically, for a certain rod $e$, the updater $\alpha_e$ is encoded as
%%%%%
\begin{equation}
       \alpha_{e} = \sum_{m=1}^{M}{r_{m}q_{e, m}} \text{ }(q \in \{0, 1\})
       \label{eq13_onoff}{.}
\end{equation}
%%%%%
%It can be seen that on-off encoding encodes the updater based on a linear summation of the predefined digits by controlling the ``0" and ``1" states of each qubit without any constraint. 
It is clear that on-off encoding has a combinatorial effect,
%Obviously, on-off encoding is a method with combinatorial effect, 
and $m$ qubits can encode $2^m$ candidate values. 
For example, if the predefined candidate vector is $\{0.1, 1.0\}$ in on-off encoding using two qubits, the updater with fousr options is encoded as $\{0.0, 0.1, 1.0, 1.1\}$.
%in on-off encoding, the updater with four options using two qubits is encoded as $\{0.0, 0.1, 1.0, 1.1\}$.} on-off encoding can encode the updater with 4 options as $\{0.0, 0.1, 1.0, 1.1\}$ using 2 qubits. 
%In this study, two qubits are deployed for each updater on the truss. 
In this study, as in the case of this example, two qubits are assigned to update the cross-sectional area of one rod in the truss structure. If one of the options is set to 1.0, QAOA does not change the cross-sectional area, so the truss structure automatically and stably converges as the truss volume approaches the volume constraint.
%By configuring one of the options as 1.0, QAOA can thus maintain the cross-sectional areas, ensuring the truss structure to automatically and stably converge as the truss volume approaches the volume constraint.


It should be noted that as the number of qubits $m$ increases, the updater can be encoded with richer expressive performance, %which transits it towards continuous variables. 
and it becomes possible to express it close to a continuous variable. 
However, this comes at the cost of an exponential expansion (to the power of 2$m$, i.e., $2^m$) in the number of search combinations.
%of the search combinations. 
The reason for encoding each updater with two qubits in this study is that we considered the
%In this study, each updater is encoded by 2 qubits, which is a 
trade-off between automatic convergence and qubit consumption. 
%Fortunately, since our framework iteratively updates the truss, and each rod is multiplied by a coefficient during an iteration. 
Fortunately, in our framework, the cross-sectional area of a rod is updated iteratively by multiplying the updater $\alpha$, so the actual representation of the cross-sectional area itself gradually becomes richer with each iteration.
In fact, the research results of Sukulthanasorn et al.\cite{sukulthanasorn2024quantum} showed that 
even with only two qubits, the change in the cross-sectional area of a rod achieved sufficient flexibility. 
At the very least, it can be said that it is not absolutely necessary to allocate additional qubits to a single rod. 
Therefore, we believe that the framework we propose can be applied to large-scale structural optimization problems and that it can also save qubits. 
In the next section, we will introduce the definition of the QAOA objective function in detail.

%%%%%%%%%%%%%%%%%%%%%%%%%%%%%%%%%%%%%%%%%%%%%%
%++++++++++++++++++++++++++++++++++++++++++++++++++++++++
\subsection{Objective function for QAOA}\label{subsec33}
%++++++++++++++++++++++++++++++++++++++++++++++++++++++++
As mentioned in \secref{subsec32}, the purpose of iterative updating is to minimize the compliance of a truss structure while satisfying a volume constraint.
Assuming that the target truss structure has a total of $N_{e}$ rods, the optimization problem that maximizes mean compliance can be formulated in the following standard format:
%%%%%
\begin{equation}
%\renewcommand{\arraystretch}{1.5} % 调整行间距为 2 倍
    \left\{
    \begin{array}{ll}
\text{find : } &\alpha \in \{ \alpha_1, \alpha_2, \ldots, \alpha_e, \ldots, \alpha_{N_e}\}\\[2ex]
\underset{\alpha_e}{\text{min : }} & \bm{F}^{T}\bm{U} 
 \text{ (compliance) }\\[2ex]
\text{s.t. : } &\bm{K}(\alpha)\bm{U} = \bm{F}\\[1ex]
   &\sum_{e=1}^{N_{e}}{V_e(\alpha_e)} \leq V^{(0)}\\[1ex]
   & 0 < \alpha_e \leq \theta\label{eq14_objective}
\end{array}{,}
    \right.
\end{equation}
%%%%%
where $\bm{F}$ is the external applied load vector, $\bm{U}$ is the global nodal displacement vector in matrix structural analysis, and $\bm{K}(\alpha)$ is the global stiffness matrix that depends on the updater $\alpha$ as in Eq. \eqref{eq11_updak}. 
Also, $\sum_{e=1}^{N_{e}}{V_e(\alpha_e)}$ refers to the volume of the entire truss structure in a specific iteration and is constrained so that it does not exceed the initial volume $V^{(0)}$. 
This is exactly the volume constraint in our framework. Usually, a slack variable $S$ is introduced to incorporate the volume constraint into the objective function. 
Specifically, the original inequality constraints can be rewritten as the following equality constraint by the slack variable: 
%%%%%
\begin{equation}
f_\textrm{cons} := \frac{\sum_{e=1}^{N_{e}}{V_e(\alpha_e)}}{V^{(0)}} + S -1 = 0\label{eq16_slack}{.}
\end{equation}
%%%%%
%In essence, 
Additionally, since minimizing compliance is equivalent to maximizing structural stiffness, 
%. Accordingly, Eq. \eqref{eq14_objective} can be reorganized as
the objective function in Eq. \eqref{eq14_objective} can be rewritten as 
%$\bm{F}^{T}\bm{U}\bm{K}(\alpha) =\bm{U}^{T}\bm{K}(\alpha)\bm{U}$ 
%{that can be evaluated on an element-by-element basis as follows: } 
\begin{align}
\bm{F}^\top \bm{U} \bm{K}(\alpha) & =\bm{U}^\top \bm{K}(\alpha) \bm{U} 
%\nonumber \\
%&
=
\sum_{e=1}^{N_e}{ \bm{U}_e^\top \frac{A_e(\alpha_e) E}{L_e} \bm{J} \bm{U}_{e}} {,} 
\end{align}
where $\bm{U}_e$ denotes the element displacement vector in matrix structural analysis conducted on a classical processor. 
%%%%%
%\begin{equation}
%f(\alpha_e) = 
%\underset{\text{Compliance}}{\bm{F}^{T}\bm{U}\vphantom{\bm{K}(\alpha)}} := \text{ }\underset{\text{Stiffness}}{\bm{U}^{T}\bm{K}(\alpha)\bm{U}}{.}\label{eq15_reorg}
%\end{equation}
%%%%%

%Jointly solving Eqs. \eqref{eq14_objective}-\eqref{eq16_slack}, the objective function can be reformulated as
Using the above expressions for equality constraint and stiffness-based objective function, we rewrite the optimization problem in Eq. \eqref{eq14_objective} as
%%%%%
\begin{equation}
%\renewcommand{\arraystretch}{1.5} % 调整行间距为 2 倍
    \left\{
    \begin{array}{ll}
\text{find : } &\alpha \in \{ \alpha_1, \alpha_2, \ldots, \alpha_e, \ldots, \alpha_{N_e}, S\}\\[2ex]
\underset{\alpha_e}{\text{min : }} & 
 -\bm{U}^{T}\bm{K}(\alpha)\bm{U} + \lambda\left(\dfrac{\sum_{e=1}^{N_{e}}{V_e(\alpha_e)}}{V^{(0)}} + S -1\right)^2 := f_\textrm{obj}\\[2ex]
\text{s.t. : } &\bm{K}(\alpha)\bm{U} = \bm{F}, \text{ } 0 < \alpha_e \leq \theta\label{eq17_ob2}
\end{array} {.}
    \right.
\end{equation}
%%%%%
Here, the negative sign preceding the first term of the objective function arises from the reformulation of the minimization into a maximization problem, aiming to identify the updater that maximizes stiffness. Also, $\lambda$ is the weight coefficient for the volume constraint defined according to the problem scenario.
The definition of $\theta$ remains the same as in Eq. \eqref{eq8_upda}, serving as the upper limit that that can be set within each iteration.
%the updaters can be set {to} within {each} iteration.

%According to the description of on-off encoding 
As described in the previous section, each updater is encoded using on-off encoding with two qubits as in Eq. \eqref{eq13_onoff}, and the pre-determined candidate vector $\bm{r}$ is defined as $\{0.1, 1\}$. 
Thus, totally four options can be generated for each updater as $\alpha_e \in \{0.0, 0.1, 1.0, 1.1\}$. 
According to Eq. \eqref{eq8_upda} and Eq. \eqref{eq13_onoff}, $\theta$ is typically set to 1.1. 
%It is worth mentioning 
It should be noted that when QAOA optimizes the value of a specific updater to 0.0, the proposed method replaces this zero with a random number $\epsilon$ in the range $[1 \times 10^{-10}, 2\times 10^{-10}]$ to prevent the stiffness matrix from becoming singular.
%will be added to prevent a singular matrix. 
Note that the slack variable likewise needs to be encoded as a discrete variable by multiple binary variables through on-off encoding. In this study, this is realized by another set of qubits $q_c$ as
%%%%%%%%%%%
\begin{equation}
S(q_c) = \frac{\sum^{N_c}_{c=1}{k_cq_c}}{\sum^{N_c}_{c=1}{k_c}}\label{eq18_slackencod}{.}
\end{equation}
%%%%%%%%%%%
Here, $N_c$ is the total number of qubits used to encode the slack variable, $k_c \;(c=1, \ldots, N_c)$ are the coefficients constructing the candidate vector $\bm{k}$ pre-determined to encode the slack variable. Generally, the components in $\bm{k}$ are often defined as the exponential form of 2, such as 2, $2^{2},\ldots, 2^{N_c}$. In this study, to reduce the total qubit consumption, only two qubits are used to encode the slack variable, i.e., $N_c = 2$ and $\bm{k} = \left\{2, 2^2 \right\}$. The denominator in Eq. \eqref{eq18_slackencod} is used to normalize the slack variable to the range [0, 1]. 

By using Eq. \eqref{eq13_onoff} and Eq. \eqref{eq18_slackencod} in Eq. \eqref{eq17_ob2}, the objective function with binary variables, $q_e$ and $q_c$, can be expressed as
%%%%%%%%%%%
\begin{align}
   f_\textrm{obj}(q_e, q_c) &= f (q_e) + \lambda f_\textrm{cons}(q_c) \nonumber \\
    & = -\sum_{e=1}^{N_e}{ \bm{U}_e^\top \frac{ \sum_{m=1}^{M} r_{m} q_{e, m} \hat{A}_e E}{L_e} \bm{J} \bm{U}_{e}} + \lambda \left(\frac{\sum_{e=1}^{N_{e}}{V_e(q_e)}}{V^{(0)}} +\frac{\sum^{N_c}_{c=1}{k_cq_c}}{\sum^{N_c}_{c=1}{k_c}} -1 \right)^2{,}
    \label{eq20_obj3}
\end{align}
%%%%%%%%%%%
%Here, $\bm{U}_e$ denotes the element displacement vector in matrix structural analysis conducted on a classical processor. 
where $\hat{A}_e$ is an updated cross-sectional area of rod $e$ in the previous iteration.
%of the rod $e$, which is solved by {matrix} structural analysis 
If the problem Hamiltonian $C$ in Eq. \eqref{eq1_comb} is constructed in the form of Eq. \eqref{eq20_obj3}, a quantum processor can be employed to solve this optimization problem. This step corresponds to the grey arrow between iterations in \fref{fig:update}. 
To provide a more intuitive and clear understanding, a detailed process and flowchart will be provided in the next section.

%%%%%%%%%%%%%%%%%%%%%%%%%%%%%%%%%%%
\begin{figure}[htbp]
    \centering
    \includegraphics[width=0.6\linewidth]{Figures/Fig4_flowchart.eps}
    \caption{Flowchart for the proposed framework.
}
\label{fig:flowchart}
\end{figure}
%%%%%%%%%%%%%%%%%%%%%%%%%%%%%%%%%%%

%%%%%%%%%%%%%%%%%%%%%%%%%%%%%%%%%%%%%%%%%%%%
%++++++++++++++++++++++++++++++++++++++++++++++++++++++++
\subsection{Overall process of proposed framework}\label{subsec34}
%++++++++++++++++++++++++++++++++++++++++++++++++++++++++
%To concisely illustrate the proposed framework, we delineate the specific procedural steps as follows:
To briefly describe the proposed framework, the specific procedures are as follows:
\begin{enumerate}
    \item \textit{Matrix structural analysis.} For prescribed updaters $\alpha_e$, perform matrix structural analysis on a classical processor to obtain basic unknowns, i.e., nodal displacement vector $\bm{U}$, which is the solution of the static equilibrium equation $\bm{K}(\alpha) \bm{U} = \bm{F}$ in Eq. \eqref{eq14_objective}.
%
    \item \textit{Encoding.} Encode the updaters $\alpha_e$ in Eq. \eqref{eq13_onoff} and the slack variable $S$ in Eq. \eqref{eq18_slackencod} with binary variables $q_e$ and $q_c$, respectively.  
%
    \item \textit{Prepare QAOA circuit.} Define the objective function in Eq. \eqref{eq20_obj3} and transform it to the corresponding Hamiltonian $C$ in Eq. \eqref{eq1_comb} for $\tilde{U}_{C}(C, \gamma)$ in the QAOA quantum circuit in Eq. \eqref{eq2_qaoa1}. 
    %. $\rightarrow$ Eq. \eqref{eq1_comb}, Eq. \eqref{eq20_obj3}, Eq. \eqref{eq2_qaoa1}
%
    \item \textit{Parameter fine-tuning and measurement.} Employ conventional optimization procedures or heuristic approach (COBYLA, Bayesian optimization, or FLRS) to adjust the variational parameters such that $(\bm{\gamma}^{*}, \bm{\beta}^{*})= \textrm{arg} \max_{\bm{\gamma}, \bm{\beta}}f(C, \ket{\psi_f})$ (Eq. \eqref{eq7_para}). 
    Then, measure the final quantum state to collapse it into a qubit string $z = z_1z_2 \cdots z_n$. 
    %to represent a potential solution in Eq. \eqref{eq6_exp}. 
%
    \item \textit{Structure update.} Decode the updaters $\alpha_e$ from the qubit string and multiply each decoded updater by the corresponding element stiffness matrix of Eq. \eqref{eq11_updak} in the previous iteration step to update the structure. 
    % ($A_e^{(0)} \rightarrow A_e^{(1)}$). $\rightarrow$ Eqs. \eqref{eq9_updato}-\eqref{eq11_updak}
%
    \item \textit{Iteration.} Repeat steps 1 to 5 until convergence is achieved, i.e., the specified conditions are met.
    %. $\rightarrow$ \fref{fig:update}
\end{enumerate}
The corresponding flowchart is shown in \fref{fig:flowchart}, and \fref{fig:update} schematizes the progression of optimization as iterations are made.


%%%%%%%%%%%%%%%%%%%%%%%%%%%%%%%%%%%%%%%%%%
\section{Numerical examples}\label{sec4}
%%%%%%%%%%%%%%%%%%%%%%%%%%%%%%%%%%%%%%%%%%%
%In \secref{sec3}, a detailed description of the proposed framework is provided. 
In this section, two numerical examples are presented to demonstrate the capabilities of the proposed framework with QAOA. 
% and explore classical optimizers for determining variational parameters.
In the first example, the framework is applied to Case 1 of the six-member truss shown in \fref{fig:truss} to compare three fine-tuning approaches for determining the variational parameters, $(\bm{\gamma}^{*}, \bm{\beta}^{*})$, in Eq. \eqref{eq2_qaoa1}.
% and determine the best one. 
Two of them are classical optimizers, COBYLA and Bayesian optimization (hereafter referred to as ``B-opt"), and the other is FLRS, which directly introduces a fixed linear ramp schedule for all variational parameter settings. 
As discussed in Section 2, we propose to use FLRS\cite{Montanez2024protocol}, and the purpose of this example is to illustrate its performance. 
%As mentioned in Section 2, this study adopts FLRS, which directly introduces a fixed linear ramp schedule for all variational parameter settings.
%For the truss structure of Case 1 in \fref{fig:truss}, optimization is performed using three different classical optimizers to determine the best one. 
%In this section, we compare the performances of different classical optimizers based on the Case 1 in \fref{fig:truss}. After determining the best classical optimizer, 
%Then, the best optimizer is used to perform optimization for the truss structure of Case 2. 
In the second example, structural optimization with QAOA using FLRS is carried out on the truss structure in Case 2 to verify the performance of the proposed framework. 
We also propose an improvement to our proposed framework and discuss its time complexity theoretically. 

%%%%%%%%%%%%%%%%%%%%%%%%%%%%%%%%%%%
\begin{figure}[htbp]
    \centering
    \includegraphics[width=0.4\linewidth]{Figures/Fig5_parameter.eps}
    \caption{Variational parameter setting for FLRS.
}
\label{fig:parameter}
\end{figure}
%%%%%%%%%%%%%%%%%%%%%%%%%%%%%%%%%%%
%%%%%%%%%%%%%%%%%%%%%%%%%%%%%%%%%%%
\begin{figure}[htbp]
    \centering
    \includegraphics[width=0.5\linewidth]{Figures/Fig6_optimize.eps}
    \caption{MAPE for 10 executions using three different fine-tuning approaches.
}
\label{fig:optimizer}
\end{figure}
%%%%%%%%%%%%%%%%%%%%%%%%%%%%%%%%%%%

%++++++++++++++++++++++++++++++++++++++++++++++++++++++++
%\subsection{{Optimizer for determining variational parameters}}
\subsection{A comparative study of parameter fine-tuning approaches for QAOA}
\label{subsec41}
%++++++++++++++++++++++++++++++++++++++++++++++++++++++++

As discussed in \secref{sec2}, among the classical optimizers for QAOA, COBYLA and B-opt are the most popular and widely used to identify appropriate patterns of variational parameters. 
%{Since these two optimization methods are well known, we omit their optimization details here. {It is a repetition of Line 326 'Note that COBYLA and...', so it is recommended that only one of these be retained.}} 
Meanwhile, some studies have also attempted to reduce the computational burden of using classical optimizers when the QAOA contains many layers. 
%As mentioned in Section 2, 
Among them, this study adopts FLRS \cite{Montanez2024protocol}, which directly introduces a fixed linear ramp schedule for all variational parameter settings.
In the following, these parameter fine-tuning approaches, COBYLA, B-opt, and FLRS, are applied to the $p$-layer QAOA circuit to optimize the configuration of rods for Case 1 in \fref{fig:truss}. 

%++++++++++++++++++++++++++++++++++++++++++++++++++++++++
%\subsection{{Truss optimization for Case 1}}\label{subsec41}
%\subsection{{Comparison of parameter fine-tuning approaches for QAOA}}
%++++++++++++++++++++++++++++++++++++++++++++++++++++++++
%\subsubsection{{Comparison of parameter fine-tuning strategies for QAOA}}
%In \secref{sec2}, the current research status of classical optimizers in QAOA are introduced. Among them, the widely used methods include COBYLA and Bayesian optimization (hereinafter referred to as ``B-opt"). Given the widespread application and familiarity of these two optimizers, the optimization details will not be provided in this study. In addition, some studies endeavored to identify the patterns of the optimal variational parameters $(\bm{\gamma}^{*}, \bm{\beta}^{*})$ to alleviate the computational burden on classical optimizers when QAOA involves a large number of layers $p$. Among them, FLRS\cite{Montanez2024protocol}, which directly deploys a fixed linear ramp schedule for all variational parameter settings, is adopted in this study. 


The number of layers, $p$, is set at 6 in this study, suggesting that there are a total of 12 variational parameters to be optimized. 
Additionally, when defining the objective function in Eq. \eqref{eq20_obj3}, 
%a coefficient $\lambda$ is required to be assigned to the weight of the volume constraint. In Case 1, this value is set at $5 \times 10^{-1}$. 
a coefficient $\lambda$ must be set as the weight of the volume constraint, and after trying several values in Case 1, $\lambda=5 \times 10^{-1}$ is adopted.
%COBYLA and B-opt both require initial values to start the optimization process. 
Note that COBYLA and B-opt need to set initial values for the variational parameters to start the optimization process.
The difference is that the former requires a single set of variational parameters, while the latter usually requires multiple sets. 
%For a specific problem, there is no standard rule to determine the initial values in advance. 
There is no standard rule for determining initial values in advance for a particular problem.
Random sampling is a commonly used method, but it should be noted that this can lead to uncertainty in the solution obtained.
%it hence also introduces uncertainty regarding the setting of initial values.
%COBYLA is initialized at random in $[0, 2\pi]$, and the remaining settings are based on the default values in Scipy 1.14.1 \cite{2020SciPyNMeth}(the commonly used Python library for implementing COBYLA). 
In this particular example problem, COBYLA is randomly initialized within $[0, 2\pi]$, and the remaining settings are based on default values from \texttt{Scipy} 1.14.1 \cite{2020SciPyNMeth}, which is a popular Python library for implementing COBYLA. 
%Also, B-opt is implemented into our in-house code based on \texttt{scikit-optimize} 0.8.1\cite{sciopt081}. Here, \texttt{scikit} is a popular machine learning Python library that is used as a subroutine in our code. 
On the other hand, in this study,
the number of initial observations for B-opt is set to 10, and a total of 100 Bayesian optimization steps are performed. 
According to Kim and Wang \cite{Kim2023to}, 
%the kernel function is set to 
the the following Matern kernel\cite{genton2001classes} can be employed for B-opt as
%%%%%%%%
\begin{equation}
   k\left(x_i, x_j\right)=\frac{1}{\Gamma(\nu) 2^{\nu-1}}\left(\frac{\sqrt{2 \nu}}{l} d\left(x_i, x_j\right)\right)^\nu K_\nu\left(\frac{\sqrt{2 \nu}}{l} d\left(x_i, x_j\right)\right){,} \label{eq20_Matern} 
\end{equation}
%%%%%%%%
where $d(\cdot, \cdot)$ is the Euclidean distance, $K_\nu(\cdot)$ is a modified Bessel function, and $\Gamma(\cdot)$ is the gamma function. In the following numerical example, the smoothing parameter $\nu$ is set to 0.5, and Upper Confidence Bound (UCB) \cite{Peter2003ucb} is adopted as an acquisition function.
B-opt with such a specification is implemented into our in-house code based on \texttt{scikit-optimize} 0.8.1\cite{sciopt081}. Here, \texttt{scikit} is a popular machine learning Python library that is used as a subroutine in our code. 

Fixed linear ramp schedule (FLRS), on the other hand, assumes a pattern of variational parameters in advance.
According to Montanez-Barrer and Michielsen\cite{Montanez2024protocol}, the variational parameters in the $p$-layer QAOA circuit can be linearly defaulted as 
% and the follow formulations.
%%%%%%%%%
\begin{equation}
    \beta_i = \left(1-\frac{i}{p}\right)\Delta\beta \quad \text{and} \quad \gamma_i = \frac{i+1}{p}\Delta\gamma,
    \quad (i = 0, \ldots, p), 
\label{eq21_vari}
\end{equation}
%%%%%%%%%
where $\Delta\beta$ and $\Delta\gamma$ represent scaling factors and are supposed to be defined individually according to the nature of the particular problem. In this study, we set both two at 1.0. 
The pattern for $p=6$ is shown in \fref{fig:parameter}. 


%In order to judge the performance of the above optimization methods, a suitable metric must be specified. 
In order to determine the performance of the above three different fine-tuning approaches, an appropriate indicator needs to be defined. 
%We first performed {truss optimization} based on the optimality criteria (OC) method on a classical processor to minimize the compliance, then defined its result as the optimal solution $y$ and 
First, we will perform truss optimization based on the OC method on a classical processor, and the result is called the reference optimal solution and denoted by $y$. Preliminarily, $y \approx 0.13$ was calculated. 
%Next, 10 independent optimization runs are performed by QAOA with each optimization method, given a general name $f(\alpha)$ to the outcomes. 
Next, 10 independent optimization runs are performed by QAOA using each approach, and the result is generically denoted by $f(\alpha)$ where $\alpha$ is the approach identifier. 
Then, mean absolute percentage error (MAPE) is used as a measure to evaluate the discrepancy between $y$ and $f(\alpha)$ as
%%%%%%%%%%
\begin{equation}
    MAPE\; [\%]= \left|\frac{f(\alpha)-y}{y}\right| \times 100{.} \label{eq22_mape}
\end{equation}
%%%%%%%%%%


The results of 10 independent runs (a total of 30 execution cases) are shown in \fref{fig:optimizer}. 
%To ensure the results can be displayed on the same figure, in cases where the MAPE exceeds 100, it will be forcibly set to 100.
In this figure, MAPE is forced to be set to 100 if it exceeds 100 so that all the results can be displayed in the same vertical range.
%It is worth mentioning that the FLRS method directly sets all the variational parameters, so its uncertainty only comes from the measurement of the QAOA circuit. 
It should be noted that the FLRS-based method sets default values for all variational parameters in advance, so its uncertainty arises only from the QAOA circuit measurements.
%Since a sufficiently large number of measurements ($1e5$) are defined, this uncertainty is insignificant. In addition to the uncertainty due to measurement, the other two optimization methods also have uncertainties due to their different initial observation arrangements. 
Since a sufficient number of measurements ($1e5$) have been set up, the uncertainty due to measurements is considered insignificant for FLRS. In contrast, the other two optimization methods have, in addition to the uncertainty due to measurements, uncertainty due to different initial parameter sets. 
As can be seen in the figure, the B-opt-based and COBYLA-based QAOA show disappointing performance, struggling to produce acceptable results in 10 independent executions. 
%In contrast, the FLRS-based QAOA shows low-level errors. This slight discrepancy arises from the encoding of the updater with binary variables. If more qubits are utilized to encode the updaters for each rod, this discrepancy will be further minimized.
In contrast, the FLRS-based QAOA exhibits a low level of error, which resulted from the process of encoding the updater with binary variables. 
%Based on the above comparison, we have determined the effectiveness of the FLRS method in our framework, particularly when dealing with a large number of variational parameters. 
From the above comparative results, it is concluded that the FLRS method has proven to be effective, especially when dealing with many variational parameters.
%Moreover, the time cost is significantly lower than that of COBYLA and B-opt, as it directly sets the variational parameters, effectively bypassing the optimization process of other classical optimizers. 
In addition, because FLRS sets the variational parameters directly, it effectively bypasses the optimization process of other classical optimizers, resulting in a much lower time cost than COBYLA or B-opt.


%%%%%%%%%%%%%%%%%%%%%%%%%%%%%%%%%%%
\begin{figure}[htbp]
    \centering
    \includegraphics[width=0.7\linewidth]{Figures/Fig7_11truss_try_c1.eps}
    \caption{Iterative updating process in Case 1 with FLRS-based QAOA.
}
\label{fig:truss11}
\end{figure}
%%%%%%%%%%%%%%%%%%%%%%%%%%%%%%%%%%%
%%%%%%%%%%%%%%%%%%%%%%%%%%%%%%%%%%%
\begin{figure}[htbp]
    \centering
    \includegraphics[width=0.7\linewidth]{Figures/Fig8_truss11process.eps}
    \caption{Convergence result in Case 1 wtih FLRS-based QAOA.
}
\label{fig:11process}
\end{figure}
%%%%%%%%%%%%%%%%%%%%%%%%%%%%%%%%%%%

%\subsubsection{{Fixed linear ramp schedule for optimizing a 6-member truss}}
In the following, we will close this section by showing the optimization result of the FLRS-based QAOA.
%In the following numerical case, FLRS is employed as the classical optimizer for TO. 
\fref{fig:truss11} illustrates the specific update process for a six-member truss structure. 
%It is evident that QAOA effectively distinguishes between critical and non-critical rods, optimizing the truss material distribution accordingly. 
It can be seen that the proposed approach effectively discriminates between critical and non-critical rods and optimizes the distribution of truss members accordingly. 
\fref{fig:11process} shows the convergence trend of the objective function, compliance, and the ratio of total volume to initial volume.
As mentioned earlier, each updater is represented only by two qubits, so four options of $\{\epsilon, 0.1, 1.0, 1.1\} \; (\epsilon \in (1 \times 10^{-10}, 2 \times 10^{-10}))$ are provided.
%Since each updater is represented by only two qubits, the four options provided are $\{\epsilon, 0.1, 1.0, 1.1\} (\epsilon \in (1e-10, 2e-10))$. 
%When the updater is optimized to $\epsilon$, the corresponding rods are almost removed instantly, resulting in a certain mutation in the structure volume. 
%Notably, when updater is optimized for $epsilon$, the corresponding rod is considered deleted. 
It is worth noting that when the updater is optimized to $epsilon$, the corresponding rod is considered deleted, but a little volume is left for numerical stability.
After convergence, the volume of the structure slightly exceeds the volume constraint, partly because of this, but also due to encoding, which does not negate the proposed framework. 
%which also stems from encoding and unrelated to the performance of QAOA itself.


%%%%%%%%%%%%%%%%%%%%%%%%%%%%%%%%%%%
\begin{figure}[htbp]
    \centering
    \includegraphics[width=0.7\linewidth]{Figures/Fig9_22truss_try_c.eps}
    \caption{Iterative updating process in Case 2 with FLRS-based QAOA.
}
\label{fig:truss22}
\end{figure}
%%%%%%%%%%%%%%%%%%%%%%%%%%%%%%%%%%%
%%%%%%%%%%%%%%%%%%%%%%%%%%%%%%%%%%%
\begin{figure}[htbp]
    \centering
    \includegraphics[width=0.75\linewidth]{Figures/Fig10_truss22process_c.eps}
    \caption{Convergence result in Case 2 with FLRS-based QAOA.
}
\label{fig:22process}
\end{figure}
%%%%%%%%%%%%%%%%%%%%%%%%%%%%%%%%%%%

%%%%%%%%%%%%%%%%%%%%%%%%%%%%%%%%%%%%%%%%%%%%%
%++++++++++++++++++++++++++++++++++++++++++++++++++++++++
\subsection{Truss optimization for Case 2}\label{subsec42}
%++++++++++++++++++++++++++++++++++++++++++++++++++++++++
%According to the analysis in the previous section, 

The discussion in the previous section validated the effectiveness of the FLRS method.
Therefore, in this section, only the FLRS-based QAOA is applied to Case 2 of the 11-member truss in \fref{fig:truss}.
%Case 2 in \fref{fig:truss} targeted  optimized based on FLRS-QAOA. 
%In accordance with the definition of the objective function in this framework, a total of $24 \; (=11 \times 2 + 2)$ qubits are adopted since Case 2 contains 11 rods. 
Since Case2 contains 11 rods, a total of $24 \; (=11 \times 2 + 2)$ qubits would be employed according to the definition of the objective function in this framework.
The depth (number of layers) of the circuit is set at 8 in this case, and the weight coefficient $\lambda$ that controls the volume constraint is set to $4.25 \times 10^{-2}$, which has been determined by trial-and-error.


\fref{fig:truss22} shows the optimization process in the resulting structural update, together with the result from the OC method on a classical processor.
%the specific process of structure updating. 
As can be seen from this figure, the FLRS-based QAOA obtains the converged structure that is consistent with that of the OC method. 
\fref{fig:22process} shows the convergence trend of the objective function, compliance, and the ratio of total volume to initial volume. 
%exhibits the transitions in structure volume and compliance during the iterations. 
It can be observed that the volume ratio $V/V^{(0)}$ when using the FLRS-based OAOA shows a slight oscillation around the volume constraint of 1.
%One can observe that the structure volume $V$ exhibits slight oscillations near the volume constraint $V^{(0)}$. 
This is due to the limited number of qubits currently available to encode the updaters and slack variable.
%This arises from the restricted number of qubits currently available for encoding the slack variable and each updater. 
%In the cases of the structure volume approaches to $V^{(0)}$, QAOA is confronted with a trade-off: it must either increase the cross-sectional areas of critical rods and potentially result in a slight violation of the volume constraint, or eliminate critical rods to maintain strict adherence with the constraint. 
In fact, when the structure volume approaches $V^{(0)}$, QAOA faces a trade-off. That is, it must choose between slightly violating the volume constraint by increasing the cross-sectional area of the critical rod, or removing the critical rod to strictly satisfy the constraint.
Obviously, the former case would be relatively more acceptable.
%Evidently, the former approach is comparatively more acceptable. 
%With future advancements in quantum hardware scalability, particularly in terms of qubit capacity, such challenge is expected to be effectively mitigated. 
In the future, as the scalability of quantum hardware, especially the capacity of qubits, improves, such a challenge is expected to be effectively mitigated.
%In addition, the compliance of the truss structure optimized by the FLRS-based QAOA is even slightly lower than that of the OC method.
Furthermore, it can be seen from \fref{fig:22process} that the compliance of the truss structure optimized by the FLRS-based QAOA is slightly lower than the OC method.
Specifically, the minimum compliance obtained by the OC method was $3.16 \times 10^{-2}$. 
%which is due to the fact its converged volume exceeds the volume constraint by about $2\%$.
This is because the converged volume exceeds the volume constraint by about $2\%$, for the same reason as the result in the previous example problem.



%After solving two topology optimization cases based on the proposed framework, 
%Finally, we will further discuss a strategy called {“early stopping update”} that is used to reduce the total cost of the iterative process in QAOA.
Finally, we would like to introduce a strategy for reducing the total cost of the iterative process in QAOA.
%a strategy is further summarized to reduce the total cost of the iterative optimization, called ``early-stopping updating". 
As mentioned previously, when using two qubits, on-off encoding provides four options for each updater.
%Based on two qubits, on-off encoding provides four options $\{\epsilon, 0.1, 1.0, 1.1\} (\epsilon \in (1e-10, 2e-10))$ for the updater. 
%If the updater on a certain rod is optimized to $\epsilon$ by QAOA or optimized to $0.1$ for multiple rods, the cross-sectional area of this rod will be reduced to less than $1\%$ of the initial area $A^{(0)}$. 
%If the QAOA optimizes the updater for a particular rod to be equal to $\epsilon$ {or the updater corresponding to this rod is optimized to 0.1 in multiple iterations}, the cross-sectional area of this rod will decrease to less than $1\%$ of the initial cross-sectional area $A^{(0)}$ and approaches a negligible value. 

If the cross-sectional area of a particular rod becomes smaller than a negligible value during the iterative update of the QAOA, e.g. less than 1\% of the initial area $A^{(0)}$, there is no need to continue updating that area.
For example, if the corresponding updater is optimized to be $\{0,0\}$ in any iteration, the cross-sectional area of this rod is multiplied by $\epsilon$.
Similarly, if the updater has never been set to $\{0,0\}$ but is optimized to be $\{0,1\}$ multiple times in succession,  the cross-sectional area of this rod decreases by a factor of 0.1 many times to less than $0.01\times A^{(0)}$.

%During the iterative process, if such a situation arises, it will lead to the cross-sectional area of the corresponding rod approaching a negligible value. 
In such cases, stopping the further optimization of these rods can effectively reduce the total number of qubits required for subsequent the iterative process.
%Halting further optimization of these rods effectively reduces the total number of qubits required in subsequent iterations. 
As is well known, the query complexity of combinatorial optimization problems can be reduced exponentially by minimizing the number of qubits. 
%This strategy not only enhances the computational efficiency of the proposed framework but also improves its scalability and competitiveness when addressing large-scale topology optimization problems in future researches.
This strategy, called ``early stopping update'', not only improves the computational efficiency of the proposed framework, but also improves its scalability and competitiveness for dealing with large-scale topology optimization problems in future research.


%%%%%%%%%%%%%%%%%%%%%%%%%%%%%%%%%%%%%%%%%
%++++++++++++++++++++++++++++++++++++++++++++++++++++++++
\subsection{Discussion on time complexity }\label{subsec43}
%++++++++++++++++++++++++++++++++++++++++++++++++++++++++
Given the scale and noise level of current quantum processors, QAOA is still far from demonstrating quantum supremacy. 
%It is anticipated that, for a certain period in the foreseeable future, applications of TO with a certain scale will primarily rely on quantum simulators. 
Therefore, it is expected that for some period of time in the foreseeable future, applications to structural optimization at a certain scale will have to rely primarily on quantum simulators.
According to Crooks\cite{crooks2018performance}, the time complexity of QAOA applied to the Max-cut problem is $\textit{O}(Np)$ where $N$ and $p$ denote the number of qubits and the number of QAOA layers, respectively. 
Since the form of the objective function in this study is the same quadratic model as the Max-cut problem, this subsection will provide a comparative study on time complexity among
%\erase{the time complexity of QAOA based on quantum simulators is $\textit{O}(2^N)$, $N$ denotes the number of qubits. Next,} 
%we will conduct a comparative analysis of the time complexity among 
the frameworks of Kim and Wang\cite{Kim2023to}, Xu et al.\cite{Xu2024Gate}, and the proposed framework based on this estimate. 
For the sake of simplicity, we refer to the first two frameworks as K-W and Xu, respectively.

In K-W, each design variable is defined as a binary variable concerning
%the design variables  as binary variables related to whether or not
the inclusion or exclusion of the corresponding rod in the truss. 
Therefore, their method is non-iterative. 
The objective function is then constructed in the form of a ground structure optimization problem as
%%%%%%%%%
\begin{equation}
    \begin{aligned}
        f(q_1, \ldots, q_N) = & \text{ } (1-q_1)(1-q_2) \cdots (1-q_N)\delta(\bm{Q}_1)\\
        &+ q_1 (1-q_2) \cdots (1-q_N)\delta(\bm{Q}_2)\\ 
        &+ \cdots + q_1 q_2 \cdots q_N\delta(\bm{Q}_N){,}
    \label{eq25_kim3}
    \end{aligned}
\end{equation}
%%%%%%%%%
where $N$ is the number of qubits, which equals to the number of rods, and $\delta(\bm{Q}_i)$ is the displacement at the target node obtained by solving a linear system $\bm{KU} =\bm{F}$ in matrix structural analysis on a classical processor.  
%when a string $\bm{Q}_i$ corresponding to a array of specific binary values, $(q_1, \ldots, q_N)$, is given. 
Here, $\bm{Q}_i$ is a binary string consisting of $(q_1, \ldots, q_N)$ to represent a specific situation $i$. 
As can be seen from this equation, a classical processor is forced to face the tedious calculation of polynomial expansion, and there are $2^N$ items in total. 
%in this polynomial
%This definition method will force classical procssors to confront a tedious polynomial expansion problem. 
%Among the total $N$ terms in Eq. \eqref{eq25_kim3}, the complexity of expanding each term is $\textit{O}(2^{N-1})$, hence the time complexity on polynomial expansion is $\textit{O}(N\cdot 2^{N-1})$. 
% {{\bf Maybe OUTDATED}
% Since the complexity of expanding each term in Eq. \eqref{eq25_kim3} is $\textit{O}(2^{N-1})$, the time complexity of the polynomial expansion is $\textit{O}(N\cdot 2^{N-1})$.
% %Additionally, the B-opt is adopted as the classical optimizer. Assuming that the QAOA circuit needs to be accessed for $k$ times to optimize the variational parameters. Therefore, the time complexity of K-W framework is $\textit{O}(pkN+N\cdot2^{N-1})$, $k$ denotes as the steps of classical optimizer.
% Additionally, when the number of accesses to the QAOA circuit is $k$ to determine the variational parameters using B-opt, the time complexity is $\textit{O}(N\cdot 2^{N-1})$.
% }
The complexity of expansion varies from term to term. For example, the time complexity is 2 for $(1-q_1)q_2 \cdot q_N$ and $2^N$ for $(1-q_1)(1-q_2)\cdots(1-q_N)$.
Hence, the time complexity for the expansion becomes $\sum_{i=1}^{N}{^N\mathbb{C}_{i}2^{i}}$ where $^N\mathbb{C}_{i}$ are combinations with $N$ items taken $i$. 
%e the combination symbol is written as $\mathbb{C}$ to distinguish it from the $C$ in Eq. \ref{eq1_comb}. 
%lly, the B-opt is adopted as the classical optimizer. Assuming that the QAOA circuit needs to be accessed for $k$ times to optimize the variational parameters. Therefore, the total time complexity of K-W framework is $\textit{O}(pkN+\sum_{i=1}^{N}{^N\mathbb{C}_{i}2^{i}})$, $k$ denotes as the steps of classical optimizer.}
Assuming that the B-opt is adopted as a classical optimizer to optimize the variational parameters and that the QAOA circuit needs to be accessed $k$ times, the total time complexity of K-W is $\textit{O}(pkN+\sum_{i=1}^{N}{^N\mathbb{C}_{i}2^{i}})$.  


In Xu, the rod cross-sectional area is represented as a discrete variable using one-hot encoding, and an objective function is defined to minimize the volume of material used to meet the design requirements for load and deformation. 
Additionally, their framework is also non-iterative and requires $h \geq 4$ to ensure the representability of the encoding when using $h$ qubits.
%Suppose they encode each rod with $h$ qubits and $h \geq 4$ to ensure the expressiveness of the encoding. 
Furthermore, a gradient-based classical optimizer, Adam \cite{Adam2014}, is employed to determine variational parameters. 
%and takes $k$ steps to determine the variational parameters.
%Again, it is assumed that $k$ steps are required to optimized the variational parameters.
%\the number of iteration steps is $k$, }
%Therefore, 
%the time complexity of Xu is estimated to be $\textit{O}(pkhN)$. 
Therefore, if we let $k$ be the number of iterations in the optimization calculation using the classical processor, then Xu's time complexity is estimated to be $\textit{O}(pkhN)$.
%$k$ denotes as the steps of classical optimizer.

In the proposed framework, the structure is iteratively updated through updaters acting on the rods, and each updater and the slack variable for volume constraints are encoded by 2 qubits. 
By adopting FLRS, the framework avoids classical optimization.
In addition to this, the major difference from K-W and Xu is that our framework is an iterative design strategy, which requires $I_{\rm d}$ times access to the QAOA circuit
%which needs to access the QAOA circuit $I_{\rm d}$ times 
to achieve convergence. 
Furthermore, thanks to our early-stop updating strategy, the consumption of qubits in $I_{\rm d}$ iterations is constantly decreasing.
%Suppose that, up to a certain iteration, $n_{\Delta}$ rods no longer need to be updated. Hence 
Denoting $n_{\Delta}$ by the number of rods that are no longer updated in a given iteration by $n_{\Delta}$, 
%until a certain number of iterations,
%\fore,} 
the time complexity of our framework is estimated as follows:
%%%%%%%%%%
\begin{equation}
    O(pI_{\rm d} \cdot \sum_{n_{\Delta} \in \Omega} {2N+2-2n_{\Delta}}) \approx O(\zeta pI_{\rm d}N), 
    %\text{ } (1<\zeta<2){,}
    \label{eq28_expl}
\end{equation}
%%%%%%%%%%
where $\zeta$ is a coefficient that satisfies the inequality $(1<\zeta<2)$ determined from $N$ and $n_{\Delta}$.
%determined from  $N$ and $n_{\Delta}$ that satisfies the inequality FFF.
%between 1 and 2 by combining $N$ and $n_{\Delta}$. 

\begin{table*}[]
\renewcommand{\arraystretch}{1.5}
	\centering
	\caption{Comparison of the three frameworks in terms of qubit usage and time complexity.}
	\label{tab:1}  
	\begin{tabular}{cccc ccc}
		\hline\noalign{\smallskip}	
		Frameworks &Qubit usage&Time complexity&Time complexity with FLRS   \\
		\noalign{\smallskip}\hline\hline\noalign{\smallskip}
%		K-W framework& $N$ & $\textit{O}(pkN+N\cdot2^{N-1})$&$\textit{O}(pN+N\cdot2^{N-1})$ \\
		K-W framework& $N$ & $\textit{O}(pkN+\sum_{i=1}^{N}{_N\mathbb{C}_{i}2^{i}}$) & $\textit{O}(pN+\sum_{i=1}^{N}{_N\mathbb{C}_{i}2^{i}}$) \\
		Xu's framework& $hN$ & $\textit{O}(pkhN)$&$\textit{O}(phN)$ \\
		Our framework& $2N$ & $\textit{O}(\zeta pI_{\rm d}N)$&-\\
		\noalign{\smallskip}\hline
	\end{tabular}
\end{table*}

\tref{tab:1} provides the information on qubit usage and time complexity of the three frameworks. 
%In addition, considering that FLRS can be easily loaded into the other two frameworks, an additional column is also provided to indicate the time complexity when the FLRS also used in other two frameworks. 
In addition, given that FLRS can be easily implemented in the other two frameworks, we have also added a column showing the time complexity when FLRS is also used in that case.
As can be seen from this comparison table, K-W shows the lowest qubit usage, but has the highest time complexity because the polynomial expansions (Eq. \ref{eq25_kim3}), is computed by a classical processor.
%As can be seen in the \tref{tab:1}, although the K-W framework demonstrates the least qubit usage, it has the highest time complexity because the expensive polynomial expansions which implemented by a classical processor. 
%After ing} FLRS into Xu, the time complexity is not significantly different from that of the proposed framework, but it requires several times the qubit usage than the proposed framework. 
Using FLRS in Xu, the time complexity is not much different from that of our framework, but it requires several times as many qubits as ours.


% \erase{From the perspective of time complexity alone, the K-W framework seems to be the most efficient for large-scale TO problems. However, its non-iterative nature limits the flexibility of the structures. Moreover, QAOA faces challenges in achieving precision for extremely high-order objective functions.}

%In this sense, the proposed framework, which balances flexibility, accuracy, qubit usage and time complexity, represents the most robust selection.
From the above discussion, it can be concluded that the proposed framework is the most robust choice, balancing flexibility, accuracy, qubit usage, and time complexity.





%%%%%%%%%%%%%%%%%%%%%%%%%%%%%%%%%%%%%%%%%%%
\section{Conclusion}\label{sec5}
%%%%%%%%%%%%%%%%%%%%%%%%%%%%%%%%%%%%%%%%%%
This study proposes an iterative structural optimization framework based on QAOA, which, to the best of our knowledge, is the first attempt to integrate FLRS into QAOA to achieve truss structure optimization.
% {which successfully obtains optimal truss structures} using 14 and 24 qubits on {the 2D truss structures of 6 and 11 members}, respectively. 
%{To the best of the author's knowledge, the proposed framework is the first to integrate the FLRS within QAOA for truss structure optimization. 
In particular, this framework overcomes the barren plateau problem caused by multiple variational parameters and improves accuracy and stability due to the FLRS configuration set in this study.
%the configuration of FLRS \erase{method} allows this framework to overcome the barren plateau problem caused by multiple variational parameters, which improves its accuracy and stability. 
As a key highlight, our framework addresses the gap in enhancing computational efficiency when applying QAOA to structural optimization, and provides a robust option that balances flexibility, accuracy, qubit usage, and time complexity. 
This is demonstrated through numerical examples that include a relatively larger number of rods compared to existing studies. 
%and proves the applicability ofoposed} framework. 
In fact, in the numerical example, we succeeded in obtaining the optimal truss structure by using 14 and 24 qubits, respectively, in the 2 truss structures with 6 and 11 rods.
%This provides valuable insights for the practical application of QAOA in truss optimization, and proves the applicability ohe proposed} framework. 
This performance is due to the incorporation of FLRS into QAOA, and and suggests the applicability of the proposed framework to practical structural optimization. It should also be noted that since FLRS is a heuristic method, the variational parameters setting adopted in this study is not necessarily applicable to other structural optimization cases. However, as suggested by He et al.\cite{he2024parameterset}, the assumed setting can be used as an initial parameter combination, and a classical optimizer can be further applied to obtain a more accurate solution for a specific optimization case. Compared with directly applying a classical optimizer, we believe that this strategy is more robust and efficient.

The limitation of the framework is that the weight coefficient $\lambda$ that controls the volume constraint is assumed to be adjusted manually. If this weight coefficient is set too small, 
% easily violated, 
it will be easy to violate the constraint, and the volume of the optimized structure will be larger than 
%resulting in the volume of the optimized structure being greater than 
the original volume $V^{(0)}$. 
Conversely, if it is too large, QAOA will focus more on satisfying the constraint than minimizing compliance, which will result in the accidental removal of important rods. 
%in important rods being incorrectly removed. 
%The efficient strategy for adjusting the $\lambda$ is a meaningful topic for the future researches. 
The search for an efficient method of adjusting the value of $\lambda$ is a subject of significance for future research. 
\backmatter

%\bmhead{Supplementary information}
%
\bmhead{Acknowledgements}
This work was supported by JST SPRING (Grant Number: JPMJSP2114).
%
\vspace{5mm}
\bmhead{On the occasion of the 10th anniversary of the journal's publication}
{\it As someone who has been part of the editorial team since the journal's launch, we are very happy that our paper will contribute to the special issue “Reflecting and anticipating the future of computational mechanics: celebrating the 10th anniversary of AMSES and honoring Prof. Ladev{\`e}ze”. I wish Pierre good health and further development for this journal, which is now led by Paco.} \\


\section*{Declarations}
\begin{itemize}
\item Funding
\\This work was supported by 
%JSPS KAKENHI Grant Number 21H01583 and 22H00507, and by 
JST SPRING, Grant Number JPMJSP2114. 
\item Competing interests
\\The authors declare that they have no competing interests.
\item Ethics approval and consent to participate
\\Not applicable
\item Consent for publication
\\Not applicable
\item Data availability 
\\The data that support the findings of the study are available from the corresponding author upon reasonable request.
\item Materials availability
\\The materials that support the findings of the study are available from the corresponding author upon reasonable request.
\item Code availability 
\\Not applicable
\item Author contribution
\\J. Xiao: Development of the QAOA framework, validation, investigation, and writing of the original draft; N. Sukulthanasorn: Code implementation for truss analysis, conceptualization of truss optimization, discussion, review, and editing of the first draft; S. Moriguchi and R. Nomura: Investigation and review of the original draft; K. Terada: Funding acquisition, conceptualization, methodology development, supervision, review, and editing.
\end{itemize}

% \noindent
% If any of the sections are not relevant to your manuscript, please include the heading and write `Not applicable' for that section. 

%%===================================================%%
%% For presentation purpose, we have included        %%
%% \bigskip command. Please ignore this.             %%
%%===================================================%%
% \bigskip
% \begin{flushleft}%
% Editorial Policies for:

% \bigskip\noindent
% Springer journals and proceedings: \url{https://www.springer.com/gp/editorial-policies}

% \bigskip\noindent
% Nature Portfolio journals: \url{https://www.nature.com/nature-research/editorial-policies}

% \bigskip\noindent
% \textit{Scientific Reports}: \url{https://www.nature.com/srep/journal-policies/editorial-policies}

% \bigskip\noindent
% BMC journals: \url{https://www.biomedcentral.com/getpublished/editorial-policies}
% \end{flushleft}

\begin{appendices}

% \section{Section title of first appendix}\label{secA1}


%%=============================================%%
%% For submissions to Nature Portfolio Journals %%
%% please use the heading ``Extended Data''.   %%
%%=============================================%%

%%=============================================================%%
%% Sample for another appendix section			       %%
%%=============================================================%%

%% \section{Example of another appendix section}\label{secA2}%
%% Appendices may be used for helpful, supporting or essential material that would otherwise 
%% clutter, break up or be distracting to the text. Appendices can consist of sections, figures, 
%% tables and equations etc.

\end{appendices}

%%===========================================================================================%%
%% If you are submitting to one of the Nature Portfolio journals, using the eJP submission   %%
%% system, please include the references within the manuscript file itself. You may do this  %%
%% by copying the reference list from your .bbl file, paste it into the main manuscript .tex %%
%% file, and delete the associated \verb+\bibliography+ commands.                            %%
%%===========================================================================================%%

\bibliography{sn-bibliography}% common bib file
%% if required, the content of .bbl file can be included here once bbl is generated
%%\input sn-article.bbl


\end{document}
