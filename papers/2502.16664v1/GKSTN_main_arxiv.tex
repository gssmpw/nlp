%%%%%%%% ICML 2025 EXAMPLE LATEX SUBMISSION FILE %%%%%%%%%%%%%%%%%

\documentclass{article}

% Recommended, but optional, packages for figures and better typesetting:
\usepackage{microtype}
\usepackage{graphicx}
\usepackage{subfigure}
\usepackage{booktabs} % for professional tables

% hyperref makes hyperlinks in the resulting PDF.
% If your build breaks (sometimes temporarily if a hyperlink spans a page)
% please comment out the following usepackage line and replace
% \usepackage{icml2025} with \usepackage[nohyperref]{icml2025} above.
\usepackage{hyperref}


% Attempt to make hyperref and algorithmic work together better:
\newcommand{\theHalgorithm}{\arabic{algorithm}}

% Use the following line for the initial blind version submitted for review:
% \usepackage{icml2025}

% If accepted, instead use the following line for the camera-ready submission:
% \usepackage[accepted]{icml2025}
\usepackage[accepted]{not_icml2025}

% For theorems and such
\usepackage{amsmath}
\usepackage{amssymb}
\usepackage{mathtools}
\usepackage{amsthm}
% \usepackage{amsthm,hyperref,cleveref}

% if you use cleveref..
\usepackage[capitalize,noabbrev]{cleveref}

%%%%%%%%%%%%%%%%%%%%%%%%%%%%%%%%
% THEOREMS
%%%%%%%%%%%%%%%%%%%%%%%%%%%%%%%%
\theoremstyle{plain}
\newtheorem{theorem}{Theorem}[section]
\newtheorem{proposition}[theorem]{Proposition}
\newtheorem{lemma}[theorem]{Lemma}
\newtheorem{corollary}[theorem]{Corollary}

\theoremstyle{definition}
\newtheorem{definition}[theorem]{Definition}
\newtheorem{assumption}[theorem]{Assumption}

% \theoremstyle{remark}
\theoremstyle{plain}
\newtheorem{remark}[theorem]{Remark}


% Todonotes is useful during development; simply uncomment the next line
%    and comment out the line below the next line to turn off comments
%\usepackage[disable,textsize=tiny]{todonotes}
\usepackage[textsize=tiny]{todonotes}

% FA start
\usepackage{bm}
\usepackage{makecell}
\usepackage{tabu, booktabs}
\usepackage{hyperref}
\usepackage{url}
\usepackage{graphicx}
\usepackage{makecell}
\usepackage{float}
%\usepackage{svg}
% \usepackage{booktabs} % To thicken table lines
\usepackage[skip=3pt]{caption}
\setlength{\textfloatsep}{10pt plus 1.0pt minus 2.0pt}
% thicker column for tabular
\newcolumntype{"}{@{\hskip\tabcolsep\vrule width 1pt\hskip\tabcolsep}}

% Optional math commands from https://github.com/goodfeli/dlbook_notation.
%%%%% NEW MATH DEFINITIONS %%%%%

\usepackage{amsmath,amsfonts,bm}
\usepackage{derivative}
% Mark sections of captions for referring to divisions of figures
\newcommand{\figleft}{{\em (Left)}}
\newcommand{\figcenter}{{\em (Center)}}
\newcommand{\figright}{{\em (Right)}}
\newcommand{\figtop}{{\em (Top)}}
\newcommand{\figbottom}{{\em (Bottom)}}
\newcommand{\captiona}{{\em (a)}}
\newcommand{\captionb}{{\em (b)}}
\newcommand{\captionc}{{\em (c)}}
\newcommand{\captiond}{{\em (d)}}

% Highlight a newly defined term
\newcommand{\newterm}[1]{{\bf #1}}

% Derivative d 
\newcommand{\deriv}{{\mathrm{d}}}

% Figure reference, lower-case.
\def\figref#1{figure~\ref{#1}}
% Figure reference, capital. For start of sentence
\def\Figref#1{Figure~\ref{#1}}
\def\twofigref#1#2{figures \ref{#1} and \ref{#2}}
\def\quadfigref#1#2#3#4{figures \ref{#1}, \ref{#2}, \ref{#3} and \ref{#4}}
% Section reference, lower-case.
\def\secref#1{section~\ref{#1}}
% Section reference, capital.
\def\Secref#1{Section~\ref{#1}}
% Reference to two sections.
\def\twosecrefs#1#2{sections \ref{#1} and \ref{#2}}
% Reference to three sections.
\def\secrefs#1#2#3{sections \ref{#1}, \ref{#2} and \ref{#3}}
% Reference to an equation, lower-case.
\def\eqref#1{equation~\ref{#1}}
% Reference to an equation, upper case
\def\Eqref#1{Equation~\ref{#1}}
% A raw reference to an equation---avoid using if possible
\def\plaineqref#1{\ref{#1}}
% Reference to a chapter, lower-case.
\def\chapref#1{chapter~\ref{#1}}
% Reference to an equation, upper case.
\def\Chapref#1{Chapter~\ref{#1}}
% Reference to a range of chapters
\def\rangechapref#1#2{chapters\ref{#1}--\ref{#2}}
% Reference to an algorithm, lower-case.
\def\algref#1{algorithm~\ref{#1}}
% Reference to an algorithm, upper case.
\def\Algref#1{Algorithm~\ref{#1}}
\def\twoalgref#1#2{algorithms \ref{#1} and \ref{#2}}
\def\Twoalgref#1#2{Algorithms \ref{#1} and \ref{#2}}
% Reference to a part, lower case
\def\partref#1{part~\ref{#1}}
% Reference to a part, upper case
\def\Partref#1{Part~\ref{#1}}
\def\twopartref#1#2{parts \ref{#1} and \ref{#2}}

\def\ceil#1{\lceil #1 \rceil}
\def\floor#1{\lfloor #1 \rfloor}
\def\1{\bm{1}}
\newcommand{\train}{\mathcal{D}}
\newcommand{\valid}{\mathcal{D_{\mathrm{valid}}}}
\newcommand{\test}{\mathcal{D_{\mathrm{test}}}}

\def\eps{{\epsilon}}


% Random variables
\def\reta{{\textnormal{$\eta$}}}
\def\ra{{\textnormal{a}}}
\def\rb{{\textnormal{b}}}
\def\rc{{\textnormal{c}}}
\def\rd{{\textnormal{d}}}
\def\re{{\textnormal{e}}}
\def\rf{{\textnormal{f}}}
\def\rg{{\textnormal{g}}}
\def\rh{{\textnormal{h}}}
\def\ri{{\textnormal{i}}}
\def\rj{{\textnormal{j}}}
\def\rk{{\textnormal{k}}}
\def\rl{{\textnormal{l}}}
% rm is already a command, just don't name any random variables m
\def\rn{{\textnormal{n}}}
\def\ro{{\textnormal{o}}}
\def\rp{{\textnormal{p}}}
\def\rq{{\textnormal{q}}}
\def\rr{{\textnormal{r}}}
\def\rs{{\textnormal{s}}}
\def\rt{{\textnormal{t}}}
\def\ru{{\textnormal{u}}}
\def\rv{{\textnormal{v}}}
\def\rw{{\textnormal{w}}}
\def\rx{{\textnormal{x}}}
\def\ry{{\textnormal{y}}}
\def\rz{{\textnormal{z}}}

% Random vectors
\def\rvepsilon{{\mathbf{\epsilon}}}
\def\rvphi{{\mathbf{\phi}}}
\def\rvtheta{{\mathbf{\theta}}}
\def\rva{{\mathbf{a}}}
\def\rvb{{\mathbf{b}}}
\def\rvc{{\mathbf{c}}}
\def\rvd{{\mathbf{d}}}
\def\rve{{\mathbf{e}}}
\def\rvf{{\mathbf{f}}}
\def\rvg{{\mathbf{g}}}
\def\rvh{{\mathbf{h}}}
\def\rvu{{\mathbf{i}}}
\def\rvj{{\mathbf{j}}}
\def\rvk{{\mathbf{k}}}
\def\rvl{{\mathbf{l}}}
\def\rvm{{\mathbf{m}}}
\def\rvn{{\mathbf{n}}}
\def\rvo{{\mathbf{o}}}
\def\rvp{{\mathbf{p}}}
\def\rvq{{\mathbf{q}}}
\def\rvr{{\mathbf{r}}}
\def\rvs{{\mathbf{s}}}
\def\rvt{{\mathbf{t}}}
\def\rvu{{\mathbf{u}}}
\def\rvv{{\mathbf{v}}}
\def\rvw{{\mathbf{w}}}
\def\rvx{{\mathbf{x}}}
\def\rvy{{\mathbf{y}}}
\def\rvz{{\mathbf{z}}}

% Elements of random vectors
\def\erva{{\textnormal{a}}}
\def\ervb{{\textnormal{b}}}
\def\ervc{{\textnormal{c}}}
\def\ervd{{\textnormal{d}}}
\def\erve{{\textnormal{e}}}
\def\ervf{{\textnormal{f}}}
\def\ervg{{\textnormal{g}}}
\def\ervh{{\textnormal{h}}}
\def\ervi{{\textnormal{i}}}
\def\ervj{{\textnormal{j}}}
\def\ervk{{\textnormal{k}}}
\def\ervl{{\textnormal{l}}}
\def\ervm{{\textnormal{m}}}
\def\ervn{{\textnormal{n}}}
\def\ervo{{\textnormal{o}}}
\def\ervp{{\textnormal{p}}}
\def\ervq{{\textnormal{q}}}
\def\ervr{{\textnormal{r}}}
\def\ervs{{\textnormal{s}}}
\def\ervt{{\textnormal{t}}}
\def\ervu{{\textnormal{u}}}
\def\ervv{{\textnormal{v}}}
\def\ervw{{\textnormal{w}}}
\def\ervx{{\textnormal{x}}}
\def\ervy{{\textnormal{y}}}
\def\ervz{{\textnormal{z}}}

% Random matrices
\def\rmA{{\mathbf{A}}}
\def\rmB{{\mathbf{B}}}
\def\rmC{{\mathbf{C}}}
\def\rmD{{\mathbf{D}}}
\def\rmE{{\mathbf{E}}}
\def\rmF{{\mathbf{F}}}
\def\rmG{{\mathbf{G}}}
\def\rmH{{\mathbf{H}}}
\def\rmI{{\mathbf{I}}}
\def\rmJ{{\mathbf{J}}}
\def\rmK{{\mathbf{K}}}
\def\rmL{{\mathbf{L}}}
\def\rmM{{\mathbf{M}}}
\def\rmN{{\mathbf{N}}}
\def\rmO{{\mathbf{O}}}
\def\rmP{{\mathbf{P}}}
\def\rmQ{{\mathbf{Q}}}
\def\rmR{{\mathbf{R}}}
\def\rmS{{\mathbf{S}}}
\def\rmT{{\mathbf{T}}}
\def\rmU{{\mathbf{U}}}
\def\rmV{{\mathbf{V}}}
\def\rmW{{\mathbf{W}}}
\def\rmX{{\mathbf{X}}}
\def\rmY{{\mathbf{Y}}}
\def\rmZ{{\mathbf{Z}}}

% Elements of random matrices
\def\ermA{{\textnormal{A}}}
\def\ermB{{\textnormal{B}}}
\def\ermC{{\textnormal{C}}}
\def\ermD{{\textnormal{D}}}
\def\ermE{{\textnormal{E}}}
\def\ermF{{\textnormal{F}}}
\def\ermG{{\textnormal{G}}}
\def\ermH{{\textnormal{H}}}
\def\ermI{{\textnormal{I}}}
\def\ermJ{{\textnormal{J}}}
\def\ermK{{\textnormal{K}}}
\def\ermL{{\textnormal{L}}}
\def\ermM{{\textnormal{M}}}
\def\ermN{{\textnormal{N}}}
\def\ermO{{\textnormal{O}}}
\def\ermP{{\textnormal{P}}}
\def\ermQ{{\textnormal{Q}}}
\def\ermR{{\textnormal{R}}}
\def\ermS{{\textnormal{S}}}
\def\ermT{{\textnormal{T}}}
\def\ermU{{\textnormal{U}}}
\def\ermV{{\textnormal{V}}}
\def\ermW{{\textnormal{W}}}
\def\ermX{{\textnormal{X}}}
\def\ermY{{\textnormal{Y}}}
\def\ermZ{{\textnormal{Z}}}

% Vectors
\def\vzero{{\bm{0}}}
\def\vone{{\bm{1}}}
\def\vmu{{\bm{\mu}}}
\def\vtheta{{\bm{\theta}}}
\def\vphi{{\bm{\phi}}}
\def\va{{\bm{a}}}
\def\vb{{\bm{b}}}
\def\vc{{\bm{c}}}
\def\vd{{\bm{d}}}
\def\ve{{\bm{e}}}
\def\vf{{\bm{f}}}
\def\vg{{\bm{g}}}
\def\vh{{\bm{h}}}
\def\vi{{\bm{i}}}
\def\vj{{\bm{j}}}
\def\vk{{\bm{k}}}
\def\vl{{\bm{l}}}
\def\vm{{\bm{m}}}
\def\vn{{\bm{n}}}
\def\vo{{\bm{o}}}
\def\vp{{\bm{p}}}
\def\vq{{\bm{q}}}
\def\vr{{\bm{r}}}
\def\vs{{\bm{s}}}
\def\vt{{\bm{t}}}
\def\vu{{\bm{u}}}
\def\vv{{\bm{v}}}
\def\vw{{\bm{w}}}
\def\vx{{\bm{x}}}
\def\vy{{\bm{y}}}
\def\vz{{\bm{z}}}

% Elements of vectors
\def\evalpha{{\alpha}}
\def\evbeta{{\beta}}
\def\evepsilon{{\epsilon}}
\def\evlambda{{\lambda}}
\def\evomega{{\omega}}
\def\evmu{{\mu}}
\def\evpsi{{\psi}}
\def\evsigma{{\sigma}}
\def\evtheta{{\theta}}
\def\eva{{a}}
\def\evb{{b}}
\def\evc{{c}}
\def\evd{{d}}
\def\eve{{e}}
\def\evf{{f}}
\def\evg{{g}}
\def\evh{{h}}
\def\evi{{i}}
\def\evj{{j}}
\def\evk{{k}}
\def\evl{{l}}
\def\evm{{m}}
\def\evn{{n}}
\def\evo{{o}}
\def\evp{{p}}
\def\evq{{q}}
\def\evr{{r}}
\def\evs{{s}}
\def\evt{{t}}
\def\evu{{u}}
\def\evv{{v}}
\def\evw{{w}}
\def\evx{{x}}
\def\evy{{y}}
\def\evz{{z}}

% Matrix
\def\mA{{\bm{A}}}
\def\mB{{\bm{B}}}
\def\mC{{\bm{C}}}
\def\mD{{\bm{D}}}
\def\mE{{\bm{E}}}
\def\mF{{\bm{F}}}
\def\mG{{\bm{G}}}
\def\mH{{\bm{H}}}
\def\mI{{\bm{I}}}
\def\mJ{{\bm{J}}}
\def\mK{{\bm{K}}}
\def\mL{{\bm{L}}}
\def\mM{{\bm{M}}}
\def\mN{{\bm{N}}}
\def\mO{{\bm{O}}}
\def\mP{{\bm{P}}}
\def\mQ{{\bm{Q}}}
\def\mR{{\bm{R}}}
\def\mS{{\bm{S}}}
\def\mT{{\bm{T}}}
\def\mU{{\bm{U}}}
\def\mV{{\bm{V}}}
\def\mW{{\bm{W}}}
\def\mX{{\bm{X}}}
\def\mY{{\bm{Y}}}
\def\mZ{{\bm{Z}}}
\def\mBeta{{\bm{\beta}}}
\def\mPhi{{\bm{\Phi}}}
\def\mLambda{{\bm{\Lambda}}}
\def\mSigma{{\bm{\Sigma}}}

% Tensor
\DeclareMathAlphabet{\mathsfit}{\encodingdefault}{\sfdefault}{m}{sl}
\SetMathAlphabet{\mathsfit}{bold}{\encodingdefault}{\sfdefault}{bx}{n}
\newcommand{\tens}[1]{\bm{\mathsfit{#1}}}
\def\tA{{\tens{A}}}
\def\tB{{\tens{B}}}
\def\tC{{\tens{C}}}
\def\tD{{\tens{D}}}
\def\tE{{\tens{E}}}
\def\tF{{\tens{F}}}
\def\tG{{\tens{G}}}
\def\tH{{\tens{H}}}
\def\tI{{\tens{I}}}
\def\tJ{{\tens{J}}}
\def\tK{{\tens{K}}}
\def\tL{{\tens{L}}}
\def\tM{{\tens{M}}}
\def\tN{{\tens{N}}}
\def\tO{{\tens{O}}}
\def\tP{{\tens{P}}}
\def\tQ{{\tens{Q}}}
\def\tR{{\tens{R}}}
\def\tS{{\tens{S}}}
\def\tT{{\tens{T}}}
\def\tU{{\tens{U}}}
\def\tV{{\tens{V}}}
\def\tW{{\tens{W}}}
\def\tX{{\tens{X}}}
\def\tY{{\tens{Y}}}
\def\tZ{{\tens{Z}}}


% Graph
\def\gA{{\mathcal{A}}}
\def\gB{{\mathcal{B}}}
\def\gC{{\mathcal{C}}}
\def\gD{{\mathcal{D}}}
\def\gE{{\mathcal{E}}}
\def\gF{{\mathcal{F}}}
\def\gG{{\mathcal{G}}}
\def\gH{{\mathcal{H}}}
\def\gI{{\mathcal{I}}}
\def\gJ{{\mathcal{J}}}
\def\gK{{\mathcal{K}}}
\def\gL{{\mathcal{L}}}
\def\gM{{\mathcal{M}}}
\def\gN{{\mathcal{N}}}
\def\gO{{\mathcal{O}}}
\def\gP{{\mathcal{P}}}
\def\gQ{{\mathcal{Q}}}
\def\gR{{\mathcal{R}}}
\def\gS{{\mathcal{S}}}
\def\gT{{\mathcal{T}}}
\def\gU{{\mathcal{U}}}
\def\gV{{\mathcal{V}}}
\def\gW{{\mathcal{W}}}
\def\gX{{\mathcal{X}}}
\def\gY{{\mathcal{Y}}}
\def\gZ{{\mathcal{Z}}}

% Sets
\def\sA{{\mathbb{A}}}
\def\sB{{\mathbb{B}}}
\def\sC{{\mathbb{C}}}
\def\sD{{\mathbb{D}}}
% Don't use a set called E, because this would be the same as our symbol
% for expectation.
\def\sF{{\mathbb{F}}}
\def\sG{{\mathbb{G}}}
\def\sH{{\mathbb{H}}}
\def\sI{{\mathbb{I}}}
\def\sJ{{\mathbb{J}}}
\def\sK{{\mathbb{K}}}
\def\sL{{\mathbb{L}}}
\def\sM{{\mathbb{M}}}
\def\sN{{\mathbb{N}}}
\def\sO{{\mathbb{O}}}
\def\sP{{\mathbb{P}}}
\def\sQ{{\mathbb{Q}}}
\def\sR{{\mathbb{R}}}
\def\sS{{\mathbb{S}}}
\def\sT{{\mathbb{T}}}
\def\sU{{\mathbb{U}}}
\def\sV{{\mathbb{V}}}
\def\sW{{\mathbb{W}}}
\def\sX{{\mathbb{X}}}
\def\sY{{\mathbb{Y}}}
\def\sZ{{\mathbb{Z}}}

% Entries of a matrix
\def\emLambda{{\Lambda}}
\def\emA{{A}}
\def\emB{{B}}
\def\emC{{C}}
\def\emD{{D}}
\def\emE{{E}}
\def\emF{{F}}
\def\emG{{G}}
\def\emH{{H}}
\def\emI{{I}}
\def\emJ{{J}}
\def\emK{{K}}
\def\emL{{L}}
\def\emM{{M}}
\def\emN{{N}}
\def\emO{{O}}
\def\emP{{P}}
\def\emQ{{Q}}
\def\emR{{R}}
\def\emS{{S}}
\def\emT{{T}}
\def\emU{{U}}
\def\emV{{V}}
\def\emW{{W}}
\def\emX{{X}}
\def\emY{{Y}}
\def\emZ{{Z}}
\def\emSigma{{\Sigma}}

% entries of a tensor
% Same font as tensor, without \bm wrapper
\newcommand{\etens}[1]{\mathsfit{#1}}
\def\etLambda{{\etens{\Lambda}}}
\def\etA{{\etens{A}}}
\def\etB{{\etens{B}}}
\def\etC{{\etens{C}}}
\def\etD{{\etens{D}}}
\def\etE{{\etens{E}}}
\def\etF{{\etens{F}}}
\def\etG{{\etens{G}}}
\def\etH{{\etens{H}}}
\def\etI{{\etens{I}}}
\def\etJ{{\etens{J}}}
\def\etK{{\etens{K}}}
\def\etL{{\etens{L}}}
\def\etM{{\etens{M}}}
\def\etN{{\etens{N}}}
\def\etO{{\etens{O}}}
\def\etP{{\etens{P}}}
\def\etQ{{\etens{Q}}}
\def\etR{{\etens{R}}}
\def\etS{{\etens{S}}}
\def\etT{{\etens{T}}}
\def\etU{{\etens{U}}}
\def\etV{{\etens{V}}}
\def\etW{{\etens{W}}}
\def\etX{{\etens{X}}}
\def\etY{{\etens{Y}}}
\def\etZ{{\etens{Z}}}

% The true underlying data generating distribution
\newcommand{\pdata}{p_{\rm{data}}}
\newcommand{\ptarget}{p_{\rm{target}}}
\newcommand{\pprior}{p_{\rm{prior}}}
\newcommand{\pbase}{p_{\rm{base}}}
\newcommand{\pref}{p_{\rm{ref}}}

% The empirical distribution defined by the training set
\newcommand{\ptrain}{\hat{p}_{\rm{data}}}
\newcommand{\Ptrain}{\hat{P}_{\rm{data}}}
% The model distribution
\newcommand{\pmodel}{p_{\rm{model}}}
\newcommand{\Pmodel}{P_{\rm{model}}}
\newcommand{\ptildemodel}{\tilde{p}_{\rm{model}}}
% Stochastic autoencoder distributions
\newcommand{\pencode}{p_{\rm{encoder}}}
\newcommand{\pdecode}{p_{\rm{decoder}}}
\newcommand{\precons}{p_{\rm{reconstruct}}}

\newcommand{\laplace}{\mathrm{Laplace}} % Laplace distribution

\newcommand{\E}{\mathbb{E}}
\newcommand{\Ls}{\mathcal{L}}
\newcommand{\R}{\mathbb{R}}
\newcommand{\emp}{\tilde{p}}
\newcommand{\lr}{\alpha}
\newcommand{\reg}{\lambda}
\newcommand{\rect}{\mathrm{rectifier}}
\newcommand{\softmax}{\mathrm{softmax}}
\newcommand{\sigmoid}{\sigma}
\newcommand{\softplus}{\zeta}
\newcommand{\KL}{D_{\mathrm{KL}}}
\newcommand{\Var}{\mathrm{Var}}
\newcommand{\standarderror}{\mathrm{SE}}
\newcommand{\Cov}{\mathrm{Cov}}
% Wolfram Mathworld says $L^2$ is for function spaces and $\ell^2$ is for vectors
% But then they seem to use $L^2$ for vectors throughout the site, and so does
% wikipedia.
\newcommand{\normlzero}{L^0}
\newcommand{\normlone}{L^1}
\newcommand{\normltwo}{L^2}
\newcommand{\normlp}{L^p}
\newcommand{\normmax}{L^\infty}

\newcommand{\parents}{Pa} % See usage in notation.tex. Chosen to match Daphne's book.

\DeclareMathOperator*{\argmax}{arg\,max}
\DeclareMathOperator*{\argmin}{arg\,min}

\DeclareMathOperator{\sign}{sign}
\DeclareMathOperator{\Tr}{Tr}
\let\ab\allowbreak



\usepackage{booktabs}
\raggedbottom
\usepackage{tabularx,ragged2e,siunitx}
\newcolumntype{Y}[1]{>{\Centering\hspace{0pt}\hsize=#1\hsize}X}

\usepackage{caption}
\usepackage{subcaption}
\usepackage{multirow}

\usepackage{comment}

\usepackage[inline]{enumitem}

% python code
\usepackage{xcolor}
% \usepackage{piton} #Requires LuaLaTex
% \usepackage{solarized-light} #not found

\usepackage{listings}

\definecolor{codegreen}{rgb}{0,0.6,0}
\definecolor{codegray}{rgb}{0.5,0.5,0.5}
\definecolor{codepurple}{rgb}{0.58,0,0.82}
\definecolor{backcolour}{rgb}{0.95,0.95,0.92}

\lstdefinestyle{mystyle}{
    backgroundcolor=\color{backcolour},   
    commentstyle=\color{codegreen},
    keywordstyle=\color{magenta},
    numberstyle=\tiny\color{codegray},
    stringstyle=\color{codepurple},
    basicstyle=\ttfamily\footnotesize,
    breakatwhitespace=false,         
    breaklines=true,                 
    captionpos=b,                    
    keepspaces=true,                 
    numbers=left,                    
    numbersep=5pt,                  
    showspaces=false,                
    showstringspaces=false,
    showtabs=false,                  
    tabsize=2
}

\lstset{style=mystyle}
% FA stop

% The \icmltitle you define below is probably too long as a header.
% Therefore, a short form for the running title is supplied here:
% \icmltitlerunning{Submission and Formatting Instructions for ICML 2025}
% \icmltitlerunning{Geometric Kolmogorov Superposition Theorem Network}
\icmltitlerunning{Geometric Kolmogorov-Arnold Superposition Theorem}

\begin{document}

\twocolumn[
% \icmltitle{Submission and Formatting Instructions for \\
%           International Conference on Machine Learning (ICML 2025)}
% \icmltitle{Geometric Kolmogorov Superposition Theorem Network}

% \icmltitle{Geometric Kolmogorov-Arnold Superposition Theorem Network}
\icmltitle{Geometric Kolmogorov-Arnold Superposition Theorem}

% Francesco Alesiani, Takashi Maruyama, Henrik Christiansen, Viktor Zaverkin

% It is OKAY to include author information, even for blind
% submissions: the style file will automatically remove it for you
% unless you've provided the [accepted] option to the icml2025
% package.

% List of affiliations: The first argument should be a (short)
% identifier you will use later to specify author affiliations
% Academic affiliations should list Department, University, City, Region, Country
% Industry affiliations should list Company, City, Region, Country

% You can specify symbols, otherwise they are numbered in order.
% Ideally, you should not use this facility. Affiliations will be numbered
% in order of appearance and this is the preferred way.
\icmlsetsymbol{equal}{*}

\begin{icmlauthorlist}
\icmlauthor{Francesco Alesiani}{equal,nle}
\icmlauthor{Takashi Maruyama}{equal,nle}
\icmlauthor{Henrik Christiansen}{nle}
\icmlauthor{Viktor Zaverkin}{nle}
\end{icmlauthorlist}

\icmlaffiliation{nle}{NEC Laboratories Europe, Heidelberg, Germany}

\icmlcorrespondingauthor{Francesco Alesiani}{Francesco.Alesiani@neclab.eu}
\icmlcorrespondingauthor{Takashi Maruyama}{Takashi.Maruyama@neclab.eu}

% You may provide any keywords that you
% find helpful for describing your paper; these are used to populate
% the "keywords" metadata in the PDF but will not be shown in the document
\icmlkeywords{Machine Learning, ICML}

\vskip 0.3in
]

% this must go after the closing bracket ] following \twocolumn[ ...

% This command actually creates the footnote in the first column
% listing the affiliations and the copyright notice.
% The command takes one argument, which is text to display at the start of the footnote.
% The \icmlEqualContribution command is standard text for equal contribution.
% Remove it (just {}) if you do not need this facility.

%\printAffiliationsAndNotice{}  % leave blank if no need to mention equal contribution
\printAffiliationsAndNotice{\icmlEqualContribution} % otherwise use the standard text.

\newcommand{\tm}[1]{{{\textcolor{red}{[TM: #1]}}}}
\newcommand{\fa}[1]{{{\textcolor{blue}{[FA: #1]}}}}

\begin{abstract}
The Kolmogorov-Arnold Theorem (KAT), or more generally, the Kolmogorov Superposition Theorem (KST), establishes that any non-linear multivariate function can be exactly represented as a finite superposition of non-linear univariate functions. Unlike the universal approximation theorem, which provides only an approximate representation without guaranteeing a fixed network size, KST offers a theoretically exact decomposition. The Kolmogorov-Arnold Network (KAN) was introduced as a trainable model to implement KAT, and recent advancements have adapted KAN using concepts from modern neural networks. However, KAN struggles to effectively model physical systems that require inherent equivariance or invariance to $E(3)$ transformations, a key property for many scientific and engineering applications. In this work, we propose a novel extension of KAT and KAN to incorporate equivariance and invariance over $O(n)$ group actions, enabling accurate and efficient modeling of these systems. Our approach provides a unified approach that bridges the gap between mathematical theory and practical architectures for physical systems, expanding the applicability of KAN to a broader class of problems. 
\end{abstract}

\section{Introduction}
Kolmogorov Arnold Networks (KANs) \cite{Liu_Wang_Vaidya_Ruehle_Halverson_Soljačić_Hou_Tegmark_2024} 
have recently risen to the interest of the machine learning community as an alternative to the well-consolidated Multi-Layer Perceptrons (MLPs) \cite{hornik1989multilayer}. MLPs have transformed machine learning for their ability to approximate arbitrary functions and have demonstrated their expressive power, theoretically guaranteed by the universal approximation theorem \cite{hornik1989multilayer}, in countless applications.  
The Kolmogorov-Arnold Theorem (KAT), developed to solve Hilbert’s 13th problem \cite{kolmogorov1961representation}, is a powerful and foundational mathematical result.
While the universal approximation theorem states that any function can be approximated with an MLP function of bounded dimension, KAT provides a way to exactly and with a finite and known number of univariate functions to represent any multivariate function. KAT has found multiple applications in mathematics \cite{laczkovich2021superposition}, fuzzy logic  \cite{kreinovichNORMALFORMSFUZZY1996}, pattern recognition \cite{koppen2002training}, and neural networks \cite{kuurkova1992kolmogorov,liu2024kan}. 

We have recently witnessed the flourishing of extensions of the use of KAT as a substitute for MLP \cite{Ji_Hou_Zhang_2024}, either as a plug-in replacement of MLP \cite{xu2024kaneffectiveidentifyingtracking, decarlo2024kolmogorovarnoldgraphneuralnetworks}, 
as a surrogate function for solving or approximating partial differentiable equations (PDE)
\cite{abueidda2024deepokandeepoperatornetwork,wang2024kolmogorovarnoldinformedneural,shuai2024physicsinformedkolmogorovarnoldnetworkspower}. Further KAN have been extended by exploring alternative basis such as Chebychev polynomials
\cite{ss2024chebyshevpolynomialbasedkolmogorovarnoldnetworks,mostajeran2024epickanselastoplasticityinformedkolmogorovarnold}, wavelet functions \cite{bozorgasl2024wavkanwaveletkolmogorovarnoldnetworks}, Fourier series \cite{xu2024fourierkangcffourierkolmogorovarnoldnetwork}, or alternative representations \cite{Guilhoto_Perdikaris_2024}. 

In applications to scientific computing, key physical symmetries are present \cite{Finzi2021,goodman2009symmetry,noether}, for example, the invariance to translations, rotations, and reflections (i.e. $E(3)$ group) of energies. These systems include fluid dynamics, partial differentiable equations (PDEs), astrophysics, material science, and biology. 
In modeling molecular systems, we want the potential energy to be invariant to rigid reflections and roto-translations of the molecules to reflect the underlying physical symmetry. While MLP-based architectures have been widely explored \cite{schutt2017schnet,Batatia_Kovács_Simm_Ortner_Csányi_2023,Satorras_Hoogeboom_Welling_2022,Liao_Smidt_2023,Zaverkin_Alesiani_Maruyama_Errica_Christiansen_Takamoto_Weber_Niepert_2024}, it is not clear how to model physical system with KAN-based architectures, especially since KAN models have shown potential to overcome the curse of dimensionality \cite{lai2021kolmogorov,poggio2022deep}. 

% \cite{somvanshiSurveyKolmogorovArnoldNetwork2024}

Our contribution are : 
\begin{itemize*}
\item to extend KAN to include $O(n)$ symmetries, thus been able to represent $O(n)$ invariant and equivariant functions (\autoref{sec:geo}). We further extend the results to include the permutation invariance with respect to input data, which reduces the parameter count of the network and improves generalization. 

\item After providing the theoretical justification, we present practical architectures (\autoref{sec:arch}) and analyze their performances with scientifically inspired experiments. We analyze the learning capability of the proposed KAN model for an idealized model (\autoref{sec:lj}), which allows us to simulate multiple particles in multiple dimensions. 

\item We experiment on real datasets for material design, the MD17 (\autoref{sec:md17}) and MD22 (\autoref{sec:md22}), and analyze the learning capability of the proposed model.
\end{itemize*}

\section{Related Works}

\paragraph{Machine Learning Interatomic Potentials and Equivariant Architectures} 
% \tm{Feels somewhat long.}
Machine learning interatomic potentials (MLIPs) 
have emerged as powerful tools for modeling interatomic interactions in molecular and materials systems, offering a computationally efficient alternative to traditional ab initio methods. Architectures like Schnet \cite{schutt2017schnet} use continuous-filter convolutional layers to capture local atomic environments and message passing, enabling accurate predictions of molecular properties. To further enhance physical expressivity, $E(3)$-equivariant architectures \cite{thomas2018tensor} have been developed, which respect the symmetries of Euclidean space (rotations, translations, and reflections) by design. These models, such as Tensor Field Networks \cite{thomas2018tensor} and NequIP \cite{batzner2022nequip}, ensure that predictions (i.e. energy and forces) are invariant or equivariant to transformations in 3D space, making them highly data-efficient for tasks like force field prediction in molecular dynamics. 
MACE \cite{Batatia_Kovács_Simm_Ortner_Csányi_2023} is a higher-order equivariant message-passing network that enhances force field accuracy and efficiency by leveraging multi-body interactions. 
E(n)-equivariant GNNs (EGNNs) \cite{Satorras_Hoogeboom_Welling_2022} implement a higher-order representation while maintaining equivariance to rotations, translations, and permutations. 
Irreducible Cartesian Tensor Potential (ICTP)  \cite{Zaverkin_Alesiani_Maruyama_Errica_Christiansen_Takamoto_Weber_Niepert_2024} introduces irreducible Cartesian tensors for equivariant message passing, offering computational advantages over spherical harmonics in the small tensor rank regime. Tensor field networks \cite{Thomas_Smidt_Kearnes_Yang_Li_Kohlhoff_Riley_2018} and Equiformer \cite{Liao_Smidt_2023} use spherical harmonics as bases for tensors. While SO3krates \cite{Frank_Unke_Müller_Chmiela_2024a} combines sparse equivariant representations with transformers to balance accuracy and speed.
Additionally, equivariant Clifford networks \cite{ruheCliffordGroupEquivariant2023b}
extend this framework by incorporating geometric algebra to build equivariant models. 
Equivariant representations mitigate cumulative errors in molecular dynamics  \cite{Unke_Chmiela_Sauceda_Gastegger_Poltavsky_Schütt_Tkatchenko_Müller_2021},  while 
directional message passing with spherical harmonics improves angular dependency modeling as implemented in DimeNet \cite{Gasteiger_Groß_Günnemann_2022}.  
Equivariant or invariant architectures enhance data efficiency, accuracy, and physical consistency in tasks where input symmetries (e.g., rotation, reflection, translation) dictate output invariance or equivariance.
While these advancements have significantly improved the accuracy and efficiency of MLIPs for applications in chemistry, physics, and materials science, the advantage of KAN architecture has not yet been explored, we thus take a fundamental step in this direction with our study. 

\paragraph{KAN Architectures}
Kolmogorov-Arnold Networks (KANs) are inspired by the Kolmogorov-Arnold representation theorem, which provides a theoretical foundation for approximating multivariate functions using univariate functions and addition. Early work by Hecht-Nielsen (1987) \cite{hecht1987kolmogorov} introduced one of the first neural network architectures based on this theorem, demonstrating its potential for efficient function approximation. 
\cite{lai2021kolmogorov} study the approximation capability of KST-based models in high dimensions and how they could potentially break the curse of dimension \cite{poggio2022deep}. 
\cite{ferdausKANICEKolmogorovArnoldNetworks2024} propose to combine  Convolutional Neural Networks (CNNs) with Kolmogorov Arnold Network (KAN) principles.
Additionally, 
\cite{yangKolmogorovArnoldTransformer2024}
explored the integration of KAN principles into transformer models, achieving improvements in efficiency for sequence modeling tasks. 
\cite{huEKANEquivariantKolmogorovArnold2024a} propose EKAN, an approximation method for incorporating matrix group equivariance into KANs. While these studies highlight the versatility of KAN architectures in adapting to various neural network frameworks, the extension to physical and geometrical symmetries has not been fully considered.

\paragraph{Application of KAN}
KANs have been applied to a range of machine learning tasks, particularly in scenarios requiring efficient function approximation. For instance, Kůrková (1991) \cite{kurkova1991kolmogorov} demonstrated the effectiveness of KANs in high-dimensional regression problems, where traditional neural networks often struggle with scalability. In the natural language processing domain, \cite{Galitsky2024} utilized KAN for word-level explanations.  
Furthermore, 
\cite{decarlo2024kolmogorovarnoldgraphneuralnetworks}
applied KANs to graph-based learning tasks, showing that their hybrid models could achieve state-of-the-art results in graph classification and node prediction. 
% survey: \cite{somvanshiSurveyKolmogorovArnoldNetwork2024} 
KAN has been used as a function approximation to solve PDE \cite{wang2024kolmogorovarnoldinformedneural,shukla2024comprehensivefaircomparisonmlp} for both forward and backward problems with highly complex boundary and initial conditions.
\cite{aghaei2024rkanrationalkolmogorovarnoldnetworks} extends KAN with rational polynomials basis to regression and classifications problems. \cite{Seydi2024} explores using Wavelet as basis functions to model hyper-spectral data. KANs have been extended to model time-series \cite{xu2024kolmogorovarnoldnetworkstimeseries,inzirillo2024sigkansignatureweightedkolmogorovarnoldnetworks} to dynamically adapt to temporal data. 
While these, and other \cite{somvanshiSurveyKolmogorovArnoldNetwork2024},  applications highlight the practical utility of KANs in solving complex real-world problems, a significant class of molecular applications remains overlooked. 

\paragraph{Theoretical Work on KAN}
The theoretical foundations of Kolmogorov–Arnold Networks (KANs) are rooted in the Kolmogorov–Arnold representation theorem, established by Andrey Kolmogorov  \citet{kolmogorov1957representation} and later refined by Vladimir Arnold \citet{arnold1959functions}. 
Building upon this foundation, David Sprecher \citet{sprecher1965structure} and George Lorentz \citet{lorentz1976approximation} provided constructive algorithms to implement the theorem, enhancing its applicability in computational contexts. 
Recent theoretical advancements have addressed challenges in training KANs, such as non-smooth optimization landscapes. Researchers have proposed various techniques to improve the stability and convergence of KAN training, including regularization methods \cite{Braun2009constructive} like dropout and weight decay, as well as optimization strategies involving adaptive learning rates, while \cite{igelnik2003kolmogorov} have proposed using cubic spline as activation and internal function for efficient approximation.  
These contributions have been instrumental in bridging the gap between the mathematical foundations of KANs and their practical implementation in machine learning.
However, training with energies requires fitting highly non-linear functions. In this work, we demonstrate how extending the KAN architecture enhances the learning capacity of KAT-based models. 


\section{Background}
\paragraph{Equivariance and invariance} We call a function $\phi: X \to Y$ {\it equivariant} or {\it invariant}, if given a set of transformation $T^X_g$ on $X$, the input space, for a given element $g$ of action group $G$, there exists an associated transformation $T^Y_g: Y \to Y$ on the output space $Y$, such that 
\begin{align}\label{eq:equivariance}
\underbrace{ \phi(T^X_g(\bm{x})) = T^Y_g(\phi(\bm{x}))}_{\text{equivariant}}, ~~~\text{or}~~~ \underbrace{ \phi(T^X_g(\bm{x})) = \phi(\bm{x})}_{\text{invariant}}.
\end{align}
An example of $\phi$ is a non-linear function of a multivariate variable $\bm{x}=(\bm{x}_1, \dots, \bm{x}_m) \in \mathbb{R}^{m \times n}$ representing a point cloud with $m$ points, where each point lives in an $n$-dimensional space $\bm{x}_i \in \mathbb{R}^n$, $\phi(\bm{x})=\bm{y} \in \mathbb{R}^{m \times n}$ the transformed points, with $T_g$ a translation of the input $T^X_g(\bm{y}) = \bm{x} + \bm{g}$ and $T^Y_g$ an associated translation in the output domain $T^Y_g(\bm{y}) = \bm{y} + \bm{g}$. When $\phi$ is equivariant with respect to the action of $G$, then first applying the translation in the input domain and then applying $\phi$, is equivalent to first applying $\phi$ and then translating for the same amount $g$, in the target domain. When $\phi$ is invariant with respect to $G$, then applying the translation or not, results in the same output $\phi(\bm{x}+\bm{g})=\phi(\bm{x})=\bm{y}$. In this work, we consider three types of symmetries, i.e. invariance and equivariance:
\begin{itemize}
    \item {\it translation symmetry}: $\phi(\bm{x}+\bm{g})=\phi(\bm{x})$ for the invariance and $\phi(\bm{x}+\bm{g})=\phi(\bm{x})+\bm{g}$ for equivariance, with $\bm{g} \in \mathbb{R}^n$ and where $\bm{x}+\bm{g}$ refers to the element-wise operation $(\bm{x}_1 +\bm{g}, \dots, \bm{x}_m + \bm{g})$;
    \item {\it rotation and reflection symmetry}: given an orthogonal matrix $\bm{Q} \in \mathbb{R}^{n \times n}$,  $\phi$ is invariant or equivariant if $\phi(\bm{Q}\bm{x})=\phi(\bm{x})$ or $\phi(\bm{Q}\bm{x})=\bm{Q}\phi(\bm{x})$, and where $\bm{Q}\bm{x}$ refers to the element-wise operation $(\bm{Q}\bm{x}_1 , \dots, \bm{Q}\bm{x}_m)$;  
    \item {\it permutation symmetry}: $\phi$ is invariant or equivariant, if $\phi(\bm{x}_1 , \dots, \bm{x}_m)=\phi(\bm{x}_{\pi_1} , \dots, \bm{x}_{\pi_m})$ and $\phi(\pi(\bm{x}))=\pi(\phi(\bm{x}))$, for any permutation $\pi: [m] \to [m]$, where $\pi(\bm{x})=\bm{x}_{\pi_1} , \dots, \bm{x}_{\pi_m}$ .
\end{itemize}
We extend KAT in \autoref{sec:geo} to functions that exhibit these symmetries.

\paragraph{Kolmogorov superposition theorem (KST)}
The Kolmogorov-Arnold representation theorem (KAT), proposed by  \citet{kolmogorov1961representation}, provides a powerful theoretical tool to represent a multivariate function $f(x_1,\dots,x_m)$ as the composition of functions of a single variable. The original form of KAT states that a given  continuous function $f: [0,1]^m \to \mathbb{R}$ can be represented exactly as 
\begin{align}\label{eq:kst}
f(x_1,\dots,x_m) = \sum_{q=1}^{2m+1} \psi_q(\sum_{p=1}^{m} \phi_{qp}(x_p))
\end{align}
with $\psi_q: \mathbb{R} \to \mathbb{R}$ and $\phi_{qp}: [0,1] \to \mathbb{R}$ uni-variate continuous functions. 

\begin{table*}
\centering
\caption{
Kolmogorov superposition formulas \cite{Guilhoto_Perdikaris_2024} for a continuous function $f(x_1, \dots, x_d)$ or $f(\bm{x}_1, \dots, \bm{x}_m)$ and their complexity in terms of parameters. 
}

\label{tab:kst}
\centering
\resizebox{\linewidth}{!}{%
\begin{tabular}{ccccc}
\toprule
    \textbf{Version} & \textbf{Formula} & \textbf{\makecell{Inner \\ Functions}} & \textbf{\makecell{Outer \\ Functions}} & \textbf{\makecell{Other \\ Parameters or functions}} \\ \midrule
    
    \makecell{Kolmogorov \\ (1957)} & $\sum_{q=1}^{2m+1}\psi_q\left( \sum_{p=1}^m \phi_{q,p}(x_p) \right)$ & $(2m+1)m$ & $2m+1$  & N/A \\ \hline

    \makecell{Ostrand \\ (1965)} & $\sum_{q=1}^{2mn+1}\psi_q\left( \sum_{p=1}^d \phi_{q,p}(\bm{x}_p) \right)$ & $(2nm+1)m$ & $2mn+1$  & N/A \\ \hline
    
    
    \makecell{Lorentz \\ (1962)} & $\sum_{q=1}^{2m+1} \psi \left( \sum_{p=1}^m \lambda_p \phi_q(x_p) \right) $ & $2m+1$ & $1$  & $\lambda\in\mathbb{R}^{m} $ \\ \hline
    
    \makecell{Sprecher \\ (1965)} & $\sum_{q=1}^{2m+1} \psi_q\left( \sum_{p=1}^m \lambda_p \phi(x_p+qa) \right) $ & $1$ & $2m+1$  & $a\in\mathbb{R},\  \lambda\in\mathbb{R}^d$ \\ \hline
        
    \makecell{Kurkova \\ (1991)} & $\sum_{q=1}^{N} \psi \left( \sum_{p=1}^m w_{pq} \phi_q(x_p) \right) $ & $2m+1 \le N$ & $1$  & $w \in \mathbb{R}^{m\times N}$ \\ \hline

    % \makecell{Braun \\ (2009)} & $\sum_{q=1}^{2m+1} \psi \left( \sum_{p=1}^m w_{p} \phi(x_p+qa) + c_q \right) $ & $N$ & $1$  & $w \in \mathbb{R}^{m}, c \in \mathbb{R}^{2m+1}, a \in \mathbb{R}$ \\ \hline
    
    \makecell{Laczkovich \\ (2021)} & $\sum_{q=1}^{N} \psi \left( \sum_{p=1}^d \lambda_{pq} \phi_q(x_p) \right) $ & $N$ & $1$  & $\lambda \in\mathbb{R}^{m\times N}$ \\ \hline

    \makecell{\textbf{This work}} & $
\sum_{q=1}^{2m^2+1} \psi_q \left(\sum_{i=1,j=1}^{m,m} \phi_{qij}(\langle \bm{x}_i,\bm{x}_j \rangle) \right)    
    $ &$ (2m^2+1) m^2$ & $2m^2+1$ & N/A\\ \hline

    \makecell{\textbf{This work}} & $
    \sum_{q=1}^{2mn+1} \psi_q \left( \sum_{i=1,j=1}^{m,n} \phi_{qij}(\langle \bm{x}_i,\bm{y}_j \rangle)  +  \sum_{i=1,j=1}^{n,n} \phi'_{qij} (\langle \bm{y}_i,\bm{y}_j\rangle)
\right)    
    $ &$(2mn+1)(mn+n^2)$ & $2mn+1$ & N/A \\ \hline

    \makecell{\textbf{This work}} & $
    \sum_{q=1}^{2mn+1} \psi_q \left( \sum_{i=1,j=1}^{m,n} \phi_{qij}(\langle \bm{x}_i,\bm{x}_j \rangle) \right)    
    $ &$(2mn+1)mn$ & $2mn+1$ & N/A \\

    
    \bottomrule
\end{tabular}
}
\end{table*}


\paragraph{Ostrand superposition theorem (OST)
}
In 1965, \citet{ostrandDIMENSIONMETRICSPACESa} proposed an extension of the original KAT to input compact domains. The theorem states that, given $X^p$ compact metric spaces of finite dimension $d_p=|X^p|$, such that $\sum_{p=1}^m d_p=M$, a function $f: \prod_{p=1}^m X^p \to \mathbb{R}$ is representable in the form 
\begin{align} \label{ostrand}
f(\bm{x}_1,\dots,\bm{x}_m) = \sum_{q=1}^{2M+1} \psi_q(\sum_{p=1}^{m} \phi_{qp}(\bm{x}_p))    
\end{align}
with $\bm{x}_p \in X^p$, and $\phi_{qp}: X^p \to \mathbb{R}$ continuous functions. When $d_p=n, \forall p$, then $M=nm$. The difference between KAT and OST, is that the building functions $\phi_{qp}$ in OST are not defined on scalars (not any more uni-variate), but defined over arbitrary compact spaces $X^p$ (thus multi-variate).

While the original formulation has been criticized \cite{Girosi1989}, other versions of the original superposition theorem have been proposed to counter-argument the smoothness and efficiency of the representation \cite{Kourkova1991}. \autoref{tab:kst} summarizes the various versions of the KAT 
\cite{kolmogorov1957representation,braun2009application,Kourkova1991,kuurkova1992kolmogorov,laczkovich2021superposition,sprecher1963dissertation,sprecher1996numerical}.

\section{Geometric Kolmogorov Superposition Theorem}
\label{sec:geo}
We want to extend the KST to invariant functions to action $g \in O(n)$. 
While the original KST already tells us that we can represent the original function as the superposition of univariate functions \autoref{eq:kst}, which requires a total of $(mn+1)(2mn+1)$ univariate functions, we would like to have a better form of this representation. OST teaches us that we only need $(m+1)(2mn+1)$ functions to represent a multivariate function on $(\mathbb{R}^n)^m$ and these functions take values from $\mathbb{R}^n \to \mathbb{R}$, therefore they are not univariate.
However, we claim that we can represent a generic invariant function $f(\bm{x})$ using only univariate functions, as  
\begin{align} \label{eq:inv_v1}
f(\bm{x}_1,\dots,\bm{x}_m) = \sum_{q=1}^{2m^2+1} \psi_q \left(\sum_{i=1,j=1}^{m,m} \phi_{qij}(\langle \bm{x}_i,\bm{x}_j \rangle) \right),    
\end{align}
more formally stated and proved in \autoref{th:on_v1}, the results is intuitive given that $\langle \bm{x}_i,\bm{x}_j^q \rangle_{i,j=1}^{n}$ represent a complete set of invariant features \cite{villarScalarsAreUniversal2023}. Unfortunately, this form is $m^4$ in the number of nodes. In \autoref{th:on_v2}, we provided an improved version of the geometric KST that grows $m^2$ with the number of nodes, since it only uses a linear number of invariant features. Indeed, if we select $\bm{y}_j^q = \alpha_j^q (\bm{x}_1,\dots,\bm{x}_m)$ a linear combination of the inputs such that they span the full space $\mathbb{R}^n$:  
\begin{align} 
% \label{eq:inv_v2}
&f(\bm{x}_1,\dots, \bm{x}_m) = \nonumber\\
&\hspace{-1mm}\sum_{q=1}^{2mn+1} \psi_q \left( \sum_{\substack{1 \leq i \leq m, \\ 1 \leq j \leq n}} \phi_{qij}(\langle \bm{x}_i,\bm{y}^q_j \rangle) \right. \nonumber 
\left. +  \sum_{\substack{1 \leq i \leq n, \\ 1 \leq j \leq n}} \phi'_{qij} (\langle \bm{y}^q_i,\bm{y}^q_j\rangle)
\right),
\end{align}
in which $\langle \bm{x}_p,\bm{y}_j^q \rangle_{j=1}^{n}= \{ \langle \bm{x}_p,\bm{y}_1^q \rangle \dots \langle \bm{x}_p,\bm{y}_n^q \rangle \}$. While the formal statement and proof are given in \autoref{th:on_v2}, the intuition is that we can project the input on the vectors $\bm{y}^q_j$. Since these vectors, built as linear combinations of the input, do not form an orthonormal basis, we need the information of their inner product $\langle \bm{y}^q_i,\bm{y}^q_j\rangle$ to reconstruct the invariant features $\langle \bm{x}_i,\bm{x}_j \rangle$. 
If we further restrict the vectors $\bm{y}^q_j$ to be a fixed subset of the input features we have that \autoref{th:on_v3}, 
\begin{align} 
\label{eq:inv_v3}
f(\bm{x}_1,\dots,\bm{x}_m) = \sum_{q=1}^{2mn+1} \psi_q \left( \sum_{i=1,j=1}^{m,n} \phi_{qij}(\langle \bm{x}_i,\bm{x}_j \rangle)
\right),    
\end{align}
which reduces further the need for the additional $n^2$ invariant features. 

\paragraph {Equivariant $O(n)$ functions} While in the supplementary material (\autoref{annex:eqivariance}), we discuss the equivariant version of these results, we can build equivariant functions, from invariant functions \cite{villarScalarsAreUniversal2023}, as
\begin{align*}
f(\bm{x}_1,\dots,\bm{x}_m) = \sum_{l=1}^m f_l (\bm{x}_1,\dots,\bm{x}_m) \bm{x}_l   
\end{align*}
with $f_l (\bm{x}_1,\dots,\bm{x}_m)$ invariant functions. Further, we can use the gradient of a geometric invariant function to build equivariant representations
\begin{align*}
f(\bm{x}_1,\dots,\bm{x}_m) = \sum_{l=1}^m \nabla_{\bm{x}_l} f_l (\bm{x}_1,\dots,\bm{x}_m)
\end{align*}
\paragraph{Translation and permutation symmetry} Translation symmetry is obtained by removing the mean of the coordinate from the input, while the permutation invariant \autoref{annex:on_perm_in} is obtained by imposing the univariate function to not depend on the node index. 

\begin{figure*}
    \centering    
    \includegraphics[width=0.8\linewidth]{diagrams/geo-kan_v6.pdf} 
    \caption{The geometric Kolmogorov superposition network is composed of layers that comprise two terms. The first term is based on the classical KST function representation, while the second term, similar to a residual path, is an almost linear term that helps the training of the non-linear functions.}
    \label{fig:geo-kst}
\end{figure*}
\section{Geometric Kolmogorov Superposition Networks (GKSN)} \label{sec:arch}
Finding the representation functions $\psi_q,\phi_{pq}$ is still a hard non-linear optimization problem. To reduce the training complexity, we consider a representation as a layer and allow the composition of multiple layers (\autoref{fig:geo-kst}). The fundamental result from \autoref{eq:inv_v3} is that we can use univariate functions on invariant features. We consider a single layer of the Geometric Kolmogorov Superposition Networks (GKSN) as the composition of the univariate functions $\phi^\ell_{pq}$ and the subsequent univariate functions $\psi^\ell_{q}$. With an abuse of notation and dropping $\ell$ dependence on the functions, we write 
\begin{align}\label{eq:geo-ksn}
\bm{z}_{\ell+1} = \overbrace{\bm{\Psi}}^{ l  \times k} \circ \overbrace{\bm{\Phi}^T}^{k \times m}(\bm{z}_{\ell}) + \overbrace{\bm{W}_\psi}^{l \times k'} \sigma ( \overbrace{\bm{W}_\phi^T}^{ k' \times m} \bm{z}_{\ell}),
\end{align}
or if we compute the $i$-th element, \\
\scalebox{0.95}{
\begin{minipage}{\linewidth}
\begin{align}
z^{\ell+1}_i = \underbrace{ \sum_{k} \psi_{ik} \left( \sum_j \phi_{jk} (z^{\ell}_j) \right) }_{\text{KST}}  
+  \underbrace{\sum_{k} \underbrace{w_{ik}^\psi \sigma }_{\psi_{ik}(.)} \left( \sum_{j} \underbrace{w_{ji}^\phi}_{\phi_{jk}(.)} z^{\ell}_j \right) }_{\text{Residue term}}, \nonumber
% _{\text{approx. linear}},
\end{align}
\end{minipage}
}
where $\circ$ is the function composition operator. 

The first term is the classical KST form, while the second is inspired by the newer forms (\autoref{tab:kst}), which contain linear terms, with a non-linear function $\sigma$ in the middle. We, therefore, assume that the original function can be represented as the sum of two functions, the first with smooth but non-linear univariate functions, the second with composition of a scaled non-linear function, and the sum of linear functions. We further assume $\sigma$ to be a fix almost everywhere smooth, continuous, and almost linear to improve the training of wide layers. The second path plays a role similar to the residual connection, which helps the training of the non-linear univariate functions. 

\section{Experimental Evaluation}
After presenting the experimental setup, we show the performance on representative datasets in molecular dynamics such as
Lennard-Jones particle system, the MD17, and MD22 datasets of the proposed architecture and compare with MLP-based approaches.
\subsection{Experimental setup and baselines}
We compare different models to learn invariant functions from data, from both synthetic and real datasets. In the test, we normalize the output to the interval $[0,1]$.
\paragraph{Symmetries} We name $O(n)$ the models with rotation and reflection symmetry, while we use $\pi$ for the models that implement permutation symmetry. 
\paragraph{Networks}  We mainly compare against the use of two layers {\bf MLP} models.  
We implemented the {\bf KAN}  model of \autoref{eq:geo-ksn}, where we use ReLU \cite{glorot2011deep} both as the basis for the KAN non-linear functions ($\psi_q,\phi_{pq}$) and for the residual connection ($\sigma$). The name of the model contains two symbols $T$=True and $F$=False; the first boolean tells us if the node index is used as an additional $O(n)$ invariant feature. The effect of adding the index of the node is to emulate the non-permutation invariant function. The second boolean is used to show if the linear ($T$) (\autoref{eq:inv_v3}) or quadratic ($F$) (\autoref{eq:inv_v1}) feature is used. Therefore, $\pi ~ O(n)$ KAN($T,T$) is a permutation invariant model based on the KAN architecture, where node index is used as a feature, where the number of features is linear in the number of nodes $m$.  
\paragraph{Invariant Features} While \autoref{eq:inv_v3} tells us that we can represent any invariant function with the inner products, nevertheless, to improve expressivity, 
we extend the invariant feature to include: 
\begin{align}
\|\bm{x}_i \|,\|\bm{y}_j\|, \|\bm{x}_i-\bm{y}_j\|, \langle \bm{x},\bm{y}_j \rangle, \sqrt{\|\bm{x}_i \|^2 \|\bm{y}_j\|^2-\langle \bm{x},\bm{y}_j \rangle^2} \nonumber
\end{align}
As additional invariant features, we optionally include the node index (first flag), and when present (experiments with MD17 and MD22), we also include the atom type. We have not explored alternative ways to embed the node's additional information as input to the network. The last term is also equivalent to $\| x \times y\|$ in $n=3$ dimensions, with $\times$ the cross product.
\paragraph{Quadratic versus Linear features} 
A consequence of \autoref{eq:inv_v3}, with the associated theorem, is that the number of invariant features that we need is linear with the number of nodes. 
We nevertheless, compare also with the quadratic version as in \autoref{eq:inv_v1}.


\begin{table}
\caption{Huber NLL ($\uparrow$, higher is better) for the LJ dataset
on different dimensions ($n \in [3,5]$) and different number of nodes $m \in [4,10,15]$. Standard deviation in parenthesis, mean computed over $3$ runs.}
\label{tab:lj1}
\begin{tabularx}{\columnwidth}{
@{} Y{1.5} Y{0.9} Y{0.9} Y{0.9} Y{0.9} @{}
}
\toprule
LJ  $m/n$ & $O(n)$ KAN & $O(n)$ MLP & $\pi~O(n)$ KAN  & $\pi~O(n)$ MLP \\
\midrule
4/3 & \textbf{8.41} & 8.00 & 7.88 & 7.59 \\
 & (0.19) & (0.12) & (0.15) & (0.14) \\
10/3 & \textbf{7.10} & 6.76 & \textbf{7.08} & 5.33 \\
 & (0.16) & (0.09) & (0.28) & (0.18) \\
10/5 & \textbf{7.15} & \textbf{6.71} & \textbf{7.23} & 3.72 \\
 & (0.37) & (0.28) & (0.41) & (0.60) \\
15/3 & \textbf{7.25} & \textbf{7.09} & \textbf{7.28} & 3.92 \\
 & (1.25) & (1.10) & (1.17) & (0.41) \\
15/5 & 6.73 & 6.56 & \textbf{6.96} & 1.76 \\
 & (0.18) & (0.13) & (0.24) & (1.33) \\
\bottomrule
\end{tabularx}
\end{table}

\begin{figure}
    \centering
    \includegraphics[width=1.0\linewidth]{fig_syn4/m15+n5+terms-lj+huber+_nll_te_v1_ipFalse_False.pdf}
    \caption{Test performance (Negative log Huber Loss) of $O(n)$ invariant models for the LJ experiment with $n=5$ and $m=15$.  
    In parenthesis, the two flags indicate if the model includes the node index $(T,*)$ or not $(F,*)$; the second flag signals if the features are linear $(*,T)$ (according to \autoref{eq:inv_v3}) or quadratic $(*,F)$ (according to \autoref{eq:inv_v1}) in the number of nodes. }
    \label{fig:lj_te_false}
\end{figure}



\begin{figure}
    \centering
    \includegraphics[width=1.0\linewidth]{fig_syn4/m15+n5+terms-lj+huber+_nll_te_v1_ipTrue_False.pdf}     
    \caption{Test performance (Negative log Huber Loss $\uparrow$) of $O(n)$ and permutation invariant models for the LJ experiment.
    }
    \label{fig:lj_te_true}
\end{figure}




\subsection{Lennard-Jones experiments}
\label{sec:lj}
% [training FF]
Lennard-Jones potential approximates inter-molecular pair interaction and models repulsive and attractive interactions. It captures key physical principles and it is widely used to model solid, fluid, and gas states. More details are in \autoref{annex:LJ}. 
\autoref{fig:lj_te_false} and \autoref{fig:lj_te_true} show the test regression loss during training for a system in $3$ dimensions and with $15$ nodes. The loss is plotted in a negative log scale. We use the Huber loss, which is quadratic if the error is less than $1$, and linear if larger. The test loss for the $O(n)$ invariant model (\autoref{fig:lj_te_false}) is regular during training and all models seem to have similar results, while in \autoref{fig:lj_te_true} the performance of permutation invariant models have quite different behavior. The MLP-based models are more unstable, while KAN-based models have a much more regular performance. 
\autoref{tab:lj1} summarizes the regression accuracy at test time for all the models. The permutation invariance reduces the performances, but more remarkably on smaller systems.  

\begin{figure}
    \centering
    \includegraphics[width=1.0\linewidth]{fig_md4/ds-md17_ethanol.npz+huber+_nll_te_v1_ipFalse_False.pdf}       
    \caption{Test performance (Negative log Huber Loss $\uparrow$) of various models for the Ethanol dataset of MD17. $O(n)$ is the model that is invariant to rotation and reflection on $\mathbb{R}^n$.
    }
    \label{fig:md17_te_false}
\end{figure}



\begin{table}
\caption{Huber NLL $\uparrow$ for the MD17 dataset (mean and standard deviation in paraenthesis)}
\label{tab:md17}
\begin{tabularx}{\columnwidth}{
@{} Y{1.5} Y{0.9} Y{0.9} Y{0.9} Y{0.9} @{}
}
\toprule
Dataset (MD17) & $O(n)$ KAN & $O(n)$ MLP & $\pi~O(n)$ KAN & $\pi~O(n)$ MLP \\
\midrule
Aspirin & {\bf 6.44} & 5.62 & 5.69 & 4.73 \\
 & (0.10) &( 0.01) & (0.02) & (0.27) \\
Benzene & {\bf 7.66} & 5.93 & 6.51 & 5.64 \\
 & (0.08) & (0.01) & (0.17) &( 0.13) \\
Ethanol & {\bf 7.57} & 5.44 & 6.09 & 5.49 \\
 & (0.04) & (0.01) & (0.13) & (0.03) \\
Malonaldehyde & {\bf 7.50} & 5.39 & 5.85 & 5.38 \\
 & (0.05 )& (0.01) & (0.04) & (0.04) \\
Naphthalene & {\bf 6.85} & 5.35 & 5.72 & 4.65 \\
 & (0.07) & (0.00) & (0.09) & (0.76) \\
Salicylic & {\bf 6.96} & 5.62 & 5.83 & 5.17 \\
 & (0.09) & (0.00) & (0.10) &( 0.24) \\
Toluene & {\bf 7.05} & 5.68 & 6.03 & 5.40 \\
 &( 0.13) &( 0.02) &( 0.10) &( 0.11) \\
Uracil & {\bf 7.54} & 5.65 & 6.10 & 5.52 \\
 &( 0.08) & (0.01) &( 0.11) & (0.05) \\
\bottomrule
\end{tabularx}
\end{table}

\begin{figure}
    \centering
    \includegraphics[width=1.0\linewidth]{fig_md4/ds-md17_ethanol.npz+huber+_nll_te_v1_ipTrue_False.pdf}    
    
    \caption{Test performance (Negative log Huber Loss $\uparrow$) of various models for the Ethanol dataset of MD17. 
    $\pi- O(n)$ are the models that are invariant to rotation, reflection, and permutation. 
    }
    \label{fig:md17_te_true}
\end{figure}



\subsection{MD17}
\label{sec:md17}
MD17 dataset contains samples from a long molecular dynamics trajectory of a few small organic molecules \cite{chmiela2017machine}. For each molecule, we split into $8,000$ training and $200$ test configurations. 
In \autoref{tab:md17} we show the negative log of the Huber loss (NLL), thus the higher the value the better, aggregated over various model options, while in \autoref{tab:md17_detailed} 
we provide the test loss for each model. 
% We report the negative log of the Huber loss (NLL), thus the higher the value, the better. 
\autoref{fig:md17_te_true} and \autoref{fig:md17_te_false} show the Huber NLL at test time for the Toluene molecule for the two classes of models. 
The test loss in negative log scale at training for $O(n)$ invariant models in \autoref{fig:md17_te_false} is stable, but reducing the number of features leads to lower performance, while KAN shows better accuracy. The training for the permutation invariant models in \autoref{fig:md17_te_true} is less stable and the overall performance reduces while keeping the model size smaller. 
\autoref{tab:md17} summarizes the performance of all models in the various atomic systems of MD17, the KAN-based models show consistently better performance, even with smaller network size. 



\begin{figure}
    \centering
    \includegraphics[width=1.0\linewidth]{fig_md4/ds-md22_buckyball-catcher.npz+huber+_nll_te_v1_ipFalse_False.pdf}  
    \caption{Test performance (Negative log Huber Loss $\uparrow$) of $O(n)$ invariant models for the Buckyball-Catcher dataset of MD22.
    }
    \label{fig:md22_te_false}
\end{figure}


\begin{table}
\caption{Performance aggregated at the level of the model type for the MD22 dataset; the performance is the negative log of the Huber loss $\uparrow$ (mean and standard deviation in parenthesis); }
\label{tab:md22}
\begin{tabularx}{\columnwidth}{
@{} Y{1.5} Y{0.9} Y{0.9} Y{0.9} Y{0.9} @{}
}
\toprule
Dataset (MD22) & $O(n)$ KAN & $O(n)$ MLP & $\pi~O(n)$ KAN & $\pi~O(n)$ MLP \\
\midrule
AT-AT-CG-CG & {\bf 8.02} & 7.61 & 7.73 & 0.82 \\
 & (0.14) &( 0.05) &( 0.05) & (0.32) \\
AT-AT & {\bf 7.32} & 6.56 & 6.62 & 0.82 \\
 &( 0.21) & (0.01) &( 0.03) & (0.40) \\
Ac-Ala3-NHMe & {\bf 5.77} & 5.57 & 5.57 & 1.48 \\
 & (0.07) & (0.00) & (0.01) & (1.08) \\
DHA & {\bf 5.64} & 5.52 & 5.50 & 0.04 \\
 & (0.07) &( 0.00) & (0.01) & (0.82) \\
Buckyball-catcher & {\bf 8.85} & 7.27 & 7.41 & 0.21 \\
 & (0.24) & (0.01) & (0.07) & (0.71) \\
Stachyose & {\bf 6.30} & 5.70 & 5.73 & 1.36 \\
 & (0.12) & (0.01) & (0.03) & (1.42) \\
\bottomrule
\end{tabularx}
\end{table}


\begin{figure}
    \centering
    \includegraphics[width=1.0\linewidth]{fig_md4/ds-md22_buckyball-catcher.npz+huber+_nll_te_v1_ipTrue_False.pdf}
    \caption{Training performance (Negative log Huber Loss $\uparrow$) of  $O(n)$ and permutation invariant models for the Buckyball-Catcher dataset of MD22.
    }
    \label{fig:md22_te_true}
\end{figure}



\subsection{MD22}
\label{sec:md22}
MD22 dataset \cite{chmiela2023accurate} contains samples from molecular dynamics trajectories of four major classes of biomolecules, as proteins, lipids, carbohydrates, nucleic acids, and supramolecules. In MD22, number of atoms ranges from $42$ to $370$. 
For each molecule, we split into $8,000$ training and $200$ test configurations. In \autoref{tab:md22} we show the NLL aggregated over various model options, while in \autoref{tab:md22_detailed} for more details information on the performance. \autoref{fig:md22_te_false} and \autoref{fig:md22_te_true} show the Huber NLL at test time 
for the Ac-Ala3-NHMe  molecule, with and without permutation invariance. 

Similar to the MD17 dataset,  the test loss in negative log scale at training for the $O(n)$ invariant models reported in \autoref{fig:md22_te_false} is stable for the KAN-based models, while MLP-based models show more unstable training and lower performances.  The training for the permutation invariant models in \autoref{fig:md22_te_true} is even less stable for the MLPs leading to low accuracy. 
\autoref{tab:md22} summarizes the performance of all models in the various atomic systems of MD22, the KAN-based models show consistently better performance, even with smaller network size. 

% \vspace{-5mm}
\section{Conclusions}
% \vspace{-2mm}

We propose an extension of the KAN architecture for invariant and equivariant function representation, which is based on  
% We presented 
the theoretical results that provide us with a lower bound on the number of functions needed for approximating invariant functions.
The theoretical results in \autoref{sec:geo}, provide a considerable improvement with previous results \cite{villarScalarsAreUniversal2023}, reducing the complexity from quadratic to linear.
We further tested the performance and compared it with MLP-based architectures on an ideal physical system, the Lennard-Jones experiment, and on two real molecular datasets, the MD17 and the MD22 datasets. The performance of the proposed network architecture shows in our experiments improved performance with respect to MLP, and further investigation will show if this architecture can be extended to implement KAN-based machine learning interatomic potentials. 



% In the unusual situation where you want a paper to appear in the
% references without citing it in the main text, use \nocite
% \nocite{langley00}

\bibliography{kolmogorov,kan,mlip,scalar}
\bibliographystyle{icml2025}



\newpage
\appendix
\onecolumn

\section*{Supplementary Material of Geometric Kolmogorov-Arnold Superposition Theorem}

\section{Main theorems for the Kolmogorov 
Superposition Theorem for invariant and equivariant functions }
\label{annex:KAT}

We first recall the Kolmogorov - Arnold
and Ostrand theorems.
\begin{theorem}{\cite{kolmogorov1961representation}} \label{th:kan}
A function $f(x_1,\dots,x_m): \mathbb{R}^{m}  \to \mathbb{R}$, with $X$ a compact space, it can be represented as $f(x_1,\dots,x_m) = \sum_{q=1}^{2m+1} \psi_q(\sum_{p=1}^{m} \phi_{qp}(x_p))$. 
with $\psi_q: \mathbb{R} \to \mathbb{R}$ and $\phi_{qp}: [0,1] \to \mathbb{R}$ uni-variate continuous functions. 
\end{theorem}


\begin{theorem}{\cite{ostrandDIMENSIONMETRICSPACESa}}
\label{thm:ostrand}
A function $f(x_1,\dots,x_m): (X)^m \to \mathbb{R}$, with $X \subset \R^n$ a compact space, it can be represented as $f(x_1,\dots,x_m) = \sum_{q=1}^{2m+1} \psi_q(\sum_{p=1}^{m} \phi_{qp}(x_p))$. 
with $\bm{x}_p \in X^n$, $\phi_{qp}: X \to \mathbb{R}$ continuous functions, and $\psi_q: \mathbb{R} \to \mathbb{R}$ .
\end{theorem}


\subsection{ Permutation invariance }
\begin{lemma}(Permutation invariance)
\label[lemma]{lm:perm}
The following function is invariant to the action of permutation group:$f(x_1,\dots,x_m) = \sum_{q=1}^{2m+1} \psi_q(\sum_{p=1}^{m} \phi_{q}(x_p))$.
\end{lemma}

\begin{proof}
Since the decomposition requires to the the same for a generic permutation $\pi$ then 
$$
\sum_{q=1}^{2m+1} \psi_{q}(\sum_{p=1}^{m} \phi_{qp}(x_p)) = \sum_{q=1}^{2m+1} \psi_{q}(
\sum_{p=1}^{m} \phi_{qp}(x_{\pi{(p)}})
)
$$
to be true, we need to drop the dependence of $\phi_{qp}$ on the node index $p$.

\end{proof}

\begin{remark}
We note that while the expression looks quite similar to KAT in appearance, it is not known whether the above expression is universal for arbitrary permutation invariant functions.    
\end{remark}

\subsection{ $O(n)$ invariance} 
\label{annex:OnInvariant}

We here consider the permutation group that acts on the input $(\bm{x}_1, \dots, \bm{x}_m)$ and present the architecture invariant to the action of the orthogonal group.

\begin{theorem}($O(n)$ invariance - v1) \label{th:on_v1}
For an $O(n)$ invariant function $f(\bm{x}_1,\dots,\bm{x}_m): X^m \to \mathbb{R}$, with $X \subset \mathbb{R}^n$ a compact space, it can be represented as 
$$
f(\bm{x}_1,\dots,\bm{x}_m) = \sum_{q=1}^{2m^2+1} \psi_q(\sum_{i,j=1}^{m,m} \phi_{qij}(\langle \bm{x}_i,\bm{x}_j \rangle)),
$$
\end{theorem}
\begin{proof}
    Lemma 1 (First Fundamental Theorem for $O(d)$) in \cite{villarScalarsAreUniversal2023} and \autoref{th:kan}.
\end{proof}


\begin{theorem}($O(n)$ invariance - v2) \label{th:on_v2}
For an $O(n)$ invariant function $f(\bm{x}_1,\dots,\bm{x}_m): X^m \to \mathbb{R}$, with $X \subset \mathbb{R}^n$ a compact space, it can be represented as 
$$
f(\bm{x}_1,\dots,\bm{x}_m) = \sum_{q=1}^{2mn+1} \psi_q \left( \sum_{i=1,j=1}^{m,n} \phi_{qij}(\langle \bm{x}_i,\bm{y}_j \rangle)
+  \sum_{i=1,j=1}^{n,n} \phi'_{qij} (\langle \bm{y}_i,\bm{y}_j\rangle)
\right),
$$
where $\bm{y}_j^q = \alpha_j^q (\bm{x}_1,\dots,\bm{x}_m) = \sum_{p=1}^{m} \alpha_p^{j} \bm{x}_p$, with $\bm{y}_j^q$ a linear combination of $\{\bm{x}_p\}$ with scalars $\alpha_{p}$ such that $\text{span}(\{\bm{y}_j^q\}_{j=1}^n)=\mathbb{R}^n$. 
\end{theorem}
\begin{proof}
The proof is based on the use of Lemma 1 (First Fundamental Theorem for $O(d)$) in \cite{villarScalarsAreUniversal2023}, 
\autoref{th:corr}
and \autoref{th:kan}. Since we define $y_j$ as linear combination of $x_p$ then also $\langle x_p,y_j \rangle$ and $\langle y_p,y_j \rangle$ are invariant to rotation, e.g. $\langle R x_p, y'_j \rangle = \langle R x_p,\sum \alpha_i R x_i \rangle = \langle R x_p, R \sum \alpha_i x_i) \rangle = \langle R x_p,R y_j \rangle = \langle x_p,y_j \rangle$. 
    
\end{proof}


\begin{theorem}($O(n)$ invariance - v3) \label{th:on_v3}
For an $O(n)$ invariant function $f(\bm{x}_1,\dots,\bm{x}_m): X^m \to \mathbb{R}$, with $X \subset \mathbb{R}^n$ a compact space, it can be represented as 
$$
f(\bm{x}_1,\dots,\bm{x}_m) = \sum_{q=1}^{2mn+1} \psi_q \left( \sum_{i=1,j=1}^{m,n} \phi_{qij}(\langle \bm{x}_i,\bm{x}_j \rangle)
\right),
$$
where we assume that  $\text{span}(\{\bm{x}_j\}_{j=1}^n)=\mathbb{R}^n$. 
\end{theorem}
\begin{proof}
    Lemma 1 (First Fundamental Theorem for $O(d)$) in \cite{villarScalarsAreUniversal2023}, 
    \autoref{th:subset}
    and \autoref{th:kan}.
\end{proof}




\subsection{$O(n)$ and permutation invariance} \label{annex:on_perm_in}
We further consider the permutation group action to the input $(\bm{x}_1, \dots, \bm{x}_m)$ and present the architecture invariant to the action of the permutation group. 

\begin{corollary}($O(n)$ and permutation invariance - v1)
\label{cor:orth_perm_v1}
The following function is invariant to the action of the permutation group and the orthogonal group $O(n)$: $f(\bm{x}_1,\dots,\bm{x}_m) = \sum_{q=1}^{2mn+1} \psi_q \left( \sum_{i=1,j=1}^{m,m} \phi_{q}(\langle \bm{x}_i,\bm{x}_j \rangle)
\right)$.

\end{corollary}

\begin{proof}
    We based this result on \autoref{th:on_v1} and \Cref{lm:perm}, by removing the dependence on the node index, the function is now permutation invariant. 
\end{proof}

\begin{remark}
We note that while the expression looks quite similar to KAT in appearance, it is not known whether the above expression is universal for arbitrary $O(n)$ and permutation invariant functions.    
\end{remark}

\subsection{ $O(n)$ equivariance} \label{annex:eqivariance}
We have the corresponding equivariant version.


\begin{theorem} ($O(n)$ equivariance - v1)
\label{th:on_eq_v1}
For an $O(n)$ equivariant function $f(x_1,\dots,x_m): X^m \to X$, with $X \subset \mathbb{R}^n$ compact space, it can be represented as 
$$
f(\bm{x}_1,\dots,\bm{x}_m) = \sum_{k=1}^m  \sum_{q=1}^{2mn+1} \psi^k_q \left( \sum_{i=1,j=1}^{m,m} \phi^k_{qij}(\langle \bm{x}_i,\bm{x}_j \rangle)
\right) \bm{x}_k,
$$

where we assume that  $\text{span}(\{\bm{x}_j\}_{j=1}^n)=\mathbb{R}^n$. 
\end{theorem}
\begin{proof}
    We based this result on \autoref{th:on_v1} and on the equivariant form (Proposition 4) from \cite{villarScalarsAreUniversal2023}. 
\end{proof}

Similar results can be obtained for the representation from  \autoref{th:on_v2} or \autoref{th:on_v3}. 


It is possible to show that we can use the gradients of invariant functions to build a generic equivariant function, in particular, if 
$f(\bm{x},\dots,\bm{x}_m)$ is invariant, then 
$$
\nabla_{\bm{x}_i} f(\bm{x},\dots,\bm{x}_m)
$$
is equivariant, as it is 
$$
\sum_{i=1}^m \alpha_i \nabla_{\bm{x}_i} f(\bm{x},\dots,\bm{x}_m)
$$
Extending the previous results with these forms is easy when $f$ is decomposed according to \autoref{th:on_v1}, \autoref{th:on_v2} or \autoref{th:on_v3}. 

\subsection{ $O(n)$ equivariance and permutation invariance} 
We have the corresponding equivariant and permutation invariant versions. 


\begin{corollary} ($O(n)$ equivariance and permutation invariance - v1)
\label{th:on_perm_eq_v1}
The following function is invariant to the action of the permutation group and the orthogonal group $O(n)$: 
$$
f(\bm{x}_1,\dots,\bm{x}_m) = \sum_{i=1}^m  \sum_{q=1}^{2mn+1} \psi_q \left( \sum_{j=1}^{m} \phi_{q}(\langle \bm{x}_i,\bm{x}_j \rangle)
\right) \bm{x}_i.
$$

\end{corollary}

\begin{proof} 
    We based this result on \autoref{th:on_eq_v1} and \Cref{lm:perm}. 
\end{proof}

\begin{remark}
We note that while the expression looks quite similar to KAT in appearance, it is not known whether the above expression is universal for arbitrary $O(n)$ and permutation invariant functions.    
\end{remark}


\subsection{Mapping invariant features}
\begin{lemma} \label{th:A14}
Suppose that we have $\bm{X} \in \R^{m \times n}$ and $\bm{Y} \in \R^{k \times n}$ with $\rho(\bm{Y}) = n, n \le k$ 
then
 $$
 \bm{X}\bm{Y}^T (\bm{Y}\bm{Y}^T)^{\dagger} \bm{Y}\bm{X}^T = \bm{X}\bm{X}^T,  
 $$
where $\rho(X)$ is the matrix rank and $^\dagger$ is the pseudo-inverse.
\end{lemma}
\begin{proof}
The equality follows from these properties:
$$
\bm{Y} = \bm{V} \bm{\Lambda} \bm{U}, ~~~ \bm{V}^T\bm{V} = \bm{I}_k, 
    ~~~ \bm{U}^T\bm{U} = \bm{I}_n = \bm{U}\bm{U}^T,
$$
$$
(\bm{Y}\bm{Y}^T)^{\dagger} = (\bm{V} \bm{\Lambda}  \bm{\Lambda}^T \bm{V}^T)^{\dagger} = \bm{V} ( \bm{\Lambda}  \bm{\Lambda}^T )^{\dagger} \bm{V}^T,
$$
$$
\bm{Y}^T = \bm{U}^T  \bm{\Lambda}^T \bm{V}^T,
$$
$$
\bm{Y}^T (\bm{Y}\bm{Y}^T)^{\dagger} \bm{Y} = \bm{U}^T  \bm{\Lambda}^T \bm{V}^T  \bm{V} ( \bm{\Lambda}  \bm{\Lambda}^T )^{\dagger} \bm{V}^T  \bm{V} \bm{\Lambda} \bm{U} = \bm{U}^T  \bm{\Lambda}^T  ( \bm{\Lambda}  \bm{\Lambda}^T )^{\dagger} \bm{\Lambda} \bm{U} = \bm{I}_n.
$$
\end{proof}

\begin{theorem} (Correlation matrix representation) 
\label{th:corr}
Given $\bm{x}_1,\dots,\bm{x}_m \in \mathbb{R}^n$ and a set of points $\bm{y}_1,\dots,\bm{y}_k \in \R^n$, such that $\rho(\bm{y}_1,\dots,\bm{y}_k) = n$, there is an invertible map between these two sets:
\begin{itemize}
    \item $\{ \langle \bm{x}_i,\bm{x}_j\rangle \}_{i,j=1}^{m,m}$, with a total number of variable equal to $m^2$
    \item $\{ \langle \bm{x}_i,\bm{y}_j\rangle \}_{i,j=1}^{m,k}$, $\{ \langle \bm{y}_i,\bm{y}_j\rangle \}_{i,j=1}^{k,k}$ with a total number of variable equal to $mk+k^2$
\end{itemize}
\end{theorem}
\begin{proof}
    Define $\bm{X} =(\bm{x}_1,\dots,\bm{x}_m)^T \in \R^{m \times n}$ and $\bm{Y} =(\bm{y}_1,\dots,\bm{y}_k)^T \in \R^{k \times n}$
    then 
    $$
    \bm{X}\bm{X}^T = \{ \langle \bm{x}_i,\bm{x}_j\rangle \}_{i,j=1}^{m,m},
    $$
    $$
    \bm{X}\bm{Y}^T = \{ \langle \bm{x}_i,\bm{y}_j\rangle \}_{i,j=1}^{m,k},
    $$ and
    $$
    \bm{Y}\bm{Y}^T = \{ \langle \bm{y}_i,\bm{y}_j\rangle \}_{i,j=1}^{k,k}.
    $$
    We then apply \autoref{th:A14} to yield
     $$
     \bm{X}\bm{X}^T  = \bm{X}\bm{Y}^T (\bm{Y}\bm{Y}^T)^{\dagger} \bm{Y}\bm{X}^T.
     $$     
    Notice that $\bm{Y}\bm{X}^T = (\bm{X}\bm{Y}^T)^T$, and   
    therefore we have the result.
\end{proof}




We define $\bm{Y}$ as a subset of $\bm{X} \in \R^{m \times n}$ of size $k$, then it is a matrix of dimension $k \times n$, which we ask to have rank $n$. We then can say,
\begin{corollary} (Special case - Subset) \label{th:subset}
If $\bm{Y}=\bm{X}[:n]$, with $n \le k$, $\rho(\bm{Y})=n$, $\bm{X} \in \R^{ m \times n}$, $m \le k \le n$, then there is an invertible map between these two sets:
\begin{itemize}
    \item $\{ \langle \bm{x}_i,\bm{x}_j\rangle \}_{i,j=1}^{m,m}$, with a total number of variable equal to $m^2$
    \item $\{ \langle \bm{x}_i,\bm{x}_j\rangle \}_{i,j=1}^{m,k}$, if $\bm{y}_j = \bm{x}_j$, with a total number of variable equal to $mk$,     
\end{itemize}    
\end{corollary}
\begin{proof}
    We use \autoref{th:corr} and notice that 
    $\bm{Y}\bm{Y}^{\operatorname{T}}$ 
    can be derived from 
    $\bm{Y}\bm{Y}^\mathrm{T} = \bm{X}[:n]\bm{X}[:n]^\mathrm{T}$, 
    which are included in the previous features. 
\end{proof}

\subsection{Informal proof of the main theorem}
There is one step in our theorem that creates concern. This step is as follows: once we change the basis for our data, we build the basis from the data itself. 
We now prove with a simple Python code that this is the case. 

\begin{lstlisting}[language=Python, caption=Python based informal proof]
# some help functions
rot_gen = lambda n: np.linalg.svd(np.random.randn(n,n))[0]
basis = lambda X: X[:n,:]
corr = lambda X: X @ X.T
inv = lambda X,Y: X @ Y.T
rot = lambda X,R: X @ R
#set the seed; it can be removed or changes
np.random.seed(42)
# the problem's dimension can be changed, but m>=n
m,n = 5,3
# this is my data
X = np.random.randn(m,n)
# the correlation matrix of the data, which is an invariant feature
C1 = corr(X)
# we build a basis that depends on the input
Y = basis(X)
# compujte invariant features
Z1 = inv(X,Y)
# compute the correlation of the new features
D1 = corr(Z1)
# some rotation
R = rot_gen(n)
# apply the rotation to the input
X = rot(X, R)
# rebuild the basis
Y = basis(X)
# compute the invariant features
Z2 = inv(X,Y)
# compute the correlation with the new invariant features
D2 = corr(Z2)
# Question: is the correlation matrix before and after the same (we know is the same):
print(np.linalg.norm(C1 - C2))
# Result: 1.934545700657722e-15 (yes, numerically the same)
# Question: is the correlation matrix with the invariant feature the same before and after (they should)
print(np.linalg.norm(D1 - D2))
# Result: 9.407543438562363e-15 (yes, numerically the same)
# Question: are the invariant features the same, before and after the rotation (they better be)?
print(np.linalg.norm(Z1 - Z2))
# Result: 1.4220500840710913e-15 (yes, numerically the same)
\end{lstlisting}

\subsection{Informal proof of \autoref{th:A14}}
\begin{lstlisting}[language=Python, caption=Python based informal proof]
import numpy as np
from numpy.linalg import norm
np.random.seed(42)
# the problem's dimension can be changed, but m>=n
m,n = 15,3
k = n+2
# create the two matrices
X = np.random.randn(m,n)
Y = np.random.randn(k,n)
# Verify Theorem A.14
print(norm(X @ Y.T @ np.linalg.pinv(Y @ Y.T) @ Y @ X.T - X@X.T))
# Result: 1.2816111681783468e-14
\end{lstlisting}



\section{Complexity}
The representation complexity of \autoref{eq:inv_v1} is $O(m^4)$, which is quite larger than the complexity we have if we apply KAT directly to the coordinates of the nodes, i.e. $O(m^2n^2)$, which ignores the symmetries of the problem.

However, in \autoref{eq:inv_v3}, we show that we can represent the invariant function $f$ with complexity $O(m^2 n^2)$, thus similar to the non-invariant KAT.


\section{Additional Experiments}

\begin{figure}
    \centering
    \includegraphics[width=0.55\linewidth]{fig_syn4/m15+n5+terms-distance-lj+huber+_nll_te_v1_ipFalse_False.pdf}
    \caption{Test performance (Negative log Huber Loss) of various models for the linear polymers. $O(n)$ is the model that is invariant to rotation and reflection on $\mathbb{R}^n$, while $\pi$ is the permutation invariant model. In parenthesis, the two flags indicate if the model includes the node index and the second if the features are linear or quadratic in the number of nodes. 
    % \tm{the figure is a bit too large?
    }
    
    \label{fig:lp_te_false}
\end{figure}

\begin{figure}
    \centering
    \includegraphics[width=0.55\linewidth]{fig_syn4/m15+n5+terms-distance-lj+huber+_nll_te_v1_ipTrue_False.pdf}     \caption{Test performance (Negative log Huber Loss) of various models for the linear polymers. $O(n)$ is the model that is invariant to rotation and reflection on $\mathbb{R}^n$, while $\pi$ is the permutation invariant model. In parenthesis, the two flags indicate if the model includes the node index and the second if the features are linear or quadratic in the number of nodes. 
    % \tm{the figure is a bit too large?}
    }
    \label{fig:lp_te_true}
\end{figure}

\begin{table}
\caption{Huber NLL for the Linear Polymer dataset, with $a_i=0$ on different dimensions ($3,5$) and different number of nodes $4,10,15$. }
\label{tab:lp1}
\begin{tabularx}{\columnwidth}{
@{} Y{1.5} Y{0.9} Y{0.9} Y{0.9} Y{0.9} @{}
}
% \begin{tabular}{lllll}
\toprule
LinPoly-1 & $O(n)$ KAN & $O(n)$ MLP & $\pi~O(n)$ KAN & $\pi~O(n)$ MLP \\
\midrule
m4/n3 & 10.85 & 11.74 & 9.07 & 9.29 \\
 & 0.52 & 0.18 & 0.40 & 0.33 \\
m10/n3 & 8.93 & 9.08 & 7.36 & 6.40 \\
 & 0.33 & 0.36 & 0.32 & 0.25 \\
m10/n5 & 9.22 & 9.06 & 7.41 & 5.97 \\
 & 0.10 & 0.16 & 0.14 & 0.12 \\
m15/n3 & 7.99 & 7.98 & 6.91 & 4.82 \\
 & 0.34 & 0.24 & 0.38 & 0.51 \\
m15/n5 & 7.99 & 7.81 & 6.76 & 4.18 \\
 & 0.33 & 0.16 & 0.39 & 0.78 \\
\bottomrule
\end{tabularx}
\end{table}

\subsection{Linear Polymer experiments}
Linear polymers are chain molecules composed of repeating structural units (monomers) linked together sequentially. Linear polymers exhibit flexibility and thermoplastic behavior. Examples include polyethylene (PE), polyvinyl chloride (PVC), and polystyrene (PS), and find applications in packaging, textiles, and plastic films due to their ease of processing, recyclability, and ability to be melted and reshaped.
\autoref{fig:lp_te_false} and \autoref{fig:lp_te_true} show the performance with $O(n)$ symmetry and with additionally permutation symmetry. 
Additional details in \autoref{annex:polymers}.

\section{Experiments}\label{annex:experiments}

\begin{table}
\caption{Huber NLL for the LJ-2 dataset}
\centering
\begin{tabular}{lllll}
\toprule
LJ (2) & $O(n)$ KAN & $O(n)$ MLP & $\pi~O(n)$ KAN & $\pi~O(n)$ MLP \\
\midrule
m4/n3 & 9.54 & 8.52 & 9.35 & 8.56 \\
& 0.82 & 0.43 & 0.62 & 0.39 \\
m10/n3 & 8.66 & 8.22 & 8.49 & 5.73 \\
 & 0.66 & 0.61 & 0.74 & 0.25 \\
m10/n5 & 7.52 & 7.02 & 7.19 & 4.84 \\
 & 0.27 & 0.10 & 0.32 & 0.19 \\
m15/n3 & 9.45 & 9.35 & 9.89 & 3.91 \\
 & 1.32 & 1.43 & 2.11 & 0.79 \\
m15/n5 & 6.66 & 6.47 & 6.74 & 2.36 \\
 & 0.23 & 0.27 & 0.25 & 1.33 \\
\bottomrule
\end{tabular}
\end{table}

\subsection{Lennard-Jonnes}\label{annex:LJ}
For the Lennard-Jonnes (LJ) experiments, we generate $m$ particles in $n$ dimensional space. The interaction between particles is described by the LJ potential, 
$$
U_\text{LJ}(r) =  f((a/r)^{12} - (a/r)^6)
$$
where $r$ is the distance between two particles and $a$ is a parameter that defines the minimum energy of the interaction, while $f(x) = x+ \sum_{l=1}^{3} a_l \sin(w_l x)$, with $a_1=1,a_2=.3,a_3=.1, w_1=11,w_2=30,w_3=50$ (or $a_1=a_2=a_3=0$), is an oscillatory term.
After generating the particles, we perform an energy minimization step to relax the system towards a lower energy state, avoiding large energy contributions caused by the random initialization of the particle positions. 

\subsection{Linear polymers}\label{annex:polymers}
As an additional experiment, we consider linear polymers of size $m$. The particles are connected to the previous and the following particle by a bond. The interaction between the bond depends quadratically on the difference between the current distance and the desired distance,
$$
U_\text{bond}(r) = f(\| d - \hat{d} \|^2) + U_\text{LJ}(r)
$$
and $f(x) = x+ \sum_{l=1}^{3} a_l \sin(w_l x)$ is an oscillatory term.
For the unbonded particle, the LJ potential is used, as before. 

\begin{table}
\caption{Huber NLL for the LinPoly-2 dataset}
\centering
\begin{tabular}{lllll}
\toprule
LinPoly-2 & $O(n)$ KAN & $O(n)$ MLP & $\pi~O(n)$ KAN & $\pi~O(n)$ MLP \\
\midrule
m4/n3 & 10.51 & 8.78 & 8.41 & 7.13 \\
 & 0.17 & 0.13 & 0.27 & 0.08 \\
m10/n3 & 8.30 & 7.50 & 7.26 & 4.73 \\
 & 0.40 & 0.18 & 0.39 & 0.78 \\
m10/n5 & 8.36 & 7.77 & 7.18 & 4.11 \\
 & 0.45 & 0.16 & 0.48 & 0.64 \\
m15/n3 & 7.40 & 7.47 & 6.95 & 2.98 \\
 & 0.42 & 0.32 & 0.48 & 0.99 \\
m15/n5 & 7.45 & 7.54 & 6.94 & 2.54 \\
 & 0.45 & 0.17 & 0.56 & 1.00 \\
\bottomrule
\end{tabular}
\end{table}

\subsection{MD17}

\autoref{tab:md17_detailed} shows in detail the performance of the different models on the MD17 dataset. 

\begin{table*}
\caption{Huber NLL for the MD17 dataset}
\label{tab:md17_detailed}
\begin{tabularx}{\textwidth}{
@{} Y{2.0} Y{0.9} Y{0.9} Y{0.9} Y{0.9} 
Y{0.9} Y{0.9} Y{0.9} Y{0.9} Y{0.9} Y{0.9} 
 Y{0.9}  Y{0.9}   Y{0.9}   Y{0.9}   Y{0.9}   Y{0.9} 
@{}
}
\toprule
Dataset (MD17) & aspirin &  & benzene2017 &  & ethanol &  & malonaldehyde &  & naphthalene &  & salicylic &  & toluene &  & uracil &  \\
\midrule
$O(n)$ KAN (F,F) & 6.77 & 0.16 & 8.02 & 0.09 & 7.94 & 0.04 & 7.84 & 0.03 & 7.42 & 0.04 & 7.54 & 0.14 & 7.60 & 0.21 & 8.03 & 0.13 \\
$O(n)$ KAN (F,T) & 6.08 & 0.01 & 7.29 & 0.03 & 7.14 & 0.03 & 7.12 & 0.04 & 6.29 & 0.08 & 6.41 & 0.03 & 6.50 & 0.06 & 7.08 & 0.00 \\
$O(n)$ KAN (T,F) & 6.83 & 0.20 & 8.06 & 0.15 & 8.06 & 0.04 & 7.90 & 0.07 & 7.39 & 0.14 & 7.53 & 0.13 & 7.54 & 0.18 & 8.04 & 0.11 \\
$O(n)$ KAN (T,T) & 6.09 & 0.04 & 7.27 & 0.06 & 7.13 & 0.03 & 7.16 & 0.07 & 6.30 & 0.03 & 6.36 & 0.05 & 6.54 & 0.06 & 7.01 & 0.09 \\
$O(n)$ MLP (F,F) & 5.63 & 0.00 & 5.98 & 0.02 & 5.47 & 0.01 & 5.40 & 0.01 & 5.37 & 0.00 & 5.65 & 0.00 & 5.69 & 0.02 & 5.70 & 0.01 \\
$O(n)$ MLP (F,T) & 5.63 & 0.01 & 5.91 & 0.00 & 5.46 & 0.03 & 5.42 & 0.02 & 5.34 & 0.00 & 5.62 & 0.01 & 5.71 & 0.00 & 5.61 & 0.00 \\
$O(n)$ MLP (T,F) & 5.61 & 0.01 & 5.93 & 0.02 & 5.41 & 0.01 & 5.37 & 0.01 & 5.36 & 0.01 & 5.61 & 0.00 & 5.64 & 0.04 & 5.69 & 0.01 \\
$O(n)$ MLP (T,T) & 5.61 & 0.00 & 5.90 & 0.00 & 5.43 & 0.01 & 5.38 & 0.01 & 5.33 & 0.00 & 5.61 & 0.01 & 5.68 & 0.01 & 5.61 & 0.00 \\
$\pi~O(n)$ KAN (F,F) & 5.68 & 0.02 & 6.73 & 0.18 & 5.95 & 0.18 & 5.83 & 0.04 & 5.82 & 0.10 & 5.81 & 0.14 & 6.11 & 0.11 & 6.21 & 0.11 \\
$\pi~O(n)$ KAN (F,T) & 5.69 & 0.02 & 6.27 & 0.10 & 6.24 & 0.13 & 5.91 & 0.01 & 5.65 & 0.06 & 5.82 & 0.00 & 5.96 & 0.12 & 5.94 & 0.14 \\
$\pi~O(n)$ KAN (T,F) & 5.69 & 0.02 & 6.69 & 0.19 & 6.01 & 0.07 & 5.80 & 0.03 & 5.84 & 0.10 & 5.90 & 0.16 & 6.06 & 0.11 & 6.32 & 0.05 \\
$\pi~O(n)$ KAN (T,T) & 5.69 & 0.02 & 6.34 & 0.20 & 6.15 & 0.16 & 5.87 & 0.07 & 5.59 & 0.11 & 5.80 & 0.10 & 5.98 & 0.05 & 5.93 & 0.16 \\
$\pi~O(n)$ MLP (F,F) & 4.28 & 0.39 & 5.73 & 0.10 & 5.55 & 0.08 & 5.41 & 0.05 & 5.07 & 0.16 & 5.27 & 0.03 & 5.41 & 0.06 & 5.58 & 0.07 \\
$\pi~O(n)$ MLP (F,T) & 5.45 & 0.05 & 5.77 & 0.05 & 5.49 & 0.01 & 5.40 & 0.03 & 5.29 & 0.02 & 5.53 & 0.06 & 5.64 & 0.04 & 5.58 & 0.02 \\
$\pi~O(n)$ MLP (T,F) & 3.84 & 0.59 & 5.47 & 0.26 & 5.44 & 0.01 & 5.34 & 0.04 & 3.08 & 2.83 & 4.41 & 0.82 & 5.07 & 0.27 & 5.47 & 0.03 \\
$\pi~O(n)$ MLP (T,T) & 5.34 & 0.06 & 5.58 & 0.11 & 5.46 & 0.01 & 5.37 & 0.03 & 5.17 & 0.03 & 5.48 & 0.07 & 5.49 & 0.05 & 5.45 & 0.08 \\
\bottomrule
\end{tabularx}
\end{table*}

\subsection{MD22}
\autoref{tab:md22_detailed} shows in detail the performance of the different models on the MD22 dataset. 

\begin{table*}
\caption{Huber NLL for the MD22 dataset}
\label{tab:md22_detailed}
\begin{tabularx}{\textwidth}{
@{} Y{1.5} Y{0.9} Y{0.9} Y{0.9} Y{0.9} 
Y{0.9} Y{0.9} Y{0.9} Y{0.9} Y{0.9} Y{0.9} 
 Y{0.9}  Y{0.9} 
@{}
}
\toprule
Dataset (MD22) & AT-AT-CG-CG &  & AT-AT & & Ac-Ala3-NHMe &  & DHA & & buckyball-catcher &  & stachyose &  \\
\midrule
$O(n)$ KAN (F,F) & NaN & NaN & 7.42 & 0.30 & 5.95 & 0.07 & 5.71 & 0.10 & NaN & NaN & NaN & NaN \\
$O(n)$ KAN (F,T) & 7.94 & 0.19 & 7.27 & 0.18 & 5.65 & 0.03 & 5.59 & 0.01 & 8.92 & 0.20 & 6.24 & 0.11 \\
$O(n)$ KAN (T,F) & NaN & NaN & 7.20 & 0.33 & 5.85 & 0.13 & 5.67 & 0.14 & NaN & NaN & NaN & NaN \\
$O(n)$ KAN (T,T) & 8.10 & 0.09 & 7.38 & 0.05 & 5.64 & 0.04 & 5.56 & 0.01 & 8.77 & 0.28 & 6.36 & 0.13 \\
$O(n)$ MLP (F,F) & 7.66 & 0.07 & 6.57 & 0.01 & 5.58 & 0.00 & 5.53 & 0.00 & 7.30 & 0.00 & 5.74 & 0.00 \\
$O(n)$ MLP (F,T) & 7.64 & 0.03 & 6.57 & 0.01 & 5.58 & 0.00 & 5.51 & 0.00 & 7.27 & 0.00 & 5.70 & 0.01 \\
$O(n)$ MLP (T,F) & 7.57 & 0.04 & 6.56 & 0.01 & 5.57 & 0.01 & 5.53 & 0.00 & 7.25 & 0.02 & 5.70 & 0.02 \\
$O(n)$ MLP (T,T) & 7.59 & 0.05 & 6.55 & 0.03 & 5.56 & 0.00 & 5.51 & 0.00 & 7.27 & 0.01 & 5.67 & 0.00 \\
$\pi~O(n)$ KAN (F,F) & NaN & NaN & 6.60 & 0.02 & 5.58 & 0.01 & 5.49 & 0.00 & NaN & NaN & NaN & NaN \\
$\pi~O(n)$ KAN (F,T) & 7.71 & 0.04 & 6.61 & 0.07 & 5.56 & 0.00 & 5.51 & 0.01 & 7.43 & 0.08 & 5.70 & 0.02 \\
$\pi~O(n)$ KAN (T,F) & 7.75 & NaN & 6.61 & 0.02 & 5.57 & 0.00 & 5.50 & 0.01 & NaN & NaN & 5.77 & NaN \\
$\pi~O(n)$ KAN (T,T) & 7.74 & 0.07 & 6.64 & 0.03 & 5.56 & 0.01 & 5.52 & 0.01 & 7.39 & 0.06 & 5.73 & 0.03 \\
$\pi~O(n)$ MLP (F,F) & NaN & NaN & NaN & NaN & -0.04 & 1.53 & -0.59 & 0.11 & NaN & NaN & NaN & NaN \\
$\pi~O(n)$ MLP (F,T) & 1.25 & 0.36 & 2.10 & 0.36 & 3.76 & 0.20 & 2.09 & 0.35 & 0.61 & 1.21 & 1.27 & 1.82 \\
$\pi~O(n)$ MLP (T,F) & NaN & NaN & -1.29 & 0.22 & 0.23 & 0.26 & -2.68 & 1.40 & NaN & NaN & NaN & NaN \\
$\pi~O(n)$ MLP (T,T) & 0.39 & 0.28 & 1.64 & 0.62 & 1.99 & 2.35 & 1.35 & 1.43 & -0.18 & 0.20 & 1.46 & 1.01 \\
\bottomrule
\end{tabularx}
\end{table*}

\section{Model parameters}

\begin{table}[]
    \centering
    \begin{tabular}{c|c| p{5cm}}
        \toprule
        \textbf{Parameter} & \textbf{Value} & \textbf{Comment} \\
        \midrule
        Number of epochs & 500  & We use 500 for the MD17 and MD22, while 1000 for the LJ experiments\\
        batch size & 4092 & \\
        loss & Huber & We selected Huber, compared to MSE, since it enables better training \\
        em lr & 0.01,  & learning rate for energy minimization for LJ experiments \\
        em niters  & 500 & number of steps for energy minimization for LJ experiments \\
        learning rate & 0.001 & we experimented with multiple rate and fix this for all experiments \\
        num samples  & 10000 & We fix the number of samples, if the dataset contains more data, we first permute the data (same for all experiments) and select the first 10000 samples. \\
        trsamples & 8000 & we split 80/20 training and testing \\
        optimizer & AdamW & \\
        weight decay & $1e-9$ & Weight decay is used to stabilize the training\\
        scheduler & ReduceLROnPlateau & The scheduler helps with different system requirement\\
        KAN layers & [input dim, 16, 16, 1] & the architecture size has been selected in the hyper-parameter search\\
        KAN orders &  [8,8,8] & This is the number of basis per function\\
        KAN Basis & ReLU& While KAN networks use Spline as basis, we experimented with ReLU, GeLU, Sigmoid, and Chebichev Polynomial, ReLU provided the most reliable solution\\
        MLP layers & [input dim, 128, 128, 1]  & the architecture size has been selected in the hyper-parameter search\\
         \bottomrule
    \end{tabular}
    \caption{Hyper-parameters used during training}
    \label{tab:hyper-params}
\end{table}

\subsection{Hyper-parameters and Hyper-parameter search}
\autoref{tab:hyper-params} show the hyper-parameters used during training for the MLP and KAN-based architectures. We implemented a separate hyper-parameter search on both MLP and KAN architecture based on the synthetic dataset, we tested the different sizes of architecture: small (128/16), medium (256/32), and large (512/64); and selected the small for both systems. 

While KAN networks use Spline as the basis, we experimented with ReLU, GeLU, Sigmoid, and Chebichev Polynomial, ReLU provided the most reliable solution across test cases. 

\subsection{LJ}
\autoref{tab:num_param_lj43} shows the number of parameters per model for the LJ experiments with $m=4$ and $n=3$. The impact of the presentation is already visible. KAN is always smaller. 
\autoref{tab:num_param_lj153} and \autoref{tab:num_param_lj155} show the network size for $m=15$ and $n=3,5$. As the input increases the KAN has more parameters than the equivalent MLP.

\begin{table}
\caption{Network sizes during the $4/3$ experiments}
\centering
\label{tab:num_param_lj43}
\begin{tabular}{lllr}
\toprule
system & model & options & size \\
\midrule
m4/n3 & $O(n)$ KAN & FF & 9911 \\
m4/n3 & $O(n)$ KAN & FT & 9911 \\
m4/n3 & $O(n)$ KAN & TF & 12044 \\
m4/n3 & $O(n)$ KAN & TT & 12044 \\
m4/n3 & $O(n)$ MLP & FF & 22145 \\
m4/n3 & $O(n)$ MLP & FT & 22145 \\
m4/n3 & $O(n)$ MLP & TF & 23681 \\
m4/n3 & $O(n)$ MLP & TT & 23681 \\
m4/n3 & $\pi O(n)$ KAN & FF & 4167 \\
m4/n3 & $\pi O(n)$ KAN & FT & 4167 \\
m4/n3 & $\pi O(n)$ KAN & TF & 4475 \\
m4/n3 & $\pi O(n)$ KAN & TT & 4475 \\
m4/n3 & $\pi O(n)$ MLP & FF & 17665 \\
m4/n3 & $\pi O(n)$ MLP & FT & 17665 \\
m4/n3 & $\pi O(n)$ MLP & TF & 17921 \\
m4/n3 & $\pi O(n)$ MLP & TT & 17921 \\
\bottomrule
\end{tabular}
\end{table}

\begin{table}
\caption{Network sizes during the $15/3$ experiments}
\label{tab:num_param_lj153}
\centering
\begin{tabular}{lllr}
\toprule
system & model & options & size \\
\midrule
m15/n3 & $O(n)$ KAN & FF & 250887 \\
m15/n3 & $O(n)$ KAN & FT & 63691 \\
m15/n3 & $O(n)$ KAN & TF & 371803 \\
m15/n3 & $O(n)$ KAN & TT & 87687 \\
m15/n3 & $O(n)$ MLP & FF & 110849 \\
m15/n3 & $O(n)$ MLP & FT & 51713 \\
m15/n3 & $O(n)$ MLP & TF & 137729 \\
m15/n3 & $O(n)$ MLP & TT & 61697 \\
m15/n3 & $\pi O(n)$ KAN & FF & 4167 \\
m15/n3 & $\pi O(n)$ KAN & FT & 4167 \\
m15/n3 & $\pi O(n)$ KAN & TF & 4475 \\
m15/n3 & $\pi O(n)$ KAN & TT & 4475 \\
m15/n3 & $\pi O(n)$ MLP & FF & 17665 \\
m15/n3 & $\pi O(n)$ MLP & FT & 17665 \\
m15/n3 & $\pi O(n)$ MLP & TF & 17921 \\
m15/n3 & $\pi O(n)$ MLP & TT & 17921 \\
\bottomrule
\end{tabular}
\end{table}

\begin{table}
\caption{Network sizes during the $15/5$ experiments}
\label{tab:num_param_lj155}
\centering
\begin{tabular}{lllr}
\toprule
system & model & options & size \\
\midrule
m15/n5 & $O(n)$ KAN & FF & 250887 \\
m15/n5 & $O(n)$ KAN & FT & 111906 \\
m15/n5 & $O(n)$ KAN & TF & 371803 \\
m15/n5 & $O(n)$ KAN & TT & 159216 \\
m15/n5 & $O(n)$ MLP & FF & 110849 \\
m15/n5 & $O(n)$ MLP & FT & 70529 \\
m15/n5 & $O(n)$ MLP & TF & 137729 \\
m15/n5 & $O(n)$ MLP & TT & 85889 \\
m15/n5 & $\pi O(n)$ KAN & FF & 4167 \\
m15/n5 & $\pi O(n)$ KAN & FT & 4167 \\
m15/n5 & $\pi O(n)$ KAN & TF & 4475 \\
m15/n5 & $\pi O(n)$ KAN & TT & 4475 \\
m15/n5 & $\pi O(n)$ MLP & FF & 17665 \\
m15/n5 & $\pi O(n)$ MLP & FT & 17665 \\
m15/n5 & $\pi O(n)$ MLP & TF & 17921 \\
m15/n5 & $\pi O(n)$ MLP & TT & 17921 \\
\bottomrule
\end{tabular}
\end{table}

\subsection{MD17}

\autoref{tab:num_param_md17} shows the number of parameters for the models used in the experiments. The permutation invariant version reduces the need for parameters considerably.

\begin{table}
\caption{Network sizes during the aspirin experiments}
\label{tab:num_param_md17}
\centering
\begin{tabular}{lllr}
\toprule
dataset & model & options & size \\
\midrule
aspirin & $O(n)$ KAN & FF & 1186625 \\
aspirin & $O(n)$ KAN & FT & 147811 \\
aspirin & $O(n)$ KAN & TF & 1692200 \\
aspirin & $O(n)$ KAN & TT & 197535 \\
aspirin & $O(n)$ MLP & FF & 258689 \\
aspirin & $O(n)$ MLP & FT & 82433 \\
aspirin & $O(n)$ MLP & TF & 312449 \\
aspirin & $O(n)$ MLP & TT & 97025 \\
aspirin & $\pi O(n)$ KAN & FF & 4475 \\
aspirin & $\pi O(n)$ KAN & FT & 4475 \\
aspirin & $\pi O(n)$ KAN & TF & 4783 \\
aspirin & $\pi O(n)$ KAN & TT & 4783 \\
aspirin & $\pi O(n)$ MLP & FF & 17921 \\
aspirin & $\pi O(n)$ MLP & FT & 17921 \\
aspirin & $\pi O(n)$ MLP & TF & 18177 \\
aspirin & $\pi O(n)$ MLP & TT & 18177 \\
\bottomrule
\end{tabular}
\end{table}

\subsection{MD22}

As for the MD17 dataset, also for MD22, \autoref{tab:num_param_md22} shows the number of parameters for the models used in the experiments. The permutation invariant version reduces the need for parameters considerably.
\begin{table}
\caption{Network sizes during the AT-AT-CG-CG experiments}
\label{tab:num_param_md22}
\centering
\begin{tabular}{lllr}
\toprule
dataset & model & options & size \\
\midrule
AT-AT-CG-CG & $O(n)$ KAN & FF & 974480535 \\
AT-AT-CG-CG & $O(n)$ KAN & FT & 2938488 \\
AT-AT-CG-CG & $O(n)$ KAN & TF & 1453151821 \\
AT-AT-CG-CG & $O(n)$ KAN & TT & 4256886 \\
AT-AT-CG-CG & $O(n)$ MLP & FF & 7969025 \\
AT-AT-CG-CG & $O(n)$ MLP & FT & 417665 \\
AT-AT-CG-CG & $O(n)$ MLP & TF & 9736193 \\
AT-AT-CG-CG & $O(n)$ MLP & TT & 506753 \\
AT-AT-CG-CG & $\pi O(n)$ KAN & FT & 4475 \\
AT-AT-CG-CG & $\pi O(n)$ KAN & TF & 4783 \\
AT-AT-CG-CG & $\pi O(n)$ KAN & TT & 4783 \\
AT-AT-CG-CG & $\pi O(n)$ MLP & FF & 17921 \\
AT-AT-CG-CG & $\pi O(n)$ MLP & FT & 17921 \\
AT-AT-CG-CG & $\pi O(n)$ MLP & TF & 18177 \\
AT-AT-CG-CG & $\pi O(n)$ MLP & TT & 18177 \\
\bottomrule
\end{tabular}
\end{table}


\end{document}

