\documentclass[sigconf]{acmart}

\AtBeginDocument{%
  \providecommand\BibTeX{{%
    Bib\TeX}}}

\copyrightyear{2025}
\acmYear{2025}
\setcopyright{acmlicensed}\acmConference[CHI '25]{CHI Conference on Human Factors in Computing Systems}{April 26-May 1, 2025}{Yokohama, Japan}
\acmBooktitle{CHI Conference on Human Factors in Computing Systems (CHI '25), April 26-May 1, 2025, Yokohama, Japan}
\acmDOI{10.1145/3706598.3713881}
\acmISBN{979-8-4007-1394-1/25/04}

\usepackage{enumitem}
\usepackage{url}
\usepackage{pifont} 
\usepackage{tabularx}
\usepackage{multirow}
\usepackage{enumitem}
\usepackage{amsmath}
\usepackage{mathtools}
\usepackage{makecell}



\newcommand{\mathTextItalics}[1]{\text{\textit{#1}}}

\begin{document}

\title[GistVis: Automatic Generation of Word-scale Visualizations from Data-rich Documents]{GistVis: Automatic Generation of Word-scale Visualizations from Data-rich Documents}
\author{Ruishi Zou}
\authornote{Both authors contributed equally to this research.}
\affiliation{%
  \institution{Tongji University}
  \city{Shanghai}
  \country{China}}
\email{zouruishi@tongji.edu.cn}

\author{Yinqi Tang}
\authornotemark[1]
\affiliation{%
  \institution{Tongji University}
  \city{Shanghai}
  \country{China}}
\email{2054136@tongji.edu.cn}

\author{Jingzhu Chen}
\affiliation{%
  \institution{Tongji University}
  \city{Shanghai}
  \country{China}}
\email{2253543@tongji.edu.cn}

\author{Siyu Lu}
\affiliation{%
  \institution{Tongji University}
  \city{Shanghai}
  \country{China}}
\email{2250898@tongji.edu.cn}

\author{Yan Lu}
\affiliation{%
  \institution{Tongji University}
  \city{Shanghai}
  \country{China}}
\email{2253887@tongji.edu.cn}

\author{Yingfan Yang}
\affiliation{%
  \institution{Tongji University}
  \city{Shanghai}
  \country{China}}
\email{yangyingfan@tongji.edu.cn}

\author{Chen Ye}
\authornote{Chen Ye is the corresponding author.}
\affiliation{%
  \institution{Tongji University}
  \city{Shanghai}
  \country{China}}
\email{yechen@tongji.edu.cn}


\renewcommand{\shortauthors}{Zou et al.}

\begin{abstract}
Data-rich documents are ubiquitous in various applications, yet they often rely solely on textual descriptions to convey data insights. Prior research primarily focused on providing visualization-centric augmentation to data-rich documents. However, few have explored using automatically generated word-scale visualizations to enhance the document-centric reading process. As an exploratory step, we propose GistVis, an automatic pipeline that extracts and visualizes data insight from text descriptions. GistVis decomposes the generation process into four modules: Discoverer, Annotator, Extractor, and Visualizer, with the first three modules utilizing the capabilities of large language models and the fourth using visualization design knowledge. 
Technical evaluation including a comparative study on Discoverer and an ablation study on Annotator reveals decent performance of GistVis. Meanwhile, the user study (N=12) showed that GistVis could generate satisfactory word-scale visualizations, indicating its effectiveness in facilitating users' understanding of data-rich documents (+5.6\% accuracy) while significantly reducing their mental demand (p=0.016) and perceived effort (p=0.033).
\end{abstract}

\begin{CCSXML}
<ccs2012>
   <concept>
       <concept_id>10003120.10003145.10003147.10010923</concept_id>
       <concept_desc>Human-centered computing~Information visualization</concept_desc>
       <concept_significance>500</concept_significance>
       </concept>
   <concept>
       <concept_id>10003120.10003145.10003151</concept_id>
       <concept_desc>Human-centered computing~Visualization systems and tools</concept_desc>
       <concept_significance>500</concept_significance>
       </concept>
 </ccs2012>
\end{CCSXML}

\ccsdesc[500]{Human-centered computing~Information visualization}
\ccsdesc[500]{Human-centered computing~Visualization systems and tools}

\keywords{Word-scale visualization, Automatic visualization, Natural language processing, Interactive article, Data document}

\begin{teaserfigure}
    \centering
    \includegraphics[width=1\textwidth]{figures/teaser.pdf}
    \caption{GistVis enables automatic generation of word-scale visualizations from data-rich documents. GistVis supports visualization generation for six data insight types: comparison \ding{182}, proportion \ding{183}, trend, rank, extreme, and value. GistVis also connects the word-scale visualizations and related entities in text snippets, allowing users to explore data-rich documents interactively.}
    \label{fig:teaser}
\end{teaserfigure}


\maketitle

\section{Introduction}

Deep Reinforcement Learning (DRL) has emerged as a transformative paradigm for solving complex sequential decision-making problems. By enabling autonomous agents to interact with an environment, receive feedback in the form of rewards, and iteratively refine their policies, DRL has demonstrated remarkable success across a diverse range of domains including games (\eg Atari~\citep{mnih2013playing,kaiser2020model}, Go~\citep{silver2018general,silver2017mastering}, and StarCraft II~\citep{vinyals2019grandmaster,vinyals2017starcraft}), robotics~\citep{kalashnikov2018scalable}, communication networks~\citep{feriani2021single}, and finance~\citep{liu2024dynamic}. These successes underscore DRL's capability to surpass traditional rule-based systems, particularly in high-dimensional and dynamically evolving environments.

Despite these advances, a fundamental challenge remains: DRL agents typically rely on deep neural networks, which operate as black-box models, obscuring the rationale behind their decision-making processes. This opacity poses significant barriers to adoption in safety-critical and high-stakes applications, where interpretability is crucial for trust, compliance, and debugging. The lack of transparency in DRL can lead to unreliable decision-making, rendering it unsuitable for domains where explainability is a prerequisite, such as healthcare, autonomous driving, and financial risk assessment.

To address these concerns, the field of Explainable Deep Reinforcement Learning (XRL) has emerged, aiming to develop techniques that enhance the interpretability of DRL policies. XRL seeks to provide insights into an agent’s decision-making process, enabling researchers, practitioners, and end-users to understand, validate, and refine learned policies. By facilitating greater transparency, XRL contributes to the development of safer, more robust, and ethically aligned AI systems.

Furthermore, the increasing integration of Reinforcement Learning (RL) with Large Language Models (LLMs) has placed RL at the forefront of natural language processing (NLP) advancements. Methods such as Reinforcement Learning from Human Feedback (RLHF)~\citep{bai2022training,ouyang2022training} have become essential for aligning LLM outputs with human preferences and ethical guidelines. By treating language generation as a sequential decision-making process, RL-based fine-tuning enables LLMs to optimize for attributes such as factual accuracy, coherence, and user satisfaction, surpassing conventional supervised learning techniques. However, the application of RL in LLM alignment further amplifies the explainability challenge, as the complex interactions between RL updates and neural representations remain poorly understood.

This survey provides a systematic review of explainability methods in DRL, with a particular focus on their integration with LLMs and human-in-the-loop systems. We first introduce fundamental RL concepts and highlight key advances in DRL. We then categorize and analyze existing explanation techniques, encompassing feature-level, state-level, dataset-level, and model-level approaches. Additionally, we discuss methods for evaluating XRL techniques, considering both qualitative and quantitative assessment criteria. Finally, we explore real-world applications of XRL, including policy refinement, adversarial attack mitigation, and emerging challenges in ensuring interpretability in modern AI systems. Through this survey, we aim to provide a comprehensive perspective on the current state of XRL and outline future research directions to advance the development of interpretable and trustworthy DRL models.
\section{Related Work}
We review three categories of previous works pertinent to our research question of designing an automatic method to generate word-scale visualizations for document-centric analysis. Specifically, we review 1) the definition, design, and application of word-scale visualizations, 2) related automatic visualization generation methods, and 3) how visualizations and texts could be integrated to enrich the reading experience.

\subsection{Word-scale Visualization}
Word-scale visualization has been discussed in the literature under several expressions, such as sparklines~\cite{tufte2006beautiful} and word-scale visualization~\cite{goffin2014exploring}.
\citet{tufte2006beautiful} introduced sparklines as concise, high-impact graphics that fit a text's typographic space. \citet{goffin2014exploring, goffin2017Exploratory} expanded Sparklines with word-sized visualizations, allowing more flexible integration of graphics and text.
Though terms may differ, one common theme between sparklines and word-scale visualizations is using text-sized graphics to augment existing documents. This study uses word-scale visualization to cover word-scale data visualization, text-sized typography, and text-sized graphics. 

In addition to discussing the word-scale visualization design space, researchers have also explored the potential of the application of word-scale visualizations~\cite{hoffswell2018Augmenting, perin2013soccerstories, brandes2013gestaltlines, beck2016visual}. For example, \citet{hoffswell2018Augmenting} implemented a design space that uses word-scale visualizations to augment the code reading experience. The SportLines interface~\cite{perin2013soccerstories} used word-scale visualizations to show the phases of players to support visual soccer analysis. These studies are formative to GistVis, as we employ the best practices integrating word-scale visualizations with text. Although prior work has explored many application scenarios, few have automated word-scale visualization generation for data-rich documents. This work contributes to the research of word-scale visualizations by proposing an automatic pipeline that generates word-scale visualizations to augment existing data-rich documents.

\vspace*{-2pt}
\subsection{Automatic Visualization Generation}
Research has explored various automatic visualization techniques, especially using tabular data as input~\cite{wang2020datashot, shi2021calliope, dibia2023lida}. For example, \citet{wang2020datashot} proposed the Datashot system that included a fact sheet generation pipeline based on a taxonomy of data facts~\cite{chen2009effective}. %
More recently, \citet{dibia2023lida} proposed the LIDA (Visua\underline{li}zation an\underline{d} Infogr\underline{a}phics) tool that incorporated large language models (LLMs) in the infographics generation process from tabular datasets. 


A subcategory of automatic visualization revolves around generating visualizations based on textual contexts~\cite{wu2022ai4vis}. Under this category, we can broadly identify two ways of automatic visualization generation:
1) using external data sources and 2) using internal data sources (text description, tables, etc.). Using external data sources, NewsViews~\cite{gao2014newsviews} generated interactive visual maps through a table database. Contextifier~\cite{hullman2013contextifier} automatically created a stock timeline graph with annotations through sourcing external news corpus and APIs. Using internal data sources, Text-to-Viz~\cite{cui2020texttoviz} used regular expressions to detect data elements for generating proportional infographics. Charagraph~\cite{masson2023Charagraph} created in situ visualizations from data-rich paragraphs to support exploratory analysis of statistical data described in the text. However, those methods primarily used rule-based approaches to extract data insights from data-rich documents, and we need further work to capture semantically implied data insights in text descriptions.

GistVis categorizes as a context-based automatic visualization technique and uses data solely from internal sources. GistVis extends previous automatic visualization research by utilizing LLMs as a proxy to capture and extract data insights from text descriptions.
We employed a data fact-based approach~\cite{chen2009effective} inspired by the automatic methods mentioned above and engineered LLMs to use such visualization knowledge to generate word-scale visualizations.

\vspace*{-2pt}
\subsection{Visualization + Text}
\label{subsec:relatedwork-vistext}
Research has demonstrated the effectiveness of dynamically connecting text segments to visualizations to enhance the reading experience~\cite{latif2022kori, bromley2024dash, masson2023Charagraph}. Prior work has proposed crowdsourcing or mixed-initiative methods to create text-visualization connections. For example, \citet{kong2014extracting} used crowdsourcing to extract references between charts and text. For mixed-initiative methods, \citet{latif2022kori} proposed the Kori system, an interactive authoring tool synthesizing text and visualization. The DASH (\underline{D}ata \underline{A}nalysis using \underline{S}emantic \underline{H}ierarchies) system~\cite{bromley2024dash} leveraged LLMs to introduce semantic levels~\cite{lundgard2022accessible} into the bidirectional analysis between text and charts. The Charagraph system~\cite{masson2023Charagraph} used regular expressions to extract statistical data from a user-selected text domain to facilitate the understanding of statistical data in data-rich documents.

GistVis incorporates the practice of linking text with visualization by supporting two types of text-visualization interaction: 1) linking entities in text descriptions to visual elements in word-scale visualizations and 2) a hover tooltip that provides essential data insights of the selected word-scale visualization. Meanwhile, GistVis extends prior research by implementing text-visualization interaction on a word-scale setting. We expect the application of text-word scale visualization interaction could provide additional data context for users while reading. 

\section{Survey on Data-rich Documents}
\label{sec:formative-study}

To better motivate the design of GistVis, we conducted a formative study based on a corpus of data-rich documents to understand the narrative features of data-rich documents. In this section, we first describe how we collected the data-rich document corpus. Then, we describe how we derived the codes we used for our deductive coding process based on prior visualization literature. Lastly, we present the findings of our study and outline several constraints we took in the design of the GistVis automatic generation pipeline.

\subsection{Corpus Collection}
Data-rich documents are widely adopted in various domains; thus, many source documents are available online. Meanwhile, data-rich documents can also come in various genres, for instance, data journalism~\cite{stalph2018classifying} and scientific articles~\cite{beck2017wordsized}. Since it is more common for the general public to access data journalism, we included only data journalism in our corpus. Specifically, data journalism refers to news reports that contain rich data content (e.g., ~\cite{data-news-example, us-k12-education}).

Three researchers, who were briefed on the definition of data journalism and familiar with data visualization, applied a snowballing approach to collect data-rich news articles to construct the corpus. During the collection process, we attempted to cover a wide variety of topics while we labeled the topic for each data article referencing Stalph's topical classification codes~\cite{stalph2018classifying}. Ultimately, we collected 44 data-rich news articles that we labeled using six topical codes: \textit{Politics} (19.30\%), \textit{Society} (19.30\%), \textit{Business} (14.04\%), \textit{Culture} (15.79 \%), \textit{Local} (8.77\%) and \textit{Other} (22.81\%). All the articles were written in English.
We documented the source link of the documents in the supplementary material.

\subsection{Qualitative Analysis}
The first step in designing an automatic method for generating word-scale in situ visualizations is ascertaining the types of insights conveyed through text description. To obtain the candidate insight types, we coded the whole corpus deductively to 1) delimit the segment that contains data narrations and 2) code the segment with one or multiple insight types. Specifically, we coded the insight types referring to the facts taxonomy~\cite{chen2009effective}, which provided formal definitions that distinguished different types of data insights (e.g., value, distribution, difference, etc.). Recent research has successfully applied this taxonomy in the design of several automatic visualization techniques~\cite{wang2020datashot, shi2021calliope, chen2024chart2vec}.

Although the fact taxonomy is designed to reflect the attribute of the data insights independently across datasets or applications, we excluded several fact types less likely to be captured from text descriptions or conveyed through word-scale visualizations to simplify the coding process. First, limited information in text descriptions prevents us from obtaining complete tabular datasets. Therefore, we excluded data fact types commonly derived from data transformation on complete datasets (e.g., aggregation, anomaly). Secondly, we considered the possible display of the data fact type and excluded fact types that might be hard to read in a word-scale setting. For example, types such as association are commonly represented with chart types like Treemaps and Sankey Diagrams. %
Treemaps or Sankey Diagrams require much more space than what word-scale visualization could offer.

Ultimately, we had eight data fact types: comparison, trend, rank, proportion, extreme, value, distribution, and categorical. Through our coding process, we observed that \textbf{data-rich documents could cover many data fact types, but several data fact types are more common than others (O1)}. The frequency of occurrence and definition of the eight fact types are as follows:
\begin{itemize}%[leftmargin=18pt, itemsep=0pt, parsep=0pt, partopsep=0pt, topsep=1pt]
    \item \textbf{Value} (33.23\%; 332): The value type is one or multiple numerical values in a sentence retrieved under some specific criterion. For example, ``\textit{Sources report that almost 10 million migrants have crossed into the country.}''
    \item \textbf{Trend} (17.34\%; 168): The trend type is a general tendency of one data attribute over time. For example, ``\textit{In the last quarter of 2023, EV sales were up 40\% from the same quarter a year before.}''
    \item \textbf{Comparison} (16.72\%; 162): The comparison type measures the difference in value between two or more entities over a shared breakdown. For example, ``\textit{EVs create 3,932 pounds of carbon per year, compared to 11,435 for gas-powered vehicles.}''
    \item \textbf{Proportion} (15.07\%; 146): The proportion type expresses how much one or multiple attributes comprise the sum. For example, ``\textit{Around 60\% of Mexico is experiencing moderate to exceptional drought.}''
    \item \textbf{Extreme} (8.46\%; 82): The extreme type is the maximum or minimum of one specific attribute. For example, ``\textit{The highest mountain in the world is Mount Chumolongma at 8848 meters.}''  
    \item \textbf{Distribution} (3.61\%; 35): The distribution type demonstrates the numerical values shared over a specific breakdown. For example, ``\textit{Monocrystalline silicon production increased by 31.6 percent, photovoltaic cell production increased by 45.6 percent, NEV increased by 46.3, and wind power generator increased by 66.4 percent.}''
    \item \textbf{Rank} (3.61\%; 35): The rank type demonstrates an order or a sorted sequence over a specific breakdown. For example, ``\textit{This figure means the province contributed the second-highest GDP in China in 2023 only following South China's Guangdong Province with 13.57 trillion yuan.}''  
    \item \textbf{Categorical} (1.96\%; 19): The categorical type is data attributes with a certain joint feature. For example, ``\textit{The employment gains were concentrated in only three sectors: health care, leisure and hospitality, and government.}''
\end{itemize}


While identifying the data fact types, we also delimited the text segment that conveys the fact types. The criterion for extracting the text segments is to find the shortest possible sentence collection that coherently conveys a data insight. In the following, we refer to this shortest text segment that conveys one data insight as the \textbf{unit segment}. We found the majority of (84.21\%, 512) unit segments contain only one sentence, while the maximum count of sentences in a unit segment is 6. This suggested that \textbf{the majority of data insights occur within one sentence, but some cases exist where data insights span multiple sentences (O2).} Meanwhile, we did not consider cross-paragraph unit segments.

Note that because we did not further break sentences into fragments, each unit segment could, in theory, contain multiple data fact types. In alignment with the fact taxonomy, we refer to this scenario as compound fact. We found 46.05\% (280) unit segments in our corpus fall within the compound fact. The above data revealed that \textbf{many unit segments contain only one data insight, yet a good portion do contain multiple insights (O3)}, which could be further divided or interpreted in various ways. However, considering our goal in facilitating document-centric reading, we chose not to further break down the original sentence structure to avoid an overdose of visualizations. Hence, we decided to keep unit segments as our smallest analysis unit.



\subsection{Design Implications}
\label{subsec:formative-constriants-implications}
The above analysis revealed a rather complicated design space to fulfill, and designing a comprehensive automatic method that generates word-scale visualizations based only on text descriptions is challenging.
Therefore, as an initial attempt, we decided to apply several constraints driven by our observations from the corpus study to provide a better definition of our tasks.
We applied the following constraints (C1 - C3) when designing our automatic generation method:
\begin{enumerate}%[label=\textbf{C\arabic*}, itemsep=0pt, parsep=0pt, partopsep=0pt, topsep=1pt]
    \item[\textbf{C1}] \textbf{We design our automatic generation method and visualization for more frequent data fact types (derived from O1).} Specifically, we support the data fact types of value, proportion, comparison, trend, rank, and extreme, the top six types in our corpus. We integrate distribution into the comparison category because they could share similar visual encoding. Additionally, we exclude categorical because of its low occurrence frequency.
    \item[\textbf{C2}] \textbf{We design our automatic generation method to be compatible with situations where a data insight can be extracted from the text in one unit segment (derived from O2).} This constraint ensures that we cover scenarios where data insights occur in consecutive sentences. Additionally, this constraint ensures that we can generate data-driven visualizations, which is the majority word-scale visualization type (79.5\%) created by human designers~\cite{goffin2017Exploratory}.
    \item[\textbf{C3}] \textbf{We design our automatic generation method to display only one data fact type (i.e., not concurrently process compound fact) (derived from O3).}  Although a moderate number of unit segments are compound facts, as a first step, we intend to evaluate the feasibility of generating one data fact type first before generalizing the method to support more complicated scenarios.
\end{enumerate}

On top of those constraints, we intended to leave the implementation flexible, which grants extensibility and enables further expansion and optimization. Considering all observations (O1 - O3), constraints (C1 - C3), and the extensibility objective, we bear three overarching design goals (DG1 - DG3):

\begin{enumerate}%[label=\textbf{DG\arabic*}, itemsep=0pt, parsep=0pt, partopsep=0pt, topsep=1pt]
    \item[\textbf{DG1}] \textbf{Establish a uniform data structure to encode both plain text and data insights.} The architecture should treat the data insights at the same level as plain text, ensuring that the word-scale visualization is generated from one unit segment from a document-centric perspective.%
    \item[\textbf{DG2}] \textbf{Apply modular design principle.} Provide abstraction to the automatic word-scale visualization generation process to support plug-and-play property for each processing module to optimize performance and extensibility.
    \item[\textbf{DG3}] \textbf{Design reusable and expressive word-scale visualization components that support interactive document-centric analysis.} The visualization design should be succinct and clear to express insights into various data fact types. They should also adhere to the space limitations of the word-scale setup while offering interaction between text and visualization to improve its clarity. Considering extensibility, the word-scale visualization components should be independent of data insights to support the easy integration of additional word-scale visualization designs.
\end{enumerate}

\section{GistVis}
\label{sec:gistvis-method}
The following sections describe the design of GistVis. We first define the uniform data representation schema \textit{data fact} (\textbf{DG1}, Sec.~\ref{sec:gistvis-gistfact}). Then, we provide a detailed description of the purpose of each module in the GistVis computational pipeline, how it could support plug-and-play property (\textbf{DG2}, Sec.~\ref{sec:computational-pipeline}), and the design choices of our word-scale visualizations (\textbf{DG3}). Lastly, we introduce the implementation details of the current iteration of GistVis (Sec.~\ref{sec:implementation-detail}).

\subsection{Data Fact}
\label{sec:gistvis-gistfact}

We define \textit{data fact} as the uniform data structure to encode all text content, either with or without data insights. Under this definition, we characterize \textit{data fact} as a declarative intermediate data structure that encapsulates all the key information required to generate word-scale visualizations. We formulate \textit{data fact} as a 2-tuple:
\begin{equation}
\mathTextItalics{fact}\, \coloneq \left \{ \mathTextItalics{unitSegmentSpec, dataSpec?} \right \} \nonumber
\end{equation}
where \textit{unitSegmentSpec} records key information related to the \textbf{unit segment}. Meanwhile, \textit{dataSpec} is a list that contains the data restored from the information provided by the unit segment. Because not all unit segments contain data insights, \textit{dataSpec} is an optional attribute in the tuple and could be used to distinguish between plain text segments and data insight segments. For clarity, we use the \textit{Typescript} notation (question mark) to represent optional elements. In the following, we explain the design of \textbf{unit segment specification} (\textit{unitSegmentSpec}) and \textbf{data specification} (\textit{dataSpec}).

\subsubsection{Unit Segment Specification}
\label{sec:gistvis-unitSegmentSpec}
We define unit segment specification as a 4-tuple:
\begin{equation}
    \mathTextItalics{unitSegmentSpec}\, \coloneq \left \{
        \mathTextItalics{type, context, attribute?, position?}
    \right \} \nonumber
\end{equation}

\paragraph{Type}
Type contains seven candidate types, including the six data insight types (i.e., value, proportion, comparison, trend, rank, and extreme) we selected from the facts taxonomy (see Sec.~\ref{sec:formative-study}), and one ``no type'' to represent plain text. 

\paragraph{Context}
Content is where we store the original text snippets that are unit segments. In this work, we define a unit segment as one or multiple sentences that collectively convey one data insight or are similar in semantics. We implement this definition to support situations where data insights span multiple adjacent sentences.

However, it is worth noting that we defined unit segments based on a strong assumption that relevant information of one data insight is described in sequential order, proximate in position, and contained within the same paragraph. Cases might exist where the same data insight could occur in numerous places across the document~\cite{goffin2020Interaction}. We argue that word-scale visualizations designed for document-centric analysis might not be the optimal solution to convey such information. Thus, we only focus on performing paragraph-level segmentation.

\smallskip
The above two entries (\textit{Type}, \textit{Context}) record contextual information directly related to the text and are ubiquitous for all unit segments. Meanwhile, \textit{Attribute} and \textit{Position} are data-related auxiliary information of the text and relevant to specific \textit{Types}. Referencing the four-level model of the semantic content of visualization proposed by \citet{lundgard2022accessible}, we viewed \textit{Attribute} and \textit{Position} to represent L2 - L3 (statistical concepts and relations, perceptual and cognitive phenomena) information and L1 (elemental and encoded properties) information respectively. Because \textit{Attribute} and \textit{Position} convey key semantic insight about the data and are closely related to text descriptions, we leave those attributes in the unit segment specification.

\paragraph{Attribute}
Attribute is an optional entry explicitly designed for data types extreme and trend. The candidate options for attributes include ``increasing'' and ``decreasing'' for the trend type and ``maximum'' and ``minimum'' for the extreme type. The attribute information is a supplementary constraint in the visualization generation process to correct potential errors in the generated visualizations and reflect the semantics of the text description.

\paragraph{Position}
The position is an optional entry to handle the extreme data fact type. It represents the original text description that should be highlighted to provide contextual information about the maximum or minimum values. For example, if we want to augment "\textit{the maximum of sales for company A,}" it is more informative to label the entire phrase rather than just "company A" (the standard highlight practice for other data fact types). If position is available in the extreme data fact type, the position attribute would override the default practice and ensure the entire phrase is highlighted.

\subsubsection{Data Specification}
\label{sec:gistvis-dataspec}
While Unit Segment Specification characterizes the textual content and higher-level data insights, data specification represents the raw data elements reconstructed from the textual content. Data Specification is designed to be an analogy of tabular datasets (Fig.~\ref{fig:dataspec-tableview}), which we define in a four-tuple:
\begin{equation}
    \mathTextItalics{dataSpec}\, \coloneq \left \{ 
        \mathTextItalics{space, breakdown, feature, value}
    \right \}[] \nonumber
\end{equation}

Specifically, \textbf{space} is a facet of analysis with a given text description. For example, if a sentence describes the market share of different car manufacturers, the analysis space would be ``car manufacture'' (Fig.~\ref{fig:dataspec-tableview} \ding{182}). Meanwhile, \textbf{breakdown} is a set of temporal or categorical data fields in which data are further divided under the space. For example, the brand name, like ``Brand A'' (Fig.~\ref{fig:dataspec-tableview} \ding{183}), would be the breakdown for ``car manufacture''. \textbf{Feature} is the measurement of breakdown. For example, we could measure the sales percentage for each manufacturer (Fig.~\ref{fig:dataspec-tableview} \ding{184}), a feature derived from annual sales of car manufacturers. Lastly, \textbf{value} is a numerical data field that could be retrieved from a combination of breakdown and feature. For example, the ``sales percentage'' of ``Brand A'' is 0.5 (Fig.~\ref{fig:dataspec-tableview} \ding{185}). All data attributes are required for each data specification entry, with the only exception being the ``not a number'' (NaN) value attributes. Cases exist when the unit segment describes a semantic data insight (e.g., increasing or decreasing for the trend type), and we make ``not a number cases'' a special condition for GistVis to process.

\begin{figure}[tb]
  \centering
  \includegraphics[width=0.8\linewidth]{figures/GistFact-Definition.pdf}
  \caption{Each element in data specification consists a four-tuple, space \ding{182}, breakdown \ding{183}, feature \ding{184} and value \ding{185}.}
  \label{fig:dataspec-tableview}
\end{figure}

In the following section, we describe how the fields in the data fact specification are filled using the GistVis pipeline.

\subsection{Computational Pipeline}
\label{sec:computational-pipeline}
Based on the data requirement of data fact, we proposed the GistVis pipeline to automatically transform data-rich text descriptions to word-scale visualizations. We utilized both LLM-based and design knowledge-driven approaches to achieve word-scale visualization generation. Specifically, we decomposed the generation process into four stages: gist discovery, fact type annotation, fact specification extraction, and fact visualization. We capitalized on LLMs' natural language understanding capability for the first three stages and applied the prompt chaining strategy~\cite{wu2022ai} to transform natural language descriptions into data facts. Meanwhile, the fact visualization stage uses a simple heuristic-driven approach to map data facts to interactive visualization components. As an automated process, all prompts and visualization heuristic rules are readily coded into the pipeline (see Supplementary Material for detailed prompt design). Fig.~\ref{fig:algorithm-pipeline} shows the GistVis computational pipeline, which consists of four modules respective to the four stages above: Discoverer (\textbf{M1}), Annotator (\textbf{M2}), Extractor (\textbf{M3}) and Visualizer \textbf{(M4)}.

\begin{figure*}
    \includegraphics[width=1\textwidth]{figures/GistVis-Pipeline.pdf}
    \caption{The GistVis pipeline consists of four modules: Discoverer (\textbf{M1}), Annotator (\textbf{M2}), Extractor (\textbf{M3}), and Visualizer (\textbf{M4}). Data flows through the four modules sequentially, where a large language model captures the insight of the data-rich document (\textbf{M1-M3}). Visualizer (\textbf{M4}) maps the captured insight into interactive visualizations, populated in situ in the text document at word scale.}
    \label{fig:algorithm-pipeline}
\end{figure*}

\subsubsection{Discoverer}
The first step in the GistVis pipeline is to divide paragraphs into unit segments. Discoverer (Fig.~\ref{fig:algorithm-pipeline}.~\textbf{M1}) leverages the zero-shot capability of LLMs~\cite{brown2020language} to perform the segmentation process. We restrain LLMs from identifying the shortest unit segment possible to better pair the text descriptions with in situ word scale visualizations. The prompt contains the six fact types to provide LLMs with more detailed segmentation requirements. Meanwhile, although LLMs are informed of the data fact types in the instructions, we do not label the data fact type at this stage to keep the task simple.
Moreover, the segmentation process shortens the context length, filtering out excess information for the subsequent data fact type labeling step. Additionally, to reduce hallucinations from LLMs during the segmentation process, we instruct the LLM to keep the text description ``as is'' without modifying the original text description or punctuation.

\subsubsection{Annotator}
The Annotator module (Fig.~\ref{fig:algorithm-pipeline}.~\textbf{M2}) aims to fill in the \textit{Type} field in the data fact specification for each segment from the prior module. Since LLMs are well calibrated to answer multiple choice and true/false questions~\cite{kadavath2022language}, we formulate the data fact type annotation as a two-stage question-answering (QA) problem. In the first stage, we ask LLMs to make a true/false judgment on whether the given segment belongs to a specific data fact type (\textbf{Type Checker}). Then, since one segment can be classified into multiple data fact types, we perform another round of prompting (\textbf{Type Moderator}) in a multiple-choice format to determine the most appropriate data fact type for each segment.

\paragraph{Type checker}
Type Checker identifies all possible fact types for each segment. To incorporate visualization knowledge into LLMs, we applied the few-shot in-context learning prompting paradigm~\cite{brown2020language}. We constructed seven individual reasoners for each of the seven fact types. Specifically, following a task description informing LLMs to return whether or not the segment can be classified as a given fact type, we define the fact type followed by three examples (two positive, one negative), allowing the LLMs to capture the definition of data fact type and forming a 3-shot prompt. After running through all seven reasoners, we record all the possible data fact types in a list for further moderation. If all reasoners responded false, the segment would be labeled a text-only segment (i.e., no type) and not proceed for further analysis.


\paragraph{Type moderator}
Taking the output from the Type Checker, the Type Moderator determines the most suitable data fact type for a given segment. We formulate the prompt in a multiple-choice pattern, with the options drawn from the previous type-checking step. Due to the varying number of possible candidates, we do not include examples and only reiterate the definition for the candidates. After the moderation step, we narrow the data fact label to only one label and fill the type field in the data fact specification in preparation for the data extraction phase.

We annotated data fact types in two instead of one round for the following reasons. First, we limit the input length and avoid excessive length prompts that could potentially harm inference capability~\cite{li2024longcontext}. Splitting the annotation task into two rounds would save space—we avoided all the samples for each data type occurring in the same prompt. Secondly, we consider the extensibility of the GistVis pipeline. Splitting the process into two stages would ensure that adding custom data fact types requires no more than writing additional prompts for the new data fact type. We also justified this design decision with an ablation study in our technical evaluation (Sec.~\ref{subsec:quant-eval-annotator}).

\subsubsection{Extractor}
Since each data fact type has different requirements, the Extractor module (Fig.~\ref{fig:algorithm-pipeline}.~\textbf{M3}) applies a case-by-case extraction strategy based on the data fact type generated from the Annotator module when extracting the data specifications. We formulate the prompts based on the visualization specification requirement described in Sec.~\ref{sec:gistvis-dataspec}.

Compared with the methods that applied regular expression to identify numbers in a text description, LLMs allow more flexibility in how the data is presented. For example, when numbers are not expressed numerically (e.g., 10 thousand, ten thousand), regular expressions would fail in extracting the correct underlying data. Instead, we prompt LLMs to convert non-numerical expressions of numbers to their numerical form (i.e., 10000), thus expanding data extraction capability. We also use a number parser to further transform all extracted data into a numerical form.


\subsubsection{Visualizer}

\begin{figure*}[tb]
  \centering
  \includegraphics[width=\linewidth]{figures/design-space.pdf}
  \caption{A collection of 14 candidate visualizations and the corresponding chart type for each data fact type. The \textbf{Example} column shows the effect of the appearance of word-scale visualization in data-rich documents. We present the examples when the mouse hovers over the word-scale visualization of focus.}
  \label{fig:visualization-design}
\end{figure*}

Visualizer (Fig.~\ref{fig:algorithm-pipeline}.~\textbf{M4}) applies a chart-based approach (as opposed to the grammar of graphics~\cite{wilkinson2012grammar} approach) in visualizing data facts. We took the chart-based approach based on the observation that various business visualization toolkits~\cite{g2plot-2024, echarts-2024} use this paradigm. 

\paragraph{Visualization Design}
We implement three basic chart types (bar chart, line chart, and icon array) and use their variants with relevant icons to tailor to the needs of different data fact types. The variants of the three chart types and icons constructed a design space including 14 candidate visualizations to represent different conditions of data insights (Fig.~\ref{fig:visualization-design}). Notably, we use horizontal stacked Bar Charts to represent proportion. Although prior works have applied Pie Charts (e.g.,~\cite{goffin2017Exploratory, huth2024eye}), we argue that the limited height could make discerning the angle difference between categories difficult. Thus, in this work, we attempt to capitalize on the relatively ample space on the horizontal dimension and use length, a more effective visual channel than angle~\cite{munzner2014visualization}, to encode proportional data. We also made several hard constraints to facilitate the readability of word-scale visualizations. For example, we limit the maximum rank for visualization to 10, avoiding situations of ultra-long vertical bar charts if the value of the ranking is large. We made this decision based on the fact that only one in 35 rank insights we identified in our corpus (Sec.~\ref{sec:formative-study}) included a rank above 10. Moreover, the definition of the rank fact type implied a sorting process on the dataset such that the rank is relative, leading to the inclination of reporting ranking with single-digit numbers. Meanwhile, to avoid situations such as rank overflow, we also designed a fallback condition, presenting a question mark icon to indicate the unit segments that might contain data insights not properly presented by GistVis. We expect that users could better perform analytical activities~\cite{amar2005lowlevel} with data-rich documents through the locally aggregated views of data in the form of word-scale visualizations.

\begin{table*}[tbp]
\small
  \centering
  \caption{The visualization knowledge the Visualizer (M4) module assumes when parsing data fact specification. C and T represent categorical and temporal data types, whereas N represents the numerical data type. The fallback condition will be hit if data is missing in the data fact specification.}
    \begin{tabularx}{\linewidth}{cccccX}
    \toprule
    \textbf{Fact Type} ($t$) & \textbf{Breakdown} ($b$) & \textbf{Feature} ($f$) & \textbf{Attribute} ($a$) & \textbf{Position} ($p$) & \textbf{Tooltip Syntax} \\
    \midrule
    Proportion & C/T   & N     & \ding{53}     & \ding{53}     & The proportion of $\{b_i\}$ is $\{v_i\}$. \\
    Value & C/T   & N     & \ding{53}     & $\geq$1 & The value of $\{b_i\}$ is $\{v_i\}$. \\
    Extreme & C/T   & N     & \makecell[c]{maximum/\\minimum} & $=$1  & The $\{a_i\}$ of $\{b_i\}$. \\
    Comparison & C/T   & N     & \ding{53}     & \ding{53}     & The difference between $\{b_i\}$ and $\{b_j\}$ is $\{| v_i - v_j|\}$. \\
    Rank  & C     & N     & \ding{53}     & \ding{53}     & Rank $\{v_i\}$: $\{b_i\}$ \\
    Trend & T     & N     & \makecell[c]{positive/\\negative} & \ding{53}     & \makecell[l]{$\{a_i\}$ \\ $\{f\}$ of $\{b_i\}$ is $\{v_i\}$. \\ The $\{a_i\}$ is $\{| v_i - v_j|\}$} \\
    \midrule
    (Fallback cond.)  & \ding{53}     & \ding{53}     & \ding{53}     & \ding{53}     & May contain data insight of $\{t\}$. \\
    \bottomrule
    \end{tabularx}
  \label{tab:gistvis-implementation}
\end{table*}


\paragraph{Linking Word-scale Visualizations with Text}
We also added interactive features to word-scale visualizations. The motivation behind making word-scale visualizations interactive is twofold: 1) to pack more information in word-scale visualizations, and 2) to enhance the reading experience by coupling visualization with text (Sec.~\ref{subsec:relatedwork-vistext}). 

Firstly, to pack more information within word-scale visualizations, we designed a drill-down interaction that pops up a tooltip every time users hover over the visualization. The tooltip contains basic data descriptions about the word-scale visualization, allowing users to grasp key information even when viewing the visualization standalone. We define a set of default syntax to describe the data insights for each data fact type according to the value types in the data fact specification (Table.~\ref{tab:gistvis-implementation}). Crucially, users access the tooltips on demand, minimizing their impact in obstructing a normal document reading process.


Secondly, to enhance the reading experience, we designed bidirectional interactions between text and word-scale visualizations. Specifically,
we highlight the ``entities'' of a sentence via matching document text with the set of extracted breakdowns in the data fact specification. When a specific visual element is selected, the corresponding entity synchronously lights up to show the correspondence between the entity and its value. Conversely, when a specific entity is selected, the corresponding visual element in the word-scale visualization would also light up to help users retrieve the entity related to the value (see Fig.~\ref{fig:visualization-design} Example column).

\subsection{Implementation}
\label{sec:implementation-detail}
We implement the GistVis pipeline using a typical web stack. For rendering the data-rich document augmented by GistVis, we use open-source libraries, including React\footnote{\url{https://react.dev/}} as the UI framework and D3.js~\cite{bostock2011datadriven} for rendering the word-scale visualizations in svg format. We took inspiration from prior jQuery-based word-scale visualization packages such as Sparklificator~\cite{sparklificator-package} and Piety~\cite{piety-package} and expanded their functionality using the React framework.

As for LLMs, we chose \texttt{DeepSeek-V2.5}~\cite{deepseek-ai2024deepseekv2}, an open-source\footnote{\url{https://huggingface.co/deepseek-ai/DeepSeek-V2.5}} Mixture-of-Experts (MoE) language model released by \texttt{DeepSeek}\footnote{\url{https://www.deepseek.com/en}}. We made this decision based on its decent performance at a low cost (overall cost less than 0.28\$ per 1M tokens). However, since we did not have the computation device to run such a large model, we resorted to the commercial API release of the exact DeepSeek model in our implementation. 
We then chained the processing steps with LangChain.js v0.1~\cite{langchainjs}, the JavaScript implementation of a popular framework for developing LLM-powered applications.

\section{Technical Evaluation}
\label{sec:technical-eval}
The effectiveness of GistVis depends on whether our computational pipeline can successfully extract key information from data-rich documents to create word-scale visualizations. Therefore, we conduct a technical evaluation to assess the performance of the GistVis submodules. Specifically, we provide quantitative results on the performance of the Discoverer and Annotator. Because Extractor and Visualizer is essentially a restoration process bound with information loss, we argue it would be challenging to provide reliable ground truth to evaluate the two modules. Hence, we demonstrate the performance of the Extractor and Visualizer with the user study results in the next section (Sec.~\ref{sec:eval-user-study}).

\subsection{Discoverer Evaluation}
The purpose of Discoverer is to delimit paragraphs into unit segments. In this section, we demonstrate the effectiveness of the Discoverer by comparing it with alternative methods that share the same algorithmic goal as the Discoverer.

\subsubsection{Experiment Setting}
We use the annotated corpus described in Sec.~\ref{sec:formative-study} to evaluate the performance of the Discoverer. Following the requirement of our computational pipeline, we annotated unit segments within the hard boundaries of paragraphs. We selected paragraphs that contained at least one data insight because we did not label unit segments for paragraphs without data insights. After filtering, our evaluation corpus included 640 paragraphs, each containing one or multiple unit segments. 

To provide a reference for comparison, we implemented three alternative methods. We first split paragraphs into sentences using the sentence tokenizer from the natural language processing toolkit nltk~\cite{bird2009natural} for all three alternative methods. Then, we applied different strategies to form unit segments. The first strategy, namely ``regex'', used regular expression to detect numbers inside a sentence. We identified a sentence as a unit segment if it includes a number. The second and third strategies involve using pre-trained language models, specifically BERT~\cite{devlin-etal-2019-bert}\footnote{\url{https://huggingface.co/google-bert/bert-base-uncased}} and sentence BERT~\cite{reimers-gurevych-2019-sentence}\footnote{\url{https://huggingface.co/sentence-transformers/all-MiniLM-L6-v2}}, to group similar sentences into unit segments. We first computed sentence embeddings, then computed the similarity between sentences, forming a unit segment if the similarity was above a certain threshold. We ran a grid search over multiple thresholds and reported the result of the best-performing threshold for each approach.

We benchmarked the performance of our Discoverer module against three alternative methods: regex, BERT, and sentence BERT, using the evaluation corpus mentioned above. %
We report the accuracy of segmentation for each condition. We define an accurate segmentation as making the same segmentation as our annotation. Our results would be a conservative indicator of the model's performance because we classify both over-segmentation and under-segmentation as incorrect, accepting only perfect matches. Over or under-segmentation of sentences without data insights would not impact the visualization result. Yet, we still require our algorithms to correctly delimit the boundaries to facilitate clear communication of our evaluation and compare different approaches.

\subsubsection{Result}
\begin{table}[tbp]
    \centering
    \caption{Comparison of segmentation accuracy for four candidate approaches.}
    \begin{tabular}{rrr}
    \toprule
       \textbf{Strategy}  & \textbf{Accuracy} & \textbf{Threshold} \\
    \midrule
       Regex  & 0.545 & - \\ 
       BERT &  0.611 & 0.81\\ 
       Sentence BERT & 0.600 & 0.68 \\ 
       Discoverer (LLM)  & \textbf{0.686} & - \\ 
    \bottomrule
    \end{tabular}
    \label{tab:discoverer-eval}
\end{table}

Results revealed moderate performance for all approaches, with the strategy applied in the Discoverer module performing the best (Table.~\ref{tab:discoverer-eval}). Specifically, the LLM-driven method we applied in the Discoverer module performed best, perfectly segmenting 68.6\% of the paragraphs. Meanwhile, the regex strategy fared worst, perfectly segmenting only 54.5\% of the paragraphs. The pre-trained language models' strategies gave a modest performance (61.1\% for BERT and 60.0\% for Sentence BERT). However, a trade-off of using the LLM-driven Discoverer is the processing time. Though the exact generation time through calling LLM APIs would be affected by internet speed and server load, we provide a rough expectation of this method's processing speed. We recorded an average of 5.22 seconds (SD=2.13s) to split each paragraph, with an average of 0.12 seconds (SD=0.03s) per-word processing time.



\begin{figure*}[tbp]
    \centering
    \includegraphics[width=0.85\linewidth]{figures/confusion-matrix-comparison.pdf}
    \caption{Normalized confusion matrices for data fact type annotation results. The left matrix (A) shows the result of our two-step Annotator (Type Checker + Type Moderator), while the right matrix (B) shows the result of the ablated condition (Type Moderator only). The horizontal axis denotes the predicted type, while the vertical axis indicates the actual type. The numbers on the diagonal line of this matrix represent the precision of classification for each category.}
    \label{fig:type-confusion-mat}
\end{figure*}


\subsection{Annotator Evaluation}
\label{subsec:quant-eval-annotator}

The goal of the Annotator is to label each unit segment with the corresponding data fact type. Thus, we demonstrate the Annotator's effectiveness by reporting its classification performance. We also justify the two-step Annotator design (Type Checker + Type Moderator) through an ablation study over a one-step design (Type Moderator only).

\subsubsection{Experiment Setting}
We use the annotated corpus described in Sec.~\ref{sec:formative-study} to evaluate the performance of the Annotator. To understand the performance of this single module, we assume the segmentation is correct and directly use the labeled unit segments as input. To suit the scope of the GistVis pipeline, we excluded types that are not yet supported and excluded meta-fact entries. After the above exclusion, we are left with 2676 unit segments as input. The dataset is highly imbalanced in its label because data insights are generally scarce even in data-rich documents. Specifically, there are 2355 unit segments without data fact (no type), 158 values, 41 trends, 58 comparisons, 7 extremes, 48 proportions, and 9 ranks, respectively. Because of the label imbalance, we report accuracy, and the weighted precision, recall, and F1-score to reflect the pipeline performance objectively. We also report the average annotation time per unit segment to provide a full picture of the performance of this module.

For our ablation study, we disabled the Type Checker in our Annotator module. Because the two-step Annotator does not pass ``no type'' segments to the Type Moderator, we slightly modified our prompt to support the output of the ``no type'' label to enable this comparison. The rest of the experiment was kept the same to eliminate confounding variables.

\subsubsection{Result}

Results revealed decent performance of the Annotator, achieving an overall accuracy of 0.79, a precision of 0.92, a recall of 0.79, and an F1 score of 0.84. Fig.~\ref{fig:type-confusion-mat}. A is the confusion matrix of the classification results. We identified several data fact types prone to misclassification, including rank, proportion, and extreme. For instance, the Annotator frequently misidentified proportion and extreme types as the value type. Since one or multiple values typically occur in proportion and extreme types, such misclassification exerts a minor impact during the visualization stage. However, visualizing proportion and extreme as the value type could potentially lead users to misunderstand the data insights. Additionally, rank was often misclassified as difference, extreme, or no type, which could undermine the expression of the rank insight during the visualization stage. We recorded an average inference time for each unit segment of 2.34 seconds (SD=2.46s). The large standard deviation in time was due to some inferences only passing one sub-stage (e.g., no type or only identified one data fact type). In contrast, others needed two passes to finalize the data fact type inference, which would significantly increase inference time.

At first glance, the ablated one-step condition (Type Moderator only) seemed to reveal comparable overall performance to the two-step Annotator, achieving an overall accuracy of 0.84, a precision of 0.89, a recall of 0.84, and an F1 score of 0.84. However, a breakdown in the classification performance of each category (Fig.~\ref{fig:type-confusion-mat}. B) revealed a significant pitfall of this design: the one-step condition was inclined to assign ``no type'' to unit segments. Specifically, the one-step condition significantly reduced the precision of the ``value'' type to only 0.11, an unacceptable result because the ``value'' type consists of as much as 33\% of the data insights in the corpus we collected. We attribute this difference between the two and one-step Annotator to the ``sifting'' effect of the Type Checker. The multiple parallel Type Checkers before the Type Moderator module filtered out most of the ``no type'' conditions and other unlikely data fact types, reducing the subsequent classification with fewer options. A smaller decision space would typically make classification easier. Another benefit of using a two-step design is that each Type Checker could be bespoke to achieve better performance. For example, we could design specific prompts using advanced prompting techniques such as In-context Learning~\cite{yao2024more} to bolster the performance further or even use different classification models for each data fact type.

\section{User Study}
% memotool as a viechle to xxx
% How to use the tool to xxx
We conducted a two-session user study to understand the feasibility of using InteRecon to create IDI and further explore the participants' approaches to using InteRecon to reconstruct their own items, collecting feedback on the challenges, future opportunities, and applications of IDI. 

% reconstructing physical memory artifacts with preserving their physical .
In the first and second sessions, we invited 16 participants from a local university campus (8 male, 8 female; age: avg = 24.13, std = 2.28). 
\revision{In the appending study, we recruited 10 participants through questionnaires from the campus, aiming for a diverse mix of professional backgrounds (4 male, 6 female; age: avg = 24.6, std = 2.5). The participants included 2 VR developers, 2 architects, 1 fashion designer, 1 product manager, 1 industrial designer, 1 hardware engineer, 1 curator, and 1 professor. Notably, three of these participants were re-invited from the previous two sessions.}
% All participants highly recognized the value of memorable personal items, showing their willingness to share these items and digitize them.
All had prior experience with using AR and VR devices with avg = 4.98, std = 1.53 on a scale from 1 (not at all familiar) to 7 (extremely familiar). 
The first two sessions took around 2 hours and the additional brainstorming took around 1 hour.
Each participant was paid an equivalent of 50 USD in local currency for compensation. 
The hardware configurations and AR deployment employed in the study were consistent with those detailed in Sec. \ref{implementation}.
The study was conducted in our laboratory and received ethical approval from the university's ethics review board.
% 我们提前联系好用,确保他们的物品所需的控件和元素都有


% We conducted a two-session user study to evaluate the user experience of utilizing MemoTool to create IDI within a mixed reality environment and further investigated the feedback and ideas on the challenges, future opportunities, and applications of MemoTool.



% After the first session, the user completed a questionnaire with Likert-type (scaled 1-7) questions according to a standard System Usability Scale (SUS) questionnaire \cite{bangor2008empirical}. After the second session, we conducted an open-ended interview to get qualitative feedback on our system. 
\begin{table}[tbh!]
\centering
\caption{\textbf{Descriptions for the atomic interactions of each function in the session one.}}
\Description{Table 2 shows descriptions of the atomic interactions of each function in the first session.}
~\label{tab:atomic_interaction}
    \vspace{-0.3cm}
    \small
    \resizebox{\linewidth}{!}{
    \begin{tabular}{l|l|l}
    \hline
        \textbf{ID} & \textbf{Function} & \textbf{Interaction Steps} \\ \hline
        A & Reconstructing 3D Appearance & \begin{tabular}[c]{@{}l@{}}1. Align the camera with the object and adjust the bounding box.\\ 2. Move the camera around the object. \\ 3. Examine and confirm the mesh.\end{tabular} \\ \hline
        B & Adding Physical Transforms & \begin{tabular}[c]{@{}l@{}}1. Segment the model for making joints.\\ 2. Touch the pre-designed joints and find a similar joint.
\\ 3. Apply the joints by mapping \textit{Base} and \textit{Movable} cubes.\end{tabular} \\ \hline
        C & Reconstructing Interface & \begin{tabular}[c]{@{}l@{}}1. Select the category of the virtual widget.\\ 2. Attach the virtual widgets on the model. \\ 3. Adjust the size and the visibility of the widgets. \end{tabular} \\ \hline
        D & Adding Embedded Content & \begin{tabular}[c]{@{}l@{}}1. Upload /edit the content with the application. \\ 2. Import the content to the model.\end{tabular} \\ \hline
    \end{tabular}
    }
\end{table}


\begin{table}[tbh!]
    \centering
    \caption{\textbf{Task scripts in the session one. }
    We employed a combination of ID letters and sequence numbers from Table  \ref{tab:atomic_interaction} to reference specific atomic interactions (e.g., the \textit{A1} corresponds to the first atomic interaction in the `Reconstructing 3D Appearance' category).}
    ~\label{tab:scripts}
    \Description{Table 2 shows the task scripts in the first session. We employed a combination of ID letters and sequence numbers from Table  \ref{tab:atomic_interaction} to reference specific atomic interactions (e.g., the \textit{A1} corresponds to the first atomic interaction in the `Reconstructing 3D Appearance' category).}
     \vspace{-0.3cm}
    \resizebox{\linewidth}{!}{
        \begin{tabular}{l|l|l}
        \hline
        \textbf{ID} & \textbf{Task} & \textbf{Script} \\ \hline
        T1 & Scan a model of a toy Panda & A1-A2-A3 \\ \hline
        T2 & Reconstruct a Moon lamp IDI’s physical transforms & B1-B2-B3 \\ \hline
        T3 & Reconstruct a TV IDI’s interface & C1-C2-C3-(D1-D2)\textasciicircum{}n \\ \hline
        \end{tabular}
    }
\end{table}




\subsection{Session One: Full Functions Experiencing}
\label{session-one}
% 1. encode functions
% 2. tasks covered all the interactions of MemoTool
% 3. goal: evaluate the usability of the MemoTool and user experience of creating IDI inside a mixed reality environment.
% and evaluate whether the user could obtain the sense of realism for the IDI created by MemoTool.
The goal of this session was to assess the feasibility of utilizing InteRecon to create the IDI within a mixed reality environment. 
To achieve this, we asked participants to experience all the functions provided by InteRecon to inspire them to propose more personal IDIs and usages in the next session.
% We designed interaction scripts to guide users' experience procedure. 
We broke down InteRecon's functions described in Sec. \ref{design_interRecon} into one-step interactions and designed scripts of one-step atomic interaction sequence for each function, as illustrated in Table \ref{tab:atomic_interaction}. 
We further designed three micro IDI-creation tasks, which collectively cover all the atomic interactions across the four functions.
Each task was structured to progress through a list of several atomic interactions, as outlined in Table \ref{tab:scripts}. 
% We employed a combination of ID letters and sequence numbers from Table \ref{tab:atomic_interaction} to reference specific atomic interactions in Table \ref{tab:scripts} (e.g., the \textit{A1} corresponds to the first atomic interaction in the `Reconstructing 3D Appearance' category).
% We illustrated the micro tasks in Fig. xx by highlighting the reconstruction component to be needed within each task model.
In the case of T3 from micro-tasks, as there were multiple contents in a TV to be added, we anticipated that the D function in Table \ref{tab:atomic_interaction} would be iterated \textit{n} times for participants to reconstruct a TV IDI.
We noted this in our task scripts in Table \ref{tab:scripts}. 
Additionally, we created tutorials for each task, available in both the AR environment and print format. 
% Each tutorial page contained written descriptions and visual instructions (e.g., specific buttons to click) for every atomic interaction.
% with each tutorial page comprising textual descriptions and visual instructions (e.g., the targeted buttons to be clicked) for each atomic interaction. 

We first asked the participants to walk through the Hololens 2 official tutorial to learn how to navigate the user interface with basic hand gestures. 
After finishing the consent form and demographic questionnaire, participants were first introduced to the study and provided with a tutorial. 
They were then guided to complete three tasks sequentially. 
Before each task, participants were invited to interact with the relevant physical item associated with the task. 
This allowed them to gain an understanding of the item's interactivity, such as the widgets on the TV and the joint mechanism between the Moon lamp's base and body, assisting them in reconstructing these interactions in AR.
After completing each task, participants provided brief feedback on their InteRecon experience and were granted a 10-minute break.
The study involved two experimenters: one was responsible for introducing the study and aiding participants in tutorial comprehension, while the other monitored participants' Hololens views and documented the feedback for each task. 
After finishing all the tasks, the participants were asked to complete a questionnaire with Likert-type (scaled 1-7) questions on evaluating the feasibility of InteRecon's four functional categories.
% We provided screenshots of each function pre-designed by our researcher within the questionnaire as a visual reminder.
The questions with metrics employed in the questionnaire are detailed in the Appendix Table \ref{tab:questionnaire}.






% After the first session, the user completed a questionnaire with Likert-type (scaled 1-7) questions according to a standard System Usability Scale (SUS) questionnaire \cite{bangor2008empirical}.

% , and the short comments for each task. The participants view in Hololens was monitored by one of our experimenter. 


% 1. two researchers, 1 for introducing the study for participants, 1 for recording the duration of each interaction step
% 2. record the participants' view in hololens, record the duration
% 3. complete the task according to the scripts, after each task the participants will taks a break for 10 mins. 

% To evaluate the efficacy of , we recorded the 



% We designed six micro tasks for the frst user study session (Figure 22). Each task contains one pair of toy-AR interaction for users to author. Half of these interactions (Task 1-3) use the toy as the trigger to actuate the AR content, and the other three (Task 4-6) use the AR content to actuate the toy. Our input and output categories are all covered in these interactions. Both continuous and discrete trigger-action types are included as well. The description of each task is detailed in Table 2. The goal of this session was to evaluate the usability of the MechARspace system and to explore the user experience of authoring toy-based AR applications inside a mixed reality environment.

% \revision{We designed interaction scripts to guide users' experience procedure. Specifically, we broke down each application into single-step interaction tasks according to the functions described in Sec 6.2 and designed scripts of a one-step atomic interaction sequence to form an integral usage flow. We summarized all possible atomic interactions into three categories - one-finger touch, one-finger swipe, and multi-finger gesture - as shown in Table \ref{tab:atomic_interaction}. The template scripts of different applications are shown in Table \ref{tab:task_description}.}

% \revision{To evaluate users' experience and subjective ratings of different hand-to-surface interaction techniques, we designed a questionnaire containing five questions derived from the system usability scale (SUS) \cite{bangor2008empirical} on willingness to use, easiness to use, integrity, learnability, and confidence regarding different applications under three techniques. Users were further asked to provide their subjective feedback towards three different settings.}


% \begin{table*}[tbh!]
% \centering
% \caption{\textbf{Descriptions for the atomic interactions of each function in the On Boarding session.}}
% \Description{Table 1 shows descriptions of the atomic interactions of each function in the first session.}
% ~\label{tab:atomic_interaction}
%     \vspace{-0.3cm}
%     \resizebox{\linewidth}{!}{
%     \begin{tabular}{l|l|l}
%     \hline
%         \textbf{ID} & \textbf{Function} & \textbf{Atomic Interaction} \\ \hline
%         A & Reconstructing 3D Appearance & \begin{tabular}[c]{@{}l@{}}1. Align the camera with the object and adjust the bounding box.\\ 2. Move the camera around the object. \\ 3. Examine and confirm the mesh.\end{tabular} \\ \hline
%         B & Adding Physical Transforms & \begin{tabular}[c]{@{}l@{}}1. Segment the model for making joints.\\ 2. Touch the pre-designed joints and find a similar joint.
% \\ 3. Apply the joints by mapping \textit{Base} and \textit{Movable} cubes.\end{tabular} \\ \hline
%         C & Reconstructing Interface & \begin{tabular}[c]{@{}l@{}}1. Select the category of the virtual widget.\\ 2. Attach the virtual widgets on the model. \\ 3. Adjust the size and the visibility of the widgets. \end{tabular} \\ \hline
%         D & Adding Embedded Content & \begin{tabular}[c]{@{}l@{}}1. Upload / edit the content with the application. \\ 2. Import the content to the model.\end{tabular} \\ \hline
%     \end{tabular}
%     }
% \end{table*}


% \begin{table}[tbh!]
%     \centering
%     \caption{\textbf{Task scripts in the On Boarding session. }}
%     % We employed a combination of ID letters and sequence numbers from Table  \ref{tab:atomic_interaction} to reference specific atomic interactions (e.g., the \textit{A1} corresponds to the first atomic interaction in the `Reconstructing 3D Appearance' category).
%     ~\label{tab:scripts}
%     \Description{Table 2 shows the task scripts in the first session. We employed a combination of ID letters and sequence numbers from Figure 1 to reference specific atomic interactions (e.g., the A1 corresponds to the first atomic interaction in the `Model' category).}
%      \vspace{-0.3cm}
%     \resizebox{\linewidth}{!}{
%         \begin{tabular}{l|l|l}
%         \hline
%         \textbf{ID} & \textbf{Task} & \textbf{Script} \\ \hline
%         T1 & Scan a model of a toy panda & A1-A2-A3 \\ \hline
%         T2 & Reconstruct a Moon lamp IDI’s physical transforms & B1-B2-B3 \\ \hline
%         T3 & Reconstruct a TV IDI’s interface & (D1-D2)\textasciicircum{}n-C1-C2-C3 \\ \hline
%         \end{tabular}
%     }
% \end{table}



\subsection{Session Two: Free Exploration on Prototyping Personalized IDI}
\label{session-two}
The goal of this session was to investigate how participants employ InteRecon to create their own IDI in an exploratory manner without specific tasks, aiming to obtain further feedback and ideas on the challenges, application scenarios, and future opportunities of InteRecon.
We conducted a phone interview to inquire about the types of items they were interested in reconstructing before participants arrived for the user study.
This ensured the InteRecon's resources, the elements within the categories of `geometry', `interface', and `content', could accommodate the potential interactivity features of reconstructing participants' items. 
Three participants said that the items they wished to reconstruct were not at hand, so with the assistance of our researcher, they downloaded similar 3D models from the internet to proceed with the next steps of creating IDI, bypassing the step of scanning the physical item. 
We also provided participants with items mentioned in our formative study in case the participants had some more creative ideas to implement.
Participants were then asked to utilize InteRecon to recreate the IDI.
% They were also allowed to reconstruct their own items. 
This exploratory session lasted approximately 30 minutes. 
% various physical mementos  such as a toy Transformer, a furry toy bear, a toy Pikachu, a toy car, a Game Boy, as shown in Fig. \ref{fig:toys} (a). 

At this stage, participants were already familiar with InteRecon and an experimenter was present to address any questions they might have had. 
Participants were asked to ``think-aloud'' \cite{van1994think} to express their thoughts promptly during the process. 
In the end, we held in-depth interviews (30-40 minutes) with our participants regarding their qualitative feedback.
The entire session was recorded on video by the experimenter. 



% \zisu{different user types and informal interview}
% \zisu{Overall, I would like to see more emphasis on how memorable items relate to the system in the revision. Additionally, please clarify the relationship between memorable items and IDI. Are memorable items just one part of what IDI encompasses, or is IDI solely focused on memorable items?}
% \zisu{1AC: One main issue is the limited scope and unclear motivation (R1, 2AC). While the focus on personal memory archiving is understandable, the paper could benefit from exploring broader applications and discussing the real-world utility of this technology beyond niche use cases (R1). Additionally, the significance of the work could be increased by emphasizing its societal or broader implications (2AC).
% }
% broader implications
\subsection{\revision{Appending Study: Gaining Design and Usage Implications from Broader Audiences}}
\revision{We conducted an appending study to investigate broader designs and applications for IDI, expanding its conceptual development beyond personal memory archiving. 
Given the diverse professional backgrounds of our participants, they were able to brainstorm potential applications and utilities of IDIs in their professional contexts, contributing ideas on new conceptual developments for IDIs in the future.}

\revision{The experimenter introduced the concept of IDI first and showcased examples of IDIs created in previous sessions, providing participants with a comprehensive overview.
Following this introduction, participants were asked to try using AR glasses and engage directly with the created IDIs. 
The introduction and try-on took around 30 minutes for each participant.}

\revision{We then conducted semi-structured interviews with the participants, focusing on the following questions: 1) benefits of digital 3D reconstruction for objects while preserving their physical interactivity, 2) potential application scenarios for IDI of their professional areas, 3) specific uses of IDI in professional or functional settings (e.g., museums, education, healthcare, and travel), and 4)extensions of the concept of IDI beyond personal memory archiving.
We specifically emphasized that the participants should incorporate their professional experiences or expertise when answering questions.
The entire procedure was recorded on video by the researcher.}

% We specifically emphasize that users should incorporate their professional experiences or expertise when answering questions.

% augmenting physical items with MemoTool in general. Specifically, we investigated participants' views of (1) the portability of operation and interaction, and (2) the completeness and expectations of system functionality. The entire session was recorded on video by the researcher.


% 1. free explore
% 2. choose an item in the scene
% 3. interview





% \begin{figure*}[]
% \centering
% \includegraphics[width=\textwidth]{Figures/exploration.jpg}
% \caption{\textbf{ Materials and sample results in the second session of our user study. (a) Physical mementos provided to explore. (b) The user is playing games through the Game Boy IDI in AR. (c) The user is manipulating the mechanical components of the Transformer IDI in AR.}}
% \Description{Figure 5 has 3 sub-figures, which show the materials and sample results in the second session of our user study. (a) Physical mementos provided to explore. (b) The user is playing games through the Game Boy IDI in AR. (c) The user is manipulating the mechanical components of the Transformer IDI in AR.}
% \label{fig:toys}
% \end{figure*}




\begin{figure}[htbp]
    \centering
    \includegraphics[width=\linewidth]{Figures/blue-barplot.png}
    \vspace{-2ex}
    \caption{\revision{\textbf{The mean duration time of each atomic interaction in three tasks in Session One.}}}
    \Description{Figure 5 shows the duration of each atomic interaction in three tasks in Session One.}
    \label{duration-time}
\end{figure}



\subsection{Results}
\label{fi:overview_results}
% We set off to find the user experiences of utilizing MemoTool to create IDI. 
We reported the results collected from our user study, including the duration time of each interaction with failure cases in Session One, the questionnaire consisting of 7-point Likert scale data regarding the user subjective ratings for four categories of InteRecon functions in Table \ref{tab:atomic_interaction}, and the qualitative feedback from the interview in Session Two. 
Fig. \ref{fig:results} illustrates examples of the reconstructed IDIs by our participants.
Except the Moon Lamp in a-1 of Fig. \ref{fig:results} was created in the pre-designed task of Session One (Section \ref{session-one}), the rest examples were created by participants in Session Two (Section \ref{session-two}). 
We conducted a thematic analysis with qualitative feedback data from 16 participants. 
We report the overall results in Sec.~\ref{sec: subj_rating} to assess the feasibility of the authoring workflow of using InteRecon to create IDI and further report the qualitative results in the following subsections.


\subsubsection{\revision{Overall Results}}
% \zisu{Discuss failure cases, the time and effort required to scan objects, and the system’s responsiveness in handling complex physical interactions.
% I am curious if the authors can have more results from the perspective of, for example, the accuracy of the 3D reconstruction, the time and effort required to scan objects, and the system’s responsiveness in handling complex physical interactions.
% Consider expanding the participant pool and incorporating more rigorous methodologies. Include technical performance metrics in the evaluation and address the concerns about the study's focus.
% They also requested more technical performance metrics in the evaluation.}
\revision{
We recorded the mean duration time to complete the atomic interaction in each task (11 data points in 3 tasks) in Session One, resulting in 176 data points in total across 16 participants, as illustrated in Fig. \ref{duration-time}. 
For the D1 and D2 interactions, which occurred multiple times in the task, we recorded the first trial for each participant.
The results show that the mean duration time of every atomic interaction is within a reasonable time phrase (under 250 seconds).
Notably, A2 and B3 posed a relatively longer duration time for users to complete. 
The difficulty of A2 lies in the fact that the user needs to move the camera slowly to ensure that enough key frames are captured to generate the reconstruction results. Additionally, in order to capture the three orbit data of the object (front, side, and bottom surface), the participant needs to hold the camera and circle around the object three times. 
The difficulty of B3 arises from the need to employ pinching gestures for selecting model segments within the AR environment. 
Participants reported challenges in this selecting process, partly due to lower pinching gesture accuracy.
We also observed 3 failure cases during B3 involving 3 participants and this may be attributed to the confusion in identifying the `Base' and `Movable' segments of the model.
Two failure cases also occurred during the B1 operation in Session Two, as participants attempted to cut overly complex mesh objects in AR (e.g., having a higher number of vertices and intricate topological structures). This complexity led to crashes in the HoloLens system.
}

% \subsubsection{Subjective Ratings}
\label{sec: subj_rating}
% 1. overall evaluation for the tool on ten metrics
% 2. the duration of each step to examine the easiness of it (check the outliers)
% 3. overall comments on each functions
% 4. 

Participants were asked to provide their ratings towards the four categories of functions through our questionnaire, as illustrated in Fig. \ref{BERT}. 
We employed five metrics: ease of use, learnability, helpfulness, expressiveness, and non-frustration. 
A detailed description of each metric's question can be found in the Appendix Table \ref{tab:questionnaire}. 
Participants found the interaction process user-friendly, as reflected in the high average rating for ease of use (avg = 5.81, std = 0.50). 
They expressed confidence in their ability to use the functions, indicated by a high learnability score (avg = 5.88, std = 0.41). 
The scores of `helpfulness' was also well received (avg = 5.55, std = 0.52). 
Moreover, participants appreciated the expressiveness of the InteRecon, giving it a high score (avg = 6.08, std = 0.42), and they experienced minimal frustration, as shown by the rating for non-frustration (avg = 5.94, std = 0.48). 
These results show that InteRecon is effective, expressive, and user-friendly in creating IDI, with its rich customization possibilities receiving positive feedback from participants. 




\begin{figure}[tbh!]
     \centering
     \includegraphics[width=\linewidth]{Figures/qua.png}
     % \includesvg[width=\linewidth]{Figures/qua.svg}
     \vspace{-2ex}
     \caption{\textbf{Average subjective rating scores for 4 categories of functions in Table \ref{tab:atomic_interaction}. The first (green), second (orange), third (purple), and fourth (pink) columns in each cluster indicate the score distribution across four function categories. 1 - strongly disagree, 7 - strongly agree.}}
     \Description{Figure 6 shows the average subjective rating scores for 4 categories of functions in Table 1. The first (green), second (orange), third (purple), and fourth (pink) columns in each cluster indicate the score distribution across four function categories. 1 - strongly disagree, 7 - strongly agree.}
     \label{BERT}
\end{figure}



\begin{figure*}[tbh!]
\centering
\includegraphics[width=\textwidth]{Figures/Results.jpg}
\caption{\textbf{Example IDIs shown in AR environment created in our user study. (a-1,2,3,4,5) IDIs of physical artifacts including reconstructed physical joints, which could be interacted by hands and generate similar movements to the real world. (b-1,2,3) IDIs of electronic devices, including reconstructed interface with widgets. (b-1) Reconstruct the slider for the DJ booth's IDI: Drag the slider to adjust its volume. (b-2) Reconstruct the screen, the buttons of on/off and directional pads for the Game Boy's IDI: Press the buttons to play the puzzle game. (b-3) Reconstruct the display screen, the viewfinder, and the shutter for the camera's IDI: By pressing the shutter button, a photo of the scene within the viewfinder is captured and displayed on the screen.}}
\Description{Figure 6 shows example IDIs shown in AR environment created in our user study. (a-1,2,3,4,5) IDIs of physical artifacts including reconstructed physical joints, which could be interacted by hands and generate similar movements to the real world. (b-1,2,3) IDIs of electronic devices, including reconstructed interface with widgets. (b-1) Reconstruct the slider for the DJ booth's IDI: Drag the slider to adjust its volume. (b-2) Reconstruct the screen, the buttons of on/off and directional pads for the Game Boy's IDI: Press the buttons to play the puzzle game. (b-3) Reconstruct the display screen, the viewfinder, and the shutter for the camera's IDI: By pressing the shutter button, a photo of the scene within the viewfinder is captured and displayed on the screen.}
\label{fig:results}
\end{figure*}

\begin{figure}[tbh!]
     \centering
     \includegraphics[width=\linewidth]{Figures/Bonus.jpg}
     \vspace{-2ex}
\caption{\textbf{Two example IDIs participants created that augmented with additional interactivity beyond the real world. (a) `Interactive Statue': the reconstructed violinist statue augmented with its beyond real-world functions of playing music by adding a ‘Play’ button and embedded content. (b) `Interactive Photo Album': the reconstructed souvenir augmented with its beyond real-world functions of displaying photos by adding a button widget and a screen widget.}}
\Description{Figure 7 shows two example IDIs participants created that augmented with additional interactivity beyond the real world. (a) `Interactive Statue': the reconstructed violinist statue augmented with its beyond real-world functions of playing music by adding a ‘Play’ button and embedded content. (b) `Interactive Photo Album': the reconstructed souvenir augmented with its real-world functions of displaying photos by adding a button widget and a screen widget.}
\label{fig:creative}
\end{figure}

%  P16 developed a feature that attaches a
% button widget and a screen widget to a souvenir purchased during
% a trip, which is shown in Fig. 7 xx. This setup is designed to display
% photos from the trip, allowing for a natural recollection of travel
% memories each time this IDI is accessed, combining photos and the
% model.

%Participants found the interaction process to be user-friendly (Easiness to use: avg = 5.73, sd = 0.48), with many expressing confidence within these functions (Confidence: avg = 6.11, sd = 0.45). 
%Participants believed that they could learn each function well through the tutorial (Learnability: avg = 5.88, sd = 0.44), allowing them to easily (Satisfaction: avg = 6, sd = 0.44) and efficiently (Efficiency: avg = 5.76, sd = 0.44) complete each function. 
%Moreover, participants confirmed that they could understand the function without confusion (Clarity: avg = 6.03, sd = 0.48).
%Our participants also agreed that InteRecon's customization and design space are rich to explore (Flexibility: avg = 5.74, sd = 0.37; Expressiveness: avg = 6.15, sd = 0.40), and we will provide more details in the following Sec. 5.3.4.
%The results show that InteRecon is an effective, expressive, and enjoyable tool for creating IDI.

% possessed the advantages of concise tutorials, easy operation, and strong helpfulness in functionality. 

% as well as their qualitative feedback in the following Sec. 5.3.4. 

% flexible (Flexibility: avg=5.74, sd=0.37) and expressive (Expressiveness: avg=6.15, sd=0.40) to use to enable them to complete more tasks with various scenarios. 


% Simultaneously, most users expressed their willingness to continue using these functions over an extended period.  

% Upon reproducing the functions, most of the participants discovered that these functions not only met their operation needs (Flexibility: avg=5.74, sd=0.37), but also achieved desired outcomes (Expressiveness: avg=6.15, sd=0.40), leading them to perceive these functions as beneficial for their future lives (Helpfulness: avg=5.7, sd=0.56). 

% In addition, we recorded the duration to complete the atomic interactions for each task (18 data points in 4 tasks) in the first session, resulting in 288 data points in total across 16 participants, as illustrated in Fig. \ref{boxplot}. 
% Fig. \ref{boxplot} presents a box-plot showing the duration in four tasks.
% A box-plot is a statistical technique summarizing a group of data and describing the discrete degree by identifying outlier data values \cite{williamson1989box}. 
% The stability of completion time can be reflected by the box-plot. 
% As can be seen from the Fig. \ref{boxplot}, only one of the 18 scripts generated two outliers in the time distribution, while the remaining 17 scripts had either no outliers or only one outlier. 
% This result indicates that each atomic interaction is generally completed within a reasonable timeframe, with the exception of D3 in Task 4, which exceeded 200 seconds. 
% Notably, D3 posed significant challenges for the majority of users (eleven out of sixteen). 
% This difficulty arises from the necessity to employ pinching gestures for selecting model segments within the AR environment. 


% The outcome reveals that each atomic interaction is generally completed within a reasonable timeframe, with the exception of D3 in Task 4, which exceeded 200 seconds. 

% Notably, D3 posed significant challenges for the majority of users (eleven out of sixteen). 
% This difficulty arises from the necessity to employ pinching gestures for selecting model segments within the augmented reality (AR) environment. 
% It is noteworthy that participants encountered initial challenges when learning to navigate the AR interface in Hololens using hand gestures, partly due to lower pinching gesture accuracy.

% During the user study, we observed that most users (fve out of six) struggled when they initially started learning how to use hand gestures to navigate the AR interface in Hololens.

% providing further evidence of the high usability of the MemoTool's overall system. 








% \revision{
% Results (qualitative feedback):

% 1. physical interactivity created by InteRecon (这部分写用interecon建立的IDI是不是重建了现实世界的交互,好在哪里)
%     a. functions enabled for electronic devices
%     b. motions mapped for physical artifacts
%     c. realism created by bare-hands-object interaction, but this still needs to be improved

% 2.potentials of IDI for enriching memory archives (这部分写IDI有哪些新的潜力)
%     a. applications
%     b. features
%     c. in-situ reconstruction for beyond personal items

% Discussion 

% 1. boundary and risks of realism and virtual augmentations
% 2. different roles played in IDI sharing virtual community through modular IDI `LEGO'
% 3. more fine-grained physical properties enabled by AI graphics techniques


% }


\subsubsection{Physical Interactivity Created by InteRecon for Realistic Experiences}
\label{fi:Realistic}
All participants agreed that they were able to create the IDI from physical items and reconstruct interactivity for them within InteRecon in the AR environment, highlighting its \engquote{memorability} (P1), and \engquote{digital longevity} (P4) of IDI. 
\revision{Additionally, with the realistic interactions of IDI created by InteRecon, participants felt a strong sense of ownership and envisioned IDIs as 3D interactive digital assets, \engquote{more memorable than photos, vlogs, or static 3D scanning items} (N=13).}
\revision{Participants also mentioned that IDI represents an `exciting advancement' compared to static 3D scanned digital objects (N=15). This is because the interactivity of personal items is entirely defined, designed, and reconstructed by the participants themselves. It more comprehensively reflects the meaning of personal items by reconstructing how users engage with their personal items, which are closely tied to personal memories. As a result, \engquote{a deeper emotional connection to IDI was built} (N=10).}



% 用户也提到了可交互数字物品和静态3d扫描数字物品相比是一个令人惊喜的进步,因为可交互数字物品的交互形式是完全由用户自己定义、设计、并完成建立的,它更全面的反映了用户的个人物品的使用方式,而使用方式是和个人记忆强相关的,所以用户对idi也具有更深的情感连接。

\textbf{Interfaces and content} reconstructed in IDI for electronic devices is vivid and relatively authentic, facilitating access to the original versions of files and allowing for the original interactive input methods of the devices.
It's akin to an AR emulator of vintage electronic devices, offering a more realistic experience than 2D emulators which lack the ability to replicate physical interactions. 
\engquote{For instance, a 2D emulator merely converts a physical button press into a screen click, diminishing its authenticity. } (P6), as he reconstructed his Game Boy in AR in Fig. \ref{fig:results}(b-2). 
After reconstructing more electronic devices (e.g., camera, music player, and DJ table) shown in Fig. \ref{fig:results}(b) and Fig. \ref{fig:teaser}(b), participants suggested that future collections of various iterations of electronic devices (like the diverse models of music players and cameras) might be supplanted by their AR counterparts, alleviating the demands on physical storage space and financial expenses. 
% This shift is expected to alleviate the demands on physical storage space and financial expenses.



\textbf{Geometric properties} reconstructed in IDI for physical artifacts enhanced the enjoyment of interacting with digital replicas of memory artifacts.
Many participants (N=15) emphasized that the reconstruction of the mechanical joints truly constituted the impressive physical features of their toys, \engquote{turning them into more functional and delightful digital keepsakes.} (P9).
However, more physical properties (e.g. texture, softness, opacity, etc. ) were mentioned by our participants to improve the realism of IDI. 
For example, as Fig. \ref{fig:results}(a-2) illustrated, P10 said \engquote{I find my toy soft and furry texture very comforting, yet its IDI version appears stiff, reducing our emotional connection with it.}



\subsubsection{Immersive Interactions Empowered IDI creation}
\label{fi:immersive}
InteRecon enabled various virtual and physical interactions, facilitating users in reconstructing the physical interactivity of memory artifacts. 
Within this spectrum, two types of prominent interactions were frequently highlighted by users as significantly impacting the reconstruction experience.

\textbf{Hand-object interactions} in AR were considered the principal resource of realism. 
Many participants (N=12) expressed their appreciation for being able to use their hands to interact with virtual objects in AR as if they were in the physical world. 
As P7 mentioned in Fig. \ref{fig:teaser}(a), \engquote{When I slightly touched the digital toy Stitch, its head moved just like the real toy Stitch.} 
Bare-hand manipulation and its corresponding realistic transformation effects in IDI constitute the most important and preferable characteristic of InteRecon.
Seven participants with VR/AR experience noted that interactions simulating bare-hand physics diverge from their usual experiences in immersive environments, which typically \engquote{involve selecting or manipulating targets with a controller} (P12).
Five participants suggested further enhancements in modeling different hand parts (e.g., finger joints, palm or back of the hand, etc.) to facilitate more detailed collisions and touch interactions with IDI.
\engquote{Since interactions between various hand surfaces and objects often result in different components' movements of the object in the real world} (P14), these advancements could also increase the interaction's realism.

\textbf{Model segmentation} is considered to have raised concerns due to being overly realistic.
Two different operations were provided for users to segment 3D models, including direct segmentation in an AR environment and segmentation using 2D software.
Within the former operation, users could witness the model being divided into components. 
Many participants expressed discomfort with this effect, describing it as cruel: \engquote{Perhaps I don't want my toy to be divided into parts; it looks as if it is broken. Maybe segmenting in a 2D interface could mitigate the realism, making it more acceptable to me.} (P15).
Especially \engquote{models of animals or human figures}, can feel overly realistic and thus \engquote{cruel} (N=13)—for example, separating the legs of a toy dog or the arms of a LEGO figure, as their IDI illustrated in Fig. \ref{fig:geo} and Fig. \ref{fig:results}(a-4). 
Although participants expressed satisfaction with the final effects of IDI to enable interactivity, they did not hope the creating process was also too realistic.  
In contrast, segmentation within a 2D environment does not evoke these sentiments. 
In 2D, the process is perceived as \engquote{a routine operation without a sense of realism} (P9), thereby avoiding feelings of cruelty.





% toy Stitch, stating, \engquote{This significantly added enjoyment to the digital reconstructed memento, transforming it into a more practical and enjoyable digital memento that exceeded my IDIgination.}
\subsubsection{Potentials of IDI to Enrich Memory Archives}
\label{fi:potentials}
Participants praised the concept of IDI in terms of enriching personal memory archives and proposed new envisioned features and scenarios to enrich the design space of InteRecon.
\revision{Participants compared the difference between simply scanning objects and objects with authoring their interactivity (N=10). They mentioned that it is similar to the difference between photos and videos; dynamic, interactive 3D models can be more expressive in life-logging scenarios. \engquote{They can be dynamic and offer greater potential for secondary creation for people} (P8), and also mentioned that \engquote{the interaction between objects, items, and people together contributes to the completeness of memories.} (P11).
}



\textbf{Creative interactivity beyond the real world} could be created by InteRecon. 
These interactions might not happen in the real world, but InteRecon can realize them in AR.
For example, for personal items, many participants wanted to create interactions about contextual elements associated with the memory (e.g., music, animations, photos, etc.) to enrich the memory of the item due to the convenience of interactivity creation by InteRecon, as shown in Fig. \ref{fig:creative}.
P4 created an IDI that could play music by pressing a `Play' button for a statue of a violinist in Mexico, as illustrated in Fig. \ref{fig:creative}(a). 
P4 added \engquote{When I bought this statue in Mexico, the environmental music was always `Remember me.' So, it is fantastic for me to reconstruct the music in my memory to my virtual statue by attaching a button.}
% Also, P16 invented the function of playing the photos taken during travel on a souvenir bought from a zoo by attaching a screen widget and a button widget and import the photos to the souvenir, which is shown in Fig. \ref{fig:results} xx. 
Also, P16 developed a feature that attaches a button widget and a screen widget to a souvenir purchased during a trip, which is shown in Fig. \ref{fig:creative}(b). This setup is designed to display photos from the trip, allowing for a natural recollection of travel memories each time this IDI is accessed, combining photos and the model.
By using InteRecon, participants could conveniently create the IDI's interactivity beyond its physical counterpart and delineate a more interactive virtual reconstruction that enriches personal memory archives.

\textbf{In-situ and life-logging scenarios} were also proposed by our participants, including using InteRecon to enable in-situ 3D interactivity reconstruction and utilizing IDI to be a social media platform to empower life-logging.
Non-personal items or items not at hand were proposed by participants to conduct in-situ interactivity reconstruction within InteRecon, such as \engquote{museum exhibits} (P8), \engquote{interactive art installations encountered during travel} and \engquote{my toy which is in my parent's house} (P4).
% They envisioned the possibility of using InteRecon to conduct in-situ interactivity reconstruction for items beyond their personal belongings, such as \engquote{museum exhibits} (P8), \engquote{interactive art installations encountered during travel} and \engquote{my toy which is in my parent's house} (P4). 
These items, though integral to their memories, are not physically transportable.
By using InteRecon remotely, anytime and anywhere, people could get access to their memorable items without spatial and temporal limitations, thus extending the digital longevity of the items.
% InteRecon offers the capability of in-situ reconstruction of interactivity of such items, providing participants with remote access to their memorable items, anytime and anywhere, thus preserving their significance in the participants' memories.
Additionally, InteRecon was recognized as a life-logging 3D content generator. 
As P5 said, \engquote{3D digital replication with interactivity offers a more vivid representation than static 2D photos. Thus I can capture memorable moments with my pets by incorporating interactive elements into our digital counterparts!}.
As life-logging 3D content thrives, InteRecon has the potential to evolve into a platform for social media sharing, as P5 said \engquote{like a 3D version of Instagram}, empowering the pervasive access of 3D content.





\subsubsection{\revision{Broader Conceptual Enrichment and Applications of IDI}}
\label{broader}
\revision{Our participants also shared ideas for the future applications of IDI, enriching its conceptual framework across various professional fields.
First, IDIs can serve as instant sharing objects in both museum and educational settings.
In museums, IDI can function as an AR digital object that integrates the historical use of ancient artifacts or brings ancient sculptures to life with dynamic animations like \engquote{making Terracotta Army of China come alive} (P3). This interactivity can make exhibits more engaging and impressive for visitors.}
\revision{In educational settings, IDI acts as a medium for instant sharing, allowing teachers to demonstrate key mathematical and physical concepts like \engquote{celestial movements} or visualize literary works by incorporating interactive scenes like \engquote{realizing Harry Potter's magic} (P2).}
\revision{In the medical field, doctors can develop IDIs that provide detailed operating procedures of \engquote{cosmetic surgery, organ transplant surgery, or traditional Chinese acupuncture} (P9). 
This comprehensive presentation allows patients to gain a clearer understanding of upcoming surgeries, while also have the potential to enable apprentice doctors to enhance their knowledge for performing these procedures. 
In the field of fashion design, our participants envisioned that IDI could offer a cost-effective way to model and experiment with materials and environments. This method allows designers to preview how garments under various conditions that are difficult to replicate in the physical world. As P7 noted, \engquote{I can reconstruct my designed coat as an IDI and even see how it performs when worn in the low gravity environment of the moon!}. Overall, the rich interactive reconstruction and customization capabilities of IDI make it highly applicable across various industries.}

% \zisu{attributes to its interactivity and customization capability}


% and apprentice doctors could have a better understanding of the surgery for efficient communication. 



% facilitate seamless doctor-patient and doctor-student communications by presenting cosmetic surgery details or teaching remote surgeries and traditional Chinese medicine techniques through 3D organ reconstructions. 



% concept extension:xxx
% 1.教育场景,一个重要的可被即时分享的工具,可以更多的激发学生的创造力,例如老师可以更身临其境的向学生演示天体的运动等等
% 2.无缝沟通工具也可以用在医疗场景中,在医生和患者沟通中,例如整容手术之前,可以更容易的让患者割双眼皮或对脸部肌肉进行微调的细节,;甚至还有远程手术的教学,中医把脉等等,重建3d器官 解剖手术教学等等,
% 2.museum场景,idi的概念可以帮助还原文物在古代是如何被使用的,make museum exhibits alive, 除此之外,idi更像是一个立体的,可以具有多种元素的重建,重建之后可以是一个可交互动画的形式,例如一个文物杯子的idi,可能包含这个杯子的主人,使用场景,文字介绍,等丰富的内容,使museum的东西更impress参观者。
% 3.服装、造型设计场景,idi的概念可以帮助更低成本的建模和切换,例如设计师希望能够预览设计的衣服在月球上如果没有重力被穿上是什么样子
% d.希望可以呈现一些效果,在月球上穿失去重力会这么样(改变了服装的物理特性),替换衣服的材质可以更方便,做更多的材料试验,redesign,节约成本。
% 这些场景中数字内容的交互性具有着决定性的作用




















% Participants proposed innovative scenarios for utilizing IDI to create, share, and enhance digital interactivity. They suggested employing InteRecon for the in-situ reconstruction of interactivity for non-personal items, such as museum exhibits, interactive art installations encountered during travels, and personal belongings located elsewhere, like a toy in a participant's parental home. These items, though integral to their memories, are not physically transportable. InteRecon enables remote access to these memorable items, allowing for interaction at any time and place, thus preserving their significance in the participants' memories. 
% Moreover, InteRecon is recognized as a versatile platform for life-logging through 3D content generation, with potential applications in social media sharing, akin to a "3D version of Instagram," thereby facilitating widespread access to 3D content.



% IDI were believed as a more vivid and multi-dimension life-logging content that can be enriched in a social media platform, \engquote{like a 3D version of Instagram} (P8).
% Through uploading their IDI to the platform, participants could post 3D contents and 



% While this statue cannot be interacted in the real world, but I can easily attach a button to it in the virtual environment. 












% Participants praised MemoTool for its creativity and generality as a tool for creating IDI in terms of specific functions and the overall workflow in MemoTool. 
% % design space
% Fig. \ref{fig:toys} (b) and (c) show some of the IDI participants created in the exploratory session.
% For example, P13 used MemoTool to create a toy Transformer IDI and commented: \engquote{Such complex joints can be mapped to the virtual model by myself!} 
% Many participants mentioned physical mementos from the toys to digital devices that can be reconstructed, such as, \engquote{all the rigid toys} (P9), \engquote{the Tamagotchi}~\footnote{https://en.wikipedia.org/wiki/Tamagotchi} (P10) (a handheld digital pet device), \engquote{old albums} (P1), \engquote{vinyl record} (P16), etc. 
% They hold a consensus that MemoTool is comprehensive for various physical mementos and their specific functions. 

% % 没有单独的去做从实物到建模的过程,添加骨骼也是做一些动画,不能支持实时的手势交互
% % workflow
% Our participants found the overall workflow of MemoTool, from 3D scanning to interaction mapping, to be clear and intuitively understandable for everyday users.
% One of our participants (P8) possessed professional expertise in modeling and anIDItion and he pointed out that the workflow of MemoTool closely resembles the professional modeling procedures, encompassing steps such as \engquote{1) Sketching, 2) Modeling (Scanning), 3) Applying textures, and 4) Rigging (creating skeletal systems).} 
% In his feedback, P8 emphasized that: \engquote{MemoTool is a user-friendly application thoughtfully tailored for everyday users, eliminating the requirement for advanced sketching skills or the burden of mastering complex cross-domain software.}

% % new functions
% Additionally, our participant also suggested expanding new functions for MemoTool. 
% For example, the function of using tangible widgets to control the mechanical components in AR (e.g., pushing a button could trigger joint movement.). 
% As demonstrated by P2, \engquote{IDIgine a vinyl record player situation where a button could initiate music playback and start the vinyl spinning.}
% Overall, we found the customization and the design space to be rich in MemoTool.




% \subsubsection{Qualitative Feedback --- Realism created by MemoTool}
% \revision{a sticker on the toy indicates the ownership }
% % a. whether the memotool could help the user to simulate the functions of the items 
% % b. tactile feedback
% % c. hand interactions (should be add to the implementation)
% % \engquote{sss}

% All participants agreed that they were able to create the IDI from physical artifacts and add real-world features to them with the support of the functions of MemoTool. 
% They all mentioned that the IDI created in each task effectively represented the physical memento in reconstructing real-world features, which enhanced its \engquote{memorability} (P1) and \engquote{longevity} (P4) of digital mementos. 
% % For instance, P11 noted, \engquote{The overall workflow of MemoTool helped me complete the reconstruction of a physical item.}
% P2 also emphasized that the reconstruction of the mechanical component truly constituted the impressive physical features of the toy Stitch, stating, \engquote{This significantly added enjoyment to the digital reconstructed memento, transforming it into a more practical and enjoyable digital memento that exceeded my IDIgination.}
 
% Furthermore, we observed that the sense of realism was largely derived from the bare-hand interactions with the virtual objects. 
% Many participants (N=12) expressed their amazement at being able to use hands to interact with virtual objects in AR as if they were in the physical world. 
% As P7 mentioned, \engquote{When I slightly touched the digital Stitch, its head moved just like the real toy Stitch.} 
% Seven participants who had the experiences with VR/AR, pointed out that bare-hand physics-simulating interaction differed from their previous experiences in the immersive environment: \engquote{Typically, hand interactions in VR/AR do not simulate physics; they are often used for manipulating objects or selecting targets, rather than making objects behave realistically.} (P9).

% We also found that the lack of tactile feedback has a negative impact on the participants’ perception of realism.
% Five participants commented that without the tactile feedback, they cannot accurately `press' the button on the virtual GameBoy, instead, they resorted to `poking' the buttons, diminishing the realism of the IDI. 
% One participant (P11) detailed the difference between pressing buttons in the real world and the virtual world. 
% In the real world, pressing buttons requires applying \engquote{force} to activate the internal mechanical structure of the button, and there is tactile feedback to confirm a successful press. 
% In contrast, in the virtual world, there is no tactile feedback when pressing buttons, which makes it difficult to gauge \engquote{how much force is needed by hands}. 
% This lack of tactile feedback can, to some extent, diminish the sense of realism for IDI.

% tactile feedback
% vritual buttons are hard to accurately touched
% cannot 'hold' the mesh and 'press' the button

% \subsubsection{Qualitative Feedback --- Ease of interactive approaches}
% a. mixed environment - mobile and AR
% b. the pre-designed functions to choose
% b. should preserve the DIY interface
% c. mesh cut method is good but there are some more smart solutions
% c. 3D manipulation on model
% d. redo


% The participants appreciated the overall interactive approaches of MemoTool. Specifically, 8 participants highlighted the seamless synchronization between mixed devices is organic and natural.
% On the one hand, mobile devices empowered the user of the freedom to upload the content as most digital files and data can be accessed through mobile devices.
% On the other hand, AR glasses maintained the real-world environment, allowing participants to \engquote{effortlessly switch to mobile devices} (P3) for uploading and editing content without compromising MemoTool's integrated interactivity. This versatility expands its potential application scenarios.

% When the participants designed the interactions for IDI (such as locating virtual widgets and attaching them to the model, or interacting with joints and mapping them onto the model), they found it convenient and easier to select options from a pre-designed set rather than creating their own. 
% An important concern that has been successfully tackled concerning participants’ non-technical backgrounds is that designing and authoring in AR no longer seems unattainable.
% P15 commented, \engquote{The presented buttons, especially the icons indicating their functions on it, facilitated my rapid identification of the target function and its attachment to the model.} P14 also suggested enhancing visual cues to the displayed joints within the scene, stated \engquote{It would be more beneficial if these showcased joints could incorporate anIDItions to indicate their relative movements, sparing me the need to interact with each one individually.} 

% Additionally, participants also put forth some suggestions of optimizations on interaction approaches.
% While the pre-set functions proved convenient for participant, there remained the possibility of unique cases. 
% Therefore, it is crucial to maintain an interface that allows end-users to program interactions for widgets or mechanical components themselves.
% The interactive segment method for models could also be optimized. 
% P10 offered feedback, stating that \engquote{utilizing a plane to cut the model appear too harsh for cherished mementos; perhaps marking three dots on the surface of the model to confirm a plane could be a gentler approach.} 



% a. mementos that could build using this tool
% b. mementos that could not build using this tool
% c. extend or create more functions on old items - update the functions、
% d. the classification is clear and comprehensive
% e. compared with 2d editing tool, 3d tool could behave more naturally and more fave to end-user community
% participants  
% 用户如何评价我们的原型系统提供的展示性的功能和内容?(上面写过了,第一个sec是关于function的,第二个是关于interaction的)
% 用户基于我们的系统希望扩展哪些功能/内容
% 我们现有的技术/系统框架是否支持用户扩展他们想要的内容?
% 用户对我们系统的泛化能力与可扩展性有何评价?
% \subsubsection{Qualitative Feedback --- Customization / Design space of MemoTool.}
% Participants praised MemoTool for its creativity and generality as a tool for creating IDI in terms of specific functions and the overall workflow in MemoTool. 
% % design space
% Fig. \ref{fig:toys} (b) and (c) show some of the IDI participants created in the exploratory session.
% For example, P13 used MemoTool to create a toy Transformer IDI and commented: \engquote{Such complex joints can be mapped to the virtual model by myself!} 
% Many participants mentioned physical mementos from the toys to digital devices that can be reconstructed, such as, \engquote{all the rigid toys} (P9), \engquote{the Tamagotchi}~\footnote{https://en.wikipedia.org/wiki/Tamagotchi} (P10) (a handheld digital pet device), \engquote{old albums} (P1), \engquote{vinyl record} (P16), etc. 
% They hold a consensus that MemoTool is comprehensive for various physical mementos and their specific functions. 

% % 没有单独的去做从实物到建模的过程,添加骨骼也是做一些动画,不能支持实时的手势交互
% % workflow
% Our participants found the overall workflow of MemoTool, from 3D scanning to interaction mapping, to be clear and intuitively understandable for everyday users.
% One of our participants (P8) possessed professional expertise in modeling and anIDItion and he pointed out that the workflow of MemoTool closely resembles the professional modeling procedures, encompassing steps such as \engquote{1) Sketching, 2) Modeling (Scanning), 3) Applying textures, and 4) Rigging (creating skeletal systems).} 
% In his feedback, P8 emphasized that: \engquote{MemoTool is a user-friendly application thoughtfully tailored for everyday users, eliminating the requirement for advanced sketching skills or the burden of mastering complex cross-domain software.}

% % new functions
% Additionally, our participant also suggested expanding new functions for MemoTool. 
% For example, the function of using tangible widgets to control the mechanical components in AR (e.g., pushing a button could trigger joint movement.). 
% As demonstrated by P2, \engquote{IDIgine a vinyl record player situation where a button could initiate music playback and start the vinyl spinning.}
% Overall, we found the customization and the design space to be rich in MemoTool.




% ------------------- virtual  community

% \subsubsection{Qualitative Feedback --- Expectations on the virtual community creation of IDI}
% \label{sec: Qualitative Feedback --- Expectations on the virtual community creation of IDI}
% % a. different roles in this community (developer, user, xx)
% % b. open sources for various assets (software, operating system, mesh/ model,)
% % c. extended functions xxx
% % D. appreciated the concept of IDI
% % e. different roles in this community (developer, user, xx) anIDItion professional 
% All of our participants expressed a positive overall reception of the Interactive Digital Memento concept with some expectations regarding IDI's future usage scenarios and visions, emphasizing its role in facilitating personal memory archiving efforts.
% A shared expectation among participants was the creation of a virtual IDI community. 
% This envisioned virtual community would serve as a platform for resource creation and sharing centered around IDI. 
% This community would facilitate collaboration among users from diverse backgrounds (e.g., modeling, anIDItion, programming, designing, etc.), allowing them to collectively build and share IDI content.
% This community should provide ready-made resources of \engquote{3D models of mementos, software, operating systems, complex mechanical components, etc.}  (P10) which users can readily access and integrate into their IDI projects. 
% For instance, seven participants expressed the desire to have more operating systems in the community for broader application across various devices. 
% Also, the virtual community should support additional editing functions for the ready-made resources to preserve the personal uniqueness of mementos for users, as P13 stated: \engquote{While it's convenient to directly use 3D models and other resources from others, the inclusion of further editing functions is essential to retain the unique personal traces in my IDI.}
% Concerns about the privacy of the virtual community was also mentioned by 5 participants, as they stated: \engquote{If it is possible to ensure that one's objects are not used indiscriminately and that privacy is guaranteed, then I'm happy to participate in discussions in the online community.}










% \textit{P15} said: \textit{``Being able to play games inside AR is invigorating, but it would be even cooler to be able to overlay different operating systems to play different old devices, like Nokia phones, Tamagotchi, that sort of thing.''}
% As P3 pointed out, \engquote{If I could find the complete operating system from the community for the old Nokia, I'd be delighted to use them, saving me the effort of creating an IDI for my old Nokia.}



% Participants came up with many ideas for building online interactive communities in the future, which aim to provide virtual object-creating and sharing functionalities between developers and users just like GitHub for programmers. 

% Firstly, end-users would like to have more scanned meshes to share and use with each other in the online community, as well as create interaction schemes, which can save the trouble of repeated scanning in many scenarios (N=14). 
% As \textit{P6} stated: \textit{``I want to be able to share models and to be able to write tools that create ways of interaction on the online community.''} 
% Besides, some participants (N=7) also expressed the expectation to share the operating systems in terms of achieving wider usage of old devices. \textit{P15} said: \textit{``Being able to play games inside AR is invigorating, but it would be even cooler to be able to overlay different operating systems to play different old devices, like Nokia phones, Tamagotchi, that sort of thing.''} 
% However, some participants (N=5) also expressed their concerns about privacy and uniqueness. 
% As \textit{P12} stated: \textit{``I would pay more attention to the signs of use of my items than sharing them in the online community. If everyone shares the same mesh, it seems to have less meaning to me.''}, and \textit{P4} said: \textit{``If it is possible to ensure that one's objects are not used indiscriminately and that privacy is guaranteed, then I'm happy to participate in discussions in the online community.''} 
% To summarize, building the virtual community of IDI could enhance the communication and operating efficiency for users, under certain privacy protection mechanisms.


% \subsubsection{Qualitative Feedback --- Ease of  using }
% (1) Object Scanning:是个很好用的app,有interactive feedback,但是也会受限于复杂度很高的物体 => LiDar Scanner的问题

% Overall, most participants (N = 14) felt that the experience and effectiveness of using object scanning met expectations. The 3D scanning of an object was written and efficiently done on the cell phone, and the real-time reconstruction was more user-friendly. \textit{P16: ``All I have to do is turn around the object and flip it over and turn it again, which I find very convenient.''} \textit{P15} also expressed appreciation for the interaction: \textit{``The real-time object reconstruction feedback was interesting to me and increased my confidence in accomplishing that task.''} However, some participants also proposed some concerns of reconstructing objects that have complex surfaces. As \textit{P11} said: \textit{``I felt it is hard for me to reconstruct objects that have many complex and spatially overlapping angles, and if an object has some joints, it might change the shape when tipping it down.''} To summarize, object scanning achieves considerable usability overall. In order to improve object reproduction, users should scan objects: (1) in a stable lighting environment, (2) choose objects with sharp edges, and (3) make sure that the object's shape remains unchanged when flipped as much as possible. 



% \subsubsection{Comprehensiveness of Tutorials}
% (2) 教程: 易于理解的程度(内容上整体上较为清晰 -- UI) 操作上的难度(每做一个task就要back to screen会造成一些麻烦),可以怎么改进(渐进式的教程,在各个components上有提示)

% \subsubsection{Button-Based Interactions}
% (3) 按键附着与功能选择: 整体功能高度还原,但是有几个问题:
% a. 容易误触,抓取与点击这两个事 --> hololens的问题
% c. mesh附着功能希望可以自动吸附
% d. 整体按键功能,有些用户希望可以fixed on screen, 而不是floating,容易覆盖或误触;也有些用户喜欢这个相对自由的interface

% \subsubsection{Physical Interactions}
% (4) 物理操作:切割这个方法很妙,但是依然有改进的空间
% a. 更用户友好的切割:例如:三点成面的切割方式
% b. 物理运动可视化很妙,但需要一些时间来学习

% \subsubsection{Trial \& Error}
% (5) trial and error 机制
% a. 【按钮】:点击后的反馈不明显,需要有个确认按钮
% b. 【切割】:切割只有一次机会,没法重复使用直到自己满意

% \subsubsection{Memory Management}
% (6) memory内容的上传 / app: 是用户友好的解决方案,但也有用户希望全部都在ar环境中完成

% \subsubsection{Glance to Online Communities}
% (7) 对community的展望:打造类似steam workshop式的community很有意义,希望能够有内容开发者提供各类轮子与内容;将内容扩展到数字人,乃至元宇宙中。





\section{Discussion}

\revision{In this section, we discuss the comparison to adversarial image-based techniques and address several challenges associated with real-world deployment.}

\revision{\noindent \textbf{Comparison to Adversarial Attacks.} Adversarial image-based techniques typically rely on the addition of carefully crafted noise to images. However, recent studies~\cite{wei2022towards} indicate that these methods often lack transferability and can be easily defeated by a novel LLM with enhanced visual capabilities. Our experimental results demonstrating the LLM's effectiveness in identifying adversarial images is available on our website~\cite{ourwebsite}.}

\revision{\noindent \textbf{Challenge of Cross-cultural Adaptability.} Our experiments reveal that individuals from different countries and age groups may exhibit varying abilities in identifying illusionary images due to cultural differences. To mitigate this issue, we propose incorporating common, everyday images—such as those of fruits, restaurants, and landscapes—to create illusionary images that are universally recognizable. By leveraging familiar objects, we aim to minimize the impact of cultural differences and ensure a consistent user experience across diverse demographics.}

\revision{\noindent \textbf{Challenge of Image Copyright.} In real-world deployment, copyright concerns may render certain images or terms (e.g., \textit{Mickey Mouse}) unsuitable for use. To mitigate these issues, we plan to employ a local AI system to generate images while carefully avoiding problematic words. This approach enables the creation of copyright-free images, thereby ensuring smoother and more compliant deployment in practical scenarios.}

\section{Conclusion}
In this paper, we proposed an automatic pipeline, GistVis, for generating word-scale visualization using LLMs. We informed our design with a formative study across 44 data-rich documents. We designed GistVis modularly to support plug-and-play property for expansion while we steer LLMs with visualization knowledge to generate word-scale visualizations to support document-centric analysis. Our technical evaluation and user study reveal that GistVis could generate satisfactory word-scale visualizations that could reduce users' workload reading data-rich documents. We discuss the limitations of our current approach and outline future directions. We believe that GistVis is a timely contribution to inspire further study from the visualization community about using automatic methods, especially LLMs, to augment data-rich documents.



\begin{acks}
This work was supported in part by the Shanghai Student Innovation Training Program \#S202410247252. We want to thank all participants for taking part in our study. Additionally, we extend our thanks to the anonymous reviewers for their constructive feedback.
\end{acks}

\balance

\bibliographystyle{ACM-Reference-Format}
\bibliography{bibs/auto-docuvis, bibs/web-link}



\end{document}
