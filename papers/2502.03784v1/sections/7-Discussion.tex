\section{Discussion and Future Work}
Our work explored using the capabilities of LLMs to generate interactive word-scale in situ visualizations from data-rich documents. In this section, we describe the limitations of GistVis and envision future research opportunities.


\paragraph{Understanding the Effectiveness of Word-scale Visualizations with Finer Granularity}
Our findings suggested that GistVis is effective in helping users understand data-rich documents following a document-centric strategy. However, the design of our evaluation limited us in obtaining finer-grained information about which part of the word-scale visualization made them particularly effective. Specifically, our qualitative data from the semi-structured interviews and the interaction log revealed interesting patterns that require further investigation. For example, we did not find strong evidence of the effectiveness of the drill-down operation with the tooltip, as no participant mentioned using those as part of their strategy. Meanwhile, our interaction log revealed that users interact with few word-scale visualizations pertinent to the question. However, exactly what they gazed at before their decisions remained unclear. Future work could use eye-tracking~\cite{huth2024eye} to explore the gaze pattern of users to understand their underlying strategy processing GistVis augmented documents.


\paragraph{Lifting the Constraints of GistVis}
We built GistVis around a series of rather stringent constraints. Initially, we defined several constraints based on the results of our formative study on data-rich documents before implementation (Sec.~\ref{subsec:formative-constriants-implications}). While this practice allowed us to conduct a proof-of-concept study, we inevitably ignored several conditions and cases in the wild that might be equally important to the conditions we supported. Additionally, we incorporated further constraints through the design of our computational pipeline. For instance, we assumed that the same data insight should be located in proximity to each other and within the same paragraph. However, there are instances where the same data insight may span across multiple paragraphs or even the entire passage. While we contend that our system supports the majority of scenarios appropriate for document-centric analysis with word-scale visualization, especially professionally written data documents (e.g., \cite{us-k12-education}), future research should aim to flip these constraints and expand GistVis to encompass a wider range of conditions.

\paragraph{Collect a Comprehensive Dataset of Data-rich Documents for Analysis and Training}
Before designing GistVis, we collected a corpus containing 44 data-rich documents to conduct a formative analysis of data-rich documents. We also utilized this annotated dataset to quantitatively evaluate the performance of GistVis. However, our corpus is far from comprehensive because we only included the genre of data journalism. Other data-rich documents, such as business reports and scientific papers, might have different narrative features to data journalism and should be thoroughly analyzed to build a firm foundation to expand the capability of our GistVis pipeline. Moreover, a larger, more reliable, and more comprehensive dataset could also benefit the design and evaluation of better algorithmic approaches. For example, future work could implement better in-context learning strategies, such as Active Learning inspired strategies (e.g., In-context Sampling~\cite{yao2024more}), to improve the few-shot capabilities of LLMs. We advocate future research to propose methods to collect and generate large-scale datasets for visualization~\cite{wu2022ai4vis} so that visualization researchers can better understand the narrative features of word-scale visualizations and improve automatic algorithms for word-scale visualization generation.

\paragraph{Expand the Design Space of Word-scale Visualizations}
Our current implementation of GistVis only contains three different chart types with 14 visualization variants, including icons. More chart types and different variants tailored for unique narrative features should be implemented to further improve the usability of the pipeline. For example, in some cases, pie charts might better convey proportional insight than the horizontal stacked bar chart we applied in this work. Future work could include pie charts in our design space while working on heuristics or algorithmic approaches to guide GistVis in selecting the better representation based on the text context. In addition to charts and icons, future work could also discuss using novel typefaces~\cite{nacenta2012fatfonts} or tiny graphics~\cite{zhao2020iconate}. We argue that the modular design of GistVis would enable us to easily expand the search space of word-scale visualizations.

\paragraph{Improve Interactivity}
Although GistVis supports basic interactivity, the scope of interaction is limited within a unit segment. One typical instance that reflects this limitation is when the same entity exists across unit segments: our current implementation would likely label the same entity with different colors. Such practice would potentially lead to confusion about the document's content. Moreover, there might be multiple word-scale visualizations related to the same entity, and our current approach could not synchronously show all the entity's related data. Future work should seek to employ universal control over the interactive components, including a color mapping system for entity label consistency and synchronous interaction for the same entities. 

Additionally, although we focus on the document-centric analysis of data-rich documents, users might benefit from being able to deploy both document-centric and visualization-centric analysis. For example, for data insights that span multiple paragraphs, interaction techniques such as view manipulation~\cite{heer2012interactive} could reorganize word-scale visualizations to support the transition between document-centric analysis and visualization-centric analysis~\cite{goffin2020Interaction}. Additionally, some data-rich documents may already have figure-size visualizations that support document-centric analysis. Although one participant (P10) believes GistVis could also be effective even when figure-size visualizations exist, we need more evidence to ascertain its effectiveness and determine a strategy to combine both forms of augmentation. Thus, future work should also explore interaction techniques to synchronize text, word-scale visualizations, and figure-size visualizations to enable users to benefit from document-centric and visualization-centric analysis.


\paragraph{Integrating GistVis into Existing Workflow}
Participants expressed a keen interest in using GistVis if it were integrated into their existing workflow. While this work validates the potential of using automatically generated word-scale visualizations from GistVis, extensive engineering efforts are required before it can be fully supported in real-world workflows. GistVis holds the potential to be applied in a wide array of use cases, such as functioning as a plug-in for document readers or enhancing the capabilities for visual analysis tools~\cite{ava-ntv-2024}.





