\section{Related Work}
We review three categories of previous works pertinent to our research question of designing an automatic method to generate word-scale visualizations for document-centric analysis. Specifically, we review 1) the definition, design, and application of word-scale visualizations, 2) related automatic visualization generation methods, and 3) how visualizations and texts could be integrated to enrich the reading experience.

\subsection{Word-scale Visualization}
Word-scale visualization has been discussed in the literature under several expressions, such as sparklines~\cite{tufte2006beautiful} and word-scale visualization~\cite{goffin2014exploring}.
\citet{tufte2006beautiful} introduced sparklines as concise, high-impact graphics that fit a text's typographic space. \citet{goffin2014exploring, goffin2017Exploratory} expanded Sparklines with word-sized visualizations, allowing more flexible integration of graphics and text.
Though terms may differ, one common theme between sparklines and word-scale visualizations is using text-sized graphics to augment existing documents. This study uses word-scale visualization to cover word-scale data visualization, text-sized typography, and text-sized graphics. 

In addition to discussing the word-scale visualization design space, researchers have also explored the potential of the application of word-scale visualizations~\cite{hoffswell2018Augmenting, perin2013soccerstories, brandes2013gestaltlines, beck2016visual}. For example, \citet{hoffswell2018Augmenting} implemented a design space that uses word-scale visualizations to augment the code reading experience. The SportLines interface~\cite{perin2013soccerstories} used word-scale visualizations to show the phases of players to support visual soccer analysis. These studies are formative to GistVis, as we employ the best practices integrating word-scale visualizations with text. Although prior work has explored many application scenarios, few have automated word-scale visualization generation for data-rich documents. This work contributes to the research of word-scale visualizations by proposing an automatic pipeline that generates word-scale visualizations to augment existing data-rich documents.

\vspace*{-2pt}
\subsection{Automatic Visualization Generation}
Research has explored various automatic visualization techniques, especially using tabular data as input~\cite{wang2020datashot, shi2021calliope, dibia2023lida}. For example, \citet{wang2020datashot} proposed the Datashot system that included a fact sheet generation pipeline based on a taxonomy of data facts~\cite{chen2009effective}. %
More recently, \citet{dibia2023lida} proposed the LIDA (Visua\underline{li}zation an\underline{d} Infogr\underline{a}phics) tool that incorporated large language models (LLMs) in the infographics generation process from tabular datasets. 


A subcategory of automatic visualization revolves around generating visualizations based on textual contexts~\cite{wu2022ai4vis}. Under this category, we can broadly identify two ways of automatic visualization generation:
1) using external data sources and 2) using internal data sources (text description, tables, etc.). Using external data sources, NewsViews~\cite{gao2014newsviews} generated interactive visual maps through a table database. Contextifier~\cite{hullman2013contextifier} automatically created a stock timeline graph with annotations through sourcing external news corpus and APIs. Using internal data sources, Text-to-Viz~\cite{cui2020texttoviz} used regular expressions to detect data elements for generating proportional infographics. Charagraph~\cite{masson2023Charagraph} created in situ visualizations from data-rich paragraphs to support exploratory analysis of statistical data described in the text. However, those methods primarily used rule-based approaches to extract data insights from data-rich documents, and we need further work to capture semantically implied data insights in text descriptions.

GistVis categorizes as a context-based automatic visualization technique and uses data solely from internal sources. GistVis extends previous automatic visualization research by utilizing LLMs as a proxy to capture and extract data insights from text descriptions.
We employed a data fact-based approach~\cite{chen2009effective} inspired by the automatic methods mentioned above and engineered LLMs to use such visualization knowledge to generate word-scale visualizations.

\vspace*{-2pt}
\subsection{Visualization + Text}
\label{subsec:relatedwork-vistext}
Research has demonstrated the effectiveness of dynamically connecting text segments to visualizations to enhance the reading experience~\cite{latif2022kori, bromley2024dash, masson2023Charagraph}. Prior work has proposed crowdsourcing or mixed-initiative methods to create text-visualization connections. For example, \citet{kong2014extracting} used crowdsourcing to extract references between charts and text. For mixed-initiative methods, \citet{latif2022kori} proposed the Kori system, an interactive authoring tool synthesizing text and visualization. The DASH (\underline{D}ata \underline{A}nalysis using \underline{S}emantic \underline{H}ierarchies) system~\cite{bromley2024dash} leveraged LLMs to introduce semantic levels~\cite{lundgard2022accessible} into the bidirectional analysis between text and charts. The Charagraph system~\cite{masson2023Charagraph} used regular expressions to extract statistical data from a user-selected text domain to facilitate the understanding of statistical data in data-rich documents.

GistVis incorporates the practice of linking text with visualization by supporting two types of text-visualization interaction: 1) linking entities in text descriptions to visual elements in word-scale visualizations and 2) a hover tooltip that provides essential data insights of the selected word-scale visualization. Meanwhile, GistVis extends prior research by implementing text-visualization interaction on a word-scale setting. We expect the application of text-word scale visualization interaction could provide additional data context for users while reading. 
