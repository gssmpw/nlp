\begin{table}[t]
\centering
\small
\caption{Clean and attacked AUROC of HALO when paired with different choices of post-processor. CIFAR-10 is used as the ID dataset and average AUROCs are reported over OpenOOD datasets. The baseline method, MSP \cite{hendrycks17baseline}, is in \textit{italics} and highest results are in \textbf{bold}. More sophisticated methods such as Energy-based OODD \cite{liu2020energy} and Generalised ENtropy \cite{liu2023gen} perform noticeably better than the baseline method.}
\begin{tabular}{lcccc}
\toprule
\multirow{2}{*}{\textbf{Model}} & \multicolumn{4}{c}{\textbf{Avg. AUROC}} \\
\cmidrule(lr){2-5}
 & \textbf{Clean} & \textbf{ID→OOD} & \textbf{OOD→ID} & \textbf{Both} \\
\midrule
%HALO + MDS & 50.64 & 55.90 & 52.32 & 57.50 \\
HALO + RMDS & 83.23 & 69.65 & 74.39 & 59.61 \\
HALO + SHE & 85.79 & 74.49 & 74.98 & 60.80 \\
\textit{HALO + MSP} & \textit{89.87} & \textit{76.80} & \textit{78.52} & \textit{61.82} \\
HALO + ODIN & 90.54 & 77.72 & 79.17 & 62.42 \\
HALO + OpenMax & 90.64 & 77.97 & 79.48 & 62.49 \\
HALO + Energy & 91.12 & \textbf{78.17} & \textbf{79.96} & \textbf{62.66} \\
\textbf{HALO + GEN} & \textbf{91.13} & \textbf{78.17} & \textbf{79.96} & \textbf{62.66} \\
\bottomrule
\end{tabular}
\label{tab:postprocessor_results}
\end{table}