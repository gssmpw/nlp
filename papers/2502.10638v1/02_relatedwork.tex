\section{Related Work}

\subsection{Process of Writing}
Writing is a non-linear process involving three key phases: planning, translating, and reviewing~\cite{hayes1980identifying}, with writers often moving back and forth between these processes\change{~\cite{flower1980cognition}}. Observational studies have shown that writers generate high-level goals and supporting sub-goals as they develop their sense of purpose, and often adjust these goals based on new insights gained during the act of writing\change{~\cite{flower1980cognition}}. These cognitive process observations reflect the idea that writing, much like other creative design tasks such as art or music, involves learning through a reflective conversation with the materials being produced~\cite{bamberger1983learning}.

Revision is a critical, recursive process in writing that benefits from supportive interfaces~\cite{fitzgerald1987research, seow2002writing}. Advanced writing often transitions from the basic knowledge-telling model, which relies on cues from topics and discourse schemas, to a more complex, knowledge-transforming model~\cite{bereiter1987psychology}, where coherence and rhetorical development become central. The act of writing requires continuously updating one's rhetorical situation, including considerations of audience and purpose~\cite{flower1980cognition}. This insight emphasizes that having strong ideas does not automatically result in well-written prose\change{~\cite{flower1980cognition}}, necessitating tools that assist writers in refining both content and structure.

Existing writing interfaces, typically designed around a single text editor layout, have been extensively studied~\cite{whiteside1982people, egan1982learner}. However, general-purpose word processors often fail to address the specific needs of professionals and creatives, who require more robust tools for maintaining consistency, managing dependencies, and organizing related information in structured documents~\cite{han2020designing}. These tools force users to rely heavily on memory to manage tasks crucial to advanced writing, such as tracking coherence across sections and adapting content to evolving goals~\cite{han2020textlets, han2020designing}. Our work builds upon these insights, addressing the gaps in current writing tools by introducing \system, a layered interface paradigm designed to support non-linear, iterative writing processes. 



\subsection{LLM Co-Writing Interfaces}
In exploring LLMs for writing~\cite{brown2020language, kenton2019bert, min2023recent, radford2018improving}, substantial work has demonstrated various ways in which LLM co-writing interfaces can be designed~\cite{Lee2024design}. While chat-based LLM interfaces have become ubiquitous and accessible~\cite{google2023gemini, openai2022chatgpt, anthropicclaude}, there remains a significant gap in effectively prompting these systems for usable responses and guiding the iterative process of generating meaningful textual outputs~\cite{zamfirescu2023johnny, kim2023towards}. Current LLM systems often lack the nuanced supports needed for writers to steer the process and maintain control over their creative work.


Several LLM co-writing systems have emerged to bridge this gap~\cite{biermann2022tool, buschek2021impact, chung2022talebrush, dang2023choice, gero2022sparks, gero2023social, giray2023prompt, goodman2022lampost, kim2023towards, lee2022coauthor, longtweetorial, mirowski2023co, shakeri2021saga, singh2023hide}. \change{While some tools focus on specific aspects of the writing process, such as brainstorming~\cite{petridis2023anglekindling, calderwoodQGC20}, content transformation~\cite{du2022read, Arnold2021GenerativeMC}, and feedback~\cite{weber2024legalwriter, hui2023lettersmith}, others offer broader support across multiple writing tasks~\cite{reza2023abscribe, kim2024diarymate, gero2022sparks, Arnold2021GenerativeMC}.}

Despite their utility, there are persistent concerns about content ownership~\cite{hoque2024hallmark}, and studies suggest that users who rely less on LLMs tend to produce writing with higher lexical density and coherence~\cite{shibani2023visual}. Furthermore, users often desire greater control over the narrative and the AI's output~\cite{ippolito2022creative, poddar2023ai}, as LLM tools have been shown to influence the writer’s opinions, raising concerns about the extent to which language models shape user thinking~\cite{jakesch2023co}. Additionally, while LLMs excel at generating content, they often fall short in creative tasks, which is why systems should focus on preserving the user's creative control, as seen in our work with \system~\cite{chakrabarty2024art}.

Recent discussions have also explored the implications of students using LLMs for academic writing, which may hinder the development of critical writing skills~\cite{bowman2022new, meyer2023chatgpt, flower1980cognition}. There is a growing consensus that the human writer should lead the process, with the LLM handling more mundane tasks~\cite{wan2022user}. This perspective highlights the importance of designing systems where AI augments, rather than dominates, the writing process.

There are also opportunities to explore tools that provide constrained support for late-stage brainstorming, planning, and reviewing, particularly for specific writing genres such as storytelling~\cite{gero2022design}. Additionally, designing AI systems that help writers understand the model’s capabilities, adapt to individual writing styles, and offer authentic feedback will be vital~\cite{gero2023social}. \change{There is a growing trend toward collages for the design of AI writing tools. \citet{buschek2024collage} identifies four key dimensions of collage-based designs: fragmentation of content, juxtaposed voices, multiple sources, and shifting roles. In their analysis, VISAR ~\cite{zhang2023visar} supports the Collage form factor best (Collage Factor=10). VISAR propose a system for argumentative writing that combines visual programming and rapid prototyping. While VISAR implements several collage dimensions, its single-page writing interface limits quick comparisons of variations. The design framework in \citet{kim2023cells} also uses text fragmentation where each separate fragment can transform or generate text. Also, there are more spatial interfaces for writing~\cite{lin2024rambler, lu2018inkplanner, park2023designing}  that satisfice the collage factor~\cite{buschek2024collage}.}

\change{Our work with \system builds on \citet{zhang2023visar}, supporting visual representation for hierarchical planning. It uses text fragmentation in the form of cells (or layers in our case) from \citet{kim2023cells}. It separates LLM content from human writing, per \citet{hoque2024hallmark}. Finally, it draws from the analytical, constructive, and critical aspects of collages \cite{buschek2024collage}. We also borrow the concept of LLM personas from ~\citet{benharrak2024writer} to improve human-centered writing. Together, these papers help us address the gaps in control and flexibility in LLM co-writing interfaces. \system uses a layered, zoomable workspace. It lets writers interact with LLMs in a non-linear, iterative way. This supports complex writing tasks and divergent thinking. It also keeps users in control of the content and structure.}