\section{Discussion}
In this section, we reflect on the key findings from the development and evaluation of \system, while also exploring future opportunities for research into accessible media authoring tools.

\subsection{Scripting \& Shifting with LLMs}

\change{Study findings indicate that a significant advantage of \system is its ability to support non-linear and divergent writing workflows, enabling users to overcome the limitations of traditional page-based tools. Unlike conventional writing applications that enforce a rigid, sequential structure, \system employs a \textbf{layered paradigm} inspired by the concept of collage \cite{buschek2024collage} and spatial interfaces. Arguably, this approach aligns with how individuals naturally think -- separating, refining, and integrating complex ideas -- and facilitates a more fluid and exploratory writing process. At the core of \system’s layered paradigm is the ability to \textbf{script and shift} between various writing strategies seamlessly. Writers can transition effortlessly between planning, editing, structuring, and reviewing, akin to assembling a collage where different elements are combined, rearranged, and refined to create a cohesive whole. The \textbf{Writer’s Friends} further enhance this experience by providing specialized functionalities tailored to distinct phases of the writing process. Each layer functions as a cognitive sandbox, dedicated to tasks such as brainstorming with Idea Ivy, expanding ideas using Detail Danny, or reorganizing content through Structure Sam. This modularity preserves creative control and encourages users to experiment with different rhetorical strategies and content structures without the constraints of a linear workflow. Further, it provides writers with affordances for iterative and divergent writing process, essential for creative work \cite{sawyer2021iterative}}

\change{A key feature that participants found compelling was \textbf{non-destructive editing}. \system allows layers to be folded, discarded, or rearranged without permanently deleting any content, providing writers with the freedom to experiment without the fear of losing their work. This capability is crucial for exploring various tones, audience alignments, and structural organizations. The \textbf{spatial organization} of content within \system also plays a key role in supporting creative workflows. By visualizing content in a layered, spatially oriented manner, writers can perform cross-content comparisons, experiment with folding or reorganizing layers, and engage in ``tear, combine, and peek'' sequences of transformations. For instance, participant P3 described an iterative process of tearing apart a layer, reordering its fragments, and recombining them to evaluate the impact on the essay’s overall flow. This spatial manipulation mirrors the collage-like assembly of ideas, allowing writers to focus on individual aspects of their work while simultaneously exploring broader structural or stylistic variations. Participants appreciated the ease with which they could swap out layers and assess the influence of their presence or absence on the narrative, facilitating a deeper understanding of how different elements interact within their writing.}

\change{In summary, \systems integration of the layered paradigm and collage-inspired affordances creates a robust framework for non-linear and divergent writing workflows. By supporting non-linear, spatially-oriented content management, \system reduces cognitive friction to 'writing ballistics'~\cite{peakurland} and fosters an environment conducive to both exploratory and goal-directed writing. This alignment between user mental models and system functionality underscores the importance of designing writing interfaces that prioritize spatial organization and direct manipulation of text.}

\subsection{Broader Utility and Non-Linear Workflows}
While \system has demonstrated its value in supporting creative writing workflows, its broader utility extends beyond individual use cases. One particularly promising area is in education, where the layered interface and LLM-assisted support could transform how writing is taught and practiced. \systems ability to scaffold \cite {pea2018social} the writing process, from brainstorming and outlining to drafting and revising, makes it an ideal tool for educators seeking to help students develop their writing skills in a more exploratory, iterative manner aligned with the cultural practices of expert writers.

Beyond education, \system holds potential for other forms of content creation, particularly where non-linear thinking is key. In areas like multimedia storytelling, screenwriting, or journalism, \system’s layered interface could be used to storyboard, organize interviews, or manage multimedia elements. Alternative designs building on the \system metaphor could be developed for other cognitive design activities such as songwriting, screenplay writing, and movie scripting. The ability to work on different parts of a project in parallel, without being tied to a linear structure, could greatly enhance workflows in these disparate fields. For instance, a journalist might use \system to organize interviews, drafts, and research into layers, while screenwriters could draft scenes in a non-linear order, rearranging them to find the optimal narrative structure.

The findings from this study demonstrate that \system effectively supports non-linear, divergent writing processes through its layered, LLM-enhanced workspace. A key strength of LLMs, as evidenced in our study, is their ability to generate \textit{diverse and unexpected alternatives}, allowing users to experiment with multiple approaches in real-time. This makes \systems layered paradigm potentially useful in other domains such as policy development, screenwriting, and even generative programming (e.g.,~\cite{angert2023spellburst}). For instance, in policy writing, \system’s ability to help users generate and compare alternative argument structures, draft revisions, and outline complex documents in layers offers significant advantages. Screenwriters could leverage LLMs to explore multiple versions of dialogue, scene settings, or character interactions, helping them iterate on scripts faster while maintaining creative control over the process. They could use template features in \system to experiment with different narrative arcs while developing their character and scene structures. By enabling users to shift between drafting, editing, and comparing different versions of their work, \system provides a versatile platform that supports not only creativity but also structured, iterative workflows across diverse fields.

\subsection{Generalizability}
\change{As discussed above, by decoupling content development and structural organization into modular layers, \system can accommodate diverse workflows, from policy analysis to creative media production. This \textit{separation of concerns} allows the architecture to support non-linear and iterative processes, which are crucial for different genres, purposes, and user preferences. By structuring content as discrete, manipulable layers, the architecture facilitates intuitive spatial reasoning, enabling users to experiment with content and structure without compromising the coherence of their overall work enterprise. Interfaces based on this architecture could move beyond traditional document-centric models to offer dynamic, zoomable workspaces where users interact with content at multiple levels of abstraction, seamlessly shifting between granular edits and high-level structural organization.} 



\change{Additionally, the architecture's extensibility through ``Writer's Friends'' makes it adaptable to varied contexts and rhetorical goals. For example, AI agents could operate within specific layers, offering suggestions tailored to the scope of the layer (e.g., tone adjustments for a narrative layer or argument alignment for a reasoning layer). At a higher level, cross-layer AI could evaluate coherence, transitions, or overarching goals, supporting interfaces that bridge the micro and macro aspects of complex projects. Finally, the architecture could also support interfaces optimized for collaborative environments, where individual contributors interact with distinct layers representing their areas of focus, supporting both autonomy and coherence.}




\subsection{Limitations and Future Work}
While \system demonstrates clear advantages in supporting flexible writing workflows, some limitations emerged during the study. One challenge lies in the initial learning curve associated with the layered interface. Some participants, particularly those with less experience using non-traditional writing tools, found the spatial organization and multi-layered features overwhelming initially. Future iterations could include enhanced onboarding and tutorials to progressively ease users into mastering the system’s capabilities. A second limitation relates to the variability in how participants used Writer’s Friends. While most found them helpful, some struggled to select the appropriate friend for their task, or felt that the generated content did not always align with their specific writing goals. This finding suggests room for improvement in tailoring the system’s AI outputs to a wider variety of writing styles and preferences. Finally, although \system is designed to preserve user voice and reduce cognitive load through in-line AI assistance, there remains a risk of LLM bias in the generated content. For instance, a few participants were skeptical of ``Research Ramesh,'' citing distrust in LLMs due to their tendency to fabricate information. To address this, the system could require users to provide context documents, restricting LLM responses to verified excerpts. While the current implementation relies on commercial LLM safety measures, integrating chain-of-thought reasoning could further reduce hallucination~\cite{ji2024chain} and bias by ensuring more transparent and grounded outputs. While the separation of user input and AI-generated content helps maintain transparency, future versions could explore ways to provide more control over LLM-generated outputs and increase the transparency of how content is produced.




