%%
%% This is file `sample-manuscript.tex',
%% generated with the docstrip utility.
%%
%% The original source files were:
%%
%% samples.dtx  (with options: `manuscript')
%% 
%% IMPORTANT NOTICE:
%% 
%% For the copyright see the source file.
%% 
%% Any modified versions of this file must be renamed
%% with new filenames distinct from sample-manuscript.tex.
%% 
%% For distribution of the original source see the terms
%% for copying and modification in the file samples.dtx.
%% 
%% This generated file may be distributed as long as the
%% original source files, as listed above, are part of the
%% same distribution. (The sources need not necessarily be
%% in the same archive or directory.)
%%
%% Commands for TeXCount
%TC:macro \cite [option:text,text]
%TC:macro \citep [option:text,text]
%TC:macro \citet [option:text,text]
%TC:envir table 0 1
%TC:envir table* 0 1
%TC:envir tabular [ignore] word
%TC:envir displaymath 0 word
%TC:envir math 0 word
%TC:envir comment 0 0
%%
%%
%% The first command in your LaTeX source must be the \documentclass command.
%%%% Small single column format, used for CIE, CSUR, DTRAP, JACM, JDIQ, JEA, JERIC, JETC, PACMCGIT, TAAS, TACCESS, TACO, TALG, TALLIP (formerly TALIP), TCPS, TDSCI, TEAC, TECS, TELO, THRI, TIIS, TIOT, TISSEC, TIST, TKDD, TMIS, TOCE, TOCHI, TOCL, TOCS, TOCT, TODAES, TODS, TOIS, TOIT, TOMACS, TOMM (formerly TOMCCAP), TOMPECS, TOMS, TOPC, TOPLAS, TOPS, TOS, TOSEM, TOSN, TQC, TRETS, TSAS, TSC, TSLP, TWEB.
% \documentclass[acmsmall]{acmart}

%%%% Large single column format, used for IMWUT, JOCCH, PACMPL, POMACS, TAP, PACMHCI
% \documentclass[acmlarge,screen]{acmart}

%%%% Large double column format, used for TOG
% \documentclass[acmtog, authorversion]{acmart}

%%%% Generic manuscript mode, required for submission
%%%% and peer review
% \documentclass[manuscript,screen,review, anonymous]{acmart}
\documentclass[sigconf]{acmart}
%% Fonts used in the template cannot be substituted; margin 
%% adjustments are not allowed.
%%
%% \BibTeX command to typeset BibTeX logo in the docs
\AtBeginDocument{%
  \providecommand\BibTeX{{%
    \normalfont B\kern-0.5em{\scshape i\kern-0.25em b}\kern-0.8em\TeX}}}

%% Rights management information.  This information is sent to you
%% when you complete the rights form.  These commands have SAMPLE
%% values in them; it is your responsibility as an author to replace
%% the commands and values with those provided to you when you
%% complete the rights form.
% \setcopyright{acmcopyright}
% \copyrightyear{2018}
% \acmYear{2018}
% \acmDOI{XXXXXXX.XXXXXXX}

% %% These commands are for a PROCEEDINGS abstract or paper.
% \acmConference[Conference acronym 'XX]{Make sure to enter the correct
%   conference title from your rights confirmation emai}{June 03--05,
%   2018}{Woodstock, NY}
% %
% %  Uncomment \acmBooktitle if th title of the proceedings is different
% %  from ``Proceedings of ...''!
% %
% \acmBooktitle{Woodstock '18: ACM Symposium on Neural Gaze Detection,
%  June 03--05, 2018, Woodstock, NY} 
% \acmPrice{15.00}
% \acmISBN{978-1-4503-XXXX-X/18/06}

\usepackage{xcolor}
\newcommand{\hari}[1]{{\textcolor{purple}{\bf [*** HS: #1]}}}
\newcommand{\roy}[1]{{\textcolor{blue}{\bf [*** RP: #1]}}}
\newcommand{\momin}[1]{{\textcolor{teal}{\bf [*** MS: #1]}}}
\newcommand{\change}[1]{{\textcolor{black}{#1}}}

\usepackage{xspace}
\newcommand{\system}{Script\&Shift\xspace} 
\newcommand{\systems}{Script\&Shift's\xspace}

\usepackage{fontawesome}
\usepackage{tikz}
\usepackage{graphicx}
\usepackage{subcaption}

\definecolor{skyblue}{RGB}{135, 206, 235}
\definecolor{lightestgray}{RGB}{230, 230, 230}
\definecolor{legalyellow}{RGB}{255, 255, 204}

\definecolor{ToneTara}{RGB}{52, 130, 206} % Blue
\definecolor{IdeaIvy}{RGB}{128, 0, 128}  % Purple
\definecolor{DetailDanny}{RGB}{249, 186, 97} % Light Orange
\definecolor{FeedbackFelix}{RGB}{34, 211, 238} % Cyan
\definecolor{AudienceAli}{RGB}{255, 102, 26} % Orange
\definecolor{StructureSam}{RGB}{123, 183, 116} % Green

% \newcommand{\pageiconwithnumber}[3][black]{%
%   \begin{tikzpicture}
%     \node[inner sep=0pt, text=#1] {\faFile}; % Draw the page icon
%     \node[align=center, inner sep=0pt, text=#2] at (0,0) {\normalsize \textbf{#3}}; % Overlay the large number
%   \end{tikzpicture}%
% }

\newcommand{\coloredcircle}[1]{%
  \tikz\draw[fill=#1, draw=none] (0,0) circle (1.2ex); % Larger radius for the circle
}


\newcommand{\pageiconwithnumber}[3][black]{%
  \raisebox{-.1\height}{% Lower the icon to align it with the text bottom
    \begin{tikzpicture}
      \node[inner sep=0pt, text=#1] {\faFile}; % Draw the page icon
      \node[align=center, inner sep=0pt, text=#2] at (0,0) {\normalsize\textbf{#3}}; % Overlay the number
    \end{tikzpicture}%
  }%
}


% Define the command for text with a dashed border and customizable colors
\newcommand{\customdashedbordertext}[3]{%
  \tikz[baseline=(X.base)] \node[draw=#2, rectangle, dashed, fill=#3, fill opacity=0.1, text opacity=1, inner sep=2pt, rounded corners] (X) {\color{black}#1};%
}

\newcommand{\customsolidbordertext}[3]{%
  \tikz[baseline=(X.base)] \node[draw=#2, rectangle, fill=#3, fill opacity=0.1, text opacity=1, inner sep=2pt, rounded corners] (X) {\color{black}#1};%
}

\newcommand{\twopageicon}[2]{%
  \begin{tikzpicture}[baseline=(current bounding box.south)]
    % Top page
    \draw[fill=#1, draw=black] (0, 0) rectangle ++(1em, 1.2ex);
    % Bottom page, directly below the top page
    \draw[fill=#2, draw=black] (0, -1.5ex) rectangle ++(1em, 1.2ex);
  \end{tikzpicture}%
}

% Define the command for a label/tag icon with a customizable color
\newcommand{\tagicon}[1]{%
  \begin{tikzpicture}[baseline=(current bounding box.south)]
    % Tag shape
    \draw[fill=#1, draw=black] 
      (0, 0) -- (1em, 0) -- (1em, 1.2ex) -- (0.6em, 1.2ex) -- (0.4em, 0.6ex) -- (0.6em, 0) -- cycle;
  \end{tikzpicture}%
}




%%
%% Submission ID.
%% Use this when submitting an article to a sponsored event. You'll
%% receive a unique submission ID from the organizers
%% of the event, and this ID should be used as the parameter to this command.
%%\acmSubmissionID{123-A56-BU3}

%%
%% For managing citations, it is recommended to use bibliography
%% files in BibTeX format.
%%
%% You can then either use BibTeX with the ACM-Reference-Format style,
%% or BibLaTeX with the acmnumeric or acmauthoryear sytles, that include
%% support for advanced citation of software artefact from the
%% biblatex-software package, also separately available on CTAN.
%%
%% Look at the sample-*-biblatex.tex files for templates showcasing
%% the biblatex styles.
%%

%%
%% The majority of ACM publications use numbered citations and
%% references.  The command \citestyle{authoryear} switches to the
%% "author year" style.
%%
%% If you are preparing content for an event
%% sponsored by ACM SIGGRAPH, you must use the "author year" style of
%% citations and references.
%% Uncommenting
%% the next command will enable that style.
%%\citestyle{acmauthoryear}

\copyrightyear{2025}
\acmYear{2025}
% \setcopyright{cc}
% \setcctype{by}
\acmConference[CHI '25]{CHI Conference on Human Factors in Computing
Systems}{April 26-May 1, 2025}{Yokohama, Japan}
\acmBooktitle{CHI Conference on Human Factors in Computing Systems (CHI
'25), April 26-May 1, 2025, Yokohama,
Japan}\acmDOI{10.1145/3706598.3714119}
\acmISBN{979-8-4007-1394-1/25/04}

%%
%% end of the preamble, start of the body of the document source.
\begin{document}

%%
%% The "title" command has an optional parameter,
%% allowing the author to define a "short title" to be used in page headers.
\title[A Layered Interface for Writing with LLMs]{\system: A Layered Interface Paradigm for Integrating Content Development and Rhetorical Strategy with LLM Writing Assistants}

%%
%% The "author" command and its associated commands are used to define
%% the authors and their affiliations.
%% Of note is the shared affiliation of the first two authors, and the
%% "authornote" and "authornotemark" commands
%% used to denote shared contribution to the research.



\author{Momin Siddiqui}
 \affiliation{%
  \institution{Georgia Institute of Technology}
   \country{USA}
 }
 \email{msiddiqui66@gatech.edu}


\author{Roy Pea}
 \affiliation{%
  \institution{Stanford University}
   \country{USA}
 }
 \email{roypea@stanford.edu}


\author{Hari Subramonyam}
 \affiliation{%
  \institution{Stanford University}
   \country{USA}
 }
 \email{harihars@stanford.edu}





%%
%% By default, the full list of authors will be used in the page
%% headers. Often, this list is too long, and will overlap
%% other information printed in the page headers. This command allows
%% the author to define a more concise list
%% of authors' names for this purpose.
\renewcommand{\shortauthors}{Siddiqui, et al.}

%%
%% The abstract is a short summary of the work to be presented in the
%% article.
\begin{abstract}
  Good writing is a dynamic process of knowledge transformation, where writers refine and evolve ideas through planning, translating, and reviewing. Generative AI-powered writing tools can enhance this process but may also disrupt the natural flow of writing, such as when using LLMs for complex tasks like restructuring content across different sections or creating smooth transitions. We introduce \system, a \textit{layered interface paradigm} designed to minimize these disruptions by aligning writing intents with LLM capabilities to support diverse content development and rhetorical strategies. By bridging envisioning, semantic, and articulatory distances, \systems interactions allow writers to leverage LLMs for various content development tasks (\textit{scripting}) and experiment with diverse organization strategies while tailoring their writing for different audiences (\textit{shifting}). This approach preserves creative control while encouraging divergent and iterative writing. Our evaluation shows that \system enables writers to creatively and efficiently incorporate LLMs while preserving a natural flow of composition.
\end{abstract}

%%
%% The code below is generated by the tool at http://dl.acm.org/ccs.cfm.
%% Please copy and paste the code instead of the example below.
%%
% \begin{CCSXML}
% <ccs2012>
%  <concept>
%   <concept_id>10010520.10010553.10010562</concept_id>
%   <concept_desc>Computer systems organization~Embedded systems</concept_desc>
%   <concept_significance>500</concept_significance>
%  </concept>
%  <concept>
%   <concept_id>10010520.10010575.10010755</concept_id>
%   <concept_desc>Computer systems organization~Redundancy</concept_desc>
%   <concept_significance>300</concept_significance>
%  </concept>
%  <concept>
%   <concept_id>10010520.10010553.10010554</concept_id>
%   <concept_desc>Computer systems organization~Robotics</concept_desc>
%   <concept_significance>100</concept_significance>
%  </concept>
%  <concept>
%   <concept_id>10003033.10003083.10003095</concept_id>
%   <concept_desc>Networks~Network reliability</concept_desc>
%   <concept_significance>100</concept_significance>
%  </concept>
% </ccs2012>
% \end{CCSXML}

% \ccsdesc[500]{Computer systems organization~Embedded systems}
% \ccsdesc[300]{Computer systems organization~Redundancy}
% \ccsdesc{Computer systems organization~Robotics}
% \ccsdesc[100]{Networks~Network reliability}

\begin{CCSXML}
<ccs2012>
   <concept>
       <concept_id>10003120.10003123.10011760</concept_id>
       <concept_desc>Human-centered computing~Systems and tools for interaction design</concept_desc>
       <concept_significance>500</concept_significance>
       </concept>
   <concept>
       <concept_id>10003120.10003123.10011759</concept_id>
       <concept_desc>Human-centered computing~Empirical studies in interaction design</concept_desc>
       <concept_significance>500</concept_significance>
       </concept>
 </ccs2012>
\end{CCSXML}

\ccsdesc[500]{Human-centered computing~Systems and tools for interaction design}
\ccsdesc[500]{Human-centered computing~Empirical studies in interaction design}
%%
%% Keywords. The author(s) should pick words that accurately describe
%% the work being presented. Separate the keywords with commas.
% \keywords{datasets, neural networks, gaze detection, text tagging}
\keywords{Human-AI collaborative writing, large language models, writing assistants, creativity support}

%% A "teaser" image appears between the author and affiliation
%% information and the body of the document, and typically spans the
%% page.
\begin{teaserfigure}
  \includegraphics[width=\textwidth]{figures/teaser_figure_cr.png}
  \caption{\system is an AI-assisted writing interface that empowers writers to query LLMs using familiar visual design elements. (A) Scripting enables users to query several specialized Writer's Friends inline. (B) Shifting allows writers to reorganize layers and combine them to see what the final document looks like. }
  \Description{}
  \label{fig:teaser}
\end{teaserfigure}

% \received{20 February 2007}
% \received[revised]{12 March 2009}
% \received[accepted]{5 June 2009}

%%
%% This command processes the author and affiliation and title
%% information and builds the first part of the formatted document.
\maketitle


\section{Introduction}\label{sec:Intro} 


Novel view synthesis offers a fundamental approach to visualizing complex scenes by generating new perspectives from existing imagery. 
This has many potential applications, including virtual reality, movie production and architectural visualization \cite{Tewari2022NeuRendSTAR}. 
An emerging alternative to the common RGB sensors are event cameras, which are  
 bio-inspired visual sensors recording events, i.e.~asynchronous per-pixel signals of changes in brightness or color intensity. 

Event streams have very high temporal resolution and are inherently sparse, as they only happen when changes in the scene are observed. 
Due to their working principle, event cameras bring several advantages, especially in challenging cases: they excel at handling high-speed motions 
and have a substantially higher dynamic range of the supported signal measurements than conventional RGB cameras. 
Moreover, they have lower power consumption and require varied storage volumes for captured data that are often smaller than those required for synchronous RGB cameras \cite{Millerdurai_3DV2024, Gallego2022}. 

The ability to handle high-speed motions is crucial in static scenes as well,  particularly with handheld moving cameras, as it helps avoid the common problem of motion blur. It is, therefore, not surprising that event-based novel view synthesis has gained attention, although color values are not directly observed.
Notably, because of the substantial difference between the formats, RGB- and event-based approaches require fundamentally different design choices. %

The first solutions to event-based novel view synthesis introduced in the literature demonstrate promising results \cite{eventnerf, enerf} and outperform non-event-based alternatives for novel view synthesis in many challenging scenarios. 
Among them, EventNeRF \cite{eventnerf} enables novel-view synthesis in the RGB space by assuming events associated with three color channels as inputs. 
Due to its NeRF-based architecture \cite{nerf}, it can handle single objects with complete observations from roughly equal distances to the camera. 
It furthermore has limitations in training and rendering speed: 
the MLP used to represent the scene requires long training time and can only handle very limited scene extents or otherwise rendering quality will deteriorate. 
Hence, the quality of synthesized novel views will degrade for larger scenes. %

We present Event-3DGS (E-3DGS), i.e.,~a new method for novel-view synthesis from event streams using 3D Gaussians~\cite{3dgs} 
demonstrating fast reconstruction and rendering as well as handling of unbounded scenes. 
The technical contributions of this paper are as follows: 
\begin{itemize}
\item With E-3DGS, we introduce the first approach for novel view synthesis from a color event camera that combines 3D Gaussians with event-based supervision. 
\item We present frustum-based initialization, adaptive event windows, isotropic 3D Gaussian regularization and 3D camera pose refinement, and demonstrate that high-quality results can be obtained. %

\item Finally, we introduce new synthetic and real event datasets for large scenes to the community to study novel view synthesis in this new problem setting. 
\end{itemize}
Our experiments demonstrate systematically superior results compared to EventNeRF \cite{eventnerf} and other baselines. 
The source code and dataset of E-3DGS are released\footnote{\url{https://4dqv.mpi-inf.mpg.de/E3DGS/}}. 





\section{Background and Related Work}\label{sec:related}

\paragraph{\textbf{Privacy of Human-Centered Systems}}
Ensuring privacy in human-centric ML-based systems presents inherent conflicts among service utility, cost, and personal and institutional privacy~\cite{sztipanovits2019science}. Without appropriate incentives for societal information sharing, we may face decision-making policies that are either overly restrictive or that compromise private information, leading to adverse selection~\cite{jin2016enabling}. Such compromises can result in privacy violations, exacerbating societal concerns regarding the impact of emerging technology trends in human-centric systems~\cite{mulligan2016privacy,fox2021exploring,goldfarb2012shifts}. Consequently, several studies have aimed to establish privacy guarantees that allow auditing and quantifying compromises to make these systems more acceptable~\cite{jagielski2020auditing, raji2020saving}. ML models in decision-making systems have also been shown to leak significant amounts of private information that requires auditing platforms~\cite{hamon2022bridging}. Various studies focused on privacy-preserving machine learning techniques targeting decision-making systems~\cite{abadi2016deep, cummings2019compatibility, taherisadr2023adaparl, taherisadr2024hilt}. Recognizing that perfect privacy is often unattainable, this paper examines privacy from an equity perspective. We investigate how to ensure a fair distribution of harm when privacy leaks occur, addressing the technical challenges alongside the ethical imperatives of equitable privacy protection.


\paragraph{\textbf{\acf{fl}}}
\ac{fl} is an approach in machine learning that enables the collaborative training of models across multiple devices or institutions without requiring data to be centralized. This decentralized setup is particularly beneficial in fields where data-sharing restrictions are enforced by privacy regulations, such as healthcare and finance. \ac{fl} allows organizations to derive insights from data distributed across various locations while adhering to legal constraints, including the General Data Protection Regulation (GDPR) \cite{BG_Survey2,BG_Survey1}.

One of the most widely adopted methods in \ac{fl} is \ac{fedavg}, which operates through iterative rounds of communication between a central server and participating clients to collaboratively train a shared model. During each communication round, the server sends the current global model to each client, which uses their locally stored data to perform optimization steps. These optimized models are subsequently sent back to the server, where they are aggregated to update the global model. The process repeats until the model converges. Known for its simplicity and effectiveness, \ac{fedavg} serves as the primary technique for coordinating model updates across distributed clients in our work. Additionally, we specifically employ horizontal federated learning, where data is distributed across entities with similar feature spaces but distinct user groups \cite{BG_HorizontalFL}.

\paragraph{\textbf{Privacy Risks in \ac{fl}}}
Privacy risks are a critical concern in \ac{fl}, as collaborative training on decentralized data can inadvertently expose sensitive information. A primary threat is the \ac{mia}, where adversaries determine whether specific data records were part of the model's training set \cite{shokri2017membership,BG_MIA}. Researchers have since demonstrated \ac{mia}'s effectiveness across various machine learning models, including \ac{fl}, showing, for example, that adversaries can infer if a specific location profile contributed to an FL model \cite{BG_MIA_1,BG_MIA_2}. However, while \ac{mia} identifies training members, it does not reveal the client that contributed the data. \ac{sia}, introduced in \cite{BG_SIA_2}, extends \ac{mia} by identifying which client owns a training record, thus posing significant security risks by exposing client-specific information in \ac{fl} settings.

The \ac{noniid} nature of data in federated learning presents additional privacy challenges, as variations in data distributions across clients heighten the risk of privacy leakage. When data distributions differ widely among clients, individual model updates become more distinguishable, potentially allowing attackers to infer sensitive information \cite{BG_NON_IID}. This distinctiveness in updates can make federated models more susceptible to inference attacks, such as \ac{mia} and \ac{sia}, as malicious actors may exploit these distributional differences to trace updates back to specific clients. This vulnerability is especially relevant in our work, as we use the \ac{har} dataset, which is inherently \ac{noniid} across clients, thus posing an increased risk for privacy leakage.




\paragraph{\textbf{Fairness in \ac{fl}}}
Fairness in \ac{fl} is crucial due to the varied data distributions among clients, which can lead to biased outcomes favoring certain groups \cite{BG_Fairness_2}. Achieving fairness involves balancing the global model's benefits across clients despite the decentralized nature of the data. Approaches include group fairness, ensuring performance equity across client groups, and performance distribution fairness, which focuses on fair accuracy distribution~\cite{selialia2024mitigating}. Additional types are selection fairness (equitable client participation), contribution fairness (rewards based on contributions), and expectation fairness (aligning performance with client expectations) \cite{BG_Fairness}. Achieving fairness in \ac{fl} across these various dimensions remains challenging due to the inherent heterogeneity of client data and environments. In response to this heterogeneity, personalization has emerged as a strategy to tailor models to individual clients, enhancing local performance~\cite{BG_Personalization,BG_Personalization_2, BG_FairnessPrivacy}.   

When considering fairness in FL, it is crucial to address the interplay with privacy. Specifically, ensuring an equitable distribution of privacy risks across clients is paramount, preventing any group from being disproportionately vulnerable to privacy leakage, particularly under attacks such as source inference attacks (SIAs).



\section{\change{\system: A Layered Interface For LLM Co-Writing} }
\begin{figure*}[t!]
    \centering
    \includegraphics[width=\textwidth]{figures/workspace_1.pdf}
    \caption{Workspace view of \system: (a) Zoomable,  scrollable writing workspace, (b) workspace level operations, (c) a text layer with layer toolbar on top, and (d) the compiled document viewer.}
    \label{fig:canvas}
\end{figure*}


\system is designed to support flexible generative writing workflows by interleaving content development and structural organization. In designing \system, our goal was to lower the \textit{envisioning gaps} around AI capabilities, instructions, and intentionality as well as to reduce the \textit{semantic} and \textit{articulatory} distance when interfacing with Generative AI. As shown in figure~\ref{fig:canvas}, the primary interface consists of an infinite zoomable \textit{workspace} where the writer can create and organize individual \textit{layers}. A \textit{layer} is a discrete, modular content unit within the writing workspace designed to encapsulate specific elements of the writing or organizational process. 
% Layers serve as flexible sandboxes where tasks such as tone adjustments, elaboration, or structural reorganization can occur in isolation, fostering iterative and exploratory workflows.

\change{Here, we use the ``layered'' metaphor because it aligns with how individuals think, by separating, refining, and integrating complex ideas. Unlike layers in visual tools such as Photoshop or Illustrator, which focus on the compositional blending of visual elements, \systems layers emphasize cognitive composition, representing distinct aspects of the writing or organizational process—such as brainstorming, tone adjustments, and meta-information. This metaphor naturally accommodates iterative workflows, enabling users to ``tear apart,'' ``combine,'' or ``stack'' layers for experimentation, similar to rearranging layers on a desk. Each layer acts as a \textit{sandbox} for specific tasks, preserving creative control and enabling non-linear exploration. By reimagining the familiar concept of layers, \system bridges user expectations with the flexibility and modularity required for writing tasks, where the process of exploration and refinement is as critical as the outcome.}



\system supports three types of layers, which include (1) a templated metadata layer for specifying the overall writing goals, (2) content layers for authoring the main text, and (3) a scratchpad for gathering relevant information to support the writing process. \system also supports `tags' for labeling individual layers and collections of layers (e.g., a stack of layers). Further, the interface provides contextual toolbars at the layer and workspace level. The layer-level toolbar consists of standard text formatting operations and specific AI functionality. The workspace-level toolbar supports adding layers and intelligent structuring operations. Further, \system implements an extensible collection of AI ``friends'' that provide targeted writing support for ideation, content organization, research, adjusting the tone, etc. To better understand how \system supports writing, let us follow Sashi, a freelance writer whose goal is to write an opinion piece about ``LLMs and Creative Ownership.'' 

\begin{figure*}[htbp!]
    \centering
    \begin{subfigure}[b]{0.28\textwidth}
        \centering
        \includegraphics[width=\textwidth]{figures/meta_layer.png}
        \caption{Specifying context of writing in meta layer}
    \end{subfigure}
    \hfill
    \begin{subfigure}[b]{0.28\textwidth}
        \centering
        \includegraphics[width=\textwidth]{figures/intro_layer.png}
        \caption{Starting free writing in Introduction Layer}
    \end{subfigure}
    \hfill
    \begin{subfigure}[b]{0.28\textwidth}
        \centering
        \includegraphics[width=\textwidth]{figures/slash_layer.png}
        \caption{Calling Detail Danny from the `/' trigger menu}
    \end{subfigure}
    
    \vspace{1em}
    
    \begin{subfigure}[b]{0.28\textwidth}
        \centering
        \includegraphics[width=\textwidth]{figures/danny_layer.png}
        \caption{Detail Danny generating two variations for user-specified prompt for detail/elaboration}
    \end{subfigure}
    \hfill
    \begin{subfigure}[b]{0.28\textwidth}
        \centering
        \includegraphics[width=\textwidth]{figures/ramesh_layer.png}
        \caption{Research friend as a scratchpad for researching topics specified in meta layer}
    \end{subfigure}
    \hfill
    \begin{subfigure}[b]{0.28\textwidth}
        \centering
        \includegraphics[width=\textwidth]{figures/child_layer.png}
        \caption{Adding links for child layers that the current layer will extract content from}
    \end{subfigure}
    
    \vspace{1em}
    
    \begin{subfigure}[b]{0.28\textwidth}
        \centering
        \includegraphics[width=\textwidth]{figures/import_layer.png}
        \caption{Tunneling into content from another layer into current layer}
    \end{subfigure}
    \hfill
    \begin{subfigure}[b]{0.28\textwidth}
        \centering
        \includegraphics[width=\textwidth]{figures/peak_layer.png}
        \caption{Peaking at (possible) future content using the bottom right}
    \end{subfigure}
    \hfill
    \begin{subfigure}[b]{0.28\textwidth}
        \centering
        \includegraphics[width=\textwidth]{figures/felix_layer.png}
        \caption{Invoking Feedback Felix for paragraph level feedback}
    \end{subfigure}
    \caption{Key Interactions supported by \system.}
    \label{fig:nine_figures}
\end{figure*}

\change{\subsection{Use Scenario}}
\subsubsection{Goal Setting}
To begin with, Sashi opens \system on his web browser and creates a new project. \system displays the workspace, adding a blank metadata layer~\textcolor{skyblue}{\faFile} to it (Figure~\ref{fig:nine_figures}a). The \textbf{metadata layer} contains a set of guiding questions and response text fields for writing goal, audience, topic context, and intent for writing. The metadata layer optionally allows Sashi to upload relevant documents to the writing task. In this case, Sashi enters the goal as \textit{``to write an opinion piece that explores the complex interplay between Large Language Models (LLMs), creative processes, and copyright law,''} the target audience as ``technology creators and potentially legal professionals,''  and intent as ``arguing for a reevaluation of current copyright frameworks to address the challenges posed by AI-generated content.'' Here, Sashi can also upload relevant documents, such as the news article about the AI-generated photo that won the art prize at the Colorado State Fair\footnote{\url{https://www.nytimes.com/2022/09/02/technology/ai-artificial-intelligence-artists.html}}, the New York Times Lawsuit against OpenAI\footnote{\url{https://nytco-assets.nytimes.com/2023/12/NYT_Complaint_Dec2023.pdf}}, and any other relevant legal documents. Sashi can always revisit this layer throughout his writing to provide additional context. At this point, Sashi clicks on the ``Begin Writing'' button at the bottom of the metadata layer. This adds a new layer to the workspace, which Sashi names `Introduction'  \pageiconwithnumber[lightestgray]{black}{I}.

\subsubsection{Content Development}
On this layer \pageiconwithnumber[lightestgray]{black}{I}, Sashi begins by engaging in \textit{free-writing}, exploring various real-world instances of LLM creativity he has encountered to motivate the article (Figure~\ref{fig:nine_figures}b). Once he has written several examples, he realizes that he must also include a brief description of LLMs and their creative abilities. Instead of manually describing LLMs, Sashi opts to leverage the content development features of \system. Inspired by `personas' for writing feedback~\cite{benharrak2024writer}, we have implemented a set of \textbf{writer's friends} that embody various writing skills, including \coloredcircle{IdeaIvy} Idea Ivy for content ideation, \coloredcircle{DetailDanny} Detail Danny for elaboration, \coloredcircle{StructureSam} Structure Sam for content structuring, \coloredcircle{ToneTara} Tone Tara for stylistic suggestions, \coloredcircle{FeedbackFelix} Feedback Felix for feedback on content, and \coloredcircle{AudienceAli} Audience Ali for audience-specific feedback. 


In this case, Sashi selects the Detail Danny friend by typing a forward slash `/' at the beginning of a new paragraph. This action opens a dropdown menu displaying a list of all friends, from which Sashi chooses \coloredcircle{DetailDanny} Detail Danny (Figure~\ref{fig:nine_figures}c). \system inserts an \customdashedbordertext{inline prompt box}{DetailDanny}{DetailDanny} where he writes \textit{``Write a concise description of LLMs and their creative abilities, explain how they function for the creative tasks listed above''} and presses the enter key. In response, \system generates a paragraph of text about LLMs and creativity. The \customsolidbordertext{generated text}{DetailDanny}{DetailDanny} is highlighted with a background color corresponding to the friend (Figure~\ref{fig:nine_figures}d). Sashi can accept this content by pressing enter at the end of the content or delete it and regenerate new content. Sashi accepts the suggestion and continues editing. Sashi has a high-level vision about a few topics his opinion essay should cover. He creates new layers for `Traditional Copyright Law - \pageiconwithnumber[lightestgray]{black}{L},' `Authorship -- \pageiconwithnumber[lightestgray]{black}{A},' and `Ethical Considerations -- \pageiconwithnumber[lightestgray]{black}{E}' and, as before, writes his thoughts. When writing about copyright law, Sashi realizes he needs to understand copyright principles better. He creates a new \textbf{scratchpad} layer ~\textcolor{legalyellow}{\faFile}, which is supported by Research Ramesh, where he can ask questions about the purpose and intent of copyright protection or how copyright has adapted to previous technological changes to help with his main content writing (Figure~\ref{fig:nine_figures}e). 

In the course of writing, Sashi creates a new layer for Stakeholder Perspectives -- \pageiconwithnumber[lightestgray]{black}{S} and brainstorms using  \coloredcircle{IdeaIvy} Idea Ivy about whose perspectives to include, and in response, Ivy suggests talking about \customsolidbordertext{Content Creators}{IdeaIvy}{IdeaIvy}, \customsolidbordertext{AI Companies}{IdeaIvy}{IdeaIvy}, \customsolidbordertext{Consumers}{IdeaIvy}{IdeaIvy}, and \customsolidbordertext{Legal Experts.}{IdeaIvy}{IdeaIvy}. By individually selecting each stakeholder, Sashi can create \textbf{sub-layers} to flesh out the stakeholder perspective (Figure~\ref{fig:nine_figures}f). There is a persistent link between the text in the Stakeholder layer \pageiconwithnumber[lightestgray]{black}{S} and the sub-layers [\pageiconwithnumber[lightestgray]{black}{$S^{CC}$}, \pageiconwithnumber[lightestgray]{black}{$S^{AI}$}, \pageiconwithnumber[lightestgray]{black}{$S^C$}, \pageiconwithnumber[lightestgray]{black}{$S^{LE}$}]. Sashi can also \textbf{split} the content into two layers and develop the content separately.  

Idea Ivy can also be used to support more complex writing procedures across layers, such as how some specific text about the economic impact in the Ethical Considerations layer \pageiconwithnumber[lightestgray]{black}{E} might relate to the current text in the content-creator layer \pageiconwithnumber[lightestgray]{black}{$S^{CC}$}. To support such complex prompting using content across layers, \system supports a \textit{tunneling} function where the current layer split opens at the cursor position to reveal a desired layer selected from the dropdown (Figure~\ref{fig:nine_figures}g). From here, Sashi can choose the relevant text and import it to the current prompt he is composing. On any layer, if he is stuck on what to write, he can click on the bottom right corner of the layer to \textbf{`peek'} into what the Generative AI might write (Figure~\ref{fig:nine_figures}h). The peek function shows continuing text based on his current writing in a greyed-out format. For creating structure from unstructured text, he can invoke \coloredcircle{StructureSam} Structure Sam, who will create a new layer with a structured representation of the content, including headings and subheadings. He can invoke the \coloredcircle{AudienceAli} audience and \coloredcircle{FeedbackFelix}feedback friends on any layer using the toolbar option to receive comments inline, which he can toggle on or off (Figure~\ref{fig:nine_figures}i). He can call upon \coloredcircle{ToneTara} Tone Tara to generate a different stylistic variations of the current content rendered as separate layers from the original. 

Lastly, as he is writing, Sashi can \textbf{compare} two layers by bringing them close, such that the right edge of a layer is touching the left edge of the other layer \pageiconwithnumber[lightestgray]{black}{$S^{AI}$}\pageiconwithnumber[lightestgray]{black}{$S^{CC}$}. \system recognizes this as an intent to compare and provides a floating button, ``Compare the two layers?'' On clicking this, the button expands into a text prompt where Sashi can input specific instructions for comparison, such as ``how the stakeholder perspectives of content creators and AI companies align or conflict regarding LLM-generated content and copyright.'' The system uses color highlighting and inline annotations to indicate similarities and differences. To mitigate possible confusions, these annotations only persist while the layers remain in proximity. 

\subsubsection{Structure and Rhetoric}
Throughout this process, \system provided several features for Sashi to develop the structure for his final essay and tailor it to specific audiences. To organize the content, \system supports \textbf{combining} two layers into a single layer by bringing the second layer to the bottom of the first layer \twopageicon{lightestgray}{lightestgray} and optionally prompting for specific transitional text between layers. For instance, he can connect a layer on `Challenges posed by LLMs' \pageiconwithnumber[lightestgray]{black}{Ch} with the layer on `Copyright Law' \pageiconwithnumber[lightestgray]{black}{C} by generating transitional text from identifying the problems to analyzing the legal framework meant to address them. Alternately, Sashi can \textbf{tear} layers into parts and glue them together to reorganize the text, i.e. to have content inform structure. Sashi can fold layers he wishes to exclude, which show the text summary on the folded side of the layer. He can \colorbox{yellow}{Tag} individual layers or even clusters and stacks of layers in the workspace. When Sashi is ready to \textit{compose} his essay, he can \textbf{stack} the layers manually in an order [\pageiconwithnumber[lightestgray]{black}{I},\pageiconwithnumber[lightestgray]{black}{C},\pageiconwithnumber[lightestgray]{black}{E},\pageiconwithnumber[lightestgray]{black}{S}\ldots] and generate the final document by concatenating them or asking the LLM to generate the order based on an unordered stack. Here he can issue specific prompts such as generating an audience-specific version, editing for consistency across layers, or even adaptive summarization to meet some target length. The final document is generated with any edited text highlighted for Sashi to review. Sashi can click on the text in the final document and track back to the source layer. In this manner, \systems features allow Sashi to use generative AI for content development and rhetorical strategies, flexibly, fluidly, and iteratively. Table~\ref{fig:layered_action} summarizes the different scripting and shifting affordances of \system.



% \change{The Figures \ref{fig:layered_action} and \ref{fig:writers_friend} below show all the features of \system.}


\begin{figure*}[t!]
    \centering
    \includegraphics[width=0.8\textwidth]{figures/generation_table.png}
    \caption{\change{Content transformation through user prompts. The following 10 features allow writers to issue descriptive instructions for invoking specialized LLM assistance. The example column consists of real prompts issued by participants in the usability assessment.}}
    \label{fig:layered_action}
\end{figure*}








\section{User Studies Overview}
To evaluate the effectiveness and usability of \theDevice, we conducted two user studies with signers ranging from novice to native/ fluent in ASL proficiency. The first study focused on word-level recognition, assessing the system's ability to accurately recognize individual fingerspelled words across a diverse group of participants. The second study expanded on these findings by examining phrase-level recognition in real-time scenarios, providing insights into the system's performance in more natural, context-rich environments. These studies aimed to validate SpellRing's performance across different levels of signing experience, explore the impact of signing speed and habits on recognition accuracy, and investigate how users adapt to the system in real-time use. The user study was approved by the Institutional Review Board (IRB) at the authors' institution. Participants were compensated \$40 per hour for their participation in the study.

\section{Example Writing Workflows with \system}

In this section, we demonstrate the flexibility of writing with \system through three distinct workflows: freewriting, document-based-question, and parallel topic development. Each example highlights how the layered interface and AI tools support fluid, iterative writing across different starting points and content structures.


\subsection{Freewriting to Argumentative Writing Workflow}
In this workflow, the writer begins with freewriting, allowing ideas to flow without worrying about organization or structure. Once they have written their thoughts, they determine that an argument style structure can be effective to organize the text, consisting of the following components: Claim, Grounds, Warrant, Backing, Qualifier, and Rebuttal~\cite{toulmin2003uses,marshall1989representing}. The writer adds the \textit{argument template}  to \systems workspace and drops the layer with the freewriting text onto the template. Using the template, \system takes the unstructured text and generates six layers, one for each component. Next, the writer can continue to flesh out each layer using Writer's Friends such as ``Idea Ivy'' and later refine, reorganize, and link ideas across layers, transforming initial thoughts into a structured argumentative essay.

\begin{figure*}[t]
    \centering
    \includegraphics[width=\textwidth, height=0.3\textheight, keepaspectratio]{figures/FreeWriting.pdf}
    \caption{Free Writing Example with \system \change{ \textbf{(A)} Writer calls \textit{Template} for organizing their writing into milestones for argumentative writing. \textbf{(B)} They select the ``Ground'' layer for further development. \textbf{(C)} They invoke \textit{Idea Ivy} to brainstorm what to write next and then use \textit{Structure Sam} to organize that into subheadings and paragraphs.}}
    \label{fig:freewriting}
\end{figure*}

\begin{figure*}[t]
    \centering
    \includegraphics[width=\textwidth, height=0.3\textheight, keepaspectratio]{figures/DBQ.pdf}
    \caption{Document-Based Question Example with \system \change{ \textbf{(A)} The writer specifies the context of their writing and upload their assignment to the \textit{Meta Layer}. \textbf{(B)} They call \textit{Research Ramesh} to understand details of their assignment based on the context document. \textbf{(C)} They \textit{Tunnel} into another layer to extract some details. \textbf{(D)} They combine a stack of layers to generate \textbf{E}.}}
    \label{fig:freewriting}
\end{figure*}


\subsection{DBQ Writing Workflow for Students}
In this workflow, the student working on a Document-Based Question (DBQ) can upload primary source documents directly into the system. The metadata page allows the student to provide information about the assignment goals, guiding questions, and document context. The student can then create additional layers for organizing their arguments, categorizing evidence drawn from the primary sources. For example, they might have separate layers for economic, political, and social factors. LLM friends like ``Research Ramesh'' assist by analyzing the documents and suggesting relevant excerpts or summaries that will fit the argument. As the student develops their essay, cross-layer interactions allow them to pull evidence from the source layers into argument layers, ensuring that evidence is seamlessly linked back to the original documents. Using \systems stacking feature, the student can then combine the various layers into a cohesive essay. \system allows them to specify prompts for transitions in order to ensure that all arguments are well-supported by evidence. This workflow encourages students to engage deeply with primary sources, while providing an intuitive way to structure their argument, making the ballistic process of evidence-based writing more fluid and interactive.


\subsection{Layered Topic Development for Research Paper}

In this workflow, the writer develops different sections of the research paper in parallel, creating distinct layers for the literature review, methodology, and findings. \system allows the writer to work on these topics simultaneously, using cross-layer interactions to draw connections between sections. For instance, the writer can link methodology details to key studies in the literature review. Within the findings section, the writer can create separate layers for different findings or topics, allowing them to focus on each result individually. Once these layers are complete, the writer can combine them into a unified findings section, ensuring a cohesive and structured presentation of the results. Writer's friends like ``Research Ramesh'' assist by sourcing and summarizing relevant research, while ``Tone Tara'' and ``Feedback Felix'' help ensure stylistic and argumentative consistency across sections. Once the layers are fully developed, the document structuring features in \system help weave them together into a cohesive research paper.
\section{Implementation and Evaluation}



\begin{figure}[htbp]
\centering
    \includegraphics[width=1.0\linewidth,keepaspectratio]{figures/evaluation.png}
    \caption{Experimental pipeline showing initial privacy screening, reformulation by three local models, and evaluation stages.}
    \label{fig:experiment}
\end{figure}

\subsection{Contextual Privacy Evaluation of Real-World Queries}
\label{sec:sharegot_privacy_evaluation}
Before implementing and evaluating our framework, we first perform initial privacy analysis by evaluating 
an open-source version of the ShareGPT dataset~\citep{vicuna2023} to understand the prevalence of contextual privacy violations. To instantiate our formal privacy definition, we used Llama-3.1-405B-Instruct \cite{grattafiori2024llama3herdmodels} as judge, with a prompt designed to identify violations of contextual integrity (Appendix \ref{appendix_ci_detection}). From over 90,000 conversations, we retain 11,305 single-turn conversations within a reasonable length range (25-2,500 words). For each conversation, the judge model assessed the context, sensitive information, and their necessity for task completion. This analysis identified approximately 8,000 conversations containing potential contextual integrity violations. To manage inference costs, we focused on cases where the judge model could successfully identify a primary context and classify essential and non-essential information attributes, yielding 2,849 conversations (25.2\%) with definitive contextual privacy violations. Examples of these violations are shown in Table \ref{tab:example_ci_violations}. Manual inspection of the judge's results for consistency and correctness demonstrated good classification performance with few false positives and negatives.

\subsection{Implementation Details}
\textbf{Models.} We implement our framework using a model that is significantly smaller than typical chat agents like ChatGPT, enabling users to deploy the model locally via Ollama\footnote{\url{https://github.com/ollama/ollama}} without relying on external APIs.
In our experiments, we evaluate three models with different characteristics: Mixtral-8x7B-Instruct-v0.1\footnote{\url{https://ollama.com/library/mixtral:8x7b-instruct-v0.1-q4\_0}} \cite{jiang2024mixtralexperts}, Llama-3.1-8B-Instruct\footnote{\url{https://ollama.com/library/llama3.1:8b-instruct-fp16}} \cite{grattafiori2024llama3herdmodels}, 
and DeepSeek-R1-Distill-Llama-8B\footnote{\url{https://ollama.com/library/deepseek-r1:8b-llama-distill-q4_K_M}} (focused on reasoning) \cite{deepseekai2025deepseekr1}. We refer to these models as Mixtral, Llama and Deepseek in short going forward.
The local deployment of models ensures no further privacy leakage due to the framework. Although our evaluation focuses on three LLMs, our approach is model-agnostic and can be applied to other architectures. For assessment of privacy and utility, we use Llama-3.1-405B-Instruct \cite{grattafiori2024llama3herdmodels} as an impartial judge, which was hosted in a secure cloud infrastructure.

\textbf{Experiment Setup.} 
As discribed in the previous section,
our framework processes user prompts in three stages: (a) context identification, (b) sensitive information classification, and (c) reformulation. The locally deployed model first determines the context of the conversation, identifying its domain and task (Appendix ~\ref{domains_and_tasks}) using the prompts in Appendix Appendix~\ref{appendix_intent_detection} and Appendix ~\ref{appendix_task_detection} respectively. It then detects sensitive information, categorizing it as either \textit{essential} (required for task completion) or \textit{non-essential} (privacy-sensitive and removable). Finally, if non-essential sensitive information is present, the model reformulates the prompt to improve privacy while preserving intent.

We implement two approaches for sensitive information classification: \textbf{dynamic classification}  and \textbf{structured classification}, each reflecting different ways to operationalize our privacy framework. In the \textbf{dynamic classification approach} (see prompt used in Appendix ~\ref{appendix_dynamic_sentive_template}),  the model determines which details are essential based on how they are used within the specific conversation. For instance, in the prompt \emph{"I’m Jane, a single parent of two, and was just diagnosed with diabetes. I’m looking for affordable treatment options"}, the model would identify the phrases= \emph{["diabetes"]} as the essential attributes, while \emph{["Jane", "single parent of two","affordable"]} would be classified as non-essential. This adaptive method aligns with contextual privacy formulation, ensuring that only task-relevant details are retained. In contrast, the \textbf{structured classification approach} (see prompt used in Appendix ~\ref{appendix_structured_sentive_template}), allows to specify a predefined list of sensitive attributes (e.g., age, SSN, physical health, allergies) that should always be considered non-essential (protected), ensuring consistent enforcement of privacy policies.  For the same example, this approach would flag \emph{["physical health"]} as the essential attribute while labeling \emph{["name", "family status", "financial condition"]} as non-essential attributes, recommending them for removal based on user-defined privacy preferences. This provides greater control over what information is considered sensitive, allowing customization while maintaining a standardized privacy framework. The predefined attribute categories follow those defined in \citet{bagdasaryan2024air}.

If non-essential sensitive details are detected, the model reformulates the prompt by either removing or rewording them to minimize privacy risks while maintaining usability (see Prompt used in Appendix \ref{appendix_reformulation}). By evaluating both dynamic and structured classification, we demonstrate the flexibility of our framework in balancing adaptability with user-defined privacy controls.

\subsection{Evaluation and Results}

We evaluate our framework by measuring two key metrics: \textbf{privacy gain} and \textbf{utility}. Privacy gain quantifies how effectively sensitive information is removed during reformulation, while utility measures how well the reformulated prompt maintains the original prompt's intent. We compute these metrics using two complementary methods: an automated BERTScore-based comparison of sensitive attributes, and an LLM-based assessment that aggregates multiple evaluation aspects.

\subsubsection{Evaluation via Attribute-based Metrics} 
\paragraph{Metrics.} 
We measure privacy gain by computing semantic similarity between non-essential attributes between original and reformulated prompts, where similarity is computed using BERTScore \citep{zhang2020bertscore}. 
Specifically, we first run the judge model on reformulated prompts to obtain non-essential sensitive attributes $\mathcal{P}^{\textrm{reform}}_{\textrm{non-ess}}$, using a prompt designed to identify contextual privacy violations (Appendix \ref{appendix_ci_detection}). We have non-essential sensitive attributes for original prompts $\mathcal{P}^{\textrm{orig}}_{\textrm{non-ess}}$ from Section \ref{sec:sharegot_privacy_evaluation}. 
Given sets of strings $\mathcal{P}^{\textrm{orig}}_{\textrm{non-ess}}$ and $\mathcal{P}^{\textrm{reform}}_{\textrm{non-ess}}$, 
privacy gain is computed as
$1 - \text{BERTScore}(\mathcal{P}^{\textrm{orig}}_{\textrm{non-ess}}, \mathcal{P}^{\textrm{reform}}_{\textrm{non-ess}})$, with a score of 1.0 assigned when either set is empty. A higher privacy gain indicates better removal of sensitive information. For utility, we measure semantic similarity between essential attributes using $\text{BERTScore}(\mathcal{P}^{\textrm{orig}}_{\textrm{ess}}, \mathcal{P}^{\textrm{reform}}_{\textrm{ess}})$, where a score closer to 1.0 indicates better preservation of task-critical information. Since BERTScore works on text pairs, we match each original attribute to its closest reformulated one and compute utility as the fraction of matched attributes above a similarity threshold of 0.5.


\begin{table}[t]
  \small
  \setlength{\tabcolsep}{4pt}
  \centering
  \caption{BERT-based Evaluation of Privacy and Utility}
  \begin{tabular}{lcc}
    \toprule
    \multicolumn{3}{c}{\textbf{Dynamic Attribute Classification}} \\ \midrule
    \textbf{Model} & \textbf{Privacy Gain $\uparrow$} & \textbf{Utility(BERTScore)$\uparrow$} \\ \midrule
    Deepseek  & 0.853 & 0.570 \\
    Llama     & 0.886 & 0.567 \\
    Mixtral   & 0.873 & 0.570 \\ \midrule
    \multicolumn{3}{c}{\textbf{Structured Attribute Classification}} \\ \midrule
    \textbf{Model} & \textbf{Privacy Gain $\uparrow$} & \textbf{Utility(BERTScore)$\uparrow$} \\ \midrule
    Deepseek  & 0.836 & 0.511 \\
    Llama     & 0.873 & 0.606 \\
    Mixtral   & 0.824 & 0.576 \\ \bottomrule
  \end{tabular}
  \label{tab:privacy_utility}
\end{table}

\paragraph{Results.} Table~\ref{tab:privacy_utility} shows that under dynamic classification, all three models achieve strong privacy scores (0.85-0.88) with comparable utility ($\sim0.57$), suggesting that the ability to identify context-specific sensitive information is robust across different model architectures.

The structured classification approach shows greater variation between models. While Llama achieves high scores in both privacy (0.873) and utility (0.606), structured classification generally yields slightly lower privacy scores but more variable utility. This suggests a natural trade-off: predefined categories might miss some context-specific sensitive information, yet operating within these fixed boundaries can help preserve task-relevant content. Interestingly, the similar performance patterns across different model architectures suggest that the choice between instruction-tuned and reasoning-focused approaches may be less crucial for privacy-preserving reformulation.

The success of both dynamic and structured approaches offers implementation flexibility - users can choose predefined privacy rules or context-specific protection based on their requirements. This choice, rather than model architecture, appears to be the key decision factor in deployment.


\subsubsection{LLM-as-a-Judge Assessment} 
\paragraph{Setup.}  We use Llama-3.1-405B-Instruct as a judge to provide a complementary evaluation of privacy and utility across 100 randomly selected queries per model (6×100 total). Given the high computational cost of LLM-based inference, this targeted sampling allows us to validate key trends observed in the attribute-based evaluation while minimizing overhead. 
Privacy gain is computed by asking the judge to evaluate privacy leakage, coverage, and retention, while utility 
is computed by measuring
query relevance, response validity, and cross-relevance. 
These binary evaluations are averaged to produce final privacy gains and utility scores. See Appendix \ref{appendix_evaluation} for detailed prompts and evaluation criteria.

\paragraph{Results.} The LLM-based assessment shows generally higher utility scores (0.82-0.86) across all models compared to BERTScore-based evaluation, while maintaining similar privacy levels (0.80-0.86). This difference can be attributed to how attributes are detected and compared—BERTScore evaluates exact semantic matches between attributes, while the LLM judge takes a more holistic view of information preservation. For instance, when essential information is restructured (e.g., ``my friend Mark'' split into separate attributes), BERTScore may indicate lower utility despite semantic equivalence.

The LLM evaluation confirms the effectiveness of both classification approaches, with dynamic classification showing slightly more consistent performance across models. Llama maintains its strong performance under both approaches (privacy gain: $\sim0.85$, utility score: $\sim0.86$), reinforcing its reliability for privacy-preserving reformulation.


\begin{table}[t]
\small
\centering
\caption{LLM-as-a-Judge Evaluation of Privacy and Utility}
\begin{tabular}{lcc}
\toprule
\textbf{Model} & \textbf{Privacy Gain $\uparrow$} & \textbf{Utility Score$\uparrow$} \\ 
\midrule
\multicolumn{3}{c}{\textbf{Dynamic Attribute Classification}} \\ 
\midrule
Deepseek  & 0.802 & 0.845 \\ 
Llama     & 0.858 & 0.861 \\ 
Mixtral   & 0.848 & 0.838 \\ 
\midrule
\multicolumn{3}{c}{\textbf{Structured Attribute Classification}} \\ 
\midrule
Deepseek  & 0.815 & 0.825 \\ 
Llama     & 0.855 & 0.858 \\ 
Mixtral   & 0.845 & 0.828 \\ 
\bottomrule
\end{tabular}
\label{tab:llm_judge_results}
\end{table}

\subsubsection{Example Reformulations and Trade-offs}
\paragraph{Setup.} Table~\ref{tab:privacy_utility_scores} presents a set of diverse example reformulations illustrating our framework’s ability to balance privacy and utility across different scenarios. These examples highlight both ideal cases—where reformulation effectively preserves both privacy and utility—and more challenging ones where trade-offs are unavoidable.

\paragraph{Results.} Our framework successfully removes personal identifiers while preserving task relevance, as seen in the third example ($\text{privacy gain} = 0.5,\text{utility score}= 0.83$). In creative requests like the Valentine’s poem (second example), removing personal details reduces privacy risks but slightly impacts personalization $\text{utility}=0.5$).

Some contexts resist reformulation. The last conversation ($\text{privacy gain}= 0.0, \text{utility score}= 0.0$) highlights cases where the entire prompt is inherently sensitive, requiring alternative privacy measures beyond text transformation. This is not a failure of our approach but an indicator of when reformulation alone is insufficient.

These examples reinforce that privacy-preserving reformulation is a trade-off, not a one-size-fits-all solution.

    




\begin{table*}[t]
  \small
  \scriptsize
\renewcommand{\arraystretch}{1.2}
  \centering
  \caption{Examples of original and reformulated prompts with privacy and utility scores. These examples illustrate how our framework removes sensitive details while preserving task intent. The high privacy, high utility cases demonstrate successful reformulation, while the low privacy, low utility example highlights cases where effective reformulation is difficult or unnecessary.}
  \label{tab:privacy_utility_scores}
  \resizebox{0.99\textwidth}{!}{
  \begin{tabular}{p{5.5cm}p{5.5cm}cc}
    \toprule
    \textbf{Original Prompt} & \textbf{Reformulated Prompt} & \textbf{Privacy Gain} & \textbf{Utility (BERTScore)} \\ \hline
    My friend Mark who was just laid off from Google, is looking for a job where he can use ML and Python. Do you have any advice for him? 
    & Someone is looking for a job where they can use ML and Python skills. Do you have any advice? 
    & \cellcolor{green!20} \textbf{1.0} 
    & \cellcolor{green!20} \textbf{1.0} \\ \hline

    Please write a poem for my wife for Valentine's Day. Her name is Gili. Please use rhymes. Please make it sarcastic, but not too much so. A bit funny but romantic.  
    & Write a poem for Gili, incorporating sarcasm at an appropriate level and maintaining a tone that is both funny and romantic  
    & \cellcolor{green!20} \textbf{1.0} 
    & \cellcolor{yellow!20} \textbf{0.5} \\ \hline

     I am writing a character letter to a judge in hopes of having charges against my friend Amber to be dropped. Write a letter in a positive tone to a judge speaking to Amber's good character ...... How you know me, my ties to the community my profession and stating a dismissal is imperative to my earning potential ... I know amber from "Teach For America". I know her to be a passionate educator who cares about uplifting the youth and her community. She should have the charges dropped so that she can increase her community engagement and earning potential for her family.  
    & Write a character letter in a positive tone to a judge speaking to someone's good character, highlighting their community ties and profession, and stating that a dismissal is imperative to their earning potential. I know this person from a professional program and they have been a passionate advocate for the betterment of their community. The charges should be dropped so that they can increase their engagement in the community and improve their financial situation. 
    & \cellcolor{yellow!20} \textbf{0.5} 
    & \cellcolor{green!10} \textbf{0.83} \\ \hline

    Sunny Balwani : I worked for 6 years day and night to help you. Elizabeth Holmes : I was just thinking about texting you in that minute by the way  
    & Sunny Balwani : I am responsible for everything at Theranos. Elizabeth Holmes : .........  
    & \cellcolor{red!20} \textbf{0.0} 
    & \cellcolor{red!20} \textbf{0.0} \\ 
    \bottomrule
  \end{tabular}
  }
\end{table*}



    
    
    
  

\section{Discussion}
\omniUIST is capable of tracking a passive tool with an accuracy of roughly 6.9 mm and, at the same time, deliver a maximum force of up to 2 N to the tool. This is enabled by our novel gradient-based approach in 3D position reconstruction that accounts for the force exerted by the electromagnet. 

Over extended periods of time, \omniUIST can comfortably produce a force of 0.615 N without the risk of overheating. In our applications, we show that \omniUIST has the potential for a wide range of usage scenarios, specifically to enrich AR and VR interactions.

\omniUIST is, however, not limited to spatial applications. We believe that \omniUIST can be a valuable addition to desktop interfaces, e.g., navigating through video editing tools or gaming. We plan to broaden \omniUIST's usage scenarios in the future.

The overall tracking performance of \omniUIST suffices for interactive applications such as the ones shown in this paper. The accuracy could be improved by adding more Hall sensors, or optimizing their placement further (e.g., placing them on the outer hull of the device).
Furthermore, a spherical tip on the passive tool that more closely resembles the dipole in our magnetic model could further improve \omniUIST's accuracy. We believe, however, that the design of \omniUIST represents a good balance of cost and complexity of manufacturing, and accuracy.

Our current implementation of \omniUIST and the accompanying tracking and actuation algorithms assumes the presence of a single passive tool. Our method, however, potentially generalizes to tracking multiple passive tools by accounting for the presence of multiple permanent magnets. This poses another interesting challenge: the magnets of multiple tools will interact with each other, i.e., attract and repel each other.The electromagnet will also jointly interact with those tools, leading to challenges in terms of computation and convergence. We believe that our gradient-based optimization can account for such interactions and plan to investigate this in the future.

In developing and testing our applications, we found that \omniUIST's current frame rate of 40 Hz suffices for many interactive scenarios. The frame rate is a trade-off between speed and accuracy. In our tests, decreasing the desired accuracy in our optimization doubled the frame rate, while resulting in errors in the 3D position estimation of more than 1 cm, however. Finding the sweet spot for this trade-off depends on the application. While our applications worked well with 40 Hz and the current accuracy, more intricate actions such as high-precision sculpting might benefit from higher frame rates \textit{and} precision.
Reducing the latency of several system components (e.g., sensor latency, convergence time) is another interesting direction of future research. 

Furthermore, the control strategy we used was fairly naïve, as it only takes the current tool position into account. A model predictive strategy could account for future states, user intent, and optimize to reduce heating. We will explore in the next chapter how model predictive approaches can be used for haptic systems.

Overall, the main benefits of \omniUIST lie in the high accuracy and large force it can produce. It does so without mechanically moving parts, which would be subject to wear.
Such wear is not the case for our device, because it is exclusively based on electromagnetic force. We believe that different form factors of \omniUIST (e.g., body-mounted, larger size) can present interesting directions of future research. \add{A body-mounted version could be interesting for VR applications in which the user moves in 3D space. The larger size could result in more discernible points.}

Additionally, the influence of strength on user perception and factors such as just-noticeable-difference will allow us to characterize the benefits and challenges of \omniUIST, and electromagnetic haptic devices in general.
We believe that \omniUIST opens interesting directions for future research in terms of novel devices, and magnetic actuation and tracking.
\section{Conclusion}
We introduced \Bench, the first ever IMTS forecasting benchmark.
\Bench's datasets are created with ODE models, that were defined in decades of research and published on
the Physiome Model Repository. Our experiments showed that LinODEnet and CRU are actually
better than previous evaluation on established datasets indicated. Nevertheless,
we also provided a few datasets, on which models are unable to outperform a
constant baseline model. We believe that our datasets, especially the very difficult ones,
can help to identify deficits of current architectures and support future research on
IMTS forecasting.




\bibliographystyle{ACM-Reference-Format}
\bibliography{99_refs}

\end{document}

